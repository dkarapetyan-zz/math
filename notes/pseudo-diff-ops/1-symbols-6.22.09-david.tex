\documentclass[12pt,reqno]{amsart}
\usepackage{amscd}
\usepackage{amsfonts}
\usepackage{amsmath}
\usepackage{amssymb}
\usepackage{amsthm}
\usepackage{appendix}
\usepackage{fancyhdr}
\usepackage{latexsym}
\usepackage{pdfsync}
\usepackage{cancel}
\usepackage{amsxtra}
\usepackage[colorlinks=true, pdfstartview=fitv, linkcolor=blue,
citecolor=blue, urlcolor=blue]{hyperref}
\input epsf
\input texdraw
\input txdtools.tex
\input xy
\xyoption{all}
%%%%%%%%%%%%%%%%%%%%%%
\usepackage{color}
\definecolor{red}{rgb}{1.00, 0.00, 0.00}
\definecolor{darkgreen}{rgb}{0.00, 1.00, 0.00}
\definecolor{blue}{rgb}{0.00, 0.00, 1.00}
\definecolor{cyan}{rgb}{0.00, 1.00, 1.00}
\definecolor{magenta}{rgb}{1.00, 0.00, 1.00}
\definecolor{deepskyblue}{rgb}{0.00, 0.75, 1.00}
\definecolor{darkgreen}{rgb}{0.00, 0.39, 0.00}
\definecolor{springgreen}{rgb}{0.00, 1.00, 0.50}
\definecolor{darkorange}{rgb}{1.00, 0.55, 0.00}
\definecolor{orangered}{rgb}{1.00, 0.27, 0.00}
\definecolor{deeppink}{rgb}{1.00, 0.08, 0.57}
\definecolor{darkviolet}{rgb}{0.58, 0.00, 0.82}
\definecolor{saddlebrown}{rgb}{0.54, 0.27, 0.07}
\definecolor{black}{rgb}{0.00, 0.00, 0.00}
\definecolor{dark-magenta}{rgb}{.5,0,.5}
\definecolor{myblack}{rgb}{0,0,0}
\definecolor{darkgray}{gray}{0.5}
\definecolor{lightgray}{gray}{0.75}
%%%%%%%%%%%%%%%%%%%%%%
%%%%%%%%%%%%%%%%%%%%%%%%%%%%
%  for importing pictures  %
%%%%%%%%%%%%%%%%%%%%%%%%%%%%
\usepackage[pdftex]{graphicx}
\usepackage{epstopdf}
% \usepackage{graphicx}
%% page setup %%
\setlength{\textheight}{20.8truecm}
\setlength{\textwidth}{14.8truecm}
\marginparwidth  0truecm
\oddsidemargin   01truecm
\evensidemargin  01truecm
\marginparsep    0truecm
\renewcommand{\baselinestretch}{1.1}
%% new commands %%
\newcommand{\bigno}{\bigskip\noindent}
\newcommand{\ds}{\displaystyle}
\newcommand{\medno}{\medskip\noindent}
\newcommand{\smallno}{\smallskip\noindent}
\newcommand{\ts}{\textstyle}
\newcommand{\rr}{\mathbb{R}}
\newcommand{\p}{\partial}
\newcommand{\zz}{\mathbb{Z}}
\newcommand{\cc}{\mathbb{C}}
\newcommand{\ci}{\mathbb{T}}
\newcommand{\ee}{\varepsilon}
\def\refer #1\par{\noindent\hangindent=\parindent\hangafter=1 #1\par}
%% equation numbers %%
\renewcommand{\theequation}{\thesection.\arabic{equation}}
%% new environments %%
%\swapnumbers
\theoremstyle{plain}  % default
\newtheorem{theorem}{Theorem}
\newtheorem{proposition}{Proposition}
\newtheorem{lemma}{Lemma}
\newtheorem{corollary}{Corollary}
\newtheorem{conjecture}[subsection]{conjecture}
\theoremstyle{definition}
\newtheorem{definition}{Definition}
\newcommand{\ve}{\varepsilon} 
\newcommand{\nin}{\noindent}
\newcommand{\oL}{\bar L}
\newcommand{\vph}{\varphi}
\begin{document}
\title{Pseudodifferential and Fourier Integral Operators}
\author{Alex Himonas, {\it Summer 2009}}
\maketitle
\section{Introduction}
\nin
Let $ X $ be an open set in $ {{\Bbb R^n}} $ and
\begin{equation}
	\begin{split}
P = \sum_{|\alpha| \le m} a_\alpha (x) D^\alpha 
		\label{1}
	\end{split}
\end{equation}
be a linear partial differential operator (pdo) of order $ m $ on $ X$. 
For $ u \in C^\infty_0 (X) $ we have 
\begin{equation*}
	\begin{split}
Pu(x) &= \sum_{|\alpha| \le m} a_\alpha (x) D^\alpha u (x)\\
&= \sum_{|\alpha| \le m} a_\alpha (x) 
\mathcal{F}^{-1} \left ( \widehat{D^\alpha u} (\xi)
\right )  (x) \quad (\mathcal{F}^{-1} \; \text{is the inverse Fourier transform})\\
&= \sum_{|\alpha| \le m} a_\alpha (x)\mathcal{F}^{-1} \left ( \xi^\alpha \hat u (\xi)
\right ) (x)\\
&= \sum_{|\alpha| \le m} a_\alpha (x) \frac{1}{(2 \pi)^n} \int_{{\Bbb R^n}}
e^{i x \cdot \xi} \xi^\alpha \hat u (\xi) d \xi
\end{split}
\end{equation*}
or
\vskip-0.2in
\begin{equation}
	\begin{split}
		\label{2b}
		Pu(x) = \frac{1}{(2 \pi)^n} \int_{{\Bbb R^n}} e^{i x \xi} p(x, \xi) \hat u
(\xi)d\xi, 
	\end{split}
\end{equation}
where
\begin{equation}
	\begin{split}
p(x, \xi) = \sum_{|\alpha| \le m} a_\alpha (x) \xi^\alpha. 
		\label{3}
	\end{split}
\end{equation}
{\bf Idea:}    If we replace the polynomial in $ \xi$,   $p(x, \xi) $ with
a more general function of polynomial growth in $ \xi $ then the
operator $ P $ in \eqref{2b} is well defined and it is not any more a pdo.  More
precisely, if $ p(x, \xi) $ is in $ C^\infty (X \times {{\Bbb R^n}} ) $ with
the property that for any compact set $ K $ in $ X $ and
multi-indices $ \alpha, \beta \in {\Bbb N^n_0} $ there exists a
constant $ C_{K, \alpha, \beta} $ such that
\begin{equation}
	\begin{split}
		\left | \partial^\alpha_x \partial^\beta_\xi p(x, \xi) \right | \le
C_{K, \alpha, \beta} (1 + |\xi|)^{m - |\beta|}, \ x \in K, \ \xi \in {\Bbb
R^n},
		\label{4}
	\end{split}
\end{equation}
then the operator $ P $ in \eqref{2b} defines a map from $ C^\infty_0 (X) $
into $ C^\infty (X) $ which is called a {\bf pseudodifferential
operator ($\boldsymbol \psi$do)} of order $ m $  with symbol $ P$. 
The inverse of an elliptic pdo of order $ m $ is a $ \psi$do of order
$ -m$.  For example,  to solve the equation
$$(- \Delta + 1) u = f, \ f \in C^\infty_0 (X)$$
we take the Fourier transform to obtain
\begin{equation*}
(|\xi|^2 + 1) \hat u (\xi) = \hat f (\xi)
\end{equation*}
or
\begin{equation*}
	\hat u (\xi) = \frac{1}{|\xi|^2+ 1} \hat f (\xi), 
\end{equation*}
and then we take inverse Fourier transform to obtain
$$u(x) = Qf(x) = \frac{1}{(2 \pi)^n} \int_{{\Bbb R^n}} e^{i x \xi}
\frac{1}{\xi^2 + 1} \hat f (\xi) d \xi. $$
The operator $ Q $ is a $ \psi$do with symbol
$$q(x, \xi) = \frac{1}{|\xi|^2 + 1}$$
which satisfies \eqref{4} with $ m = -2$; i.e. $  Q $ is a $ \psi$do of order
$-2$.
\smallskip
\nin
Such  pseudodifferential operators appear in the writing 
of several non-linear equations. For example, the Camassa-Holm equation
$$
\partial_t u -\partial_x^2\partial_t u+ 3u\partial_x u
-2 \partial_x u\partial_x^2 u
-  u\partial_x^3 u
= 0, 
$$
can be written as
$$
\partial_t u + u\partial_x u
+
\partial_x
\left(
1 - \partial_x^2
\right)^{-1}
 \Big[
u^2 + \frac{1}{2}(\partial_x u)^2
\Big]
= 0.
$$
Also the Euler equations
\begin{equation*}
	\begin{split}
&\partial_{t}u + \nabla_{u}u +\nabla p =0\\
&\text{div}\, u = 0\\   
&u(0, x)= u_0(x), 
\quad 
x\in {{\Bbb R^n}}, \; t\in{\Bbb R}  
\end{split}
\end{equation*}
can be written in the form
%
$$
\partial_{t}u + \nabla_{u}u - \nabla  \Delta^{-1}  {\text{div}} \nabla_{u}u 
=
0.
$$
%
\smallskip
\nin
The operator $ P $ in \eqref{2b} can also be written in the form of the
following {\bf repeated} integral
\begin{equation}
	\begin{split}
		Pu(x) = \frac{1}{(2 \pi)^n} \int_{{\Bbb R^n}} \int_{{\Bbb R^n}} e^{i(x-y)\xi}
p(x, \xi) u(y) dy d \xi 
		\label{5}
	\end{split}
\end{equation}
where $ y $ is the integration is to  be performed first.  In general, this in
{\bf not} an absolutely convergent integral on $ {\Bbb R^n} \times \Bbb
R^n$.  Such integrals are called {\bf oscillatory} integrals.  The most
general form of a $ \psi$do is expressed by \eqref{5} where  $p(x, \xi) $ is
replaced by a function which also depends on $ y, p(x,y, \xi) $, and
satisfies an estimate similar to \eqref{4}.  Also, if the phase $ (x-y)
\cdot \xi $ is replaced by a more general phase function $ \vph (x,
y, \xi) $ which is positively homogeneous in $ \xi $ of degree 1; i.e.
$$\vph(x, y, t \xi) = t \vph (x, y , \xi), \ t > 0,$$ then $ P $ defines a
{\bf Fourier integral operator (FIO)}.
\smallskip
\nin
For defining $ \psi$do's and FIO's precisely, we need first to define
the symbols.
\section{Classes of Symbols}
\medskip
\nin
Let $ X $ be an open set $ {\Bbb R^n} $ and $ N \in \Bbb N $.
\medskip
\nin
\begin{definition}  For $ m \in \Bbb R $ the class of symbols $
S^m (X \times \Bbb R^N) $ is the space of all functions $ p(x, \xi)
$ in $ C^\infty (X \times \Bbb R^N) $ such that for every compact
set $ K $ in $ X, \alpha \in \Bbb N^n_0, \beta \in \Bbb N^N_0 $ there
exists a constant $ C_{K, \alpha, \beta} $ with
\begin{equation}
	\begin{split}
		\left | \partial^\alpha_x \partial^\beta_\xi p(x, \xi) \right | \le C_{K,
\alpha, \beta} (1 + |\xi|)^{m - |\beta |}, \ x \in K, \ \xi \in \Bbb R^N. 
		\label{1'}
	\end{split}
\end{equation}
For a fixed $ K, \alpha, \beta $ the quantity
\begin{equation*}
	\begin{split}
		\mathcal{N}_{K, \alpha, \beta} (p) = 
		\sup_{x \in K , \; \xi \in \rr^N}   \frac{\left |
\partial^\alpha_x \partial^\beta_\xi p(x, \xi) \right |}{(1 + |\xi|)^{m -
|\beta |}}
	\end{split}
\end{equation*}
defines a seminorm on the space $ S^m (X \times \Bbb R^N)$.  In fact, if
$ K_1 \subset K_2 \subset \cdots$ is an increasing family of compact
sets  in $ X $ with $ X = \bigcup^n_{j=1} K_j $  then the
countable family of seminorms 
$$N_{j, \alpha, \beta} = N_{K_j, \alpha, \beta}, \ j \in \Bbb N_0, \alpha
\in \Bbb N^n_0, \beta \in \Bbb N^N_0$$
\nin
makes $ S^m (X \times \Bbb R^N) $ a {\bf Fr\'echet} space.
\end{definition}
\vskip0.1in
\nin
{\bf Remark.} The classes  $S^m (X \times \Bbb R^N) $ were introduced by
Kohn and Nirenberg \cite{}.
 Most of the results we shall describe in these
lectures hold for more general classes of symbols.  For $ 0 \le \rho, \; \delta
\le 1$, and $ m \in \Bbb R$, such a class is defined by the inequality
\begin{equation}
	\begin{split}
		\left | \partial^\alpha_x \partial^\beta_\xi p(x, \xi) \right | \le C_{K,
\alpha, \beta} (1 + |\xi|)^{m-\rho |\beta| + \delta |\alpha|}, \ x \in K, \ \xi
\in \Bbb R^N. 
		\label{1''}
	\end{split}
\end{equation}
Such a symbol $ p $ is said to belong in the space of symbols of order $
m $ and type $ (\rho, \delta)$, denoted by $ S^m_{\rho, \delta} (X
\times \Bbb R^N)$.  Thus the symbols $ S^m (X \times \Bbb R^N) $ are
of type (1, 0). The  type $ (\rho, \delta)$ symbols were introduced by
H\"ormander \cite{}. More classes of symbols were introduced
by  R. Beals, C. Fefferman and others for studing various PDE problems.
\\
\\
{\bf Examples:}
\vskip0.1in
\nin
{\bf 1.}  For $ m \in \Bbb N $ and $ a_\alpha \in C^\infty(X) $ a
polynomial in $ \xi \in {\Bbb R^n} $ of the form
$$p(x, \xi) = \sum_{|\alpha| \le m} a_\alpha (x) \xi^\alpha$$
is a symbol in $ S^m (X \times {\Bbb R^n})$.  It can be considered as the
symbol of a pdo of order $ m $ on $ X$.
\vskip0.1in
\nin
{\bf 2.}  Let $ s \in \Bbb R$.  Then
$$\lambda_s (\xi) = (1 + |\xi|^2)^{s/2}, \ \xi \in {\Bbb R^n} $$
is a symbol in $ S^s({\Bbb R^n} \times {\Bbb R^n})$.  The operator defined
by
$$\widehat{\Lambda^s u} (\xi) = \lambda_s (\xi) \hat u (\xi) $$
is an isomorphism from $ H^s ({\Bbb R^n}) $ onto $ H^0 ({\Bbb R^n}) = L^2
({\Bbb R^n})$.  We have
$$\| \Lambda^su\| = \|u\|_s. $$
More generally,  for $ a(x) \in C^\infty (X) $ the function
$$a(x) (1 + |\xi|^2)^{\frac{s}{2}}, \ x \in X, \xi \in \Bbb R^N $$
defines a symbol in $ S^s (X \times \Bbb R^N)$.
\vskip0.1in
\nin
{\bf 3.}  For $ m \in \Bbb N $ the function
$$\lambda_{-2m} (\xi) = (1 + |\xi|^2)^{-m}$$
defines a symbol in $ S^{-2m} ({\Bbb R^n} \times {\Bbb R^n})$, which is the
symbol of the inverse of $ (1- \Delta)^m$.
\medskip
\nin
{\bf 4.}  %
Let $ a \in C^\infty(X \times \dot \rr^N) $ and positively
homogeneous of degree $ m \in \rr $ in $ \xi;.$ i.e.
\begin{equation*}
	a(x, t \xi) = t^m a(x, \xi), \ t > 0, \xi \in \dot \rr^N. 
\end{equation*}
Also let $ \vph \in C^\infty (\rr ) $ such that, $ 0 \le \vph \le 1$, and
\begin{equation}
	\begin{split}
		\vph (s) = \begin{cases}  0, \text{ if } s<1\\
	1, \text{ if } s>2.\end{cases} 
		\label{2'}
	\end{split}
\end{equation}
({\bf Sketch the graph of $\vph$})
\vskip 0.1in
\nin
Then the function
$$p(x, \xi) = \vph (|\xi|) a(x, \xi) $$
defines a symbol in $ S^m (X \times {\Bbb R^N})$.
\medskip
\nin
A special case of such a symbol is obtained in the following way.  For $
m \in \Bbb N $ let
$$q(x, \xi) = \sum_{|\alpha|=m} a_\alpha (x) \xi^\alpha, \ a_\alpha \in
C^\infty(X). $$
Assume that $ q(x, \xi) $ is elliptic, i.e.
$$q(x, \xi) \ne 0, \ x \in X, \ \xi \in \dot {{\Bbb R^n}}.$$
Then
$$a(x, \xi) = \frac{1}{q(x, \xi)}$$
is in $ C^\infty (X \times \dot {{\Bbb R^n}}) $ and homogeneous in $ \xi $ of
degree $ -m$.  The function
$$p(x, \xi) = \frac{\vph (|\xi|)}{q(x, \xi)}$$
defines a symbol in $ S^{-m} (X \times {\Bbb R^n})$.  This symbol
appears in the construction of an inverse for the elliptic pdo $ q(x, D)$.
\vskip0.1in
\nin
{\bf 5.}  Let $ m \in \Bbb N $ and
$$q(x, \xi) = \sum_{|\alpha| \le m} a_\alpha (x) \xi^\alpha, \ a_\alpha
\in C^\infty (X).$$
Assume that
$$q_m (x, \xi) = \sum_{|\alpha|=m} a_\alpha (x) \xi^\alpha \ne 0, \ x \in
X, \xi \in \dot {{\Bbb R^n}}, $$
i.e. $ q_m $ is elliptic in $ X$.
\medskip
\nin
If $ \Omega $ is an open subset of $ X $ with $ \bar \Omega $ compact,
then there exist constants $ C = C(\Omega) > 0 $ and $ R = R(\Omega)
> 0 $ such that
\begin{equation}
	\begin{split}
|q(x, \xi)| \ge C |\xi|^m, \ x \in \Omega, \ |\xi| \ge R. 
		\label{3'}
	\end{split}
\end{equation}
In fact, since $ \bar \Omega \times \{\xi \in \dot {{\Bbb R^n}}: \ |\xi| = 1\} $
is compact and $ q_m $ is continuous and elliptic, we have that there
exist $ C = C(\Omega) > 0 $ such that
$$\left |q_m \left (x, \frac{\xi}{|\xi|} \right ) \right | \ge 2C, \ x \in
\Omega, \ \xi \in \dot {{\Bbb R^n}}. $$
Therefore
$$|q_m (x, \xi)| \ge 2C |\xi|^m, \ x \in \Omega, \ \xi \in {\Bbb R^n}. $$
Now if $ x \in \Omega $ and $ |\xi| > 1 $ then
\begin{equation*}
	\begin{split}
|q(x, \xi)| &\ge |q_m (x, \xi)| -|\sum_{|\alpha| \le m-1} a_\alpha (x)
\xi^\alpha | \\
&\ge 2C |\xi|^m - C^\prime |\xi|^{m-1}\\
&= 2C |\xi|^{m-1} (|\xi| - C^\prime).
\end{split}
\end{equation*}
If we choose $ R $ large enough such that  $ |\xi| > R $ implies 
$|\xi|-C^\prime > \frac{|\xi|}{2} $ then from last relation
we obtain \eqref{3'}.  Now with $ \vph $ as in \eqref{2'} we let
$$p(x, \xi) = \frac{\vph( \frac{1}{2R} |\xi|)}{q (x, \xi)}.$$
{\bf Exercise:}  $ p(x, \xi) $ belongs to $ S^{-m} (\Omega \times \Bbb
R^n)$.
\vskip0.1in
\nin
{\bf 6.}  The function
$$p(x, \xi) = e^{i x \xi}$$
does not belong to any $ S^m ({\Bbb R^n} \times {\Bbb R^n})$.  However, it
does belongs to $ S^m_{(0,1)} ({\Bbb R^n} \times {\Bbb R^n})$.
\medskip
\nin
We shall use the following notation:
\begin{equation*}
	\begin{split}
		S^\infty(X \times \Bbb R^N) &\doteq \underset{m}{\cup} \, S^m (X \times \Bbb R^N)\\
		S^{- \infty} (X \times \Bbb R^N) &\doteq \underset{m}{\cap} \, S^m (X \times \Bbb R^N) 
\end{split}
\end{equation*}
and
\begin{equation*}
	\begin{split}
		S_c (X \times \Bbb R^N) = 
&\{p \in C^\infty (X \times
\Bbb R^N): \text{ for every compact }\\
& K \subset X \text{ there exists
} R > 0 \text{ such that }\\
&p(x, \xi) = 0 \text{ for } x \in K \text{ and }
|\xi| > R\}.
	\end{split}
\end{equation*}
Note that $ S^{- \infty} (X \times {\Bbb R^n}) $ is a Fr\'echet space and 
$$S_c (X \times \Bbb R^N) \subset S^{- \infty} (X \times \Bbb R^N). $$
Also if $ a(x) \in C^\infty(X) $ and $ \vph (\xi) \in S(\Bbb R^N) $ then
$$p(x, \xi) = a(x) \vph (\xi) \in S^{- \infty} (X \times \Bbb R^N). $$
Next we shall state the basic properties of the symbols.  For
convenience we shall write $ S^m $ instead of $ S^m (X \times \Bbb
R^N)$.  Also recall the notion of convergence in the Fr\'echet space $
S^m$:
$$p_j \to 0 \text{ in } S^m \text{ if } N_{K, \alpha, \beta} (p_j)
\underset{j \to \infty}\longrightarrow 0, \text{ for all } K, \alpha,
\beta. $$
{\bf Properties of Symbols.}  The following hold:
\vskip0.1in
\nin
{\bf 1.}  $ \partial^\alpha_x \partial^\beta_\xi $ is continuous from $
S^m $ into $ S^{m- |\beta|}$
\vskip0.1in
\nin
{\bf 2.}  If $ m^\prime \le m $ then $ S^{m^\prime} \subset S^m $ and the
inclusion map is continuous.  Moreover if $ m^\prime < m $ then the
inclusion map is compact.
\vskip0.1in
\nin
{\bf 3.}  For any $ m, k \in \Bbb R $ the product map $ (p,q) \longmapsto
pq $ is a bilinear continuous map from $ S^m \times S^k $ into $ S^{m +
k}$.
\vskip0.2in
\nin
The first property is easy to prove.  We shall prove only (2) and
(3). 
\vskip0.1in
\nin
{\bf Proof of 2:}  If $m'\le m$ then for all $K, \alpha, \beta $ fixed 
$$\mathcal{N}^m_{K, \alpha, \beta} (p) \le \mathcal{N}^{m^\prime}_{K, \alpha,
\beta} (p).$$
This shows the inclusion relation, and that the inclusion map is continuous.
\vskip0.1in
\nin
{\bf Compactness of the inclusion:}   Let $ m^\prime < m $ and $ \{p_j\} $ be a bounded
sequence in $ S^{m^\prime}$;  i.e. for every $ K, \alpha, \beta $ there
exists $ M = M(K, \alpha, \beta) $ such that
$$\mathcal{N}^{m^\prime}_{K, \alpha, \beta} (p_j) \le M, \ \forall j = 1, 2, 3,
\dots$$
This implies that $ \{p_j\} $ is a bounded sequence in $ C^\infty (X
\times \Bbb R^N)$, which is also a Fr\'echet space.  By applying the
Ascoli-Arzela theorem (any bounded set in $ C^\infty (X \times \Bbb
R^N) $ is relatively compact) we obtain a subsequence $ \{p_{j_k}\} $
such that $ p_{j_k} \longrightarrow p $ in $ C^\infty (X \times \Bbb
R^N)$.  To show that $ p_{j_k} \longrightarrow  p $ in $ S^m$, we let $ R
> 0 $ and write
$$\mathcal{N}^m_{K, \alpha, \beta} (p_{j_k} - p) \le A_k (R) + B_k (R)$$
where
$$A_k (R) = \substack{ \text{ sup }\\
x \in K\\
|\xi| \le R} \ \frac{ \left |\partial^\alpha_x \partial^\beta_\xi
(p_{j_k} - p) \right |}{(1 + |\xi|)^{m-|\beta|}}\ , \  B_k (R) = \substack{\text{ sup
}\\ x \in K\\
|\xi| \ge R } \frac{\left | \partial^\alpha_x \partial^\beta_\xi
(p_{j_k} - p) \right |}{(1 + |\xi|)^{m-|\beta|}}.$$
Since
$$B_k (R) \le \frac{1}{(1 + R)^{m-m^\prime}} \left [ \mathcal
N^{m^\prime}_{K, \alpha, \beta} (p_{j_k} - p) \right ] \le  \frac{  2M  }{(1
+ R)^{m-m^\prime}}$$ 
we obtain that $ B_k (R) \longrightarrow 0$, as $ R \to
\infty$, uniformly on $ k$.  On the other hand, since $ p_{j_k}
\longrightarrow p $ in $ C^\infty(X \times \Bbb R^N) $ we obtain that $
A_k (R) \longrightarrow 0 $ as $ k \to \infty $ for any fixed $ R > 0$. 
Therefore given $ \ve > 0 $ we can choose $ R > 0 $ and large enough such that
$ \mathcal N^m_{K, \alpha, \beta} (p_{j_k} - p) < \ve $ for all $ k \ge k_0 $ for
some $ k_0$.  This completes the proof of (2).
\vskip0.1in
\nin
{\bf Proof of 3:}   By  Leibniz's formula we have
$$\partial^\alpha_x \partial^\beta_\xi (pq) = \sum_{\gamma \le
\beta} c_\gamma \sum_{\delta \le \alpha} c_\delta \partial^{\alpha -
\delta}_x \left ( \partial^{\beta - \gamma}_\xi p \right )
\partial^\delta_x \left ( \partial^\gamma_\xi q \right ). $$
Therefore we obtain that
$$\mathcal N^{m+k}_{K, \alpha, \beta} (pq) \le \sum_{\gamma \le \beta}
\sum_{\delta \le \alpha} c_\gamma c_\delta \mathcal N^m_{K, \alpha -
\delta, \beta - \gamma} (p) \mathcal N^k_{K, \delta, \gamma} (q). $$
This inequality implies (3).  
\vskip0.2in
\nin
{\bf Remark:}  It follows from the proof of the compactness of the
inclusion map from $ S^{m^\prime} \longrightarrow S^m, \ m^\prime < m$,
that on bounded sets of $ S^{m^\prime} $ the topologies of $ S^m (X \times
\dot {\Bbb R^N}), C^\infty (X \times {\Bbb R^n})$, and pointwise
convergence coincide.
\medskip
\nin
Next we shall show that $ S_c$, and therefore $ S^{- \infty}$, is dense
in any $ S^m$ in the topology of  $ S^{m^\prime}, \ m^\prime > m$.  More
precisely we have:
\medskip
\nin
\begin{proposition}
	\label{prop1}
Let $ \psi \in C^\infty_0 (\Bbb R^N), \ 0 < \psi <
1, $ with $ \psi (\xi) = 1 $ for $ |\xi| \le 1 $ and $ \psi (\xi) = 0 $ for $ |\xi|
\ge 2$.  
Then
\vskip0.1in
\nin
{\bf 1.}  $ \psi_j (\xi) = \psi (\xi/j) \in S_c $ is
bounded in $ S^0 (X \times \Bbb R^{N})$.
\vskip0.2in
\nin
{\bf 2.}  For any $ p \in S^m $ the sequence
$$p_j (x, \xi) = \psi_j (\xi) p(x, \xi), \ j = 1, 2, 3, \dots$$
converges to $ p $ in $ S^{m^\prime} $ for any $ m^\prime > m$.
\end{proposition}
{\bf Proof of 1.}   Since $ \psi_j $ is independent of $ x $ it suffices to
estimate the derivatives with respect to $ \xi$.  We have
$$\partial^\beta_\xi \psi_j (\xi) = j^{- |\beta|} \partial^\beta \psi (\xi/j). $$
For $ \beta = 0 $ we have $ N^0_\beta (\psi_j) = \text{sup } \psi (\xi/j)  \le 1$. 
Let $ \beta \neq 0$. Since $ \partial^\beta \psi (\xi/j)= 0 $ unless $ j
\le |\xi| \le 2j $ we obtain that
\begin{equation*}
	\begin{split}
N^0_\beta (\psi_j) &\le \underset j \le |\xi| \le 2j\to{\text{ sup  }}
j^{- |\beta|} \frac{\left | \partial^\beta \psi (\xi/j) \right |}{(1 + |\xi|)^{-
|\beta|}}\\
 &\le \frac{(1 + 2j)^{|\beta|}}{j^{|\beta|} }\
\ \|\partial^\beta \psi \|_\infty 
\le 3^{|\beta|} \|\partial^\beta \psi\|_\infty  \ ,
\end{split}
\end{equation*}
which implies that $ \psi_j \in S^0$.
\vskip0.2in
\nin
{\bf Proof of 2. }  Since $ \psi_j $ is bounded in $ S^0 $ it follows that $ \psi_j p $
is bounded in $ S^m$.  Since also $ p_j = \psi_j p \longrightarrow p $ in
$ C^\infty (X \times \Bbb R^N) $ by the last remark $ p_j
\longrightarrow p $ in $ S^{m^\prime} $ for any $ m^\prime > m$.  This
completes the proof of the proposition.
\vskip0.2in
\nin
{\bf Remark:  } The sequence of symbols $ p_j = \psi_j p \in S^m $ may
not converge in the topology of $ S^m$.  To see this let $ p = 1 $ and
then $ p_j = \psi_j $ which does not converge in $ S^0 $ since
$$\sup_{\xi} |\psi_j (\xi) - 1| = 1
\nrightarrow 0 \text{ as } j \to \infty. $$
Therefore the condition $ m^\prime > m $ in (2)  is necessary.
Next we shall describe a realization of a formal sum
$$\sum^\infty_{j=0} p_j, \text{ where } p_j \in S^{m_j} \text{ with } m_j
\searrow - \infty.$$
\begin{proposition}(Hormander) 
	\label{prop2}
	Let $ p_j (x, \xi) \in S^{m_j}
(X \times \Bbb R^N) $ with $ m_0 > m_1> \dots$, and $ m_j
\longrightarrow - \infty $ as $ j \to \infty$.  Then there exists $ p \in
S^{m_0} (X \times \Bbb R^N) $ such that for every $ J \in \Bbb N $
\begin{equation}
	\begin{split}
		(p - \sum^{J-1}_{j=0} p_j) \in S^{m_J} (X \times {\Bbb R^n}). 
		\label{1'''}
	\end{split}
\end{equation}
Moreover $ p $ is unique modulo $ S^{- \infty} (X \times {\Bbb R^n}).$
\end{proposition}
\vskip0.1in
\nin
{\bf Proof.}  First we show uniqueness.  If $ p' $ is another symbol
satisfying (1) then for any $ J $
$$p - p^\prime = \left ( p - \sum^{J-1}_{j=0} p_j\right)  - \left (
p^\prime - \sum^{J-1}_{j=0} p_j \right ) \in S^{m_J}. $$
 Since $ m_J \to - \infty $ as
$ J \longrightarrow \infty $ we obtain that $ p - p^\prime \in S^{-
\infty}$.  Next we construct $ p$.  Let $ \vph = 1 - \psi $ where $ \psi $
is as in Proposition 1 ; i.e. $ \vph \in C^\infty (\Bbb R^N) $ such that $ \vph
(\xi) = 0 $ if $ |\xi| \le 1 $ and $ \vph (\xi) = 1 $ if $ |\xi | > 2 $.
Also let $ 1 < t_0 < t_1 < \dots < t_j \longrightarrow \infty $ a
sequence of number $ t_j $ to be determined later.  Then we define $
p(x, \xi)$ by
\begin{equation}
	\begin{split}
		p(x, \xi) \doteq \sum^\infty_{j=0} \vph ( \frac{\xi}{t_j} ) p_j (x, \xi).
		\label{2'''}
	\end{split}
\end{equation}
For $ \xi $ in a compact set of $ \Bbb R^N $ the series in \eqref{2'''} is a finite
sum.  Therefore $ p(x, \xi) $ is in $ C^\infty (X \times \Bbb R^N)$. Now
let $ \{K_j\} $ be an increasing sequence of compact sets in $ X $ with $
U^\infty_{j=0} K_j = X$.  We shall show that for any $ j \in \Bbb N $
there exists $ t_j \nearrow \infty $ such that
\begin{equation}
	\begin{split}
		\mathcal N^{m_{j-1}}_{K_j, \alpha, \beta} \left (\vph (\frac{\xi}{t_j}) p_j (x,
\xi)\right ) \le \frac{1}{2^j}, \quad  |\alpha| + |\beta| \le j. 
		\label{3'''}
	\end{split}
\end{equation}
In fact, by Leibniz's rule we have
$$\partial^\alpha_x \partial^\beta_\xi [\vph (\frac{\xi}{t_j}) p_j (x,
\xi)] = \sum_{\beta^\prime + \beta^{\prime \prime} = \beta}
C_{\beta^\prime, \beta^{\prime \prime}} \partial^{\beta^\prime}_\xi
[\vph (\frac{\xi}{t_j})] \partial^\alpha_x
\partial^{\beta^{\prime\prime}}_\xi p_j (x, \xi) $$
Since by Proposition 1 $ \vph (\frac{\xi}{t_j}) $ is bounded,  $ \vph
(\frac{\xi}{t_j}) = 0 $ for  $ |\xi| \le t_j$, and  $ p_j \in S^{m_j}, $ we obtain
that for $ x \in K_j $ 
$$\frac{\big |\partial^\alpha_x \partial^\beta_\xi[
\vph(\frac{\xi}{t_j})p_j(x, \xi)] \big|}{(1 + |\xi|)^{m_{j-1} - |\beta|}}
\le C_{j, \alpha, \beta} \frac{(1 + |\xi|)^{m_j - |\beta|}}{(1 +
|\xi|)^{m_{j-1}- |\beta|}} \quad   \mathcal X_{\{|\xi| \ge t_j\}} (\xi) $$
where $ \mathcal X_{\{|\xi| \ge t_j\}} $ is the characteristic function of the
set $ \{ \xi \in \Bbb R^N: |\xi| \ge t_j\}$.  Then the last relation implies
that
 $$\mathcal N_{K_j, \alpha, \beta}^{m_{j-1}} \le C_{j, \alpha, \beta}
\frac{1}{(1 + t_j)^{m_{j-1} -m_j}}. $$
Since $ m_{j-1} - m_j > 0 $ we can choose $ t_j $ large enough so that
\eqref{3'''} holds.  
\smallskip
\nin
Now we shall use estimate \eqref{3'''} to prove \eqref{1'''}.  For any $ J \in
\Bbb N $ fixed, we have
$$p(x, \xi) - \sum^J_{j=0} p_j(x, \xi) = \sum^{J-1}_{j=1} [\vph
(\frac{\xi}{t_j}) -1] p_j (x, \xi) + r_J (x, \xi),$$ 
where
$$r_J (x, \xi) = \sum^\infty_{j=J} \vph  
(\xi / t_j)  p_j (x, \xi). $$
Since $ \vph (\xi/t_j) -1 =0 $ for $ |\xi| \ge 2 t_j $ we obtain that
the term $ \sum^{J-1}_{j=0} [\vph (\xi/t_j) -1] p_j (x, \xi) $
is in $ S_c \subset S^{m_J}$.  It remains to show that
 $ r_J (x, \xi)$ is  in $S^{m_J}$.  
For this let $ K $ be
a compact set in $ X $ and $ \alpha, \beta $ two multi-indices.  We must
show that  
\begin{equation}
	\begin{split}
\mathcal N_{K, \alpha, \beta}^{m_J}  (r_J) < \infty 
		\label{4'''}
	\end{split}
\end{equation}
To prove \eqref{4'''} we choose $ \ell \in \Bbb N $
such that 
\begin{equation}
	\begin{split}
		J + 1 < \ell,\quad K \subset K_\ell, \quad\text{ and } 
\quad |\alpha| + |\beta| \le\ell, 
		\label{5'''}
	\end{split}
\end{equation}
and we write the remainder $ r_J$ in 
the form
 $$r_J (x, \xi) = \sum^{\ell - 1}_{j=J}
\vph (\frac{\xi}{t_j}) p_j (x, \xi) +r_\ell(x, \xi),$$ 
where
$$r_\ell(x, \xi)  = \sum^\infty_{j=\ell}
\vph (\frac{\xi}{t_j}) p_j (x, \xi).$$
Since $ r_J-r_\ell \in S^{m_J} $ it suffices to show \eqref{4'''} with $r_J$ replaced
with $ r_\ell$.  Since by \eqref{5'''} we have that $
|\alpha| + |\beta| \le j $ for all $ j \ge \ell $ we use inequality
\eqref{3'''} to obtain   
\begin{equation*}
	\begin{split}
		\mathcal N^{m_J}_{K, \alpha,
\beta} (r_\ell)  
&\le \sum^\infty_{j=\ell} \mathcal N^{m_J}_{K_j, \alpha, \beta} 
\left (\vph  (\xi/t_j) p_j (x, \xi)\right )\\
 &\le \sum^\infty_{j= \ell} \mathcal N^{m_{j-1}}_{K_j, \alpha, \beta} 
 \left (\vph (\xi/t_j) p_j (x, \xi)\right )\\ 
 & \overset{(3)} \le \sum^\infty_{j =\ell} \frac{1}{2^j} < \infty.
\end{split}
\end{equation*}
This completes the proof of Proposition \ref{prop2}. 
\vskip0.2in
\nin
{\bf Remark:}  A less constructive but shorter proof is given by using
the density of $ S^{-\infty } $ in any $ S^m $ for the topology of any $
S^{m+\ve} $ proved in Proposition \ref{prop1}.  More precisely if for fixed $ j $
we denote by $ \mathcal N_{j,0}, N_{j,1}, N_{j,3}, \dots $ a sequence of
seminorms defining  the topology of $ S^{m_j} $ then by Proposition
\ref{prop1} for any $ j $ there exists $ q_j \in S^{- \infty} $ such that
$$\mathcal N_{k, \ell} (p_j - q_j) < \frac{1}{2^j}, \quad  k, \ell < j-1.$$
Then for any $ J $ the series
$$\sum^\infty_{j=J} (p_j - q_j)$$
converges in $ S^{m_J}$.  If we let
$$p = \sum^\infty_{j=0} (p_j - q_j) $$
then $ p \in S^{m_0} $ and
$$p - \sum^{J-1}_{j=0} p_j = - \sum^{J-1}_{j=0} q_j +
\sum^\infty_{j=J} (p_j - q_j) \in S^{- \infty} + S^{m_J} \subset
S^{m_J}.$$
Next we give a more general statement of Proposition \ref{prop2}.
\begin{corollary}  Let $ p_j \in S^{m_j} $ with $ m_j \longrightarrow
- \infty $ as $ j \to \infty$.  Then there exists $ p \in S^{\bar m_0}$,
with $ \bar m_0 = \underset{j \ge 0} \max \; m_j $ such that for
any $ J \in \Bbb N $
$$p(x, \xi) - \sum^{J-1}_{j=0} p_j (x, \xi) \in S^{\bar m_J},$$
where $ \bar m_J = \underset{j \ge J} \max \; m_j.$
\end{corollary}
\medskip
\nin
{\bf Proof.}  It follows by the last proposition applied to the
sequence $ q_j $ defined by
$$q_j = \sum_{p_k \in S^{\bar m_j}} p_k. $$
\begin{definition}  Let $ p_j \in S^{m_j} $ with $ m_j \longrightarrow
- \infty$.  A function $ p(x, \xi) \in C^\infty (X \times {\Bbb R^n}) $ is
called \lq\lq the\rq\rq \   {\bf asymptotic sum}  of $ p_j $ if for any $ J
\in \Bbb N $ we have
$$\left ( p - \sum^{J-1}_{j=0} p_j \right ) \in S^{\bar m_J}, $$
where $ \bar m_J = \underset{j \ge J} \max \; m_j$.  Then $ p $
must be in $ S^{\bar m_0}$.
\end{definition}
\vskip0.1in
\nin
Proposition \ref{prop1} is similar to Borel theorem where a $ C^\infty $ function
is constructed with a given formal power series.  More precisely we
have the theorem.
\medskip
\begin{theorem}(Borel)  If $ \{a_j\} $ is a sequence of complex
numbers then there exists a function $ f $ in $ C^\infty (\Bbb R) $ such
that $ f^{(j)} (0) = a_j$.
\end{theorem}
\vskip0.1in
\nin
{\bf Proof}  First we note that such a function $ f $ is not unique
since $ f(x) + h(x) $ with $ h^{(j)} (0) = 0 $ (say $ h(x) = e^{-
\frac{1}{x^2}}) $  is another function with the same derivatives at 0.  To
construct $ f(x) $ we choose $ \vph \in C^\infty_0 (\Bbb R), \ 0 \le \vph \le
1, \ \vph (x) = 1, \ |x| < \frac{1}{2} $ and $ \vph (x) = 0, \ |x| > 1$.
If $ \{t_j\} $ is sequence of positive number with $ t_j \longrightarrow
\infty $ (to be chosen later) then the formula
$$f(x) = \sum^\infty_{j=0} \frac{a_j}{j!} x^j \vph (t_j x) $$
gives a finite number for any fixed $ x \in \Bbb R $ since then it
becomes a finite sum.  Next we shall choose $ t_j $ appropriately such
that $ f \in C^\infty (\Bbb R)$.  Since $ \vph (t_j x) = 0 $ unless $ |x| <
\frac{1}{t_j} $ we have that
$$|f(x)| \le \sum^\infty_{j=0} \frac{|a_j|}{j!t_j^j}. $$
Therefore the series converges uniformly in $ \Bbb R $ if 
\begin{equation}
	\begin{split}
		\sum^\infty_{j=1} \frac{|a_j| }{j!}
 \frac{1}{(t_j)^j} <\infty.
		\label{0a}
	\end{split}
\end{equation}
By differentiating formally we obtain
$$
f'(x) = \sum^\infty_{j=1} \frac{a_j}{(j-1)!} x^{j-1} \vph(t_jx) +
\sum^\infty_{j=0} \frac{a_j}{j!} x^j  t_j \vph '(t_jx),
$$
and  then we have
$$
|f'(x)|\le \sum^\infty_{j=1} \frac{|a_j|}{(j-1)!} \frac{1}{t_j^{j-1}} +
\sum^\infty_{j=0} \frac{|a_j|}{j!} \frac{1}{t_j^{j-1}} \|\vph'\|_\infty
$$
In order that $f'(x)$ converges uniformly we should choose $t_j$ such that
$$\sum^\infty_{j=1} \frac{|a_j|}{(j-1)!} \frac{1}{t_j^{j-1}} +
 \sum^\infty_{j=0} \frac{|a_j|}{j!} \frac{1}{t_j^{j-1}} < \infty.$$
\smallskip
\nin
But if the first series converges then also the second
 series converges, so it suffices that
 \begin{equation}
	 \begin{split}
		  \sum^\infty_{j=1} \frac{|a_j|}{(j-1)!}
\frac{1}{t_j^{j-1}} < \infty.
		 \label{1a}
	 \end{split}
 \end{equation}
By differentiating again we obtain
$$
f''(x) = \sum^\infty_{j=2} \frac{a_j}{(j-2)!} x^{j-2} \vph
(t_jx)+2\sum^\infty_{j=1} \frac{a_j}{(j-1)!} x^{j-1}t_j
\vph '(t_jx) +
\sum^\infty_{j=0} \frac{a_j}{j!} x^j t^2_j \vph''(t_jx). $$
Again for uniform convergence of $f''(x)$  we should have
$$\sum^\infty_{j=2} \frac{|a_j|}{(j-2)!} \frac{1}{t_j^{j-2}} +
\sum^\infty_{j=1} \frac{|a_j|}{(j-1)!} \frac{1}{t_j^{j-2}} \|
\vph'\|_\infty + \sum^\infty_{j=0} \frac{|a_j|}{j!} \frac{1}{t^{j-2}_j}\|
\vph'' \|_\infty< \infty$$
%
But the convergence of the first series implies the convergence of the others.
 So  it suffices that
 %
 \begin{equation}
	 \begin{split}
		 \sum^\infty_{j=2} \frac{|a_j|}{(j-2)!}
\frac{1}{t_j^{j-2}} < \infty. 
		 \label{2a}
	 \end{split}
 \end{equation}
%
Continuing to take derivatives we find that
\begin{equation}
	\begin{split}
		 \sum^\infty_{j=k}\frac{|a_j|}{(j-k)!}
 \frac{1}{t_j^{j-k}}<\infty
 \quad  k = 0,1,\dots,k
		\label{3a}
	\end{split}
\end{equation}
%
 is a sufficient condition for $f^{(k)}(x)$ to converge uniformly.
\medskip
\nin
Now we choose 
$$t_j \ge 1+|a_j| \text{ and }t_j \longrightarrow \infty.$$  Then
by \eqref{3a} we obtain 
 \begin{equation*}
	 \begin{split}
 \sum^\infty_{j=k} \frac{|a_j|}{(j-k)!}
\frac{1}{(1+|a_j|)^{j-k}} &\le \sum^\infty_{j=k} \frac{1}{(j-k)!}
\frac{1+|a_j|}{(1+|a_j|)^{j-k}}\\
&=1+|a_j| + \sum^\infty_{j=k+1} \frac{1}{(j-k)!} \frac{1}{(1+|a_j|)^{j-k-1}}\\
&\le 1+|a_j| +\sum^\infty_{j=k+1}\frac{1}{(j-k)!} <\infty,
\end{split}
\end{equation*}
which completes the proof of the Theorem.
\vskip0.2in
\nin
{\bf Remark:}
If $ x $ is such that 
$ |t_jx| \le \frac{1}{2},\,  j=1,\dots,N,$
 or  $ |x| \le\frac{1}{2t_j}$ or   
$$ |x| \le \frac{1}{2}(\text{max}\{t_1,\dots,t_N\})^{-1}=\ve_N,$$
then
\begin{equation*}
	\begin{split}
f(x) &= \sum^N_{j=0} \frac{a_j}{j!} x^j + \sum^\infty_{j=N+1}\frac{a_j}{j!}x^j
\vph(t_jx)\\ 
&= \sum^N_{j=0} \frac{a_j}{j!} x^j + |x|^{N+1}\cdot(\text{ bounded
function}),
\end{split}
\end{equation*}
or for $|x| \le \ve_N$ we get
$$f(x)-\sum^N_{j=0} \frac{a_j}{j!} x^j = O (|x|^{N+1}).$$
\end{document}
