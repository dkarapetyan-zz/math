\documentclass[12pt,reqno]{amsart}
\usepackage{amscd}
\usepackage{amsfonts}
\usepackage{amsmath}
\usepackage{amssymb}
\usepackage{amsthm}
\usepackage{appendix}
\usepackage{fancyhdr}
\usepackage{latexsym}
\usepackage{pdfsync}
\usepackage{cancel}
\usepackage{amsxtra}
\usepackage[colorlinks=true, pdfstartview=fitv, linkcolor=blue,
citecolor=blue, urlcolor=blue]{hyperref}
\input epsf
\input texdraw
\input txdtools.tex
\input xy
\xyoption{all}
%%%%%%%%%%%%%%%%%%%%%%
\usepackage{color}
\definecolor{red}{rgb}{1.00, 0.00, 0.00}
\definecolor{darkgreen}{rgb}{0.00, 1.00, 0.00}
\definecolor{blue}{rgb}{0.00, 0.00, 1.00}
\definecolor{cyan}{rgb}{0.00, 1.00, 1.00}
\definecolor{magenta}{rgb}{1.00, 0.00, 1.00}
\definecolor{deepskyblue}{rgb}{0.00, 0.75, 1.00}
\definecolor{darkgreen}{rgb}{0.00, 0.39, 0.00}
\definecolor{springgreen}{rgb}{0.00, 1.00, 0.50}
\definecolor{darkorange}{rgb}{1.00, 0.55, 0.00}
\definecolor{orangered}{rgb}{1.00, 0.27, 0.00}
\definecolor{deeppink}{rgb}{1.00, 0.08, 0.57}
\definecolor{darkviolet}{rgb}{0.58, 0.00, 0.82}
\definecolor{saddlebrown}{rgb}{0.54, 0.27, 0.07}
\definecolor{black}{rgb}{0.00, 0.00, 0.00}
\definecolor{dark-magenta}{rgb}{.5,0,.5}
\definecolor{myblack}{rgb}{0,0,0}
\definecolor{darkgray}{gray}{0.5}
\definecolor{lightgray}{gray}{0.75}
%%%%%%%%%%%%%%%%%%%%%%
%%%%%%%%%%%%%%%%%%%%%%%%%%%%
%  for importing pictures  %
%%%%%%%%%%%%%%%%%%%%%%%%%%%%
\usepackage[pdftex]{graphicx}
\usepackage{epstopdf}
% \usepackage{graphicx}
%% page setup %%
\setlength{\textheight}{20.8truecm}
\setlength{\textwidth}{14.8truecm}
\marginparwidth  0truecm
\oddsidemargin   01truecm
\evensidemargin  01truecm
\marginparsep    0truecm
\renewcommand{\baselinestretch}{1.1}
%% new commands %%
\newcommand{\bigno}{\bigskip\noindent}
\newcommand{\ds}{\displaystyle}
\newcommand{\medno}{\vskip0.1in\noindent}
\newcommand{\smallno}{\vskip0.1in\noindent}
\newcommand{\ts}{\textstyle}
\newcommand{\rr}{\mathbb{R}}
\newcommand{\p}{\partial}
\newcommand{\zz}{\mathbb{Z}}
\newcommand{\cc}{\mathbb{C}}
\newcommand{\ci}{\mathbb{T}}
\newcommand{\ee}{\varepsilon}
\def\refer #1\par{\noindent\hangindent=\parindent\hangafter=1 #1\par}
%% equation numbers %%
\renewcommand{\theequation}{\thesection.\arabic{equation}}
%% new environments %%
%\swapnumbers
\theoremstyle{plain}  % default
\newtheorem{theorem}{Theorem}
\newtheorem{proposition}{Proposition}
\newtheorem{lemma}{Lemma}
\newtheorem{corollary}{Corollary}
\newtheorem{conjecture}[subsection]{conjecture}
\theoremstyle{definition}
\newtheorem{definition}{Definition}
\newcommand{\ve}{\varepsilon} 
\newcommand{\nin}{\noindent}
\newcommand{\oL}{\bar L}
\newcommand{\vph}{\varphi}
\begin{document}
\title{Pseudodifferential and Fourier Integral Operators}
\author{Alex Himonas, {\it Summer 2009}}
\maketitle
\setcounter{section}{4}
\section{Asymptotic Expansion Formula (AEF)}
Let $ X $ be an open set in $ \Bbb R^n $ and 
$$P\in \Psi^m(X) \quad\text {with symbol}
 \quad p(x, y, \xi) \in S^m(X\times X\times \Bbb R^N).$$ 
We want to
study the quantity \begin{equation*}
	\begin{split}
b(x, \theta) &= e^{-i \psi(x, \theta)} P(a(\cdot, \theta) e^{i \psi (\cdot,
\theta)}) (x)\\
&= e^{-i \psi(x, \theta)} \frac{1}{( 2 \pi)^n} \int_{\xi \in \Bbb R^N} \int_{y \in X}
e^{i(x-y)\xi} p(x, y, \xi) a(y, \theta) e^{i \psi (y, \theta)} dy d\xi,
\end{split}
\end{equation*}
where $ \psi \in C^\infty (X \times \dot{\rr}^N)$, real-valued,
positively homogeneous in $ \theta $ of degree 1, with
$$\psi^\prime_x (x,\theta) \ne 0 \text{ if } (x, \theta) \in \text{ conic supp } a, \
\theta \ne 0$$ 
and
$$ a(y, \theta) \in S^q (X \times \Bbb R^N),$$
  and we shall
assume that $ P $ is {\bf  properly supported}
 (therefore it maps  $ C^\infty (X)
\longrightarrow C^\infty (X) $ continuously)
 or that 
$$\text{supp } a(\cdot, \theta)\;\; \text{is
compact}. $$ 
Then we have the following theorem: 
\begin{theorem}  If the above assumptions hold,  then:
	\begin{equation}
		\label{4}
		b(x, \theta) \in S^{m+q} (X\times \Bbb R^N),
	\end{equation}
	\begin{equation}
		\label{5}
		b(x, \theta) \sim \sum_{\alpha, \beta} \frac{1}{\alpha ! \beta!} \left (
\partial^{\alpha + \beta}_\xi D^\beta_y p \right ) (x, x, \psi^\prime (x, \theta) )
D^\alpha_y (a(y, \theta) e^{i r(x, y, \theta)} ) \big |_{y=x}  
\end{equation}
where $r(x, y, \theta) = \psi (y, \theta) - \psi (x, \theta) - <y-x, \psi^\prime_x
(x, \theta) >,$
\smallskip
\nin 
and
\begin{equation}
	\label{6}
	b(x, \theta) - p(x, x, \psi^\prime (x, \theta)) a(x, \theta) \in S^{m+q-1} (X
\times \Bbb R^N). 
\end{equation}
\end{theorem}
\nin
{\bf Proof.} As a first step we shall prove the theorem in the very special
case where
$$P = Q \cdot g$$
where $ Q  $ is a differential operator in $ X $ of order $ m $ and $ g \in C^\infty
(X)$.  Then
$$p(x, y, \xi) = Q(x, \xi) g(y)$$
is a symbol of $ P$.  In this case we have
\begin{equation*}
	\begin{split}
b(x, \theta) &= e^{-i \psi(x, \theta)} \left [ Q(y, D) \left ( g(y) a(y, \theta)
e^{i \psi (y, \theta)} \right ) \big |_{y=x} \right ]\\
&\substack{\text{Leibniz}\\
=\\
\text{ H\"ormander}}e^{-i \psi (x, \theta)} \sum_\beta \frac{1}{\beta!}
\left ( \partial^\beta_\xi Q \right ) (y, D) \left ( a(y, \theta) e^{i \psi (y, 
\theta)} \right  ) D^\beta_y g(y) \big |_{y=x}\\
&= \sum_\beta \frac{1}{\beta!}
\left ( \partial^\beta_\xi Q \right ) (y, D) \left ( e^{i < y - x, \psi^\prime_x (x,
\theta)>} a(y, \theta) e^{ir(x, y, \theta)} \right ) D^\beta_y g(y) \big
|_{y=x}\\
&= \sum_{\alpha, \beta} \frac{1}{\alpha! \beta!} \left ( \partial^{\beta +
\alpha}_\xi Q\right )(x, \psi^\prime_x (x, \theta))\  D^\alpha_y \left ( a(y,
\theta) e^{ir(x, y, \theta)} \right ) D^\beta_y g(y) \big |_{y=x}\end{split}
\end{equation*}
or
\begin{equation}
	\label{7}
	b(x, \theta) = \sum_{\alpha, \beta} \frac{1}{\alpha! \beta!} \left (
\partial^{\beta + \alpha}_\xi D^\beta_y Qg \right ) (x, x, \psi^\prime_x (x,
\theta) ) D^\alpha_y \left ( a(y, \theta) e^{ir(x, y, \theta)} \right ) \big |_{y=x}
,
\end{equation}
which is the desired formula when $p(x, y, \xi) = Q(x, \xi) g(y). \qquad
\Box$
\vskip0.1in
\nin
{\bf The general case:}  We write $ b(x, \theta) $ in the form
\begin{equation}
	\label{8}
	b(x, \theta) = \frac{1}{(2 \pi)^n} \int \int e^{i[< x-y, \xi> + \psi (y, \theta) -
\psi (x, \theta)]} \ p(x, y, \xi) a(y, \theta) dy d \xi. 
\end{equation}
We shall assume that $ a(y, \theta) $ is compactly supported in $ y $ (since the
case that $ P $ is properly supported is similar), i.e. 
$$ a(y, \theta) = 0 \text{  if } y \notin K \text{ for some compact set in } X.$$
 Since
the unit sphere in $ \Bbb R^N $ is compact it suffices to prove the theorem
locally in $ (x, \theta)$.  Therefore we shall assume that we are given a point $ (
x_0, \theta_0) \in X \times \Bbb R^N, \ |\theta_0| = 1$, and that $ (x, \theta) $ is
in a conic neighborhood of $ (x_0, \theta_0) $ in $ X \times \Bbb R^N$.  We start
the proof by writing $ \theta $ in polar coordinates; i.e.
$$\theta = \lambda \omega, \ \lambda = |\theta| \text{ and } \omega \in
S^{N-1}.$$
Then we let
$$\xi = \lambda \eta, $$
and $ b(x, \theta) $ takes the form
\begin{equation}
	\label{9}
	b(x, \omega, \lambda) = \left (\frac{\lambda}{2 \pi} \right )^n \int_{\eta \in
\Bbb R^n} \int_{y \in K} e^{ i \lambda \vph (x, \omega, y, \eta)} p(x, y, \lambda
\eta) a(y, \lambda \omega) dy d\eta, 
\end{equation}
where
\begin{equation}
	\label{10}
	\vph(x, \omega, y, \eta) = <x-y, \eta > + \psi (y, \omega) - \psi (x, \omega).
\end{equation}
Since $ \psi^\prime_x (x, \theta)  \ne 0 $ on (conic \ supp $a) \cap
(X \times S^{N-1}) $ there exists $ r > 0 $ such that
\begin{equation}
	\label{11}
	| \psi^\prime_y (y, \omega) | \ge r \text{ for all } y \in K, \omega \text{ near
} \omega_0. 
\end{equation}
By \eqref{10} and \eqref{11} we obtain that
\begin{equation}
	\label{12}
	|\vph^\prime_y (x, \omega, y, \eta)| \ge \frac{r}{2} > 0 \text{ if } |\eta| <
\frac{r}{2}, \ y \in K. 
\end{equation}
Now we choose a cut-off function $ \chi \in C^\infty_0 (\Bbb R^n) $ such that
\begin{equation}
	\begin{split}
\chi(\eta) &= 1 \text { if } |\eta | < \frac{r}{4}\\
&= 0 \text{ if } |\eta| > \frac{r}{2} \end{split}
\end{equation}
Then we write
\begin{equation}
	\label{13}
	b(x, \omega, \lambda) = I(x, \omega, \lambda) + R(x, \omega, \lambda),
\end{equation}
where
\begin{equation}
	\label{14}I(x, \omega, \lambda) =  \left ( \frac{\lambda}{2 \pi} \right )^n \int_{\Bbb R^n} \int_K e^{i
\lambda \vph (x, \omega, y, \eta)} p(x, y, \lambda \eta) a(y, \lambda \omega)
(1- \chi(\eta)) dy d \eta, 
\end{equation}
and
$$R_1(x, \omega, \lambda) =  \left ( \frac{\lambda}{2 \pi} \right )^n \int_{\Bbb
R^n} \int_K e^{i \lambda \vph (x, \omega, y, \eta)} p(x, y, \lambda \eta) a(y,
\lambda \omega) \chi (\eta)dy d\eta $$
\begin{proposition}  $ R_1(x, \omega, \lambda) \in S^{- \infty} (X \times S^{N-1}
\times \Bbb R^+)$.  
\end{proposition}
\nin
In fact, by \eqref{12} we have
$$\vph^\prime_y \ne 0 \text{ on } supp\  \chi$$
\vskip0.1in
\nin
Therefore the operator
$$L_y =  \frac{<- \eta + \psi^\prime_y (y, \omega), \ D_y>}{|-
\eta + \psi^\prime_y (y, \omega)|^2} $$
is well defined on supp $ \chi $ and we have
$$\frac{1}{\lambda} L_y (e^{i \lambda \vph}) = e^{i \lambda \vph}. $$
If we multiply by   $\chi $ then we obtain the relation
$$ \chi \frac{1}{\lambda}  L_y (e^{i \lambda \vph}) = e^{i \lambda \vph} \chi, \
y \in K, \eta \in \Bbb R^n.$$
Therefore by integrating by parts (as usual) for any $ k = 1, 2, 3, \dots $ we
obtain
$$R_1(x, \omega, \lambda) = \left ( \frac{\lambda}{2 \pi} \right )^n \int_{|\eta|
\le \frac{r}{2}} \int_{y \in K}\  e^{i \lambda \vph (x, \omega, y, \eta)}
\left (\frac{1}{\lambda} {}^tL_y \right )^k [p(x, y, \lambda \eta) a (y, \lambda
\omega)] \chi (\eta) dy d \eta$$ 
Thus for any $ k  \in \Bbb N $ we obtain
\begin{align}
|R_1(x, \omega, \lambda)| &\le C^\prime \lambda^n \lambda^{-k}
\int_{|\eta| \le \frac{r}{2}} \int_{y \in K} \ (1 + |\lambda \eta|)^m (1 + |\lambda
\omega|)^q dy d\eta\\
&\le C \lambda^n \lambda^{-k} \lambda^m \lambda^q \end{align}
Since for each $ \alpha, \beta, j  $ we can prove similar inequality for $
\partial^\alpha_x \partial^\beta_\omega \partial^j_\lambda R(x, \omega,
\lambda) $ we conclude that $ R_1 \in S^{- \infty} (X \times S^{N-1}\times \Bbb
R^+) $. In the $ (x, \theta) $
variables this is equivalent to $ R_1(x, \theta) \in S^{- \infty} (X \times \Bbb
R^N)$. \vskip0.1in
\nin
{\bf Estimation of} \boldmath $I(x, \omega, \lambda).$ 
\unboldmath
To estimate the integral $ I(x, \omega,
\lambda) $ we shall use a cut-off one more time to localize in both $ y $ and $
\eta$,  and then we shall apply the stationary phase theorem.  We have $$d_{y,
\eta} \vph (x, \omega, y, \eta) = (x-y) d \eta + \left [- \eta + \psi^\prime_y (y,
\omega) \right ] dy$$ and
$$\vph^{\prime \prime}_{y, \eta} = \begin{pmatrix}
\psi^{\prime \prime}_{yy} &-I\\
\\
-I &0 \end{pmatrix}$$
The critical points of $ \vph $ are
$$y = x, \eta = \psi^\prime_x (x, \omega)$$
Now we apply Morse's lemma near the point
$$(x_0, \omega_0, y_0 = x_0, \ \eta_0 = \psi'_x (x_0, \omega_0)). $$
There exist:  a neighborhood $ U \times \Omega $ of  $(x_0, \omega_0)$, a
neighborhood $ V $ of $ (y_0 = x_0, \ \eta_0 = \vph_x (x_0, \omega_0)) $ and a
morse map
$$h: (U \times \Omega)\times V \longrightarrow \Bbb R^{2n}. $$
By shrinking $ U \times \Omega $ we can arrange so that the set of critical
values
\begin{equation}
	\label{15}
	V_0 = \{ (y, \eta): y = x, \ \eta = \psi^\prime_x (x, \omega) \} \subset
\subset V. 
\end{equation}
Then we choose a cut-off function $ \rho(y, \eta) \in C^\infty_0 (V) $ such that
$ \rho = 1 $ on $ V_0$, and write
$$I = J + R_2, $$
where
\begin{equation}
	\label{16}J(x, \omega, \lambda) =
\left ( \frac{\lambda}{2 \pi} \right )^n \int_{|\eta| \ge \frac{r}{4}} \int_K e^{i
\lambda \vph(x, \omega, y, \eta)} p(x, y, \lambda \eta) a(y, \lambda \omega) (1
- \chi(\eta)) \rho(y, \eta) dy d\eta, 
\end{equation}
and
$$R_2 (x, \omega, \lambda) =
\left ( \frac{\lambda}{2 \pi} \right )^n \int_{|\eta| \ge \frac{r}{4}} \int_K e^{i
\lambda \vph(x, \omega, y, \eta)} 
p(x, y, \lambda \eta) a(y, \lambda \omega) (1
- \chi(\eta)) (1-\rho(y, \eta)) dy d\eta.$$
\nin
{\bf Claim: }  $ R_2 (x, \omega, \lambda) \in S^{- \infty} (\cup \times
\Omega \times \Bbb R^+)$.
\vskip0.1in
\nin
In fact, by \eqref{15} and the choice of $ \rho $ we have that there exists $ \ve > 0 $
such that
$$|\vph^\prime_{y, \eta}|^2 = |x-y|^2 + |- \eta + \psi^\prime_y (y, \omega)|^2
\ge \ve > 0 \text{ on supp } (1 - \rho). $$
Therefore the operator
$$L_{y, \eta} = \frac{1}{|\vph^\prime_{y, \eta}|^2} \left ( <x-y, D_\eta> + <- \eta
+ \psi^\prime_y (y, \omega), D_y> \right )$$
is well defined on supp $(1 - \rho) $ and we have
$$\frac{1}{\lambda} L_{y, \eta} \left ( e^{i \lambda \vph} \right ) = e^{i \lambda
\vph}$$
and thus
$$(1 - \rho) \frac{1}{\lambda} L_{y, \eta} \left ( e^{i \lambda \vph} \right ) =
e^{i \lambda \vph} (1 - \rho) \text{ all } y, \eta. $$
Then for any $ k \in \Bbb N $
\begin{equation}
	\begin{split}
R_2 (x, \omega, \lambda) =  \left (
\frac{\lambda}{2 \pi} \right )^n \int\limits_{|\eta| \ge \frac{r}{4}} \int\limits_K
& e^{i \lambda \vph(x, \omega,  y, \eta)}\left  ( \frac{1}{\lambda} {}^tL_{y, \eta} 
\right )^k  \big [ p(x, y, \lambda \eta  ) a(y, \lambda \omega) \  \cdot \\
& (1
- \chi (\eta) ) (1 - \rho(y, \eta) \big ] dy d \eta. \end{split}
\end{equation}
Since the coefficients of $ L_{y, \eta} $ are in $ S^{-1} $ in $ \eta $, we obtain the
following estimate for $ y \in K$: 
\begin{equation}
	\begin{split}
& \big |\left ({}^t L_{j,\eta} \right )^k [p(x, y, \lambda \eta) a(y, \lambda
\omega) (1 - \chi(\eta )) (1 - \rho(y, \eta) )]\big |\\
&\le C(1 + |\eta|)^{-k}
\sum_{|\beta + \gamma | \le k} \lambda^{|\beta|} \big | \left ( \partial^\alpha_y
\partial^\beta_\eta p \right )(x, y, \lambda \eta) \partial^\gamma_y a(y,
\lambda \omega) \big |\\
&\lesssim (1 + |\eta|)^{-k} \sum_{|\beta + \gamma| \le
k} \lambda^{|\beta|} (1 + |\lambda \eta|)^{m - |\beta|} (1 + |\lambda \omega|)^q
\end{split}
\end{equation}
\nin
 $\bullet$ Since $ (1 + |\lambda \eta|) \le (1 + \lambda) (1 + |\eta|) $ if
$ m- |\beta| \ge 0 $ the general term in the sum is bounded by $
\lambda^{|\beta|} (1 + \lambda)^{m - |\beta|} (1 + |\eta|)^{m - |\beta|} (1 +
|\lambda \omega|)^q$.  Since $ \omega $ is near $ \omega_0, |\omega_0| = 1 $
the above quantity is bounded for $ \lambda \ge 1 $ by 
$$C \lambda^{m + q} (1 + |\eta|)^{-k+m}$$
$\bullet$  Since $ |\eta| \ge \frac{r}{4} > 0 $ if $ m - |\beta| < 0 $ then
$$(1 + |\lambda \eta|)^{m - |\beta |} \le (1 + \frac{r}{4} \lambda)^{m - |\beta |}
\lesssim \lambda^{m - |\beta|}, \ \lambda \ge 1 $$
Therefore in this case the general term in the above sum is bounded by $ C
\lambda^{|\beta|} \lambda^{m - |\beta|} |\lambda|^q = \lambda^{m+q}$. 
Therefore the above quantity is bounded by
$$C \lambda^{m + q} (1 + |\eta|)^{-k} $$
Therefore for any $ k \in \Bbb N $ and large we obtain
\begin{equation*}
	\begin{split}
|R_2 (x, \omega, \lambda)| &\le C \lambda^{- k + n + m + q} \int_{\Bbb R^n} (1 +
|\eta|)^{-k + |m|} d \eta\\
&\le C \lambda^{-k + n + m + q} \int_{\Bbb R^n} (1 + |\eta|)^{-n-1} d \eta
\end{split}
\end{equation*}
if $ -k + |m| < -n - 1$.  Therefore for $ k > |m| + n + 1 $ we obtain
$$|R_2 (x, \omega, \lambda)| \le C \lambda^{- k + n + m +q}, \ \lambda \ge 1. $$
Since similar inequalities can be obtained for $ \partial^\alpha_x
\partial^\beta_\omega  \partial^\gamma_\lambda R_2 $ we conclude that $
R_2 \in S^{- \infty}$, and this completes the proof of the claim. $\qquad
\Box$
\vskip0.1in
\nin
{\bf Estimation of the integral} \boldmath $J ( x,
\omega,  \lambda)${\bf :}  \unboldmath For $ J(x, \omega, \lambda)$  we
are in the ideal situation for applying the stationary phase theorem.  We
have
$$J(x, \omega, \lambda) \sim \left ( \frac{\lambda}{2 \pi} \right )^n \left (
\frac{2 \pi}{\lambda} \right )^{\frac{2n}{2}} \frac{e^{i \frac{\pi}{4}sgn
\vph^{\prime \prime}_{y, \eta} (x, \omega, x, \psi^\prime_x (x,
\omega))}}{\sqrt{|\text{det}\  \vph^{\prime \prime}_{y, \eta} (x, \omega, x,
\psi'_x (x, \omega))}} \ e^{i \lambda \vph (x, \omega, x, \psi^\prime_x (x,
\omega))}\cdot$$ $$\sum^\infty_{k=0} \ \frac{\lambda^{-k}}{k!} R^k (x,
\omega, y, \eta, D_y, D_\eta) \bigg [p(x, y, \lambda \eta) a(y, \lambda \omega)
(1 - \chi(\eta)) \rho (y, \eta) \bigg ]_{\begin{matrix} y &= x\hskip.5in  \\
	\eta &= \psi^\prime_x (x, \omega).\end{matrix}} $$
 Since
 \begin{equation*}
	 \begin{split}
\bullet \det \vph^{\prime \prime}_{y, \eta} (x, \omega, x, \psi'_x (x, \omega))
& = 1 
\bullet  \ sgn \vph^{\prime \prime}_{y, \eta} (x, \omega, x, \psi'_x (x,
\omega)) \\
&= 0 
\bullet \ \vph (x, \omega, x, \psi^\prime_x (x, \omega)) \\
&= 0 
\end{split}
\end{equation*}
the above formula takes the form
\begin{equation}
	\label{17}
	\begin{split}
&J(x, \omega, \lambda) \\
&\sim \sum^\infty_{k=0} \frac{\lambda^{-k}}{k!}
R^k (x, \omega, y, \eta, D_y, D_\eta) \left [p(x, y, \lambda \eta) a(y, \lambda
\omega) (1 - \chi(\eta) \rho(y, \eta) \right ]_{\begin{matrix}
y &= x\hskip .45 in  \\
\eta &= \psi^\prime_x (x, \omega).\end{matrix}}\end{split}
\end{equation}
Now we notice that the operator $ R^k $ is a linear differential operator which
depends only on the morse map or equivalently on $ \varphi$, which is an
expression of $ \psi$.  It does not depend on the symbol $ p$.  Also notice that:
\vskip0.1in
\noindent
$\bullet (\chi(\eta) = 0 $ near points where $ \eta = \psi^\prime_x (x,
\omega)$, since by the choice of cut-off $ \chi,\  \chi(\eta) = 0 $ if $ |\eta| >
\frac{r}{2} $ and when $ x \in U, \omega \in \Omega $ we have $
|\psi^\prime_x (x, \omega) | > \frac{r}{2}$.
\vskip0.1in
\noindent
$\bullet \rho(y, \eta) = 1 $ near the critical points $ y = x, \eta =
\psi^\prime_x (x, \omega)$.
\vskip0.1in
\noindent
Therefore in \eqref{17} $ (1 - \chi(\eta)) \rho(y, \eta) $ can be taken equal to 1.
\vskip0.1in
\noindent
Taking all the above into consideration and for each $ k $ expanding the
operator $ R^k $ of order $ 2k $ we obtain
\begin{align}
&J(x, \omega, \lambda) \sim\sum^\infty_{k=0} \frac{\lambda^{-k}}{k!}
\sum_{|\alpha + \beta + \gamma| \le 2k} \\
&C_{\alpha, \beta, \gamma, k} (x, \omega, y, \eta) (D^\alpha_y
\partial^\beta_\xi p) (x, y, \lambda \eta) \lambda^{|\beta|} D^\gamma_y a(y,
\lambda \omega) \bigg |_{\begin{matrix} y &=x \hskip .45 in\\
 \eta &=\psi^\prime_x(x, \omega) \end{matrix}} \end{align} 
$$\sim \sum_{k=0}
\frac{\lambda^{-k}}{k!} \sum_{|\alpha + \beta + \gamma| \le 2k}
C^\prime_{\alpha, \beta, \gamma, k} (x, \omega) (D^\alpha_y
\partial^\beta_\xi p) (x, x, \psi^\prime_x (x, \lambda \omega))
\lambda^{|\beta|} (D^\gamma_y a)(x, \lambda \omega). $$ 
Now by letting
$$\omega = \frac{\theta}{\lambda} $$
we go back to the original variables $ (x, \theta)$, and $ J(x, w, \lambda) $
takes the form
\begin{equation}
	\label{18}J(x, \theta) \sim \sum^\infty_{k=0} \sum_{|\alpha + \beta + \gamma| \le 2k}
C_{\alpha, \beta, \gamma, k} (x, \theta) (D^\alpha_y \partial^\beta_\xi p) (x, x,
\psi^\prime_x (x, \theta) D^\gamma_y a(x, \theta),
\end{equation}
where $ C_{\alpha, \beta, \gamma, k} $ depends only on $ \psi $ and is
independent of $ p $ and $ a $ and
$$C_{\alpha, \beta, \gamma, k} (x, \theta) \text{ is homogeneous of degree }
|\beta| - k. $$
This follows by the form of $ \varphi^{\prime \prime} $ at the critical points $
y = x, \eta = \psi^\prime_x (x, \theta) $ and the fact that in the Morse
variables
$$R = \frac{i}{2} < \left ( \varphi^{\prime \prime}_{y, \eta} \right )^{-1} (x, x,
\omega, \psi^\prime_x (x, \omega) \partial_z, \partial_z > $$
and the entries of $ \left ( \varphi^{\prime \prime}_{y, \eta} \right )^{-1} $ are
homogeneous of degree zero since the entries of $ - \varphi^{\prime
\prime}_{y, \eta} $ are homogeneous of degree zero evaluated at the critical
points.  
\smallskip
\noindent
Now we see that the general term of the sum in \eqref{18} is in
$$S^{|\beta| - k + m - |\beta| + q} = S^{m+q-k}$$
Since
$$m+q-k \le m + q - \frac{1}{2} |\alpha + \beta + \gamma|$$
and it tends to $ - \infty $ when $ |\alpha| $ or $ |\beta|$ or $ |\gamma| $ goes to
$ \infty $ the asymptotic sum \eqref{18} can be written in the form
\begin{equation}
	\label{19}
	J(x, \omega) \sim \sum_{\alpha, \beta} \left ( D^\alpha_y
\partial^\beta_\xi p \right ) (x, x, \psi^\prime_x (x, \theta)) \
\sigma_{\alpha, \beta} (x, \theta), 
\end{equation}
\begin{equation}
	\label{20}
	\sigma_{\alpha, \beta} (x, \theta) \sim \sum_{k, \gamma} C_{\alpha, \beta,
\gamma, k} (x, \theta) D^\gamma_y a(x, \theta). 
\end{equation}
Since each term $ C_{\alpha, \beta, \gamma, k} (x, \theta) D^\gamma_ya (x,
\theta) $ is in
$$S^{m+q-k-(m -|\beta|)} $$
and
\begin{equation*}
	\begin{split}
m+q-k-(m - |\beta|) &= q - k + |\beta| \\
&\le q - \frac{1}{2} |\alpha| - \frac{1}{2} |\beta| - \frac{1}{2} |\gamma| + |\beta|\\
&= q + \frac{1}{2} |\beta| - \frac{1}{2} |\alpha| - \frac{1}{2} |\gamma|, \\
&\le q + \frac{1}{2} |\beta| - \frac{1}{2} |\alpha|, \end{split}
\end{equation*}
we see that
\begin{equation*}
	\begin{split}
	\sigma_{\alpha, \beta} (x, \theta) \in S^{q + \frac{1}{2} |\beta| - \frac{1}{2}
|\alpha|}. 
\end{split}
\end{equation*}
Since $ \sigma_{\alpha, \beta} $ is independent of the symbol $ p $ it can be
computed by using any easy symbol $ p(x, y, \xi)$.  Therefore we have
computed them in the special case \eqref{7} at the beginning of the proof of the
theorem.  This completes the proof of the theorem. $\Box$
%
\end{document}
                                                                                                                                                                   

                                                                                                                                                                                                                                                                            
 

