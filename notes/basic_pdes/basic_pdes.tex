\chapter{Four Important PDEs}
\section{The Transport Equation}
The general form of the transport equation is given by
\begin{equation*}
\begin{split}
	u_{t}  + b\Delta u = 0,
\end{split}
\end{equation*}
with $u = u(x,t)$, constant $b \in \rr^n$, $x \in \rr^{n}$, $t \in \rr$.
As an example for how to find a classical solution to this equation, we set
$n = 1$ and $b=2$. That is
\begin{equation}
	\label{constant-behav}
\begin{split}
	0 = u_{t} + 2 u _{x} = (2,1) \cdot (u_{x}, u_{t}).
\end{split}
\end{equation}
Hence, the directional derivative of $u$ in the direction of vector $(2,1)$ is
0; that is, $D_{(1,2)} = 0$. Hence, $u$ is constant on each line parallel to
$(2,1)$, since they have the same direction as $(2,1)$. Since the lines parallel
to $(2,1)$ traverse the entire $(x,t)$ plane, the solution $u$ must be
determined by them. Let $(0,c)$ be a point on the $t$ axis, and $(x,t)$ be some
point in $\rr^{2}$ on the unique line parallel to $(2,1)$ and containing point
$(0,c)$. Since $(1,-2) \perp (2,1)$, we want
\begin{equation*}
\begin{split}
	& (1,-2) \cdot \left[ (x,t) - (0,c) \right] = 0
	\implies -\frac{1}{2}(x-2t) = c
\end{split}
\end{equation*}
Observe that $u$ is constant for $c$ satisfying the above, by
\eqref{constant-behav}. Since $c$ was arbitrary, we can vary it over all $\rr$,
and so $u$ is constant on all lines $\{ c = -(x-2t)/2\}_{c \in \rr}$. The above holds for any fixedHence, for any given $c$, there exis Shifting our perspective, it follows 

