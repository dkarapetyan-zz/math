% !TeX root = ../notes.tex
\chapter{The Transport Equation}
\section{The Homogeneous Transport Equation}
The general form of the homogenous transport equation is given by
\begin{equation*}
\begin{split}
	& u_{t}  + b\Delta u = 0
	\\
	& u(x,0) = f(x)
\end{split}
\end{equation*}
with $u = u(x,t): \rr^{n} \times \rr \to \rr$, and constant $b \in \rr^n$.
As an example for how to find a classical solution to this equation, we set
$n = 1$ and $b=2$. That is
\begin{equation}
	\label{constant-behav}
\begin{split}
	0 = u_{t} + 2 u _{x} = (2,1) \cdot (u_{x}, u_{t}).
\end{split}
\end{equation}

Hence, the directional derivative of $u$ in the direction of vector $(2,1)$ is
0; that is, $D_{(1,2)} = 0$. Hence, $u$ is constant on each line parallel to
$(2,1)$, since they have the same direction as $(2,1)$. Since the lines parallel
to $(2,1)$ traverse the entire $(x,t)$ plane, the solution $u$ must be
determined by them. Let $(0,c)$ be a point on the $t$ axis, and $(x,t)$ be some
point in $\rr^{2}$ on the unique line parallel to $(2,1)$ and containing point
$(0,c)$. Since $(1,-2) \perp (2,1)$, we want
\begin{equation*}
\begin{split}
	& (1,-2) \cdot \left[ (x,t) - (0,c) \right] = 0
	\implies -\frac{1}{2}(x-2t) = c
\end{split}
\end{equation*}

Observe that $u$ is constant for $c$ satisfying the above, by
\eqref{constant-behav}. Since $c$ was arbitrary, we can vary it over all $\rr$,
and hence $u$ is a solution to the transport equation if and only if it is
constant on each line $\{ c = x-2t\}_{c \in \rr}$.
On each of these lines, $u(x,t)$ takes a particular value, determined entirely
by the point $(0,c)$ at which this line (in the direction $(2,1)$ intersects the
$t$-axis. That is, $u(x,t) = u(0, t-x/2) = f(t-x/2)$.

Using similar logic as in this example, one can show that the generalized
solution to the transport equation is
\begin{equation*}
\begin{split}
	u(x,t) = f(x-bt).
\end{split}
\end{equation*}
\section{The Inhomogenous Transport Equation}

The general form of the inhomogenous transport equation is given by
\begin{equation*}
\begin{split}
	& u_{t}  + b\Delta u = g(x)
	\\
	& u(x,0) = f(x)
\end{split}
\end{equation*}
with $u = u(x,t)$, constant $b \in \rr^n$, $x \in \rr^{n}$, $t \in \rr$.
As an example for how to find a classical solution to this equation, we set
$n = 1$, $b=2$, and $g(x) = \cos(x+t)$. That is
\begin{equation}
\begin{split}
	\cos(x + t) = u_{t} + 2 u _{x} = (2,1) \cdot (u_{x}, u_{t}).
\end{split}
\end{equation}
with $u = u(x,t): \rr^{n} \times \rr \to \rr$, constant $b \in \rr^n$, and
$g = g(x): \rr^{n} \to \rr$.

Heuristically, what we would like to do is integrate both sides along an
appropriate curve, and invoke the fundamental theorem of calculus to obtain our
solution. To make this argument rigorous, we will define a parametrization of
$u$, for fixed $x,t$, that is essentially the values of $u$ along the line
containing $x, t$ and parallel to $(-1,2)$.

Proceeding, we set $z(s) = u(x-s, t+2s)$. Then
\begin{equation*}
\begin{split}
	z ' (s) & = (-1,2) \cdot
(u_{x}(x-s, t+2s), u_{t}(x-s, t+2s))
\\
& =  \cos((x-s) + (t + 2s))
\\
& = \cos(x + s + t).
\end{split}
\end{equation*}
Observe that $z(-t/2)$ = $f(x-s)$. Hence, integrating both sides, we obtain
\begin{equation*}
\begin{split}
	\int_{-t/2}^{s} z'(q) dq  = \int_{-t/2}^{s} \cos(x + q + t) dq
\end{split}
\end{equation*}
which implies
\begin{equation*}
\begin{split}
	z(s) = f(x-s) + \sin(x + s + t) - \sin(x + t/2)
\end{split}
\end{equation*}
Setting $s = 0$, we conclude that
\begin{equation*}
\begin{split}
	u(x,t) = f(x) + \sin(x + t) - \sin(x + t/2).
\end{split}
\end{equation*}
The general solution, given initial condition $u(x,0) = g(x)$, is given by
\begin{equation*}
\begin{split}
	u(x,t) = g(x-tb) + \int_{0}^{t} f(x + (s-t)b, s)ds
\end{split}
\end{equation*}

\chapter{The Laplace Equation}
\section{The Homogeneous Laplace Equation}
In $\rr$, the equation takes the form
\begin{equation*}
\begin{split}
	u_{xx} = 0.
\end{split}
\end{equation*}
To find the solution, we note that
\begin{gather*}
	\frac{\partial}{\partial_{x}}(u_{x}) = u_{xx} = 0
	\\
	\implies u_{x} = b
	\\
	\implies u = bx + c.
\end{gather*}
In $\rr^{2}$, the equation takes the form
\begin{equation*}
\begin{split}
	u_{xx} + u_{yy} = 0.
\end{split}
\end{equation*}
\begin{definition}
	For $f: V \to W$, where $V$ and $W$ are vector spaces, let $\tau_{z}(f)(x) = f(x-z)$. Then $\tau_{z}$ is called a \emph{translation}
	of $f$. It is said to translates $f$ forward by $z$ units. Similarly,
	$\tau_{-z}$ is said to translate $f$ backward by $z$ units.
\end{definition}

\begin{definition}
Let \begin{equation*}
\begin{split}
A = \begin{bmatrix}
& \cos \theta & \sin \theta
\\ & -\sin \theta & \cos \theta.
\end{bmatrix}
\end{split}
\end{equation*}
Then for $z = (x,y) \in \rr^{2}$, we say that $Az$ is a $\theta$-degree
clockwise \emph{rotation}.
\end{definition}

\begin{lemma}
	Solutions to the Laplace equation in $\rr^{2}$ are invariant under
	translations and rotations
\end{lemma}
\begin{remark}
One can generalize this result to arbitrary
dimensions.
\end{remark}

\begin{proof}
	To prove invariance under translation, we let $u(x,y)$ be a solution to
	Laplace's equation, and consider $v(x+a, y+b) \doteq u(x,y)$. Then by the
	chain rule, we obtain
	\begin{equation*}
	\begin{split}
		0 = u_{xx} + u_{yy} = v_{xx}(x + a, y+b) + v_{yy}(x + a, y+b)
		= \tau_{(a,b)}u_{xx} + \tau_{(a,b)}u_{yy}.
	\end{split}
	\end{equation*}
	To prove invariance under rotations, we first observe that
	\begin{equation}
		\label{eq:rotation}
	\begin{split}
		A(x,y)^{T} = \begin{bmatrix}
		& x \cos \theta + y \sin \theta
		\\
		& -x \sin \theta + y \cos \theta
		\end{bmatrix}
	\end{split}
	\end{equation}
	Define $v(x \cos \theta + y \sin \theta, -x \sin \theta + y \cos \theta)
	\doteq u(x,y)$. It follows from \eqref{eq:rotation} that $v$ is a rotation of
	solution $u(x,y)$ clockwise by angle $\theta$. Setting
	$x' = x \cos \theta + y\sin \theta$ and $y' = -x \sin \theta + y \cos \theta$
	and applying the chain rule, we obtain
	\begin{equation*}
	\begin{split}
		& u_{x} = u_{x'} \cos \theta - u_{y'} \sin \theta
		\\
		& u_{y} = u_{x'} \sin \theta + u_{y'} \cos \theta
	\end{split}
	\end{equation*}
and
\begin{equation*}
\begin{split}
	& u_{xx} = (u_{x'} \cos \theta - u_{y'} \sin \theta)_{x'} \cos \theta -
	(u_{x'} \cos \theta - u_{y'} \sin \theta)_{y'} \sin \theta
	\\
	& u_{yy} = (u_{x'} \sin \theta + u_{y'} \cos \theta)_{x'} \sin \theta
	+ (u_{x'} \sin \theta + u_{y'} \cos \theta)_{y'} \cos \theta.
\end{split}
\end{equation*}
Summing, we obtain
\begin{equation*}
\begin{split}
	0 = u_{xx} + u_{yy} = \left( u_{x'x'} + u_{y'y'} \right)\left( \cos^{2} \theta
		+ \sin^{2} \theta
	\right) = u_{x'x'} + u_{y'y'}.
\end{split}
\end{equation*}
\end{proof}

\chapter{The Wave Equation}
\section{The Homogeneous Wave Equation}
The homogeneous wave equation is written
\begin{equation*}
\begin{split}
	u_{tt} - u_{xx} = 0
\end{split}
\end{equation*}
subject to initial and boundary conditions. Here we have $u\doteq u(x,t)$,
where $x \in \rr$. $u(x,t)$ represents the amplitude of the wave at time $t$ and
position $x$. We think of motion being induced in a string, and propogating
through it, via a force $F(t)$ applied to the end of the string.

Using this framework, we now derive the wave equation. First, we assume the
absence of gravity. Note that
\begin{equation*}
\begin{split}
	\frac{\partial^{2}}{\partial t^{2}} \int_{0}^{\infty} u \ dx =
	\int_{0}^{\infty} u_{tt} \ dx.
\end{split}
\end{equation*}
We must choose an orientation; it is natural to let ``upwards'' be our
``positive direction''. Hence, the force used used to generate the wave is
$-F(t)$. Taking the mass density of our string to be equal to $1$, we have
$a = F$ be Newton's law, and hence
\begin{equation*}
\begin{split}
	\int_{0}^{\infty} u_{tt} \ dx = -F.
\end{split}
\end{equation*}
By the fundamental theorem of calculus, it follows that
\begin{equation*}
\begin{split}
	u_{tt} = -F_{x}.
\end{split}
\end{equation*}
In particular, $F$ is a function of velocity, and if $F$ is constant when our
hand is at the uppermost and lowermost points, we can write
$F = -a u_{x}$. Substituting, we obtain
\begin{equation*}
\begin{split}
	u_{tt} - au_{xx} = 0.
\end{split}
\end{equation*}
Without loss of generality, we can set $a = 1$ (otherwise, we simply scale
appropriately in the $x$ variable), which gives us the wave equation.
\subsection{Derivation of Solution}
We can factor the wave equation as
\begin{equation*}
\begin{split}
	0 = u_{tt} - u_{xx} = \left( \frac{\partial}{\partial t} +
	\frac{\partial}{\partial x} \right)\left( \frac{\partial}{\partial t} -
	\frac{\partial}{\partial x} \right) u.
\end{split}
\end{equation*}
Let $v(x,t) \doteq \left( \frac{\partial}{\partial t} -
\frac{\partial}{\partial x} \right) u $. Then the wave equation can be rewritten
as
\begin{equation*}
\begin{split}
	\left( \frac{\partial}{\partial t} + \frac{\partial}{\partial x} \right) v
	= v_{t} + v_{x} = 0.
\end{split}
\end{equation*}
But this is just the transport equation. Hence, it admits a traveling wave
solution $v(x,t) = g(x-t)$, where $g$ is a function of one variable.
We now work with an initial condition, to make things clearer.
Set
\begin{equation*}
\begin{split}
	u(x,0) = 0, \quad u_{t}(x,0) = \cos x.
\end{split}
\end{equation*}
Then
\begin{equation*}
\begin{split}
	u_{t}(x,t) - u_{x}(x,t) = v(x,t) = g(x-t).
\end{split}
\end{equation*}
If $t=0$, then
\begin{equation*}
\begin{split}
	u_{t}(x,0) - u_{x}(x,0) = g(x).
\end{split}
\end{equation*}

But $u_{x}(x,0) = 0$ and $u_{t}(x,0) = \cos x$ by assumption. Therefore,
we must have $g(x) = \cos x$, and so we have reduced the problem of solving the
wave equation to solving
\begin{equation*}
\begin{split}
	u_{t}(x,t) - u_{x}(x,t) = \cos (x-t).
\end{split}
\end{equation*}
But this is just a non-homogeneous transport equation, with solution
\begin{equation*}
\begin{split}
\frac{1}{2c} \left[ \sin(x + ct) - \sin(x-ct) \right].
\end{split}
\end{equation*}
In general,
\begin{equation*}
\begin{split}
	u(x,t) = \frac{1}{2} \left[ u(x + ct, 0) + u(x-ct, 0) \right] +
\frac{1}{2c} \int_{x-ct}^{x+ct} u_{t}(x,0) \ ds.
\end{split}
\end{equation*}
This is called \emph{d'Alembert's Formula}.

\chapter{The Heat Equation}
\section{The Homogeneous Heat Equation}
In $\rr$, the heat equation takes the form
\begin{equation*}
\begin{split}
	u_{t} - u_{xx} = 0,
\end{split}
\end{equation*}
subject to initial and boundary conditions. Note that $u \doteq u(x,t)$.
\subsection{Derivation}
Let $U$ be a closed surface (i.e. a rod). in $\rr$. If $u \doteq u(x,t)$ denotes
temperature density at time $t$, the temperature of our rod at time $t$ is
\begin{equation*}
\begin{split}
	\int_{V} u \ dx.
\end{split}
\end{equation*}
The rate of change of the temperature within $V$ equals the negative of the
rate that heat escapes from the ends of the rod (i.e. the net flux through
$\partial V$). Hence,
\begin{equation*}
\begin{split}
	\frac{d}{dt} \int_{V} u \ dx = -\left[ F \cdot  \right]_{a}^{b}
\end{split}
\end{equation*}
where $F$ is the flux density, $a$ is the leftmost end of the rod, $b$ is the
rightmost end, and $v$ is a unit vector parallel to the $x$-axis on which the
rod lies. By the fundamental theorem of Calculus, we infer that
\begin{equation*}
\begin{split}
	u_{t} = -\partial_{x} F = -\text{div}F.
\end{split}
\end{equation*}
In many situations, $F$ is proportional to $\partial_{x} u$, but points in
opposite direction (since heat spreads from regions of high concentration to
regions with low concentration). Hence
\begin{equation*}
\begin{split}
	F = -a \partial_{u}u, \quad (a > 0).
\end{split}
\end{equation*}
Substituting, we get
\begin{equation*}
\begin{split}
	u_{t} = a \partial_{x}(\partial_{x} u ) = a u_{xx}.
\end{split}
\end{equation*}
Without loss of generality (we can always scaled in the $x$ variable), we set $a
= 1$, obtaining the heat equation.
\subsection{Derivation of the Fundamental Solution}
A simple computation (i.e. ``plug and chug'') shows that if $u(x,t)$ is a solution to the heat equation, then
so is $u(\lambda x, \lambda^{2} t)$.
\begin{equation*}
\begin{split}
\end{split}
\end{equation*}
Using this information, we ask if there are \emph{traveling wave} solutions to
the heat equation. That is, solutions of form
\begin{equation*}
\begin{split}
	u(x,t) = g(x/\sqrt{t}),
\end{split}
\end{equation*}
where $g$ is some function of one variable. We callLet $p doteq x / \sqrt{t}$.
Note that
\begin{equation*}
\begin{split}
	u_{t} = -\frac{1}{2t} \frac{x}{\sqrt{t}}g'(p), \quad u_{xx} =
	\frac{1}{t}g'(p)
\end{split}
\end{equation*}
and so
\begin{equation*}
\begin{split}
	0 = u_{t} - u_{xx} = \frac{1}{t}\left( -\frac{1}{2}pg'(p) -
	\frac{1}{4} g''(p) \right).
\end{split}
\end{equation*}
Simplifying, we have
\begin{equation*}
\begin{split}
	g'' + 2pg' = 0.
\end{split}
\end{equation*}
The general solution to this ODE is
\begin{equation*}
\begin{split}
	g(p) = c_{1} \int_{0}^{x / \sqrt{t}} e^{-p^{2}} \ dp + c_{2}.
\end{split}
\end{equation*}
Specifying the initial condition
\begin{equation*}
\begin{split}
	Q(x,0) = \begin{cases}
		1, \quad x > 0 \\
		0, \quad x < 0,
	\end{cases}
\end{split}
\end{equation*}
we obtain
\begin{equation*}
\begin{split}
	0 & =  -c_{1} \frac{\sqrt{\pi}}{2} + c_{2}
	\\
	1 & = c_{1} \frac{\sqrt{\pi}}{2} + c_{2}
\end{split}
\end{equation*}
and so $c_{1} = 1/\sqrt{\pi}$ and $c_{2} = 1/2$.
Therefore,

\begin{equation*}
\begin{split}
	u(x,t) = \frac{1}{2} + \frac{1}{\sqrt{\pi}} \int_{0}^{x / \sqrt{t}}
	e^{-p^{2}} \ dp.
\end{split}
\end{equation*}
Observe that if $u(x,t)$ is a solution to the heat equation, then so is
$u_{x}$. Differentiating the above, we obtain
\begin{equation*}
\begin{split}
u_{x}(x,t) = \frac{1}{\sqrt{\pi t}} e^{-x^{2}/t}
\end{split}
\end{equation*}
which is called the \emph{fundamental solution } to the heat equation.

\chapter{What are PDE's?}
A partial differential equation (PDE) is any equation involving a function of 
two or more variables and some of its partial derivaties. Some examples:
\begin{example}
Laplace's equation:
\begin{equation*}
\begin{split}
	\Delta u = \sum_{i=1}^{n} u_{x_{i}} u _{x_{i}} = 0
\end{split}
\end{equation*}
\end{example}
\begin{example}
	Linear equation equation:
	\begin{equation*}
	\begin{split}
		u_{t} + \sum_{i=1}^{n}b^{i}u_{x_{i}} = 0.
	\end{split}
	\end{equation*}
\end{example}
\begin{definition}
An expression of form
\begin{equation*}
\begin{split}
	F(D^{k}u (x), D^{k-1}u(x), \dots, Du(x), u(x), x) = 0, \quad x \in U
\end{split}
\end{equation*}
is called a \emph{kth order PDE}, where
\begin{equation*}
\begin{split}
	F: \rr^{n^{k}} \times \rr^{n^{k-1}} \times \dots \times \rr^{n} \times \rr
	\times U \to \rr
\end{split}
\end{equation*}
is given, and $u: U \to \rr$ is the unknown.
\end{definition}
\begin{example}
	Let $u\doteq u(m,t)$, where $m = (x,y,z) in \rr^{3}$. Then
	\begin{equation*}
	\begin{split}
		u_{xx} + u_{y} = 0
	\end{split}
	\end{equation*}
	is a second order PDE.
\end{example}
\begin{definition}
A PDE is called \emph{linear} if it has form
\begin{equation*}
\begin{split}
	\sum_{|alpha| \le k} a_{\alpha}(x) D^{\alpha} u = f(x)
\end{split}
\end{equation*}
for given functions $a_{\alpha}, f$. The PDE is \emph{homogeneous} if $f \equiv
0$.
\end{definition}


