%
\documentclass[12pt,reqno]{amsart}
\usepackage{amssymb}
\usepackage{cancel}  %for cancelling terms explicity on pdf
\usepackage{yhmath}   %makes fourier transform look nicer, among other things
\usepackage[alphabetic, msc-links]{amsrefs} %for the bibliography; uses cite pkg
%\usepackage{showkeys}  %shows source equation labels on the pdf
\usepackage[margin=3cm]{geometry}  %page layout
%\usepackage[pdftex]{graphicx} %for importing pictures into latex--pdf compilation
\setcounter{secnumdepth}{1} %number only sections, not subsections
\hypersetup{colorlinks=true,
linkcolor=blue,
citecolor=blue,
urlcolor=blue,
}
\synctex=1
\numberwithin{equation}{section}  %eliminate need for keeping track of counters
\numberwithin{figure}{section}
\setlength{\parindent}{0in} %no indentation of paragraphs after section title
\renewcommand{\baselinestretch}{1.1} %increases vert spacing of text
%
\newcommand{\ds}{\displaystyle}
\newcommand{\ts}{\textstyle}
\newcommand{\nin}{\noindent}
\newcommand{\rr}{\mathbb{R}}
\newcommand{\nn}{\mathbb{N}}
\newcommand{\zz}{\mathbb{Z}}
\newcommand{\cc}{\mathbb{C}}
\newcommand{\ci}{\mathbb{T}}
\newcommand{\zzdot}{\dot{\zz}}
\newcommand{\wh}{\widehat}
\newcommand{\p}{\partial}
\newcommand{\ee}{\varepsilon}
\newcommand{\vp}{\varphi}
%
%
\theoremstyle{plain}  
\newtheorem{theorem}{Theorem}
\newtheorem{proposition}{Proposition}
\newtheorem{lemma}{Lemma}
\newtheorem{corollary}{Corollary}
\newtheorem{claim}{Claim}
\newtheorem{conjecture}[subsection]{conjecture}
%
\theoremstyle{definition}
\newtheorem{definition}{Definition}
%
\theoremstyle{remark}
\newtheorem{remark}{Remark}
%
%
%
\def\makeautorefname#1#2{\expandafter\def\csname#1autorefname\endcsname{#2}}
\makeautorefname{equation}{Equation}
\makeautorefname{footnote}{footnote}
\makeautorefname{item}{item}
\makeautorefname{figure}{Figure}
\makeautorefname{table}{Table}
\makeautorefname{part}{Part}
\makeautorefname{appendix}{Appendix}
\makeautorefname{chapter}{Chapter}
\makeautorefname{section}{Section}
\makeautorefname{subsection}{Section}
\makeautorefname{subsubsection}{Section}
\makeautorefname{paragraph}{Paragraph}
\makeautorefname{subparagraph}{Paragraph}
\makeautorefname{theorem}{Theorem}
\makeautorefname{theo}{Theorem}
\makeautorefname{thm}{Theorem}
\makeautorefname{addendum}{Addendum}
\makeautorefname{add}{Addendum}
\makeautorefname{maintheorem}{Main theorem}
\makeautorefname{corollary}{Corollary}
\makeautorefname{lemma}{Lemma}
\makeautorefname{sublemma}{Sublemma}
\makeautorefname{proposition}{Proposition}
\makeautorefname{property}{Property}
\makeautorefname{scholium}{Scholium}
\makeautorefname{step}{Step}
\makeautorefname{conjecture}{Conjecture}
\makeautorefname{question}{Question}
\makeautorefname{definition}{Definition}
\makeautorefname{notation}{Notation}
\makeautorefname{remark}{Remark}
\makeautorefname{remarks}{Remarks}
\makeautorefname{example}{Example}
\makeautorefname{algorithm}{Algorithm}
\makeautorefname{axiom}{Axiom}
\makeautorefname{case}{Case}
\makeautorefname{claim}{Claim}
\makeautorefname{assumption}{Assumption}
\makeautorefname{conclusion}{Conclusion}
\makeautorefname{condition}{Condition}
\makeautorefname{construction}{Construction}
\makeautorefname{criterion}{Criterion}
\makeautorefname{exercise}{Exercise}
\makeautorefname{problem}{Problem}
\makeautorefname{solution}{Solution}
\makeautorefname{summary}{Summary}
\makeautorefname{operation}{Operation}
\makeautorefname{observation}{Observation}
\makeautorefname{convention}{Convention}
\makeautorefname{warning}{Warning}
\makeautorefname{note}{Note}
\makeautorefname{fact}{Fact}
%
\begin{document}
\title{Notes on Numerical Computation }
\author{David Karapetyan}
\address{Department of Mathematics  \\
         University  of Notre Dame\\
				          Notre Dame, IN 46556 }
									\date{08/12/10}
									%
									\maketitle
									%
									%
									%
									%
									%
									%
									\section{Hermite Interpolation}
Let $f(x) \in C^m(\rr)$ with 
%
%
\begin{equation}
	\label{hinterp-data}
	\begin{split}
		f^i(x_j) = b_{i j }, \qquad 0 \le j \le k, \ 0 \le i \le n_j
	\end{split}
\end{equation}
%
%
where $b_{j k}$ is a constant. Can we find a polynomial $p(x)$ such that
%
%
\begin{equation}
	\label{h-poly-data}
	\begin{split}
		p^i(x_j) = b_{i j}, \qquad 0 \le j \le k, \ 0 \le i \le n_j, 
	\end{split}
\end{equation}
%
%
holds? If so, we call $p(x)$ a \emph{Hermite polynomial}, and we say that $p(x)$
\emph{interpolates} $f(x)$ at the \emph{nodes} $x_k$. 
%
%
%%%%%%%%%%%%%%%%%%%%%%%%%%%%%%%%%%%%%%%%%%%%%%%%%%%%%
%
%
%				 Unique Hermite Poly Theorem
%
%
%%%%%%%%%%%%%%%%%%%%%%%%%%%%%%%%%%%%%%%%%%%%%%%%%%%%%
%
%
\begin{theorem}
	If $\sum_{j=0}^{k} n_{j} = m-k$, then there exists a unique polynomial of
	degree less than or equal to $m$ satisfying \eqref{h-poly-data}.
\end{theorem}
%
%
%
{\bf Proof.} Write
%
%
\begin{equation*}
	\begin{split}
		p(x) = a_0 + a_{1}x + a_{2}x^2 +\cdots+ a_{m}x^m
	\end{split}
\end{equation*}
%
%
where the constants $\{a_i\}_{0 \le i \le m}$ are to be determined. We wish to solve
the system given by
%
%

\begin{gather*}
	a_{0}+a_{1}x_{j} + a_{2}x_{j}^2 +\cdots+a_{m}x_{j}^{m} = b_{0 j}
	\\
	a_{1} + 2 a_{2}x_{j} + 3a_{3}x_{j}^2 + \cdots+ m a_{m}x_{j}^{m-1} =
	b_{1j}
	\\
	\vdots
	\\
	n_j! a_{ n_{j}} x_{j}^{m -n_{j}} +
	(n_j + 1)! a_{n_{j}+1}
	x_{j}^{m - n_{j}-1} + \cdots + \frac{m!}{(m-n_j)!} a_{m}x_{j} = b_{n_j j}
\end{gather*}
ranging over $0 \le j \le k$. In matrix form, this is expressed by $A\vec{a}=B$, where
%
%
%
%
\begin{gather*}
		A= \begin{bmatrix}
		A_1
		\\
		A_2
		\\
		\vdots
		\\
		A_k
	\end{bmatrix}
	, \ A_j=\begin{bmatrix}
		1 & x_{j} & x_{j}^{2} & \cdots &\cdots &\cdots & \cdots &x_{j}^{m}\\
	0 &1 &2x_{j} & \cdots &\cdots &\cdots &\cdots &mx_{j}^{m-1}\\
	0 &0 &2 & \cdots &\cdots &\cdots &\cdots &m(m-1)x_{j}^{m-2}\\
	\vdots & \vdots &\vdots &\ddots &\\
	0 &0 &0 &\cdots & n_j! \  & (n_j + 1)! x_j & \cdots &\frac{m!}{(m-n_{j})!} x_j^{m -n_{j}}
\end{bmatrix},
\\
	 \vec{a}=\begin{bmatrix}
		a_{0}\\
	a_{2}\\
	\vdots\\
	a_{m}
\end{bmatrix}
\\
 B = \begin{bmatrix}
	B_1\\
	B_2\\
	\vdots \\
	B_k
\end{bmatrix}
, \ B_j=
	\begin{bmatrix}
	b_{0j}\\
	b_{1j}\\
	\vdots \\
	b_{n_j j}.
\end{bmatrix}
\end{gather*}
%
%
%
%
Note that $A$ is a square $(m+1) \times (m+1)$ matrix. Therefore, to complete
the proof it will be enough to show that $A$ is invertible, or, equivalently,
that if $A \vec{a}=\vec{0}$, then $\vec{a} = \vec{0}$. Proceeding, suppose
$A \vec{a}=\vec{0}$. Then $\vec{a}$ defines a polynomial $p(x)$ with root $x_j$ of
multiplicity $n_j +1$. Hence,
%
%
\begin{equation*}
	\begin{split}
		p(x) = q(x) \prod_{j=0}^{k}(x-x_{j})^{n_{j}+1}.
	\end{split}
\end{equation*}
%
%
However, note that $\prod_{j=0}^{k}(x-x_{j})^{n_{j}+1}$ is of order $m+1$,
whereas the coefficients of $\vec{a}$ define $p(x)$ to be of order at most $m$.
It follows that $q \equiv 0$. \qquad \qedsymbol
%
%
\section{Finite Difference Method}
\subsection{Burgers Equation}
Recall the Burgers initial value problem (ivp)
%
%
\begin{gather}
	\label{burgers-eqn}
	u_{t} = \frac{1}{2}(u^{2})_{x},
	\\
	\label{burgers-init-data}
	u(x, 0) = u_{0}(x), \qquad x, t \in \rr.
\end{gather}
%
%
Note that for small \emph{stepsize} (or \emph{mesh}) $h, k >0$, 
%
%
\begin{equation*}
	\begin{split}
		 u_{t}(x,t) &\approx \frac{u(x, t+k) - u(x, t)}{k},
		\\
		 u_{x}(x, t) & \approx \frac{u(x+h, t) - u(x, t)}{k}.
	\end{split}
\end{equation*}
%
%
Set
\begin{gather*}
	x_i = ih, \quad i \in \mathbb{Z}
	\\
	t_{j}=jk, \quad j \in \mathbb{N}
\end{gather*}
and let
\begin{gather*}
	u_{i,j} = u(x_{i}, t_{j}).
\end{gather*}
%
%
Then the discretized Burgers ivp takes the form
%
%
\begin{gather*}
	\frac{u_{i, j+1}- u_{i,j}}{k}=\frac{1}{2h}\left( u_{i+1,j}^{2} -
	u_{i,j}^{2} \right),
	\\
	u_{i,0} = u_{0}(ih)
\end{gather*}
%
%
or
\begin{gather}
	\label{burgers-discrete}
	u_{i, j+1}=u_{i,j} + \frac{k}{2h}\left( u_{i+1,j}^{2} -
	u_{i,j}^{2} \right),
	\\
	\label{burgers-discrete-init}
	u_{i,0} = u_{0}(ih).
\end{gather}
Note that \eqref{burgers-discrete}-\eqref{burgers-discrete-init} gives us an
explicit numerical solution to the Burgers ivp
\eqref{burgers-eqn}-\eqref{burgers-init-data}. To illustrate this, we set
$j=0$, and obtain
%
%
\begin{equation}
	\label{case-j=0}
	\begin{split}
		u_{i,1} = \underbrace{u_{i,0} + \frac{k}{2h}\left( u_{i+1,0}^{2} -
		u_{i,0}^{2} \right)}_{\text{known from initial data
		\eqref{burgers-discrete-init}}}.
	\end{split}
\end{equation}
%
%
Similarly, for $j=1$, we have
%
%
\begin{equation*}
	\begin{split}
		u_{i,2} = \underbrace{u_{i,1}}_{\text{known from
		\eqref{case-j=0}}} + \frac{k}{2h}\left(
		\underbrace{u_{i+1,1}^{2}}_{?} -
		\underbrace{u_{i,1}^{2}}_{\text{known from
		\eqref{case-j=0}}} \right).
	\end{split}
\end{equation*}
%
%
But \eqref{case-j=0} implies that
\begin{equation*}
	\begin{split}
		u_{i+1,1} = \underbrace{u_{i+1,0} + \frac{k}{2h}\left( u_{i+2,0}^{2} -
		u_{i+1,0}^{2} \right)}_{\text{known from \eqref{burgers-discrete-init}}}.
	\end{split}
\end{equation*}
Hence, \eqref{burgers-discrete-init} and \eqref{case-j=0} give us the value of $u_{i,2}$. This process can be
continued indefinitely to find $u_{i, j}$ for any $j \in \mathbb{N}$, and is
called an \emph{explicit finite difference method} for numerically solving the
Burgers ivp. An important drawback to this method is that it converges very
slowly to solutions of the Burgers ivp. 
%
%
%\nocite{*}
%\bibliography{/Users/davidkarapetyan/Documents/math/bib-files/numerics}
% \bib, bibdiv, biblist are defined by the amsrefs package.
\begin{bibdiv}
\begin{biblist}

\bib{Kincaid-Cheney-1996-Numerical-analysis}{book}{
      author={Kincaid, David},
      author={Cheney, Ward},
       title={Numerical analysis},
     edition={Second},
   publisher={Brooks/Cole Publishing Co.},
     address={Pacific Grove, CA},
        date={1996},
        ISBN={0-534-33892-5},
        note={Mathematics of scientific computing},
      review={\MR{MR1388777 (97g:65003)}},
}

\bib{LeVeque-1992-Numerical-methods-for-conservation}{book}{
      author={LeVeque, Randall~J.},
       title={Numerical methods for conservation laws},
     edition={Second},
      series={Lectures in Mathematics ETH Z{\"u}rich},
   publisher={Birkh{\"a}user Verlag},
     address={Basel},
        date={1992},
        ISBN={3-7643-2723-5},
      review={\MR{MR1153252 (92m:65106)}},
}

\end{biblist}
\end{bibdiv}
									\end{document}  

