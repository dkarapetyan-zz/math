\chapter{Well Posedness for the KDV}
%
%
%
%
%
\section{Introduction}
We consider the Korteweg-de Vries (KDV) initial value problem (ivp)
%
%
\begin{gather}
	\label{2KDV-eq}
	\p_t u + \p_x^{3} u + u \p_x u = 0,
	\\
	\label{2KDV-init-data}
	u(x,0) = u_0(x), \quad x \in \ci, \ t \in \rr.
\end{gather}
%
%
%
%
\begin{definition}
	We say that the KDV ivp \eqref{2KDV-eq}-\eqref{2KDV-init-data} is
	\emph{locally well posed} in
	$X$ if 
	\begin{enumerate}
		\item For every $\vp(x) \in
	B_R$ there exists $T>0$ depending on $R$ and a unique function
	\\
	$u \in C([-T, T],
	X)$ satisfying \eqref{2KDV-eq} for all $t \in [-T, T]$. 
\item The flow map $u_0 \mapsto u(t)$ is locally uniformly continuous. That is, if $u_0
	\in B_R$, $\{u_{0,n}\} \subset B_R$, and 
	$\|u_0 - u_{0, n} \|_{H^{s}(\ci)} \to 0$, then there exists $T >0$ depending
	on $R$ such that $\|u(\cdot, t) - u_{n}(\cdot,t) \|_{X} \to
	0$ for $t \in [-T, T]$. 
	\end{enumerate}
	Otherwise, we say that the KDV ivp is \emph{ill-posed}.
\end{definition}
%
%
We are now prepared to state the following result.

%%%%%%%%%%%%%%%%%%%%%%%%%%%%%%%%%%%%%%%%%%%%%%%%%%%%%
%
%
%				 Well Posedness Theorem
%
%
%%%%%%%%%%%%%%%%%%%%%%%%%%%%%%%%%%%%%%%%%%%%%%%%%%%%%
%
%
\begin{theorem}
	\label{2thm:main}
	The KDV is well-posed in $\dot{H}^s(\ci)$ for $s \ge -1/2$.  
\end{theorem}
%
%
%%%%%%%%%%%%%%%%%%%%%%%%%%%%%%%%%%%%%%%%%%%%%%%%%%%%%
%
%
%				Outline
%
%
%%%%%%%%%%%%%%%%%%%%%%%%%%%%%%%%%%%%%%%%%%%%%%%%%%%%%
%
%
\section{Outline of the Proof of Main Theorem}
%
%
%
%
%
We first derive a weak formulation of the KDV ivp. 
Let $\ci = [0, 2 \pi]$, and use
the following notation for the Fourier transform
%
%
%
%
\begin{equation*}
	\begin{split}
		\widehat{f}(n) = \int_{\ci} e^{-ix n} f(x) \, dx.
	\end{split}
\end{equation*}
Let $w(x,t) = u \p_x u$. Applying 
the Fourier transform to the KDV ivp in the space variable we obtain 
%
%
\begin{gather}
	\label{2four-trans-pde}
	\p_t \widehat{u}(n, t) = -i n^3 \widehat{u}(n, t) + \lambda i  
	\widehat{w} (n, t),
	\\
	\notag
	\widehat{u} (n,0) = \widehat{\vp}(n)
\end{gather}
%
%
which is a globally well-defined relation in $t$ 
and $n$. Multiplying \eqref{2four-trans-pde} by the integrating factor $e^{itn^3}$ then yields
%%
%%
\begin{equation*}
	\begin{split}
		\left[e^{it n^3} \widehat{u}(n) \right]_t = i
		 e^{it n^3} \widehat{w} (n, t).	
	\end{split}
\end{equation*}
%
%
Integrating from $0$ to $t$, we obtain
%
%
\begin{equation*}
	\begin{split}
		\wh{u}(n, t) = \wh{\vp}(n) e^{- it n^3} + i  
		\int_0^t e^{i(t' - t) n^3} \wh{w}(n, t') \ 
		dt'.
	\end{split}
\end{equation*}
%
%
Therefore, by Fourier inversion 
%
%
\begin{equation}
	\label{2KDV-integral-form}
	\begin{split}
		u(x,t) & = \sum_{n \in \zz} \wh{\vp}(n) e^{i\left( xn - t n^3 
		\right)} 
		\\
		& + i \sum_{n \in \zz} \int_0^t e^{i\left[ xn + \left( t' - t 
		\right) n^3 \right]} \wh{w}(n, t') \ dt'.
	\end{split}
\end{equation}
%
%
Then it is immediate that \eqref{2KDV-integral-form} is a weaker 
restatement of the Cauchy-problem \eqref{2KDV-eq}-\eqref{2KDV-init-data}, 
since by construction any classical solution of the KDV 
ivp is a solution to \eqref{2KDV-integral-form}. 
\\
\\
%
%
We now derive an integral 
equation global in $t$ and equivalent to \eqref{2KDV-integral-form} for $t 
\in [-T, T]$. Let $\psi(t)$ be a cutoff function symmetric about the 
origin such that $\psi(t) = 1$ for $|t| \le T$ and $\text{supp} \, \psi 
= [-2T, 2T ]$. Multiplying the right hand side of expression
$\eqref{2KDV-integral-form}$ by $\psi(t)$, we obtain
%
%
\begin{equation}
	\begin{split}
		\label{2cutoff-int-eq}
		u(x, t)
		& = \frac{1}{2 \pi} \psi(t) \sum_{n \in \zz} e^{i(xn - t n^3)} \widehat{\vp}(n) 
		\\
		& + \frac{i }{2 \pi} \psi(t) \int_0^t \sum_{n \in \zz} 
		e^{i\left[ xn + (t - t')n^3 \right]} \wh{w}(n, t') \ dt'.
	\end{split}
\end{equation}
%
%
Noting that $e^{i\left( xn + tn^3 \right)}$ 
does not depend on $t'$, we may rewrite
%
%
\begin{equation}
	\label{2pre-prim-int-form}
	\begin{split}
		& \frac{i }{2 \pi} \psi(t) \int_0^t \sum_{n \in \zz} 
		e^{i\left[ xn + (t - t') n^3 \right]} \wh{w}(n, t') \ dt'
		\\
		& = \frac{i}{2 \pi} \psi(t) \sum_{n \in \zz} e^{i\left( xn + t 
		 n^3 
		\right)} \int_0^t e^{- it'n^3} \wh{w}(n, t') \ dt'.
	\end{split}
\end{equation}
%%
%%
We remark that this is a \emph{global} relation in $t$. Therefore, by Fourier 
inversion
%
%
%
%
%
%
%
\begin{equation*}
	\begin{split}
		\text{rhs of} \; \eqref{2pre-prim-int-form}
		& = \frac{i}{4 \pi^2} \psi (t) \sum_{n \in \zz} e^{i\left( xn + t 
		 n^3
		\right)} \int_0^t \int_\rr e^{it'\left( \tau - n^3 \right) }
		\wh{w}(n, \tau) d \tau dt'
		\\
		& = \frac{i}{4 \pi^2} \psi(t) \sum_{n \in \zz} \int_\rr 
		e^{i\left( xn + tn^3 \right)} \frac{e^{it\left( \tau - n^3 
		\right)}-1}{\tau - n^3} \wh{w}(n, \tau) d \tau
	\end{split}
\end{equation*}
%
%
where the last step follows from integration. Substituting
into \eqref{2cutoff-int-eq} we obtain
%
%
\begin{equation}
	\begin{split}
		\label{2cutoff-int-eq-2}
		u(x, t)
		& = \frac{1}{2 \pi} \psi(t) \sum_{n \in \zz} e^{i(xn - tn^3)} \widehat{\vp}(n) 
		\\
		& + \frac{1}{4 \pi^2} \psi(t) \sum_{n \in \zz} \int_\rr
		e^{i(xn + t n^3)} \frac{e^{it(\tau - n^3)}- 1}{\tau - n^3} 
		\wh{w}(n, \tau) \ d \tau.
	\end{split}
\end{equation}
%
%
%
%
%
Next, we localize near the singular curve $\tau =  n^3$.  Multiplying the
summand of the second term of \eqref{2cutoff-int-eq-2} by $1 + \psi(\tau -
n^3) - \psi(\tau -
n^3) $ and
rearranging terms, we have
%
%
\begin{equation*}
	\begin{split}
		 u(x, t)
		& = \frac{1}{2 \pi} \psi(t) \sum_{n \in \zz} e^{i(xn + t n^{m 
		})} \widehat{\vp}(n) 
		\\
		& + \frac{1}{4 \pi^2} \psi(t) \sum_{n \in \zz} \int_\rr e^{ixn}  
		e^{it \tau} \frac{1 - \psi(\tau - n^3) 
		}{\tau - n^3} \wh{w}(n, \tau) \ d \tau
		\\
		& - \frac{1}{4 \pi^2} \psi(t) \sum_{n \in \zz} \int _\rr e^{i(xn + 
		t n^3)}
		 \frac{1- \psi(\tau - n^3)}{\tau - n^3} \wh{w}(n, \tau) \ d \tau
		\\
		& + \frac{1}{4 \pi^2} \psi(t) \sum_{n \in \zz} \int_\rr
		e^{i(xn + t n^3)}
		\frac{\psi(\tau - n^3)\left[ e^{it(\tau - n^3)}-1 
		\right]}{\tau - n^3} \wh{w}(n, \tau) \ d \tau
	\end{split}
\end{equation*}
%
%
which by a power series expansion of $[e^{it(\tau - n^3)}-1]$ simplifies  
to
%
%
\begin{align}
	\label{2main-int-expression-0}
	& u(x, t) 
		\\
		\label{2main-int-expression-1}
		& = \frac{1}{2 \pi} \psi(t) \sum_{n \in \zz} e^{i(xn + tn^{m 
		})} \widehat{\vp}(n) 
		\\
		\label{2main-int-expression-2}
		& + \frac{1}{4 \pi^2} \psi(t) \sum_{n\in \zz} \int_\rr e^{ixn}  
		e^{it \tau} \frac{1 - \psi(\tau -  n^3) 
		}{\tau -  n^3} \wh{w}(n, \tau) \ d \tau
		\\
		\label{2main-int-expression-3}
		& - \frac{1}{4 \pi^2} \psi(t) \sum_{n\in \zz} \int_\rr e^{i(xn + 
		t n^3)}
		 \frac{1- \psi(\tau -  n^3)}{\tau -  n^3} \wh{w}(n, \tau) \ d \tau
		\\
		\label{2main-int-expression-4}
		& + \frac{1}{4 \pi^2} \psi(t) \sum_{k \ge 1} \frac{i^k t^k}{k!}
		\sum_{n \in \zz} \int_\rr e^{i(xn + t n^3 )}
		\psi(\tau -  n^3) (\tau -  n^3)^{k-1} \wh{w}(n, \tau)  
		\\
		& \doteq T(u) \notag
\end{align}
%
%
where $T = T_{\vp}$. We now introduce the following spaces. 

\begin{definition}
	Denote $\dot{Y}^s$ to be the space of all
	functions $u$ on $\ci \times \rr$ with
	bounded norm
\begin{equation}
	\label{2Y-s-norm}
	\begin{split}
		\|u\|_{\dot{Y}^s} = \|u\|_{\dot{X}^s} + \|n^s \wh{u}\|_{\dot{\ell}^2_n L^1_\tau }
	\end{split}
\end{equation}
%
%
%
%
where
%
\begin{equation}
	\label{2Xs-norm}
	\begin{split}
		& \|u\|_{\dot{X}^s}
		= \left ( \sum_{n\in \zz} |n|^{2s} \int_\rr \left ( 1 + | 
		\tau - n^3 \right ) | \wh{u} ( n, \tau ) |^2
		\right )^{1/2}
	\end{split}
\end{equation}
and
%
%
\begin{equation}
	\label{2E-norm}
	\|n^s \wh{u}\|_{\dot{\ell}^2_n L^1_\tau } = \left[ \sum_{n \in \zzdot}| n |^{2s} \left(
	\int_{\rr}| \wh{u}(n, \tau) |d \tau \right)^{2} \right]^{1/2}.
\end{equation}
%
%
%
%
\end{definition}
The $\dot{Y}^s$ spaces have the following important property, whose proof
is provided in the appendix.
\begin{lemma}
	\label{2lem:cutoff-loc-soln}
	Let $\psi(t)$ be a smooth cutoff function with $\psi(t) =1$ for $t \in [-T, T]$. If
	$\psi(t)u(x,t) \in \dot{Y}^s$, then $u \in C([-T, T], \dot{H}^s(\ci))$.
\end{lemma}
%
%
We will 
show that for initial data $\vp \in \dot{H}^s(\ci)$, $T$ is a contraction on $B_M 
\subset \dot{Y}^s$, where $B_M$ is the ball centered at the origin of radius $M = 
M_{\vp}> 0$, by estimating the $\dot{Y}^s$
norm of \eqref{2main-int-expression-1}-\eqref{2main-int-expression-4}. The 
Picard fixed point theorem will
then yield a unique solution to
\eqref{2main-int-expression-0}-\eqref{2main-int-expression-4}. An application of
\cref{2lem:cutoff-loc-soln} will then imply the existence of a unique, local
solution $u \in C([-T, T], \dot{H}^s(\ci))$ to the KDV ivp which coincides with the solution to
\eqref{2main-int-expression-0}-\eqref{2main-int-expression-4} on the interval $[-T, T]$. Local Lipschitz continuity of the flow map will follow
from estimates used to establish the contraction mapping. %
%
%%%%%%%%%%%%%%%%%%%%%%%%%%%%%%%%%%%%%%%%%%%%%%%%%%%%%
%
%
%			Proof of Theorem	
%
%
%%%%%%%%%%%%%%%%%%%%%%%%%%%%%%%%%%%%%%%%%%%%%%%%%%%%%
%
%
\section{Proof of Main Theorem}
%
%
%
%%%%%%%%%%%%%%%%%%%%%%%%%%%%%%%%%%%%%%%%%%%%%%%%%%%%%
%
%
%		Estimation of Integral Equality Part 1		
%
%
%%%%%%%%%%%%%%%%%%%%%%%%%%%%%%%%%%%%%%%%%%%%%%%%%%%%%
%
%
%
%
\subsection{Estimate for \ref{2main-int-expression-1}.}
%
%
Letting $f(x,t) = \psi(t) \sum_{n \in \zz} e^{i(xn + tn^{3})} 
\wh{\vp}(n)$, we have $\wh{f}(n,t) = \psi(t) \wh{\vp}(n) e^{itn^{3}}$,
from which we obtain
%
%
\begin{equation}
	\label{2fourier-trans-calc}
	\begin{split}
		\wh{f}(n, \tau)
		& = \wh{\vp}(n) \int_\rr e^{-it( \tau - n^{3})} 
		\psi(t) \ d t
		= \wh{\psi}(\tau - n^{3}) \wh{\vp}(n).
	\end{split}
\end{equation}
%
%
%
%
%
%
Since $\wh{\psi}(\xi)$ is Schwartz for $|\xi| \ge T$, we see that 
%
%
\begin{equation}
	\begin{split}
	\label{2main-int1-est}
		\|\eqref{2main-int-expression-1}\|_{\dot{Y}^s}
		& = \left (  \sum_{n\in \zz} |n|^s \int_\rr \left( 1 + | \tau - n^{3} 
		| \right )
		| \wh{\psi}(\tau - n^{3}) \wh{\vp}(n) |^2 d \tau \right)^{1/2} 
		\\
		& + \left[ \sum_{n \in \zz }\left( 1 + | n | \right)^{2s} \left( \int_{\rr} |
		\wh{\psi}(\tau - n^{3})\wh{\vp}(n) | d \tau
		\right)^{2} \right]^{1/2}
		\\
		& \le c_{\psi}
		\|\vp\|_{\dot{H}^s(\ci)}.
	\end{split}
\end{equation}
%
%
%
%
\subsection{Estimate for \ref{2main-int-expression-2}.}
We now need the following lemma, whose proof is provided in the appendix.
%
%
%%%%%%%%%%%%%%%%%%%%%%%%%%%%%%%%%%%%%%%%%%%%%%%%%%%%%
%
%
%			Schwartz Multiplier	
%
%
%%%%%%%%%%%%%%%%%%%%%%%%%%%%%%%%%%%%%%%%%%%%%%%%%%%%%
%
%
\begin{lemma}
\label{2lem:schwartz-mult}
	For $\psi \in S(\rr)$,
%
%
\begin{equation}
	\label{2schwartz-mult}
	\begin{split}
		\|\psi f \|_{\dot{X}^s} \le c_{\psi} \|f \|_{\dot{X}^s}.
	\end{split}
\end{equation}
%
%
\end{lemma}
%
%
Hence,
%
%
\begin{equation}
	\label{2main-int2-est-X-s-part}
	\begin{split}
		\|\eqref{2main-int-expression-2}\|_{\dot{X}^s} 
		& \lesssim 
		\left( \| \sum_{n \in \zz} e^{ixn} \int_\rr 
		e^{it \tau} \frac{1 - \psi (\tau - n^{3} ) 
		}{\tau - n^{3}} \wh{w}(n, \tau) \ 
		d \tau\|_{\dot{X}^s} \right)^{1/2}
		\\
		& =  \left( \sum_{n \in \zz} |n|^{2s} \int_\rr
		(1 + |\tau - n^{3}|) \left | \frac{1 - \psi(\tau - n^{2 
		})}{\tau - n^{3}} 
		\wh{w}(n, \tau) \right |^2 \ d 
		\tau \right)^{1/2}
		\\
		& \le \left( \sum_{n \in \zz} |n|^{2s} \int_{| \tau - n^3| \ge 1}
		(1 + |\tau - n^{3}|) \frac{|\wh{w}(n, \tau)|^2 }{|\tau - n^3|^2} 
		\ d 
		\tau \right)^{1/2}
		\\
		& \lesssim  \left( \sum_{n \in 
		\zz} |n|^{2s} \int_\rr
		\frac{|\wh{w}(n, \tau) |^2}{1+ |\tau - 
		n^{3}|} 
		 \ d \tau 
		\right)^{1/2}
		\\
		& \lesssim  \|u\|_{\dot{X}^s}^3
	\end{split}
\end{equation}
%
%
where the last two steps follow from the inequality 
%
\begin{equation}
	\label{2one-plus-ineq}
	\begin{split}
		\frac{1}{|\tau - n^{3}| } \le \frac{2}{1 + |\tau - n^{3}| }, 
		\qquad |\tau - n^{3}| \ge 1
	\end{split}
\end{equation}
%
%
and the following bilinear estimate, whose proof we leave for later.
%
%%%%%%%%%%%%%%%%%%%%%%%%%%%%%%%%%%%%%%%%%%%%%%%%%%%%%
%
%
%				 Bilinear Estimates
%
%
%%%%%%%%%%%%%%%%%%%%%%%%%%%%%%%%%%%%%%%%%%%%%%%%%%%%%
%
%
\begin{proposition}
	\label{2prop:prim-bilin-est}
	For any $s \ge -1/2$ we have
	\begin{equation}
		\label{2prim-bilin-est}
		\left( \sum_{n \in \dot{\zz}} |n|^{2s} \int_\rr
		\frac{|\wh{w_{fg}}(n, \tau) |^2}{1+ |\tau - 
		n^{3}| } 
		 \ d \tau 
		\right)^{1/2}
		\lesssim \|f\|_{\dot{X}^s} \|g\|_{\dot{X}^s}
	\end{equation}
	where $w_{fg}(x,t)$ = $\p_x(fg)(x,t)$.
\end{proposition}
Furthermore,
%
%
%
%
\begin{equation}
	\label{2main-int-expression-2-Y-s-part}
	\begin{split}
		\|\wh{\eqref{2main-int-expression-2}} \|_{\dot{\ell}^2_n L^1_\tau}
		& \lesssim \left( \| \sum_{n \in \zz} e^{ixn} \int_\rr 
		e^{it \tau} \frac{1 - \psi (\tau - n^{3} ) 
		}{\tau - n^{3}} \wh{w}(n, \tau) \ 
		d \tau\|_{\dot{\ell}^2_n L^1_\tau} \right)^{1/2}
		\\
		& = \left[ \sum_{n \in \zz}|n|^{2s} \left(
		\int_{\rr}\frac{1 - \psi(\tau - n^{3})}{\tau - n^{3}} \wh{w}(n, \tau) d
		\tau \right)^{2} \right]^{1/2}
		\\
		& \lesssim \|f\|_{X^s} \|g\|_{X^s}
	\end{split}
\end{equation}
%
%
where the last step follows from the following bilinear estimate.
%
%%%%%%%%%%%%%%%%%%%%%%%%%%%%%%%%%%%%%%%%%%%%%%%%%%%%%
%
%
%				Second trilinear Estimate 
%
%
%%%%%%%%%%%%%%%%%%%%%%%%%%%%%%%%%%%%%%%%%%%%%%%%%%%%%
%
%
\begin{proposition}
\label{2prop:bilinear-estimate2}
For any $s \ge -1/2$ we have
%
%
\begin{equation}
	\label{2bilinear-estimate2}
	\begin{split}
		\left( \sum_{n \in \zzdot} |n|^{2s}  \left ( \int_\rr 
		\frac{|\wh{w_{fg}}(n, \tau) |}{1 + | \tau - n^3 |}
		 \ d\tau \right)^2  \right)^{1/2} \lesssim \|f\|_{\dot{X}^s} \|g\|_{\dot{X}^s}.
	\end{split}
\end{equation}
\end{proposition}
%
%
Combining \eqref{2main-int2-est-X-s-part} and
\eqref{2main-int-expression-2-Y-s-part}, we conclude that
%
%
%
%
\begin{equation}
	\label{2main-int2-est}
	\begin{split}
		\|\eqref{2main-int-expression-2}\|_{\dot{Y}^s} \le c_{\psi}\|f\|_{\dot{X}^s} \|g\|_{\dot{X}^s}.
	\end{split}
\end{equation}
%
%
\subsection{Estimate for \ref{2main-int-expression-3}.}
Letting $$f(x,t) = \psi(t) \sum_{n \in \zzdot} e^{i\left( xn + tn^{3} \right)} 
\int_\rr \frac{1 - \psi\left( \lambda - n^{3} \right)}{\lambda - n^{3}} 
\wh{w} \left( n, \lambda \right) \ d \lambda,$$ we have
%
%
\begin{equation*}
	\begin{split}
		& \wh{f^x}(n, t) = \psi(t) e^{itn^{3}} \int_\rr
		\frac{1 - \psi\left( \lambda - n^{3} \right)}{\lambda - n^{3}} 
		\wh{w}(n, \lambda) \ d \lambda
	\end{split}
\end{equation*}
and
\begin{equation*}
	\begin{split}
		 \wh{f}\left( n, \tau \right)
		 & = \int_\rr e^{-it\left( \tau - n^{3} 
		\right)} \psi(t) \int_\rr \frac{1 - \psi\left( 
		\lambda - n^{3} 
		\right)}{\lambda - n^{3}} \wh{w}(n, \lambda) \ d \lambda d \tau
		\\
		& = \wh{\psi}\left( \tau - n^{3} \right) \int_\rr 
		\frac{1 - \psi\left( 
		\lambda - n^{3} 
		\right)}{\lambda - n^{3}} \wh{w}(n, \lambda) \ d \lambda.
	\end{split}
\end{equation*}
Therefore,
%
%
\begin{equation*}
	\begin{split}
		& \| \eqref{2main-int-expression-3} \|_{\dot{X}^s} 
		\\
		& = \left( \sum_{n \in \zzdot} |n|^{2s} \int_\rr \left( 1 + | \tau - n^{m
		} \right ) | | \wh{\psi}\left( \tau - n^3 \right) |^2 \ d \tau
		\right.
		\\
		& \times \left . |
		\int_\rr \frac{1 - \psi\left( \lambda - n^3 \right)}{\lambda -
		n^3} \wh{w}(n, \lambda) \ d \lambda |^2  \right)^{1/2}
		\\
		& \lesssim \left( \sum_{n \in \zzdot} |n|^{2s} | \int_\rr
		\frac{1 - \psi\left( \lambda - n^3 \right)}{\lambda - n^3}
		\wh{w}(n, \lambda) \ d\lambda |^2 \right)^{1/2}
		\\
		& \le \left( \sum_{n \in \zzdot} |n|^{2s}  \left ( \int_\rr
		\frac{1 - \psi\left( \lambda - n^3 \right)}{|\lambda - n^3|}
		|\wh{w}(n, \lambda) | \ d\lambda \right )^2 \right)^{1/2}
		\\
		& \le \left( \sum_{n \in \zzdot} |n|^{2s}  \left ( \int_{| \lambda - 
		n^3 | \ge 1}
		\frac{|\wh{w}(n, \lambda) | }{|\lambda - n^3|}
		\ d\lambda \right )^2 \right)^{1/2}.
	\end{split}
\end{equation*}
%
%
Applying estimate \eqref{2one-plus-ineq} then gives
%
%%
\begin{equation}
	\label{2main-int3-est-X-s-part}
	\begin{split}
		\| \eqref{2main-int-expression-3} \|_{\dot{X}^s}
		& \lesssim \left( \sum_{n \in \zzdot} |n|^{2s}  \left ( \int_\rr
		\frac{|\wh{w}(n, \lambda)| }{1 + |\lambda - n^3|}
		 \ d\lambda \right )^2 \right)^{1/2}
		 \\
		& \lesssim \|u\|_{\dot{X}^s}^3
	\end{split}
\end{equation}
%
%%
where the last step follows from \cref{2prop:bilinear-estimate2}.
Furthermore, 
%
%
\begin{equation}
	\label{2main-int-estimate-3-Y-s-part}
	\begin{split}
		\|\eqref{2main-int-expression-3}\|_{\dot{\ell}^2_n L^1_\tau}
		& = \left[ \sum_{n \in \zzdot} |n|^{2s} \int_{\rr} |
		\wh{\psi}(\tau - n^{3}) |^{2} \left( \int_{\rr}\frac{1 - \psi(\lambda -
		n^{3})}{\lambda - n^{3}} \wh{w}(n, \lambda) d \lambda \right)^{2} d \tau
		\right]^{1/2}
		\\
		& \le c_{\psi} \left[ \sum_{n \in \zzdot} |n|^{2s} \left(
		\int_{\rr} \frac{1 - \psi(\lambda - n^{3})}{\lambda - n^{3}}
		\wh{w}(n, \lambda) d \lambda
		\right)^{2}\right]^{1/2}
		\\
		& \le 2 c_{\psi} \left[ \sum_{n \in \zzdot} |n|^{2s} \left(
		\int_{\rr} \frac{\wh{w}(n, \lambda) }{1 + |\lambda - n^{3}|}
		d \lambda
		\right)^{2}\right]^{1/2}
		\\
		& \lesssim \|f\|_{\dot{X}^s} \|g\|_{\dot{X}^s} 
	\end{split}
\end{equation}
%
%
where the last two steps follow from \eqref{2one-plus-ineq} and
\cref{2prop:bilinear-estimate2}, respectively. Combining
\eqref{2main-int3-est-X-s-part} and \eqref{2main-int-estimate-3-Y-s-part}, we
conclude that
%
%
\begin{equation}
	\label{2main-int3-est}
	\begin{split}
		\|\eqref{2main-int-expression-3}\|_{\dot{Y}^s} 
		\lesssim \|f\|_{\dot{X}^s} \|g\|_{\dot{X}^s}.
	\end{split}
\end{equation}
%
%
%
\subsection{Estimate for \ref{2main-int-expression-4}.}
Note that
%
%
\begin{equation}
	\label{21n}
	\begin{split}
		\eqref{2main-int-expression-4} \simeq \sum_{k \ge 1}
		\frac{i^k}{k!}g_k(x,t)
	\end{split}
\end{equation}
%
%
where 
%
%
\begin{equation*}
	\begin{split}
		& g_k(x,t) = t^k \psi(t) \sum_{n \in \zzdot} e^{i\left( xn + tn^{3}
		\right)} h_k(n),
		\\
		& h_k(n) = \int_\rr \psi \left( \tau - n^3 \right) \times \left(
		\tau - n^3 \right)^{k -1} \wh{w}(n, \tau) \ d \tau.
	\end{split}
\end{equation*}
%
%
Hence
%
%
\begin{equation*}
	\begin{split}
		\wh{g_k^x}(n, t) = t^{k} \psi(t) e^{i t n^3} h_k(n)
	\end{split}
\end{equation*}
%
%
which gives
%
%
\begin{equation*}
	\begin{split}
		\wh{g_k}(n, \tau)
		& = h_k(n) \int_\rr e^{-it\left( \tau - n^3 \right)}
		t^{k}\psi(t) \ dt
		\\
		& = h_k(n) \wh{t^{k}\psi(t)} \left( \tau - n^3 \right).
	\end{split}
\end{equation*}
%
%
Applying this to \eqref{21n}, we obtain
%
%
\begin{equation}
	\label{22n}
	\begin{split}
		\|\eqref{2main-int-expression-4}\|_{\dot{X}^s} 
		& \simeq \left( \sum_{n \in \zzdot} |n|^{2s} \int_\rr \left( 1 + | \tau -
		n^3
		|
		\right) | \wh{\sum_{k \ge 1} \frac{i^k}{k!}g_k(x,t)} |^2 \ d \tau
		\right)^{1/2}
		\\
		& \le \sum_{k \ge 1} \frac{1}{k!}\left( \sum_{n \in \zzdot} |n|^{2s}
		\int_\rr \left( 1 + | \tau - n^3 | \right) | \wh{g_k}(n, \tau) |^2 \
		d \tau \right)^{1/2}
		\\
		& = \sum_{k \ge 1} \frac{1}{k!} \left( \sum_{n \in \zzdot} |n|^{2s}
		\int_\rr \left( 1 + | \tau - n^3 | \right) | h_k(n) \wh{t^k
		\psi(t)} \left( \tau - n^3 \right) |^2 \ d \tau \right)^{1/2}
		\\
		& = \sum_{k \ge 1} \frac{1}{k!} \left( \sum_{n \in \zzdot} |n|^{2s} |
		h_k(n) |^2 \int_\rr \left( 1 + | \tau - n^3 | \right) | \wh{t^k
		\psi(t)} \left( \tau - n^3 \right) |^2 \ d \tau \right)^{1/2}.
	\end{split}
\end{equation}
%
%
Notice that for fixed $n$, the change of variable $\tau - n^3 \to \tau'$
gives
%
%
\begin{equation}
	\label{23n}
	\begin{split}
		\int_\rr \left( 1 + | \tau - n^3 | \right) | \wh{t^{k}
		\psi(t)}\left( \tau - n^3 \right) |^2 \ d \tau
		& = \int_\rr \left( 1 + |\tau'| \right) | \wh{t^k \psi(t)}(\tau') |^2 \
		d \tau'
		\\
		& \le \int_\rr \left( 1 + |\tau'| \right)^2 | \wh{t^k \psi(t)}(\tau')
		|^2 \ d \tau'
		\\
		& \lesssim \int_\rr \left( 1 + | \tau' |^2 \right) | \wh{t^{k}
		\psi(t)}(\tau') |^2 \ d \tau'
		\\
		& = \|t^k \psi(t) \|_{H^1(\rr)}^2.
	\end{split}
\end{equation}
%
%
But
%
%
\begin{equation}
	\label{24n}
	\begin{split}
		\|t^k \psi(t) \|_{H^1(\rr)}^2
		& = \left( \|t^k \psi(t)\|_{L^2(\rr)} + \|\p_t \left( t^k \psi(t)
		\right)\|_{L^2(\rr)} \right)^2
		\\
		& \lesssim \|t^{k}\psi(t) \|_{L^2(\rr)}^2 + \|\p_t \left (t^{k}
		\psi(t) \right )\|_{L^2(\rr)}^2
		\\
		& \le \|t^k \psi(t) \|_{L^2(\rr)}^2 + \|t^k \p_t \psi(t)
		\|_{L^2(\rr)}^2 + \|k t^{k -1} \psi(t) \|_{L^2(\rr)}^2
		\\
		& = c_{\psi} + c_{\psi}' + k^2 c_{\psi}''
		\\
		& \lesssim k^2.
	\end{split}
\end{equation}
%
%
Hence, applying \eqref{23n} and \eqref{24n} to \eqref{22n}, we obtain
%
%%
\begin{equation}
	\label{25n}
	\begin{split}
		\|\eqref{2main-int-expression-4} \|_{\dot{X}^s}
		& \lesssim
		\sum_{k \ge 1} \frac{k}{k!} \left( \sum_{n \in \zzdot} |n|^{2s} | h_k(n) |^2 
		\right)^{1/2}
		\\
		& \le \sum_{k \ge 1} \frac{k}{k!}
		 \sup_{k \ge 1} \left( \sum_{n \in \zzdot} |n|^{2s} | 
		h_k(n) |^2 \right)^{1/2}
		\\
		& = \sum_{k \ge 1} \frac{k}{k!}  \sup_{k \ge 1} 
		\left( \sum_{n \in \zzdot} |n|^{2s} \int_\rr 
		\psi\left( \tau - n^3 \right) \cdot \left( \tau - n^3 
		\right)^{k -1} \wh{w}(n, \tau) \ d \tau \right)^{1/2}.
	\end{split}
\end{equation}
%
%%
Recall that $\text{supp} \, |\psi| \subset [0, T ]$. Pick $T \le 1$. 
Then $| \psi\left( \tau - n^3 \right) \cdot \left( \tau - n^3 \right)^{k 
-1} | \le \chi_{| \tau - n^3 | \le 1}$ for all $k \ge 1$. Hence, \eqref{25n} gives
%
%%
\begin{equation*}
	\begin{split}
		\|\eqref{2main-int-expression-4} \|_{\dot{X}^s} 
		& \lesssim \sum_{k \ge 1} \frac{k}{k!}  \left( \sum_{n \in \zzdot} | 
		\int_{| \tau - n^{3}  |\le 1} | \wh{w}(n, \tau) \ d \tau |^2 
		\right)^{1/2}
	\end{split}
\end{equation*}
%
%%
which by the inequality
%
%%
\begin{equation*}
	\begin{split}
		\frac{1 + | \tau - n^3 |}{1 + | \tau  - n^3 |} \le 
		\frac{2}{1 + | \tau - n^3 |}, \qquad | \tau - n^3  | \le 1
	\end{split}
\end{equation*}
%
%%
implies
%
%%
\begin{equation}
\label{2main-int4-est-X-s-part}
	\begin{split}
		\|\eqref{2main-int-expression-4}\|_{\dot{X}^s}
		& \lesssim \left( \sum_{n \in \zzdot} | \int_{| \tau - n^{3}| \le 1 }
		\frac{\wh{w}(n, \tau)}{1 + | \tau - n^3 |} \ d \tau |^2 
		\right)^{1/2}
		\\
		& \le \left( \sum_{n \in \zzdot} | \int_\rr
		\frac{\wh{w}(n, \tau)}{1 + | \tau - n^3 |} \ d \tau |^2 
		\right)^{1/2} \\
		& \le \left( \sum_{n \in \zzdot} \left( \int_\rr 
		\frac{|\wh{w}(n, \tau)|}{1 + | \tau - n^3 |}  \ d \tau  \right)^2
		\right)^{1/2} \\
		& \lesssim \|u\|_{\dot{X}^s}^3
	\end{split}
\end{equation}
%
%%
where the last step follows from \cref{2prop:bilinear-estimate2}. Similarly,
we have
%
%
\begin{equation}
\label{2main-int4-est-Y-s-part}
	\begin{split}
		\|\eqref{2main-int-expression-4}\|_{\dot{\ell}^2_n L^1_\tau}
		& \simeq \left[ \sum_{n \in
		\zzdot}|n|^{2s} \left( \int_{\rr} | \sum_{k \ge 1}
		\wh{\frac{i^{k}}{k!}g_{k}(x,t)(n, \tau)} |d \tau \right)^{2} \right]^{1/2}
		\\
		& \le \sum_{k \ge 1} \frac{1}{k!} \left[ \sum_{n \in \zzdot} (1 + | n
		|)^{2s} \left( \int_{\rr} | \wh{g}(n, \tau) | d \tau \right)^{2}
		\right]^{1/2}
		\\
		& = \sum_{k \ge 1} \frac{1}{k!} \left[ \sum_{n \in \zzdot} (1 + | n
		|)^{2s} | h_{k}(n) |^2 \left( \int_{\rr} | \wh{t^{k} \psi(t)}(\tau -
		n^{3}) |d \tau \right)^{2} \right]^{1/2}
		\\
		& = c_{\psi} \sum_{k \ge 1} \frac{1}{k!} \left[ \sum_{n \in \zzdot} (1 + | n
		|)^{2s} | h_{k}(n) |^2 \right]^{1/2}
		\\
		& \lesssim \|u\|_{\dot{X}^s}^{3}
	\end{split}
\end{equation}
%
%
where the last step follows from the computations starting from \eqref{25n}
through \eqref{2main-int4-est-X-s-part}.
Combining \eqref{2main-int4-est-X-s-part} and \eqref{2main-int4-est-Y-s-part}, we
have
%
%
\begin{equation}
\label{2main-int4-est}
	\begin{split}
		\|\eqref{2main-int-expression-4}\|_{\dot{Y}^s} \lesssim \|u\|_{\dot{X}^s}^{3}.
	\end{split}
\end{equation}
%
%
Collecting estimates \eqref{2main-int1-est}, \eqref{2main-int2-est}, 
\eqref{2main-int3-est}, and \eqref{2main-int4-est}, and recalling 
\eqref{2main-int-expression-1}-\eqref{2main-int-expression-4}, we see that
$$\|Tu\|_{\dot{Y}^s} \le c_\psi \left( \|\vp \|_{\dot{H}^s(\ci)} + \|u\|_{\dot{X}^s}^3 \right )$$ 
which by the inequality $\|u\|_{\dot{X}^s} \le \|u\|_{\dot{Y}^s}$ yields the following.
%%
%%%%%%%%%%%%%%%%%%%%%%%%%%%%%%%%%%%%%%%%%%%%%%%%%%%%%
%
%% Contraction Proposition
%				 
%%%%%%%%%%%%%%%%%%%%%%%%%%%%%%%%%%%%%%%%%%%%%%%%%%%%%%
%%
%%
%
\begin{proposition}
\label{2prop:contraction}
Let $s \ge -1/2$. Then
%
%%
\begin{equation*}
	\begin{split}
		\|Tu\|_{\dot{Y}^s} \le c_\psi \left( \|\vp \|_{\dot{H}^s(\ci)} + \|u\|_{\dot{Y}^s}^3 
		\right).
	\end{split}
\end{equation*}
%
%%
\end{proposition}
We will now use \cref{2prop:contraction} to prove local well-posedness for the 
KDV ivp. Let $c = c_{\psi}^{1/2}$. For given $\vp$, we may choose $\psi$ such
that 
%
%%
\begin{equation*}
	\begin{split}
		\|\vp\|_{\dot{H}^s(\ci)} \le \frac{15}{64c^3}.
	\end{split}
\end{equation*}
%
%%
Then if $\|u\|_{\dot{Y}^s} \le \frac{1}{4c}$, we have
%
%%
\begin{equation*}
	\begin{split}
		\|T u \|_{\dot{Y}^s} 
		& \le c^2 \left[ \frac{15}{64c^3} + \left( 
		\frac{1}{4c} \right)^3 \right]
		=  \frac{1}{4c}.
	\end{split}
\end{equation*}
%
%%
Hence, $T=T_{\vp}$ maps the ball $B\left( 0, \frac{1}{4c} \right) \subset \dot{Y}^s$ into 
itself. Next, note that
%
%%
\begin{equation*}
	\begin{split}
		Tu - Tv = \eqref{2main-int-expression-2} + \eqref{2main-int-expression-3} 
		+ \eqref{2main-int-expression-4}
	\end{split}
\end{equation*}
%
%%
where now $w = u | u |^2 - v | v |^{2}$. Rewriting
%
%%
\begin{equation*}
	\begin{split}
		u | u |^{2} - v | v |^{2}
		& = | u |^2 \left( u -v \right) + v\left( | u 
		|^2 - | v |^2
		\right)
		\\
		& = u \bar u \left( u -v \right) + v u \bar u - v v \bar v
		\\
		& = u \bar u \left( u - v \right) + v \bar u\left( u - v \right) + v 
		\bar u v - v v \bar v
		\\
		& = u \bar u \left( u -v \right) + v \bar u\left( u - v \right) + v v 
		\left( \overline{u -v} \right)
	\end{split}
\end{equation*}
%
%%
the triangle inequality and linearity of the Fourier transform then give
%
%%
\begin{equation*}
	\begin{split}
		| \wh{w}(n, \tau) | = | \mathcal{F}(u | u |^2 - v| v |^2) |
		& \le | \wh{u \overline{u} \left (u -v \right )} | +
		| \wh{v \overline{u} (u -v)} | + |\wh{v v 
		(\overline{u-v})}|
		\\
		& \doteq | \wh{w_1} | + | \wh{w_2} | + | \wh{w_3} |
	\end{split}
\end{equation*}
%
%%
where
%
%%
\begin{equation*}
	\begin{split}
		w_1 = u \bar u \left( u -v \right), \qquad w_2 = v \bar u \left( u -v 
		\right), \qquad w_3 = v v \left( \overline{u -v} \right).
	\end{split}
\end{equation*}
%
%%
Hence, $Tu - Tv = \sum_{\ell=1, 2, 3} 
T_\ell(u, v)$, where
\begin{align}
	\label{2main-int-exp-mod1}
	& \frac{1}{4 \pi^2} \psi(t) \sum_{n\in \zzdot} \int_\rr e^{ixn}  
		e^{it \tau} \frac{1 - \psi(\tau - n^{3}) 
		}{\tau - n^{3}} \wh{w_\ell}(n, \tau) \ d \tau
		\\
		\label{2main-int-exp-mod2}
		- & \frac{1}{4 \pi^2} \psi(t) \sum_{n\in \zzdot} \int_\rr e^{i(xn + 
		tn^{3})}
		 \frac{1- \psi(\tau - n^{3})}{\tau - n^{3}} \wh{w_\ell}(n, \tau) \ d \tau
		\\
		\label{2main-int-exp-mod3}
		+ & \frac{1}{4 \pi^2} \psi(t) \sum_{k \ge 1} \frac{i^k t^k}{k!}
		\sum_{n \in \zzdot} \int_\rr e^{i(xn + tn^{3} )}
		\psi(\tau - n^{3}) (\tau - n^{3})^{k-1} \wh{w_\ell}(n, \tau)  
		\\
		\doteq & T_\ell(u). \notag
\end{align}
Repeating the arguments used to estimate 
\eqref{2main-int-expression-2}-\eqref{2main-int-expression-4}, we obtain
%
%%
\begin{equation*}
	\begin{split}
		& \|T_1\|_{\dot{Y}^s} \le c_\psi \|u -v \|_{\dot{Y}^s} \|u\|^2_{\dot{Y}^s}
		\\
		& \|T_2\|_{\dot{Y}^s} \le c_\psi \|u -v \|_{\dot{Y}^s} \|u\|_{\dot{Y}^s} \|v\|_{\dot{Y}^s}
		\\
		& \|T_3\|_{\dot{Y}^s} \le c_\psi \|u -v \|_{\dot{Y}^s} \|v\|_{\dot{Y}^s}^2.
	\end{split}
\end{equation*}
%
%%
Therefore,
%
%%
\begin{equation}
	\label{220a}
	\begin{split}
		\|Tu - Tv \|_{\dot{Y}^s} = & \| \sum T_\ell(u, v) \|_{\dot{Y}^s}
		\\
		& \le c_\psi \|u -v \|_{\dot{Y}^s} \left( \|u\|_{\dot{Y}^s}^2 + 
		\|u\|_{\dot{Y}^s} \|v\|_{\dot{Y}^s} + \|v\|_{\dot{Y}^s}^2 \right)
		\\
		& \le c_\psi \|u -v\|_{\dot{Y}^s} \left( \|u\|_{\dot{Y}^s} + \|v\|_{\dot{Y}^s} \right)^2
		\\
		& = c^2 \|u -v\|_{\dot{Y}^s} \left( \|u\|_{\dot{Y}^s} + \|v\|_{\dot{Y}^s} \right)^2.
	\end{split}
\end{equation}
%
%%
If $u, v \in B(0, \frac{1}{4c}) \subset \dot{Y}^s$, it follows that
%
%%
\begin{equation}
	\label{221a}
	\begin{split}
		\|Tu - Tv \|_{\dot{Y}^s}
		& \le c^2 \|u -v \|_{\dot{Y}^s} \left( \frac{1}{4c} + 
		\frac{1}{4c} \right)^2
		\\
		& = \frac{1}{4} \|u -v \|_{\dot{Y}^s}. 
	\end{split}
\end{equation}
%
%%
We conclude that $T = T_{\vp}$ is a contraction on the ball $B(0, 
\frac{1}{4c}) \subset \dot{Y}^s$. A Picard iteration and application of 
\cref{2lem:cutoff-loc-soln} then yield a unique, local
solution to the KDV ivp \eqref{2KDV-eq}-\eqref{2KDV-init-data}.
\begin{definition}
	We say that the flow map $u_0 \mapsto u(t)$ is \emph{locally Lipschitz} in a Banach
	space $X$ if for
	$$u_0, v_0 \in B_R \doteq \{f: \|f\|_X < R\},$$ there exist $C, T>0$
	depending on $R$ such that $\|u(\cdot, t) - v(\cdot, t)
	\|_X \le C \|u_{0} - v_0 \|_{X}$ for $t \in [-T, T]$. We
	say the flow map is \emph{locally uniformly
	continuous} in $X$ if for
	$u_0, v_0 \in B_R$ there exists $T >0$ depending on $R$ such that for
	$t \in [-T, T]$, $\|u(\cdot, t) - v(\cdot, t) \|_{X} \to
	0$ if $\|u_0 - v_0 \|_{H^{s}(\ci)} \to 0$. 
\end{definition}
%
%
Clearly any locally Lipschitz flow map is locally uniformly continuous. 
Next, we shall establish local Lipschitz continuity in $\dot{Y}^s$ of the flow
map. Let $\vp_1, \vp_2 \subset \dot{H}^s(\ci)$ be given. Choose $\psi$ such that
$\vp_1, \vp_2 \subset B(0, \frac{15}{64c^{3}})$.  Then there exist $u_1, u_2 \in
\dot{Y}^s$ such that $u_1 = T_{\vp_1}$, $u_2 = T_{\vp_2}$, and so
%
%
\begin{equation*}
	\begin{split}
		T_{\vp_1}(u) - T_{\vp_2}(v) = \frac{1}{2\pi} \psi(t) \sum_{n \in
		\zzdot}e^{i\left( xn + tn^{3} \right)} \wh{\vp_1 - \vp_2}(n) + \sum_{\ell
		= 1,2,3} T_{\ell}(u).
	\end{split}
\end{equation*}
%
%
Using an argument similar to \eqref{2fourier-trans-calc}-\eqref{2main-int1-est},
we obtain
%
%
\begin{equation*}
	\begin{split}
		\| \frac{1}{2\pi} \psi(t) \sum_{n \in
		\zzdot}e^{i\left( xn + tn^{3} \right)} \wh{\vp_1 - \vp_2}(n)\|_{\dot{Y}^s}
		\le c_\psi \|\vp_{1} - \vp_{2}\|_{\dot{Y}^s}.
	\end{split}
\end{equation*}
%
%
Hence, \eqref{220a}-\eqref{221a} gives
%
%
\begin{equation*}
	\begin{split}
		\sum_{\ell=1,2,3} T_{\ell}(u,v) \le \frac{1}{4}\|u-v\|_{\dot{Y}^s}.
	\end{split}
\end{equation*}
%
%
Therefore,
%
%
\begin{equation*}
	\begin{split}
		\|u -v \|_{\dot{Y}^s} = \|T_{\vp_1}(u) - T_{\vp_2}(v) \|_{\dot{Y}^s} \le c_\psi
		\|\vp_{1} - \vp_{2} \|_{\dot{H}^s\left( \ci \right)}\| +
		\frac{1}{4} \|u -v \|_{\dot{Y}^s}
	\end{split}
\end{equation*}
%
%
which implies
%
%
\begin{equation*}
	\begin{split}
		\frac{3}{4} \|u-v\|_{\dot{Y}^s} \le c_\psi \|\vp_1 - \vp_2 \|_{\dot{H}^s(\ci)}
	\end{split}
\end{equation*}
%
%
or
%
%
\begin{equation*}
	\begin{split}
		\|u -v \|_{\dot{Y}^s} \le \frac{4}{3} c_\psi \|\vp_1 - \vp_2 \|_{\dot{H}^s(\ci)}.
	\end{split}
\end{equation*}
%
%
Applying \cref{2lem:cutoff-loc-soln}, we then obtain
%
%
	 %
	 %
	 \begin{equation*}
		 \begin{split}
			\|u(\cdot, t) -v(\cdot, t) \|_{\dot{H}^s(\ci)} \le \frac{4}{3} c_\psi \|\vp_1 -
			\vp_2 \|_{\dot{H}^s(\ci)}, \qquad t \in [-T, T].
		 \end{split}
	 \end{equation*}
	 %
	 %
Hence, the flow map of the KDV ivp is locally Lipschitz continuous in
$\dot{H}^s(\ci)$. This
concludes the proof of \cref{2thm:main}. \qquad \qedsymbol
%
%
%
%
\section{Proof of First Bilinear Estimate}
Note first that $|\wh{w_{fg}}(n, \tau) |  = | n\wh{f} *  \wh{g} 
(n, \tau)|$. From this and the conservation of mass, it follows that
%
%
\begin{equation}
	\label{2non-lin-rep}
	\begin{split}
		| \wh{w_{fg}}(n, \tau)|
		& = | \sum_{\substack{n_1 \neq 0, n_2 \neq 0 \\n_1 +n_2 =n}}  \int_{\tau_1 + \tau_2 = \tau}n\wh{f}\left( n_1,  \tau_1 
\right) \wh{g}\left( n_2, \tau_2  
\right) d \tau_1 d \tau_2 |
\\
& = | \sum_{\substack{n_1 \neq0, n_2 \neq 0 \\n_1 + n_2 =n}}  \int_{\tau_1 + \tau_2 = \tau}n\wh{f}\left( n_1,  \tau_1 
\right) \wh{g}\left( n_2, \tau_2  
\right) d \tau_1 d \tau_2 | 
\\
& \le \sum_{\substack{n_1 \neq0, n_2 \neq 0 \\n_1 + n_2 =n}}   \int_{\tau_1 + \tau_2 = \tau}| n | \times | \wh{f}\left( n_1, \tau_1 
\right) | \times  | \wh{g}\left( n_2, \tau_2 
\right) |   d \tau_1 d \tau_2  
\\
& = \sum_{\substack{n_1 \neq0, n_2 \neq 0 \\n_1 + n_2 =n}} \int_{\tau_1 + \tau_2 = \tau}| n | \times \frac{c_f\left( n_1, \tau_1 
\right)}{|n_1|^s \left( 1 + | \tau_1 - n_1^3 | \right)^{1/2}}
\\
& \times \frac{c_{g}\left( n_2, \tau_2 \right)}{|n_2|^s\left( 1 + | \tau_2 -  n_2^3| 
\right)^{1/2}}
  \ d \tau_1 d \tau_2 
\end{split}
\end{equation}
%
%
where 
%
%
\begin{equation*}
	\begin{split}
		c_h(n, \tau) =
		\begin{cases}
			|n|^s \left( 1 + | \tau - n^3 |  
			\right)^{1/2} | \wh{h}\left( n, \tau \right) |, \qquad & n \neq 0
		\\
		0, \qquad & n = 0.
	\end{cases}
	\end{split}
\end{equation*}
%
%
From our work above, it follows that 
%
%
\begin{equation}
	\label{2convo-est-starting-pnt}
	\begin{split}
		 & |n|^s \left( 1 + | \tau - n^3 | \right)^{-1/2} | \wh{w_{fg}}\left( 
		n, \tau \right) |
		\\
		& \le \left( 1 + | \tau - n^3 | \right)^{-1/2}
		\sum_{\substack{n_1 \neq0, n_2 \neq 0 \\n_1 + n_2 =n}} \int_{\tau_1 + \tau_2 = \tau}\frac{|n|^{s+1}}{|n_1|^s | n_2|^s} 
		\times \frac{c_f(n_1, \tau_1)}{\left( 1 + | \tau_1 - n_1^3 | 
		\right)^{1/2}}
		\\
		& \times
		\frac{c_g(n_2, \tau_2)}{\left( 1 + | \tau_2 - n_2^3 | 
		\right)^{1/2}}\ d \tau_1 d \tau_2.
	\end{split}
\end{equation}
%
%
Unlike the NLS, we must use the smoothing properties of the
principal symbol $\tau - n^3$ regardless of the choice of $s$, since the quantity
%
%
\begin{equation}
	\label{2convo-multiplier}
	\begin{split}
		\frac{|n|^{s+1}}{|n_1|^s |n_2|^s }
	\end{split}
\end{equation}
%
%
blows up in general, due to the presence of the extra power of $|n|$ coming from the derivative on
the nonlinearity. To utilize the smoothing effects of the principal symbol, we will need the following, whose
proof is provided in the appendix.
%
%
\begin{lemma}
	\label{2lem:number-theory}
	Let $n=n_1 + n_2$ and suppose that $n, n_1, n_2\neq
	0$. Then for any integer $c \ge 0$
%
%
\begin{equation}
	\begin{split}
		\label{2number-theory}
		| - n^{3} + n_1^3 + n_2^3| \ge 2^{-c/2} | n |^{\frac{2+c}{2}} | n_{1}
		|^{\frac{2-c}{2}}| n_2 |^{\frac{2-c}{2}}.
	\end{split}
\end{equation}
%
%
\end{lemma}
%
%
\begin{remark}
	In~\cite{Bourgain:1993ju}, Bourgain obtains the lower bound $n^2$ for
	the left hand side of \eqref{2number-theory}. This is too coarse an estimate,
	as we shall see.
\end{remark}
%
%
Since $$| \tau - n^{3} - \left( \tau_1 - n_1^3 
+ \tau_2 - n_2^3  \right ) | = | - n^{3} + n_1^3 +
n_2^3|,$$ by \cref{2lem:number-theory} and
the pigeonhole principle we must have one of the 
following.
%
%
\begin{align}
	\label{2pigeon-case-1}
	& |\tau - n^3| \ge \frac{2^{-c/2}}{3} | n |^{\frac{2+c}{2}} | n_{1}
		|^{\frac{2-c}{2}}| n_2 |^{\frac{2-c}{2}}		\\
		\label{2pigeon-case-2}
		& | \tau_1 - n_1^3 | \ge \frac{2^{-c/2}}{3} | n |^{\frac{2+c}{2}} | n_{1}
		|^{\frac{2-c}{2}}| n_2 |^{\frac{2-c}{2}},  
		\\
		\label{2pigeon-case-3}
		& | \tau_2 - n_2^3 | \ge
		\frac{2^{-c/2}}{3} | n |^{\frac{2+c}{2}} | n_{1}
		|^{\frac{2-c}{2}}| n_2 |^{\frac{2-c}{2}}.  
\end{align}
%
%
By the symmetry of the convolution, it will be enough to consider only
\eqref{2pigeon-case-1} and \eqref{2pigeon-case-2}.
%
%
%
\subsection{Case \ref{2pigeon-case-1}.} 
We have, for nonzero $ n, n_1, n_2 $
%
%%
\begin{equation}
	\label{2convo-deriv-bound}
	\begin{split}
		& \frac{|n|^{s+1}}{|n_1|^s 
		| n_2|^s}
		\times
		\frac{1}{(1 + | \tau -n^{3} |)^{1/2}}
		\\
		& \lesssim | n |^{s+1}| n_1 |^{-s}| n_2 |^{-s} \times | n
		|^{-\frac{2+c}{4}}| n_1 |^{-\frac{2-c}{4}}| n_2 |^{-\frac{2-c}{4}} 
		\\
		& = | n |^{\frac{4s +2 -c}{4}} | n_1 |^{\frac{-4s -2 +c}{4}} | n_2
		|^{\frac{-4s -2 +c}{4}}
		\\
		& \le 1, \qquad s \ge -1/2.
	\end{split}  
\end{equation}
%
%
\begin{remark}
	\label{2rem:s-val}
	The last line follows from the following reasoning: Set $(4s + 2 -c) = 0$
or, equivalently, $-4s -2 +c = 0$. Then for any $c \ge 0$ such that $c = 4s+2$
the left hand side of
\eqref{2convo-deriv-bound} is bounded by $1$. Of course such a $c$ exists, as long as $s \ge -1/2$. 
\end{remark}
%
%
%
Hence, recalling \eqref{2convo-est-starting-pnt} and applying estimates 
\eqref{2pigeon-case-1} and \eqref{2convo-deriv-bound}, we obtain
%
%
\begin{equation}
	\label{2non-lin-rep-with-bound}
	\begin{split}
		& |n|^s \left( 1 + | \tau - n^3 | \right)^{-1/2} | 
		\wh{w_{fg}}(n, \tau) | 
		\\
		& \lesssim \sum_{\substack{n_1 \neq0, n_2 \neq 0 \\n_1 + n_2 =n}} \int_{\tau_1 + \tau_2 = \tau}\frac{c_f(n_1, \tau_1)}{\left( 1 + | 
		\tau_1 -  n_1^3| \right)^{1/2}}
		\times \frac{c_g\left( n_2, \tau_2\right)}{\left( 1 + | \tau_2 -n_2^3|
		\right)^{1/2}}
		\\
		& = \wh{C_f C_g}(n, \tau)
	\end{split}
\end{equation}
%
%
where
\begin{equation*}
	\begin{split}
		C_h(x,t) =
		\left[ \frac{c_h(n, \tau)}{\left( 1 + | \tau - n^3 | 
		\right)^{1/2}}\right]^\vee.	
	\end{split}
\end{equation*}

%
%
Therefore, from \eqref{2non-lin-rep-with-bound}, Plancherel, and generalized 
H\"{o}lder, we obtain
%
%
\begin{equation}
	\label{2gen-holder-bound}
	\begin{split}
		& \| |n|^s \left( 1 + | \tau - n^3 | \right )^{-1/2}  \wh{w_{fg}}\left( 
		n, \tau \right) \|_{L^2(\ci \times \rr)}
		\\
		& \lesssim \|\wh{C_f C_g }\left( n, \tau \right) 
		\|_{L^2\left( \zzdot \times \rr \right)}
		\\
		& \simeq \|C_f C_g \|_{L^2\left( \ci \times \rr \right)}
		\\
		& \le \|C_f \|_{L^4(\ci \times \rr)} \|C_g \|_{L^4(\ci \times \rr)}.
	\end{split}
\end{equation}
%
We now need the following Fourier multiplier estimate. 
\begin{lemma}
	\label{2lem:four-mult-est-L4}
	Let $(x, t) \in \ci \times \rr $ and $(k, \tau) \in \zz \times \rr$ be
	the dual variables. Let $v$ be a positive even integer, and fix $\ee >
	0$. Then for $b$ satisfying the relations $b \ge (\ee v + 1/2)/v$ and $b \ge (v+1)/(4v)$, we have
\begin{equation}
	\label{2four-mult-est-L4}
	\begin{split}
		\|f\|_{L^4(\ci \times \rr)} \le c_\ee \|\left( 1 + | \tau - n^v | 
		\right)^b \wh{f} \|_{L^2( \zz \times \rr)}
	\end{split}
\end{equation}
for every test function $f(x, t)$. 
%
%
%
%
\end{lemma}
\begin{framed}
%
%
\begin{remark}
  See \cite{Himonas:2007qf} for a similar theorem. The above theorem is a slight improvement. 
\end{remark}
%
%
\end{framed}
From the lemma, we see that
%
%
\begin{equation}
	\label{2four-mult-conseq}
	\begin{split}
		\|C_h\|_{L^4(\ci \times \rr)} 
		& \lesssim \|(1 + | \tau - n^3 |)^{1/2} \wh{C_h}
		\|_{L^2(\zz \times \rr)}
		\\
		& = \|c_{h} \|_{L^2(\zz \times \rr)} 
		\\
		& = \|h \|_{\dot{X}^s}. 
	\end{split}
\end{equation}
%
%
Applying this to \eqref{2gen-holder-bound} we
conclude that
\begin{equation*}
	\begin{split}
		\| |n|^s \left( 1 + | \tau - n^3 | \right ) ^{-1/2} \wh{w_{fg}}\left( 
		n, \tau \right) \|_{L^2(\zzdot \times \rr)}
		& \lesssim \|f\|_{\dot{X}^s} \|g\|_{\dot{X}^s}.
	\end{split}
\end{equation*}
%
%
%
\subsection{Case \ref{2pigeon-case-2}.}
Using a similar argument to that in Case \eqref{2pigeon-case-1}, we obtain
%
%
\begin{equation}
	\label{21f}
	\begin{split}
		& |n|^s  | \wh{w_{fg}}\left( 
		n, \tau \right) |
		\\
		& \lesssim 
		\sum_{\substack{n_1 \neq0, n_2 \neq 0 \\n_1 + n_2 =n}} \int_{\tau_1 + \tau_2 = \tau}		c_f(n_1, \tau_1)
		\times
		\frac{c_g(n_2, \tau_2)}{\left( 1 + | \tau_2 - n_2^3 | 
		\right)^{1/2}} 
		\\
		& = \wh{\overset{\sim}{C_f} C_g}
	\end{split}
\end{equation}
%
%%
where
%
%
\begin{equation*}
	\begin{split}
		\overset{\sim}{C_h}(x,t) = \left[ c_h(n, \tau) \right]^\vee.
	\end{split}
\end{equation*}
%
%
Hence
%
%%
\begin{equation}
	\label{23f}
	\begin{split}
		& \| |n|^s \left( 1 + | \tau - n^3 | \right)^{-1/2} \wh{w_{fg}}(n, \tau) 
		\|_{L^2(\zzdot \times \rr)}
		\\
		& \lesssim \|\left( 1 + | \tau - n^{3} | \right)^{-1/2} 
		\wh{\overset{\sim}{C_f} C_g } \|_{L^2(\zzdot \times \rr)}
		\\
		& =  \|\left( 1 + | \tau - n^{3} | \right)^{-1/2} 
		\wh{\overset{\sim}{C_f} C_g } \|_{L^2(\zz \times \rr)}
		\\
		& \lesssim  \|\overset{\sim}{C_f} C_g  \|_{L^{4/3}(\ci \times \rr)}
	\end{split}
\end{equation}
%
%%
where the last step follows by dualizing \cref{2lem:four-mult-est-L4}. More
precisely, we have the following.
\begin{corollary}
	\label{2cor:four-mult-est-L4}
	Let $(x, t) \in \ci \times \rr $ and $(n, \tau) \in \zzdot \times \rr$ be 
	the dual variables. Let $v$ be a positive even integer. Then there is a 
	constant $c_v > 0$ such that
%
%
\begin{equation}
	\label{2four-mult-est-L4st}
	\begin{split}
		\| \left( 1 + | \tau - n^v | 
		\right)^{-\frac{v+1}{4v}}
		\wh{f}\|_{L^2(\zz \times \rr)} \le c_v \|f \|_{L^{4/3}( \ci \times \rr)}.
	\end{split}
\end{equation}
%
%
\end{corollary}
%
Applying H\"{o}lder's inequality to the right hand side of
\eqref{23f}, we obtain the bound
%
%%
\begin{equation}
	\label{24f}
	\begin{split}
		\|\overset{\sim}{C_f} \|_{L^2(\ci \times \rr)} \|C_g \|_{L^4\left( \ci 
		\times \rr 
		\right)}. 
	\end{split}
\end{equation}
%
%%
By Plancherel we have
%
%%
%
%%
\begin{equation}
	\label{25f}
	\begin{split}
		\|\overset{\sim}{C_f} \|_{L^2(\ci \times \rr)}
		& \simeq \|c_f\|_{L^2(\zz \times \rr)}
		\\
		& = \|f \|_{\dot{X}^s}
	\end{split}
\end{equation}
%
%%
while \eqref{2four-mult-conseq} gives
%
%
\begin{equation}
	\label{26f}
	\begin{split}
		\|C_g \|_{L^4(\ci \times \rr)} \lesssim \|g\|_{\dot{X}^s}.
	\end{split}
\end{equation}
%
%
We conclude from \eqref{23f}-\eqref{26f} that
%
%
\begin{equation*}
	\begin{split}
		\| |n|^s \left( 1 + | \tau - n^3 | \right)^{-1/2} \wh{w_{fg}}(n, \tau) 
		 \|_{L^2(\zzdot \times \rr)}
		 \lesssim \|f\|_{\dot{X}^s} \|g\|_{\dot{X}^s}
	\end{split}
\end{equation*}
%
%
which completes the proof.  \qquad \qedsymbol
%
%

\section{Proof of Second Bilinear Estimate}
Recall that for the NLS, one obtains one trilinear estimate as a corollary of
another. Using this as motivation, let us see if we can obtain
\cref{2prop:bilinear-estimate2} as a corollary of
\cref{2prop:prim-bilin-est}. By
duality, it suffices to show that
%
%%
\begin{equation}
	\label{2duality-est}
	\begin{split}
		\sum_{n \in \zzdot}  |n|^{s}
		a_n \int_{\rr} \frac{|\wh{w_{fg}}(n, \tau)|}{1 
		+ | \tau - n^3 |} \ d \tau \lesssim \|f\|_{\dot{X}^s} \|g\|_{\dot{X}^s}
		\|a_n \|_{\ell^2}, \qquad s \ge -1/2.
	\end{split}
\end{equation}
%
%%
By the triangle inequality 
and Cauchy-Schwartz,
%
%%
\begin{equation}
	\label{21m}
	\begin{split}
		& | \sum_{n \in \zzdot} |n|^{s} a_n
		\int_{\rr}\frac{| \wh{w_{fg}}(n, \tau) |}{(1 + | \tau - n^3 |)} \ d \tau |
		\\
		& \le \sum_{n \in \zzdot} \int_{\rr} \frac{| a_n |}{\left( 1 + 
		| \tau - n^3 |
		\right)^{1/2 + \eta}} \times \frac{| n|^s  |
		\wh{w_{fg}}(n, \tau) |}{\left( 
		1 + | \tau - n^3 | \right)^{1/2 - \eta}} \ d \tau
		\\
		& \le \left( \sum_{n \in \zzdot} | a_{n} |^2\int_{\rr} \frac{1}{\left( 1 + |
		\tau - n^3 | \right)^{1 + 2 \eta}} \ d \tau  
		\right)^{1/2} 
		\left ( \sum_{n \in \zzdot}\int_{\rr} \frac{|n|^{2s} | \wh{w_{fg}}(n, \tau) 
		|^2}{\left( 1 + | \tau - n^3 | \right)^{1 -2 \eta}}\ d \tau 
		\right)^{1/2}.
	\end{split}
\end{equation}
%
%%
Applying the change of variable $\tau - n^3
\mapsto \tau'$ we obtain  
%%

\begin{equation*}
	\begin{split}
		& \left( \sum_{n \in \zzdot} | a_{n} |^2\int_{\rr} \frac{1}{\left( 1 + | \tau -
		n^3 | \right)^{1 + 2 \eta}} \ d \tau  
		\right)^{1/2} 
		\\
		& = \left ( \sum_{n \in \zzdot}
		| a_n |^2 
		\int_{\rr} \frac{1}{\left( 1 + | \tau' | \right)^{1 + 2 \eta}} \ d 
		\tau \right)^{1/2}
		\\
		& \simeq \|a_n\|_{\ell^2}, \qquad \eta >0.
		\end{split}
\end{equation*}
However, if we assume $\eta >0$, then
we cannot use \cref{2prop:prim-bilin-est} to bound
\begin{equation*}
	\begin{split}
		\left ( \sum_{n \in \zzdot}\int_{\rr} \frac{|n|^{2s} | \wh{w_{fg}}(n, \tau) 
		|^2}{\left( 1 + | \tau - n^3 | \right)^{1 - 2\eta}}\ d \tau
		\right)^{1/2}. 
	\end{split}
\end{equation*}
%%
%%
\begin{remark}
Hence, unlike the NLS, we have not been able to obtain a second bilinear
estimate as a corollary from the first. Heuristically, this is due to the
derivative in nonlinearity, which is not present in the NLS nonlinearity.
However, one can obtain \eqref{2bilinear-estimate2} for $s>1/2$ as a
corollary of \cref{2prop:prim-bilin-est} by using the ideas
above and by modifying the proof of \cref{2prop:prim-bilin-est} slightly (i.e.,
showing that if $b = \frac{1}{2}^-$, then \eqref{2prim-bilin-est} holds for
$s\ge-\frac{1}{2}^+$). To show that \eqref{2bilinear-estimate2} holds for the
case $s=1/2$, we will have to resort to Kenig-Ponce-Vega~\cite{Kenig:1996yn} techniques.
\end{remark}
%
%
Proceeding, note that by duality, to prove \cref{2prop:bilinear-estimate2} it
suffices to show \eqref{2duality-est} for $s \ge -1/2$. By the symmetry of the convolution, we
consider only cases \eqref{2pigeon-case-1} and \eqref{2pigeon-case-2}.
%
%
\subsection{Case \ref{2pigeon-case-1}.} From the triangle inequality and \eqref{2non-lin-rep-with-bound}, we have
%
%
\begin{equation*}
	\begin{split}
	 |\eqref{2duality-est}|
	& \lesssim \sum_{n \in \zzdot} |a_{n}| \int_{\rr} \sum_{\substack{n_1 \neq 0, n_2 \neq 0
		\\ n_1 +n_2 =n}} \int_{\tau_1 + \tau_2 = \tau} c_f(n_1, \tau_1)
		c_g(n_2, \tau_2)
		\\
		& \times \frac{1}{(1 + | \tau - n^{3} |)^{1/2}(1 + |
		\tau_{1}-n_{1}^{3} |)^{1/2}(1 + | \tau-n_{2}^{3} |^{1/2})} d \tau_1 d \tau_2
		d \tau
	\end{split}
\end{equation*}
%
%
which by Cauchy-Schwartz is bounded by
%
%
\begin{equation}
	\label{210g}
	\begin{split}
		& \sum_{n \in \zzdot} |a_n| \int_{\rr} \left(  \sum_{\substack{n_1 \neq 0, n_2
		\neq 0 \\n_1 +n_2 =n}} \int_{\tau_1 + \tau_2 = \tau} c_{f}^{2}(n_1, \tau_1)
		c_{g}^{2} (n_2, \tau_2) d \tau_1 d \tau_2 \right)^{1/2} 
		\\
		& \times \left( \sum_{\substack{n_1 \neq 0, n_2 \neq 0 \\n_1 +n_2 =n}}
		\int_{\tau_1 + \tau_2 = \tau} \frac{1}{(1 + | \tau - n^{3} |)(1 + | \tau_{1}-n_{1}^{3} |)(1 + |
		\tau_2 -n_{2}^{3} |)} d \tau_1 d \tau_2
		\right)^{1/2} d \tau.
	\end{split}
\end{equation}
%
%
Applying Cauchy-Schwartz again, \eqref{210g} is bounded by
%
%
\begin{align}
	\notag
		& \|\left( \sum_{\substack{n_1 \neq 0, n_2 \neq 0 \\n_1 +n_2 =n}}\int_{\tau_1 + \tau_2 = \tau} c_{f}^{2}(n_1, \tau_1)
		c_{g}^{2} (n_2, \tau_2) d \tau_1 d \tau_2 \right)^{1/2} \|_{L^{2}(\zz \times
		\rr)}
		\\
		\notag
		& \times  \|a_{n}
		\left( \sum_{\substack{n_1 \neq 0, n_2 \neq 0 \\n_1 +n_2
		=n}}\int_{\tau_1 + \tau_2 = \tau} \frac{1}{(1 + | \tau - n^{3} |)(1 + |
		\tau_{1}-n_{1}^{3} |)(1 + | \tau_2 -n_{2}^{3} |)} d \tau_1 d \tau_2
		\right)^{1/2} \|_{L^2(\zz \times \rr)}
		\\
		\notag
		& = \|f\|_{\dot{X}^s} \|g\|_{\dot{X}^s}
		\\
		\label{2holder-term}
		& \times 
		\|a_{n}
		\left( \sum_{\substack{n_1 \neq 0, n_2 \neq 0 \\n_1 +n_2
		=n}}\int_{\tau_1 + \tau_2 = \tau} \frac{1}{(1 + | \tau - n^{3} |)(1 + |
		\tau_{1}-n_{1}^{3} |)(1 + | \tau_2 -n_{2}^{3} |)} d \tau_1 d \tau_2
		\right)^{1/2} \|_{L^2(\zz \times \rr)}.
\end{align}
%
Applying H{\"o}lder then gives
%
%
\begin{equation*}
	\begin{split}
		& \eqref{2holder-term}
		 \le \| a_{n} \|_{\ell^2}
		\\
		& \times \left( \sup_{n \neq 0} \int_{\rr}
		\sum_{\substack{n_1 \neq 0, n_2 \neq 0 \\n_1 +n_2 =n}} \int_{\tau_1 + \tau_2
		= \tau} \frac{1}{(1 + | \tau - n^{3} |)(1 + |
		\tau_{1}-n_{1}^{3} |)(1 + | \tau_2 -n_{2}^{3} |)} d \tau_1 d \tau_2 d \tau
		\right)^{1/2}.
	\end{split}
\end{equation*}
%
%
Hence, to complete the proof for case \eqref{2pigeon-case-1}, it will be enough
to show that 
%
%
%
%
\begin{equation*}
	\begin{split}
		 \sup_{n \neq 0} \int_{\rr}
		\sum_{\substack{n_1 \neq 0, n_2 \neq 0 \\n_1 +n_2 =n}} \int_{\tau_1 + \tau_2
		= \tau} \frac{1}{(1 + | \tau - n^{3} |)(1 + |
		\tau_{1}-n_{1}^{3} |)(1 + | \tau_2 -n_{2}^{3} |)} d \tau_1 d \tau_2 d \tau <\infty
	\end{split}
\end{equation*}
%
%
or, equivalently, that
%
%
\begin{equation}
	\label{212g}
	\begin{split}
		\sup_{n \neq 0} \sum_{\substack{n_1 \neq 0, n_2 \neq 0 \\n_1 +n_2 =n}} \int_{\rr}
		\int_\rr  \frac{1}{(1 + | \tau - n^{3} |)(1 + | \tau_1 - n_{1}^{3} |)(1 + | \tau - \tau_1 -
		n_2^3 |)} d \tau_1 d \tau < \infty.
	\end{split}
\end{equation}
%
%
Following Kenig~\cite{Kenig:1996yn}, we now need the following Calculus lemma.
%
%
%%%%%%%%%%%%%%%%%%%%%%%%%%%%%%%%%%%%%%%%%%%%%%%%%%%%%
%
%
%				 Calculus Lemma
%
%
%%%%%%%%%%%%%%%%%%%%%%%%%%%%%%%%%%%%%%%%%%%%%%%%%%%%%
%
%
\begin{lemma}
	\label{2lem:calc}
 %
 %
 \begin{equation}
	 \label{2calc}
	 \begin{split}
		 \int_{\rr} \frac{1}{(1 + | \theta |)(1 + | a - \theta |)} d \theta \lesssim
		 \frac{\log(2 + | a |)}{1 + | a |}.
	 \end{split}
 \end{equation}
 %
 %
 \end{lemma}
%
%
Applying the lemma with $\theta = \tau_1 - n_1^3$ and $a = \tau - n_1^3 -
n_2^3$, we see that
%
%
\begin{equation*}
	\begin{split}
	\int_{\rr}
		\int_\rr  \frac{1}{(1 + | \tau - n^{3} |)(1 + | \tau - \tau_1 -
		n_2^3 |)} d \tau_1 d \tau \lesssim \frac{\log(2 + | \tau - n_{1}^{3} -
		n_{2}^{3} |)}{1 + | \tau - n_{1}^{3} - n_{2}^{3} |}.
	\end{split}
\end{equation*}
%
%
%
Hence, the left hand side of \eqref{212g} is bounded by
%
\begin{equation*}
	\begin{split}
		\sup_{n \neq 0} \sum_{\substack{n_1 \neq 0, n_2 \neq 0 \\n_1 +n_2 =n}}
		\int_{\rr} \frac{\log(2 + | \tau - n_{1}^{3} -
		n_{2}^{3} |)}{(1 + | \tau - n_{1}^{3} - n_{2}^{3} |)(1 + | \tau - n^{3} |)}
		d \tau	
	\end{split}
\end{equation*}
%
%
or, equivalently, by
%
%
\begin{equation}
	\label{213g}
	\begin{split}
		\sup_{n \neq 0} \sum_{n_1 \neq 0} \int_{\rr} \frac{\log(2 + | \tau -
		n_{1}^{3} - (n - n_1)^{3} |)}{(1 + | \tau - n_{1}^{3} - (n - n_{1})^{3} |)(1
		+ | \tau - n^{3} |)} d \tau.
	\end{split}
\end{equation}
%
%
Now by assumption, for $m \ge 3$
we have  $ |\tau - n^m| \ge
\frac{2^{-2s+1}}{3} | n |^{2s+2} | n_{1} |^{-2s}| n_2
|^{-2s}$. This follows from  \eqref{2number-theory}
with $c = 4s+2$, where the choice of $c$ is motivated by \cref{2rem:s-val}. Hence, \eqref{213g} is bounded by a constant times
%
%
%
%
\begin{equation}
	\label{214g}
	\begin{split}
		& \sup_{n \neq 0} \sum_{n_1 \neq 0}
		\frac{1}{|n|^{+(2s +2)}{|n_1 n_2 |^{+(-s)}}}\int_{\rr} \frac{\log(2 + | \tau
		- n_{1}^m - (n - n_1)^m |)}{(1 + | \tau - n_{1}^m - (n - n_{1})^m
		|)(1 + | \tau - n^m |)^{\frac{1}{2}-}}
		d \tau
		\\
		& \le \sup_{n \neq 0} \sum_{n_1 \neq 0}
		\frac{1}{|n|^{+(2s +2)}{|n_1 n_2 |^{+(-s)}}}		\\
		& \times \sup_{n \neq 0} \sum_{n_1 \neq 0}
		\int_{\rr} \frac{\log(2 + | \tau
		- n_{1}^m - (n - n_1)^m |)}{(1 + | \tau - n_{1}^m - (n - n_{1})^m
		|)(1 + | \tau - n^m |)^{\frac{1}{2}-}}
		d \tau
	\end{split}
\end{equation}
%
%
Observe that for the first sum, the supremum is attained at $n=1$. But then $n_2
= 1 - n_1$, and so $| n_1 n_2 | \gtrsim | n_1 |^2$. Furthermore, we know that 
for any $\ee > 0$, we have $\log (2 + | a |) \le c_{\ee}(1 + | a
|)^{\ee}$. Hence, we bound \eqref{214g} by
%
%
%
%
\begin{equation*}
	\begin{split}
		c_{\ee}  \sum_{n_1 \neq 0} \frac{1}{|n_1|^{+(-2s)}}
		\sup_{n \neq 0} \sum_{n_1 \neq 0} \int_{\rr} \frac{1}{(1 +
		| \tau - n_{1}^m - (n - n_{1})^m |)^{1- \ee}(1 + | \tau - n^m
		|)^{\frac{1}{2}-}} d \tau
	\end{split}
\end{equation*}

%
which due to the estimate
%
%
\begin{equation}
	\label{216g}
	\begin{split}
		(1 + | \tau - n^{3} |)
		& = 1 + \frac{1}{4}| \tau - n^{3} | + \frac{3}{4}| \tau - n^{3} |
		\\
		& \ge 1 + \frac{1}{4}| \tau - n^{3} | + \frac{3}{4} \times
		\frac{1}{3}| n | |n_1 | n - n_1 |
		\\
		& = 1 + \frac{1}{4}| \tau - n^{3} | + \frac{1}{4}| -n^3 + n_1^3 + (n -
		n_1)^3 |
		\\
		& \ge \frac{1}{4}| \tau -n^3 + n_1^3 + (n - n_1)^{3} |
	\end{split}
\end{equation}
%
%
is bounded by
%
%
\begin{equation}
	\label{215g}
	\begin{split}
		& 4 c_{\ee} \sum_{n_1 \neq 0} \frac{1}{|n_1|^{(-2s)+}} \sup_{n \neq 0} \sum_{n_1 \neq 0}\int_{\rr} \frac{1}{(1 +
		| \tau - n_{1}^{3} - (n - n_{1})^{3} |)^{\frac{3}{2}^- - \ee}} d \tau
		\\
		& \lesssim \sum_{n_1 \neq 0} \frac{1}{|n_{1}|^{(-2s)+}} < \infty, \qquad s \ge
		-1/2,
	\end{split}
\end{equation}
%
%
completing the proof. \qquad \qedsymbol
%
%
\subsection{Case \ref{2pigeon-case-2}.} Recalling \eqref{21f}, we have
%
\begin{equation}
	\begin{split}
		& \sum_{n \neq 0} \int_{\rr} a_n |n|^s \left( 1 + | \tau - n^3 | \right)^{-1} | 
		\wh{w_{fg}}(n, \tau) | d \tau
		\\
		& \lesssim \sum_{n \neq 0}  \int_{\rr} a_{n} (1+ | \tau - n^{3} |)^{-1} \wh{\overset{\sim}{C_f} C_g} d
		\tau
	\\	
	& = \sum_{n \neq 0} \int_{\rr} a_{n} (1+ | \tau - n^{3} |)^{-5/8} (1 + | \tau - n^{3}
	|)^{-3/8} \wh{\overset{\sim}{C_f} C_g} d
		\tau
		\\
		& \le \|a_{n} (1 + | \tau - n^{3} |)^{-5/8}\|_{L^2(\zz \times \rr)}  \| (1 +
		| \tau - n^{3} |)^{-3/8} \wh{\overset{\sim}{C_f} C_g}  \|_{L^2(\zz \times
		\rr)}
		%\\
		%&\wh{\overset{\sim}{C_f} C_g}(n, \tau) d \tau
		%\\
		%& \le \|a_n\|_{\ell^2} \|\wh{\overset{\sim}{C_f} C_g}(n, \tau)\|_{L^2(\zz \times \rr)}
		%\\
		%& \simeq \|a_n\|_{\ell^2} \|\overset{\sim}{C_f} C_g\|_{L^2(\ci \times \rr)}
		%\\
		%& \le \|a_n\|_{\ell^2} \|\overset{\sim}{C_f}\|_{L^4(\ci \times \rr)} \|C_g\|_{L^4(\ci \times \rr)}
	\end{split}
\end{equation}
%
%
where the last step follows from Cauchy-Schwartz. A change of variable shows
that
%
%
\begin{equation*}
	\begin{split}
		\|a_{n} (1 + | \tau - n^{3} |)^{-5/8}\|_{L^2(\zz \times \rr)} \lesssim
		\|a_{n}\|_{\ell^2}
	\end{split}
\end{equation*}
%
%
while \eqref{23f}-\eqref{26f} yields the bound
%
%
\begin{equation*}
	\begin{split}
	\| (1 + | \tau - n^{3} |)^{-3/8} \wh{\overset{\sim}{C_f} C_g}  \|_{L^2(\zz
	\times \rr)} \lesssim \|f\|_{\dot{X}^s} \|g\|_{\dot{X}^s}
	\end{split}
\end{equation*}
%
%
completing the proof. \qquad \qedsymbol
%
%
%
\section{Proofs of Lemmas and Estimates}
\begin{proof}[Proof of Cutoff Lemma]
%
%
\begin{equation*}
	\begin{split}
		\lim_{t_{n} \to t} \|u(\cdot, t) - u(\cdot, t_{n})\|_{\dot{H}^s(\ci)} 
		& = \lim_{t_{n} \to t} \|\psi(t) u(\cdot, t) - \psi(t_n) u(\cdot,
		t_{n})\|_{\dot{H}^s(\ci)} 
		\\
		& = \lim_{t_n \to t} \left[ \sum_{n \in \zzdot}| n |
		^{2s} | \psi(t)  \wh{u}(n, t) - \psi(t_n) \wh{u}(n, t_n) |^2 \right]^{1/2}
		\\
		& = \lim_{t_n \to t} \left[ \sum_{n \in \zzdot} | n |^{2s} | \int_{\rr} (e^{it \tau} - e^{it_{n} \tau}) \wh{\psi u}(n,
		\tau) d \tau |^2 \right]^{1/2}.
	\end{split}
\end{equation*}
		It is clear that
		%
		%
		\begin{equation*}
			\begin{split}
				| n |
				^{2s} | \int_{\rr} (e^{it \tau} - e^{it_{n}\tau}) \wh{\psi u}(n, \tau) d \tau |^2 
		& \le 4  | n |^{2s} \left ( \int_{\rr} |\wh{\psi u}(n, \tau)| d \tau
		\right )^2 
	\end{split}
\end{equation*}
and 
%
%
\begin{equation*}
	\begin{split}
 \sum_{n \in \zzdot} | n |^{2s} \left ( \int_{\rr} |\wh{\psi u}(n, \tau)| d \tau
		\right ) ^2 
		& = \|\wh{\psi u}\|_{\dot{\ell}_n^2 L_\tau^1}
		\\
		& \le \|\psi u \|_{Y^s}^2 
	\end{split}
\end{equation*}
which is bounded by assumption.
Applying dominated convergence completes the proof. 
\end{proof}
%
%
%
\begin{proof}[Proof of General Multiplier Estimate]
  We are motivated by a proof of the case $v=2$ by Tzvetkov, as outlined by Tao \cite{Tao:2006el}. Observe that
  %
  %
  \begin{equation*}
  \begin{split}
    \| u \|_{L^{4}_{x}L^{4}_{t}}^{2} 
    &= \| u^{2} \|_{L^{2}_{x}, L^{2}_{t}}
    \\
    & = \| (\sum_{M}u_{M})^{2} \|_{L^{2}_{x}L^{2}_{t}}
  \end{split}
  \end{equation*}
  %
  %
  where $M$ is a dyadic integer, and $u_{M}$ is the portion of $u$ localized to the spacetime frequency region $M \le \langle \tau - k^{2} \rangle \le 2M$. Now 
  %
  %
  \begin{equation*}
  \begin{split}
    | \left( \sum_{M} u_{M} \right)^{2} |
    & = | \sum_{M} u_{M} \sum_{M'}u_{M'} |
    \\
    & = \sum_{M, M'} u_{M} u_{M'}
    \\
    & \simeq \sum_{m \ge 0} \sum_{M} u_{M}u_{2^{m}M}.
  \end{split}
  \end{equation*}
  %
  \begin{framed}
For fixed $M$, we have
  %
  %
  \begin{equation*}
  \begin{split}
    \sum_{M'} u_{M} u_{M'} = \sum_{m \in \zz} u_{M} u_{2^{m}M}
  \end{split}
  \end{equation*}
  %
  %
  and so
  %
  %
  \begin{equation*}
  \begin{split}
    \sum_{M, M'} u_{M}u_{M'}
    & = \sum_{m \in \zz} \sum_{M} u_{M} u_{2^{m}M}
    \\
    & = 2 \sum_{m \ge 0} \sum_{M} u_{M}u_{2^{m}M}.
  \end{split}
  \end{equation*}
  %
\end{framed}
  %
  %
  Also,
  %
  %
  \begin{equation*}
  \begin{split}
    \| u \|^{2}_{X^{0, b}} \sim \sum_{M} M^{2b} \| u_{M} \|^{2}_{L^{2}_{x}L^{2}_{t}}.
  \end{split}
  \end{equation*}
  %
  %
  So, it suffices to show
%
%
\begin{equation*}
\begin{split}
  \| \sum_{m \ge 0} \sum_{M} u_{M} u_{2^{m}M} \|_{L^{2}_{x}L^{2}_{t}}
  \lesssim M^{2b} \| u_{M} \|_{L^{2}_{x}L^{2}_{t}}^{2}
\end{split}
\end{equation*}
%
%
or, by the triangle inequality, that
%
%
\begin{equation*}
\begin{split}
  \sum_{M} \| u_{M} u_{2^{m}M} \|_{L^{2}_{x}L^{2}_{t}}
\lesssim 2^{-\ee m} \sum_{M} M^{2b} \| u_{M} \|_{L^{2}_{x}L^{2}_{t}}^{2}
\end{split}
\end{equation*}
where $\ee > 0$ is some fixed constant.
But by Cauchy-Schwartz,
%
%
\begin{equation*}
\begin{split}
  \sum_{M} M^{b} \| u_{M} \|_{L^{2}_{x}L^{2}_{t}} (2^{m}M)^{b} \| u_{M} \|_{L^{2}_{x}L^{2}_{t}} 
  & \le (\sum_{M} M^{2b} \| u_{M} \|_{L^{2}_{x}L^{2}_{t}}^{2})^{1/2}
  (\sum_{M} (2^{m}M)^{2b} \| u_{2^{m}M} \|_{L^{2}_{x}L^{2}_{t}}^{2})^{1/2}
  \\
  & = \sum_{M} M^{2b} \| u_{M} \|_{L^{2}_{x}L^{2}_{t}}^{2}
\end{split}
\end{equation*}
%
%
and so it will suffice to show
\begin{equation*}
\begin{split}
\| u_{M} u_{2^{m}M} \|_{L^{2}_{x}L^{2}_{t}}
& \lesssim 2^{-\ee m} M^{b} \| u_{M} \|_{L^{2}_{x}L^{2}_{t}}
(2^{m} M)^{b} \| u_{2^{m} M} \|_{L^{2}_{x}L^{2}_{t}}
\\
& = 2^{m(b - \ee)} M^{2b} \| u_{M} \|_{L^{2}_{x}L^{2}_{t}}
\| u_{2^{m} M} \|_{L^{2}_{x}L^{2}_{t}}.
\end{split}
\end{equation*}
%
Now by Parseval, Cauchy-Schwartz, and H\"older
%
%
\begin{equation*}
\begin{split}
\| u_{M} u_{2^{m}M} \|_{L^{2}_{x}L^{2}_{t}}
& \simeq \| \int_{\rr} \sum_{k_{1}} \wh{u_{M}}(\tau - \tau_{1}, k - k_{1}) \wh{u_{2^{m}M}}(\tau_{1}, k_{1}) \|_{L^{2}_{\tau} \ell^{2}_{k}}
\\
& \le \| \left( \int_{\rr} \sum_{k_{1}} | \wh{u_{M}}(\tau - \tau_{1}, k - k_{1}) |^{2} | \wh{u_{2^{m}M}}(\tau_{1}, k_{1}) |^{2} d \tau_{1} \right)^{1/2} 
\\
& \times \left ( \int_{\rr} \sum_{k_{1}} \chi_{M}(\tau - \tau_{1}, k - k_{1}) \chi_{2^{m}M}(\tau_{1}, k_{1}) d \tau_{1} \right ) ^{1/2} \|_{\ell^{2}_{k} L^{2}_{\tau}}
\\
& \le \left ( \sup_{\tau \in \rr, k \in \zz} \int_{\rr} \sum_{k_{1}}
\chi_{M}(\tau - \tau_{1}, k - k_{1}) \chi_{2^{m}M}(\tau_{1}, k_{1}) d \tau_{1} \right )^{1/2}  
\\
& \times \| \left( \int_{\rr} \sum_{k_{1}} | \wh{u_{M}}(\tau - \tau_{1}, k - k_{1}) |^{2} | \wh{u_{2^{m}M}}(\tau_{1}, k_{1}) |^{2} d \tau_{1} \right)^{1/2} \|_{\ell^{2}_{k} L^{2}_{\tau}}. 
\end{split}
\end{equation*}
%
%
where $\chi_{N}(\lambda, n)$ is an indicator function with support on the set of pairs $(\lambda, n)$ satisfying $N \le \langle \lambda - n^{v} \rangle  \le 2N$.  But
%
%
\begin{equation*}
\begin{split}
  \| \left( \int_{\rr} \sum_{k_{1}} | \wh{u_{M}}(\tau - \tau_{1}, k - k_{1}) |^{2} | \wh{u_{2^{m}M}}(\tau_{1}, k_{1}) |^{2} d \tau_{1} \right)^{1/2} \|_{\ell^{2}_{k}L^{2}_{\tau}} 
  & = \| u_{M} \|_{\ell^{2}_{k}L^{2}_{\tau}} \| u_{2^{m} M} \|_{\ell^{2}_{k}L^{2}_{\tau}} 
\end{split}
\end{equation*}
%
%
and so we have reduced to proving that for fixed $k \in \zz, \tau \in \rr$
%
%
\begin{equation*}
\begin{split}
\int_{\rr} \sum_{k_{1}}
\chi_{M}(\tau - \tau_{1}, k - k_{1}) \chi_{2^{m}M}(\tau_{1}, k_{1}) d \tau_{1}  \lesssim 2^{m(2b - 2 \ee)} M^{4b}
\end{split}
\end{equation*}
%
%
where the bound does not depend upon $k$ or $\tau$. Now for fixed $k, \tau$, we invoke Fubini, the symmetry of the convolution, and the support of $\chi_{N}$ to obtain
%
%
\begin{equation*}
\begin{split}
\int_{\rr} \sum_{k_{1}}
\chi_{M}(\tau - \tau_{1}, k - k_{1}) \chi_{2^{m}M}(\tau_{1}, k_{1}) d \tau_{1}  
& = 
\sum_{k_{1}}\int_{\rr} 
\chi_{M}(\tau_{1}, k_{1}) \chi_{2^{m}M}(\tau - \tau_{1}, k - k_{1}) d \tau_{1} 
\\
& = \sum_{k_{1 \in S}} \int_{k_{1}^{v} + M}^{k_{1}^{v} + 2M} d \tau_{1}
\\
& = M | S |
\end{split}
\end{equation*}
%
%
where
%
%
\begin{equation*}
\begin{split}
  S =  \{ & k_{1} \in \zz: \text{for every} \ \tau_{1} \in \rr \ \text{satisfying} \ 
  k_{1}^{v} + M \le \tau_{1} \le k_{1}^{v} + 2M, \ \text{we have}
  \\
  & 2^{m} M \le \langle \tau - \tau_{1} - (k - k_{1})^{v} \rangle  \le 2^{m+1}M \}.
\end{split}
\end{equation*}
%
Therefore, to complete the proof, it will be enough to show that
%
%
\begin{equation*}
\begin{split}
  | S | \lesssim 2^{m(2b - 2 \ee)}M^{4b-1}.
\end{split}
\end{equation*}
%
Proceeding, we observe that we reduce to counting the number of $k_{1}$ such that
%
%
%
\begin{equation*}
\begin{split}
  2^{m} M \le | \tau - k_{1}^{v} - cM  - (k - k_{1})^{v}| \le 2^{m+1}M
\end{split}
\end{equation*}
%
for all $c \in [1,2]$.
This is equivalent to counting the number of $k_{1}$ satisfying one of two equations 
%
%
%
\begin{equation*}
\begin{split}
  & 2^{m}M + cM - \tau \le k_{1}^{v} + (k - k_{1})^{v} \le 2^{m+1}M + cM - \tau,
  \\
  & 
  - 2^{m}M - cM  + \tau \ge   k_{1}^{v} + (k - k_{1})^{v}  \ge -2^{m+1}M - cM + \tau
\end{split}
\end{equation*}
%
%
%
%
If $v$ is even, then $k_{1}^{v}$ and $(k - k_{1})^{v}$ are positive, so we can restrict our attention to counting the number of $k_{1}$ satisfying one of two equations
\begin{equation}
  \label{ijj}
\begin{split}
  & 2^{m}M + cM - \tau \le k_{1}^{v}  \le 2^{m+1}M + cM - \tau,
  \\
  & 
  - 2^{m}M - cM  + \tau \ge   k_{1}^{v}  \ge -2^{m+1}M - cM + \tau
\end{split}
\end{equation}
For this relation to be satisfied, $k_{1}^{v}$ must live in an interval of length $2^{m}M$. Observe that there is an upper bound of  $4 \times \lfloor 2^{m/v} M^{1/v} \rfloor$ on the number of choices of $k_{1}$ in this case, where $\lfloor \cdot \rfloor$ denotes the nearest integer. Restricting $v$ such that $1/v \le (2b - 2 \ee)$ and $1/v \le 4b-1$ completes the proof.
\end{proof}
%
\begin{framed} The case of odd $v$ results in infinitely many $k_{1}$ when $k=0$. Hence, the proof falls apart in this case.
\end{framed}
%
%
%
%
%
\begin{proof}[Proof of \cref{2lem:schwartz-mult}]
Note that
%
%
\begin{equation*}
	\begin{split}
		\wh{\psi f}\left( n, \tau \right)
		& = \wh{\psi}(\cdot) * \wh{f}(n,
		\cdot)(\tau)
		= \int_\rr \wh{\psi}(\tau_1) \wh{f} \left( n, \tau - \tau_1 \right) 
		d\tau_1
	\end{split}
\end{equation*}
%
%
and hence
%
%
\begin{equation}
	\label{19b}
	\begin{split}
		\|\psi f\|_{\dot{X}^s} 
		& = \left( \sum_{n \in \zzdot} |n|^{2s} \int_\rr \left( 1 + | \tau -
		n^{m} | \right) | \int_\rr \wh{\psi}(\tau_1) \wh{f}\left( n, \tau -
		\tau_1
		\right)  d \tau_1 d \tau |^2 \right)^{1/2}
		\\
		& \le \left( \sum_{n \in \zzdot} |n|^{2s} \int_\rr \left( 1 + | \tau -
		n^{m }
		|
		\right) \left( \int_\rr \wh{\psi}\left( \tau_1 \right) \wh{f}\left( n,
		\tau - \tau_1
		\right)  d \tau_1 d \tau \right)^2 \right)^{1/2}.
	\end{split}
\end{equation}
%
%
Using the relation
%
%
\begin{equation*}
	\begin{split}
		1 + | \tau - n^{m } |
		& = 1 + | \tau + \tau_1 - n^{m} |
		\\
		& \le 1 + | \tau_1 | + | \tau - \tau_1 - n^{m} |
		\\
		& \le \left( 1 + | \tau_1 | \right)\left( 1 + | \tau - \tau_1 -
		n^{m} | \right),
	\end{split}
\end{equation*}
%
%
we obtain
%
%
\begin{equation*}
	\begin{split}
		\eqref{19b}
		& \le \left( \sum_{n \in \zzdot} |n|^{2s} \right.
		\\
		& \times \left . \int_\rr \left(
		\int_\rr \left( 1 + | \tau_1 | \right)^{1/2} | \wh{\psi}(\tau_1) |
		\left( 1 + | \tau - \tau_1 - n^{m} | \right)^{1/2} \wh{f}\left( n, \tau
		- \tau_1
		\right)d \tau_1
		\right)^2 d \tau \right)^{1/2}
	\end{split}
\end{equation*}
%
%
which by Minkowski's inequality is bounded by
%
%
\begin{equation}
	\label{18a}
	\begin{split}
		& \left( \sum_{n \in \zzdot} |n|^{2s}  \right.
		\\
		& \times \left. \left( \int_\rr \left[ \int_\rr
		\left( 1 + | \tau_{1} | \right) | \wh{\psi}(\tau_1) |^2 \left( 1 + |
		\tau - \tau_1 - n^{m} |
		\right) | \wh{f}\left( n, \tau - \tau_1 \right) |^2 d \tau_1 
		\right]^{1/2} d \tau \right)^2 \right)^{1/2}.
	\end{split}
\end{equation}
%
%
Using the change of variable $\tau - \tau_1 \to \lambda$ gives
%
%
\begin{equation*}
	\begin{split}
		\eqref{18a}
		& = \left( \sum_{n \in \zzdot} |n|^{2s}\right.
		\\
		& \times \left.  \left( \int_\rr \left[
		\int_\rr \left( 1 + | \tau_1 | \right) | \wh{\psi}\left( \tau_1
		\right) |^2 \left( 1 + | \lambda - n^{m} | \right) | \wh{f} \left( n,
		\lambda
		\right)|^2 d \tau_1 \right]^{1/2} d \lambda \right)^2 \right)^{1/2}
		\\
		& =  \left( \sum_{n \in \zzdot} |n|^{2s} \right.
		\\
		& \times \left. \left( \int_\rr \left( 1 + |
		\tau_1 |
		\right)^{1/2} | \wh{\psi}(\tau_1) | d \tau_1 \left[ \int_\rr \left( 1 + |
		\lambda - n^{m} |
		\right) | \wh{f}\left( n, \lambda \right) |^2 d \lambda \right]^{1/2}
		\right)^2 \right)^{1/2}
		\\
		& = c_{\psi} \left( \sum_{n \in \zzdot} |n|^{2s} \left( \left[ \int_\rr
		\left( 1 + | \lambda - n^{m} | \right) | \wh{f}\left( n, \lambda
		\right) |^2 d \lambda
		\right]^{\cancel{1/2}} \right)^{\cancel{2}} \right)^{1/2}
		\\
		& = c_{\psi} \|f\|_{\dot{X}^s},
	\end{split}
\end{equation*}
%
%
concluding the proof. 
\end{proof}

%
\begin{proof}[Proof of Number Theory Lemma]
First note that
%
\begin{equation*}
		| - n^{3} + n_1^3 + n_2^3|
		 = 3 | n | |n_1 | |n_2 |.
\end{equation*}
%
%
Hence, it will be enough to show that for $c \ge 0$
%
%
\begin{equation*}
	\begin{split}
		| n | |n_1 | |n_2 | \gtrsim | n |^{\frac{2 + c}{2}}| n_1
		|^{\frac{2-c}{2}}| n_2 |^{\frac{2-c}{2}}
	\end{split}
\end{equation*}
%
%
or, dividing through on both sides by $|n| | n_1 | | n_2 |$ and rearranging terms
%
%
\begin{equation*}
	\begin{split}
		| n |^{c/2} \lesssim | n_1 |^{c/2} | n_2 |^{c/2}.
	\end{split}
\end{equation*}
%
%
But
%
%
\begin{equation*}
	\begin{split}
		| n |^{c/2} &= | n_1 + n_2 |^{c/2}
		\\
		& \le (| n_1 | + |n_2|)^{c/2} 
		\\
		& \le (2\max\{|
		n_1 |, | n_2 |)^{c/2}
		\\
		& \le (2|
		n_1 | | n_2 |)^{c/2}
		\\
		& = 2^{c/2} | n_1 |^{c/2} | n_2 |^{c/2}
	\end{split}
\end{equation*}
%
%
where the last step follows from the fact that, for $a, b \in \zz$ 
%\cref{1lem:splitting}. \qquad \qed
%
%\subsection{Proof of \cref{1lem:splitting}.} We have
%%
%%
\begin{equation}
	\label{16a}
	\begin{split}
		| a + b | 
		& \le | a | + | b | 
		\\
		& \le 2\left( \max\{| a |, | b | \}\right)
		\\
		& \le 2 |a| |b|.
	\end{split}
\end{equation} 
This concludes the proof.
\end{proof}
%%
%
