\chapter{Well-Posedness for the mNLS}
%\author{Alex Himonas, David Karapetyan, and Gerson Petronilho}
%\address{Department of Mathematics  \\
%University  of Notre Dame\\
%Notre Dame, IN 46556 }
%\address{Department of Mathematics \\
%University  of Notre Dame\\
%Notre Dame, IN 46556 }
%\address{Departamento de Matemática \\
%Universidade Federal de São
%Carlos \\
%Rodovia Washington Luiz, Km 235, São Carlos, SP,
%13565-905, Brasil}
				  %
				  %
				  %
				  %
				  %
				  %
				  \section{Introduction}
				  We consider the modified nonlinear Schr\"{o}dinger (mNLS) 
				  initial value problem (ivp)
%
%
\begin{gather}
	\label{nmNLS-eq}
	i \p_t u + \p_x^{m} u + \lambda |u|^2 u =0,
		\\
		\label{nmNLS-init-data}
		u(x,0) = \vp(x) \in H^s(\ci), \ \ t \in \rr, \ \ x \in \ci
\end{gather}
%
%
where $m \in \{2,4,6,\dots\}$ and $\lambda \in \{-1, 1\}$. 
%
%
%
%
%
%
%%%%%%%%%%%%%%%%%%%%%%%%%%%%%%%%%%%%%%%%%%%%%%%%%%%%%
%
%
%				Outline
%
%
%%%%%%%%%%%%%%%%%%%%%%%%%%%%%%%%%%%%%%%%%%%%%%%%%%%%%
%
%
%
%
%
%
%
To derive a weak formulation of the mNLS ivp, we first let
$\ci = [0, 2 \pi]$, and use
the following notation for the Fourier transform
%
%
%
%
\begin{equation}
	\label{nfour-trans-pde}
	\begin{split}
    \widehat{f}(n) = \int_{\ci} e^{-ix n} f(x) \, dx.
	\end{split}
\end{equation}
Applying the spatial Fourier transform to the mNLS ivp we obtain
%
%
\begin{gather}
  \p_t \widehat{u}(n, t) = i^{m+1} n^m \widehat{u}(n, t) + \lambda i  
	\widehat{w} (n, t),
  \label{neq-1}
	\\
	\widehat{u} (n,0) = \widehat{\vp}(n)
  \label{neq-1-init-data}.
\end{gather}
%
%
Similarly, we have 
\begin{gather*}
  \p_t [\widehat{u}(n, -t)] =
  i^{m+1} n^m \widehat{u}(n, -t) + \lambda i  
	\widehat{w} (n, -t),
  \\
  \widehat{u} (n,0) = \widehat{\vp}(n)
\end{gather*}
or
\begin{gather}
  \label{neq-2}
  \p_t \widehat{u}(n, -t) = -i^{m+1} n^m \widehat{u}(n, -t) - \lambda i  
	\widehat{w} (n, -t)
  \\
\widehat{u} (n,0) = \widehat{\vp}(n)
  \label{neq-2-init-data}.
\end{gather}
Since \eqref{neq-1}-\eqref{neq-1-init-data} and
\eqref{neq-2}-\eqref{neq-2-init-data} are equivalent initial value problems, we
may without loss of generality restrict our attention to
\eqref{neq-1}-\eqref{neq-1-init-data} when $m \in \left\{ 2, 6, 10, \dots
\right\}$, and \eqref{neq-2}-\eqref{neq-2-init-data} when $m \in \left\{ 4, 8, 12,
\dots
\right\}$. That is, it is enough to consider the initial value problem   
%
%
\begin{gather}
  \label{nhk}
  \p_t \widehat{u}(n, t) = -i n^m \widehat{u}(n, t) + (-1)^{\frac{m+2}{2}} \lambda i  
  \widehat{w} (n, t)	\\
	\widehat{u} (n,0) = \widehat{\vp}(n)
\end{gather}
for $m \in \left\{ 2, 4, 6, \dots \right\}$.
Assume $\lambda = (-1)^{\frac{m+2}{2}}$; we will address the case $\lambda =
(-1)^{\frac{m}{2}}$ later. Then from the above, we obtain 
the ivp
\begin{gather}
  \label{nfour}
  \p_t \widehat{u}(n, t) = -i n^m \widehat{u}(n, t) + i  
  \widehat{w} (n, t)
	\\
  \label{nfour-init-data}
	\widehat{u} (n,0) = \widehat{\vp}(n)
\end{gather}
for $ m \in \left\{ 2, 4, 6, \dots \right\}$.
Multiplying \eqref{nfour} by the integrating factor $e^{itn^m}$ then yields
%%
%%
\begin{equation*}
	\begin{split}
		\left[ e^{ it n^m} \widehat{u}(n) \right]_t = i
		 e^{ it n^m} \widehat{w} (n, t).	
	\end{split}
\end{equation*}
%
%
Integrating from $0$ to $t$, we obtain
%
%
\begin{equation*}
	\begin{split}
		\wh{u}(n, t) = \wh{\vp}(n) e^{- it n^m} + i  
		\int_0^t e^{ i(t' - t) n^m} \wh{w}(n, t') \ 
		dt'.
	\end{split}
\end{equation*}
%
%
Therefore, by Fourier inversion 
%
%
\begin{equation}
	\label{nmNLS-integral-form}
	\begin{split}
		u(x,t) & = \frac{1}{2\pi} \sum_{n \in \zz} \wh{\vp}(n) e^{i\left( xn - t n^m 
		\right)} 
		\\
    & + \frac{i}{2 \pi} \sum_{n \in \zz} \int_0^t e^{i\left[ xn + \left( t' - t 
		\right) n^m \right]} \wh{w}(n, t') \ dt'.
	\end{split}
\end{equation}
%
%
Note that \eqref{nmNLS-integral-form} is a weaker 
restatement of the Cauchy-problem \eqref{nmNLS-eq}-\eqref{nmNLS-init-data}, 
since by construction any classical solution of the mNLS 
ivp is a solution to \eqref{nmNLS-integral-form}. 
%
%
We now derive an integral 
equation global in $t$ and equivalent to \eqref{nmNLS-integral-form} for $t 
\in [-\delta, \delta]$. Let $\psi(t)$ be a cutoff function symmetric about the 
origin such that $\psi(t) = 1$ for $|t| \le 1/2$ and $\text{supp} \, \psi 
= [-1, 1 ]$. Define $\psi_{\delta}(t) = \psi(2t/\delta)$.  Multiplying both sides of expression
$\eqref{nmNLS-integral-form}$ by $\psi_{\delta}(t)$, we obtain
%
%
\begin{equation}
	\begin{split}
		\label{ncutoff-int-eq}
    \psi_{\delta} u(x, t)
		& = \frac{1}{2 \pi} \psi_{\delta}(t) \sum_{n \in \zz} e^{i(xn - t n^{m })} \widehat{\vp}(n) 
		\\
		& + \frac{i }{2 \pi} \psi_{\delta}(t) \int_0^t \sum_{n \in \zz} 
		e^{i\left[ xn + (t - t')n^m \right]} \wh{w}(n, t') \ dt'.
	\end{split}
\end{equation}
%
%
Noting that $e^{i\left( xn + tn^{m } \right)}$ 
does not depend on $t'$, we may rewrite the second term
%
%
\begin{equation}
	\label{npre-prim-int-form}
	\begin{split}
		& \frac{i }{2 \pi} \psi_{\delta}(t) \int_0^t \sum_{n \in \zz} 
		e^{i\left[ xn + (t - t') n^m \right]} \wh{w}(n, t') \ dt'
		\\
		& = \frac{i}{2 \pi} \psi_{\delta}(t) \sum_{n \in \zz} e^{i\left( xn + t 
		 n^{m } 
		\right)} \int_0^t e^{- it'n^{m }} \wh{w}(n, t') \ dt'.
	\end{split}
\end{equation}
%%
%%
We remark that this is a \emph{global} relation in $t$. Therefore, by Fourier 
inversion
%
%
%
%
%
%
%
\begin{equation*}
	\begin{split}
		\text{rhs of} \; \eqref{npre-prim-int-form}
		& = \frac{i}{4 \pi^2} \psi (t) \sum_{n \in \zz} e^{i\left( xn + t 
		 n^m
		\right)} \int_0^t \int_\rr e^{it'\left( \tau - n^m \right) }
		\wh{w}(n, \tau) d \tau dt'
		\\
		& = \frac{1}{4 \pi^2} \psi_{\delta}(t) \sum_{n \in \zz} \int_\rr 
		e^{i\left( xn + tn^m \right)} \frac{e^{it\left( \tau - n^m 
		\right)}-1}{\tau - n^m} \wh{w}(n, \tau) d \tau
	\end{split}
\end{equation*}
%
%
where the last step follows from Fubini and integration. Substituting
into \eqref{ncutoff-int-eq} we obtain
%
%
\begin{equation}
	\begin{split}
		\label{ncutoff-int-eq-2}
    \psi_{\delta} u(x, t)
		& = \frac{1}{2 \pi} \psi_{\delta}(t) \sum_{n \in \zz} e^{i(xn - tn^{m })} \widehat{\vp}(n) 
		\\
		& + \frac{1}{4 \pi^2} \psi_{\delta}(t) \sum_{n \in \zz} \int_\rr
		e^{i(xn + t n^m)} \frac{e^{it(\tau - n^m)}- 1}{\tau - n^m} 
		\wh{w}(n, \tau) \ d \tau.
	\end{split}
\end{equation}
%
%
%
Next, we localize near the singular curve $\tau =  n^m$.  Multiplying the
summand of the second term of \eqref{ncutoff-int-eq-2} by $1 + \psi(\tau -
n^m) - \psi(\tau -
n^m) $ and
rearranging terms, we have
%
%
\begin{equation*}
	\begin{split}
    \psi_{\delta} u(x, t)
		& = \frac{1}{2 \pi} \psi_{\delta}(t) \sum_{n \in \zz} e^{i(xn + t n^{m 
		})} \widehat{\vp}(n) 
		\\
		& + \frac{1}{4 \pi^2} \psi_{\delta}(t) \sum_{n \in \zz} \int_\rr e^{ixn}  
		e^{it \tau} \frac{ 1 - \psi(\tau - n^m) 
		}{\tau - n^m} \wh{w}(n, \tau) \ d \tau
		\\
		& - \frac{1}{4 \pi^2} \psi_{\delta}(t) \sum_{n \in \zz} \int _\rr e^{i(xn + 
		t n^m)}
		 \frac{1- \psi(\tau - n^m)}{\tau - n^m} \wh{w}(n, \tau) \ d \tau
		\\
		& + \frac{1}{4 \pi^2} \psi_{\delta}(t) \sum_{n \in \zz} \int_\rr
		e^{i(xn + t n^m)}
		\frac{\psi(\tau - n^m)\left[ e^{it(\tau - n^m)}-1 
		\right]}{\tau - n^m} \wh{w}(n, \tau) \ d \tau
	\end{split}
\end{equation*}
%
%
which by a power series expansion of $[e^{it(\tau - n^m)}-1]$ simplifies  
to
%
%
\begin{align}
	\label{nmain-int-expression-0}
  & \psi_{\delta} u(x, t) 
		\\
		\label{nmain-int-expression-1}
		& = \frac{1}{2 \pi} \psi_{\delta}(t) \sum_{n \in \zz} e^{i(xn + tn^{m 
		})} \widehat{\vp}(n) 
		\\
		\label{nmain-int-expression-2}
		& + \frac{1}{4 \pi^2} \psi_{\delta}(t) \sum_{n\in \zz} \int_\rr e^{ixn}  
		e^{it \tau} \frac{ 1 - \psi(\tau -  n^m) 
		}{\tau -  n^m} \wh{w}(n, \tau) \ d \tau
		\\
		\label{nmain-int-expression-3}
		& - \frac{1}{4 \pi^2} \psi_{\delta}(t) \sum_{n\in \zz} \int_\rr e^{i(xn + 
		t n^m)}
		 \frac{1- \psi(\tau -  n^m)}{\tau -  n^m} \wh{w}(n, \tau) \ d \tau
		\\
		\label{nmain-int-expression-4}
		& + \frac{1}{4 \pi^2} \psi_{\delta}(t) \sum_{k \ge 1} \frac{i^k t^k}{k!}
		\sum_{n \in \zz} \int_\rr e^{i(xn + t n^m )}
		\psi(\tau -  n^m) (\tau -  n^m)^{k-1} \wh{w}(n, \tau)  
		\\
		& \doteq T(u) \notag
\end{align}
%
%
where $T = T_{\vp, \psi, \delta}$. 
\begin{definition}
  Given a Banach space $X$, denote $B_{X}(R) \doteq \left\{ f: \| f \|_{X} < R
      \right\}$. We say that the mNLS ivp \eqref{nmNLS-eq}-\eqref{nmNLS-init-data} is
	\emph{locally well posed} in
	$H^s(\ci)$ if 
	\begin{enumerate}
    \item For every $\vp(x) \in B_{H^{s}(\ci)}(R)$
      there exists $\delta>0$ depending on $R$ and a unique $u \in C([-\delta,
      \delta], H^s(\ci))$ satisfying
      \eqref{nmain-int-expression-0}-\eqref{nmain-int-expression-4}.
    \item
      The flow map $u_0 \mapsto u(t)$ is uniformly continuous from
      $B_{H^{s}(\ci)}(R)$ 
      to $C(\left[ -\delta, \delta \right], H^s(\ci))$. That is, if
      $\{u_{0,n} \}, \{v_{0,n}\} \subset B_{H^{s}(\ci)}(R)$ such that $\|u_{0,n} -
      v_{0,n} \|_{H^{s}(\ci)} \to 0$, then \\
      $\sup_{t \in [-\delta, \delta]}
      \|u_{n}(\cdot, t) - v_{n}(\cdot, t) \|_{H^s(\ci)} \to 0$.
  \end{enumerate}
	Otherwise, we say that the mNLS ivp is \emph{ill-posed}.
\end{definition}
%
\begin{definition}
  Let $\mathcal{Y}$ be the space of functions $F(\cdot)$ such that
  \begin{enumerate}[(i)]
   \item{$F: \ci \times \rr \to \cc$ }.
   \item{ $F(x, \cdot) \in \mathcal{S}(\rr)$ for each $x \in \ci$}.
   \item{ $F(\cdot, t) \in C^{\infty}(\ci)$for each $t \in \rr$}.
  \end{enumerate}
  Let $Y^{s}$ denote the completion of $\mathcal{Y}$ with
  respect to the norm
  %
  %
  \begin{equation}
	\label{nY-s-norm}
	\begin{split}
		\|u\|_{Y^s} = \|u\|_{X^s} + \|\wh{u}\|_{ \ell^2_n L^1_\tau }
	\end{split}
\end{equation}
  %
    %
    where
\begin{equation}
	\label{nX^s-norm}
	\begin{split}
		& \|u\|_{X^s}
		= \left ( \sum_{n\in \zz} \left (1 + |n| \right )^{2s} \int_\rr \left ( 1 + | 
		\tau - n^{m } | \right ) | \wh{u} ( n, \tau ) |^2
		\right )^{1/2}
	\end{split}
\end{equation}
and
%
%
\begin{equation}
	\label{nE-norm}
	\|\wh{u}\|_{ \ell^2_n L^1_\tau } = \left[ \sum_{n \in \zz}(1 + | n |)^{2s} \left(
	\int_{\rr}| \wh{u}(n, \tau) |d \tau \right)^{2} \right]^{1/2}.
\end{equation}
    %
  \end{definition}

%
The $Y^s$ spaces have the following important property, whose proof
is provided in the appendix.
\begin{lemma}
	\label{nlem:cutoff-loc-soln}
	Let $\psi(t)$ be a smooth cutoff function with $\psi(t) =1$ for $t \in [-1,
  1]$, and define $\psi_{\delta}(t) = \psi(t/\delta)$. If
  $\psi_{\delta}(t)u(x,t) \in Y^s$, then $u \in C([-\delta, \delta], H^s(\ci))$.
\end{lemma}

We are now prepared to state the main result of this paper.
%
%
%
%
%%%%%%%%%%%%%%%%%%%%%%%%%%%%%%%%%%%%%%%%%%%%%%%%%%%%%
%
%
%	Main Result				
%
%
%%%%%%%%%%%%%%%%%%%%%%%%%%%%%%%%%%%%%%%%%%%%%%%%%%%%%
%
%
\begin{theorem}
\label{nthm:main}
The initial value problem 
\eqref{nmNLS-eq}-\eqref{nmNLS-init-data} is locally well-posed in $H^s(\ci)$ for $s \ge
0$, and ill-posed for $s <0$. %
%
\end{theorem} 
%
%
%
%
%
%
%%%%%%%%%%%%%%%%%%%%%%%%%%%%%%%%%%%%%%%%%%%%%%%%%%%%%
%
%
%			Proof of Theorem	
%
%
%%%%%%%%%%%%%%%%%%%%%%%%%%%%%%%%%%%%%%%%%%%%%%%%%%%%%
%
%
\section{Proof of Main Theorem}
%
%
To prove well-posedness for the mNLS ivp we we will 
show that for initial data $\vp \in B_{H^{s}(\ci)}(R)$, $T$ is a contraction on
$B_{Y^{s}}(M_{R})$ by estimating the $Y^s$
norm of \eqref{nmain-int-expression-1}-\eqref{nmain-int-expression-4}. The 
Picard fixed point theorem will
then yield a unique solution to
$u = Tu$. An application of
\cref{nlem:cutoff-loc-soln} will then imply the existence of a unique, local
solution $u \in C([-\delta, \delta], H^s(\ci))$ to
$\psi_{\delta} u = Tu$. Lipschitz continuity of the flow map (and hence, uniform
continuity) will follow from estimates used to establish the contraction
mapping. The proof of ill-posedness will follow via counterexample 
by selecting an appropriate sequence of
solutions to \eqref{nmNLS-eq}-\eqref{nmNLS-init-data}. 
%
\begin{framed}
\begin{remark}
  To cover the case $\lambda = (-1)^{\frac{m}{2}}$ in \eqref{nhk}, we need only
  replace $i$ by $-i$ in
  expressions \eqref{nmain-int-expression-2}-\eqref{nmain-int-expression-4}. This
  does not change their $Y^s$ norms. Hence, we may assume $\lambda =
  (-1)^{\frac{m+2}{2}}$ without loss of generality throughout this paper.
\label{nrem:lambda-arb}
\end{remark}
\end{framed}

%%%%%%%%%%%%%%%%%%%%%%%%%%%%%%%%%%%%%%%%%%%%%%%%%%%%%
%
%
%		Estimation of Integral Equality Part 1		
%
%
%%%%%%%%%%%%%%%%%%%%%%%%%%%%%%%%%%%%%%%%%%%%%%%%%%%%%
%
%
%
%
\subsection{Estimate for \eqref{nmain-int-expression-1}.}
%
%
Letting $f(x,t) = \psi_{\delta}(t) \sum_{n \in \zz} e^{i(xn + tn^{m})} 
\wh{\vp}(n)$, we have $\wh{f}(n,t) = \psi_{\delta}(t) \wh{\vp}(n) e^{itn^{m}}$,
from which we obtain
%
%
\begin{equation}
	\label{nfourier-trans-calc}
	\begin{split}
		\wh{f}(n, \tau)
		& = \wh{\vp}(n) \int_\rr e^{-it( \tau - n^{m})} 
		\psi_{\delta}(t) \ d t
    = \wh{\psi_{\delta}}(\tau - n^{m}) \wh{\vp}(n).
	\end{split}
\end{equation}
%
%
%
%
%
%
Therefore
%
\begin{equation}
	\begin{split}
	\label{nmain-int1-est}
		\|\eqref{nmain-int-expression-1}\|_{Y^s}
		& \simeq \left (  \sum_{n\in \zz} \left (1 + |n| \right )^s \int_\rr \left( 1 + | \tau - n^{m} 
		| \right )
    | \wh{\psi_{\delta}}(\tau - n^{m}) \wh{\vp}(n) |^2 d \tau \right)^{1/2} 
		\\
		& + \left[ \sum_{n \in \zz }\left( 1 + | n | \right)^{2s} \left( \int_{\rr} |
    \wh{\psi_{\delta}}(\tau - n^{m})\wh{\vp}(n) | d \tau
		\right)^{2} \right]^{1/2}
		\\
    & = c_{\psi, \delta}
		\|\vp\|_{H^s(\ci)}.
	\end{split}
\end{equation}
%
%
%
%
\subsection{Estimate for \eqref{nmain-int-expression-2}.}
We now need the following lemma, whose proof is provided in the appendix.
%
%
%%%%%%%%%%%%%%%%%%%%%%%%%%%%%%%%%%%%%%%%%%%%%%%%%%%%%
%
%
%			Schwartz Multiplier	
%
%
%%%%%%%%%%%%%%%%%%%%%%%%%%%%%%%%%%%%%%%%%%%%%%%%%%%%%
%
%
\begin{lemma}
\label{nlem:schwartz-mult}
	For $\psi \in S(\rr)$,
%
%
\begin{equation}
	\label{nschwartz-mult}
	\begin{split}
		\|\psi f \|_{Y^s} \le c_{\psi} \|f \|_{Y^s}.
	\end{split}
\end{equation}
%
%
\end{lemma}
%
%
Hence,
%
%
\begin{equation}
  \label{nyu}
	\begin{split}
		\|\eqref{nmain-int-expression-2}\|_{Y^s} 
    & \le c_{\psi, \delta}
		\| \sum_{n \in \zz} e^{ixn} \int_\rr 
		e^{it \tau} \frac{ 1 - \psi (\tau - n^{m} ) 
		}{\tau - n^{m}} \wh{w}(n, \tau) \ 
		d \tau\|_{Y^s}.
			\end{split}
\end{equation}
%
We first estimate
%
%
\begin{equation}
\label{nmain-int2-est-X-s-part}
\begin{split}
  & \| \sum_{n \in \zz} e^{ixn} \int_\rr 
		e^{it \tau} \frac{ 1 - \psi (\tau - n^{m} ) 
		}{\tau - n^{m}} \wh{w}(n, \tau) \ 
		d \tau\|_{X^s}
		\\
    & = \left( \sum_{n \in \zz} \left (1 + |n| \right )^{2s} \int_\rr
		(1 + |\tau - n^{m}|) \left | \frac{1 - \psi(\tau - n^{2 
		})}{\tau - n^{m}} 
		\wh{w}(n, \tau) \right |^2 \ d 
		\tau \right)^{1/2}
		\\
		& \le \left( \sum_{n \in \zz} \left (1 + |n| \right )^{2s} \int_{| \tau - n^{m }| \ge 1}
		(1 + |\tau - n^{m}|) \frac{|\wh{w}(n, \tau)|^2 }{|\tau - n^{m }|^2} 
		\ d 
		\tau \right)^{1/2}
		\\
		& \lesssim  \left( \sum_{n \in 
		\zz} \left (1 + |n| \right )^{2s} \int_\rr
		\frac{|\wh{w}(n, \tau) |^2}{1+ |\tau - 
		n^{m}|} 
		 \ d \tau 
		\right)^{1/2}
		\\
		& \lesssim  \|u\|_{X^s}^3
\end{split}
\end{equation}
%
%
%
where the last two steps follow from the inequality 
%
\begin{equation}
	\label{none-plus-ineq}
	\begin{split}
		\frac{1}{|\tau - n^{m}| } \le \frac{2}{1 + |\tau - n^{m}| }, 
		\qquad |\tau - n^{m}| \ge 1
	\end{split}
\end{equation}
%
%
and the following trilinear estimate, whose proof we leave for later.
%
%
%%%%%%%%%%%%%%%%%%%%%%%%%%%%%%%%%%%%%%%%%%%%%%%%%%%%%
%
%
%				Proposition
%
%
%%%%%%%%%%%%%%%%%%%%%%%%%%%%%%%%%%%%%%%%%%%%%%%%%%%%%
%
%
\begin{proposition}
\label{nprop:trilinear-est}
	%
	%
	For any $s \ge 0$ and $b \ge 3/8$, we have
	\begin{equation}
		\left( \sum_{n \in \zz} \left (1 + |n| \right )^{2s} \int_\rr
		\frac{|\wh{w_{fgh}}(n, \tau) |^2}{\left (1+ |\tau - 
    n^{m}| \right )^{2b}} 
		 \ d \tau 
		\right)^{1/2}
		\lesssim \|f\|_{X^s} \|g\|_{X^s}\|h\|_{X^s}
	\end{equation}
	where $w_{fgh}(x,t)$ = $fg \bar h (x,t)$.
%
%
%
%
\end{proposition}
%
%
Furthermore,
%
%
%
%
\begin{equation}
	\label{nmain-int-expression-2-Y-s-part}
	\begin{split}
    & \| \wh{\eqref{nmain-int-expression-2}}\|_{\ell^{2}_{n}L^{1}_{\tau}}
		\\
    & \simeq \left[ \sum_{n \in \zz}(1 + | n |)^{2s} \left(
		\int_{\rr}\frac{|1 - \psi(\tau - n^{m})|}{|\tau - n^{m}|} |\wh{w}(n, \tau)| d
		\tau \right)^{2} \right]^{1/2}
		\\
    & \le \left[ \sum_{n \in \zz}(1 + | n |)^{2s} \left(
    \int_{| \tau - n^{m} | \ge 1 }\frac{|\wh{w}(n, \tau)|}{|\tau - n^{m}|}  d
		\tau \right)^{2} \right]^{1/2}
    \\
    & \lesssim \left[ \sum_{n \in \zz}(1 + | n |)^{2s} \left(
    \int_{\rr}\frac{|\wh{w}(n, \tau)|}{1 + |\tau - n^{m}|}  d
		\tau \right)^{2} \right]^{1/2}
    \\
		& \lesssim \|f\|_{X^s} \|g\|_{X^s}\|h\|_{X^s}
	\end{split}
\end{equation}
%
%
where the last two steps follow from \eqref{none-plus-ineq} and the following
corollary to \cref{nprop:trilinear-est}.
%
%
%%%%%%%%%%%%%%%%%%%%%%%%%%%%%%%%%%%%%%%%%%%%%%%%%%%%%
%
%
%				Second trilinear Estimate 
%
%
%%%%%%%%%%%%%%%%%%%%%%%%%%%%%%%%%%%%%%%%%%%%%%%%%%%%%
%
%
\begin{corollary}
\label{ncor:trilinear-estimate2}
	For $s \ge 0$ we have
%
%
\begin{equation}
	\label{ntrilinear-estimate2}
	\begin{split}
		\left( \sum_{n \in \zz} \left (1 + |n| \right )^{2s}  \left ( \int_\rr 
		\frac{|\wh{w_{fgh}}(n, \tau) |}{1 + | \tau - n^{m } |}
		 \ d\tau \right)^2  \right)^{1/2} \lesssim \|f\|_{X^s} \|g\|_{X^s}\|h\|_{X^s}.
	\end{split}
\end{equation}
\end{corollary}
%
%
Combining \eqref{nyu}, \eqref{nmain-int2-est-X-s-part}, and
\eqref{nmain-int-expression-2-Y-s-part}, we conclude that
%
%
%
%
\begin{equation}
	\label{nmain-int2-est}
	\begin{split}
		\|\eqref{nmain-int-expression-2}\|_{Y^s} \le c_{\psi, \delta}\|f\|_{X^s} \|g\|_{X^s}\|h\|_{X^s}.
	\end{split}
\end{equation}
%
%
\subsection{Estimate for \eqref{nmain-int-expression-3}.}
Letting $$f(x,t) = \psi_{\delta}(t) \sum_{n \in \zz} e^{i\left( xn + tn^{m} \right)} 
\int_\rr \frac{1 - \psi\left( \lambda - n^{m} \right)}{\lambda - n^{m}} 
\wh{w} \left( n, \lambda \right) \ d \lambda,$$ we have
%
%
\begin{equation*}
	\begin{split}
		& \wh{f^x}(n, t) = \psi_{\delta}(t) e^{itn^{m}} \int_\rr
		\frac{1 - \psi\left( \lambda - n^{m} \right)}{\lambda - n^{m}} 
		\wh{w}(n, \lambda) \ d \lambda
	\end{split}
\end{equation*}
and
\begin{equation*}
	\begin{split}
		 \wh{f}\left( n, \tau \right)
		 & = \int_\rr e^{-it\left( \tau - n^{m} 
		\right)} \psi_{\delta}(t) \int_\rr \frac{1 - \psi\left( 
		\lambda - n^{m} 
		\right)}{\lambda - n^{m}} \wh{w}(n, \lambda) \ d \lambda d \tau
		\\
    & = \wh{\psi_{\delta}}\left( \tau - n^{m} \right) \int_\rr 
		\frac{1 - \psi\left( 
		\lambda - n^{m} 
		\right)}{\lambda - n^{m}} \wh{w}(n, \lambda) \ d \lambda.
	\end{split}
\end{equation*}
Therefore,
%
%
\begin{equation}
  \label{niu}
	\begin{split}
		& \| \eqref{nmain-int-expression-3} \|_{X^s} 
		\\
		& \simeq \left( \sum_{n \in \zz} \left (1 + |n| \right )^{2s} \int_\rr \left( 1 + | \tau - n^{m
    } \right ) | | \wh{\psi_{\delta}}\left( \tau - n^{m } \right) |^2 \ d \tau
		\right.
		\\
		& \times \left . |
		\int_\rr \frac{1 - \psi\left( \lambda - n^{m } \right)}{\lambda -
		n^{m }} \wh{w}(n, \lambda) \ d \lambda |^2  \right)^{1/2}
		\\
    & \le c_{\psi, \delta}\left( \sum_{n \in \zz} \left (1 + |n| \right )^{2s} | \int_\rr
		\frac{1 - \psi\left( \lambda - n^{m } \right)}{\lambda - n^{m }}
		\wh{w}(n, \lambda) \ d\lambda |^2 \right)^{1/2}
		\\
		& \simeq \left( \sum_{n \in \zz} \left (1 + |n| \right )^{2s}  \left ( \int_\rr
		\frac{1 - \psi\left( \lambda - n^{m } \right)}{|\lambda - n^{m }|}
		|\wh{w}(n, \lambda) | \ d\lambda \right )^2 \right)^{1/2}
		\\
		& \le \left( \sum_{n \in \zz} \left (1 + |n| \right )^{2s}  \left ( \int_{| \lambda - 
		n^{m } | \ge 1}
		\frac{|\wh{w}(n, \lambda) | }{|\lambda - n^{m }|}
		\ d\lambda \right )^2 \right)^{1/2}.
	\end{split}
\end{equation}
%
%
Applying estimate \eqref{none-plus-ineq}, we bound this by
%
%%
\begin{equation}
	\label{nmain-int3-est-X-s-part}
	\begin{split}
		& 2 \left( \sum_{n \in \zz} \left (1 + |n| \right )^{2s}  \left ( \int_\rr
		\frac{|\wh{w}(n, \lambda)| }{1 + |\lambda - n^{m }|}
		 \ d\lambda \right )^2 \right)^{1/2}
		 \\
		& \lesssim \|u\|_{X^s}^3
	\end{split}
\end{equation}
%
%%
where the last step follows from \cref{ncor:trilinear-estimate2}.
Furthermore, 
%
%
\begin{equation}
	\label{nmain-int-estimate-3-Y-s-part}
	\begin{split}
    \|\wh{\eqref{nmain-int-expression-3}}\|_{\ell^{2}_{n}L^{1}_{\tau}}
		& \simeq \left[ \sum_{n \in \zz} (1 + | n |)^{2s} \int_{\rr} |
    \wh{\psi_{\delta}}(\tau - n^{m}) |^{2} \left( \int_{\rr}\frac{1 - \psi(\lambda -
		n^{m})}{\lambda - n^{m}} \wh{w}(n, \lambda) d \lambda \right)^{2} d \tau
		\right]^{1/2}
		\\
		& \le c_{\psi, \delta} \left[ \sum_{n \in \zz} (1 + | n |)^{2s} \left(
		\int_{\rr} \frac{1 - \psi(\lambda - n^{m})}{\lambda - n^{m}}
		\wh{w}(n, \lambda) d \lambda
		\right)^{2}\right]^{1/2}
		\\
		& \le 2 c_{\psi, \delta} \left[ \sum_{n \in \zz} (1 + | n |)^{2s} \left(
		\int_{\rr} \frac{\wh{w}(n, \lambda) }{1 + |\lambda - n^{m}|}
		d \lambda
		\right)^{2}\right]^{1/2}
		\\
		& \lesssim \|f\|_{X^s} \|g\|_{X^s} \|h\|_{X^s}
	\end{split}
\end{equation}
%
%
where the last two steps follow from \eqref{none-plus-ineq} and
\cref{ncor:trilinear-estimate2}, respectively. Combining \eqref{niu},
\eqref{nmain-int3-est-X-s-part}, and \eqref{nmain-int-estimate-3-Y-s-part}, we
conclude that
%
%
\begin{equation}
	\label{nmain-int3-est}
	\begin{split}
		\|\eqref{nmain-int-expression-3}\|_{Y^s} 
    \le c_{\psi, \delta} \|f\|_{X^s} \|g\|_{X^s} \|h\|_{X^s}.
	\end{split}
\end{equation}
%
%
%
\subsection{Estimate for \eqref{nmain-int-expression-4}.}
Note that
%
%
\begin{equation}
	\label{n1n}
	\begin{split}
		\eqref{nmain-int-expression-4} \simeq \sum_{k \ge 1}
		\frac{i^k}{k!}g_k(x,t)
	\end{split}
\end{equation}
%
%
where 
%
%
\begin{equation*}
	\begin{split}
		& g_k(x,t) = t^k \psi_{\delta}(t) \sum_{n \in \zz} e^{i\left( xn + tn^{m}
		\right)} h_k(n),
		\\
		& h_k(n) = \int_\rr \psi \left( \tau - n^{m } \right) \cdot \left(
		\tau - n^{m } \right)^{k -1} \wh{w}(n, \tau) \ d \tau.
	\end{split}
\end{equation*}
%
%
Hence
%
%
\begin{equation*}
	\begin{split}
		\wh{g_k^x}(n, t) = t^{k} \psi_{\delta}(t) e^{i t n^{m }} h_k(n)
	\end{split}
\end{equation*}
%
%
which gives
%
%
\begin{equation*}
	\begin{split}
		\wh{g_k}(n, \tau)
		& = h_k(n) \int_\rr e^{-it\left( \tau - n^{m } \right)}
		t^{k}\psi_{\delta}(t) \ dt
		\\
		& = h_k(n) \wh{t^{k}\psi_{\delta}(t)} \left( \tau - n^{m } \right).
	\end{split}
\end{equation*}
%
%
Applying this to \eqref{n1n}, we obtain
%
%
\begin{equation}
	\label{n2n}
	\begin{split}
		\|\eqref{nmain-int-expression-4}\|_{X^s} 
		& \simeq \left( \sum_{n \in \zz} \left (1 + |n| \right )^{2s} \int_\rr \left( 1 + | \tau -
		n^{m }
		|
		\right) | \wh{\sum_{k \ge 1} \frac{i^k}{k!}g_k(x,t)} |^2 \ d \tau
		\right)^{1/2}
		\\
		& \le \sum_{k \ge 1} \frac{1}{k!}\left( \sum_{n \in \zz} \left (1 + |n| \right )^{2s}
		\int_\rr \left( 1 + | \tau - n^{m } | \right) | \wh{g_k}(n, \tau) |^2 \
		d \tau \right)^{1/2}
		\\
		& = \sum_{k \ge 1} \frac{1}{k!} \left( \sum_{n \in \zz} \left (1 + |n| \right )^{2s}
		\int_\rr \left( 1 + | \tau - n^{m } | \right) | h_k(n) \wh{t^k
		\psi_{\delta}(t)} \left( \tau - n^{m } \right) |^2 \ d \tau \right)^{1/2}
		\\
		& = \sum_{k \ge 1} \frac{1}{k!} \left( \sum_{n \in \zz} \left (1 + |n| \right )^{2s} |
		h_k(n) |^2 \int_\rr \left( 1 + | \tau - n^{m } | \right) | \wh{t^k
		\psi_{\delta}(t)} \left( \tau - n^{m } \right) |^2 \ d \tau \right)^{1/2}.
	\end{split}
\end{equation}
%
%
Notice that for fixed $n$, the change of variable $\tau - n^{m } = \tau'$
gives
%
%
\begin{equation}
	\label{n3n}
	\begin{split}
		\int_\rr \left( 1 + | \tau - n^{m } | \right) | \wh{t^{k}
		\psi_{\delta}(t)}\left( \tau - n^{m } \right) |^2 \ d \tau
		& = \int_\rr \left( 1 + |\tau'| \right) | \wh{t^k \psi_{\delta}(t)}(\tau') |^2 \
		d \tau'
		\\
		& \le \int_\rr \left( 1 + |\tau'| \right)^2 | \wh{t^k \psi_{\delta}(t)}(\tau')
		|^2 \ d \tau'
		\\
		& \lesssim \int_\rr \left( 1 + | \tau' |^2 \right) | \wh{t^{k}
		\psi_{\delta}(t)}(\tau') |^2 \ d \tau'
		\\
		& = \|t^k \psi_{\delta}(t) \|_{H^1(\rr)}^2.
	\end{split}
\end{equation}
%
%
But
%
%
\begin{equation}
	\label{n4n}
	\begin{split}
		\|t^k \psi_{\delta}(t) \|_{H^1(\rr)}^2
		& = \left( \|t^k \psi_{\delta}(t)\|_{L^2(\rr)} + \|\p_t \left( t^k \psi_{\delta}(t)
		\right)\|_{L^2(\rr)} \right)^2
		\\
		& \lesssim \|t^{k}\psi_{\delta}(t) \|_{L^2(\rr)}^2 + \|\p_t \left (t^{k}
		\psi_{\delta}(t) \right )\|_{L^2(\rr)}^2
		\\
		& \le \|t^k \psi_{\delta}(t) \|_{L^2(\rr)}^2 + \|t^k \p_t \psi_{\delta}(t)
		\|_{L^2(\rr)}^2 + \|k t^{k -1} \psi_{\delta}(t) \|_{L^2(\rr)}^2
		\\
		& = c_{\psi, \delta} + c_{\psi, \delta}' + c_{\psi, \delta}''k^2 
		\\
    & \lesssim c_{\psi, \delta} k^2.
	\end{split}
\end{equation}
%
%
Hence, applying \eqref{n3n} and \eqref{n4n} to \eqref{n2n}, we obtain
%
%%
\begin{equation}
	\label{n5n}
	\begin{split}
		\|\eqref{nmain-int-expression-4} \|_{X^s}
		& \lesssim
    c_{\psi, \delta}\sum_{k \ge 1} \frac{k}{k!} \left( \sum_{n \in \zz} \left (1 + |n| \right )^{2s} | h_k(n) |^2 
		\right)^{1/2}
		\\
		& \lesssim \sum_{k \ge 1} \frac{1}{(k-1)!}
		\times \sup_{k \ge 1} \left( \sum_{n \in \zz} \left (1 + |n| \right )^{2s} | 
		h_k(n) |^2 \right)^{1/2}
		\\
		& = \sum_{k \ge 1} \frac{1}{(k-1)!} \times \sup_{k \ge 1} 
		\left( \sum_{n \in \zz} \left (1 + |n| \right )^{2s} |\int_\rr 
		\psi\left( \tau - n^{m } \right) \cdot \left( \tau - n^{m } 
    \right)^{k -1} \wh{w}(n, \tau) \ d \tau|^{2} \right)^{1/2}.
    	\end{split}
\end{equation}
%
%%
Recall that $0 \le \psi \le 1, \text{ supp} \, \psi \subset [-1,1 ]$. 
This implies $| \psi\left( \tau - n^{m } \right) \cdot \left( \tau - n^{m } \right)^{k 
-1} | \le \chi_{| \tau - n^{m } | \le 1}$ for all $k \ge 1$. Hence, we bound the
right hand side of \eqref{n5n} by
%
%%
\begin{equation*}
	\begin{split}
		& \sum_{k \ge 1} \frac{1}{(k-1)!} \times \left( \sum_{n \in \zz} (1 + | n |)^{2s}| 
		\int_{| \tau - n^{m}  |\le 1}  \wh{w}(n, \tau) \ d \tau |^2 
		\right)^{1/2}
    \\
    & = e \left( \sum_{n \in \zz} (1 + | n |)^{2s}| 
		\int_{| \tau - n^{m}  |\le 1}  \wh{w}(n, \tau) \ d \tau |^2 
		\right)^{1/2}
    \\
    & \le e \left[ \sum_{n \in \zz} (1 + | n |)^{2s}\left (  
		\int_{| \tau - n^{m}  |\le 1} | \wh{w}(n, \tau) | \ d \tau \right ) ^2 
		\right]^{1/2}
	\end{split}
\end{equation*}
%
%%
which by the inequality
%
%%
\begin{equation*}
	\begin{split}
		\frac{1 + | \tau - n^{m } |}{1 + | \tau  - n^{m } |} \le 
		\frac{2}{1 + | \tau - n^{m } |}, \qquad | \tau - n^{m }  | \le 1
	\end{split}
\end{equation*}
%
%%
is bounded by 
%
%%
\begin{equation}
\label{nmain-int4-est-X-s-part}
	\begin{split}
		& 2e \left[ \sum_{n \in \zz} (1 + | n |)^{2s}\left ( \int_\rr
		\frac{|\wh{w}(n, \tau)|}{1 + | \tau - n^{m } |} \ d \tau \right ) ^2 
		\right]^{1/2} \\
		& \lesssim \|u\|_{X^s}^3
	\end{split}
\end{equation}
%
%%
where the last step follows from \cref{ncor:trilinear-estimate2}. Similarly,
we have
%
%
\begin{equation}
\label{nmain-int4-est-Y-s-part}
	\begin{split}
    \|\wh{\eqref{nmain-int-expression-4}}\|_{ \ell^2_n L^1_\tau }
		& \simeq \left[ \sum_{n \in
		\zz}(1 + | n |)^{2s} \left( \int_{\rr} | \sum_{k \ge 1}
		\wh{\frac{i^{k}}{k!}g_{k}(x,t)(n, \tau)} |d \tau \right)^{2} \right]^{1/2}
		\\
		& \le \sum_{k \ge 1} \frac{1}{k!} \left[ \sum_{n \in \zz} (1 + | n
    |)^{2s} \left( \int_{\rr} | \wh{g_{k}}(n, \tau) | d \tau \right)^{2}
		\right]^{1/2}
		\\
		& = \sum_{k \ge 1} \frac{1}{k!} \left[ \sum_{n \in \zz} (1 + | n
		|)^{2s} | h_{k}(n) |^2 \left( \int_{\rr} | \wh{t^{k} \psi_{\delta}(t)}(\tau -
		n^{m}) |d \tau \right)^{2} \right]^{1/2}
		\\
		& = c_{\psi, \delta} \sum_{k \ge 1} \frac{1}{k!} \left[ \sum_{n \in \zz} (1 + | n
		|)^{2s} | h_{k}(n) |^2 \right]^{1/2}
		\\
		& \lesssim \|u\|_{X^s}^{3}
	\end{split}
\end{equation}
%
%
where the last step follows from the computations starting from \eqref{n5n}
through \eqref{nmain-int4-est-X-s-part}.
Combining \eqref{nmain-int4-est-X-s-part} and \eqref{nmain-int4-est-Y-s-part}, we
have
%
%
\begin{equation}
\label{nmain-int4-est}
	\begin{split}
    \|\eqref{nmain-int-expression-4}\|_{Y^s} \le c_{\psi, \delta} \|u\|_{X^s}^{3}.
	\end{split}
\end{equation}
%
%
Collecting estimates \eqref{nmain-int1-est}, \eqref{nmain-int2-est}, 
\eqref{nmain-int3-est}, and \eqref{nmain-int4-est}, and recalling 
\eqref{nmain-int-expression-1}-\eqref{nmain-int-expression-4}, we see that
$$\|Tu\|_{Y^s} \le c_{\psi,\delta} \left( \|\vp \|_{H^s(\ci)} + \|u\|_{X^s}^3 \right )$$ 
which by the inequality $\|u\|_{X^s} \le \|u\|_{Y^s}$ yields the following.
%%
%%%%%%%%%%%%%%%%%%%%%%%%%%%%%%%%%%%%%%%%%%%%%%%%%%%%%
%
%% Contraction Proposition
%				 
%%%%%%%%%%%%%%%%%%%%%%%%%%%%%%%%%%%%%%%%%%%%%%%%%%%%%%
%%
%%
%
\begin{proposition}
\label{nprop:contraction}
	Let $s \ge0$. Then
%
%%
\begin{equation*}
	\begin{split}
		\|Tu\|_{Y^s} \le c_{\psi,\delta} \left( \|\vp \|_{H^s(\ci)} + \|u\|_{Y^s}^3 
		\right).
	\end{split}
\end{equation*}
%
%%
\end{proposition}
We will now use \cref{nprop:contraction} to prove local well-posedness for the 
mNLS ivp. Let $c = c_{\psi, \delta}^{1/2}$. For given $\vp$, we may choose $\delta$ such
that 
%
%%
\begin{equation*}
	\begin{split}
		\|\vp\|_{H^s(\ci)} \le \frac{15}{64c^3}.
	\end{split}
\end{equation*}
%
%%
Then if $$\|u\|_{Y^s} \le \frac{1}{4c}$$ we have
%
%%
\begin{equation*}
	\begin{split}
		\|T u \|_{Y^s} 
		& \le c^2 \left[ \frac{15}{64c^3} + \left( 
		\frac{1}{4c} \right)^3 \right]
		=  \frac{1}{4c}.
	\end{split}
\end{equation*}
%
%%
Hence, $T=T_{\vp, \psi, T}$ maps the ball $B_{Y^{s}}\left( \frac{1}{4c} \right)$ into 
itself. Next, note that
%
%%
\begin{equation*}
	\begin{split}
		Tu - Tv = \eqref{nmain-int-expression-2} + \eqref{nmain-int-expression-3} 
		+ \eqref{nmain-int-expression-4}
	\end{split}
\end{equation*}
%
%%
where now $w = u | u |^2 - v | v |^{2}$. Rewriting
%
%%
\begin{equation*}
	\begin{split}
		u | u |^{2} - v | v |^{2}
		& = | u |^2 \left( u -v \right) + v\left( | u 
		|^2 - | v |^2
		\right)
		\\
		& = u \bar u \left( u -v \right) + v u \bar u - v v \bar v
		\\
		& = u \bar u \left( u - v \right) + v \bar u\left( u - v \right) + v 
		\bar u v - v v \bar v
		\\
		& = u \bar u \left( u -v \right) + v \bar u\left( u - v \right) + v v 
		\left( \overline{u -v} \right),
	\end{split}
\end{equation*}
%
%%
the triangle inequality and linearity of the Fourier transform give
%
%%
\begin{equation*}
	\begin{split}
		| \wh{w}(n, \tau) | = | \mathcal{F}(u | u |^2 - v| v |^2) |
		& \le | \wh{u \overline{u} \left (u -v \right )} | +
		| \wh{v \overline{u} (u -v)} | + |\wh{v v 
		(\overline{u-v})}|
		\\
		& = | \wh{w_1} | + | \wh{w_2} | + | \wh{w_3} |
	\end{split}
\end{equation*}
%
%%
where
%
%%
\begin{equation*}
	\begin{split}
		w_1 = u \bar u \left( u -v \right), \qquad w_2 = v \bar u \left( u -v 
		\right), \qquad w_3 = v v \left( \overline{u -v} \right).
	\end{split}
\end{equation*}
%
%%
Hence, $Tu - Tv = \sum_{\ell=1}^{3} 
T_\ell(u, v)$, where
\begin{align}
	\label{nmain-int-exp-mod1}
	& \frac{i}{4 \pi^2} \psi_{\delta}(t) \sum_{n\in \zz} \int_\rr e^{ixn}  
		e^{it \tau} \frac{ 1 - \psi(\tau - n^{m}) 
		}{\tau - n^{m}} \wh{w_\ell}(n, \tau) \ d \tau
		\\
		\label{nmain-int-exp-mod2}
		- & \frac{i}{4 \pi^2} \psi_{\delta}(t) \sum_{n\in \zz} \int_\rr e^{i(xn + 
		tn^{m})}
		 \frac{1- \psi(\tau - n^{m})}{\tau - n^{m}} \wh{w_\ell}(n, \tau) \ d \tau
		\\
		\label{nmain-int-exp-mod3}
		+ & \frac{i}{4 \pi^2} \psi_{\delta}(t) \sum_{k \ge 1} \frac{i^k t^k}{k!}
		\sum_{n \in \zz} \int_\rr e^{i(xn + tn^{m} )}
		\psi(\tau - n^{m}) (\tau - n^{m})^{k-1} \wh{w_\ell}(n, \tau)  
		\\
		\doteq & T_\ell(u). \notag
\end{align}
Repeating the arguments used to estimate 
\eqref{nmain-int-expression-2}-\eqref{nmain-int-expression-4}, we obtain
%
%%
\begin{equation*}
	\begin{split}
    & \|T_1\|_{Y^s} \le c_{\psi,\delta} \|u -v \|_{Y^s} \|u\|^2_{Y^s}
		\\
    & \|T_2\|_{Y^s} \le c_{\psi,\delta} \|u -v \|_{Y^s} \|u\|_{Y^s} \|v\|_{Y^s}
		\\
    & \|T_3\|_{Y^s} \le c_{\psi,\delta} \|u -v \|_{Y^s} \|v\|_{Y^s}^2.
	\end{split}
\end{equation*}
%
%%
Therefore,
%
%%
\begin{equation}
	\label{n20a}
	\begin{split}
    \|Tu - Tv \|_{Y^s} = & \| \sum_{\ell =1}^{3} T_\ell(u, v) \|_{Y^s}
		\\
    & \le c_{\psi,\delta} \|u -v \|_{Y^s} \left( \|u\|_{Y^s}^2 + 
		\|u\|_{Y^s} \|v\|_{Y^s} + \|v\|_{Y^s}^2 \right)
		\\
		& \le c_{\psi,\delta} \|u -v\|_{Y^s} \left( \|u\|_{Y^s} + \|v\|_{Y^s} \right)^2
		\\
		& = c^2 \|u -v\|_{Y^s} \left( \|u\|_{Y^s} + \|v\|_{Y^s} \right)^2.
	\end{split}
\end{equation}
%
%%
If $u, v \in B_{Y^{s}}(\frac{1}{4c})$, it follows that
%
%%
\begin{equation}
	\label{n21a}
	\begin{split}
		\|Tu - Tv \|_{Y^s}
		& \le c^2 \|u -v \|_{Y^s} \left( \frac{1}{4c} + 
		\frac{1}{4c} \right)^2
		\\
		& = \frac{1}{4} \|u -v \|_{Y^s}. 
	\end{split}
\end{equation}
%
%%
We conclude that $T = T_{\vp, \psi, \delta}$ is a contraction on the ball
$B_{Y^{s}}(\frac{1}{4c})$. A Picard iteration, coupled with
\cref{nlem:cutoff-loc-soln} then yields a unique solution $u \in C(\left[ -\delta,
\delta \right], H^{s}(\ci))$ to the integral equation $\psi_{\delta} u = Tu$.\\
\\
We now establish Lipschitz continuity of the flow map from
$B_{H^{s}(\ci)}(\frac{15}{64c^{3}})$ to $C(\left[ -\delta, \delta \right], H^{s}(\ci))$.
Assume $\vp_1, \vp_2
\subset B_{H^s(\ci)}(\frac{15}{64c^{3}})$.
Then by the preceding arguments there exist $u_1, u_2 \in Y^s$ such that 
$u_1 = T_{\vp_1, \psi, \delta}$, $u_2 = T_{\vp_2, \psi, \delta}$, and so
%
%
\begin{equation*}
	\begin{split}
		T_{\vp_1, \psi, \delta}(u) -
    T_{\vp_2, \psi, \delta}(v) = \frac{1}{2\pi} \psi_{\delta}(t) \sum_{n \in
		\zz}e^{i\left( xn + tn^{m} \right)} \wh{\vp_1 - \vp_2}(n) + \sum_{\ell=1
    }^{3} T_{\ell}(u).
	\end{split}
\end{equation*}
%
%
Using an argument similar to \eqref{nfourier-trans-calc}-\eqref{nmain-int1-est},
we obtain
%
%
\begin{equation*}
	\begin{split}
		\| \frac{1}{2\pi} \psi_{\delta}(t) \sum_{n \in
		\zz}e^{i\left( xn + tn^{m} \right)} \wh{\vp_1 - \vp_2}(n)\|_{Y^s}
		\le c_{\psi,\delta} \|\vp_{1} - \vp_{2}\|_{Y^s}.
	\end{split}
\end{equation*}
%
%
Hence, \eqref{n20a}-\eqref{n21a} gives
%
%
\begin{equation*}
	\begin{split}
    \sum_{\ell=1}^{3} T_{\ell}(u,v) \le \frac{1}{4}\|u-v\|_{Y^s}.
	\end{split}
\end{equation*}
%
%
Hence,
%
%
\begin{equation*}
	\begin{split}
    \|\psi_{\delta} u - \psi_{\delta} v \|_{Y^s} = \|T_{\vp_1}(u) - T_{\vp_2}(v) \|_{Y^s} \le c_{\psi,\delta}
		\|\vp_{1} - \vp_{2} \|_{H^{s}\left( \ci \right)}\| +
		\frac{1}{4} \|u -v \|_{Y^s}
	\end{split}
\end{equation*}
%
%
which implies
%
%
\begin{equation*}
	\begin{split}
    \frac{3}{4} \|\psi_{\delta} u- \psi_{\delta} v\|_{Y^s} \le c_{\psi,\delta} \|\vp_1 - \vp_2 \|_{H^s(\ci)}
	\end{split}
\end{equation*}
%
%
or
%
%
\begin{equation*}
	\begin{split}
    \|\psi_{\delta} u - \psi_{\delta} v \|_{Y^s} \le \frac{4}{3} c_{\psi,\delta} \|\vp_1 - \vp_2 \|_{H^{s}(\ci)}.
	\end{split}
\end{equation*}
%
%
Applying \cref{nlem:cutoff-loc-soln}, we then obtain
%
%
	 %
	 %
	 \begin{equation*}
		 \begin{split}
       \sup_{t \in \left[ -\delta, \delta \right]}\|u(\cdot, t) -v(\cdot, t) \|_{H^s(\ci)} \le \frac{4}{3} c_{\psi,\delta} \|\vp_1 -
			\vp_2 \|_{H^{s}(\ci)}, \qquad t \in [-\delta, \delta].
		 \end{split}
	 \end{equation*}
	 %
	 %
   Hence, the flow map of the mNLS ivp is Lipschitz continuous from
   $B_{H^{s}(\ci)}( \frac{15}{64c^{3}})$ to  $C(\left[ -\delta, \delta \right],
   H^{s}(\ci))$. Since this implies uniform continuity of the flow map from
   $B_{H^{s}(\ci)}( \frac{15}{64c^{3}})$ to  $C(\left[ -\delta, \delta \right],
   H^{s}(\ci))$, the proof of \cref{nthm:main} is complete. \qquad
   \qedsymbol
%
%
%
%
\section{Proof of Trilinear Estimate}
%
%
%
%
%
%
Note first that $|\wh{w_{fgh}}(n, \tau) |  = | \wh{f} * ( \wh{g} 
* \wh{\bar h})(n, \tau)|$ and $| \wh{\bar{h}}(n, \tau) | = |\overline{ \wh{\overline{h}} 
}(n, \tau)| = | \wh{h}(-n, -\tau) |$. It follows that
%
%
\begin{equation}
	\label{nnon-lin-rep}
	\begin{split}
		& | \wh{w_{fgh}}(n, \tau)|
    \\
    & = |  \sum_{n_{1} } \int_{\tau_{1}} \sum_{n_{2}}
    \int_{\tau_{2}} \wh{f}\left( n
    -n_1,  \tau - \tau_1 \right) \wh{g}\left( n_{1} - n_2, \tau_{1} - \tau_2  
\right) \wh{\bar h}\left( n_2, \tau_2 \right) d \tau_2 d \tau_1 |
\\
& \le \sum_{n_{1} } \int_{\tau_{1}} \sum_{n_{2}}
\int_{\tau_{2}}  | \wh{f}\left( n - n_1, \tau - \tau_1 
\right) | \times  | \wh{g}\left( n_1 - n_2, \tau_1 - \tau_2 
\right) | \times | \wh{\bar h}\left( n_{2}, \tau_{2} \right) | d \tau_2 d \tau_1
\\
& = \sum_{n_{1} } \int_{\tau_{1}} \sum_{n_{2}}
\int_{\tau_{2}}  | \wh{f}\left( n - n_1, \tau - \tau_1 
     \right) | \times | \wh{g}\left( n_{1} - n_2, \tau_{1} - \tau_2 
\right) | \times | \wh{h}\left( -n_{2}, - \tau_2 \right) | d \tau_2 d \tau_1
\\
& = \sum_{n_{1} } \int_{\tau_{1}} \sum_{n_{2}}
\int_{\tau_{2}} \frac{c_f\left( n - n_1, \tau - \tau_1 
\right)}{\left (1 + |n - n_{1}| \right )^s \left( 1 + | \tau - \tau_1 - (n - n_{1})^{m} | \right)^{b}}
\\
& \times \frac{c_{g}\left( n_1 - n_2, \tau_1 - \tau_2 \right)}{\left (1 + |n_1 -
n_2| \right ) 
^s\left( 1 + | \tau_1 - \tau_2 -  (n_1 - n_2)^{m }| 
\right)^{b}}
 \times \frac{c_{h}\left( -n_{2}, -\tau_2 \right)}{\left (1 + |n_{2}| \right ) ^s\left( 1 + | 
\tau_2 + n_{2}^{m } | \right)^{b}} \ d \tau_2 d \tau_1
\end{split}
\end{equation}
%
%
where
%
%
\begin{equation*}
	\begin{split}
		c_\sigma(n, \tau) = \left (1 + |n| \right ) ^s \left( 1 + | \tau - n^{m } |  
		\right)^{b} | \wh{\sigma}\left( n, \tau \right) | .
	\end{split}
\end{equation*}
%
%
Hence
%
%
\begin{equation}
	\label{nconvo-est-starting-pnt}
	\begin{split}
		 & \left (1 + |n| \right )^s \left( 1 + | \tau - n^{m } | \right)^{-b} | \wh{w_{fgh}}\left( 
		n, \tau \right) |
		\\
		& \le \left( 1 + | \tau - n^{m } | \right)^{-b}
    \sum_{n_{1} } \int_{\tau_{1}} \sum_{n_{2}}
    \int_{\tau_{2}}     \\
    & \frac{\left (1 + |n| \right )^s}{\left (1 +
		|n - n_{1}| \right )^s \left (1 + | n_1 - n_2| \right )^s \left (1 + |n_{2}| \right )^s} 
		\times \frac{c_f(n - n_{1}, \tau - \tau_1)}{\left( 1 + | \tau - \tau_1 - (n - n_{1})^{m } | 
		\right)^{b}}
		\\
		& \times
		\frac{c_g(n_1 - n_2, \tau_1 - \tau_2)}{\left( 1 + | \tau_1 - \tau_2 - (n_1 - n_2)^{m } | 
		\right)^{b}} \times
		\frac{c_h(-n_{2}, -\tau_2)}{\left( 1 + | \tau_2 + n_{2}^{m } | 
		\right)^{b}}\ d \tau_2 d \tau_1 .
	\end{split}
\end{equation}
%
%
For $s \ge 0$, observe that
%
%
\begin{equation}
	\label{nderiv-bound-easy-s}
	\begin{split}
		\frac{\left (1 + |n| \right ) ^s}{\left (1 + |n - n_{1}| \right ) ^s \left (1 + |n_1 - n_2| \right ) ^s \left (1 + |n_2| \right ) ^s} 
		\le 3^{s}
	\end{split}
\end{equation}
%
%
by the following lemma, whose proof is provided in the appendix.
%
%
\begin{lemma}
\label{nlem:splitting}
	For $v \ge 0$ we have
%
%
\begin{equation}
	\label{nsplitting}
	\begin{split}
		\left ( 1 + |a +b + c| \right)^v \le 3^v \left(1 + | a | \right)^v \left(
		1 + | b | \right)^v \left( 1 + | c | \right)^v.
	\end{split}
\end{equation}
%
%
\end{lemma}
%
%
Hence, from \eqref{nconvo-est-starting-pnt} and \eqref{nderiv-bound-easy-s}, we 
obtain
%
\begin{equation*}
	\begin{split}
		& \left (1 + |n| \right )^s \left( 1 +  | \tau - n^{m }  | \right)^{-b} | 
		\wh{w_{fgh}}\left( n, \tau \right) | 
		\\
    & \lesssim  \frac{1}{\left( 1 +
		| \tau - n^{m}| 
		\right)^{b}}  
		\times
    \sum_{n_{1} } \int_{\tau_{1}} \sum_{n_{2}}
    \int_{\tau_{2}} \frac{c_f\left( n - n_{1}, \tau - \tau_1 
		\right)}{\left (1 + |n - n_{1}| \right )^s \left( 1 + | \tau - \tau_1 - (n - n_{1})^{m} |
		\right)^{b}}
		\\
		& \times \frac{c_{g}\left( n_1 - n_2, \tau_1 - \tau_2 \right)}{\left (1 + |n_1 - n_2| \right ) 
		^s\left( 1 + | \tau_1 - \tau_2 -  (n_1 - n_2)^{m }| 
		\right)^{b}}
    \times \frac{c_{h}\left( -n_2, -\tau_2 \right)}{\left (1 + |n_2| \right )
    ^s\left( 1 + | \tau_2 + n_2^{m } | \right)^{b}} \ d \tau_2 d \tau_1 
    \\
		& = \left( 1 + | \tau - n^{m } | \right)^{-b}
		\wh{C_f C_{g} C^+_{h}} \left( n, \tau \right)
	\end{split}
\end{equation*}
%
%
where
%
%
\begin{equation*}
	\begin{split}
		C_\sigma(x, t) = \left[ \frac{c_\sigma\left( n, \tau \right)}{\left( 
		1 + | \tau - n^{m } | \right)^{b}} \right]^\vee,
		\ \ C^+_\sigma(x, t) = \left[ \frac{c_\sigma\left( -n, -\tau \right)}{\left( 
		1 + | \tau + n^{m } | \right)^{b}} \right]^\vee.
	\end{split}
\end{equation*}
%
%
Therefore
%
%
\begin{equation}
	\label{ngen-holder-pre-estimate}
	\begin{split}
		& \| \left( 1 + |n | \right)^s
		\left( 1 + | \tau - n^{m } | \right)^{-b} \wh{w_{fgh}}(n, 
		\tau)		
		\|_{L^2(\zz \times \rr)}
		\\
		& \lesssim \| \left( 1 + | \tau - n^{m } | \right)^{-b}
		\wh{C_f C_{g} C^+_{h}} \|_{L^2(\zz \times \rr)}.
	\end{split}
\end{equation}
%
We now require the following multiplier estimate, whose proof can be found in 
\cite{Himonas:2001db}.
%
%
%%%%%%%%%%%%%%%%%%%%%%%%%%%%%%%%%%%%%%%%%%%%%%%%%%%%%
%
%
%			Four Mult Est	
%
%
%%%%%%%%%%%%%%%%%%%%%%%%%%%%%%%%%%%%%%%%%%%%%%%%%%%%%
%
%
%
%
%
%
%
%
\begin{lemma}
	\label{nlem:four-mult-est-L4}
	Let $(x, t) \in \ci \times \rr $ and $(n, \tau) \in \zz \times \rr$ be 
	the dual variables. Let $v$ be a positive even integer. Then there is a 
	constant $c_v > 0$ such that
%
%
\begin{equation}
	\label{nfour-mult-est-L4*}
	\begin{split}
		\| \left( 1 + | \tau - n^v | 
		\right)^{-\frac{v+1}{4v}}
		\wh{f}\|_{L^2(\zz \times \rr)} \le c_v \|f \|_{L^{4/3}( \ci \times \rr)}.
	\end{split}
\end{equation}
%
%
\end{lemma}
%
%
Applying \cref{ncor:four-mult-est-L4} and generalized H\"{o}lder to the 
right-hand-side of \eqref{ngen-holder-pre-estimate} gives
%
%
\begin{equation}
	\label{ngen-holder-piece-1}
	\begin{split}
		\|\left( 1 + | \tau - n^{m } | \right)^{-b} \wh{C_f C_{ 
		g } C^+_{h}}\|_{L^2(\zz \times \rr)}
		& \lesssim  \|C_f C_{g} C^+_{h} \|_{L^{4/3}(\ci \times \rr)}
		\\
		& \le \|C_f \|_{L^4(\ci \times \rr)} \|C_{g}\|_{L^4(\ci \times \rr)} 
		\|C^+_{h}\|_{L^4(\ci \times \rr)}.
	\end{split}
\end{equation}
%
%
Note that a change of variable gives
%
%
\begin{equation*}
	\begin{split}
		C_\sigma^+(x, t)
		& = \sum_{n \in \zz} \int_\rr e^{i(nx +  \tau t)} \frac{c_\sigma\left( -n, -\tau \right)}{\left( 
		1 + | \tau + n^{m } | \right)^{b}} \ d \tau
		\\
		& = - \sum_{n \in \zz} \int_\rr e^{-i(nx +   \tau t )}
		\frac{c_\sigma\left( n, \tau \right)}{\left( 
		1 + | \tau - n^{m } | \right)^{b}} \ d \tau
	\end{split}
\end{equation*}
%
%
and so
%
%
\begin{equation*}
	\begin{split}
		C_\sigma^+(-x, -t) = -C_\sigma(x, t).
	\end{split}
\end{equation*}
%
%
We will now the need the following dual estimate of
\cref{nlem:four-mult-est-L4}.
%
\begin{corollary}
	\label{ncor:four-mult-est-L4}
	Let $(x, t) \in \ci \times \rr $ and $(n, \tau) \in \zz \times \rr$ be 
	the dual variables. Let $v$ be a positive even integer. Then there is a 
	constant $c_v > 0$ such that
%
%
\begin{equation}
	\label{nfour-mult-est-L4}
	\begin{split}
		\|f\|_{L^4(\ci \times \rr)} \le c_v \|\left( 1 + | \tau - n^v | 
		\right)^\frac{v+1}{4v} \wh{f} \|_{L^2( \zz \times \rr)}
	\end{split}
\end{equation}
for every test function $f(x, t)$. 
%
%
%
%
\end{corollary}
%
%
Recalling that $L^4(\ci \times \rr)$ is invariant under the transformation $(x, 
t) \mapsto (-x,-t)$ and applying 
\cref{ncor:four-mult-est-L4}, we obtain
%
%
\begin{equation}
	\label{nC-sig-estimate}
	\begin{split}
		\| C^+_\sigma \|_{L^4(\ci \times \rr)} = \|C_\sigma \|_{L^4(\ci \times \rr)} 
		& \lesssim \|\left( 1 + | \tau - n^{m } | 
		\right)^{(m +1)/4m} \left( 1 + | \tau - n^{m } | 
		\right)^{-b} c_\sigma \|_{L^2(\zz \times \rr)}
		\\
		& = \|\left( 1 + | \tau - n^{m } | 
		\right)^{[m(1 - 4b) + 1]/4m } c_\sigma \|_{L^2(\zz \times \rr)}
		\\
		& \le \|c_\sigma \|_{L^2(\zz \times \rr)}  \qquad (\text{since  } [m(1 - 4b) + 
		1]/4m \le 0)
		\\
		& = \|\sigma\|_{X^s}.
	\end{split}
\end{equation}
%
%
We conclude from \eqref{ngen-holder-pre-estimate}, \eqref{ngen-holder-piece-1}, 
and \eqref{nC-sig-estimate} that
%
%
%
%
\begin{equation*}
	\begin{split}
		\| \left( 1 + |n | \right)^s \left( 1 + | \tau - n^{m} | \right)^{-b} \wh{w_{fgh}} 
		(n, \tau) \|_{L^2(\zz \times \rr)} \lesssim 
		\|f\|_{X^s}\|g\|_{X^s}\|h\|_{X^s}.
	\end{split}
\end{equation*}
%
%
%
%
%
%
%
\section{Proof of Corollary to Trilinear Estimate}
By duality, it suffices to show that 
%
%%
\begin{equation*}
	\begin{split}
		| \sum_{n \in \zz} \left (1 + |n| \right )^{s}
		a_n \int_{\rr} \frac{|\wh{w_{fgh}}(n, \tau)|}{1 
		+ | \tau - n^{m } |} \ d \tau | \lesssim \|f\|_{X^s} \|g\|_{X^s} \|h\|_{X^s}
		\|a_n \|_{\ell^2}
	\end{split}
\end{equation*}
%
%%
for $\{a_n\} \in \ell^2$. By the triangle inequality 
and Cauchy-Schwartz,
%
%%
\begin{equation}
	\label{n1m}
	\begin{split}
		& | \sum_{n \in \zz} \left (1 + |n| \right )^{s} a_n
		\int_{\rr}\frac{| \wh{w_{fgh}}(n, \tau) |}{1 + | \tau - n^{m } |} \ d \tau |
		\\
		& \le \sum_{n \in \zz} \int_{\rr} \frac{| a_n |}{\left( 1 + 
		| \tau - n^{m } |
		\right)^{1/2 + \eta}} \cdot \frac{\left( 1 + | n| \right)^s  |
		\wh{w_{fgh}}(n, \tau) |}{\left( 
		1 + | \tau - n^{m } | \right)^{1/2 - \eta}} \ d \tau
		\\
		& \le \left( \sum_{n \in \zz} | a_{n} |^2\int_{\rr} \frac{1}{\left( 1 + | \tau - n^{m } | \right)^{1 + 2 \eta}} \ d \tau  
		\right)^{1/2} 
		\left ( \sum_{n \in \zz}\int_{\rr} \frac{\left (1 + |n| \right )^{2s} | \wh{w_{fgh}}(n, \tau) 
		|^2}{\left( 1 + | \tau - n^{m } | \right)^{1 - 2 \eta}}\ d \tau 
		\right)^{1/2}
	\end{split}
\end{equation}
%
%%
Restrict $\eta \in (0, 1/8)$. Applying the change of variable $\tau - n^{m }
= \tau'$ we obtain  %
%%

\begin{equation*}
	\begin{split}
		& \left( \sum_{n \in \zz} | a_{n} |^2\int_{\rr} \frac{1}{\left( 1 + | \tau -
		n^{m } | \right)^{1 + 2 \eta}} \ d \tau  
		\right)^{1/2} 
		\\
		& = \left ( \sum_{n \in \zz}
		| a_n |^2 
		\int_{\rr} \frac{1}{\left( 1 + | \tau' | \right)^{1 + 2 \eta}} \ d 
		\tau \right)^{1/2}
		\\
		& \simeq \|a_n\|_{\ell^2}
		\end{split}
\end{equation*}
while \cref{nprop:trilinear-est} gives the bound
\begin{equation*}
	\begin{split}
		\left ( \sum_{n \in \zz}\int_{\rr} \frac{\left (1 + |n| \right )^{2s} | \wh{w_{fgh}}(n, \tau) 
		|^2}{\left( 1 + | \tau - n^{m } | \right)^{1 - 2 \eta}}\ d \tau 
		\right)^{1/2} \lesssim \|f\|_{X^s} \|g\|_{X^s} \|h\|_{X^s}
	\end{split}
\end{equation*}
%
%%
completing the proof.
\qquad \qedsymbol
%
%
%
\section{Proof of Ill-Posedness}
We adapt an argument from~\cite{Burq:2002xd}. For $s<0$,
$m \in \{4, 8, 12, \dots\}$, set
%
%
%
%
\begin{equation}
	\label{nill-soln}
	\begin{split}
		u_{n}(x,t)=\frac{1}{2 }n^{-s}e^{it\left( n^{m}+\frac{1}{4}n^{-2s}
		\right)}e^{inx}.
	\end{split}
\end{equation}
%
%
Then
%
%
\begin{equation*}
	\begin{split}
		& i \p_t u_{n}
		= -\frac{1}{2 }n^{-s}\left( n^{m}+\frac{1}{4}n^{-2s} \right)e^{it\left(
		n^{m}+ \frac{1}{4}n^{-2s} \right)}e^{inx},
		\\
		& \p_x^{m}u_{n}  = \frac{1}{2 }n^{-s+m}e^{it\left(
		n^{m}+\frac{1}{4}n^{-2s} \right)}e^{inx},
		\\
		& | u_{n} |^{2}u_{n}  = \frac{1}{8 }n^{-3s}e^{it\left(
		n^{m}+\frac{1}{4}n^{-2s} \right)}e^{inx}.
	\end{split}
\end{equation*}
%
%
Hence,
%
%
\begin{equation*}
	\begin{split}
		i \p_t u_{n} + \p_x^{m}u_{n} + | u_{n} |^{2} u_{n}
		=0.
	\end{split}
\end{equation*}
%
%
Therefore, $u_{n}(x,t)$ solves the initial value problem
%
%
\begin{gather*}
	\begin{split}
		i \p_t u + \p_x^m u + | u |^{2}u = 0,
		\\
    u(x,0) = \frac{1}{2 }n^{-s}e^{inx}.
	\end{split}
\end{gather*}
%
%
Next, we show that $u_{n}(\cdot, t) \in H^{s}(\ci), \text{ } s < 0$ for all $t
\in \rr$.  First, we compute
%
%
\begin{equation*}
	\begin{split}
		\|e^{inx}\|_{H^{s}(\ci)}
		& =  \left[ \sum_{\xi \in \zz} \left( 1+ | \xi |
		\right)^{2s} | \wh{e^{inx}}(\xi) |^{2} \right]^{1/2}
		\\
		& =  \left[ \sum_{\xi \in \zz} \left( 1 + | \xi | \right)^{2s} |
		\int_{\ci}e^{ix(n- \xi)}dx |^{2}\right]^{1/2}.
	\end{split}
\end{equation*}
%
%
Noting that
%
\begin{equation*}
	\begin{split}
		\int_{\ci}e^{ix(n - \xi)}dx =
		\begin{cases}0, \qquad & \xi \neq n 
			\\
			2 \pi, \qquad & \xi = n 
		\end{cases}
	\end{split}
\end{equation*}
%
%
we obtain
%
%
\begin{equation}
	\label{noscill-bound}
	\begin{split}
		\|e^{inx}\|_{H^{s}(\ci)} & = 2 \pi (1 + | n |)^{s}
	\end{split}
\end{equation}
%
%
and so
%
%
\begin{equation*}
	\begin{split}
		\|u_{n}(\cdot, t)\|_{H^s{(\ci)}}
    & = \frac{1}{2}|n|^{-s}\|e^{inx}\|_{H^{s}(\ci)}
    \\
    & = \pi |n|^{-s} (1 + | n |)^{s} 
    \\
    & \le \pi, \qquad s < 0.
	\end{split}
\end{equation*}
%
%
Next, let
%
%
\begin{equation*}
	\begin{split}
		u_{k_{n}}(x,t) = k_{n}n^{-s}e^{it\left( n^{m} + k_{n}^2 n^{-2s}
		\right)}e^{inx}.
	\end{split}
\end{equation*}
%
%
Following our preceding computations, one can check that $u_{n, k_{n}}$ is a solution to the ivp
%
%
\begin{equation}
	\label{nfamily-ivp}
	\begin{split}
		i\p_t u + \p_x^{m} + | u |^{2}u = 0,
		\\
		u(x,0) = k_{n}n^{-s}e^{inx}
	\end{split}
\end{equation}
%
%
and satisfies 
%
%
\begin{equation*}
	\begin{split}
		\|u_{n, k_{n}}(\cdot, t)\|_{H^{s}(\ci)} \le 2 \pi |k_{n}|.
	\end{split}
\end{equation*}
%
%
Furthermore, choosing $\{k_{n}\}_{n} \subset (0, 1/2)$ to be a family of
rational numbers converging to $k =1/2$ and recalling \eqref{noscill-bound}, we
have
%
%
\begin{equation*}
	\begin{split}
		\|u(x,0) - u_{n, k_{n}}(x, 0) \|_{H^s(\ci)} 
		& =
		\|\frac{1}{2}n^{-s}e^{inx} - k_{n}n^{-s}e^{inx} \|_{H^s(\ci)}
		\\
		& = | n |^{-s} \|e^{inx}\|_{H^s(\ci)}|\frac{1}{2} - k_{n}|
		\\
    & = 2 \pi |n|^{-s}(1+| n |)^{s} |\frac{1}{2} - k_{n}|
    \\
    & \to 0, \qquad s < 0
	\end{split}
\end{equation*}
%
%
and
%
%
\begin{equation*}
	\begin{split}
		& \|u_{n}(\cdot, t) - u_{n, k_{n}}(\cdot, t) \|_{H^{s}(\ci)}
		\\
		& = \|\frac{1}{2}n^{-s}e^{it\left( n^{m} + \frac{1}{4}n^{-2s}
		\right)}e^{inx} - k_{n}n^{-s}e^{it\left( n^{m} + k_{n}^{2}n^{-2s}
		\right)}e^{inx} \|_{H^{s}(\ci)}
		\\
		& = | n |^{-s} \|e^{it\left( n^{m} + \frac{1}{4}n^{-2s}
		\right)}e^{inx}\left( \frac{1}{2} - k_{n}e^{it\left(
		k_{n}^{2}n^{-2s}-\frac{1}{4}n^{-2s} \right)} \right)\|_{H^{s}(\ci)}
		\\
		& = | n|^{-s} \|e^{inx} \|_{H^{s}(\ci)}
    | \frac{1}{2} - k_{n}e^{it\left(
    k_{n}^{2}n^{-2s} - \frac{1}{4}n^{-2s} \right )}|
		\\
    & = 2 \pi | n |^{-s} (1 + | n |)^{s} | \frac{1}{2} - k_{n}e^{itn^{-2s}\left(
    k_{n}^{2}- \frac{1}{4}\right)} |.
	\end{split}
\end{equation*}
%
%
Hence, in order for uniform continuity of the flow map to hold for $s < 0$, we must have
%
%
\begin{equation*}
	\begin{split}
		\lim_{n \to \infty}  k_{n} e^{itn^{-2s}\left( k_{n}^{2} -
		\frac{1}{4} \right)}  = \frac{1}{2}.
	\end{split}
\end{equation*}
%
%
But setting $k_{n} = \left( \frac{1}{4} + n^{2s + \ee} \right)^{1/2}$ where  $\ee >
0$, we see that 
%
%
\begin{equation*}
	\begin{split}
		\lim_{n \to \infty} k_{n} e^{itn^{-2s}\left( k_{n}^{2} - \frac{1}{4}
		\right)}
    & = \lim_{n \to \infty} \left( \frac{1}{4} + n^{2s + \ee} \right)^{1/2}
    e^{itn^{\ee}} \\ & = \frac{1}{2} \lim_{n \to \infty} e^{itn^{\ee}}, \qquad s
    < 0
    \\
    & \neq \frac{1}{2}.
	\end{split}
\end{equation*}
%
%
In fact, the above limit does not converge at all. This concludes the proof for
the case $m \in \{4, 8, 12, \dots \}$. For the case $m \in \{2, 6, 10, \dots \}$, we take
%
%
\begin{equation*}
	\begin{split}
		u_{n}(x,t) = \frac{1}{2}n^{-s}e^{it\left( -n^{m} + \frac{1}{4}n^{-2s}
		\right)}e^{inx},
		\\ u_{n, k_{n}}(x,t) = k_{n}n^{-s}e^{it\left( -n^{m} + k_{n}^{2}n^{-2s}
		\right)}e^{inx} 
	\end{split}
\end{equation*}
and then repeat the above arguments. \qquad \qedsymbol
%
%
\begin{framed}
\begin{remark}
	Note that this result implies that it will be impossible to use a Picard
	iteration type argument to prove existence and uniqueness of solutions to the
	mNLS ivp for $s<0$, since this technique would imply uniform
	continuity of the flow map.
\end{remark}
\end{framed}
%
%
%\section{Failure of Continuity of the Flow Map}
%We shall prove the following.
%%
%%
%%%%%%%%%%%%%%%%%%%%%%%%%%%%%%%%%%%%%%%%%%%%%%%%%%%%%%
%%
%%
%%				 Failure of Continuity Theorem
%%
%%
%%%%%%%%%%%%%%%%%%%%%%%%%%%%%%%%%%%%%%%%%%%%%%%%%%%%%%
%%
%%
%\begin{theorem}
%The flow map $u_0 \mapsto u(t)$ of the mNLS ivp is not continuous for $s<0$, for
%any $t \neq 0$. More precisely, there exists initial data $u_0 \in L^2(\ci)$ and
%a sequence of intial data $\{u_{0,n} \} \subset L^2(\ci)$ such that
%%
%%
%\begin{equation*}
%	\begin{split}
%		u_{0,n} \to u_{0} \ \ \text{in} \ \ H^s(\ci)
%	\end{split}
%\end{equation*}
%%
%%
%and
%%
%%
%\begin{equation*}
%	\begin{split}
%		u_{n} \to e^{ \frac{it \gamma}{\pi}\left( \alpha^2 - \|u_{0}\|_{L^2(\ci)} 
%		\right)}u(t) \ \ \text{in} \ \ H^s(\ci)
%	\end{split}
%\end{equation*}
%%
%%
%where $\alpha \in \rr \setminus \|u_0\|_{L^2(\ci)}$ and $u$ is the unique solution to the mNLS ivp with
%initial data $u_{0}$.
%\end{theorem}
%%
%%
%{\bf Proof.} Define the trilinear operator
%%
%%
%\begin{equation*}
%	\begin{split}
%		g(u,v,w) \doteq \bar{u} v w.
%	\end{split}
%\end{equation*}
%%
%%
%Following Molinet, we rewrite this as
%%
%%
%\begin{equation*}
%	\begin{split}
%		g(u,v,w) 
%		&= \sum_{k_1, k_2, k_3 \in \zz}
%		\wh{\bar{u}}(k_{1})\wh{v}(k_2)\wh{w}(k_3)e^{i(k_1 + k_2 + k_3)x}
%		\\
%		& = \sum_{k_1, k_3 \in \zz} \wh{\bar{u}}(k_1)\wh{v}(-k_1)
%		\wh{w}(k_3)e^{ik_3 x} + \sum_{k_1, k_2 \in \zz} \wh{\bar{u}}(k_1)
%		\wh{v}(k_2) \wh{w}(-k_1)e^{ik_2x}
%		\\
%		& - \sum_{ k \in \zz} \wh{\bar{u}}(k) \wh{v}(-k)\wh{w}(-k)e^{-ikx}
%		+
%		\sum_{\substack{k_1, k_2, k_3 \in \zz\\ (k_1 + k_2)(k_1 + k_3) \neq 0}}
%		\wh{\bar{u}}(k_1) \wh{v}(k_2)
%		\wh{w}(k_3)e^{i(k_1 + k_2 + k_3)x}.
%	\end{split}
%\end{equation*}
%%
%%
%In particular, if $u=v=w$, we obtain
%%
%%
%\begin{equation*}
%	\begin{split}
%		g(u,u,u) = \frac{1}{\pi} \|u\|_{L^2}^2 u + \Lambda_1(u, u, u) + \Lambda_2
%		(u, u, u)	
%	\end{split}
%\end{equation*}
%%
%%
%where
%%
%%
%\begin{equation*}
%	\begin{split}
%		& \Lambda_1(u, v, w)
%		 = \sum_{\substack{k_1, k_2, k_3 \in \zz\\ (k_1 + k_2)(k_1 + k_3) \neq 0}}
%		\wh{\bar{u}}(k_1) \wh{v}(k_2)
%		\wh{w}(k_3)e^{i(k_1 + k_2 + k_3)x}
%		\\
%		& \Lambda_2(u, v, w) = - \sum_{ k \in \zz} \wh{\bar{u}}(k) \wh{v}(-k)\wh{w}(-k)e^{-ikx}
%	\end{split}
%\end{equation*}
%
%
%
\section{Proofs of Lemmas}
\begin{proof}[Proof of \cref{nlem:cutoff-loc-soln}]
%
%
\begin{equation}
  \label{ndm}
	\begin{split}
		\lim_{t_{k} \to t} \|u(\cdot, t) - u(\cdot, t_{k})\|_{H^s(\ci)} 
    & = \lim_{t_{k} \to t} \|\psi_{\delta}(t) u(\cdot, t) - \psi_{\delta}(t_{k}) u(\cdot, t_{k})\|_{H^s(\ci)} 
		\\
		& = \lim_{t_{k} \to t} \left[ \sum_{n}\left( 1 + | n |
    \right)^{2s} | \psi_{\delta}(t)  \wh{u}(n, t) - \psi_{\delta}(t_{k}) \wh{ u}(n, t_{k}) |^2 \right]^{1/2}
		\\
		& = \lim_{t_{k} \to t} \left[ \sum_{n} \left( 1 + | n |
    \right)^{2s} | \int_{\rr} (e^{it \tau} - e^{it_{k} \tau})
    \wh{\psi_{\delta} u}(n,
		\tau) d \tau |^2 \right]^{1/2}.
	\end{split}
\end{equation}
First note that
%
%
%
%
\begin{equation*}
\begin{split}
& \lim_{t_{k} \to t}  | \int_{\rr} (e^{it \tau} - e^{it_{k} \tau})
    \wh{\psi_{\delta} u}(n,
		\tau) d \tau |^2 
    \\
    = 
     & \lim_{t_{k} \to t}  \int_{\rr} (e^{it \tau} - e^{it_{k} \tau})
    \wh{\psi_{\delta} u}(n,
    \tau) d \tau \times \lim_{t_{k} \to t} \overline{\int_{\rr} (e^{it \tau} - e^{it_{k} \tau})
    \wh{\psi_{\delta} u}(n,
    \tau) d \tau }  
    \\
    = 
    &  \lim_{t_{k} \to t}  \int_{\rr} (e^{it \tau} - e^{it_{k} \tau})
    \wh{\psi_{\delta} u}(n,
    \tau) d \tau \times \lim_{t_{k} \to t} \int_{\rr} (e^{-it \tau} - e^{-it_{k} \tau})
    \overline{\wh{\psi_{\delta} u}}(n,
    \tau) d \tau.   
    \end{split}
\end{equation*}
%
%
But for fixed $n$ 
%
%
\begin{equation*}
\begin{split}
|(e^{it \tau} - e^{it_{k} \tau})  
    \wh{\psi_{\delta} u}(n, \tau) | \le 2 |\wh{\psi_{\delta} u(n, \tau)} |
\end{split}
\end{equation*}
%
%
and
%
%
%
\begin{equation*}
\begin{split}
  \int_{\rr} |2 \wh{\psi_{\delta} u(n, \tau)} | d \tau < \infty.
\end{split}
\end{equation*}
%
%
Hence, by dominated convergence
%
%
\begin{equation*}
\begin{split}
\lim_{t_{k} \to t}  \int_{\rr} (e^{it \tau} - e^{it_{k} \tau})
    \wh{\psi_{\delta} u}(n,
    \tau) d \tau =  \int_{\rr} \lim_{t_{k} \to t} (e^{it \tau} - e^{it_{k} \tau})
    \wh{\psi_{\delta} u}(n,
    \tau) d \tau = 0. 
\end{split}
\end{equation*}
%
%
Similarly, 
%
%
%
\begin{equation*}
\begin{split}
\lim_{t_{k} \to t} \int_{\rr} (e^{-it \tau} - e^{-it_{k} \tau})
    \overline{\wh{\psi_{\delta} u}}(n,
    \tau) d \tau  =\int_{\rr}  \lim_{t_{k} \to t} (e^{-it \tau} - e^{-it_{k} \tau})
    \overline{\wh{\psi_{\delta} u}}(n,
    \tau) d \tau  = 0.
\end{split}
\end{equation*}
%
%
Hence
%
%
%
\begin{equation}
  \label{ngh}
\begin{split}
  \lim_{t_{k} \to t} | \int_{\rr} (e^{it \tau} - e^{it_{k} \tau})
    \wh{\psi_{\delta} u}(n,
		\tau) d \tau |^2 = 0.
\end{split}
\end{equation}
%
%
		Furthermore,
    %
    %
    \begin{equation*}
    \begin{split}
      (1 + | n |)^{2s} | \int_{\rr} \left( e^{it\tau} - e^{it_{k} \tau} \right)
      \wh{\psi_{\delta}u}(n, \tau) d \tau|^{2} \le 4 (1 + | n |)^{2s} \left(
      \int_{\rr} | \wh{\psi_{\delta} u}(n, \tau)  | d \tau
      \right)^{2}
    \end{split}
    \end{equation*}
    %
    %
    and
		%
		%
		\begin{equation*}
			\begin{split}
         \sum_{n}  \left( 1 + | n |
        \right)^{2s} \left ( \int_{\rr} |\wh{\psi_{\delta} u}(n, \tau)| d \tau
        \right )^2  
        & = \|\wh{\psi_{\delta} u}\|_{\ell^{2}_{n}L^{1}_{\tau}}^2
		\\
		& \le \|\psi_{\delta} u \|_{X_{s,b}}^2 
	\end{split}
\end{equation*}
which is bounded by assumption. Therefore, applying dominated convergence and
\eqref{ngh}, we
obtain 
%
%
\begin{equation*}
\begin{split}
  \text{rhs of \eqref{ndm}} = \left[ \sum_{n} \left( 1 + | n |
    \right)^{2s} \lim_{t_{k} \to t} | \int_{\rr} (e^{it \tau} - e^{it_{k} \tau})
    \wh{\psi_{\delta} u}(n,
		\tau) d \tau |^2 \right]^{1/2} = 0
\end{split}
\end{equation*}
%
%
completing the proof. 
\end{proof}
%
%
\begin{proof}[Proof of \cref{nlem:schwartz-mult}]
Note that
%
%
\begin{equation*}
	\begin{split}
		\wh{\psi f}\left( n, \tau \right)
		& = \wh{\psi}(\cdot) * \wh{f}(n,
		\cdot)(\tau)
		= \int_\rr \wh{\psi}(\tau_1) \wh{f} \left( n, \tau - \tau_1 \right) 
		d\tau_1
	\end{split}
\end{equation*}
%
%
and hence
%
%
\begin{equation}
	\label{n1b}
	\begin{split}
		\|\psi f\|_{X^s} 
		& = \left( \sum_{n \in \zz} \left (1 + |n| \right )^{2s} \int_\rr \left( 1 + | \tau -
		n^{m} | \right) | \int_\rr \wh{\psi}(\tau_1) \wh{f}\left( n, \tau -
		\tau_1
		\right)  d \tau_1 d \tau |^2 \right)^{1/2}
		\\
		& \le \left( \sum_{n \in \zz} \left (1 + |n| \right )^{2s} \int_\rr \left( 1 + | \tau -
		n^{m }
		|
		\right) \left( \int_\rr |\wh{\psi}\left( \tau_1 \right) | |\wh{f}\left( n,
		\tau - \tau_1
		\right) |  d \tau_1 d \tau \right)^2 \right)^{1/2}.
	\end{split}
\end{equation}
%
%
Using the relation
%
%
\begin{equation*}
	\begin{split}
		1 + | \tau - n^{m } |
    & = 1 + | \tau - \tau_1 + \tau_{1} - n^{m} |
		\\
		& \le 1 + | \tau_1 | + | \tau - \tau_1 - n^{m} |
		\\
		& \le \left( 1 + | \tau_1 | \right)\left( 1 + | \tau - \tau_1 -
		n^{m} | \right)
	\end{split}
\end{equation*}
%
%
we obtain
%
%
\begin{equation*}
	\begin{split}
		\eqref{n1b}
		& \le \left( \sum_{n \in \zz} \left (1 + |n| \right )^{2s} \right.
		\\
		& \times \left . \int_\rr \left(
		\int_\rr \left( 1 + | \tau_1 | \right)^{1/2} | \wh{\psi}(\tau_1) |
		\left( 1 + | \tau - \tau_1 - n^{m} | \right)^{1/2} \wh{f}\left( n, \tau
		- \tau_1
		\right)d \tau_1
		\right)^2 d \tau \right)^{1/2}
	\end{split}
\end{equation*}
%
%
which by Minkowski's inequality is bounded by
%
%
\begin{equation}
	\label{n2b}
	\begin{split}
		& \left( \sum_{n \in \zz} \left (1 + |n| \right )^{2s}  \right.
		\\
		& \times \left. \left( \int_\rr \left[ \int_\rr
		\left( 1 + | \tau_{1} | \right) | \wh{\psi}(\tau_1) |^2 \left( 1 + |
		\tau - \tau_1 - n^{m} |
		\right) | \wh{f}\left( n, \tau - \tau_1 \right) |^2 d \tau 
    \right]^{1/2} d \tau_{1} \right)^2 \right)^{1/2}.
	\end{split}
\end{equation}
%
%
Using the change of variable $\tau - \tau_1 = \lambda$ gives
%
%
\begin{equation*}
	\begin{split}
		\eqref{n2b}
		& = \left( \sum_{n \in \zz} \left (1 + |n| \right )^{2s}\right.
		\\
		& \times \left.  \left( \int_\rr \left[
		\int_\rr \left( 1 + | \tau_1 | \right) | \wh{\psi}\left( \tau_1
		\right) |^2 \left( 1 + | \lambda - n^{m} | \right) | \wh{f} \left( n,
		\lambda
    \right)|^2 d \lambda \right]^{1/2} d \tau_{1} \right)^2 \right)^{1/2}
		\\
		& =  \left( \sum_{n \in \zz} \left (1 + |n| \right )^{2s} \right.
		\\
		& \times \left. \left( \int_\rr \left( 1 + |
		\tau_1 |
		\right)^{1/2} | \wh{\psi}(\tau_1) | d \tau_1 \left[ \int_\rr \left( 1 + |
		\lambda - n^{m} |
		\right) | \wh{f}\left( n, \lambda \right) |^2 d \lambda \right]^{1/2}
		\right)^2 \right)^{1/2}
		\\
		& = c_{\psi} \left( \sum_{n \in \zz} \left (1 + |n| \right )^{2s} \left( \left[ \int_\rr
		\left( 1 + | \lambda - n^{m} | \right) | \wh{f}\left( n, \lambda
		\right) |^2 d \lambda
		\right]^{\cancel{1/2}} \right)^{\cancel{2}} \right)^{1/2}
		\\
		& = c_{\psi} \|f\|_{X^s}.
	\end{split}
\end{equation*}
%
Also, by Young's inequality we have the estimate 
%
%
\begin{equation*}
\begin{split}
  \|\wh{\psi f}\|_{\ell^{2}_{n} L^{1}_{\tau}} 
  & = \left[ \sum_{n \in \zz} \left (1 + |n| \right )^{2s} \left (
  \int_{\rr} | \wh{\psi}(\cdot) * \wh{f}(n, \cdot)(\tau) | d \tau  \right ) ^2 \right]^{1/2}
  \\
  & \le  \left[ \sum_{n \in \zz} (1 + | n |)^{2s} \left( \int_{\rr} |
    \wh{\psi}(\tau) | d \tau  \times \int_{\rr} | \wh{f}(n, \tau) | d \tau
    \right)^{2}\right]^{1/2}
  \\
  & = c_{\psi} \| \wh{f} \|_{\ell^{2}_{n} L^{1}_{\tau}}
\end{split}
\end{equation*}
%
%
%
concluding the proof. 
\end{proof}
%
%
\begin{proof}[Proof of \cref{nlem:splitting}] We have
%
%
\begin{equation}
	\label{n6a}
	\begin{split}
		1 + | a + b + c| 
		& \le 1 + | a | + | b | + | c |
		\\
		& \le 1 + | a | + 1 + | b | + 1 + | c |
		\\
		& \le 3\left( \max\{1+| a |, 1+| b |, 1+ | c | \}\right)
		\\
		& \le 3 \left( 1 + | a | \right)\left( 1 + | b | \right) \left( 1 + |
		c |
		\right).
	\end{split}
\end{equation}
%
%
Raising both sides of expression $\eqref{n6a}$ to the $v$ power completes 
the proof. 
\end{proof}
%
%
%
\section{Remarks about mNLS and Related Equations}
%
\begin{proof}[Why \cref{nprop:trilinear-est} fails when dealing
with nonlinearity $\frac{1}{3} \p_x u^3$]
%
%
%
Recalling \eqref{nnon-lin-rep}, we have
\begin{equation}
	\begin{split}
		| \wh{w_{fgh}}(n, \tau)|
    & = | \sum_{n_1, n_2, n_3 = n}  \int_{\tau_{1} + \tau_{2} + \tau_{3} = \tau}
    n \wh{f}\left( n_1,  \tau_1 
\right) \wh{g}\left( n_2, \tau_2  
\right) \wh{h}\left( n_3, \tau_3 \right) d \tau_1 d \tau_2 d \tau_3 |
\\
& \le \sum_{n_1, n_2, n_3 = n}  \int_{\tau_{1} + \tau_{2} + \tau_{3} = \tau}
| n | \times | \wh{f}\left( n_1, \tau_1 
\right) | \times  | \wh{g}\left( n_2, \tau_2 
\right) | \times | \wh{ h}\left( n_3, \tau_3 \right) | d \tau_1 d \tau_2 d 
\tau_3
\\
& = \sum_{n_1, n_2, n_3 = n}  \int_{\tau_{1} + \tau_{2} + \tau_{3} = \tau} \frac{| n |c_f\left( n_1, \tau_1 
\right)}{\left (1 + |n_1| \right )^s \left( 1 + | \tau_1 - n_1^{m} | \right)^{b}}
\\
& \times \frac{c_{g}\left( n_2, \tau_2 \right)}{\left (1 + |n_2| \right ) 
^s\left( 1 + | \tau_2 -  n_2^{m }| 
\right)^{b}}
 \times \frac{c_{h}\left( n_3, \tau_3 \right)}{\left (1 + |n_3| \right ) ^s\left( 1 + | 
\tau_3 - n_3^{m } | \right)^{b}} \ d \tau_1 d \tau_2 d \tau_3
\end{split}
\end{equation}
where 
%
%
\begin{equation*}
	\begin{split}
		c_\sigma(n, \tau) = \left (1 + |n| \right ) ^s \left( 1 + | \tau - n^{m } |  
		\right)^{b} | \wh{\sigma}\left( n, \tau \right) | .
	\end{split}
\end{equation*}
%
%
Hence
%
%
\begin{equation*}
	\begin{split}
		 & \left (1 + |n| \right )^s \left( 1 + | \tau - n^{m } | \right)^{-b} | \wh{w_{fgh}}\left( 
		n, \tau \right) |
		\\
		& \le \left( 1 + | \tau - n^{m } | \right)^{-b}
		\sum_{n_1, n_2, n_3 = n}  \int_{\tau_{1} + \tau_{2} + \tau_{3} = \tau}
    \frac{\left (1 + |n| \right )^s}{\left (1 +
		|n_1| \right )^s \left (1 + | n_2| \right )^s \left (1 + |n_3| \right )^s} 
		\\
    & \times \frac{c_f(n_1, \tau_1)}{\left( 1 + | \tau_1 - n_1^{m } | 
		\right)^{b}}
		\times
		\frac{c_g(n_2, \tau_2)}{\left( 1 + | \tau_2 - n_2^{m } | 
		\right)^{b}} \times
		\frac{c_h(n_3, \tau_3)}{\left( 1 + | \tau_3 - n_3^{m } | 
		\right)^{b}}\ d \tau_1 d \tau_2 d \tau_3.
	\end{split}
\end{equation*}
%
%
For $s \ge 0$, observe that the quantity 
%
%
\begin{equation}
	\label{nunbounded-quan}
	\begin{split}
		\frac{| n | \left (1 + |n| \right ) ^s}{\left (1 + |n_1| \right ) ^s \left (1 + |n_2| \right ) ^s \left (1 + |n_3| \right ) ^s} 
	\end{split}
\end{equation}
is unbounded (take $n_1 = n_2 = 0$ and $n_3$ arbitrarily large). Hence
we hope that the  principal symbol $\tau - n^m$ offers enough
decay to give control of \eqref{nunbounded-quan}. Following the Bourgain
approach we consider the quantity
%
%
\begin{equation*}
	\begin{split}
		| \tau - n^{m} - \left( \tau_{1} - n_{1}^m + \tau_{2} - n_{2}^m +
		\tau_{3} - n_{3}^m \right) | = |n_{1}^m + n_2^m + n_3^m - n^m|
	\end{split}
\end{equation*}
%
%
and seek a lower bound that is a function of $n$. No such bound exists (this
becomes evident if one sets $n_1 = n_2$). In the case of the KDV, to prove well-posedness we need to bound
\begin{equation}
	\label{nKDV-bound-term}
	\begin{split}
		\frac{| n | \left (1 + |n| \right ) ^s}{\left (1 + |n_1| \right ) ^s \left (1 + |n_2| \right ) ^s} 
	\end{split}
\end{equation}
where $n_1 + n_2 = n$. 
Consider 
%
%
\begin{equation}
  \label{nah}
	\begin{split}
		| \tau - n^{3} - \left( \tau_{1} - n_{1}^3 + \tau_{2} - n_{2}^3 \right) | = |n_{1}^3 + n_2^3 - n^3|
	\end{split}
\end{equation}
where  $\tau_1 + \tau_2 = \tau$. Unlike the mNLS with derivative nonlinearity
considered above, in the case of the KDV we can use the conservation of mass to
\emph{exclude} the pathological cases $n_1=0$ and $n_2=0$, allowing us to obtain
a lower bound of $|n|^{2}$ for \eqref{nah}.
By the pigeonhole principle, we then have three
cases, each with the lower bound
%
%
\begin{equation*}
	\begin{split}
		\frac{1}{| \tau_{i} - n_{i}^{m} |^{b}} \gtrsim \frac{1}{|n|^{b}}	
	\end{split}
\end{equation*}
%
%
where $\tau_0 =\tau, n_0 = n$. 
Hence, we must set $b \ge 1$ to offset the $|n|$ in the numerator of 
\eqref{nKDV-bound-term}.
Lastly, in the case of the mNLS, we are able to bound 
\begin{equation*}
	\begin{split}
    \frac{\left (1 + |n| \right ) ^s}{\left (1 + |n_1| \right ) ^s \left (1 +
    |n_2| \right ) ^s \left (1 + |n_3| \right ) ^s}, \quad s \ge 0 
	\end{split}
\end{equation*}
without relying on any potential smoothing from the principal symbol.
This is due to the
absence of a $|n|$ term in the
numerator. A consequence is that we have more freedom in how
we choose $b$. In fact, we can choose $b$ all the way down to $3/8$, but no
lower, since we must have $b \ge 3/8$ in order to be able to apply
\cref{ncor:four-mult-est-L4}. 
\end{proof}
%
%
\begin{proof}[Conservation of the $L_x^2$ norm.] 
We have
%
%
\begin{equation*}
	\begin{split}
		\frac{d}{dt} \int_\ci | u |^2  dx
		& = \int_\ci \frac{d}{dt} | u |^2  dx
		\\
		& = \int_\ci \frac{d}{dt} \left( u \overline{u} \right)  dx
		\\
		& = \int_\ci \left( u \p_t \overline{u} + \overline{u} \p_t u \right) dx
		\\
		& = \int_\ci \left( u \overline{\p_t u} + \overline{u} \p_t u \right)dx.
	\end{split}
\end{equation*}
%
%
Substituting in $\p_t u = i\left( \p_x^{m} u + | u |^2 u \right)$ we obtain
%
%
\begin{equation*}
	\begin{split}
		& \int_{\ci} \left\{ u\left[ -i\left( \p_x^{m} \overline{u} + | u |^2
		\overline{u} \right) \right] + \overline{u}\left[ i\left( \p_x^{m} u + | u
		|^2 u \right) \right] \right\}dx
		\\
		& = \int_\ci \left[ -iu \p_x^{m} \overline{u} - i| u |^4 + i \overline{u}
		\p_x^{m} u + i | u |^4 \right]dx
		\\
		& = i \int_{\ci}\left( \overline{u} \p_x^{m} u - u \p_x^{m } \overline{u}
		\right)dx.
	\end{split}
\end{equation*}
%
%
Integrating by parts $m/2$ times and using
the spatial periodicity of $u$, the right
hand side simplifies to
%
%
\begin{equation*}
	\begin{split}
    i (-1)^{m/2}\int_\ci \left( \p_x^{m/2} \overline{u} \p_x^{m/2} u - \p_x^{m/2} u
		\p_x^{m/2 } 
		\overline{u} \right) dx = 0.
	\end{split}
\end{equation*}
%
%
Therefore, the $L_x^2(\ci)$ norm of solutions to the mNLS is conserved. 
\end{proof}
%
%
\begin{proof}[Why Assuming Mean Initial Data is Problematic]
Recall the NLS ivp
%
%
\begin{equation*}
	\begin{split}
		&i \p_t u = \p_x^2 u - | u |^2 u,
		\\
		& u(x,0) = \vp(x).
	\end{split}
\end{equation*}
%
%
This is equivalent to the ivp
%
%
\begin{gather*}
		 i \p_t [u - \wh{\vp}(0)]
		  = -\p_x^2 [u - \wh{\vp}(0)] - [u -
		\wh{\vp}(0)][u - \wh{\vp}(0)][\bar{u} - \bar{\wh{\vp}}(0)]
		\\
		- 2| u |^2
		\wh{\vp}(0) + \bar{u}\left[ \wh{\vp}(0) \right]^2 - u^{2}
		\bar{\wh{\vp}}(0) + 2 u | \wh{\vp}(0) |^2 - | \wh{\vp}(0) |^2
		\wh{\vp}(0),
		\\
		u(x,0) = \vp(x) - \wh{\vp}(0)
\end{gather*}
or
%
%
\begin{gather*}
		 i \p_t u 
		  = -\p_x^2 u - [u -
		\wh{\vp}(0)][u - \wh{\vp}(0)][\bar{u} - \bar{\wh{\vp}}(0)]
		\\
		- 2| u |^2
		\wh{\vp}(0) + \bar{u}\left[ \wh{\vp}(0) \right]^2 - u^{2}
		\bar{\wh{\vp}}(0) + 2 u | \wh{\vp}(0) |^2 - \boxed{| \wh{\vp}(0) |^2
		\wh{\vp}(0)},
		\\
		u(x,0) = \vp(x) - \wh{\vp}(0).
\end{gather*}
%
The boxed term is problematic. 
\end{proof}
%
\section{Classical Well-Posedness for the mNLS}
%
%
%%%%%%%%%%%%%%%%%%%%%%%%%%%%%%%%%%%%%%%%%%%%%%%%%%%%%
%
%
%			Alternate WP Theorem	
%
%
%%%%%%%%%%%%%%%%%%%%%%%%%%%%%%%%%%%%%%%%%%%%%%%%%%%%%
%
%
%
%
We have introduced the spaces $Y_s$ in part because well-posedness
in $H^s(\ci)$ for the mNLS becomes problematic as $s$ becomes small. 
On the other hand, for $s > 1/2$, well-posedness in $H^s(\ci)$ is a direct 
consequence of the algebra property of Sobolev spaces and the fact that the operator 
$e^{it \p_x^2}$ isometrically preserves Sobolev spaces. Stated more 
precisely, we have the following result:
%
%
\begin{proposition}
  Let $B_R \doteq \{f \in H^s : \|f\|_{H^s} < R \}$.
  Then the generalized mNLS ivp
\begin{gather}
  \label{general-mNLS-eq}
    i \p_t v = - \p_x^2 u - \lambda |u|^{\alpha -1} u, \ \ \alpha > 
    1, \lambda > 1
    \\
    \label{general-mNLS-init-data}
    u(x,0) = \vp(x), \ \ t \in \rr, \ \ x \in \ci \ \text{or} \ \rr
\end{gather}
  is locally well-posed in $H^s$ for $s > 1/2$ for 
  sufficiently small initial data $\vp \in B_R$, where the lifespan $T$ 
  satisfies 
%
%
\begin{equation*}
  \begin{split}
    T < 1/c
  \end{split}
\end{equation*}
%
%
for some constant $c = c(s, \lambda, \alpha, R, \vp)$.
\end{proposition}
%
%
\begin{proof} We will only provide a proof on the circle; the case on 
the line is nearly identical. The key ingredient
will be to establish that $L$ is a 
contraction on $C([-T, T], B_R)$. For the sake of clarity, we let $H^s_x 
= H^s_x(\ci)$. Let $e^{it \p_x^2}: \mathcal{E}'(\ci) \to 
\mathcal{E}'(\ci)$ be an operator defined by  
%
%
\begin{equation}
  \label{unit-op}
  \begin{split}
    e^{it \p_x^2} f(x) = \left[ e^{(-1)^j i t n^2} \wh{f}(n)
    \right]^{\vee} = 
    \sum_{n \in \zz} e^{i(nx + (-1)^j it n^2)} \wh{f}(n).
  \end{split}
\end{equation}
%
%
First, note that $e^{it \p_x^2}$ 
is unitary on $H^s(\ci)$; that is
%
%
\begin{equation}
  \label{unitary-op}
  \begin{split}
    \|e^{it \p_x^2} f \|_{H^s_x} & = \sum_{n \in \zz} |e^{(-1)^j it n^2} 
    \wh{f}(n)|^2 (1 + n^2)^s  
    \\
    & = \sum_{n \in \zz} |\wh{f}(n)|^2 (1 + n^2)^s 
    \\
    & = \|f\|_{H^s_x}.
  \end{split}
\end{equation}

Rewriting \eqref{general-mNLS-eq}-\eqref{general-mNLS-init-data} in its 
integral form
%
%
\begin{equation}
  \label{mNLS-int-form-with-op}
  \begin{split}
    u(x,t) = e^{it \p_x^2} \vp + i \lambda \int_0^t e^{i(t - 
    t')\p_x^2} |u|^{\alpha -1} u(x, t') \ dt' 
  \end{split}
\end{equation}
%
%
and applying the triangle inequality, Minkowski's inequality, and 
\eqref{unitary-op}, we obtain
%
%
\begin{equation}
  \label{bound-for-L}
  \begin{split}
    & \|Lu\|_{L^\infty_t[-T, T] H^s_x}
    \\
    & \le \|e^{t \p_x^2}
    \vp\|_{L^\infty_t[-T, T] H^s_x} + \|i \lambda \int_0^t e^{i(t - 
    t')\p_x^2} |u|^{\alpha -1} u(x, t') \ dt' \|_{L^\infty_t[-T, T] 
    H^s_x} 
    \\
    & \le \|\vp\|_{H^s_x} + |\lambda| \int_0^T \|e^{i(t 
    -t')\p_x^2} |u|^{\alpha -1} u \|_{L^\infty_t[-T, T] H^s_x} \ 
    dt'
    \\
    & = \|\vp\|_{H^s} + T |\lambda| \|u^\alpha \|_{L^\infty_t[-T, T] H^s_x}.
  \end{split}
\end{equation}
%
%
We now need the following lemma, whose proof can be found in Taylor 
\cite{Taylor:1991eg}:
%
%
%
\begin{lemma}
  \label{lem:algebra-prop}
  The Sobolev space $H^s$ is an algebra for $s>1/2$. More precisely, 
%
%
\begin{equation}
  \label{algebra-prop}
  \begin{split}
    \|fg\|_{H^s} \le c_s \|f\|_{H^s} \|g\|_{H^s}.
  \end{split}
\end{equation}
%
%
%
\end{lemma}
%
%
Applying \cref{lem:algebra-prop} to estimate \eqref{bound-for-L} gives
%
%
%
%
\begin{equation}
  \label{Tu-space-bound}
  \begin{split}
    \|Lu\|_{L^\infty_t[-T, T] H^s_x}
    & \le \|\vp\|_{H^s_x} + Tc_s | \lambda| \|u\|_{L^\infty_t[-T, T] 
    H^s_x}^\alpha
  \end{split}
\end{equation}
%
%
and since $u \in C([-T, T], B_R)$ a priori, it follows
that for sufficiently small $\vp$ and $T = T(s, \lambda, \alpha, R, \vp)$ we must 
have $Lu \in L^\infty([-T, T], B_R)$. To improve the regularity of 
$Lu$, let $\{t_n\} \subset [-T, T]$ and suppose that $t_n \to t \in [-T, 
T]$. Then
%
%
\begin{equation}
  \label{befo-dom}
  \begin{split}
    & \lim_{n \to \infty} \|Lu(\cdot, t) - Lu(\cdot, t_n)\|_{H^s_x} 
    \\
    & = \lim_{n \to \infty} \| \left \{ i \lambda \int_0^{t - t_n} e^{i(t  
    - t') \p_x^2} \left [|u|^{\alpha -1}u( \cdot, t') \right ]
    \ dt'\right \} \|_{H^s_x}
    \\
    & \le |\lambda|
    \lim_{n \to \infty}  \left \{  \int_0^{t - t_n} \| e^{i(t  
    - t') \p_x^2} \left [ |u|^{\alpha -1}u( \cdot, t') \right ]  
    \|_{H^s_x} \ dt' \right \}
    \\
    & = |\lambda|
    \lim_{n \to \infty}  \left[  \int_\rr \chi_{[0, t-t_n]}
    \| u^{\alpha}(\cdot, t') \|_{H^s_x} \ dt' \right ]
  \end{split}
\end{equation}
%
%
where the last step follows from \eqref{unitary-op}. 
By the algebra property and our a priori
assumption $u \in C([-T, T], B_R )$, we have 
%
%
\begin{equation*}
  \begin{split}
    \chi_{[0,t-t_n]}	\|u^\alpha(\cdot, t')\|_{H^s_x}
    \lesssim \chi_{[0,t-t_n]} 	\|u(\cdot, t')\|_{H^s_x}^\alpha 
    \le \chi_{[0,T]}\|u\|^\alpha_{L^\infty_t(\rr) H^s_x} \in 
    L^1_t(\rr).
    \end{split}
\end{equation*}
%
%
%
%
Hence, applying dominated 
convergence to \eqref{befo-dom}, we may pass the limit inside the integral,  
giving
%
\begin{equation*}
  \begin{split}
    \lim_{n \to \infty} \|Lu(\cdot, t)  - Lu(\cdot, t_n)\|_{H^s_x} 
    \le |\lambda|
    \int_\rr \lim_{n \to \infty} \left [ \chi_{[0, t-t_n]}
    \| u^{\alpha}(\cdot, t') \|_{H^s_x} \ dt' \right ]
    = 0
  \end{split}
\end{equation*}
%
%
which implies $Lu \in C([-T, T], B_R)$. Furthermore, for 
$u, v \in C([-T, T], B_R)$, we have
%
%
\begin{equation*}
  \begin{split}
    & \|Lu-Lv\|_{L^\infty_t[-T, T] H^s_x}
    \\
    & = \|i \lambda \int_0^t e^{i(t 
    -t')\p_x^2} (|u|^{\alpha - 1}u -|v|^{\alpha -1} v ) \ dt'
    \|_{L^\infty_t[-T, T] H^s_x}
    \\
    & \le |\lambda| \int_0^T \||u^{\alpha-1 }| u - | v^{\alpha - 1}| v
    \|_{L^\infty_t[-T, T] H^s_x} \ dt'
    \\
    & = |\lambda| T \cdot \|(u-v)(|u^{\alpha -1}| + |v^{\alpha -1}|) 
    + |u^{\alpha -1}|v 
    + u |v^{\alpha -1}| \|_{L^\infty_t[-T, T] H^s_x}
  \end{split}
\end{equation*}
%
%
which by the triangle inequality and algebra property simplifies to
%
%
\begin{equation}
  \label{L-contract}
  \begin{split}
    & \|Lu-Lv\|_{L^\infty_t[-T, T] H^s_x}
    \\
    & \le  T c_s |\lambda| \cdot \big [ \|u-v\|_{L^\infty_t[-T, T]
    H^s_x}(\|u\|^{\alpha -1}_{L^\infty_t[-T, T] H^s_x} +
    \|v\|^{\alpha -1}_{L^\infty_t[-T, T] H^s_x})
    \\
    & + \|u\|^{\alpha-1}_
    {L^\infty_t[-T, T] H^s_x} \|v\|_{L^\infty_t[-T, T] H^s_x}
    + \|u\|_
    {L^\infty_t[-T, T] H^s_x} \|v\|^{\alpha -1}_
    {L^\infty_t[-T, T] H^s_x} \big ]
    \\
    & \le T c_s |\lambda| \cdot \left[  2R^{\alpha -1} 
    \|u -v\|_{L^\infty_t[-T, T] H^s_x} + 2R^{\alpha} \right]
    \\
    & \le T c' \|u -v \|_{L^\infty_t[-T, T] 
    H^s_x}
  \end{split}
\end{equation}
%
%
where $c' = c'(s, \lambda, \alpha, R)$ is a constant.
Since it was established earlier that $T = T(s, \lambda, \alpha, R, \vp)$,  
we conclude that for $$T < 1/c,  \qquad c = c(s, \lambda, \alpha, R, 
\vp),$$ $L$ is a 
contraction on $C([-T, T], B_R)$. 
\end{proof}

%
%
%

