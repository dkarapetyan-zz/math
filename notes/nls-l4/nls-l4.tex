\chapter{Well-Posedness for the Cubic NLS (via Bourgain Approach in $L^{4}$) }
			  %
				  %
				  \section{Introduction}
				  We consider the cubic nonlinear Schr\"{o}dinger (NLS) 
				  initial value problem (ivp)
%
%
\begin{gather}
	\label{NLS-eq}
		i \p_t u = - \p_x^2 u - |u|^{2} u,
		\\
		\label{NLS-init-data}
		u(x,0) = \vp(x), \ \ x \in \ci, \ \ t \in \rr.
\end{gather}
%
%
and prove the following result.
%
%
%
%
%%%%%%%%%%%%%%%%%%%%%%%%%%%%%%%%%%%%%%%%%%%%%%%%%%%%%
%
%
%	Main Result				
%
%
%%%%%%%%%%%%%%%%%%%%%%%%%%%%%%%%%%%%%%%%%%%%%%%%%%%%%
%
%
\begin{theorem}
	\label{thm:main}
	For any $s \ge 0$, the initial value problem 
	\eqref{NLS-eq}-\eqref{NLS-init-data} is locally well-posed for 
	initial data $\vp(x) \in H^s(\ci)$.
%
%
\end{theorem} 
%
%
%
%
%%%%%%%%%%%%%%%%%%%%%%%%%%%%%%%%%%%%%%%%%%%%%%%%%%%%%
%
%
%				Outline
%
%
%%%%%%%%%%%%%%%%%%%%%%%%%%%%%%%%%%%%%%%%%%%%%%%%%%%%%
%
%
\section{Outline of the Proof of Theorem}
%
%
%
%
%
%
%
Restrict $t \in [0, 2\delta]$, where $\delta = \pi/m_0$ for some $m_0 \in 
\mathbb{N}$. Let $\psi_1(t)$ be a cutoff function symmetric about the 
origin such that $\psi_1(t) = 1$ for $|t| \le \delta$ and $\text{supp} \, \psi_1 
= 
[-2\delta, 2\delta ]$. Let $\psi_2(t)$ be a cutoff function symmetric about the origin such that 
$\psi_2(t) = 1$ for $|t| \le B$ and $\text{supp} \, \psi_2 = 
[-2B, 2 B]$.
Then, using analogous computations to those done for the mNLS IVP, we see that the cubic NLS IVP is equivalent to 
%
\begin{align}
	\label{main-int-expression-1}
	& u(x, t) \notag
		\\
		& = \frac{1}{2 \pi} \psi_1(t) \sum_{n \in \zz} e^{i(xn + tn^{2 
		j})} \widehat{\vp}(n) 
		\\
		\label{main-int-expression-2}
		& + \frac{1}{4 \pi^2} \psi_1(t) \sum_{n \in \zz} e^{ixn} \sum_{\tau \in \zz} 
		e^{it \tau} \frac{ 1 - \psi_2(\tau - n^2) 
		}{\tau - n^2} \wh{w}(n, \tau) 
		\\
		\label{main-int-expression-3}
		& - \frac{1}{4 \pi^2} \psi_1(t) \sum_{n \in \zz} e^{i(xn + 
		tn^2)}
		\sum_{\tau \in \zz} \frac{1- \psi_2(\tau - n^2)}{\tau - n^2} \wh{w}(n, \tau) 
		\\
		\label{main-int-expression-4}
		& + \frac{1}{8 B \pi^2} \psi_1(t) \sum_{k \ge 1} \frac{i^k (2Bt)^k}{k!}
		\sum_{n \in \zz} e^{i(xn + tn^2 )}
		\sum_{\tau \in \zz}	\psi_2 (\tau - n^2) \left(\frac{\tau - 
		n^2}{2B} \right)^{k -1} \wh{w}(n, \tau) 
		\\
		& \doteq T(u). \notag
\end{align}
where $T = T(\vp)$. Note that 
\eqref{main-int-expression-1}-\eqref{main-int-expression-4} is a global 
relation in $t$; furthermore, the fixed point solution $Tu=u$ gives rise to a 
local solution of the NLS ivp by simply restricting the time 
variable to 
the $[0, \delta]$ interval. Hence, we focus our attention on 
\eqref{main-int-expression-1}-\eqref{main-int-expression-4}. In 
Section 3 we will show that for initial data $\vp \in H^s(\ci)$, $T$ is 
a contraction on $B_M 
\subset L^4(\ci^2)$, where $B_M$ is the ball centered at 
the origin of radius $M = M(\vp) > 0$, by estimating the $L^4(\ci^2)$
norm of 
\eqref{main-int-expression-1}-\eqref{main-int-expression-4}. The Picard 
fixed point theorem and time restriction will
then yield a unique local solution to the NLS ivp in the time interval
$I = [0, \delta]$. Continuous 
dependence will follow from estimates used to establish the contraction 
mapping.  In Section 4 a similar argument for handling well-posedness for 
\eqref{NLS-integral-form} in $H^s(\ci)$, $s > 0$ will be discussed.
%In the 
%appendix we establish necessary conditions under which a local solution to 
%\eqref{NLS-integral-form} can be viewed as a distributional solution of 
%\eqref{NLS-eq}-\eqref{NLS-init-data}.
%
%
%
%
%%%%%%%%%%%%%%%%%%%%%%%%%%%%%%%%%%%%%%%%%%%%%%%%%%%%%
%
%
%	Case \ell^2			
%
%
%%%%%%%%%%%%%%%%%%%%%%%%%%%%%%%%%%%%%%%%%%%%%%%%%%%%%
%
%
\section{The case $\vp \in L^2(\ci)$}
\label{sec:s=0}
%
%
%
%
%%%%%%%%%%%%%%%%%%%%%%%%%%%%%%%%%%%%%%%%%%%%%%%%%%%%%
%
%
%		Estimation of Integral Equality Part 1		
%
%
%%%%%%%%%%%%%%%%%%%%%%%%%%%%%%%%%%%%%%%%%%%%%%%%%%%%%
%
%
\subsection{Estimate for \eqref{main-int-expression-1}}
Letting $f(x,t) = \psi(t) \sum_{n \in \zz} e^{i(xn + tn^2)} 
\wh{\vp}(n)$, we have $\wh{f}(n,t) = \psi(t) \wh{\vp}(n) e^{itn^2}$,
from which we obtain
%
%
\begin{equation}
	\label{fourier-trans-calc}
	\begin{split}
		\wh{f}(n, \tau)
		& = \wh{\vp}(n) \int_\rr e^{-it( \tau - n^2}) 
		\wh{\psi_1}(\tau) \ d\tau
		= \wh{\psi_1}(\tau - n^2) \wh{\vp}(n).
	\end{split}
\end{equation}
%
%
We now require the following multiplier estimate, whose proof can be found in 
\cite{Bourgain:1993ju}. %
%
%%%%%%%%%%%%%%%%%%%%%%%%%%%%%%%%%%%%%%%%%%%%%%%%%%%%%
%
%
%			Fourier Multiplier Estimate	
%
%
%%%%%%%%%%%%%%%%%%%%%%%%%%%%%%%%%%%%%%%%%%%%%%%%%%%%%
%
%
\begin{lemma}[Bourgain]
	\label{lem:four-mult-est}
	Let $(x, t) \in \ci^2$ and let $(n, \tau) \in \zz^2$ 
	be the dual variables. Then for $h \in L^2(\ci^2)$,
%
\begin{equation}
	\label{four-mult-est}
	\begin{split}
		\|h\|_{L^4(\ci^2)} \le c \|(1 + |\tau - 
		n^2|)^{3/8} \, \wh{h}(n, \tau) \|_{\ell^2(\zz^2)}.
	\end{split}
\end{equation}
%
%
\end{lemma}
%
%
%
%
%
%
%
Applying \cref{lem:four-mult-est} and \eqref{schwartz-bound} to
\eqref{fourier-trans-calc}, we obtain
%
%
\begin{equation}
	\label{main-int-est-part-1}
	\begin{split}
		\|\eqref{main-int-expression-1}\|_{L^4(\ci^2)} 
		& \le c  \|(1 + |\tau - 
		n^2|)^{3/8} \wh{\psi}(\tau - n^2) 		\wh{\vp}(n) \|_{\ell^2(\zz^2)}
		\\
		& \le c \|(1 + |\xi|)  \wh{\psi_1}(\xi)\|_{L^\infty(\zz)} 
		\|\wh{\vp} \|_{\ell^2(\zz)}
		\\
		& \lesssim \|\vp \|_{L^2(\ci)}.
	\end{split}
\end{equation}
%
%
%
%
%
%
%
%
\subsection{Estimate for \eqref{main-int-expression-2}.}
We have
%
%
\begin{equation}
	\label{1ag}
	\begin{split}
		\|\eqref{main-int-expression-2}\|_{L^4(\ci^2)} 
		& \lesssim \| \sum_{n \in \zz} e^{ixn} \sum_{\tau \in
		\zz} e^{it \tau} \frac{1- \psi_2\left( \tau - n^2 \right)}{\tau - 
		n^2} \wh{w}\left( n, \tau \right) \|_{L^4(\ci^2)}
		\\
		& \lesssim  \|\left( 1 + |\tau - n^2| \right)^{3/8}
		\frac{1- \psi_2\left( \tau - n^2 \right)}{\tau - 
		n^2} \wh{w}(n, \tau) \|_{\ell^2(\zz^2)}
	\end{split}
\end{equation}
%
%
where the last step follows from \cref{lem:four-mult-est}. Applying the 
estimate
%
%
\begin{equation}
	\label{2ag}
	\begin{split}
		\left[ 1 - \psi_2\left( \tau - n^2 \right) \right]
		\left( \tau - n^2 \right)^{-1}
		& \le \left( 1 + 1 /B \right)\left( 1 + |\tau - n^2| 
		\right)^{-1}
		\\
		& = \left( 1 + 1/B \right)\left( 1 + B \right)^{-k }
		\left( 1 + B \right)^k \left( 1 + |\tau - n^2| \right)^{-1} 
				\\
		& \le \left( 1 + 1/B \right)\left( 1 + B \right)^{-k 
		} \left( 1 + |\tau - n^2| \right)^{k-1}
		\\
		& = \frac{\left( 1 + B \right)^{1-k }}{B} \left( 1 + |\tau - 
		n^2|
		\right)^{k - 1}
		\\
		& \le 2^{k -1} B^{-k } \left( 1 + |\tau - n^2| \right)^{k - 1}, 
		\qquad k \le 1
	\end{split}
\end{equation}
%
%
with $k = 1/4$ to \eqref{1ag}, we obtain 
%
%
\begin{equation}
	\label{3ag}
	\begin{split}
		\|\eqref{main-int-expression-2}\|_{L^4(\ci^2)}
		& \lesssim  
		B^{-1/4} \|\left( 1 + |\tau - n^2| \right)^{-3/8} \wh{w} (n, 
		\tau) \|_{\ell^2(\zz^2)}.
	\end{split}
\end{equation}
%
%
We now need the following dual estimate of 
\cref{lem:four-mult-est}, whose proof is provided in the appendix:
%
%
\begin{corollary}
	\label{cor:four-mult-est-dual}
Let $(n, \tau) \in \zz^2$ and let  $(x, t) \in \ci^2$
be the dual variables. Then for $h \in L^2(\ci^2)$
%
%
\begin{equation}
	\label{four-mult-est-dual}
	\|(1 + |\tau - 
	n^2|)^{-3/8} \wh{h}(n, \tau) \|_{\ell^2(\zz^2)} \le 
	c \|h\|_{L^{4/3}(\ci^2)}.
\end{equation}
%
%
\end{corollary}
%%
%%
Applying \cref{cor:four-mult-est-dual} to \eqref{3ag}, we conclude that 
%
%
\begin{equation}
	\label{main-int-est-2}
	\begin{split}
		\|\eqref{main-int-expression-2}\|_{L^4(\ci^2)} 
		& \le c B^{-1/4} \|w\|_{L^{4/3}(\ci^2)}
		\\
		& = c B^{-1/4} \|u^3\|_{L^{4/3}(\ci^2)}
		\\
		& = c B^{-1/4} \|u\|^3_{L^4(\ci^2)}.
	\end{split}
\end{equation}
%%
%%

\subsection{Estimate for \eqref{main-int-expression-3}.} 
Letting $$f(x,t) = \psi(t) \sum_{n \in \zz} e^{i\left( xn + tn^2 \right)} 
\sum_{\lambda \in \zz} \frac{1 - \psi\left( \tau - n^2 \right)}{\tau - n^2} 
\wh{w} \left( n, \tau \right),$$ we have
%
%
\begin{equation*}
	\begin{split}
		& \wh{f^x}(n, t) = \psi(t) e^{itn^2} \sum_{\lambda \in \zz} 
		\frac{1 - \psi\left( \lambda - n^2 \right)}{\lambda - n^2} 
		\wh{w}(n, \lambda)
	\end{split}
\end{equation*}
and
\begin{equation*}
	\begin{split}
		 \wh{f}\left( n, \tau \right)
		 & = \int_\ci e^{-it\left( \tau - n^2 
		\right)} \psi(t) \sum_{\lambda \in \zz} \frac{1 - \psi\left( 
		\lambda - n^2 
		\right)}{\lambda - n^2} \wh{w}(n, \lambda)
		\\
		& = \wh{\psi_1}\left( \tau - n^2 \right) \sum_{\lambda \in \zz}
		\frac{1 - \psi\left( 
		\lambda - n^2 
		\right)}{\lambda - n^2} \wh{w}(n, \lambda).
	\end{split}
\end{equation*}
%
%
Therefore, applying \cref{lem:four-mult-est} and 
\eqref{schwartz-bound} gives 
%
%
\begin{equation}
	\label{4hh}
	\begin{split}
		\|f\|_{L^4(\ci^2)}
		& \lesssim \|\left( 1 + | \tau - n^2| \right)^{3/8} 
		\wh{\psi_1}\left( \tau - n^2 \right) \sum_{\lambda \in \zz} 
		\frac{1 - \psi_2\left( \lambda - n^2 \right)}{\lambda - n^2} 
		\wh{w}\left( n, \lambda \right) 
		\|_{\ell^2\left( \zz^2 \right)}
		\\
		& \lesssim \|\sum_{\lambda \in \zz}
		\frac{1 - \psi_2\left( \lambda - n^2 \right)}{\lambda - n^2} \wh{w} 
		\left( n, \lambda \right) \|_{\ell^2\left( \zz \right)}.
	\end{split}
\end{equation}
%
%
Next, note 
that 
%
%
\begin{equation}
	\label{apply-ortho}
	\begin{split}
		\|\sum_{\lambda \in \zz}
		\frac{1 - \psi_2\left( \lambda - n^2 \right)}{\lambda - n^2} \wh{w} 
		\left( n, \lambda \right) \|_{\ell^2\left( \zz \right)}
		& = \left( \sum_{n \in \zz}  | \sum_{\lambda \in \zz}
		\frac{1 - \psi_2\left( \lambda - n^2 \right)}{\lambda - n^2} \wh{w} 
		\left( n, \lambda \right) |^2 \right )^{1/2}
		\\
		& \lesssim \left( \sum_{n \in \zz} \sum_{\lambda \in \zz} |
		\frac{1 - \psi_2\left( \lambda - n^2 \right)}{\lambda - n^2} 
		\wh{w}\left( n, \lambda \right) |^2 
		\right)^{1/2} 
	\end{split}
\end{equation}
%
%
where the last step follows from orthogonality. 
But by \eqref{2ag} and \cref{cor:four-mult-est-dual}, we have
\begin{equation*}
	\begin{split}
		\left( \sum_{n, \tau \in \zz} |
		\frac{1 - \psi_2\left( \lambda - n^2 \right)}{\lambda - n^2} 
		\wh{w}\left( n, \lambda \right) |^2 
		\right)^{1/2}
		& = \|\frac{1 - \psi_2 \left( \lambda - n^2 
		\right)}{\lambda - n^2} \wh{w}\left( n, \lambda \right) 
		\|_{\ell^2(\zz^2)}
		\\
		& \lesssim B^{-5/8} \|\left( 1 + |\tau - n^2| \right)^{-3/8} 
		\wh{w}\left( n, \lambda \right) \|_{\ell^2(\zz^2)}
		\\
		& \lesssim B^{-5/8} \|w\|_{L^{4/3}(\ci^2)}
		\\
		& = B^{-5/8} \|u\|_{L^4(\ci^2)}^3.
	\end{split}
\end{equation*}
%
%
Substituting back into \eqref{4hh}, and letting $c$ absorb all superficial 
constants, we obtain
%
%
\begin{equation}
	\label{main-int-3-est}
	\begin{split}
		\|f\|_{L^4(\ci^2)} \le c \delta B^{-5/8} 
		\|u\|_{L^4(\ci^2)}^3.
	\end{split}
\end{equation}
%
%
\subsection{Estimate for \eqref{main-int-expression-4}.}
Noting that
%
%
\begin{equation*}
	\begin{split}
		|\sum_{k \ge 1} \frac{i^k (2Bt)^k}{k!}| = |e^{2iBt} - 1| \le 2,
	\end{split}
\end{equation*}
%
%
we obtain
%
%
\begin{equation}
	\label{10aa}
	\begin{split}
		\|\eqref{main-int-expression-4}\|_{L^4(\ci^2)} 
		\le 2 \sup_k \|f \|_{L^4(\ci^2)}
	\end{split}
\end{equation}
%
%
where $$f(x,t) = \psi_1(t) \sum_{n \in \zz} e^{i\left( xn + tn^2 \right)} 
		\sum_{\tau \in \zz} \psi_2\left( \tau - n^2 
		\right)\left (\frac{ \tau - n^2}{2B}\right)^{k - 1} \wh{w}\left( n, \tau 
		\right).$$
%
%
Calculating the Fourier transform of $f(x, t)$, we have
\begin{equation*}
	\begin{split}
		& \wh{f^x}(n, t) = \psi_1(t) e^{itn^2} \sum_{\tau \in \zz} 
		\psi_2\left (\frac{ \tau - n^2}{2B}\right)^{k - 1}
		\wh{w}\left( n, \tau \right)
	\end{split}
\end{equation*}
which gives
\begin{equation*}
	\begin{split}
		\wh{f}(n, \lambda)
		& = \int_\ci e^{-it\left( \lambda - n^2 \right)} 
		\psi_1(t) \ dt \sum_{\tau \in \zz} \psi_2
		\left (\frac{ \tau - n^2}{2B}\right)^{k - 1}
\wh{w}\left( n, \tau 
		\right)
		\\
		&  = \wh{\psi_1}\left( \lambda - n^2 \right) \sum_{\tau \in \zz} 
		\psi_2\left( \tau - n^2 \right)
		\left (\frac{ \tau - n^2}{2B}\right)^{k - 1}
		\wh{w} \left(n, \tau \right).
	\end{split}
\end{equation*}
%
%
Therefore, applying \cref{lem:four-mult-est} and 
\eqref{schwartz-bound} to \eqref{10aa} yields
%
%
\begin{equation}
	\label{main-int-4-est-prelim}
	\begin{split}
		& \|f\|_{L^4(\ci^2)} 
		\\
		& \le c \sup_{k \ge 1} \|\left( 1 + |\tau - n^2| 
		\right)^{3/8} \wh{\psi_1}\left( \tau - n^2 \right) \sum_{\lambda 
		\in \zz} \psi_2\left( \lambda - n^2 \right)\left( \frac{\lambda - 
		n^2}{2B} 
		\right)^{k - 1} \wh{w}\left( n, \lambda \right) \|_{\ell^2(\zz^2)}
		\\
		& \le c \delta \| \sum_{\lambda \in \zz} 
		\psi_2\left( \lambda - n^2 \right) \wh{w}\left( n, \lambda \right)\|_{\ell^2(\zz)}
		\\
		& = c \delta \left( \sum_{n \in \zz} | 
		\sum_{\lambda \in \zz} \psi_2\left( \lambda - n^2 \right) 
		\wh{w}\left( n, \lambda \right)|^2\right)^{1/2}
		\\
		& \lesssim \delta \left( \sum_{n \in \zz} \wh{w}(n, 
		n^2) \right)^{1/2}
	\end{split}
\end{equation}
%
%
where the last step follows by the definition of 
$\psi_2$. 
Applying \cref{cor:four-mult-est-dual}, and letting $c$ absorb all 
superficial constants, we conclude
that 
%
%
\begin{equation}
	\label{main-int-4-est}
	\begin{split}
		\|f\|_{L^4(\ci^2)}
		& \le c \delta B \|w\|_{L^{4/3}(\ci^2)}
		\\
		& \le c \delta B \|u^3\|_{L^{4/3}(\ci^2)}
		\\
		& \le c \delta B |u\|_{L^4(\ci^2)}^3.
	\end{split}
\end{equation}
%
%
Collecting estimates \eqref{main-int-est-part-1},\eqref{main-int-est-2}, 
\eqref{main-int-3-est}, and \eqref{main-int-4-est}, we obtain
%
%
\begin{equation}
	\label{1gh}
	\begin{split}
		\|Tu\|_{L^4(\ci^2)} \le c\left( \|\vp\|_{L^2(\ci)}
		+ B^{-1/4} \|u\|_{L^4(\ci^2)}^3 + \delta B \|u\|_{L^4(\ci^2)}^3 \right).
	\end{split}
\end{equation}
%
%
Let $B_M = \left\{ u \in L^4(\ci^2): \|u\|_{L^4(\ci^2)} \le 2c \|\vp 
\|_{L^2(\ci)} \right\}$. Setting $B = \delta^{-1/2}$ and restricting $u \in B_M$, we 
obtain %
%
\begin{equation*}
	\begin{split}
		\|Tu\|_{L^4(\ci^2)}
		& \le c \left[ \|\vp\|_{L^2(\ci)} + \left( \delta^{1/8} 
		+ \delta^{1/2} \right) M^3 \right]
		\\
		& \le c \left[ \|\vp\|_{L^2(\ci)} + 2 \delta^{1/2} M^3 \right]
		\\
		& \le c \left[ \|\vp\|_{L^2(\ci)} + 16c^3 \delta^{1/2} \|\vp\|_{L^2(\ci)}^3 
		\right].
	\end{split}
\end{equation*}
 %
%
Recalling that  $\delta = \pi/n, \ n \in \mathbb{N}$ and choosing $n$ large enough 
such that $\delta < 1/(10^6 c^6 \|\vp\|_{L^2(\ci)}^4)$ then gives
%
%
\begin{equation}
	\label{ball-to-ball}
	\begin{split}
		\|Tu\|_{L^4(\ci^2)} < 2c\|\vp\|_{L^2(\ci)} = M, \qquad u \in B_M.
	\end{split}
\end{equation}
%
%
Hence, $T: B_M \to B_M$. Furthermore, from our definition of $T$ in 
\eqref{main-int-expression-1}-\eqref{main-int-expression-4}, we obtain
%
%
\begin{equation*}
	\begin{split}
		Tu - Tv = \eqref{main-int-expression-2} + 
		\eqref{main-int-expression-3} + \eqref{main-int-expression-4}
	\end{split}
\end{equation*}
%
%
where now $w = u|u|^2 - v|v|^2$. By the triangle inequality and generalized
H\"{o}lder
%
%
\begin{equation}
	\label{gen-holder}
	\begin{split}
		\|u |u|^2 - v |v|^2\|_{4/3}
		& = \| |u|^2\left( u -v \right) + v\left( |u|^2 - |v|^2 
		\right)\|_{4/3}
		\\
		& \le \|u^2\left( u -v \right)\|_{4/3} + \|v\left( |u| + |v| 
		\right)\left( |u| - |v| \right) \|_{4/3}
		\\
		& \le \|u\|_4^2 \|u -v \|_4 + \|v\|_4  \| |u| + |v| \|_4 
		\| |u| - |v |\|_4 
		\\
		& \le \|u \|_4^2 \|u -v\|_4 + \|v\|_4\left( \|u\|_4 + \|v\|_4 
		\right) \|u -v \|_4
		\\
		& \le 2 \|u -v \|_4 \left(  \|u\|_4 + \|v\|_4 \right)^2.
	\end{split}
\end{equation}
%
%
Substituting $w = u|u|^2 - v|v|^2$ into the first line of \eqref{main-int-est-2}, 
\eqref{main-int-3-est}, and \eqref{main-int-4-est}, and estimating using  
\eqref{gen-holder}, we conclude that
%
%
\begin{equation*}
	\begin{split}
		\|Tu - Tv\|_{L^4(\ci^2)}
		& \le 2c\left( \delta B + B^{-1/4} 
		\right)\left( \|u\|_{L^4(\ci^2)} +
		\|v\|_{L^4(\ci^2)}\right)^2 \|u -v \|_{L^4(\ci^2)}
		\\
		& \le 8 c M^2(\delta B + B^{-1/4}) \|u - v\|_{L^4(\ci^2)}
		\\
		& \le 16cM^2 \delta^{1/2} \|u-v\|_{L^4(\ci^2)}
		\\
		& < 64 \cancel{c^3} \cancel{\|\vp\|_{L^2(\ci)}^2}
		\times \frac{1}{10^3 \cancel{c^3} \cancel{\|\vp\|_{L^2(\ci)}^2}}
		\|u-v\|_{L^4(\ci^2)}
	\end{split}
\end{equation*}
%
%
which yields the estimate
%
%
\begin{equation}
	\label{contract-est}
	\begin{split}
		\|Tu - Tv\|_{L^4(\ci^2)} < \frac{1}{2} \|u -v \|_{L^4(\ci^2)}.
	\end{split}
\end{equation}
%
%
By \eqref{ball-to-ball} and \eqref{contract-est}, we conclude that
$T = T(\vp)$ is a contraction on $B_M \subset L^4(\ci^2)$. 
%
%
%

%
%
\begin{proof}[Proof of \cref{cor:four-mult-est-dual}]
By duality,
%
%
\begin{equation}
  \label{1g}
  \begin{split}
    \|f\|_{L^{\frac{4}{3}}(\ci^2)}
    & = \sup_{g \in L^4(\ci^2)} 
    \frac{|\ \int_{\ci^2} f \bar{g} \ dx dt|}{\|g\|_{L^4(\ci^2)}}
    \\
    & \ge \frac{|\ \int_{\ci^2} f \bar{g} \ dx 
    dt|}{\|g\|_{L^4(\ci^2)}}, \qquad g \in L^4(\ci^2)
    \\
    &  =   \frac{|\sum_{m,n} \wh{f}(m,n) 
    \bar{\wh{g}}\left( m,n \right)| }{ 4 \pi^2 \|g\|_{L^4(\ci^2)}}
  \end{split}
\end{equation}
%
%
where the last step follows from Parseval's theorem. Choose $g$ such that
%
%
\begin{equation*}
  \begin{split}
    \wh{g}(m,n) = \left( 1 + |n - m^2| \right)^{-3/4} \wh{f}(m,n).
  \end{split}
\end{equation*}
%
%
Then by \cref{lem:four-mult-est},
%
%
\begin{equation*}
  \begin{split}
    \|g\|_{L^4(\ci^2)}
    & \le \|\left( 1 + |n-m^2| \right)^{-3/8} 
    \wh{g}(m,n) \|_{\ell^2(\zz^2)}
    \\
    & = \|\left( 1 + |n-m^2| \right)^{-3/8} 
    \left( 1 + |n - m^2| \right)^{-3/4} \wh{f}(m,n)  \|_{\ell^2(\zz^2)}
    \\
    & \lesssim \|f\|_{\ell^2(\ci^2)} < \infty.
  \end{split}
\end{equation*}
%
%
Therefore, substituting into \eqref{1g}, we obtain
%
%
\begin{equation*}
  \begin{split}
    \|f\|_{L^{4/3}(\ci^2)}
    & \ge \frac{\sum_{m,n} |\wh{f}(m,n)|^2 \left( 
    1 + |n-m^2| \right)^{-3/4}}{4 \pi^2 \|g\|_{L^4(\ci^2)}}
    \\
    & \ge \frac{\sum_{m,n} |\wh{f}(m,n)|^2 \left( 
    1 + |n-m^2| \right)^{-3/4}}{4 \pi^2 c \|\left( 1 + |n - m^2| 
    \right)^{3/8} \wh{g}(n,m) \|_{\ell^2(\zz^2)}}
    \\
    & = \frac{ \sum_{m,n} |\wh{f}(m,n)|^2 \left( 1 + |n - m| 
    \right)^{-3/4}}{4 \pi^2 c \left( \sum_{n,m} |\wh{f}(n,m)|^2 \left( 
    1 + |n-m^2| \right)^{-3/4} \right)^{1/2}}
    \\
    & \simeq \|\left( 1 + |n-m^2| \right)^{-3/8} \wh{f}(m,n) \|_{\ell^2(\zz^2)}
  \end{split}
\end{equation*}
%
%
completing the proof. 
\end{proof}
%
%
