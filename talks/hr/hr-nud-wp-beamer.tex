\documentclass{beamer}
%\let\Tiny=\tiny
\usepackage{amscd}
\usepackage{amsfonts}
\usepackage{amsmath}
\usepackage{amssymb}
\usepackage{amsthm}
\usepackage{fancyhdr}
\usepackage{latexsym}
\usepackage{lmodern}
\synctex=1
\input epsf
\input texdraw
\input txdtools.tex
\input xy
\xyoption{all}
\usepackage{color}

\definecolor{Red}{rgb}{1.00, 0.00, 0.00}
\definecolor{DarkGreen}{rgb}{0.00, 1.00, 0.00}
\definecolor{Blue}{rgb}{0.00, 0.00, 1.00}
\definecolor{Cyan}{rgb}{0.00, 1.00, 1.00}
\definecolor{Magenta}{rgb}{1.00, 0.00, 1.00}
\definecolor{DeepSkyBlue}{rgb}{0.00, 0.75, 1.00}
\definecolor{DarkGreen}{rgb}{0.00, 0.39, 0.00}
\definecolor{SpringGreen}{rgb}{0.00, 1.00, 0.50}
\definecolor{DarkOrange}{rgb}{1.00, 0.55, 0.00}
\definecolor{OrangeRed}{rgb}{1.00, 0.27, 0.00}
\definecolor{DeepPink}{rgb}{1.00, 0.08, 0.57}
\definecolor{DarkViolet}{rgb}{0.58, 0.00, 0.82}
\definecolor{SaddleBrown}{rgb}{0.54, 0.27, 0.07}
\definecolor{Black}{rgb}{0.00, 0.00, 0.00}
\definecolor{dark-magenta}{rgb}{.5,0,.5}
\definecolor{myblack}{rgb}{0,0,0}
\definecolor{darkgray}{gray}{0.5}
\definecolor{lightgray}{gray}{0.75}
%
\newcommand{\bigno}{\bigskip\noindent}
\newcommand{\ds}{\displaystyle}
\newcommand{\medno}{\medskip\noindent}
\newcommand{\smallno}{\smallskip\noindent}
\newcommand{\nin}{\noindent}
\newcommand{\ts}{\textstyle}
\newcommand{\rr}{\mathbb{R}}
\newcommand{\p}{\partial}
\newcommand{\zz}{\mathbb{Z}}
\newcommand{\cc}{\mathbb{C}}
\newcommand{\ci}{\mathbb{T}}
\newcommand{\tor}{\mathbb{T}}
\newcommand{\ee}{\varepsilon}
\newcommand{\wh}{\widehat}
\newcommand{\weak}{\rightharpoonup}
\newcommand{\vp}{\varphi}
%
%
\newtheorem{proposition}{Proposition}
\newtheorem{claim}{Claim}
\newtheorem{remark}{Remark}
\newtheorem{conjecture}[subsection]{conjecture}

\def\refer #1\par{\noindent\hangindent=\parindent\hangafter=1 #1\par}

%% Equation Numbers %%

\renewcommand{\theequation}{\thesection.\arabic{equation}}





%%%%%%%%%%%%%%%%%%%%%%

\date{}
\title{Non-Uniform Dependence and Well-posedness for the Hyperelastic Rod Equation}
\author{\it David Karapetyan}
\begin{document}
 %
 \begin{frame}
	 \titlepage
 \end{frame}

\section*{Table of Contents}
\begin{frame}
	\frametitle{Table Of Contents}
	\tableofcontents
\end{frame}


 \begin{frame}
	 \frametitle{Abstract}
	 We consider the Cauchy problem for the Hyperelastic Rod (HR) equation
	 \begin{equation*} 
		 \label{hr}
		 \partial_t u + \gamma u\partial_x u + (1-\p_x^2)^{-1} \p_x \left
		 [\frac{3- \gamma}{2} u^2 + \frac{\gamma}{2}(\p_x u)^2 \right ] = 0,
		 \ x \in  \mathbb{T} \  \text{or}  \ \rr,  \ t \in \mathbb{R},
	 \end{equation*}
	 \begin{equation*} 
		 \label{hr-data} 
		 u(x, 0) = u_0 (x)
	 \end{equation*}
%
and prove the following result:
%
%
\pause
%
\begin{theorem}
\label{hr-non-unif-dependence}
Let $\gamma$ be a nonzero constant. Then 
the data-to-solution map $u(0) \mapsto u(t)$ of the Cauchy-problem
for the HR equation is not uniformly continuous
from any bounded subset of  $H^s$ into $C([-T, T], H^s)$
for $s>1$ on the line  and for $s>3/2$ on the circle.
%
\end{theorem}
\end{frame}
%
%
\section{Strategy}
\begin{frame}
	\frametitle{Strategy}
We begin by outlining the method of the proof,
as it has been applied for the case $\gamma=1$ in \cite{Himonas_2009_Non-uniform-dep}.
We will show that there there exist two sequences of solutions 
$u_n(t)$
and $v_n(t)$ in $C([-T, T], H^s)$ such that
\pause
%
%
%
%
\begin{equation*}
\label{h-s-bdd}
\| u_n(t)  \|_{H^s}
+
\| v_n(t)  \|_{H^s}
\lesssim
1,
\end{equation*}
%
%
%
%
%
\begin{equation*}
\label{zero-limit-at-0}
\lim_{n\to\infty}
\|
u_n(0)
-
v_n(0)
\|_{H^s}
=
0,
\end{equation*}
%
%
%
%
and
%
%
%
%
\begin{equation*}
\label{bdd-away-from-0}
\liminf_{n\to\infty}
\|
u_n(t)
-
v_n(t)
\|_{H^s}
\gtrsim
|\sin ( \gamma t)|,
\quad
| \gamma t|\le 1.
\end{equation*}%
%
%
\pause
We accomplish this in two steps.
First, we will construct two sequences of approximate solutions
satisfying the above properties.
Then, we will construct two sequences of actual solutions 
coinciding with the approximate solutions at time zero.
The key point of this method is that 
the difference between solutions and approximate solutions
must decay.
\end{frame}
%
%
\section{Well-Posedness Theorem}
\begin{frame}
	\frametitle{Well-Posedness}
We will require the following.
\begin{theorem}
\label{hr-wp}
If $u_0(x) \in  H^s$ for some $s >3/2$,  then there is  a $T>0$
depending only on  $\|u_0\|_{H^s}$ such that there exists a unique
function $u(x, t)$ solving  the HR Cauchy problem
in the sense of distributions with  $u \in C([0, T]; H^s)$.
The solution $u$ depends continuously on the initial data $u_0$
in the sense that the mapping of the initial data to the solution 
is continuous from the Sobolev space $H^s$ to the space $C([0, T]; H^s)$.
Furthermore, the  lifespan (the maximal existence time)
 is greater than 
%
     \begin{equation*}
   T
   \doteq
   \frac{1}{2c_s}
   \frac{1}{ \|u_0 \|_{H^s(\ci)}},
 \end{equation*}
%
where $c_s$  is a constant depending only on $s$.
Also, we have 
%
  \begin{equation*}
   \label{u-u0-Hs-bound}
\|u(t)\|_{H^s(\ci))}
  \le
  2
  \|u_0 \|_{H^s(\ci)},
  \quad
  0\le t \le T.
   \end{equation*}
  %
\end{theorem}
\end{frame}
%
%
%
\section{Approximate solutions on the Line}
\begin{frame}
	\frametitle{Approximate Solutions on the Line}
Following \cite{Himonas_2009_Non-uniform-dep}, our approximate solutions
\\ $u^{\omega, \lambda} = u^{\omega,
\lambda}(x,t)$ will
consist of a low frequency and a high frequency part,
i.e.
%
%
%
%
\begin{equation*}
\label{apple1}
u^{\omega,\lambda} = u_\ell + u^h
\end{equation*}
%
%
%
%
where $\omega$ is in a bounded set of $\rr$ and $\lambda > 0$. The high frequency part is given by 
\pause
%
%
%
%
\begin{equation*}
\begin{split}
u^h = u^{h,\omega,\lambda}(x,t) =
\lambda^{-\frac{\delta}{2} -s}
\phi \left (\frac{x}{\lambda^\delta}\right )
\cos(\lambda x - \gamma \omega t)
\end{split}
\end{equation*}
%
%
%
%
where $\phi$ is a $C^\infty$ cut-off function such that
%
%
%
%
\begin{equation*}
\phi = \begin{cases}
1, &\text{if $|x|<1$,} \\
0, &\text{if $|x| \ge 2,$} \end{cases}
\end{equation*}
%
%
%
%
\pause
and by the existence portion of well-posedness,
we let the low frequency part $u_\ell = u_{l,
\omega, \lambda}(x,t)$ be the unique solution to the Cauchy problem
\end{frame}
%
%
\begin{frame}
\begin{align*}
& \p_t u_\ell = -\gamma u_\ell \p_x u_\ell -
\Lambda^{-1} \left[ \frac{3-\gamma}{2}(u_\ell)^2 +
\frac{\gamma}{2} \left( \p_x u_\ell \right)^2
\right],
\\
& u_\ell(x,0) = \omega \lambda^{-1} \tilde{\phi} \left(
\frac{x}{\lambda^{\delta}}
\right), \quad x \in \rr, \quad t \in \rr
\end{align*}
%
%
%
%
where $\tilde{\phi}$ is a $C^{\infty}_0(\rr)$ function such that
%
%
%
%
\begin{equation*}
\label{apple1***}
\tilde{\phi}(x) = 1 \; \;  \text{if} \; \;
x \in \text{supp} \; \phi.
\end{equation*}
\end{frame}
%
%
\section{Error of Approximate Solutions on the Line}
\begin{frame}
	\frametitle{Error of Approximate Solutions on the Line}
Substituting the
approximate solution $u^{\omega, \lambda} = u_\ell + u^h$ into the HR
equation, we see that the error
$E$ of our approximate solution is given by
%
%
\begin{equation*}
E=E_1 + E_2 + \dots + E_8
\end{equation*}
%
%
\pause
where
%
%
\begin{equation*}
\label{all_errors_together}
\begin{split}
E_1 & = \gamma \lambda^{1 -\frac{\delta}{2}-s}  \left[ u_\ell(x,0) - u_\ell(x,t)
\right] \phi\left(
\frac{x}{\lambda^ \delta}
\right)\sin(\lambda x - \gamma \omega t),
\\
E_2 & = \gamma \lambda^{-\frac{3\delta}{2}-s}
u_\ell(x,t) \cdot \phi'\left( \frac{x}{\lambda^\delta} \right)\cos\left( \lambda
x - \gamma \omega t
\right),
\\
E_3 & = \gamma u^h \p_x u_\ell, \; \; E_4 = \gamma u^h \p_x u^h, \ E_5  = 
 \frac{3-\gamma}{2} \Lambda^{-1} \left[  \left( u^h \right)^2 \right], \\
E_6 & = (3- \gamma)\Lambda^{-1}
  \left[ u_\ell u^h \right], \  E_7 = \frac{\gamma}{2} \Lambda^{-1} \left[ 
 \left(
\p_x u^h \right)^2 \right ], \; \;
\\
E_8 & = \gamma \Lambda^{-1} \left[  \p_x u_\ell \p_x u^h \right]
.
\end{split}
\end{equation*}
%
%
\end{frame}
%
%
%
\begin{frame}
%
%
\begin{proposition}
Let $1<\delta<2$. Then for $s > 1$, bounded $\omega$, and
$\lambda >>1$ we are assured the decay of the error $E$ of the
approximate solutions to the HR equation. Specifically
%
%
%
\begin{equation*}
\label{E-est}
\|E(t)\|_{H^1(\rr)} \lesssim \lambda^{\frac{\delta}{2} -s}, \quad |t| \le 
T.
\end{equation*}
%
%
%
\end{proposition}
%
%
\end{frame}
%
%
\section{Construction of Solutions on the Line}
\begin{frame}
	\frametitle{Construction of Solutions on the Line}
We wish now to estimate the difference between approximate and actual 
solutions to
the HR ivp with common initial data. Let
$u_{\omega,\lambda}(x,t)$ be the unique solution to the HR equation
with initial data $u^{\omega,\lambda}(x,0)$. That is,
$u_{\omega,\lambda}$ solves the initial value problem
\begin{gather*}
 \p_t u_{\omega,\lambda} = - \gamma u_{\omega,\lambda} \p_x 
u_{\omega,\lambda} - \Lambda^{-1} \left[
\frac{3- \gamma}{2}\left( u_{\omega,\lambda} \right)^2 + 
\frac{\gamma}{2}\left(
\p_x u_{\omega,\lambda} \right)^2
\right], 
\\
 u_{\omega,\lambda}(x, 0) = u^{\omega,\lambda}(x,0) = \omega \lambda^{-1}
\tilde{\phi} \left( \frac{x}{\lambda^\delta} \right)
+ \lambda^{-\frac{\delta}{2} -s}
\phi\left( \frac{x}{\lambda^\delta} \right) \cos(\lambda x).
\end{gather*}
%
%
%
We will now prove that the $H^1(\rr)$ norm of the difference decays: 
%
\end{frame}
%%%%%%%%%%%%%%%%%%%%%%%%%%%%%%%%%%
%
%
%
%
%
%    : H^1 bound_for_difference-of-approx-and-actual-soln
%
%
%
%
%
%
%%%%%%%%%%%%%%%%%%%%%%%%%%%%%%%%%
%
\begin{frame}
%
%
\begin{proposition}
\label{applelem:bound_for_difference-of-approx-and-actual-soln}
%
Let $v = u^{\omega,\lambda} - u_{\omega,\lambda}$, with $\lambda >>1$.
Then, for $s > 1$ and $1<\delta<2$ we have
%
%
\begin{equation*} \|
v(t)
\|_{H^1(\rr)}
\lesssim \lambda^{\frac{\delta}{2} -s}, \quad
|t| \le T.
\end{equation*}
%
%
\end{proposition}
%
%
\pause
{\bf   Proof.}
First we observe that $v$ satisfies 
%
%
\begin{equation*}
\begin{split}
\p_t v & = E + \gamma(v \p_x v - v \p_x u^{\omega,\lambda} - 
u^{\omega,\lambda} \p_x v) \\
& + \Lambda^{-1}  \left[ \frac{3-
\gamma}{2}v^2 + \frac{\gamma}{2}\left( \p_x v \right)^2 - \left(
3 - \gamma \right)u^{\omega,\lambda} v -
\gamma \p_x u^{\omega,\lambda} \p_x v \right].
\end{split}
\end{equation*}
%%
\pause
It follows immediately that
		\begin{equation*}
			\label{applev-dtv-pseudo-functional-equality*}
			\begin{split}
			v(1-\p_x^2)\p_t v &= v(1- \p_x^2)E + v\gamma(1- \p_x^2)(v\p_x v 
			- v\p_x u^{\omega,\lambda} -
			u^{\omega,\lambda} \p_x v)
			\\
			&+ v\p_x \left[ \frac{3-\gamma}{2}v^2 + \frac{\gamma}{2}(\p_x v)^2 -
			(3-\gamma)u^{\omega,\lambda} v - \gamma \p_x u^{\omega,\lambda} \p_x v \right].
		\end{split}
	\end{equation*}
\end{frame}
%
\begin{frame}
	\pause
	Applying the relation $v\p_t v = v(1-\p_x^2) \p_t v + v\p_x^2 \p_t v$ we obtain
	\begin{equation*}
		\label{pre-int}
		\begin{split}
		v \p_t v &= v(1- \p_x^2)E + v\gamma(1- \p_x^2)(v\p_x v - v\p_x u^{\omega,\lambda} -
			u^{\omega,\lambda} \p_x v)
			\\
			&+ v\p_x \left[ \frac{3-\gamma}{2}v^2 + \frac{\gamma}{2}(\p_x v)^2 -
			(3-\gamma)u^{\omega,\lambda} v - \gamma \p_x u^{\omega,\lambda} \p_x v
			\right] 
		\\
		& + v\p_x^2 \p_t v.
		\end{split}
	\end{equation*}
	\pause
	Adding $\p_x v \p_t \p_x v$ to both sides and 
	integrating then gives
	\begin{equation*}
		\label{appleenergy-est*}
		\begin{split}
			&\frac{1}{2} \frac{d}{dt} \|v\|_{H^1(\rr)}^2  
			\\
		& =  \int_{\rr} \left[ v(1-\p_x^2)E \right]dx
		\\
		& - \gamma \int_{\rr} \left[ v(1-\p_x^2)(v\p_x u^{\omega,\lambda} + u^{\omega,\lambda} \p_x v) \right]dx
		\\
		&- \int_{\rr}\left[ \left( 3-\gamma \right)v \p_x\left( u^{\omega,\lambda}v \right) + \gamma v
		\p_x \left( \p_x u^{\omega,\lambda} \p_x v \right)\right]dx
		\\
		&+  \int_{\rr}
		\left[ \gamma v \left( 1-\p_x^2 \right)\left( v \p_x v \right) + v
		\p_x \left( \frac{3-\gamma}{2} v^2 + \frac{\gamma}{2}\left( \p_x v \right)^2
		\right) \right . +  v \p_x^2 \p_t v 
		\\
		& + \p_x v \p_t \p_x v\bigg]dx.
	\end{split}
\end{equation*}
\end{frame}
%
\begin{frame}
Noting that the the last integral can be rewritten as 
\begin{equation*}
	\begin{split}
	\int_{\rr} \left[ \p_x (v^3) - \gamma \p_x (v^2 \p_x^2 v) + \p_x\left( v \p_t
	\p_x v
	\right) \right]dx  = 0
\end{split}
\end{equation*}
%
we can simplify to obtain
%
%
\begin{equation*}
\label{appleenergy-est}
\begin{split}
\frac{1}{2} \frac{d}{dt} \|v\|_{H^1(\rr)}^2  
& = 
 \int_{\rr} \left[ v(1-\p_x^2)E \right]dx\\
 &-
 \gamma \int_{\rr} \left[ v(1-\p_x^2)(v\p_x u^{\omega,\lambda} + 
u^{\omega,\lambda} \p_x v) \right]dx
\\
&- \int_{\rr}\left[ \left( 3-\gamma \right)v \p_x\left( u^{\omega,\lambda}v 
\right) + \gamma v
\p_x \left( \p_x u^{\omega,\lambda} \p_x v \right)\right]dx.
\end{split}
\end{equation*}
%
%
\pause
We now estimate the three integrals on the right-hand side. Integrating by parts and applying Cauchy-Schwartz,  
we obtain
%
%
%
\begin{equation*}
	\begin{split}
\label{applefirst_piece}
& \left |\int_{\rr} \left [v (1- \p_x^2)E \right ] dx \right |
 \lesssim
\|v\|_{H^1(\rr)} \|E\|_{H^1(\rr)}
\end{split}
\end{equation*}
\end{frame}
%
%
\begin{frame}
for the first integral,
%
%
\pause
\begin{equation*}
	\begin{split}
\label{applesecond-piece-final}
& \left | -\gamma \int_{\rr}
\left[ v\left( 1-\p_x^2 \right)\left( v \p_x u^{\omega,\lambda} + 
u^{\omega,\lambda} \p_x v
\right) \right] dx \right |
\\
& \lesssim \left( \|u^{\omega,\lambda}\|_{L^\infty(\rr)}\| + \|\p_x 
u^{\omega,\lambda}
\|_{L^\infty(\rr)} \right .  + \|\p_x^2 u^{\omega,\lambda} 
\|_{L^\infty(\rr)}
\big )\|v\|_{H^1(\rr)}^2
\end{split}
\end{equation*}
for the second integral, and
%
%
\pause
%
%
\begin{equation*}
	\begin{split}
\label{applelast_piece_final}
& \left | -\int_{\rr} \left[ \left( 3-\gamma \right)v
\p_x \left( u^{\omega,\lambda} v \right) + \gamma
v \p_x \left( \p_x u^{\omega,\lambda} \p_x v \right)\right]dx \right |
\\
& \lesssim \big(
\|u^{\omega,\lambda}\|_{L^\infty(\rr)}
+ \|\p_x u^{\omega,\lambda} \|_{L^\infty(\rr)} \big)
\|v\|_{H^1(\rr)}^2
\end{split}
\end{equation*}
%
%
%
for the third integral.
\pause
Combining these estimates, we 
obtain
%
%
\begin{equation*}
\begin{split}
\label{appleenergy-estimate-best}
\frac{d}{dt} \|v(t)\|_{H^1(\rr)}^2
& \lesssim \left( \|u^{\omega,\lambda}\|_{L^\infty(\rr)} + \|
\p_x u^{\omega,\lambda} \|_{L^\infty(\rr)} + \|\p_x^2 u^{\omega,\lambda} 
\|_{L^\infty (\rr)} \right)
\\
& \times \|v\|_{H^1(\rr)}^2 + \|v\|_{H^1(\rr)} \|E\|_{H^1(\rr)}.
\end{split}
\end{equation*}
%
%
Assume $\lambda >>1$. A straightforward calculation of derivatives yields
%
%
\begin{equation*}
\begin{split}
\|u^h\|_{L^\infty(\rr)} + \|\p_x u^h\|_{L^\infty(\rr)} + \|\p_x^2
u^h\|_{L^\infty(\rr)} \lesssim \lambda^{- \frac{\delta}{2} - s +2 }.
\label{apple53}
\end{split}
\end{equation*}
%
%
\end{frame}
%
%
\begin{frame}
Furthermore, by the Sobolev Imbedding Theorem and our well-posedness energy 
estimate, we have
%
%
%
%
\begin{equation*}
\begin{split}
\|u_\ell\|_{L^\infty(\rr)} + \|\p_x u_\ell \|_{L^\infty(\rr)} + \|\p_x^2
u_\ell\|_{L^\infty(\rr)}
& \le c_s \|u_\ell\|_{H^3(\rr)} 
 \lesssim \lambda^{-1 + \frac{\delta}{2}}. 
\label{apple55}
\end{split}
\end{equation*}
%
%
Hence
%
%
\begin{equation*}
\begin{split}
\|u^{\omega,\lambda}\|_{L^\infty(\rr)} + \|\p_x 
u^{\omega,\lambda}\|_{L^\infty(\rr)} + \|\p_x^2
u^{\omega,\lambda}\|_{L^\infty(\rr)}
& \lesssim \lambda^{-\rho_s}, \quad |t| \le T
\label{apple56}
\end{split}
\end{equation*}
%
%
where $\rho_s = \text{min} \Big\{ \frac{\delta}{2} + s -2, \; 1-
\frac{\delta}{2} \Big\}$.  Note that for $s>1$, we can assure $\rho_s > 0$
by choosing a suitable $1<\delta<2$.
Therefore, 
%
\begin{equation*}
\label{apple58}
\frac{d}{dt} \|v(t)\|_{H(\rr)}^2 \lesssim \lambda^{-\rho_s}
\|v\|_{H^1(\rr)}^2 + \lambda^{\frac{\delta}{2} -s}
\|v \|_{H^1(\rr)}, \quad |t| \le T
\end{equation*}
%
%
and applying Gronwall's Inequality completes the proof. \qquad \qedsymbol%
%
%
\end{frame}

\section{Non-Uniform Dependence for $s>1$ on the Line}
\begin{frame}
	\frametitle{Non-Uniform Dependence for $s>1$ on the Line}

Let $u_{\pm 1,\lambda}$ be solutions to the HR ivp with initial 
data $u^{\pm 1,
n}(0)$. We wish to show that the $H^s$ norm of the difference of $u_{\pm 1,
n}$ and the associated approximate solution $u^{\pm 1,\lambda}$
decays as $\lambda \to \infty$. Note that
\pause
%
%
\begin{equation*}
\begin{split}
\label{apple62}
 \|u^{\pm 1, \lambda}(t)\|_{H^{2s-1}(\rr)}
 & \le \|u_{\ell, \pm 1, \lambda}\|_{H^{2s-1}(\rr)}
\\
& +
\| \lambda^{-\frac{\delta}{2} -s} \phi \left(
\frac{x}{\lambda^\delta} \right) \cos(\lambda x \mp \gamma \omega t)
\|_{H^{2s-1}(\rr)}
\\
& \lesssim \lambda^{s-1}, \quad |t| \le T
\end{split}
\end{equation*}
%
%
and
%
\begin{equation*}
\begin{split}
\|u_{\pm 1,\lambda} (t) \|_{H^{2s-1}(\rr)}
& \le 2 \|u^{\pm 1,\lambda}(0) \|_{H^{2s-1}(\rr)}, \quad
|t| \le T.
\label{apple60}
\end{split}
\end{equation*}
%
%
%
%
\pause
Hence
%
\begin{equation*}
\begin{split}
\|u^{\pm 1, \lambda}(t) - u_{\pm 1, \lambda}(t) \|_{H^{2s-1}(\rr)}
\lesssim \lambda^{s-1}, \quad |t| \le T.
\label{apple63}
\end{split}
\end{equation*}
%
%
\end{frame}

\begin{frame}

Furthermore, by the proposition we just proved 
%
%
\begin{equation*}
\begin{split}
\|u^{\pm 1, \lambda}(t) - u_{\pm 1, \lambda} \|_{H^1(\rr)} \lesssim
\lambda^{\frac{\delta}{2} -s}, \quad |t| \le T.
\label{apple64}
\end{split}
\end{equation*}
%
%
%
\pause
%
%
%
%
%
Interpolating using 
the inequality
\begin{equation*}
\label{apple403}
\|\psi \|_{H^s (\rr)} \leq  (\| \psi \|_{H^1 (\rr)} \| \psi
\|_{H^{2s-1}(\rr)})^\frac12
\end{equation*}
%
%
gives
%
%
\begin{equation*}
\begin{split}
\|u^{\pm 1, \lambda}(t) - u_{\pm 1, \lambda}(t)
\|_{H^s(\rr)}
\lesssim \lambda^{\frac{\delta -2}{4}}, \quad |t| \le T.
\label{apple65}
\end{split}
\end{equation*}
%
%
Next, we will use this estimate to prove non-uniform
dependence when $s > 1$.
%%%%%%%%%%%%% Behavior at time  t = 0  %%%%%%%%%%%% 

%
\end{frame}

\begin{frame}
	\frametitle{Behavior at Time $t=0$}  We have
%
%
%
%
\begin{equation*}
\begin{split}
\|u_{1,\lambda}(0) - u_{-1,\lambda}(0) \|_{H^s(\rr)} & = \|u^{1,\lambda}(0) 
- u^{-1,\lambda}(0) \|_{H^s(\rr)}
\\
& = 2 \lambda^{-1} \| \tilde{\phi}\left( \frac{x}{\lambda^\delta} \right) 
\|_{H^s(\rr)}
\\
& = 2
\lambda^{\frac{\delta}{2}-1} \|\tilde{\phi} \|_{H^s(\rr)} \to 0
\; \; \text{as} \; \; \lambda \to \infty.
\end{split}
\end{equation*}


\end{frame}
%
%

\begin{frame}
	\frametitle{ Behavior at time  $t>0$}
%
%
%%%%%%%%%%%%%% Behavior at time  t >0  %%%%%%%%%%%% 
%  
%

Using the reverse triangle inequality, we 
have
%
%
%
%
%
\begin{equation*} \label{appleHR-slns-differ-t-pos}
\begin{split}
\|
u_{1,\lambda}(t)
-
u_{- 1,\lambda}(t)
\|_{H^s(\rr)}
&
\ge
\|
u^{1,\lambda}(t)
-
u^{- 1,\lambda}(t)
\|_{H^s(\rr)}
\\
& -
\|
u^{1,\lambda}(t)
-
u_{1,\lambda}(t)
\|_{H^s(\rr)}
\\
& -
\|
-u^{-1,\lambda}(t)
+
u_{-1,\lambda}(t)
\|_{H^s(\rr)} .
\end{split}
\end{equation*}
%
%
%
%
from which it follows that
%
%
%
%
\begin{equation*} \label{appleHR-slns-differ-t-pos-est}
\|
u_{1,\lambda}(t)
-
u_{- 1,\lambda}(t)
\|_{H^s(\rr)}
\ge
\|
u^{1,\lambda}(t)
-
u^{- 1,\lambda}(t)
\|_{H^s(\rr)}
-
c \lambda^{\frac{\delta - 2}{4}}
\end{equation*}
%
%
where c is a positive, non-zero constant. Letting $\lambda$ go to $\infty$ 
then yields
%
%
%
\begin{equation*} \label{appleHR-slns-to-ap-est}
\liminf_{n\to\infty}
\|
u_{1,\lambda}(t)
-
u_{- 1,\lambda}(t)
\|_{H^s(\rr)}
\ge
\liminf_{n\to\infty}
\|
u^{1,\lambda}(t)
-
u^{- 1,\lambda}(t)
\|_{H^s(\rr)}.
\end{equation*}
%
%
%
\end{frame}

\begin{frame}
%
Using the identity $$
\cos \alpha -\cos \beta
=
-2
\sin(\frac{\alpha + \beta}{2})
\sin(\frac{\alpha - \beta}{2})
$$
gives
%
%
\begin{equation*}
\label{apple80}
\begin{split}
u^{1,\lambda}(t)
-
u^{- 1,\lambda}(t)
& =
u_{\ell,1,\lambda}(t) - u_{\ell,-1,\lambda}(t)
\\
& + 
2\lambda^{-\frac{\delta}{2}-s}
\phi\left( \frac{x}{\lambda^\delta} \right)\sin(\lambda x) \sin(\gamma t).
\end{split}
\end{equation*}
%
%
Now, by our well-posedness energy estimate we have
%
%
\begin{equation*}
\begin{split}
\|u_{\ell,1,\lambda}(t) - u_{\ell,-1,\lambda}(t)\|_{H^s(\rr)} \lesssim
\lambda^{-1 + \frac{\delta}{2}}.
\end{split}
\end{equation*}
%
%
Hence, applying the reverse triangle inequality we 
obtain
%
%
\begin{equation*} 
\begin{split}
& \|
u^{1,\lambda}(t)
-
u^{- 1,\lambda}(t)
\|_{H^s(\rr)}
\\
& \ge 2 \lambda^{-\frac{\delta}{2}-s} \|\phi\left(
\frac{x}{\lambda^\delta} \right) \sin(\lambda x) \|_{H^s(\rr)} |\sin \gamma 
t|
- \|u_{\ell,-1,\lambda}(t) - u_{\ell,1,\lambda}(t)\|_{H^s(\rr)} \\
& \gtrsim \lambda^{-\frac{\delta}{2}-s} \|\phi\left(
\frac{x}{\lambda^\delta} \right ) \sin(\lambda x) \|_{H^s(\rr)} |\sin 
\gamma t| -
\lambda^{-1 + \frac{\delta}{2}}.
\end{split}
\end{equation*}
%
%
%
\end{frame}

\begin{frame}
%
Letting $\lambda$ go to $\infty$,  
gives
%
%
%
%
\begin{equation*} \label{apple91}
\liminf_{\lambda \to\infty}
\|
u^{1,\lambda}(t)
-
u^{- 1,\lambda}(t)
\|_{H^s(\rr)}
\gtrsim
|\sin \gamma t|, \quad |t| \le T.
\end{equation*}
%
%
This completes 
the proof of non-uniform dependence on the line. \qquad \qedsymbol
\end{frame}


\section{Approximate Solutions on the Circle}
\begin{frame}
	\frametitle{Approximate Solutions on the Circle}
%
%
%
In this case the  approximate solutions are of the form
%
%
\begin{equation*}
\label{approx-solutions-form}
u^{\omega,n}(x,t) = \omega n^{-1} + n^{-s} \cos \left( nx - \gamma \omega t
\right), 
\end{equation*}
where $n$ is a positive integer and $\omega$ is in a bounded subset of 
$\rr$. We remark that the approximate 
solutions are in $C^\infty(\ci)$ for all $t \in \rr$, and hence have 
infinite lifespan in $H^s(\ci)$ for $s  \ge 0$. 
\pause
Furthermore, for $n>>1$ we 
have 
%
%
\begin{equation*}
	\label{bound-approx}
	\begin{split}
		\|u^{\omega,n} \|_{H^s(\ci)} \approx 1	
	\end{split}
\end{equation*}
%
%
from the inequality
\begin{equation*}
\label{1m}
\begin{split}
	\|\cos(k(nx-c))\|_{H^s(\ci)} \simeq n^s, \quad k \in \rr \setminus
	\{0\}.
\end{split}
\end{equation*}


\end{frame}

\section{Error of Approximate Solutions on the Circle}
\begin{frame}
	\frametitle{Error of Approximate Solutions on the Circle}
Substituting the approximate solutions into the HR equation, we obtain the error
%
%
\begin{equation*}
\begin{split}
E=
E_1 + E_2 + E_3 \label{57}
\end{split}
\end{equation*}
%
%
where
\begin{align*}
& E_1 =
- \frac{\gamma}{2}n^{-2s+1}\sin\left[ 2\left( nx - \gamma \omega t \right)
\right],
\\
& E_2 = - \Lambda^{-1} \bigg[ \frac{3-\gamma}{2} \bigg (
n^{-2s+1} \sin\left( 2(nx - \gamma \omega t \right)
\\
& + 2\omega n^{-s} \sin( 
2(nx - \gamma \omega t))
\bigg )
\bigg ],
\\
& E_3 = \frac{\gamma}{4}
n^{-2s+2} \left [ 1- \cos \left (\frac{nx - \gamma \omega t}{2} \right) 
\right ].
\label{90}
\end{align*}

\end{frame}
%
%
%
%
%%%%%%%%%%%%%%%%%%%%%%%%%%%%%%%%%
%
%
%
%                      
%
%
%
%
%%%%%%%%%%%%%%%%%%%%%%%%%%%%%%%%

\begin{frame}

\begin{lemma}
\label{lem:error_of_approx_solution}
Let $u^{\omega,n}$ be an approximate solution to the HR ivp, with 
$\sigma \le 1$,  $\omega$ bounded, and $n >> 1$.
Then for the error $E$ we have
%
%
\begin{equation*}
\label{total-error-approx-solution}
\begin{split}
	\|E(t)\|_{H^\sigma(\ci)} \lesssim n^{-r_s} \ \ \text{where} \ \ r_s = 
\begin{cases}
2(s-1)   & \text{if} \quad s \le 3,\\  s+1  & \text{if} \quad s > 3. \\
\end{cases}
\end{split}
\end{equation*}
%
%
%
%
\end{lemma}

\end{frame}

\section{Construction of Solutions on the Circle}
\begin{frame}
	\frametitle{Construction of Solutions on the Circle}
We are now prepared to prove a decaying estimate for the difference of 
approximate and actual solutions:
%
%
\begin{proposition}
\label{prop:bound_for_difference-of-approx-actual-soln}
Let $v=u^{\omega,n} - u_{\omega,n}$, $n >>1$,
where $u_{\omega,n}$ denotes a solution to the HR
Cauchy-problem with
initial data $u_0(x) = u^{\omega,n}(x,0)$.
If \ $s > 3/2 $ and $\sigma = 1/2 + \ee$ for a sufficiently
small $\ee = \ee(s) > 0$, then 
%
%
\begin{equation*} \label{differ-Hsigma-est} \|
v(t)
\|_{H^\sigma(\ci)}
\lesssim n^{-r_s}, \quad |t| \le T.
\end{equation*}
%
%
\end{proposition}
%
%
\pause
{\bf Proof.} The difference $v = u^{\omega,n} - u_{\omega,n}$ satisfies 
the i.v.p
\begin{align*}
\label{1.7}
& \p_t v  =  E - \frac{\gamma}{2} \p_x
\left[ \left( u^{\omega,n} + u_{\omega,n} \right)v \right]
\\
\notag & \phantom{\p_t v} - \Lambda^{-1} \left[
\frac{3-\gamma}{2} \left( u^{\omega,n} + u_{\omega,n}
\right) v +
\frac{\gamma}{2}\left( \p_x u^{\omega,n} +
\p_x u_{\omega,n}
\right) \p_x v
\right], \\
& v(x,0)=0.
\end{align*}


\end{frame}

\begin{frame}
	For any $\sigma \in \rr$ let   $D^\sigma=(1-\p_x^2)^{\sigma/2}$ be the  operator
defined by 
%
$$ \widehat{D^\sigma f}(\xi) \doteq (1 + \xi^2)^{\sigma/2} \widehat{f}(\xi), $$
%
where $ \widehat{f}$ is the Fourier transform
%
$$ \widehat{f}(\xi) =  \frac{1}{2\pi}\int_{\ci} e^{-i \xi x} f(x) \ dx.  $$
%
%
Applying $D^\sigma$ to both sides of the Cauchy problem above, multiplying by
$D^\sigma v$, and integrating, we obtain the
relation
%
%
\begin{equation*}
\begin{split}
\frac{1}{2}\frac{d}{dt}\|v(t)\|_{H^\sigma(\ci)}^2
& = \int_{\ci} D^\sigma E \cdot D^\sigma
v \ dx
\\
&-
 \frac{\gamma}{2}\int_{\ci} D^\sigma
\p_x \left[ \left( u^{\omega,n} + u_{\omega,n} \right)v
\right]\cdot D^\sigma v \ dx
\\
& -
 \frac{3-\gamma}{2}\int_{\ci} D^{\sigma
-2} \p_x \left[ \left( u^{\omega,n} + u_{\omega,n}
\right)v \right] \cdot D^\sigma v \ dx
\\
& - \frac{\gamma}{2}\int_{\ci} D^{\sigma
-2}
\p_x \left[ \left( \p_x u^{\omega,n} + \p_x u_{\omega,n}
\right)\cdot \p_x v \right] \cdot
D^\sigma v \ dx.
\label{X}
\end{split}
\end{equation*}
%
%
We now estimate each integral of the right-hand side.
\end{frame}
%
%
\begin{frame}

{\bf Estimate of Integral 1.} Applying Cauchy-Schwartz, we obtain
%
%
\begin{equation*}
\begin{split}
\left |\int_{\ci} D^\sigma E \cdot D^\sigma v \ dx \right |
 \le \|E\|_{H^\sigma(\ci)} \|v\|_{H^\sigma(\ci)}.
\label{est_for_1}
\end{split}
\end{equation*}
%
%
%
\pause
%
%
{\bf Estimate of Integral 2.} We can rewrite
%
%
\begin{equation*}
\begin{split}
& -\frac{\gamma}{2} \int_{\ci} D^\sigma \p_x \left[ \left( u^{\omega,n} + 
u_{\omega,n}
\right)v \right] \cdot D^\sigma v \ dx
 \\
 & = -\frac{\gamma}{2}\int_{\ci} \left[ D^\sigma \p_x , u^{\omega,n} + 
u_{\omega,n}
\right]v \cdot D^\sigma v \ dx
\\
& - \frac{\gamma}{2} \int_{\ci} (u^{\omega,n} + u_{\omega,n})
D^\sigma \p_x v \cdot
D^\sigma v \ dx.
\label{est_for_2}
\end{split}
\end{equation*}
%
%
Integration 
by parts and Cauchy-Schwartz gives 
%
%
\begin{equation*}
\begin{split}
\left | \frac{\gamma}{2} \int_{\ci} (u^{\omega,n} + u_{\omega,n})
D^\sigma \p_x v \cdot
D^\sigma v \ dx \right |
& \lesssim \|\p_x(u^{\omega,n} + u_{\omega,n}) \|_{L^\infty(\ci)}
\|v\|_{H^\sigma(\ci)}^2.
\label{2'}
\end{split}
\end{equation*}
%
%
 We now need the following result
 taken from \cite{Taylor_2003_Commutator-esti}.
%
\end{frame}
%
%
\begin{frame}
	%
	%
\vskip0.1in
\begin{lemma}
\label{cor1}
If $\rho > 3/2$ and $0 \le \sigma + 1 \le \rho$, then
%
%
\begin{equation*}
\begin{split}
\|[D^\sigma \p_x ,f]v\|_{L^2} \le C \|f\|_{H^\rho} \|v\|_{H^\sigma}.
\label{15}
\end{split}
\end{equation*}
%
%
\end{lemma}
%
\pause
Let $\sigma = 1/2 + \ee$ and $\rho = 3/2 + \ee$, where 
$\ee > 0$ is
arbitrarily small. Applying Cauchy-Schwartz and the above commutator 
estimate, we obtain 
%
%
%
%
%
\begin{equation*}
\begin{split}
\left | -\frac{\gamma}{2} \int_{\ci} [D^\sigma \p_x , u^{\omega,n} + 
u_{\omega,n}]v
\cdot D^\sigma v \ dx \right | \lesssim \|u^{\omega,n} +
u_{\omega,n}\|_{H^{\rho}(\ci)} \|v\|_{H^\sigma(\ci)}^2.
\label{7}
\end{split}
\end{equation*}
%
%
\pause
Combining our estimates, we conclude that
%
%
\begin{equation*}
\begin{split}
& \left | -\frac{\gamma}{2} \int_{\ci} D^\sigma \p_x \left[ \left( 
u^{\omega,n} + u_{\omega,n}
\right)v \right]  \cdot D^\sigma v \ dx \right |
\\
& \lesssim (\|u^{\omega,n} + u_{\omega,n}\|_{H^{\rho}(\ci)} + \|\p_x 
u^{\omega,n} +
\p_x u_{\omega,n}\|_{L^\infty(\ci)} ) \cdot \|v\|_{H^\sigma(\ci)}^2.
\label{8}
\end{split}
\end{equation*}
%
%
%
\end{frame}
%
%
\begin{frame}
	\frametitle{}
{\bf Estimate of Integral 3.} Using Cauchy-Schwartz, and recalling that
$\sigma = 1/2 + \ee$,  we obtain
%
%
\begin{equation*}
\begin{split}
& \bigg | -\frac{3-\gamma}{2} \int_{\ci} D^{\sigma -2} \p_x \left[
(u^{\omega,n} + u_{\omega,n})v \right]
\cdot D^\sigma v \ dx \bigg |
\\
& \lesssim \|u^{\omega,n} + u_{\omega,n} \|_{L^\infty(\ci)} 
\|v\|_{H^\sigma(\ci)}^2.
\label{9}
\end{split}
\end{equation*}
%
%
%
\pause
{\bf Estimate of Integral 4.}
We will need the following result whose proof can be found in 
\cite{Himonas_2009_Non-uniform-dep-per}:
%
%
%
\begin{lemma}
\label{impo}
If  $1/2 < \sigma < 1 $ then
%
%
\begin{equation*}
\begin{split}
\|fg\|_{H^{\sigma - 1}} \le C \|f\|_{H^{\sigma}}
\|g\|_{H^{\sigma -1}}.
\label{11}
\end{split}
\end{equation*}
%
%
\end{lemma}
%
\pause
Applying Cauchy-Schwartz the above lemma, we obtain
%
%
\begin{equation*}
\begin{split}
& \left | -\frac{\gamma}{2} \int_{\ci} D^{\sigma -2 } \p_x \left[
\left( \p_x u^{\omega,n} + \p_x u_{\omega,n} \right) \cdot \p_x v
\right] \cdot D^\sigma v \ dx \right |
\\
& \lesssim \|\p_x u^{\omega,n} + \p_x u_{\omega,n}
\|_{H^\sigma(\ci)} \|v\|_{H^\sigma(\ci)}^2.
\label{12}
\end{split}
\end{equation*}
%
%
\end{frame}
%
%
\begin{frame}
	%
	%
Collecting our estimates for integrals 1-4 and
applying the Sobolev Imbedding Theorem, we deduce
%
%
\begin{equation*}
\begin{split}
\frac{1}{2}\frac{d}{dt} \|v\|_{H^\sigma(\ci)}^2
& \lesssim
\|u^{\omega,n} + u_{\omega,n}\|_{H^{\rho}(\ci)} \|v\|_{H^\sigma(\ci)}^2
+ \|E\|_{H^\sigma(\ci)}
\|v\|_{H^\sigma(\ci)}.
\label{10}
\end{split}
\end{equation*}
%
%
\pause
It follows from the well-posedness energy estimate
that the solutions $u_{\omega,n}$ have a common 
lifespan $T$. Hence, applying it in conjunction with
the triangle inequality yields
%
%
\begin{equation*}
	\|u^{\omega,n} + u_{\omega,n}\|_{H^\rho(\ci)} \lesssim n^{\rho -s}, 
	\quad |t| \le T.
\label{3r}
\end{equation*}
%
%
%
%
%
%
Substituting, we obtain
%
%
\begin{equation*}
\begin{split}
\frac{1}{2}\frac{d}{dt}\|v\|_{H^\sigma(\ci)}^2 \lesssim n^{\rho - s}
\|v\|_{H^\sigma(\ci)}^2 + n^{-r_s}\|v\|_{H^\sigma(\ci)}, \quad |t| \le T.
\label{200r}
\end{split}
\end{equation*}
%
%
and applying Gronwall's inequality concludes the proof. \qquad \qedsymbol

\end{frame}

\section{Non-Uniform Dependence for $s > 3/2$ on the Circle}

\begin{frame}
	\frametitle{Non-Uniform Dependence for $s > 3/2$ on the Circle}
%
%
%
Let $u_{\pm 1, n}$ be solutions to the HR ivp with common initial data 
$u^{\pm 1,
n}(0)$, respectively.
We wish to show that the $H^s$ norm of the difference of $u_{\pm 1,
n}$ and the associated approximate solution $u^{\pm 1, n}$ decays.
We assume
$s > 3/2 $ and $\sigma = 1/2 + \ee$ \ for a sufficiently small
$\ee= \ee(s) > 0$. 
Then by the lemma we just proved it follows that 
%
%
\begin{equation*}
\begin{split}
\|u^{\pm 1, n}(t) - u_{\pm 1, n} (t) \|_{H^\sigma (\ci)} \lesssim n^{-r_s}
\label{6h}, \quad |t| \le T.
\end{split}
\end{equation*}
%
%
\pause
Furthermore, by our well-posedness estimate we obtain
%
%
\begin{equation*}
\begin{split}
\|u^{\pm 1, n} (t) \|_{H^{2s - \sigma} (\ci)}
& \lesssim n^{s-\sigma}
\label{4}
\end{split}
\end{equation*}
%
%
and 
\begin{equation*}
	\begin{split}
\|u_{\pm 1, n} (t) \|_{H^{2s - \sigma}(\ci)}
& \lesssim n^{s- \sigma}, \quad |t| \le T.
\label{final-est-Hk-norm-sol}
\end{split}
\end{equation*}
%
%
%
%
%
\end{frame}
%
%
\begin{frame}
Therefore
%
\begin{equation*}
\begin{split}
\|u^{\pm 1, n} (t) - u_{\pm 1, n}(t)\|_{H^{2s - \sigma}(\ci)}
\lesssim n^{s-\sigma}, \quad |t| \le T.
\label{5h}
\end{split}
\end{equation*}
%
%
\pause
%
%
Interpolating using the
inequality
%
\begin{equation*}
\|\psi \|_{H^s (\ci)} \leq  (\| \psi \|_{H^\sigma (\ci)} \| \psi
\|_{H^{2s-\sigma}(\ci)})^\frac12
\end{equation*}
%
%
we obtain
%
%
\begin{equation*}
\begin{split}
\|u^{\pm 1,n}(t) - u_{\pm 1, n}(t) \|_{H^s (\ci)} \lesssim
n^{-\ee(s)/2}, \quad |t| \le T.
\label{10h}
\end{split}
\end{equation*}
%
%
The remainder of the proof of non-uniform dependence on the circle is
analogous to that on the real line. \qquad \qedsymbol

\end{frame}

\section{Proof of Well-Posedness for HR}
\begin{frame}[allowframebreaks]
	\frametitle{Proof of Well-Posedness for HR}

%
%
%
%
%
We will now prove well-posedness for HR.
Since the proof for the circle and the line are similar,
we will provide it only for circle.
For the line, we present only the
needed modifications.
%
%
%
%
%

We will prove existence by using an abstract ODE theorem (see Dieudonn\'e 
\cite{Dieudonne_1969_Foundations-of-}) in $H^s$. 
Unfortunately, the 
right-hand side of the HR ivp is not a map from $H^s$ to $H^s$.
For this we will consider the following mollification of the HR equation
%
%
\begin{align*}
& \p_t  u_\ee =
-\gamma J_\ee(J_\ee u_\ee \partial_x  J_\ee  u_\ee) - \Lambda^{-1} \left 
[\frac{3-\gamma}{2}(u_\ee)^2 + \frac{\gamma}{2}(\p_x u_\ee)^2 \right ],
\\
& u_\ee(x, 0) = u_0 (x),
\label{hr-moli-data}
\end{align*}
%
% 
%
%
%
%
%
%
where $J_\ee$ is defined  by
%
\begin{equation*}
\begin{split}
J_\ee f(x) = j_\ee * f(x), \quad \ee>0
\end{split}
\end{equation*}
%
\end{frame}

\begin{frame}
with 
%
\begin{equation*}
\begin{split}
j_\ee(x) = \frac{1}{\ee}j\left( \frac{x}{\ee} \right)
\end{split}
\end{equation*}
%
for non-negative $j(x) \in
\mathcal{S}(\rr)$. Notice that  $f_\ee$ given by 
%
%
\begin{equation*}
\label{f_ep}
f_{\ee}(u) = - \gamma J_\ee(J_\varepsilon u \partial_x J_\varepsilon u)
- \Lambda^{-1} \left
[\frac{3-\gamma}{2}u^2 + \frac{\gamma}{2}(\p_x u)^2 \right ]
\end{equation*}
%
is a map from $H^s(\ci)$  into $H^s(\ci)$. 
Therefore, the mollified HR equation 
is an ODE in $H^s$.
%
%
Furthermore, $f_\ee$ has a continuous total derivative $D f_\ee (u): 
H^s(\ci) \to H^s(\ci)$ given by
\begin{equation*}
	\label{total-deriv}
	\begin{split}
		[Df_{\ee}(u)](w)
		=
		& -\gamma J_\ee (J_\varepsilon w \partial_x J_\varepsilon u +
		J_\varepsilon u \partial_x J_\varepsilon w)
		\\
		& - \Lambda^{-1} \left [(3-\gamma)w u + \gamma\p_x w \p_x u \right ].
	\end{split}
\end{equation*}
Hence, by the Cauchy Existence Theorem (see \cite{Dieudonne_1969_Foundations-of-}), for each 
$\ee > 0$ there exists a
unique solution $u_\ee \in C(I, H^s(\ci))$ satisfying the mollified HR 
Cauchy-problem.
Next, we analyze the size and
lifespan of the family $\{u_\ee\}$ of solutions.

\end{frame}
%%%%%%%%%%%%%%%%%%%%%%%%
%
%     Estimates  for Lifespan and Sobolev norm of $u_\ee$
%
%%%%%%%%%%%%%%%%%%%%%%%%
%
%

\begin{frame}
\frametitle{Estimates for the Lifespan and Sobolev norm of $u_\ee$.}
%
We will show that there is a lower bound  $T$
for $T_\ee$ which is independent of $\ee\in(0, 1]$.
This is based on the differential
inequality 
%
%
%
\begin{equation*} \label{B-diff-ineq}
\frac 12
\frac{d}{dt}
\|u_\ee(t)\|_{H^{s}(\ci)}^2
\le
c_s
\|u_\ee(t)\|_{H^{s}(\ci)}^3,
\quad
|t| \le T_\ee
\end{equation*}
%
%
%
%
which we now prove by
following the approach used for quasilinear symmetric
hyperbolic systems in Taylor  \cite{Taylor_1991_Pseudodifferent}. In what 
follows we will suppress the
$t$ parameter for the sake of clarity.
%
%
%
Applying $D^s$ to both sides of the mollified HR equation,
multiplying the resulting equation by $D^s u_\ee$,
integrating it for $x\in\ci$, and noting that 
$D^s$ and $J_\ee$ commute
and that  $J_\ee$ satisfies 
%
%
\begin{equation*} 
\label{J-e-inner-prod-property}
(J_\ee f, g)_{L^2}=( f, J_\ee g)_{L^2}
\end{equation*}
%
%
we obtain

\end{frame}

\begin{frame}
%
%
%
\begin{equation*} \begin{split}
\label{B-moli-int}
 \frac 12
\frac{d}{dt} \|u_\ee \|_{H^s(\ci)}^2
=
& -
\gamma \int_{\ci}  D^s(J_\ee u_\ee \partial_x J_\ee u_\ee) \cdot
D^s J_\ee u_\ee  \  dx
\\
&- \frac{3 -\gamma}{2} \int_{\ci} D^{s-2} \p_x (u_{\ee})^2 \cdot D^s J_\ee 
u_{\ee} \ dx
\\
& - \frac{\gamma}{2} \int_{\ci}  D^{s-2} \p_x (\p_x u_\ee)^2
\cdot D^s J_\ee u_\ee  \ dx.
\end{split}
\end{equation*}
%
%
%
We will estimate the right-hand side in parts.  
Letting $v=J_\ee u_\ee$ we can rewrite the first integral on the right-hand 
side as 
%
%
%
\begin{equation*} \begin{split}
\label{B-moli-int-v}
& -  \gamma \int_{\ci}   D^s (J_{\ee} u_{\ee} \p_x J_\ee u_\ee)
\cdot D^s
J_{\ee}u_\ee \ dx
\\
& = - \gamma \int_\ci
\left [ D^s(v\p_x v)  -  v D^s (\p_xv)
\right ] \cdot D^s v \ dx
- \gamma \int_\ci
v D^s (\p_xv) 
\cdot D^s v \ dx.
\end{split}
\end{equation*}
%
%
%
%
%
We now estimate this in parts. Applying the Cauchy-Schwarz 
inequality gives
%
%
%
\end{frame}

\begin{frame}

\begin{equation*} \label{int1-est-calc2}
\begin{split}
& \Big|
- \gamma \int_\ci
\big[ D^s(v\p_x v)  -  v D^s (\p_xv)
\big]
\cdot D^s v   \, dx
\Big|
\\
& \lesssim
\|
D^s(v\p_x v)  -  v D^s (\p_xv)
\|_{L^2(\ci)}
\|
v
\|_{H^s(\ci)}
\lesssim \| \p_x v \|_{L^\infty(\ci)} \| v \|_{H^s(\ci)}^2,
\end{split}
\end{equation*}
%
%
%
where the last step follows from 
%
%
%
\begin{equation*} \label{int1-est-calc3}
\| D^s(v\p_x v)  -  v D^s (\p_xv) \|_{L^2(\ci)}
\le
2 c_s    \| \p_x v \|_{L^\infty(\ci)} \| v \|_{H^s(\ci)},
\end{equation*}
%
%
which is a simple corollary of the following Kato-Ponce commutator 
estimate, whose proof can be found in \cite{Kato_1988_Commutator-esti}: 
%
%
\begin{lemma}[Kato-Ponce] \label{KP-lemma}
If  $s>0$ then there is $c_s>0$ such that 
%
%
%
\begin{equation*} \begin{split}
& \| D^{s} \big(fg) -  f D^s g\|_{L^2(\ci)}
\\
& \le
c_s \big(
\| D^{s}f \|_{L^2(\ci)}    \| g \|_{L^\infty(\ci)} 
+ \| \p_xf \|_{L^\infty(\ci)}    \| D^{s-1}g \|_{L^2(\ci)}   \big).
\end{split}
\end{equation*}
%
%
%
\end{lemma}
%
%
\end{frame}

\begin{frame}
	
	A proof by Kato and Ponce can be found in \cite{Kato_1988_Commutator-esti}.
We apply Cauchy-Schwartz
to estimate the remaining integral
%
%
%
%
%
\begin{equation*} \label{int1-est-calc5}
\begin{split}
\Big|
-\gamma \int_\ci
v D^s (\p_x v)
\cdot  D^s v \ dx
\Big|
\lesssim \| \p_x v \|_{L^\infty(\ci)} \| v \|_{H^s(\ci)}^2.
\end{split}
\end{equation*}
%
%
%
%
%
Recalling that $v = J_\ee u_\ee$, combining our previous inequalities for 
the integrals, and applying the Sobolev Imbedding Theorem and the 
estimate
%
%
\begin{equation*}
	\begin{split}
		\|J_\ee u_\ee \|_{H^s(\ci)} \le \|u_\ee\|_{H^s(\ci)}	\end{split}
\end{equation*}
%
%
we obtain
%
%
%
\begin{equation*} \label{burgers_est'}
\begin{split}
\Big|
-\gamma \int_\ci
D^s(J_\ee u_\ee \partial_x J_\ee u_\ee) \cdot   D^s J_\ee u_\ee \, dx  
\Big|
& \lesssim \| u_\ee \|_{H^s(\ci)}^3.
\end{split}
\end{equation*}
%
%
%
For the remaining integrals, 
Cauchy-Schwartz and the algebra property of Sobolev spaces give 
%
%
\end{frame}

\begin{frame}


\begin{equation*}
\label{hl1}
\begin{split}
\left | - \frac{3 -\gamma}{2} \int_\ci D^{s-2} \p_x u_\ee^2 \cdot
D^s J_\ee u_\ee \; dx \right |
\lesssim \| u_\ee \|^3_{H^s(\ci)}
\end{split}
\end{equation*}
%
%
%
%
%
%
and 
%
%
\begin{equation*}
\label{hl2}
\begin{split}
\left | - \frac{\gamma}{2} \int_\ci D^{s-2} (\p_x u_\ee)^2 \cdot
D^s J_\ee u_\ee \; dx \right |
& \lesssim \|u_\ee\|_{H^s(\ci)}^3.
\end{split}
\end{equation*}
%
%
%
Combining all these estimates, we obtain
the desired ordinary differential inequality. Solving  
it, we obtain the following:

%
%
\end{frame}
%
%
%
\begin{frame}
%
%
\begin{lemma}
\label{hr_wp}
Let  $u_0(x) \in  H^s(\ci)$, $s >3/2$. Then for any $\ee\in (0, 1]$
the i.v.p. for the mollified HR equation 
%
%
%
\begin{align*} 
& \partial_t  u_\ee =
-\gamma J_\ee (J_\ee u_\ee \partial_x  J_\ee  u_\ee) - \Lambda^{-1} \left
[\frac{3-\gamma}{2}(u_\ee)^2 + \frac{\gamma}{2}(\p_x u_\ee)^2
\right ], 
\\
&  u_\ee(x, 0) = u_0 (x)
\label{burgers-moli-data-2}
\end{align*}
%
% 
%
%
%
%
%
%
has a unique solution $u_\ee( t)\in C([-T, T], H^s(\ci))$.  In particular,
%
%
%
\begin{equation*} \label{life-est}
T
=
\frac{1}{ 2 c_s \|u_0\|_{H^s(\ci)}}
\end{equation*}
%
%
%
is a lower bound for the lifespan of $u_\ee( t)$ and
%
%
%
\begin{equation*}
\label{u-e-Hs-bound}
\|u_\ee(t)\|_{H^s(\ci)}
\le
2 \|u_0 \|_{H^s(\ci)},
\quad
|t| \le T.
\end{equation*}
%
%
%
Furthermore,  $u_\ee( t)\in C^1([T, T], H^{s-1}(\ci))$ and satisfies
%
%
\begin{equation*}
\label{dt-u-e-Hs-bound}
\|\p_t u_\ee(t)\|_{H^{s-1}(\ci)}
\lesssim
\|u_0 \|_{H^s(\ci)}^2,
\quad
|t| \le T.
\end{equation*}
%
%
% 
\end{lemma}
%
%
\end{frame}
%%%%%%%%%%%%%%%%%%%%%%%%
%
%     Choosing  a convergent subsequence
%
%%%%%%%%%%%%%%%%%%%%%%%%
\begin{frame}
	\frametitle{Choosing  a convergent subsequence.}
%
Next we shall show that the family $\{ u_\ee\}$ has a convergent 
subsequence
whose limit $u$ solves the HR i.v.p.  Let I= [-T, T]. By the previous lemma and the compactness of $I$ we have a uniformly bounded 
family
%
%
%
\begin{equation*}
\label{Lip-1-fam}
\{u_\ee\}\subset C(I, H^s(\ci))\cap C^1(I,
H^{s-1}(\ci)).
\end{equation*}
%
%
%
%
By the Riesz Lemma, we can identify $H^s(\rr)$ with
$(H^s(\rr))^*$, where for $w, \psi \in H^s(\rr)$ the duality is
defined by 
\begin{equation*}
	T_w(\psi) = <w, \psi>_{H^s(\rr)} = \int_{\rr}
\widehat{w}(\xi, t) \overline{\widehat{\psi}}(\xi, t) \cdot (1
+ \xi^2)^s \ d \xi.
\end{equation*}
Applying the Riesz Representation Theorem, it follows that we 
can identify \\ $L^\infty(I, H^s(\ci)) $ with the dual space of $L^1(I,
H^{s}(\ci))$, where for $v\in L^\infty(I, H^s(\ci)) $ and $ \phi \in
L^1(I, H^{s}(\ci))$ the duality is defined by  
%
%
%
\begin{equation*}
T_v(\phi) = \int_I <v (t), \phi (t)>_{H^s(\rr)} dt  = \int_I
\int_{\rr}
\widehat{v}(\xi, t) \overline{\widehat{\phi}}(\xi, t) \cdot (1
+ \xi^2)^s \ d \xi dt.
\end{equation*}
%
%
%
\end{frame}

\begin{frame}
	By Aloaglu's Theorem (see Folland \cite{Folland_1999_Real-analysis}), the bounded family $\{u_\ee\}$ is compact in the weak$^*$ topology of 
$L^\infty(I, H^s(\ci))$. More precisely,
there is a subsequence  $\{ u_{\ee_k} \}$ converging
weakly to a $ u\in L^{\infty}(I, H^s(\ci))$.
That is 
%
%
%
\begin{equation*}
\label{weak-conv}
\lim_{n\to \infty} T_{u_{\ee_k}}(\phi)  =  T_u (\phi) \; \;		\text{ for 
all } \;\;  \phi \in L^1(I, H^{s}(\ci)).
\end{equation*}
%
%
%
In order to show that $u$ solves the HR i.v.p. we need to obtain a 
subsequence of $\{u_{\ee}\}$ with a stronger convergence, so that we can take 
the limit in the mollified HR equation. First we will need the following 
interpolation result:

\end{frame}
%%%%%%%%%%%%%%%%%%%%%%%%%%%
%
%
%                 Interpolation 
%
%
%%%%%%%%%%%%%%%%%%%%%%%%%%%
\begin{frame}

\begin{lemma}[Interpolation]
\label{interpolation-lem}
Let  $s > \frac{3}{2}$.
If $v \in C(I, H^s(\ci)) \cap C^1(I, H^{s-1}(\ci))$
then $v \in C^\sigma (I, H^{s- \sigma}(\ci))$ for  $0 < \sigma < 1$.
\end{lemma}
%
Applying this gives 
%
%
%
\begin{equation*}
\label{equic-1}
\sup_{t \neq t'} \frac { \|u_\ee(t) - u_\ee(t') \|_{H^{s -
\sigma}(\ci)}}{|t - t'|^\sigma} < c 
\end{equation*}
%
%
%
or
%
%
%
\begin{equation*}
\label{equic-2}
\|u_\ee(t) - u_\ee(t') \|_{H^{s - \sigma}(\ci)}< c|t - t'|^\sigma, \text{ 
for all }  \,\,  t, t'\in I,
\end{equation*}
%
%
%
%
which shows that the family  $\{u_\ee\}$ is equicontinuous in $C(I, 
H^{s-\sigma}(\ci))$. Furthermore, since the inclusion $H^s(\ci) \subset H^{s-
\sigma }(\ci)$ is compact and 
$\{u_\ee(t)\}$ is a uniformly bounded family, it follows that
$\{u_\ee(t)\}$ is precompact in 
$H^{s-\sigma}(\ci)$. Hence, we can apply Ascoli's Theorem  
\cite{Dieudonne_1969_Foundations-of-} to conclude that there exists a 
subsequence $\left\{ u_{\ee_n} \right\}$
such that
%
%
\begin{equation*}
\label{strong-conv-of-u_ep}
u_{\ee_n} \to u \; \; \text{in} \; \; C(I, H^{s-\sigma}(\ci)).
\end{equation*}


\end{frame}
%
%
%
%
%
%%%%%%%%%%%%%%%%%%%%%%%%%%%%%%%%%
%
%
%     Verifying that the limit $u$ solves Burgers equation
%
%
%%%%%%%%%%%%%%%%%%%%%%%%%%%%%%%%%
\begin{frame}
	\frametitle{Verifying that $u$ solves the HR equation} 
Using the previous result and the Sobolev Imbedding Theorem, we see that
%
%
%
%
\begin{equation*}
\begin{split}
& -\gamma J_\ee (J_{\ee_n} u_{\ee_n}  J_{\ee_n} \p_x
u_{\ee_n}) - \Lambda^{-1} \left( \frac{3-\gamma}{2}
(u_{\ee_n})^2
+ \frac{\gamma}{2} (\p_x u_{\ee_n})^2 \right )
\\
\to & -\gamma u \p_x u -
\Lambda^{-1} \left( \frac{3-\gamma}{2} u^2
+ \frac{\gamma}{2} (\p_x u)^2 \right ) \ \
\text{in} \ \ C(I, C(\ci)).
\label{loc-non-loc-tog}
\end{split}
\end{equation*}
%
%
Furthermore, the strong convergence of our family $\left\{ u_{\ee_n} 
\right\}$  gives 
%
%
%
\begin{equation*}
\label{weak-conv-2}
T_{u_{\ee_n}}(\phi)  \longrightarrow  T_u(\phi) \;
\text{ for all } \;  \phi \in L^1(I, H^{s}(\ci))
\end{equation*}
%
%
%
which implies
%
%
%
%
\begin{equation*}
\label{distib-conv-2}
T_{\p_t u_{\ee_n}}(\phi)  \longrightarrow  T_{\p_t u}(\phi) \;
\text{ for all } \;  \phi \in L^1(I, H^{s}(\ci)).
\end{equation*}
%
%
%
%
It follows from the uniqueness of the limit 
that
%
%
%
\begin{equation*}
\label{1000y}
\partial_t u =- \gamma u \partial_x u- \Lambda^{-1} \left
[\frac{3-\gamma}{2}u^2 + \frac{\gamma}{2}(\p_x u)^2 \right ].
\end{equation*}
%
%
%
Thus we have constructed a solution $u \in L^\infty(I, H^s(\ci))$
to the HR i.v.p. It remains to prove that $u \in C(I, H^s(\ci)).$

\end{frame}
%%%%%%%%%%%%%%%%%%%%%%%%%%
%
%
%Proof that  $u \in C(I, H^s(\ci)) \bigcap C^1(I, H^{s-1}(\ci))$.
%
%
%
%%%%%%%%%%%%%%%%%%%%%%%%%%
\begin{frame}
	\frametitle{\bf Proof that $u \in C(I, H^s(\ci))$}
Since $u \in L^\infty(I, 
H^s(\ci))$, it is a
continuous function from $I$ to $H^s(\ci)$ with respect to the weak
topology on $H^s(\ci)$. That is, for $\{t_n\} \subset I$ such that $t_n \to t$, we
have
%
%
\begin{equation*}
\begin{split}
<u(t_n), \ v>_{H^s(\ci)} \ \longrightarrow \
<u(t), \ v>_{H^s(\ci)}, \quad \forall
v \in H^s(\ci).
\label{1ff}
\end{split}
\end{equation*}
%
%
Next, note that
%
%
\begin{equation*}
\begin{split}
 \|u(t) - u(t_n) \|_{H^s(\ci)}^2
& = <u(t) - u(t_n), \ u(t) -
u(t_n)>_{H^s(\ci)}
\\
& = \|u(t)\|_{H^s(\ci)}^2 + \|u(t_n)\|_{H^s(\ci)}^2
 - <u(t_n), \
u(t) >_{H^s(\ci)} 
\\
& - <u(t), u(t_n)>_{H^s(\ci)}.
\label{2ff}
\end{split}
\end{equation*}
%
%
%
%
Hence
%
%
\begin{equation*}
\begin{split}
\lim_{n \to \infty} \|u(t) - u(t_n)\|_{H^s(\ci)}^2 = \left[ \lim_{n
\to \infty} \|u(t_n)\|_{H^s(\ci)}^2
\right] - \|u(t)\|_{H^s(\ci)}^2.
\label{3ff}
\end{split}
\end{equation*}
%
%
Therefore, to prove that $u \in C(I, H^s(\ci))$, it will be
enough to show that the map $t \mapsto \|u(t)\|_{H^s(\ci)}$ is a continuous
function of $t$. This will follow from the energy
estimate

\end{frame}

\begin{frame}
%
%
\begin{equation*}
\label{en-est-u}
\frac{1}{2} \frac{d}{dt} \|u(t)\|_{H^s(\ci)}^2
\le c_s \|u(t)\|_{H^s(\ci)}^3, \quad |t| \le T
\end{equation*}
%
%
which we now derive. Applying $D^s$ to both sides of the HR ivp,
multiplying the
resulting equation by $D^s u$, and integrating for $x\in \ci$, we obtain
%
%
\begin{equation*}
\begin{split}
\label{bound-int}
 \frac 12
 \frac{d}{dt} \|u \|_{H^s(\ci)}^2
 =
& -
\gamma \int_{\ci}   D^s (u \p_x u) \cdot
D^s u \  dx
\\
& - \frac{3 -\gamma}{2} \int_{\ci}  D^{s-2} \p_x (u^2) \cdot D^s u \ dx
\\
& - \frac{\gamma}{2} \int_{\ci}   D^{s-2} \p_x (\p_x u)^2
\cdot D^s u \ dx.
\end{split}
\end{equation*}
%
%
Using estimates analogous to those in our previous energy-estimate 
computations, we 
obtain the desired differential inequality.
Derivating the left hand side and simplifying gives
%
%

\end{frame}

\begin{frame}
\begin{equation*}
\label{en-est-u-simplified}
\frac{d}{dt} \|u(t)\|_{H^s(\ci)} \le c_s \|u(t)\|_{H^s(\ci)}^2, \quad |t| 
\le T.
\end{equation*}
%
Solving this ordinary differential inequality yields an upper bound
%
%
\begin{equation*}
\label{uniform_bound_for_u}
\|u(t)\|_{H^s(\ci)}
\le
2 \|u_0\|_{H^s(\ci)},
\quad |t| \le T
\end{equation*}
%
%
%


for the size of the solution.  Since $\|u(t)\|_{H^s(\ci)}$
is uniformly bounded for $|t| \le T$, we conclude from
that the map $t \mapsto
\|u(t)\|_{H^s(\ci)}$ is Lipschitz continuous in $t$, for $|t| \le T$. \qquad \qedsymbol
%
%
%
\end{frame}

\begin{frame}
	\frametitle{Uniqueness}
%
%
Let $u,\omega \in C(I, H^s(\ci)), \ s>3/2,$ be two solutions to the HR
Cauchy-problem with
common initial data. Set $v=u-w$. Then $v$ solves the Cauchy-problem
%
%
\begin{align*}
& \p_t v
=  -\frac{\gamma}{2} \p_x [v(u + w)] 
\\
\notag
& \phantom{\p_t v = }- D^{-2} \p_x \left\{
\frac{3-\gamma}{2}[v(u+w)] + \frac{\gamma}{2}[\p_x v \cdot \p_x (u+w)]
\right\},
\\
& v(x,0) = 0.
\label{uniqueness-init-data}
\end{align*}
%
%
%
%
Applying $D^\sigma$ to both sides of this new Cauchy-problem, then 
multiplying both sides by $D^\sigma v$ and integrating, we obtain
%
%
\begin{equation*}
\begin{split}
 \frac{1}{2} \frac{d}{dt} \|v\|_{H^\sigma(\ci)}^2
 = & -\frac{\gamma}{2} \int_{\ci} D^\sigma \p_x [v(u+w)] \cdot
D^\sigma v \ dx
\\
& - \frac{3-\gamma}{2} \int_{\ci}  D^{\sigma -2}
\p_x[v(u+w)] \cdot
D^\sigma v \ dx  
\\
& - \frac{\gamma}{2} \int_{\ci} D^{\sigma 
-2} \p_x [ \p_x v
\cdot \p_x (u+w)]\cdot D^\sigma v \ dx .
\label{2v}
\end{split}
\end{equation*}
%
%
We now estimate in parts.

\end{frame}

\begin{frame}

{\bf Estimate of Integral 1.} Note that
%
%
\begin{equation*}
\begin{split}
& \left |  -\frac{\gamma}{2} \int_{\ci} D^\sigma \p_x [v(u+w)] \cdot
D^\sigma v \ dx \right |
\\
& =
\left |
-\frac{\gamma}{2} \int_{\ci} \left[ D^\sigma \p_x, \ u+w \right]v \cdot
D^\sigma v \ dx - \frac{\gamma}{2} \int_{\ci} (u+w) D^\sigma
\p_x v \cdot D^\sigma v\ dx
\right | \\
& \le \left |
-\frac{\gamma}{2} \int_{\ci} \left[ D^\sigma \p_x, \ u+w \right]v \cdot
D^\sigma v \ dx \right |
+ \left | \frac{\gamma}{2} \int_{\ci} (u+w) D^\sigma \p_x v
\cdot D^\sigma v\
dx \right |.
\label{4v}
\end{split}
\end{equation*}
%
%
Observe that by integrating by parts gives
%
%
\begin{equation*}
\begin{split}
\left | \frac{\gamma}{2}\int_{\ci} (u+w) D^\sigma \p_x v \cdot
D^\sigma v \ dx \right |
\lesssim \|\p_x (u+w)\|_{L^\infty(\ci)}
\|v\|_{H^\sigma(\ci)}^2.
\label{4'v}
\end{split}
\end{equation*}
%
%
%
%
To estimate the remaining integral, we first 
choose $\ 3/2 < \rho
< s$ and $ 1/2 < \sigma \le \rho -1$. An application of 
Cauchy-Schwartz and the Taylor commutator estimate then yields 
%
%
\begin{equation*}
\begin{split}
 \left | -\frac{\gamma}{2} \int_{\ci} [D^\sigma \p_x, \ u+w] v
\cdot D^\sigma v \ dx \right |
& \lesssim \|u+w\|_{H^\rho(\ci)} 
\|v\|_{H^\sigma(\ci)}^2.
\label{7v}
\end{split}
\end{equation*}
%
%
\end{frame}

\begin{frame}


Combining our results and applying the Sobolev Imbedding 
Theorem, we obtain the estimate
%
%
\begin{equation*}
\begin{split}
\left |  -\frac{\gamma}{2} \int_{\ci} D^\sigma \p_x [v(u+w)] \cdot
D^\sigma v \ dx \right |
 \lesssim \|u+w\|_{H^\rho(\ci)} \|v\|_{H^\sigma(\ci)}^2.
\label{8v}
\end{split}
\end{equation*}
%
%

{\bf Estimate of Integral 2.} Applying Cauchy-Schwartz, the algebra property of Sobolev spaces, and the 
Sobolev Imbedding Theorem, we obtain 
%
%
%
%
\begin{equation*}
\begin{split}
\left | - \frac{3-\gamma}{2} \int_{\ci}  D^{\sigma -2}
\p_x[v(u+w)] \cdot
D^\sigma v \ dx  \right |
 \lesssim \|u+w\|_{H^{\sigma -1}(\ci)} \|v\|_{H^\sigma(\ci)}^2.
\label{3v}
\end{split}
\end{equation*}
%
%
%

{\bf Estimate of Integral 3.} We first apply
Cauchy-Schwartz and the Sobolev Imbedding Theorem to obtain
%
%
\begin{equation*}
\begin{split}
\left | - \frac{\gamma}{2} \int_{\ci} D^{\sigma 
-2} \p_x [ \p_x v
\cdot \p_x (u+w)]\cdot D^\sigma v \ dx \right | 
 \lesssim 
\|[\p_x v \cdot \p_x (u+w)] \|_{H^{\sigma -1}(\ci)}
\|v\|_{H^\sigma(\ci)}.
\end{split}
\end{equation*}
%
%
Restrict $1/2 < \sigma < 1$. Then applying the Himonas-Kenig fractional 
derivative estimate and the Sobolev 
Imbedding Theorem, we conclude that
%
%
\begin{equation*}
\begin{split}
\left | - \frac{3-\gamma}{2} \int_{\ci}  D^{\sigma -2}
\p_x[v(u+w)] \cdot
D^\sigma v \ dx  \right |
 \lesssim \|u+w \|_{H^{\sigma - 1}(\ci)}
\|v\|_{H^\sigma(\ci)}^2.
\label{3'v}
\end{split}
\end{equation*}
%
%
%
\end{frame}
%

\begin{frame}

Grouping our estimates, and
applying
the Sobolev Imbedding Theorem, we obtain
%
%
\begin{equation*}
\begin{split}
\frac{1}{2} \frac{d}{dt}
\|v\|_{H^\sigma(\ci)}^2 \lesssim \|u+w\|_{H^\rho(\ci)}
\|v\|_{H^\sigma(\ci)}^2.
\label{9v}
\end{split}
\end{equation*}
%
%
%
%
%
%
Since $v_0 = 0$ and $\|u + w \|_{H^\rho}
\le \|u + w \|_{H^s(\ci)} < \infty$ for $|t| \le T$, we deduce from 
an application of  Gronwall's 
inequality that $v = 0$. \qquad \qedsymbol
%
%
%
\end{frame}

%\begin{frame}
%	\frametitle{Continuous Dependence} 
%Let $\left\{ u_{0, n} \right\}_n \subset H^s(\ci)$ be a uniformly bounded
%sequence converging to $u_0$ in $H^s(\ci)$.
%Consider solutions $u $, $u^\ee$, $u^\ee_n$, and $u_n$ to the 
%Cauchy-problem
%\eqref{hyperelastic-rod-equation*}-\eqref{init-cond}
%with associated initial data $u_0$, $J_\ee u_0$,
%$J_\ee u_{0,n}$, and $u_{0,n}$, respectively, where $J_\ee$ is the operator 
%defined by
%\begin{equation*}
%\label{0'u}
%\begin{split}
%J_\ee f(x) = j_\ee * f(x), \quad \ee>0.
%\end{split}
%\end{equation*}
%%
%%
%Here
%\begin{equation*}
%\begin{split}
%j_\ee (x) = \sum_{\xi \in \zz}
%\widehat{j}(\ee \xi) e^{i \xi x}, \quad \ee > 0
%\label{parseval-def}
%\end{split}
%\end{equation*}
%where $\widehat{j}(\xi) \in \mathcal{S}(\rr)$ is chosen such that 
%%
%\begin{equation*}
%\label{0u}
%\begin{split}
%	 0 \le \widehat{j}(\xi) \le 1  \ \ \text{and} \ \
% \widehat{j}(\xi) = 1 \ \ \text{if} \ \ |\xi| \le 1.
%\end{split}
%\end{equation*}
%%
%%
%%
%%
%%
%%
%%
%%
%%
%%
%We remark that it follows immediately from \eqref{parseval-def} that
%\begin{equation*}
%\begin{split}
%	\widehat{j_\ee}(\xi)  = \widehat{j }(\ee \xi), \quad \ee > 0.
%\label{widehat-def}
%\end{split}
%\end{equation*}
%This will prove
%crucial later on.
%%
%Next, applying
%the triangle inequality, we obtain
%%
%%
%\begin{equation*}
%\begin{split}
%\|u - u_n\|_{H^s(\ci)}
%& \le \|u - u^\ee\|_{H^s(\ci)}
%+ \|u^\ee - u^{\ee}_n \|_{H^s(\ci) }
%+  \|u^{\ee}_n - u_n \|_{H^s(\ci)}.
%\end{split}
%\end{equation*}
%%
%%
%Let $\eta > 0$. To prove continuous dependence, it will be enough to show that 
%we can find $\ee > 0$ and $N \in \mathbb{N}$ such that for all $n > N$ 
%\begin{align*}
%	 \|u(t) - u^\ee(t)\|_{H^s(\ci)}
%	& < \eta/3, \quad |t| \le T,
%\\
%  \|u^\ee(t) - u^{\ee}_n(t)
%\|_{H^s(\ci)} & < \eta/3, \quad |t| \le T,
%\\
%  \|u^{\ee}_n(t) - u_n(t) \|_{H^s(\ci)} & < \eta/3, \quad |t| \le T.
%\end{align*}
%%
%%
%The proof of \eqref{enough_to_prove3} will be analogous to that of 
%\eqref{enough_to_prove1}, so we will omit the details.
%%
%%
%%
%%
%
%{\bf Proof of \eqref{enough_to_prove1}.}
%Consider two solutions $u $ and $u^\ee$ to the Cauchy-problem
%\eqref{hyperelastic-rod-equation*}-\eqref{init-cond}
%with associated initial data $u_0$ and
%$J_\ee u_0$, respectively. Set $v= u -u^\ee $. Then $v$ solves the
%Cauchy-problem
%\begin{align*}
%& \p_t v  =  - \gamma (v \p_x v + v \p_x u^\ee + u^\ee \p_x v)  \\
%& \phantom{ \p_t v  =} - D^{-2} \p_x \left\{ \left (\frac{3-\gamma}{2} \right )(v^2 +
%2u^\ee v) + \frac{\gamma}{2}\left[ (\p_x v)^2 + 2 \p_x u^\ee \p_x v \right]
%\right\}, \notag
%\\
%& v(0) = (I- J_\ee)u_0.
%\label{5u}
%\end{align*}
%Applying the operator $D^s$ to both sides of \eqref{4u}, then multiplying by
%$D^s v$ and integrating gives
%%
%%
%\begin{equation*}
%\begin{split}
%\frac{1}{2}\frac{d}{dt} \|v\|_{H^s(\ci)} = A + B
%\label{6u}
%\end{split}
%\end{equation*}
%%
%%
%where
%%
%%
%\begin{equation*}
%\begin{split}
%A
% =  & -\gamma \int_{\ci} D^s(v \p_x v) \cdot D^s v \
%dx
%\\
%& - \frac{3- \gamma}{2} \int_\ci D^{s-2} \p_x (v^2) \cdot D^s v
%\ dx
%\\
%& - \frac{\gamma}{2}\int_\ci D^{s-2} \p_x (\p_x v)^2 \cdot D^s
%v \ dx
%\label{7u}
%\end{split}
%\end{equation*}
%%
%%
%and
%%
%%
%\begin{equation*}
%\begin{split}
%B = & -\gamma \int_\ci D^s (v \p_x u^\ee ) \cdot D^s v \
%dx \\
%& -\gamma \int_\ci D^s (u^\ee \p_x v) \cdot D^s v \
%dx
%\\
%& - \ ( 3- \gamma) \int_\ci D^{s-2} \p_x (u^\ee v) \cdot D^s
%v \ dx
%\\
%& -\gamma \int_\ci D^{s-2} \p_x
%(\p_x u^\ee \cdot \p_x v) \cdot D^s v \
%dx.
%\label{8u}
%\end{split}
%\end{equation*}
%%
%%
%Using estimates analogous to those in 
%\eqref{B-moli-int-v}-\eqref{hl2}, we 
%obtain 
%%
%%
%\begin{equation*}
%\begin{split}
%|A| \lesssim \|v\|_{H^s(\ci)}^3, \quad |t| \le T.
%\label{8'u}
%\end{split}
%\end{equation*}
%%
%%
%%
%Next we estimate $B$ in parts:
%%
%%
%%
%%
%%
%
%{\bf Estimate of Integral 1.} We can rewrite
%%
%%
%\begin{equation*}
%\begin{split}
%-\gamma \int_\ci D^s (v \p_x u^\ee ) \cdot D^s v \
%dx = & -\gamma \int_\ci \left[ D^s(v \p_x u^\ee) - v D^s
%\p_x u^\ee \right] \cdot D^s v \ dx
%\\
%& -  \gamma \int_\ci v D^s \p_x u^\ee \cdot D^s v \ dx.
%\label{1wap'}
%\end{split}
%\end{equation*}
%%
%%
%%
%%
%%
%%
%Applying Cauchy-Schwartz, the Kato-Ponce estimate \eqref{KP-com-est}, and the Sobolev 
%Imbedding Theorem, we obtain 
%%
%%
%%
%%
%\begin{equation*}
%\begin{split}
%| -\gamma \int_\ci \left[ D^s(v \p_x u^\ee ) - v D^s
%\p_x u^\ee \right] \cdot D^s v \ dx |
%\lesssim \|u^\ee \|_{H^s(\ci)} \|v\|_{H^s(\ci)}^2.
%\label{2wap}
%\end{split}
%\end{equation*}
%%
%%
%For the remaining integral of \eqref{1wap'}, Cauchy-Schwartz and the Sobolev 
%Imbedding Theorem give
%%
%%
%%
%\begin{equation*}
%\begin{split}
%| - \gamma \int_\ci u^\ee D^s \p_x v \cdot D^s v \ dx |
%\lesssim \|u^\ee \|_{H^{s+1}(\ci)} \|v\|_{H^{s-1}(\ci)}
%\|v\|_{H^s(\ci)}.
%\label{3wap}
%\end{split}
%\end{equation*}
%%
%%
%Combining estimates \eqref{2wap} and \eqref{3wap} we conclude that
%%
%%
%\begin{equation*}
%\begin{split}
%\left | -\gamma \int_\ci D^s (v \p_x u^\ee ) \cdot D^s v \
%dx \right | \lesssim \|u^\ee \|_{H^s(\ci)} \|v\|_{H^s(\ci)}^2 + \|u^\ee 
%\|_{H^{s+1}(\ci)} \|v\|_{H^{s-1}(\ci)}
%\|v\|_{H^s(\ci)}.
%\label{4wap}
%\end{split}
%\end{equation*}
%%
%%
%%
%{\bf Estimate of Integral 2}. We can rewrite
%%
%%
%\begin{equation*}
%\begin{split}
%-\gamma \int_\ci D^s (u^\ee \p_x v) \cdot D^s v \
%dx
%= & -\gamma \int_\ci \left[ D^s(u^\ee \p_x v) - u^\ee D^s
%\p_x v \right] \cdot D^s v \ dx
%\\
%& -  \gamma \int_\ci u^\ee D^s \p_x v \cdot D^s v \ dx.
%\label{1wa'}
%\end{split}
%\end{equation*}
%%
%%
%Applying Cauchy-Schwartz, the Kato-Ponce estimate \eqref{KP-com-est}, and 
%the Sobolev \\ Imbedding Theorem to 
%the first integral, we obtain
%%
%%
%%
%%
%\begin{equation*}
%\begin{split}
%| -\gamma \int_\ci \left[ D^s(u^\ee \p_x v) - u^\ee D^s
%\p_x v \right] \cdot D^s v \ dx |
%\lesssim \|u^\ee \|_{H^s(\ci)} \|v\|_{H^s(\ci)}^2.
%\label{2wa}
%\end{split}
%\end{equation*}
%%
%%
%For the remaining integral of \eqref{1wa'}, integration by parts, 
%Cauchy-Schwartz, and the Sobolev Imbedding Theorem give
%%
%%
%\begin{equation*}
%\begin{split}
%| - \gamma \int_\ci u^\ee D^s \p_x v \cdot D^s v \ dx |
%\lesssim \|u^\ee \|_{H^s(\ci)} \|v\|_{H^s(\ci)}^2.
%\label{3wa}
%\end{split}
%\end{equation*}
%%
%%
%Combining estimates \eqref{2wa} and \eqref{3wa} we conclude that
%%
%%
%\begin{equation*}
%\begin{split}
%\left | -\gamma \int_\ci D^s (u^\ee \p_x v) \cdot D^s v \
%dx \right |
% \lesssim \|u^\ee \|_{H^s(\ci)} \|v\|_{H^s(\ci)}^2.
%\label{4wa}
%\end{split}
%\end{equation*}
%%
%%
%{\bf Estimate of Integral 3.} Applying Cauchy-Schwartz, the 
%algebra property of Sobolev spaces, and the Sobolev Imbedding Theorem gives
%%
%%
%\begin{equation*}
%\begin{split}
%\left |- \ ( 3- \gamma) \int_\ci D^{s-2} \p_x (u^\ee v) \cdot D^s
%v \ dx \right |  \lesssim \|u^\ee\|_{H^{s}(\ci)} \|v\|_{H^{s}(\ci)}^2.
%\label{13u}
%\end{split}
%\end{equation*}
%%
%%
%%
%%
%%
%%
%{\bf Estimate of Integral 4.} Applying Cauchy-Schwartz, the 
%algebra property of Sobolev spaces, and the Sobolev Imbedding Theorem, we 
%obtain
%%
%%
%\begin{equation*}
%\begin{split}
%\left |-\gamma \int_\ci D^{s-2} \p_x
%(\p_x u^\ee \cdot \p_x v) \cdot D^s v \
%dx \right |
% \lesssim \|u^\ee\|_{H^s(\ci)} \|v\|_{H^s(\ci)}^2.
%\end{split}
%\end{equation*}
%%
%%
%Hence, collecting our estimates for integrals 1-4, we obtain 
%%
%%
%\begin{equation*}
%\begin{split}
%|B| & \lesssim
%\|u^\ee\|_{H^s(\ci)}
%\|v\|_{H^s(\ci)}^2 + \|u^\ee\|_{H^{s+1}(\ci)}
%\|v\|_{H^{s-1}(\ci)} \|v\|_{H^s(\ci)}.
%\label{14u}
%\end{split}
%\end{equation*}
%%
%%
%Combining estimates \eqref{8'u} and \eqref{14u} and recalling
%\eqref{6u}, we obtain
%%
%%
%\begin{equation*}
%\begin{split}
%\frac{1}{2}\frac{d}{dt}\|v\|_{H^{s}(\ci)}^2
%& \lesssim \|v\|_{H^s(\ci)}^3 + \|u^\ee\|_{H^s(\ci)}
%\|v\|_{H^s(\ci)}^2
% + \|u^\ee\|_{H^{s+1}(\ci)}
%\|v\|_{H^{s-1}(\ci)} \|v\|_{H^s(\ci)}
%\end{split}
%\end{equation*}
%%
%%
%which simplifies to 
%\begin{equation*}
%\begin{split}
%\frac{d}{dt}\|v\|_{H^{s}(\ci)}
%& \lesssim \|v\|_{H^s(\ci)}^2 + 
%\|v\|_{H^s(\ci)}
%+ \ee^{-1}
%\|v\|_{H^{s-1}(\ci)} 
%\label{15u}
%\end{split}
%\end{equation*}
%by differentiating the left-hand side and applying the following lemma:
%%
%%
%%
%\begin{lemma}
%\label{lem5r}
%For $r \ge s > 3/2$ and $0 < \ee <1$, 
%%
%%
%\begin{equation*}
%\begin{split}
%\|u^{\ee} (t) \|_{H^r(\ci)} \lesssim \ee^{s-r}.
%\label{700r}
%\end{split}
%\end{equation*}
%%
%%
%\end{lemma}
%%
%%
%{\bf Proof.} Recalling the construction of $J_\ee$ in 
%\eqref{0'u}-\eqref{widehat-def},  we have
%%
%%
%\begin{equation*}
%\label{schwartz}
%\begin{split}
%	|\widehat{J_\ee u_0}(\xi)| = |\widehat{j_\ee}(\xi) \widehat{u_0}(\xi)|
%	= |\widehat{j }(\ee \xi) \widehat{u_0}(\xi)| \le c_r |\ee \xi 
%	|^{s-r} \widehat{u_0}(\xi), \ \ r \ge s, \ \ \xi \neq 0.
%\end{split}
%\end{equation*}
%%
%%
%Applying \eqref{u_x-Linfty-Hs} and \eqref{schwartz}, the result follows.
%\qquad \qedsymbol
%%
%%
%%
%
%We now aim to prove decay for the $\ee^{-1}
%\|v\|_{H^{s-1}(\ci)} $ term in \eqref{15u}. To do so, we 
%will first obtain an estimate for
%$\|v\|_{H^\sigma(\ci)}$ for suitably chosen $\sigma < s-1$. Then, 
%interpolating between $\|v\|_{H^\sigma(\ci)}$
%and $\|v\|_{H^s(\ci)}$, we will show that 
%$\|v\|_{H^{s-1}(\ci)}$ experiences $o(\ee)$ decay. This will imply
%$o(1)$ decay of $\ee^{-1}
%\|v\|_{H^{s-1}(\ci)} $.
%%
%%
%\begin{proposition} \label{prop:6r}
%For $\sigma$ such that $1/2 < \sigma < 1$ and $\sigma + 1 \le s$, we have
%%
%%
%\begin{equation*}
%\begin{split}
%\|v\|_{H^{\sigma}(\ci)} = o(\ee^{s- \sigma }), \quad |t| \le T.
%\end{split}
%\end{equation*}
%%
%%
%\end{proposition}
%%
%%
%%
%{\bf Proof.}
%Recall that $v$ solves the Cauchy-problem \eqref{4u}-\eqref{5u}.
%Applying $D^\sigma$ to both sides of \eqref{4u}, then multiplying by
%$D^\sigma v$ and integrating, we obtain 
%%
%%
%\begin{equation*}
%\begin{split}
%\frac{1}{2}\frac{d}{dt}\|v(t)\|_{H^\sigma(\ci)}^2
%= & - \frac{\gamma}{2}\int_{\ci} D^\sigma
%\p_x \left[ \left( u + u^\ee \right)v
%\right]\cdot D^\sigma v \ dx
%\\
%& - \frac{3-\gamma}{2}\int_{\ci} D^{\sigma
%-2} \p_x \left[ \left( u + u^\ee
%\right)v \right] \cdot D^\sigma v \ dx
%\\
%& - \frac{\gamma}{2}\int_{\ci} D^{\sigma
%-2}
%\p_x \left[ \left( \p_x u + \p_x u^\ee
%\right)\cdot \p_x v \right] \cdot
%D^\sigma v \ dx.
%\end{split}
%\end{equation*}
%%
%%
%Repeating calculations \eqref{X}-\eqref{12}, with $E$ set to zero,
%$u^{\omega,n}$ replaced by $u$, $u_{\omega,n}$ replaced by $u^\ee$, and
%$\sigma$ and $\rho$ chosen such that
%%
%%
%%
%\begin{equation*}
%\label{size_of_sigma}
% 1/2 < \sigma < 1,
% \quad 
% \text{and}
% \quad
% \sigma + 1 \le \rho \le s 
%\end{equation*}
%%
%%
%yields
%%
%%
%\begin{equation*}
%\begin{split}
%\frac{1}{2}\frac{d}{dt} \|v\|_{H^\sigma(\ci)}^2
%& \lesssim
%(\|u^{\ee} + u\|_{H^{\rho}(\ci)} +
%\|\p_x(u^{\ee} + u) \|_{H^\sigma(\ci)})
%\cdot \|v\|_{H^\sigma(\ci)}^2.
%\end{split}
%\end{equation*}
%%
%%
%By the Sobolev Imbedding Theorem , it follows that 
%%
%%
%\begin{equation*}
%\begin{split}
%\frac{1}{2}\frac{d}{dt} \|v\|_{H^{\sigma}(\ci)}^2
%& \lesssim
%\|u^{\ee}
%+ u\|_{H^{s}(\ci)} \cdot \|v\|_{H^{\sigma}(\ci)}^2.
%\label{10x}
%\end{split}
%\end{equation*}
%%
%%
%Hence, applying the triangle inequality, \eqref{u_x-Linfty-Hs}, and the estimate
%%
%%
%\begin{equation*}
%\begin{split}
%	\|J_\ee f\|_{H^s(\ci)} \le \|f\|_{H^s(\ci)}
%\label{lem100u}
%\end{split}
%\end{equation*}
%%
%%
%%
%%
%to the right-hand side of \eqref{10x} yields
%%
%%
%%
%%
%%
%\begin{equation*}
%\begin{split}
%\label{12x}
%\frac{1}{2}\frac{d}{dt} \|v\|_{H^{\sigma}(\ci)}^2
%& \le
%C \|v\|_{H^{\sigma}(\ci)}^2
%\end{split}
%\end{equation*}
%%
%%
%where $C = C(\|u_0\|_{H^s(\ci)})$. Gronwall's inequality then gives
%%
%%
%%
%\begin{equation*}
%\label{conc-lemma}
%\begin{split}
%\|v\|_{H^{\sigma}(\ci)}
% \le e^{C t}\|v(0)\|_{H^{\sigma}(\ci)}
% = e^{C t}\|u_0 - J_\ee u_0 \|_{H^{\sigma}(\ci)} = o(\ee^{s-r})
%\end{split}
%\end{equation*}
%%
%%
%%
%%
%%
%where the last step follows from the operator norm estimate provided below. 
%\qquad \qedsymbol
%%
%%
%\begin{lemma}
%\label{lem4r}
%For $r \le s$ and $\ee>0$
%%
%%
%\begin{equation*}
%\label{0r}
%\begin{split}
%\|I - J_\ee\|_{L(H^s(\ci), H^r(\ci))} = o(\ee^{s-r}).
%\end{split}
%\end{equation*}
%%
%%
%\end{lemma}
%%
%%
%{\bf Proof.}
%Let $u \in H^s(\ci)$ and $r, s \in \rr$ such that $r \le s$. 
%Recalling the construction of $J_\ee$ in
%\eqref{0'u}-\eqref{widehat-def}, we have
%%
%%
%\begin{align*}
%& \|u - J_\ee u\|_{H^r(\ci)}^2 = \sum_{\xi \in \zz} | [1- \widehat{j}(\ee 
%\xi)] \cdot \widehat{u}(\xi) |^2
%(1+\xi^2)^r, \ \ \text{and}
%\\
%& |1 - \widehat{ j }(\ee \xi)| \le |\ee \xi |^{s-r}, \quad 
%\xi \in \rr, \ \ee > 0.
%\label{2r}
%\end{align*}
%%
%%
%Applying \eqref{2r} to \eqref{1r} we obtain
%%
%%
%\begin{equation*}
%\label{2pr}
%\begin{split}
%\|u - J_\ee u\|_{H^r(\ci)}
%\lesssim \ee^{(s-r)}
%\end{split}
%\end{equation*}
%%
%%
%%
%%
%while a dominated convergence argument gives
%%
%%
%\begin{equation*}
%\label{o1}
%\begin{split}
%\|u - J_\ee u \|_{H^s(\ci) } & = o(1).
%\end{split}
%\end{equation*}
%%
%%
%%
%%
%Applying the interpolation estimate 
%%
%%
%\begin{equation*}
%\begin{split}
%\|f\|_{H^{k_2}(\ci)} \le
%\|f\|_{H^{k_1}(\ci)}^{(s-k_2)/(s -k_1 )}
%\|f\|_{H^s(\ci)}^{1 - (s-k_2)/(s -k_1 )}, \quad k_1 < k_2 \le s
%\label{16u}
%\end{split}
%\end{equation*}
%%
%%
%%
%%
%%
%%
%completes the proof. \qquad \qedsymbol 
%%
%
%We now return to analyzing the $\ee^{-1}
%\|v\|_{H^{s-1}(\ci)} $ term of \eqref{15u}.
%Applying \eqref{16u} and \autoref{prop:6r}, 
%we obtain
%%
%%
%%
%%
%$$
%\|v\|_{H^{s-1}(\ci)}  \lesssim o(\ee) 
%\|v\|_{H^s(\ci)}^{1-
%1/(s- \sigma)}.
%$$
%
%%
%%
%Note
%that $\|v(t)\|_{H^s(\ci)}$ is uniformly bounded for all $\ee > 0$. More 
%precisely, by
%the triangle inequality,  \eqref{lem100u}, and \eqref{u_x-Linfty-Hs},
%we have
%%
%%
%\begin{equation*}
%	\label{bound-no-ep}
%\begin{split}
%\|v(t) \|_{H^s(\ci)}
%\le 4 \|u_0\|_{H^s(\ci)}, \quad |t| \le T.
%\end{split}
%\end{equation*}
%%
%%
%Hence
%\begin{equation*}
%\|v\|_{H^{s-1}(\ci)} = o(\ee)
%\end{equation*}
%%
%%
%which implies
%\begin{equation*}
%	\label{e-decay}
%	\ee^{-1} \|v\|_{H^{s-1}(\ci)} = o(1).
%\end{equation*}
%%
%%
%Substituting \eqref{e-decay} into \eqref{15u}, we obtain
%%
%%
%\begin{equation*}
%\begin{split}
%\frac{d}{dt} \|v\|_{H^s(\ci)} \lesssim
%\|v\|_{H^s(\ci)}^2 + \|v\|_{H^s(\ci)} + o(1).
%\label{202x}
%\end{split}
%\end{equation*}
%%
%%
%Letting $y(t) = \|v(t)\|_{H^s(\ci)}$, we can factor the right-hand side to 
%obtain
%%
%%
%\begin{equation*}
%	\label{y-express}
%	\begin{split}
%		\frac{dy}{dt} \lesssim (y-\alpha)(y-\beta)	
%	\end{split}
%\end{equation*}
%%
%%
%where
%%
%%
%\begin{equation*}
%	\begin{split}
%		\alpha = \frac{-1 + \sqrt{1-o(1)}}{2} \qquad \text{and} \qquad
%		\beta = \frac{-1 - \sqrt{1-o(1)}}{2}.
%	\end{split}
%\end{equation*}
%%
%%
%Rewriting \eqref{y-express} yields
%%
%%
%\begin{equation*}
%	\begin{split}
%		\left( \frac{1}{y-\alpha} - \frac{1}{y-\beta} 
%		\right) \frac{dy}{dt} \lesssim \sqrt{1- o(1)} \approx 1.
%	\end{split}
%\end{equation*}
%%
%%
%Noting that $1/(y - \alpha) - 1/(y - \beta)$ is positive, and integrating 
%from $0$ to $t$, we obtain 
%%
%%
%\begin{equation*}
%	\begin{split}
%		\ln \left (\frac{y(t) - \alpha}{y(t) - \beta} \cdot 
%		\frac{y(0) - \beta}{y(0) - \alpha} \right ) \le ct.	
%	\end{split}
%\end{equation*}
%%
%%
%Exponentiating both sides and rearranging gives
%%
%%
%\begin{equation*}
%	\begin{split}
%		\frac{y(t) - \alpha}{y(t) - \beta} \le e^{ct} \cdot
%		\frac{y(0) - \alpha}{y(0) - \beta}	
%	\end{split}
%\end{equation*}
%%
%%
%which implies
%%
%%
%\begin{equation*}
%	\begin{split}
%		y(t) 
%		& \le e^{ct} \cdot \frac{ \left [y(0) - \alpha \right ]
%		\left [y(t) - \beta \right ]}{y(0) - 
%		\beta} + \alpha
%		 \lesssim \left [y(0) - \alpha \right ] \left [y(t) - \beta \right ] + \alpha, \quad |t| \le T
%	\end{split}
%\end{equation*}
%%
%%
%where the last step follows from the fact that $1/2 \le -\beta \le 1$.  Substituting back in $\|v\|_{H^s(\ci)}$, we obtain
%\begin{equation*}
%	\label{key-decay-ineq}
%	\begin{split}
%		\|v\|_{H^s(\ci)}  \lesssim \left [\|v(0)\|_{H^s(\ci)} - 
%		\alpha \right ] \left [\|v\|_{H^s(\ci)} - \beta \right ] + \alpha.
%	\end{split}
%\end{equation*}
%Noting that $\|v\|_{H^s(\ci)}$ is uniformly bounded in $\ee$ by 
%\eqref{bound-no-ep}, $\alpha \to 0$, and
%%
%%
%\begin{equation*}
%\label{303''qx}
%\begin{split}
%\|v(0)\|_{H^s(\ci)} = \|u_0 - J_\ee u_0 \|_{H^s(\ci)} \to 0 \end{split}
%\end{equation*}
%by  \autoref{lem4r}, we conclude from \eqref{key-decay-ineq} that
%%
%%
%\begin{equation*}
%\label{304qx}
%\begin{split}
%\|v(t)\|_{H^s(\ci)} = 
%\|u(t) - u^\ee(t) \|_{H^s(\ci)}= o(1), \quad |t| \le T.
%\end{split}
%\end{equation*}
%%
%%
%Choosing $\ee$ sufficiently small gives $\|v(t)\|_{H^s(\ci)} < \eta/3$, 
%completing the proof of \eqref{enough_to_prove1}. \qquad \qedsymbol
%%
%%
%%
%%
%%
%
%{\bf Proof of \eqref{enough_to_prove2}.} Let $v = u^\ee_n - u^\ee$. Then 
%$v$ solves the Cauchy problem
%\begin{align*}
%& \p_t v  =  -\gamma (v \p_x v + v \p_x u^\ee + u^\ee \p_x v)  \\
%& \phantom{ \p_t v  =} - D^{-2} \p_x \left\{ \left (\frac{3-\gamma}{2} \right )(v^2 +
%2u^\ee v) + \frac{\gamma}{2}\left[ (\p_x v)^2 + 2 \p_x u^\ee \p_x v \right]
%\right\}, \notag
%\\
%& v(0) =J_\ee(u_{0,n} - u_0).
%\label{5qu}
%\end{align*}
%Applying the operator $D^s$ to both sides of \eqref{4qu}, multiplying by
%$D^s$ and integrating, and estimating as in \eqref{8'u}-\eqref{14u}, we 
%obtain
%%
%%
%\begin{equation*}
%\begin{split}
%\frac{1}{2}\frac{d}{dt}\|v\|_{H^{s}(\ci)}^2
%& \lesssim \|v\|_{H^s(\ci)}^3 + \|u^\ee\|_{H^s(\ci)}
%\|v\|_{H^s(\ci)}^2
% + \|u^\ee\|_{H^{s+1}(\ci)}
%\|v\|_{H^{s-1}(\ci)} \|v\|_{H^s(\ci)}
%\end{split}
%\end{equation*}
%%
%%
%which by differentiating the left-hand side and applying \autoref{lem5r} to 
%the right-hand side simplifies to
%\begin{equation*}
%\begin{split}
%\frac{d}{dt}\|v\|_{H^{s}(\ci)}
%& \lesssim \|v\|_{H^s(\ci)}^2 + \|v\|_{H^s(\ci)}
%+ \ee^{-1}
%\|v\|_{H^{s-1}(\ci)}.
%\label{15qu}
%\end{split}
%\end{equation*}
%%
%%
%We now aim to control of the growth the $\ee^{-1}
%\|v\|_{H^{s-1}(\ci)}$ term of \eqref{15qu}. To do so, we will need an estimate for
%$\|v\|_{H^{s-1}(\ci)}$, which we will obtain using interpolation. First, 
%we will need the following:
%%
%%
%%
%%
%\begin{proposition}
%\label{prop:left}
%For $\sigma$ such that $1/2 < \sigma < 1$ and $\sigma + 1 \le s$, 
%%
%%
%\begin{equation*}
%\label{prop:6rq}
%\begin{split}
%\|v\|_{H^{\sigma}(\ci)} = \|u^\ee_n - u^\ee\|_{H^\sigma(\ci)}
%\lesssim \|u_0 - u_{0,n} \|_{H^s(\ci)}, \quad |t| \le T.
%\end{split}
%\end{equation*}
%%
%%
%\end{proposition}
%%
%%
%%
%{\bf Proof.}
%Repeating calculations \eqref{X}-\eqref{12}, with $E$ set to zero, 
%$u^{\omega,n}$
%replaced by $u^\ee_n$, $u_{\omega,n}$ replaced by $u^\ee$, and $\sigma$ and 
%$\rho$ chosen such that
%%
%%
%\begin{equation*}
%\label{size_of_sigma'}
%\begin{split}
%	& 1/2 < \sigma < 1 \ \ \text{and} \ \  \sigma + 1 \le \rho \le s
%\end{split}
%\end{equation*}
%%
%%
%yields
%%
%%
%\begin{equation*}
%\begin{split}
%\frac{1}{2}\frac{d}{dt} \|v\|_{H^\sigma(\ci)}^2
%& \lesssim
%(\|u^{\ee}_n + u^\ee \|_{H^{\rho}(\ci)} +
%\|\p_x(u^{\ee}_n + u^\ee) \|_{H^\sigma(\ci)})
%\cdot \|v\|_{H^\sigma(\ci)}^2.
%\end{split}
%\end{equation*}
%%
%%
%Since $\{u_{0,n}\}$ belongs to a bounded subset of $H^s(\ci)$, it follows that %
%%
%\begin{equation*}
%\begin{split}
%\label{12qx}
%\frac{1}{2}\frac{d}{dt} \|v\|_{H^{\sigma}(\ci)}^2
%& \le
%C \|v\|_{H^{\sigma}(\ci)}^2.
%\end{split}
%\end{equation*}
%%
%%
%where $C = C(\|u_0\|_{H^s(\ci)}, \ \limsup_{n \to \infty} 
%\|u_{0,n}\|_{H^s(\ci)})$. 
%Applying Gronwall's inequality to \eqref{12qx}, we obtain
%%
%%
%\begin{equation*}
%\begin{split}
%\|v\|_{H^{\sigma}(\ci)}
%& \le e^{C t}\|v(0)\|_{H^{\sigma}(\ci)}
%= e^{C t}\|u^\ee(0) - u^\ee_n(0) \|_{H^{\sigma}(\ci)} \le e^{C t} \|u_0 - 
%u_{0,n}\|_{H^\sigma(\ci)}
%\end{split}
%\end{equation*}
%%
%%
%concluding the proof. \qquad \qedsymbol
%%
%%
%%
%
%We now return to analyzing the $\ee^{-1}
%\|v\|_{H^{s-1}(\ci)}$ term of \eqref{15qu}.
%Applying the 
%interpolation estimate \eqref{16u} and
%\autoref{prop:left} gives
%%
%%
%%
%%
%\begin{equation*}
%\begin{split}
%\label{200qx}
%\|v\|_{H^{s-1}(\ci)} 
%& \lesssim  
%\|u_0-u_{0,n}\|_{H^s(\ci)}^{1/(s-\sigma)}\|v\|_{H^s(\ci)}^{1- 
%1/(s-\sigma)}.
%\end{split}
%\end{equation*}
%%
%%
%%
%Note that the triangle inequality, \eqref{u_x-Linfty-Hs},
%and \eqref{lem100u} 
%imply that $\|v\|_{H^s(\ci)}$ is uniformly bounded in $n$ \emph{and} $\ee$. 
%That is
%%
%%
%\begin{equation*}
%\begin{split}
%	\|v\|_{H^s(\ci)} \le 2 \left[ \|u_0 \|_{H^s(\ci)} + \limsup_{n \to 
%	\infty} 
%	\|u_{0,n}\|_{H^s(\ci)} 
%\right], \quad |t| \le T.
%\label{growth_v}
%\end{split}
%\end{equation*}
%%
%%
%Hence, \eqref{200qx} gives
%%
%%
%\begin{equation*}
%\begin{split}
%\label{200qxr}
%\|v\|_{H^{s-1}(\ci)} 
%& \lesssim  
%\|u_0-u_{0,n}\|_{H^s(\ci)}^{1/(s-\sigma)}.
%\end{split}
%\end{equation*}
%
%Fix $\ee, \rho > 0$. Since $\|u_0 -
%u_{0,n} \|_{H^s(\ci)} \to 0$, we
%can find $N \in \mathbb{N}$ such that for all $n > N$
%%
%%
%\begin{equation*}
%\begin{split}
%\ee^{-1} \|u_0-u_{0,n}\|_{H^s(\ci)}^{1/(s-\sigma)}
%& < \rho
%\label{uniform_n}
%\end{split}
%\end{equation*}
%%
%%
%which by \eqref{200qxr} implies
%%
%%
%\begin{equation*}
%	\label{end-decay}
%	\begin{split}
%		\ee^{-1} \|v\|_{H^{s-1}(\ci)} \lesssim \rho.
%	\end{split}
%\end{equation*}
%%
%%
%%
%%
%Since $\rho$ can be chosen to be arbitrarily small, the remainder of the 
%proof is analogous to that of \eqref{enough_to_prove1}. \qquad \qedsymbol
%%
%%
%%
%%
\begin{frame}
{\bf Extending Well-Posedness to the Non-Periodic Case.}
In the proof of existence on the line, we will have difficulties in 
arranging
that the solutions $\{u_\ee\}$ to the mollified HR i.v.p. converge in $C(I,
H^{s- \sigma}(\rr))$, $0 < \sigma < 1$ to a candidate solution $u$ of the 
HR i.v.p., since the inclusion $H^s(\rr) \subset H^{s-\sigma}(\rr)$ is not 
compact for $\sigma > 0$ (contrast this with the situation on the circle).  
However, by Rellich's Theorem, the map $f \mapsto \vp f$
is a compact operator from $H^s(\rr)$ to  $H^{s-\sigma}(\rr)$ for any $\vp 
\in \mathcal{S}(\rr)$.
Hence, considering the family $\left\{ \varphi
u_\ee \right\}$ instead, it can be shown that 
for arbitrary $k \in \mathbb{N}$
%
%
\begin{equation*}
\begin{split}
& \vp \Lambda^{-1} [(u_{\ee_n})^k] \to \vp \Lambda^{-1} [u^k] \ \ \text{in}  \ \ C(I,
H^{s-\sigma }(\rr)),
\\
& \vp \Lambda^{-1} [(\p_x u_{\ee_n})^k] \to \vp \Lambda^{-1} [(\p_x u)^k] \ \
\text{in}  \ \ C(I,
H^{s-\sigma -1}(\rr)).
\label{hhdx_vp_u_ep_conv}
\end{split}
\end{equation*}
%
%
%
%
To utilize this result, we multiply both sides of the mollified HR equation by 
$\varphi$ and rewrite to obtain the Cauchy-problem
%
%
\begin{align*}
 & \p_t(\vp u_{\ee_n} )  = -\gamma \vp J_\ee (J_{\ee_n} u_{\ee_n} 
J_{\ee_n} \p_x u_{\ee_n}) - \vp \Lambda^{-1} \left( \frac{3-\gamma}{2} 
(u_{\ee_n})^2
+ \frac{\gamma}{2} (\p_x u_{\ee_n})^2 \right ),
\\
& u_{\ee_n}(x, 0) \phantom{\,,} = \phantom{\,} u_0 (x).
\label{real-line-moli-data-2} 
\end{align*}
%
%
\end{frame}

\begin{frame}

Choosing $\vp \in \mathcal{S}(\rr)$ such that
$\vp^\frac{1}{2} \in \mathcal{S}(\rr)$ then gives 
%
%
%
\begin{equation*}
\begin{split}
	& -\gamma \vp J_\ee (J_{\ee_n} u_{\ee_n} 
J_{\ee_n} \p_x u_{\ee_n}) - \vp \Lambda^{-1} \left( \frac{3-\gamma}{2}
(u_{\ee_n})^2
+ \frac{\gamma}{2} (\p_x u_{\ee_n})^2 \right )
\\
& \to -\gamma \vp u \p_x u - \vp \Lambda^{-1} \left( \frac{3-\gamma}{2} u^2
+ \frac{\gamma}{2} (\p_x u)^2 \right ) \ \
\text{in} \ \ C(I, C(\rr)).
\label{llnon-local-convergence}
\end{split}
\end{equation*}
%
%
%
Restricting $\vp$ to be non-zero, and using an argument analogous to that 
in the periodic case, it follows that 
$u$ is a solution to the HR ivp. Proofs of 
$u \in C(I, H^s(\rr))$ and uniqueness are analogous to the proofs in 
the periodic case.
%
%
\end{frame}

\begin{thebibliography}{HKM09}

		\providecommand{\bysame}{\leavevmode\hbox to3em{\hrulefill}\thinspace}
\providecommand{\MR}{\relax\ifhmode\unskip\space\fi MR }
% \MRhref is called by the amsart/book/proc definition of \MR.
\providecommand{\MRhref}[2]{%
  \href{http://www.ams.org/mathscinet-getitem?mr=#1}{#2}
}
\providecommand{\href}[2]{#2}

\bibitem[Die69]{Dieudonne_1969_Foundations-of-}
J.~Dieudonn{{\'e}}, \emph{Foundations of modern analysis}, Academic Press, New
  York, 1969, Enlarged and corrected printing, Pure and Applied Mathematics,
  Vol. 10-I. \MR{MR0349288 (50 \#1782)}

\bibitem[Fol99]{Folland_1999_Real-analysis}
Gerald~B. Folland, \emph{Real analysis}, second ed., Pure and Applied
  Mathematics (New York), John Wiley \& Sons Inc., New York, 1999, Modern
  techniques and their applications, A Wiley-Interscience Publication.
  \MR{MR1681462 (2000c:00001)}

\bibitem[HK09]{Himonas_2009_Non-uniform-dep}
Alex Himonas and Carlos~E. Kenig, \emph{Non-uniform dependence on initial data
  for the ch equation on the line}, Differential Integral Equations \textbf{22}
  (2009), no.~3-4, 201--224.

\bibitem[HKM09]{Himonas_2009_Non-uniform-dep-per}
Alex Himonas, Carlos~E. Kenig, and G.~Misio{\l}ek, \emph{Non-uniform dependence
  for the periodic ch equation.}, To appear in Communications in Partial
  Differential Equations (2009).

\bibitem[KP88]{Kato_1988_Commutator-esti}
Tosio Kato and Gustavo Ponce, \emph{Commutator estimates and the {E}uler and
  {N}avier-{S}tokes equations}, Comm. Pure Appl. Math. \textbf{41} (1988),
  no.~7, 891--907. \MR{MR951744 (90f:35162)}

\bibitem[Tay91]{Taylor_1991_Pseudodifferent}
Michael~E. Taylor, \emph{Pseudodifferential operators and nonlinear {PDE}},
  Progress in Mathematics, vol. 100, Birkh{\"a}user Boston Inc., Boston, MA,
  1991. \MR{MR1121019 (92j:35193)}

\bibitem[Tay03]{Taylor_2003_Commutator-esti}
Michael~Eugene Taylor, \emph{Commutator estimates}, Proc. Amer. Math. Soc.
  \textbf{131} (2003), no.~5, 1501--1507 (electronic). \MR{MR1949880
  (2003k:35261)}

\end{thebibliography}


%\bibliographystyle{amsalpha}
%\bibliography{/Users/davidkarapetyan/Documents/math/references.bib}
%\nocite{*}

 \end{document}


