\documentclass{beamer}
%\let\Tiny=\tiny
\usepackage{mathtools}
\mathtoolsset{showonlyrefs} %number only equations that are referenced
%\usepackage{amscd}
%\usepackage{amsfonts}
%\usepackage{amsmath}
%\usepackage{amssymb}
%\usepackage{amsthm}
%\usepackage{fancyhdr}
%\usepackage{latexsym}
\usepackage{lmodern}
\synctex=1
%\input epsf
%\input texdraw
%\input txdtools.tex
%\input xy
%\xyoption{all}
%\usepackage{color}

%\definecolor{Red}{rgb}{1.00, 0.00, 0.00}
%\definecolor{DarkGreen}{rgb}{0.00, 1.00, 0.00}
%\definecolor{Blue}{rgb}{0.00, 0.00, 1.00}
%\definecolor{Cyan}{rgb}{0.00, 1.00, 1.00}
%\definecolor{Magenta}{rgb}{1.00, 0.00, 1.00}
%\definecolor{DeepSkyBlue}{rgb}{0.00, 0.75, 1.00}
%\definecolor{DarkGreen}{rgb}{0.00, 0.39, 0.00}
%\definecolor{SpringGreen}{rgb}{0.00, 1.00, 0.50}
%\definecolor{DarkOrange}{rgb}{1.00, 0.55, 0.00}
%\definecolor{OrangeRed}{rgb}{1.00, 0.27, 0.00}
%\definecolor{DeepPink}{rgb}{1.00, 0.08, 0.57}
%\definecolor{DarkViolet}{rgb}{0.58, 0.00, 0.82}
%\definecolor{SaddleBrown}{rgb}{0.54, 0.27, 0.07}
%\definecolor{Black}{rgb}{0.00, 0.00, 0.00}
%\definecolor{dark-magenta}{rgb}{.5,0,.5}
%\definecolor{myblack}{rgb}{0,0,0}
%\definecolor{darkgray}{gray}{0.5}
%\definecolor{lightgray}{gray}{0.75}
%
\newcommand{\bigno}{\bigskip\noindent}
\newcommand{\ds}{\displaystyle}
\newcommand{\medno}{\medskip\noindent}
\newcommand{\smallno}{\smallskip\noindent}
\newcommand{\nin}{\noindent}
\newcommand{\ts}{\textstyle}
\newcommand{\rr}{\mathbb{R}}
\newcommand{\p}{\partial}
\newcommand{\zz}{\mathbb{Z}}
\newcommand{\cc}{\mathbb{C}}
\newcommand{\ci}{\mathbb{T}}
\newcommand{\tor}{\mathbb{T}}
\newcommand{\ee}{\varepsilon}
\newcommand{\wh}{\widehat}
\newcommand{\weak}{\rightharpoonup}
\newcommand{\vp}{\varphi}
%
%
\newtheorem{proposition}{Proposition}
\newtheorem{claim}{Claim}
\newtheorem{remark}{Remark}
\newtheorem{conjecture}[subsection]{conjecture}

\def\refer #1\par{\noindent\hangindent=\parindent\hangafter=1 #1\par}

%% Equation Numbers %%

\renewcommand{\theequation}{\thesection.\arabic{equation}}





%%%%%%%%%%%%%%%%%%%%%%

\date{}
\title{On the Cauchy Problem for the Hyperelastic Rod Equation}
\author{\it David Karapetyan}
\begin{document}
%
\begin{frame}
\titlepage
\end{frame}

\begin{frame}
\frametitle{Abstract}
We consider the Cauchy problem for the Hyperelastic Rod (HR) equation
\begin{equation*} 
\label{hr}
\partial_t u + \gamma u\partial_x u + (1-\p_x^2)^{-1} \p_x \left
[\frac{3- \gamma}{2} u^2 + \frac{\gamma}{2}(\p_x u)^2 \right ] = 0,
\ x \in  \mathbb{T} \  \text{or}  \ \rr,  \ t \in \mathbb{R},
\end{equation*}
\begin{equation*} 
\label{hr-data} 
u(x, 0) = u_0 (x)
\end{equation*}
%
and prove a portion of the following result:
%
%
%
%
%
\frametitle{Well-Posedness}
\end{frame}
\begin{frame}
\begin{theorem}
\label{hr-wp}
If $u_0(x) \in  H^s$ for some $s >3/2$,  then there is  a $T>0$
depending only on  $\|u_0\|_{H^s}$ such that there exists a unique
function $u(x, t)$ solving  the HR Cauchy problem
in the sense of distributions with  $u \in C([0, T]; H^s)$.
The solution $u$ depends continuously on the initial data $u_0$
in the sense that the mapping of the initial data to the solution 
is continuous from the Sobolev space $H^s$ to the space $C([0, T]; H^s)$.
Furthermore, the  lifespan (the maximal existence time)
is greater than 
%
\begin{equation*}
T
\doteq
\frac{1}{2c_s}
\frac{1}{ \|u_0 \|_{H^s(\rr)}},
\end{equation*}
%
where $c_s$  is a constant depending only on $s$.
Also, we have 
%
\begin{equation*}
\label{u-u0-Hs-bound}
\|u(t)\|_{H^s(\rr))}
\le
2
\|u_0 \|_{H^s(\rr)},
\quad
0\le t \le T.
\end{equation*}
%
\end{theorem}
\end{frame}
%
%
%
\section{Proof of Well-Posedness for HR}
\begin{frame}[allowframebreaks]
\frametitle{Proof of Well-Posedness for HR}
%
%
%
%
%
%
%
%
%
%
Consider the following mollification of the HR equation
%
%
\begin{align*}
& \p_t  u_\ee =
-\gamma J_\ee(J_\ee u_\ee \partial_x  J_\ee  u_\ee) - \Lambda^{-1} \left 
[\frac{3-\gamma}{2}(u_\ee)^2 + \frac{\gamma}{2}(\p_x u_\ee)^2 \right ],
\\
& u_\ee(x, 0) = u_0 (x),
\label{hr-moli-data}
\end{align*}
%
% 
%
%
%
%
%
%
where $J_\ee$ is defined  by
%
\begin{equation*}
\begin{split}
J_\ee f(x) = j_\ee * f(x), \quad \ee>0
\end{split}
\end{equation*}
%
with 
%
\begin{equation*}
\begin{split}
j_\ee(x) = \frac{1}{\ee}j\left( \frac{x}{\ee} \right)
\end{split}
\end{equation*}
%
for non-negative $j(x) \in
\mathcal{S}(\rr)$.
\end{frame}
%\begin{frame}
%By the Cauchy Existence Theorem, for each 
%$\ee > 0$ there exists a
%unique solution $u_\ee \in C(I, H^s(\rr))$ satisfying the mollified HR 
%Cauchy-problem.
%\end{frame}
%%%%%%%%%%%%%%%%%%%%%%%%
%
%     Estimates  for Lifespan and Sobolev norm of $u_\ee$
%
%%%%%%%%%%%%%%%%%%%%%%%%
%
%

\begin{frame}
\frametitle{Estimates for the Lifespan and Sobolev norm of $u_\ee$.}
%
We will show that there is a lower bound  $T$
for $T_\ee$ which is independent of $\ee\in(0, 1]$.
This is based on the differential
inequality 
%
%
%
\begin{equation*} \label{B-diff-ineq}
\frac 12
\frac{d}{dt}
\|u_\ee(t)\|_{H^{s}(\rr)}^2
\le
c_s
\|u_\ee(t)\|_{H^{s}(\rr)}^3,
\quad
|t| \le T_\ee
\end{equation*}
%
%
%
%
%
%
%
Applying $D^s$ to both sides of the mollified HR equation,
multiplying the resulting equation by $D^s u_\ee$,
integrating it for $x\in\rr$, and noting that 
$D^s$ and $J_\ee$ commute
and that  $J_\ee$ satisfies 
%
%
\begin{equation*} 
\label{J-e-inner-prod-property}
(J_\ee f, g)_{L^2}=( f, J_\ee g)_{L^2}
\end{equation*}
%
%
we obtain

\end{frame}

\begin{frame}
%
%
%
\begin{equation*} \begin{split}
\label{B-moli-int}
\frac 12
\frac{d}{dt} \|u_\ee \|_{H^s(\rr)}^2
=
& -
\gamma \int_{\rr}  D^s(J_\ee u_\ee \partial_x J_\ee u_\ee) \cdot
D^s J_\ee u_\ee  \  dx
\\
&- \frac{3 -\gamma}{2} \int_{\rr} D^{s-2} \p_x (u_{\ee})^2 \cdot D^s J_\ee 
u_{\ee} \ dx
\\
& - \frac{\gamma}{2} \int_{\rr}  D^{s-2} \p_x (\p_x u_\ee)^2
\cdot D^s J_\ee u_\ee  \ dx.
\end{split}
\end{equation*}
%
%
%
Letting $v=J_\ee u_\ee$ we can rewrite the first integral on the right-hand 
side as 
%
%
%
\begin{equation*} \begin{split}
\label{B-moli-int-v}
& -  \gamma \int_{\rr}   D^s (J_{\ee} u_{\ee} \p_x J_\ee u_\ee)
\cdot D^s
J_{\ee}u_\ee \ dx
\\
& = - \gamma \int_\rr
\left [ D^s(v\p_x v)  -  v D^s (\p_xv)
\right ] \cdot D^s v \ dx
- \gamma \int_\rr
v D^s (\p_xv) 
\cdot D^s v \ dx.
\end{split}
\end{equation*}
%
%
%
%
%
We now estimate this in parts. Applying the Cauchy-Schwarz 
inequality gives
%
%
%
\end{frame}

\begin{frame}

\begin{equation*} \label{int1-est-calc2}
\begin{split}
& \Big|
- \gamma \int_\rr
\big[ D^s(v\p_x v)  -  v D^s (\p_xv)
\big]
\cdot D^s v   \, dx
\Big|
\\
& \lesssim
\|
D^s(v\p_x v)  -  v D^s (\p_xv)
\|_{L^2(\rr)}
\|
v
\|_{H^s(\rr)}
\lesssim \| \p_x v \|_{L^\infty(\rr)} \| v \|_{H^s(\rr)}^2,
\end{split}
\end{equation*}
%
%
%
where the last step follows from 
%
%
%
\begin{equation*} \label{int1-est-calc3}
\| D^s(v\p_x v)  -  v D^s (\p_xv) \|_{L^2(\rr)}
\le
2 c_s    \| \p_x v \|_{L^\infty(\rr)} \| v \|_{H^s(\rr)},
\end{equation*}
%
%
which is a simple corollary of the following Kato-Ponce commutator 
estimate.
%
%
\begin{lemma}[Kato-Ponce] \label{KP-lemma}
If  $s>0$ then there is $c_s>0$ such that 
%
%
%
\begin{equation*} \begin{split}
& \| D^{s} \big(fg) -  f D^s g\|_{L^2(\rr)}
\\
& \le
c_s \big(
\| D^{s}f \|_{L^2(\rr)}    \| g \|_{L^\infty(\rr)} 
+ \| \p_xf \|_{L^\infty(\rr)}    \| D^{s-1}g \|_{L^2(\rr)}   \big).
\end{split}
\end{equation*}
%
%
%
\end{lemma}
%
%
\end{frame}

%\begin{frame}
%%
%The remaining integrals can be estimated using Cauchy-Schwartz, the algebra
%property, and Sobolev Imbedding. Grouping all the estimates, we obtain the
%desired ordinary differential inequality. Solving  it, we obtain the following:
%%
%%
%\end{frame}
%
%
%
\begin{frame}
%
%
\begin{lemma}
\label{hr_wp}
Let  $u_0(x) \in  H^s(\rr)$, $s >3/2$. Then for any $\ee\in (0, 1]$
the i.v.p. for the mollified HR equation 
%
%
%
\begin{align*} 
& \partial_t  u_\ee =
-\gamma J_\ee (J_\ee u_\ee \partial_x  J_\ee  u_\ee) - \Lambda^{-1} \left
[\frac{3-\gamma}{2}(u_\ee)^2 + \frac{\gamma}{2}(\p_x u_\ee)^2
\right ], 
\\
&  u_\ee(x, 0) = u_0 (x)
\label{burgers-moli-data-2}
\end{align*}
%
% 
%
%
%
%
%
%
has a unique solution $u_\ee( t)\in C([-T, T], H^s(\rr))$.  In particular,
%
%
%
\begin{equation*} \label{life-est}
T
=
\frac{1}{ 2 c_s \|u_0\|_{H^s(\rr)}}
\end{equation*}
%
%
%
is a lower bound for the lifespan of $u_\ee( t)$ and
%
%
%
\begin{equation*}
\label{u-e-Hs-bound}
\|u_\ee(t)\|_{H^s(\rr)}
\le
2 \|u_0 \|_{H^s(\rr)},
\quad
|t| \le T.
\end{equation*}
%
%
%
Furthermore,  $u_\ee( t)\in C^1([T, T], H^{s-1}(\rr))$ and satisfies
%
%
\begin{equation*}
\label{dt-u-e-Hs-bound}
\|\p_t u_\ee(t)\|_{H^{s-1}(\rr)}
\lesssim
\|u_0 \|_{H^s(\rr)}^2,
\quad
|t| \le T.
\end{equation*}
%
%
% 
\end{lemma}
%
%
\end{frame}
%%%%%%%%%%%%%%%%%%%%%%%%
%
%     Choosing  a convergent subsequence
%
%%%%%%%%%%%%%%%%%%%%%%%%
\begin{frame}
\frametitle{Choosing  a convergent subsequence.}
%
Next we shall show that the family $\{ u_\ee\}$ has a convergent 
subsequence
whose limit $u$ solves the HR i.v.p.  Let I= [-T, T]. By the previous lemma and the compactness of $I$ we have a uniformly bounded 
family
%
%
%
\begin{equation*}
\label{Lip-1-fam}
\{u_\ee\}\subset C(I, H^s(\rr))\cap C^1(I,
H^{s-1}(\rr)).
\end{equation*}
%
%
%
%
By the Riesz Lemma, we can identify $H^s(\rr)$ with
$(H^s(\rr))^*$, where for $w, \psi \in H^s(\rr)$ the duality is
defined by 
\begin{equation*}
T_w(\psi) = <w, \psi>_{H^s(\rr)} = \int_{\rr}
\widehat{w}(\xi, t) \overline{\widehat{\psi}}(\xi, t) \cdot (1
+ \xi^2)^s \ d \xi.
\end{equation*}
Applying the Riesz Representation Theorem, it follows that we 
can identify $L^\infty(I, H^s(\rr)) $ with the dual space of $L^1(I,
H^{s}(\rr))$, where for $v\in L^\infty(I, H^s(\rr)) $ and $ \phi \in
L^1(I, H^{s}(\rr))$ the duality is defined by  
%
%
%
\begin{equation*}
T_v(\phi) = \int_I <v (t), \phi (t)>_{H^s(\rr)} dt  = \int_I
\int_{\rr}
\widehat{v}(\xi, t) \overline{\widehat{\phi}}(\xi, t) \cdot (1
+ \xi^2)^s \ d \xi dt.
\end{equation*}
%
%
%
\end{frame}

\begin{frame}
By Aloaglu's Theorem, the bounded family $\{u_\ee\}$ is compact in the weak$^*$ topology of $L^\infty(I, H^s(\rr))$. More precisely,
there is a subsequence  $\{ u_{\ee_k} \}$ converging
weakly to a $ u\in L^{\infty}(I, H^s(\rr))$.
That is 
%
%
%
\begin{equation*}
\label{weak-conv}
\lim_{n\to \infty} T_{u_{\ee_k}}(\phi)  =  T_u (\phi) \; \;		\text{ for 
all } \;\;  \phi \in L^1(I, H^{s}(\rr)).
\end{equation*}
%
In order to show that $u$ solves the HR i.v.p., it would
suffice to obtain a stronger convergence for  $u_{\ee_n}$ so that 
we could take the limit in the mollified HR equation. However,
this is difficult, and unnecessary. Rather, our approach will be to
show that for any pseudo-differential operator
$P \in \Psi^0$ and arbitrary $\vp \in S(\rr)$, $k \in
\mathbb{N}$, $0< \sigma < 1$, we have
\end{frame}
%
\begin{frame}
%
%
\begin{align}
\label{hhstrong-conv}
& \varphi P [(u_{\ee_n})^k] \longrightarrow \varphi P [u^k]  
\quad
\text{ in } \,\,   C(I, H^{s-\sigma}(\rr)), \ \,
\\
\label{hhstrong-conv-next}
& \varphi P [(\p_x u_{\ee_n})^k] \longrightarrow \varphi P
[(\p_x u)^k]  
\quad
\text{ in } \,\,   C(I, H^{s-\sigma -1}(\rr)), \ \ 
\end{align}
%
\end{frame}
%%%%%%%%%%%%%%%%%%%%%%%%%%%
%
%
%                 Interpolation 
%
%
%%%%%%%%%%%%%%%%%%%%%%%%%%%
\begin{frame}
\frametitle{Interpolation}
\begin{lemma}
\label{interpolation-lem}
Let  $s > \frac{3}{2}$.
If $v \in C(I, H^s(\rr)) \cap C^1(I, H^{s-1}(\rr))$
then $v \in C^\sigma (I, H^{s- \sigma}(\rr))$ for  $0 < \sigma < 1$.
\end{lemma}
%
%
\end{frame}
%
\begin{frame}
Fix $k \in \mathbb{N}$. Using the previous lemma, we
will show that the family
\begin{equation*}
\begin{split}
\{\varphi P[(u_\ee)^k]\}_\ee
\end{split}
\end{equation*}
is equicontinuous in $C(I, H^{s-\sigma}(\rr))$ 
for $0 < \sigma < 1$ and $\varphi = \varphi(x) \in \mathcal{S}(\rr)$.
We will follow this by proving that
there exists a sub-family $\{\varphi P[(u_{\ee_n}(t))^k]\}_n$
that is precompact in $H^{s-\sigma}(\rr)$ for $\sigma > 0$. 
These two facts, in conjunction with Ascoli's Theorem, will
yield
\begin{equation*}
\label{hhstrong-conv2}
\varphi P[(u_\ee)^k] \to \tilde{u}
\; \; \text{in} \; \; C(I,H^{s-\sigma}(\rr))
\end{equation*}
for $0 < \sigma < 1$.
We will then show that $\tilde{u} = \varphi P[u^k]$, from which it will
follow that
\begin{equation*}
\label{hhphiplus}
\begin{split}
\varphi P[(u_\ee)^k] \to \varphi P[u^k]
\; \; \text{in} \; \; C(I,H^{s-\sigma}(\rr)).
\end{split}
\end{equation*}
\end{frame}



%%%%%%%%%%%%%%%%%%%%%%
%
%
%       Equicontinuity
%
%
%%%%%%%%%%%%%%%%%%%%%%

%
\vskip0.1in
\nin
\begin{frame}
\frametitle{Equicontinuity of $\{ \varphi P [(u_\ee)^k]\}_\ee$  in $C(I,
H^{s-\sigma}(\rr))$}
%
%
Since $\varphi \in \mathcal{S}(\rr)$, the map $u \mapsto \vp u$
is a bounded linear function on $H^s(\rr)$, for arbitrary $s \in
\rr$, where  
\begin{equation}
\begin{split}
\|\varphi u\|_{H^s(\rr)} \le C(s, \varphi)
\|u\|_{H^s(\rr)}, \quad \forall s\in \rr.
\label{hhschwartz-estimate}
\end{split}
\end{equation}
Furthermore, $$P: H^s(\rr) \to H^s(\rr)$$ is bounded and linear,
with 
\begin{equation}
\label{operator-normaa}
\|P\|_{L(H^s(\rr), H^s(\rr))} \le 1.
\end{equation}
Hence, the map 
\begin{equation}
\label{the-map}
\begin{split}
& T: H^s(\rr) \to H^s(\rr),
\\
& T(u) = \vp P u 
\end{split}
\end{equation}
is bounded and linear, with 
\begin{equation}
\begin{split}
\|T\|_{L(H^s(\rr), H^s(\rr))} \le C(s, \vp).
\label{op-norm-product}
\end{split}
\end{equation}
Therefore, applying the interpolation lemma
gives 
\end{frame}
\begin{frame}
%
\begin{equation*}
\begin{split}
\label{hhequic-1}
& \sup_{t \neq t'} \frac {\| \varphi P [(u_\ee(t))^k] - \varphi
P [(u_\ee(t'))^k] \|_{H^{s -
\sigma  }(\rr)}}{|t - t'|}
\\
& \le \sup_{t \neq t'}  \frac { \|\vp P \|_{L(H^{s-\sigma}(\rr),
H^{s-\sigma}(\rr))} \cdot \|   [u_\ee(t)]^k  - 
[u_\ee(t')]^k \|_{H^{s -
\sigma }(\rr)}}{|t - t'|}
\\
& \le C(s, \vp) \cdot \sup_{t \neq t'}  \frac { \|   [u_\ee(t)]^k  - 
[u_\ee(t')]^k \|_{H^{s -
\sigma }(\rr)}}{|t - t'|}
\\
&< c
\end{split}
\end{equation*}
%
or
%
\begin{equation*}
\label{hhequic-2}
\|\varphi P [(u_\ee(t))^k] - \varphi
P [(u_\ee(t'))^k \|_{H^{s - \sigma }(\rr)}< c|t -
t'|, 
\text{ for all }  \,\,  t, t'\in I,
\end{equation*}
%
which shows that  the family  $\{\varphi P [(u_\ee)^k]\}_\ee$ is
equicontinuous in $C(I, H^{s-\sigma }(\rr))$.  $\quad \Box$
%
\vskip0.1in
\nin
%
%%%%%%%%%%%%%%%%%%%%%%
%
%
%      PreCompactness
%
%
%%%%%%%%%%%%%%%%%%%%%%%%%%
%
%
%
%
%		
\end{frame}
\begin{frame}
\frametitle{Precompactness of $\{\varphi P [(u_\ee(t))^k]\}_\ee$ in
$H^{s-\sigma  }(\rr)$}
Recalling the algebra property of Sobolev
spaces and our energy estimates, we have
\begin{equation*}
\begin{split}
\|P [(u_\ee(t))^k]\|_{H^{s}(\rr)}
\lesssim_s \|u_0 \|^k_{H^s(\rr)} < \infty.
\end{split}
\end{equation*}
Therefore, by Reillich's Theorem, the family $\left\{
\varphi P [(u_\ee(t))^k] \right\}_\ee$ is
precompact in $H^{s- \sigma }(\rr)$ for all $\sigma > 0$ and $|t| \le T$. $\quad
\Box$ 
\vskip0.1in
\end{frame}
\begin{frame}
Hence, compiling our previous results on equicontinuity and precompactness
and applying Ascoli's Theorem, we
conclude that we can find $\tilde{u}$ and a subfamily 
\\ $\left\{
\varphi P [(u_{\ee_n})^k]
\right\}_n$ such that
\begin{equation}
\label{hhstrong-conv-of-u_ep}
\varphi P [(u_{\ee_n})^k] \to \tilde{u}
\; \; \text{in} \; \; C(I, H^{s-\sigma}(\rr)).
\end{equation}
%
%

\vskip0.1in
We would now like to find out what $\tilde{u}$ is:
\end{frame}
\begin{frame}
%
%
%
\vskip0.1in
\begin{lemma}
\label{hhlem:crit-conv}
For arbitrary $k \in \mathbb{N}$,
\begin{equation}
\begin{split}
\varphi P [(u_{\ee_n})^k] \xrightarrow{weak^*}
\varphi P [u^k] \ \ \text{on} \ \ L^\infty(I,
H^{s-\sigma}(\rr)).
\label{hhcrit-conv-est}
\end{split}
\end{equation}
\end{lemma}
{\bf Proof.} 
Fix $k \in \mathbb{N}$ and recall that the operators 
\begin{equation*}
\begin{split}
& T_\varphi: H^s(\rr) \to H^s(\rr)\\
& T_\varphi u = \varphi u
\end{split}
\end{equation*}
and 
\begin{equation*}
\begin{split}
P:H^s(\rr) \to H^s(\rr)
\end{split}
\end{equation*}
are continuous; therefore 
\end{frame}
\begin{frame}
\begin{equation*}
\begin{split}
T_\vp P: H^s(\rr) \to H^s(\rr)
\end{split}
\end{equation*}
continuously. Hence, its adjoint  $(T_\varphi P)^*$
exists and
\begin{equation*}
(T_\varphi P)^*: H^s(\rr) \to H^s(\rr) 
\end{equation*}
continuously. Therefore, we conclude that
\begin{equation}
\label{widpseudo}
\begin{split}
& \int_I <\varphi P[u^k] - \varphi
P [(u_{\ee_n})^k],\  f>_{H^{s-\sigma }(\rr)} dt
\\
&= \int_I <u^k - 
(u_{\ee_n})^k, \ (T_\vp P)^* f>_{H^{s-\sigma }(\rr)} \to 0
\end{split}
\end{equation}
completing the proof. $\quad \Box$
\vskip0.1in
%
%
We summarize our result below:
\end{frame}
\begin{frame}
\begin{proposition}
\label{hhthm:crit1}
Let $P \in \Psi^0$ be a pseudo-differential operator. Then for
arbitrary $k \in \mathbb{N}$, 
\begin{equation}
\begin{split}
& \vp P [(u_{\ee_n})^k] \to \vp P [u^k] \ \ \text{in}  \ \ C(I,
H^{s-\sigma }(\rr)),
\\
& 
\vp P [(\p_x u_{\ee_n})^k] \to \vp P [(\p_x u)^k] \ \
\text{in}  \ \ C(I,
H^{s-\sigma -1}(\rr)).
\label{hhdx_vp_u_ep_conv}
\end{split}
\end{equation}
\end{proposition}
\vskip0.1in
\end{frame}
\begin{frame}
\frametitle{Verifying that $u$ solves the HR equation.} 
Restricting the
choice of $\vp$ such that $\vp^\frac{1}{2} \in S(\rr)$ and applying the
proposition it follows that
\begin{equation}
\begin{split}
& -\gamma \vp (J_{\ee_n} u_{\ee_n} \cdot J_{\ee_n} \p_x
u_{\ee_n}) -
\vp \p_x(1- \p_x^2)^{-1} \left( \frac{3-\gamma}{2}
(u_{\ee_n})^2
+ \frac{\gamma}{2} (\p_x u_{\ee_n})^2 \right )
\\
\to & -\gamma \vp u \p_x u -
\vp \p_x(1- \p_x^2)^{-1} \left( \frac{3-\gamma}{2} u^2
+ \frac{\gamma}{2} (\p_x u)^2 \right ) \ \
\text{in} \ \ C(I, C(\rr)).
\label{llloc-non-loc-tog}
\end{split}
\end{equation}
%
\end{frame}
\begin{frame}
Next, we note that the convergence  
%
\begin{equation}
\label{hhweak-conv-2}
T_{\vp u_{\ee_n}}(f)  \longrightarrow  T_{\vp u} (f) \;
\text{ for all } \;  f \in L^1(I, H^{-s}(\rr))
\end{equation}
%
can be restated as 
%
\begin{equation}
\vp u_{\ee_n}  \longrightarrow  \vp u
\quad
\text{ in }  \,\,
\mathcal{D}'(I\times \rr).
\end{equation}
%
\end{frame}
\begin{frame}
This implies 
%
\begin{equation}
\label{hhdistib-conv-2}
\p_t(\vp u_{\ee_n})  \longrightarrow  \p_t (\vp u)
\quad
\text{ in }  \,\, \mathcal{D}'(I\times \rr).
\end{equation}
%
Since for all $n$ we have 
%
\begin{equation}
\begin{split}
\p_t (\vp u_{\ee_n})
= & -\gamma \vp
(J_{\varepsilon_n} u_{\varepsilon_n}  \cdot
J_{\varepsilon_n}\partial_x u_{\varepsilon_n})
\\
& -
\vp \p_x(1- \p_x^2)^{-1} \left( \frac{3-\gamma}{2} (u_{\ee_n})^2
+ \frac{\gamma}{2} (\p_x u_{\ee_n})^2 \right )
\end{split}
\end{equation}
%
it follows from the uniqueness of the weak limit on the left hand side
and the uniqueness of the
$C(I, C(\rr))$ limit of the right hand side that 
\begin{equation}
\begin{split}
\p_t (\vp u)
= & -\gamma \vp
u \p_x u - \vp \p_x(1- \p_x^2)^{-1} \left( \frac{3-\gamma}{2} u^2
+ \frac{\gamma}{2} (\p_x u)^2 \right ).
\label{hhadone}
\end{split}
\end{equation}
Further restricting $\vp \in \mathcal{S}(\rr)$ to be nonzero in
$\rr$, we
can divide both sides by $\vp$ to obtain
\begin{equation}
\label{hh2yy}
\begin{split}
\p_t  u
= & -\gamma
u \p_x u - \p_x(1- \p_x^2)^{-1} \left( \frac{3-\gamma}{2} u^2
+ \frac{\gamma}{2} (\p_x u)^2 \right ).
\end{split}
\end{equation}
Thus we have constructed a solution $u \in L^\infty(I, H^s(\rr))$
to the HR i.v.p. 
\vskip0.1in

%
%

\end{frame}
%
%
%
%
%
%\begin{thebibliography}{HKM09}

%\providecommand{\bysame}{\leavevmode\hbox to3em{\hrulefill}\thinspace}
%\providecommand{\MR}{\relax\ifhmode\unskip\space\fi MR }
%% \MRhref is called by the amsart/book/proc definition of \MR.
%\providecommand{\MRhref}[2]{%
%\href{http://www.ams.org/mathscinet-getitem?mr=#1}{#2}
%}
%\providecommand{\href}[2]{#2}

%\bibitem[Die69]{Dieudonne_1969_Foundations-of-}
%J.~Dieudonn{{\'e}}, \emph{Foundations of modern analysis}, Academic Press, New
%York, 1969, Enlarged and corrected printing, Pure and Applied Mathematics,
%Vol. 10-I. \MR{MR0349288 (50 \#1782)}

%\bibitem[Fol99]{Folland_1999_Real-analysis}
%Gerald~B. Folland, \emph{Real analysis}, second ed., Pure and Applied
%Mathematics (New York), John Wiley \& Sons Inc., New York, 1999, Modern
%techniques and their applications, A Wiley-Interscience Publication.
%\MR{MR1681462 (2000c:00001)}

%\bibitem[HK09]{Himonas_2009_Non-uniform-dep}
%Alex Himonas and Carlos~E. Kenig, \emph{Non-uniform dependence on initial data
%for the ch equation on the line}, Differential Integral Equations \textbf{22}
%(2009), no.~3-4, 201--224.

%\bibitem[HKM09]{Himonas_2009_Non-uniform-dep-per}
%Alex Himonas, Carlos~E. Kenig, and G.~Misio{\l}ek, \emph{Non-uniform dependence
%for the periodic ch equation.}, To appear in Communications in Partial
%Differential Equations (2009).

%\bibitem[KP88]{Kato_1988_Commutator-esti}
%Tosio Kato and Gustavo Ponce, \emph{Commutator estimates and the {E}uler and
%{N}avier-{S}tokes equations}, Comm. Pure Appl. Math. \textbf{41} (1988),
%no.~7, 891--907. \MR{MR951744 (90f:35162)}

%\bibitem[Tay91]{Taylor_1991_Pseudodifferent}
%Michael~E. Taylor, \emph{Pseudodifferential operators and nonlinear {PDE}},
%Progress in Mathematics, vol. 100, Birkh{\"a}user Boston Inc., Boston, MA,
%1991. \MR{MR1121019 (92j:35193)}

%\bibitem[Tay03]{Taylor_2003_Commutator-esti}
%Michael~Eugene Taylor, \emph{Commutator estimates}, Proc. Amer. Math. Soc.
%\textbf{131} (2003), no.~5, 1501--1507 (electronic). \MR{MR1949880
%(2003k:35261)}

%\end{thebibliography}


%\bibliographystyle{amsalpha}
%\bibliography{/Users/davidkarapetyan/Documents/math/references.bib}
%\nocite{*}

\end{document}


