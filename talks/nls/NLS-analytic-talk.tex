%
\documentclass{beamer}
\setbeamersize{text margin left=0.4cm, text margin right=0.4cm}
%\documentstyle{Boadilla}
\usepackage{amscd}
\usepackage{amsfonts}
\usepackage{amsmath}
\usepackage{amssymb}
\usepackage{amsthm}
\usepackage{fancyhdr}
\usepackage{latexsym}
\usepackage{lmodern}
\usepackage{cancel}
\usepackage{hyperref}
\synctex=1
\numberwithin{equation}{section}
%
\newcommand{\bigno}{\bigskip\noindent}
\newcommand{\ds}{\displaystyle}
\newcommand{\medno}{\medskip\noindent}
\newcommand{\smallno}{\smallskip\noindent}
\newcommand{\nin}{\noindent}
\newcommand{\ts}{\textstyle}
\newcommand{\rr}{\mathbb{R}}
\newcommand{\p}{\partial}
\newcommand{\zz}{\mathbb{Z}}
\newcommand{\cc}{\mathbb{C}}
\newcommand{\ci}{\mathbb{T}}
\newcommand{\tor}{\mathbb{T}}
\newcommand{\ee}{\varepsilon}
\newcommand{\wh}{\widehat}
\newcommand{\weak}{\rightharpoonup}
\newcommand{\vp}{\varphi}
%
%
\newtheorem{proposition}{Proposition}
\newtheorem{claim}{Claim}
\newtheorem{remark}{Remark}
\newtheorem{conjecture}[subsection]{conjecture}

\def\refer #1\par{\noindent\hangindent=\parindent\hangafter=1 #1\par}

%% Equation Numbers %%

\renewcommand{\theequation}{\thesection.\arabic{equation}}





%%%%%%%%%%%%%%%%%%%%%%
%
\date{}
\title
{ Well-posedness of  periodic NLS  \\in  analytic  spaces}
  
\author{{\it A.  Himonas,\, D. Karapetyan   \& G. Petronilho}}
%\author{Alex Himonas}
%\address{Department of Mathematics  \\
%       University of Notre Dame     \\
%         Notre Dame, IN 46556}

\begin{document}

\begin{frame}
\titlepage
\end{frame}


%\section*{Table of Contents}
%\begin{frame}
%\frametitle{Table Of Contents}
%\tableofcontents
%\end{frame}
%
%
\begin{frame}
  \frametitle{Cubic NLS}
%
%
\begin{align*}
	&i \p_t u + \p_x^{2} u \pm
  |u|^2 u =0, \quad x\in \mathbb{T},\,\,t\in \mathbb{R}
		\\
		&u(x,0) = \vp(x).
\end{align*}
%
Some Introductory Material on the history of the equation.
\end{frame}
%
%
\begin{frame}
  \frametitle{The Analytic Spaces}
For $\delta >0$ and $s\ge 0$, we define 
$$G^{\delta, s}(\mathbb{T})=\{f\in L^2(\mathbb{T}): ||f||^2_{G^{\delta,s}(\mathbb{T})}=\sum_{k\in \mathbb{Z}} 
(1+ |k|)^{2s}e^{2\delta|k|}|\widehat{f}(k)|^2<\infty\}.$$
%
%
%
%
\[Y_{\delta,s}=Y_{\delta,s}(\mathbb{T}\times \mathbb{R})=\{v\in L^2(\mathbb{T}\times \mathbb{R}):
||v||_{Y_{\delta,s}}<\infty\}\]
where 
\begin{equation*}
  \begin{split}
    ||v||_{Y_{\delta,s}}
    & =\left( \sum_{k\in \mathbb{Z}}\int_{\mathbb{R}}(1+|\tau
-k^2|)(1+|k|)^{2s}
e^{2\delta |k|}|\widehat{v}(k,\tau)|^2d\tau\right)^{1/2}
\\
& +\Big (\sum_{k\in \mathbb{Z}}(1+|k|)^{2s}e^{2\delta |k|}
\Big [\int_{\mathbb{R}}
|\widehat{v}(k,\tau)|d\tau\Big ]^2 \Big )^{\frac{1}{2}} 
\end{split}
\end{equation*}
%
\end{frame}
%
%
\begin{frame}
  \frametitle{Facts}
  \begin{enumerate}
    \item If $\vp \in G^{\delta, s}(\ci)$, then it is real analytic, with a complex
  analytic extension to a symmetric strip around the real axis with width $\delta$.
\item
  For a cutoff function $\psi(t)$ symmetric about the origin with
  $\psi(t)=1$ for $| t | \le T$, we have
$$Y_{\delta,s}(\mathbb{T}\times \mathbb{R})
\xrightarrow{\psi} C([0,T],G^{\delta,s}(\mathbb{T}))$$.
  \end{enumerate}
\end{frame}

\begin{frame}
  \frametitle{Main Result}
%
\begin{theorem}
  \begin{enumerate}
%
\item
Let $s\ge 0$. For initial data in $G^{\delta,s}(\mathbb{T})$, $\delta >0$,
there exists a positive time $T$, such that the initial-value problem
is well-posed in the space $C([0,T], G^{\delta,s}(\mathbb{T}))$.
%
\item
Moreover, the regularity of solutions in the time variable is Gevrey of order
two ($G^2$).
\end{enumerate}
%
\end{theorem}
\end{frame}
%
\begin{frame}
  \frametitle{NLS Localized Integral Form}
\begin{equation*}
  \begin{split}
u(x,t) & =
 \psi(t) \underset{n\in \mathbb{Z}}{\sum} e^{i(nx +n^2t)}  \widehat{\varphi} (n)
\\
& + \psi(t) \underset{n\in \mathbb{Z}}{\sum} e^{i(nx+n^2t)}  \int_{\mathbb{R}}
\frac{e^{i(\lambda-n^2 )t}-1}{\lambda-n^2}
\widehat{w}(n,\lambda) \ d \lambda
\end{split}
\end{equation*}
where $w=|u|^2\overline{u}=u\overline{u}u$ and 
$\psi$ is a cutoff function symmetric about the origin with
$\psi(t)=1$ for $| t | \le T$. Computing the $Y_{\delta, s}$ norm of the right
hand side, the proof of well-posedness reduces to showing the following.
\end{frame}
\begin{frame}
  \frametitle{Trilinear Estimates}
\begin{lemma}\label{bilinear1}
 For $s\ge 0$  
$$\left ( \sum_{n \in {\mathbb{Z}}} (1+|n|)^{2s}e^{2\delta |n|}
\int_{\mathbb{R}}
\frac{|\widehat{w_{fgh}}(n, \lambda)|^2}{1+|\lambda -n^2|}
d\lambda
\right )^{1/2}\le C||u||^3_{{X}_{\delta,s}}$$
%
%
\begin{equation*}
	\begin{split}
		\left( \sum_{n \in \zz} \left (1 + |n| \right )^{2s}  \left ( \int_\rr 
		\frac{|\wh{w_{fgh}}(n, \lambda) |}{1 + | \lambda - n^{2} |}
		 \ d\lambda \right)^2  \right)^{1/2} \lesssim \|f\|_{X_{\delta, s}} \|g\|_{X_{\delta, s}}\|h\|_{X_{\delta, s}}
	\end{split}
\end{equation*}
for all $f, g, h \in Y_{\delta,s}$.
\end{lemma}
\end{frame}
%
%
\begin{frame}
  \frametitle{Key Point in Proof}
We will prove only the first trilinear estimate. The proof of the second is
similar. Start by setting
\begin{equation*}
c_u(n,\lambda)=|n|^s e^{\delta |n|}(1+|\lambda-n^2|)^{1/2}\widehat{u}(n,\lambda)
\end{equation*}
then we have
\begin{equation*}
||u||_{{X}_{\delta,s}}=\Big (\sum_{n\in \mathbb{{Z}}}\int_{\mathbb{R}}|c_u(n,\lambda)|^2d\lambda \Big )^{1/2}
=||c_u(n,\lambda)||_{l^2_nL^2_{\lambda}}.
\end{equation*}
\end{frame}
\begin{frame}
It follows that

\begin{eqnarray*}
&&\sum_{n } (1+|n|)^{2s}e^{2\delta |n|}
\int_{}
     \frac{|\widehat{w}(n, \lambda)|^2}{1+|\lambda -n^2|}
d\lambda
\\
&&
=\sum_{n  {}} (1+|n|)^{2s}e^{2\delta |n|}
\int_{}
     \frac{1}{1+|\lambda -n^2|}\Big \vert 
         \notag\\
&&
  \times  \sum_{n_1 {}}
     \int_{_{\lambda_1}}\sum_{n_2 {}}
     \int_{_{\lambda_2}}\frac{e^{-\delta |n-n_1|}c_u(n-n_1,\lambda-\lambda_1)}{(1+|n-n_1|)^s(1+|(\lambda-\lambda_1)-(n-n_1)^2|)^{1/2}}\notag\\
&&\times
\frac{e^{-\delta|n_1-n_2|}c_u(n_1-n_2,\lambda_1-\lambda_2)}{(1+|n_1-n_2|)^s(1+|(\lambda_1-\lambda_2)-(n_1-n_2)^2|)^{1/2}}
\\
&& \times \frac{e^{-\delta |n_2|}\overline{c_u(-n_2,-\lambda_2)}}{(1+|n_2|)^s(1+|(-\lambda_2)-(-n_2)^2|)^{1/2}}
d\lambda_2d\lambda_1 \Big \vert^2d\lambda
     \notag\\
   \end{eqnarray*}
   which is bounded by
 \end{frame}
   \begin{frame}
     \begin{eqnarray*}
      &&=\sum_{n  {}} 
\int_{\lambda}
     \Big \vert \sum_{n_1 {}}
     \int_{_{\lambda_1}}\sum_{n_2 {}}
     \int_{_{\lambda_2}}
     \\
     && \frac{(1 + |n|)^{s}|(1 + |n-n_1|)^{-s}(1 + |n_1-n_2|)^{-s}(1 + |n_2|)^{-s}}{(1+|\lambda -n^2|)^{1/2}(1+|(\lambda-\lambda_1)-(n-n_1)^2|)^{1/2}}\\
     &&\times \frac{c_u(n-n_1,\lambda-\lambda_1)c_u(n_1-n_2,\lambda_1-\lambda_2)\overline{c_u(-n_2,-\lambda_2)}}{(1+|(\lambda_1-\lambda_2)-(n_1-n_2)^2|)^{1/2}
     (1+|(-\lambda_2)-(-n_2)^2|)^{1/2}}d\lambda_1d\lambda_2 \Big \vert^2
\end{eqnarray*}
since $|n|\le |n-n_1|+(1+|n_1-n_2|)+|n_2|$.
From here the proof is the same as the NLS trilinear estimate. \qed
\end{frame}
\end{document}
