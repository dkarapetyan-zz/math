%
\documentclass[12pt,reqno]{amsart}
\usepackage{amssymb}
\usepackage{appendix}
\usepackage[showonlyrefs=true]{mathtools} %amsmath extension package
\usepackage{cancel}  %for cancelling terms explicity on pdf
\usepackage{yhmath}   %makes fourier transform look nicer, among other things
\usepackage{framed}  %for framing remarks, theorems, etc.
\usepackage{enumerate} %to change enumerate symbols
\usepackage[margin=2.5cm]{geometry}  %page layout
\setcounter{tocdepth}{1} %must come before secnumdepth--else, pain
\setcounter{secnumdepth}{1} %number only sections, not subsections
%\usepackage[pdftex]{graphicx} %for importing pictures into latex--pdf compilation
\numberwithin{equation}{section}  %eliminate need for keeping track of counters
%\numberwithin{figure}{section}
\setlength{\parindent}{0in} %no indentation of paragraphs after section title
\renewcommand{\baselinestretch}{1.1} %increases vert spacing of text
%
\usepackage{hyperref}
\hypersetup{colorlinks=true,
linkcolor=blue,
citecolor=blue,
urlcolor=blue,
}
\usepackage[alphabetic, initials, msc-links]{amsrefs} %for the bibliography; uses cite pkg. Must be loaded after hyperref, otherwise doesn't work properly (conflicts with cref in particular)
\usepackage{cleveref} %must be last loaded package to work properly
%
%
\newcommand{\ds}{\displaystyle}
\newcommand{\ts}{\textstyle}
\newcommand{\nin}{\noindent}
\newcommand{\rr}{\mathbb{R}}
\newcommand{\nn}{\mathbb{N}}
\newcommand{\zz}{\mathbb{Z}}
\newcommand{\cc}{\mathbb{C}}
\newcommand{\ci}{\mathbb{T}}
\newcommand{\zzdot}{\dot{\zz}}
\newcommand{\wh}{\widehat}
\newcommand{\p}{\partial}
\newcommand{\ee}{\varepsilon}
\newcommand{\vp}{\varphi}
\newcommand{\wt}{\widetilde}
%
%
%
%
\newtheorem{theorem}{Theorem}[section]
\newtheorem{lemma}[theorem]{Lemma}
\newtheorem{corollary}[theorem]{Corollary}
\newtheorem{claim}[theorem]{Claim}
\newtheorem{prop}[theorem]{Proposition}
\newtheorem{proposition}[theorem]{Proposition}
\newtheorem{no}[theorem]{Notation}
\newtheorem{definition}[theorem]{Definition}
\newtheorem{remark}[theorem]{Remark}
\newtheorem{examp}{Example}[section]
\newtheorem {exercise}[theorem]{Exercise}
\newtheorem {example}[theorem]{Example}
%
%\makeatletter \renewenvironment{proof}[1][\proofname] {\par\pushQED{\qed}\normalfont\topsep6\p@\@plus6\p@\relax\trivlist\item[\hskip\labelsep\bfseries#1\@addpunct{.}]\ignorespaces}{\popQED\endtrivlist\@endpefalse} \makeatother%
%makes proof environment bold instead of italic
\newcommand{\uol}{u^\omega_\lambda}
\newcommand{\lbar}{\bar{l}}
\renewcommand{\l}{\lambda}
\newcommand{\R}{\mathbb R}
\newcommand{\RR}{\mathcal R}
\newcommand{\al}{\alpha}
\newcommand{\ve}{q}
\newcommand{\tg}{{tan}}
\newcommand{\m}{q}
\newcommand{\N}{N}
\newcommand{\ta}{{\tilde{a}}}
\newcommand{\tb}{{\tilde{b}}}
\newcommand{\tc}{{\tilde{c}}}
\newcommand{\tS}{{\tilde S}}
\newcommand{\tP}{{\tilde P}}
\newcommand{\tu}{{\tilde{u}}}
\newcommand{\tw}{{\tilde{w}}}
\newcommand{\tA}{{\tilde{A}}}
\newcommand{\tX}{{\tilde{X}}}
\newcommand{\tphi}{{\tilde{\phi}}}
\renewcommand{\cref}{\Cref} %capitalize option for cleveref doesn't work on all
\synctex=1
\begin{document}
\title{Differential Calculus}
\author{David Karapetyan}
\address{Department of Mathematics  \
University  of Notre Dame\
Notre Dame, IN 46556 }
\date{\today}
%
\maketitle
%
%
\section{Differentiation in Banach Spaces}
We remark that the theory presented below can be extended to arbitrary
topological vector spaces with a topology induced by a metric (in particular,
Fr\'echet spaces) by replacing norms with metrics in what follows.
\begin{definition}
\label{def:diff}
Let $X,Y$ be Banach spaces over the real numbers, $U \subset X$ open,
and consider the map $f: U \to Y$.
Then $f$ is differentiable at $x_0 \in U$ at $x_{0} \in U$ if there
is a continuous linear map $Df(x_0): X \to Y$ such that
%
%
%
%
\begin{equation}
\label{diff-limit}
\begin{split}
\lim_{h \to 0} \frac{\|f(x_0+ h) - f(x_0) -
Df(x_0)(h) \|_Y}{\|h\|_{X}} = 0.
\end{split}
\end{equation}
%
%
This map, which we call the \emph{total derivative} or \emph{Fr\'echet derivative} of $f$ at $t_0$, is 
unique. If $Df(x_0)$ exists for all $x_0 \in U$,
then we say that $f$ is
\emph{differentiable} in $U$. If $f$ is differentiable in $U$, and 
$\|Df(x_0 + h) - Df(x_0) \|_Y \to 0$ as $\|h\|_{X} \to 0$ for all $x_0 \in U$,
then we say that $f$ is \emph{continuously differentiable in $U$}, or $C^{1}$ in $U$. If $X, Y$
are Banach spaces over the complex numbers, we say that $f$ is
\emph{holomorphic} (or \emph{continuously holomorphic}) in $U$.
\end{definition}
%
%
%
\begin{framed}
%
%
\begin{remark}
Equivalently, $f$ is differentiable at $x_{0}$ if there exists a continuous
linear map $Df(x_{0}): X \to Y$ such that
%
%
\begin{equation*}
\begin{split}
f(x_{0} + h) = f(x_{0}) + Df(x_{0}) \circ h + o(h).
\end{split}
\end{equation*}
%
To see this, suppose
%
%
\begin{equation*}
\begin{split}
\lim_{h \to \infty} \frac{\| f(x_{0} + h) - f(x_{0}) - Df(x_{0}) \circ h
\|_{Y}}{\| h \|_{X}} = 0.
\end{split}
\end{equation*}
%
%
Then we can find $\delta > 0$ such that
%
%
\begin{equation*}
\begin{split}
\frac{\| f(x_{0} + h) - f(x_{0}) - Df(x_{0}) \circ h
\|_{Y}}{\| h \|_{X}}  < \ee
\end{split}
\end{equation*}
%
%
for all $h \in X$ with $\| h \|_{X} < \delta$. Hence
%
%
\begin{equation*}
\begin{split}
\| f(x_{0} + h) - f(x_{0}) - Df(x_{0}) \circ h \|_{Y} < \ee \| h
\|_{X}.
\end{split}
\end{equation*}
%
%
This implies
%
%
\begin{equation*}
\begin{split}
f(x_{0} + h) - f(x_{0}) - Df(x_{0}) \circ h = g, \quad g \in Y, \ \| g
\|_{Y} < \ee \| h \|_{X}.
\end{split}
\end{equation*}
%
%
Hence, 
%
%
\begin{equation*}
\begin{split}
f(x_{0} + h) 
& = f(x_{0}) + Df(x_{0}) \circ h + g
\\
& =  f(x_{0}) + Df(x_{0})\circ h + o(\| h \|_{X}).
\end{split}
\end{equation*}
%
%
For the reverse direction, if
%
%
\begin{equation*}
\begin{split}
f(x_{0} + h) = f(x_{0}) + Df(x_{0}) \circ h + o(\| h \|_{X})
\end{split}
\end{equation*}
%
%
then
%
%
\begin{equation*}
\begin{split}
\| f(x_{0} + h) - f(x_{0}) - Df(x_{0}) \circ h \|_{Y} = o(\| h
\|_{X})
\end{split}
\end{equation*}
%
%
which implies
%
%
\begin{equation*}
\begin{split}
\lim_{h \to \infty} \frac{\| f(x_{0} + h) - f(x_{0}) - Df(x_{0}) \circ h
\|_{Y}}{\| h \|_{X}}  = \lim_{h \to \infty} \frac{o(\| h \|_{X})}{\| h
\|_{X}} = 0.
\end{split}
\end{equation*}
%
%
%
\label{rem:equiv-def}
\end{remark}
%
%
\end{framed}
%
%
%
\begin{framed}
%
%
\begin{remark}
Observe that if $F$ is differentiable at $x_{0}$, then it is continuous at $x_{0}$.
\label{rem:deriv-imp-cont}
\end{remark}
%
%
\end{framed}
\begin{framed}
\begin{exercise}
Compute the total
derivative of the function $f:L^{\infty} \to L^{\infty}$ at $u_{0}$,
where $f(u) = u^{2}$. 
\end{exercise}
Answer:       %
%
\begin{equation*}
\begin{split}
\frac{\| (u_{0} + h)^{2} - u_{0}^{2} - 2u_{0}h \|_{L^{\infty}}}{\| h
\|_{L^{\infty}}}
&=         \frac{\| h^{2} \|_{L^{\infty}}}{\| h \|_{L^{\infty}}}
\\
& \le \| h \|_{L^{\infty}} 
\\
& \to 0.
\end{split}
\end{equation*}
%
%
Therefore, $D_{u}(f(u))(u_{0}) = 2u_{0}I$. 
\end{framed}
Recall that if $F$ is continuously differentiable in $U$, then $DF : U \to L(U,
X)$ is a continuous map. Since $X$ is Banach, it follows that $L(U, X)$ is also
Banach under the operator norm, hence it makes sense to talk about derivatives
of $DF$ at a point. More precisely, if $DF$ is differentiable at $x_{0}$, then
we denote its derivative at $x_{0}$ by $D^{2}F(x_{0})$ and say that 
is $F$ is $2$-times differentiable at $x_{0}$.
Observe then that $D^{2}F(x_{0}) \in L(U, L(U,X))$. If $DF$ is continuously
differentiable in $U$, then we say that $F$ is $C^{2}$ in $U$, and observe that
$D^{2}F : U \to L(U, L(U, X))$ continuously. In general, we say $F$ $k$-times
differentiable at $x_{0}$ if $D^{k-1}F$ is differentiable at $x_{0}$, and
observe that $D^{k}F(x_{0}) \in L^{k}(U, X)$, where $L^{k}(U, X)$ is
defined recursively via the relations $L^{0}(U, X) \doteq L(U, X)$, $L^{1}(U,
X)$ = $L(U, L^{0}(U, X))$, \ldots, $L^{k}(U, X)$ = $L(U, L^{k-1}(U, X))$.
If $D^{k-1}F$ is
continuously differentiable in $U$, then we say $F$ is $C^{k}$ in $U$. 

Notice that our definition of $D^{k}F$ becomes more and more complicated as $k$
increases. However, there is a special case in which our definition of the
Banach space derivative becomes easier to deal with. More precisely, for an
arbitrary Banach space $X$, open interval $(a,b) \subset \rr$, and map $f:(a,b)
\to X$, we can identify $Df(x_0)$ with an element of $X$ via the following.
%
%
%%%%%%%%%%%%%%%%%%%%%%%%%%%%%%%%%%%%%%%%%%%%%%%%%%%%%
%
%
%			Lemma Isometry
%
%
%%%%%%%%%%%%%%%%%%%%%%%%%%%%%%%%%%%%%%%%%%%%%%%%%%%%%
%
%
\begin{lemma}
\label{lem:isometry} Let $(a,b) \subset \rr$ be an open interval, $X$ a Banach
space, with $x \in X$. Define the map $T_x \in L\left ( \rr , X \right )$ by
$T_x(t_0) = x t_0$. Then the map $x \mapsto T_x$ is an
isometric isomorphism from
$X$ to $L(\rr , X)$. 
\end{lemma}
%
%
\begin{proof} Note that 
%
%
\begin{equation*}
\begin{split}
| | | T_x | | |
& = \sup_{|t_0| = 1} \| T_x (t_0) \|_X
= \| x t_0\|_X
= \|x\|_X.
\end{split}
\end{equation*}
%
%
Hence, the map $x \mapsto T_x$ is an isometry from $X$ into $L(\rr,
X)$. It remains to show that it is onto. Let $V \in L( \rr, X)$. Then
by linearity
%
%
\begin{equation*}
\begin{split}
V(t_0) = V(1)t_0. 
\end{split}
\end{equation*}
%
%
Hence, $V = T_{V(1)}$, completing the proof. 
\end{proof}
%
%
Applying the lemma, we see that if $Df(t_0)$ exists for a map $f: (a,b) \to X$,
then it can be viewed as an
element of $X$. Similarly, if $Df(t_0)$ exists for all $t_0 \in (a,b)$, then
we may view the map $t_0 \to Df(t_0)$ as an
element of $L( \rr, X)$. Hence, for a
map $f:(a,b) \to X$, we see that the following is an equivalent
reformulation of \autoref{def:diff}. 
\begin{definition}
\label{def:diff-simp}
Let $X$ be Banach space, $(a,b) \subset \rr$ an open interval, and
consider the map $f: (a,b) \to X$.
Then $f$ is \emph{differentiable at $t_0$} if there exists a map
$f': \rr \to X$ such that 
%
%
%
%
\begin{equation}
\label{diff-limit-simp}
\begin{split}
\lim_{h \to 0} \| \frac{f(t_0+ h) - f(t_0) 
}{h} - f'(t_0) \|_X = 0.
\end{split}
\end{equation}
%
%
We call $f'(t_0)$ the \emph{derivative of $f$ at $t_0$}.
If \eqref{diff-limit-simp}
holds for all $t_0 \in (a,b)$, then we say $f$ is \emph{differentiable in
$(a,b)$}, and call $f'$ the
\emph{derivative of $f$ in $(a,b)$}.  
\end{definition}
\begin{framed}
\begin{example}
Let $f: (a,b) \to \rr^2$ be defined by $f(t) = (1, t^2)$. Then $Df$ maps the
interval $(a,b)$ to $L((a,b), \rr^{2})$ by the relation $Df(t) = (0, 2t)$, and
$Df(t)(h) = (0,2th)$. Hence, we may view $Df(t)$ as the point $(0,2t) \in
\rr^2$, and view its action on $(a,b)$ as nothing more than standard scalar
multiplication in $\rr^{2}$. That is, $f'(t) = (0,2t)$, and $f'(t)(s) = s(0,
2t)$. More generally, composition can be \emph{identified} with the dot product in certain contexts. 
\end{example}
\end{framed}
We also have the following important result, which will make higher order
Fre\`chet derivatives of functions easier to deal with.
%
%
%%%%%%%%%%%%%%%%%%%%%%%%%%%%%%%%%%%%%%%%%%%%%%%%%%%%%
%
%
%			isomorphism
%
%
%%%%%%%%%%%%%%%%%%%%%%%%%%%%%%%%%%%%%%%%%%%%%%%%%%%%%
%
%
\begin{lemma}
Let $U_{1}, U_{2}, \cdots,
U_{k}$ open subsets of the Banach spaces $X_{1}, X_{2}, \cdots, X_{k}$
respectively. Let $Y$ be Banach. 
Then the space of continuous $k$-linear maps $L(X_{1}, X_{2}, \cdots, X_{k};
Y)$ is isometrically isomorphic to $L(X_{1}, L(X_{2}, \cdots L(X_{k-1},
X_{k}))))$.
\label{lem:iso}
\end{lemma}
%
%
%
\begin{proof}
We handle the case $k=2$; the proof for arbitrary $k$ is analogous. Proceeding,
for each $u \in L(X_{1}, X_{2}; Y)$, we constructed the bounded linear map
$u_{x} \in L(X_{2}, Y)$ where $u_{x_{1}}(x_{2}) = u(x_{1}, x_{2})$. For
$u_{x}$, we in turn construct the linear bounded map $\tilde{u}: L(X_{1},
L(X_{2}, Y))$ where $\tilde{u}(x_{1}) = u_{x_{1}}$. We claim the map $f:
L(X_{1}, X_{2}; Y) \to L(X_{1}, L(X_{2}, Y))$ given by $f(u) = \tilde{u}$ is an
isometry and onto.
\begin{enumerate}[i)]
\item{}Onto: Let $v \in L(X_{1}, L(X_{2}, X_{3}))$. Then it is enough to
show that the map $(x_{1}, x_{2}) \to [v(x_{1})](x_{2}) \doteq
u(x_{1}, x_{2})$ is bilinear and continuous. Observe that
%
%
\begin{equation*}
\begin{split}
u(ax_{1} + b, x_{2} )
& = [v(ax_{1} + b)](x_{2})
\\
& = [av(x_{1}) + v(b)](x_{2})
\\
& = [av(x_{1})](x_{2}) + [v(b)](x_{2})
\\
& = a[v(x_{1})](x_{2}) + [v(b)](x_{2})
\\
& = au(x_{1}, x_{2}) + u(b, x_{2}).
\end{split}
\end{equation*}
%
%
Similarly, 
%
%
\begin{equation*}
\begin{split}
u(x_{1}, ax_{2} + b) = au(x_{1}, x_{2}) + u(b, x_{2})
\end{split}
\end{equation*}
%
%
and so $u$ is bilinear. For the boundedness, we observe that
%
%
\begin{equation*}
\begin{split}
\| u(x_{1}, x_{2}) \|_{Y} 
& = \| [v(x_{1})](x_{2}) \|_{Y}
\\
& \le \| v(x_{1}) \|_{L(X_{2}, Y)} \| x_{2} \|_{X_{2}}
\\
& \le \| v \|_{L(X_{1}, L(X_{2}, Y))} \| x_{1} \|_{X_{1}} \| x_{2}
\|_{X_{2}}
\\
& = C\| x_{1} \|_{X_{1}} \| x_{2}
\end{split}
\end{equation*}
%
%
\item{}Isometry:
%
%
\begin{equation*}
\begin{split}
\| \tilde{u} \|_{L(X_{1}, L(X_{2}, Y))}
& = \sup_{\| x_{1} \|_{X_{1}} = 1} \| \tilde{u} \|_{L(X_{2}, Y)}
\\
& = \sup_{\| x_{1} \|_{X_{1}} = 1} \| \tilde{u}_{x} \|_{L(X_{2},
Y)}
\\
& = \sup_{\| x_{1} \|_{X_{1}} = 1, \| x_{2} \|_{X_{2}} = 1} \|
u(x_{1}, x_{2}) \|_{Y}
\\
& = \| u \|_{L(X_{1}, X_{2}; Y)}
\end{split}
\end{equation*}
%
%
which concludes the proof. 

\end{enumerate}
\end{proof}
%
%
\begin{framed}
In particular, if $f: U \to X$ is $C^{k}$, where $X$ is Banach and $U
\subset X$ is open, \cref{lem:iso} implies
%
%
\begin{equation*}
\begin{split}
D^{k}f = L^{k}(U, X) \cong L(\underbrace{U, U, \cdots, U}_{k \ \text{copies}}; X).
\end{split}
\end{equation*}
%
This will prove useful later on when computing Fr\`echet derivatives.
%
\end{framed}
%
Lastly, we include the following.
%
%
\begin{lemma}[Chain Rule]
Let $X,Y,Z$ be Banach spaces, $f: X \to Y$ and $g: Y \to Z$
continuously differentiable maps. Then for all $x_0 \in X$ we have
%
\begin{equation*} (g \circ f)' (x_0) = g'(f(x_0)) \circ (f'(x_0)).
\end{equation*} 
%
\end{lemma}
\begin{proof} Since $f$ and $g$ are continuously differentiable, we can write
\begin{equation*}
f(x_0 + s) = f(x_0) + f'(x_0)(s) + o_{1}(s)
\end{equation*}
and
\begin{equation*}
g(x_0 + t) = g(x_0) + g'(x_0)(t)+ o_{2}(t).
\end{equation*}
Hence
\begin{equation}
\label{pre-order}
\begin{split}
h(x_0 + s) &= g(f(x_0 +s)
\\
&= g(f(x_0) + f'(x_0)(s) + o_{1}(s))
\\
&= g(f(x_0)) + g'(f(x_0)) \circ [f'(x_0)(s) + o_{1}(s)] 
+ o_{2}(f'(x_0)(s) + o_{1}(s))
\\
&= g(f(x_0)) + g'(f(x_0)) \circ f'(x_0)(s) +
g'(f(x_0)) \circ o_{1}(s) + o_{2}(f'(x_0)(s) + o_{1}(s)).
\end{split}
\end{equation}
Since $f'(x_0)$
and $g'(f(x_0))$ are continuous linear operators, we have
\begin{equation*}
\begin{split}
&  f'(x_0)(s)
= o_{3}(s)
\\
& g'(f(x_0))(t) = o_{4}(t).
\end{split}
\end{equation*}
Therefore,
%
%
\begin{equation*}
\begin{split}
g'(f(x_0)) \circ o_{1}(s)
& = o_{4}(o_{1}(s))
\\
& = o_{5}(s)
\end{split}
\end{equation*}
%
%
and
%
%
\begin{equation*}
\begin{split}
o_{2}(f'(x_0)(s) + o_{1}(s))
& = o_{2}(o_{3}(s) + o_{1}(s))
\\
& =o_{6}(s).
\end{split}
\end{equation*}
%
%
Substituting, we conclude that
\begin{equation*}
\begin{split}
h(x_0 + s) = g(f(x_0)) + g'(f(x_0))(f'(x_0)(s)) +
o(s)
\end{split}
\end{equation*}
completing the proof. 
\end{proof}
%
%
%%%%%%%%%%%%%%%%%%%%%%%%%%%%%%%%%%%%%%%%%%%%%%%%%%%%%
%
%
%				product rule
%
%
%%%%%%%%%%%%%%%%%%%%%%%%%%%%%%%%%%%%%%%%%%%%%%%%%%%%%
%
%
%\begin{lemma}[Product Rule]
%Let $X$, $Y$, $Z$ be Banach spaces, and $B: X \times Y \to Z$ is a continuous
%bilinear operator, where $X \times Y$ is Banach under the usual product
%norm $\| (f,g) \|_{X \times Y} = \| f \|_{X} + \| g \|_{Y}$.  Then $B$ is differentiable in $X \times Y$, with derivative at $(x_{0}, y_{0})$ given
%by the bilinear map $DB(x_{0}, y_{0}): X \times Y \to Z$ defined by
%%
%%
%\begin{equation}
%\label{ansatz}
%\begin{split}
%DB(x_{0},y_{0})(x,y) = B(x, y_{0}) + B(x_{0}, y)
%\end{split}
%\end{equation}
%%
%%
%\label{lem:product-rule}
%\end{lemma}
%%
%%
%%
%%
%\begin{proof}
%%
%It will be enough to verify that \eqref{ansatz} is the derivative of $B$.
%Using the bilinearity of $B$, we see that 
%%
%\begin{equation*}
%\begin{split}
%\frac{\| B(x_{0} + h, y_{0} + k) - B(x_{0}, y_{0})  - B(h, y_{0}) - B(x_{0}, k)\|_{Z}}{\| (h, k) \|_{X \times Y}}
%& = \frac{\| B(h, k)\|_{Z}}{\| (h, k) \|_{X \times Y}}
%\\
%& \le \frac{\| h \|_{X} \| k \|_{Y} ||| B |||}{\| (h, k) \|_{X \times Y}}
%\\
%& \simeq \frac{\| h \|_{X} \| k \|_{Y}}{\| (h, k) \|_{X \times Y}}
%\end{split}
%\end{equation*}
%%
%which goes to $0$ as $\| (h, k) \|_{X \times Y}$ goes to $0$. This concludes the proof.
%%
%\end{proof}
%%
%%
%\begin{framed}
%\begin{example} Let $u \in L^{\infty}$. Then multiplication is a continuous bilinear mapping taking $L^{\infty} \times L^{\infty}$ to $L^{\infty}$. Hence
%%
%%
%\begin{equation*}
%\begin{split}
%D(u^{2})(h, k) 
%\end{split}
%\end{equation*}
%%
%%
%
%\end{example}
%\end{framed}
%%%%%%%%%%%%%%%%%%%%%%%%%%%%%%%%%%%%%%%%%%%%%%%%%%%%%
%
%
%                Dependence on Params
%
%
%%%%%%%%%%%%%%%%%%%%%%%%%%%%%%%%%%%%%%%%%%%%%%%%%%%%%
%
%
%
\section{Regularity of the Data-to-Solution Map for ODEs} 
\label{sec:dep-param-smooth}
%
%
Following Taylor \cite{Taylor:1995kx}, let $X$ be a Banach space, $U \subset X$ an open subset, and $F: U \to X$. Consider the non-autonomous system
%
%
\begin{gather}
\label{aa-sm}
\frac{dy}{dt} = F(y(t), t),
\\
y(0)= y_{0} \in X, \quad t \in \rr.
\label{bb-sm}
\end{gather}
%
Letting $z = (y(t), t)$, we get
%
%
\begin{equation*}
\begin{split}
\frac{dz}{dt}  
& = \left (\frac{dy}{dt}, 1 \right )
\\
& = \left( F(y(t), t), 1 \right)
\\
& = \left( F(z), 1 \right)
\\
& \doteq G(z).
\end{split}
\end{equation*}
%
%
Hence, without loss of generality, we may restrict our attention to the
autonomous system
%
%
\begin{gather}
\frac{dy}{dt} = F\left( y(t) \right)
\label{ode-eq-sm}
\\
y(0) = y_{0}, \quad t \in \rr
\label{ode-init-data-sm}
\end{gather}
%
%
%
for which we shall prove the following.
%
%
%%%%%%%%%%%%%%%%%%%%%%%%%%%%%%%%%%%%%%%%%%%%%%%%%%%%%
%
%
%                dep on init cond
%
%
%%%%%%%%%%%%%%%%%%%%%%%%%%%%%%%%%%%%%%%%%%%%%%%%%%%%%
%
%
\begin{proposition}
Let $U \subset X$ be an open subset, and $F: U  \to X$.
If $F$ is Lipschitz, then the flow map $\vp \mapsto u(t, \vp) \doteq
S_{t}(\vp)$ is Lipschitz. If $F$ is $C^{k}$, where $k \in \mathbb{N} \cup
\{\infty \}$, then the flow map is $C^{k}$.
\label{prop:reg-result}
\end{proposition}
%
%
%
%
\begin{proof}
We divide our work into cases.
\subsection*{Case $k = 0$.} The ivp \eqref{ode-eq-sm}-\eqref{ode-init-data-sm} is equivalent to 
%
%
\begin{equation*}
\begin{split}
u(t, \vp) = \vp + \int_{0}^{t} F[u(\tau, \vp)] d \tau
\end{split}
\end{equation*}
%
and so
%
%
%
\begin{equation*}
\begin{split}
u(t, \vp) - u(t, \psi) = \vp - \psi + \int_{0}^{t} \{F[u(\tau, \vp)] - F[u(\tau, \psi)]\} d \tau.
\end{split}
\end{equation*}
%
Recall that the Metric Space ODE Theorem tells us that for any $\vp \in B_{r} \subset U$ there exists a unique solution $u(t, \vp)$ for $t \in [-h, h]$, where $h = h(r)$. Therefore, applying the triangle inequality and the Lipschitz condition, we obtain the estimate
\begin{equation*}
\begin{split}
\sup_{t \in [-h, h]} \|u(t, \vp) - u(t, \psi)\|_{L^{\infty}}
& \le \|\vp -
\psi\|_{L^{\infty}} +  h \sup_{t \in [-h, h]} \|F[u(t, \vp)] - F[u(t, \psi)] \|_{L^{\infty}}
\\
& \le \|\vp - \psi\|_{L^{\infty}} +  ch \sup_{t \in [-h, h]} \|[u(t, \vp)] -
[u(t, \psi)] \|_{L^{\infty}}.
\end{split}
\end{equation*}
Restricting $h < 1/c$ and rearranging terms, we obtain
%
%
\begin{equation*}
\begin{split}
\sup_{t \in [-h, h]} \| u(t, \vp) - u(t, \psi) \|_{L^{\infty}} \le \frac{1}{1-ch} \| \vp - \psi \|_{L^{\infty}}
\end{split}
\end{equation*}
%
%
concluding the proof.
\subsection*{Case $k=1$} 
\label{ssec:case-k}
% 
%
Formally differentiate
\eqref{ode-eq-sm}-\eqref{ode-init-data-sm} with respect to $\vp$ and obtain
%
%
%
\begin{gather}
W' = DF(y(\vp,t)) \circ W
\label{ode-lin}
\\
W(0) = I
\label{ode-lin-init-data}
\end{gather}
%
%
where $W \doteq D_{\vp} y(\vp,t)$ and $y$ is the unique solution to
\eqref{ode-eq-sm}-\eqref{ode-init-data-sm}. Let $w(\vp,t) = D_{\vp}y(\vp,t) \circ h$, where $h \in U$ is
arbitrary. Then
from \eqref{ode-lin}-\eqref{ode-lin-init-data} we
obtain
%
%
\begin{gather}
w' = DF(y(\vp, t)) \circ w 
\label{ode-lin-h}
\\
w(0) = h.
\label{ode-lin-init-data-h}
\end{gather}
%
%
Since $F$ is $C^{1}$, $DF(y(\vp, t)): U \to X$ continuously. Since $Df(y(\vp,
t))$ is also linear, it is bounded, and hence Lipschitz on $U$. Therefore, by
the Metric Space ODE theorem, there exists a unique solution $w = w(\vp,t)$ to
\eqref{ode-lin-h}-\eqref{ode-lin-init-data-h} for $t$ in some open subset of
$\rr$. We now show that this solution is in fact $D_{\vp} y(\vp,t)$ as follows.
Let $z(\vp,t) = y(\vp + h, t) - y(\vp,t)$. Note that
%
%
\begin{equation}
\label{uhh}
\begin{split}
\frac{dz}{dt} 
& = F\left[ y(\vp + h, t) \right] - F\left[ y(\vp,t) \right]
\\
& = \int_{0}^{1} \frac{d}{ds} F\left[ sy(\vp+h, t) + (1-s)y(\vp,t) \right]ds
\\
& = \int_{0}^{1} \frac{d}{ds} F\left[ sz(\vp,t) + y(\vp,t) \right] ds
\\
& = F[y(\vp,t) +z(\vp, t)] - F[y(\vp, t)]
\end{split}
\end{equation}
%
%
which by the definition of the Banach space derivative implies
%
%
\begin{equation*}
\begin{split}
\frac{dz}{dt} = DF(y) \circ z + R_{1}(y) \circ z, \quad R_{1}(y) \circ z = o(\|
z(t) \|_{X})
\end{split}
\end{equation*}
%
%
and so
%
%
\begin{equation*}
\begin{split}
\frac{d}{dt}(z-w) = DF(y) \circ (z-w) + R_{1}(y) \circ z.
\end{split}
\end{equation*}
%
%
Noting that %
%
\begin{equation*}
\begin{split}
(z-w)(0) & = 0
\end{split}
\end{equation*}
%
%
we now seek to analyze the the inhomogeneous ode
%
%
\begin{gather}
\frac{d}{dt}(z-w)  = DF(y) \circ (z-w) + R_{1}(y) \circ z
\label{inhom-ode}
\\
(z-w)(0) = 0.
\label{inhom-ode-init}
\end{gather}
%
Since $DF(y)$ and $R_{1}(y)$ are Lipschitz on $U$, the Metric Space ODE theorem
guarantees a unique solution to this equation for $t$ in a sufficiently small
interval. Taking norms of both sides, and using the fact that $DF(y)$ is
Lipschitz, we obtain
%
%
%
%
\begin{gather}
\label{yt}
\frac{d}{dt} \| z - w \|_{X} \lesssim  \| z - w \|_{X}  + o(\|z\|_{X}),
\\
(z-w)(0) = 0.
\label{ytt}
\end{gather}
%
We now bound $\| z \|_{X}$. From the
second line of \eqref{uhh}, we get
%
%
\begin{equation*}
\begin{split}
\frac{d}{dt} \| z(t) \|_{X}
& \le \| \int_{0}^{1} \frac{d}{ds} F\left[ sy(\vp+h, t) + (1-s)y(\vp,t) \right]ds
\|_{X}
\\
& \le \| \int_{0}^{1} DF\left[  sy(\vp+h, t) + (1-s)y(\vp,t) \right] \circ z ds
\|_{X}
\\
& \le \| z(t) \|_{X} | | | DF | | |_{L(U, X)}
\\
& = M \| z(t) \|_{X}
\end{split}
\end{equation*}
%
where we emphasize that $M$ does not depend on $h$, $z$, or $w$.
Hence, setting $\vp(t) =  \| z(t) \|_{X}$, we wish to solve
%
%
\begin{gather*}
\frac{d \vp}{dt} = M \vp
\\
\vp(0) = \| h \|_{X}
\end{gather*}
%
%
which gives
%
%
\begin{equation*}
\begin{split}
\vp(t) & = \vp(0)e^{Mt}
\\
& = \| h \|_{X} e^{Mt}
\\
& \lesssim \| h \|_{X}, \quad 0 \le t < T.
\end{split}
\end{equation*}
%
%
Hence,
%
%
\begin{equation*}
\begin{split}
\| z(t) \|_{X}  \lesssim  \| h \|_{X}  \ \text{for all } \ t \in I.
\end{split}
\end{equation*}
Applying this relation to the Cauchy problem \eqref{yt}-\eqref{ytt}, we obtain
%
\begin{gather*}
\frac{d}{dt} \| z - w \|_{X} \lesssim  \| z - w \|_{X}  + o(\| h \|_{X}),
\\
(z-w)(0) = 0
\end{gather*}
which by Gronwall's inequality gives
%
%
%
%
%
%
%
\begin{equation*}
\begin{split}
\| z(t) - w(t) \|_{X} 
= o(\| h \|_{X}),
\end{split}
\end{equation*}
%
%
that is
%
%
\begin{equation*}
\begin{split}
\| y(\vp + h, t) - y(\vp,t) - w(\vp,t) \|_{X} = o(\| h \|_{X}).
\end{split}
\end{equation*}
%
%
Hence, by the definition of the Banach space derivative, $D_{\vp}y(\vp,t)$
exists and is equal to $w(\vp,t)$. The proof for the case $k=1$ is complete.
\subsection*{Case $k >1$} 
\label{ssec:case-kg1}
For finite $k$, formally differentiating
\eqref{ode-eq-sm}-\eqref{ode-init-data-sm}
$k$ times with respect to $\vp$, we obtain
\begin{gather*}
W' = DG(y) \circ W
\\
W(0) = I
\end{gather*}
where $G \doteq D^{k-1}F$, $W \doteq D_{\vp}^{k} y$. Hence, the proof of
$k$-time differentiability in $x$ reduces to the proof in the case $k=1$, which
completes the proof for finite $k$. If $k = \infty$, we run an induction on $k$ to
obtain the corresponding result.
\end{proof}
%
%
%
%
%
%
%
\section{The BBM Equation} 
\label{sec:}
Setting $\gamma = 0$ in the HR equation, we obtain the Benjamin, Bona,
Mahoney (BBM) equation. That is
%
%
\begin{gather*}
\p_{t} u = F(u(t))
\\
u(x,0) = u_0(x), \; \; x \in \rr, \; \; t \in \rr
\end{gather*}
where
%
%
\begin{equation*}
\begin{split}
F(u(t)) \doteq -\frac{3}{2} \p_{x}(1 - \p_{x}^{2})^{-1} u^{2}.
\end{split}
\end{equation*}
%
Observe that $F$ maps open sets in $C^{k}$ to $C^{k}$,  $k \in \mathbb N \cup
\{\infty \}$, and is hence an ODE on the space of $C^{k}$ functions. Following
\cite{Bona_2009_Sharp-well-pose}, we note that the BBM is also an ODE on
$H^{s}$, $s \ge 0$ due to the following bilinear estimate.
%
%
%%%%%%%%%%%%%%%%%%%%%%%%%%%%%%%%%%%%%%%%%%%%%%%%%%%%%
%
%
%				bbm bilinear estimate
%
%
%%%%%%%%%%%%%%%%%%%%%%%%%%%%%%%%%%%%%%%%%%%%%%%%%%%%%
%
%
\begin{lemma}
For $s \ge 0$
%
%
\begin{equation*}
\begin{split}
\| \p_{x}(1 - \p_{x}^{2})^{-1} (uv) \|_{H^{s}} \lesssim \| u \|_{H^{s}} \| v \|_{H^{s}}.
\end{split}
\end{equation*}
%
%
\label{lem:bbm-bilin-est}
\end{lemma}
%
%
%
\begin{proof}
We diverge slightly from the proof in \cite{Bona_2009_Sharp-well-pose}. Observe that
\begin{equation*}
\begin{split}
\| \p_{x}(1 - \p_{x}^{2})^{-1} (uv) \|_{H^{s}}
& = \left [ \int_{\rr} \frac{\xi^{2}}{(1 + \xi^{2})^{2}} (1 + |\xi|)^{2s} | \wh{u} * \wh{v}(\xi) |^{2} d \xi \right ]^{1/2}
\\
& = \left [ \int_{\rr} \frac{\xi^{2}}{(1 + \xi^{2})^{2}} | (1 + |\xi|)^{s}\wh{u} * \wh{v}(\xi) |^{2} d \xi \right ]^{1/2}
\\
& \le \| \xi^{2}/(1 + \xi^{2})^{2} \|_{L^{2}}^{1/2} \| (1 + |\xi|)^{s} \wh{u} * \wh{v}(\xi) \|_{L^{\infty}}
\\
& \simeq \| (1 + |\xi|)^{s} \wh{u} * \wh{v}(\xi) \|_{L^{\infty}}.
\end{split}
\end{equation*}
%
Applying Peetre's inequality (valid only for $s \ge 0$), followed by Young's inequality and Parseval, we bound this by
%
%
%
%
\begin{equation*}
\begin{split}
\| (1 + |\cdot| )^{s}\wh{u}(\cdot) * (1 + |\cdot| )^{s}\wh{v}(\cdot)(\xi) \|_{L^{\infty}} 
& \lesssim \| (1 + |\xi| )^{s}\wh{u}(\xi)\|_{L^{2}} \| (1 + |\xi| )^{s}\wh{v}(\xi) \|_{L^{2}}
\\
&  = \| u \|_{H^{s}} \| v \|_{H^{s}}
\end{split}
\end{equation*}
%
%
which concludes the proof.
%
\end{proof}
%
%
%
%
%
%
By the ODE theory we have covered, for $u_{0}(x) \in C^{k}$, the BBM admits a
unique solution $u(x,t)$ which is $C^{k}$ in the space variable for $0\le t <
T$, where $T > 0$ is sufficiently small. (In fact, classical ODE theory
guarantees that it is $C^{k}$ in time as well, but this is outside our
discussion). Furthermore, by the same theory, we see that for $u_{0}(x) \in
H^{s}$, $s \ge 0$, the BBM admits a unique solution $u(x,t) \in C^{1}([0, T],
H^{s})$ for sufficiently small $T > 0$. In both cases, we claim that the data to
solution map is smooth. To prove this, by the ODE theory it will suffice to show
that $F$ is smooth. Adopt
the notation
\begin{gather*}
F(u) = -\frac{3}{2}Lf(u), \ \text{where} \ 
\\
\Lambda \doteq \p_{x}(1 - \p_{x}^{2})^{-1}, \ f(u) = u^{2}.
\end{gather*}
Note that $\Lambda$ is a bounded linear operator on both $C^{k}$ and $H^{s}$,
and hence $D \Lambda(f) = \Lambda$ for $f$ in $C^{k}$ or $H^{s}$. Applying the chain
rule, we see that for $h \in H^{s}$ or $h \in C^{k}$ we get
%
%
\begin{equation*}
\begin{split}
DF(u) & = -\frac{3}{2}D \Lambda (f(u))Df(u)\\
& = -\frac{3}{2} \Lambda (2uI_{1})
\\
& = -3 \Lambda (uI_{1})
\end{split}
\end{equation*}
%
%
where $\Lambda (uI_{1}) \in L(H^{s}, H^{s})$ via the relation
$\Lambda (uI_{1})(h_{1}) = \Lambda (u h_{1})$ and as a consequence of 
\cref{lem:bbm-bilin-est}. 
%
We now compute the second derivative of $F$ at $u_{0}$. Applying the chain rule, we have
%
%
\begin{equation*}
\begin{split}
D^{2}_{u}F(u_{0}) 
& = D_{u} \left[ \Lambda (u I_{1}) \right](u_{0})
\\
& = (D \Lambda)(u_{0}) \circ \left[ D_{u}(u I_{1}) \right](u_{0})
\\
& = \Lambda \circ I_{2}I_{1}
\end{split}
\end{equation*}
%
%
where $\Lambda \circ I_{2} I_{1}: H^{s} \to L(H^{s}, H^{s}))$ via the
relation
%
%
\begin{equation*}
\begin{split}
\Lambda \circ I_{2} I_{1} (h_{2}) = \Lambda \circ h_{2} I_{1}.
\end{split}
\end{equation*}
%
Alternatively, due to \cref{lem:iso}, we may view $\Lambda \circ I_{2} I_{1}$ as
a member of $L(H^{s}, H^{s}; H^{s})$, where $\Lambda \circ I_{2} I_{1}(h_{1},
h_{2}) = \Lambda (h_{1}h_{2})$. Observe that $D_{u}(I_{2} I_{1})(u_{0}) = 0$, since $I_{2} I_{1}$ does not depend on $u$. Hence, by the
chain rule
%
%
\begin{equation*}
\begin{split}
D^{k}_{u} F(u) = 0 \quad k \ge 3.
\end{split}
\end{equation*}
%
%
%
%We now run an induction. Assume $F$ is $k$-times differentiable. Then by the
%Leibniz rule
%%
%%
%\begin{equation*}
%\begin{split}
%D^{k+1}F(u)(h_{1}, h_{2}, \dots, h_{k+1})
%& = -3 D^{k} L(uI)(h_{1}, h_{2}, h_{3}, \dots, h_{k+1})
%\\
%& = -3 \sum_{j = 0}^{k} {k \choose j} D^{j}u D^{k - j}F(u)
%\\
%& = 2 \left[ u D^{k}F(u) + \sum_{j=1}^{k} {k \choose j} I D^{k-j}F(u)
%\right].
%\end{split}
%\end{equation*}
%%
%which closes the inductive loop.
We conclude that $F$ is smooth. In fact, it is real analytic, since it has only
finitely many derivatives that do not vanish. 
%That is
%%
%%
%\begin{equation*}
%\begin{split}
%F(u + h) & = \sum_{n = 0}^{\infty} \frac{D^{n}F(u)(h^{n})}{n!}
%\\
%& = 
%\end{split}
%\end{equation*}
%
%
\begin{framed}
Notice that the map $(u, v) \to \Lambda (uv)$ is bilinear. In general, for
any $f \in L(U_{1}, U_{2}, \cdots, U_{k}; Y)$, where $U_{i} \subset X_{1}$
open, $X_{i}$ Banach, and $Y$ Banach, we have $D^{n}f = 0$ for $n \ge k+1$.
For more details, see \cite{Dieudonne_1969_Foundations-of-}. For an 
example in the finite dimensional case, consider the bi-linear map $f: \rr^{2} \to \rr$ given by $f(x, y) = xy$. Then $D^{2}(f)(x,y)$ is equivalent to the Hessian at $(x,y)$, which is given by
%
%
\begin{equation*}
\begin{split}
H(x,y) = 
\begin{bmatrix}
f_{xx}(x,y) & f_{(x,y)}(x,y)
\\
f_{yx}(x,y) & f_{yy}(y,y)
\end{bmatrix}
= 
\begin{bmatrix}
0 & 1 
\\
1 & 0. 
\end{bmatrix}
\end{split}
\end{equation*}
%
Since $H$ is a constant matrix (i.e.\ not a function of $(x,y)$), it follows
that $D^{3}(f)(x,y) = D(H)(x,y) = 0$. 
%
\end{framed}
%
Therefore, the data-to-solution map for the BBM equation is smooth (in fact
analytic) for $C^{k}$ and $H^{s}$ initial data, $s \ge 0$. Hence, the method of
proof developed to prove
non-uniform dependence for HR does not apply to the BBM.
%
%
%
\nocite{Dieudonne_1969_Foundations-of-}
\nocite{Jost-1998-Postmodern-analysis}
\nocite{Yosida:1980fk}
%\nocite{*}
\bibliography{/Users/davidkarapetyan/math/bib-files/references}
\end{document}
