%
\documentclass{beamer}
\usetheme{Berkeley}
\let\Tiny=\tiny
\usepackage{amscd}
\usepackage{amsfonts}
\usepackage{amsmath}
\usepackage{amssymb}
\usepackage{amsthm}
\usepackage{fancyhdr}
\usepackage{latexsym}
\input epsf
\input texdraw
\input txdtools.tex
\input xy
\xyoption{all}
\usepackage{color}
\synctex=1

\definecolor{Red}{rgb}{1.00, 0.00, 0.00}
\definecolor{DarkGreen}{rgb}{0.00, 1.00, 0.00}
\definecolor{Blue}{rgb}{0.00, 0.00, 1.00}
\definecolor{Cyan}{rgb}{0.00, 1.00, 1.00}
\definecolor{Magenta}{rgb}{1.00, 0.00, 1.00}
\definecolor{DeepSkyBlue}{rgb}{0.00, 0.75, 1.00}
\definecolor{DarkGreen}{rgb}{0.00, 0.39, 0.00}
\definecolor{SpringGreen}{rgb}{0.00, 1.00, 0.50}
\definecolor{DarkOrange}{rgb}{1.00, 0.55, 0.00}
\definecolor{OrangeRed}{rgb}{1.00, 0.27, 0.00}
\definecolor{DeepPink}{rgb}{1.00, 0.08, 0.57}
\definecolor{DarkViolet}{rgb}{0.58, 0.00, 0.82}
\definecolor{SaddleBrown}{rgb}{0.54, 0.27, 0.07}
\definecolor{Black}{rgb}{0.00, 0.00, 0.00}
\definecolor{dark-magenta}{rgb}{.5,0,.5}
\definecolor{myblack}{rgb}{0,0,0}
\definecolor{darkgray}{gray}{0.5}
\definecolor{lightgray}{gray}{0.75}

\newcommand{\tf}{\tilde{f}}
\newcommand{\ti}{\tilde}
\newcommand{\bigno}{\bigskip\noindent}
\newcommand{\ds}{\displaystyle}
\newcommand{\medno}{\medskip\noindent}
\newcommand{\smallno}{\smallskip\noindent}
\newcommand{\nin}{\noindent}
\newcommand{\ts}{\textstyle}
\newcommand{\rr}{\mathbb{R}}
\newcommand{\p}{\partial}
\newcommand{\zz}{\mathbb{Z}}
\newcommand{\cc}{\mathbb{C}}
\newcommand{\ci}{\mathbb{T}}
\newcommand{\ee}{\varepsilon}
\newcommand{\vp}{\varphi}
\def\refer #1\par{\noindent\hangindent=\parindent\hangafter=1 #1\par}

%% Equation Numbers %%

\renewcommand{\theequation}{\thesection.\arabic{equation}}

%
%
\begin{document}
%
%
\date{\today}
%
%
\title{On the Uniqueness of Solutions to the Burgers Equation in Sobolev Spaces}
\author{{\it David Karapetyan}}
\begin{frame}
	\titlepage
\end{frame}
%
%
%
%
%%%%%%%%%%%%%%%%%%%%%%%%
%
%      introduction
%
%%%%%%%%%%%%%%%%%%%%%%%%

%
%
\begin{frame}
	\frametitle{ The Burgers Cauchy Problem}
	
Let $u,\omega \in C(I, H^s(\ci)), \ s>3/2$ be two solutions to the
Cauchy-problem 
%
%
\begin{align}
	\label{burgers-equation}
		& \p_t u = -u \p_x u
		\\
		\label{init-cond}
		& u(x,0) = u_0, \quad x \in \ci
\end{align}
%
%
with common initial data. Let $v=u-w$; then $v$ solves the Cauchy-problem
%
%
\begin{align}
	\label{uniqueness-exp}
 & \p_t v  =  -\frac{1}{2} \p_x [v(u + w)],
\\
\label{uniqueness-init-data}
& v(x,0) = 0.
\end{align}
%
%
%
%
Applying $D^\sigma$ to both sides of \eqref{uniqueness-exp}, then 
multiplying both sides by $D^\sigma v$ and integrating, we obtain
\end{frame}
%
%
\begin{frame}
	\frametitle{An Energy Estimate for the Difference of Solutions}
%
%
\begin{equation}
\begin{split}
 \frac{1}{2} \frac{d}{dt} \|v\|_{H^\sigma(\ci)}^2
 = & -\frac{1}{2} \int_{\ci} D^\sigma \p_x [v(u+w)] \cdot
D^\sigma v \ dx.
\label{2v}
\end{split}
\end{equation}
%
%
Unfortunately, it will not be enough to simply use Cauchy-Schwartz, the 
Sobolev Imbedding Theorem, or the algebra property of Sobolev Spaces to 
estimate the right hand side of this equation. We require more heavy 
machinery. 
\end{frame}
%
%
\begin{frame}
	\frametitle{Rough Estimates}
Proceeding, we rewrite
%
%
\begin{equation}
\begin{split}
|i|  & =
\Big |
 -\frac{1}{2} \int_{\ci} \left[ D^\sigma \p_x, \ u+w \right]v \cdot
D^\sigma v \ dx 
\\
& \phantom{ \ = \Big |}- \frac{1}{2} \int_{\ci} (u+w) D^\sigma
\p_x v \cdot D^\sigma v\ dx
\Big | \\
& \le \left |
-\frac{1}{2} \int_{\ci} \left[ D^\sigma \p_x, \ u+w \right]v \cdot
D^\sigma v \ dx \right |
\\
& + \left | \frac{1}{2} \int_{\ci} (u+w) D^\sigma \p_x v
\cdot D^\sigma v\
dx \right |.
\label{4v}
\end{split}
\end{equation}
%
%
\end{frame}
%
%
\begin{frame}
	\frametitle{Rough Estimates}
Observe that by integrating by parts
and applying Cauchy-Schwartz we have
%
%
\begin{equation}
\begin{split}
\left | \frac{1}{2}\int_{\ci} (u+w) D^\sigma \p_x v \cdot
D^\sigma v \ dx \right |
\lesssim \|\p_x (u+w)\|_{L^\infty(\ci)}
\|v\|_{H^\sigma(\ci)}^2.
\label{4'v}
\end{split}
\end{equation}
%
%
Furtheremore, 
\begin{equation}
\begin{split}
& \left | -\frac{1}{2} \int_{\ci} [D^\sigma \p_x, \ u+w] v
\cdot D^\sigma v \ dx \right |
\\
& \lesssim \|[D^\sigma \p_x, \ u+w]v\|_{L^2(\ci)}
\|v\|_{H^\sigma(\ci)} \lesssim \|u+w\|_{H^\rho(\ci)} 
\|v\|_{H^\sigma(\ci)}^2
\label{7v}
\end{split}
\end{equation}

where the last step follows from the following commutator
estimate:
%
\end{frame}
%
%
\begin{frame}
	\frametitle{Commutator Estimates (Sharp Estimates)}
	%
%
\begin{theorem}
\label{cor1}
If $\rho > 3/2$ and $0 \le \sigma + 1 \le \rho$, then
%
%
\begin{equation}
\begin{split}
\|[D^\sigma \p_x ,f]v\|_{L^2} \le C \|f\|_{H^\rho} \|v\|_{H^\sigma}.
\label{15}
\end{split}
\end{equation}
%
%
\end{theorem}
A proof of a more general version of Theorem \ref{cor1} can be found in a survey article by Michael Taylor on 
commutator estimates. We note that the proof relies heavily on the  
Kato-Ponce commutator estimate. %
\end{frame}
%
%
\begin{frame}
	\frametitle{An ODE From a PDE}
Recallling \eqref{2v}, combining our previous estimates and applying the Sobolev Imbedding 
Theorem, we obtain 
%
%

%
%
%
\begin{equation}
\begin{split}
\frac{1}{2} \frac{d}{dt}
\|v\|_{H^\sigma(\ci)}^2 \lesssim \|u+w\|_{H^\rho(\ci)}
\|v\|_{H^\sigma(\ci)}^2.
\label{9v}
\end{split}
\end{equation}
%
%
\end{frame}
%
%
%
%
\begin{frame}
	\frametitle{Gronwall's Inequality}
	%
By Gronwall's inequality, \eqref{9v} gives

%
%
\begin{equation}
\label{10lv}
\begin{split}
\|v\|_{H^\sigma(\ci)}
& \lesssim e^{\int_0^t \|u+w\|_{H^{\rho}}}
\|v_0\|_{H^\sigma(\ci)}, \qquad |t| \le T.
\end{split}
\end{equation}
%
%
Since $v_0 = 0$ and $\|u + w \|_{H^\rho}
\le \|u + w \|_{H^s(\ci)} < \infty$ for $|t| \le T$, we conclude from 
\eqref{10lv} that solutions to the HR i.v.p. with initial data in
$H^s(\ci)$ are unique for $s > 3/2$.  
%
%
\end{frame}
%
%
%
\end{document}
