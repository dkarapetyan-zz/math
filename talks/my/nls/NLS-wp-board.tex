%
\documentclass[12pt,reqno]{amsart}
\usepackage{amssymb}
\usepackage{cancel}  %for cancelling terms explicity on pdf
\usepackage{yhmath}   %makes fourier transform look nicer, among other things
\usepackage[alphabetic, msc-links]{amsrefs} %for the bibliography; uses cite pkg
%\usepackage[notcite, notref]{showkeys}
\usepackage[margin=3cm]{geometry}  %page layout
%\usepackage[pdftex]{graphicx} %for importing pictures into latex--pdf compilation
\setcounter{secnumdepth}{1} %number only sections, not subsections
\hypersetup{colorlinks=true,
linkcolor=blue,
citecolor=blue,
urlcolor=blue,
}
\synctex=1
\numberwithin{equation}{section}  %eliminate need for keeping track of counters
\numberwithin{figure}{section}
\setlength{\parindent}{0in} %no indentation of paragraphs after section title
\renewcommand{\baselinestretch}{1.1} %increases vert spacing of text
%
\newcommand{\ds}{\displaystyle}
\newcommand{\ts}{\textstyle}
\newcommand{\nin}{\noindent}
\newcommand{\rr}{\mathbb{R}}
\newcommand{\nn}{\mathbb{N}}
\newcommand{\zz}{\mathbb{Z}}
\newcommand{\cc}{\mathbb{C}}
\newcommand{\ci}{\mathbb{T}}
\newcommand{\zzdot}{\dot{\zz}}
\newcommand{\wh}{\widehat}
\newcommand{\p}{\partial}
\newcommand{\ee}{\varepsilon}
\newcommand{\vp}{\varphi}
\newcommand{\quod}{\qquad \qedsymbol}
%
%
\theoremstyle{plain}  
\newtheorem{theorem}{Theorem}
\newtheorem{proposition}{Proposition}
\newtheorem{lemma}{Lemma}
\newtheorem{corollary}{Corollary}
\newtheorem{claim}{Claim}
\newtheorem{conjecture}[subsection]{conjecture}
%
\theoremstyle{definition}
\newtheorem{definition}{Definition}
%
\theoremstyle{remark}
\newtheorem{remark}{Remark}
%
%
%
\def\makeautorefname#1#2{\expandafter\def\csname#1autorefname\endcsname{#2}}
\makeautorefname{equation}{Equation}
\makeautorefname{footnote}{footnote}
\makeautorefname{item}{item}
\makeautorefname{figure}{Figure}
\makeautorefname{table}{Table}
\makeautorefname{part}{Part}
\makeautorefname{appendix}{Appendix}
\makeautorefname{chapter}{Chapter}
\makeautorefname{section}{Section}
\makeautorefname{subsection}{Section}
\makeautorefname{subsubsection}{Section}
\makeautorefname{paragraph}{Paragraph}
\makeautorefname{subparagraph}{Paragraph}
\makeautorefname{theorem}{Theorem}
\makeautorefname{theo}{Theorem}
\makeautorefname{thm}{Theorem}
\makeautorefname{addendum}{Addendum}
\makeautorefname{add}{Addendum}
\makeautorefname{maintheorem}{Main theorem}
\makeautorefname{corollary}{Corollary}
\makeautorefname{lemma}{Lemma}
\makeautorefname{sublemma}{Sublemma}
\makeautorefname{proposition}{Proposition}
\makeautorefname{property}{Property}
\makeautorefname{scholium}{Scholium}
\makeautorefname{step}{Step}
\makeautorefname{conjecture}{Conjecture}
\makeautorefname{question}{Question}
\makeautorefname{definition}{Definition}
\makeautorefname{notation}{Notation}
\makeautorefname{remark}{Remark}
\makeautorefname{remarks}{Remarks}
\makeautorefname{example}{Example}
\makeautorefname{algorithm}{Algorithm}
\makeautorefname{axiom}{Axiom}
\makeautorefname{case}{Case}
\makeautorefname{claim}{Claim}
\makeautorefname{assumption}{Assumption}
\makeautorefname{conclusion}{Conclusion}
\makeautorefname{condition}{Condition}
\makeautorefname{construction}{Construction}
\makeautorefname{criterion}{Criterion}
\makeautorefname{exercise}{Exercise}
\makeautorefname{problem}{Problem}
\makeautorefname{solution}{Solution}
\makeautorefname{summary}{Summary}
\makeautorefname{operation}{Operation}
\makeautorefname{observation}{Observation}
\makeautorefname{convention}{Convention}
\makeautorefname{warning}{Warning}
\makeautorefname{note}{Note}
\makeautorefname{fact}{Fact}
%
%
%
\begin{document}
\title{Well-Posedness for the Nonlinear Schr\"{o}dinger Equation }
\author{David Karapetyan}
\address{Department of Mathematics  \\
         University  of Notre Dame\\
		          Notre Dame, IN 46556 }
				  \date{09/01/10}
				  %
				  \maketitle
				  %
				  %
				  %
				  %
				  %
				  %
				  \section{Introduction}
				  We consider the  nonlinear Schr\"{o}dinger (NLS) 
				  initial value problem (ivp)
%
%
\begin{gather}
	\label{NLS-eq}
	i \p_t u + \p_x^2 u +  |u|^2 u =0,
		\\
		\label{NLS-init-data}
		u(x,0) = \vp(x) \in H^s(\ci), \ \ t \in \rr, \ \ x \in \ci.
\end{gather}
%
%
\begin{definition}
	We say that the flow map is \emph{locally Lipschitz} in $H^s(\ci)$ if for
	$$u_0, v_0 \in B_R \doteq \{f: \|f\|_{H^s} < R\},$$ there exist $C, T>0$
	depending on $R$ such that $\|u(\cdot, t) - v(\cdot, t)
	\|_{H^s(\ci)} \le C \|u_{0} - v_0 \|_{H^s(\ci)}$ for $t \in [-T, T]$. We
	say the flow map is \emph{locally uniformly
	continuous} in $H^s(\ci)$ if for
	$u_0, v_0 \in B_R$ there exists $T >0$ depending on $R$ such that for
	$t \in [-T, T]$, $\|u(\cdot, t) - v(\cdot, t) \|_{H^s(\ci)} \to
	0$ if $\|u_0 - v_0 \|_{H^{s}(\ci)} \to 0$. 
\end{definition}
%
%
Clearly any locally Lipschitz flow map is locally uniformly continuous. 
\begin{definition}
	We say that the NLS ivp \eqref{NLS-eq}-\eqref{NLS-init-data} is
	\emph{locally well posed} in
	$H^s(\ci)$ if 
	\begin{enumerate}
		\item For every $\vp(x) \in
	B_R$ there exists $T>0$ depending on $R$ and a unique function
	\\
	$u \in C([-T, T],
	H^s(\ci))$ satisfying \eqref{NLS-eq} for all $t \in [-T, T]$. 
\item The flow map $u_0 \mapsto u(t)$ is locally uniformly continuous. That is, if $u_0
	\in B_R$, $\{u_{0,n}\} \subset B_R$, and 
	$\|u_0 - u_{0, n} \|_{H^{s}(\ci)} \to 0$, then there exists $T >0$ depending
	on $r$ such that $\|u(\cdot, t) - u_{n}(\cdot,t) \|_{H^s(\ci)} \to
	0$ for $t \in [-T, T]$. 
	\end{enumerate}
	Otherwise, we say that the NLS ivp is \emph{ill-posed}.
\end{definition}
%
%
We are now prepared to state the following result.
%
%
%
%
%%%%%%%%%%%%%%%%%%%%%%%%%%%%%%%%%%%%%%%%%%%%%%%%%%%%%
%
%
%	Main Result				
%
%
%%%%%%%%%%%%%%%%%%%%%%%%%%%%%%%%%%%%%%%%%%%%%%%%%%%%%
%
%
\begin{theorem}
\label{thm:main}
The initial value problem 
\eqref{NLS-eq}-\eqref{NLS-init-data} is locally well-posed in $H^s(\ci)$ for $s \ge
0$, and ill-posed for $s <0$. %
%
\end{theorem} 
%
%
%
%
%%%%%%%%%%%%%%%%%%%%%%%%%%%%%%%%%%%%%%%%%%%%%%%%%%%%%
%
%
%				Outline
%
%
%%%%%%%%%%%%%%%%%%%%%%%%%%%%%%%%%%%%%%%%%%%%%%%%%%%%%
%
%
\section{Outline of the Proof of \autoref{thm:main}}
%
%
%
%
%
We first derive a weak formulation of the NLS ivp. 
Let $\ci = [0, 2 \pi]$, and use
the following notation for the Fourier transform
%
%
%
%
\begin{equation}
	\label{four-trans-pde}
	\begin{split}
		\widehat{f}(n) = \int_{\ci} e^{-ix n} f(x) \, dx.
	\end{split}
\end{equation}
Using the Fourier transform, we may rewrite the NLS ivp as an integral
equation 
%
%
\begin{equation}
	\label{NLS-integral-form}
	\begin{split}
		u(x,t) & = \sum_{n \in \zz} \wh{\vp}(n) e^{i\left( xn - t n^2 
		\right)} 
		\\
		& + i \sum_{n \in \zz} \int_0^t e^{i\left[ xn + \left( t' - t 
		\right) n^2 \right]} \wh{w}(n, t') \ dt'.
	\end{split}
\end{equation}
%
%
Localizing in the time variable, we obtain
%
%
\begin{align}
	\label{main-int-expression-0}
	& u(x, t) 
		\\
		\label{main-int-expression-1}
		& = \frac{1}{2 \pi} \psi(t) \sum_{n \in \zz} e^{i(xn + tn^{m 
		})} \widehat{\vp}(n) 
		\\
		\label{main-int-expression-2}
		& + \frac{1}{4 \pi^2} \psi(t) \sum_{n\in \zz} \int_\rr e^{ixn}  
		e^{it \tau} \frac{ 1 - \psi(\tau -  n^2) 
		}{\tau -  n^2} \wh{w}(n, \tau) \ d \tau
		\\
		\label{main-int-expression-3}
		& - \frac{1}{4 \pi^2} \psi(t) \sum_{n\in \zz} \int_\rr e^{i(xn + 
		t n^2)}
		 \frac{1- \psi(\tau -  n^2)}{\tau -  n^2} \wh{w}(n, \tau) \ d \tau
		\\
		\label{main-int-expression-4}
		& + \frac{1}{4 \pi^2} \psi(t) \sum_{k \ge 1} \frac{i^k t^k}{k!}
		\sum_{n \in \zz} \int_\rr e^{i(xn + t n^2 )}
		\psi(\tau -  n^2) (\tau -  n^2)^{k-1} \wh{w}(n, \tau)  
		\\
		& \doteq T(u) \notag
\end{align}
%
%
where $T = T_{\vp}$. We now introduce the following spaces.
\begin{definition}
	Denote $Y^s$ to be the space of all
	functions $u$ on $\ci \times \rr$ with
	bounded norm
\begin{equation}
	\label{Y-s-norm}
	\begin{split}
		\|u\|_{Y^s} = \|u\|_{X^s} + \|\wh{u}\|_{ \ell^2_n L^1_\tau }
	\end{split}
\end{equation}
%
%
%
%
where
%
\begin{equation}
	\label{X^s-norm}
	\begin{split}
		& \|u\|_{X^s}
		= \left ( \sum_{n\in \zz} \left (1 + |n| \right )^{2s} \int_\rr \left ( 1 + | 
		\tau - n^2 \right ) | \wh{u} ( n, \tau ) |^2
		\right )^{1/2}
	\end{split}
\end{equation}
and
%
%
\begin{equation}
	\label{E-norm}
	\|\wh{u}\|_{ \ell^2_n L^1_\tau } = \left[ \sum_{n \in \zz}(1 + | n |)^{2s} \left(
	\int_{\rr}| \wh{u}(n, \tau) |d \tau \right)^{2} \right]^{1/2}.
\end{equation}
%
%
%
%
\end{definition}
The $Y^s$ spaces have the following important property, whose proof
is provided in the appendix.
\begin{lemma}
	\label{lem:cutoff-loc-soln}
	Let $\psi(t)$ be a smooth cutoff function with $\psi(t) =1$ for $t \in [-T, T]$. If
	$\psi(t)u(x,t) \in Y^s$, then $u \in C([-T, T], H^s(\ci))$.
\end{lemma}
%
%
We we will 
show that for initial data $\vp \in H^s(\ci)$, $T$ is a contraction on $B_M 
\subset Y^s$, where $B_M$ is the ball centered at the origin of radius $M = 
M_{\vp}> 0$, by estimating the $Y^s$
norm of \eqref{main-int-expression-1}-\eqref{main-int-expression-4}. The 
Picard fixed point theorem will
then yield a unique solution to
\eqref{main-int-expression-0}-\eqref{main-int-expression-4}. An application of
\autoref{lem:cutoff-loc-soln} will then imply the existence of a unique, local
solution $u \in C([-T, T], H^s(\ci))$ to the NLS ivp which coincides with the solution to
\eqref{main-int-expression-0}-\eqref{main-int-expression-4} on the interval $[-T, T]$. Local Lipschitz continuity of the flow map will follow
from estimates used to establish the contraction mapping. %
%
%%%%%%%%%%%%%%%%%%%%%%%%%%%%%%%%%%%%%%%%%%%%%%%%%%%%%
%
%
%			Proof of Theorem	
%
%
%%%%%%%%%%%%%%%%%%%%%%%%%%%%%%%%%%%%%%%%%%%%%%%%%%%%%
%
%
\section{Proof of \autoref{thm:main}}
%
%
%
%%%%%%%%%%%%%%%%%%%%%%%%%%%%%%%%%%%%%%%%%%%%%%%%%%%%%
%
%
%		Estimation of Integral Equality Part 1		
%
%
%%%%%%%%%%%%%%%%%%%%%%%%%%%%%%%%%%%%%%%%%%%%%%%%%%%%%
%
%
%
%
%
%
It will be enough to establish the following trilinear estimates.
%
%
%%%%%%%%%%%%%%%%%%%%%%%%%%%%%%%%%%%%%%%%%%%%%%%%%%%%%
%
%
%				Proposition
%
%
%%%%%%%%%%%%%%%%%%%%%%%%%%%%%%%%%%%%%%%%%%%%%%%%%%%%%
%
%
\begin{proposition}
\label{prop:trilinear-est}
	%
	%
	For any $s \ge 0$ and $b \ge 3/8$, we have
	\begin{equation}
		\left( \sum_{n \in \zz} \left (1 + |n| \right )^{2s} \int_\rr
		\frac{|\wh{w_{fgh}}(n, \tau) |^2}{\left (1+ |\tau - 
		n^2| \right ) ^{2b}} 
		 \ d \tau 
		\right)^{1/2}
		\lesssim \|f\|_{X^s} \|g\|_{X^s}\|h\|_{X^s}
	\end{equation}
	where $w_{fgh}(x,t)$ = $fg \bar h (x,t)$.
%
%
%
%
\end{proposition}
%
%
%
%
%%%%%%%%%%%%%%%%%%%%%%%%%%%%%%%%%%%%%%%%%%%%%%%%%%%%%
%
%
%				Second trilinear Estimate 
%
%
%%%%%%%%%%%%%%%%%%%%%%%%%%%%%%%%%%%%%%%%%%%%%%%%%%%%%
%
%
\begin{corollary}
\label{cor:trilinear-estimate2}
	For $s \ge 0$ we have
%
%
\begin{equation}
	\label{trilinear-estimate2}
	\begin{split}
		\left( \sum_{n \in \zz} \left (1 + |n| \right )^{2s}  \left ( \int_\rr 
		\frac{|\wh{w_{fgh}}(n, \tau) |}{1 + | \tau - n^2 |}
		 \ d\tau \right)^2  \right)^{1/2} \lesssim \|f\|_{X^s} \|g\|_{X^s}\|h\|_{X^s}.
	\end{split}
\end{equation}
\end{corollary}
%
%
If we establish these estimates, routine computations of the $Y^s$ norms of
\eqref{main-int-expression-0}-\eqref{main-int-expression-4} will yield the
following.
%%
%%%%%%%%%%%%%%%%%%%%%%%%%%%%%%%%%%%%%%%%%%%%%%%%%%%%%
%
%% Contraction Proposition
%				 
%%%%%%%%%%%%%%%%%%%%%%%%%%%%%%%%%%%%%%%%%%%%%%%%%%%%%%
%%
%%
%
\begin{proposition}
\label{prop:contraction}
	Let $s \ge0$. Then
%
%%
\begin{equation*}
	\begin{split}
		\|Tu\|_{Y^s} \le c_\psi \left( \|\vp \|_{H^s(\ci)} + \|u\|_{Y^s}^3 
		\right).
	\end{split}
\end{equation*}
%
%%
\end{proposition}
We will later use \autoref{prop:contraction} to prove local well-posedness for the 
NLS ivp. %
%
%
\section{Proof of \autoref{prop:trilinear-est}.}
%
%
%
%
%
%
Note first that $|\wh{w_{fgh}}(n, \tau) |  = | \wh{f} * ( \wh{g} 
* \wh{\bar h})(n, \tau)|$ and $| \wh{\bar{h}}(n, \tau) | = |\overline{ \wh{\overline{h}} 
}(n, \tau)| = | \wh{h}(-n, -\tau) |$. It follows that
%
%
\begin{equation}
	\label{non-lin-rep}
	\begin{split}
		| \wh{w_{fgh}}(n, \tau)|
		& = | \sum_{\substack{n_1,n_2,n_3\\n = n_1 + n_2 + n_3}}  \int_{\tau=\tau_1 + \tau_2 + \tau_3} \wh{f}\left( n_1,  \tau_1 
\right) \wh{g}\left( n_2, \tau_2  
\right) \wh{\bar h}\left( n_3, \tau_3 \right) d \tau_1 d \tau_2 d \tau_3 |
\\
& \le \sum_{\substack{n_1,n_2,n_3\\n = n_1 + n_2 + n_3}}  \int_{\tau=\tau_1 + \tau_2 + \tau_3} | \wh{f}\left( n_1, \tau_1 
\right) | \times  | \wh{g}\left( n_2, \tau_2 
\right) | \times | \wh{\bar h}\left( n_3, \tau_3 \right) | d \tau_1 d \tau_2 d 
\tau_3
\\
& \le \sum_{\substack{n_1,n_2,n_3\\n = n_1 + n_2 + n_3}}  \int_{\tau=\tau_1 + \tau_2 + \tau_3} | \wh{f}\left( n_1, \tau_1 
\right) | \times | \wh{g}\left( n_2, \tau_2 
\right) | \times | \wh{h}\left( -n_3, - \tau_3 \right) | d \tau_1 d \tau_2 d 
\tau_3
\\
& = \sum_{\substack{n_1,n_2,n_3\\n = n_1 + n_2 + n_3}} \int_{\tau=\tau_1 + \tau_2 + \tau_3} \frac{c_f\left( n_1, \tau_1 
\right)}{\left (1 + |n_1| \right )^s \left( 1 + | \tau_1 - n_1^2 | \right)^{b}}
\\
& \times \frac{c_{g}\left( n_2, \tau_2 \right)}{\left (1 + |n_2| \right ) 
^s\left( 1 + | \tau_2 -  n_2^2| 
\right)^{b}}
 \times \frac{c_{h}\left( -n_3, -\tau_3 \right)}{\left (1 + |n_3| \right ) ^s\left( 1 + | 
\tau_3 + n_3^2 | \right)^{b}} \ d \tau_1 d \tau_2 d \tau_3
\end{split}
\end{equation}
%
%
where 
%
%
\begin{equation*}
	\begin{split}
		c_\sigma(n, \tau) = \left (1 + |n| \right ) ^s \left( 1 + | \tau - n^2 |  
		\right)^{b} | \wh{\sigma}\left( n, \tau \right) | .
	\end{split}
\end{equation*}
%
%
Hence
%
%
\begin{equation}
	\label{convo-est-starting-pnt}
	\begin{split}
		 & \left (1 + |n| \right )^s \left( 1 + | \tau - n^2 | \right)^{-b} | \wh{w_{fgh}}\left( 
		n, \tau \right) |
		\\
		& \le \left( 1 + | \tau - n^2 | \right)^{-b}
		\sum_{\substack{n_1,n_2,n_3\\n = n_1 + n_2 + n_3}} \int_{\tau=\tau_1 + \tau_2 + \tau_3} \frac{\left (1 + |n| \right )^s}{\left (1 +
		|n_1| \right )^s \left (1 + | n_2| \right )^s \left (1 + |n_3| \right )^s} 
		\times \frac{c_f(n_1, \tau_1)}{\left( 1 + | \tau_1 - n_1^2 | 
		\right)^{b}}
		\\
		& \times
		\frac{c_g(n_2, \tau_2)}{\left( 1 + | \tau_2 - n_2^2 | 
		\right)^{b}} \times
		\frac{c_h(-n_3, -\tau_3)}{\left( 1 + | \tau_3 + n_3^2 | 
		\right)^{b}}\ d \tau_1 d \tau_2 d \tau_3.
	\end{split}
\end{equation}
%
%
For $s \ge 0$, observe that
%
%
\begin{equation}
	\label{deriv-bound-easy-s}
	\begin{split}
		\frac{\left (1 + |n| \right ) ^s}{\left (1 + |n_1| \right ) ^s \left (1 + |n_2| \right ) ^s \left (1 + |n_3| \right ) ^s} 
		\le 3^{s}
	\end{split}
\end{equation}
%
%
by the following lemma, whose proof is provided in the appendix.
%
%
\begin{lemma}
\label{lem:splitting}
	For $v \ge 0$ and $a, b, c \in \zz$, we have
%
%
\begin{equation}
	\label{splitting}
	\begin{split}
		\left ( 1 + |a +b + c| \right)^v \le 3^v \left(1 + | a | \right)^v \left(
		1 + | b | \right)^v \left( 1 + | c | \right)^v.
	\end{split}
\end{equation}
%
%
\end{lemma}
%
%
Hence, from \eqref{convo-est-starting-pnt} and \eqref{deriv-bound-easy-s}, we 
obtain
%
\begin{equation*}
	\begin{split}
		& \left (1 + |n| \right )^s \left( 1 +  | \tau - n^2  | \right)^{-b} | 
		\wh{w_{fgh}}\left( n, \tau \right) | 
		\\
		& \lesssim \sum_{n_1, n_2,n_3} \int_{\tau=\tau_1 + \tau_2 + \tau_3} \frac{1}{\left( 1 +
		| \tau - n^2| 
		\right)^{b}}  
		\\
		& \times
		\sum_{\substack{n_1,n_2,n_3\\n = n_1 + n_2 + n_3}} \int_{\tau=\tau_1 + \tau_2 + \tau_3} \frac{c_f\left( n_1, \tau_1 
		\right)}{\left (1 + |n_1| \right )^s \left( 1 + | \tau_1 - n_1^2 |
		\right)^{b}}
		\\
		& \times \frac{c_{g}\left( n_2, \tau_2 \right)}{\left (1 + |n_2| \right ) 
		^s\left( 1 + | \tau_2 -  n_2^2| 
		\right)^{b}}
		\\
		& \times \frac{c_{h}\left( -n_3, -\tau_3 \right)}{\left (1 + |n_3| \right ) ^s\left( 1 + | 
		\tau_3 + n_3^2 | \right)^{b}} \ d \tau_1 d \tau_2 d \tau_3
		\\
		& = \left( 1 + | \tau - n^2 | \right)^{-b}
		\wh{C_f C_{g} C^+_{h}} \left( n, \tau \right)
	\end{split}
\end{equation*}
%
%
where
%
%
\begin{equation*}
	\begin{split}
		C_\sigma(x, t) = \left[ \frac{c_\sigma\left( n, \tau \right)}{\left( 
		1 + | \tau - n^2 | \right)^{b}} \right]^\vee,
		\ \ C^+_\sigma(x, t) = \left[ \frac{c_\sigma\left( -n, -\tau \right)}{\left( 
		1 + | \tau + n^2 | \right)^{b}} \right]^\vee.
	\end{split}
\end{equation*}
%
%
Therefore
%
%
\begin{equation}
	\label{gen-holder-pre-estimate}
	\begin{split}
		& \| \left( 1 + |n | \right)^s
		\left( 1 + | \tau - n^2 | \right)^{-b} \wh{w_{fgh}}(n, 
		\tau)		
		\|_{L^2(\zz \times \rr)}
		\\
		& \lesssim \| \left( 1 + | \tau - n^2 | \right)^{-b}
		\wh{C_f C_{g} C^+_{h}} \|_{L^2(\zz \times \rr)}.
	\end{split}
\end{equation}
%
We now require the following multiplier estimate, whose proof can be found in 
\cite{Himonas-Misiolek-2001-A-priori-estimates-for-Schrodinger}.
%
%
%%%%%%%%%%%%%%%%%%%%%%%%%%%%%%%%%%%%%%%%%%%%%%%%%%%%%
%
%
%			Four Mult Est	
%
%
%%%%%%%%%%%%%%%%%%%%%%%%%%%%%%%%%%%%%%%%%%%%%%%%%%%%%
%
%
%
%
%
%
%
%
\begin{lemma}
	\label{lem:four-mult-est-L4}
	Let $(x, t) \in \ci \times \rr $ and $(n, \tau) \in \zz \times \rr$ be 
	the dual variables. Let $v$ be a positive even integer. Then there is a 
	constant $c_v > 0$ such that
%
%
\begin{equation}
	\label{four-mult-est-L4*}
	\begin{split}
		\| \left( 1 + | \tau - n^v | 
		\right)^{-\frac{v+1}{4v}}
		\wh{f}\|_{L^2(\zz \times \rr)} \le c_v \|f \|_{L^{4/3}( \ci \times \rr)}.
	\end{split}
\end{equation}
%
%
\end{lemma}
%
%
Applying \autoref{cor:four-mult-est-L4} and generalized H\"{o}lder to the 
right-hand-side of \eqref{gen-holder-pre-estimate} gives
%
%
\begin{equation}
	\label{gen-holder-piece-1}
	\begin{split}
		\|\left( 1 + | \tau - n^2 | \right)^{-b} \wh{C_f C_{ 
		g } C^+_{h}}\|_{L^2(\zz \times \rr)}
		& \lesssim  \|C_f C_{g} C^+_{h} \|_{L^{4/3}(\ci \times \rr)}
		\\
		& \le \|C_f \|_{L^4(\ci \times \rr)} \|C_{g}\|_{L^4(\ci \times \rr)} 
		\|C^+_{h}\|_{L^4(\ci \times \rr)}.
	\end{split}
\end{equation}
%
%
Note that a change of variable gives
%
%
\begin{equation*}
	\begin{split}
		C_\sigma^+(x, t)
		& = \sum_{n \in \zz} \int_\rr e^{i(nx +  \tau t)} \frac{c_\sigma\left( -n, -\tau \right)}{\left( 
		1 + | \tau + n^2 | \right)^{b}} \ d \tau
		\\
		& = - \sum_{n \in \zz} \int_\rr e^{-i(nx +   \tau t )}
		\frac{c_\sigma\left( n, \tau \right)}{\left( 
		1 + | \tau - n^2 | \right)^{b}} \ d \tau
	\end{split}
\end{equation*}
%
%
and so
%
%
\begin{equation*}
	\begin{split}
		C_\sigma^+(-x, -t) = -C_\sigma(x, t).
	\end{split}
\end{equation*}
%
%
We will now the need the following dual estimate of
\autoref{lem:four-mult-est-L4}.
%
\begin{corollary}
	\label{cor:four-mult-est-L4}
	Let $(x, t) \in \ci \times \rr $ and $(n, \tau) \in \zz \times \rr$ be 
	the dual variables. Let $v$ be a positive even integer. Then there is a 
	constant $c_v > 0$ such that
%
%
\begin{equation}
	\label{four-mult-est-L4}
	\begin{split}
		\|f\|_{L^4(\ci \times \rr)} \le c_v \|\left( 1 + | \tau - n^v | 
		\right)^\frac{v+1}{4v} \wh{f} \|_{L^2( \zz \times \rr)}
	\end{split}
\end{equation}
for every test function $f(x, t)$. 
%
%
%
%
\end{corollary}
%
%
Recalling that $L^4(\ci \times \rr)$ is invariant under the transformation $(x, 
t) \mapsto (-x,-t)$ and applying 
\autoref{cor:four-mult-est-L4}, we obtain
%
%
\begin{equation}
	\label{C-sig-estimate}
	\begin{split}
		\| C^+_\sigma \|_{L^4(\ci \times \rr)} = \|C_\sigma \|_{L^4(\ci \times \rr)} 
		& \lesssim \|\left( 1 + | \tau - n^2 | 
		\right)^{3/8} \left( 1 + | \tau - n^2 | 
		\right)^{-b} c_\sigma \|_{L^2(\zz \times \rr)}
		\\
		& 
		\le \|c_\sigma \|_{L^2(\zz \times \rr)}  \qquad (\text{since  } b \ge 3/8)
		\\
		& = \|\sigma\|_{X^s}.
	\end{split}
\end{equation}
%
%
We conclude from \eqref{gen-holder-pre-estimate}, \eqref{gen-holder-piece-1}, 
and \eqref{C-sig-estimate} that
%
%
%
%
\begin{equation*}
	\begin{split}
		\| \left( 1 + |n | \right)^s \left( 1 + | \tau - n^2 | \right)^{-b} \wh{w_{fgh}} 
		(n, \tau) \|_{L^2(\zz \times \rr)} \lesssim 
		\|f\|_{X^s}\|g\|_{X^s}\|h\|_{X^s}.
	\end{split}
\end{equation*}
%
%
%
\subsection{Why \autoref{prop:trilinear-est} fails when dealing
with nonlinearity $\frac{1}{3} \p_x u^3$}
%
%
%
Recalling \eqref{non-lin-rep}, we have
\begin{equation}
	\begin{split}
		| \wh{w_{fgh}}(n, \tau)|
		& = | \sum_{\substack{n_1,n_2,n_3\\n = n_1 + n_2 + n_3}}  \int_{\tau=\tau_1 + \tau_2 + \tau_3} n \wh{f}\left( n_1,  \tau_1 
\right) \wh{g}\left( n_2, \tau_2  
\right) \wh{h}\left( n_3, \tau_3 \right) d \tau_1 d \tau_2 d \tau_3 |
\\
& \le \sum_{\substack{n_1,n_2,n_3\\n = n_1 + n_2 + n_3}}  \int_{\tau=\tau_1 + \tau_2 + \tau_3} | n | \times | \wh{f}\left( n_1, \tau_1 
\right) | \times  | \wh{g}\left( n_2, \tau_2 
\right) | \times | \wh{ h}\left( n_3, \tau_3 \right) | d \tau_1 d \tau_2 d 
\tau_3
\\
& = \sum_{\substack{n_1,n_2,n_3\\n = n_1 + n_2 + n_3}} \int_{\tau=\tau_1 + \tau_2 + \tau_3} \frac{| n |c_f\left( n_1, \tau_1 
\right)}{\left (1 + |n_1| \right )^s \left( 1 + | \tau_1 - n_1^2 | \right)^{b}}
\\
& \times \frac{c_{g}\left( n_2, \tau_2 \right)}{\left (1 + |n_2| \right ) 
^s\left( 1 + | \tau_2 -  n_2^2| 
\right)^{b}}
 \times \frac{c_{h}\left( n_3, \tau_3 \right)}{\left (1 + |n_3| \right ) ^s\left( 1 + | 
\tau_3 - n_3^2 | \right)^{b}} \ d \tau_1 d \tau_2 d \tau_3
\end{split}
\end{equation}
where $n = n_1 + n_2 + n_3$, $\tau = \tau_1 + \tau_2 + \tau_3$, and 
%
%
\begin{equation*}
	\begin{split}
		c_\sigma(n, \tau) = \left (1 + |n| \right ) ^s \left( 1 + | \tau - n^2 |  
		\right)^{b} | \wh{\sigma}\left( n, \tau \right) | .
	\end{split}
\end{equation*}
%
%
Hence
%
%
\begin{equation*}
	\begin{split}
		 & \left (1 + |n| \right )^s \left( 1 + | \tau - n^2 | \right)^{-b} | \wh{w_{fgh}}\left( 
		n, \tau \right) |
		\\
		& \le \left( 1 + | \tau - n^2 | \right)^{-b}
		\sum_{\substack{n_1,n_2,n_3\\n = n_1 + n_2 + n_3}} \int_{\tau=\tau_1 + \tau_2 + \tau_3} \frac{\left (1 + |n| \right )^s}{\left (1 +
		|n_1| \right )^s \left (1 + | n_2| \right )^s \left (1 + |n_3| \right )^s} 
		\times \frac{c_f(n_1, \tau_1)}{\left( 1 + | \tau_1 - n_1^2 | 
		\right)^{b}}
		\\
		& \times
		\frac{c_g(n_2, \tau_2)}{\left( 1 + | \tau_2 - n_2^2 | 
		\right)^{b}} \times
		\frac{c_h(n_3, \tau_3)}{\left( 1 + | \tau_3 - n_3^2 | 
		\right)^{b}}\ d \tau_1 d \tau_2 d \tau_3.
	\end{split}
\end{equation*}
%
%
For $s \ge 0$, observe that the quantity 
%
%
\begin{equation}
	\label{unbounded-quan}
	\begin{split}
		\frac{| n | \left (1 + |n| \right ) ^s}{\left (1 + |n_1| \right ) ^s \left (1 + |n_2| \right ) ^s \left (1 + |n_3| \right ) ^s} 
	\end{split}
\end{equation}
is unbounded (simply take $n_1 = n_2 = 0$ and $n_3$ arbitrarily large). Hence,
we hope that the principal symbol $\tau - n^2$ offers enough
decay to give control of \eqref{unbounded-quan}. Following the Bourgain
approach we consider the quantity
%
%
\begin{equation*}
	\begin{split}
		| \tau - n^2 - \left( \tau_{1} - n_{1}^2 + \tau_{2} - n_{2}^2 +
		\tau_{3} - n_{3}^2 \right) | = |n_{1}^2 + n_2^2 + n_3^2 - n^2|
	\end{split}
\end{equation*}
%
%
and seek a lower bound that is a function of $n$. No such bound exists (Simply
take $n_1 = n_2 =0$.) In the case of the KDV, to prove well-posedness we need to bound
\begin{equation}
	\label{KDV-bound-term}
	\begin{split}
		\frac{| n | \left (1 + |n| \right ) ^s}{\left (1 + |n_1| \right ) ^s \left (1 + |n_2| \right ) ^s} 
	\end{split}
\end{equation}
where $n_1 + n_2 = n$. 
Consider 
%
%
\begin{equation*}
	\begin{split}
		| \tau - n^{3} - \left( \tau_{1} - n_{1}^3 + \tau_{2} - n_{2}^3 \right) | = |n_{1}^3 + n_2^3 - n^3|
	\end{split}
\end{equation*}
where  $\tau_1 + \tau_2 = \tau$. Unlike the linear Schr{\"o}dinger with
derivative nonlinear forcing considered above, in the case of the KDV we can use the conservation of mass to
\emph{exclude} the pathological cases $n=0$, $n_1=0$ and $n_2=0$, allowing us to obtain
a lower bound of $|n|^{2}$. By the pigeonhole principle, we then have three
cases, each with the lower bound
%
%
\begin{equation*}
	\begin{split}
		\frac{1}{| \tau_{i} - n_{i}^2 |^{b}} \gtrsim \frac{1}{|n|^{b}}	
	\end{split}
\end{equation*}
%
%
where $\tau_0 =\tau, n_0 = n$. 
Hence, we must set $b \ge 1$ to offset the $|n|$ in the numerator of 
\eqref{KDV-bound-term}.
Lastly, in the case of the NLS, we are dealing with the quantity
\begin{equation*}
	\begin{split}
		\frac{\left (1 + |n| \right ) ^s}{\left (1 + |n_1| \right ) ^s \left (1 + |n_2| \right ) ^s \left (1 + |n_3| \right ) ^s} 
	\end{split}
\end{equation*}
which for $s \ge 0$
is already bounded by $3^s$ (note the absence of a $|n|$ term in the
numerator). Hence, we have more ``wiggle room'' with how
we choose $b$ for the NLS. In fact, we can choose $b$ all the way down to $3/8$, but no
lower, since we must have $b \ge 3/8$ in order to be able to apply
\autoref{cor:four-mult-est-L4}. 
%
%
%
%
\section{Proof of \autoref{cor:trilinear-estimate2}.}
By duality, it suffices to show that 
%
%%
\begin{equation*}
	\begin{split}
		\sum_{n \in \zz} \left (1 + |n| \right )^{s}
		a_n \int_{\rr} \frac{|\wh{w_{fgh}}(n, \tau)|}{1 
		+ | \tau - n^2 |} \ d \tau \lesssim \|f\|_{X^s} \|g\|_{X^s} \|h\|_{X^s}
		\|a_n \|_{\ell^2}
	\end{split}
\end{equation*}
%
%%
for $\{a_n\} \in \ell^2$. By the triangle inequality 
and Cauchy-Schwartz,
%
%%
\begin{equation}
	\label{1m}
	\begin{split}
		& | \sum_{n \in \zz} \left (1 + |n| \right )^{s} a_n
		\int_{\rr}\frac{| \wh{w_{fgh}}(n, \tau) |}{1 + | \tau - n^2 |} \ d \tau |
		\\
		& \le \sum_{n \in \zz} \int_{\rr} \frac{| a_n |}{\left( 1 + 
		| \tau - n^2 |
		\right)^{1/2 + \eta}} \cdot \frac{\left( 1 + | n| \right)^s  |
		\wh{w_{fgh}}(n, \tau) |}{\left( 
		1 + | \tau - n^2 | \right)^{1/2 - \eta}} \ d \tau
		\\
		& \le \left( \sum_{n \in \zz} | a_{n} |^2\int_{\rr} \frac{1}{\left( 1 + | \tau - n^2 | \right)^{1 + 2 \eta}} \ d \tau  
		\right)^{1/2} 
		\left ( \sum_{n \in \zz}\int_{\rr} \frac{\left (1 + |n| \right )^{2s} | \wh{w_{fgh}}(n, \tau) 
		|^2}{\left( 1 + | \tau - n^2 | \right)^{1 - 2 \eta}}\ d \tau 
		\right)^{1/2}
	\end{split}
\end{equation}
%
%%
Restrict $\eta \in (0, 1/8]$. Applying the change of variable $\tau - n^2
\mapsto \tau'$ we obtain  %
%%
%
\begin{equation*}
	\begin{split}
		& \left( \sum_{n \in \zz} | a_{n} |^2\int_{\rr} \frac{1}{\left( 1 + | \tau -
		n^2 | \right)^{1 + 2 \eta}} \ d \tau  
		\right)^{1/2} 
		\\
		& = \left ( \sum_{n \in \zz}
		| a_n |^2 
		\int_{\rr} \frac{1}{\left( 1 + | \tau' | \right)^{1 + 2 \eta}} \ d 
		\tau \right)^{1/2}
		\\
		& \simeq \|a_n\|_{\ell^2}
		\end{split}
\end{equation*}
while \autoref{prop:trilinear-est} gives the bound
\begin{equation*}
	\begin{split}
		\left ( \sum_{n \in \zz}\int_{\rr} \frac{\left (1 + |n| \right )^{2s} | \wh{w_{fgh}}(n, \tau) 
		|^2}{\left( 1 + | \tau - n^2 | \right)^{1 - 2 \eta}}\ d \tau 
		\right)^{1/2} \lesssim \|f\|_{X^s} \|g\|_{X^s} \|h\|_{X^s}
	\end{split}
\end{equation*}
%
%%
completing the proof.
\qquad \qedsymbol
%
%

%
\subsection{Proof of Well-Posedness Using  \autoref{prop:contraction}.} 
Let $c = c_{\psi}^{1/2}$. For given $\vp$, we may choose $\psi$ such
that 
%
%%
\begin{equation*}
	\begin{split}
		\|\vp\|_{H^s(\ci)} \le \frac{15}{64c^3}.
	\end{split}
\end{equation*}
%
%%
Then if $\|u\|_{Y^s} \le \frac{1}{4c}$, we have
%
%%
\begin{equation*}
	\begin{split}
		\|T u \|_{Y^s} 
		& \le c^2 \left[ \frac{15}{64c^3} + \left( 
		\frac{1}{4c} \right)^3 \right]
		=  \frac{1}{4c}.
	\end{split}
\end{equation*}
%
%%
Hence, $T=T_{\vp}$ maps the ball $B\left( 0, \frac{1}{4c} \right) \subset Y^s$ into 
itself. Next, note that
%
%%
\begin{equation*}
	\begin{split}
		Tu - Tv = \eqref{main-int-expression-2} + \eqref{main-int-expression-3} 
		+ \eqref{main-int-expression-4}
	\end{split}
\end{equation*}
%
%%
where now $w = u | u |^2 - v | v |^{2}$. Rewriting
%
%%
\begin{equation*}
	\begin{split}
		u | u |^{2} - v | v |^{2}
		& = | u |^2 \left( u -v \right) + v\left( | u 
		|^2 - | v |^2
		\right)
		\\
		& = u \bar u \left( u -v \right) + v u \bar u - v v \bar v
		\\
		& = u \bar u \left( u - v \right) + v \bar u\left( u - v \right) + v 
		\bar u v - v v \bar v
		\\
		& = u \bar u \left( u -v \right) + v \bar u\left( u - v \right) + v v 
		\left( \overline{u -v} \right)
	\end{split}
\end{equation*}
%
%%
the triangle inequality and linearity of the Fourier transform then give
%
%%
\begin{equation*}
	\begin{split}
		| \wh{w}(n, \tau) | = | \mathcal{F}(u | u |^2 - v| v |^2) |
		& \le | \wh{u \overline{u} \left (u -v \right )} | +
		| \wh{v \overline{u} (u -v)} | + |\wh{v v 
		(\overline{u-v})}|
		\\
		& \doteq | \wh{w_1} | + | \wh{w_2} | + | \wh{w_3} |
	\end{split}
\end{equation*}
%
%%
where
%
%%
\begin{equation*}
	\begin{split}
		w_1 = u \bar u \left( u -v \right), \qquad w_2 = v \bar u \left( u -v 
		\right), \qquad w_3 = v v \left( \overline{u -v} \right).
	\end{split}
\end{equation*}
%
%%
Hence, $Tu - Tv = \sum_{\ell=1, 2, 3} 
T_\ell(u, v)$, where
\begin{align}
	\label{main-int-exp-mod1}
	& \frac{1}{4 \pi^2} \psi(t) \sum_{n\in \zz} \int_\rr e^{ixn}  
		e^{it \tau} \frac{ 1 - \psi(\tau - n^2) 
		}{\tau - n^2} \wh{w_\ell}(n, \tau) \ d \tau
		\\
		\label{main-int-exp-mod2}
		- & \frac{1}{4 \pi^2} \psi(t) \sum_{n\in \zz} \int_\rr e^{i(xn + 
		tn^2)}
		 \frac{1- \psi(\tau - n^2)}{\tau - n^2} \wh{w_\ell}(n, \tau) \ d \tau
		\\
		\label{main-int-exp-mod3}
		+ & \frac{1}{4 \pi^2} \psi(t) \sum_{k \ge 1} \frac{i^k t^k}{k!}
		\sum_{n \in \zz} \int_\rr e^{i(xn + tn^2 )}
		\psi(\tau - n^2) (\tau - n^2)^{k-1} \wh{w_\ell}(n, \tau)  
		\\
		\doteq & T_\ell(u). \notag
\end{align}
Repeating the arguments used to estimate 
\eqref{main-int-expression-2}-\eqref{main-int-expression-4}, we obtain
%
%%
\begin{equation*}
	\begin{split}
		& \|T_1\|_{Y^s} \le c_\psi \|u -v \|_{Y^s} \|u\|^2_{Y^s}
		\\
		& \|T_2\|_{Y^s} \le c_\psi \|u -v \|_{Y^s} \|u\|_{Y^s} \|v\|_{Y^s}
		\\
		& \|T_3\|_{Y^s} \le c_\psi \|u -v \|_{Y^s} \|v\|_{Y^s}^2.
	\end{split}
\end{equation*}
%
%%
Therefore,
%
%%
\begin{equation}
	\label{20a}
	\begin{split}
		\|Tu - Tv \|_{Y^s} = & \| \sum T_\ell(u, v) \|_{Y^s}
		\\
		& \le c_\psi \|u -v \|_{Y^s} \left( \|u\|_{Y^s}^2 + 
		\|u\|_{Y^s} \|v\|_{Y^s} + \|v\|_{Y^s}^2 \right)
		\\
		& \le c_\psi \|u -v\|_{Y^s} \left( \|u\|_{Y^s} + \|v\|_{Y^s} \right)^2
		\\
		& = c^2 \|u -v\|_{Y^s} \left( \|u\|_{Y^s} + \|v\|_{Y^s} \right)^2.
	\end{split}
\end{equation}
%
%%
If $u, v \in B(0, \frac{1}{4c}) \subset Y^s$, it follows that
%
%%
\begin{equation}
	\label{21a}
	\begin{split}
		\|Tu - Tv \|_{Y^s}
		& \le c^2 \|u -v \|_{Y^s} \left( \frac{1}{4c} + 
		\frac{1}{4c} \right)^2
		\\
		& = \frac{1}{4} \|u -v \|_{Y^s}. 
	\end{split}
\end{equation}
%
%%
We conclude that $T = T_{\vp}$ is a contraction on the ball $B(0, 
\frac{1}{4c}) \subset Y^s$. A Picard iteration, coupled with
\autoref{lem:cutoff-loc-soln} then yields a unique, local
solution to the NLS ivp \eqref{NLS-eq}-\eqref{NLS-init-data}.
\\
\\
To establish local Lipschitz continuity of the flow map, let $\vp_1, \vp_2
\subset H^s(\ci)$ be given. Choose $\psi$ such that $\vp_1, \vp_2 \subset
B(0, \frac{15}{64c^{3}})$.  Then there exist $u_1, u_2 \in Y^s$ such that 
$u_1 = T_{\vp_1}$, $u_2 = T_{\vp_2}$, and so
%
%
\begin{equation*}
	\begin{split}
		T_{\vp_1}(u) - T_{\vp_2}(v) = \frac{1}{2\pi} \psi(t) \sum_{n \in
		\zz}e^{i\left( xn + tn^2 \right)} \wh{\vp_1 - \vp_2}(n) + \sum_{\ell
		= 1,2,3} T_{\ell}(u).
	\end{split}
\end{equation*}
%
%
A routine computation shows that
%
%
\begin{equation*}
	\begin{split}
		\| \frac{1}{2\pi} \psi(t) \sum_{n \in
		\zz}e^{i\left( xn + tn^2 \right)} \wh{\vp_1 - \vp_2}(n)\|_{Y^s}
		\le c_\psi \|\vp_{1} - \vp_{2}\|_{Y^s} 
	\end{split}
\end{equation*}
%
%
Hence, \eqref{20a}-\eqref{21a} gives
%
%
\begin{equation*}
	\begin{split}
		\sum_{\ell=1,2,3} T_{\ell}(u,v) \le \frac{1}{4}\|u-v\|_{Y^s}.
	\end{split}
\end{equation*}
%
%
Hence,
%
%
\begin{equation*}
	\begin{split}
		\|u -v \|_{Y^s} = \|T_{\vp_1}(u) - T_{\vp_2}(v) \|_{Y^s} \le c_\psi
		\|\vp_{1} - \vp_{2} \|_{H^{s}\left( \ci \right)}\| +
		\frac{1}{4} \|u -v \|_{Y^s}
	\end{split}
\end{equation*}
%
%
which implies
%
%
\begin{equation*}
	\begin{split}
		\frac{3}{4} \|u-v\|_{Y^s} \le c_\psi \|\vp_1 - \vp_2 \|_{H^s(\ci)}
	\end{split}
\end{equation*}
%
%
or
%
%
\begin{equation*}
	\begin{split}
		\|u -v \|_{Y^s} \le \frac{4}{3} c_\psi \|\vp_1 - \vp_2 \|_{H^{s}(\ci)}.
	\end{split}
\end{equation*}
%
%
Applying \autoref{lem:cutoff-loc-soln}, we then obtain
%
%
	 %
	 %
	 \begin{equation*}
		 \begin{split}
			\|u(\cdot, t) -v(\cdot, t) \|_{H^s(\ci)} \le \frac{4}{3} c_\psi \|\vp_1 -
			\vp_2 \|_{H^{s}(\ci)}, \qquad t \in [-T, T].
		 \end{split}
	 \end{equation*}
	 %
	 %
Hence, the flow map of the NLS ivp is locally Lipschitz continuous in
$H^s(\ci)$. This
concludes the proof of \autoref{thm:main}. \qquad \qedsymbol
%
%
%
%\section{Proof of Ill-Posedness}
%We adapt an argument from \cite{Burq_Gerad_Tzvetkov-An-instability-}. For $s<0$,
%$m \in \{4, 8, 12, \dots\}$, set
%%
%%
%%
%%
%\begin{equation}
%	\label{ill-soln}
%	\begin{split}
%		u_{n}(x,t)=\frac{1}{2}n^{-s}e^{it\left( n^2+\frac{1}{4}n^{-2s}
%		\right)}e^{inx}.
%	\end{split}
%\end{equation}
%%
%%
%Then
%%
%%
%\begin{equation*}
%	\begin{split}
%		& i \p_t u_{n}
%		= -\frac{1}{2}n^{-s}\left( n^2+\frac{1}{4}n^{-2s} \right)e^{it\left(
%		n^2+ \frac{1}{4}n^{-2s} \right)}e^{inx},
%		\\
%		& \p_x^2u_{n}  = \frac{1}{2}n^{-s+m}e^{it\left(
%		n^2+\frac{1}{4}n^{-2s} \right)}e^{inx},
%		\\
%		& | u_{n} |^{2}u_{n}  = \frac{1}{8}n^{-2s-s}e^{it\left(
%		n^2+\frac{1}{4}n^{-2s} \right)}e^{inx}.
%	\end{split}
%\end{equation*}
%%
%%
%Hence,
%%
%%
%\begin{equation*}
%	\begin{split}
%		i \p_t u_{n} + \p_x^2u_{n} + | u_{n} |^{2} u_{n}
%		=0.
%	\end{split}
%\end{equation*}
%%
%%
%Therefore, $u_{n}(x,t)$ solves the initial value problem
%%
%%
%\begin{gather*}
%	\begin{split}
%		i \p_t u + \p_x^2 u + | u |^{2}u = 0,
%		\\
%		u(x,0) = \frac{1}{2}n^{-s}e^{inx}.
%	\end{split}
%\end{gather*}
%%
%%
%Next, we show that $u_{n}(\cdot, t) \in H^{s}(\ci)$ for all $t \in \rr$.
%First, we compute
%%
%%
%\begin{equation*}
%	\begin{split}
%		\|e^{inx}\|_{H^{s}(\ci)}
%		& =  \left[ \sum_{\xi \in \zz} \left( 1+ | \xi |
%		\right)^{2s} | \wh{e^{in(\cdot)}}(\xi) |^{2} \right]^{1/2}
%		\\
%		& =  \left[ \sum_{\xi \in \zz} \left( 1 + | \xi | \right)^{2s} |
%		\int_{\ci}e^{ix(n- \xi)}dx |^{2}\right]^{1/2}.
%	\end{split}
%\end{equation*}
%%
%%
%Noting that
%%
%\begin{equation*}
%	\begin{split}
%		\int_{\ci}e^{ix(n - \xi)}dx =
%		\begin{cases}0, \qquad & n \neq \xi
%			\\
%			2 \pi, \qquad & n = \xi
%		\end{cases}
%	\end{split}
%\end{equation*}
%%
%%
%we obtain
%%
%%
%\begin{equation}
%	\label{oscill-bound}
%	\begin{split}
%		\|e^{inx}\|_{H^{s}(\ci)} & = \sqrt{2 \pi}(1 + | n |)^{s}
%		\\
%		& \le | n |^{s}
%	\end{split}
%\end{equation}
%%
%%
%and so
%%
%%
%\begin{equation*}
%	\begin{split}
%		\|u_{n}(\cdot, t)\|_{H^s{(\ci)}} = \frac{1}{2}|n|^{-s}
%		\|e^{inx}\|_{H^{s}(\ci)} \le \frac{1}{2}.
%	\end{split}
%\end{equation*}
%%
%%
%Next, let
%%
%%
%\begin{equation*}
%	\begin{split}
%		u_{k_{n}}(x,t) = k_{n}n^{-s}e^{it\left( n^2 + k_{n}^2 n^{-2s}
%		\right)}e^{inx}.
%	\end{split}
%\end{equation*}
%%
%%
%Following our preceding computations, it is easy to show that $u_{n, k_{n}}$ is a solution to the ivp
%%
%%
%\begin{equation}
%	\label{family-ivp}
%	\begin{split}
%		i\p_t u + \p_x^2 + | u |^{2}u = 0,
%		\\
%		u(x,0) = k_{n}n^{-s}e^{inx}
%	\end{split}
%\end{equation}
%%
%%
%and satisfies 
%%
%%
%\begin{equation*}
%	\begin{split}
%		\|u_{n, k_{n}}(\cdot, t)\|_{H^{s}(\ci)} \le k_{n}.
%	\end{split}
%\end{equation*}
%%
%%
%Furthermore, choosing $\{k_{n}\}_{n} \subset (0, 1/2)$ to be a family of
%rational numbers converging to $k =1/2$, we have
%%
%%
%\begin{equation*}
%	\begin{split}
%		\|u(x,0) - u_{n, k_{n}}(x, 0) \|_{H^s(\ci)} 
%		& =
%		\|\frac{1}{2}n^{-s}e^{inx} - k_{n}n^{-s}e^{inx} \|_{H^s(\ci)}
%		\\
%		& = | n |^{-s} \|e^{inx}(\frac{1}{2} - k_{n})\|_{H^s(\ci)}
%		\\
%		& = \frac{(1+| n |)^{s}}{|n|^{s}} |\frac{1}{2} - k_{n}| \to 0
%	\end{split}
%\end{equation*}
%%
%%
%and
%%
%%
%\begin{equation*}
%	\begin{split}
%		& \|u_{n}(\cdot, t) - u_{n, k_{n}}(\cdot, t) \|_{H^{s}(\ci)}
%		\\
%		& = \|\frac{1}{2}n^{-s}e^{it\left( n^2 + \frac{1}{4}n^{-2s}
%		\right)}e^{inx} - k_{n}n^{-s}e^{it\left( n^2 + k_{n}^{2}n^{-2s}
%		\right)}e^{inx} \|_{H^{s}(\ci)}
%		\\
%		& = | n |^{-s} \|e^{it\left( n^2 + \frac{1}{4}n^{-2s}
%		\right)}e^{inx}\left( \frac{1}{2} - k_{n}e^{it\left(
%		k_{n}^{2}n^{-2s}-\frac{1}{4}n^{-2s} \right)} \right)\|_{H^{s}(\ci)}
%		\\
%		& = | n^{-s} |\|e^{inx}\left( \frac{1}{2} - k_{n}e^{it\left(
%		k_{n}^{2}n^{-2s} - \frac{1}{4}n^{-2s}
%		\right)} \right) \|_{H^{s}(\ci)}
%		\\
%		& = \sqrt{2 \pi} \frac{(1 + | n |^s)}{|n|^s} | \frac{1}{2} - k_{n}e^{itn^{-2s}\left( k_{n}^{2}- \frac{1}{4}\right)} |
%	\end{split}
%\end{equation*}
%%
%%
%where the last step follows from \eqref{oscill-bound}. Hence, in order for uniform continuity of the flow map to hold, we must have
%%
%%
%\begin{equation*}
%	\begin{split}
%		\lim_{n \to \infty}  k_{n} e^{itn^{-2s}\left( k_{n}^{2} -
%		\frac{1}{4} \right)}  = \frac{1}{2}.
%	\end{split}
%\end{equation*}
%%
%%
%But setting $k_{n} = \left( \frac{1}{4} + n^{2s + \ee} \right)^{1/2}$ where  $\ee >
%0$, we see that 
%%
%%
%\begin{equation*}
%	\begin{split}
%		\lim_{n \to \infty} k_{n} e^{itn^{-2s}\left( k_{n}^{2} - \frac{1}{4}
%		\right)} = \lim_{n \to \infty} k_{n} e^{itn^{\ee}} \neq \frac{1}{2}.
%	\end{split}
%\end{equation*}
%%
%%
%In fact, the above limit does not converge at all. This concludes the proof for
%the case $m \in \{4, 8, 12, \dots \}$. For the case $m \in \{2, 6, 10, \dots \}$, we take
%%
%%
%\begin{equation*}
%	\begin{split}
%		u_{n}(x,t) = \frac{1}{2}n^{-s}e^{it\left( -n^2 + \frac{1}{4}n^{-2s}
%		\right)}e^{inx},
%		\\ u_{n, k_{n}}(x,t) = k_{n}n^{-s}e^{it\left( -n^2 + k_{n}^{2}n^{-2s}
%		\right)}e^{inx} 
%	\end{split}
%\end{equation*}
%and then repeat the above arguments. \qquad \qedsymbol
%%
%%
%\begin{remark}
%	Note that this result implies that it will be impossible to use a Picard
%	iteration type argument to prove existence and uniqueness of solutions to the
%	NLS ivp for $s<0$, since this technique would imply uniform
%	continuity of the flow map.
%\end{remark}
%
%
%\section{Failure of Continuity of the Flow Map}
%We shall prove the following.
%%
%%
%%%%%%%%%%%%%%%%%%%%%%%%%%%%%%%%%%%%%%%%%%%%%%%%%%%%%%
%%
%%
%%				 Failure of Continuity Theorem
%%
%%
%%%%%%%%%%%%%%%%%%%%%%%%%%%%%%%%%%%%%%%%%%%%%%%%%%%%%%
%%
%%
%\begin{theorem}
%The flow map $u_0 \mapsto u(t)$ of the NLS ivp is not continuous for $s<0$, for
%any $t \neq 0$. More precisely, there exists initial data $u_0 \in L^2(\ci)$ and
%a sequence of intial data $\{u_{0,n} \} \subset L^2(\ci)$ such that
%%
%%
%\begin{equation*}
%	\begin{split}
%		u_{0,n} \to u_{0} \ \ \text{in} \ \ H^s(\ci)
%	\end{split}
%\end{equation*}
%%
%%
%and
%%
%%
%\begin{equation*}
%	\begin{split}
%		u_{n} \to e^{ \frac{it \gamma}{\pi}\left( \alpha^2 - \|u_{0}\|_{L^2(\ci)} 
%		\right)}u(t) \ \ \text{in} \ \ H^s(\ci)
%	\end{split}
%\end{equation*}
%%
%%
%where $\alpha \in \rr \setminus \|u_0\|_{L^2(\ci)}$ and $u$ is the unique solution to the NLS ivp with
%initial data $u_{0}$.
%\end{theorem}
%%
%%
%{\bf Proof.} Define the trilinear operator
%%
%%
%\begin{equation*}
%	\begin{split}
%		g(u,v,w) \doteq \bar{u} v w.
%	\end{split}
%\end{equation*}
%%
%%
%Following Molinet, we rewrite this as
%%
%%
%\begin{equation*}
%	\begin{split}
%		g(u,v,w) 
%		&= \sum_{k_1, k_2, k_3 \in \zz}
%		\wh{\bar{u}}(k_{1})\wh{v}(k_2)\wh{w}(k_3)e^{i(k_1 + k_2 + k_3)x}
%		\\
%		& = \sum_{k_1, k_3 \in \zz} \wh{\bar{u}}(k_1)\wh{v}(-k_1)
%		\wh{w}(k_3)e^{ik_3 x} + \sum_{k_1, k_2 \in \zz} \wh{\bar{u}}(k_1)
%		\wh{v}(k_2) \wh{w}(-k_1)e^{ik_2x}
%		\\
%		& - \sum_{ k \in \zz} \wh{\bar{u}}(k) \wh{v}(-k)\wh{w}(-k)e^{-ikx}
%		+
%		\sum_{\substack{k_1, k_2, k_3 \in \zz\\ (k_1 + k_2)(k_1 + k_3) \neq 0}}
%		\wh{\bar{u}}(k_1) \wh{v}(k_2)
%		\wh{w}(k_3)e^{i(k_1 + k_2 + k_3)x}.
%	\end{split}
%\end{equation*}
%%
%%
%In particular, if $u=v=w$, we obtain
%%
%%
%\begin{equation*}
%	\begin{split}
%		g(u,u,u) = \frac{1}{\pi} \|u\|_{L^2}^2 u + \Lambda_1(u, u, u) + \Lambda_2
%		(u, u, u)	
%	\end{split}
%\end{equation*}
%%
%%
%where
%%
%%
%\begin{equation*}
%	\begin{split}
%		& \Lambda_1(u, v, w)
%		 = \sum_{\substack{k_1, k_2, k_3 \in \zz\\ (k_1 + k_2)(k_1 + k_3) \neq 0}}
%		\wh{\bar{u}}(k_1) \wh{v}(k_2)
%		\wh{w}(k_3)e^{i(k_1 + k_2 + k_3)x}
%		\\
%		& \Lambda_2(u, v, w) = - \sum_{ k \in \zz} \wh{\bar{u}}(k) \wh{v}(-k)\wh{w}(-k)e^{-ikx}
%	\end{split}
%\end{equation*}
%
%
%
%
%
%
\appendix
\section{}
\subsection{Proof of \autoref{lem:cutoff-loc-soln}}
%
%
\begin{equation*}
	\begin{split}
		\lim_{t_{n} \to t} \|u(\cdot, t) - u(\cdot, t_{n})\|_{H^s(\ci)} 
		& = \lim_{t_{n} \to t} \|\psi(t) u(\cdot, t) - \psi(t_n) u(\cdot, t_{n})\|_{H^s(\ci)} 
		\\
		& = \lim_{t_n \to t} \left[ \sum_{n \in \zz}\left( 1 + | n |
		\right)^{2s} | \psi(t)  \wh{u}(n, t) - \psi(t_n) \wh{ u}(n, t_n) |^2 \right]^{1/2}
		\\
		& = \lim_{t_n \to t} \left[ \sum_{n \in \zz} \left( 1 + | n |
		\right)^{2s} | \int_{\rr} (e^{it \tau} - e^{it_{n} \tau}) \wh{\psi u}(n,
		\tau) d \tau |^2 \right]^{1/2}.
	\end{split}
\end{equation*}
		It is clear that
		%
		%
		\begin{equation*}
			\begin{split}
				\left( 1 + | n |
				\right)^{2s} | \int_{\rr} (e^{it \tau} - e^{it_{n}\tau}) \wh{\psi u}(n, \tau) d \tau |^2 
		& \le 4  \left( 1 + | n |
		\right)^{2s} \left ( \int_{\rr} |\wh{\psi u}(n, \tau)| d \tau
		\right )^2 
	\end{split}
\end{equation*}
and 
%
%
\begin{equation*}
	\begin{split}
 \sum_{n \in \zz} \left( 1 + | n |
		\right)^{2s} \left ( \int_{\rr} |\wh{\psi u}(n, \tau)| d \tau
		\right ) ^2 
		& = \|\wh{\psi u}\|_{\ell_{n}^2 L^{1}_{\tau}}
		\\
		& \le \|\psi u \|_{Y^s}^2 
	\end{split}
\end{equation*}
which is bounded by assumption.
Applying dominated convergence completes the proof. \qquad \qedsymbol
%
%
%
%
%
%\subsection{Conservation of the $L_x^2$ norm.} 
%We have
%%
%%
%\begin{equation*}
%	\begin{split}
%		\frac{d}{dt} \int_\ci | u |^2  dx
%		& = \int_\ci \frac{d}{dt} | u |^2  dx
%		\\
%		& = \int_\ci \frac{d}{dt} \left( u \overline{u} \right)  dx
%		\\
%		& = \int_\ci \left( u \p_t \overline{u} + \overline{u} \p_t u \right) dx
%		\\
%		& = \int_\ci \left( u \overline{\p_t u} + \overline{u} \p_t u \right)dx.
%	\end{split}
%\end{equation*}
%%
%%
%Substituting in $\p_t u = i\left( \p_x^2 u + | u |^2 u \right)$ we obtain
%%
%%
%\begin{equation*}
%	\begin{split}
%		& \int_{\ci} \left\{ u\left[ -i\left( \p_x^2 \overline{u} + | u |^2
%		\overline{u} \right) \right] + \overline{u}\left[ i\left( \p_x^2 u + | u
%		|^2 u \right) \right] \right\}dx
%		\\
%		& = \int_\ci \left[ -iu \p_x^2 \overline{u} - i| u |^4 + i \overline{u}
%		\p_x^2 u + i | u |^4 \right]dx
%		\\
%		& = i \int_{\ci}\left( \overline{u} \p_x^2 u - u \p_x^2 \overline{u}
%		\right)dx.
%	\end{split}
%\end{equation*}
%%
%%
%Integrating by parts $m/2$ times and using
%the spatial periodicity of $u$, the right
%hand side simplifies to
%%
%%
%\begin{equation*}
%	\begin{split}
%		i \int_\ci \left( \p_x^{m/2} \overline{u} \p_x^{m/2} u - \p_x^{m/2} u
%		\p_x^{m/2 } 
%		\overline{u} \right) dx = 0.
%	\end{split}
%\end{equation*}
%%
%%
%Therefore, the $L_x^2(\ci)$ norm of solutions to the NLS is conserved. \quod
%
%
\subsection{Proof of \autoref{lem:splitting}.} We have
%
%
\begin{equation}
	\label{6a}
	\begin{split}
		1 + | a + b + c| 
		& \le 1 + | a | + | b | + | c |
		\\
		& \le 1 + | a | + 1 + | b | + 1 + | c |
		\\
		& \le 3\left( \max\{1+| a |, 1+| b |, 1+ | c | \}\right)
		\\
		& \le 3 \left( 1 + | a | \right)\left( 1 + | b | \right) \left( 1 + |
		c |
		\right), \qquad a, b, c \in \zz.
	\end{split}
\end{equation}
%
%
Raising both sides of expression $\eqref{6a}$ to the $v$ power completes 
the proof. \qquad \qedsymbol 
%
%
%
%
% \bib, bibdiv, biblist are defined by the amsrefs package.
\begin{bibdiv}
\begin{biblist}
\bib{Burq_Gerad_Tzvetkov-An-instability-}{article}{
      author={Burq, N.},
      author={G{{\'e}}rad, P.},
      author={Tzvetkov, N.},
       title={An instability property of the nonlinear {S}chr{\"o}dinger
  equation on {$S^d$}},
        date={2002},
        ISSN={1073-2780},
     journal={Math. Res. Lett.},
      volume={9},
      number={2-3},
       pages={323\ndash 335},
      review={\MR{MR1909648 (2003c:35144)}},
}
\bib{Bourgain-Fourier-transfo-1}{article}{
      author={Bourgain, J.},
       title={Fourier transform restriction phenomena for certain lattice
  subsets and applications to nonlinear evolution equations. {I}.
  {S}chr{\"o}dinger equations},
        date={1993},
        ISSN={1016-443X},
     journal={Geom. Funct. Anal.},
      volume={3},
      number={2},
       pages={107\ndash 156},
         url={http://dx.doi.org/10.1007/BF01896020},
      review={\MR{MR1209299 (95d:35160a)}},
}
\bib{Bourgain-Fourier-transfo}{article}{
      author={Bourgain, J.},
       title={Fourier transform restriction phenomena for certain lattice
  subsets and applications to nonlinear evolution equations. {II}. {T}he
  {K}d{V}-equation},
        date={1993},
        ISSN={1016-443X},
     journal={Geom. Funct. Anal.},
      volume={3},
      number={3},
       pages={209\ndash 262},
         url={http://dx.doi.org/10.1007/BF01895688},
      review={\MR{MR1215780 (95d:35160b)}},
}
\bib{Bourgain-1999-Global-solutions-of-nonlinear}{book}{
      author={Bourgain, J.},
       title={Global solutions of nonlinear {S}chr{\"o}dinger equations},
      series={American Mathematical Society Colloquium Publications},
   publisher={American Mathematical Society},
     address={Providence, RI},
        date={1999},
      volume={46},
        ISBN={0-8218-1919-4},
      review={\MR{MR1691575 (2000h:35147)}},
}
\bib{Colliander_Keel_Staffilani_Takaoka_Tao-A-refined-globa}{article}{
      author={Colliander, J.},
      author={Keel, M.},
      author={Staffilani, G.},
      author={Takaoka, H.},
      author={Tao, T.},
       title={A refined global well-posedness result for {S}chr{\"o}dinger
  equations with derivative},
        date={2002},
        ISSN={0036-1410},
     journal={SIAM J. Math. Anal.},
      volume={34},
      number={1},
       pages={64\ndash 86 (electronic)},
         url={http://dx.doi.org/10.1137/S0036141001394541},
      review={\MR{MR1950826 (2004c:35381)}},
}
\bib{Colliander_Keel_Staffilani_Takaoka_Tao-Multilinear-est}{article}{
      author={Colliander, J.},
      author={Keel, M.},
      author={Staffilani, G.},
      author={Takaoka, H.},
      author={Tao, T.},
       title={Multilinear estimates for periodic {K}d{V} equations, and
  applications},
        date={2004},
        ISSN={0022-1236},
     journal={J. Funct. Anal.},
      volume={211},
      number={1},
       pages={173\ndash 218},
         url={http://dx.doi.org/10.1016/S0022-1236(03)00218-0},
      review={\MR{MR2054622 (2005a:35241)}},
}
\bib{Folland_1999_Real-analysis}{book}{
      author={Folland, Gerald~B.},
       title={Real analysis},
     edition={Second},
      series={Pure and Applied Mathematics (New York)},
   publisher={John Wiley \& Sons Inc.},
     address={New York},
        date={1999},
        ISBN={0-471-31716-0},
        note={Modern techniques and their applications, A Wiley-Interscience
  Publication},
      review={\MR{MR1681462 (2000c:00001)}},
}
\bib{Gorsky_2007_Well-posedness-}{article}{
      author={Gorsky, Jennifer},
      author={Himonas, Alex},
       title={Well-posedness for a class of nonlinear dispersive equations},
        date={2007},
        ISSN={1201-3390},
     journal={Dyn. Contin. Discrete Impuls. Syst. Ser. A Math. Anal.},
      volume={14},
      number={Advances in Dynamical Systems, suppl. S2},
       pages={85\ndash 90},
      review={\MR{MR2384111 (2008m:35305)}},
}
\bib{Grunrock_Herr-Low-regularity-}{article}{
      author={Gr{{\"u}}nrock, Axel},
      author={Herr, Sebastian},
       title={Low regularity local well-posedness of the derivative nonlinear
  {S}chr{\"o}dinger equation with periodic initial data},
        date={2008},
        ISSN={0036-1410},
     journal={SIAM J. Math. Anal.},
      volume={39},
      number={6},
       pages={1890\ndash 1920},
         url={http://dx.doi.org/10.1137/070689139},
      review={\MR{MR2390318 (2009a:35233)}},
}
\bib{Grunrock-thesis}{thesis}{
      author={Gr{{\"u}}nrock, Axel},
       title={New applications of the fourier restriction norm method to
  wellposedness problems for nonlinear evolution equations},
        type={Ph.D. Thesis},
        date={2002},
}
\bib{Grunrock-Bi--and-triline}{article}{
      author={Gr{{\"u}}nrock, Axel},
       title={Bi- and trilinear {S}chr{\"o}dinger estimates in one space
  dimension with applications to cubic {NLS} and {DNLS}},
        date={2005},
        ISSN={1073-7928},
     journal={Int. Math. Res. Not.},
      number={41},
       pages={2525\ndash 2558},
      review={\MR{MR2181058 (2007b:35298)}},
}
\bib{Herr-On-the-Cauchy-p}{article}{
      author={Herr, Sebastian},
       title={On the {C}auchy problem for the derivative nonlinear
  {S}chr{\"o}dinger equation with periodic boundary condition},
        date={2006},
        ISSN={1073-7928},
     journal={Int. Math. Res. Not.},
       pages={Art. ID 96763, 33},
      review={\MR{MR2219223 (2007e:35258)}},
}
\bib{Himonas-Misiolek-2001-A-priori-estimates-for-Schrodinger}{article}{
      author={Himonas, A.~Alexandrou},
      author={Misiolek, Gerard},
       title={A priori estimates for {S}chr{\"o}dinger type multipliers},
        date={2001},
        ISSN={0019-2082},
     journal={Illinois J. Math.},
      volume={45},
      number={2},
       pages={631\ndash 640},
      review={\MR{MR1878623 (2002j:42018)}},
}
\bib{Himonas_Misioek-The-Cauchy-prob}{article}{
      author={Himonas, A.},
      author={Misio{\l}ek, Gerard},
       title={The {C}auchy problem for a shallow water type equation},
        date={1998},
        ISSN={0360-5302},
     journal={Comm. Partial Differential Equations},
      volume={23},
      number={1-2},
       pages={123\ndash 139},
         url={http://dx.doi.org/10.1080/03605309808821340},
      review={\MR{MR1608504 (99b:35176)}},
}
\bib{Schlotthauer-Hannah-Well-posedness-}{thesis}{
      author={Schlotthauer-Hannah, Heather},
       title={Well-posedness and regularity for a higher order periodic mkdv
  equation},
        type={Dissertation},
     address={Notre Dame, Indiana},
        date={2007},
}
\bib{Tao-Multilinear-wei}{article}{
      author={Tao, Terence},
       title={Multilinear weighted convolution of {$L^2$}-functions, and
  applications to nonlinear dispersive equations},
        date={2001},
        ISSN={0002-9327},
     journal={Amer. J. Math.},
      volume={123},
      number={5},
       pages={839\ndash 908},
  url={http://muse.jhu.edu/journals/american_journal_of_mathematics/v123/123.5%
tao.pdf},
      review={\MR{MR1854113 (2002k:35283)}},
}
\bib{Taylor_1991_Pseudodifferent}{book}{
      author={Taylor, Michael~E.},
       title={Pseudodifferential operators and nonlinear {PDE}},
      series={Progress in Mathematics},
   publisher={Birkh{\"a}user Boston Inc.},
     address={Boston, MA},
        date={1991},
      volume={100},
        ISBN={0-8176-3595-5},
      review={\MR{MR1121019 (92j:35193)}},
}
\bib{Tzvetkov_2006_Ill-posedness-i}{incollection}{
      author={Tzvetkov, Nikolay},
       title={Ill-posedness issues for nonlinear dispersive equations},
        date={2006},
   booktitle={Lectures on nonlinear dispersive equations},
      series={GAKUTO Internat. Ser. Math. Sci. Appl.},
      volume={27},
   publisher={Gakk\=otosho},
     address={Tokyo},
       pages={63\ndash 103},
      review={\MR{MR2404974}},
}
\end{biblist}
\end{bibdiv}
%
%
%\nocite{*}
%\bibliography{/Users/davidkarapetyan/Documents/Math/schrodinger.bib}
\end{document}
