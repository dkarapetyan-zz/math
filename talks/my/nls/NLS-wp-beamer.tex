\documentclass[handout]{beamer}
\hypersetup{colorlinks=true,
linkcolor=blue,
citecolor=red,
urlcolor=red,
}

% \usetheme{default}
 \usetheme{Boadilla}
 %\usetheme{Madrid}
% \usetheme{Montpellier}
% \usetheme{Warsaw}
% \usetheme{Copenhagen}
% \usetheme{Goettingen}
% \usetheme{Hannover}
%\usetheme{Berkeley}
\numberwithin{equation}{section}
\setbeamertemplate{footline}[page number]{}
\usepackage{lmodern}
\usepackage{cancel}
\synctex=1

\setlength{\parindent}{0in} %no indentation of paragraphs after section title
%
\newcommand{\bigno}{\bigskip\noindent}
\newcommand{\ds}{\displaystyle}
\newcommand{\medno}{\medskip\noindent}
\newcommand{\smallno}{\smallskip\noindent}
\newcommand{\nin}{\noindent}
\newcommand{\ts}{\textstyle}
\newcommand{\rr}{\mathbb{R}}
\newcommand{\p}{\partial}
\newcommand{\zz}{\mathbb{Z}}
\newcommand{\cc}{\mathbb{C}}
\newcommand{\ci}{\mathbb{T}}
\newcommand{\tor}{\mathbb{T}}
\newcommand{\ee}{\varepsilon}
\newcommand{\wh}{\widehat}
\newcommand{\weak}{\rightharpoonup}
\newcommand{\vp}{\varphi}
%
%

\newtheorem{proposition}{Proposition}
\newtheorem{claim}{Claim}
\newtheorem{remark}{Remark}
\newtheorem{conjecture}[subsection]{conjecture}

%%%%%%%%%%%%%%%%%%%%%%
%
\date{}
\title{ Well-Posedness for the Nonlinear Schr\"{o}dinger Equation  } 
\author{David Karapetyan}
\institute{Notre Dame}
\begin{document}

\begin{frame}
	\titlepage
\end{frame}


\section*{Table of Contents}
\setcounter{section}{0}
\begin{frame}
		\frametitle{Table Of Contents}
			\tableofcontents
		\end{frame}

		\section{Introduction}
		\begin{frame}
			\frametitle{Introduction}
We consider the nonlinear Schr\"{o}dinger (NLS) 
				  initial value problem (ivp)
%
%
\begin{gather}
	\label{NLS-eq}
	i \p_t v = - \p_x^{2} u - |u|^2 u,
		\\
		\label{NLS-init-data}
		u(x,0) = \vp(x), \ \ t \in \rr, \ \ x \in \ci
\end{gather}
%
%
and prove the following result.
\end{frame}


\begin{frame}
%
%
%
%
%%%%%%%%%%%%%%%%%%%%%%%%%%%%%%%%%%%%%%%%%%%%%%%%%%%%%
%
%
%	Main Result				
%
%
%%%%%%%%%%%%%%%%%%%%%%%%%%%%%%%%%%%%%%%%%%%%%%%%%%%%%
%
%
\begin{theorem}
	\label{thm:main}
	For $s\ge 0$, the initial value problem 
	\eqref{NLS-eq}-\eqref{NLS-init-data} has a unique solution in the space   
	$X_s$, which is defined to be the completion of 
	all functions Schwartz in the time variable and $L^2$ in the space 
	variable under the norm
%
%
%
%
\begin{equation}
	\label{X_s-norm}
	\begin{split}
		& \|u\|_{X_s}
		= \left( \sum_{n\in \zz} \left(1 + |n| \right)^{2s} \int_\rr \left( 1 + | 
		\tau - n^{2 } \right ) | \wh{u}(n, \tau) |^2
		\right )^{1/2}.
	\end{split}
\end{equation}
%
%
This norm is understood as a restriction norm with respect to the time 
variable, i.e. $u = \psi u$ for a suitable cutoff $\psi(t)$. Furthermore, the 
flow map from $H^s(\ci)$ to $X_s$ is locally Lipschitz continuous. 
%
%
\end{theorem} 

\end{frame}

\section{A Weak Formulation of the NLS ivp}
\begin{frame}
	\frametitle{A Weak Formulation of the NLS ivp}
We first derive a weak formulation of the NLS ivp. 
Let $\ci = [0, 2 \pi]$, and use
the following notation for the Fourier transform
%
%
%
%
\begin{equation*}
	\begin{split}
		\widehat{f}(n) = \int_{\ci} e^{-ix n} f(x) \ dx.
	\end{split}
\end{equation*}
Assume 
$u(x,t)$ is a classical solution of \eqref{NLS-eq}-\eqref{NLS-init-data}.
Then $u(\cdot ,   t): \rr \to C^{2}(\ci)$ is a $C^1$ map. Therefore, letting 
$w = | u |^2 u$ and applying 
the Fourier transform to the NLS ivp in the space variable we obtain 
%
%
\end{frame}

\begin{frame}
\begin{gather*}
	\p_t \widehat{u}(n, t) = -i n^2 \widehat{u}(n, t) + i  
	\widehat{w} (n, t),
	\\
	\widehat{u} (n,0) = \widehat{\vp}(n)
\end{gather*}
%
%
which is a globally well-defined relation in $t$ 
and $n$. Multiplication by the integrating factor $e^{itn^{2}}$ then yields
%%
%%
\begin{equation*}
	\begin{split}
		\left[ e^{it n^{2}} \widehat{u}(n) \right]_t = i
		 e^{it n^{2}} \widehat{w} (n, t).	
	\end{split}
\end{equation*}
%
%
Integrating from $0$ to $t$, we obtain
%
%
\begin{equation*}
	\begin{split}
		\wh{u}(n, t) = \wh{\vp}(n) e^{-it n^{2}} + i  
		\int_0^t e^{i(t' - t)n^{2}} \wh{w}(n, t') \ 
		dt'.
	\end{split}
\end{equation*}
%
%
By assumption, $u(t) \in C^2(\ci)$, which implies $u(t) \in L^6(\ci)$. This 
in turn implies $w(t) \in L^2(\ci)$. Therefore, by Fourier inversion 
%
%
\end{frame}

\begin{frame}
\begin{equation}
	\label{NLS-integral-form}
	\begin{split}
		u(x,t) & = \sum_{n \in \zz} \wh{\vp}(n) e^{i\left( xn - t n^{2} 
		\right)} 
		\\
		& + i \sum_{n \in \zz} \int_0^t e^{i\left[ xn + \left( t' - t 
		\right)n^{2} \right]} \wh{w}(n, t') \ dt'
	\end{split}
\end{equation}
%
%
where the second series converges in the usual $L^2$ sense.
%
%
Hence, 
\end{frame}

\begin{frame}

\begin{equation*}
	\begin{split}
		u(x,-t) & = \sum_{n \in \zz} \wh{\vp}(n) e^{i\left( xn + t n^{2} 
		\right)} 
		+ i \sum_{n \in \zz} \int_0^{-t} e^{i\left[ xn + \left( t' + t 
		\right)n^{2} \right]} \wh{w}(n, t') \ dt'
		\\
		& = \sum_{n \in \zz} \wh{\vp}(n) e^{i\left( xn + t n^{2} 
		\right)} 
		- i \sum_{n \in \zz} \int_0^t e^{i\left[ xn + \left( t - t' 
		\right)n^{2} \right]} \wh{w}(n, -t') \ dt'
	\end{split}
\end{equation*}
It is immediate that \eqref{NLS-integral-form} is a weaker 
restatement of the Cauchy-problem \eqref{NLS-eq}-\eqref{NLS-init-data}, 
since by construction any classical solution of the NLS 
ivp is a solution to \eqref{NLS-integral-form}. This motivates the 
following.
%
%
\end{frame}


\begin{frame}

\begin{definition}
	%
	%
	We say $u(x,t)$ is a \emph{local solution} of the NLS ivp
\eqref{NLS-eq}-\eqref{NLS-init-data} if it satisfies the weak formulation 
\eqref{NLS-integral-form} for $t \in I \subset \rr$. If $I = \rr$, then we say 
$u$ is a \emph{global solution}. 
%
%
\end{definition}
%
%
%
\end{frame}

\begin{frame}

We now derive an integral 
equation global in $t$ and equivalent to \eqref{NLS-integral-form} for $t 
\in [0, \delta]$. Let $\psi(t)$ be a cutoff function symmetric about the 
origin such that $\psi(t) = 1$ for $|t| \le \delta$ and $\text{supp} \, \psi 
= [-2\delta, 2\delta ]$. Multiplying the right hand side of expression
$\eqref{NLS-integral-form}$ by $\psi(t)$, we obtain
%
%
\begin{equation}
	\begin{split}
		\label{cutoff-int-eq}
		u(x, t)
		& = \frac{1}{2 \pi} \psi(t) \sum_{n \in \zz} e^{i(xn - tn^{2
		})} \widehat{\vp}(n) 
		\\
		& + \frac{i }{2 \pi} \psi(t) \int_0^t \sum_{n \in \zz} 
		e^{i\left[ xn + (t - t')n^{2} \right]} \wh{w}(n, t') \ dt'.
	\end{split}
\end{equation}
%
%
Noting that $e^{i\left( xn + tn^{2 } \right)}$ 
does not depend on $t'$, we may rewrite

\end{frame}


\begin{frame}
%
%
\begin{equation}
	\label{pre-prim-int-form}
	\begin{split}
		& \frac{i }{2 \pi} \psi(t) \int_0^t \sum_{n \in \zz} 
		e^{i\left[ xn + (t - t')n^{2} \right]} \wh{w}(n, t') \ dt'
		\\
		& = \frac{i}{2 \pi} \psi(t) \sum_{n \in \zz} e^{i\left( xn + t 
		n^{2 } 
		\right)} \int_0^t e^{-it'n^{2 }} \wh{w}(n, t') \ dt'.
	\end{split}
\end{equation}
%%
%%
We remark that this is a \emph{global} relation in $t$. Therefore, by Fourier 
inversion
\end{frame}
%
%
%
%
%
\begin{frame}
%
%
\begin{equation*}
	\begin{split}
		\text{rhs of} \; \eqref{pre-prim-int-form}
		& = \frac{i}{4 \pi^2} \psi (t) \sum_{n \in \zz} e^{i\left( xn + t 
		n^{2 }
		\right)} \int_0^t \int_\rr e^{it'\left( \tau - n^{2 } \right) }
		\wh{w}(n, \tau) d \tau dt'
		\\
		& = \frac{i}{4 \pi^2} \psi(t) \sum_{n \in \zz} \int_\rr 
		e^{i\left( xn + tn^{2} \right)} \frac{e^{it\left( \tau - n^{2 } 
		\right)}-1}{\tau - n^{2 }} \wh{w}(n, \tau) d \tau
	\end{split}
\end{equation*}
%
%
where the last step follows from integration. Substituting
into \eqref{cutoff-int-eq} we obtain

\end{frame}

\begin{frame}
%
%
\begin{equation}
	\begin{split}
		\label{cutoff-int-eq-2}
		u(x, t)
		& = \frac{1}{2 \pi} \psi(t) \sum_{n \in \zz} e^{i(xn - tn^{2  
		})} \widehat{\vp}(n) 
		\\
		& + \frac{1}{4 \pi^2} \psi(t) \sum_{n \in \zz} \int_\rr
		e^{i(xn + tn^{2})} \frac{e^{it(\tau - n^{2})}- 1}{\tau - n^{2}} 
		\wh{w}(n, \tau) \ d \tau.
	\end{split}
\end{equation}
%
%
\end{frame}

\begin{frame}
%
%
%
Next, we localize near the singular curve $\tau = n^2$.  Multiplying the summand of the
second term of \eqref{cutoff-int-eq-2} by $1 \pm \psi(\tau - n^2)$ and
rearranging terms, we have

%
%
\begin{equation*}
	\begin{split}
		 u(x, t)
		& = \frac{1}{2 \pi} \psi(t) \sum_{n \in \zz} e^{i(xn + tn^{2 
		})} \widehat{\vp}(n) 
		\\
		& + \frac{1}{4 \pi^2} \psi(t) \sum_{n \in \zz} \int_\rr e^{ixn}  
		e^{it \tau} \frac{ 1 - \psi(\tau - n^{2}) 
		}{\tau - n^{2}} \wh{w}(n, \tau) \ d \tau
		\\
		& - \frac{1}{4 \pi^2} \psi(t) \sum_{n \in \zz} \int _\rr e^{i(xn + 
		tn^{2})}
		 \frac{1- \psi(\tau - n^{2})}{\tau - n^{2}} \wh{w}(n, \tau) \ d \tau
		\\
		& + \frac{1}{4 \pi^2} \psi(t) \sum_{n \in \zz} \int_\rr
		e^{i(xn + tn^{2})}
		\frac{\psi(\tau - n^{2})\left[ e^{it(\tau - n^{2})}-1 
		\right]}{\tau - n^{2}} \wh{w}(n, \tau) \ d \tau
	\end{split}
\end{equation*}
%
%
which by a power series expansion of $[e^{it(\tau - n^{2})}-1]$ simplifies  
to
\end{frame}

\subsection{NLS Integral Equation}
\begin{frame}
	\frametitle{NLS Integral Equation}
%
%
\begin{align}
	\label{main-int-expression-0}
	& u(x, t) 
		\\
		\label{main-int-expression-1}
		& = \frac{1}{2 \pi} \psi(t) \sum_{n \in \zz} e^{i(xn + tn^{2 
		j})} \widehat{\vp}(n) 
		\\
		\label{main-int-expression-2}
		& + \frac{1}{4 \pi^2} \psi(t) \sum_{n\in \zz} \int_\rr e^{ixn}  
		e^{it \tau} \frac{ 1 - \psi(\tau - n^{2}) 
		}{\tau - n^{2}} \wh{w}(n, \tau) \ d \tau
		\\
		\label{main-int-expression-3}
		& - \frac{1}{4 \pi^2} \psi(t) \sum_{n\in \zz} \int_\rr e^{i(xn + 
		tn^{2})}
		 \frac{1- \psi(\tau - n^{2})}{\tau - n^{2}} \wh{w}(n, \tau) \ d \tau
		\\
		\label{main-int-expression-4}
		& + \frac{1}{4 \pi^2} \psi(t) \sum_{k \ge 1} \frac{i^k t^k}{k!}
		\sum_{n \in \zz} \int_\rr e^{i(xn + tn^{2} )}
		\psi(\tau - n^{2}) (\tau - n^{2})^{k-1} \wh{w}(n, \tau)  
		\\
		& \doteq T(u) \notag
\end{align}
%
%
where $T = T(\vp)$.

\end{frame}


\begin{frame}

Note that 
\eqref{main-int-expression-0}-\eqref{main-int-expression-4} is a global 
relation in $t$; furthermore, the fixed point solution $Tu=u$ gives rise to a 
local solution of the NLS ivp by simply restricting the time 
variable to 
the $[-\delta, \delta]$ interval. Hence, we focus our attention on 
\eqref{main-int-expression-0}-\eqref{main-int-expression-4}. We we will 
show that for initial data $\vp \in H^s(\ci)$, $T$ is a contraction on $B_M 
\subset X_s$, where $B_M$ is the ball centered at the origin of radius $M = 
M_{\vp}> 0$, by estimating the $X_s$
norm of \eqref{main-int-expression-0}-\eqref{main-int-expression-4}. The 
Picard fixed point theorem and time restriction will
then yield a unique local solution to the NLS ivp in the time interval
$I = [-\delta, \delta]$. Local Lipschitz continuity of the flow map will follow
from estimates used to establish the contraction mapping.
%
%
%
\end{frame}

\section{Proof of the Theorem}
%
%
%
%%%%%%%%%%%%%%%%%%%%%%%%%%%%%%%%%%%%%%%%%%%%%%%%%%%%%
%
%
%		Estimation of Integral Equality Part 1		
%
%
%%%%%%%%%%%%%%%%%%%%%%%%%%%%%%%%%%%%%%%%%%%%%%%%%%%%%
%
%
%
\subsection{Estimate for 1st Term}
\begin{frame}
	\frametitle{Estimate for 1st Term}
%
%
Letting $f(x,t) = \psi(t) \sum_{n \in \zz} e^{i(xn + tn^{2})} 
\wh{\vp}(n)$, we have $\wh{f}(n,t) = \psi(t) \wh{\vp}(n) e^{itn^{2}}$,
from which we obtain
%
%
\begin{equation}
	\label{fourier-trans-calc}
	\begin{split}
		\wh{f}(n, \tau)
		& = \wh{\vp}(n) \int_\rr e^{-it( \tau - n^{2})} 
		\psi(t) \ d t
		= \wh{\psi}(\tau - n^{2}) \wh{\vp}(n).
	\end{split}
\end{equation}
%
%
%
%
%
%
Since $\wh{\psi}(\xi)$ is Schwartz for $|\xi| \ge \delta$, we conclude that 

\end{frame}
%
%
%
\begin{frame}

\begin{equation}
	\begin{split}
	\label{main-int1-est}
		\|\eqref{main-int-expression-1}\|_{X_s}
		& = \left(  \sum_{n\in \zz} \left(1 + |n| \right)^s \int_\rr \left( 1 + | \tau - n^{2} | \right)
		| \wh{\psi}(\tau - n^{2}) \wh{\vp}(n) |^2 d \tau \right)^{1/2} 
		\\
		& \le
		c_\psi \|\vp\|_{H^s(\ci)}.
	\end{split}
\end{equation}
%
%
\end{frame}

\subsection{Estimate for 2nd Term}

\begin{frame}
	\frametitle{Estimate for Second Term}
We now need the following lemma, whose proof is provided in the appendix.
%
%
%%%%%%%%%%%%%%%%%%%%%%%%%%%%%%%%%%%%%%%%%%%%%%%%%%%%%
%
%
%			Schwartz Multiplier	
%
%
%%%%%%%%%%%%%%%%%%%%%%%%%%%%%%%%%%%%%%%%%%%%%%%%%%%%%
%
%
\begin{lemma}
	\label{lem:schwartz-mult}
	For $\psi \in S(\rr)$,
%
%
\begin{equation}
	\label{schwartz-mult}
	\begin{split}
		\|\psi f \|_{X_s} \le c_{\psi} \|f \|_{X_s}.
	\end{split}
\end{equation}
%
%
\end{lemma}
%
%
Hence,

\end{frame}
%
%
%
\begin{frame}

\begin{equation}
	\label{main-int2-est}
	\begin{split}
		\|\eqref{main-int-expression-2}\|_{X_s} 
		& \lesssim 
		\left( \| \sum_{n \in \zz} e^{ixn} \int_\rr 
		e^{it \tau} \frac{ 1 - \psi(\tau - n^{2}) 
		}{\tau - n^{2}} \wh{w}(n, \tau) \ 
		d\tau\|_{X_s} \right)^{1/2}
		\\
		& =  \left( \sum_{n \in \zz} \left(1 + |n| \right)^{2s} \int_\rr
		(1 + |\tau - n^{2}|) \left | \frac{1 - \psi(\tau - n^{2 
		})}{\tau - n^{2}} 
		\wh{w}(n, \tau) \right |^2 \ d 
		\tau \right)^{1/2}
		\\
		& \le \left( \sum_{n \in \zz} \left(1 + |n| \right)^{2s} \int_{| \tau - n^{2 }| \ge 1}
		(1 + |\tau - n^{2}|) \frac{|\wh{w}(n, \tau)|^2 }{|\tau - n^{2 }|^2} 
		\ d 
		\tau \right)^{1/2}
		\\
		& \lesssim  \left( \sum_{n \in 
		\zz} \left(1 + |n| \right)^{2s} \int_\rr
		\frac{|\wh{w}(n, \tau) |^2}{1+ |\tau - 
		n^{2}|} 
		 \ d \tau 
		\right)^{1/2}
		\\
		& \lesssim  \|u\|_{X_s}^3
	\end{split}
\end{equation}
%
%
where the last two steps follow from the inequality 

\end{frame}

%
\begin{frame}
\begin{equation}
	\label{one-plus-ineq}
	\begin{split}
		\frac{1}{|\tau - n^{2}| } \le \frac{2}{1 + |\tau - n^{2}| }, 
		\qquad |\tau - n^{2}| \ge 1
	\end{split}
\end{equation}
%
%
and the following trilinear estimate, whose proof we leave for later.

\end{frame}


\begin{frame}
%
%
%%%%%%%%%%%%%%%%%%%%%%%%%%%%%%%%%%%%%%%%%%%%%%%%%%%%%
%
%
%				Proposition
%
%
%%%%%%%%%%%%%%%%%%%%%%%%%%%%%%%%%%%%%%%%%%%%%%%%%%%%%
%
\subsubsection{Trilinear Estimate}
%
\begin{proposition}
	\label{prop:trilinear-est}
	%
	%
	For $b \ge 3/4$, we have
	\begin{equation}
		\left( \sum_{n \in \zz} \left(1 + |n| \right)^{2s} \int_\rr
		\frac{|\wh{w_{fgh}}(n, \tau) |^2}{\left (1+ |\tau - 
		n^{2}| \right ) ^b} 
		 \ d \tau 
		\right)^{1/2}
		\lesssim \|f\|_{X_s} \|g\|_{X_s}\|h\|_{X_s}
	\end{equation}
	where $w_{fgh}$ = $fg \bar h$.
%
%
%
%
\end{proposition}


\end{frame}
%
%
%
\subsection{Estimate for 3rd Term}
\begin{frame}
	\frametitle{Estimate for 3rd Term}
Letting $$f(x,t) = \psi(t) \sum_{n \in \zz} e^{i\left( xn + tn^{2} \right)} 
\int_\rr \frac{1 - \psi\left( \lambda - n^{2} \right)}{\lambda - n^{2}} 
\wh{w} \left( n, \lambda \right) \ d \lambda,$$ we have
%
%
\begin{equation*}
	\begin{split}
		& \wh{f^x}(n, t) = \psi(t) e^{itn^{2}} \int_\rr
		\frac{1 - \psi\left( \lambda - n^{2} \right)}{\lambda - n^{2}} 
		\wh{w}(n, \lambda) \ d \lambda
	\end{split}
\end{equation*}
and
\begin{equation*}
	\begin{split}
		 \wh{f}\left( n, \tau \right)
		 & = \int_\rr e^{-it\left( \tau - n^{2} 
		\right)} \psi(t) \int_\rr \frac{1 - \psi\left( 
		\lambda - n^{2} 
		\right)}{\lambda - n^{2}} \wh{w}(n, \lambda) \ d \lambda d \tau
		\\
		& = \wh{\psi}\left( \tau - n^{2} \right) \int_\rr 
		\frac{1 - \psi\left( 
		\lambda - n^{2} 
		\right)}{\lambda - n^{2}} \wh{w}(n, \lambda) \ d \lambda.
	\end{split}
\end{equation*}

Therefore,
\end{frame}
%
%
\begin{frame}
\begin{equation*}
	\begin{split}
		& \| \eqref{main-int-expression-3} \|_{X_s}  
		\\
		& = \left( \sum_{n \in \zz} \left(1 + |n| \right)^{2s} \int_\rr \left( 1 + | \tau - n^{2
		} \right ) | | \wh{\psi}\left( \tau - n^{2 } \right) |^2 \ d \tau 
		\right .
		\\
		& \times \left . |
		\int_\rr \frac{1 - \psi\left( \lambda - n^{2 } \right)}{\lambda -
		n^{2 }} \wh{w}(n, \lambda) \ d \lambda |^2  \right)^{1/2}
		\\
		& \lesssim \left( \sum_{n \in \zz} \left(1 + |n| \right)^{2s} | \int_\rr
		\frac{1 - \psi\left( \lambda - n^{2 } \right)}{\lambda - n^{2 }}
		\wh{w}(n, \lambda) \ d\lambda |^2 \right)^{1/2}
		\\
		& \le \left( \sum_{n \in \zz} \left(1 + |n| \right)^{2s}  \left ( \int_\rr
		\frac{1 - \psi\left( \lambda - n^{2 } \right)}{|\lambda - n^{2 }|}
		|\wh{w}(n, \lambda) | \ d\lambda \right )^2 \right)^{1/2}
		\\
		& \le \left( \sum_{n \in \zz} \left(1 + |n| \right)^{2s}  \left ( \int_{| \lambda - 
		n^{2 } | \ge 1}
		\frac{|\wh{w}(n, \lambda) | }{|\lambda - n^{2 }|}
		\ d\lambda \right )^2 \right)^{1/2}.
	\end{split}
\end{equation*}

\end{frame}
%
%
\begin{frame}
Applying estimate \eqref{one-plus-ineq}, we conclude that
%
%%
\begin{equation}
	\label{main-int3-est}
	\begin{split}
		\| \eqref{main-int-expression-3} \|_{X_s}
		& \lesssim \left( \sum_{n \in \zz} \left(1 + |n| \right)^{2s}  \left ( \int_\rr
		\frac{|\wh{w}(n, \lambda)| }{1 + |\lambda - n^{2 }|}
		 \ d\lambda \right )^2 \right)^{1/2}
		 \\
		& \lesssim \|u\|_{X_s}^3
	\end{split}
\end{equation}
%
%%
where the last step follows from the following trilinear estimate, whose proof we
leave for later.

\end{frame}
%
%
%%%%%%%%%%%%%%%%%%%%%%%%%%%%%%%%%%%%%%%%%%%%%%%%%%%%%
%
%
%				Second trilinear Estimate 
%
%
%%%%%%%%%%%%%%%%%%%%%%%%%%%%%%%%%%%%%%%%%%%%%%%%%%%%%
%
\begin{frame}

\subsubsection{Corollary to Trilinear Estimate}
%
\begin{corollary}
	\label{prop:trilinear-estimate2}
	For $s \ge 0$ we have
%
%
\begin{equation}
	\label{trilinear-estimate2}
	\begin{split}
		\left( \sum_{n \in \zz} \left(1 + |n| \right)^{2s}  \left ( \int_\rr 
		\frac{|\wh{w_{fgh}}(n, \tau) |}{1 + | \tau - n^{2 } |}
		 \ d\tau \right)^2  \right)^{1/2} \lesssim \|f\|_{X_s} \|g\|_{X_s}\|h\|_{X_s}.
	\end{split}
\end{equation}
%
%
\end{corollary}

\end{frame}

\subsection{Estimate for 4th Term}
\begin{frame}
	\frametitle{Estimate for 4th Term}
Note that
%
%
\begin{equation}
	\label{1n}
	\begin{split}
		\eqref{main-int-expression-4} \simeq \sum_{k \ge 1}
		\frac{i^k}{k!}g_k(x,t)
	\end{split}
\end{equation}
%
%
where 
%
%
\begin{equation*}
	\begin{split}
		& g_k(x,t) = t^k \psi(t) \sum_{n \in \zz} e^{i\left( xn + tn^{2}
		\right)} h_k(n),
		\\
		& h_k(n) = \int_\rr \psi \left( \tau - n^{2 } \right) \cdot \left(
		\tau - n^{2 } \right)^{k -1} \wh{w}(n, \tau) \ d \tau.
	\end{split}
\end{equation*}
%
%
Hence
%
%
\begin{equation*}
	\begin{split}
		\wh{g_k^x}(n, t) = t^{k} \psi(t) e^{i t n^{2 }} h_k(n)
	\end{split}
\end{equation*}
%
%
which gives

\end{frame}
%
%
\begin{frame}

\begin{equation*}
	\begin{split}
		\wh{g_k}(n, \tau)
		& = h_k(n) \int_\rr e^{-it\left( \tau - n^{2 } \right)}
		t^{k}\psi(t) \ dt
		\\
		& = h_k(n) \wh{t^{k}\psi(t)} \left( \tau - n^{2 } \right).
	\end{split}
\end{equation*}

Applying this to \eqref{1n}, we obtain
\end{frame}
%
%
\begin{frame}

%
%
\begin{equation}
	\label{2n}
	\begin{split}
		& \|\eqref{main-int-expression-4}\|_{X_s} 
		\\
		& \simeq \left( \sum_{n \in \zz} \left(1 + |n| \right)^{2s} \int_\rr \left( 1 + | \tau -
		n^{2 }
		|
		\right) | \wh{\sum_{k \ge 1} \frac{i^k}{k!}g_k(x,t)} |^2 \ d \tau
		\right)^{1/2}
		\\
		& \le \sum_{k \ge 1} \frac{1}{k!}\left( \sum_{n \in \zz} \left(1 + |n| \right)^{2s}
		\int_\rr \left( 1 + | \tau - n^{2 } | \right) | \wh{g_k}(n, \tau) |^2 \
		d \tau \right)^{1/2}
		\\
		& = \sum_{k \ge 1} \frac{1}{k!} \left( \sum_{n \in \zz} \left(1 + |n| \right)^{2s}
		\int_\rr \left( 1 + | \tau - n^{2 } | \right) | h_k(n) \wh{t^k
		\psi(t)} \left( \tau - n^{2 } \right) |^2 \ d \tau \right)^{1/2}
		\\
		& = \sum_{k \ge 1} \frac{1}{k!} \left( \sum_{n \in \zz} \left(1 + |n| \right)^{2s} |
		h_k(n) |^2 \int_\rr \left( 1 + | \tau - n^{2 } | \right) | \wh{t^k
		\psi(t)} \left( \tau - n^{2 } \right) |^2 \ d \tau \right)^{1/2}.
	\end{split}
\end{equation}
%
%
\end{frame}

\begin{frame}
Notice that for fixed $n$, the change of variable $\tau - n^{2 } \to \tau'$
gives
%
%
\begin{equation}
	\label{3n}
	\begin{split}
		& \int_\rr \left( 1 + | \tau - n^{2 } | \right) | \wh{t^{k}
		\psi(t)}\left( \tau - n^{2 } \right) |^2 \ d \tau
		\\
		& = \int_\rr \left( 1 + |\tau'| \right) | \wh{t^k \psi(t)}(\tau') |^2 \
		d \tau'
		\\
		& \le \int_\rr \left( 1 + |\tau'| \right)^2 | \wh{t^k \psi(t)}(\tau')
		|^2 \ d \tau'
		\\
		& \lesssim \int_\rr \left( 1 + | \tau' |^2 \right) | \wh{t^{k}
		\psi(t)}(\tau') |^2 \ d \tau'
		\\
		& = \|t^k \psi(t) \|_{H^1(\rr)}^2.
	\end{split}
\end{equation}
%
%
But
%
%
\begin{equation}
	\label{4n}
	\begin{split}
		\|t^k \psi(t) \|_{H^1(\rr)}^2
		& = \left( \|t^k \psi(t)\|_{L^2(\rr)} + \|\p_t \left( t^k \psi(t)
		\right)\|_{L^2(\rr)} \right)^2
		\\
		& \lesssim \|t^{k}\psi(t) \|_{L^2(\rr)}^2 + \|\p_t \left (t^{k}
		\psi(t) \right )\|_{L^2(\rr)}^2
		\\
		& \le \|t^k \psi(t) \|_{L^2(\rr)}^2 + \|t^k \p_t \psi(t)
		\|_{L^2(\rr)}^2 + \|k t^{k -1} \psi(t) \|_{L^2(\rr)}^2
		\\
		& = c_{\psi} + c_{\psi}' + k^2 c_{\psi}''
		\\
		& \lesssim k^2.
	\end{split}
\end{equation}

\end{frame}
%
%
\begin{frame}

Hence, applying \eqref{3n} and \eqref{4n} to \eqref{2n}, we obtain
%
%%
\begin{equation}
	\label{5n}
	\begin{split}
		& \|\eqref{main-int-expression-4} \|_{X_s}
		\\
		& \lesssim
		\sum_{k \ge 1} \frac{k}{k!} \left( \sum_{n \in \zz} \left(1 + |n| \right)^{2s} | h_k(n) |^2 
		\right)^{1/2}
		\\
		& \le \sum_{k \ge 1} \frac{k}{k!}
		\cdot \sup_{k \ge 1} \left( \sum_{n \in \zz} \left(1 + |n| \right)^{2s} | 
		h_k(n) |^2 \right)^{1/2}
		\\
		& = \sum_{k \ge 1} \frac{k}{k!} \cdot \sup_{k \ge 1} 
		\left( \sum_{n \in \zz} \left(1 + |n| \right)^{2s} \int_\rr 
		\psi\left( \tau - n^{2 } \right) \cdot \left( \tau - n^{2 } 
		\right)^{k -1} \wh{w}(n, \tau) \ d \tau \right)^{1/2}.
	\end{split}
\end{equation}
%
%%
Recall that $\text{supp} \, |\psi| \subset [0, \delta ]$. Pick $\delta \le 1$. 
Then $| \psi\left( \tau - n^{2 } \right) \cdot \left( \tau - n^{2 } \right)^{k 
-1} | \le \chi_{| \tau - n^{2 } | \le 1}$ for all $k \ge 1$. Hence, \eqref{5n} gives
%
\end{frame}


\begin{frame}



\begin{equation*}
	\begin{split}
		\|\eqref{main-int-expression-4} \|_{X_s} 
		& \lesssim \sum_{k \ge 1} \frac{k}{k!} \cdot \left( \sum_{n \in \zz} | 
		\int_{| \tau - n^{2}  |\le 1} | \wh{w}(n, \tau) \ d \tau |^2 
		\right)^{1/2}
	\end{split}
\end{equation*}
%
%%
which by the inequality
%
%%
\begin{equation*}
	\begin{split}
		\frac{1 + | \tau - n^{2 } |}{1 + | \tau  - n^{2 } |} \le 
		\frac{2}{1 + | \tau - n^{2 } |}, \qquad | \tau - n^{2 }  | \le 1
	\end{split}
\end{equation*}
%
%%
implies
%
%%
\begin{equation}
\label{main-int4-est}
	\begin{split}
		\|\eqref{main-int-expression-4}\|_{X_s}
		& \lesssim \left( \sum_{n \in \zz} | \int_{| \tau - n^{2}| \le 1 }
		\frac{\wh{w}(n, \tau)}{1 + | \tau - n^{2 } |} \ d \tau |^2 
		\right)^{1/2}
		\\
		& \le \left( \sum_{n \in \zz} | \int_\rr
		\frac{\wh{w}(n, \tau)}{1 + | \tau - n^{2 } |} \ d \tau |^2 
		\right)^{1/2} \\
		& \le \left( \sum_{n \in \zz} \left( \int_\rr 
		\frac{|\wh{w}(n, \tau)|}{1 + | \tau - n^{2 } |}  \ d \tau  \right)^2
		\right)^{1/2} \\
		& \lesssim \|u\|_{X_s}^3
	\end{split}
\end{equation}

\end{frame}

\begin{frame}
%
%%
where the last step follows from the 2nd trilinear estimate.
Collecting estimates \eqref{main-int1-est}, \eqref{main-int2-est}, 
\eqref{main-int3-est}, and \eqref{main-int4-est}, and recalling 
\eqref{main-int-expression-1}-\eqref{main-int-expression-4}, we obtain the 
following.

\end{frame}
%
%%
%%%%%%%%%%%%%%%%%%%%%%%%%%%%%%%%%%%%%%%%%%%%%%%%%%%%%
%
%% Contraction Proposition
%				 
%%%%%%%%%%%%%%%%%%%%%%%%%%%%%%%%%%%%%%%%%%%%%%%%%%%%%%
%%
%%
%
%
\begin{frame}
\begin{proposition}
	\label{prop:contraction}
	For $s \ge 0$, 
%%
\begin{equation*}
	\begin{split}
		\|Tu\|_{X_s} \le c_\psi \left( \|\vp \|_{H^s(\ci)} + \|u\|_{X_s}^3 
		\right).
	\end{split}
\end{equation*}
%
%%
\end{proposition}

\end{frame}

\subsection{Setting up the Contraction Map}
\begin{frame}
	\frametitle{Setting up the Contraction Map}

We now use the proposition to prove local well-posedness for the 
NLS ivp. Let $c = c_\psi^{1/2}$ and restrict 
%
%%
\begin{equation*}
	\begin{split}
		\|\vp\|_{H^s(\ci)} \le \frac{15}{64c^3}, \qquad \|u\|_{X_s} \le 
		\frac{1}{4c}.
	\end{split}
\end{equation*}
%
%%
Then
%
%%
\begin{equation*}
	\begin{split}
		\|T u \|_{X_s} 
		& \le c^2 \left[ \frac{15}{64c^3} + \left( 
		\frac{1}{4c} \right)^3 \right]
		=  \frac{1}{4c}.
	\end{split}
\end{equation*}
%
%%
Hence, $T$ maps the ball $B\left( 0, \frac{1}{4c} \right) \subset X_s$ into 
itself. Next, note that
%
%%
\begin{equation*}
	\begin{split}
		Tu - Tv = \eqref{main-int-expression-2} + \eqref{main-int-expression-3} 
		+ \eqref{main-int-expression-4}
	\end{split}
\end{equation*}
%
%%
where now $w = u | u |^2 - v | v |^{2}$. Rewriting

\end{frame}

\begin{frame}
%
%%
\begin{equation*}
	\begin{split}
		u | u |^{2} - v | v |^{2}
		& = | u |^2 \left( u -v \right) + v\left( | u 
		|^2 - | v |^2
		\right)
		\\
		& = u \bar u \left( u -v \right) + v u \bar u - v v \bar v
		\\
		& = u \bar u \left( u - v \right) + v \bar u\left( u - v \right) + v 
		\bar u v - v v \bar v
		\\
		& = u \bar u \left( u -v \right) + v \bar u\left( u - v \right) + v v 
		\left( \overline{u -v} \right)
	\end{split}
\end{equation*}
%
%%
the triangle inequality and linearity of the fourier transform then give
%
%%
\begin{equation*}
	\begin{split}
		| \wh{w}(n, \tau) | = | \wh{u | u |^2 - v| v |^2} |
		& \le | \wh{u \overline{u} \left (u -v \right )} | +
		| \wh{v \overline{u} (u -v)} | + |\wh{v v 
		(\overline{u-v})}|
		\\
		& \doteq | \wh{w_1} | + | \wh{w_2} | + | \wh{w_3} |
	\end{split}
\end{equation*}
%
%%
where
%
%%
\begin{equation*}
	\begin{split}
		w_1 = u \bar u \left( u -v \right), \qquad w_2 = v \bar u \left( u -v 
		\right), \qquad w_3 = v v \left( \overline{u -v} \right).
	\end{split}
\end{equation*}
%
%%
\end{frame}


\begin{frame}
Hence, $\|Tu - Tv\|_{X_s} \le \sum_{\ell=1, 2, 3} 
T_\ell(u)$, where



\begin{align}
	\label{main-int-exp-mod1}
	& \frac{1}{4 \pi^2} \psi(t) \sum_{n\in \zz} \int_\rr e^{ixn}  
		e^{it \tau} \frac{ 1 - \psi(\tau - n^{2}) 
		}{\tau - n^{2}} \wh{w_\ell}(n, \tau) \ d \tau
		\\
		\label{main-int-exp-mod2}
		- & \frac{1}{4 \pi^2} \psi(t) \sum_{n\in \zz} \int_\rr e^{i(xn + 
		tn^{2})}
		 \frac{1- \psi(\tau - n^{2})}{\tau - n^{2}} \wh{w_\ell}(n, \tau) \ d \tau
		\\
		\label{main-int-exp-mod3}
		+ & \frac{1}{4 \pi^2} \psi(t) \sum_{k \ge 1} \frac{i^k t^k}{k!}
		\sum_{n \in \zz} \int_\rr e^{i(xn + tn^{2} )}
		\psi(\tau - n^{2}) (\tau - n^{2})^{k-1} \wh{w_\ell}(n, \tau)  
		\\
		\doteq & T_\ell(u). \notag
\end{align}

\end{frame}

\begin{frame}
Repeating the arguments used to estimate 
\eqref{main-int-expression-2}-\eqref{main-int-expression-4}, we obtain
%
%%
\begin{equation*}
	\begin{split}
		& \|T_1\|_{X_s} \le c_\psi \|u -v \|_{X_s} \|u\|^2_{X_s}
		\\
		& \|T_2\|_{X_s} \le c_\psi \|u -v \|_{X_s} \|u\|_{X_s} \|v\|_{X_s}
		\\
		& \|T_3\|_{X_s} \le c_\psi \|u -v \|_{X_s} \|v\|_{X_s}^2.
	\end{split}
\end{equation*}
%
%%
Therefore,
%
%%
\begin{equation*}
	\begin{split}
		\|Tu - Tv \|_{X_s}
		& \le c_\psi \|u -v \|_{X_s} \left( \|u\|_{X_s}^2 + 
		\|u\|_{X_s} \|v\|_{X_s} + \|v\|_{X_s}^2 \right)
		\\
		& \le c_\psi \|u -v\|_{X_s} \left( \|u\|_{X_s} + \|v\|_{X_s} \right)^2
		\\
		& = c^2 \|u -v\|_{X_s} \left( \|u\|_{X_s} + \|v\|_{X_s} \right)^2.
	\end{split}
\end{equation*}

\end{frame}
%
%%

\begin{frame}

Restricting $u, v \in B(0, \frac{1}{4c}) \subset X_s$, it follows that
%
%%
\begin{equation*}
	\begin{split}
		\|Tu - Tv \|_{X_s}
		& \le c^2 \|u -v \|_{X_s} \left( \frac{1}{4c} + 
		\frac{1}{4c} \right)^2
		\\
		& = \frac{1}{4} \|u -v \|_{X_s}. 
	\end{split}
\end{equation*}
%
%%
We conclude that $T = T_{\vp}$ is a contraction on the ball $B(0, 
\frac{1}{4c}) \subset X_s$. A Picard iteration then yields a local, unique
solution to the NLS ivp.
\end{frame}

\begin{frame}
	To establish local Lipschitz continuity of the flow map, let $u, v \in B(0,
\frac{1}{4c}) \subset X_s$ be solutions to
the integral NLS \eqref{main-int-expression-0}-\eqref{main-int-expression-4}
with initial data $u(0) = \vp_1, v(0) =
\vp_{2}$. Then $u = T_{\vp_1}$, $v = T_{\vp_2}$, and so
%
%
\begin{equation*}
	\begin{split}
		T_{\vp_1}(u) - T_{\vp_2}(v) = \frac{1}{2\pi} \psi(t) \sum_{n \in
		\zz}e^{i\left( xn + tn^{2j} \right)} \wh{\vp_1 - \vp_2}(n) + \sum_{\ell
		= 1,2,3} T_{\ell}(u).
	\end{split}
\end{equation*}
%
%
Using an argument similar to \eqref{fourier-trans-calc}-\eqref{main-int1-est},
we obtain
%
%
\end{frame}

\begin{frame}

\begin{equation*}
	\begin{split}
		\| \frac{1}{2\pi} \psi(t) \sum_{n \in
		\zz}e^{i\left( xn + tn^{2j} \right)} \wh{\vp_1 - \vp_2}(n)\|_{X_s}
		\le c_\psi \|\vp_{1} - \vp_{2}\|_{X_s} 
	\end{split}
\end{equation*}
%
%
Furthermore, estimating as before, we have
%
%
\begin{equation*}
	\begin{split}
		\sum_{\ell=1,2,3} T_{\ell}(u,v) \le \frac{1}{4}\|u-v\|_{X_s}.
	\end{split}
\end{equation*}
%
%
\end{frame}

\begin{frame}

Hence,
%
%
\begin{equation*}
	\begin{split}
		\|u -v \|_{X_s} = \|T_{\vp_1}(u) - T_{\vp_2}(v) \|_{X_s} \le c_\psi
		\|\vp_{1} - \vp_{2} \|_{H^{s}\left( \ci \right)}\| +
		\frac{1}{4} \|u -v \|_{X_s}
	\end{split}
\end{equation*}
%
%
which implies
%
%
\begin{equation*}
	\begin{split}
		\frac{3}{4} \|u-v\|_{X_s} \le c_\psi \|\vp_1 - \vp_2 \|_{H^s(\ci)}
	\end{split}
\end{equation*}
%
%
or
%
%
\begin{equation*}
	\begin{split}
		\|u -v \|_{X_s} \le \frac{4}{3} c_\psi \|\vp_1 - \vp_2 \|_{H^{s}(\ci)}.
	\end{split}
\end{equation*}
%
%
Recalling that $\text{supp} \, \psi = [-2\delta, 2\delta ]$ and choosing
$\delta$ sufficiently small, we obtain

\end{frame}

\begin{frame}
%
%
\begin{equation*}
	\begin{split}
		\|u -v \|_{X_s} \le \frac{1}{2} \|\vp_1 - \vp_2 \|_{H^{s}(\ci)}.
	\end{split}
\end{equation*}
%
%
Hence, the flow map of the mNLS ivp is locally Lipschitz continuous from
$H^s(\ci)$ to $X_s$ for $s=0$ and $s > - \frac{(2j -1)(j-1)}{16j}$. This
concludes the proof of \autoref{thm:main}. \qquad \qedsymbol

\end{frame}


\section{Proof of Trilinear Estimate}

\begin{frame}
	\frametitle{Proof of Trilinear Estimate}

	 Note first that $|\wh{w_{fgh}}(n, \tau) |  = | \wh{f} * ( \wh{g} 
* \wh{\bar h})(n, \tau)|$ and $| \wh{\bar{h}}(n, \tau) | = |\overline{ \wh{\overline{h}} 
}(n, \tau)| = | \wh{h}(-n, -\tau) |$. It follows that

\end{frame}

%
%
%
\begin{frame}


\begin{equation}
	\label{non-lin-rep}
	\begin{split}
		& | \wh{w_{fgh}}(n, \tau)|
		\\
		& = | \sum_{n_1, n_2, n_3}  \int \wh{f}\left( n_1,  \tau_1 
\right) \wh{g}\left( n_2, \tau_2  
\right) \wh{\bar h}\left( n_3, \tau_3 \right) d \tau_1 d \tau_2 d \tau_3 |
\\
& \le \sum_{n_1,n_2,n_3}  \int | \wh{f}\left( n_1, \tau_1 
\right) | \times  | \wh{g}\left( n_2, \tau_2 
\right) | \times | \wh{\bar h}\left( n_3, \tau_3 \right) | d \tau_1 d \tau_2 d 
\tau_3
\\
& \le \sum_{n_1,n_2,n_3}  \int | \wh{f}\left( n_1, \tau_1 
\right) | \times | \wh{g}\left( n_2, \tau_2 
\right) | \times | \wh{u}\left( -n_3, - \tau_3 \right) | d \tau_1 d \tau_2 d 
\tau_3
\\
& = \sum_{n_1,n_2,n_3} \int \frac{c_f\left( n_1, \tau_1 
\right)}{\left(1 + |n_1| \right)^s \left( 1 + | \tau_1 - n_1^{2} | \right)^{b/2}}
\\
& \times \frac{c_{g}\left( n_2, \tau_2 \right)}{\left(1 + |n_2|\right) 
^s\left( 1 + | \tau_2 -  n_2^{2 }| 
\right)^{b/2}}
 \times \frac{c_{h}\left( -n_3, -\tau_3 \right)}{\left(1 + |n_3|\right) ^s\left( 1 + | 
\tau_3 + n_3^{2 } | \right)^{b/2}} \ d \tau_1 d \tau_2 d \tau_3
\end{split}
\end{equation}


%
%
where $n = n_1 + n_2 + n_3$, $\tau = \tau_1 + \tau_2 + \tau_3$, and 
%
%
\begin{equation*}
	\begin{split}
		c_\sigma(n, \tau) = \left(1 + |n| \right) ^s \left( 1 + | \tau - n^{2 } |  
		\right)^{b/2} | \wh{\sigma}\left( n, \tau \right) | .
	\end{split}
\end{equation*}

\end{frame}

\begin{frame}
%
%
Hence
%
%
\begin{equation}
	\label{convo-est-starting-pnt}
	\begin{split}
		 & \left(1 + |n| \right)^s \left( 1 + | \tau - n^{2 } | \right)^{-b/2} | \wh{w_{fgh}}\left( 
		n, \tau \right) |
		\\
		& \le \left( 1 + | \tau - n^{2 } | \right)^{-b/2}
		\sum_{n_1, n_2, n_3} \int \frac{\left(1 + |n| \right)^s}{\left(1 + |n_1|
		\right)^s \left( 1 + | n_2 
				|\right)^s \left(1 + |n_3| \right)^s} 
		\\
		& \times \frac{c_f(n_1, \tau_1)}{\left( 1 + | \tau_1 - n_1^{2 } | 
		\right)^{b/2}}
		\times
		\frac{c_g(n_2, \tau_2)}{\left( 1 + | \tau_2 - n_2^{2 } | 
		\right)^{b/2}} \times
		\frac{c_h(-n_3, -\tau_3)}{\left( 1 + | \tau_3 + n_3^{2 } | 
		\right)^{b/2}}\ d \tau_1 d \tau_2 d \tau_3.
	\end{split}
\end{equation}
%
%
For $s \ge 0$, observe that
%
%
\begin{equation}
	\label{deriv-bound-easy-s}
	\begin{split}
		\frac{\left(1 + |n| \right) ^s}{\left(1 + |n_1| \right) ^s
		\left(1 + |n_2| \right) ^s \left(1 + |n_3| \right) ^s} 
		\lesssim 1
	\end{split}
\end{equation}
%
%
by the following lemma.

\end{frame}
%
%
%
\begin{frame}

\begin{lemma}
	\label{lem:splitting}
	For $v \ge 0$ and $a, b, c \in \zz$, we have 
%
%
\begin{equation*}
	\label{splitting}
	\begin{split}
		1 + | a + b + c | \le \left[ \left( 1 + | a | \right)\left( 1 + | b |
		\right) \left( 1 + | c | \right) \right].
	\end{split}
\end{equation*}
%
%
%
%
\end{lemma}

\end{frame}
%
%
%
%
\begin{frame}

Hence, from \eqref{convo-est-starting-pnt} and \eqref{deriv-bound-easy-s}, we 
obtain
%
\begin{equation*}
	\begin{split}
		& \left(1 + |n| \right)^s \left( 1 +  | \tau - n^{2 }  | \right)^{-b/2} | 
		\wh{w_{fgh}}\left( n, \tau \right) | 
		\\
		& \lesssim \sum_{n_1, n_2,n_3} \int \frac{1}{\left( 1 +
		| \tau - n^{2}| 
		\right)^{b/2}}  
		\\
		& \times
		\sum_{n_1,n_2,n_3} \int \frac{c_f\left( n_1, \tau_1 
		\right)}{\left(1 + |n_1| \right)^s \left( 1 + | \tau_1 - n_1^{2} | \right)^{b/2}}
		\\
		& \times \frac{c_{g}\left( n_2, \tau_2 \right)}{\left(1 + |n_2|  \right) 
		^s\left( 1 + | \tau_2 -  n_2^{2 }| 
		\right)^{b/2}}
		\\
		& \times \frac{c_{h}\left( -n_3, -\tau_3 \right)}{\left(1 + |n_3|
		\right) ^s\left( 1 + | \tau_3 + n_3^{2 } | \right)^{b/2}} \ d \tau_1
		d \tau_2 d \tau_3
		\\
		& = \left( 1 + | \tau - n^{2 } | \right)^{-b/2}
		\wh{C_f C_{g} C^-_{h}} \left( n, \tau \right)
	\end{split}
\end{equation*}
%
%
%
\end{frame}

\begin{frame}

where
%
%
\begin{equation*}
	\begin{split}
		C_\sigma(x, t) = \left[ \frac{c_\sigma\left( n, \tau \right)}{\left( 
		1 + | \tau - n^{2 } | \right)^{b/2}} \right]^\vee,
		\ \ C^-_\sigma(x, t) = \left[ \frac{c_\sigma\left( -n, -\tau \right)}{\left( 
		1 + | \tau + n^{2 } | \right)^{b/2}} \right]^\vee.
	\end{split}
\end{equation*}
%
%
Therefore
%
%
\begin{equation}
	\label{gen-holder-pre-estimate}
	\begin{split}
		& \| \left ( 1 + |n | \right )^s\left( 1 + | \tau - n^{2 } | \right)^{-b/2} \wh{w_{fgh}}(n, 
		\tau)		
		\|_{L^2(\zz \times \rr)}
		 \\
		 & \lesssim \| \left( 1 + | \tau - n^{2 } | \right)^{-b/2}
		\wh{C_f C_{g} C^-_{h}} \|_{L^2(\zz \times \rr)}.
	\end{split}
\end{equation}
%
We now require the following multiplier estimate, whose proof can be found in
\cite{himonas_Misiolek-A-priori-estima}.  
\end{frame}
%
%
%%%%%%%%%%%%%%%%%%%%%%%%%%%%%%%%%%%%%%%%%%%%%%%%%%%%%
%
%
%			Four Mult Est	
%
%
%%%%%%%%%%%%%%%%%%%%%%%%%%%%%%%%%%%%%%%%%%%%%%%%%%%%%
%
%
\subsubsection{A Key Multiplier Estimate}

\begin{frame}
	\frametitle{A Key Multiplier Estimate}

\begin{lemma}
	\label{lem:four-mult-est-L4}
	Let $(x, t) \in \ci \times \rr $ and $(n, \tau) \in \zz \times \rr$ be 
	the dual variables. Then
	%
%
\begin{equation}
	\label{four-mult-est-L4}
	\begin{split}
		\|f\|_{L^4(\ci \times \rr)} \le c \|\left( 1 + | \tau - n^2 | 
		\right)^\frac{3}{8} \wh{f} \|_{L^2( \zz \times \rr)}
	\end{split}
\end{equation}
for every test function $f(x, t)$. 
%
%
%
%
\end{lemma}
%
%
Dualizing, we obtain the following.
%
%
%%%%%%%%%%%%%%%%%%%%%%%%%%%%%%%%%%%%%%%%%%%%%%%%%%%%%
%
%
%			Dual Lemma	
%
%
%%%%%%%%%%%%%%%%%%%%%%%%%%%%%%%%%%%%%%%%%%%%%%%%%%%%%
%
%
\begin{corollary}
	\label{cor:four-mult-est-L4}
	Let $(x, t) \in \ci \times \rr $ and $(n, \tau) \in \zz \times \rr$ be the
	dual variables. Then %
%
\begin{equation}
	\label{four-mult-est-L4*}
	\begin{split}
		\| \left( 1 + | \tau - n^2 | 
		\right)^{-\frac{3}{8}}
		\wh{f}\|_{L^2(\zz \times \rr)} \le c \|f \|_{L^{4/3}( \ci \times \rr)}.
	\end{split}
\end{equation}
%
%

\end{corollary}
%
%
\end{frame}


\begin{frame}

Applying the corollary and generalized H\"{o}lder to the 
right-hand-side of \eqref{gen-holder-pre-estimate} gives
%
%
\begin{equation}
	\label{gen-holder-piece-1}
	\begin{split}
		& \|\left( 1 + | \tau - n^{2 } | \right)^{-b/2} \wh{C_f C_{ 
		g } C^-_{h}}\|_{L^2(\zz \times \rr)}
		\\
		& \lesssim  \|C_f C_{g} C^-_{h} \|_{L^{4/3}(\ci \times \rr)}
		\\
		& \le \|C_f \|_{L^4(\ci \times \rr)} \|C_{g}\|_{L^4(\ci \times \rr)} 
		\|C^-_{h}\|_{L^4(\ci \times \rr)}.
	\end{split}
\end{equation}
%
%
Note that by a change of variable
%
%
\begin{equation*}
	\begin{split}
		C_\sigma^-(x, t)
		& = \sum_{n \in \zz} \int_\rr e^{i(nx +  \tau t)} \frac{c_\sigma\left( -n, -\tau \right)}{\left( 
		1 + | \tau + n^{2 } | \right)^{b/2}} \ d \tau
		\\
		& = - \sum_{n \in \zz} \int_\rr e^{-i(nx +   \tau t )}
		\frac{c_\sigma\left( n, \tau \right)}{\left( 
		1 + | \tau - n^{2 } | \right)^{b/2}} \ d \tau
	\end{split}
\end{equation*}
%
%
from which it follows that
%
%
\begin{equation*}
	\begin{split}
		C_\sigma^-(-x, -t) = -C_\sigma(x, t).
	\end{split}
\end{equation*}
%
%
%
\end{frame}

\begin{frame}
Recalling that $L^4(\ci \times \rr)$ is
invariant under the transformation $(x, 
t) \mapsto (-x,-t)$, we obtain
%
%
\begin{equation}
	\label{C-sig-estimate}
	\begin{split}
		\| C_\sigma^- \|_{L^4(\ci \times \rr)}
		& = \|C_\sigma \|_{L^4(\ci \times 
		\rr)} 
		\\
		& \lesssim \|\left( 1 + | \tau - n^{2 } | 
		\right)^{3/8} \left( 1 + | \tau - n^{2 } | 
		\right)^{-b/2} c_\sigma \|_{L^2(\zz \times \rr)}
		\\
		& = \|\left( 1 + | \tau - n^{2 } | 
		\right)^{(3 - 4b )/8 } c_\sigma \|_{L^2(\zz \times \rr)}
		\\
		& \le \|c_\sigma \|_{L^2(\zz \times \rr)}  \qquad (\text{since  } (3-4b)/8 \le 0 )
		\\
		& = \|\sigma\|_{X_s}.
	\end{split}
\end{equation}
%
%
We conclude from \eqref{gen-holder-pre-estimate}, \eqref{gen-holder-piece-1}, 
and \eqref{C-sig-estimate} that
%
%
%
\begin{equation*}
	\begin{split}
		\| \left ( 1 + |n | \right) ^s \left( 1 + | \tau - n^{2} | \right)^{-b/2} \wh{w_{fgh}} 
		(n, \tau) \|_{L^2(\zz \times \rr)} \lesssim 
		\|f\|_{X_s}\|g\|_{X_s}\|h\|_{X_s}.
	\end{split}
\end{equation*}
%
%
%
%
This completes the proof.  \qquad \qedsymbol
\end{frame}


\section{Proof of Corollary to Trilinear Estimate}
\begin{frame}
	\frametitle{Proof of Corollary to Trilinear Estimate}

By duality, it suffices to bound
%
%%
\begin{equation*}
	\begin{split}
		\sum_{n \in \zz} \left(1 + |n| \right)^{2s} \wh{f}(n) \int_{\rr} \frac{|\wh{w_{fgh}}(n, \tau)|}{1 
		+ | \tau - n^{2 } |} \ d \tau
	\end{split}
\end{equation*}
%
%%
where $f \in H^s(\ci)$ with $\|f\|_{H^s(\ci)} = 1$. By the triangle inequality 
and Cauchy-Schwartz,
\end{frame}

\begin{frame}
%
%%
\begin{equation}
	\label{1m}
	\begin{split}
		& | \sum_{n \in \zz} \left(1 + |n| \right)^{2s} \wh{f}(n)  
		\int_{\rr}\frac{| \wh{w_{fgh}}(n, \tau) |}{1 + | \tau - n^{2 } |} \ d \tau |
		\\
		& \le \sum_{n \in \zz} \int_{\rr} \frac{\left(1 + |n| \right)^s |\wh{f}(n)|}{\left( 1 + 
		| \tau - n^{2 } |
		\right)^{1/2 + \eta}} \times \frac{\left( 1 + | n | \right)^s |\wh{w_{fgh}}(n, \tau) |}{\left( 
		1 + | \tau - n^{2 } | \right)^{1/2 - \eta}} \ d \tau
		\\
		& \le \left( \sum_{n \in \zz} \int_{\rr} \frac{\left(1 + |n| \right)^{2s} | \wh{f}(n) 
		|^2}{\left( 1 + | \tau - n^{2 } | \right)^{1 + \ee}} \ d \tau  
		\right)^{1/2} 
		\\
		& \times \left ( \sum_{n \in \zz}\int_{\rr} \frac{\left(1 + |n| \right)^{2s} | \wh{w_{fgh}}(n, \tau) 
		|^2}{\left( 1 + | \tau - n^{2 } | \right)^{1 - \ee}}\ d \tau 
		\right)^{1/2}
	\end{split}
\end{equation}
%
where $\ee = 2 \eta > 0$. Recalling that
$1 = \|f\|_{H^s(\ci)} = \left( \sum_{n \in \zz} \left(1 + |n| \right)^{2s} | \wh{f}(n) |^2 
\right)^{1/2}$, we must have

\end{frame}
%
%%
\begin{frame}


%
%%
\begin{equation*}
	\begin{split}
		& \text{rhs of } \eqref{1m} 
		\\
		& \le \left ( \int_{\rr} \frac{1}{\left( 1 + | \tau' | \right)^{1 + \ee}} \ d 
		\tau \right)^{1/2}
		\left ( \sum_{n \in \zz}\int_{\rr} \frac{\left(1 + |n| \right)^{2s} | \wh{w_{fgh}}(n, \tau) 
		|^2}{\left( 1 + | \tau - n^{2 } | \right)^{1 - \ee}}\ d \tau 
		\right)^{1/2}
		\\
		& \lesssim
		\left ( \sum_{n \in \zz}\int_{\rr} \frac{\left(1 + |n| \right)^{2s} | \wh{w_{fgh}}(n, \tau) 
		|^2}{\left( 1 + | \tau - n^{2 } | \right)^{1 - \ee}}\ d \tau 
		\right)^{1/2}
		\\
		& \lesssim \|f\|_{X_s}\|g\|_{X_s}\|h\|_{X_s} 
	\end{split}
\end{equation*}
%
%%
where the last step follows from the first trilinear estimate. Since  
$\ee = 2\eta>0$ can be made arbitrarily small, the proof is complete. 
\qquad \qedsymbol

\end{frame}

%\nocite{*}
%\bibliographystyle{custom}
%\bibliography{/Users/davidkarapetyan/Documents/Math/bib-files/schrodinger}
\section{References}
\begin{thebibliography}{10}
\newcommand{\enquote}[1]{``#1''}

\begin{frame}
	\frametitle{References}
\bibitem{Bourgain-Fourier-transfo-1}
\text{J.~Bourgain}.
\newblock \enquote{Fourier transform restriction phenomena for certain lattice
  subsets and applications to nonlinear evolution equations. {I}.
  {S}chr{\"o}dinger equations.}
\newblock \emph{Geom. Funct. Anal.}, \textbf{3} (1993), no.~2, 107--156.

\bibitem{Himonas_Misioek-The-Cauchy-prob}
\text{A.~Himonas and G.~Misio{\l}ek}.
\newblock \enquote{The {C}auchy problem for a shallow water type equation.}
\newblock \emph{Comm. Partial Differential Equations}, \textbf{23} (1998), no.
  1-2, 123--139.

\bibitem{himonas_Misiolek-A-priori-estima}
\text{A.~A. Himonas and G.~Misiolek}.
\newblock \enquote{A priori estimates for {S}chr{\"o}dinger type multipliers.}
\newblock \emph{Illinois J. Math.}, \textbf{45} (2001), no.~2, 631--640.

\bibitem{Tao-Multilinear-wei}
\text{T.~Tao}.
\newblock \enquote{Multilinear weighted convolution of {$L^2$}-functions, and
  applications to nonlinear dispersive equations.}
\newblock \emph{Amer. J. Math.}, \textbf{123} (2001), no.~5, 839--908.
\end{frame}
\end{thebibliography}


		\end{document}


