%
\documentclass[12pt,reqno]{amsart}
\usepackage{amssymb}
\usepackage{amsmath} 
\usepackage{cancel}  %for cancelling terms explicity on pdf
\usepackage{yhmath}   %makes fourier transform look nicer, among other things
\usepackage{framed}  %for framing remarks, theorems, etc.
\usepackage[shortalphabetic, initials, msc-links]{amsrefs} %for the bibliography; uses cite pkg
\usepackage{enumerate} %to change enumerate symbols
%\usepackage{showkeys}  %shows source equation labels on the pdf
\usepackage[margin=3cm]{geometry}  %page layout
%\usepackage[pdftex]{graphicx} %for importing pictures into latex--pdf compilation
%\setcounter{secnumdepth}{1} %number only sections, not subsections
\hypersetup{colorlinks=true,
linkcolor=blue,
citecolor=blue,
urlcolor=blue,
}
\synctex=1
\numberwithin{equation}{section}  %eliminate need for keeping track of counters
\numberwithin{figure}{section}
\setlength{\parindent}{0in} %no indentation of paragraphs after section title
\renewcommand{\baselinestretch}{1.1} %increases vert spacing of text
%
%
\newcommand{\ds}{\displaystyle}
\newcommand{\ts}{\textstyle}
\newcommand{\nin}{\noindent}
\newcommand{\rr}{\mathbb{R}}
\newcommand{\nn}{\mathbb{N}}
\newcommand{\zz}{\mathbb{Z}}
\newcommand{\cc}{\mathbb{C}}
\newcommand{\ci}{\mathbb{T}}
\newcommand{\zzdot}{\dot{\zz}}
\newcommand{\wh}{\widehat}
\newcommand{\p}{\partial}
\newcommand{\ee}{\varepsilon}
\newcommand{\vp}{\varphi}
\newcommand{\wt}{\widetilde}
%
%
\theoremstyle{plain}  
\newtheorem{theorem}{Theorem}
\newtheorem{proposition}{Proposition}
\newtheorem{lemma}{Lemma}
\newtheorem{corollary}{Corollary}
\newtheorem{claim}{Claim}
\newtheorem{conjecture}[subsection]{conjecture}
%
\theoremstyle{definition}
\newtheorem{definition}{Definition}
%
\theoremstyle{remark}
\newtheorem{remark}{Remark}
%
%
%\newtheorem{theorem}{Theorem}[section]
%\newtheorem{lemma}[theorem]{Lemma}
%\newtheorem{corollary}[theorem]{Corollary}
%\newtheorem{claim}[theorem]{Claim}
%\newtheorem{prop}[theorem]{Proposition}
%\newtheorem{proposition}[theorem]{Proposition}
%\newtheorem{no}[theorem]{Notation}
%\newtheorem{definition}[theorem]{Definition}
%\newtheorem{remark}[theorem]{Remark}
%\newtheorem{examp}{Example}[section]
%\newtheorem {exercise}[theorem] {Exercise}
%
\def\makeautorefname#1#2{\expandafter\def\csname#1autorefname\endcsname{#2}}
\makeautorefname{equation}{Equation}
\makeautorefname{footnote}{footnote}
\makeautorefname{item}{item}
\makeautorefname{figure}{Figure}
\makeautorefname{table}{Table}
\makeautorefname{part}{Part}
\makeautorefname{appendix}{Appendix}
\makeautorefname{chapter}{Chapter}
\makeautorefname{section}{Section}
\makeautorefname{subsection}{Section}
\makeautorefname{subsubsection}{Section}
\makeautorefname{paragraph}{Paragraph}
\makeautorefname{subparagraph}{Paragraph}
\makeautorefname{theorem}{Theorem}
\makeautorefname{theo}{Theorem}
\makeautorefname{thm}{Theorem}
\makeautorefname{addendum}{Addendum}
\makeautorefname{add}{Addendum}
\makeautorefname{maintheorem}{Main theorem}
\makeautorefname{corollary}{Corollary}
\makeautorefname{lemma}{Lemma}
\makeautorefname{sublemma}{Sublemma}
\makeautorefname{proposition}{Proposition}
\makeautorefname{property}{Property}
\makeautorefname{scholium}{Scholium}
\makeautorefname{step}{Step}
\makeautorefname{conjecture}{Conjecture}
\makeautorefname{question}{Question}
\makeautorefname{definition}{Definition}
\makeautorefname{notation}{Notation}
\makeautorefname{remark}{Remark}
\makeautorefname{remarks}{Remarks}
\makeautorefname{example}{Example}
\makeautorefname{algorithm}{Algorithm}
\makeautorefname{axiom}{Axiom}
\makeautorefname{case}{Case}
\makeautorefname{claim}{Claim}
\makeautorefname{assumption}{Assumption}
\makeautorefname{conclusion}{Conclusion}
\makeautorefname{condition}{Condition}
\makeautorefname{construction}{Construction}
\makeautorefname{criterion}{Criterion}
\makeautorefname{exercise}{Exercise}
\makeautorefname{problem}{Problem}
\makeautorefname{solution}{Solution}
\makeautorefname{summary}{Summary}
\makeautorefname{operation}{Operation}
\makeautorefname{observation}{Observation}
\makeautorefname{convention}{Convention}
\makeautorefname{warning}{Warning}
\makeautorefname{note}{Note}
\makeautorefname{fact}{Fact}
%
%


\newcommand{\uol}{u^\omega_\lambda}
\newcommand{\lbar}{\bar{l}}
\renewcommand{\l}{\lambda}
\newcommand{\R}{\mathbb R}
\newcommand{\RR}{\mathcal R}
\newcommand{\al}{\alpha}
\newcommand{\ve}{q}
\newcommand{\tg}{{tan}}
\newcommand{\m}{q}
\newcommand{\N}{N}
\newcommand{\ta}{{\tilde{a}}}
\newcommand{\tb}{{\tilde{b}}}
\newcommand{\tc}{{\tilde{c}}}
\newcommand{\tS}{{\tilde S}}
\newcommand{\tP}{{\tilde P}}
\newcommand{\tu}{{\tilde{u}}}
\newcommand{\tw}{{\tilde{w}}}
\newcommand{\tA}{{\tilde{A}}}
\newcommand{\tX}{{\tilde{X}}}
\newcommand{\tphi}{{\tilde{\phi}}}


\begin{document}
\title{Modified Boussinesq equation}

\author{Dan-Andrei Geba, Alexandrou Himonas, and David Karapetyan}

\address{Department of Mathematics, University of Rochester, Rochester, NY 14627}
\address{Department of Mathematics, University of Notre Dame, Notre Dame, IN 46556}
\address{Department of Mathematics, University of Notre Dame, Notre Dame, IN 46556}
\date{}

%\begin{abstract}
%\end{abstract}

\subjclass[2000]{35B30, 35Q55, 35Q72}
\keywords{local well-posedness; ill-posedness.}

\maketitle
%
\section{Introduction}


%

%
%%%%%%%%%%%%%%%%%%%%%%%%%%%%%%%%%%%%%%%%%%%%%%%%%%%%%
%
%
%             Fourth order Modified Boussinesq  equation
%
%
%%%%%%%%%%%%%%%%%%%%%%%%%%%%%%%%%%%%%%%%%%%%%%%%%%%%%


%
\section{Fourth order Modified Boussinesq  equation}
%
We consider the initial value problem (ivp) for the fourth order modified Boussinesq
($B_4$) equation 
\begin{gather}
  u_{tt}   + u_{xxxx} + (u^2)_{xx} = 0,
  \label{eqn:mb-2}
  \\
  u(x,0) = u_{0}(x), \quad \p_t u(x, 0) = u_1(x), 
  \label{eqn:mb-init-data-2}
  \\
  \notag
  (u_0, u_1) \in
  H^{s}(\ci) \times
  H^{s-1}(\ci)  
\end{gather}
This has the scaling
   \[
    u_{\lambda}(t,x)\,=\,\frac{1}{\lambda^2}u\left(\frac{t}{\lambda^2}, \frac{x}{\lambda}\right),
    \]
    which implies that the critical index is $s_c=-\frac 32$;
\begin{framed}
\begin{remark}
We now prove this.
Let $u(x, t)$ be a solution to the $B_4$ equation, that is
%
$$
B_4(u)=
 \partial_t^2u + \partial^4_x u + \partial_x^2(u^2)  = 0
$$
%
We would like to find the constants
$a, b, c$ such that
\[
u_\lambda (x, t) = \lambda^a u(\lambda^b x, \lambda^c t)
\]
is also a solution to $B_4$.  Since 
$$
B_4(u_\lambda)=
\lambda^{a+2c} \partial_t^2u 
+
 \lambda^{a+4b} \partial^4_x u 
 +
  \lambda^{2a+2b}
  \partial_x^2(u^2),  
$$
we see that $u_\lambda$ is a $B_4$ solution only if
$$
a+2c=a+4b=2a+2b,
$$
or
$
c= 2b =a.
$
  Thus
\[
u_\lambda (x, t) = \lambda^{2b} u(\lambda^{b}x,  \lambda^{2b} t).
\]
%
%
Therefore, replacing  $ \lambda^b$ with  $ \lambda$ gives the following scaling:
%
\begin{equation}
\label{DP-scal}
\boxed{
u(x, t) \text{ solution to }  B_4
 \Longrightarrow 
u_\lambda (x, t) = \lambda^2 u(\lambda x, \lambda^2 t)  \text { is also a
solution to }  B_4. 
}
\end{equation}
  \label{rem:scaling}
To find the critical Sobolev index, we compute
%
%
\begin{equation}
\begin{split}
  \| u_{\lambda} \|_{\dot{H}^s(\ci)} 
  & = \lambda^{2} \| u(\lambda x, \lambda^2 t) \|_{\dot{H}^{s}(\ci)}
  \\
  & = \lambda^{2} \left( \int_{\rr} | \xi |^{2s} | \wh{u(\lambda x,
  \lambda^{2} t)}^x (\xi, t)| \right)^{1/2}.
\end{split}
\label{crit-ind-comp}
\end{equation}
%
But
%
%
\begin{equation*}
\begin{split}
  \wh{u(\lambda x, \lambda^{2}t)^x}(\xi, t)
  & = \int_{\rr}e^{-i\xi x}u(\lambda x, \lambda^2 t) dx
  \\
  & = \frac{1}{\lambda} \int_{\rr}e^{-i \frac{n}{\lambda} x'}u(x',
  \lambda^{2} t) dx'
  \\
  & = \frac{1}{\lambda} \wh{u(\cdot, \lambda^{2}t)}(\frac{\xi}{\lambda})
\end{split}
\end{equation*}
%
%
Substituting back into \eqref{crit-ind-comp}, we obtain
%
%
\begin{equation*}
\begin{split}
  \| u_{\lambda} \|_{\dot{H}^s(\rr)} 
  & = \lambda^{2} \left( \int_{\rr} | \xi |^{2s} |
  \frac{1}{\lambda}\wh{u(\cdot, \lambda^{2}t)}(\frac{\xi}{\lambda}) |^2 d \xi
  \right)^{1/2}
  \\
  & = \lambda \left( \int_{\rr}| \xi |^{2s} | \wh{u(\cdot,
  \lambda^{2}t)}(\frac{\xi}{\lambda}) |^2 d \xi  \right)^{1/2}
  \\
  & = \lambda \left( \int_{\rr} | \lambda \xi' |^{2s} 
  \wh{u(\cdot, \lambda^{2}t)}(\xi') |^2 \lambda d \xi
  \right)^{1/2}
  \\
  & = \lambda^{s + 3/2} \|u(\cdot, \lambda^{2}t) \|_{\dot{H}^s (\ci)}.
\end{split}
\end{equation*}
%
%
Therefore, $\| u_{\lambda(0)} \|_{\dot{H}^s(\rr)} = \lambda^{s + 3/2} \|
u_{0} \|_{\dot{H}^{s}(\rr)}$, and so $s=-3/2$ is the critical Sobolev index.
\end{remark}
\end{framed}
%
%
We shall prove the following.
%
%
%%%%%%%%%%%%%%%%%%%%%%%%%%%%%%%%%%%%%%%%%%%%%%%%%%%%%
%
%
%                Main Theorem
%
%
%%%%%%%%%%%%%%%%%%%%%%%%%%%%%%%%%%%%%%%%%%%%%%%%%%%%%
%
%
\begin{theorem}
  If $s>s_c = -1/4$ then 
  then the $B_{4}$ ivp is well-posed
  \begin{enumerate}[(i)]
    \item In $H^s(\rr)$ if $s > s_c$
    \item In $H^{s}(\ci)$ if $s > s_c$,
  \end{enumerate}
  and the data-to-solution map is locally Lipschitz. Furthermore, these results
  are optimal in the sense that uniform continuity of the flow map fails for $s
  \le -1/4$. 
  \label{thm:wp-2}
\end{theorem}
%

%
%
%%%%%%%%%%%%%%%%%%%%%%%%%%%%
%
%
%           Scaling for B4
%
%
%%%%%%%%%%%%%%%%%%%%%%%%%%%
%
%
%



%
%
%%%%%%%%%%%%%%%%%%%%%%%%%%%%%%%%%%%%%%%%%%%%%%%%%%%%%
%
%
%                The Periodic Case
%
%
%%%%%%%%%%%%%%%%%%%%%%%%%%%%%%%%%%%%%%%%%%%%%%%%%%%%%
%
%
\subsection{The Periodic Case} 
\label{ssec:periodic-case}
We will first rewrite the $B_4$ ivp
\eqref{eqn:mb-2}-\eqref{eqn:mb-init-data-2} in integral form. Consider
the linear $B_4$ ivp
\begin{gather}
  u_{tt} + u_{xxxx} = 0,
  \label{lin-mb}
  \\
  u(x, 0)=u_{0}(x), \quad u_{t}(x,0) = u_{1}(x).
  \label{lin-mb-init-data-1}
\end{gather}
Taking the spatial Fourier transform yields the ivp
\begin{gather*}
  \wh{u_{tt}^{x}} + n^{4} \wh{u^{x}} = 0,
  \\
  \wh{u^{x}}(\cdot, 0) = \wh{u_{0}}(n), \quad
  \wh{u_{t}^{x}}(\cdot, 0) = \wh{u_{1}}(n)
\end{gather*}
which admits the unique solution
%
%
\begin{equation*}
  \begin{split}
    \wh{u^{x}}(n, t) = \wh{u_{0}}(n) \frac{e^{in^{2}t} + e^{in^{2}t}}{2} + 
    \wh{u_{1}}(n) \frac{e^{in^{2}t} - e^{-in^{2}t}}{2i n^{2}}.
  \end{split}
\end{equation*}
%
%
%
%
\begin{framed}
\begin{remark}
  Note that $$g(n) \doteq \frac{e^{in^{2}t} - e^{-in^{2}t}}{2i n^{2}}$$ has a removable
  singularity at $n=0$. Since $$\lim_{n \to 0} g(n) = 0$$ we may analytically
  extend $g(n)$ to the entire complex plane. 
\label{rem:analytic-extension}
\end{remark}
\end{framed}
%
%
%
Therefore,
%
%
\begin{equation*}
  \begin{split}
    u(x,t) = R_t u_{0} + S_{t}u_{1}
  \end{split}
\end{equation*}
%
is the unique solution to the ivp
\eqref{lin-mb}-\eqref{lin-mb-init-data-1}, where $R_{t}$ and $S_{t}$ are linear operators defined via the relation
%
%
\begin{gather*}
  \wh{R_{t}\vp} = \wh{\vp}(n) \frac{e^{in^{2}t} + e^{-in^{2}t}}{2} , \quad 
  \wh{S_{t}\vp} = \wh{\vp}(n) \frac{e^{in^{2}t} - e^{-in^{2}t}}{2i n^{2}}.
\end{gather*}

%
By Duhamel's principle, it
follows that the $B_4$ ivp \eqref{eqn:mb-2}-\eqref{eqn:mb-init-data-2} can
be written in the integral form
%
%
\begin{equation}
  \begin{split}
    u(x,t) = R_{t}u_{0} + S_{t}u_{1} + \int_{0}^{t} S_{t-t'}
    (u^{2})_{xx} dt'
  \end{split}
  \label{eqn:integral-form}
\end{equation}
%
%
which we will now localize in time.
Let $\psi(t)$ be a cutoff function symmetric about the 
origin such that $\psi(t) = 1$ for $|t| \le T$ and $\text{supp} \, \psi 
= [-2T, 2T ]$. Multiplying the right hand side of expression
$\eqref{eqn:integral-form}$ by $\psi(t)$, we obtain
%
%
\begin{equation}
  \begin{split}
    \psi(t) u(x,t)
    & = \psi(t) R_{t} u_{0} + \psi(t) S_{t}u_{1} +
    \psi(t) \int_{0}^{t} S_{t-t'}
    (u^{2})_{xx} dt'
    \\
    & \doteq Tu
  \end{split}
  \label{localized-int-eqn}
\end{equation}
where $T=T_{u_0, u_1}$.We now introduce the following spaces. 
%
%
\begin{definition}
  Let $\mathcal{Y}$ be the space of functions $F(\cdot)$ such that
  \begin{enumerate}[(i)]
   \item{$F: \ci \times \rr \to \cc$ }.
   \item{ $F(x, \cdot) \in S(\rr)$ for each $x \in \ci$}.
   \item{ $F(\cdot, t) \in C^{\infty}(\ci)$for each $t \in \rr$}.
  \end{enumerate}
  For $s, b \in \rr$, $X_{s,b}$ denotes the completion of $\mathcal{Y}$ with
  respect to the norm
  %
  %
  \begin{equation}
  \begin{split}
    \|F\|_{X_{s,b}} = \left( \sum_{n \in \zz} (1 + n^{2})^{s} \int_{\rr}
    (1 + | | \tau | - n^{2} |)^{2b} \wh{F}(n, \tau) d \tau\right)^{1/2}.
  \end{split}
  \label{eqn:bous-norm}
  \end{equation}
  %
  %
  %
  %
\end{definition}
%
The $X_{s,b}$ spaces have the following important embedding, whose proof is
provided in the appendix.
%
%
%%%%%%%%%%%%%%%%%%%%%%%%%%%%%%%%%%%%%%%%%%%%%%%%%%%%%
%
%
%               Embedding 
%
%
%%%%%%%%%%%%%%%%%%%%%%%%%%%%%%%%%%%%%%%%%%%%%%%%%%%%%
%
%
\begin{lemma}
  Let $b > 1/2$. Then $X_{s, b} \subset C(\rr, H^s)$ continuously. That is, there exists $c>0$ depending only on $b$ such that
%
%
\begin{equation*}
\begin{split}
  \| u \|_{C(\rr, H^s) } \le c \| u \|_{X_{s,b}}.
\end{split}
\end{equation*}
%
\label{lem:embedding}
\end{lemma}
%
%
We will 
show that for initial data $\vp \in {H}^s(\ci)$, $T$ is a contraction on $B_M 
\subset {X}_{s,b}$, where $B_M$ is the ball centered at the origin of radius $M = 
M_{\vp}> 0$, by estimating the $X_{s,b}$
norm of \eqref{localized-int-eqn}. The Picard fixed point theorem will
then yield a unique solution to
\eqref{localized-int-eqn}. An application of \autoref{lem:embedding}
will then imply the existence of a unique, local
solution $u \in C([-T, T], H^s(\ci))$ to the $B_4$ ivp. Local Lipschitz continuity of the flow map will follow from estimates used to establish the contraction mapping. %
%
%
%
%
%
%
%
%
%
\subsubsection{Estimate for $\psi(t) R_{t}u_{0}$.} 
\label{sssec:est-init-term-1}
We have
%
%
\begin{equation*}
  \begin{split}
    \wh{\psi(t)R_{t}u_{0}}^{x}(n, t)
    & = \psi(t) \wh{u_{0}}(n) \frac{e^{in^2 t} + e^{-in^{2}t}}{2}
    \\
    & = \frac{\psi(t) \wh{u_{0}}(n)e^{in^{2}t}}{2} + \frac{\psi(t)
    \wh{u_{0}}(n)e^{-in^{2}t}}{2}  
  \end{split}
\end{equation*}
%
%
and
%
%
\begin{equation*}
  \begin{split}
    \wh{\psi(t) R_{t}u_{0}}(n, \tau) = \frac{\wh{\psi}(\tau -
    n^{2})\wh{u_{0}}(n)}{2} + \frac{\wh{\psi}(\tau - n^{2})\wh{u_{0}}(n)}{2}.
  \end{split}
\end{equation*}
%
%
Hence,
%
%
\begin{align}
    & \| \psi(t) R_{t}u_{0} \|_{X_{s,b}}^{2} 
    \notag
    \\
    & = \sum_{n \in \zz}(1 + n^{2})^{s} \int_{\rr}\left( 1 + | | \tau
    |-n^{2} | \right)^{2b} | \frac{\wh{\psi}(\tau - n^{2})\wh{u_{0}(n)}}{2} +
    \frac{\wh{\psi}(\tau + n^{2})\wh{u_{0}}(n)}{2} |^{2} d \tau
    \notag
    \\
    & \simeq \sum_{n \in \zz}(1 + n^{2})^{s} | \wh{u_{0}(n)} |^{2} \int_{\rr}
    \left( 1 + | | \tau | - n^{2} | \right)^{2b} | \wh{\psi}(\tau - n^{2}) +
    \wh{\psi}(\tau + n^{2}) |^{2} d \tau
    \notag
    \\
    & \le 4 \left ( \sum_{n \in \zz} \left( 1 + n^{2} \right)^{s} | \wh{u_{0}}(n)
    |^{2} \int_{\rr} | \wh{\psi}(\tau - n^{2}) |^{2}\left( 1 + | | \tau | -
    n^{2} | \right)^{2b} d \tau \right .
    \label{u-0-norm-comp-1}
    \\
    & + \left.\sum_{n \in \zz} \left( 1 + n^{2} \right)^{s} | \wh{u_{0}}(n)
    |^{2} \int_{\rr} | \wh{\psi}(\tau + n^{2}) |^{2}\left( 1 + | | \tau | -
    n^{2} | \right)^{2b} d \tau \right ).
    \label{u-0-norm-comp-3}
\end{align}
%
Noting that
\begin{equation}
  \begin{split}
    | | \tau | - n^{2} | \le \min\left\{ | \tau - n^{2} |, | \tau + n^{2} | \right\}
  \end{split}
  \label{eqn:norm-key-ineq}
\end{equation}
%
%
and that $\wh{\psi}$ is Schwartz, we bound \eqref{u-0-norm-comp-1}  
%
%
\begin{equation*}
  \begin{split}
    & \sum_{n \in \zz} \left( 1 + n^{2} \right)^{s} | \wh{u_{0}}(n)
    |^{2} \int_{\rr} | \wh{\psi}(\tau - n^{2}) |^{2}\left( 1 + | | \tau | -
    n^{2} | \right)^{2b} d \tau
    \\
    & \le  \sum_{n \in \zz} \left( 1 + n^{2} \right)^{s} | \wh{u_{0}}(n)
    |^{2} \int_{\rr} | \wh{\psi}(\tau - n^{2}) |^{2}\left( 1 +  | \tau  -
    n^{2} | \right)^{2b} d \tau
    \\
    & = \sum_{n \in \zz} \left( 1 + n^{2} \right)^{s} | \wh{u_{0}}(n)
    |^{2} \int_{\rr} | \wh{\psi}(\tau') |^{2}\left( 1 +  | \tau'| \right)^{2b} d \tau
    \\
    & = c_{\psi, b} \sum_{n \in \zz} \left( 1 + n^{2} \right)^{s} | \wh{u_{0}}(n)
    |^{2} 
    \\
    & = c_{\psi, b} \| u_{0} \|_{H^{s}}^{2}
  \end{split}
\end{equation*}
%
%
where $c_{\psi, b}$ is a constant depending only on $\psi$ and $b$. The
terms \eqref{u-0-norm-comp-1}-\eqref{u-0-norm-comp-3} are bounded in similar fashion. Therefore, 
$\|\psi(t) R_{t} u_{0}\|_{X_{s,b}}^{2} = c_{\psi, b}
\|u_{0}\|_{H^s}^2$ and
taking square roots of both sides gives
%
%
\begin{equation}
  \begin{split}
    \|\psi(t) R_{t} u_{0}\|_{X_{s,b}} = c_{\psi, b}
    \|u_{0}\|_{H^s}.
  \end{split}
  \label{eqn:u-0-fin-est}
\end{equation}
%
%

\subsubsection{Estimate for $\psi(t) S_{t}u_{1}$.}
\label{sssec:estimate-init-term-2}
We have
%
%
\begin{equation*}
  \begin{split}
    \wh{\psi(t)S_{t}u_{1}}^{x}(n, t)
    & = \psi(t) \wh{u_{1}}(n) \frac{e^{in^2 t} - e^{-in^{2}t}}{2i n^{2}}
    \\
    & = \frac{\psi(t) \wh{u_{1}}(n)e^{in^{2}t}}{2i n^{2}} - \frac{\psi(t)
    \wh{u_{1}}(n)e^{-in^{2}t}}{2i n^{2}}  
  \end{split}
\end{equation*}
%
%
and
%
%
\begin{equation*}
  \begin{split}
    \wh{\psi(t) S_{t}u_{1}}(n, \tau) = \frac{\wh{\psi}(\tau -
    n^{2})\wh{u_{1}}(n)}{2i n^{2}} + \frac{\wh{\psi}(\tau - n^{2})\wh{u_{1}}(n)}{2i
    n^{2}}.
  \end{split}
\end{equation*}
%
%
Hence,
%
%
\begin{equation*}
  \begin{split}
    \| \psi(t) S_{t}u_{1} \|_{X_{s,b}}^{2} 
    & = \sum_{n \in \zz}(1 + n^{2})^{s} \int_{\rr}\left( 1 + | | \tau
    |-n^{2} | \right)^{2b} | \frac{\wh{\psi}(\tau - n^{2})\wh{u_{1}(n)}}{2i
    n^{2}} -
    \frac{\wh{\psi}(\tau + n^{2})\wh{u_{1}}(n)}{2i n^{2}} |^{2} d \tau.
    \end{split}
\end{equation*}
%
Applying \autoref{rem:analytic-extension}, this simplifies to
%
%
\begin{equation}
\begin{split}
  & \sum_{n \in \dot{\dot{\zz}}}(1 + n^{2})^{s} \int_{\rr}\left( 1 + | | \tau
    |-n^{2} | \right)^{2b} | \frac{\wh{\psi}(\tau - n^{2})\wh{u_{1}(n)}}{2i
    n^{2}} -
    \frac{\wh{\psi}(\tau + n^{2})\wh{u_{1}}(n)}{2i n^{2}} |^{2} d \tau
    \\
    & \lesssim \sum_{n \in \dot{\zz}}(1 + n^{2})^{s-1} | \wh{u_{1}(n)} |^{2} \int_{\rr}
    \left( 1 + | | \tau | - n^{2} | \right)^{2b} | \wh{\psi}(\tau - n^{2}) -
    \wh{\psi}(\tau + n^{2}) |^{2} d \tau
    \\
    & \le \sum_{n \in \dot{\zz}} \left( 1 + n^{2} \right)^{s-1} | \wh{u_{1}}(n)
    |^{2} \int_{\rr} | \wh{\psi}(\tau - n^{2}) |^{2}\left( 1 + | | \tau | -
    n^{2} | \right)^{2b} d \tau
    \\
    & + \sum_{n \in \dot{\zz}} \left( 1 + n^{2} \right)^{s-1} | \wh{u_{1}}(n)
    |^{2} \int_{\rr} | 2\wh{\psi}(\tau - n^{2})\wh{\psi}(\tau + n^{2}) |^{2}\left( 1 + | | \tau | -
    n^{2} | \right)^{2b} d \tau
    \\
    & + \sum_{n \in \dot{\zz}} \left( 1 + n^{2} \right)^{s-1} | \wh{u_{1}}(n)
    |^{2} \int_{\rr} | \wh{\psi}(\tau + n^{2}) |^{2}\left( 1 + | | \tau | -
    n^{2} | \right)^{2b} d \tau.
\end{split}
\label{u-1-norm-comp}
\end{equation}
%
%
%
Recalling \eqref{eqn:norm-key-ineq} 
and that $\wh{\psi}$ is Schwartz, we bound the first term of
\eqref{u-1-norm-comp}  
%
%
\begin{equation*}
  \begin{split}
    & \sum_{n \in \dot{\zz}} \left( 1 + n^{2} \right)^{s-1} | \wh{u_{1}}(n)
    |^{2} \int_{\rr} | \wh{\psi}(\tau - n^{2}) |^{2}\left( 1 + | | \tau | -
    n^{2} | \right)^{2b} d \tau
    \\
    & \le  \sum_{n \in \dot{\zz}} \left( 1 + n^{2} \right)^{s-1} | \wh{u_{1}}(n)
    |^{2} \int_{\rr} | \wh{\psi}(\tau - n^{2}) |^{2}\left( 1 +  | \tau  -
    n^{2} | \right)^{2b} d \tau
    \\
    & = \sum_{n \in \dot{\zz}} \left( 1 + n^{2} \right)^{s-1} | \wh{u_{1}}(n)
    |^{2} \int_{\rr} | \wh{\psi}(\tau') |^{2}\left( 1 +  | \tau'| \right)^{2b} d \tau
    \\
    & = c_{\psi, b} \sum_{n \in \dot{\zz}} \left( 1 + n^{2} \right)^{s-1} | \wh{u_{1}}(n)
    |^{2} 
    \\
    & \le c_{\psi, b} \| u_{1} \|_{H^{s-1}}^{2}
  \end{split}
\end{equation*}
%
%
where $c_{\psi, b}$ is a constant depending only on $\psi$ and $b$. The
remaining terms of \eqref{u-1-norm-comp} are bounded in similar fashion.  
Therefore, 
$\|\psi(t) S_{t} u_{1}\|_{X_{s,b}}^{2} = c_{\psi, b}
\|u_{1}\|_{H^{s-1}}^2$ and
taking square roots of both sides gives
%
%
\begin{equation}
  \begin{split}
    \|\psi(t) S_{t} u_{1}\|_{X_{s,b}} = c_{\psi, b}
    \|u_{1}\|_{H^{s-1}}.
  \end{split}
  \label{eqn:u-1-fin-est}
\end{equation}

\subsubsection{Estimate for $\psi(t) \int_{0}^{t} S_{t-t'} (u^{2})_{xx} dt'$.}
\label{sssec:non-lin-term}
We define the spatial Fourier transform by 
%
%
\begin{equation*}
\begin{split}
  \tilde{f}(n, t) = \int_{\ci} e^{-inx}f(x,t) dx
\end{split}
\end{equation*}
%
%
and the spacetime Fourier transform by
\begin{equation*}
\begin{split}
  \wh{f}(n, \tau) = \int_{\rr} \int_{\ci} e^{-inx-it\tau}f(x,t) dx dt
\end{split}
\end{equation*}
%
%
Let $f(x,t) \doteq \psi(t) \int_{0}^{t} S_{t-t'} (u^{2})_{xx} dt'$. 
Then
%
%
\begin{equation}
  \begin{split}
    \wt{f}(n, t)
    & = \frac{\psi(t)}{2i} \int_{0}^{t}\wt{u^{2}}(n, t') \left[
    e^{in^{2}(t-t')} - e^{-in^{2}(t-t')}
    \right] dt'
    \\
    & \simeq e^{in^{2}t} \int_{0}^{t} \psi(t) \wt{u^{2}}(n, t') e^{-in^{2}t'}
    dt' - 
    e^{-in^{2}t} \int_{0}^{t} \psi(t) \wt{u^{2}}(n, t') e^{in^{2}t'} dt'
    \\
    & \doteq e^{in^{2}t} \wt{w_1}(n, t) - e^{-in^{2}t} \wt{w_2}(n, t)
  \end{split}
  \label{space-four-trans}
\end{equation}
%
where
%
%
\begin{gather*}
  w_{1}(x,t) = \frac{1}{2 \pi} \sum_{n \in \zz} e^{inx} \left[ \int_{0}^{t} \psi(t) \wt{u^{2}}(n, t') e^{in^{2}t'}
  dt'\right],
  \\
  w_{2}(x,t) = \frac{1}{2 \pi} \sum_{n \in \zz} e^{inx} \left[ \int_{0}^{t} \psi(t) \wt{u^{2}}(n, t') e^{-in^{2}t'} dt'
 \right].
\end{gather*}
%
%
%
Notice that \eqref{space-four-trans} is a \emph{global} relation in $t$.
Hence, taking its time Fourier transform gives
%
%
\begin{equation}
  \label{full-fourier-trans-exp}
\begin{split}
  \wh{f}(n, \tau) = \wh{w_{1}}(n, \tau - n^{2}) - \wh{w_{2}}(n, \tau +
  n^{2}).
\end{split}
\end{equation}
%
%
Therefore, using the definition of the $X_{s,b}$ spaces, and applying the
inequality 
%
%
\begin{equation*}
\begin{split}
  (a + b)^{2} \le 4(a^{2} + b^{2})
\end{split}
\end{equation*}
%
%
gives 
%
%
\begin{equation*}
\begin{split}
  \| f \|_{X_{s,b}}^{2}
  & = \sum_{n \in \zz} (1 + n^{2})^{s} \int_{\rr} (1 + |
  | \tau | - n^{2} |)^{2b} | \wh{w_{1}}(n, \tau - n^{2}) - \wh{w_{2}}(n, \tau +
  n^{2}) |^{2} d \tau
  \\
  & \lesssim  \sum_{n \in \zz} (1 + n^{2})^{s} \int_{\rr} (1 + |
  | \tau | - n^{2} |)^{2b} | \wh{w_{1}}(n, \tau - n^{2}) d \tau
  \\
  & + \sum_{n \in \zz} (1 + n^{2})^{s} \int_{\rr} (1 + |
  | \tau | - n^{2} |)^{2b} | \wh{w_{1}}(n, \tau + n^{2}) d \tau.
\end{split}
\end{equation*}
%
%
Applying a change of variable implies
%
%
%
%
\begin{equation}
\begin{split}
  \| f \|_{X_{s,b}}^{2}
  & \lesssim  \sum_{n \in \zz} (1 + n^{2})^{s} \int_{\rr} (1 + |
  | \tau + n^{2} | - n^{2} |)^{2b} | \wh{w_{1}}(n, \tau) |^2 d \tau
  \\
  & + \sum_{n \in \zz} (1 + n^{2})^{s} \int_{\rr} (1 + |
  | \tau - n^{2} | - n^{2} |)^{2b} | \wh{w_{2}}(n, \tau )|^2 d \tau.
\end{split}
\label{comp-pre-lemma}
\end{equation}
%
%
We now need the following lemma, whose proof is provided in the appendix.
%
%
%%%%%%%%%%%%%%%%%%%%%%%%%%%%%%%%%%%%%%%%%%%%%%%%%%%%%
%
%
%                Bound for modified principal symbol
%
%
%%%%%%%%%%%%%%%%%%%%%%%%%%%%%%%%%%%%%%%%%%%%%%%%%%%%%
%
%
\begin{lemma}
For any $n, \tau \in \rr$, we have
\label{lem:mod-princ-symb-bound}
%
%
\begin{equation*}
\begin{split}
  \max\left\{ | | \tau + n^{2} | - n^{2} |, | | \tau - n^{2} | - n^{2} |
  \right\} \le | \tau |.
\end{split}
\end{equation*}
%
%
\end{lemma}
%
%
Applying \autoref{lem:mod-princ-symb-bound} to \eqref{comp-pre-lemma} yields
%
%
\begin{equation*}
\begin{split}
\| f \|_{X_{s,b}}^{2}
  & \lesssim \sum_{j=1}^{2}  \sum_{n \in \zz} (1 + n^{2})^{s} \int_{\rr} (1 + |
  \tau|)^{2b} | \wh{w_{j}}(n, \tau)|^2 d \tau
  \\
  & = \sum_{j=1}^{2} \sum_{n \in \zz} (1 + n^{2})^{s} \|\wt{w_{j}}(n, t)
  \|^{2}_{H_{t}^{b}}
  \\
  & \simeq \sum_{n \in \zz}\left( \| \psi(t) \int_{0}^{t} \wt{u^2}(n, t')
  e^{in^{2}t'}dt'  \|_{H_{t}^{b}} + \| \psi(t) \int_{0}^{t} \wt{u^2}(n, t')
  e^{-in^{2}t'}dt'  \|_{H_{t}^{b}} \right).
\end{split}
\end{equation*}
%
We now need the following, whose proof if provided in the appendix.
%
%
%%%%%%%%%%%%%%%%%%%%%%%%%%%%%%%%%%%%%%%%%%%%%%%%%%%%%
%
%
%                Lemma to Reduce to Bilinear Est Form
%
%
%%%%%%%%%%%%%%%%%%%%%%%%%%%%%%%%%%%%%%%%%%%%%%%%%%%%%
%
%
\begin{lemma}
Let $-1/2 < b' \le 0 < b \le b' + 1$. Then
%
%
\begin{equation*}
\begin{split}
  \| \psi(t) \int_{0}^{t} g(t') dt' \|_{H^{b}_{t}} \le T^{1-(b-b')} \| g
  \|_{H_{t}^{b'}}.
\end{split}
\end{equation*}
%
%
\label{lem:pre-bilin-est}
\end{lemma}
%
%
Applying the lemma, we bound the right hand side of \eqref{comp-pre-lemma} by
%
%
\begin{equation*}
\begin{split}
  & T^{1-(b - b')} \left( \sum_{n \in \zz} (1 + |n|)^{s} \| \wt{u^{2}}(n, t')
  e^{in^{2}t'} \|_{H_{t}^{b'}}  +
  \sum_{n \in \zz} (1 + |n|)^{s} \| \wt{u^{2}}(n, t')
  e^{-in^{2}t'} \|_{H_{t}^{b'}} \right)
  \\
  & = T^{1-(b -b')}\left( \sum_{n \in \zz} (1 + |n|)^{s} \int_{\rr} (1 + | \tau
  |)^{2b'} \wh{u^{2}}(n, \tau - n^{2}) d \tau \right. 
  \\
  & +
  \left . \sum_{n \in \zz} (1 + |n|)^{s} \int_{\rr} (1 + | \tau
  |)^{2b'} \wh{u^{2}}(n, \tau + n^{2}) d \tau  \right)
  \\
  & = T^{1-(b -b')}\left( \sum_{n \in \zz} (1 + |n|)^{s} \int_{\rr} (1 + | \tau
  + n^{2}
  |)^{2b'} \wh{u^{2}}(n, \tau ) d \tau \right .
  \\
  & +
  \left. \sum_{n \in \zz} (1 + |n|)^{s} \int_{\rr} (1 + | \tau
  - n^{2} |)^{2b'} \wh{u^{2}}(n, \tau) d \tau  \right).
\end{split}
\end{equation*}
%
%
Hence, applying \autoref{lem:mod-princ-symb-bound} we obtain
%
%
%
\begin{equation*}
\begin{split}
  & \| f \|_{X_{s,b}}^{2} \lesssim T^{1-(b - b')} 
  \sum_{n \in \zz} (1 + |n|)^{s} \int_{\rr} (1 + | |\tau|
  - n^{2} |)^{2b'} \wh{u^{2}}(n, \tau) d \tau  
  \\
  & \simeq T^{(1-b - b')} \|u^{2} \|_{X_{s,b}}^{2}.
\end{split}
\end{equation*}
%
%
Taking square roots and substituting back in for $f$ gives
%
%
\begin{equation}
\begin{split}
  \|\psi(t) \int_{0}^{t} S_{t-t'} (u^{2})_{xx} dt'\|_{X_{s,b}} \le c_{\psi, b,
  b'} \| u^{2} \|_{X_{s,b'}}.
\end{split}
\label{eqn:non-lin-bound}
\end{equation}
%
%
To bound the right hand side, we now require a crucial bilinear
estimate.
%
%
%%%%%%%%%%%%%%%%%%%%%%%%%%%%%%%%%%%%%%%%%%%%%%%%%%%%%
%
%
%                Bilinear Estimate
%
%
%%%%%%%%%%%%%%%%%%%%%%%%%%%%%%%%%%%%%%%%%%%%%%%%%%%%%
%
%
\begin{proposition}
\label{prop:bilin-est}
  Let $s > -1/4$ and $u,v \in X_{s, -a}$. If either
  \begin{enumerate}[(i)]
   \item{$s \ge 0$, $b > 1$, and $1/4 < a< 1/2$ }
     \label{first-it}
   \item{ $-1/4 < s< 0$, $b > 1/2$, and $1/4 < a < 1/2$ such that $| s | <
     a/2$,}
     \label{sec-item}
  \end{enumerate}
 then there exists $c > 0$ depending only on $a$, $b$, and $s$ such that
  %
  %
  \begin{equation*}
  \begin{split}
    \| uv \|_{X_{s,-a}} \le c \| u \|_{X_{s,b}} \| v \|_{X_{s,b}}.
  \end{split}
  \end{equation*}
  %
  %
\end{proposition}
%
%
Applying the bilinear estimate to \eqref{eqn:non-lin-bound}, we conclude that
for given $s > -1/4$, we can choose $b, b'$ such that
%
%
\begin{equation}
\begin{split}
\|\psi(t) \int_{0}^{t} S_{t-t'} (u^{2})_{xx} dt'\|_{X_{s,b}} \le c_{\psi, b,
  b'} \| u \|^2_{X_{s,b}}.
\end{split}
\label{eqn:nonlinear-term-bound}
\end{equation}
%
%
\subsubsection{Proof of Existence and Uniqueness in the Periodic Case}
\label{sssec:proof-b4-per-case}
%
%
Collecting estimates \eqref{eqn:u-0-fin-est}, \eqref{eqn:u-1-fin-est}, and
\eqref{eqn:nonlinear-term-bound}, we 
we obtain the following.
%%
%%%%%%%%%%%%%%%%%%%%%%%%%%%%%%%%%%%%%%%%%%%%%%%%%%%%%
%
%% Contraction Proposition
%				 
%%%%%%%%%%%%%%%%%%%%%%%%%%%%%%%%%%%%%%%%%%%%%%%%%%%%%%
%%
%%
%
\begin{proposition}
\label{prop:contraction}
Let $s > -\frac{1}{4}$. Then
%
%%
\begin{equation*}
	\begin{split}
    \|Tu\|_{X_{s,b}} \le c_{\psi, b} \left( \|u_0 \|_{H^s(\ci)} + \|u_1 \|_{H^{s-1}(\ci)}
    + \|u\|_{X_{s,b}}^2 
		\right).
	\end{split}
\end{equation*}
%
%%
\end{proposition}
We will now use \autoref{prop:contraction} to prove local well-posedness for the 
$B_4$ ivp. Let $c = c_{\psi, b}$. For given $u_0, u_1$, we may choose $\psi$ such
that 
%
%%
\begin{equation*}
	\begin{split}
    \|u_0\|_{H^s(\ci)} \le \frac{3}{32c^2}, \quad \|u_1\|_{H^{s-1}(\ci)} \le \frac{3}{32c^2}.
	\end{split}
\end{equation*}
%
%%
Then $$\|u\|_{X_{s,b}} \le \frac{1}{4c}$$ implies
%
%%
\begin{equation*}
	\begin{split}
		\|T u \|_{X_{s,b}} 
		& \le c \left[ \frac{3}{32c^2} + \frac{3}{32c^2}+ \left( 
		\frac{1}{4c} \right)^2 \right]
		=  \frac{1}{4c}.
	\end{split}
\end{equation*}
%
%%
Hence, $T=T_{u_0, u_1}$ maps the ball $B\left( 0, \frac{1}{4c} \right) \subset
X_{s,b}$ into itself. Next, note that 
%
%%
\begin{equation*}
	\begin{split}
		Tu - Tv = 
    \int_{0}^{t} S_{t-t'}
    (u^{2} - v^{2})_{xx} dt'.
  \end{split}
  \label{eqn:integral-form-dif}
\end{equation*}
%
%%
Rewriting
%
%%
\begin{equation*}
	\begin{split}
	\p_x^2 (u^2 - v^2)	
		& = \p_x^2[(u-v)(u+v)]
		\end{split}
\end{equation*}
%
%%
and repeating the arguments used in \autoref{sssec:non-lin-term},
we obtain
%
%%
%%
\begin{equation}
	\label{20a}
	\begin{split}
		\|Tu - Tv \|_{X_{s,b}}  
		& \le c_{\psi, b} \|u -v\|_{X_{s,b}} \|u + v \|_{X_{s,b}}
		\\
		& \le c_{\psi, b} \|u -v\|_{X_{s,b}} (\|u\|_{X_{s,b}}+ \|v \|_{X_{s,b}}).
	\end{split}
\end{equation}
%
%%
If $$ u, v \in B(0, \frac{1}{4c}) \subset X_{s,b},$$ then
%
%%
\begin{equation}
	\label{21a}
	\begin{split}
		\|Tu - Tv \|_{X_{s,b}}
		& \le c \|u -v \|_{X_{s,b}} \left( \frac{1}{4c} + 
		\frac{1}{4c} \right)
		\\
		& = \frac{1}{2} \|u -v \|_{X_{s,b}}. 
	\end{split}
\end{equation}
%
%%
We conclude that $T = T_{\vp}$ is a contraction on the ball $B(0, 
\frac{1}{4c}) \subset X_{s,b}$. A Picard iteration then yields a unique solution
$u \in X_{s,b}$ to \eqref{localized-int-eqn}. Applying
\autoref{lem:embedding}, it follows that $u(x,t) \subset C( [-T, T], H^s$ is a unique
solution of the B4 ivp \eqref{eqn:mb-2}-\eqref{eqn:mb-init-data-2} for $t
\in [-T, T]$.
%
%
\subsubsection{Proof of Lipschitz Continuity in the Periodic Case} 
\label{sssec:lip-continuity}
%
%
We first define our notion of continuity.
%
%
\begin{definition}
  Let $X, Y$ be Banach spaces, and equip $X \times Y$ with the product
  topology (i.e. if $\|(f_0, f_1)\|_{X \times Y} = \|f_0\|_{X} + \|f_1\|_{Y}$).
  We say that the data to solution
  map $(f_0, f_1) \mapsto f(t)$ of the ivp $T_{f_0, f_1} f =
  0$ is \emph{locally Lipschitz} in $X \times Y$ if for
  $(u_0, u_1), (v_0, v_1) \in B_R \doteq \{(f_0,f_1) \in X \times Y: \|f_0\|_{X} +
  \|f_1\|_{Y}< R\},$ there exist $C, T>0$ depending on $R$ and local solutions
  $u(x,t), v(x,t)$
  for $t \in [-T, T]$ of $T_{u_0, u_1}u=0, T_{v_0, v_1}v=0$ satisfying
	$$\|u(\cdot, t) - v(\cdot, t)
  \|_X \le C \left( \|u_{0} - v_0 \|_{X} + \|u_{1} - v_1 \|_{Y}
  \right), \quad t \in [-T, T].$$ We
	say the flow map is \emph{locally uniformly
	continuous} in $X$ if for
	$u_0, v_0 \in B_R$ there exists $T >0$ depending on $R$ and local solutions
  $u(x,t), v(x,t)$
  for $t \in [-T, T]$ of $T_{u_0, u_1}u=0, T_{v_0, v_1}v=0$ such that 
	$$ \|u(\cdot, t) - v(\cdot, t) \|_{X} \to
  0 \ \ \text{if}  \ \ \|u_0 - v_0 \|_{X}, \|u_1 - v_1 \|_{Y} \to 0, \quad
  t \in
  [-T, T]. $$ 
\end{definition}
%
%
Notice that any locally Lipschitz flow map is locally uniformly continuous. 
Next, we shall establish local Lipschitz continuity in $X_{s,b}$ of the flow
map. Let $(u_0, u_1), (v_0, v_1) \subset \in H^{s}(\ci) \times H^{s-1}(\ci)  $
be given. Choose $\psi$ such that $$(u_0, u_1), (v_0, v_1)  \subset B(0,
\frac{15}{64c^{3}}).$$ Then there exist $u, v \in X_{s,b}$ such that $u =
T_{u_0, u_1}$, $v = T_{v_0, v_1}$, and so
%
%
\begin{equation}
	\label{gen-1a}
	\begin{split}
		& T_{u_0, u_1}(u) - T_{v_0, v_1}(v)
		\\
    & = \psi(t ) R_{t}(u_{0} - v_0) + \psi(t) S_{t}(u_{1} - v_1)
    + \psi(t) \int_{0}^{t} S_{t-t'}
    (u^{2} - v^{2} )_{xx} dt'.
		\end{split}
\end{equation}
%
%
Using arguments similar to those in 
\autoref{sssec:est-init-term-1}-\autoref{sssec:estimate-init-term-2}
we obtain
%
%
\begin{equation}
	\label{gen-2a}
	\begin{split}
		& \| \psi(t ) R_t (u_0 - v_0)\|_{X_{s,b}}
		\le c_{\psi, b} \|u_0 -v_0\|_{H^s},
    \\
    & \| \psi(t) S_t (u_1 - v_1)\|_{X_{s,b}}
    \le c_{\psi, b} \|u_1 -v_1\|_{H^{s-1}}.
	\end{split}
\end{equation}
%
%
Therefore, from \eqref{21a}-\eqref{gen-2a}, we obtain
%
%
\begin{equation*}
	\begin{split}
    \|u -v \|_{X_{s,b}}
    & = \|T_{u_0, v_0}(u) - T_{u_1, v_1}(v) \|_{X_{s,b}}
    \\
    & \le
    c_{\psi, b} \left( \|u_0 -v_0 \|_{H^s\left( \ci \right)} +\|u_1 -v_1
        \|_{H^{s-1}\left( \ci \right)} + \frac{1}{2} \|u -v \|_{X_{s,b}}\right)
  \end{split}
\end{equation*}
%
%
which implies
%
%
\begin{equation*}
	\begin{split}
		\frac{1}{2} \|u-v\|_{X_{s,b}} \le
    c_{\psi, b} \left( \|u_0 -v_0 \|_{H^s\left( \ci \right)} +\|u_1 -v_1
        \|_{H^{s-1}\left( \ci \right)} \right )
      \end{split}
\end{equation*}
%
%
or
%
%
\begin{equation}
	\begin{split}
		\|u -v \|_{X_{s,b}} \le 2 c_{\psi, b} \left( \|u_0 -v_0 \|_{H^s\left( \ci \right)} +\|u_1 -v_1
        \|_{H^{s-1}\left( \ci \right)} \right ).
	\end{split}
  \label{pre-lem-estimate}
\end{equation}
%
%
Applying  
\autoref{lem:embedding} to \eqref{pre-lem-estimate}, we obtain 

%
%
%
	 %
	 %
	 \begin{equation*}
		 \begin{split}
			\|u(\cdot, t) -v(\cdot, t) \|_{H^s(\ci)} \le
      2 c_{\psi, b} \left( \|u_0 -v_0 \|_{H^s\left( \ci \right)} +\|u_1 -v_1
        \|_{H^{s-1}\left( \ci \right)} \right ).
		 \end{split}
	 \end{equation*}
	 %
	 %
Since $u,v$ are the unique solutions to the ivp
\eqref{localized-int-eqn}, it follows that $u(x,t), v(x,t), t \in [-T, T]$ are unique
local solutions to \eqref{eqn:mb-2} with
initial data $(u_0, u_1), (v_0, v_1)$, respectively.
Hence, the flow map of the $B_4$ ivp is locally Lipschitz continuous in
$H^s(\ci)$. This
concludes the proof of well-posedness for the $B_4$ ivp
\eqref{eqn:mb-2}-\eqref{eqn:mb-init-data-2}. \qquad \qedsymbol

%
%
%%%%%%%%%%%%%%%%%%%%%%%%%%%%%%%%%%%%%%%%%%%%%%%%%%%%%
%
%
%                Proof of Bilinear Estimate B4 Per
%
%
%%%%%%%%%%%%%%%%%%%%%%%%%%%%%%%%%%%%%%%%%%%%%%%%%%%%%
%
%
\subsubsection{Proof of \autoref{prop:bilin-est}} 
\label{sssec:proof-bilin-est}
By
duality, it suffices to show that when either \eqref{first-it} or
\eqref{sec-item} are satisfied, we have
%
%%
\begin{equation}
	\label{duality-est}
	\begin{split}
	|	\sum_{n \in \zzdot}  (1 + |n|)^{s}
		\int_{\rr} \phi(n, \tau) \wh{uv}(n, \tau)(1 
    + | |\tau| - n^{2} |^{-a}) d \tau | \lesssim \|u\|_{X_{s,b}}
    \|v\|_{X_{s,b}}
    \|\phi \|_{L^{2}_{n, \tau}}.
	\end{split}
\end{equation}
Note first that $|\wh{uv}(n, \tau) |  = | \wh{u} *  \wh{v} 
(n, \tau)|$. From this it follows that
%
%
\begin{equation}
	\label{non-lin-rep}
	\begin{split}
		| \wh{uv}(n, \tau)|
    & = | \sum_{n_{1} \in \zz }  \int
    \wh{u}\left( n_1,  \tau_1 \right) \wh{v}\left( n - n_1 , \tau - \tau_1   
\right) d \tau_1 |
\\
& \le  \sum_{n_{1} \in \zz }  \int
    |\wh{u}\left( n_1,  \tau_1 \right)| |\wh{v}\left( n - n_1 , \tau - \tau_1   
\right)| d \tau_1 
\\
& = \sum_{n_1 \in \zz } \int \frac{c_u\left( n_1, \tau_1 
\right)}{\langle n_1 \rangle ^s \langle |\tau_1| - n_1^{2} | \rangle ^{b}}
\\
& \times \frac{c_{v}\left( n - n_1, \tau - \tau_1 \right)}{\langle n -
n_1 \rangle ^s\ \langle |\tau - \tau_1 | -  (n - n_1)^{2} \rangle^{b}}
  \ d \tau_1 
\end{split}
\end{equation}
%
%
where for clarity of notation we have introduced 
%
%
%
\begin{equation*}
\begin{split}
\langle k \rangle \doteq 1 + |k|
\end{split}
\end{equation*}
%
%
and
%
\begin{equation*}
	\begin{split}
		c_h(n, \tau) =
			\langle n \rangle ^s \langle |\tau| - n^{2} \rangle ^{b} | \wh{h}\left( n, \tau \right) |.
	\end{split}
\end{equation*}
%
%
From our work above, it follows that 
%
%
\begin{equation}
	\label{convo-est-starting-pnt}
	\begin{split}
		 & \langle n \rangle^s \langle \tau - n^{2} \rangle^{-a} | \wh{uv}\left( 
		n, \tau \right) |
		\\
		& \le \langle |\tau| - n^{2} \rangle^{-a}
		\sum_{n_1 \in \zz} \int \frac{\langle n \rangle^{s}}{\langle n_1 \rangle^s
    \langle n - n_1 \rangle^s} 
		\times \frac{c_f(n_1, \tau_1)}{\langle |\tau_1| - n_1^{2} \rangle ^{b}}
		\\
		& \times
		\frac{c_g(n - n_1, \tau - \tau_1 )}{\langle |\tau - \tau_1| - (n - n_1)^{2}
    \rangle^{b}}\ d \tau_1.
	\end{split}
\end{equation}
%
%
Hence, 
%
%
\begin{equation*}
	\begin{split}
    |\text{lhs of} \ \eqref{duality-est}|
	& \lesssim \sum_{n \in \zz} \int_{\rr} \phi(n, \tau) \langle n \rangle^s \langle \tau - n^{2} \rangle^{-a}
  \sum_{n_1 \in \zz}
  \int_{\rr} c_f(n_1, \tau_1)
		c_g(n - n_1, \tau - \tau_1 )
		\\
    & \times \frac{\langle n \rangle ^{s}}{\langle n_{1} \rangle ^{s} \langle
    n-n_{1} \rangle ^{s}} \times \frac{1}{\langle |\tau| - n^{2} \rangle
    ^{b}\langle |\tau_{1}|-n_{1}^{2} \rangle ^{-b}\langle | \tau|-n_{2}^{2}
    \rangle ^{b}} d \tau_1 d \tau
	\end{split}
\end{equation*}
%
%
which by Cauchy-Schwartz is bounded by
%
%
\begin{equation}
	\label{10g}
	\begin{split}
    & \sum_{n \in \zz} \int_{\rr} \phi(n, \tau) \langle | \tau | - n^{2} \rangle
    ^{-a} \langle n \rangle ^{s}
    \\
    & \times \left( \sum_{n_{1} \in \zz} \int_{\rr}
    \frac{1}{\langle n_{1} \rangle ^{2s} \langle n-n_{1} \rangle ^{2s} \langle |
    \tau_{1} | - n_{1}^{2}\rangle ^{2b} \langle | \tau - \tau_{1} | -
    (n - n_{1})^{2} \rangle ^{2b}} d \tau_{1} \right)^{1/2}
    \\
    & \times \left( \sum_{n_{1} \in \zz} \int_{\rr} c_{u}^{2}(n, \tau_{1})
    c_{v}^{2}(n - n_{1}, \tau - \tau_{1}) d \tau_{1} \right)^{1/2} d \tau
  \end{split}
\end{equation}
%
%
Applying Cauchy-Schwartz again, \eqref{10g} is bounded by
%
%
\begin{equation*}
  \begin{split}
  & \|\left( \sum_{n_{1} \in \zz }\int_{\rr } c_{u}^{2}(n_1, \tau_1)
  c_{v}^{2} (n - n_1, \tau - \tau_{1} ) d \tau_1  \right)^{1/2} \|_{L^{2}(\zz \times
		\rr)}
		\\
    & \times  \|\phi(n, \tau) \langle | \tau | - n^{2} \rangle ^{-a} \langle n
    \rangle ^{s}
		\\
    & \times \left( \sum_{n_{1} \in \zz} \int_{\rr} \frac{1}{ \langle n_{1}
    \rangle ^{2s} \langle n-n_{1} \rangle ^{2s} \langle | \tau_{1}|-n_{1}^{2}
    \rangle^{2b} \langle  |\tau -
    \tau_{1} | -(n - n_{1}^{2}
    \rangle^{2b} } d \tau_1 \right)^{1/2} \|_{L^2(\zz \times \rr)}
		\\
    & = \|u\|_{X_{s,b}} \|v\|_{X_{s,b}} \label{holder-term}
     \|\phi(n, \tau)     \\
    & \times \left( \langle | \tau | - n^{2} \rangle ^{-2a} \langle n
    \rangle ^{2s}
\sum_{n_{1} \in \zz} \int_{\rr} \frac{1}{ \langle n_{1} \rangle ^{2s} \langle
n-n_{1} \rangle ^{2s}  \langle | \tau_{1}|-n_{1}^{2} \rangle^{2b} \langle  |\tau -
    \tau_{1} | -(n - n_{1}^{2}
    \rangle^{2b} } d \tau_1 \right)^{1/2} \|_{L^2(\zz \times \rr)}.
  \end{split}
\end{equation*}
%
Applying H{\"o}lder, we bound this by 
%
%
\begin{equation*}
	\begin{split}
    & \|u\|_{X_{s,b}} \|v\|_{X_{s,b}} \| \phi \|_{L^{2}_{n, \tau}}
    \\
    & \times \|\left( \langle | \tau | - n^{2} \rangle ^{-2a} \langle n
    \rangle ^{2s}
\sum_{n_{1} \in \zz} \int_{\rr} \frac{1}{ \langle n_{1} \rangle ^{2s} \langle
n-n_{1} \rangle ^{2s} \langle | \tau_{1}|-n_{1}^{2} \rangle^{2b} \langle  |\tau -
    \tau_{1} | -(n - n_{1}^{2}
    \rangle ^{2b} } d \tau_1 \right)^{1/2} \|_{L^\infty_{n, \tau}}
	\end{split}
\end{equation*}
%
%
Hence, to complete the proof, it will be enough
to show that 
%
%
%
%
\begin{equation}
  \label{key-sup-estimate}
	\begin{split}
		 \| \langle | \tau | - n^{2} \rangle ^{-2a} \langle n
    \rangle ^{2s}
\sum_{n_{1} \in \zz} \int_{\rr} \frac{1}{  \langle n_{1} \rangle ^{2s} \langle
n-n_{1} \rangle ^{2s} \langle | \tau_{1}|-n_{1}^{2} \rangle^{2b}  \langle  |\tau -
    \tau_{1} | -(n - n_{1}^{2}
    \rangle ^{2b} } d \tau_1 \|_{L^\infty_{n, \tau}} < \infty.
	\end{split}
\end{equation}
%
%
By the triangle inequality and the fact that 
%
%
\begin{equation*}
\begin{split}
& | \tau | =
\begin{cases}
  - \tau, \quad & \tau < 0, 
\\
\tau, \quad & \tau > 0
\end{cases}
\end{split}
\end{equation*}
%
%
\eqref{key-sup-estimate} will be proved if we can bound the
$L^{\infty}_{\tau, n}$ norm of the quantity
%
%
\begin{equation}
  \label{sup-est-gen}
\begin{split}
		  \langle \sigma \rangle ^{-2a} \langle n
    \rangle ^{2s}
\sum_{n_{1} \in \zz} \int_{\rr} \frac{1}{ \langle n_{1} \rangle ^{2s} \langle n-n_{1} \rangle ^{2s} 
\langle \sigma_{1} \rangle^{2b} \langle  \sigma_{2} \rangle^{2b} }
d \tau_1 
	\end{split}
\end{equation}
%
for the following cases.
\begin{enumerate}[(i)]
    \item $ \sigma=\tau+n^2,\quad \sigma_1=\tau_1+n_1^2,\quad \sigma_2=\tau -
      \tau_1+(n - n_1)^2$,
\label{it-1}
    \item $ \sigma=\tau-n^2,\quad \sigma_1=\tau_1-n_1^2,\quad \sigma_2=\tau - \tau_1+(n - n_1)^2$,
\label{it-2}
    \item  $\sigma=\tau+n^2,\quad \sigma_1=\tau_1-n_1^2,\quad \sigma_2=\tau - \tau_1+(n - n_1)^2$,
      \label{it-3}
    \item $\sigma=\tau-n^2,\quad \sigma_1=\tau_1+n_1^2,\quad \sigma_2=\tau - \tau_1-(n - n_1)^2$,
\label{it-4}
    \item $\sigma=\tau+n^2,\quad \sigma_1=\tau_1+n_1^2,\quad \sigma_2=\tau - \tau_1-(n - n_1)^2$,
\label{it-5}
    \item $\sigma=\tau-n^2,\quad \sigma_1=\tau_1-n_1^2,\quad \sigma_2=\tau - \tau_1-(n - n_1)^2$.
\label{it-6}
\end{enumerate}
%
%
\begin{framed}
\begin{remark}
Note that the cases $\sigma=\tau+n^2,\quad \sigma_1=\tau_1-n_1^2,\quad
\sigma_2=\tau - \tau_1-(n - n_1)^2$ and $\sigma=\tau-n^2,\quad
\sigma_1=\tau_1+n_1^2,\quad \sigma_2=\tau - \tau_1+(n - n_1)^2$ cannot occur, since
$\tau_1< 0, \tau-\tau_1< 0$ implies $\tau<0$ and $\tau_1\geq 0, \tau-\tau_1\geq
0$ implies $\tau\geq 0$.
\end{remark}
\end{framed}
%
Observe that the transformation $(n, \tau, n_{1}, \tau_{1}) \mapsto -(n, \tau,
n_{1}, \tau_{1})$ reduces \eqref{it-3} to \eqref{it-4}, \eqref{it-2} to
\eqref{it-5}, and \eqref{it-1} to \eqref{it-6}. Furthermore, the change of
variables $\tau_{2} = \tau - \tau_{1}, n_{2} = n - n_{1}$, and the
transformation $(n, \tau, n_{2}, \tau_{2}) \mapsto - (n, \tau, n_{2},
\tau_{2})$ reduces \eqref{it-5} to \eqref{it-4}. Since $L^{2}$ is invariant
under change of variables and reflections, we may without loss of generality
restrict our attention to cases \eqref{it-4} and \eqref{it-6}.
 \subsubsection{Case \eqref{it-6}} 
\label{sssec:case-it-6}
Let 
%
%
\begin{align*}
A_1&=\{(n, n_1, \tau, \tau_1)\in A: n=0\},\\
A_2&=\{(n, n_1, \tau, \tau_1)\in A: n_1 = n \},\\
A_3&=\{(n, n_1, \tau, \tau_1)\in A: n_1=0 \},\\
A_4&=\{(n, n_1, \tau, \tau_1)\in A: n \neq 0, n_1 \neq 0 \text{ and } n_1 \neq n \}.
\end{align*} 
%
%
Then 
%
%
\begin{equation}
  \label{n=0}
\begin{split}
  |  \eqref{sup-est-gen} \chi_{A_{1}}| = | \langle \tau \rangle ^{-2a} \sum_{n_{1}} \langle
  n_{1}\rangle ^{-4s} \int_{\rr} \frac{1}{\langle \tau_{1} - n_{1}^{2} \rangle ^{2b}\langle
  \tau - \tau_{1} - n_{1}^{2}\rangle ^{2b}}d \tau_{1} |
\end{split}
\end{equation}
%
%
Following Ginibre, Tsutsumi, Velo --- and Kenig \cite{Kenig:1996aa}, and others,
we now need the following Calculus lemma.
%
%
%%%%%%%%%%%%%%%%%%%%%%%%%%%%%%%%%%%%%%%%%%%%%%%%%%%%%
%
%
%				 Calculus Lemma
%
%
%%%%%%%%%%%%%%%%%%%%%%%%%%%%%%%%%%%%%%%%%%%%%%%%%%%%%
%
%
\begin{lemma}
	\label{lem:calc}
 %
 %
 For $p, q>0$ and $r=\min\{ p, q, p+q-1\}$ with $p+q>1$, there exists $c>0$ such that
\begin{equation*}
\int_{-\infty}^{+\infty}\frac{dx} {\langle x-\alpha\rangle^{p}\langle x-\beta\rangle^{q}}\leq\dfrac{c} {\langle \alpha-\beta\rangle^{r}}.
\end{equation*}
 %
 %
 \end{lemma}


%
%%%%%%%%%%%%%%%%%%%%%%%%%%%%%%%%%%%%%%%%%%%%%%%%%%%%%
%
%
%             Secod order Modified Boussinesq  equation
%
%
%%%%%%%%%%%%%%%%%%%%%%%%%%%%%%%%%%%%%%%%%%%%%%%%%%%%%

Applying the calculus lemma, it follows from \eqref{n=0} that
%
\begin{equation*}
\begin{split}
  |\eqref{sup-est-gen} \chi_{A_{1}} |  & \lesssim \| \langle \tau \rangle ^{-2a} \sum_{n_{1} \in \zz} \frac{\langle n_{1} \rangle
  ^{-4s}}{ \langle \tau - 2n_{1}^{2}  \rangle ^{2b}} \|_{L^{\infty}_{\tau}}
  \\
  & \simeq \sum_{n_{1}} \langle n_{1} \rangle ^{-4s - 4b}
  \\
  & < \infty, \quad s > \frac{1-4b}{4}.
\end{split}
\end{equation*}
%
%
%
%
%
%
Similarly, substitution and \autoref{lem:calc} give
%
%
\begin{equation}
\begin{split}
  |  \eqref{sup-est-gen} \chi_{A_{2}}|
  & = | \langle \tau -n^{2} \rangle ^{-2a}\int_{\rr} \frac{1}{\langle \tau_{1} -
  n^{2} \rangle ^{2b}\langle
  \tau - \tau_{1}\rangle ^{2b}}d \tau_{1} |
  \\
  & \lesssim |  \langle \tau - n^{2} \rangle ^{-2a-2b} |
  \\
  & < \infty, \quad b \ge -a,
\end{split}
\end{equation}
%
%
%
%
\begin{equation}
\begin{split}
  |  \eqref{sup-est-gen} \chi_{A_{3}}|
  & = \langle \tau - n^{2} \rangle ^{-2a}
  \int_{\rr} \frac{1}{ \langle \tau_{1} \rangle^{2b}  \langle \tau -
  \tau_{1} - n^{2} \rangle^{2b}}
d \tau_1 
\\
  & \lesssim |  \langle \tau - n^{2} \rangle ^{-2a-2b} |
  \\
  & < \infty, \quad b \ge -a.
	\end{split}
\end{equation}
%
%
and
%
%
\begin{equation*}
\begin{split}
  | \eqref{sup-est-gen} \chi_{A_{4}} |
  & = 
  \langle \tau - n^{2}  \rangle ^{-2a} \langle n
    \rangle ^{2s}
    \sum_{n_{1} \in \zz} \int_{\rr} \frac{\chi_{A_{4}}}{ \langle n_{1} \rangle ^{2s} \langle n-n_{1} \rangle ^{2s} 
\langle \tau_{1} - n_{1}^{2}  \rangle \langle  \tau - \tau_{1} - (n -
n_{1})^{2}  \rangle}
d \tau_1 
\\
& = \langle \tau - n^{2}  \rangle ^{-2a} \langle n
    \rangle ^{2s}
    \sum_{j=1}^{2} \sum_{n_{1} \in \zz} \int_{\rr} \frac{\chi_{A_{4,j}}}{ \langle n_{1} \rangle ^{2s} \langle n-n_{1} \rangle ^{2s} 
\langle \tau_{1} - n_{1}^{2}  \rangle \langle  \tau - \tau_{1} - (n -
n_{1})^{2}  \rangle}
d \tau_1 
\end{split}
\end{equation*}
%
%
where we have partioned $ A_{4}$ into two parts
\begin{align*}
A_{4,1}&=\{(n, n_1, \tau, \tau_1)\in A_3: |\tau_1-n_1^2|\leq|\tau-n^2|\},\\
A_{4,2}&=\{(n, n_1, \tau, \tau_1)\in A_3: |\tau-n^2|\leq|\tau_1-n_1^2| \}.
\end{align*} 
Furthermore, by the symmetry of the convolution, we may assume without loss of
generality that
$$|(\tau-\tau_1)-(n-n_1)^2|\leq|\tau_1-n_1^2|\}.$$
Noting that 
%
%
\begin{equation*}
\begin{split}
  \tau - n^{2} - \left[ (\tau - n_{1}^{2}) + (\tau - \tau_{1}) - (n -
  n_{1})^{2} \right] = 2n_{1}(n - n_{1}).
\end{split}
\end{equation*}
%
%
we have in region $A_{4}$ 
%
%
\begin{enumerate}[(i)]
  \item{$\langle \tau - n^{2} \rangle  \gtrsim \langle n_{1}(n - n_{1}) \rangle$, or}
    \\
  \item{  $\langle \tau_{1} - n_{1}^{2} \rangle  \gtrsim \langle n_{1}(n -
    n_{1}) \rangle$, or}
    \\
  \item{ $\langle \tau - \tau_{1} - (n - n_{1})^{2} \rangle  \gtrsim \langle n_{1}(n -
    n_{1}) \rangle$}.
\end{enumerate}
In region $A_{4,1}$, these reduce to the single case $ \langle \tau - n^{2} \rangle
\gtrsim \langle n_{1}(n - n_{1}) \rangle$, and in region $A_{4,2}$ to the single
case $ \langle \tau_{1} - n_{1}^{2} \rangle
\gtrsim \langle n_{1}(n - n_{1}) \rangle$. Estimating first in region
$A_{4,1}$, we apply \autoref{lem:calc} and obtain
%
%
%
%
\begin{equation}
  \label{region-a41}
\begin{split}
& \langle \tau - n^{2}  \rangle ^{-2a} \langle n
    \rangle ^{2s}
    \sum_{n_{1} \in \zz} \int_{\rr} \frac{\chi_{A_{4,1}}}{ \langle n_{1} \rangle ^{2s} \langle n-n_{1} \rangle ^{2s} 
\langle \tau_{1} - n_{1}^{2}  \rangle \langle  \tau - \tau_{1} - (n -
n_{1})^{2}  \rangle}
d \tau_1 
\\
& \lesssim \langle \tau - n^{2} \rangle ^{-2a} \langle n \rangle ^{2s}
\sum_{n_{1} \in
\zz}  \frac{\chi_{A_{4,1}}}{\langle n_{1} \rangle ^{2s} \langle n - n_{1} \rangle
^{2s} \langle \tau - n^{2} - 2n_{1}^{2} + 2nn_{1}  \rangle ^{2b}}
\\
& \lesssim 
\sum_{n_{1} \in
\zz}  \frac{\langle n_1 \rangle ^{-2s} \langle n - n_{1} \rangle ^{-2s}}{\langle
n \rangle ^{-2s} \langle n_{1}(n - n_{1}) \rangle
^{2a}} \times \frac{\chi_{A_{4,1}}}{\langle \tau - n^{2} - 2n_{1}^{2} + 2nn_{1}
\rangle ^{2b}}.
\end{split}
\end{equation}
%
%
But
%
%
\begin{equation}
  \label{prelim-int-est}
\begin{split}
& \frac{\langle n_1 \rangle ^{-2s} \langle n - n_{1} \rangle ^{-2s}}{\langle
n \rangle ^{-2s} \langle n_{1}(n - n_{1}) \rangle
^{2a}}
\\
& \lesssim  \frac{| n_1 | ^{-2s} | n - n_{1} | ^{-2s}}{|
n | ^{-2s} | n_{1}(n - n_{1}) |
^{2a}}, \quad n \neq 0, n_1 \neq 0, n \neq n_1 
\\
& = | n - n_{1} |^{-2s-2a}|n_{1}|^{-2s-2a}| n |^{2s}
\\
& \lesssim \begin{cases}
  | n |^{\cancel{-2s}-2a}| n_{1} |^{-2s-2a} | n_{1} |^{-2s-2a}\cancel{| n
  |^{2s}}, \quad & s < 0, \quad a \ge -s
\\
| n - n_{1} |^{\cancel{-2s}-2a}|n_{1}|^{\cancel{-2s}-2a}{\cancel{| n
-n_{1}|^{2s}}\cancel{| n_{1}
|^{2s}}}, \quad & s \ge 0. 
\end{cases}
\end{split}
\end{equation}
%
%
where the last step follows from the following.
%
%
%
%%%%%%%%%%%%%%%%%%%%%%%%%%%%%%%%%%%%%%%%%%%%%%%%%%%%%
%
%
%                Integer Bound
%
%
%%%%%%%%%%%%%%%%%%%%%%%%%%%%%%%%%%%%%%%%%%%%%%%%%%%%%
%
%
\begin{lemma}
  Let $n, n_1 \in \zz$ such that $n_{1} \neq 0$ and $n_{1} \neq n$.
  Then
  %
  %
  \begin{equation*}
  \begin{split}
    | n | \le | n - n_{1} | | n_{1} |.
  \end{split}
  \end{equation*}
  %
  %
\label{lem:integer-bound}
\end{lemma}
%
Since $a \ge 0$, it follows from \eqref{prelim-int-est} that 
%
\begin{equation*}
\frac{\langle n_1 \rangle ^{-2s} \langle n - n_{1} \rangle ^{-2s}}{\langle
n \rangle ^{-2s} \langle n_{1}(n - n_{1}) \rangle
^{2a}} \lesssim 1, \quad s \ge 0 \text{ or } s < 0, a \ge s
\end{equation*}
%
%
which we use to bound the right hand side of \eqref{region-a41} by
%
%
\begin{equation*}
\begin{split}
\sum_{n_{1} \in
\zz} 
\frac{\chi_{A_{4,1}}}{\langle \tau - n^{2} - 2n_{1}^{2} + 2nn_{1}  \rangle ^{2b}}
\end{split}
\end{equation*}
%
%
%
which is finite for $b > 1/4$, due to the following lemma, which can be found in. 
\begin{lemma}
  \label{lem:sum-estimate}
If $\gamma>1/2$, then
\begin{equation}\label{CI2}
\sup_{(n,\tau)\in \zz \times \rr}\sum_{n_1\in \zz}\frac{1}{(1+|\tau\pm n_1(n-n_1)|)^{\gamma}}<\infty. 
\end{equation}
\end{lemma}
%
Recalling that 
$$ \langle \tau_{1} - n_{1}^{2} \rangle
\gtrsim \langle n_{1}(n - n_{1}) \rangle$$
%
in region $A_{4,2}$, we have
%
%
\begin{equation*}
\begin{split}
& \langle \tau - n^{2}  \rangle ^{-2a} \langle n
    \rangle ^{2s}
    \sum_{n_{1} \in \zz} \int_{\rr} \frac{\chi_{A_{4,1}}}{ \langle n_{1} \rangle ^{2s} \langle n-n_{1} \rangle ^{2s} 
\langle \tau_{1} - n_{1}^{2}  \rangle \langle  \tau - \tau_{1} - (n -
n_{1})^{2}  \rangle}
d \tau_1 
\\
& \lesssim \langle \tau - n^{2}  \rangle ^{-2a}     \sum_{n_{1} \in \zz} \int_{\rr} \frac{\chi_{A_{4,1}} \langle n
    \rangle ^{2s}
}{ \langle n_{1} \rangle ^{2s} \langle n-n_{1} \rangle ^{2s} 
\langle n_{1}(n - n_{1}) \rangle ^{2b} \langle  \tau - \tau_{1} - (n -
n_{1})^{2}  \rangle}
d \tau_1 
\end{split}
\end{equation*}
%
%
which by computation similar to \eqref{prelim-int-est}
is bounded by
%
%
\begin{equation}
  \label{prelim-int-est-2}
\begin{split}
\langle \tau - n^{2}  \rangle ^{-2a} \sum_{n_{1} \in \zz} \int_{\rr} \frac{\chi_{A_{4,1}} }{ \langle  \tau - \tau_{1} - (n -
n_{1})^{2}  \rangle^{2b}}
d \tau_1 
\end{split}
\end{equation}
%
%
for $s \ge 0$ or $s \le 0, b \ge -2s$. If $b > 1/2$, then integrating in
$\tau_{1}$ and discarding the  $\langle \tau - n^{2}  \rangle ^{-2a}$ term, we
then bound \eqref{prelim-int-est-2} by
%
%
\begin{equation*}
\begin{split}
  C \sum_{n_{1} \in \zz} \frac{\chi_{A_{4,1}}}{\langle \tau - (n -
  n_{1})^{2} \rangle ^{2b -1}}
  & \simeq
  \sum_{n_{1} \in \zz} \frac{\chi_{A_{4,1}}}{\langle \tau - n^{2} +
  2nn_{1} - n_{1}^{2}
  \rangle ^{2b -1}}
  \\
  & < \infty, \quad b > 1/2
\end{split}
\end{equation*}
%
%
where the last step follows from  \autoref{lem:sum-estimate}.
\subsubsection{Case \eqref{it-4}} 
\label{sssec:case-it-4}
Let 
%
%
\begin{align*}
B_1&=\{(n, n_1, \tau, \tau_1)\in B: n=0\},\\
B_2&=\{(n, n_1, \tau, \tau_1)\in B: n_1 = 0 \},\\
B_3&=\{(n, n_1, \tau, \tau_1)\in B: n \neq 0, n_1 \neq 0 \}.
\end{align*} 
%
%
Then 
%
%
\begin{equation}
\begin{split}
  |  \eqref{sup-est-gen} \chi_{B_{1}}| = | \langle \tau \rangle ^{-2a} \sum_{n_{1}} \langle
  n_{1}\rangle ^{-4s} \int_{\rr} \frac{1}{\langle \tau_{1} + n_{1}^{2} \rangle ^{2b}\langle
  \tau - \tau_{1} - n_{1}^{2}\rangle ^{2b}}d \tau_{1} |
\end{split}
\end{equation}
Applying \autoref{lem:calc}, it follows that
%
\begin{equation*}
\begin{split}
  |\eqref{sup-est-gen} \chi_{B_{1}} |  & \lesssim \| \langle \tau \rangle ^{-2a} \sum_{n_{1} \in \zz} \frac{\langle n_{1} \rangle
  ^{-4s}}{ \langle \tau  \rangle ^{2b}} \|_{L^{\infty}_{\tau}}
  \\
  & \le \sum_{n_{1}} \langle n_{1} \rangle ^{-4s}
  \\
  & < \infty, \quad s > \frac{1}{4}.
\end{split}
\end{equation*}
%
%
%
%
Similarly, substitution and \autoref{lem:calc} give
%
%
\begin{equation}
\begin{split}
  |  \eqref{sup-est-gen} \chi_{B_{2}}|
  & = | \langle \tau +n^{2} \rangle ^{-2a}\int_{\rr} \frac{1}{\langle \tau_{1}\rangle ^{2b}\langle
  \tau - \tau_{1} - n^{2} \rangle ^{2b}}d \tau_{1} |
  \\
  & \lesssim |  \langle \tau - n^{2} \rangle ^{-2a-2b} |
  \\
  & < \infty, \quad b \ge -a.
\end{split}
\end{equation}
and
%
%
\begin{equation*}
\begin{split}
  | \eqref{sup-est-gen} \chi_{B_{3}} |
  & = 
  \langle \tau - n^{2}  \rangle ^{-2a} \langle n
    \rangle ^{2s}
    \sum_{n_{1} \in \zz} \int_{\rr} \frac{\chi_{B_{3}}}{ \langle n_{1} \rangle ^{2s} \langle n-n_{1} \rangle ^{2s} 
\langle \tau_{1} - n_{1}^{2}  \rangle \langle  \tau - \tau_{1} - (n -
n_{1})^{2}  \rangle}
d \tau_1 
\\
& = \langle \tau - n^{2}  \rangle ^{-2a} \langle n
    \rangle ^{2s}
    \sum_{j=1}^{2} \sum_{n_{1} \in \zz} \int_{\rr} \frac{\chi_{B_{3,j}}}{ \langle n_{1} \rangle ^{2s} \langle n-n_{1} \rangle ^{2s} 
\langle \tau_{1} - n_{1}^{2}  \rangle \langle  \tau - \tau_{1} - (n -
n_{1})^{2}  \rangle}
d \tau_1 
\end{split}
\end{equation*}
%
%
where we have partioned $ B_{3}$ into two parts
\begin{align*}
B_{3,1}&=\{(n, n_1, \tau, \tau_1)\in B_3: |\tau_1-n_1^2|\leq|\tau-n^2|\},\\
B_{3,2}&=\{(n, n_1, \tau, \tau_1)\in B_3: |\tau-n^2|\leq|\tau_1-n_1^2| \}.
\end{align*} 
Furthermore, by the symmetry of the convolution, we may assume without loss of
generality that
$$|(\tau-\tau_1)-(n-n_1)^2|\leq|\tau_1-n_1^2|\}.$$
Noting that 
%
%
\begin{equation*}
\begin{split}
  \tau - n^{2} - \left[ (\tau + n_{1}^{2}) + (\tau - \tau_{1}) - (n -
  n_{1})^{2} \right] = 2n_{1}n
\end{split}
\end{equation*}
%
%
we have in region $B_{3}$ 
%
%
\begin{enumerate}[(i)]
  \item{$\langle \tau - n^{2} \rangle  \gtrsim \langle n_{1}n \rangle$, or}
    \\
  \item{  $\langle \tau_{1} - n_{1}^{2} \rangle  \gtrsim \langle n_{1}n \rangle$, or}
    \\
  \item{ $\langle \tau - \tau_{1} - (n - n_{1})^{2} \rangle  \gtrsim \langle n_{1}n
    \rangle$}.
\end{enumerate}
In region $B_{3,1}$, these reduce to the single case $ \langle \tau - n^{2} \rangle
\gtrsim \langle n_{1}n \rangle$, and in region $B_{3,2}$ to the single
case $ \langle \tau_{1} - n_{1}^{2} \rangle
\gtrsim \langle n_{1}n \rangle$. Estimating first in region
$B_{3,1}$, we apply \autoref{lem:calc} and obtain
%
%
%
%
\begin{equation}
  \label{region-b31}
\begin{split}
& \langle \tau - n^{2}  \rangle ^{-2a} \langle n
    \rangle ^{2s}
    \sum_{n_{1} \in \zz} \int_{\rr} \frac{\chi_{B_{3,1}}}{ \langle n_{1} \rangle ^{2s} \langle n-n_{1} \rangle ^{2s} 
\langle \tau_{1} + n_{1}^{2}  \rangle \langle  \tau - \tau_{1} - (n -
n_{1})^{2}  \rangle}
d \tau_1 
\\
& \lesssim \langle \tau - n^{2} \rangle ^{-2a} \langle n \rangle ^{2s}
\sum_{n_{1} \in
\zz}  \frac{\chi_{B_{3,1}}}{\langle n_{1} \rangle ^{2s} \langle n - n_{1} \rangle
^{2s} \langle \tau - n^{2} + 2nn_{1}  \rangle ^{2b}}
\\
& \lesssim 
\sum_{n_{1} \in
\zz}  \frac{\langle n_1 \rangle ^{-2s} \langle n - n_{1} \rangle ^{-2s}}{\langle
n \rangle ^{-2s} \langle n_{1}n \rangle
^{2a}} \times \frac{\chi_{B_{3,1}}}{\langle \tau - n^{2} + 2nn_{1}
\rangle ^{2b}}.
\end{split}
\end{equation}
%
%
But for $n = n_{1}$, 
%
%
\begin{equation}
  \label{prelim-int-est-b-1}
\begin{split}
\frac{\langle n_1 \rangle ^{-2s} \langle n - n_{1} \rangle ^{-2s}}{\langle
n \rangle ^{-2s} \langle n_{1}n \rangle
^{2a}} \le 1
\end{split}
\end{equation}
while for $n \neq 0, n_{1} \neq 0, n \neq n_{1}$, 
\begin{equation}
  \begin{split}
  \label{prelim-int-est-b-2}
& \frac{\langle n_1 \rangle ^{-2s} \langle n - n_{1} \rangle ^{-2s}}{\langle
n \rangle ^{-2s} \langle n_{1}n \rangle
^{2a}} 
\\
& \le \begin{cases}
  | n_{1} |^{-4s-2a} , \quad & s < 0
\\
1, \quad  & s \ge 0. 
\end{cases}
\end{split}
\end{equation}
%
%
where the last step follows from \autoref{lem:integer-bound}. Since $a \ge 0$,
it follows from \eqref{prelim-int-est-b-1}-\eqref{prelim-int-est-b-2} that 
%
\begin{equation*}
\frac{\langle n_1 \rangle ^{-2s} \langle n - n_{1} \rangle ^{-2s}}{\langle
n \rangle ^{-2s} \langle n_{1}(n - n_{1}) \rangle
^{2a}} \le 1, \quad s \ge 0 \text{ or } s<0, s \ge -a/2 
\end{equation*}
%
%
which we use to bound the right hand side of \eqref{region-b31} by
%
%
\begin{equation*}
\begin{split}
\sum_{n_{1} \in
\zz} 
\frac{1}{\langle \tau - n^{2} + 2nn_{1}  \rangle ^{2b}}
\end{split}
\end{equation*}
%
%
%
which is finite for $b > 1/4$, due to \autoref{lem:sum-estimate}.
Recalling that 
$$ \langle \tau_{1} - n_{1}^{2} \rangle
\gtrsim \langle n_{1}n \rangle$$
%
in region $B_{3,2}$, we have
%
%
\begin{equation*}
\begin{split}
& \langle \tau - n^{2}  \rangle ^{-2a} \langle n
    \rangle ^{2s}
    \sum_{n_{1} \in \zz} \int_{\rr} \frac{\chi_{B_{3,1}}}{ \langle n_{1} \rangle ^{2s} \langle n-n_{1} \rangle ^{2s} 
\langle \tau_{1} + n_{1}^{2}  \rangle \langle  \tau - \tau_{1} - (n -
n_{1})^{2}  \rangle}
d \tau_1 
\\
& \lesssim \langle \tau - n^{2}  \rangle ^{-2a}     \sum_{n_{1} \in \zz} \int_{\rr} \frac{\chi_{B_{3,1}} \langle n
    \rangle ^{2s}
}{ \langle n_{1} \rangle ^{2s} \langle n-n_{1} \rangle ^{2s} 
\langle n_{1}n \rangle ^{2b} \langle  \tau - \tau_{1} - (n -
n_{1})^{2}  \rangle}
d \tau_1 
\end{split}
\end{equation*}
%
%
which by computation similar to \eqref{prelim-int-est-b-2}
is bounded by
%
%
\begin{equation}
  \label{prelim-int-est-2-b}
\begin{split}
\langle \tau - n^{2}  \rangle ^{-2a} \sum_{n_{1} \in \zz} \int_{\rr} \frac{\chi_{B_{3,1}} }{ \langle  \tau - \tau_{1} - (n -
n_{1})^{2}  \rangle^{2b}}
d \tau_1 
\end{split}
\end{equation}
%
%
for $s \ge 0$ or $s \le 0, b \ge -2s$. If $b > 1/2$, then integrating in
$\tau_{1}$ and discarding the  $\langle \tau - n^{2}  \rangle ^{-2a}$ term, we
then bound \eqref{prelim-int-est-2-b} by
%
%
\begin{equation*}
\begin{split}
  C \sum_{n_{1} \in \zz} \frac{\chi_{B_{3,1}}}{\langle \tau - (n -
  n_{1})^{2} \rangle ^{2b -1}}
  & \simeq
  \sum_{n_{1} \in \zz} \frac{\chi_{B_{3,1}}}{\langle \tau - n^{2} +
  2nn_{1} - n_{1}^{2}
  \rangle ^{2b -1}}
  \\
  & < \infty, \quad b > 1/2
\end{split}
\end{equation*}
%
%
where the last step follows from  \autoref{lem:sum-estimate}.

\section{Second order Modified Boussinesq  equation}
\label{sec:intro}
We consider the initial value problem (ivp) for a modified Boussinesq
equation ($B_4$) equation 
\begin{gather}
  u_{tt} - u_{xx} + (u^2)_{xx} = 0,
  \label{eqn:mb}
  \\
  u(x,0) = u_{0}(x), \quad u_{0} \in H^{s}
  \label{eqn:mb-init-data}
\end{gather}
and conjecture the following.
%
%
%%%%%%%%%%%%%%%%%%%%%%%%%%%%%%%%%%%%%%%%%%%%%%%%%%%%%
%
%
%                Main Theorem
%
%
%%%%%%%%%%%%%%%%%%%%%%%%%%%%%%%%%%%%%%%%%%%%%%%%%%%%%
%
%
\begin{theorem}
  If $s>s_c$ then then the  i.v.p, for the fourth order modified
  Boussinesq  equation is well-posed
  \begin{itemize}
    \item In $H^s(\rr)$ if $s > s_c$
    \item In $H^{s}(\ci)$ if $s > s_c + 1/4$,
  \end{itemize}
  and the data-to-solution map is  ?? (continuous?, Lip?, smooth, analytic?). 
  \label{thm:wp}
\end{theorem}
%
%
%
%
%
Since the scaling conserves data in $\dot{H}^{-3/2}$......
It seems that this equation is ``like KdV".
So one may expect KdV type theorems...
That is, $s_c=-3/4$ on the line and $s_c=-1/2$ on the circle,
if one uses bilinear estimates.
But, Kappeler and collaborators went all the way to $-1$ for KdV.
However KdV is integrable. Is this equation integrable?
Also, people conjecture that the critical index for KdV well-posedness 
in some appropriate sense should be the scaling index which is  $-3/2$.

\newpage
\appendix
\section{}
%\subsection{Proof of \autoref{lem:embedding}}
%%
%%
%\begin{equation*}
%\begin{split}
  %\| u(t) - u(t') \|_{H^s}^{2}
  %& = \sum_{n \in \zz} (1 + n^{2})^{s} [\wt{u}(n, t) - \wt{u}(n, t')]
  %\\
  %& = \sum_{n \in \zz} (1 + n^{2})^{s} \int_{\rr} (e^{it\tau} - e^{it'
  %\tau})\wh{u}(n, \tau) d \tau
  %\\
  %& \le 2 \sum_{n \in \zz} (1 + n^{2})^{s} \int_{\rr} \wh{u}(n, \tau) d \tau
  %\\
  %& \simeq \sum_{n \in \zz} (1 + n^{2})^{s} \int_{\rr} (1 + | | \tau | -
  %n^{2} |)^{b}(1 + | | \tau | - n^{2} |)^{-b} | \wh{u}(n, \tau) | d \tau.
%\end{split}
%\end{equation*}
%%
%%
%Applying Cauchy-Schwartz in $\tau$, we bound this by
%%
%%
%\begin{equation*}
%\begin{split}
  %& \sum_{n \in \zz} (1 + n^{2})^{s} \left[ \int_{\rr} (1 + | | \tau | -
  %n^{2} |)^{2b} | \wh{u}(n, \tau) |^{2} d \tau \right]^{1/2} \left[ \int_{\rr}
  %(1 + | | \tau | - n^{2} |)^{-2b} d \tau \right]^{1/2}
  %\\
  %& = \sum_{n \in \zz} (1 + n^{2})^{s} \left[ \int_{\rr} (1 + | | \tau | -
  %n^{2} |)^{2b} | \wh{u}(n, \tau) |^{2} d \tau \right]^{1/2} \left[ \int_{\rr}
  %(1 +  | \tau' | )^{-2b} d \tau' \right]^{1/2}
  %\\
  %& = c \sum_{n \in \zz} (1 + n^{2})^{s} \left[ \int_{\rr} (1 + | | \tau | -
  %n^{2} |)^{2b} | \wh{u}(n, \tau) |^{2} d \tau \right]^{1/2} \qquad (b > 1/2). 
%\end{split}
%\end{equation*}
%%
%%
%%
%%
%Applying Cauchy-Schwartz in $n$ then gives the bound
%%
%%
%\begin{equation*}
%\begin{split}
  %\sum_{n \in \zz} (1 + n^{2})^{s} \int_{\rr} (1 + | | \tau | - n^{2}
  %|)^{2b} \wh{u}(n, \tau) d \tau = \| u \|_{X_{s,b}}^{2}.
%\end{split}
%\end{equation*}
%%
%%
%An application of dominated convergence completes the proof. \qquad \qedsymbol
%
\subsection{Proof of \autoref{lem:mod-princ-symb-bound}} 
\label{ssec:pf-mod-princ}
By the reverse triangle inequality, we have
%
%
\begin{equation*}
\begin{split}
  | \tau | = | \tau + n^{2} - n^{2} | \ge | | \tau + n^{2} | - n^{2} |.
\end{split}
\end{equation*}
%
%
Furthermore, if $\tau - n^{2} < 0$, then
%
%
\begin{equation*}
\begin{split}
  | | \tau - n^{2} | - n^{2} | = | n^{2} - \tau - n^{2} | = | \tau |
\end{split}
\end{equation*}
%
%
while if $\tau - n^{2} > 0$, then
%
%
\begin{equation*}
\begin{split}
  | | \tau - n^{2} | - n^{2} | \le n^{2} \le \tau = |\tau|
\end{split}
\end{equation*}
%
%
completing the proof. \qquad \qedsymbol
%
%


%\nocite{*}
%\bibliography{/Users/davidkarapetyan/math/bib-files/references.bib}
%
%
% \bib, bibdiv, biblist are defined by the amsrefs package.
\begin{bibdiv}
\begin{biblist}

\bib{Kenig:1996aa}{article}{
      author={Kenig, Carlos~E.},
      author={Ponce, Gustavo},
      author={Vega, Luis},
       title={A bilinear estimate with applications to the {K}d{V} equation},
        date={1996},
        ISSN={0894-0347},
     journal={J. Amer. Math. Soc.},
      volume={9},
      number={2},
       pages={573\ndash 603},
         url={http://dx.doi.org/10.1090/S0894-0347-96-00200-7},
      review={\MR{1329387 (96k:35159)}},
}

\end{biblist}
\end{bibdiv}\end{document}
