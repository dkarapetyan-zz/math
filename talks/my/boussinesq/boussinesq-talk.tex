\documentclass{beamer}
\setbeamersize{text margin left=0.4cm, text margin right=0.4cm}
%\usetheme{Boadilla}
\usetheme{Madrid}
\setbeamertemplate{theorems}[numbered]
\setbeamercovered{transparent}
\usepackage{amscd}
\usepackage{amsfonts}
\usepackage{amsmath}
\usepackage{amssymb}
\usepackage{amsthm}
\usepackage{fancyhdr}
\usepackage{latexsym}
\usepackage{lmodern}
\usepackage{cancel}
\usepackage{hyperref}
\synctex=1    
\numberwithin{equation*}{section}
\newcommand{\bigno}{\bigskip\noindent}
\newcommand{\ds}{\displaystyle}
\newcommand{\medno}{\medskip\noindent}
\newcommand{\smallno}{\smallskip\noindent}
\newcommand{\nin}{\noindent}
\newcommand{\ts}{\textstyle}
\newcommand{\rr}{\mathbb{R}}
\newcommand{\p}{\partial}
\newcommand{\zz}{\mathbb{Z}}
\newcommand{\cc}{\mathbb{C}}
\newcommand{\ci}{\mathbb{T}}
\newcommand{\tor}{\mathbb{T}}
\newcommand{\ee}{\varepsilon}
\newcommand{\wh}{\widehat}
\newcommand{\weak}{\rightharpoonup}
\newcommand{\vp}{\varphi}
\newtheorem{proposition}{Proposition}
\newtheorem{claim}{Claim}
\newtheorem{remark}{Remark}
\newtheorem{conjecture}[subsection]{conjecture}
%% Equation Numbers %%
%%%%%%%%%%%%%%%%%%%%%%
\title[Ill-posedness for Boussinesq]{New Ill-posedness Results for the ``Good'' Boussinesq Equation }
\author[Geba, Himonas, Karapetyan]{Dan Geba (University of Rochester), \\ Alex Himonas (University of Notre Dame), \\ David Karapetyan (University of Rochester)}
\institute[UR, ND]{}
\date{}
\begin{document}
\begin{frame}
\titlepage
\end{frame}
\section{``Good''Boussinesq: Introduction}
\begin{frame}
  \frametitle{``Good'' Boussinesq equation}
\begin{gather*}
u_{tt} - u_{xx} + u_{xxxx} + (u^2)_{xx} = 0, \quad x \in \rr \ \text{or} \
\ci, \ t \in \rr
\label{eqn:mb-2}
\\
u(x,0) = u_{0}(x), \quad \p_t u(x, 0) = u_{1}(x) = \vp_{x}(x)
\label{eqn:mb-init-data-2}
\\
\notag
(u_0, \vp) \in
H^{s}\times
H^{s-1}.
\end{gather*}
\pause
\begin{itemize}
\item{}
Why is it good? Answer lies in the principal symbol of the linear part.
\begin{gather*}
  \tau^{2} - (\xi^{4} + \xi^{2}).
\end{gather*}
\pause
Real roots. ``Bad Boussinesq'' has symbol
\begin{gather*}
 \tau^{2} + (\xi^{4} + \xi^{2}). 
\end{gather*}
\pause
Imaginary roots. Exponential blowup of $\wh{u}(x, \tau)$ as $\xi \to \infty$ for root $i \sqrt{\xi^{4} + \xi^{2}}$.
\end{itemize}
\end{frame}
\begin{frame}
\frametitle{Some Features}
\begin{itemize}
\item{}
Original formulation of ``good'' Boussinesq is two component Hamiltonian system
\begin{gather*}
  u_{t} = v_{x}, \quad v_{t} = (u - u_{xx} - (u^{2})_{x}),
  \\
  u(x,0) = u_{0}(x), \quad v(x,0) = v_{0}(x).
\end{gather*}
\pause
\item Zakharov showed has infinitely many conservation laws and is completely integrable for real line. McKean did the same for the circle.
\pause
\item{}
Solitary waves extensively studied. Bona and Sachs, Alexander and Sachs, Liu, and others.
\pause
\item{}Using traveling wave ansatz $u_{c}(x,t) = f(x - ct)$, a priori assumption of soliton-like behavior, and calculus, can show existence of soliton
\begin{gather*}
  u_{c}(x,t) = f(x - ct) = \frac{3}{2}(1-c^{2}) \sech^{2} \left
  [\frac{\sqrt{(1-c^{2})}}{2}(x -ct -x_{0})
  \right ]
\end{gather*}
\end{itemize}
\end{frame}
\begin{frame}
\frametitle{Some Features}
\begin{itemize}
\item{}
Conserves energy
\begin{gather*}
  \int [v^{2} + u^{2} + u_{x}^{2} - 2F] dx
\end{gather*}
where $F' = u^{2}$.
\pause
\item{} Conserves momentum
\begin{gather*}
  \int uv dx.
\end{gather*}
\pause
\item{}
Conserves mean
\begin{gather*}
  \int u dx.
\end{gather*}
\end{itemize}
\end{frame}
\section{History or Local Well-posedness}
\begin{frame}
  \frametitle{History of Local Well-posedness}
  \begin{itemize}
  \item{}
  Bourgain, Kenig-Ponce-Vega, Fang and Grillakis, Linares, Farah, Farah and Scialom are the relevant names.
  \pause
  \item{}
  Key Bourgain spaces are those of Grillakis and Fang
  \end{itemize}
  \end{frame}
 \begin{frame} 
  \begin{definition}
Let $\mathcal{Y}$ be the space of functions $F(\cdot)$ such that
\begin{enumerate}[(I)]
\item{$F: \ci \times \rr \to \cc$}.
\item{$F(x, \cdot) \in \mathcal{S}(\rr)$ for each $x \in \ci$}.
\item{$F(\cdot, t) \in C^{\infty}(\ci)$for each $t \in \rr$}.
\end{enumerate}
For $s, b \in \rr$, $X_{s,b}$, and $Y_{s,b}$ denote the completion of $\mathcal{Y}$ with
respect to the norms
\begin{gather*}
\|F\|_{X_{s,b}} = \| \langle n \rangle ^{s} \langle | \tau | - n^{2} \rangle \wh{u}(n, \tau) \|_{\ell^{2}_{n}L^{2}_{\tau}}
\end{gather*}
and
\begin{gather*}
\|F\|_{Y_{s,b}} = \|F\|_{X_{s,b}} + \|n^s \wh{F}\|_{\ell^{2}_{n} L^1_\tau }.
\end{gather*}
\end{definition}
\end{frame}
\begin{frame}
\frametitle{History of Local Well-posedness}
Why these spaces? Becomes clear when you write Boussinesq in localized integral form
\begin{align*}
& u(x,t)
\label{main1-rel-term-0}
\\
\label{main1-rel-term-1}
& = \psi(t) \sum_{n \in \zzdot} e^{inx} \wh{u_{0}}(n) \frac{e^{i\gamma(n)t} + e^{-i\gamma(n)t}}{2} 
\\
\label{main1-rel-term-2}
& + \psi(t) \sum_{n \in \zzdot} e^{inx}
\wh{u_{1}}(n)\frac{e^{i\gamma(n)t} - e^{-i\gamma(n)t}}{2 i \gamma(n)} 
\\
\label{main1-rel-term-3}
& +  \psi(t)\sum_{a = \pm 1} \sum_{n \in \zzdot} \frac{n^{2}}{\gamma(n)}\int_\rr e^{ixn}  
e^{it \tau} \frac{1 - \psi(\tau -  a\gamma(n)) 
}{\tau -  a\gamma(n)} \wh{w}(n, \tau) \ d \tau
\\
\label{main1-rel-term-4}
& + \psi(t) \sum_{a = \pm 1} \sum_{n \in \zzdot} \frac{n^{2}}{\gamma(n)}\int_\rr e^{i(xn + 
t a\gamma(n))}
\frac{1- \psi(\tau -  a\gamma(n))}{\tau -  a\gamma(n)} \wh{w}(n, \tau) \ d \tau
\\
\label{main1-rel-term-4.5}
& +  \psi(t) \sum_{a = \pm 1}  \sum_{k \ge 1} \frac{i^k t^k}{k!}
\sum_{n \in \zzdot} \frac{n^{2}}{\gamma(n)}\int_\rr e^{i(xn + t a\gamma(n) )}
\psi(\tau -  a\gamma(n)) (\tau -  a\gamma(n))^{k-1} \wh{w}(n, \tau)
\end{align*}
\end{frame}
\begin{frame}
\frametitle{History of Local Well-posedness}
\begin{itemize}
\item{}Local well-posedness (Picard iterative sense) for $s > 1/4$ on $\rr$ and $\ci$ by Farah and Farah and Scialom using these spaces.
\pause
\item{} Kishimoto recently posted a paper on the arxiv improving this to $s \ge -1/2$ for line and circle by transforming Boussinesq to NLS type equation, then using a Besov endpoint type modification of Bourgain type spaces adapted to NLS with additional weights on specific dyadic components, following spaces Tao and Bejenaru used for quadratic NLS.
\end{itemize}
\end{frame}
\begin{frame}
\frametitle{Ill-Posedness}
Bejanaru-Tao Framework:
\begin{equation*}
u\,=\,L(f)\,+\,N(u,u),
\end{equation*}
\begin{itemize}
\item
$f=(u_0, u_1)$ is an initial data lying in a data space $D$ (e.g., $H^s \times H^{s-2}$)
\pause
\item solution $u$ takes values in a solution space $S \subseteq	 C([0, T]; H^s)$,
\pause
\item $L: D \to S$ is a linear operator
\pause
\item $N:S\times S \to S$ is a bilinear form, both of which are densely defined. \pause
\item 
If  $(D,\|\cdot \|_D)$ and $(S,\|\cdot \|_S)$ are a pair of Banach spaces satisfying 
\begin{equation*}
\|L(f)\|_S \leq C \|f\|_D,\qquad \|N(u,v)\|_S \leq C \|u\|_S \|v\|_S,
\label{estim}
\end{equation*}
where $C>0$ is an absolute constant, standard contraction argument shows that, for all  $\|f\|_D<\frac{1}{16C^2}$, there exists a unique solution $u$ with $\|u\|_D<\frac{1}{4C}$. 
\end{itemize}
\end{frame}
\begin{frame}
\begin{itemize}
\item In fact, solution is the sum of an absolutely convergent series in S, 
\begin{equation*}
u\,=\,\sum_{n=1}^{\infty} A_n(f), \qquad (\forall) \|f\|_D<\frac{1}{16C^2},
\label{series}
\end{equation*}
where the nonlinear maps $A_n: D\to S$ ($n\geq 1$) are defined recursively by
\begin{equation*}
A_1(f)\,=\,L(f), \qquad A_n(f)\,=\,\sum_{k=1}^{n-1} N(A_k(f),A_{n-k}(f)), \qquad (\forall)n\geq 2.
\label{An}
\end{equation*}
\pause
\item{}
To prove ill-posedness, enough to show
\begin{equation*}
\limsup_{N\to \infty} \|f_N\|_D\,\ll\,1, \qquad \quad \sup_N \frac{\|A_n(f_N)\|_{S'}}{\|f_N\|_{D'}}\,=\,\infty.
\end{equation*}
\pause
\item{}
Drawback is you must have local well-posedness theory to prove ill-posedness. 
\pause
\item{}
Advantage over Kishimoto ill-posedness method ($s < -1/2$) is that we don't need to have endpoint well-posedness to prove ill-posedness.
\end{itemize}  
\end{frame}
\begin{frame}
\frametitle{Proof of Ill-posedness in Periodic Case}
\begin{equation*}
f_N\,=\,(g_N, 0), \qquad \widehat{g_N}(n)\,=\,\frac{r}{N^s} \,\left(\chi_{\{n=N\}}\,+\,\chi_{\{n=1-N\}}\right).
\end{equation*}
\pause
Obtain
\[
\|f_N\|_{H^{s'}\times H^{s'-2}}\,\simeq\,r\,N^{s'-s}, \qquad \widehat{A_2(f_N)}(t,n)\,=\,0, \ (\forall) n \neq 1,\]
and
\[
\widehat{A_2(f_N)}(t,1)\,\sim\,\frac{1}{N^{2s}}\,\int_0^t \sin((t-t') \lambda(1))\cos(t' \lambda(N-1))\cos(t'\lambda(N))\,dt'.
\]
\pause
\[
\widehat{A_2(f_N)}(t,1)\,\sim\,a(t,N)\cdot O(N^{-2s-4})\,+\,b(t,N)\cdot O(N^{-2s-2}),
\]
where
\[\aligned
a(t,N)&= \cos (t\lambda(1))-\cos[t(\lambda(N-1) + \lambda(N))],\\ b(t,N)&= \cos (t\lambda(1))-\cos[t(\lambda(N) - \lambda(N-1))].
\endaligned
\]
\end{frame}
\begin{frame}
\frametitle{Proof of Ill-posedness in Periodic Case}
In comparison with the non-periodic case, we have a little bit more flexibility in our choice of $t=t_N$. We can take $t_N \sim \frac{1}{N}$, but we can also set $t_N=1$, as the divergence of the sequence $(\cos n)_n$ allows one to identify a subsequence $(N_k)_k\to\infty$ such that 
\[
\liminf_k \left|b(1, N_k)\right|\,>\,0.
\]
\pause
Any of these two avenues leads to the same conclusion:
\[
\sup_N\,\frac{\|A_2(f_N)\|_{C([0,1], H^{s'}(\mathbb{T}))}}{\|f_N\|_{H^{s'}\times H^{s'-2}}}\,=\,\infty,\]
for $s'<\min\{s,-2-s\}$, which concludes this argument.
\end{frame}
\begin{frame}
\begin{center}\Large Thank you \\ for your attention
\end{center}
\end{frame}
\end{document}
