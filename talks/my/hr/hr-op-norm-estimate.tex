\documentclass[12pt,reqno]{amsart}
\usepackage{amscd}
\usepackage{amsfonts}
\usepackage{amsmath}
\usepackage{amssymb}
\usepackage{amsthm}
\usepackage{appendix}
\usepackage{fancyhdr}
\usepackage{latexsym}
\usepackage{pdfsync}
\usepackage{cancel}
\usepackage[colorlinks=true, pdfstartview=fitv, linkcolor=blue,
citecolor=blue, urlcolor=blue]{hyperref}
\input epsf
\input texdraw
\input txdtools.tex
\input xy
\xyoption{all}
%%%%%%%%%%%%%%%%%%%%%%
\usepackage{color}
\definecolor{red}{rgb}{1.00, 0.00, 0.00}
\definecolor{darkgreen}{rgb}{0.00, 1.00, 0.00}
\definecolor{blue}{rgb}{0.00, 0.00, 1.00}
\definecolor{cyan}{rgb}{0.00, 1.00, 1.00}
\definecolor{magenta}{rgb}{1.00, 0.00, 1.00}
\definecolor{deepskyblue}{rgb}{0.00, 0.75, 1.00}
\definecolor{darkgreen}{rgb}{0.00, 0.39, 0.00}
\definecolor{springgreen}{rgb}{0.00, 1.00, 0.50}
\definecolor{darkorange}{rgb}{1.00, 0.55, 0.00}
\definecolor{orangered}{rgb}{1.00, 0.27, 0.00}
\definecolor{deeppink}{rgb}{1.00, 0.08, 0.57}
\definecolor{darkviolet}{rgb}{0.58, 0.00, 0.82}
\definecolor{saddlebrown}{rgb}{0.54, 0.27, 0.07}
\definecolor{black}{rgb}{0.00, 0.00, 0.00}
\definecolor{dark-magenta}{rgb}{.5,0,.5}
\definecolor{myblack}{rgb}{0,0,0}
\definecolor{darkgray}{gray}{0.5}
\definecolor{lightgray}{gray}{0.75}
%%%%%%%%%%%%%%%%%%%%%%
%%%%%%%%%%%%%%%%%%%%%%%%%%%%
%  for importing pictures  %
%%%%%%%%%%%%%%%%%%%%%%%%%%%%
\usepackage[pdftex]{graphicx}
\usepackage{epstopdf}
% \usepackage{graphicx}
%% page setup %%
\setlength{\textheight}{20.8truecm}
\setlength{\textwidth}{14.8truecm}
\marginparwidth  0truecm
\oddsidemargin   01truecm
\evensidemargin  01truecm
\marginparsep    0truecm
\renewcommand{\baselinestretch}{1.1}
%% new commands %%
\newcommand{\tf}{\tilde{f}}
\newcommand{\ti}{\tilde}
\newcommand{\bigno}{\bigskip\noindent}
\newcommand{\ds}{\displaystyle}
\newcommand{\medno}{\medskip\noindent}
\newcommand{\smallno}{\smallskip\noindent}
\newcommand{\nin}{\noindent}
\newcommand{\ts}{\textstyle}
\newcommand{\rr}{\mathbb{R}}
\newcommand{\p}{\partial}
\newcommand{\zz}{\mathbb{Z}}
\newcommand{\cc}{\mathbb{C}}
\newcommand{\ci}{\mathbb{T}}
\newcommand{\ee}{\varepsilon}

\def\refer #1\par{\noindent\hangindent=\parindent\hangafter=1 #1\par}
%% equation numbers %%
\renewcommand{\theequation}{\thesection.\arabic{equation}}
%% new environments %%
%\swapnumbers
\theoremstyle{plain}  % default
\newtheorem{theorem}{Theorem}
\newtheorem{proposition}{Proposition}
\newtheorem{lemma}{Lemma}
\newtheorem{corollary}{Corollary}
\newtheorem{claim}{Claim}
\newtheorem{remark}{Remark}
\newtheorem{conjecture}[subsection]{conjecture}
\theoremstyle{definition}
\newtheorem{definition}{Definition}
%
\begin{document}
%\begin{titlepage}
\title{ The Critical Operator Norm Estimate in $H^s$}
\author{David Karapetyan}
\address{Department of Mathematics  \\
         University  of Notre Dame\\
		          Notre Dame, IN 46556 }
				  \date{09/03/2009}
				  %
				  \maketitle
				  %
				  %
				  \parindent0in
				  \parskip0.1in
				  %
				  %\end{titlepage}
				  %
				  %
				  %
				  \setcounter{equation}{0}
				 \begin{proposition}
	\label{lem4r}
	For $r \le s$ and $\ee>0$
	\begin{equation}
	\label{0r}
		\begin{split}
			\|I - J_\ee\|_{L(H^s, H^r)} \le o(\ee^{s-r}).
		\end{split}
	\end{equation}
\end{proposition}
%
\vskip0.1in
We will split our proof into two parts:
\vskip0.1in
{\bf Proof of Proposition \ref{0r} on $\ci$.}
Pick a function $\widehat{j}(\xi) \in \mathcal{S}(\rr)$ such that
	\begin{equation}
		\label{0u}
		\begin{split}
			& 0 \le \widehat{j}(\xi) \le 1,
			\\
			& \widehat{j}(\xi) = 1 \ \ \text{if} \ \ |\xi| \le 1.
		\end{split}
	\end{equation}
	Then by Parseval's identity, we can define
	\begin{equation}
		\begin{split}
			j_\ee (x) \overset{L^2(\ci)}{=} \sum_{\xi \in \zz}
			\widehat{j}(\ee \xi) e^{\pi i x}, \quad \ee > 0
			\label{parseval-def}
		\end{split}
	\end{equation}
	which is equivalent to stating that we can find $\left\{ j_\ee
	\right\} \subset L^2(\ci)$ such that
	\begin{equation}
		\begin{split}
			\widehat{j_\ee} = \widehat{j }(\ee \xi), \quad \ee > 0.
			\label{widehat-def}
		\end{split}
	\end{equation}
	Hence, we can define $J_\ee$ to be the Friedrichs mollifier 
	\begin{equation}
		\label{0'u}
		\begin{split}
			J_\ee f(x) = j_\ee * f(x), \quad \ee>0.
		\end{split}
	\end{equation}
Pick an arbitrary unit vector $u \in H^s(\ci)$ and
	$r, s \in \rr$ such that $r \le s$. Using the fact that
$\widehat{j_\ee}(\xi) = \widehat{j}(\ee \xi)$, we have
\begin{equation}
	\begin{split}
		\|u - J_\ee u\|_{H^r(\ci)}^2 
		& = \sum_{\xi \in \zz} |\widehat{u}(\xi) - \widehat{j_\ee * u}(\xi) |^2
		(1+\xi^2)^r
		\\
		& = \sum_{\xi \in \zz} |\widehat{u}(\xi) - \widehat{j_\ee}(\xi)
		\widehat{u}(\xi) |^2 (1+\xi^2)^r
		\\
		& = \sum_{\xi \in \zz} | [1- \widehat{j_\ee}(\xi] \widehat{u}(\xi) |^2
		(1+\xi^2)^r
		\\
		& = \sum_{\xi \in \zz} | [1- \widehat{j}(\ee \xi)] \widehat{u}(\xi) |^2
		(1+\xi^2)^r.
		\label{1r}
	\end{split}
\end{equation}
Assume $r \le s$. Then by construction we have
\begin{equation*}
	\begin{split}
		|1 - \widehat{j } (\xi) | \le |\xi|^{s-r}
	\end{split}
\end{equation*}
for all $\xi \in \rr$; hence
\begin{equation}
	\begin{split}
		|1 - \widehat{ j }(\ee \xi)| \le |\ee \xi |^{s-r}, \quad \forall
		\xi \in \rr, \ \ee > 0.
		\label{2r}
	\end{split}
\end{equation}
Applying \eqref{2r} to \eqref{1r} and recalling that $r \le s$, we obtain
\begin{equation}
	\label{2pr}
	\begin{split}
	\|u - J_\ee u\|_{H^r(\ci)}^2 
	& \le \sum_{\xi \in \zz}  |\ee \xi |^{2(s-r)}
	|\widehat{u}(\xi)|^2 (1 + \xi^2)^r
	\\
	& = \ee^{2(s-r)} \sum_{\xi \in \zz} |\widehat{u}(\xi)|^2  \cdot (\xi^2)^{s-r}
	(1 + \xi^2)^{r-s} (1 + \xi^2)^{s}
	\\
	& \le \ee^{2(s-r)}
	\sum_{\xi \in \zz} |\widehat{u}(\xi)|^2 (1 + \xi^2)^s
	\\
	& =  \ee^{2(s-r)}.
	\end{split}
\end{equation}
Furthermore,
\begin{equation*}
	\begin{split}
		& |[1- \widehat{j_\ee}(\xi)] \widehat{u}(\xi)|^2 (1 + \xi^2)^r \le
		|\widehat{u}(\xi)|^2 (1 + \xi^2)^r, \quad \forall \ee > 0, \ 
		\text{and}
		\\
		& \sum_{\xi \in \zz} |\widehat{u}(\xi)|^2 (1 + \xi^2)^r < \infty,
		\quad r \le s;
	\end{split}
\end{equation*}
therefore, by the dominated convergence theorem for series
\begin{equation}
	\label{o1}
	\begin{split}
		\lim_{\ee \to 0} \|u - J_\ee u \|_{H^r }^2 
		& = \lim_{\ee \to 0} \sum_{\xi \in \zz} |[1-\widehat{j_\ee}(\xi)]
		\widehat{u}(\xi) |^2 (1 + \xi^2)^r
		\\
		& = \lim_{\ee \to 0} \sum_{\xi \in \zz} |[1-\widehat{j}(\ee \xi)]
		\widehat{u}(\xi) |^2 (1 + \xi^2)^r
		\\
		& = \sum_{\xi \in \zz} \lim_{\ee \to 0} |[1-\widehat{j}(\ee \xi)]
		\widehat{u}(\xi) |^2 (1 + \xi^2)^r
		\\
		& = 0.
	\end{split}
\end{equation}
To complete the proof of Proposition \ref{lem4r}, we take note of the following interpolation result:
\begin{remark}
	\label{lem2r}
	For $\sigma < r \le s$ and arbitrary $u \in {S'}$,
	\begin{equation}
		\begin{split}
			\|u\|_{H^{r}} \le
			\|u\|_{H^\sigma}^{(r-s)/(\sigma -s)}
			\|u\|_{H^s}^{1 - (r-s)/(\sigma -s)}.
			\label{16u}
		\end{split}
	\end{equation}
\end{remark}
%
%
%
%
We have two cases:

{\bf Proof of Remark \ref{lem2r} on $\ci$.}
Assuming $u \in \mathcal{S'(\ci)}$,
we rewrite and apply Holder's inequality:
\begin{equation*}
	\begin{split}
		\|u\|_{H^{r}(\ci)}^2
		& = \sum_{\xi \in \zz} |\widehat{u}(\xi)|^2 (1 + \xi^2)^{r}
		\\
		& = \sum_{\xi \in \zz}
		\left [|\widehat{u}(\xi)|^2 (1 + \xi^2)^\sigma \right ]^{(r-s)/(\sigma -s)}
		\cdot \left [ |\widehat{u}(\xi )
		|^2 (1+ \xi^2)^s \right ] ^{1 - (r-s)/(\sigma -s)} 
		\\
		& \le \|\left[ |\widehat{u}(\xi)|^2 (1 + \xi^2)^\sigma
		\right]^{(r-s)/(\sigma -s)} \|_{l^{(\sigma -s)/(r-s)}(\zz)}
		\\
		& \cdot \|\left[ |\widehat{u}(\xi)|^2 (1 + \xi^2)^\sigma
		\right]^{1- (r-s)/(\sigma -s)} \|_{l^{1/[1 -(\sigma -s)/(r-s)]}(\zz)}
		\\
		& = \|v\|_{H^\sigma(\ci)}^{2(r-s)/(\sigma -s)}
		\|v\|_{H^s(\ci)}^{2[1 - (r-s)/(\sigma -s)]}
	\end{split}
\end{equation*}
from which the result follows.
\vskip0.1in
{\bf Proof of Remark \ref{lem2r} on $\rr$.}
Assuming $u \in \mathcal{S'(\rr)}$,
we rewrite and apply Holder's inequality:
\begin{equation*}
	\begin{split}
		&\|u\|_{H^{r}(\rr)}^2
		\\
		& = \int_\rr |\widehat{u}(\xi)|^2 (1 + \xi^2)^{r}
		\\
		& = \int_\rr
		\left [|\widehat{u}(\xi)|^2 (1 + \xi^2)^\sigma \right ]^{(r-s)/(\sigma -s)}
		\cdot \left [ |\widehat{u}(\xi )
		|^2 (1+ \xi^2)^s \right ] ^{1 - (r-s)/(\sigma -s)} 
		\\
		& \le \|\left[ |\widehat{u}(\xi)|^2 (1 + \xi^2)^\sigma
		\right]^{(r-s)/(\sigma -s)} \|_{L^{(\sigma -s)/(r-s)}(\rr)}
		\\
		& \cdot \|\left[ |\widehat{u}(\xi)|^2 (1 + \xi^2)^\sigma
		\right]^{1- (r-s)/(\sigma -s)} \|_{L^{1/[1 -(\sigma 
		-s)/(r-s)]}(\rr)}
		\\
		& = \|v\|_{H^\sigma(\rr)}^{2(r-s)/(\sigma -s)}
		\|v\|_{H^s(\rr)}^{2[1 - (r-s)/(\sigma -s)]}
	\end{split}
\end{equation*}
from which the result follows. This completes the proof of Remark 
\ref{lem2r}. $\quad \Box$
\vskip0.1in
Applying Remark \ref{lem2r}, and estimates \eqref{2pr} and \eqref{o1}, we conclude that
\begin{equation*}
	\begin{split}
		\|u - J_\ee u \|_{H^r(\ci)}^2
		& \le \|u - J_\ee u
		\|_{L^2(\ci)}^{(s-r)/s} \|u - J_\ee u \|_{H^s(\ci)}^{1 -
		(s-r)/s}
		\\
		& = \left( \ee^{2s} \right)^{(s-r)/s} \cdot o(1)
		\\
		& = o(\ee^{2(s-r)})
	\end{split}
\end{equation*}
fom which \eqref{0r} follows (on $\ci$).
\vskip0.1in
%
%
%
{\bf Proof of Proposition \ref{0r} on $\rr$.}
Pick a function $\widehat{j}(\xi) \in \mathcal{S}(\rr)$ such that
	\begin{equation}
		\label{p0u}
		\begin{split}
			& 0 \le \widehat{j}(\xi) \le 1,
			\\
			& \widehat{j}(\xi) = 1 \ \ \text{if} \ \ |\xi| \le 1.
		\end{split}
	\end{equation}
	Then by the Fourier Inversion Theorem, we can define
	\begin{equation}
		\begin{split}
			j_\ee (x) {=} \int_{\rr}
			\widehat{j}(\ee \xi) e^{\pi i x}, \quad \ee > 0
			\label{pparseval-def}
		\end{split}
	\end{equation}
	which is equivalent to stating that we can find $\left\{ j_\ee
	\right\} \subset \mathcal{S}(\rr)$ such that
	\begin{equation}
		\begin{split}
			\widehat{j_\ee} = \widehat{j }(\ee \xi), \quad \ee > 0.
			\label{pwidehat-def}
		\end{split}
	\end{equation}
	Hence, we can define $J_\ee$ to be the Friedrichs mollifier 
	\begin{equation}
		\label{p0'u}
		\begin{split}
			J_\ee f(x) = j_\ee * f(x), \quad \ee>0.
		\end{split}
	\end{equation}
Pick an arbitrary unit vector $u \in H^s(\rr)$ and
	$r, s \in \rr$ such that $r \le s$. Using the fact that
$\widehat{j_\ee}(\xi) = \widehat{j}(\ee \xi)$, we have
\begin{equation}
	\begin{split}
		\|u - J_\ee u\|_{H^r(\rr)}^2 
		& = \int_{\rr} |\widehat{u}(\xi) - \widehat{j_\ee * u}(\xi) |^2
		(1+\xi^2)^r
		\\
		& = \int_{\rr} |\widehat{u}(\xi) - \widehat{j_\ee}(\xi)
		\widehat{u}(\xi) |^2 (1+\xi^2)^r
		\\
		& = \int_{\rr} | [1- \widehat{j_\ee}(\xi] \widehat{u}(\xi) |^2
		(1+\xi^2)^r
		\\
		& = \int_{\rr} | [1- \widehat{j}(\ee \xi)] \widehat{u}(\xi) |^2
		(1+\xi^2)^r.
		\label{p1r}
	\end{split}
\end{equation}
Assume $r \le s$. Then by construction we have
\begin{equation*}
	\begin{split}
		|1 - \widehat{j } (\xi) | \le |\xi|^{s-r}
	\end{split}
\end{equation*}
for all $\xi \in \rr$; hence
\begin{equation}
	\begin{split}
		|1 - \widehat{ j }(\ee \xi)| \le |\ee \xi |^{s-r}, \quad \forall
		\xi \in \rr, \ \ee > 0.
		\label{p2r}
	\end{split}
\end{equation}
Applying \eqref{p2r} to \eqref{p1r} and recalling that $r \le s$, we obtain
\begin{equation}
	\label{p2pr}
	\begin{split}
	\|u - J_\ee u\|_{H^r(\rr)}^2 
	& \le \int_{\rr}  |\ee \xi |^{2(s-r)}
	|\widehat{u}(\xi)|^2 (1 + \xi^2)^r
	\\
	& = \ee^{2(s-r)} \int_{\rr} |\widehat{u}(\xi)|^2  \cdot (\xi^2)^{s-r}
	(1 + \xi^2)^{r-s} (1 + \xi^2)^{s}
	\\
	& \le \ee^{2(s-r)}
	\int_{\rr} |\widehat{u}(\xi)|^2 (1 + \xi^2)^s
	\\
	& =  \ee^{2(s-r)}.
	\end{split}
\end{equation}
Furthermore,
\begin{equation*}
	\begin{split}
		& |[1- \widehat{j_\ee}(\xi)] \widehat{u}(\xi)|^2 (1 + \xi^2)^r \le
		|\widehat{u}(\xi)|^2 (1 + \xi^2)^r, \quad \forall \ee > 0, \ 
		\text{and}
		\\
		& \int_{\rr} |\widehat{u}(\xi)|^2 (1 + \xi^2)^r < \infty,
		\quad r \le s;
	\end{split}
\end{equation*}
therefore, by the dominated convergence theorem
\begin{equation}
	\label{po1}
	\begin{split}
		\lim_{\ee \to 0} \|u - J_\ee u \|_{H^r }^2 
		& = \lim_{\ee \to 0} \int_{\rr} |[1-\widehat{j_\ee}(\xi)]
		\widehat{u}(\xi) |^2 (1 + \xi^2)^r
		\\
		& = \lim_{\ee \to 0} \int_{\rr} |[1-\widehat{j}(\ee \xi)]
		\widehat{u}(\xi) |^2 (1 + \xi^2)^r
		\\
		& = \int_{\rr} \lim_{\ee \to 0} |[1-\widehat{j}(\ee \xi)]
		\widehat{u}(\xi) |^2 (1 + \xi^2)^r
		\\
		& = 0.
	\end{split}
\end{equation}
Applying Remark \ref{lem2r}, and estimates \eqref{p2pr} and \eqref{po1}, we 
conclude that
\begin{equation*}
	\begin{split}
		\|u - J_\ee u \|_{H^r(\rr)}^2
		& \le \|u - J_\ee u
		\|_{L^2(\rr)}^{(s-r)/s} \|u - J_\ee u \|_{H^s(\rr)}^{1 -
		(s-r)/s}
		\\
		& = \left( \ee^{2s} \right)^{(s-r)/s} \cdot o(1)
		\\
		& = o(\ee^{2(s-r)})
	\end{split}
\end{equation*}
from which \eqref{0r} follows (on $\rr$).
This completes the proof of Proposition
\ref{lem4r}.  $\quad \Box$
\vskip0.1in
 

				  \end{document}


