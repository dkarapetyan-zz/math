\documentclass[12pt, oneside]{amsart}
\usepackage{graphicx}
\usepackage[nohead, margin=0.5in]{geometry}
\usepackage{enumerate}
%\geometry{left=0.5in,right=0.5in,top=0.3in,bottom=0.5in} 

\pagestyle{empty}
%Next are definitions of "\one" etc. 
%This is an easy way of assigning points to your questions and to 
%your table all at once. 
\newcommand{\one}{10}
\newcommand{\two}{10}
\newcommand{\three}{10}
\newcommand{\four}{4}
\newcommand{\five}{4}
\newcommand{\six}{4}
\newcommand{\seven}{4}
\newcommand{\eight}{4}
\newcommand{\nine}{4}
\newcommand{\ten}{6}
\synctex=1

\begin{document}


%feel free to change the title page
%
 
\begin{center}
\hrulefill\\
{\bf \textsf{\raisebox{-0.10cm}{Fall 2012: MATH 143} \hspace{\fill} 
\raisebox{-0.10cm}{Calculus III} \hspace{\fill}
\raisebox{-0.10cm}{David Karapetyan}}}\\
\hrulefill\\
{\large \rule{0cm}{1.2cm} \textsf{Sunday 12/16/2012} \hfill
\textsf{Final Exam} \hfill  \textsf{180 minutes}}\\
{\large\rule{0cm}{1.2cm}\textsf{Name: \framebox[2.9in]{\rule{0cm}{0.8cm}} 
\hspace{\fill}
Student ID: \framebox[2.1in]{\rule{0cm}{0.8cm}}}}\\
\end{center}
\vspace{0.8cm}

\noindent
{\bf \textsf{Instructions.}}

\begin{enumerate}
\item Don't panic.
\item Ok, panic a little.
\item Attempt all questions.   
\item Show all the steps of your work clearly.  
\item Good luck. 
%The method (reasoning) used to 
%obtain an answer is worth more than the answer itself.   
\end{enumerate}

\vfill

%The \rule commands create vertical space, which makes things sit nicely in 
%vertical way in boxes of table below.

\begin{center}
{\large
\begin{tabular}{|c|c|c|}
\hline
\rule[-0.3cm]{0cm}{1cm}
\textsf{Question} & \textsf{Points} &  \textsf{Your Score} \\
\hline
\hline
\rule[-0.3cm]{0cm}{1cm}
\textsf{Q1} & \one &\\
\hline
\rule[-0.3cm]{0cm}{1cm}
\textsf{Q2} & \two &\\
\hline
\rule[-0.3cm]{0cm}{1cm}
\textsf{Q3} & \three &\\
\hline
\rule[-0.3cm]{0cm}{1cm}
\textsf{Q4} & \four &\\
\hline
\rule[-0.3cm]{0cm}{1cm}
\textsf{Q5} & \five &\\
\hline
\rule[-0.3cm]{0cm}{1cm}
\textsf{Q6} & \six &\\
\hline
\rule[-0.3cm]{0cm}{1cm}
\textsf{Q7} & \seven &\\
\hline
\rule[-0.3cm]{0cm}{1cm}
\textsf{Q8} & \eight &\\
\hline
\rule[-0.3cm]{0cm}{1cm}
\textsf{Q9} & \nine &\\
\hline
\rule[-0.3cm]{0cm}{1cm}
\textsf{Q10} & \ten &\\
 \hline 
\rule[-0.3cm]{0cm}{1cm}
 \textsf{TOTAL} & 60 & \\
 \hline
 \end{tabular}
} 

\end{center}

\vfill


\newpage
\noindent
\textbf{Q1}.\\ \\ Determine whether each of the sequences converges or diverges. If it converges, compute it's limit. If it diverges, state whether it diverges to $+\infty$, $-\infty$, or neither. \\

\begin{enumerate}[a)]
  \item
     $ \left\{ \displaystyle{\frac{7^{n}}{10^{n}}} \right\}_{n=1}^{\infty}$
     \vspace{6cm}
   \item
     $\left\{ \cos n \right\}_{n=1}^{\infty}$
     \vspace{6cm}
   \item
     $\left\{ \displaystyle{\frac{(n+1)!}{(2n-1)!}} \right\}_{n=1}^{\infty}$
\newpage
  \item
     $ \left\{ (1 + 6/n)^{n} \right\}_{n=1}^{\infty}$
     \vspace{9cm}
   \item
     $\left\{ (-1/5)^{n}\right\}_{n=1}^{\infty}$
   \end{enumerate}
\newpage
\noindent
\textbf{Q2}. \\ \\ If the series converges, explain why, and compute its sum. If the series diverges, state whether it diverges to $\infty$, $-\infty$, or neither, and explain why. \\

\begin{enumerate}[a)]
  \item
    $\displaystyle{\sum_{n=1}^{\infty} 1}$
    \vspace{6cm}
  \item
    $\displaystyle{\sum_{n=1}^{2} (1 + 1/n)^{n}}$
\vspace{6cm}
\item
  $\sum_{n=1}^{\infty} [e^{-n} - e^{-(n+1)}]$
  \newpage
   \item
     $\sum_{n=0}^{\infty} \displaystyle{(1/4)^{n+1}}$
     \vspace{9cm}
   \item $\sum_{n=1}^{\infty} \displaystyle{2^{n}}$
  \vspace{6cm}
\end{enumerate}
\newpage
\noindent
\textbf{Q3}. \\ \\ Use a convergence or divergence test to show whether the following series converge or diverge. Make sure to state which test you are using, and show all work. \\
\begin{enumerate}[a)]
  \item
$\displaystyle{\sum_{n=1}^{\infty} \frac{1}{n^{2} + 1}}$
    \vspace{6cm}
  \item
$\displaystyle{\sum_{n=3}^{\infty} \frac{\ln n}{n}}$
\vspace{6cm}
\item
  $ \displaystyle{\sum_{n=1}^{\infty} \frac{1}{n!}}$
  \newpage
\item
  $ \displaystyle{\sum_{n=1}^{\infty} \frac{(-5)^{n}}{n} }$
  \vspace{9cm}
\item
  $ \displaystyle{\sum_{n=1}^{\infty} \frac{e^{n}}{n^{n}} }$
\end{enumerate}



\newpage
\noindent
\textbf{Q4}. \\ \\ Use the integral test to show whether the
series converges or diverges. \\
\begin{enumerate}[a)]
\item
  $ \displaystyle{\sum_{n=1}^{\infty} \frac{{1}}{n^{2}}}$
  \vspace{9cm}
\item
Now, approximate the sum in a) using the first $2$
terms, and state an upper bound for the error (Hint: use the remainder estimate
for the integral test).
  \newpage
\end{enumerate}




\newpage
\noindent
\textbf{Q5}. \\ \\ State whether the series diverges, converges conditionally, or converges absolutely, and explain why. \\
\begin{enumerate}[a)]
  \item
$\displaystyle{\sum_{n=2}^{\infty} \frac{(-1)^{n}}{\log n}}$
    \vspace{9cm}
\item
Now, approximate the sum in a) using the first $2$ terms, and state an upper bound for the error (Hint: use the remainder estimate for alternating series).
  \newpage
\end{enumerate}
\textbf{Q6}. \\ \\ Find the interval of \emph{conditional} convergence for the following power series.\\
\begin{enumerate}[a)]
  \item
$\displaystyle{\sum_{n=1}^{\infty} \frac{n^{2} x^{n}}{2^{n}}}$
\vspace{9cm}
\item
  $ \displaystyle{\sum_{n=1}^{\infty} \frac{{(5-x)^{n}}}{n^{2n}}}$
  \newpage
\end{enumerate}




\newpage
\noindent
\textbf{Q7}. \\ \\ Find the Maclaurin series of the following functions.\\
\begin{enumerate}[a)]
  \item
    $\displaystyle{f(x) = \frac{1}{1-x}}$
    \vspace{9cm}
  \item
    $\displaystyle{f(x) = e^{x^{2}}}$
  \newpage
\end{enumerate}

\noindent
\textbf{Q8}. \\ \\ Find the Taylor series about $a=\pi$ of the following functions.\\
\begin{enumerate}[a)]
  \item
    $\displaystyle{f(x) = \cos x}$
    \vspace{9cm}
  \item
    $\displaystyle{f(x) = (x-\pi)^{2} + 3(x-\pi)^{5}}$
  \newpage
\end{enumerate}

\noindent
\textbf{Q9}. \\ \\ Approximate the following function using the first two terms of a Taylor series expansion. \\
\begin{enumerate}[a)]
  \item
    $\displaystyle{f(x)=x^{2/3}} \ \ \text{evaluated at} \ \ x=8.1$
    \vspace{11cm}
  \item
What is an upper bound for the error? Show all work. 
  \newpage
\end{enumerate}

\noindent
\textbf{Q10}. \\ \\ Convert the following curve to polar coordinates.  \\ 
\begin{enumerate}[a)]
  \item
    $\displaystyle{x^{2} + 2y^{2} = 4}$
\vspace{9cm}
\item
Now, compute the area between the curve in a) and the $x$-axis from $\theta = 0$ to $\theta = \pi$.
  \newpage
\end{enumerate}


\end{document}
