\documentclass[12pt, oneside]{amsart}
\usepackage{graphicx}
\usepackage[nohead, margin=0.5in]{geometry}
\usepackage{enumerate}
%\geometry{left=0.5in,right=0.5in,top=0.3in,bottom=0.5in} 

\pagestyle{empty}
%Next are definitions of "\one" etc. 
%This is an easy way of assigning points to your questions and to 
%your table all at once. 
\newcommand{\one}{10}
\newcommand{\two}{10}
\newcommand{\three}{20}
\newcommand{\four}{10}
\newcommand{\five}{20}
\newcommand{\six}{10}
\newcommand{\seven}{10}
\newcommand{\eight}{10}
\newcommand{\nine}{10}
\newcommand{\ten}{10}
\newcommand{\eleven}{10}
\newcommand{\twelve}{10}
\synctex=1

\begin{document}


%feel free to change the title page
%
 
\begin{center}
\hrulefill\\
{\bf \textsf{\raisebox{-0.10cm}{Fall 2012: MATH 143} \hspace{\fill} 
\raisebox{-0.10cm}{Calculus III} \hspace{\fill}
\raisebox{-0.10cm}{David Karapetyan}}}\\
\hrulefill\\
{\large \rule{0cm}{1.2cm} \textsf{Thursday 11/15/2012} \hfill
\textsf{Midterm #2} \hfill  \textsf{90 minutes}}\\
{\large\rule{0cm}{1.2cm}\textsf{Name: \framebox[2.9in]{\rule{0cm}{0.8cm}} 
\hspace{\fill}
Student ID: \framebox[2.1in]{\rule{0cm}{0.8cm}}}}\\
\end{center}
\vspace{0.8cm}

\noindent
{\bf \textsf{Algorithm for Success.}}

\begin{enumerate}
\item Attempt all questions.   
\item Show all the steps of your work clearly.  
\item Don't freak out.
%The method (reasoning) used to 
%obtain an answer is worth more than the answer itself.   
\end{enumerate}

\vfill

%The \rule commands create vertical space, which makes things sit nicely in 
%vertical way in boxes of table below.

\begin{center}
{\large
\begin{tabular}{|c|c|c|}
\hline
\rule[-0.3cm]{0cm}{1cm}
\textsf{Question} & \textsf{Points} &  \textsf{Your Score} \\
\hline
\hline
\rule[-0.3cm]{0cm}{1cm}
\textsf{ Q1} & \one &\\
\hline
\rule[-0.3cm]{0cm}{1cm}
\textsf{ Q2} & \two &\\
\hline
\rule[-0.3cm]{0cm}{1cm}
\textsf{ Q3} & \three &\\
\hline
\rule[-0.3cm]{0cm}{1cm}
\textsf{ Q4} & \four &\\
\hline
\rule[-0.3cm]{0cm}{1cm}
\textsf{ Q5} & \five &\\
\hline
\rule[-0.3cm]{0cm}{1cm}
\textsf{ Q6} & \six &\\
\hline
\rule[-0.3cm]{0cm}{1cm}

 \textsf{ TOTAL} & 80 & \\
 \hline
 \end{tabular}
} 

\end{center}

\vfill


\newpage
\noindent
\textbf{Q1}.\\ \\ Find the Maclaurin series for the following functions. What is their radius of convergence, and why? \\

\begin{enumerate}[a)]
  \item
    $\displaystyle{f(x) = \frac{2}{1 - 3x^{2}}}$
          \vspace{6cm}
   \item
         $\displaystyle{f(x) = \frac{6}{2 - 4x^{2}}}$
     \vspace{6cm}
   \item
     $\displaystyle{f(t)=\ln | 1 - t |}$
     \vspace{6cm}
     \newpage
     \end{enumerate}
\newpage
\noindent
\textbf{Q2}. \\ \\ Give the Taylor series about $a=1$ of the following functions. \\ 

\begin{enumerate}[a)]
  \item
    $\displaystyle{f(x) = 1 + x^{2} + 2x^{3}}$
    \vspace{6cm}
  \item
    $\displaystyle{f(x) = 3(x-1)^{2} + 5(x-1)^{5}}$
\vspace{6cm}
\item
  $f(x) = e^{x}$
  \newpage
   \end{enumerate}
\newpage
\noindent
\textbf{Q3}. \\ \\ Approximate the following functions using the first three terms of a Taylor series expansion. What is an upper bound for the error? Show all work. \\
\begin{enumerate}[a)]
  \item
    $\displaystyle{f(x)=x^{1/3}} \ \ \text{evaluated at} \ \ x=8.1$
    \vspace{12cm}
  \item
   $\displaystyle{f(x)=e^{x}} \ \ \text{evaluated at} \ \ x=0.1$
  \newpage
\end{enumerate}



\newpage
\noindent
\textbf{Q4}. \\ \\ Find the Maclaurin series of the following functions. \\ 
\begin{enumerate}[a)]
  \item
$\displaystyle{f(x) = x^{2} \cos x}$
    \vspace{6cm}
  \item
    $\displaystyle{\ln|1-x^{2}|}$
\vspace{6cm}
\item
  $ \displaystyle{(1-x)\sin x}$
  \newpage
\end{enumerate}




\newpage
\noindent
\textbf{Q5}. \\ \\ Approximate the following integrals using two terms from a Taylor series. What is an upper bound for your error? \\
\begin{enumerate}[a)]
  \item
$\displaystyle{\int_{0}^{1} e^{-x^{2}}dx}$
    \vspace{6cm}
  \item
$\displaystyle{\int_{0}^{2} \cos(x^{2})dx}$
  \newpage
\end{enumerate}
\textbf{Q6}. \\ \\ Convert a) to to cartesian coordinates, and b) to polar coordinates. Then compute the arc length of these curves from $\theta = 0$ to $\theta = \pi$. \\ 
\begin{enumerate}[a)]
  \item
$\displaystyle{r=4 \cos \theta}$
    \vspace{6cm}
  \item
    $\displaystyle{x^{2} + y^{2} = 1}$
\vspace{6cm}
  \newpage
\end{enumerate}

\end{document}
