\documentclass[12pt, oneside]{amsart}
\usepackage{graphicx}
\usepackage[nohead, margin=0.5in]{geometry}
\usepackage{enumerate}
%\geometry{left=0.5in,right=0.5in,top=0.3in,bottom=0.5in} 

\pagestyle{empty}
%Next are definitions of "\one" etc. 
%This is an easy way of assigning points to your questions and to 
%your table all at once. 
\newcommand{\one}{20}
\newcommand{\two}{20}
\newcommand{\three}{20}
\newcommand{\four}{20}
\newcommand{\five}{20}
\newcommand{\rr}{\mathbb{R}}
\synctex=1

\begin{document}


%feel free to change the title page
%

\begin{center}
    \hrulefill\\
    {\bf \textsf{\raisebox{-0.10cm}{Fall 2013: MATH 163} \hspace{\fill} 
            \raisebox{-0.10cm}{Ordinary Differential Equations} \hspace{\fill}
            \raisebox{-0.10cm}{David Karapetyan}}}\\
    \hrulefill\\
    {\large \rule{0cm}{1.2cm} \textsf{Friday 11/15/2013} \hfill
        \textsf{Midterm} \hfill  \textsf{50 minutes}}\\
    {\large\rule{0cm}{1.2cm}\textsf{Name: \framebox[2.9in]{\rule{0cm}{0.8cm}} 
            \hspace{\fill}
            Student ID: \framebox[2.1in]{\rule{0cm}{0.8cm}}}}\\
\end{center}
\vspace{0.8cm}

\noindent
{\bf \textsf{Instructions.}}

\begin{enumerate}
    \item Attempt all questions.   
    \item Show all the steps of your work clearly.  
    \item Good luck 
        %The method (reasoning) used to 
        %obtain an answer is worth more than the answer itself.   
\end{enumerate}

\vfill

%The \rule commands create vertical space, which makes things sit nicely in 
%vertical way in boxes of table below.

\begin{center}
    {\large
        \begin{tabular}{|c|c|c|}
            \hline
            \rule[-0.3cm]{0cm}{1cm}
            \textsf{Question} & \textsf{Points} &  \textsf{Your Score} \\
            \hline
            \hline
            \rule[-0.3cm]{0cm}{1cm}
            \textsf{Q1} & \one &\\
            \hline
            \rule[-0.3cm]{0cm}{1cm}
            \textsf{Q2} & \two &\\
            \hline
            \rule[-0.3cm]{0cm}{1cm}
            \textsf{Q3} & \three &\\
            \hline
            \rule[-0.3cm]{0cm}{1cm}
            \textsf{Q4} & \four &\\
            \hline
            \rule[-0.3cm]{0cm}{1cm}

            \textsf{TOTAL} & 80 & \\
            \hline
        \end{tabular}
    } 

\end{center}

\vfill


\newpage
\noindent
\textbf{Q1}. \\ \\ 
\begin{enumerate}[a)]
    \item
Find the general solution to the equation
\begin{equation*}
\begin{split}
y'' + 2y' - 3y =0
\end{split}
\end{equation*}
without converting it to a first order system. What is its characteristic polynomial?


\vspace{5in}
\item
    Now find the unique solution satisfying $y(1) = 1$ and $y'(1) = 2$.
\end{enumerate}

\newpage
\noindent
\textbf{Q2}. \\ \\ 
\begin{enumerate}[a)]
    \item
Find the general solution to the system
\begin{equation*}
\begin{split}
& x' = -2x + 4y
\\
& y' = -2x+ 2y 
\end{split}
\end{equation*}



\vspace{5in}
\item
    Now find the unique solution satisfying $x(0) = 0$ and $y(0) = 0$. 
\end{enumerate}

\newpage
\noindent
\textbf{Q3}. \\ \\ 
Repeat both a) and b) in Q$1$, but now solve by converting to a first order system.
        \newpage

\noindent
\textbf{Q4}. \\ \\ 
Consider the second order equation $y'' + p(t)y' + q(t) = 0$. Suppose that $u$
and $v$ are solutions. Show that $u + v$ is also a solution.


\end{document}



