\documentclass[12pt, oneside]{amsart}
\usepackage{graphicx}
\usepackage[nohead, margin=0.5in]{geometry}
\usepackage{enumerate}
%\geometry{left=0.5in,right=0.5in,top=0.3in,bottom=0.5in} 

\pagestyle{empty}
%Next are definitions of "\one" etc. 
%This is an easy way of assigning points to your questions and to 
%your table all at once. 
\newcommand{\one}{20}
\newcommand{\two}{20}
\newcommand{\three}{10}
\newcommand{\four}{20}
\newcommand{\five}{20}
\newcommand{\six}{10}
\newcommand{\seven}{20}
\newcommand{\rr}{\mathbb{R}}
\synctex=1

\begin{document}


%feel free to change the title page
%

\begin{center}
    \hrulefill\\
    {\bf \textsf{\raisebox{-0.10cm}{Fall 2013: MATH 163} \hspace{\fill} 
            \raisebox{-0.10cm}{Ordinary Differential Equations} \hspace{\fill}
            \raisebox{-0.10cm}{David Karapetyan}}}\\
    \hrulefill\\
    {\large \rule{0cm}{1.2cm} \textsf{Thursday 12/19/2013} \hfill
        \textsf{Final} \hfill  \textsf{180 minutes}}\\
    {\large\rule{0cm}{1.2cm}\textsf{Name: \framebox[2.9in]{\rule{0cm}{0.8cm}} 
            \hspace{\fill}
            Student ID: \framebox[2.1in]{\rule{0cm}{0.8cm}}}}\\
\end{center}
\vspace{0.8cm}

\noindent
{\bf \textsf{Instructions.}}

\begin{enumerate}
    \item Attempt all questions.   
    \item Show all the steps of your work clearly.  
    \item Good luck 
        %The method (reasoning) used to 
        %obtain an answer is worth more than the answer itself.   
\end{enumerate}

\vfill

%The \rule commands create vertical space, which makes things sit nicely in 
%vertical way in boxes of table below.

\begin{center}
    {\large
        \begin{tabular}{|c|c|c|}
            \hline
            \rule[-0.3cm]{0cm}{1cm}
            \textsf{Question} & \textsf{Points} &  \textsf{Your Score} \\
            \hline
            \hline
            \rule[-0.3cm]{0cm}{1cm}
            \textsf{Q1} & \one &\\
            \hline
            \rule[-0.3cm]{0cm}{1cm}
            \textsf{Q2} & \two &\\
            \hline
            \rule[-0.3cm]{0cm}{1cm}
            \textsf{Q3} & \three &\\
            \hline
            \rule[-0.3cm]{0cm}{1cm}
            \textsf{Q4} & \four &\\
            \hline
            \rule[-0.3cm]{0cm}{1cm}
            \textsf{Q5} & \five &\\
            \hline
            \rule[-0.3cm]{0cm}{1cm}
            \textsf{Q6} & \six &\\
            \hline
                        \rule[-0.3cm]{0cm}{1cm}
            \textsf{Q7} & \seven &\\
            \hline
            \rule[-0.3cm]{0cm}{1cm}
            \textsf{TOTAL} & 120 & \\
            \hline
        \end{tabular}
    } 

\end{center}

\vfill


\newpage
\noindent
\textbf{Q1}. \\ \\ 
\begin{enumerate}[a)]
    \item
Find the general solution to the equation
\begin{equation*}
\begin{split}
y' - 2xy = \cos x, \qquad y = y(x).
\end{split}
\end{equation*}


\vspace{5in}
\item
    Now find the unique solution satisfying $y(0) = 1$ .
\end{enumerate}

\newpage
\noindent
\textbf{Q2}. \\ \\ 
\begin{enumerate}[a)]
    \item
Find the general solution to the system
\begin{equation*}
\begin{split}
& x_1' = -3x_1 - 2x_2
\\
& x_2' = -2x_1 - 2 x_2
\end{split}
\end{equation*}
where $x_1 = x_1(t)$ and $x_2 = x_2(t)$.
\vspace{5in}
\item
    Now find the unique solution satisfying $x_1(0) = 1$ and $x_2(0) = 0$. 
\end{enumerate}

\newpage
\noindent
\textbf{Q3}. \\ \\ 
Convert
\begin{align*}
y'' + 2y' + y = t^2, \qquad y = y(t)
\end{align*}
to a first order system. Explain how you recover the solution $y$
to the above via the solution to your first order system.
        \newpage

\noindent
\textbf{Q4}. \\ \\ 
\begin{enumerate}[a)]
\item
Find the eigenvalues and corresponding eigenvectors of the matrix
\begin{align*}
A = \begin{bmatrix}
2 & -4 \\
-1 & -1
\end{bmatrix}
\end{align*}
\vspace{3in}
\item
Find the general solution to
\begin{align*}
\vec{y}' = Ay + \vec{b}, \qquad \vec{y} = \vec{y}(t)
\end{align*}
where
\begin{align*}
\vec{y} = \begin{bmatrix}
y_1(t) \\
y_2(t)
\end{bmatrix}, \quad
\vec{b} = \begin{bmatrix}
1 \\
2
\end{bmatrix}
\end{align*}
\end{enumerate}
and $A$ is as in part $a)$.
        \newpage

\noindent
\textbf{Q5}. \\ \\ 
\begin{enumerate}[a)]
	\item
Find the unique solution to
\begin{align*}
y' = y^2 t^3 \\
y(0)=1.
\end{align*}
\vspace{3in}
\item
Now, using Euler's method with stepsize $h = 1$, compute the value
of $y(1)$ and $y(2)$. Compare these
with the actual values you obtain using your
solution to $a)$, and explain why they differ.
\end{enumerate}
\newpage
\noindent
\textbf{Q6}. \\ \\ 
State, with explanation, whether the following ODEs are linear or nonlinear. What order are they?

\vspace{1in}
\begin{enumerate}[a)]
    \item
        $y'''' + 4y' + y^{3} = 5$, \quad $y = y(x)$
        \vspace{2in}

    \item
        $x^3y' + y = 5$, \quad $y = y(x)$
        \vspace{2in}

    \item 
        $y' + 7y'' = e^t$, \quad $y = y(t)$
        \vspace{2in}
\end{enumerate}
\newpage
\noindent
\textbf{Q7}. \\ \\ 
\begin{enumerate}[a)]
	\item
Use the method of undetermined coefficients to find the
general solution for
\begin{align*}
y'' + 2y' -3y = 5e^t, \qquad y = y(t)
\end{align*}
\vspace{3in}
\item

Use variation of parameters to find the general solution for
\begin{align*}
y'' -3y' + 4y = \sin t, \qquad y = y(t)
\end{align*}
\end{enumerate}
\end{document}



