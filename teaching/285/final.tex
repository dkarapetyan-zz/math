\documentclass[12pt, oneside]{amsart}
\usepackage{graphicx}
\usepackage[nohead, margin=0.5in]{geometry}
\usepackage{enumerate}
%\geometry{left=0.5in,right=0.5in,top=0.3in,bottom=0.5in} 
\usepackage{listings}
\pagestyle{empty}
%Next are definitions of "\one" etc. 
%This is an easy way of assigning points to your questions and to 
%your table all at once. 
\newcommand{\one}{20}
\newcommand{\two}{20}
\newcommand{\three}{20}
\newcommand{\four}{20}
\newcommand{\five}{20}
\newcommand{\rr}{\mathbb{R}}
\synctex=1

\begin{document}


%feel free to change the title page
%

\begin{center}
    \hrulefill\\
    {\bf \textsf{\raisebox{-0.10cm}{Spring 2014: MATH 285} \hspace{\fill} 
            \raisebox{-0.10cm}{Introduction to Applied Mathematics} \hspace{\fill}
            \raisebox{-0.10cm}{David Karapetyan}}}\\
    \hrulefill\\
    {\large \rule{0cm}{1.2cm} \textsf{Wednesday 04/30/2014} \hfill
        \textsf{Final} \hfill  \textsf{75 minutes}}\\
    {\large\rule{0cm}{1.2cm}\textsf{Name: \framebox[2.9in]{\rule{0cm}{0.8cm}} 
            \hspace{\fill}
            Student ID:\@\framebox[2.1in]{\rule{0cm}{0.8cm}}}}\\
\end{center}
\vspace{0.8cm}

\noindent
{\bf \textsf{Instructions.}}

\begin{enumerate}
    \item Attempt all questions.   
    \item Show all the steps of your work clearly.  
    \item Good luck 
        %The method (reasoning) used to 
        %obtain an answer is worth more than the answer itself.   
\end{enumerate}

\vfill

%The \rule commands create vertical space, which makes things sit nicely in 
%vertical way in boxes of table below.

\begin{center}
    {\large
        \begin{tabular}{|c|c|c|}
            \hline
            \rule[-0.3cm]{0cm}{1cm}
            \textsf{Question} & \textsf{Points} &  \textsf{Your Score} \\
            \hline
            \rule[-0.3cm]{0cm}{1cm}
            \textsf{Q1} & \one &\\
            \hline
            \rule[-0.3cm]{0cm}{1cm}
            \textsf{Q2} & \two &\\
            \hline
            \rule[-0.3cm]{0cm}{1cm}
            \textsf{Q3} & \three &\\
            \hline
            \rule[-0.3cm]{0cm}{1cm}
            \textsf{Q4} & \four &\\
            \hline
            \rule[-0.3cm]{0cm}{1cm}
            \textsf{Q5} & \five &\\
            \hline
            \rule[-0.3cm]{0cm}{1cm}

            \textsf{TOTAL} & 100 & \\
            \hline
        \end{tabular}
    } 

\end{center}

\vfill

\newpage
\noindent
\textbf{Q1}. \\ \\ 
Find the characteristic function of a random variable $X$ when it:
\\
\begin{enumerate}[(a)]
	\item is uniformly distributed in the interval $[0,1]$.
		\vspace{4in}
		\item is equal to $\mu$ with probability $1$.
\end{enumerate}
\newpage
\noindent
\textbf{Q2}. \\ \\
\begin{enumerate}[(a)]
	\item What is the definition of arbitrage?
		\vspace{2in}
	\item 
		State and prove the monotonicity theorem,
		assuming a no-arbitrage economy.
	\end{enumerate}
\newpage
\noindent
\textbf{Q3}.\\ \\ 
 Let $C(K, T, S(t), t)$ denote the price at time $t$ of a call option
			with underlying $S$, expiry $T$, and strike $K$. 
			\\ \\
\begin{enumerate}[(a)]
	\item Is $C$ an increasing function of $S$, decreasing function of $S$,
		or neither? Why? (please supply a proof).
			\vspace{5in}
    \item Is $C$ an increasing function of $K$, decreasing function of $K$,
			or neither? Why? (please supply a proof).
\end{enumerate}

\newpage
\noindent
\textbf{Q4}. \\ \\ 
\noindent
\begin{enumerate}[(a)]
    \item
			Suppose we have a one-step, two state economy with zero interest.
			Let $S_{0}$ denote the price of a stock at time $t=0$,
			and $S_{T}$ be the random variable
			denoting stock price after one step, where 
			\begin{equation*}
			\begin{split}
				S_{T} = \begin{cases}
					S_{0} + 2u, \quad & \text{with probability } p \\
					S_{0} - u, \quad & \text{with probability } 1-p. \\
				\end{cases}
			\end{split}
			\end{equation*}
			Construct a portfolio consisting of $S$ and a call option $C$ with $S$
			as an underlying and strike $S_{0}$
			such that the portfolio is \emph{risk-free}.
			\vspace{4in}
		\item Find the risk-neutral probability for $S$, and use it to compute
			the value of $C$ at time $t=0$. 

		\end{enumerate}

\newpage

\noindent \textbf{Q5}. \\ \\ 
State and prove the weak law of large numbers.





\end{document}



