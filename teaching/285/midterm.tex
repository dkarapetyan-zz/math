\documentclass[12pt, oneside]{amsart}
\usepackage{graphicx}
\usepackage[nohead, margin=0.5in]{geometry}
\usepackage{enumerate}
%\geometry{left=0.5in,right=0.5in,top=0.3in,bottom=0.5in} 
\usepackage{listings}
\pagestyle{empty}
%Next are definitions of "\one" etc. 
%This is an easy way of assigning points to your questions and to 
%your table all at once. 
\newcommand{\one}{20}
\newcommand{\two}{20}
\newcommand{\three}{20}
\newcommand{\four}{20}
\newcommand{\five}{20}
\newcommand{\rr}{\mathbb{R}}
\synctex=1

\begin{document}


%feel free to change the title page
%

\begin{center}
    \hrulefill\\
    {\bf \textsf{\raisebox{-0.10cm}{Spring 2014: MATH 285} \hspace{\fill} 
            \raisebox{-0.10cm}{Introduction to Applied Mathematics} \hspace{\fill}
            \raisebox{-0.10cm}{David Karapetyan}}}\\
    \hrulefill\\
    {\large \rule{0cm}{1.2cm} \textsf{Wednesday 03/05/2014} \hfill
        \textsf{Midterm} \hfill  \textsf{75 minutes}}\\
    {\large\rule{0cm}{1.2cm}\textsf{Name: \framebox[2.9in]{\rule{0cm}{0.8cm}} 
            \hspace{\fill}
            Student ID:\@\framebox[2.1in]{\rule{0cm}{0.8cm}}}}\\
\end{center}
\vspace{0.8cm}

\noindent
{\bf \textsf{Instructions.}}

\begin{enumerate}
    \item Attempt all questions.   
    \item Show all the steps of your work clearly.  
    \item Good luck 
        %The method (reasoning) used to 
        %obtain an answer is worth more than the answer itself.   
\end{enumerate}

\vfill

%The \rule commands create vertical space, which makes things sit nicely in 
%vertical way in boxes of table below.

\begin{center}
    {\large
        \begin{tabular}{|c|c|c|}
            \hline
            \rule[-0.3cm]{0cm}{1cm}
            \textsf{Question} & \textsf{Points} &  \textsf{Your Score} \\
            \hline
            \rule[-0.3cm]{0cm}{1cm}
            \textsf{Q1} & \one &\\
            \hline
            \rule[-0.3cm]{0cm}{1cm}
            \textsf{Q2} & \two &\\
            \hline
            \rule[-0.3cm]{0cm}{1cm}
            \textsf{Q3} & \three &\\
            \hline
            \rule[-0.3cm]{0cm}{1cm}
            \textsf{Q4} & \four &\\
            \hline
            \rule[-0.3cm]{0cm}{1cm}
            \textsf{Q5} & \five &\\
            \hline
            \rule[-0.3cm]{0cm}{1cm}

            \textsf{TOTAL} & 100 & \\
            \hline
        \end{tabular}
    } 

\end{center}

\vfill


\newpage
\noindent
\textbf{Q1}. \\ \\ 
A rare disease affects one person in $10^5$. A test for the disease
is wrong with probability $1/100$; that is, it is positive with probability
$1/100$ for someone who is in fact healthy, and negative
with probability $1/100$ for someone who is in fact ill. What is the probability
that you have the disease given that you took the test and it is positive?
\newpage
\noindent
\textbf{Q2}.\\ \\ 
\begin{enumerate}[(a)]
    \item Give the definition of a $\sigma$-algebra on a sample space. What are the properties of a probability measure on this $\sigma$-algebra? 
        \vspace{5in}
    \item Give the definition of a real-valued random variable.
\end{enumerate}

\newpage

\noindent
\textbf{Q3}. \\ \\ 

Let $X$ and $Y$ be two continuous independent random variables with associated densities $f_X$ and $f_Y$, respectively. Derive the density functions of $X+Y$ and $X^{1/2}$, where we are taking the \emph{positive} root of $X$.  
\newpage
\noindent
\textbf{Q4}. \\ \\ 
\noindent
\begin{enumerate}[(a)]
    \item
Suppose we have a biased coin ($3/4$ probability for flipping heads) and assign $-1$ points to flipping tails, and $+1$ points to
flipping heads. We play a game where, starting with $0$ points, we flip this coin repeatedly until we obtain $-1$ points, or $2$ points. What is the expected number of flips before the
game terminates? 
\vspace{4in}
\item Let getting to $-1$ points be considered a loss, and to $2$ points a win.
    What is the probability that you win the game? 
\end{enumerate}
\newpage
\noindent

\noindent \textbf{Q5}. \\ \\ 
\noindent
Prove that, for discrete random variables $X$ and $Y$ 
\begin{align*}
f_{Y|X}(y|x) = \frac{f_{(X,Y)}(x,y)}{f_X(x)}.
\end{align*}
Show that the same equality holds if $X$ and $Y$ are continuous random
variables.

\vspace{2in}




\end{document}



