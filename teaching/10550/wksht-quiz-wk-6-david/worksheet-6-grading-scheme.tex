\documentclass[12pt]{article}
\usepackage{tikz}
\setlength{\topmargin}{.5in}
\setlength{\textheight}{215mm}
\setlength{\textwidth}{142 mm}
\evensidemargin 0.3in
\oddsidemargin 0.3in
\topmargin -.2in
\setcounter{page}{1}
\usepackage{amssymb}
\usepackage{enumerate}
\usepackage{amsmath}
\usepackage{amsfonts}
\newtheorem{theorem}{Theorem}[section]
\newtheorem{lemma}[theorem]{Lemma}
\newtheorem{corollary}[theorem]{Corollary}
\newtheorem{claim}[theorem]{Claim}
   \newtheorem{prop}[theorem]{Proposition}
\newtheorem{no}[theorem]{Notation}
\newtheorem{definition}[theorem]{Definition}
\newtheorem{re}[theorem]{Remark}
\newtheorem{examp}{Example}[section]
\newtheorem {exercise}[theorem] {Exercise}
\newcommand{\Section}[1]{\section{#1}\setcounter{equation}{0}}
\renewcommand{\theequation}{\thesection.\arabic{equation}}
%\renewcommand{\baselinestretch}{1.4}
\textwidth 7.3 in\oddsidemargin -0.35in
\textheight 9.6in\topmargin -1in

\def\ds{\displaystyle}
\def\medno{\medskip \noindent}
\def\bigno{\bigskip \noindent}
\def\smno{\smallskip \noindent}
\def\smallno{\smallskip \noindent}
\synctex=1


\begin{document}

\hskip0.01in Name \underline{\hskip 4.5in}
\hskip0.2in
  Date \underline{\hskip 1in}


\begin{center}
\textbf{Math 10550 Worksheet 6}

Show all your work to receive credit.
\end{center}


\boxed{{\bf Problem \ 1}} Sketch the curve $\displaystyle y = \frac{x^{2}}{x^2 +
    4}$ on the following axes. Indicate the local extrema (i.e.\ local max's and
local min's) and the intervals where the function is
increasing, decreasing, concave, and convex. 
\vspace{-16em}
\begin{center}
\includegraphics[scale=0.9]{prob-1.pdf}
\end{center}
\vspace{-20em}

\noindent
\textcolor{red}{\Large 10 points total. 2 for sketching the graph properly, 4 for identifying the intervals of decrease and increase, and 4 for
        identifying intervals of convexity and concavity.}
\\
\\
To plot $y$ correctly, we gather some facts:
\begin{enumerate}[i)]
\item{} Asymptote $y=1$, since $\displaystyle \frac{x^{2}}{x^{2} + 4} \approx 1$ for large $x$. 
\item{}
    %
    %
    $ \displaystyle y'
    = \frac{(x^{2} + 4)(2x) - x^{2}(2x)}{(x^{2} + 4)^{2}}
    = \frac{8x}{(x^{2} + 4)^{2}}$, and so
    %
    %
    $ \displaystyle y' =0$ at $x = 0$, $y' < 0 \ \text{for} \  x  < 0 \
    \text{and} \ y' > 0$ for $x > 0$.
    %
    \item{}
      $\displaystyle y'' = \frac{8(x^{2} + 4)^{2} - 8x(2)(x^{2} +
    4)(2x)}{(x^{2} + 4)^{4}} = \frac{32 - 24x^{2}}{(x^{2} + 4)^{3}}$, and so
    $y'' = 0$ at $\displaystyle x = \pm \frac{2}{\sqrt{3}}$, $y'' > 0$ for
    $ \displaystyle
    -\frac{2}{\sqrt{3}} < x < \frac{2}{\sqrt{3}}$ and $y'' <0$ otherwise. 
    %
  %
  \item{}
    $y>0$ for all $x \neq 0$, $y(0) = 0$, $y(x) = y(-x)$, $y(2) = y(-2) = 1/2$.
\end{enumerate}
\newpage
\noindent
\boxed{{\bf Problem \ 2}} Sketch the curve $\displaystyle y=x - 2\sin x$ on $[0, 2\pi]$on the following axes. Indicate the local extrema (i.e.\ local max's and
local min's) and the intervals where the function is
increasing, decreasing, concave, and convex. 
\vspace{-16em}
\begin{center}
    \includegraphics[scale=0.9]{prob-2.pdf}
\end{center}
\vspace{-16em}
\textcolor{red}{\Large 10 points total. 2 for sketching the graph properly, 4 for identifying the intervals of decrease and increase, and 4 for
        identifying intervals of convexity and concavity.}
\\
\\
To plot $y$ correctly, we gather some facts:
\begin{enumerate}[i)]
\item{}
        $ \displaystyle y' = 1 - 2 \cos x$, and so
        $\displaystyle y'=0$ at $x = \pi/3, 5\pi/3$, $y' > 0$ for $\pi/3< x < 5\pi/3$, and $y' < 0$ for $0 < x < \pi/3$. 
        \item{}
        $y'' = 2 \sin x$, and so $y''= 0$ at $x = 0, \pi$, $y'' > 0$ for $0 < x < \pi$ and $y'' < 0$ for $\pi < x < 2 \pi$.
        \item{}
        $y(0) = 0$.
\end{enumerate}

\newpage
\noindent
\boxed{{\bf Problem \ 3}} Sketch the curve $xy = x^2 + 3$.
Indicate the local extrema (i.e.\ local max's and
local min's) and the intervals where the function is
increasing, decreasing, concave, and convex. 
\vspace{-16em}
\begin{center}
    \includegraphics[scale=0.9]{prob-3.pdf}
\end{center}
\vspace{-16em}

\noindent
\textcolor{red}{\Large 10 points total. 2 for sketching the graph properly, 4 for identifying the intervals of decrease and increase, and 4 for
        identifying intervals of convexity and concavity.}
\\
\\
To plot $y$ correctly, we gather some facts:
\begin{enumerate}[i)]
\item{}
    Asymptote $y= x + 3/x \approx x$ for large $x$. 
    \item{}
      $ \displaystyle y' = 1 - 3/x^{2}$ and so $y'=0$ for $x = \pm \sqrt{3}$, $y' > 0$ for $x > \sqrt{3}$ or $x < -\sqrt{3}$, and $y' < 0$ for all nonzero $x$ satisfying $- \sqrt{3} < x < \sqrt{3}$. Notice that $y'$ approaches $- \infty$ as $x \to 0$ from either the right or the left.
    \item{} $\displaystyle y'' = \frac{6}{x^{3}}$, and so $y'' > 0$ for $x > 0$ and $y'' < 0$ for $x < 0$. Notice that $y''$ approaches $-\infty$ as $x \to 0^{-}$ and approaches $\infty$ as $x \to 0^{+}$.
        \item{}
       $y(x) = -y(x)$ with $y(1) = 4, \ y(-1) = -4, \ y(3) = 4$ and  $y(-3) = -4$.
\end{enumerate}
\newpage
\noindent
\boxed{{\bf Problem \ 4}} Evaluate the following limits. 

\vspace{2em}
\noindent
\textcolor{red}{\Large 20 points total. 4 for each question. If they obtain the wrong answer, mark 1 point. If their computations are wrong, use your judgement. For example, silly mistakes should get marked at most 1 point. Conceptual mistakes get marked more, per your discretion.}
\\
\\
\begin{enumerate}[a)]
\item{}
  \qquad $ \lim_{x \to 0} \displaystyle \frac{\sin x}{\sqrt{x}} = 0$, since 
  \begin{equation*}
    \begin{split}
\lim_{x \to 0} \displaystyle \frac{\sin x}{\sqrt{x}} & =
    \lim_{x \to 0} \sqrt{x} \frac{\sin x}{x} 
    \\
    & = \lim_{x \to 0 } \sqrt{x} \times
  \lim_{x \to 0} \frac{\sin x}{x} 
  \\
  & = \lim_{x \to 0 } \sqrt{x} 
  \\
  & = 0.
\end{split}
\end{equation*}
    \item{}
        \vspace{4em}
    \qquad $ \displaystyle \lim_{x \to \infty} \displaystyle \frac{\sin
    x}{\sqrt{x}}= 0$. That is, by the sqeeze theorem, $ | \displaystyle \lim_{x \to \infty}\frac{\sin x}{\sqrt{x}}|= 0$, since
    %
    %
    \begin{equation*}
      | \displaystyle \frac{\sin x}{\sqrt{x}}| \le \frac{1}{\sqrt{x}} \to 0 \ \text{as} \ x
      \to \infty \ \text{and} \ | \displaystyle \frac{\sin x}{\sqrt{x}}| \ge 0.
    \end{equation*} 
         %
\qquad But $ | \displaystyle \lim_{x \to \infty}\frac{\sin x}{\sqrt{x}}|= 0$ implies 
$ \displaystyle \lim_{x \to \infty} \displaystyle \frac{\sin
    x}{\sqrt{x}}= 0$, and so we're done.
    %
    \item{}
        \vspace{4em}
        $ \qquad \displaystyle \lim_{x \to -\infty}(\sqrt{x^2 + x} - \sqrt{x^2
        - x}) =1$. That is, we compute
        %
        %
        \begin{equation*}
        \begin{split}
        \sqrt{x^2 + x} - \sqrt{x^2
        - x} & = \sqrt{x}\left( \sqrt{x+1} - \sqrt{x-1} \right)
        \\
        & = \sqrt{x}\left( \sqrt{x+1} - \sqrt{x-1} \right)\left( \frac{\sqrt{x+1} + \sqrt{x-1}}{\sqrt{x+1} + \sqrt{x-1}} \right)
        \\
        & = \sqrt{x} \left[ \frac{x+1 - (x-1)}{\sqrt{x+1} + \sqrt{x-1}} \right]
        \\
        & = \frac{2 \sqrt{x}}{\sqrt{x+1} + \sqrt{x-1}}
        \\
        & = \frac{2 \sqrt{x}}{\sqrt{x+1} + \sqrt{x-1}} \times
        \frac{1/\sqrt{x}}{1/\sqrt{x}}
        \\
        & = \frac{2}{\sqrt{1 + 1/x} + \sqrt{1 - 1/x}} 
        \\
        & \to \frac{2}{1 + 1} \ \text{as} \ x \to \infty
        \\
        & = 1.
        \end{split}
        \end{equation*}
        %
        %
        \item{}
            \vspace{8em}
        $\qquad \displaystyle \lim_{x \to 2} \frac{x^{2} - 4}{ \sqrt{x -2}}= 0 $, since
        %
        %
        \begin{equation*}
        \begin{split}
          \frac{x^{2} - 4}{\sqrt{x-2}} & = \frac{(x+2)(x-2)}{\sqrt{x-2}} 
          \\
          & = (x+2)\sqrt{x-2}
          \\
          & \to 0 \ \text{as} \ x \to 2.
        \end{split}
        \end{equation*}
        %
        %
        \item{}
            \vspace{8em}
            $\qquad \displaystyle \lim_{x \to -\infty} \frac{3x^{5} + 2x^{2} + 1}{\sqrt{81x^{10}  + 4x^{3} + 2}} = \frac{1}{3}$, since
            %
            %
            \begin{equation*}
            \begin{split}
            \frac{3x^{5} + 2x^{2} + 1}{\sqrt{81x^{10}  + 4x^{3} + 2}} 
            & = \frac{3x^{5} + 2x^{2} + 1}{\sqrt{81x^{10}  + 4x^{3} + 2}}\left( \frac{1/x^{5}}{1/x^{5}} \right) 
            \\
            & = \frac{3 + 2/x^{3} + 1/x^{5}}{\sqrt{81  + 4/x^{7} + 2/x^{10}}} 
            \\
            & \to \frac{3}{\sqrt{81}} \ \text{as} \ x \to -\infty
            \\
            & = \frac{1}{3}.
            \end{split}
            \end{equation*}
            %
            %
\end{enumerate}

\end{document}
