\documentclass[12 pt]{article}
\usepackage{amsmath,amsfonts,amsthm,amscd}
\newtheorem{thm}{Theorem}[section]
\newtheorem{pro}[thm]{Proposition}
\newtheorem{cor}[thm]{Corollary}
\newtheorem{lem}[thm]{Lemma}
\newtheorem{conj}[thm]{Conjecture}
\newtheorem{ques}[thm]{Question}
\newtheorem{statement}[thm]{Statement}
\newtheorem{claim}[thm]{Claim}
\newtheorem{ex}[thm]{Example}
\newtheorem{rem}[thm]{Remark}
\newtheorem{defn}[thm]{Definition}
\newtheorem{prob}[thm]{Problem}
\newtheorem{quot}[]{Result}
\newcommand{\dime}{\operatorname{dim}}
\newcommand{\Obj}{\operatorname{Obj}}
\newcommand{\Mor}{\operatorname{Mor}}
\newcommand{\chan}{\operatorname{char}}
\newcommand{\degr}{\operatorname{deg}}
\newcommand{\expo}{\operatorname{exp}}
\newcommand{\spant}{\operatorname{span}}
\newcommand{\res}{\operatorname{res}}
\newcommand{\Term}{\operatorname{Term}}
\newcommand{\Init}{\operatorname{Init}}
\newcommand{\spec}{\operatorname{spec}}
\newcommand{\Endo}{\operatorname{End}}
\newcommand{\Supp}{\operatorname{Supp}}
\newcommand{\Order}{\operatorname{ord}}
\newcommand{\kerl}{\operatorname{Ker}}
\newcommand{\Img}{\operatorname{Image}}
\newcommand{\sine}{\operatorname{sin}}

\begin{document}

\centerline{\bf MATH 280: Homework 12 (Written=20 points). }
\centerline{\bf Due during final exam: Monday, Dec 17}

\bigskip

\noindent
[6 points]1. \\ Find a formula for approximating $\int_{0}^3 f(x) dx$ 
via $Af(0) + Bf(1) + Cf(2)$. Your formula should be exact for all $f$ in $\Pi_2$, i.e., for all polynomials 
of degree $\leq 2$.

\medskip

\noindent
[6 points] 2. \\ In Theorem 1, on page 487 of the book, it is shown that the choice of nodes $x_0, \dots x_n$ that minimize 
the error term in general for the approximation 
$$
\int_{-1}^1 f(x)dx \sim \sum_{i=0}^n A_i f(t_i)
$$
are the Chebyshev nodes i.e. the roots $t_i=\frac{(i+1)\pi}{n+2}$, $0 \leq i \leq n$ of $U_{n+1}$, the $(n+1)$st Chebyshev polynomial.
These nodes can be obtained by uniformly spacing points on the upper unit circle and projecting these points to the $x$-axis.

Find the Chebyshev nodes $t_0, t_1, \dots, t_5$ when $6$ nodes are used in the interval $[-1,1]$ to two decimal places. Draw a reasonably accurate picture 
indicating their relative placement in the interval.
Also find the best nodes $x_0, x_1, \dots, x_5$ to use for the interval $[2,4]$ to minimize error in general in the approximation
$$
\int_2^4 f(x)dx \sim \sum_{i=0}^5 A_i f(x_i).
$$

\medskip

\noindent
[8 points] 3. \\ In this exercise you will derive the Gaussian Quadrature formula for the interval $[-1,1]$ when $4$ nodes are used.
Recall a Gaussian Quadrature formula is one of the form 
$$
\int_a^b f(x) dx \sim \sum_{i=0}^n A_i f(x_i)
$$
which is exact on $\Pi_{2n+1}$ the vector space of degree $\leq 2n+1$ polynomials. (In general such formulas are always exact on $\Pi_n$ if 
they come from interpolation at $n+1$ nodes, but Gaussian quadrature improves this by choosing the nodes correctly to make the formula exact on $\Pi_{2n+1}$.) \\
(a) Recall to find Gaussian Quadrature formulas we introduce the inner product 
$<f,g> = \int_{-1}^1 f(x) g(x) dx$ on the vector space of continuous real valued functions on $[-1,1]$. We seek to find a nonzero polynomial 
of degree $4$ that is orthogonal to every polynomial in $\Pi_3$ under this inner product. By Theorem 1 on Page 493 of the book, the roots of this polynomial will 
give us a Gaussian Quadrature formula with $4$ nodes $x_0, \dots, x_3$ that is exact on $\Pi_{2(3)+1}=\Pi_7$ i.e. polynomials of degree 7 or less. 
{\bf Use the Gram Schmidt process on the basis $\{1,x, x^2, x^3, x^4\}$ of $\Pi_4$ so as to find an orthonormal basis (under the inner product above) for 
$\Pi_4$. Use your answer to write down a nonzero polynomial $q(x)$ in $\Pi_4$ that is orthogonal to everything in $\Pi_3$.}

\noindent
(b) Find the roots of the polynomial $q(x)$ found in part (a). These are the nodes of the Gaussian quadrature formula we seek. 

\noindent
(c) The final formula is 
$$
\int_{-1}^{1} f(x) dx \sim A_0f(x_0) + A_1f(x_1) + A_2f(x_2) + A_3f(x_3)
$$
where $x_0, \dots, x_3$ are the nodes you found in (b). This formula will be exact on $\Pi_7$. Use this to set up a system of $4$ linear equations 
in the $4$ unknowns $A_0, \dots A_3$. You do not have to solve for them but in principle you could. 

\end{document}


