\documentclass[12 pt]{article}
\usepackage{amsmath,amsfonts,amsthm,amscd}
\newtheorem{thm}{Theorem}[section]
\newtheorem{pro}[thm]{Proposition}
\newtheorem{cor}[thm]{Corollary}
\newtheorem{lem}[thm]{Lemma}
\newtheorem{conj}[thm]{Conjecture}
\newtheorem{ques}[thm]{Question}
\newtheorem{statement}[thm]{Statement}
\newtheorem{claim}[thm]{Claim}
\newtheorem{ex}[thm]{Example}
\newtheorem{rem}[thm]{Remark}
\newtheorem{defn}[thm]{Definition}
\newtheorem{prob}[thm]{Problem}
\newtheorem{quot}[]{Result}
\newcommand{\dime}{\operatorname{dim}}
\newcommand{\Obj}{\operatorname{Obj}}
\newcommand{\Mor}{\operatorname{Mor}}
\newcommand{\chan}{\operatorname{char}}
\newcommand{\degr}{\operatorname{deg}}
\newcommand{\expo}{\operatorname{exp}}
\newcommand{\spant}{\operatorname{span}}
\newcommand{\res}{\operatorname{res}}
\newcommand{\Term}{\operatorname{Term}}
\newcommand{\Init}{\operatorname{Init}}
\newcommand{\spec}{\operatorname{spec}}
\newcommand{\Endo}{\operatorname{End}}
\newcommand{\Supp}{\operatorname{Supp}}
\newcommand{\Order}{\operatorname{ord}}
\newcommand{\kerl}{\operatorname{Ker}}
\newcommand{\Img}{\operatorname{Image}}
\newcommand{\sine}{\operatorname{sin}}

\begin{document}

\centerline{\bf MATH 280: Homework 6 (Written=8 points). }
\centerline{\bf Due in class, Monday, Oct 29}

\bigskip

\noindent
[2 points]1. \\ Thru a heuristic, it was argued that an $n \times n$ matrix has the hope of a unique $LU$-factorization if $L$ say is required 
to be unit lower triangular. However this heuristic does not work well often for singular matrices. You will study this issue in this exercise: \\
(a) Show that every matrix of the form $\mathbb{A}=\begin{bmatrix} 0 & a \\ 0 & b \end{bmatrix}$ has a $LU$-factorization. (Note it might be best to find the factorization directly by writing equations for the entries of $L$ and $U$). However also 
show that even if $L$ is required to be unit lower triangular, the factorization is not unique in general. \\
(b) Show that every matrix of the form $\mathbb{A}=\begin{bmatrix} 0 & 0 \\ a & b \end{bmatrix}$ has a $LU$-factorization. Does it have a $LU$ factorization 
where $L$ is unit lower triangular?

\vspace{0.2 in}

\noindent
[3 points]2. \\ Recall we showed in class that if $\mathbb{A}$ is a symmetric matrix and $\mathbb{A}=LU$ then $LU=\mathbb{A}=\mathbb{A}^T=U^TL^T$ which yields $UL^{-T}=D=L^{-1}U^T$
with $D$ a diagonal matrix. Then $\mathbb{A}=LU=LDL^T$. If furthermore $\mathbb{A}$ is positive definite, then $D$ has positive entries on the diagonal and we can 
find a decomposition $D=D^{\frac{1}{2}} D^{\frac{1}{2}}$. Setting $\mathbb{\hat{L}}=LD^{\frac{1}{2}}$ we then find the Cholesky decomposition 
$\mathbb{A}=\hat{\mathbb{L}}\hat{\mathbb{L}}^T$. \\
Use this to find the Cholesky decomposition of the following matrices: \\
(a)
$$
\mathbb{A}=\begin{bmatrix} 1 & 2 \\ 2 & 5 \end{bmatrix}.
$$

\noindent
(b)
$$
\mathbb{A}=\begin{bmatrix} 4 & \frac{1}{2} & 1 \\ \frac{1}{2} & \frac{17}{16} & \frac{1}{4} \\  1 & \frac{1}{4} & \frac{33}{64} \end{bmatrix}.
$$

\noindent
[3 points]3. \\ Use the pivoting method to choose pivots in each column in a way to minimize computational error magnification to find decompositions 
of the form $P\mathbb{A} = LU$ where $P$ is a permutation matrix, $L$ is unit lower triangular and $U$ is upper triangular for the following matrices 
$\mathbb{A}$. In each case compute the initial "scale" of each row carefully, show your work to decide on pivots and circle pivot choices during the computation, and state the final $L$, $U$ and permutation $P$ clearly. \\
(a) \\
$$
\mathbb{A} = \begin{bmatrix} -1 & 1 & -4 \\ 2 & 2 & 0 \\ 3 & 3 & 2 \end{bmatrix}
$$
\noindent
(b) \\
$$
\mathbb{A} = \begin{bmatrix} 1 & 6 & 0 \\ 2 & 1 & 0 \\ 0 & 2 & 1 \end{bmatrix}
$$

\end{document}


