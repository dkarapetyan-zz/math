\documentclass[12 pt]{article}
\usepackage{amsmath,amsfonts,amsthm,amscd}
\newtheorem{thm}{Theorem}[section]
\newtheorem{pro}[thm]{Proposition}
\newtheorem{cor}[thm]{Corollary}
\newtheorem{lem}[thm]{Lemma}
\newtheorem{conj}[thm]{Conjecture}
\newtheorem{ques}[thm]{Question}
\newtheorem{statement}[thm]{Statement}
\newtheorem{claim}[thm]{Claim}
\newtheorem{ex}[thm]{Example}
\newtheorem{rem}[thm]{Remark}
\newtheorem{defn}[thm]{Definition}
\newtheorem{prob}[thm]{Problem}
\newtheorem{quot}[]{Result}
\newcommand{\dime}{\operatorname{dim}}
\newcommand{\Obj}{\operatorname{Obj}}
\newcommand{\Mor}{\operatorname{Mor}}
\newcommand{\chan}{\operatorname{char}}
\newcommand{\degr}{\operatorname{deg}}
\newcommand{\expo}{\operatorname{exp}}
\newcommand{\spant}{\operatorname{span}}
\newcommand{\res}{\operatorname{res}}
\newcommand{\Term}{\operatorname{Term}}
\newcommand{\Init}{\operatorname{Init}}
\newcommand{\spec}{\operatorname{spec}}
\newcommand{\Endo}{\operatorname{End}}
\newcommand{\Supp}{\operatorname{Supp}}
\newcommand{\Order}{\operatorname{ord}}
\newcommand{\kerl}{\operatorname{Ker}}
\newcommand{\Img}{\operatorname{Image}}
\newcommand{\sine}{\operatorname{sin}}

\begin{document}

\centerline{\bf MATH 280: Homework 5 }

\bigskip

\noindent 1. \\
For each of the following functions, show they are a contraction mapping on the given interval and find the smallest positive value of 
$\lambda$ such that $|f(x)-f(y)| \leq \lambda |x-y|$ for all $x,y$ in the interval for each. It might be helpful to recall that the mean value theorem 
shows that $|f(x)-f(y)|=|f'(c)||x-y|$ for some $c$ between $x$ and $y$. 
\\
(a) $f(x)=\frac{1}{1+x^2}$ on the interval $[0,10]$. 
\\
(b) $g(x)=x^{3/2}$ on the interval $[0,0.4]$. \\
\medskip

\noindent What restriction on the magnitude of the derivative of $f$
in an interval $[a,b]$ guarantees that $f$ is a contraction mapping on that
interval?

\medskip

\noindent
2. \\
(a) Take any real number and repeatedly apply the cosine function to it, i.e. look at the sequence $x_0, \cos(x_0), \cos(\cos(x_0)), \dots$. 
What happens? State your answer up to 2 decimal digits. \\
(b) Now explain theoretically why for any real number $x_0$, $x_1=\cos(x_0)$ lies in the interval $[-1,1]$ and $x_2=\cos(x_1)$ lies in the interval
$[0,1]$. Show that $\cos$ maps $[0,1]$ back to itself and that it is a contraction on this interval. Explain why this proves the behavior you observed 
in (a).

\medskip

\noindent
3. \\
Fix a real number $p > 1$. Let us consider a recursive sequence where $x_1=1$ and $x_n=\frac{1}{p+x_{n-1}}$ for $n \geq 2$. \\
(a) Write out expressions for $x_2, x_3, x_4$. \\
(b) Notice if we can show the sequence converges then it would converge to something that we could interpret as the value of 
$$
\frac{1}{p+\frac{1}{p+\frac{1}{p+\frac{1}{p+\dots}}}}.
$$
Consider the function $f(x)=\frac{1}{p+x}$. Show that $f$ maps the closed interval $[0,1]$ back to itself and is a contraction on this interval.
Use this to explain why your sequence $x_1,x_2,\dots$ in (a) must converge to a number in $[0,1]$. \\
(c) If we call this limit number $\alpha$ show that $\alpha = \frac{1}{p+\alpha}$ and use this to find a formula for $\alpha$ in terms of $p$. 
(Notice this step is only technically valid once we have shown $\alpha$ exists as we did in (b)).

\medskip

\noindent
 4. \\
Use Horner's method to evaluate the polynomial $p(x)=x^4 + 3x^3 + 2x^2 -5x + 7$ at $x=8$ and to find 
$q(x)=\frac{p(x)-p(8)}{x-8}$. Please display the method as the book does when computing. 


\medskip

\noindent
5. \\
(a) Use the complete Horner method to find the Taylor expansion of $p(x)=x^4+3x^3+2x^2-5x+7$ about $c=3$. 
Please do this using this method and not Taylor's formula or some other method. \\
(b) Use your results in (a) to evaluate $p(3), p'(3), p''(3), p'''(3)$ and $p^{(4)}(3)$.

\medskip

\noindent
6. \\
(a) Use the localization of zeros theorem to find an open disk in the complex plane where all the zeros of the 
polynomial $p(z)=3z^5-7z^4-5z^3+z^2-8z+2$ are guaranteed to lie. \\
(b) Write down the reverse polynomial $q(z)=z^5p(\frac{1}{z})$ of $p(z)$. \\
(c) Use the localization of zeros theorem to find an open disk in the complex plane where all the zeros of the 
reverse polynomial $q(z)$ are guaranteed to lie. \\
(d) Notice if the reverse polynomial roots have $|z| < R$ then the original polynomial has roots with $|z| > \frac{1}{R}$ as 
they are reciprocals. 
Use this to find an annulus region where all the roots of the polynomial $p(z)$ must lie. 

\medskip

\medskip

\noindent
 7. \\
Code a function that uses Horner's complete method to output the Taylor
expansion of $p(x)=10x^7+12x^6 - 2x^5 +3x^4-x^3+2x^2 -x - 23$ around the
point $x=2$. 


\end{document}


