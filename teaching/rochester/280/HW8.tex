\documentclass[12 pt]{article}
\usepackage{amsmath,amsfonts,amsthm,amscd}
\newtheorem{thm}{Theorem}[section]
\newtheorem{pro}[thm]{Proposition}
\newtheorem{cor}[thm]{Corollary}
\newtheorem{lem}[thm]{Lemma}
\newtheorem{conj}[thm]{Conjecture}
\newtheorem{ques}[thm]{Question}
\newtheorem{statement}[thm]{Statement}
\newtheorem{claim}[thm]{Claim}
\newtheorem{ex}[thm]{Example}
\newtheorem{rem}[thm]{Remark}
\newtheorem{defn}[thm]{Definition}
\newtheorem{prob}[thm]{Problem}
\newtheorem{quot}[]{Result}
\newcommand{\dime}{\operatorname{dim}}
\newcommand{\Obj}{\operatorname{Obj}}
\newcommand{\Mor}{\operatorname{Mor}}
\newcommand{\chan}{\operatorname{char}}
\newcommand{\degr}{\operatorname{deg}}
\newcommand{\expo}{\operatorname{exp}}
\newcommand{\spant}{\operatorname{span}}
\newcommand{\res}{\operatorname{res}}
\newcommand{\Term}{\operatorname{Term}}
\newcommand{\Init}{\operatorname{Init}}
\newcommand{\spec}{\operatorname{spec}}
\newcommand{\Endo}{\operatorname{End}}
\newcommand{\Supp}{\operatorname{Supp}}
\newcommand{\Order}{\operatorname{ord}}
\newcommand{\kerl}{\operatorname{Ker}}
\newcommand{\Img}{\operatorname{Image}}
\newcommand{\sine}{\operatorname{sin}}

\begin{document}

\centerline{\bf MATH 280: Homework 8 (Written=20 points). }
\centerline{\bf Due in class, Monday, Nov 12}

\bigskip

\noindent
[2 points]1. \\ FREE POINTS!
\medskip

\noindent
[5 points]2. \\ Recall the power method to find the largest eigenvalue of a $n \times n$ matrix $\mathbb{A}$. It works as long as $\mathbb{A}$ is non-defective 
and $|\lambda_1| > |\lambda_2| \geq \dots \geq |\lambda_n|$. Here let $\mathbb{A} u_i = \lambda_i x_i$ where $\{ u_1, \dots, u_n \}$ is an eigenbasis for 
$\mathbb{C}^n$.
In this case for generic choice of $\hat{x}_0$ and generic choice of linear functional $\phi$, one computes a recursive sequence $\hat{x}_n = \mathbb{A} \hat{x}_{n-1}$ and $r_n = \frac{\phi(\hat{x}_{n})}{\phi(\hat{x}_{n-1})}$. In this case we showed $r_n \to \lambda_1$ as $n \to \infty$. \\

\noindent
(a) Recall for $\hat{x}_0$ to be generic, we required $\hat{x}_0=C_1 u_1 + \dots + C_n u_n$ with $C_1 \neq 0$. Show that for a $2 \times 2$ matrix $\mathbb{A}$ 
satisfying the conditions of this problem, any nonzero vector is either generic or an eigenvector of $\mathbb{A}$. Show that if a nongeneric $\hat{x}_0$ 
is chosen, that the sequence $r_n$ above will converge to $\lambda_2$ in this case, and in fact it will converge after one step.

\noindent
(b) In general, for a non-defective matrix $\mathbb{A}$ with $|\lambda_1| > |\lambda_2| > \dots > |\lambda_n|$, show that for {\bf any} nonzero $\hat{x}_0$, 
and generic $\phi$, the sequence $r_n$ will converge to some eigenvalue of $\mathbb{A}$. (Hint: Assume $C_1=\dots=C_{r-1}=0$ and $C_r \neq 0$ 
and mimic the proof of the original theorem).

\noindent
(c) Let $\mathbb{A}=\begin{bmatrix} 1 & 2 \\ 1 & 4 \end{bmatrix}$. Let $\hat{x}_0 = \begin{bmatrix} 1 \\ 1 \end{bmatrix}$ and choose functional 
$\phi=\pi_1$ the projection to the 1st coordinate. With these choices compute the first 3 vectors $\hat{x}_1, \hat{x}_2, \hat{x}_3$ in the recursion and 
the corresponding numbers $r_1, r_2, r_3$. On the other hand, find the characteristic polynomial and its roots using the quadratic formula. (In this case this can be done exactly). Compare the largest eigenvalue to $r_3$, how big is the absolute error?

\medskip

\noindent
[4 points]3. \\ (a) Find the Gershgorin disks for each row of the matrix $\mathbb{A} = \begin{bmatrix} 6 & 2 & 1 \\ 1 & -5 & 0 \\ 2 & 1 & 4 \end{bmatrix}$. 
Carefully shade the area given by these disks in the complex plane. Recall all the eigenvalues of $\mathbb{A}$ must lie in this shaded area. \\

\noindent
(b) The spectral radius $\rho(\mathbb{A})$ of a square matrix $\mathbb{A}$ is the modulus of the largest eigenvalue of $\mathbb{A}$. Equivalently 
it is the radius of the smallest closed disk centered at the origin which contains all the eigenvalues of $\mathbb{A}$. Find the Gershgorin 
disks of the matrix $\mathbb{A}=\begin{bmatrix} 0 & 2 & -1 \\ -2 & -10 & 0 \\ -1 & -1 & 4 \end{bmatrix}$, shade the area given by these disks in the complex plane 
and use your answer to give an upper bound for $\rho(\mathbb{A})$ without finding the eigenvalues directly. 

\medskip


\noindent
[4 points]4. \\ Let $\hat{x}_1 = \begin{bmatrix} 1 \\ 0 \\ 0 \\ 1 \end{bmatrix}$, $\hat{x}_2=\begin{bmatrix} 1 \\ 1 \\ 0 \\ 0 \end{bmatrix}$, $\hat{x}_3=\begin{bmatrix} 0 \\ 1 \\ 1 \\ 0 \end{bmatrix}$. \\

\noindent
(a) Perform the Gram-Schmidt orthogonalization process to find an orthonormal set $\hat{u}_1, \hat{u}_2, \hat{u}_3$. (Notice in this case, all the vectors are real so the Hermitian inner product just becomes the usual dot product of $\mathbb{R}^4$.) \\

\noindent
(b) Express each $\hat{x}_i$ in terms of $\hat{u}_1, \dots, \hat{u}_i$ for 
$i=1,2,3$. Use this to write $A=BT$ where $A$ is the matrix with columns $\hat{x}_i$, $B$ is the matrix with columns $\hat{u}_i$ and 
$T$ is upper triangular with positive diagonal entries.

\noindent
(c) Solve the least squares problem $Ax=b$ for $b=\begin{bmatrix} 1 \\ 1 \\ 2 \\ 1 \end{bmatrix}$. What is $||Ax-b||_2$ for this solution?

\medskip

\noindent
[5 points]5. \\ Let us suppose we have $N$ data points $(x_i,y_i)$ for $1 \leq i \leq N$ in the plane and wish to fit a straight line $y=mx+c$ thru these as best as we can. For any choice of $m$ and $c$ note that $mx_i+c$ is the $y$-value on the line when $x=x_i$. Thus $y_i-mx_i-c$ is the error between the $y$ value 
predicted by the line and the actual $y$ value. The least square line fit is the choice of line which minimizes the quantity 
$\sum_{i=1}^N (y_i-(mx_i+c))^2$ i.e., the sum of the errors squared. \\

\noindent
(a) If $\hat{y}$ is the column vector with $i$th entry $y_i$ and 
$\mathbb{A} =\begin{bmatrix} x_1 & 1 \\ x_2 & 1 \\ \dots \\ x_N & 1 \end{bmatrix}$, show that the least square line fit is given by the least square solution 
$\begin{bmatrix} m \\ c \end{bmatrix}$ to the equation $\mathbb{A} \begin{bmatrix} m \\ c \end{bmatrix} = \hat{y}$. \\

\noindent
(b) Use part (a) to conclude that the least square fit line $y=mx+c$ is determined (uniquely if $rank(\mathbb{A})=2$ i.e. some $x_i \neq x_j$ ) by the fact that $\begin{bmatrix} m \\ c \end{bmatrix}$ 
is the solution to the normal equations $\mathbb{A}^* \mathbb{A} \begin{bmatrix} m \\ c \end{bmatrix} =\mathbb{A}^* \hat{y}$. Simplify $\mathbb{A}^*\mathbb{A}$ and 
$\mathbb{A}^* \hat{y}$ until they are explicit functions of $N$, $x_i$ and $y_i$. \\

\noindent
(c) Use (b) to find the least squares line fit thru the points $(0,1), (1,2), (4,4)$.


\end{document}


