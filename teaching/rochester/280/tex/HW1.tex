\documentclass[12 pt]{article}
\usepackage{amsmath,amsfonts,amsthm,amscd}
\newtheorem{thm}{Theorem}[section]
\newtheorem{pro}[thm]{Proposition}
\newtheorem{cor}[thm]{Corollary}
\newtheorem{lem}[thm]{Lemma}
\newtheorem{conj}[thm]{Conjecture}
\newtheorem{ques}[thm]{Question}
\newtheorem{statement}[thm]{Statement}
\newtheorem{claim}[thm]{Claim}
\newtheorem{ex}[thm]{Example}
\newtheorem{rem}[thm]{Remark}
\newtheorem{defn}[thm]{Definition}
\newtheorem{prob}[thm]{Problem}
\newtheorem{quot}[]{Result}
\newcommand{\dime}{\operatorname{dim}}
\newcommand{\Obj}{\operatorname{Obj}}
\newcommand{\Mor}{\operatorname{Mor}}
\newcommand{\chan}{\operatorname{char}}
\newcommand{\degr}{\operatorname{deg}}
\newcommand{\expo}{\operatorname{exp}}
\newcommand{\spant}{\operatorname{span}}
\newcommand{\res}{\operatorname{res}}
\newcommand{\Term}{\operatorname{Term}}
\newcommand{\Init}{\operatorname{Init}}
\newcommand{\spec}{\operatorname{spec}}
\newcommand{\Endo}{\operatorname{End}}
\newcommand{\Supp}{\operatorname{Supp}}
\newcommand{\Order}{\operatorname{ord}}
\newcommand{\kerl}{\operatorname{Ker}}
\newcommand{\Img}{\operatorname{Image}}
\newcommand{\sine}{\operatorname{sin}}

\begin{document}

\centerline{\bf MATH 280: Homework I}

\bigskip

\noindent
1. Consider the function $f(x)$ defined on the real line by
$$
f(x) = \begin{cases}
x \sine(1/x) \ \text{if} \  x \neq 0 \\
0 \ \text{if} \  x = 0
\end{cases}
$$
(a) Show that $|f(x)| \leq |x|$ for all $x$ and use this to argue that 
$f(x)$ is continuous at $x=0$ i.e., $\displaystyle\lim_{x \to 0} f(x) = f(0)$. \\

\noindent
(b) Recall that the derivative $f'(a)$ is formally defined as 
$$
f'(a) = \lim_{x \to a} \frac{f(x)-f(a)}{x-a}
$$
if it exists and that $f$ is not differentiable at $a$ if the limit does not exist. 
Explain why for the function $f$ above, $f$ is not differentiable at $a=0$.
[Note that putting parts (a) and (b) together, you have shown that $f$ is continuous 
but not differentiable at $0$.]

\medskip

\noindent
2. 
(a) Order the functions $x$, $e^x$, $ln(x)$, $x^2$, $x^3$ from left to right such 
that $f$ is put to the left of $g$ if $f=O(g)$ as $x \to \infty$. \\

\noindent
(b) Order the functions $x$, $e^x$, $ln(x)$, $x^2$, $x^3$ from left to right 
such that $f$ is put to the left of $g$ if $f=O(g)$ as $x \to 0$. \\

\noindent
3. State whether the following assertions are true or false and give a brief explanation of your answer: \\
(a) $\frac{n+1}{n^2} = o(1/n)$ as $n \to \infty$. \\
(b) $\frac{n+1}{\sqrt{n}}=o(1)$ as $n \to \infty$. \\
(c) $\frac{1}{ln(n)} = O(1/n)$ as $n \to \infty$. \\
(d) $\frac{1}{n ln(n)} = o(1/n)$ as $n \to \infty$. \\
(e) $\frac{e^n}{n^5} = O(1/n)$ as $n \to \infty$. \\
(f) $e^x-1 = O(x^2)$ as $x \to 0$. \\
(g) $cos(x)-1 = O(x^2)$ as $x \to 0$. \\

\noindent
4. (Programming)
Write code that outputs "Hello World" to the user upon execution.


\end{document}


