\documentclass[12 pt]{article}
\usepackage{amsmath,amsfonts,amsthm,amscd}
\newtheorem{thm}{Theorem}[section]
\newtheorem{pro}[thm]{Proposition}
\newtheorem{cor}[thm]{Corollary}
\newtheorem{lem}[thm]{Lemma}
\newtheorem{conj}[thm]{Conjecture}
\newtheorem{ques}[thm]{Question}
\newtheorem{statement}[thm]{Statement}
\newtheorem{claim}[thm]{Claim}
\newtheorem{ex}[thm]{Example}
\newtheorem{rem}[thm]{Remark}
\newtheorem{defn}[thm]{Definition}
\newtheorem{prob}[thm]{Problem}
\newtheorem{quot}[]{Result}
\newcommand{\dime}{\operatorname{dim}}
\newcommand{\Obj}{\operatorname{Obj}}
\newcommand{\Mor}{\operatorname{Mor}}
\newcommand{\chan}{\operatorname{char}}
\newcommand{\degr}{\operatorname{deg}}
\newcommand{\expo}{\operatorname{exp}}
\newcommand{\spant}{\operatorname{span}}
\newcommand{\res}{\operatorname{res}}
\newcommand{\Term}{\operatorname{Term}}
\newcommand{\Init}{\operatorname{Init}}
\newcommand{\spec}{\operatorname{spec}}
\newcommand{\Endo}{\operatorname{End}}
\newcommand{\Supp}{\operatorname{Supp}}
\newcommand{\Order}{\operatorname{ord}}
\newcommand{\kerl}{\operatorname{Ker}}
\newcommand{\Img}{\operatorname{Image}}
\newcommand{\sine}{\operatorname{sin}}

\begin{document}

\centerline{\bf MATH 280: Homework 11 (Written=20 points). }
\centerline{\bf Due in class: Friday, Dec 7}

\bigskip

\noindent
[10 points]1. \\ Recall thru any set of data $(t_0, y_0), \dots, (t_n, y_n)$ with $t_0 < t_1 < \dots < t_n$, we have seen that there is a 
unique regular cubic spline $S$ that interpolates the data. 
Consider the data: \\
$$
\begin{matrix} t & | & 0 & 1 & 2 & 3 \\
y & | & 1 & 1 & 0 & 10 
\end{matrix}
$$
with $t_i=i, 0 \leq i \leq 3$. \\
(a) Compute $h_i=t_{i+1}-t_i$ for $i=0,1,2$. \\

\noindent
(b) Recall the notation $z_i=S''(t_i)$ so $z_0=z_3=0$ as we are looking for a {\bf regular} cubic spline. The other $z_i$ values were shown in the book and class 
to satisfy the linear system:

$$
\begin{bmatrix} 
2(h_0+h_1) & h_1 \\
h_1 & 2(h_1+h_2) 
\end{bmatrix}
\begin{bmatrix}
z_1 \\
z_2
\end{bmatrix}
=
\begin{bmatrix}
v_1 \\
v_2 
\end{bmatrix}
$$
where
$v_i=\frac{6}{h_i} (y_{i+1}-y_{i}) - \frac{6}{h_{i-1}}(y_i - y_{i-1})$.

Use this to compute $z_1, z_2$ for this data. 

\noindent
(c) $S=S_i$ on the interval $[t_i, t_{i+1}]$ where we have shown 
$$
S_i(x)=\frac{z_i}{6h_i}(t_{i+1}-x)^3 + \frac{z_{i+1}}{6h_i}(x-t_i)^3 + (\frac{y_{i+1}}{h_i}-\frac{z_{i+1}h_i}{6})(x-t_i) + (\frac{y_i}{h_i}-\frac{z_ih_i}{6})(t_{i+1}-x)
$$
Use this to find the cubics $S_0, S_1, S_2$ as functions of $x$ (all other numbers should be plugged in and simplified).

\bigskip

\noindent
[10 points]2. \\ Derive the following two formulas for approximating derivatives and show that they both have $O(h^4)$ error terms by explicitly finding 
formulas for these error terms also. You may assume $f$ has continuous sixth derivative - try to consolidate your error terms to involve only one 
derivative term evaluated at a single point.
$$
f'(x) \sim \frac{1}{12h}[-f(x+2h)+8f(x+h)-8f(x-h)+f(x-2h)] 
$$
$$
f''(x) \sim \frac{1}{12h^2}[-f(x+2h)+16f(x+h)-30f(x)+16f(x-h)-f(x-2h)]
$$


\end{document}


