\documentclass[12 pt]{article}
\usepackage{amsmath,amsfonts,amsthm,amscd}
\usepackage{enumerate}
\newtheorem{thm}{Theorem}[section]
\newtheorem{pro}[thm]{Proposition}
\newtheorem{cor}[thm]{Corollary}
\newtheorem{lem}[thm]{Lemma}
\newtheorem{conj}[thm]{Conjecture}
\newtheorem{ques}[thm]{Question}
\newtheorem{statement}[thm]{Statement}
\newtheorem{claim}[thm]{Claim}
\newtheorem{ex}[thm]{Example}
\newtheorem{rem}[thm]{Remark}
\newtheorem{defn}[thm]{Definition}
\newtheorem{prob}[thm]{Problem}
\newtheorem{quot}[]{Result}
\newcommand{\dime}{\operatorname{dim}}
\newcommand{\Obj}{\operatorname{Obj}}
\newcommand{\Mor}{\operatorname{Mor}}
\newcommand{\chan}{\operatorname{char}}
\newcommand{\degr}{\operatorname{deg}}
\newcommand{\expo}{\operatorname{exp}}
\newcommand{\spant}{\operatorname{span}}
\newcommand{\res}{\operatorname{res}}
\newcommand{\Term}{\operatorname{Term}}
\newcommand{\Init}{\operatorname{Init}}
\newcommand{\spec}{\operatorname{spec}}
\newcommand{\Endo}{\operatorname{End}}
\newcommand{\Supp}{\operatorname{Supp}}
\newcommand{\Order}{\operatorname{ord}}
\newcommand{\kerl}{\operatorname{Ker}}
\newcommand{\Img}{\operatorname{Image}}
\newcommand{\sine}{\operatorname{sin}}

\begin{document}

\centerline{\bf MATH 280: Homework 6} 

\bigskip

\noindent

\noindent
1. Recall that if $\mathbb{A}$ is a symmetric matrix and $\mathbb{A}=LU$ then
\\ $LU=\mathbb{A}=\mathbb{A}^T=U^TL^T$ which yields $UL^{-T}=D=L^{-1}U^T$ with
$D$ a diagonal matrix. Then $\mathbb{A}=LU=LDL^T$. If furthermore $\mathbb{A}$ is positive definite, then $D$ has positive entries on the diagonal and we can find a decomposition $D=D^{\frac{1}{2}} D^{\frac{1}{2}}$. Setting $\mathbb{\hat{L}}=LD^{\frac{1}{2}}$ we then find the Cholesky decomposition
$\mathbb{A}=\hat{\mathbb{L}}\hat{\mathbb{L}}^T$. \\
Use this to find the Cholesky decomposition of the following matrices:
\begin{enumerate}[a)]
  \item
$$
\mathbb{A}=\begin{bmatrix} 1 & 2 \\ 2 & 5 \end{bmatrix}
$$
\item
$$
\mathbb{A}=\begin{bmatrix} 4 & \frac{1}{2} & 1 \\ \frac{1}{2} & \frac{17}{16} & \frac{1}{4} \\  1 & \frac{1}{4} & \frac{33}{64} \end{bmatrix}
$$
\end{enumerate}
2. Use the pivoting method to choose pivots in each column in a way to minimize computational error magnification to find decompositions
of the form $P\mathbb{A} = LU$ where $P$ is a permutation matrix, $L$ is unit lower triangular and $U$ is upper triangular for the following matrices
$\mathbb{A}$. In each case compute the initial "scale" of each row carefully, show your work to decide on pivots and circle pivot choices during the computation, and state the final $L$, $U$ and permutation $P$ clearly.
\begin{enumerate}[a)]
  \item
$$
\mathbb{A} = \begin{bmatrix} -1 & 1 & -4 \\ 2 & 2 & 0 \\ 3 & 3 & 2 \end{bmatrix}
$$
\item
$$
\mathbb{A} = \begin{bmatrix} 1 & 6 & 0 \\ 2 & 1 & 0 \\ 0 & 2 & 1 \end{bmatrix}
$$
\end{enumerate}
3. Code a function that takes two matrices and either returns their
product or an error if their product is not
well-defined. If both matrices are $n \times n$, what is the computational
complexity of your function? If a better algorithm exists, what is the best
computational complexity we can hope for?
\end{document}