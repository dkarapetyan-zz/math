\documentclass[12 pt]{article}
\usepackage{amsmath,amsfonts,amsthm,amscd}
\newtheorem{thm}{Theorem}[section]
\newtheorem{pro}[thm]{Proposition}
\newtheorem{cor}[thm]{Corollary}
\newtheorem{lem}[thm]{Lemma}
\newtheorem{conj}[thm]{Conjecture}
\newtheorem{ques}[thm]{Question}
\newtheorem{statement}[thm]{Statement}
\newtheorem{claim}[thm]{Claim}
\newtheorem{ex}[thm]{Example}
\newtheorem{rem}[thm]{Remark}
\newtheorem{defn}[thm]{Definition}
\newtheorem{prob}[thm]{Problem}
\newtheorem{quot}[]{Result}
\newcommand{\dime}{\operatorname{dim}}
\newcommand{\Obj}{\operatorname{Obj}}
\newcommand{\Mor}{\operatorname{Mor}}
\newcommand{\chan}{\operatorname{char}}
\newcommand{\degr}{\operatorname{deg}}
\newcommand{\expo}{\operatorname{exp}}
\newcommand{\spant}{\operatorname{span}}
\newcommand{\res}{\operatorname{res}}
\newcommand{\Term}{\operatorname{Term}}
\newcommand{\Init}{\operatorname{Init}}
\newcommand{\spec}{\operatorname{spec}}
\newcommand{\Endo}{\operatorname{End}}
\newcommand{\Supp}{\operatorname{Supp}}
\newcommand{\Order}{\operatorname{ord}}
\newcommand{\kerl}{\operatorname{Ker}}
\newcommand{\Img}{\operatorname{Image}}
\newcommand{\sine}{\operatorname{sin}}

\begin{document}

\centerline{\bf MATH 280: Homework 10 (Written=20 points). }
\centerline{\bf Due in class: Friday, Nov 30}

\bigskip

\noindent
[4 points]1. \\ 
(a) Find the Lagrange cardinal functions $\ell_i(x), 0 \leq i \leq 3$ for the nodes $x_0=1, x_1=0, x_2=-1, x_3=4$. 
Simplify your answers. \\

\noindent
(b) Use your answer in (a) to find the unique cubic polynomial with $p(1)=2, p(0)=1, p(-1)=3, p(4)=0$. Use the Lagrange interpolation form. \\

\medskip

\noindent
[3 points]2. \\
Let $\ell_i(x)$ be the Lagrange cardinal functions for the distinct nodes $x_i, 0 \leq i \leq n$. Thus 
$\ell_i(x)$ is the polynomial of degree $\leq n$ such that $\ell_i(x_j) = \delta_{i,j}$, the Kronecker delta function, for $0 \leq i,j \leq n$.
Recall that $\ell_i(x) = \prod_{j=0, j \neq i}^n \frac{x-x_j}{x_i-x_j}$. Show that 
$\sum_{i=0}^n \ell_i(x) = 1$ for {\bf all} $x$. (Hint: First show this for $x=x_0, \dots x_n$ and use degree arguments to finish your proof.)

\medskip

\noindent
[3 points]3. \\
Find the maximum error made by interpolating the hyperbolic cosine function $$f(x)=cosh(x)=\frac{e^x+e^{-x}}{2}$$ at 21 nodes by a polynomial $p(x)$ of degree 20, on the interval $[-1,1]$. 

\medskip

\noindent
[4 points]4. \\
Recall given a function $f$ and distinct nodes $x_0, x_1, \dots, x_n$, the unique degree $\leq n$ polynomial $p(x)$ with $p(x_i)=f(x_i), 0 \leq i \leq n$ 
is given by 
$$
\sum_{k=0}^n f[x_0,\dots,x_k] \prod_{i < k} (x-x_i)
$$
where $f[x_0,\dots,x_k]$ is a divided difference. Use the recursive algorithm discussed in class and the book to find the divided differences and 
the interpolating polynomial for the following data points: \\
(a) \\
$\begin{matrix} x & | & 1 & 2 & 4 
\\  y & | & 3 & 4 & 0
\end{matrix}$

\noindent
(b) \\
$\begin{matrix} x & | & 1.5 & 2.7 & 3.1 & -2.1 \\
y & | & 1.0 & 4.6 & 2.1 & 0.0 
\end{matrix}
$

\medskip

\noindent
[3 points]5. If $x_0, \dots, x_n$ are distinct nodes and $f, g$ two functions, use the recursion for divided differences and mathematical induction to 
prove the {\bf Leibniz} formula:
$$
(fg)[x_0, \dots, x_n] = \sum_{k=0}^n f[x_0,\dots,x_k]g[x_k, x_{k+1}, \dots, x_n].
$$

\medskip

\noindent
[3 points]6. Find the unique polynomial $p(x)$ of degree $\leq 5$ which interpolates $f(x)=cos(x)$ at $0,\frac{\pi}{2},\frac{\pi}{2}, \pi, \pi, \pi$ i.e., find 
$p(x)$ such that $p(0)=cos(0), p(\frac{\pi}{2})=cos(\frac{\pi}{2}), p'(\frac{\pi}{2})=cos'(\frac{\pi}{2}), p(\pi)=cos(\pi), p'(\pi)=cos'(\pi), p''(\pi)=cos''(\pi)$. 
Do this by using the Hermite algorithm to find the divided differences \\ $f[0], f[0, \frac{\pi}{2}], f[0, \frac{\pi}{2}, \frac{\pi}{2}]$ etc. and using the interpolation formula 
$$
p(x) = \sum_{k=0}^n f[x_0,\dots, x_k] \prod_{i < k} (x-x_i).
$$
Recall this algorithm is based on a refinement of the algorithm you used in problem 4 where $x$ values are repeated. In this case the recursion becomes
for $x_0 \leq x_1 \leq \dots \leq x_n$:

$$
f[x_0, \dots, x_n] = \begin{cases} \frac{f[x_1,\dots,x_n] - f[x_0, \dots, x_{n-1}]}{x_n-x_0} \text{ if } x_n \neq x_0 \\
\frac{f^{(n)}(x_0)}{n!} \text{ if } x_0=x_n
\end{cases}
$$

\end{document}


