\documentclass[12 pt]{article}
\usepackage{amsmath,amsfonts,amsthm,amscd}
\newtheorem{thm}{Theorem}[section]
\newtheorem{pro}[thm]{Proposition}
\newtheorem{cor}[thm]{Corollary}
\newtheorem{lem}[thm]{Lemma}
\newtheorem{conj}[thm]{Conjecture}
\newtheorem{ques}[thm]{Question}
\newtheorem{statement}[thm]{Statement}
\newtheorem{claim}[thm]{Claim}
\newtheorem{ex}[thm]{Example}
\newtheorem{rem}[thm]{Remark}
\newtheorem{defn}[thm]{Definition}
\newtheorem{prob}[thm]{Problem}
\newtheorem{quot}[]{Result}
\newcommand{\dime}{\operatorname{dim}}
\newcommand{\Obj}{\operatorname{Obj}}
\newcommand{\Mor}{\operatorname{Mor}}
\newcommand{\chan}{\operatorname{char}}
\newcommand{\degr}{\operatorname{deg}}
\newcommand{\expo}{\operatorname{exp}}
\newcommand{\spant}{\operatorname{span}}
\newcommand{\res}{\operatorname{res}}
\newcommand{\Term}{\operatorname{Term}}
\newcommand{\Init}{\operatorname{Init}}
\newcommand{\spec}{\operatorname{spec}}
\newcommand{\Endo}{\operatorname{End}}
\newcommand{\Supp}{\operatorname{Supp}}
\newcommand{\Order}{\operatorname{ord}}
\newcommand{\kerl}{\operatorname{Ker}}
\newcommand{\Img}{\operatorname{Image}}
\newcommand{\sine}{\operatorname{sin}}

\begin{document}

\centerline{\bf MATH 280: Homework 2 }

\bigskip

\noindent
1. Let $V$ denote the vector space of all complex sequences as used in the book. Let $E: V \to V$ denote 
the shift operator i.e., $E(x_1,x_2, x_3, \dots) = (x_2, x_3, \dots)$ for all sequences $\vec{x} = (x_1,x_2,x_3,\dots) \in V$.
$E$ is easily seen to be a linear map.

\noindent
(a) Let $\vec{x}$ be an eigenvector of $E$ with eigenvalue $\lambda$, then $E(\vec{x}) = \lambda \vec{x}$. 
If $x_1=C$, find a formula for $x_n$ in terms of $C, n$ and $\lambda$.

\noindent
(b) What sort of sequences are eigenvectors of $E$? Describe all the possible eigenvalues of $E$.

\medskip

\noindent
2. 
(a) Consider the following lhcc (linear, homogeneous, constant coefficient) recurrence:
$$
(4E^0 - 3E^2 + E^3)\vec{x} = \vec{0}
$$
Find an equivalent formula for $x_{n+3}$ in terms $x_n, x_{n+1}$ and $x_{n+2}$. 
What is the characteristic polynomial of this recurrence? Is the recurrence stable?

\noindent
(b) Find the roots of the characteristic polynomial and use them to write down a general 
solution to the recurrence. (This should be in terms of $n$ and $3$ unknown arbitrary constants that would be determined by initial conditions).

\medskip

\noindent
3. (a) If $p$ is a polynomial with real coefficients and $p(E) \vec{x} = \vec{0}$ is a linear recurrence, explain why if $\vec{x} = \vec{z}$ 
is a solution using complex numbers, then $\vec{x} = \bar{\vec{z}}$, is also a solution where $\bar{\vec{z}}$ is the sequence obtained 
by taking the complex conjugate of each entry of $\vec{z}$. 

\medskip

\noindent
(b) Recall if $z=a+ib$ is a complex number, then $\bar{z}=a-ib$ and $a=Re(z) = \frac{z+\bar{z}}{2}, b=Im(z)=\frac{z-\bar{z}}{2i}$ are the real 
and imaginary part respectively of $z$. Explain why if $\vec{z}$ is a sequence of complex numbers solving $p(E) \vec{x}=\vec{0}$, 
then $Re(\vec{z})$, the sequence of real parts of the coordinates of $\vec{z}$ and $Im(\vec{z})$ the sequence of imaginary parts of 
$\vec{z}$ also solve $p(E)\vec{x}=\vec{0}$.

\medskip

\noindent
(c) If $z=a+ib=re^{i \theta}$, calculate the real and imaginary parts of $z^n$ in terms of $r, \theta$ and $n$. 
If $x_n = Az^n + B(\bar{z})^n$, show that $x_n = Cr^n\cos(n \theta) + Dr ^n\sin(n \theta)$ for constants $C, D$ depending only 
on $A$ and $B$. Give explicit formulas expressing $C$ and $D$ in terms of $A$ and $B$. (Notice that the formulas $r^n \cos(n \theta)$ 
and $r^n \sin(n \theta)$ are real valued quantities.)

\medskip

\noindent
4.
(a) Consider the following lhcc recurrence:
$$
(E^2 + E^0)\vec{x} = \vec{0}
$$
Find an equivalent formula for $x_{n+2}$ in terms of $x_n$ and $x_{n+1}$. What is the characteristic polynomial of this recurrence?

\noindent
(b) The characteristic polynomial has two distinct complex conjugate roots. Find them and write down the general solution in usual form. 
Then use problem 3 to write the general solution as a linear combination of two real formulas.

\noindent
5. (Programming)
Implement Horner's algorithm (see section 1.2) for the following polynomial 
$$
p(x) = 3x^5 + 2x^4 - 3x^3 + x^2 + 7x + 1.
$$
What is $p(5)$? How many multiplications and additions does it take to compute this using Horner's algorithm? How many multiplications and additions would it take if we computed each term ($3x^5$, $2x^4$, etc.) individually, and then summed? 

\end{document}


