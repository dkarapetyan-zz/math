\documentclass[12 pt]{article}
\usepackage{amsmath,amsfonts,amsthm,amscd}
\newtheorem{thm}{Theorem}[section]
\newtheorem{pro}[thm]{Proposition}
\newtheorem{cor}[thm]{Corollary}
\newtheorem{lem}[thm]{Lemma}
\newtheorem{conj}[thm]{Conjecture}
\newtheorem{ques}[thm]{Question}
\newtheorem{statement}[thm]{Statement}
\newtheorem{claim}[thm]{Claim}
\newtheorem{ex}[thm]{Example}
\newtheorem{rem}[thm]{Remark}
\newtheorem{defn}[thm]{Definition}
\newtheorem{prob}[thm]{Problem}
\newtheorem{quot}[]{Result}
\newcommand{\dime}{\operatorname{dim}}
\newcommand{\Obj}{\operatorname{Obj}}
\newcommand{\Mor}{\operatorname{Mor}}
\newcommand{\chan}{\operatorname{char}}
\newcommand{\degr}{\operatorname{deg}}
\newcommand{\expo}{\operatorname{exp}}
\newcommand{\spant}{\operatorname{span}}
\newcommand{\res}{\operatorname{res}}
\newcommand{\Term}{\operatorname{Term}}
\newcommand{\Init}{\operatorname{Init}}
\newcommand{\spec}{\operatorname{spec}}
\newcommand{\Endo}{\operatorname{End}}
\newcommand{\Supp}{\operatorname{Supp}}
\newcommand{\Order}{\operatorname{ord}}
\newcommand{\kerl}{\operatorname{Ker}}
\newcommand{\Img}{\operatorname{Image}}
\newcommand{\sine}{\operatorname{sin}}

\begin{document}

\centerline{\bf MATH 280: Homework 4 (Written=20 points). }
\centerline{\bf Due in class, Friday, Oct 12}

\bigskip

\noindent
[3 points]1. \\
(a) Recall that in the bisection method with initial interval $[a_0,b_0]$, one has 
that the $n$th midpoint $c_n$ and root $r$ of $f(x)=0$ satisfy 
$$
|c_n - r | \leq \frac{1}{2} (b_n-a_n) = \frac{1}{2} \frac{1}{2^n} (b_0-a_0) = \frac{1}{2^{n+1}} (b_0-a_0).
$$
Let $\epsilon > 0$. If $N$ is the number of steps that must be taken in the bisection method to guarantee that $|r-c_N| \leq \epsilon$, 
show that
$$
N \geq \frac{ln(b_0-a_0)-ln(\epsilon)}{ln(2)} - 1.
$$
(b) Assuming $a_0 > 0$, find a similar inequality as in part (a) for the number $N$ of steps that must be taken in the bisection method 
to ensure the root is found with relative accuracy $\leq \epsilon$, i.e., $|\frac{r-c_N}{r}| \leq \epsilon$. In your final expression, you should 
replace any occurrence of $r$ with either $a_0$ or $b_0$ (whichever works correctly) as apriori $r$ is not known.

\medskip

\noindent
[4 points]2. Consider $f(x)=x^2-4xsin(x) + sin^2(x)$. Note $r=0$ is a root of $f(x)=0$. 
We wish to find a {\bf positive} root $r$ of $f(x)=0$ using the bisection method. In this problem 
you are free to use a calculator/computer to compute the evaluations required but otherwise will do the bisection method "by hand". \\
(a) Find suitable values $a_0, b_0 > 0$ to choose as a starting interval $[a_0,b_0]$ for the bisection method.  \\
(b) Use the bisection method to find a {\bf positive} root accurate to two decimal digits. In each step state the value of the evaluation at the midpoint $f(c_n)$ 
and which half interval you keep clearly. Also state clearly the root's value up to two decimal digits.




\medskip

\noindent
[4 points]3. Let $q > 0$ be a fixed positive real number and $f(x)=x^2-q$. Notice the positive root of $f(x)$ is $\sqrt{q}$. \\
(a) Find the Newton-Raphson iteration function $N_f(x) = x - \frac{f(x)}{f'(x)}$ for this example. \\
(b) For what real $x$ is this Newton-Raphson function not defined? Show that if $x > 0$ then $N_f(x) > 0$ and $x<0$ then $N_f(x)<0$. 
Based on your results, which "seeds" $x_0$ lead to values where $N_f(x)$ is not defined when $N_f$ is applied iteratively? \\
(c) Let $e_n=x_n-r$ be the error on the $n$th iteration of the Newton-Raphson method. In class and in the book it is shown that 
$e_{n+1} \leq C e_n^2$ as long as $x_n$ is close enough to a root $r$. The quantity $C$ is given by 
$$
\frac{1}{2} \frac{ max | f''(x) | }{ min |f'(y)|}
$$
where the max and min are over an interval $[r-\delta, r+\delta]$. Any point in this interval has to converge to the root as long 
as $C \delta < 1$ i.e. $\delta < \frac{1}{C}$.  Compute the quantity $\frac{1}{2} \frac{ max |f''(x)|}{min |f'(y)| }$ for the function in this question. 
Then find suitable $\delta$ and corresponding $C$ such that convergence of the method is guaranteed on the interval 
$[\sqrt{q} - \delta, \sqrt{q} + \delta]$. (Hint: Show that $\delta=\frac{\sqrt{q}}{2}$ works. ) \\
(d) Explain why for seeds in the interval $[\frac{\sqrt{q}}{2}, \frac{3\sqrt{q}}{2}]$, Newton-Raphson iteration converges to the root $\sqrt{q}$ 
and furthermore if the $n$th iterate $x_n$ is correct to $k$ binary digits, then $x_{n+1}$ is correct to at least $2k+1$ binary digits as long as 
$q \geq 1$.



\medskip

\noindent
[2 points]4. If Newton's method is used to compute a root of $f(x)=x^3-2$ starting with $x_0=1$, what is $x_2$? How close is $(x_2)^3$ to $2$?

\medskip

\noindent
[2 points]5. If the secant method is applied to the function $f(x)=x^2-2$ with $x_0=0$ and $x_1=1$, what is $x_2$?

\medskip

\noindent
[2 points]6. Show that for $f(x)=x$, the secant method will give $x_3=x_4=\dots = 0$ for any choice of $x_0$ and $x_1$. 

\noindent
[3 points]7. (a) Show that for $f(x)=x^2$, the secant method will yield the recursion
$$
x_{n+1} = \frac{x_n x_{n-1}}{x_n + x_{n-1}}.
$$
Explain why if $x_0, x_1 > 0$ then $x_n > 0$ for all $n$. \\
(b) Show that if $x_0, x_1 > 0$ then $x_{n+1} < x_n$ for all $n \geq 1$. Thus $x_1, x_2, x_3, \dots$ is a decreasing sequence, bounded below by zero 
and hence the monotone convergence theorem says it has a limit $\alpha$. Explain why $\alpha$ must satisfy 
$$
\alpha = \frac{\alpha^2}{2\alpha}
$$
if it were nonzero and use this to deduce $\alpha$. What do your results say about the convergence of the secant method to a root in this instance?
 
\end{document}


