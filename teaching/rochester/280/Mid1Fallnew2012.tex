
% Essential Formatting
   
\documentclass[12pt]{article}
\usepackage{epsfig,amsmath,amsthm,amssymb,amscd,latexsym}
\usepackage[multipart,letter]{urmathexam}

% Basic User Def's

\def\ds{\displaystyle}
\def\t{\theta}
% Beginning of the Document

\begin{document}
\examtitle{MATH 280}{MIDTERM EXAM}{Oct 25, 2012}
\studentinfo
\instructions{
  \begin{itemize}
  \item
    {\bf No calculators are allowed on this exam.}
  \item
  {\bf $8" \times 11"$ (both sides) note sheet is allowed on this exam.}
  \item
    {\bf Please show all your work.
         You may use back pages if necessary.
         You may not receive full credit for
         a correct answer if there is no work shown.}
    {\bf Attempt all questions.}
    \end{itemize}
}
\finishfirstpage

% Problems Start Here % ----------------------------------------------------- %
\newpage


\problem{12} 
\noindent 
(a) Order the functions $x, , \sqrt{x}, e^x, ln(x), x^2, x^3$ from left to right such that $f$ is to the left of $g$ if $f=O(g)$ as $x \to 0$. \\

\vspace{1 in}

\noindent
(b) Order the functions $x, \sqrt{x}, e^x, ln(x), x^2, x^3$ from left to right such that $f$ is to the left of $g$ if $f=O(g)$ as $x \to \infty$. \\

\vspace{1 in}

\noindent
(c) State if the following assertions are true or false and give a brief explanation of your answer: \\
(i) $\frac{1}{n ln(n)} = O(\frac{1}{n})$ as $n \to \infty$. \\

\vspace{1in}

\noindent
(ii) $\frac{1}{n ln(n)}=o(\frac{1}{n})$ as $n \to \infty$. \\

\vspace{1in}

\noindent
(iii) $\sin(x) = 1 - \frac{1}{2}x^2 + O(x^3)$ as $x \to \infty$. \\

\vspace{1in}

\noindent
(iv) $\frac{1}{1+x^2} = 1- x^2+  O(x^3)$ as $x \to 0$. \\


 
    
\newpage
\problem{12}
Consider the recurrence 
$$
x_{n+2}=x_{n+1}+2x_n
$$
for $n \geq 0$. \\

\noindent
(a) If $E$ is the shift operator on the vector space of complex sequences, and $\hat{x}=(x_0,x_1,x_2,\dots)$, rewrite the recurrence in the form 
$p(E)\hat{x}=\hat{0}$. What is the characteristic polynomial $p(x)$? 

\vspace{2 in}

\noindent
(b) Find the roots of the characteristic polynomial and use them to describe the general solution to this recurrence. 

\vspace{1.5in}

\noindent
(c) If $x_0=0$ and $x_1=1$, find $x_2, x_3, x_4, x_5$ directly from the recurrence. \\

\vspace{1in}

\noindent
(d) Find the particular solution to this recurrence when $x_0=0$ and $x_1=1$. In particular express $x_n$ as a function of $n$.

\newpage

\problem{12}
(a) Find the binary expansion of the number $272$. \\

\vspace{3 in}

\noindent
(b) Find the binary expansion of the number $\frac{1}{5}$. \\

\vspace{3 in}

\noindent
(c) Find the hexadecimal representation of the binary number $110.110010110010$.

\newpage

\problem{6}
\noindent
Recall the typical $32$ bit computer representation of a floating point number as discussed in the book, reserves $1$ bit (1st bit) for the sign of the number, 
$8$ bits (bits 2 thru 9) for the biased exponent, and $23$ bits (bits 10 thru 32) for the mantissa. Express the representation 
$11011000010000000000000000000000$ in the form $\pm q 2^m$. Give the value of this number in decimal.

\vspace{2.5 in}

\problem{6}
\noindent
Let $f(x)=x^5$ and $x_0=10$ with absolute error at most $\delta_x=0.1$. Compute $f'(x_0)$ and the condition number and use them to estimate 
the absolute and relative error in $y_0=f(x_0)$. 

\newpage
\problem{6}
In the bisection method, if $c_n$ is accurate to $20$ binary digits, then $c_{n+1}$ is accurate to at least $\underline{~~~~~~~~~~~~}$ binary digits. \\
In the Newton-Raphson method, if $x_n$ is accurate to $15$ decimal digits, then a reasonable estimate for the accuracy of $x_{n+1}$ is that it is 
accurate to $\underline{~~~~~~~~~~~~~~~}$ decimal digits. \\ 

\problem{6}
(a) What is the Newton-Raphson function $N_f(x)$ for $f(x)=x^3-7$. Simplify $N_f(x)$ until it is of the form 
$ax+\frac{b}{x^2}$ for suitable constants $a$ and $b$. For which $x$ is it undefined? Explain why if we start with a positive seed $x_0$, we never 
encounter this problem.

\vspace{2 in}

\noindent
(b) If $x_0=1$ compute $x_1, x_2$.

\newpage
\problem{10}
(a) Write down the recursion one obtains when one applies the secant method to the function $f(x)=x^3-10$. 

\vspace{2 in}

\noindent
(b) Prove that in the secant method, to obtain $x_{n+2}$ one can draw a secant line between the points $(x_n,f(x_n))$ and $(x_{n+1},f(x_{n+1}))$ on the 
graph of $f$ and see where it intersects the $x$-axis.

\newpage
\problem{12}
Use Horner's method to do the following tasks. Please do not use a different method. \\
(a) Evaluate the polynomial $p(x)=x^4+3x^3+2x^2+x+1$ at $x=2$ and find $q(x)=\frac{p(x)-p(2)}{x-2}$. 

\vspace{ 2 in}

\noindent
(b) Verify that $-1$ is a root of the polynomial $p(x)=x^4+3x^3+2x^2+x+1$ and deflate the polynomial at that root. \\

\vspace{2 in}

\noindent
(c) Find $p'(2),p''(2), p'''(2)$ for the same polynomial in examples (a) and (b).

\newpage
\problem{10} Find an annulus in the complex plane such that all the zeros of the polynomial 
$p(z)=4z^6 + 8z^5 + 4z^4 + 3z^3+2z^2+z+13$ must lie in it. Use the localization theorem and explain all your work.


\newpage
\problem{8} 
(a) Explain why $f(x)=\cos(x)$ maps $[0,1] \to [0,1]$ and is a contraction on this interval. 

\vspace{3 in}

\noindent
(b) Explain why for any starting seed $x_0 \in \mathbb{R}$ we have the recursive sequence given by $x_{n+1}=\cos(x_n)$ converges to the 
same limit (independent of $x_0$).



% Problems End Here % ------------------------------------------------------- %

\problemsdone
\end{document}

% End of the Document
