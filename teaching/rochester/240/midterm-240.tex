\documentclass[12pt, oneside]{amsart}
\usepackage{graphicx}
\usepackage[nohead, margin=0.5in]{geometry}
\usepackage{enumerate}
%\geometry{left=0.5in,right=0.5in,top=0.3in,bottom=0.5in} 

\pagestyle{empty}
%Next are definitions of "\one" etc. 
%This is an easy way of assigning points to your questions and to 
%your table all at once. 
\newcommand{\one}{20}
\newcommand{\two}{20}
\newcommand{\three}{20}
\newcommand{\four}{20}
\newcommand{\five}{20}
\newcommand{\rr}{\mathbb{R}}
\synctex=1

\begin{document}


%feel free to change the title page
%
 
\begin{center}
\hrulefill\\
{\bf \textsf{\raisebox{-0.10cm}{Spring 2013: MATH 240} \hspace{\fill} 
\raisebox{-0.10cm}{Topology} \hspace{\fill}
\raisebox{-0.10cm}{David Karapetyan}}}\\
\hrulefill\\
{\large \rule{0cm}{1.2cm} \textsf{Wednesday 03/06/2012} \hfill
\textsf{Midterm} \hfill  \textsf{50 minutes}}\\
{\large\rule{0cm}{1.2cm}\textsf{Name: \framebox[2.9in]{\rule{0cm}{0.8cm}} 
\hspace{\fill}
Student ID: \framebox[2.1in]{\rule{0cm}{0.8cm}}}}\\
\end{center}
\vspace{0.8cm}

\noindent
{\bf \textsf{Instructions.}}

\begin{enumerate}
\item Attempt all questions.   
\item Show all the steps of your work clearly.  
\item Good luck 
%The method (reasoning) used to 
%obtain an answer is worth more than the answer itself.   
\end{enumerate}

\vfill

%The \rule commands create vertical space, which makes things sit nicely in 
%vertical way in boxes of table below.

\begin{center}
{\large
\begin{tabular}{|c|c|c|}
\hline
\rule[-0.3cm]{0cm}{1cm}
\textsf{Question} & \textsf{Points} &  \textsf{Your Score} \\
\hline
\hline
\rule[-0.3cm]{0cm}{1cm}
\textsf{ Q1} & \one &\\
\hline
\rule[-0.3cm]{0cm}{1cm}
\textsf{ Q2} & \two &\\
\hline
\rule[-0.3cm]{0cm}{1cm}
\textsf{ Q3} & \three &\\
\hline
\rule[-0.3cm]{0cm}{1cm}
\textsf{ Q4} & \four &\\
\hline
\rule[-0.3cm]{0cm}{1cm}

 \textsf{ TOTAL} & 80 & \\
 \hline
 \end{tabular}
} 

\end{center}

\vfill


\newpage
\noindent
 \textbf{Q1}. \\ \\ 
\begin{enumerate}[a)]
  \item
    What is the definition of a topology on a set?
    \vspace{3cm}
  \item
    What is the definition of a basis for a topology?
    \vspace{3cm}
  \item 
    What is the definition of a subbasis for a topology?
        \vspace{3cm}

  \item
    Choose any set you wish, and equip it with any topology you like. Prove that it is indeed a topology. Construct a basis and subbasis for your topology, and prove that they are indeed a basis and subbasis, respectively. (Note: if you tackle the components of this question in the correct order, you will reduce the amount of work considerably). 
\end{enumerate}

\newpage
\noindent
\textbf{Q2}.\\ \\ Consider the sequence
 \left \{(1/2)^{n} \right\}_{n=1}^{\infty} \subset \rr.

 \vspace{1cm}
\begin{enumerate}[a)]
  \item
     Show that the above sequence has exactly one limit point in the standard order topology on $\rr$. 
     \vspace{8cm}
   \item
     Construct a topology on $\rr$ such that the above sequence has more than one limit point, and show why.
     \vspace{8cm}
   \item
     Construct a topology on $\rr$ such that the above sequence has no limit points, and show why.
   \end{enumerate}
\newpage
\noindent
\textbf{Q3}.
\vspace{1cm}
\begin{enumerate}[a)]
  \item
    What is the definition of an ordered set?    \vspace{6cm}
  \item
    Is there a way to construct an order topology for $\rr \times \rr$? If so, provide the construction and prove that it indeed is an order topology. If not, give a counterexample. 
    \newpage
\item
  Let $X$ be equipped with an order topology, and let $Y$ be a subset of $X$. Prove that the order relation on $X,$ when restricted to $Y,$ makes $Y$ into an ordered topological space. 
  \vspace{12cm}

\item 
  Are the subspace topology on $Y$ and the restricted order topology on $Y$ in general the same? If so, provide an explanation. If not, provide a counterexample (with proof that it is a counterexample), and the condition or conditions on $Y$ necessary to have the subspace topology on $Y$ and the restricted order topology on $Y$ coincide.  
  \newpage
   \end{enumerate}
\newpage
\noindent

\newpage
\noindent
\textbf{Q4}. \\ \\ \begin{enumerate}[a)]
  \item
Prove that the set of rational numbers is countable. 
\vspace{10cm}
\item
Let $X$ denote the two element set $\{0,1 \}$, and let $X^{\omega}$ denote the cartesian product of countably many copies of $X$. Prove that $X^{\omega}$ is uncountable.
  \newpage
\end{enumerate}







\end{document}
