\documentclass[12pt, oneside]{amsart}
\usepackage{graphicx}
\usepackage[nohead, margin=0.5in]{geometry}
\usepackage{enumerate}
%\geometry{left=0.5in,right=0.5in,top=0.3in,bottom=0.5in} 

\pagestyle{empty}
%Next are definitions of "\one" etc. 
%This is an easy way of assigning points to your questions and to 
%your table all at once. 
\newcommand{\one}{20}
\newcommand{\two}{20}
\newcommand{\three}{20}
\newcommand{\four}{20}
\newcommand{\five}{20}
\newcommand{\rr}{\mathbb{R}}
\synctex=1

\begin{document}


%feel free to change the title page
%
 
\begin{center}
\hrulefill\\
{\bf \textsf{\raisebox{-0.10cm}{Spring 2013: MATH 240} \hspace{\fill} 
\raisebox{-0.10cm}{Topology} \hspace{\fill}
\raisebox{-0.10cm}{David Karapetyan}}}\\
\hrulefill\\
{\large \rule{0cm}{1.2cm} \textsf{Tuesday 05/07/2013} \hfill
\textsf{Final} \hfill  \textsf{90 minutes}}\\
{\large\rule{0cm}{1.2cm}\textsf{Name: \framebox[2.9in]{\rule{0cm}{0.8cm}} 
\hspace{\fill}
Student ID: \framebox[2.1in]{\rule{0cm}{0.8cm}}}}\\
\end{center}
\vspace{0.8cm}

\noindent
{\bf \textsf{Instructions.}}

\begin{enumerate}
\item Attempt all questions.   
\item Show all the steps of your work clearly.  
\item Good luck 
%The method (reasoning) used to 
%obtain an answer is worth more than the answer itself.   
\end{enumerate}

\vfill

%The \rule commands create vertical space, which makes things sit nicely in 
%vertical way in boxes of table below.

\begin{center}
{\large
\begin{tabular}{|c|c|c|}
\hline
\rule[-0.3cm]{0cm}{1cm}
\textsf{Question} & \textsf{Points} &  \textsf{Your Score} \\
\hline
\hline
\rule[-0.3cm]{0cm}{1cm}
\textsf{ Q1} & \one &\\
\hline
\rule[-0.3cm]{0cm}{1cm}
\textsf{ Q2} & \two &\\
\hline
\rule[-0.3cm]{0cm}{1cm}
\textsf{ Q3} & \three &\\
\hline
\rule[-0.3cm]{0cm}{1cm}
\textsf{ Q4} & \four &\\
\hline
\rule[-0.3cm]{0cm}{1cm}
\textsf{ Q5} & \four &\\
\hline
\rule[-0.3cm]{0cm}{1cm}
 \textsf{ TOTAL} & 100 & \\
 \hline
 \end{tabular}
} 

\end{center}

\vfill


\newpage
\noindent
 \textbf{Q1}. \\ \\ 
\begin{enumerate}[a)]
  \item
    Let $X$ be a non-empty set with at least two elements. Construct a topology on $X$ such that every map $f: X \to \rr$ is continuous. Explain why.

    \vspace{10cm}
  \item 
    Now, construct a topology on $X$ such that there exists at least one continuous map $f: X \to \rr$ and one discontinuous map $f: X \to \rr$. Explain why.  
        \vspace{3cm}

\end{enumerate}

\newpage
\noindent
\textbf{Q2}. 
\\
\\
\begin{enumerate}[a)]
  \item
Let $X$ be a Banach space.
      What is the definition of its dual space $X^*$?  
      
     \vspace{6cm}
   \item
     Equip $X^*$ with any norm you like. Prove that it is indeed a norm. 
     \vspace{10cm}
 \item State a subbasis for the metric topology on $X^*$ induced by its norm. How do we construct the metric topology on $X^*$ from this subbasis?   

       
        \end{enumerate}
\newpage
\noindent
\textbf{Q3}. 
\\
\\ 
Let $X$ and $Y$ be topological spaces. Suppose that $f : X \to Y$ is an injective open
map. Prove that if $f$ is continuous, then there exists a subspace of $Y$ homeomorphic to $X$.
\vspace{10cm}
\newpage
\noindent
\textbf{Q4}.
\\
\\
Let $X$ and $Y$ be topological spaces. Show that if $A \subset X$ and $B \subset Y$ are compact subspaces, then $A \times B$ is compact under the product topology.   

\newpage

\noindent
\textbf{Q5}.
\\
\\
\begin{enumerate}[a)]
    \item Let $X$ be a connected topological space, $Y$ a topological space, and $f: X \to Y$ a continuous map. Prove that $f(X)$ is connected.
	\vspace{10cm}
    \item	
Let $A \subset X$ be compact. Prove that $f(A)$ is compact. 
\end{enumerate}
\newpage





\end{document}
