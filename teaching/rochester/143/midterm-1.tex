\documentclass[12pt, oneside]{amsart}
\usepackage{graphicx}
\usepackage[nohead, margin=0.5in]{geometry}
\usepackage{enumerate}
%\geometry{left=0.5in,right=0.5in,top=0.3in,bottom=0.5in} 

\pagestyle{empty}
%Next are definitions of "\one" etc. 
%This is an easy way of assigning points to your questions and to 
%your table all at once. 
\newcommand{\one}{20}
\newcommand{\two}{20}
\newcommand{\three}{20}
\newcommand{\four}{10}
\newcommand{\five}{10}
\newcommand{\six}{10}
\newcommand{\seven}{10}
\newcommand{\eight}{10}
\newcommand{\nine}{10}
\newcommand{\ten}{10}
\newcommand{\eleven}{10}
\newcommand{\twelve}{10}
\synctex=1

\begin{document}


%feel free to change the title page
%
 
\begin{center}
\hrulefill\\
{\bf \textsf{\raisebox{-0.10cm}{Fall 2012: MATH 143} \hspace{\fill} 
\raisebox{-0.10cm}{Calculus III} \hspace{\fill}
\raisebox{-0.10cm}{David Karapetyan}}}\\
\hrulefill\\
{\large \rule{0cm}{1.2cm} \textsf{Thursday 10/18/2012} \hfill
\textsf{Midterm #1} \hfill  \textsf{90 minutes}}\\
{\large\rule{0cm}{1.2cm}\textsf{Name: \framebox[2.9in]{\rule{0cm}{0.8cm}} 
\hspace{\fill}
Student ID: \framebox[2.1in]{\rule{0cm}{0.8cm}}}}\\
\end{center}
\vspace{0.8cm}

\noindent
{\bf \textsf{Instructions.}}

\begin{enumerate}
\item Attempt all questions.   
\item Show all the steps of your work clearly.  
\item Good luck 
%The method (reasoning) used to 
%obtain an answer is worth more than the answer itself.   
\end{enumerate}

\vfill

%The \rule commands create vertical space, which makes things sit nicely in 
%vertical way in boxes of table below.

\begin{center}
{\large
\begin{tabular}{|c|c|c|}
\hline
\rule[-0.3cm]{0cm}{1cm}
\textsf{Question} & \textsf{Points} &  \textsf{Your Score} \\
\hline
\hline
\rule[-0.3cm]{0cm}{1cm}
\textsf{ Q1} & \one &\\
\hline
\rule[-0.3cm]{0cm}{1cm}
\textsf{ Q2} & \two &\\
\hline
\rule[-0.3cm]{0cm}{1cm}
\textsf{ Q3} & \three &\\
\hline
\rule[-0.3cm]{0cm}{1cm}
\textsf{ Q4} & \four &\\
\hline
\rule[-0.3cm]{0cm}{1cm}
\textsf{ Q5} & \five &\\
\hline
\rule[-0.3cm]{0cm}{1cm}
\textsf{ Q6} & \six &\\
\hline
\rule[-0.3cm]{0cm}{1cm}
\textsf{ Q7} & \seven &\\
 \hline
 \hline

 \textsf{ TOTAL} & 100 & \\
 \hline
 \end{tabular}
} 

\end{center}

\vfill


\newpage
\noindent
\textbf{Q1}.\\ \\ Determine whether each of the sequences converges or diverges. If it converges, compute it's limit. If it diverges, state whether it diverges to $+\infty$, $-\infty$, or neither. \\

\begin{enumerate}[a)]
  \item
     $ \left\{ 1 + \displaystyle{\frac{9^{n}}{10^{n}}} \right\}$
     \vspace{6cm}
   \item
     $\left\{ \cos(n/2) \right\}$
     \vspace{6cm}
   \item
     $\left\{ \displaystyle{\frac{(2n+1)!}{(2n-1)!}} \right\}$
\newpage
  \item
     $ \left\{ (1 + 3/n)^{n} \right\}$
     \vspace{6cm}
   \item
     $\left\{ (-1/2)^{n}\right\}$
     \vspace{6cm}
   \item
     $\left\{ \displaystyle{\sqrt{\frac{n+1}{8n+1}}} \right\}$
   \end{enumerate}
\newpage
\noindent
\textbf{Q2}. \\ \\ If the series converges, explain why, and compute its sum. If the series diverges, state whether it diverges to $\infty$, $-\infty$, or neither, and explain why.

\begin{enumerate}[a)]
  \item
    $\displaystyle{\sum_{n=1}^{\infty} (1 + 1/n)^{n}}$
    \vspace{6cm}
  \item
    $\displaystyle{\sum_{n=1}^{3} (1 + 1/n)^{n}}$
\vspace{6cm}
\item
  $\sum_{n=1}^{\infty} [e^{-n} - e^{-(n+1)}]$
  \newpage
   \item
     $\sum_{n=1}^{\infty} \displaystyle{(1/3)^{n+1}}$
     \vspace{6cm}
   \item $\sum_{n=1}^{\infty} \displaystyle{(3/2)^{n}}$
  \vspace{6cm}
\item $\sum_{n=1}^{\infty} \displaystyle{[3^{n} - 2^{n}]}$
\end{enumerate}
\newpage
\noindent
\textbf{Q3}. \\ \\ Use a convergence or divergence test to show whether the following series converge or diverge. Make sure to state which test you are using, and show all work. 
\begin{enumerate}[a)]
  \item
$\displaystyle{\sum_{n=1}^{\infty} \frac{1}{n^{2} + 1}}$
    \vspace{6cm}
  \item
$\displaystyle{\sum_{n=2}^{\infty} \frac{5}{n - 1}}$
\vspace{6cm}
\item
  $ \displaystyle{\sum_{n=1}^{\infty} \frac{2^{n}}{n!}}$
  \newpage
\item
  $ \displaystyle{\sum_{n=1}^{\infty} \frac{(-3)^{n}}{n} }$
  \vspace{6cm}
\item
  $ \displaystyle{\sum_{n=1}^{\infty} \frac{e^{n}}{n^{2}} }$
  \vspace{6cm}
\item
  $ \displaystyle{\sum_{n=1}^{\infty} \frac{3^{n}}{(2n +1)!} }$
\end{enumerate}



\newpage
\noindent
\textbf{Q4}. \\ \\ For a) and b), use the integral test to show whether the
series converge or diverge. For c), approximate the sum using the first $3$
terms, and state an upper bound for the error (Hint: Use the remainder estimate
for the integral test).
\begin{enumerate}[a)]
  \item
$\displaystyle{\sum_{n=1}^{\infty} \frac{n^{2}}{n^{3} + 1}}$
    \vspace{6cm}
  \item
    $\displaystyle{\sum_{n=2}^{\infty} \frac{1}{n (\log n)^{2}}}$
\vspace{6cm}
\item
  $ \displaystyle{\sum_{n=1}^{\infty} \frac{{1}}{n^{2}}}$
  \newpage
\end{enumerate}




\newpage
\noindent
\textbf{Q5}. \\ \\ State whether the series in a), b), and c) diverge, converge conditionally, or converge absolutely, and explain why. For c), also approximate the sum using the first $2$ terms, and state an upper bound for the error (Hint: use the remainder estimate for alternating series).
\begin{enumerate}[a)]
  \item
$\displaystyle{\sum_{n=1}^{\infty} \frac{(-1)^{n}}{\log n}}$
    \vspace{6cm}
  \item
    $\displaystyle{\sum_{n=2}^{\infty} \frac{(-1)^{n} e^{1/n}}{n^{2}}}$
\vspace{6cm}
\item
  $ \displaystyle{\sum_{n=1}^{\infty} \frac{{(-1)^{n}}}{n^{2}}}$
  \newpage
\end{enumerate}
\textbf{Q6}. \\ \\ Find the radius of convergence and interval of \emph{conditional} convergence for the following power series.
\begin{enumerate}[a)]
  \item
$\displaystyle{\sum_{n=1}^{\infty} \frac{n^{2} x^{n}}{2^{n}}}$
    \vspace{6cm}
  \item
    $\displaystyle{\sum_{n=2}^{\infty} \frac{(x-2)^{n}}{n}}$
\vspace{6cm}
\item
  $ \displaystyle{\sum_{n=1}^{\infty} \frac{{(2-x)^{n}}}{n^{n}}}$
  \newpage
\end{enumerate}







\end{document}
