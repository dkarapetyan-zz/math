\documentclass[12pt, oneside]{amsart}
\usepackage{graphicx}
\usepackage[nohead, margin=0.5in]{geometry}
\usepackage{enumerate}
%\geometry{left=0.5in,right=0.5in,top=0.3in,bottom=0.5in} 
\usepackage{listings}
\pagestyle{empty}
%Next are definitions of "\one" etc. 
%This is an easy way of assigning points to your questions and to 
%your table all at once. 
\newcommand{\p}{\partial}
\newcommand{\one}{20}
\newcommand{\two}{20}
\newcommand{\three}{20}
\newcommand{\four}{20}
\newcommand{\five}{20}
\newcommand{\six}{20}
\newcommand{\rr}{\mathbb{R}}
\synctex=1

\begin{document}


%feel free to change the title page
%

\begin{center}
    \hrulefill\\
    {\bf \textsf{\raisebox{-0.10cm}{Fall 2013: MATH 280} \hspace{\fill} 
            \raisebox{-0.10cm}{Numerical Analysis} \hspace{\fill}
            \raisebox{-0.10cm}{David Karapetyan}}}\\
    \hrulefill\\
    {\large \rule{0cm}{1.2cm} \textsf{Wednesday 12/18/2013} \hfill
        \textsf{Final} \hfill  \textsf{180 minutes}}\\
    {\large\rule{0cm}{1.2cm}\textsf{Name: \framebox[2.9in]{\rule{0cm}{0.8cm}} 
            \hspace{\fill}
            Student ID: \framebox[2.1in]{\rule{0cm}{0.8cm}}}}\\
\end{center}
\vspace{0.8cm}

\noindent
{\bf \textsf{Instructions.}}

\begin{enumerate}
    \item Attempt all questions.   
    \item Show all the steps of your work clearly.  
    \item Good luck 
        %The method (reasoning) used to 
        %obtain an answer is worth more than the answer itself.   
\end{enumerate}

\vfill

%The \rule commands create vertical space, which makes things sit nicely in 
%vertical way in boxes of table below.

\begin{center}
    {\large
        \begin{tabular}{|c|c|c|}
            \hline
            \rule[-0.3cm]{0cm}{1cm}
            \textsf{Question} & \textsf{Points} &  \textsf{Your Score} \\
            \hline
            \hline
            \rule[-0.3cm]{0cm}{1cm}
            \textsf{Q1} & \one &\\
            \hline
            \rule[-0.3cm]{0cm}{1cm}
            \textsf{Q2} & \two &\\
            \hline
            \rule[-0.3cm]{0cm}{1cm}
            \textsf{Q3} & \three &\\
            \hline
            \rule[-0.3cm]{0cm}{1cm}
            \textsf{Q4} & \four &\\
            \hline
            \rule[-0.3cm]{0cm}{1cm}
            \textsf{Q5} & \five &\\
            \hline
            \rule[-0.3cm]{0cm}{1cm}
            \textsf{Q6} & \six &\\
            \hline
            \rule[-0.3cm]{0cm}{1cm}
            \textsf{TOTAL} & 120 & \\
            \hline
        \end{tabular}
    } 

\end{center}

\vfill


\newpage
\noindent
\textbf{Q1}. \\ \\ 
Let $A$ be a square matrix, and let $\sigma(A)$ denote its spectrum.
Assuming Schur's Lemma, prove that
\begin{align*}
\sigma(A) = \inf_{\| \cdot \|} \| A \|
\end{align*}

\newpage

\noindent
\textbf{Q2}.\\ \\ 
Write a finite difference scheme algorithm in C or C++ for solving the
\emph{transport equation}
\begin{align*}
& \p_t u - \p_x u = t
\\
& u(x,0) = x
\end{align*}
where $u = u(x,t)$, $x \in \rr$, and $t \in \rr^{+}$.
\newpage

\noindent
\textbf{Q3}. \\ \\ 
Find the pseudoinverse for the matrix
\begin{align*}
A = 
\begin{bmatrix}
1 & 2 & 3\\
2 & 4 & 6 \\
2 & 3 & 4 \\
\end{bmatrix}.
\end{align*}
\newpage
\noindent
\textbf{Q4}. \\ \\ 
What is iterative refinement? State an iterative refinement scheme, and explain
(without proof) the conditions necessary for it to converge. 
\newpage
\noindent \textbf{Q5}. \\ \\ 
\noindent
Derive the composite trapezoid formula via Newton-Coates by using
the uniform spacing $h = (b-a)/n$, where $x_i = a + ih$, for $0 \le i \le n$.
\vspace{2in}
\newpage
\noindent \textbf{Q6}. \\ \\ 
\noindent
Prove that a Hermitian, positive semidefinite matrix has real, positive
eigenvalues.
\vspace{2in}





\end{document}



