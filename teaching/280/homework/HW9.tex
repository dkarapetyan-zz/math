\documentclass[12 pt]{article}
\usepackage{amsmath,amsfonts,amsthm,amscd}
\newtheorem{thm}{Theorem}[section]
\newtheorem{pro}[thm]{Proposition}
\newtheorem{cor}[thm]{Corollary}
\newtheorem{lem}[thm]{Lemma}
\newtheorem{conj}[thm]{Conjecture}
\newtheorem{ques}[thm]{Question}
\newtheorem{statement}[thm]{Statement}
\newtheorem{claim}[thm]{Claim}
\newtheorem{ex}[thm]{Example}
\newtheorem{rem}[thm]{Remark}
\newtheorem{defn}[thm]{Definition}
\newtheorem{prob}[thm]{Problem}
\newtheorem{quot}[]{Result}
\newcommand{\dime}{\operatorname{dim}}
\newcommand{\Obj}{\operatorname{Obj}}
\newcommand{\Mor}{\operatorname{Mor}}
\newcommand{\chan}{\operatorname{char}}
\newcommand{\degr}{\operatorname{deg}}
\newcommand{\expo}{\operatorname{exp}}
\newcommand{\spant}{\operatorname{span}}
\newcommand{\res}{\operatorname{res}}
\newcommand{\Term}{\operatorname{Term}}
\newcommand{\Init}{\operatorname{Init}}
\newcommand{\spec}{\operatorname{spec}}
\newcommand{\Endo}{\operatorname{End}}
\newcommand{\Supp}{\operatorname{Supp}}
\newcommand{\Order}{\operatorname{ord}}
\newcommand{\kerl}{\operatorname{Ker}}
\newcommand{\Img}{\operatorname{Image}}
\newcommand{\sine}{\operatorname{sin}}

\begin{document}

\centerline{\bf MATH 280: Homework 9}

\bigskip

\noindent
1. \\ Recall a $m \times n$ matrix $\mathbb{A}$ has $n$ singular values $\sigma_1, \dots, \sigma_n$ which are the 
positive square roots of the eigenvalues of the $n \times n$ matrix $\mathbb{A}^* \mathbb{A}$. Furthermore the matrix $\mathbb{A}^* \mathbb{A}$ is Hermitian and thus there 
is an orthonormal basis $\{ \hat{u}_1, \dots, \hat{u}_n \}$ of $\mathbb{C}^n$ (or $\mathbb{R}^n$ if working with real matrices) consisting of eigenvectors of $\mathbb{A}^*\mathbb{A}$. In particular $\mathbb{A}^*\mathbb{A} \hat{u}_i = \sigma_i^2 \hat{u}_i$ for $1 \leq i \leq n$. The rank $r$ of $\mathbb{A}^* \mathbb{A}$ 
is hence the same as the number of nonzero singular values of $\mathbb{A}$. \\
Find the singular values, corresponding orthonormal basis $\{ \hat{u}_1, \dots, \hat{u}_n \}$ and rank($\mathbb{A}^*\mathbb{A}$) of the following matrices: \\
(a) $\begin{bmatrix} 4 & 0 & 0 \\ 0 & 0 & 0 \\ 0 & 0 & 7 \\ 0 & 0 & 0 \end{bmatrix} $\\

\noindent
(b) $\begin{bmatrix} 2 & 1 \end{bmatrix}$ \\

\noindent
(c) $\begin{bmatrix} 5 \\ - 4 \end{bmatrix} $


\medskip

\noindent
2. \\ For a $m \times n$ matrix $\mathbb{A}$ show that $\mathbb{A}^*\mathbb{A}$ and $\mathbb{A}$ have the same nullspace.
(Hint: If $\mathbb{A}^*\mathbb{A}\hat{x}=\hat{0}$, consider $\langle\mathbb{A}^*\mathbb{A}\hat{x}, \hat{x} \rangle$ and use it to show 
$\mathbb{A}\hat{x}=\hat{0}$.) Explain how this and the rank-nullity theorem shows that $rank(\mathbb{A}^*\mathbb{A})=rank(\mathbb{A})$ in general.

\medskip

\noindent
3. \\Recall given a $m \times n$ matrix $\mathbb{A}$, we can write $A=PDQ$ (singular-value decomposition) where $P$ is a $m \times m$ unitary matrix, $Q$ is a $n \times n$ unitary matrix and $D$ is a (pseudo) diagonal $m \times n$ matrix whose nonzero "diagonal" entries are $\sigma_1, \dots, \sigma_r$, the nonzero singular values of $\mathbb{A}$. Recall the rows of $Q$ are given by the adjoints of the set $\{ \hat{u}_1, \dots, \hat{u}_n \}$. The first $r$ columns of $P$ are computed 
via the formula $\hat{v}_i = \frac{1}{\sigma_i} \mathbb{A} \hat{u}_i$ and the remaining $m-r$ are an extension to an orthonormal basis. The pseudoinverse 
$\mathbb{A}^+$ of $\mathbb{A}$ is given by $Q^*D^+P^*$ where $D^+$ is the (pseudo) diagonal $n \times m$ matrix whose nonzero 
"diagonal" entries are $\frac{1}{\sigma_1}, \dots, \frac{1}{\sigma_r}$. \\
(a) Find the singular-value decomposition and the pseudoinverse of the matrix in 1(a). \\
(b) Find the singular-value decomposition and the pseudoinverse of the matrix in 1(b). \\
(c) Find the singular-value decomposition and the pseudoinverse of the matrix in 1(c). \\

\medskip

\noindent
4. \\
Recall a minimal solution to $A\hat{x}=\hat{b}$ is  a solution $\hat{x}$ that minimizes $||A\hat{x}-b||_2$ (least squares solution) such that it is of minimal length $||\hat{x}||_2$ amongst such least square solutions. Find the minimal solutions to the following systems: \\
(a) 
$$\begin{bmatrix} 1 \\ 1 \\1 \end{bmatrix} x_1 = \begin{bmatrix} 2 \\ 4 \\ 0 \end{bmatrix}.$$ \\
(b) 
$$\begin{bmatrix} 4 & 0 & 0 \\ 0 & 0 & 0 \\ 0 & 0 & 7 \\ 0 & 0 & 0 \end{bmatrix} \begin{bmatrix} x_1 \\ x_2 \\ x_3 \end{bmatrix} 
= \begin{bmatrix} 1 \\ 2 \\ 4 \\ 0 \end{bmatrix}.$$

\noindent
5. \\
Explain how in practice one could devise an algorithm to compute the singular
value decomposition of a matrix. Hint: break your algorithm into pieces, and attack
each piece with an algorithm we've learned previously. 

\end{document}


