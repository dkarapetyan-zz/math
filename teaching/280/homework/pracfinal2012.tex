
% Essential Formatting
   
\documentclass[12pt]{article}
\usepackage{epsfig,amsmath,amsthm,amssymb,amscd,latexsym}
\usepackage[multipart,letter]{urmathexam}

% Basic User Def's

\def\ds{\displaystyle}
\def\t{\theta}
% Beginning of the Document

\begin{document}
\examtitle{MATH 280}{PRACTICE FINAL EXAM}{Dec 17, 2012}
\studentinfo
\instructions{
  \begin{itemize}
  \item
    {\bf No calculators are allowed on this exam.}
  \item
  {\bf $8" \times 11"$ (both sides) note sheet is allowed on this exam.}
  \item
    {\bf Please show all your work.
         You may use back pages if necessary.
         You may not receive full credit for
         a correct answer if there is no work shown.}
    {\bf Attempt all questions.}
    \end{itemize}
}
\finishfirstpage

% Problems Start Here % ----------------------------------------------------- %
\newpage


\problem{12} 
\noindent 
(a) Order the functions $x, , \sqrt{x}, e^x, ln(x), x^2, x^3$ from left to right such that $f$ is to the left of $g$ if $f=O(g)$ as $x \to 0$. \\

\vspace{1 in}

\noindent
(b) Order the functions $x, \sqrt{x}, e^x, ln(x), x^2, x^3$ from left to right such that $f$ is to the left of $g$ if $f=O(g)$ as $x \to \infty$. \\

\vspace{1 in}

\noindent
(c) State if the following assertions are true or false and give a brief explanation of your answer: \\
(i) $\frac{1}{n ln(n)} = O(\frac{1}{n})$ as $n \to \infty$. \\

\vspace{1in}

\noindent
(ii) $\frac{1}{n ln(n)}=o(\frac{1}{n})$ as $n \to \infty$. \\

\vspace{1in}

\noindent
(iii) $\sin(x) = 1 - \frac{1}{2}x^2 + O(x^3)$ as $x \to \infty$. \\

\vspace{1in}

\noindent
(iv) $\frac{1}{1+x^2} = 1- x^2+  O(x^3)$ as $x \to 0$. \\


 
    
\newpage
\problem{12}
Consider the recurrence 
$$
x_{n+2}=x_{n+1}+2x_n
$$
for $n \geq 0$. \\

\noindent
(a) If $E$ is the shift operator on the vector space of complex sequences, and $\hat{x}=(x_0,x_1,x_2,\dots)$, rewrite the recurrence in the form 
$p(E)\hat{x}=\hat{0}$. What is the characteristic polynomial $p(x)$? 

\vspace{2 in}

\noindent
(b) Find the roots of the characteristic polynomial and use them to describe the general solution to this recurrence. 

\vspace{1.5in}

\noindent
(c) If $x_0=0$ and $x_1=1$, find $x_2, x_3, x_4, x_5$ directly from the recurrence. \\

\vspace{1in}

\noindent
(d) Find the particular solution to this recurrence when $x_0=0$ and $x_1=1$. In particular express $x_n$ as a function of $n$.

\newpage

\problem{12}
(a) Find the binary expansion of the number $272$. \\

\vspace{3 in}

\noindent
(b) Find the binary expansion of the number $\frac{1}{5}$. \\

\vspace{3 in}

\noindent
(c) Find the hexadecimal representation of the binary number $110.110010110010$.

\newpage

\problem{6}
\noindent
Recall the typical $32$ bit computer representation of a floating point number as discussed in the book, reserves $1$ bit (1st bit) for the sign of the number, 
$8$ bits (bits 2 thru 9) for the biased exponent, and $23$ bits (bits 10 thru 32) for the mantissa. Express the representation 
$11011000010000000000000000000000$ in the form $\pm q 2^m$. Give the value of this number in decimal.

\vspace{2.5 in}

\problem{6}
\noindent
Let $f(x)=x^5$ and $x_0=10$ with absolute error at most $\delta_x=0.1$. Compute $f'(x_0)$ and the condition number and use them to estimate 
the absolute and relative error in $y_0=f(x_0)$. 

\newpage
\problem{6}
In the bisection method, if $c_n$ is accurate to $20$ binary digits, then $c_{n+1}$ is accurate to at least $\underline{~~~~~~~~~~~~}$ binary digits. \\
In the Newton-Raphson method, if $x_n$ is accurate to $15$ decimal digits, then a reasonable estimate for the accuracy of $x_{n+1}$ is that it is 
accurate to $\underline{~~~~~~~~~~~~~~~}$ decimal digits. \\ 

\problem{6}
(a) What is the Newton-Raphson function $N_f(x)$ for $f(x)=x^3-7$. Simplify $N_f(x)$ until it is of the form 
$ax+\frac{b}{x^2}$ for suitable constants $a$ and $b$. For which $x$ is it undefined? Explain why if we start with a positive seed $x_0$, we never 
encounter this problem.

\vspace{2 in}

\noindent
(b) If $x_0=1$ compute $x_1, x_2$.

\newpage
\problem{10}
(a) Write down the recursion one obtains when one applies the secant method to the function $f(x)=x^3-10$. 

\vspace{2 in}

\noindent
(b) Prove that in the secant method, to obtain $x_{n+2}$ one can draw a secant line between the points $(x_n,f(x_n))$ and $(x_{n+1},f(x_{n+1}))$ on the 
graph of $f$ and see where it intersects the $x$-axis.

\newpage
\problem{12}
Use Horner's method to do the following tasks. Please do not use a different method. \\
(a) Evaluate the polynomial $p(x)=x^4+3x^3+2x^2+x+1$ at $x=2$ and find $q(x)=\frac{p(x)-p(2)}{x-2}$. 

\vspace{ 2 in}

\noindent
(b) Verify that $-1$ is a root of the polynomial $p(x)=x^4+3x^3+2x^2+x+1$ and deflate the polynomial at that root. \\

\vspace{2 in}

\noindent
(c) Find $p'(2),p''(2), p'''(2)$ for the same polynomial in examples (a) and (b).

\newpage
\problem{10} Find an annulus in the complex plane such that all the zeros of the polynomial 
$p(z)=4z^6 + 8z^5 + 4z^4 + 3z^3+2z^2+z+13$ must lie in it. Use the localization theorem and explain all your work.


\newpage
\problem{8} 
(a) Explain why $f(x)=\cos(x)$ maps $[0,1] \to [0,1]$ and is a contraction on this interval. 

\vspace{3 in}

\noindent
(b) Explain why for any starting seed $x_0 \in \mathbb{R}$ we have the recursive sequence given by $x_{n+1}=\cos(x_n)$ converges to the 
same limit (independent of $x_0$).

\anotherpart

\problem{8}
\noindent
Find the LU factorization of the matrix $A=\begin{bmatrix}   -4 & - 3 \\ -4 & - 8 \end{bmatrix}$ where $L$ is unit lower triangular and $U$ is upper triangular.

\vspace{3in}

\problem{10}
Recall on $\mathbb{R}^n$ we have a few different norms. For example
$$
||x||_{\infty}=max\{ |x_i| | 1 \leq i \leq n \}
$$
$$
||x||_{1}=\sum_{i=1}^n |x_i| 
$$
and the Euclidean norm 
$$
||x||_{2}=\sqrt{\sum_{i=1}^n x_i^2}.
$$
(a) Show that $$||x||_{\infty} \leq ||x||_2 \leq ||x||_1$$ for all vectors $x \in \mathbb{R}^n$. \\

\vspace{2in}

\noindent
(b) Show that $$||x||_1 \leq n ||x||_{\infty}$$ and $$||x||_2 \leq \sqrt{n} ||x||_{\infty}$$ for all $x \in \mathbb{R}^n$. \\

\vspace{2in}

\problem{12}
Let $\mathbb{A} = \begin{bmatrix} 1 & 2 & 3 \\ 0 & 1 & 2 \\ 0 & 1 & 1 \end{bmatrix}$. \\
(a)Compute $||\mathbb{A}||_{\infty}$ and $\kappa_{\infty}(\mathbb{A})$ the condition number of $\mathbb{A}$ computed 
using the matrix norm $||\cdot ||_{\infty}$ subordinate to the vector norm $|| \cdot ||_{\infty}$. \\

\vspace{2in}

\noindent
(b) If $b=(1,0,2)^T$ and $b'=(1.01,-0.02, 2.03)^T$ and $\mathbb{A}x=b, \mathbb{A}x'=b'$, 
what is an upper bound for $||x-x'||_{\infty}$? (Give an upper bound that does not require you to solve for $x$ and $x'$.) What is an upper bound for $\frac{||x-x'||}{||x||}$?

\vspace{ 2in}

\noindent
(c) Let $\mathbb{B}= \begin{bmatrix} 1.2 & 0.4 & -0.3 \\ 0.2 & 0.9 & -0.1 \\ 0.1 & 0.3 & 0.8 \end{bmatrix}$.
Find $\mathbb{A}=\mathbb{I} - \mathbb{B}$ and compute $||\mathbb{A}||_{\infty}$. 
Use this to explain why $\mathbb{B}^{-1}$ exists and give a Neumann series (matrix power series) that converges to $\mathbb{B}^{-1}$. \\

\vspace{ 3 in}

\problem{12}
\noindent
Consider the system of linear equations given by:
$$
\begin{bmatrix} 3 & 1  \\ 1 & 3   \end{bmatrix}
\begin{bmatrix} x_1 \\ x_2  \end{bmatrix}
=
\begin{bmatrix} 5 \\ 3  \end{bmatrix}
$$
(a) In the Jacobi Iterative Method, find the splitting matrix $\mathbb{Q}$ for this system and compute 
$\mathbb{I} - \mathbb{Q}^{-1} \mathbb{A}$. Write down the corresponding recursion system and 
find $|| \mathbb{I}-\mathbb{Q}^{-1}\mathbb{A}||_{\infty}$. Does this method converge to a solution 
for every initial seed in this case? \\

\newpage

\noindent
(b) In the Gauss-Seidel Method, find the splitting matrix $\mathbb{Q}$ for this system and compute 
$\mathbb{I} - \mathbb{Q}^{-1} \mathbb{A}$. Write down the corresponding recursion system and 
find $|| \mathbb{I}-\mathbb{Q}^{-1}\mathbb{A}||_{\infty}$. Does this method converge to a solution for every initial seed in this case?\\

\vspace{3in}

\problem{8}
Find the Gershgorin disks for each row of the matrix $\mathbb{A} = \begin{bmatrix} 6 & 2 & 1 \\ 1 & -5 & 0 \\ 2 & 1 & 4 \end{bmatrix}$. 
Carefully shade the area given by these disks in the complex plane. Recall all the eigenvalues of $\mathbb{A}$ must lie in this shaded area.

\vspace{3in}

\problem{12}
Recall a $m \times n$ matrix $\mathbb{A}$ has $n$ singular values $\sigma_1, \dots, \sigma_n$ which are the 
positive square roots of the eigenvalues of the $n \times n$ matrix $\mathbb{A}^* \mathbb{A}$. Furthermore the matrix $\mathbb{A}^* \mathbb{A}$ is Hermitian and thus there 
is an orthonormal basis $\{ \hat{u}_1, \dots, \hat{u}_n \}$ of $\mathbb{C}^n$ (or $\mathbb{R}^n$ if working with real matrices) consisting of eigenvectors of $\mathbb{A}^*\mathbb{A}$. In particular $\mathbb{A}^*\mathbb{A} \hat{u}_i = \sigma_i^2 \hat{u}_i$ for $1 \leq i \leq n$. The rank $r$ of $\mathbb{A}^* \mathbb{A}$ 
is hence the same as the number of nonzero singular values of $\mathbb{A}$. \\
(a) Find the singular values, corresponding orthonormal basis $\{ \hat{u}_1, \dots, \hat{u}_n \}$ and rank($\mathbb{A}^*\mathbb{A}$) of the matrix 
$\begin{bmatrix} 2 & 1 \end{bmatrix}$.

\vspace{3in}
\noindent
(b) Recall given a $m \times n$ matrix $\mathbb{A}$, we can write $A=PDQ$ (singular-value decomposition) where $P$ is a $m \times m$ unitary matrix, $Q$ is a $n \times n$ unitary matrix and $D$ is a (pseudo) diagonal $m \times n$ matrix whose nonzero "diagonal" entries are $\sigma_1, \dots, \sigma_r$, the nonzero singular values of $\mathbb{A}$. Recall the rows of $Q$ are given by the adjoints of the set $\{ \hat{u}_1, \dots, \hat{u}_n \}$. The first $r$ columns of $P$ are computed 
via the formula $\hat{v}_i = \frac{1}{\sigma_i} \mathbb{A} \hat{u}_i$ and the remaining $m-r$ are an extension to an orthonormal basis. The pseudoinverse 
$\mathbb{A}^+$ of $\mathbb{A}$ is given by $Q^*D^+P^*$ where $D^+$ is the (pseudo) diagonal $n \times m$ matrix whose nonzero 
"diagonal" entries are $\frac{1}{\sigma_1}, \dots, \frac{1}{\sigma_r}$. Find the singular value decomposition and pseudo inverse of the matrix 
in part(a). \\

\newpage

\problem{12}
(a) Find the Lagrange cardinal functions $\ell_i(x), 0 \leq i \leq 2$ for the nodes $x_0=1, x_1=0, x_2=-1$. 
Simplify your answers. 

\vspace{2 in}

\noindent
(b) Recall given a function $f$ and distinct nodes $x_0, x_1, \dots, x_n$, the unique degree $\leq n$ polynomial $p(x)$ with $p(x_i)=f(x_i), 0 \leq i \leq n$ 
is given by 
$$
\sum_{k=0}^n f[x_0,\dots,x_k] \prod_{i < k} (x-x_i)
$$
where $f[x_0,\dots,x_k]$ is a divided difference. Use the recursive algorithm discussed in class and the book to find the divided differences and 
the interpolating polynomial for the following data points: \\
$\begin{matrix} x & | & 1 & 2 & 4 
\\  y & | & 3 & 4 & 0
\end{matrix}$

\newpage

\noindent
(c) Recall in Hermite interpolation where values of a function $f(x)$ and its derivatives need to be interpolated, this can be viewed as an algorithmic interpolation  where $x$ values are repeated. In this case the recursion becomes
for $x_0 \leq x_1 \leq \dots \leq x_n$:

$$
f[x_0, \dots, x_n] = \begin{cases} \frac{f[x_1,\dots,x_n] - f[x_0, \dots, x_{n-1}]}{x_n-x_0} \text{ if } x_n \neq x_0 \\
\frac{f^{(n)}(x_0)}{n!} \text{ if } x_0=x_n
\end{cases}
$$
Use this to find a quadratic interpolating polynomial $f(x)$ such that \\
$f(0)=1, f'(0)=2, f(1)=-1$. Though this case can be done directly by other methods, please use the Hermite interpolating algorithm.

\vspace{3in}

\problem{8}
Show that the following approximation holds for any $f \in C^5[a,b]$ (i.e. $f$ has continuous 5th derivative on the interval $[a,b]$) where the error term is $O(h^4)$.
$$
f'(x) \sim \frac{1}{12h}[-f(x+2h)+8f(x+h)-8f(x-h)+f(x-2h)] 
$$

\newpage

\problem{8}
In numerical integration formulas $\int_a^b f(x) dx \sim \sum_{i=0}^n A_i f(x_i)$ derived from interpolating $f$ by a degree $\leq n$ polynomial at 
the nodes $x_0, \dots, x_n$, it is known that the formula has zero error for $f \in \Pi_n$ where $Pi_n$ is the vector space of degree $\leq n$ polynomials. \\
(a) In general it is known that the error term in this approximation is minimized over all $f$ when the nodes are chosen as the zeros of the Chebyshev polynomial.
Draw a picture indicating how one can geometrically find the placement of these nodes in the case where $n=3$ i.e. $4$ nodes $x_0, x_1, x_2, x_3$ are used 
for the interval $[-2,2]$.

\vspace{3 in}

\noindent
(b) On the other hand, one can also ask to choose nodes such that the error is zero for a large a vector space as possible. These are called 
Gauss Quadrature Formulas. In general the formula 
$$
\int_a^b f(x) dx ~ \sum_{i=0}^n A_i f(x_i)
$$
can be made to have zero error for all $f \in \Pi_k$ where $k=$~\underline{\hskip 1.5 in}. (Please fill in the blanks with the correct function of $n$).
In the case of $6$ nodes $x_0, x_1, \dots x_5$, and the interval $[0,1]$ please describe a process to choose the nodes correctly to get a 
Gaussian Quadrature formula. You do not have to carry out the computations as they would be very tedious but please describe the process so 
that in principle someone could follow your process to get the answer.

\newpage

\problem{10}
Given the first order differential equation initial condition problem:
$$
\frac{dx}{dt} = x^2+t^2
$$
$x(0)=1$.
(a) Explain why there must exist a unique solution $x(t)$ for $t$ in some open interval about $t=0$. 

\vspace{2 in}

\noindent
(b) Use the 1st order Taylor method (Euler method) to derive an approximate recursive formula for 
$x(t+h)$ in terms of $x(t),t,h$. For any given step size $h$, this can then be used to generate approximate values of the solution.

% Problems End Here % ------------------------------------------------------- %

\problemsdone
\end{document}

% End of the Document
