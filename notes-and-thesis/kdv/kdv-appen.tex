\begin{proof}[Proof of \cref{1lem:cutoff-loc-soln}]
%
%
\begin{equation*}
	\begin{split}
		\lim_{t_{n} \to t} \|u(\cdot, t) - u(\cdot, t_{n})\|_{\dot{H}^s(\ci)} 
		& = \lim_{t_{n} \to t} \|\psi(t) u(\cdot, t) - \psi(t_n) u(\cdot,
		t_{n})\|_{\dot{H}^s(\ci)} 
		\\
		& = \lim_{t_n \to t} \left[ \sum_{n \in \zzdot}| n |
		^{2s} | \psi(t)  \wh{u}(n, t) - \psi(t_n) \wh{ u}(n, t_n) |^2 \right]^{1/2}
		\\
		& = \lim_{t_n \to t} \left[ \sum_{n \in \zzdot} | n |^{2s} | \int_{\rr} (e^{it \tau} - e^{it_{n} \tau}) \wh{\psi u}(n,
		\tau) d \tau |^2 \right]^{1/2}.
	\end{split}
\end{equation*}
		It is clear that
		%
		%
		\begin{equation*}
			\begin{split}
				| n |
				^{2s} | \int_{\rr} (e^{it \tau} - e^{it_{n}\tau}) \wh{\psi u}(n, \tau) d \tau |^2 
		& \le 4  | n |^{2s} \left ( \int_{\rr} |\wh{\psi u}(n, \tau)| d \tau
		\right )^2 
	\end{split}
\end{equation*}
and 
%
%
\begin{equation*}
	\begin{split}
 \sum_{n \in \zzdot} | n |^{2s} \left ( \int_{\rr} |\wh{\psi u}(n, \tau)| d \tau
		\right ) ^2 
		& = \|\wh{\psi u}\|_{\dot{\ell}_n^2 L_\tau^1}
		\\
		& \le \|\psi u \|_{Y^s}^2 
	\end{split}
\end{equation*}
which is bounded by assumption.
Applying dominated convergence completes the proof. 
\end{proof}
%
%
\begin{proof}[Proof of \cref{1lem:schwartz-mult}]
Note that
%
%
\begin{equation*}
	\begin{split}
		\wh{\psi f}\left( n, \tau \right)
		& = \wh{\psi}(\cdot) * \wh{f}(n,
		\cdot)(\tau)
		= \int_\rr \wh{\psi}(\tau_1) \wh{f} \left( n, \tau - \tau_1 \right) 
		d\tau_1
	\end{split}
\end{equation*}
%
%
and hence
%
%
\begin{equation}
	\label{19b}
	\begin{split}
		\|\psi f\|_{\dot{X}^s} 
		& = \left( \sum_{n \in \zzdot} |n|^{2s} \int_\rr \left( 1 + | \tau -
		n^{m} | \right) | \int_\rr \wh{\psi}(\tau_1) \wh{f}\left( n, \tau -
		\tau_1
		\right)  d \tau_1 d \tau |^2 \right)^{1/2}
		\\
		& \le \left( \sum_{n \in \zzdot} |n|^{2s} \int_\rr \left( 1 + | \tau -
		n^{m }
		|
		\right) \left( \int_\rr \wh{\psi}\left( \tau_1 \right) \wh{f}\left( n,
		\tau - \tau_1
		\right)  d \tau_1 d \tau \right)^2 \right)^{1/2}.
	\end{split}
\end{equation}
%
%
Using the relation
%
%
\begin{equation*}
	\begin{split}
		1 + | \tau - n^{m } |
		& = 1 + | \tau + \tau_1 - n^{m} |
		\\
		& \le 1 + | \tau_1 | + | \tau - \tau_1 - n^{m} |
		\\
		& \le \left( 1 + | \tau_1 | \right)\left( 1 + | \tau - \tau_1 -
		n^{m} | \right),
	\end{split}
\end{equation*}
%
%
we obtain
%
%
\begin{equation*}
	\begin{split}
		\eqref{19b}
		& \le \left( \sum_{n \in \zzdot} |n|^{2s} \right.
		\\
		& \times \left . \int_\rr \left(
		\int_\rr \left( 1 + | \tau_1 | \right)^{1/2} | \wh{\psi}(\tau_1) |
		\left( 1 + | \tau - \tau_1 - n^{m} | \right)^{1/2} \wh{f}\left( n, \tau
		- \tau_1
		\right)d \tau_1
		\right)^2 d \tau \right)^{1/2}
	\end{split}
\end{equation*}
%
%
which by Minkowski's inequality is bounded by
%
%
\begin{equation}
	\label{18a}
	\begin{split}
		& \left( \sum_{n \in \zzdot} |n|^{2s}  \right.
		\\
		& \times \left. \left( \int_\rr \left[ \int_\rr
		\left( 1 + | \tau_{1} | \right) | \wh{\psi}(\tau_1) |^2 \left( 1 + |
		\tau - \tau_1 - n^{m} |
		\right) | \wh{f}\left( n, \tau - \tau_1 \right) |^2 d \tau_1 
		\right]^{1/2} d \tau \right)^2 \right)^{1/2}.
	\end{split}
\end{equation}
%
%
Using the change of variable $\tau - \tau_1 \to \lambda$ gives
%
%
\begin{equation*}
	\begin{split}
		\eqref{18a}
		& = \left( \sum_{n \in \zzdot} |n|^{2s}\right.
		\\
		& \times \left.  \left( \int_\rr \left[
		\int_\rr \left( 1 + | \tau_1 | \right) | \wh{\psi}\left( \tau_1
		\right) |^2 \left( 1 + | \lambda - n^{m} | \right) | \wh{f} \left( n,
		\lambda
		\right)|^2 d \tau_1 \right]^{1/2} d \lambda \right)^2 \right)^{1/2}
		\\
		& =  \left( \sum_{n \in \zzdot} |n|^{2s} \right.
		\\
		& \times \left. \left( \int_\rr \left( 1 + |
		\tau_1 |
		\right)^{1/2} | \wh{\psi}(\tau_1) | d \tau_1 \left[ \int_\rr \left( 1 + |
		\lambda - n^{m} |
		\right) | \wh{f}\left( n, \lambda \right) |^2 d \lambda \right]^{1/2}
		\right)^2 \right)^{1/2}
		\\
		& = c_{\psi} \left( \sum_{n \in \zzdot} |n|^{2s} \left( \left[ \int_\rr
		\left( 1 + | \lambda - n^{m} | \right) | \wh{f}\left( n, \lambda
		\right) |^2 d \lambda
		\right]^{\cancel{1/2}} \right)^{\cancel{2}} \right)^{1/2}
		\\
		& = c_{\psi} \|f\|_{\dot{X}^s},
	\end{split}
\end{equation*}
%
%
concluding the proof. 
\end{proof}

%
\begin{proof}[Proof of \cref{1lem:number-theory}]
First note that
%
\begin{equation*}
		| - n^{3} + n_1^3 + n_2^3|
		 = 3 | n | |n_1 | |n_2 |.
\end{equation*}
%
%
Hence, it will be enough to show that for $c \ge 0$
%
%
\begin{equation*}
	\begin{split}
		| n | |n_1 | |n_2 | \gtrsim | n |^{\frac{2 + c}{2}}| n_1
		|^{\frac{2-c}{2}}| n_2 |^{\frac{2-c}{2}}
	\end{split}
\end{equation*}
%
%
or, dividing through on both sides by $|n| | n_1 | | n_2 |$ and rearranging terms
%
%
\begin{equation*}
	\begin{split}
		| n |^{c/2} \lesssim | n_1 |^{c/2} | n_2 |^{c/2}.
	\end{split}
\end{equation*}
%
%
But
%
%
\begin{equation*}
	\begin{split}
		| n |^{c/2} &= | n_1 + n_2 |^{c/2}
		\\
		& \le (| n_1 | + |n_2|)^{c/2} 
		\\
		& \le (2\max\{|
		n_1 |, | n_2 |)^{c/2}
		\\
		& \le (2|
		n_1 | | n_2 |)^{c/2}
		\\
		& = 2^{c/2} | n_1 |^{c/2} | n_2 |^{c/2}
	\end{split}
\end{equation*}
%
%
where the last step follows from the fact that, for $a, b \in \zz$ 
%\cref{1lem:splitting}. \qquad \qed
%
%\subsection{Proof of \cref{1lem:splitting}.} We have
%%
%%
\begin{equation}
	\label{16a}
	\begin{split}
		| a + b | 
		& \le | a | + | b | 
		\\
		& \le 2\left( \max\{| a |, | b | \}\right)
		\\
		& \le 2 |a| |b|.
	\end{split}
\end{equation} 
This concludes the proof.
\end{proof}
%%
%
