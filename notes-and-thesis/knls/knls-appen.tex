%
%
\begin{proof}[Proof of \cref{llem:number-theory}] Define
%
\begin{equation*}
	\begin{split}
		| - n^{m} + n_1^{m} + n_2^{m }|
		& = | n_{1}^{m} - n^{m} + (n-n_{1})^{m}| 
		\\
		& \doteq f(n).
	\end{split}
\end{equation*}
%
%
Then the absolute minima
of $f(n)$ occur only on the line $n = 1+n_{1}$ of lattice points 
(the line $n = n_1$ is not available by assumption). Next, note that
%
%
\begin{equation*}
	\begin{split}
		f(1+ n_{1}) = | n_{1}^{m} - (1 + n_{1})^m + 1 |
		& = | (1 + n_{1} )^{m} - n_{1}^{m} -1 |
		\\
		& = | \sum_{1 \le k \le m-1} c_{k} n_1^{k}|, \qquad \{c_k\} \in \mathbb{N}
		\setminus 0.
	\end{split}
\end{equation*}
If $n_1 = -1$, then clearly $| (1 + n_{1})^m - n_1^m -1 | = 2 \ge 0 = |n|^2
|n_1|^{m-3}$. Hence, by symmetry, and the fact that $n_1 \neq 0$ by assumption,
we may further assume
$n_1 >0$ without loss of generality.
Then 
%
%
\begin{equation*}
	\begin{split}
	  | \sum_{1 \le k \le m-1} c_{k} n_1^{k}|
	 & = \sum_{1 \le k \le m-1} |c_{k}| |n_1|^{k}
	 \\
	 & = |n_1| \sum_{0 \le \ell \le m-2} |c_{\ell}| |n_1|^{k}, \qquad \{c_\ell\}
	 \subset \mathbb{N} \setminus 0
	 \\
	 & \ge c_{m-2}|n_1| | n_1|^{m-2}
	 \\
	 & = c_{m-2}| n_1 |^2 | n_1 |^{m-3}
	 \\
	 & \ge \frac{c_{m-2}}{4} (1 + | n_1 |)^2 | n_1|^{m-3}
	 \\
	 & \simeq n^2 | n_1 |^{m-3}
	\end{split}
\end{equation*}
%
%
completing the proof. 
\end{proof}
%
%
%
\begin{proof}[Proof of \cref{llem:splitting}] We have
%
%
\begin{equation}
	\label{l6a}
	\begin{split}
		1 + | a + b | 
		& \le 1 + | a | + | b | 
		\\
		& \le 1 + | a | + 1 + | b | 
		\\
		& \le 2\left( \max\{1+| a |, 1+| b | \}\right)
		\\
		& \le 2 \left( 1 + | a | \right)\left( 1 + | b | \right), \qquad a, b \in {\zz}.
	\end{split}
\end{equation}
%
%
Raising both sides of expression $\eqref{l6a}$ to the $k$ power completes 
the proof. 
\end{proof}
