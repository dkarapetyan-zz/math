%
\documentclass[12pt,reqno]{amsart}
\usepackage{amssymb}
\usepackage{cancel}  %for cancelling terms explicity on pdf
\usepackage{yhmath}   %makes fourier transform look nicer, among other things
\usepackage{framed}  %for framing remarks, theorems, etc.
\usepackage[shortalphabetic, initials, msc-links]{amsrefs} %for the bibliography; uses cite pkg
%\usepackage{showkeys}  %shows source equation labels on the pdf
\usepackage[margin=3cm]{geometry}  %page layout
%\usepackage[pdftex]{graphicx} %for importing pictures into latex--pdf compilation
\setcounter{secnumdepth}{1} %number only sections, not subsections
\hypersetup{colorlinks=true,
linkcolor=blue,
citecolor=blue,
urlcolor=blue,
}
\synctex=1
\numberwithin{equation}{section}  %eliminate need for keeping track of counters
\numberwithin{figure}{section}
\setlength{\parindent}{0in} %no indentation of paragraphs after section title
\renewcommand{\baselinestretch}{1.1} %increases vert spacing of text
%
%
\newcommand{\ds}{\displaystyle}
\newcommand{\ts}{\textstyle}
\newcommand{\nin}{\noindent}
\newcommand{\rr}{\mathbb{R}}
\newcommand{\nn}{\mathbb{N}}
\newcommand{\zz}{\mathbb{Z}}
\newcommand{\cc}{\mathbb{C}}
\newcommand{\ci}{\mathbb{T}}
\newcommand{\zzdot}{\dot{\zz}}
\newcommand{\wh}{\widehat}
\newcommand{\p}{\partial}
\newcommand{\ee}{\varepsilon}
\newcommand{\vp}{\varphi}
%
%
\theoremstyle{plain}  
\newtheorem{theorem}{Theorem}
\newtheorem{proposition}{Proposition}
\newtheorem{lemma}{Lemma}
\newtheorem{corollary}{Corollary}
\newtheorem{claim}{Claim}
\newtheorem{conjecture}[subsection]{conjecture}
%
\theoremstyle{definition}
\newtheorem{definition}{Definition}
%
\theoremstyle{remark}
\newtheorem{remark}{Remark}
\newtheorem{case}{Case}
%
%
%
\def\makeautorefname#1#2{\expandafter\def\csname#1autorefname\endcsname{#2}}
\makeautorefname{equation}{Equation}
\makeautorefname{footnote}{footnote}
\makeautorefname{item}{item}
\makeautorefname{figure}{Figure}
\makeautorefname{table}{Table}
\makeautorefname{part}{Part}
\makeautorefname{appendix}{Appendix}
\makeautorefname{chapter}{Chapter}
\makeautorefname{section}{Section}
\makeautorefname{subsection}{Section}
\makeautorefname{subsubsection}{Section}
\makeautorefname{paragraph}{Paragraph}
\makeautorefname{subparagraph}{Paragraph}
\makeautorefname{theorem}{Theorem}
\makeautorefname{theo}{Theorem}
\makeautorefname{thm}{Theorem}
\makeautorefname{addendum}{Addendum}
\makeautorefname{add}{Addendum}
\makeautorefname{maintheorem}{Main theorem}
\makeautorefname{corollary}{Corollary}
\makeautorefname{lemma}{Lemma}
\makeautorefname{sublemma}{Sublemma}
\makeautorefname{proposition}{Proposition}
\makeautorefname{property}{Property}
\makeautorefname{scholium}{Scholium}
\makeautorefname{step}{Step}
\makeautorefname{conjecture}{Conjecture}
\makeautorefname{question}{Question}
\makeautorefname{definition}{Definition}
\makeautorefname{notation}{Notation}
\makeautorefname{remark}{Remark}
\makeautorefname{remarks}{Remarks}
\makeautorefname{example}{Example}
\makeautorefname{algorithm}{Algorithm}
\makeautorefname{axiom}{Axiom}
\makeautorefname{case}{Case}
\makeautorefname{claim}{Claim}
\makeautorefname{assumption}{Assumption}
\makeautorefname{conclusion}{Conclusion}
\makeautorefname{condition}{Condition}
\makeautorefname{construction}{Construction}
\makeautorefname{criterion}{Criterion}
\makeautorefname{exercise}{Exercise}
\makeautorefname{problem}{Problem}
\makeautorefname{solution}{Solution}
\makeautorefname{summary}{Summary}
\makeautorefname{operation}{Operation}
\makeautorefname{observation}{Observation}
\makeautorefname{convention}{Convention}
\makeautorefname{warning}{Warning}
\makeautorefname{note}{Note}
\makeautorefname{fact}{Fact}
%
\begin{document}
\title{Interpolation Theorems}
\author{David Karapetyan}
\address{Department of Mathematics  \\
University  of Notre Dame\\
Notre Dame, IN 46556 }
\date{}
%
\maketitle
%
%
%
%
%
%
\section{Riesz-Thorin}
\label{sec:riesz-thorin}
%
%
%%%%%%%%%%%%%%%%%%%%%%%%%%%%%%%%%%%%%%%%%%%%%%%%%%%%%
%
%
%				Riesz-Thorin 
%
%
%%%%%%%%%%%%%%%%%%%%%%%%%%%%%%%%%%%%%%%%%%%%%%%%%%%%%
%
%
\begin{theorem}
\label{thm:riesz-thorin} 
Let $T$ be a linear operator from $L^{p_{0}}(U) + L^{p_{1}}(U)$ to
$L^{q_{0}}(V) + L^{q_{1}}(V)$ such that $T$ is bounded
from $L^{p_0}(U)$ to $L^{q_0}(V)$ with operator norm $M_{p_{0}q_{0}}$ and from
$L^{p_{1}}(U)$ to $L^{q_1}(V)$ with operator norm $M_{p_{1}q_{1}}$. Then $T$ is
bounded from $L^{p_{\lambda}}(U)$ to $L^{q_{\lambda}}(V)$ with operator norm
%
%
\begin{equation}
\label{riesz-op-norm}
M_{p_{\lambda}q_{\lambda}} \le M_{p_{0}q_{0}}^{1-\lambda}M_{p_{1}q_{1}}^{\lambda}
\end{equation}
%
%
where 
%
%
\begin{equation}
\begin{split}
\label{riesz-op-norm-params}
& \frac{1}{p_\lambda}  = \frac{1-\lambda}{p_0} + \frac{\lambda}{p_1},
\quad  \frac{1}{q_\lambda} = \frac{1-\lambda}{q_0} + \frac{\lambda}{q_1}, \qquad
\lambda \in (0,1).
\end{split}
\end{equation}
%
%
%
%
\end{theorem}
%
%
{\bf Proof.} The case $p_0 = p_1 \doteq p$ follows immediately from the
following lemma, whose proof is provided in the appendix.
%
%
%%%%%%%%%%%%%%%%%%%%%%%%%%%%%%%%%%%%%%%%%%%%%%%%%%%%%
%
%

%				simple riesz thorin
%
%
%%%%%%%%%%%%%%%%%%%%%%%%%%%%%%%%%%%%%%%%%%%%%%%%%%%%%
%
%
\begin{lemma}
\label{lem:simp-riesz-thorin}
If $0 < p < q < r \le \infty$, then $L^{p} \cap L^{r} \subset L^{q} \subset
L^{p} + L^{r}$, with
%
%
\begin{equation}
\label{embed-relation}
\| f \|_{q} \le \| f \|_{p}^{\lambda} \| f \|_{r}^{1-\lambda}
\end{equation}
%
%
where $\lambda \in (0,1)$ is defined by
%
%
\begin{align}
\label{embed-param-fin}
& \frac{1}{q} = \frac{1-\lambda}{r} + \frac{\lambda}{p}, \quad r < \infty, \qquad
\\
\label{embed-param-infty}
& \frac{1}{q} = \frac{\lambda}{p}, \quad r = \infty.
\end{align}
%
%
\end{lemma}
%
%
Hence, we assume without loss of generality that $p_0 < p_1$. From this it
follows that $p_\lambda < \infty$ for all $\lambda \in (0,1)$. This is a crucial
observation, due to the following result, whose proof is provided in the
appendix.
%
%
%
%
%%%%%%%%%%%%%%%%%%%%%%%%%%%%%%%%%%%%%%%%%%%%%%%%%%%%%
%
%
%	 Simple functions Dense			
%
%
%%%%%%%%%%%%%%%%%%%%%%%%%%%%%%%%%%%%%%%%%%%%%%%%%%%%%
%
%
\begin{lemma}
\label{lem:simp-func}
The simple functions with compact support are dense in $L^{p}$, $1 \le p <
\infty$.
\end{lemma}
%
%
\begin{framed}
\begin{remark}
\label{rem:simplify-proof}
Hence, it will be enough to prove \autoref{thm:riesz-thorin} for simple
functions. To see this, let $f \in L^{p_0} + L^{p_1}$, and let $f_n$ be a
sequence of simple functions converging to $f$ in $L^{p_\lambda}$. Assume
%
%
\begin{equation*}
	\begin{split}
		\| Tf_{n} \|_{q_\lambda} \le M_{p_\lambda q_\lambda} \| f_n \|_{p_\lambda}.
	\end{split}
\end{equation*}
%
%
Define $E = \left\{ x: f(x) >1 \right\}$. Then $\|(f -f_n)\chi_{E}\|_{p_0} \le
\| (f - f_n) \chi_{E} \|_{p_\lambda} \le \| f -f_n \|_{p_\lambda} \to 0$. Similarly,
$\|(f -f_n) \chi_{E^c}\|_{p_1} \le  \|(f -f_n) \chi_{E^c} \|_{p_\lambda} \le \|f -f_n
\|_{p_\lambda} \to 0$. Furthermore, since $T$ is bounded from $L^{p_0}$ to
$L^{q_0}$ and $L^{p_1}$ to $L^{q_1}$, it is continuous from $L^{p_0}$ to
$L^{q_0}$ and $L^{p_1}$ to $L^{q_1}$. Hence
%
%
\begin{equation*}
	\begin{split}
		& T(f\chi_{E} - f_n \chi_{E}) \to 0  \ \ \text{in} \ \ L^{q_0},
		\\
		&  T(f\chi_{E^c} - f_n \chi_{E^c}) \to 0 \ \ \text{in} \ \ L^{q_1}.
	\end{split}
\end{equation*}
%
%
Hence, $Tf_n \chi_{E} \to Tf \chi_{E}$ almost everywhere, and $Tf_n
\chi_{E^c} \to Tf \chi_{E^c}$ almost everywhere. By the linearity of
$T$ it follows that $Tf_n \to Tf$ almost
everywhere. Applying Fatou's Lemma, we conclude
%
%
%
%
\begin{equation*}
	\begin{split}
		\|Tf\|_{q_\lambda} \le \liminf \| Tf_n \|_{q_\lambda} \le \liminf M_{p_\lambda q_\lambda} \| f_n
		\|_{p_\lambda} = M_{p_\lambda q_\lambda} \| f \|_{p_\lambda}.
	\end{split}
\end{equation*}
\end{remark}
\end{framed}
%
%
We adopt the notation
%
%
\begin{equation*}
\begin{split}
<f, g> = \int_{V}f(x)g(x)dx.
\end{split}
\end{equation*}
%
%
By \autoref{rem:simplify-proof} and duality, it is enough to show that for a
simple function $f \in L^{p_\lambda}$ with compact support
%
%
\begin{equation*}
\begin{split}
| \langle Tf, g \rangle |  \le M_{p_0q_0}^{1-\lambda}M_{p_1q_1}^{\lambda}
\|f\|_{p_\lambda},
\quad \|g\|_{\tilde{q}_{\lambda}} =1, \quad \frac{1}{q_{\lambda}} +
\frac{1}{\tilde{q}_{\lambda}} =1
\end{split}
\end{equation*}
%
%
or equivalently, by normalizing $f$
%
%
\begin{equation*}
\begin{split}
| \langle Tf, g \rangle |  \le M_{p_0q_0}^{1-\lambda}M_{p_1q_1}^{\lambda},
\quad \|g\|_{\tilde{q}_{t}} =1, \quad \frac{1}{q_{\lambda}} +
\frac{1}{\tilde{q_{\lambda}}} =1.
\end{split}
\end{equation*}
%
%
Furthermore, by \autoref{lem:simp-func}, we can find compactly supported simple
functions $g_n$ converging to $g$ in $L^{q_{\lambda}}$.
It follows that the $g_n$ converge to $g$ almost everywhere.
Applying a dominated convergence
argument then gives
%
%
\begin{equation*}
\begin{split}
\lim_{n \to \infty}  \langle Tf, g_n \rangle
= \langle Tf, g \rangle.
\end{split}
\end{equation*}
%
%
Hence, to complete the proof, it will suffice to show that for simple functions
$f, g$ with compact support
%
%
\begin{equation}
\label{duality-relation}
\begin{split}
|\langle Tf, g \rangle | \le M_{p_0 q_0}^{1- \lambda} M_{p_1 q_1}^{\lambda},
\quad \| f \|_{p_\lambda} = \| g \|_{\tilde{q}_\lambda} =1.
\end{split}
\end{equation}
%
%
Write
%
%
\begin{equation*}
\begin{split}
	& f(x) = \sum_j a_j \chi_{A_j}(x), \quad A_j \ \text{disjoint}, \ m(A_j) <
	\infty
	\\
	& g(x) = \sum_{k}b_{k} \chi_{B_k}(x), \quad B_j \ \text{disjoint}, \ m(B_j) <
	\infty.
\end{split}
\end{equation*}
%
%
For $0 \le \text{Re}\, z
\le 1$, define
%
%
\begin{gather*}
\frac{1}{p_z} = \frac{1-z}{p_0} + \frac{z}{p_1}
,\quad \frac{1}{\tilde{q}_z} =
\frac{1-z}{\tilde{q}_0} + \frac{z}{\tilde{q}_1},
\\
\vp_{f}(z) = \vp_{f}(x, z) = \sum_{j} | a_j |^{p_\lambda / p_z} e^{i
\arg(a_j)} \chi_{A_j}(x),
\\
\vp_{g}(z) = \vp_{g}(x, z) = \sum_{k} | b_k |^{\tilde{q}_\lambda / \tilde{q}_z} e^{i
\arg(b_k)} \chi_{B_k}(x),
\\
F(z) = \langle T\vp_{f}(z), \vp_{g}(z) \rangle.
\end{gather*}
%
%
By the chain rule, triangle inequality, and H{\"o}lder
%
%
\begin{equation*}
\begin{split}
| F'(z) |
& \le \langle | (T{\vp_f})'(z) |, | \vp_{g}(z) | \rangle + \langle | T
\vp_{f} |, |\vp'_{g}| \rangle
\\
& \le \| (T\vp_{f})'(z) \|_{L_x^{q_0}} \|\vp_{g}(z)\|_{L_x^{\tilde{q_0}}} + \|
T\vp_{f}(z) \|_{L_x^{q_0}} \|\vp_{g}'(z) \|_{L_x^{\tilde{q_0}}}
\\
& = \| [T( \vp'_{f})](z) \|_{L_x^{q_0}} \|\vp_{g}(z)\|_{L_x^{\tilde{q_0}}} + \|
T\vp_{f}(z) \|_{L_x^{q_0}} \|\vp_{g}'(z) \|_{L_x^{\tilde{q_0}}}
\\
& \le M_{p_0 q_0}
\| \vp'_{f}(z)] \|_{L_x^{p_0}} \|\vp_{g}(z)\|_{L_x^{\tilde{q_0}}} + M_{p_0 q_0}\|
\vp_{f}(z) \|_{L_x^{p_0}} \|\vp_{g}'(z) \|_{L_x^{\tilde{q_0}}}.
\end{split}
\end{equation*}
%
%
Note that
%
%
\begin{gather*}
	\vp_{f}(z) \in L^{q_0}_x, \ \vp_{g}(z) \in L_{x}^{\tilde{q_0}},
\\
\vp_{f}'(z) = \frac{p_z'}{p_\lambda} \sum_{j} \log |a_j| \cdot |
a_j|^{p_z/p_\lambda}e^{i \arg(a_j)} \chi_{A_j} \in L^{p_0}_x,
\\
\vp'_{g}(z) =  \frac{\tilde{q_\lambda}'}{\tilde{q_z}} \sum_{k} \log|b_k| \cdot
| b_k |^{\tilde{q_\lambda} / \tilde{q_z}} e^{i
\arg(b_k)} \chi_{B_k} \in L_{x}^{\tilde{q_0}}.
\end{gather*}
%
%
Hence, $F(z)$ is analytic on the interior of the strip $0 \le \text{Re}\, z \le
1$ ($p_z' \neq \infty$ for $z$ in the interior). Furthermore,
%
%
\begin{equation*}
\begin{split}
| T \vp_{f}(z_n) \vp_{g}(z_n) | 
& \le \sum_{j} | a_j |^{p_\lambda / \text{Re}\, p_z} T \chi_{A_j}
\sum_{k} | b_k |^{\tilde{q_\lambda} / \text{Re}\, \tilde{q_z}} \chi_{B_k}
\\
& \le \sum_{j} | a_j |^{\gamma_1} T \chi_{A_j}
\sum_{k} | b_k |^{\gamma_2} \chi_{B_k}\end{split}
\end{equation*}
%
%
where
%
%
\begin{gather*}
\gamma_1 = p_\lambda/p_0, \gamma_2 = \tilde{q_\lambda}/\tilde{q_0} \quad
\text{if} \quad \text{Re}\, z=0
\\
\gamma_1 = p_\lambda/p_1, \gamma_2 = \tilde{q_\lambda}/\tilde{q_1}\quad
\text{if} \quad \text{Re}\, z=1
\end{gather*}
%
%
while H\"older and Minkowski give
%
%
\begin{equation*}
\begin{split}
	\int_{V} \sum_{j} | a_j |^{\gamma_1} T \chi_{A_j}
	\sum_{k} | b_k |^{\gamma_2} \chi_{B_k} dx
	& \le \sum_{j} \| a_j^{\gamma_1} T
	\chi_{A_j}\|_{L_x^{q_0}}  \sum_{k} \|  b_k^{\gamma_2} \chi_{B_k}
	\|_{L_x^{\tilde{q_0}}} 
	\\
	& \le M_{p_0 q_0}\sum_{j} \|| a_j |^{\gamma_1} 
	\chi_{A_j}\|_{L_x^{p_0}}  \sum_{k} \| | b_k |^{\gamma_2} \chi_{B_k}
	\|_{L_x^{\tilde{q_0}}} 
	\\
	& < \infty
\end{split}
\end{equation*}
%
%
so by dominated convergence, $F(z)$ is bounded and continuous on the boundary of
the strip $0 \le \text{Re}\, \le 1$. Next, applying H\"older again gives
%
%
\begin{equation}
\label{key-duality-est}
\begin{split}
	& \int_{V} | T \vp_{f}(z) \vp_{g}(z) | dx 
	\le  \|T \vp_{f}(z)\|_{L_x^{q_0}} \|\vp_{g}(z)\|_{L_x^{\tilde{q_0}}}
	\le M_{p_0 q_0} \|\vp_{f}(z)\|_{L_x^{p_0}} \|\vp_{g}(z)\|_{L_x^{\tilde{q_0}}}
	\\
	& \int_{V} | T \vp_{f}(z) \vp_{g}(z) | dx
	\le  \|T \vp_{f}(z)\|_{L_x^{q_1}} \|\vp_{g}(z)\|_{L_x^{\tilde{q_1}}}
	\le M_{p_1 q_1} \|\vp_{f}(z)\|_{L_x^{p_1}} \|\vp_{g}(z)\|_{L_x^{\tilde{q_1}}}.
\end{split}
\end{equation}
Note that
%
%
\begin{equation*}
\begin{split}
	| \vp_f(it) |
	& = | \sum_{j}| a_j |^{p_\lambda / p_{it}} e^{i \arg(aj)}
	\chi_{A_j} |
	\\
	& = | \sum_{j}| a_j |^{p_\lambda / p_0} e^{i \arg(aj)}
	\chi_{A_j} |
	\\
	& =  \sum_{j}| a_j^{p_\lambda / p_0} e^{i \arg(aj)}
	\chi_{A_j} | \qquad \text{(since the $A_j$ are disjoint)}
	\\
	& = \sum_{j}| |a_j|^{p_\lambda / p_0}
	\chi_{A_j} | 
	\\
	& = \left( \sum_{j} | a_j | \chi_{A_j} \right)^{p_\lambda / p_0}
	\qquad \text{(since the $A_j$ are disjoint)}
	\\
	& = | \sum_{j} | a_j |e^{i \arg(a_j)} \chi_{A_j} |^{p_\lambda / p_0}
	\qquad \text{(since the $A_j$ are disjoint)}.
\end{split}
\end{equation*}
Hence, 
\begin{equation}
	\label{modulus-1}
\begin{split}
	| \vp_f(it) |
	= | f |^{p_\lambda / p_0}.
\end{split}
\end{equation}
%
%
Similarly,
%
%
\begin{align}
	\label{modulus-2}
	& | \vp_f(1 + it) | = | f |^{p_\lambda / p_1},
	\\
	\label{modulus-3}
	& | \vp_{g}(it) | = | g |^{\tilde{q_\lambda} / \tilde{q_0}}
	\\
	\label{modulus-4}
	& | \vp_{g}(1 + it) | = | g |^{\tilde{q_\lambda} / \tilde{q_1}}.
\end{align}
%
%
Hence, applying \eqref{modulus-1}-\eqref{modulus-4} to \eqref{key-duality-est},
we obtain
%
%
\begin{equation}
\label{bound-for-apply-3-line-lem}
\begin{split}
	& \text{Re}\,z = 0: \qquad \int_{V} | T \vp_{f}(z) \vp_{g}(z) | dx \le M_{p_0
	q_0},
	\\
	& \text{Re}\,z =1: \qquad \int_{V} | T \vp_{f}(z) \vp_{g}(z) | dx \le M_{p_1
	q_1}.
\end{split}
\end{equation}
%
%
Since
%
%
\begin{equation*}
\begin{split}
	|\langle T\vp_{f}(z), \vp_{g}(z) \rangle | \le \int_{V} | T \vp_{f}(z)
	\vp_{g}(z) | dx,
\end{split}
\end{equation*}
%
%
an application of the following lemma (with proof provided in the appendix)
completes the proof of \autoref{thm:riesz-thorin}. \qquad \qedsymbol
%%%%%%%%%%%%%%%%%%%%%%%%%%%%%%%%%%%%%%%%%%%%%%%%%%%%%
%
%
%				Three Lines Lemma
%
%
%%%%%%%%%%%%%%%%%%%%%%%%%%%%%%%%%%%%%%%%%%%%%%%%%%%%%
%
%
\begin{lemma}[Three Lines Lemma]
\label{lem:three-lines}
Let $F(z)$ be a continuous bounded function on the strip $0 \le \text{Re} \, z \le
1$ that is analytic on the interior of the strip. If $| F(z) | \le M_0$ on the
line $\text{Re}\, z=0$, and $| F(z) | \le M_1$ on the line $\text{Re}\, z=1$, then
\end{lemma}
%
%
%
%
\begin{equation}
\label{three-lines}
\begin{split}
	| F(z) | \le M_{0}^{1-\lambda} M_{1}^{\lambda} \ \ \text{on the line} \ \
	\text{Re}\,z=\lambda, \quad 0 \le \lambda \le 1.
\end{split}
\end{equation}
%
%
%
%%%%%%%%%%%%%%%%%%%%%%%%%%%%%%%%%%%%%%%%%%%%%%%%%%%%%
%
%
%				Stein
%
%
%%%%%%%%%%%%%%%%%%%%%%%%%%%%%%%%%%%%%%%%%%%%%%%%%%%%%
%
%
\section{Stein Interpolation} 
\label{sec:stein-interp}
Following Klainerman \cite{Klainerman:fk}, we introduce the following.  
\begin{definition}
Let $T_{z}$ be a family of linear operators indexed by $z \in D$. We say the
$T_{z}$ are an analytic family of operators if
\begin{enumerate}
\item{$T_{z}$ maps simple functions into measurable functions}
\item{The map $z \mapsto T_{z}$ is analytic in the interior of the strip
%
%
\begin{equation*}
\begin{split}
0 \le \text{Re}\, z \le 1
\end{split}
\end{equation*}
%
%
and bounded and continuous on the boundary.}
\end{enumerate}
\end{definition}
%
%
%%%%%%%%%%%%%%%%%%%%%%%%%%%%%%%%%%%%%%%%%%%%%%%%%%%%%
%
%
%                stein-interp
%
%
%%%%%%%%%%%%%%%%%%%%%%%%%%%%%%%%%%%%%%%%%%%%%%%%%%%%%
%
%
\begin{lemma}[Stein Complex Interpolation]
Let $T_{z}$ be an analytic family of operators and assume there are positive
constants $M_{0}, M_{1}$ such that, for every $b \in \rr$
%
%
\begin{equation*}
\begin{split}
\| T_{ib} \|_{L^{q_{0}}} \le M_{0} \| f \|_{L^{p_{0}}}, \quad \|
T_{1 + ib} f \|_{L^{q_{1}}} \le M \| f \|_{L^{p_{1}}}
\end{split}
\end{equation*}
%
%
with $1 \le q_{0}, p_{0}, q_{1}, p_{1} \le \infty$. Then, for $z = a + ib \in
D$, $T_{z}$ extends to a bounded operator from $L^{p}$ to $L^{q}$ and
%
%
\begin{equation*}
\begin{split}
\| T_{z} f \|_{L^{q}} \le M_{0}^{1-a} M_{1}^{a} \| f \|_{L^{p}}
\end{split}
\end{equation*}
%
%
where
%
%
\begin{equation*}
\begin{split}
\frac{1}{p} = \frac{1-a}{p_{0}} + \frac{a}{p_{1}}, \quad \frac{1}{q} =
\frac{1-a}{q_{0}} + \frac{a}{q_{1}}.
\end{split}
\end{equation*}
%
%
\label{lem:stein}
\end{lemma}
%
%
%
To apply the lemma, we need some preliminaries. First, recall that for $r \in
\rr$
%
%
\begin{equation*}
\begin{split}
(1 - \p_x^{2})^{r}h(x) 
\overset{\text{it.}}{=} & \int_{\rr} e^{ix
\xi}(1 + \xi^{2})^{r} \wh{h}(\xi) d \xi
\\
\overset{\text{abs.\ conv.}}{=}  & \lim_{j \to \infty} \frac{1}{2 \pi} \int_{\rr}
\int_{\rr} e^{i(x-y)}
\chi( \xi/j) (1 + \xi^{2})^{r}dy d \xi
\end{split}
\end{equation*}
%
%
where $\chi(\xi)$ is a cutoff function symmetric about the origin.\ 
we fix $g \in H^{s}$ with $\| g\|_{H^{s}} = 1$
%
%
\appendix
\section{}
{\bf Proof of \autoref{lem:simp-riesz-thorin}.} 
%
First, we show that $L^q \subset L^p + L^r$. If $f \in L^q$, let $E = \left\{ x:
f(x) <1 \right\}$. Clearly $\|f \chi_{E}\|_{\infty} < 1$, while for for finite $r$, $\| f \chi_{E} \|_{r} \le \|f \chi_{E} \|_{q} \le \| f
\|_{q}$. Similarly, $\| f \chi_{E^c} \|_{p} \le \|f \chi_{E^c} \|_{q} \le \|f\|_{q}$. 
\\
\\
Next, we show that $L^{p} \cap L^{r} \subset L^{q}$ by proving embedding relation
\eqref{embed-relation}.
\begin{case}[$r < \infty$]
Compute
%
\begin{equation}
	\label{simp-riesz-comp-finite}
	\begin{split}
		\| f \|_{q}^{q}
		& = \int_{\rr} | f |^{q}dx
		\\
		& = \int_{\rr} | f |^{\lambda q} | f |^{(1-\lambda)q}, \qquad \lambda \in
		(0,1)
		\\
		& \le \left( \int_{\rr} | f |^{\lambda q \cdot \frac{p}{\lambda q}}dx
		\right)^{\frac{\lambda q}{p}}
		\left( \int_{\rr}| f |^{(1- \lambda)q \left(
		\frac{1}{1-\frac{\lambda q}{p}}
		\right)} dx \right)^{1- \frac{\lambda q}{p}} \qquad \text{(by H{\"o}lder)}
		\\
		& =\| f \|_{p}^{\lambda q} \left( \int_{\rr} | f |^{\frac{(1 -
		\lambda)q}{(1 - \frac{\lambda q}{p})}}dx \right)^{1- \frac{\lambda q}{p}}.
	\end{split}
\end{equation}
%
%
Define $\lambda$ by the relation	%
%
\begin{equation*}
	\begin{split}
		\frac{(1 - \lambda)q}{1 - \frac{\lambda q}{p}} = r
	\end{split}
\end{equation*}
%
%
which by algebra is equivalent to \eqref{embed-param-fin}.
%
%
%
%
%
Substituting into \eqref{simp-riesz-comp-finite} we obtain
%
%
\begin{equation*}
	\begin{split}
		\|f\|_{q}^{q} \le \| f \|_{p}^{\lambda q} \| f \|_{r}^{(1 - \lambda)q}.
	\end{split}
\end{equation*}
%
%
Taking $q$th roots of both sides gives \eqref{embed-relation}.
\end{case}	
%
%
\begin{case}[$r = \infty$]
Compute
%
\begin{equation}
	\label{simp-riesz-comp-infty}
	\begin{split}
		\| f \|_{q}^{q}
		& = \int_{\rr} | f |^{q}dx
		\\
		& = \int_{\rr} | f |^{\lambda q} | f |^{(1-\lambda)q} dx, \qquad \lambda \in
		(0,1)
		\\
		& \le \left( \int_{\rr} | f |^{\lambda q} dx \right)
		\| f \|_{\infty}^{(1 - \lambda)q}.
	\end{split}
\end{equation}
Defining $\lambda$ by the relation
%
%
\begin{equation*}
	\begin{split}
		\frac{1}{q} = \frac{\lambda}{p}
	\end{split}
\end{equation*}
and substituting into the right hand side of \eqref{simp-riesz-comp-infty}, we obtain
%
%
\begin{equation*}
	\begin{split}
		\| f \|_{q}^{q}
		& \le \left( \int_{\rr} | f |^{p} dx \right) \|f \|_{\infty}^{(1 -
		\lambda)q}.
	\end{split}
\end{equation*}
%
%
Taking $q$th roots of both sides gives
%
%
\begin{equation*}
	\begin{split}
		\|f\|_{q} \le \left( \int_{\rr} | f |^{p} dx \right)^{\frac{1}{q}}
		\|f \|_{\infty}^{1 -\lambda} = \|f\|_{p}^{\frac{p}{q}}
		\|f\|_{\infty}^{1 - \lambda} = \|f\|_{p}^{\lambda} \|f\|_{\infty}^{1-
		\lambda}.
	\end{split}
\end{equation*}
%
%
This completes the proof of \autoref{lem:simp-riesz-thorin}.	\qquad \qedsymbol		
\end{case}
%
%
{\bf Proof of \autoref{lem:simp-func}.}
Since the simple functions are dense in $L^p$, we can find a sequence of simple
functions $f_n=\sum a_{j} \chi_{E_j}$, where the $E_j$ are disjoint, that
converge to $f$ in $L^p$. But then we must have $m(E_j) < \infty$, since
%
%
\begin{equation*}
\begin{split}
	\left[ \sum | a_j |^{p} m(E_j) \right]^{1/p} =  \|f\|_{p} < \infty. \qquad
	\qed
\end{split}
\end{equation*}
%
%
%
%
\begin{framed}
\begin{remark}
	\label{rem:density}
	The simple functions with compact support are NOT dense in $L^\infty(\rr)$. As a
	counterexample, consider $f(x) = \sin x$. Then for any compactly supported simple
	function $f_n(x)$ on $\rr$, we have $\| f_n(x) - \sin x \|_{L^\infty(\rr)} \ge 1$.
\end{remark}
\end{framed}
%
%
{\bf Proof of \autoref{lem:three-lines}.} Let $$G(z) = \frac{F(z)}{| M_0^{1-z} | |M_1^z |}.$$ Then on the
line $\text{Re}\, z =0$, we have
%
%
\begin{equation*}
\begin{split}
	| G(z) | = \frac{| F(z) |}{| M_0^{1-z} | M_1^z |} = \frac{| F(z) |}{|
	M_0|^{1-\text{Re}\, z}  |M_1|^{\text{Re}\, z}} = \frac{| F(z) |}{M_0} \le 1.
\end{split}
\end{equation*}
%
%
Similarly, on the line $\text{Re}\, z=1$, we have
\begin{equation*}
\begin{split}
	| G(z) | = \frac{| F(z) |}{| M_0^{1-z} | M_1^z |} = \frac{| F(z) |}{|
	M_0|^{1-\text{Re}\, z}  |M_1|^{\text{Re}\, z}} = \frac{| F(z) |}{M_1} \le 1
\end{split}
\end{equation*}
while on
the line $\text{Re}\, z=\lambda$, $$| G(z) | = \frac{| F(z) |}{| M_0^{1-z} |
|M_1^{z}
|} = \frac{| F(z) |}{| M_0|^{1-\text{Re}\, z}  |M_1|^{\text{Re}\, z}} 
= \frac{| F(z) |}{| M_0|^{1-\lambda}  |M_1|^{\lambda}}  \le 1.$$
%
%
Hence, it suffices to prove \autoref{lem:three-lines} for the case $M_0 = M_1 =
1$. Define 
%
%
\begin{equation*}
\begin{split}
	F_n(z) = F(z) e^{\frac{z^2 -1}{n}}, \quad n \in \zz^+
\end{split}
\end{equation*}
and compute
%
%
\begin{equation}
\label{three-line-comp}
\begin{split}
	| F_n(z) |
	& = | F(z) | e^{\frac{z^{2}-1}{n}} |
	\\
	& = | F(z) | |e^{\frac{(x+iy)^2 -1}{n}}|
	\\
	& = | F(z) |  e^{\frac{x^2}{n}}e^{-\frac{y^2}{n}}e^{-\frac{1}{n}}, \quad n
	\in \zz^+
	\\
	& \le | F(z) |e^{-\frac{y^2}{n}} \quad \text{for} \quad 0 \le \text{Re}\,z \le
	1, \quad n \in \zz^+.
\end{split}
\end{equation}
%
%
Since $| F(z) | \le 1$ on the lines $\text{Re}\, z =0$ and $\text{Re}\, z =1$,
it follows from \eqref{three-line-comp} that $| F_n(z) | \le 1$ on the lines 
$\text{Re}\, z =0$ and $\text{Re}\, z =1$. Furthermore, from the boundedness of
$F(z)$ on the strip $0 \le \text{Re}\, z \le 1$ and \eqref{three-line-comp}, we
see that $F_n(x + iy) \to 0$ as $|y| \to \infty$
uniformly in $x$ for $0 \le x \le 1$. Hence, by the maximum modulus principle,
it follows that 
%
%
%
%
\begin{equation}
\label{pre-limit-estimate}
\begin{split}
	|F_n(z)| \le 1 \quad  \text{in the strip} \quad 0 \le \text{Re}\, z \le 1.
\end{split}
\end{equation}
Observe that from the boundedness of $F(z)$ in the strip $0 \le \text{Re}\, \le
1$, we have \\ $F_n(z) \to F(z)$ uniformly in $z$
in the strip $0 \le \text{Re}\, z\le 1$. Taking the
limit on the left hand side of \eqref{pre-limit-estimate} completes the proof.
\qquad \qedsymbol




%\nocite{*}
\bibliography{/Users/davidkarapetyan/math/bib-files/references.bib}

\end{document}
