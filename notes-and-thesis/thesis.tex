%\ProvidesFile{template.tex}
\documentclass[final]{nddiss2e}
                     % Available options are
                     % (a) draft + 10pt/11pt/12pt + twoadvisors + textrefs
                     % (b) review + noinfo + twoadvisors + textrefs
                     % (c) final + noinfo + twoadvisors + textrefs
%\let\bibsection\relax 

\usepackage{titlesec}
\titleformat{\chapter}[display]{\centering}
{\chaptertitlename\ \thechapter}{20pt}{\uppercase} %capitalize
%chapters--required by thesis format
\usepackage{titletoc}  %capitalize chapters in toc--required by thesis
\titlecontents{chapter}
[0pt]
{\addvspace{1pc}%
\normalfont}%
{\contentsmargin{0pt}%
\normalfont
\makebox[0pt][r]{\huge\thecontentslabel\enspace}%
\uppercase}
{\contentsmargin{0pt}%
\uppercase}
{\titlerule*[1pc]{.}\contentspage}
[\addvspace{0.1pc}]

\setcounter{tocdepth}{1} %must come before secnumdepth--else, pain
\setcounter{secnumdepth}{1} %number only sections, not subsections
\usepackage[T1]{fontenc}
\usepackage{ae,aecompl} 
\usepackage{appendix}
\usepackage{tikz}
\usepackage{fix-cm} %allow arbitrary size fonts
\usepackage{type1ec} %ditto
\usepackage{amsmath}
\usepackage{amsthm}
\usepackage{amssymb}
\usepackage{amsfonts}
\usepackage[showonlyrefs=true]{mathtools} %amsmath extension package
\usepackage{cancel}  %for cancelling terms explicity on pdf
\usepackage{yhmath}   %makes fourier transform look nicer, among other things
\usepackage{framed}  %for framing remarks, theorems, etc.
\usepackage{appendix}
\synctex=1
\usepackage{enumerate} %to change enumerate symbols
%\usepackage[initials]{amsrefs} %for the bibliography; uses cite pkg. Must be loaded after hyperref, otherwise doesn't work properly (conflicts with cref in particular)
\usepackage{hyperref}
\hypersetup{colorlinks=true,
linkcolor=blue,
citecolor=blue,
urlcolor=blue,
}
\usepackage{cleveref} %must be last loaded package to work properly
\renewcommand{\cref}{\Cref}
\newtheorem{theorem}{Theorem}[chapter]
\newtheorem{lemma}[theorem]{Lemma}
\newtheorem{corollary}[theorem]{Corollary}
\newtheorem{claim}[theorem]{Claim}
\newtheorem{prop}[theorem]{Proposition}
\newtheorem{proposition}[theorem]{Proposition}
\newtheorem{no}[theorem]{Notation}
\newtheorem{definition}[theorem]{Definition}
\newtheorem{remark}[theorem]{Remark}
\newtheorem{example}{Example}[section]
\newtheorem {exercise}[theorem] {Exercise}

\newcommand{\ds}{\displaystyle}
\newcommand{\ts}{\textstyle}
\newcommand{\nin}{\noindent}
\newcommand{\rr}{\mathbb{R}}
\newcommand{\nn}{\mathbb{N}}
\newcommand{\zz}{\mathbb{Z}}
\newcommand{\cc}{\mathbb{C}}
\newcommand{\ci}{\mathbb{T}}
\newcommand{\zzdot}{\dot{\zz}}
\newcommand{\wh}{\widehat}
\newcommand{\p}{\partial}
\newcommand{\ee}{\varepsilon}
\newcommand{\vp}{\varphi}
\newcommand{\wt}{\widetilde}
%
%
%
%
%
%\makeatletter \renewenvironment{proof}[1][\proofname] {\par\pushQED{\qed}\normalfont\topsep6\p@\@plus6\p@\relax\trivlist\item[\hskip\labelsep\bfseries#1\@addpunct{.}]\ignorespaces}{\popQED\endtrivlist\@endpefalse} \makeatother%
%%makes proof environment bold instead of italic
\newcommand{\uol}{u^\omega_\lambda}
\newcommand{\lbar}{\bar{l}}
\renewcommand{\l}{\lambda}
\newcommand{\R}{\mathbb R}
\newcommand{\RR}{\mathcal R}
\newcommand{\al}{\alpha}
\newcommand{\ve}{q}
\newcommand{\tg}{{tan}}
\newcommand{\m}{q}
\newcommand{\N}{N}
\newcommand{\ta}{{\tilde{a}}}
\newcommand{\tb}{{\tilde{b}}}
\newcommand{\tc}{{\tilde{c}}}
\newcommand{\tS}{{\tilde S}}
\newcommand{\tP}{{\tilde P}}
\newcommand{\tu}{{\tilde{u}}}
\newcommand{\tw}{{\tilde{w}}}
\newcommand{\tA}{{\tilde{A}}}
\newcommand{\tX}{{\tilde{X}}}
\newcommand{\tphi}{{\tilde{\phi}}}
\renewcommand{\qed}{\qquad \qedsymbol}
\renewcommand{\chaptername}{CHAPTER}
\begin{document}

\frontmatter         % All the items before Chapter 1 go in ``frontmatter''

\title{WELL-POSEDNESS AND ILL-POSEDNESS FOR STRONGLY AND WEAKLY DISPERSIVE
EQUATIONS}            % Title of Work
\author{David Karapetyan, B.S., M.S.}           % Author's name
\work{Dissertation}             % ``Dissertation'' or ``Thesis''
\degprior{}         % All prior degrees
\degaward{Ph.D}         % Degree you're aiming for
\advisor{Alex Himonas}          % Advisor's name
 % \secondadvisor{ } % Second advisor, if used option ``twoadvisors''
\department{Department of Mathematics, University of Notre Dame}       % Name of the department

\maketitle           % The title page is created now

 % \copyrightholder{ } % If you're not the copyright holder
 \copyrightyear{2011}   % If the copyright is not for the current year
\makecopyright      % If not making your work public domain
                       % uncomment out \makecopyright
 % \makepublicdomain   % Uncomment this to make your work public domain

\begin{abstract}

It is shown that the solution map for the hyperelastic rod equation is not 
uniformly continuous on bounded sets of Sobolev spaces with exponent 
greater than 3/2 in the periodic case and greater than 1 in the 
non-periodic case. The proof is based on the method of approximate 
solutions and well-posedness estimates for the solution and its lifespan.
Building upon this work, we also prove H\"older continuity of the data to
solution map for the hyperelastic rod equation for
initial data in Sobolev spaces with eponent greater than 3/2 in
weaker Sobolev topologies with index greater than -1. Following this, we
consider a generalized non-linear Schr\"odinger equation, and prove a
well-posedness result. Lastly, we produce numerics for all these equations.
\end{abstract}
 %                         % Either place the text between begin/end, or
 % \include{abstract}  % put it in a file to be included

% \renewcommand{\dedicationname}{\mbox{}} % Replace \mbox{} if you want
                                           % something else
 %\begin{dedication}
   %To my mama. You have been patient. Be patient a little longer -- this is only the
   %beginning.
   %\\
%To Cynthia -- loyal, and fierce, and loving until the end. To zemer Amelda, whose kindnesses
%give me the strength to never bend, never break.
  %\end{dedication}
 %%                       % Use one of the two choices to add dedication text
  %\include{dedication}

\tableofcontents
%\listoffigures
%\listoftables

 %% \renewcommand{\symbolsname}{newsymname} % Replace ``newsymname'' with
                                           % the name you want, and uncomment
 % \begin{symbols}
 % \end{symbols}
 %                       % Use one of the two choices to add symbols text
 % \include{symbols}

 %% \renewcommand{\prefacename}{ } % If you want another Preface name, add
                                   % something else, and uncomment
 % \begin{preface}
 % \end{preface}
 %                       % Use one of the two choices to add preface text
 % \chapter*{Preface}
This monograph is based on the author's work during graduate study at the University of Notre Dame from August 2007-March 2012.
\aufm{David Karapetyan}



 %% \renewcommand{\acknowledgename}{ } % If you want another Acknowledgement name
                                       % add something else, and uncomment
 % \begin{acknowledge}
 % \end{acknowledge}
 %                       % Use one of the two choices to add acknowledge text
 % \include{acknowledgement}

\mainmatter
 % Place the text body here.
\chapter{Non-Uniform Dependence for the
Hyperelastic Rod Equation}
%\begin{abstract}
%The solution map for the Hyperelastic Rod equation is not uniformly continuous
	%from any bounded set of $H^s$ into $C([-T, T]; H^s)$
	%for $s>3/2$ in the periodic case and for $s>1$ in the non-periodic case.
	%The proof is based on the method of approximate solutions.
%\end{abstract}
%\maketitle
%\markboth{Non-Uniform Dependence for the Hyperelastic Rod Equation}{David Karapetyan}
%\parindent0in
%\parskip0.1in
%\end{titlepage}
%%%%%%%%%%%%%%%%%%%%%%%%
%
%      introduction
%
%%%%%%%%%%%%%%%%%%%%%%%%
\section{Introduction}
%
We consider the periodic initial value problem for
the hyperelastic-rod (HR)  equation
\begin{equation}
	\label{hr}
	\p_t u
	-
	\p_t \p_x^2 u
	+
	3u\p_x u
	=
	\gamma \big (
	2\p_x u \p_x^2 u
	+
	u \p_x^3 u
	\big ),
\end{equation}
\begin{equation}
	\label{hr-data} u(x, 0) = u_0 (x),
	\quad x  \in \ci, \text{  or  } \rr \quad t \in \rr,
\end{equation}
where $\gamma$ is a constant. Equation \ref{hr} was first
derived by Dai in \cite{Dai_1998_Model-equations} as a model for finite-length and
small-amplitude axial deformation waves in thin cylindrical
rods composed of a compressible isotropic hyperelastic
material. Local well-posedness and blow-up criteria for
solutions were established by  Zhou in \cite{Liu_2008_Blow-up-phenome}. Orbital
stability of a class of solitary waves for this equation was
proved in \cite{Constantin_2000_Stability-of-a-} by Constantin and Strauss.
\\
\\
Motivated by the work of Himonas and Kenig \cite{Himonas:2009fk} and
Himonas and Misiolek \cite{Himonas:2005kx} we prove that 
the solution map for the HR equation is not uniformly
continuous in both the periodic and non-periodic case.
%
\begin{theorem}
	\label{hr-non-unif-dependence}
	For $s>3/2$ in the periodic case and for $s>1$ in
	the non-periodic case, the flow map $u_0 \to u(t)$ of the
	Cauchy-problem \eqref{hr}-\eqref{hr-data} is not uniformly continuous
	from any bounded set of $H^s$ into $C([-T, T]; H^s)$.
	More precisely, in both cases there exist two sequences of solutions $u_n(t)$
	and $v_n(t)$ in $C([-T, T]; H^s)$ such that
	%
	%
	\begin{equation}
		\label{h-s-bdd}
		\| u_n(t)  \|_{H^s}
		+
		\| v_n(t)  \|_{H^s}
		\lesssim
		1,
	\end{equation}
	%
	\begin{equation}
		\label{zero-limit-at-0}
		\lim_{n\to\infty}
		\|
		u_n(0)
		-
		v_n(0)
		\|_{H^s}
		=
		0,
	\end{equation}
	%
	%
	and
	%
	%
	\begin{equation}
		\label{bdd-away-from-0}
		\liminf_{n\to\infty}
		\|
		u_n(t)
		-
		v_n(t)
	\|_{H^s}
		\gtrsim
		\sin ( \gamma t),
		\quad
		|t|\le T.
	\end{equation}
	%
	%
	\end{theorem}
	For $\gamma \neq 3$ this result was proved by Olson in
  \cite{Olson_2006_The-initial-val} using
	traveling wave solutions. In this paper
	we will eliminate the restriction on $\gamma$ using approximate solutions to the HR
	equation. In Section 1 we introduce the approximate
	solutions we will be using, and derive a functional representation of
	the error in Section 2. In Section 3 we find an upper bound for the
	error. This leads to a crucial lemma in Section 4 that gives a
	decaying bound for the difference of approximate and actual solutions.
	In Section 5 the lemma is used in an interpolation which allows us to 
	conclude the non-uniform dependence on initial
	data of solutions to the HR equation, independent of the choice of 
	$\gamma$.
\section{The non-periodic case}
We consider the Cauchy problem for the Hyperelastic Rod Equation (HR)
\begin{equation}
	\begin{split}
		\p_t u + \gamma u \p_x u + \p_x \left( 1 - \p_x^2
		\right)^{-1}  \left[ \frac{3-\gamma}{2}u^2 +
		\frac{\gamma}{2} \left( \p_x u \right)^2
		\right] = 0,
		\label{apple1'}
	\end{split}
\end{equation}
%
\begin{equation}
	\begin{split}
		u(x,0) = u_0(x), \; \; x \in \rr, \; \; t \in \rr. 
		\label{apple2'}
	\end{split}
\end{equation}
Our approximate solutions $u^{\omega, \lambda} = u^{\omega,
\lambda}(x,t)$ to \eqref{apple1'}-\eqref{apple2'} will
consist of a low frequency and a high frequency part,
i.e.
\begin{equation}
	\label{apple1}
	u^{\omega,\lambda} = u_\ell + u^h.
\end{equation}
The high frequency part is given by 
\begin{equation}
	\begin{split}
		u^h = u^{h,\omega,\lambda}(x,t) =
		\lambda^{-\frac{\delta}{2} -s}
		\phi \left (\frac{x}{\lambda^\delta}\right )
		\cos(\lambda x - \gamma \omega t)
	\end{split}
\end{equation}
where $\phi$ is a $C^\infty$ cutoff function such that
\begin{equation*}
	\phi = 
	\begin{cases}
		1, &\text{if $|x|<1$;} \\
		0, &\text{if $|x| \ge 2$.} 
	\end{cases}
\end{equation*}
By Theorem \ref{thm:HR_existence_continous_dependence} ,
we let the low frequency part $u_\ell = u_{l,
\omega, \lambda}(x,t)$ be the unique solution to the HR equation
\begin{equation}
	\p_t u_\ell + \gamma u_\ell \p_x u_\ell + \p_x (1- \p_x^2)^{-1}  \left[
	\frac{3- \gamma}{2}(u_\ell)^2 + \frac{\gamma}{2}\left( \p_x u_\ell
	\right)^2 \right] = 0
	\label{apple1*}
\end{equation}
with initial data
\begin{equation}
	u_\ell(x,0) = \omega \lambda^{-1} \tilde{\phi} \left(
	\frac{x}{\lambda^{\delta}}
	\right), \quad x \in \rr, \quad t \in \rr
	\label{apple1**}
\end{equation}
where $\tilde{\phi}$ is a $C^{\infty}_0(\rr)$ function such that
\begin{equation}
	\label{apple1***}
	\tilde{\phi}(x) = 1 \; \;  \text{if} \; \;
	x \in \text{supp} \; \phi.
\end{equation}
Let $\Lambda^{-1} = \p_x (1 - \p_x^2)^{-1} $. Substituting the
approximate solution $u^{\omega, \lambda} = u_\ell + u^h$ into the HR
equation, and recalling that $u_\ell$ is a solution to the
HR equation, we obtain
\begin{equation}
	\begin{split}
		E 
		& = \p_t u^h + \gamma u_\ell \p_x u^h + \gamma u^h \p_x u_\ell +
		\gamma u^h \p_x u^h
		\\
		& + \Lambda^{-1} \left\{ \frac{3-\gamma}{2}\left[ \left( u^h
		\right)^2 + 2u_\ell u^h
		\right]+ \frac{\gamma}{2}\left[ \left( \p_x u^h \right)^2 + 2
		\p_x u_\ell \p_x u^h\right] \right\}.
		\label{apple2star}
	\end{split}
\end{equation}
Using a straightforward calculation of derivatives, and
noting that $\tilde{\phi} (x) = 1$ for $x \in \text{supp} \;
\phi$,
we deduce
\begin{equation}
	\begin{split}
		\p_t u^h + \gamma u_\ell \p_x u^h 
		& = \gamma \lambda\left[ u_\ell(x,0) - u_\ell(x,t)
		\right]\lambda^{-\frac{\delta}{2}-s} \phi\left(
		\frac{x}{\lambda^\delta}
		\right) \sin(\lambda x - \gamma \omega t)
		\\
		& + \gamma u_\ell(x,t) \cdot \lambda^{-\frac{3\delta}{2}-s}
		\phi'\left( \frac{x}{\lambda^\delta} \right)\cos\left( \lambda
		x - \gamma \omega t
		\right).
						 \label{apple5}
					 \end{split}
				 \end{equation}
				 Therefore, applying \eqref{apple5} to \eqref{apple2star}, we see that the error
				 $E$ of our approximate solution is given by
				 \begin{equation*}
					 E=E_1 + E_2 + \dots + E_8
				 \end{equation*}
				 where
				 \begin{equation}
					 \label{all_errors_together}
					 \begin{split}
						  E_1 & = \gamma \lambda \left[ u_\ell(x,0) - u_\ell(x,t)
						 \right] \lambda^{-\frac{\delta}{2}-s} \phi\left(
						 \frac{x}{\lambda^ \delta}
						 \right)\sin(\lambda x - \gamma \omega t)
						 \\
						 E_2 & = \gamma u_\ell(x,t) \cdot \lambda^{-\frac{3\delta}{2}-s}
						 \phi'\left( \frac{x}{\lambda^\delta} \right)\cos\left( \lambda
						 x - \gamma \omega t
						 \right)
						 \\
						 E_3 & = \gamma u^h \p_x u_\ell, \; \; E_4 = \gamma u^h \p_x u^h
						 \\
						 E_5 & = \Lambda^{-1}\left[ \frac{3-\gamma}{2} \left(
						 u^h \right)^2 
						 \right], \; \; E_6 = \Lambda^{-1}
						 \left[ (3- \gamma)u_\ell u^h \right]
						 \\
						 E_7 & = \Lambda^{-1} \left[ \frac{\gamma}{2} \left(
						 \p_x u^h \right)^2 \right ], \; \;
						 E_8 = \Lambda^{-1} \left[ \gamma \p_x u_\ell \p_x u^h \right]
						 .
						 \end{split}
				 \end{equation}
				 %
				 %
				 %
				 \subsection{An Upper Bound in $H^1$ For the Error of the Approximate
				 Solutions}
         For estimating the $H^1(\ci)$ norm of $u^h$, we need the
				 following result, whose proof can be found in
         \cite{Himonas:2009fk}:
				  \begin{lemma}
					 \label{applea}
					 Let $\psi \in S(\rr)$, $\alpha \in \rr$. Then for $s \ge 0$ we have
					 \begin{equation}
						 \begin{split}
							 \lim_{\lambda \to \infty} \lambda^{-\frac{\delta}{2}-s}
							 \|\psi \left( \frac{x}{\lambda^\delta} \right)\cos(\lambda
							 x - \alpha) \|_{H^s(\rr)} = \frac{1}{\sqrt
							 2}\|\psi\|_{L^2(\rr)}.
							 \label{apple6}
						 \end{split}
					 \end{equation}
					 Relation \eqref{apple6} remains true if $\cos$ is
					 replaced by $\sin$.
				 \end{lemma}
				 Next, we provide an upper bound for the $H^1(\ci)$ norm of
				 $u_l$; details can again be found in
         \cite{Himonas:2009fk}:
				 %
				\begin{lemma}
					\label{appleb}
					Let $0<\delta<2$, with $\omega$ belonging to a bounded
					subset of $\rr$. Then the initial value problem
					\eqref{apple1*}-\eqref{apple1**} has a unique solution
					$u_\ell \in C\left( [0,T], H^s(\rr) \right)$ for all $s
					\ge 0$, which 
					satisfies
					\begin{equation}
						\label{apple10'}
						\|u_\ell(t)\|_{H^s(\rr)} \le c_s \lambda^{-1 +
						\frac{\delta}{2}}, \quad |t| \le T.
					\end{equation}
				\end{lemma}
								We will also need the following:
\begin{lemma}
	\label{applec}
	For any $f,g \in L^2(\rr)$,
	\begin{equation*}
		\|fg\|_{H^1(\rr)} \le \sqrt{2} \|f\|_{C^1(\rr)} \|g\|_{H^1(\rr)}.
	\end{equation*}
\end{lemma}
%
We are now prepared to estimate the $H^1$ norms of each $E_i$.
\subsection{Estimating the $H^1$ norm of $\hyperref[all_errors_together]{E_1}$.} We have
\begin{equation*}
	\begin{split}
		\|E_1\|_{H^1(\rr)}
		& = \| \gamma \lambda \left[ u_\ell(x,0) - u_\ell(x,t) \right]
		\lambda^{-\frac{\delta}{2}-s} \phi\left( \frac{x}{\lambda^\delta}
		\right ) \sin (\lambda x - \gamma \omega t )\|_{H^1(\rr)}
		\\
		& = |\gamma| \lambda^{1 -\frac{\delta}{2} -s } \|\left[ u_\ell(x,0) - u_\ell(x,t)
		\right] \phi\left( \frac{x}{\lambda^\delta} \right )
		\sin\left( \lambda x - \gamma \omega t
		\right) \|_{H^1(\rr)}.
	\end{split}
\end{equation*}
Applying Lemma \ref{applec}, we obtain
\begin{equation}
	\begin{split}
		\|E_1\|_{H^1(\rr)} \le |\gamma| \lambda^{1 - \frac{\delta}{2} -s } \|\phi
		\left( \frac{x}{\lambda^\delta} \right) \sin (\lambda x - \gamma \omega t)
		\|_{C^1(\rr)} \|[u_\ell (x,0) - u_\ell (x,t) ] \|_{H^1(\rr)}.
		\label{apple14}
	\end{split}
\end{equation}
We now estimate the right hand side of \eqref{apple14} in pieces. For the first piece, we
have
\begin{equation*}
	\begin{split}
		& \|\phi \left( \frac{x}{\lambda^\delta} \right) \sin (\lambda x - \gamma \omega t)
		\|_{C^1(\rr)} 
		\\
		&
		\le \|\phi \left( \frac{x}{\lambda^\delta} \right) \|_{L^\infty(\rr)} + \lambda
		\|\phi\left( \frac{x}{\lambda^\delta} \right)\|_{L^\infty(\rr)} +
		\lambda^{-\delta} \|\phi'\left( \frac{x}{\lambda^\delta}
		\right)\|_{L^\infty(\rr)}
	\end{split}
\end{equation*}
which gives
\begin{equation}
	\begin{split}
		\|\phi\left( \frac{x}{\lambda^\delta} \right) \sin(\lambda x - \gamma \omega t)
		\|_{C^1(\rr)}
		\lesssim \lambda.
		\label{apple15}
	\end{split}
\end{equation}
For the next piece of \eqref{apple14}, we observe that the fundamental theorem
of calculus gives
\begin{equation*}
	u_\ell(x,t) - u_\ell(x,0) = \int_0^t \p_\tau u_\ell(x,\tau) \; d \tau.
\end{equation*}
Hence
\begin{equation}
	\begin{split}
		\|u_\ell(x,t) - u_\ell(x,0)\|_{H^1(\rr)}
		& = \left \| \int_0^t \p_\tau
		u_\ell(x,\tau) \; d \tau \right \|_{H^1(\rr)}
		\\
		& \le  \int_0^t \|\p_\tau u_\ell (x,\tau) \|_{H^1(\rr)} \; d \tau.
		\label{apple100}
	\end{split}
\end{equation}
We want to estimate the right hand side of \eqref{apple100}. Recalling
\eqref{apple1'}, we have
\begin{equation}
	\label{apple101}
	\begin{split}
		\|\p_\tau u_\ell(x,\tau) \|_{H^1(\rr)}
		& =  \|-\gamma u_\ell \p_x u_\ell + \Lambda^{-1} \left[
		\frac{3-\gamma}{2}(u_\ell)^2 + \frac{\gamma}{2} \left( \p_x u_\ell \right)^2
		\right] \|_{H^1(\rr)}
		\\
		& \le \|\gamma u_\ell \p_x u_\ell \|_{H^1(\rr)} + \|\Lambda^{-1} \left[
		\frac{3-\gamma}{2} (u_\ell)^2 + \frac{\gamma}{2} \left( \p_x u_\ell \right)^2
		\right] \|_{H^1(\rr)}.
	\end{split}
\end{equation}
Applying the algebra property of Sobolev spaces, we obtain
\begin{equation*}
	\begin{split}
		\|\gamma u_\ell \p_x u_\ell \|_{H^1(\rr)} &
		= \|\gamma \p_x (u_\ell)^2 \|_{H^1(\rr)}
		\\
		& \le |\gamma| \cdot \| (u_\ell)^2 \|_{H^2(\rr)}
		\\
		& \lesssim \|u_\ell\|_{H^2(\rr)}^2
	\end{split}
\end{equation*}
which by \eqref{apple10'} reduces to
\begin{equation}
	\begin{split}
		\|\gamma u_\ell \p_x u_\ell \|_{H^1(\rr)} \lesssim \lambda^{-2 + \delta}.
		\label{apple102}
	\end{split}
\end{equation}
Next, note that, for any $u \in L^2(\rr)$, we
have
\begin{equation}
	\begin{split}
		\|\Lambda^{-1} u \|_{H^1(\rr)} 
		\le \|u\|_{L^2(\rr)}.
		\label{apple27}
	\end{split}
\end{equation}
Hence, applying \eqref{apple27}, the algebra property of Sobolev spaces,
and the Sobolev Imbedding Theorem, we obtain
\begin{equation*}
	\begin{split}
		\|\Lambda^{-1} \left[ \frac{3-\gamma}{2}u^2 +
		\frac{\gamma}{2}\left( \p_x u \right)^2 \right] \|_{H^1(\rr)}
		& \lesssim \|u\|_{H^2(\rr)}^2
	\end{split}
\end{equation*}
which by \eqref{apple10'} reduces to 
\begin{equation}
	\begin{split}
		\|\Lambda^{-1} \left[ \frac{3-\gamma}{2}u^2 +
		\frac{\gamma}{2}\left( \p_x u \right)^2 \right] \|_{H^1(\rr)}
		\lesssim \lambda^{-2 + \delta}.
		\label{apple104}
	\end{split}
\end{equation}
Substituting \eqref{apple102} and \eqref{apple104} into the right hand side of
\eqref{apple101}, and recalling \eqref{apple100}, we obtain
\begin{equation}
	\begin{split}
		\|u_\ell(x,t) - u_\ell(x,0)\|_{H^1(\rr)} \lesssim \lambda^{-2 + \delta}.
		\label{apple105}
	\end{split}
\end{equation}
By estimates \eqref{apple14}, \eqref{apple15}, and \eqref{apple105}, we conclude that 
\begin{equation}
	\begin{split}
		\|E_1\|
		& \lesssim \lambda^{\frac{\delta}{2}-s}.
		\label{apple106}
	\end{split}
\end{equation}
%
\subsection{Estimating the $H^1$ norm of $\hyperref[all_errors_together]{E_2}$.} Applying Lemma \ref{applec}, we have
\begin{equation}
	\begin{split}
		\|E_2\|_{H^1(\rr)} 
		& = \gamma \lambda^{-\frac{3 \delta}{2} -s } \|u_\ell(x,t) \cdot
		\phi'\left( \frac{x}{\lambda^\delta} \right) \cos (\lambda x - \gamma \omega t)
		\|_{H^1(\rr)}
		\\
		& \le c_s \gamma \lambda^{\frac{-3 \delta}{2} -s } \|u_\ell(x,t) \|_{H^1(\rr)}
		\|\phi'\left( \frac{x}{\lambda^\delta} \right )
		\cos(\lambda x - \gamma \omega t 
		\|_{C^1(\rr)}.
		\label{apple18}
	\end{split}
\end{equation}
We note that
\begin{equation*}
	\begin{split}
		& \|\phi'\left( \frac{x}{\lambda^\delta} \right) \cos(\lambda x - \gamma \omega t)
		\|_{C^1(\rr)}
		\\
		& \le \|\phi' \left( \frac{x}{\lambda^\delta} \right)\|_{L^\infty(\rr)} +
		\lambda \|\phi'\left( \frac{x}{\lambda^\delta} \right)\|_{L^\infty(\rr)}
		+ \lambda^{-\delta} \|\phi''\left( \frac{x}{\lambda^\delta} \right)
		\|_{L^\infty(\rr)}
	\end{split}
\end{equation*}
which gives
\begin{equation}
	\begin{split}
		\|\phi'\left( \frac{x}{\lambda^\delta} \right) \cos(\lambda x - \gamma \omega t)
		\|_{C^1(\rr)} \lesssim \lambda.
		\label{apple19}
	\end{split}
\end{equation}
Applying estimates \eqref{apple19} and \eqref{apple10'} to \eqref{apple18}, we obtain
\begin{equation*}
	\begin{split}
	\label{apple20}
	\|E_2\|_{H^1(\rr)} \lesssim \lambda^{-\delta -s }.
\end{split}
\end{equation*}
%
%
%
%
\subsection{Estimating the $H^1$ norm of $\hyperref[all_errors_together]{E_3}$.} 
By Lemma \ref{applec}, we deduce
\begin{equation}
	\begin{split}
		\|\gamma u^h \p_x u_\ell \| \le \sqrt{2} |\gamma| \cdot \|u^h\|_{C^1(\rr)}
		\|u_\ell\|_{H^1(\rr)}.
		\label{apple21}
	\end{split}
\end{equation}
Now, note that
\begin{equation}
	\begin{split}
		\|u^h\|_{L^\infty(\rr)} 
		& = \lambda^{-\frac{\delta}{2} -s } \|\phi\left( \frac{x}{\lambda^\delta}
		\right) \cos \left( \lambda x - \gamma \omega t \right) \|_{L^\infty(\rr)}
		\\
		& \lesssim \lambda^{-\frac{\delta}{2} -s }.
		\label{apple22}
	\end{split}
\end{equation}
and 
\begin{equation}
	\begin{split}
		& \|\p_x u^h \|_{L^\infty(\rr)}
		\\
		& = \lambda^{-\frac{\delta}{2}-s} \|\phi\left(
		\frac{x}{\lambda^\delta}
		\right) \cdot -\lambda \sin(\lambda x - \gamma \omega t) + \lambda^{-\delta}
		\phi'\left( \frac{x}{\lambda^\delta}\right) \cos(\lambda x - \gamma \omega
		t) \|_{L^\infty(\rr)}
		\\
		& \lesssim \lambda^{1 - \frac{\delta}{2} -s }.
		\label{apple23}
	\end{split}
\end{equation}
Therefore, from \eqref{apple22} and \eqref{apple23} it follows that
\begin{equation}
	\begin{split}
		\|u^h\|_{C^1(\rr)} \lesssim \lambda^{-\frac{\delta}{2} -s } + \lambda^{1
		-\frac{\delta}{2} -s}
		\approx \lambda^{1- \frac{\delta}{2} -s}.
		\label{apple24}
	\end{split}
\end{equation}
Substituting estimates \eqref{apple24} and  \eqref{apple10'} into \eqref{apple21} we obtain
\begin{equation}
	\begin{split}
		\|\gamma u^h \p_x u_\ell \|_{H^1(\rr)} \lesssim \lambda^{-s}.
		\label{apple24'}
	\end{split}
\end{equation}
\subsection{Estimating the $H^1$ norm of $\hyperref[all_errors_together]{E_4}$.} Applying Lemma \ref{applec} we have
\begin{equation}
	\begin{split}
		\|\gamma u^h \p_x u^h\|_{H^1(\rr)} \lesssim \|u^h\|_{C^1(\rr)}
		\|u^h\|_{H^1(\rr)}.
		\label{apple25}
	\end{split}
\end{equation}
Substituting in \eqref{apple24}, and recalling that $\|u^h\|_{H^1(\rr)} \approx 1$ by Lemma
\ref{applea}, we obtain
\begin{equation}
	\begin{split}
		\|u^h \p_x u^h \|_{H^1(\rr)} \lesssim \lambda^{1-\frac{\delta}{2}-s}.
		\label{apple26}
	\end{split}
\end{equation}
%
%
\subsection{Estimating the $H^1$ norm of $\hyperref[all_errors_together]{E_5}$.}
Applying \eqref{apple27}, we obtain
\begin{equation}
	\begin{split}
		\|E_5\|_{H^1(\rr)}
		& = \|\Lambda^{-1}\left[ \frac{3-\gamma}{2}(u^h)^2
		\right]\|_{H^1(\rr)}
		\\
		& \lesssim \|u^h\|_{L^\infty(\rr)} \|u^h\|_{L^2(\rr)}.
		\label{apple28}
	\end{split}
\end{equation}
Substituting \eqref{apple6} and \eqref{apple22} into \eqref{apple28}, we conclude that
\begin{equation}
	\begin{split}
		\|E_5\|_{H^1(\rr)} \lesssim \lambda^{-\frac{\delta}{2}-s}.
		\label{apple29}
	\end{split}
\end{equation}
%
%
%
%
\subsection{Estimating the $H^1$ norm of $\hyperref[all_errors_together]{E_6}$.} Applying \eqref{apple27}, we obtain
\begin{equation}
	\begin{split}
		\|E_6\|_{H^1(\rr)} 
		& = \|\Lambda^{-1} \left[ (3 -\gamma) u_\ell u^h \right]\|_{H^1(\rr)}
		\\
		& \lesssim \|u_\ell\|_{L^2(\rr)} \|u^h\|_{L^\infty(\rr)}.
		\label{apple30}
	\end{split}
\end{equation}
which by Lemma \ref{appleb} and \eqref{apple22} reduces to
\begin{equation}
	\begin{split}
		\|E_6\|_{H^1(\rr)} \lesssim \lambda^{-1-s}.
		\label{apple31}
	\end{split}
\end{equation}
%
%
%
%
\subsection{Estimating the $H^1$ norm of $\hyperref[all_errors_together]{E_7}$.} Applying \eqref{apple27}, we obtain
\begin{equation}
	\begin{split}
		\|E_7\|_{H^1(\rr)} 
		& = \|\Lambda^{-1} \left[ \frac{\gamma}{2}\left( \p_x u \right)^2
		\right]\|_{H^1(\rr)}
		\\
		& \lesssim  \|\p_x u^h\|_{L^\infty(\rr)} \|u^h\|_{H^1(\rr)}.
		\label{apple32}
	\end{split}
\end{equation}
which by Lemma \ref{applea} and \eqref{apple23} reduces to
\begin{equation}
	\begin{split}
		\|E_7\|_{H^1(\rr)} \lesssim \lambda^{1-\frac{\delta}{2}-s}.
		\label{apple33'}
	\end{split}
\end{equation}
%
%
%
%
\subsection{Estimating the $H^1$ norm of $\hyperref[all_errors_together]{E_8}$.} Applying \eqref{apple27}, we have
\begin{equation}
	\begin{split}
		\|E_8\|_{H^1(\rr)}
		& = \|\Lambda^{-1}\left[ \gamma \p_x u_\ell \p_x u^h \right]\|_{H^1(\rr)}
		\\
		& \lesssim \|u_\ell\|_{H^2(\rr)} \|\p_x u^h\|_{L^\infty(\rr)}.
		\label{apple33}
	\end{split}
\end{equation}
which by Lemma \ref{appleb} and \eqref{apple23} reduces to
\begin{equation}
	\begin{split}
		\|E_8\|_{H^1(\rr)} \lesssim \lambda^{-s}.
		\label{apple34}
	\end{split}
\end{equation}
Collecting all our estimates for the $E_i$ and recalling that we have assumed
$1<\delta<2$, we obtain
\begin{equation*}
	\begin{split}
		\|E\|_{H^1(\rr)}
		 \lesssim \lambda^{\frac{\delta}{2} -s }, \qquad \lambda >>1.
	\end{split}
\end{equation*}
We now summarize our result:
%
%
\begin{proposition}
	Let $1<\delta<2$. Then for $s > 1$, bounded $\omega$, and
	$\lambda >>1$ we are assured the decay of the error $E$ of the
	approximate solutions to the HR equation; specifically
	\begin{equation}
		\begin{split}
			\|E(t)\|_{H^1(\rr)} \lesssim \lambda^{-r_s}
			\label{apple35}
		\end{split}
	\end{equation}
	where
	\begin{equation}
		\begin{split}
			r_s = s - \frac{\delta}{2}.   
			\label{appler_s}
		\end{split}
	\end{equation}
\end{proposition}
%
%
%
%
%
\subsection{Estimating the $H^1$ norm of the difference 
between approximate and actual
	solutions.}
	We wish now to estimate the difference between approximate and actual solutions to
	the HR i.v.p with common initial data $u_0 \in H^s$. Let
	$u_{\omega,\lambda}(x,t)$ be the unique solution to the HR equation
	with initial data $u^{\omega,\lambda}(x,0)$; that is,
	$u_{\omega,\lambda}$ solves the initial value problem
	\begin{align}
		& \p_t u_{\omega,\lambda} + \gamma u_{\omega,\lambda} \p_x u_{\omega,\lambda} + \Lambda^{-1} \left[
		\frac{3- \gamma}{2}\left( u_{\omega,\lambda} \right)^2 + \frac{\gamma}{2}\left(
		\p_x u_{\omega,\lambda} \right)^2
		\right], \; \; x\in \rr, \; \; t \in \rr,
		\label{apple50}
		\\
		& u_{\omega,\lambda} = u^{\omega,\lambda}(x,0)=\omega \lambda^{-1}
		\tilde{\phi} \left( \frac{x}{\lambda^\delta} \right)
		+ \lambda^{-\frac{\delta}{2} -s}
		\phi\left( \frac{x}{\lambda^\delta} \right) \cos(\lambda x).
		\label{apple41}
	\end{align}
	%
%
%
Letting $v = u^{\omega,\lambda} - u_{\omega,\lambda}$, we will now prove the following critical lemma:
\begin{lemma}
	\label{applelem:bound_for_difference-of-approx-and-actual-soln}
	For $s > 1$ and $1<\delta<2$ we have 
			\begin{equation} 
				\|
				v(t)
				\|_{H^1(\rr)}
				\doteq
				\label{applediffer-H1-est} 
				\|
				u^{\omega,\lambda}(t) 
				- 
				u_{\omega,\lambda}(t)
				\|_{H^1(\rr)}
				\lesssim 
				n^{-r_s}, 
				\quad
				|t| \le T
			\end{equation}
			where $r_s$ is defined as in \eqref{appler_s}.
			%
			\end{lemma}
      \begin{proof} Our plan will be to calculate an energy-estimate for $v$.
			To do so, we must first calculate $\p_t v$. Subtracting $\p_t
			u_{\omega, \lambda}$ from $\p_t u^{\omega,\lambda}$ and
			recalling that $u_{\omega,t}$ is a solution to the HR Cauchy
			problem \eqref{apple50}-\eqref{apple41},
			and $u_{\omega,\lambda}$ is an approximate solution, we obtain
			\begin{equation*}
				\begin{split}
					\p_t v 
					& = E + \gamma(v \p_x v - v \p_x u^{\omega,\lambda} - u^{\omega,\lambda} \p_x v) 
					\\
					& + \p_x \left( 1 - \p_x^2 \right)^{-1}  \left[ \frac{3-
					\gamma}{2}v^2 + \frac{\gamma}{2}\left( \p_x v \right)^2 - \left(
					3 - \gamma \right)u^{\omega,\lambda} v -
					\gamma \p_x u^{\omega,\lambda} \p_x v \right].
				\end{split}
			\end{equation*}
			It follows immediately that
		\begin{equation}
			\label{applev-dtv-pseudo-functional-equality}
			\begin{split}
			v(1-\p_x^2)\p_t v &= v(1- \p_x^2)E + v\gamma(1- \p_x^2)(v\p_x v 
			- v\p_x u^{\omega,\lambda} -
			u^{\omega,\lambda} \p_x v)
			\\
			&+ v\p_x \left[ \frac{3-\gamma}{2}v^2 + \frac{\gamma}{2}(\p_x v)^2 -
			(3-\gamma)u^{\omega,\lambda} v - \gamma \p_x u^{\omega,\lambda} \p_x v \right].
		\end{split}
	\end{equation}
	Applying the equality $v\p_t v = v(1-\p_x^2) \p_t v + v\p_x^2 \p_t v$ to
	\eqref{applev-dtv-pseudo-functional-equality}, we obtain
	\begin{equation*}
		\begin{split}
		v \p_t v &= v(1- \p_x^2)E + v\gamma(1- \p_x^2)(v\p_x v - v\p_x u^{\omega,\lambda} -
			u^{\omega,\lambda} \p_x v)
			\\
			&+ v\p_x \left[ \frac{3-\gamma}{2}v^2 + \frac{\gamma}{2}(\p_x v)^2 -
			(3-\gamma)u^{\omega,\lambda} v - \gamma \p_x u^{\omega,\lambda} \p_x v
			\right] + v\p_x^2 \p_t v.
		\end{split}
	\end{equation*}
	Hence
	\begin{equation}
		\label{appleenergy-est}
		\begin{split}
			&\frac{1}{2} \frac{d}{dt} \|v\|_{H^1(\rr)}^2  
			\\
		& =  \int_{\rr} \left[ v(1-\p_x^2)E \right]dx
		- \gamma \int_{\rr} \left[ v(1-\p_x^2)(v\p_x u^{\omega,\lambda} + u^{\omega,\lambda} \p_x v) \right]dx
		\\
		&- \int_{\rr}\left[ \left( 3-\gamma \right)v \p_x\left( u^{\omega,\lambda}v \right) + \gamma v
		\p_x \left( \p_x u^{\omega,\lambda} \p_x v \right)\right]dx
		\\
		&+  \int_{\rr}
		\left[ \gamma v \left( 1-\p_x^2 \right)\left( v \p_x v \right) + v
		\p_x \left( \frac{3-\gamma}{2} v^2 + \frac{\gamma}{2}\left( \p_x v \right)^2
		\right) \right . +  v \p_x^2 \p_t v + \p_x v \p_t \p_x v\bigg]dx.
	\end{split}
\end{equation}
We compute the last term of \eqref{appleenergy-est} first:
\begin{equation*}
	\begin{split}
	&  \int_{\rr} \bigg[ \gamma v \left( 1-\p_x^2 \right)(v\p_x v) + v \p_x\left(
	\frac{3-\gamma}{2}v^2 + \frac{\gamma}{2}\left( \p_x v \right )^2 \right)
	+ v\p_x^2 \p_t v + \p_x v \p_t \p_x v
	\bigg]dx
	\\
	& =  \int_{\rr} \left[ 3v^2 \p_x v - \gamma v^2 \p_x^3 v - 2 \gamma v \p_x v \p_x^2
	v + v \p_x^2 \p_t v + \p_x v \p_t \p_x v \right]dx
	\\
	&=  \int_{\rr} \left[ \p_x (v^3) - \gamma \p_x (v^2 \p_x^2 v) + \p_x\left( v \p_t
	\p_x v
	\right) \right]dx
	\\
	& = 0.
\end{split}
\end{equation*}
Therefore			
\begin{equation}
	\label{appleenergy-estimate-simplified}
	\begin{split}
		\frac{1}{2}	\frac{d}{dt} \|v(t)\|_{H^1(\rr)}^2
		& =  \int_{\rr}
		 v\left( 1-\p_x^2
	\right)E \; dx
	\\
	&-  \gamma  \int_{\rr}  v\left( 1-\p_x^2 \right)\left( v \p_x u^{\omega,\lambda}
	+ u^{\omega,\lambda} \p_x v
	\right) \; dx
	\\
	& -  \int_{\rr} \left[ \left( 3-\gamma \right)v \p_x \left( u^{\omega,\lambda}v \right) + \gamma v
	\p_x \left( \p_x u^{\omega,\lambda} \p_x v \right)\right]dx.
\end{split}
\end{equation}
We now estimate the right-hand-side of \eqref{appleenergy-estimate-simplified}:
%
\begin{equation}
	\begin{split}
		\label{applefirst_piece}
	\left |\int_{\rr} \left [v (1- \p_x^2)E \right ] dx \right |
	& \lesssim
	\left( \|v\|_{L^2(\rr)}
	\|E\|_{L^2(\rr)} + \|\p_x v \|_{L^2(\rr)}
		\|\p_xE\|_{L^2(\rr)}\right)
	\\
	&
	\lesssim
	\|v\|_{H^1(\rr)} \|E\|_{H^1(\rr)}.
\end{split}
\end{equation}
For the second piece of \eqref{appleenergy-estimate-simplified} we have
\begin{equation}
	\begin{split}
		\label{applesecond-piece}
		& -\gamma \int_{\rr} v\left( 1-\p_x^2 \right)\left( v \p_x u^{\omega,\lambda} +
		u^{\omega,\lambda} \p_x v
		\right) \; dx
		\\
		& = -\gamma \int_{\rr} \left[ v^2 \p_x u^{\omega,\lambda} - v \p_x^2\left( v \p_x u^{\omega,\lambda}
		\right) \right]dx
		\\
		&   -\gamma \int_{\rr}\left[ vu^{\omega,\lambda} \p_x v - v \p_x^2\left( u^{\omega,\lambda} \p_x v \right)
		\right]dx.
	\end{split}
\end{equation}
We estimate the first term of \eqref{applesecond-piece} in parts:
\begin{equation*}
	\begin{split}
		\left | -\gamma \int_{\rr} \left[ v^2 \p_x u^{\omega,\lambda} \right]dx \right |
		& \lesssim \|\p_x u^{\omega,\lambda} \|_{L^\infty(\rr)} \|v\|_{H^1(\rr)}^2
	\end{split}
\end{equation*}
and by the product rule, Cauchy-Schwartz, and the Sobolev Imbedding Theorem,
we also have 
\begin{equation*}
	\begin{split}
		\left |-  \gamma \int_{\rr} \left [-v \p_x^2 \left(v \p_x u^{\omega,\lambda}
		\right) \right ]  dx \right |
		& \lesssim \left ( \|v\|_{L^2(\rr)} \|\p_x v\|_{L^2(\rr)} \|\p_x^2
		u^{\omega,\lambda} \|_{L^\infty(\rr)} \right .
		\\
		&+ \|\p_x v \|_{L^2(\rr)}^2 \|\p_x u^{\omega,\lambda}\|_{L^\infty(\rr)} \left )
		\right .
		\\
		& \lesssim \left( \|\p_x u^{\omega,\lambda} \|_{L^\infty(\rr)}+ \|\p_x^2
		u^{\omega,\lambda} \|_{L^\infty(\rr)} \right )\|v\|_{H^1(\rr)}^2 .
	\end{split}
\end{equation*}
Hence,
\begin{equation}
	\label{applepart1}
	\begin{split}
		& \left | - \gamma \int_{\rr} \left[ v^2 \p_x u^{\omega,\lambda} - v\p_x^2 \left( v \p_x u^{\omega,\lambda} \right)
		\right]dx \right |
		 \\
		 &  \lesssim  \left( \|\p_x u^{\omega,\lambda} \|_{L^\infty (\rr)} + \| \p_x^2 u^{\omega,\lambda}
		\|_{L^\infty(\rr)}
		\right)
		\|v \|_{H^1 (\rr)}^2.
	\end{split}
\end{equation}
Next, we estimate the second term of \eqref{applesecond-piece} in parts. For the first part, we apply Cauchy-Schwartz to obtain:
\begin{equation}
	\label{applefirst-part}
	\begin{split}
		\left |  -\gamma \int_{\rr} \left[ v u^{\omega,\lambda} \p_x v \right] dx \right |
		& \lesssim \|u^{\omega,\lambda}\|_{L^\infty(\rr)} \|v\|_{H^1(\rr)}^2
	\end{split}
\end{equation}
For the second part, we use integration by parts and the product rule to
obtain
\begin{equation}
	\label{appleboo}
	\begin{split}
		 \left | -\gamma \int_{\rr} 
		 -v \p_x^2 \big ( u^{\omega,\lambda} \p_x v \big )
		\; dx \right | 
		& \simeq \left | \int_{\rr} \left[ 
		\p_x\left( u^{\omega,\lambda}\left( \p_x v
		\right)^2 \right) + \p_x u^{\omega,\lambda}\left( \p_x v \right)^2
		\right]dx \right |.
	\end{split}
\end{equation}
Since $u^{\omega,\lambda}$ is compactly supported on $\rr$, \eqref{appleboo}
gives
\begin{equation}
	\label{applesecond-part}
	\begin{split}
		 \left | -\gamma \int_{\rr} \left [-v \p_x^2 \big ( u^{\omega,\lambda} \p_x v \big )\right
		] dx \right | 
		& \lesssim \|\p_x u^{\omega,\lambda} \|_{L^\infty(\rr)} \|v \|_{H^1(\rr)}^2.
	\end{split}
\end{equation}
Grouping \eqref{applefirst-part} and \eqref{applesecond-part} we obtain
\begin{equation}
	\label{applepart2}
	\left |  -\gamma \int_{\rr} \left[ v u^{\omega,\lambda} \p_x v - v \p_x^2\left( u^{\omega,\lambda} \p_x v \right)
	\right]dx \right | \lesssim \left( \|u^{\omega,\lambda}\|_{L^\infty(\rr)} + \|\p_x u^{\omega,\lambda}
	\|_{L^\infty(\rr)}
	\right)\|v\|_{H^1(\rr)}^2.
\end{equation}
which, combined with \eqref{applepart1} yields an estimate for
\eqref{applesecond-piece}:
\begin{equation}
	\begin{split}
		\label{applesecond-piece-final}
		\left | -\gamma \int_{\rr}
		\left[ v\left( 1-\p_x^2 \right)\left( v \p_x u^{\omega,\lambda} + u^{\omega,\lambda} \p_x v
		\right) \right] dx \right |
		&\lesssim \left( \|u^{\omega,\lambda}\|_{L^\infty(\rr)}\| + \|\p_x u^{\omega,\lambda}
		\|_{L^\infty(\rr)} \right . 
		\\
		& + \|\p_x^2 u^{\omega,\lambda} \|_{L^\infty(\rr)}
		\big )\|v\|_{H^1(\rr)}^2.
	\end{split}
\end{equation}
We now estimate the final piece of the right-hand-side of
\eqref{appleenergy-estimate-simplified}, i.e.
\begin{equation}
	\label{applelast_piece}
	-\int_{\rr} \left[ \left( 3 -\gamma \right)v \p_x \left( u^{\omega,\lambda} v \right) + \gamma
	v \p_x \left( \p_x u^{\omega,\lambda} \p_x v \right)\right]dx.
\end{equation}
We will estimate in parts:
\begin{equation}
	\begin{split}
		\label{applelast_piece_part1}
		\left | -\int_{\rr}  \left( 3- \gamma \right)v \p_x\left( u^{\omega,\lambda} v \right)
		 dx \right | 
		& \lesssim \|\p_x v \|_{L^2(\rr)} \|u^{\omega,\lambda} v \|_{L^2(\rr)}
		\\
		& \lesssim \|u^{\omega,\lambda}\|_{L^\infty(\rr)} \|v\|_{H^1(\rr)}^2
	\end{split}
\end{equation}
and
\begin{equation}
	\begin{split}
		\label{applelast_piece_part2}
		\left | -\int_{\rr}  \gamma v \p_x \left( \p_x u^{\omega,\lambda} \p_x v
		\right) dx  \right | 
		& \lesssim \|\p_x v \|_{L^2(\rr)} \| \p_x u^{\omega,\lambda} \p_x v \|_{L^2(\rr)}
		\\
		& \lesssim \|\p_x u^{\omega,\lambda} \|_{L^\infty(\rr)} \|v \|_{H^1(\rr)}^2.
	\end{split}
\end{equation}
Using \eqref{applelast_piece_part1} and \eqref{applelast_piece_part2}, we now have the
following estimate for \eqref{applelast_piece}:
\begin{equation}
	\begin{split}
	\label{applelast_piece_final}
	& \left | -\int_{\rr} \left[ \left( 3-\gamma \right)v
	\p_x \left( u^{\omega,\lambda} v \right) + \gamma
	v \p_x \left( \p_x u^{\omega,\lambda} \p_x v \right)\right]dx \right |
	\\
	& \lesssim \big(
	\|u^{\omega,\lambda}\|_{L^\infty(\rr)}
	 + \|\p_x u^{\omega,\lambda} \|_{L^\infty(\rr)} \big)
	\|v\|_{H^1(\rr)}^2.
\end{split}
\end{equation}
Combining \eqref{applefirst_piece}, \eqref{applesecond-piece-final},
and \eqref{applelast_piece_final}, we can
simplify \eqref{appleenergy-estimate-simplified} to obtain
\begin{equation}
	\begin{split}
		\label{appleenergy-estimate-best}
		\frac{d}{dt} \|v(t)\|_{H^1(\rr)}^2
		& \lesssim \left( \|u^{\omega,\lambda}\|_{L^\infty(\rr)} + \|
		\p_x u^{\omega,\lambda} \|_{L^\infty(\rr)} + \|\p_x^2 u^{\omega,\lambda} \|_{L^\infty (\rr)} \right)
		\|v\|_{H^1(\rr)}^2 
		\\
		&+ \|v\|_{H^1(\rr)} \|E\|_{H^1(\rr)}.
	\end{split}
\end{equation}
Now, observe that
\begin{equation}
	\begin{split}
		\p_x^2 u^h 
		& = \lambda^{-\frac{\delta}{2}-s} \Big[ - \lambda^2 \phi\left(
		\frac{x}{\lambda^\delta} \right ) \cos(\lambda x - \gamma \omega t) \\
		& - 2\lambda^{1 -\delta } \phi'\left( \frac{x}{\lambda^\delta}
		\right )
		\sin(\lambda x - \gamma \omega t ) + \lambda^{-2\delta} \phi''\left(
		\frac{x}{\lambda^\delta} \right) \cos (\lambda x - \gamma \omega t) \Big].
		\label{apple51}
	\end{split}
\end{equation}
Since the $\lambda^2$ term dominates inside the brackets, \eqref{apple51} yields
\begin{equation}
	\begin{split}
		\|\p_x^2 u^h \|_{L^\infty(\rr)} \lesssim
		\lambda^{2-\frac{\delta}{2}-s}.
		\label{apple52}
	\end{split}
\end{equation}
Combining \eqref{apple22}, \eqref{apple23}, and \eqref{apple52}, we obtain
\begin{equation}
	\begin{split}
		\|u^h\|_{L^\infty(\rr)} + \|\p_x u^h\|_{L^\infty(\rr)} + \|\p_x^2
		u^h\|_{L^\infty(\rr)} \lesssim \lambda^{-\left(
		\frac{\delta}{2} + s -2 \right)}.
		\label{apple53}
	\end{split}
\end{equation}
Furthermore, we have
\begin{equation}
	\begin{split}
		\|u_\ell\|_{L^\infty(\rr)} + \|\p_x u_\ell \|_{L^\infty(\rr)} + \|\p_x^2
		u_\ell\|_{L^\infty(\rr)} \le c_s \|u_\ell\|_{H^3(\rr)}.
		\label{apple54}
	\end{split}
\end{equation}
Applying Lemma \ref{appleb} to \eqref{apple54}, we see that
\begin{equation}
	\begin{split}
		\|u_\ell\|_{L^\infty(\rr)} + \|\p_x u_\ell \|_{L^\infty(\rr)} + \|\p_x^2
		u_\ell\|_{L^\infty(\rr)} \lesssim \lambda^{-(1 - \frac{\delta}{2})},
		\quad |t| \le T.
		\label{apple55}
	\end{split}
\end{equation}
Combining \eqref{apple53} and \eqref{apple55}, we obtain
\begin{equation}
	\begin{split}
		\|u^{\omega,\lambda}\|_{L^\infty(\rr)} + \|\p_x u^{\omega,\lambda}\|_{L^\infty(\rr)} + \|\p_x^2
		u^{\omega,\lambda}\|_{L^\infty(\rr)}
		& \lesssim \lambda^{-\rho_s},
		\label{apple56}
	\end{split}
\end{equation}
where
\begin{equation}
	\begin{split}
		\rho_s = \text{min} \left\{ \frac{\delta}{2} + s -2, \; 1-
		\frac{\delta}{2} \right\}.
		\label{apple57}
	\end{split}
\end{equation}
Note that for $s>1$, we can assure $\rho_s > 0$
by choosing a suitable $1<\delta<2$.
Substituting \eqref{apple35} and \eqref{apple56} into \eqref{appleenergy-estimate-best},
we get
\begin{equation}
	\label{apple58}
	\frac{d}{dt} \|v(t)\|_{H^(\rr)}^2 \lesssim \lambda^{-\rho_s}
	\|v\|_{H^1(\rr)}^2 + \lambda^{-r_s}
	\|v \|_{H^1(\rr)}
\end{equation}
where we recall the definition of $r_s$ from \eqref{appler_s}. By Gronwall's Inequality, we conclude that for $s>1$ and
suitably chosen $1<\delta<2$, we are assured the
decay of $\|v(t)\|_{H^1(\rr)}$, i.e. 
\begin{equation}
	\label{appleen-est-fin!}
	\|v(t)\|_{H^1(\rr)} 
	\lesssim
	\lambda^{-r_s}, \quad |t| \le T,\quad \lambda>>1 . 
\end{equation}
This completes the proof. 
\end{proof}
%
%
%
\subsection{Non-Uniform Dependence for $s > 1$.}
Let $u_{\pm 1,\lambda}$ be solutions to the HR i.v.p. with common initial data $u^{\pm 1,
n}(0)$, respectively.
We wish to show that the $H^s$ norm of the difference of $u_{\pm 1,
n}$ and the associated approximate solution $u^{\pm 1,\lambda}$
decays as $n \to \infty$. In order for \eqref{appleen-est-fin!} to hold,
we assume $s > 1$; recalling Theorem \ref{thm:HR_existence_continous_dependence}
, we
have
\begin{equation}
	\begin{split}
		\|u_{\pm 1,\lambda} (t) \|_{H^{2s-1}(\rr)}
		& \le 2 \|u^{\pm 1,\lambda}(0) \|_{H^{2s-1}(\rr)}, \qquad
		|t| \le T.
		\label{apple60}
	\end{split}
\end{equation}
Furthermore, recalling \eqref{apple1}-\eqref{apple1***}, we have 
\begin{equation}
	\begin{split}
		& \|u^{\pm 1, \lambda}(t)\|_{H^{2s-1}(\rr)}
		\\
		& \le \|u_{\ell, \pm \omega, \lambda}\|_{H^{2s-1}(\rr)} +
		 \| \lambda^{-\frac{\delta}{2} -s} \phi \left(
		\frac{x}{\lambda^\delta} \right) \cos(\lambda x \mp \gamma \omega t)
		\|_{H^{2s-1}(\rr)}
		\\
		& = \|u_{\ell, \pm \omega, \lambda}\|_{H^{2s-1}(\rr)}
		+
		\lambda^{s-1} \cdot
		\lambda^{-\frac{\delta}{2}-(2s-1)} \|\phi \left(
		\frac{x}{\lambda^\delta} \right) \cos(\lambda x \mp \gamma \omega t)
		\|_{H^{2s-1}(\rr)}.
		\label{apple61}
	\end{split}
\end{equation}
Applying Lemma \ref{applea} and Lemma \ref{appleb} to \eqref{apple61}, we obtain
\begin{equation}
	\begin{split}
		\|u^{\pm 1, \lambda}(t) \|_{H^{2s-1}(\rr)}
		& \lesssim \lambda^{s-1}.
		\label{apple62}
	\end{split}
\end{equation}
Hence, using \eqref{apple60}, \eqref{apple62}, and the triangle inequality, we deduce
\begin{equation}
	\begin{split}
		\|u^{\pm 1, \lambda}(t) - u_{\pm 1, \lambda}(t) \|_{H^{2s-1}(\rr)}
		\lesssim \lambda^{s-1}.
		\label{apple63}
	\end{split}
\end{equation}
Furthermore, by Lemma
\ref{applelem:bound_for_difference-of-approx-and-actual-soln}, we have
\begin{equation}
	\begin{split}
		\|u^{\pm 1, \lambda}(t) - u_{\pm 1, \lambda} \|_{H^1(\rr)} \lesssim
		\lambda^{-r_s}.
		\label{apple64}
	\end{split}
\end{equation}
		We now wish to interpolate in order to obtain an estimate of the $H^s (\rr)$
		norm of the difference of the approximate and actual solutions:
		\begin{lemma}
			\label{apple403}
			For all $\psi \in L^2(\rr)$,
			\begin{equation*}
				\|\psi \|_{H^s (\rr)}^2 \leq  \| \psi \|_{H^1 (\rr)} \| \psi
				\|_{H^{2s-1}}. 
			\end{equation*}
		\end{lemma}
			Hence, using Lemma \ref{apple403} to interpolate between estimates
			\eqref{apple63} and \eqref{apple64}, we obtain
			\begin{equation}
				\begin{split}
					\|u^{\pm 1, \lambda}(t) - u_{\pm 1, \lambda}(t)
					\|_{H^s(\rr)}
					\lesssim \lambda^{\frac{\delta -2}{4}}.
					\label{apple65}
				\end{split}
			\end{equation}
			Next, we will use estimate \eqref{apple65} to prove non-uniform
			dependence when $s > 1$.
%%%%%%%%%%%%% Behavior at time  t = 0  %%%%%%%%%%%% 
%
\subsection{Behavior at time $t=0$.}  We have
%
%
\begin{equation}
	\begin{split}
		\|u_{1,\lambda}(0) - u_{-1,\lambda}(0) \|_{H^s(\rr)} 
		& = \|u^{1,\lambda}(0) - u^{-1,\lambda}(0) \|_{H^s(\rr)}
		\\
		& = 2 \lambda^{-1} \| \tilde{\phi}\left( \frac{x}{\lambda^\delta}
		\right) \|_{H^s(\rr)}.
		\label{apple}
	\end{split}
\end{equation}
Applying the estimate
\begin{equation}
	\begin{split}
		\|\tilde{\phi}\left( \frac{x}{\lambda^\delta}
		\right)\|_{H^{k}(\rr)} \le
		\lambda^{\frac{\delta}{2}}\|\tilde{\phi}\|_{H^{k}(\rr)},
		\qquad k\ge 0,
	\end{split}
\end{equation}
and recalling that $1<\delta<2$, we conclude that
\begin{equation}
	\begin{split}
		\|u_{1,\lambda}(0) - u_{-1,\lambda}(0) \| \le 2
		\lambda^{\frac{\delta}{2}-1} \|\tilde{\phi} \|_{H^s(\rr)} \to 0
		\; \; \text{as} \; \; \lambda \to \infty.
		\label{apple70}
	\end{split}
\end{equation}
%
%
%%%%%%%%%%%%%% Behavior at time  t >0  %%%%%%%%%%%% 
%  
%
\subsection{Behavior at time  $t>0$.}  Using the reverse triangle inequality, we have
%
%
%
\begin{equation} 
	\label{appleHR-slns-differ-t-pos}
	\begin{split}
		\|
		u_{1,\lambda}(t)
		-
		u_{- 1,\lambda}(t)
		\|_{H^s(\rr)}
		&
		\ge
		\|
		u^{1,\lambda}(t)
		-
		u^{- 1,\lambda}(t)
		\|_{H^s(\rr)}
		\\
		&
		-
		\|
		u^{1,\lambda}(t)
		-
		u_{1,\lambda}(t)
		\|_{H^s(\rr)}
		\\
		&
		-
		\|
		-u^{-1,\lambda}(t)
		+
		u_{-1,\lambda}(t)
		\|_{H^s(\rr)} .
	\end{split}
\end{equation}
%
%
Using estimate \eqref{apple65} for the last two terms 
in \eqref{appleHR-slns-differ-t-pos} we obtain
%
%
%
\begin{equation} 
	\label{appleHR-slns-differ-t-pos-est}
	\|
	u_{1,\lambda}(t)
	-
	u_{- 1,\lambda}(t)
	\|_{H^s(\rr)}
	\ge
	\|
	u^{1,\lambda}(t)
	-
	u^{- 1,\lambda}(t)
	\|_{H^s(\rr)}
	-
	c \lambda^{\frac{\delta - 2}{4}}
\end{equation}
where c is a positive, non-zero constant. Letting $\lambda$ go to $\infty$ in
\eqref{appleHR-slns-differ-t-pos-est}
yields
%
\begin{equation} 
	\label{appleHR-slns-to-ap-est}
	\liminf_{n\to\infty}
	\|
	u_{1,\lambda}(t)
	-
	u_{- 1,\lambda}(t)
	\|_{H^s(\rr)}
	\ge
	\liminf_{n\to\infty}
	\|
	u^{1,\lambda}(t)
	-
	u^{- 1,\lambda}(t)
	\|_{H^s(\rr)}.
\end{equation}
%
%
Hence, by \eqref{appleHR-slns-to-ap-est}, we see we have reduced the problem of
analyzing the growth of the difference of actual solutions to the more
manageable problem of analyzing the growth of the associated approximate
solutions. Using the identity 
$$
\cos \alpha -\cos \beta
=
-2
\sin(\frac{\alpha + \beta}{2})
\sin(\frac{\alpha - \beta}{2})
$$
gives
\begin{equation}
	\label{apple80}
	\begin{split}
u^{1,\lambda}(t)
-
u^{- 1,\lambda}(t)
=
u_{\ell,1,\lambda}(t) - u_{\ell,-1,\lambda}(t) + 2\lambda^{-\frac{\delta}{2}-s}
\phi\left( \frac{x}{\lambda^\delta} \right)\sin(\lambda x) \sin(\gamma t).
\end{split}
\end{equation}
Now, by Lemma \ref{appleb}, we have
\begin{equation*}
	\begin{split}
	\|u_{\ell,-1,\lambda}(t) - u_{\ell,1,\lambda}(t)\|_{H^s(\rr)} \lesssim
	\lambda^{-1 + \frac{\delta}{2}};
	\end{split}
\end{equation*}
hence applying the reverse triangle inequality to \eqref{apple80}, we obtain
\begin{equation} 
	\label{apple90}
	\begin{split}
	& \|
	u^{1,\lambda}(t)
	-
	u^{- 1,\lambda}(t)
	\|_{H^s(\rr)}
	\\
	& \ge 2 \lambda^{-\frac{\delta}{2}-s} \|\phi\left(
	\frac{x}{\lambda^\delta} \right) \sin(\lambda x) \|_{H^s(\rr)} |\sin \gamma t|
	\\
	& - \|u_{\ell,-1,\lambda}(t) - u_{\ell,1,\lambda}(t)\|_{H^s(\rr)} 
	\\
	& \gtrsim \lambda^{-\frac{\delta}{2}-s} \|\phi\left(
	\frac{x}{\lambda^\delta} \right ) \sin(\lambda x) \|_{H^s(\rr)} |\sin \gamma t| -
	\lambda^{-1 + \frac{\delta}{2}}.
\end{split}
\end{equation}
%
%
Letting $\lambda$ go to $\infty$, Lemma \ref{applea}
with \eqref{apple90}  gives
%
%
\begin{equation} 
	\label{apple91}
	\liminf_{\lambda \to\infty}
	\|
	u^{1,\lambda}(t)
	-
	u^{- 1,\lambda}(t)
	\|_{H^s(\rr)}
	\gtrsim
	|\sin \gamma t|.
\end{equation}
Combining \eqref{appleHR-slns-to-ap-est} with \eqref{apple91}, we see that
\begin{equation}
	\begin{split}
		\liminf_{\lambda \to \infty} \|u_{1,\lambda}(t) -
		u_{-1,\lambda}(t) \|_{H^s(\rr)} \gtrsim |\sin
		\gamma t |, \qquad |t| \le T,
		\label{apple92}
	\end{split}
\end{equation}
proving \eqref{bdd-away-from-0}. Furthermore, a computation analogous
to that in \eqref{apple60}-\eqref{apple62} yields
\begin{equation}
	\label{apple93}
	\begin{split}
		\|u_{\pm 1, \lambda} (t) \|_{H^{s}(\rr)}
		& \lesssim 1.
	\end{split}
\end{equation}
Collecting \eqref{apple70}, \eqref{apple92}, and \eqref{apple93}, we
conclude that we have proven Theorem \ref{hr-non-unif-dependence} for the
non-periodic case.
	%
	%%%%%%%%%%%%%%%%%%%%%%%%%%%%%%%%%%
	%
	%
	%
	%             Proof of Theorem in the Periodic case
	%
	%
	%
	%%%%%%%%%%%%%%%%%%%%%%%%%%%%%%%%%%
	%
	\section{Proof of Non-Uniform Dependence in the Periodic case} 
	%
	%
	%
	%
	We will consider approximate solutions of form
	\begin{equation}
		\label{approx-solutions-form}
		u^{\omega,n}(x,t) = \omega n^{-1} + n^{-s} \cos \left( nx - \gamma \omega t
		\right) 
	\end{equation}
	with integer valued $n >>1$, bounded $\omega \in \rr$, and a fixed constant $\gamma \in \rr$. We rewrite 
	the HR i.v.p as
	\begin{align}
			 \p_t u + \gamma u \p_x u \ + & \ \p_x (1 - \p_x^2)^{-1} 
			 \left[ \frac{3 - \gamma}{2}u^2 +
			\frac{\gamma}{2}(\p_x u)^2 \right] = 0,
			\label{hyperelastic-rod-equation}
			\\
			& {u(x,0) = u_0(x)},
			\label{init-cond}
			\end{align}
	and aim to substitute our approximate solution into the left hand side 
	in order to obtain a functional representation of its error. Hence, some 
	preliminary calculations are necessary. We omit the superscripts $w,n$ for clarity: 
	\begin{equation}
		\begin{split}
			 \p_t u
			 & = \gamma \omega n^{-s} \sin\left( nx - \gamma \omega t \right),
			\\
			 \p_x u
			 & = -n^{-s+1} \sin(nx - \gamma \omega t),
			\\
			\gamma u \p_x u
			& = - \gamma \omega n^{-s} \sin\left( nx - \gamma
			\omega t \right) - \frac{\gamma}{2}n^{-2s+1}\sin\left( 2\left( nx - \gamma
			\omega t
			\right) \right),
			\\
			u^2 
			& = \omega^2 n^{-2} + 2\omega n^{-s -1} \cos \left( nx - \gamma \omega t
			\right) + n^{-2s} \cos^2\left( nx - \gamma \omega t \right),
			\\
			(\p_x u)^2 
			& = n^{-2s+2} \sin^2\left( nx - \gamma \omega t \right).
			\label{calculation of functional representation of error}
		\end{split}
	\end{equation}
	Using these relations, we obtain
	\begin{equation}
		\begin{split}
			\p_t u + \gamma u \p_x u
			& + \p_x(1- \p_x^2)^{-1} \left[
			\frac{3-\gamma}{2}u^2 + \frac{\gamma}{2}(\p_x u)^2 \right]
			\\
			& = \cancel{\gamma \omega n^{-s} \sin(nx - \gamma \omega t)} -
			\cancel{\gamma \omega n^{-s}\sin\left( nx - \gamma \omega t \right)}
			\\
			& -
			\frac{\gamma}{2}n^{-2s+1}\sin\left( 2\left( nx - \gamma \omega t \right)
			\right)
			\\
			& + \p_x \left( 1-\p_x^2 \right)^{-1}\bigg[ \frac{3-\gamma}{2} \bigg (
			\omega^2 n^{-2} + 2 \omega n^{-s -1}\cos( nx - \gamma \omega t )
			\\
			& + n^{-2s}\cos^2\left( nx - \gamma \omega t \right) \bigg ) + \frac{\gamma}{2}
			n^{-2s+2}\sin^2\left( nx - \gamma \omega t \right)
			\bigg]
			\\
			& \doteq E.
			\label{functional-representation-of-error}
		\end{split}
	\end{equation}
	Since $\frac{(3-\gamma)}{2}w^2 n^{-2}$ is a constant, it's derivative vanishes;
	hence we can rewrite the error $E$ as
	\begin{equation}
		\begin{split}
			E= E_1 + E_2 + E_3 + E_4
			\label{57}
		\end{split}
	\end{equation}
	where
	\begin{align}
		\label{90*}
			& E_1 =
			- \frac{\gamma}{2}n^{-2s+1}\sin\left[ 2\left( nx - \gamma 
			\omega t \right)
			\right],
			\\
			\label{90**}
			& E_2 = \p_x \left( 1-\p_x^2 \right)^{-1}\bigg[ \frac{3-\gamma}{2} \bigg (
			2 \omega n^{-s -1}\cos( nx - \gamma \omega t )
			\bigg )
			\bigg ],
			\\
			\label{90***}
			& E_3 = \p_x \left( 1-\p_x^2 \right)^{-1}\bigg[ 
			\frac{3-\gamma}{2} \bigg (
			 n^{-2s}\cos^2\left( nx - \gamma \omega t \right) \bigg )
			\bigg ],
			\\
			& E_4 = \frac{\gamma}{2}
			n^{-2s+2}\sin^2\left( nx - \gamma \omega t \right).
			\label{90}
	\end{align}
%
%
%
\noindent
\subsection{Estimate for the  Error of the Approximate Solutions.}
%
%
First, we will need the following two lemmas:
%
%
%
	 \begin{proposition}
		 \label{1n}
		 For nonzero $k \in \rr$, we have
		 \begin{equation}
			 \begin{split}
				 \|\sin(k(nx-c))\|_{H^\sigma(\ci)} \simeq n^\sigma.
				 \label{1m}
			 \end{split}
		 \end{equation}
		The same result holds with $\sin$ replaced by $\cos$.
	\end{proposition}
		%
    \begin{proof} The Fourier transform of $\psi_n(x) = \sin[k(nx-c)]$
		is
		\begin{equation*}
			\begin{split}
				\widehat{\psi_n}(\xi)
				& = \int_0^{2\pi} e^{-ix \xi} \sin [k(nx-c)]
				\ dx
				\\
				& = \int_0^{2\pi} e^{-ix \xi} \left( \frac{e^{ik(nx-c)} -
				e^{-ik(nx-c)}}{2i} \right) \ dx
				\\
				& = \frac{1}{2i} \int_0^{2\pi} e^{i[x(kn- \xi)-kc]} \ dx
				- \frac{1}{2i}\int_0^{2\pi} e^{-i[x(kn+\xi) - kc]} \ dx.
			\end{split}
		\end{equation*}
		Therefore
		\begin{equation*}
			\begin{split}
				\widehat{\psi_n}(\xi) =
				\begin{cases}
					- i \pi e^{-ikc}, \qquad & \xi = kn\\
					i \pi e^{ikc}, \qquad & \xi = -kn\\
					0,  \qquad & \xi \neq \pm kn.
				\end{cases}
			\end{split}
		\end{equation*}
		Hence
		\begin{equation*}
			\begin{split}
				\|\psi_n\|_{H^\sigma(\ci)}^2
				& = \sum_{\xi \in \zz}
				(1+\xi^2)^{\sigma} \widehat{\psi_n}(\xi) 
				\\
				& = i \pi (1+k^2 n^2)^{\sigma} (e^{ikc} - e^{-ikc})
				\\
				& \simeq n^{2 \sigma}, \qquad n>>1
			\end{split}
		\end{equation*}
		from which we obtain \eqref{1m}. Furthermore, since
		\begin{equation*}
			\begin{split}
				\cos[k(nx-c)]
				&= \sin[k(nx-c)- \pi/2] \\
				& = \sin\{k[nx - (c + \pi/2k)]\},
			\end{split}
		\end{equation*}
		we conclude \eqref{1m} holds when $\sin$ is replaced by $\cos$. This
    completes the proof. 
  \end{proof}
	%	
		\begin{proposition}
			\label{2n}
			For nonzero $k \in \rr$, $n >>1$,
			\begin{equation}
				\label{2m}
				\begin{split}
					\|\sin^2[k(nx-c)] \|_{H^\sigma(\ci)} \simeq
					\begin{cases}
					1, \qquad & \sigma \le 0
					\\
					n^\sigma,\qquad &\sigma > 0.
				\end{cases}
				\end{split}
			\end{equation}
			The same result holds with $\sin$ replaced by $\cos$.
		\end{proposition}
		\begin{proof} The Fourier transform of $\psi_n(x) = \sin^2[k(nx-c)]$
		is
		\begin{equation*}
			\begin{split}
				\widehat{\psi_n}(\xi) 
				& = \int_0^{2\pi} e^{-ix \xi} \sin^2[k(nx-c)] \ dx
				\\
				& = \int_0^{2\pi} e^{-ix \xi} \left( \frac{e^{ik(nx-c)} -
				e^{-ik(nx-c)}}{2i} \right)^2 \ dx
				\\
				& = -\frac{1}{4} \int_0^{2\pi} e^{-ix \xi} (e^{2ik(nx-c)} +
				e^{-2ik(nx-c)} -2) \ dx
				\\
				& = -\frac{1}{4} \int_0^{2\pi} e^{ix(2kn - \xi) - 2ikc} \
				dx - \frac{1}{4} \int_0^{2\pi} e^{-ix(2kn + \xi) + 2ikc} \ dx
				\\
				& + \frac{1}{2} \int_0^{2\pi} e^{-ix \xi} \ dx.
			\end{split}
		\end{equation*}
	Hence, for nonzero $k \in \rr$ and $n >>1$ we have
	\begin{equation*}
		\begin{split}
			\widehat{\psi_n}(\xi) = 
			\begin{cases}
				-\frac{\pi}{2}e^{-2ikc}, \qquad & \xi=2kn
				\\
				-\frac{\pi}{2}e^{2ikc}, \qquad & \xi = -2kn
				\\
				\pi, \qquad & \xi = 0
				\\
				0, \qquad & \xi \neq 0, \ \pm 2kn.
			\end{cases}
		\end{split}
	\end{equation*}
	which gives
	\begin{equation*}
		\begin{split}
			\|\psi_n\|_{H^\sigma(\ci)}^2 
			& = \sum_{\xi \in \zz} (1+ \xi^2)^\sigma \widehat{\psi_n}(\xi)
			\\
			& = -\frac{\pi}{2}(1+4k^2n^2)^\sigma (e^{2ikc} + e^{-2ikc}) +
			\pi
			\\
			& \simeq 
			\begin{cases}
				1,  \qquad & \sigma \le 0 
				\\
				n^{2 \sigma}, &  \sigma > 0
			\end{cases}
		\end{split}
	\end{equation*}
	from which \eqref{2m} follows. Since $\sin^2[k(nx-c)] = 1-
	\cos^2[k(nx-c)]$, \eqref{2m} holds with $\sin$ replaced by
	$\cos$. 
\end{proof}
	%
	\subsection{An Estimate for $\hyperref[90*]{E_1}$.}
	We apply Proposition \ref{1n} to obtain
	\begin{equation}
		\label{85}
		\begin{split}
			\|E_1\|_{H^\sigma(\ci)}
			& = 
			\left\| - \frac{\gamma}{2}n^{-2s+1}\sin\left( 2\left(
			nx - \gamma \omega t \right)\right )
			\right\|_{H^\sigma(\ci)}
			\\
			& \lesssim
			n^{-2s + \sigma + 1}
		\end{split}
	\end{equation}
	\subsection{An Estimate for $\hyperref[90**]{E_2}$.}
	We will need the following:
	%
	\begin{remark}
		\label{lem:operator-norm-lemma}
		For $u \in L^2(\ci)$ and arbitrary $k$, we have
		\begin{equation}
			\begin{split}
				\|\p_x (1 -\p_x^2)^{-1}u \|_{H^{k}(\ci))} \le
				\|u\|_{H^{k-1}(\ci)}.
				\label{operator norm of pseudo-diff operator we use}
			\end{split}
		\end{equation}
	\end{remark}
	\begin{proof} Let $u \in L^2(\ci)$. Then
	\begin{equation*}
		\begin{split}
			\|\p_x \left( 1- \p_x^2 \right)^{-1} u \|_{H^{k}(\ci)}
			& = \sum_{\xi \in \zz}  \left[ \xi\left( 1+\xi^2 \right)^{-1} \right]^2
			\cdot \left( 1 + \xi^2 \right)^{k} \cdot |\hat{u}(\xi)|^2 
			\\
			& \le \sum_{\xi \in \zz}  \left( 1+ \xi^2 \right)^{k-1} \cdot 
			|\hat{u}(\xi)|^2 
			\\
			& \le \|u\|_{H^{k-1}(\ci)}.
			\qquad \Box
		\end{split}
	\end{equation*}
	Applying Remark \ref{lem:operator-norm-lemma}, we obtain
		\begin{equation}
		\begin{split}
			 \|E_2\|_{H^\sigma(\ci)} & = \left \|\p_x(1-\p_x^2)^{-1}
			\left[ \frac{3-\gamma}{2}
			\left( 2 \omega n^{-s -1}\cos( nx - \gamma \omega t )
			\right) \right] \right \|_{H^\sigma(\ci)}
			\\
			& \le \left |\frac{3-\gamma}{2}\right |
			\left \|2 \omega n^{-s -1}\cos( nx - \gamma \omega t )
			\right \|_{H^{\sigma -1 }(\ci)}
			\label{non-local_term_first_piece_without_constant}
		\end{split}
	\end{equation}
	which by Proposition \ref{1n} gives
		\begin{equation}
			\label{3.10}
		\begin{split}
			\|E_2\|_{H^\sigma(\ci)}
			& \lesssim n^{-s + \sigma -2}.
		\end{split}
	\end{equation}
	%
  This completes the proof. 
\end{proof}
\subsection{An Estimate for $\hyperref[90]{E_3}$.}
Applying Remark \ref{lem:operator-norm-lemma}, we obtain
		\begin{equation}
		\begin{split}
			\|E_3\|_{H^\sigma(\ci)} & = \left \|\p_x(1-\p_x^2)^{-1}
			\left[ \frac{3-\gamma}{2}\left( n^{-2s}\cos^2\left( nx - \gamma 
			\omega
			t\right)\right) \right] \right \|_{H^\sigma(\ci)}
			\\
			& \le \left |\frac{3-\gamma}{2}\right |
			\left \| n^{-2s}\cos^2\left( nx - \gamma \omega t \right)
			\right \|_{H^{\sigma -1 }(\ci)}
			\end{split}
	\end{equation}
which by Proposition \ref{2n} gives
		\begin{equation}
			\label{yuoo}
		\begin{split}
			\|E_3\|_{H^\sigma(\ci)}
			& \lesssim 
			\begin{cases}
				n^{-2s}, \qquad & \sigma \le 1  \\
				n^{-2s +\sigma -1}, \qquad & \sigma
				> 1.
			\end{cases}
		\end{split}
	\end{equation}
	\subsection{An Estimate for $\hyperref[90]{E_4}$.}
	We now apply Remark \ref{lem:operator-norm-lemma} and Proposition 
	\ref{2n} to obtain
	\begin{equation*}
		\begin{split}
			\|E_4\|_{H^\sigma(\ci)}
			& = \Big \|\p_x \left( 1 - \p_x^2 \right)^{-1} 
			\frac{\gamma}{2}n^{-2s+2}\sin^2\left(
			nx - \gamma \omega t \right) \Big \|_{H^\sigma(\ci)}
			\\
			& \leq \left | \frac{\gamma}{2} \right |  \| 
			n^{-2s+2}\sin^2\left( nx - \gamma \omega t
			\right)\|_{H^{\sigma -1}(\ci)}
			\end{split}
	\end{equation*}
	which by Proposition \ref{2n} gives
	\begin{equation}
			\label{estimate_for_second_piece_of_non_local_final}
		\begin{split}
			\|E_4\|_{H^\sigma(\ci)} \lesssim 
			\begin{cases}
				n^{-2s+2}, \qquad & \sigma \le 1
				\\
				n^{-2s+ \sigma + 1}, \qquad & \sigma >1.
			\end{cases}
		\end{split}
	\end{equation}
	Grouping estimates \eqref{85}, \eqref{3.10}, \eqref{yuoo}, and
	\eqref{estimate_for_second_piece_of_non_local_final} we see that for bounded $\omega$ and $n >> 1$
	we have the following upper bound for the $H^\sigma(\ci)$ error
	of our approximate solutions:
	\begin{equation*}
		\begin{split}
			\|E(t)\|_{H^{\sigma}(\ci)} \lesssim 
			  \begin{cases}
				  n^{-s-1} + n^{-2s +2}, \qquad & \sigma \le 1
				  \\
				  n^{-s + \sigma - 2} + n^{-2s + \sigma + 1}, \qquad &
				  \sigma > 1.
			  \end{cases}
		\end{split}
	\end{equation*}
		Noting that $\|E(t)\|_{H^\sigma(\ci)}$ blows up as $\sigma$ 
		increases,
	regardless of the choice of $s$, we restrict our attention to the case $\sigma \le 1$
	and obtain the following:
	  \begin{lemma}
		  \label{lem:error_of_approx_solution}
		  Let $u^{\omega,n}$ be an approximate solution to the HR i.v.p., 
		  with $\sigma \le 1$,  $\omega$ bounded, and $n >> 1$.
		  Then for the error $E$ we have
		  \begin{equation}
			  \begin{split}
				  \|E(t)\|_{H^\sigma(\ci)} \lesssim n^{-r_s}
				  \label{total-error-approx-solution}
			  \end{split}
		  \end{equation}
		  where
		  \begin{equation}
			  \begin{split}
			r_s = 
			\begin{cases}
				2(s-1)   & \text{if} \quad s \le 3,\\  
				s+1  & \text{if} \quad s > 3. \\
			\end{cases}
			\label{r_s-definition}
			  \end{split}
		  \end{equation}
	  \end{lemma}
	 % 
	 %%%%%%%%%%%%%%%%%%%%%%%%%%%%%%%%%
	 %
	 %
	 %
	 %   Proof of  Theorem in periodic case for s between 3/2 and 2
	 %
	 %
	 %
	 %%%%%%%%%%%%%%%%%%%%%%%%%%%%%%%%%%%
	 \subsection{A Critical Lemma.}
	We wish now to estimate the difference between approximate and actual solutions to
	the HR i.v.p with common initial data $u_0 \in H^s$. To do so, we must first
	establish the existence and lifespan of solutions $u(x,t)$ with initial data $u_0$.
	We have the following:
\begin{theorem}
	\label{thm:HR_existence_continous_dependence}
For  $s>3/2$  the following  results  hold:
%
\noindent
(i) If $u_0\in H^s(\ci)$  then  there exists a unique solution to
the Cauchy problem  \eqref{hr}--\eqref{hr-data} 
  in $C([-T, T]; H^s(\ci))$, where the life-span  $T$ depends on the size
  of the initial data $u_0$, that is
  $T=T(\|u_0 \|_{H^s(\ci)})$.
  \noindent
(ii)
 The flow  map $u_0 \to u(t)$  is continuous from
 bounded sets of $H^s(\ci)$ into $C([-T, T]; H^s(\ci))$.
%
 \noindent
\\
(iii)  The  lifespan $T$ satisfies the lower bound estimate 
%
     \begin{equation}
   \label{Life-span-est}
T
\ge
\frac{1}{2c_s}
\frac{1}{\|
u_0
  \|_{H^s(\ci)}},
   \end{equation}
   %
and the solution $u$ satisfies the estimate
%
     \begin{equation}
   \label{u_x-Linfty-Hs}
\|
u(t)
  \|_ {H^s(\ci))}
  \le
  2
  \|
u_0
  \|_{H^s(\ci)},\,\, |t|\le T.
   \end{equation}
   %
 \end{theorem}
%
A proof of these results is provided in the appendix.
%
%
%
Let $v=u^{\omega,n} -
u_{\omega,n}$, where $u_{\omega,n}$ denotes a solution to
the Cauchy-problem \eqref{hyperelastic-rod-equation}-\eqref{init-cond} with
initial data $u_0(x) = u^{\omega,n}(x,0)$. We are now prepared to prove the following critical lemma:
\begin{lemma}
	\label{lem:bound_for_difference-of-approx-actual-soln}
	If \ $s > 3/2 $ and $\sigma = 1/2 + \ee$ for an appropriately
	chosen $\ee = \ee(s) > 0$, then 
			\begin{equation} 
				\|
				v(t)
				\|_{H^\sigma(\ci)}
				\doteq
				\label{differ-Hsigma-est} 
				\|
				u^{\omega, n}(t) 
				- 
				u_{\omega, n}(t)
				\|_{H^\sigma(\ci)}
				\lesssim 
				n^{-r_s}, 
				\quad
				|t| \le T.
			\end{equation}
			\end{lemma}
      \begin{proof} Recall the HR Cauchy problem
\begin{align}
	\label{1.1}
	&\p_t u  = -\gamma u \p_x u - \p_x\left( 1-\p_x^2
		\right)^{-1}\left[ \frac{3-\gamma}{2}
		u^2 + \frac{\gamma}{2}\left( \p_x
		u
		\right)^2 \right],
		\\
		\label{1.2}
		& u(x,0) = u_0.
\end{align}
and its approximate solutions of form
	\begin{equation}
		\label{approx-solns-form}
		u^{\omega,n}(x,t) = \omega n^{-1} + n^{-s} \cos \left( nx - \gamma \omega t
		\right) 
	\end{equation}
	with integer valued $n >>1$, bounded $\omega \in \rr$, and a fixed 
	constant $\gamma \in \rr$.	Then the approximate solution 
	$u^{\omega,n}$ satisfies the equation
\begin{equation}
	\begin{split}
		\label{1.4}
		\p_t u^{\omega,n} = E - \gamma u^{\omega,n} \p_x u^{\omega,n} 
		- \p_x\left( 1-\p_x^2 \right)^{-1} \left[
		\frac{3-\gamma}{2} \left( u^{\omega,n} \right)^2 +
		\frac{\gamma}{2}\left( \p_x u^{\omega,n} \right)^2 \right].
	\end{split}
\end{equation}
Let $u_{\omega,n}$ denote the solution to the i.v.p
\begin{align}
	\label{1.5}
	&\p_t u_{\omega,n}  = -\gamma u_{\omega,n} \p_x u_{\omega,n} - 
	\p_x\left( 1-\p_x^2
		\right)^{-1}\left[ \frac{3-\gamma}{2}\left(
		u_{\omega,n} \right)^2 + \frac{\gamma}{2}\left( \p_x
		u_{\omega,n}
		\right)^2 \right],
		\\
		& u_{\omega,n}(x,0) = u^{\omega,n}(x,0).
\end{align}
Subtracting \eqref{1.5} from \eqref{1.4}, we see that the
difference $v = u^{\omega,n} - u_{\omega,n}$ satisfies the i.v.p
\begin{align}
		\label{1.7}
		& \p_t v  =  E - \frac{\gamma}{2} \p_x
		\left[ \left( u^{\omega,n} + u_{\omega,n} \right)v \right]
		 \notag
		\\
		& - \p_x(1-\p_x^2)^{-1} \left[
		\frac{3-\gamma}{2} \left( u^{\omega,n} + u_{\omega,n}
		\right) v +
		\frac{\gamma}{2}\left( \p_x u^{\omega,n} +
		\p_x u_{\omega,n}
		\right) \p_x v
		\right], 
		\\
		& v(x,0)=0.
\end{align}
Applying $D^\sigma$ to both sides of \eqref{1.7}, multiplying by
$D^\sigma v$, and integrating, we obtain the
relation
\begin{equation}
	\begin{split}
		\frac{1}{2}\frac{d}{dt}\|v(t)\|_{H^\sigma(\ci)}^2
		& = \int_{\ci} D^\sigma E \cdot D^\sigma
		v \ dx
		\\
		& - \frac{\gamma}{2}\int_{\ci} D^\sigma
		\p_x \left[ \left( u^{\omega,n} + u_{\omega,n} \right)v
		\right]\cdot D^\sigma v \ dx
		\\
		& - \frac{3-\gamma}{2}\int_{\ci} D^{\sigma
		-2} \p_x \left[ \left( u^{\omega,n} + u_{\omega,n}
		\right)v \right] \cdot D^\sigma v \ dx
		\\
		& - \frac{\gamma}{2}\int_{\ci} D^{\sigma
		-2}
		\p_x \left[ \left( \p_x u^{\omega,n} + \p_x u_{\omega,n}
		\right)\cdot \p_x v \right] \cdot
		D^\sigma v \ dx.
		\label{X}
	\end{split}
\end{equation}
We now estimate each term of the right hand side
of \eqref{X}.
\textbf{Estimate for Term 1.} Applying Cauchy-Schwartz, we obtain
\begin{equation}
	\begin{split}
	\left |\int_{\ci} D^\sigma E \cdot D^\sigma v \ dx \right |
		& \le \|D^\sigma E \cdot D^\sigma v \|_{L^1(\ci)}
		\\
		& \le \|E\|_{H^\sigma(\ci)} \|v\|_{H^\sigma(\ci)}.
		\label{est_for_1}
	\end{split}
\end{equation}
%
\textbf{Estimate for Term 2.} We can rewrite
\begin{equation}
	\begin{split}
		-\frac{\gamma}{2} \int_{\ci} D^\sigma \p_x \left[ \left( u^{\omega,n} + u_{\omega,n}
		\right)v \right] \cdot D^\sigma v \ dx
		 = & -\frac{\gamma}{2}\int_{\ci} \left[ D^\sigma \p_x , u^{\omega,n} + u_{\omega,n}
		\right]v \cdot D^\sigma v \ dx
		\\
		& - \frac{\gamma}{2} \int_{\ci} (u^{\omega,n} + u_{\omega,n})
		D^\sigma \p_x v \cdot
		D^\sigma v \ dx.
		\label{est_for_2}
	\end{split}
\end{equation}
We now estimate \eqref{est_for_2} in parts. For the first term, we have
\begin{equation}
	\begin{split}
		\left | \frac{\gamma}{2} \int_{\ci} (u^{\omega,n} + u_{\omega,n})
		D^\sigma \p_x v \cdot
		D^\sigma v \ dx \right |
		& = \bigg | \frac{\gamma}{4}\int_{\ci} (u^{\omega,n} +
		u_{\omega,n}) \cdot \p_x (D^\sigma v)^2 \ dx \bigg |
		\\
		& = \left | -\frac{\gamma}{4} \int_{\ci} \p_x(u^{\omega,n} + u_{\omega,n}) \cdot
		(D^\sigma v)^2  \ dx \right |
		\\
		& \lesssim \|\p_x(u^{\omega,n} + u_{\omega,n}) \|_{L^\infty(\ci)}
		\|v\|_{H^\sigma(\ci)}^2.
		\label{2'}
	\end{split}
\end{equation}
To deal with the remaining term, we will need the following commutator
estimate:
\begin{theorem}
	\label{thm10}
	Assume $1<p<\infty$, $m \ge 0$. Consider a pseudo-differential operator $P
	\in \Psi^m$; then for $\rho >n/p + 1$, $s \ge 0$, and $s+m \le \rho$ we
	have
	\begin{equation}
		\begin{split}
			\|[P,f]v\|_{H^{s,p}} \le C \|f\|_{H^{\rho,p}}
			\|v\|_{H^{s+m-1,p}}.
			\label{5}
		\end{split}
	\end{equation}
\end{theorem}
%
A proof of \eqref{5} can be found in \cite{Taylor_2003_Commutator-esti}. We also have the following
corollary:
\begin{corollary}
	\label{cor1}
If $\rho > 3/2$ and $0 \le \sigma + 1 \le \rho$, then
\begin{equation}
	\begin{split}
		\|[D^\sigma \p_x ,f]v\|_{L^2(\ci)} \le C \|f\|_{H^\rho} \|v\|_{H^\sigma}.
		\label{15}
	\end{split}
\end{equation}
\end{corollary}
\begin{proof} It follows from Theorem \ref{thm10} by setting $n=1$, $p=2$,
$s=0$, and noting  that $D^\sigma \p_x \in
\Psi^{\sigma + 1}$. 
\end{proof}
Let $\sigma = 1/2 + \ee$ and $\rho = 3/2 + \ee$, where $\ee > 0$ is
arbitrarily small. Then
applying Corollary \ref{cor1}, we obtain
\begin{equation}
	\begin{split}
		\|[D^\sigma \p_x, u^{\omega,n} + u_{\omega,n}]v\|_{L^2(\ci)} \le C \|u^{\omega,n} + u_{\omega,n}
		\|_{H^{\rho}(\ci)} \|v\|_{H^\sigma(\ci)}.
		\label{6}
	\end{split}
\end{equation}
Applying Cauchy-Schwartz and estimate \eqref{6} gives
\begin{equation}
	\begin{split}
		\left | -\frac{\gamma}{2} \int_{\ci} [D^\sigma \p_x , u^{\omega,n} + u_{\omega,n}]v
		\cdot D^\sigma v \ dx \right | \lesssim \|u^{\omega,n} +
		u_{\omega,n}\|_{H^{\rho}(\ci)} \|v\|_{H^\sigma(\ci)}^2.
		\label{7}
	\end{split}
\end{equation}
Combining estimates \eqref{2'} and \eqref{7} we conclude that
\begin{equation}
	\begin{split}
		& \left | -\frac{\gamma}{2} \int_{\ci} D^\sigma \p_x \left[ \left( u^{\omega,n} + u_{\omega,n}
		\right)v \right]  \cdot D^\sigma v \ dx \right |
		\\
		& \lesssim (\|u^{\omega,n} + u_{\omega,n}\|_{H^{\rho}(\ci)} 
		 + \|\p_x u^{\omega,n} +
		\p_x u_{\omega,n}\|_{L^\infty(\rr)} ) \cdot \|v\|_{H^\sigma(\ci)}^2.
		\label{8}
	\end{split}
\end{equation}
%
\subsection{Estimate for Term 3.} Using Cauchy-Schwartz, and recalling that
$\sigma = 1/2 + \ee$,  we obtain
\begin{equation}
	\begin{split}
		\bigg | -\frac{3-\gamma}{2} \int_{\ci} D^{\sigma -2} \p_x \left[
		(u^{\omega,n} + u_{\omega,n})v \right]
		& \cdot D^\sigma v \ dx \bigg |
		\\
		& \lesssim
		\|D^{\sigma -2 } \p_x [(u^{\omega,n} + u_{\omega,n})v] \cdot D^\sigma v
		\|_{L^1(\ci)}
		\\
		& \lesssim \|D^{\sigma -2 } \p_x [(u^{\omega,n} +
		u_{\omega,n})v\|_{L^2(\ci)} \cdot \|D^\sigma v \|_{L^2(\ci)}
		\\
		& \lesssim \|(u^{\omega,n} + u_{\omega,n})v \|_{H^{\sigma -1 }(\ci)}
		\|v\|_{H^\sigma(\ci)}
		\\
		& \lesssim \|(u^{\omega,n} + u_{\omega,n})v \|_{L^2(\ci)} \|v\|_{H^\sigma(\ci)}
		\\
		& \lesssim \|u^{\omega,n} + u_{\omega,n} \|_{L^\infty(\ci)} \|v\|_{H^\sigma(\ci)}^2.
		\label{9}
	\end{split}
\end{equation}
%
\subsection{Estimate for Term 4.}
We will need the following:
\begin{lemma}
	\label{impo}
	For $1/2 < \sigma < 1 $, $f,g \in \mathcal{S'}$,
	\begin{equation}
		\begin{split}
			\|fg\|_{H^{\sigma - 1}} \le C \|f\|_{H^{\sigma}}
			\cdot \|g\|_{H^{\sigma -1}}.
			\label{11}
		\end{split}
	\end{equation}
\end{lemma}
%
	Hence, applying Cauchy-Schwartz and Lemma \ref{impo}, we obtain
	\begin{equation}
		\begin{split}
			& \left | -\frac{\gamma}{2} \int_{\ci} D^{\sigma -2 } \p_x \left[
			\left( \p_x u^{\omega,n} + \p_x u_{\omega,n} \right) \cdot \p_x v
			\right] \cdot D^\sigma v \ dx \right |
			\\
			& \lesssim \|D^{\sigma -2} \p_x [(\p_x u^{\omega,n} + \p_x
			u_{\omega,n}) \cdot \p_x v]\|_{L^2(\ci)} \cdot \|D^\sigma v
			\|_{L^2(\ci)}
			\\
			& \lesssim \|(\p_x u^{\omega,n} + \p_x u_{\omega,n}) \cdot \p_x v
			\|_{H^{\sigma -1}(\ci)} \| \cdot \|v\|_{H^\sigma (\ci)} 
			\\
			& \lesssim \|\p_x u^{\omega,n} + 
			\p_x u_{\omega,n}
			\|_{H^\sigma(\ci)} \|v\|_{H^\sigma(\ci)}^2.
			\label{12}
		\end{split}
	\end{equation}
Collecting estimates \eqref{est_for_1}, \eqref{8}, \eqref{9}, and
\eqref{12}, and applying the Sobolev Imbedding Theorem, we deduce
\begin{equation}
	\begin{split}
		\frac{1}{2}\frac{d}{dt} \|v\|_{H^\sigma(\ci)}^2
		& \lesssim
		(\|u^{\omega,n} + u_{\omega,n}\|_{H^{\rho}(\ci)} +
		\|\p_x(u^{\omega,n} + u_{\omega,n}) \|_{H^\sigma(\ci)})
		\cdot \|v\|_{H^\sigma(\ci)}^2
		\\
		& + \|E\|_{H^\sigma(\ci)}
		\|v\|_{H^\sigma(\ci)}.
		\label{10}
	\end{split}
\end{equation}
We now estimate the right hand side of \eqref{10} in parts. Applying
Proposition \ref{1n} and Theorem \ref{thm:HR_existence_continous_dependence} 
gives
\begin{equation}
	\begin{split}
		 \|u^{\omega,n} + u_{\omega,n} \|_{H^\rho(\ci)}
		 & \le
		\|u^{\omega,n}\|_{H^\rho(\ci)} + \|u_{\omega,n}\|_{H^\rho(\ci)}
		\\
		& \lesssim n^{\rho -s} + \|u^{\omega,n}(0)\|_{H^\rho(\ci)}
		\\
		& \lesssim n^{\rho -s}
		\label{3r}
	\end{split}
\end{equation}
and
\begin{equation}
	\begin{split}
		\|\p_x(u^{\omega,n} + u_{\omega,n}) \|_{H^\sigma(\ci)} 
		& \le \|u^{\omega,n} + u_{\omega,n}\|_{H^{\sigma + 1}(\ci)}
		\\
		& = \|u^{\omega,n} + u_{\omega,n}\|_{H^{\rho}(\ci)}
		\\
		& \le \|u^{\omega,n} \|_{H^{\rho}(\ci)} + \|u_{\omega,n}
		\|_{H^{\rho}(\ci)}
		\\
		& \lesssim n^{\rho -s} + \|u^{\omega,n}(0)\|_{H^{\rho}(\ci)}
		\\
		& \lesssim n^{\rho -s}.
		\label{4r}
	\end{split}
\end{equation}
Applying Lemma \ref{lem:error_of_approx_solution}, and
substituting \eqref{total-error-approx-solution}, \eqref{3r},
and \eqref{4r} into \eqref{10}, we obtain
\begin{equation}
	\begin{split}
		\frac{1}{2}\frac{d}{dt}\|v\|_{H^\sigma(\ci)}^2 \lesssim n^{\rho - s}
		\|v\|_{H^\sigma(\ci)}^2 + n^{-r_s}\|v\|_{H^\sigma(\ci)}.
		\label{200r}
	\end{split}
\end{equation}
Setting $y(t) = \|v(t)\|_{H^\sigma(\ci)}$  in \eqref{200r}, derivating the
left-hand-side, and dividing through by $y(t)$ yields
\begin{equation}
	\label{sub8}
	\frac{d}{dt} y(t) \le cn^{\rho - s}y + cn^{-r_s}, \quad c\in \rr^{+}.
\end{equation}
We multiply both sides of \eqref{sub8} by the integrating factor $e^{-tcn^{\rho - s}}$,
which gives
\begin{equation*}
	e^{-tcn^{\rho - s}}\frac{d}{dt} y(t) \le cn^{\rho - s} e^{-tcn^{\rho - s}}y
	+ cn^{-r_s}e^{-tcn^{\rho - s}}.
\end{equation*}
It follows that
\begin{equation*}
	\frac{d}{dt}\left (e^{-tcn^{\rho - s}} y \right ) \le cn^{-r_s}e^{-tcn^{\rho - s}} .
\end{equation*}
Hence,
\begin{equation*}
	\begin{split}
	\int_0^t  \frac{d}{d \tau} \left[ e^{-\tau cn^{\rho - s}} y(\tau)
	\right] \; d \tau
	& \le \int_0^t  c n^{-r_s} e^{-\tau cn^{\rho - s}}  \; d \tau
	\\
	& \le \int_0^t cn^{-r_s} \; d \tau
\end{split}
\end{equation*}
from which we obtain
\begin{equation}
	\label{almost8}
	e^{-tcn^{\rho - s}} y(t) - y(0) \le ctn^{-r_s}.
\end{equation}
Noting that $y(0)=0$, we can simplify \eqref{almost8} to obtain
\begin{equation}
	\label{gronwall-ineq8}
	y(t) \le ctn^{-r_s} e^{tcn^{\rho - s}}.
\end{equation}
Substituting back in $\|v(t)\|_{H^\sigma(\ci)}$ for $y$, we see that we are assured the
decay of $\|v(t)\|_{H^\sigma(\ci)}$ only when $s \ge \rho$. Recall that
we previously set $\sigma = 1/2 + \ee$, $\rho = 3/2 + \ee$, where $\ee$ was
chosen to be arbitrarily small. Hence, we conclude that for $s>3/2$
\begin{equation}
	\label{en-est-fin!8}
	\|v(t)\|_{H^\sigma(\ci)} 
	\lesssim
	n^{-r_s}, \quad |t| \le T,\quad n>>1 . 
\end{equation}
This concludes the proof of Lemma
\ref{lem:bound_for_difference-of-approx-actual-soln}.
\end{proof}
%
%
%
%
\subsection{Non-Uniform Dependence for $3/2<s<2$.}
%
%
%
Let $u_{\pm 1, n}$ be solutions to the HR i.v.p. with common initial data $u^{\pm 1,
n}(0)$, respectively.
We wish to show that the $H^s$ norm of the difference of $u_{\pm 1,
n}$ and the associated approximate solution $u^{\pm 1, n}$
decays as $n \to \infty$. Due to Lemma
\ref{lem:bound_for_difference-of-approx-actual-soln} we assume
$s > 3/2 $ and $\sigma = 1/2 + \ee$ for an appropriately
chosen $\ee= \ee(s) > 0$. Then by Proposition \ref{1n} and
Theorem \ref{thm:HR_existence_continous_dependence}
we obtain
\begin{equation}
	\begin{split}
		\|u_{\pm 1, n} (t) \|_{H^{2s - \sigma}(\ci)}
		& \le 2 \|u^{\pm 1, n}(0) \|_{H^{2s - \sigma}(\ci)}
		\\
		& \lesssim n^{s- \sigma}.
			\label{final-est-Hk-norm-sol}
	\end{split}
\end{equation}
Furthermore, by Proposition \ref{1n}, we have
\begin{equation}
	\begin{split}
		\|u^{\pm 1, n} (t) \|_{H^{2s - \sigma} (\ci)}
		& = \|\pm n^{-1} + n^{-s} \cos(nx \mp \gamma \omega t) \|_{H^{2s - \sigma}(\ci)}
		\\
		& \le \| \pm n^{-1} \|_{H^{2s - \sigma}(\ci)} +
		\|n^{-s} \cos(nx \mp \gamma \omega
		t) \|_{H^{2s - \sigma}(\ci)}
		\\
		& \lesssim n^{-1} + n^{s-\sigma}
		\\
		& \lesssim n^{s-\sigma}.
		\label{4}
	\end{split}
\end{equation}
		Therefore, \eqref{final-est-Hk-norm-sol}, \eqref{4}, and the triangle
		inequality yield
		\begin{equation}
			\begin{split}
				\|u^{\pm 1, n} (t) - u_{\pm 1, n}(t)\|_{H^{2s - \sigma}(\ci)}
				\lesssim n^{s-\sigma}.
				\label{5h}
			\end{split}
		\end{equation}
		Recalling
		Lemma \ref{lem:bound_for_difference-of-approx-actual-soln}, we also
		have
		\begin{equation}
			\begin{split}
				\|u^{\pm 1, n}(t) - u_{\pm 1, n} (t) \|_{H^\sigma (\ci)} 
				\lesssim n^{-r_s}
				\label{6h}.
			\end{split}
		\end{equation}
		We now wish to interpolate in order to obtain an estimate of the $H^s (\ci)$
		norm of the difference of the approximate and actual solutions:
		\begin{lemma}
			\label{7h}
			For all $\psi \in L^2(\ci)$,
			\begin{equation*}
				\|\psi \|_{H^s (\ci)} \leq \left( \| \psi \|_{H^k (\ci)} \| \psi
				\|_{H^{2s - k}} \right)^{\frac{1}{2}}.
			\end{equation*}
		\end{lemma}
		\begin{proof}
			\begin{equation*}
				\begin{split}
					\|\psi\|^2_{H^s (\ci)} & = \sum_{\xi \in \zz} (1 + 
					\xi^2)^s |\hat{\psi}(\xi) |^2
				\\
				& = \sum_{\xi \in \zz} (1 + \xi^2)^{s - 
				\frac{k}{2}}|\hat{\psi}(\xi)| \cdot
				( 1 + \xi^2)^{\frac{k}{2}}|\hat{\psi}(\xi)|
				\\
				& \le \|\psi\|_{H^k(\ci)}
				\|\psi\|_{H^{2s - k}(\ci)}
			\end{split}
			\end{equation*}
			from which we obtain 
		\begin{equation*}
				\|\psi \|_{H^s (\ci)} \leq \left( \| \psi \|_{H^k (\ci)} \| \psi
				\|_{H^{2s - k}} \right)^{\frac{1}{2}}. \qquad \Box
			\end{equation*}
			Applying Lemma \ref{7h}, and estimates \eqref{5h} and \eqref{6h}
			yields
			\begin{equation}
				\label{comp-200}
				\begin{split}
					\|u^{\pm 1,n}(t) - u_{\pm 1, n}(t) \|_{H^s (\ci)}
					& \le ( \| u^{\pm 1,n}(t)
					- u_{\pm 1, n}(t) \|_{H^\sigma (\ci)}
					\\
					& \cdot \| u^{\pm 1,n}(t)
					- u_{\pm 1, n}(t)\|_{H^{2s - \sigma}(\ci)} )^{\frac{1}{2}}
					\\
					& \lesssim (n^{-r_s} \cdot n^{s-\sigma})^{\frac{1}{2}}.
				\end{split}
			\end{equation}
			Recalling \eqref{r_s-definition}, we see that for $s \le 3$,
			\eqref{comp-200} reduces to 
			\begin{equation}
				\begin{split}
					\|u^{\pm 1,n}(t) - u_{\pm 1, n}(t) \|_{H^s (\ci)}
					& \lesssim (n^{-2(s-1)} \cdot n^{s-\sigma})^{\frac{1}{2}}
					\\
					& \lesssim n^{(2-s - \sigma)/2}
					\label{8h}
				\end{split}
			\end{equation}
			and for $s > 3$ reduces to 
			\begin{equation}
				\begin{split}
					\|u^{\pm 1,n}(t) - u_{\pm 1, n}(t) \|_{H^s (\ci)}
					& \lesssim \left( n^{-\left( s+1 \right)} \cdot
					n^{s-\sigma}
					\right)^{\frac{1}{2}}
					\\
					& \lesssim n^{(-1-\sigma)/2}.
					\label{9h}
				\end{split}
			\end{equation}
			Since $s > 3/2 $ by assumption, we now recall \eqref{8h},
			\eqref{9h}, and the
			relation $\sigma = 1/2 + \ee(s)$ to obtain 
			\begin{equation}
				\begin{split}
					\|u^{\pm 1,n}(t) - u_{\pm 1, n}(t) \|_{H^s (\ci)} \lesssim
					n^{-\ee(s)/2}.
					\label{10h}
				\end{split}
			\end{equation}
      which concludes the proof. 
    \end{proof}
%
%
%%%%%%%%%%%%%% Behavior at time  t = 0  %%%%%%%%%%%% 
%  
%
\textbf{Behavior at time $t=0$.}  We have
%
%
\begin{equation} 
	\label{HR-slns-differ-t-0} 
	\|
	u_{1, n}(0)
	-
	u_{-1, n}(0)
	\|_{H^s(\ci)}
	=
	\|
	2   n^{-1}
	\|_{H^s(\ci)}
	\simeq
	n^{-1}
	\longrightarrow 
	0
	\,\,
	\text{as}
	\,\,
	n \to \infty.
\end{equation}
%
%
%%%%%%%%%%%%%% Behavior at time  t >0  %%%%%%%%%%%% 
%  
%
\textbf{Behavior at time  $t>0$.}  Using the reverse triangle inequality, we 
have
%
%
%
\begin{equation} 
	\label{HR-slns-differ-t-pos}
	\begin{split}
		\|
		u_{1, n}(t)
		-
		u_{- 1, n}(t)
		\|_{H^s(\ci)}
		&
		\ge
		\|
		u^{1, n}(t)
		-
		u^{- 1, n}(t)
		\|_{H^s(\ci)}
		\\
		&
		-
		\|
		u^{1, n}(t)
		-
		u_{1, n}(t)
		\|_{H^s(\ci)}
		\\
		&
		-
		\|
		-u^{-1, n}(t)
		+
		u_{-1, n}(t)
		\|_{H^s(\ci)} .
	\end{split}
\end{equation}
%
%
Using estimate \eqref{10h} for the last two terms 
in \eqref{HR-slns-differ-t-pos} we obtain
%
%
%
\begin{equation} 
	\label{HR-slns-differ-t-pos-est}
	\|
	u_{1, n}(t)
	-
	u_{- 1, n}(t)
	\|_{H^s(\ci)}
	\ge
	\|
	u^{1, n}(t)
	-
	u^{- 1, n}(t)
	\|_{H^s(\ci)}
	-
	c n^{- \ee(s)/2}
\end{equation}
where $c \in \rr^+$. Letting $n$ go to $\infty$ in
\eqref{HR-slns-differ-t-pos-est}
yields
%
\begin{equation} 
	\label{HR-slns-to-ap-est}
	\liminf_{n\to\infty}
	\|
	u_{1, n}(t)
	-
	u_{- 1, n}(t)
	\|_{H^s(\ci)}
	\ge
	\liminf_{n\to\infty}
	\|
	u^{1, n}(t)
	-
	u^{- 1, n}(t)
	\|_{H^s(\ci)}.
\end{equation}
%
%
Hence, by \eqref{HR-slns-to-ap-est}, we see that in order to find a lower bound for
the difference of the unknown solution sequences it is
sufficient to find a lower bound for the difference of the
associated approximate solutions. We set out to do so; using the identity 
$$
\cos \alpha -\cos \beta
=
-2
\sin(\frac{\alpha + \beta}{2})
\sin(\frac{\alpha - \beta}{2})
$$
we obtain
$$
u^{1, n}(t)
-
u^{- 1, n}(t)
=
2
n^{-1}
+
2
n^{-  s}
\sin (n x) \sin (\gamma t).
$$
Therefore
%
%
\begin{equation} 
	\label{B--ap-below-est-1}
	\|
	u^{1, n}(t)
	-
	u^{- 1, n}(t)
	\|_{H^s(\ci)}
	\ge
	2
	n^{  -  s}
	\|
	\sin (n x)
	\|_{H^s(\ci)}
	|\sin \gamma t|.
\end{equation}
%
%
Letting $n$ go to $\infty$, Proposition \ref{1n}
and \eqref{B--ap-below-est-1} imply
%
%
\begin{equation} 
	\label{HR-ap-below-est}
	\liminf_{n\to\infty}
	\|
	u^{1, n}(t)
	-
	u^{- 1, n}(t)
	\|_{H^s(\ci)}
	\gtrsim
	|\sin \gamma t|.
\end{equation}
%
%
Combining  \eqref{HR-slns-to-ap-est} and  \eqref{HR-ap-below-est}
yields
%
%
\begin{equation} 
	\label{HR-slns-below-est-fin}
	\liminf_{n\to\infty}
	\|
	u_{1, n}(t)
	-
	u_{- 1, n}(t)
	\|_{H^s(\ci)}
	\gtrsim
	|\sin \gamma t|
\end{equation}
%
%
proving \eqref{bdd-away-from-0}. Furthermore, by Proposition \ref{1n} and
Theorem \ref{thm:HR_existence_continous_dependence}  we obtain
\begin{equation}
	\begin{split}
		\label{solutions-are-small}
		\|u_{\pm 1, n} (t) \|_{H^{s}(\ci)}
		& \lesssim \|u^{\pm 1, n}(0) \|_{H^{s}(\ci)}
		\\
		& \lesssim 1
	\end{split}
\end{equation}
Collecting  \eqref{HR-slns-differ-t-0}, \eqref{HR-slns-below-est-fin}, and
\eqref{solutions-are-small}, we conclude that we have proven Theorem
\ref{hr-non-unif-dependence} for the periodic case.
%
%
%
%
%
%	
	 %%%%%%%%%%%%%%%%%%%%%%%%%%%%%%%%%
	 %
	 %
	 %
	 %   Proof of  Theorem in periodic case for s greater or equal to 2
	 %
	 %
	 %
	 %%%%%%%%%%%%%%%%%%%%%%%%%%%%%%%%%%%
	\section{A Proof of Non-Uniform Dependence In the Periodic Case For $s\ge 2$}
	In this situation we use the following initial value problem
	satisfied  by the difference
	$
	v=
		u^{\omega, n}(t) 
				- 
				u_{\omega, n}(t)
				$:
				%
				%
				%
	\begin{align}
	\label{v-eq}
					\p_t v 
					& = E + \gamma(v \p_x v - v \p_x u^{\omega,n} - u^{\omega,n} \p_x v) 
					\\
					& + \p_x\left( 1 - \p_x^2 \right)^{-1} \left[ \frac{3-
					\gamma}{2}v^2 + \frac{\gamma}{2}\left( \p_x v \right)^2 - \left(
					3 - \gamma \right)u^{\omega,n} v -
					\gamma \p_x u^{\omega,n} \p_x v \right],
					\notag
					\\
					v(0) & =0
					\label{v-data}
			\end{align}
							%
	%
	%
and prove the following key result:
%
\begin{lemma}
	\label{lem:bound_for_difference-of-approx-and-actual-soln}
If $s \ge 2$ then
			\begin{equation} 
				\|
				v(t)
				\|_{H^1(\ci)}
				=
				\label{differ-H1-est} 
				\|
				u^{\omega, n}(t) 
				- 
				u_{\omega, n}(t)
				\|_{H^1(\ci)}
				\lesssim 
				n^{-r_s}, 
				\quad
				|t| \le T.
			\end{equation}
			%
			\end{lemma}
%
			\subsection{{Proof.}} 
Straightforward computations give		
\begin{equation}
	\label{energy-estimate-simplified}
	\begin{split}
		\frac{d}{dt} \|v(t)\|_{H^1(\ci)}^2
		& = 2 \int_{\ci}
		 v\left( 1-\p_x^2
	\right)E \; dx
	\\
	&+ 2 \gamma  \int_{\ci}  v\left( 1-\p_x^2 \right)\left( v \p_x u^{\omega,n}
	- u^{\omega,n} \p_x v
	\right) \; dx
	\\
	& - 2 \int_{\ci} \left[ \left( 3-\gamma \right)v \p_x \left( u^{\omega,n}v \right) + \gamma v
	\p_x \left( \p_x u^{\omega,n} \p_x v \right)\right] \; dx.
\end{split}
\end{equation}
Applying Cauchy-Schwartz and the inequality $ab + cd \le (a^2 +
c^2)^{\frac{1}{2}}(b^2 + d^2)^{\frac{1}{2}}$ for $a,b,c,d \in \rr$, we
obtain
\begin{equation}
	\begin{split}
		\label{energy-estimate-best}
		\frac{d}{dt} \|v(t)\|_{H^1(\ci)}^2
		& \lesssim \left( \|u^{\omega,n}\|_{L^\infty(\ci)} + \|
		\p_x u^{\omega,n} \|_{L^\infty(\ci)} + \|\p_x^2 u^{\omega,n} \|_{L^\infty (\ci)} \right)
		\|v\|_{H^1(\ci)}^2 
		\\
		&+ \|v\|_{H^1(\ci)} \|E\|_{H^1(\ci)}.
	\end{split}
\end{equation}
Since we have
\begin{equation}
	\label{L-infty-error}
	\begin{split}
		\|u^{\omega,n} \|_{L^\infty(\ci)} + \|\p_x u^{\omega,n} \|_{L^\infty(\ci)}
		+ \|\p_x^2 u^{\omega,n} \|_{L^\infty(\ci)}
		& \lesssim 
		n^{-s} + n^{-s+1} + n^{-s + 2} \\
		& \lesssim n^{-s + 2}  ,
	\end{split}
\end{equation}
substituting \eqref{L-infty-error} and \eqref{total-error-approx-solution} into
\eqref{energy-estimate-best} gives
\begin{equation}
	\label{en-est-fin!}
	\frac{d}{dt} \|v(t)\|_{H^1(\ci)}^2 \lesssim n^{-s+2} \|v\|_{H^1(\ci)}^2 + n^{-r_s}
	\|v \|_{H^1(\ci)}
\end{equation}
where $r_s$ is defined in \eqref{r_s-definition}.
Applying Gronwall's inequality completes the proof. $\Box$
%
Next, note that Proposition \ref{1n}, Theorem
\ref{thm:HR_existence_continous_dependence}, and the triangle inequality
yield
\begin{equation}
	\begin{split}
		\|u^{\pm 1, n} (t) - u_{\pm 1, n}(t)\|_{H^{2s - 1}(\ci)}
		\lesssim n^{s-1}.
		\label{5hprimus}
	\end{split}
\end{equation}
Hence, interpolating and applying Lemma
\ref{lem:bound_for_difference-of-approx-and-actual-soln} and
\eqref{5hprimus}, we obtain
		%
			\begin{equation*}
				\begin{split}
					\|u^{\pm 1,n}(t) - u_{\pm 1, n}(t) \|_{H^s (\ci)}
					& \le ( \| u^{\pm 1,n}(t)
					- u_{\pm 1, n}(t) \|_{H^1 (\ci)}
					\\
					& \cdot \| u^{\pm 1,n}(t)
					- u_{\pm 1, n}(t)\|_{H^{2s-1}(\ci)} )^{\frac{1}{2}}
					\\
					& \lesssim (n^{-r_s} \cdot n^{s-1})^{\frac{1}{2}}.
				\end{split}
			\end{equation*}
			Recalling \eqref{r_s-definition}, we see that for $s \ge 2$ this reduces to
			\begin{equation}
				\begin{split}
					\|u^{\pm 1,n}(t) - u_{\pm 1, n}(t) \|_{H^s (\ci)} \lesssim
					n^{-\frac{1}{2}}.
					\label{10v}
				\end{split}
			\end{equation}
%
The rest of the proof is the same as in the case $s>3/2$.
%
%
%	
%
%
%
%
%
%


\section{Well-Posedness for HR in the Periodic Case}
%
%
%
%
We will now prove well-posedness for the periodic case, after which we will
provide the necessary details to extend the argument to the non-periodic case.
\subsection{Existence.}
\label{existence}
Here we will prove the existence of a solution to the HR i.v.p. and inequalities
\eqref{Life-span-est} and \eqref{u_x-Linfty-Hs}.  We begin by mollifying the HR equation, so that we may apply the following ODE
theorem: 
%
\begin{theorem}
	\label{ode_theorem}
	Let  $Y$  be a Banach space, $X\subset Y$ be an open subset,
	$I' \subset \rr$, and $f: I' \times X\to Y$ a continuously differentiable
	map.  Then for any $t_{0} \in I'$ and $x_{0} \in X$ there exists an
	open ball $I \subset I'$ and a unique differentiable mapping $u:I
	\to Y$ such that for all $t \in I$,  $u'(t) = f(t, u)$
	and $u(t_{0}) = x_{0}.$
\end{theorem}
%
To see why we cannot apply the Banach Space ODE Theorem to the HR equation as is,
we use a counterexample. Let $u=x^{-1/2} \chi_{[0,1]}$ and $s=0$. Then $u \in H^s$ but
$u\p_x u \notin H^s$. Hence, returning to the general case, we see that the
HR equation as is can not be thought as an ODE on the space $H^s$.  To
deal with this problem we will replace the i.v.p \eqref{hr}--\eqref{hr-data} by  
\begin{equation}
	\label{hr-moli}
	\p_t  u_\ee =
	-\gamma J_\ee u_\ee \partial_x  J_\ee  u_\ee - \p_x (1-\p_x^2)^{-1} 
	\left [\frac{3-\gamma}{2}u^2 + \frac{\gamma}{2}(\p_x u)^2 \right ],
\end{equation} 
%
\begin{equation} 
	\label{hr-moli-data} 
	u_\ee(x, 0) = u_0 (x),
\end{equation}
%
where $J_\ee$ is defined as follows: Pick a non-negative $j(x) \in
\mathcal{S}(\rr)$ and let
\begin{equation*}
	\begin{split}
		j_\ee(x) = \frac{1}{\ee}j\left( \frac{x}{\ee} \right).
	\end{split}
\end{equation*}
	We then define $J_\ee$ to be the ``Friedrichs mollifier''
	\begin{equation}
		\begin{split}
			J_\ee f(x) = j_\ee * f(x), \quad \ee>0.
		\end{split}
	\end{equation}
%
%
Notice that the right hand side of \eqref{hr-moli} is a map from $H^s(\ci)$
to $H^s(\ci)$.  In order to apply the ODE Theorem, we will also need to
show that it is a continuously differentiable map:
%
%
%
\begin{lemma}
	Let $f_\ee:H^s(\ci) \to H^s(\ci)$ be given by 
	\begin{equation}
		\label{f_ep}
		f_{\ee}(u) = -\gamma  J_\varepsilon u \partial_x J_\varepsilon u
		- \p_x (1-\p_x^2)^{-1} \left
		[\frac{3-\gamma}{2}u^2 + \frac{\gamma}{2}(\p_x u)^2 \right ].
	\end{equation}
	Then $f_\ee$  is a continuously differentiable map.
\end{lemma}
%
%
\subsection{ Proof.} We explicitly calculate the derivative of $f_\ee$ at an
arbitrary $w \in H^s(\ci)$:
\begin{equation*}
	\begin{split}
		[Df_{\ee}(u)](w)
		=
		& -\gamma (J_\varepsilon w \cdot \partial_x J_\varepsilon u +
		J_\varepsilon u \cdot \partial_x J_\varepsilon w)
		\\
		& - (1-\p_x ^2)^{-1}
		\p_x \left [(3-\gamma)w u + \gamma\p_x w \p_x u \right ].
	\end{split}
\end{equation*}
Let $w_n \xrightarrow{H^s(\ci)} w$. Then it is easy to check that
%
\begin{equation}
	\begin{split}
		& -\gamma (J_\varepsilon w_n \cdot \partial_x J_\varepsilon u 
		+ J_\varepsilon u \cdot \partial_x J_\varepsilon w_n)
		+ (1-\p_x ^2)^{-1}
		\p_x \left [(3-\gamma)w_n u + \gamma\p_x w_n \p_x u \right ]
		\\
		& \xrightarrow{H^s(\ci)} 
		 -\gamma (J_\varepsilon w \cdot \partial_x J_\varepsilon u 
		+ J_\varepsilon u \cdot \partial_x J_\varepsilon w) + (1-\p_x ^2)^{-1}
		\p_x \left [(3-\gamma)w u + \gamma\p_x w \p_x u \right ].
	\end{split}
\end{equation}
This concludes the proof. $\quad \square$
Hence, by Theorem \ref{ode_theorem}, for each $\ee > 0$ there exists a
unique solution $u_\ee \in C(I, H^s(\ci))$ satisfying the Cauchy-problem
\eqref{hr-moli}-\eqref{hr-moli-data}. Next, we analyze the size and
lifespan of the family $\{u_\ee\}$ of solutions.
%%%%%%%%%%%%%%%%%%%%%%%%
%
%     Estimates  for Life-span and Sobolev norm of $u_\ee$
%
%%%%%%%%%%%%%%%%%%%%%%%%
%
%
\subsection{ Estimates  for Life-span and Sobolev norm of $u_\ee$.}
%
We will show that there is a lower bound  $T$
for $T_\ee$, which is  independent of $\ee\in(0, 1]$.
This is based on the following differential
inequality for the solution $u_\ee$:
%
\begin{equation} 
	\label{B-diff-ineq}
	\frac 12
	\frac{d}{dt}
	\|u_\ee(t)\|_{H^{s}(\ci)}^2
	\le
	c_s
	\|u_\ee(t)\|_{H^{s}(\ci)}^3,
	\quad
	|t| \le T_\ee.
\end{equation}
%
%
We will prove this inequality  by
following the approach used for quasilinear symmetric
hyperbolic systems in Taylor \cite{Taylor_1991_Pseudodifferent}. In what follows we will suppress the
$t$ parameter for the sake of clarity.
%
For any $s\in \ci$ let   $D^s=(1-\p_x^2)^{s/2}$ be the  operator
defined by 
%
$$ \widehat{D^s f}(\xi) \doteq (1 + \xi^2)^{s/2} \widehat{f}(\xi), $$
%
where $ \widehat{f}$ is the Fourier transform
%
$$ \widehat{f}(\xi) =  \frac{1}{2\pi}\int_{\ci} e^{-i \xi x} f(x) \ dx.  $$
%
Applying the operator $D^s$ to  both sides of  \eqref{hr-moli},
then  multiplying the resulting equation by $D^s J_\ee u_\ee$
and integrating it for $x\in\ci$ gives
%
\begin{equation} 
	\begin{split}
		\label{B-moli-int}
		\frac 12
		\frac{d}{dt} \|u_\ee \|_{H^s}^2
		=
		&-
		\gamma \int_{\ci}  D^s(J_\ee u_\ee \partial_x J_\ee u_\ee) \cdot
		D^s J_\ee u_\ee  \  dx
		\\
		&- \frac{3 -\gamma}{2} \int_{\ci} D^{s-2} \p_x (u_{\ee}^2) 
		\cdot D^s J_\ee u_{\ee} \ dx
		\\
		&- \frac{\gamma}{2} \int_{\ci}  D^{s-2} \p_x (\p_x u_\ee)^2
		\cdot D^s J_\ee u_\ee  \ dx.
	\end{split}
\end{equation}
%
We will estimate the right hand side of \eqref{B-moli-int} in parts. In
what follows next we use the fact that  $D^s$ and $J_\ee$ commute and
that  $J_\ee$ satisfies the properties 
%
\begin{equation} 
	\label{J-e-inner-prod-property}
	(J_\ee f, g)_{L^2(\ci)}=( f, J_\ee g)_{L^2(\ci)}
\end{equation}
%
and
%
\begin{equation} 
	\label{Je-u-Hs}
	\| J_\ee u \|_{H^s(\ci)}
	\le
	\|  u \|_{H^s(\ci)}.
\end{equation}
%
%%%%%%%%%%%% Burgers term energy estimate %%%%%%%%%%%%
%
%
%
\noindent
Letting 
%
\begin{equation} 
	\label{v-Je-ue}
	v=J_\ee u_\ee
\end{equation}
%
%
we have
%
\begin{equation} 
	\begin{split}
		\label{B-moli-int-v}
		& -  \gamma \int_{\ci}   D^s (J_{\ee} u_{\ee} \p_x J_\ee u_\ee)
		 \cdot D^s
		J_{\ee}u_\ee \ dx  
		\\
		& = - \gamma \int_\ci
		 D^s(v \partial_x v) \cdot   D^s v \ dx
		\\
		& = - \gamma \int_\ci
		\left [ 
		D^s(v\p_x v)  -  v D^s (\p_xv)
		\right ] 
		D^s v \ dx - \gamma \int_\ci
		v D^s (\p_xv)
		D^s v \ dx.
	\end{split}
\end{equation}
%
%
%
We now estimate \eqref{B-moli-int-v} in parts. Applying the Cauchy-Schwarz inequality gives
%
\begin{equation} 
	\label{int1-est-calc2}
	\begin{split}
		& \Big|
		- \gamma \int_\ci
		\big[ 
		D^s(v\p_x v)  -  v D^s (\p_xv)
		\big]
		D^s v   \, dx
		\Big|
		\\
		& \le
		|\gamma| \cdot \|
		D^s(v\p_x v)  -  v D^s (\p_xv)
		\|_{L^2(\ci)}
		\|
		D^s v 
		\|_{L^2(\ci)}
		\\
		&\le
		|\gamma| \cdot \|
		D^s(v\p_x v)  -  v D^s (\p_xv)
		\|_{L^2(\ci)}
		\|
		v
		\|_{H^s(\ci)}
		\\
		&\le c_s \| \p_x v \|_{L^\infty(\ci)} 
		\| v \|_{H^s(\ci)}^2,
	\end{split}
\end{equation}
%
where the last step follows from 
%
\begin{equation} 
	\label{int1-est-calc3}
	\| D^s(v\p_x v)  -  v D^s (\p_xv) \|_{L^2(\ci)}
	\le
	2 c_s^{\prime}    \| \p_x v \|_{L^\infty(\ci)} 
	\| v \|_{H^s(\ci)},
\end{equation}
which we prove below by using the following Kato-Ponce commutator 
estimate:  
\begin{lemma} 
	\label{KP-lemma}
	[Kato-Ponce]
	If  $s>0$ then there is $c_s^{\prime}>0$ such that 
	%
	\begin{equation} 
		\label{KP-com-est}
		\| D^{s} \big(fg) -  f D^s g\|_{L^2(\ci)}
		\le
		c_s^{\prime}\big(
		\| D^{s}f \|_{L^2(\ci)}    \| g \|_{L^\infty(\ci)} 
		+
		\| \p_xf \|_{L^\infty(\ci)}    \| D^{s-1}g \|_{L^2(\ci)}   
		\big).
	\end{equation}
	%
	\end{lemma}
	%
	%
	In fact, applying  this estimate with $f=v$ and $g=\p_xv$ gives 
	%
	\begin{equation} 
		\label{int1-est-calc4}
		\begin{split}
			& \| D^s(v\p_x v)  -  v D^s (\p_xv) \|_{L^2(\ci)}
			\\
			& \le
			{c_s}^\prime \big(
			\| D^{s}v \|_{L^2(\ci)}    \| \p_x v \|_{L^\infty(\ci)} 
			+
			\| \p_xv \|_{L^\infty(\ci)}    \| D^{s-1}\p_x v \|_{L^2(\ci)}   
			\big)
			\\
			& \le
			2{c_s}^\prime    \| \p_x v \|_{L^\infty(\ci)} 
			\| v \|_{H^s(\ci)}, 
		\end{split}
	\end{equation}
	%
	which  is the desired estimate  \eqref{int1-est-calc3}.
	Next, we have
	%
	%
	%
	\begin{equation} 
		\label{int1-est-calc5}
		\begin{split}
			\Big|
			-\gamma \int_\ci
			v D^s (\p_x v)
			\cdot  D^s v \ dx
			\Big|
			& =
			\left | \frac{\gamma}{2} \right | \cdot \Big|
			\int_\ci
			v \p_x\left(D^s v\right)^2  dx
			\Big|
			\\
			& =
			\left | \frac{\gamma}{2} \right | \cdot \Big | \int_\ci
			\p_x v \, (D^s v)^2 \ dx
			\Big|
			\\
			& \le
			\left | \frac{\gamma}{2} \right |  \cdot \int_\ci
			\Big | \p_x v \, (D^s v)^2   
			\Big| \ dx
			\\
			& \lesssim
			\| \p_x v \|_{L^\infty(\ci)} 
			\| v \|_{H^s(\ci)}^2.
		\end{split}
	\end{equation}
	%
	%
	%
	Combining inequalities  \eqref{int1-est-calc2} and
	\eqref{int1-est-calc5} and applying the Sobolev Imbedding Theorem, we
	have
	%
	\begin{equation} 
		\label{burgers_est'}
		\begin{split}
			\Big|
			-\gamma \int_\ci
			D^s(v \partial_x v) \cdot   D^s v \, dx  
			\Big|
			&\le
			{c_s}^\prime
			\| \p_x v \|_{L^\infty(\ci)} 
			\|  v \|_{H^s(\ci)}^2
			\\
			& \le {c_s}^\prime \| v \|_{C^1(\ci)} \| v \|_{H^s(\ci)}^2
			\\
			& \le {c_s}^{\prime \prime} \| v \|_{H^s(\ci)}^3
			\\
			& \le {c_s}^{\prime \prime} \| u_\ee \|_{H^s(\ci)}^3.
		\end{split}
	\end{equation}
	%
	Next we estimate
	\begin{equation}
		\begin{split}
			\left | - \frac{3 -\gamma}{2} \int_\ci D^{s-2} \p_x u_\ee^2 \cdot
			D^s J_\ee u_\ee \; dx \right |
			& \le \left | \frac{3- \gamma}{2} \right | \int_\ci \left |
			D^{s-2} \p_x u_\ee^2 \cdot D^s J_\ee u_\ee \; dx \right | 
			\\
			& \le \left | \frac{3- \gamma}{2} \right |
			\|D^{s-2} \p_x u_\ee^2 \|_{L^2(\ci)} 
			\|D^s J_\ee u_\ee \|_{L^2(\ci)}
			\\
			& \le \left | \frac{3- \gamma}{2} \right |
			\|D^{s-1} u_\ee^2 \|_{L^2(\ci)} 
			\|D^s u_\ee \|_{L^2(\ci)}
			\\
			& \lesssim \| u_\ee^2 \|_{H^s(\ci)} \| u_\ee \|_{H^s(\ci)}.
		\end{split}
	\end{equation}
	%
	%
	Applying the algebra property, we obtain
	%
	\begin{equation}
		\label{hl1}
		\begin{split}
			\left | - \frac{3 -\gamma}{2} \int_\ci D^{s-2} \p_x u_\ee^2 \cdot
			D^s J_\ee u_\ee \; dx \right |
			\lesssim \| u_\ee \|_{H^s(\ci)}^3.
		\end{split}
	\end{equation}
	%
	%
	We also have
	\begin{equation}
		\begin{split}
			\left |- \frac{\gamma}{2} \int_\ci D^{s-2} \p_x (\p_x u_\ee)^2 \cdot
			D^s J_\ee u_\ee \; dx \right |
			& \le \left | \frac{\gamma}{2} \right | \int_\ci \left | D^{s-2} \p_x (\p_x u_\ee)^2 \right |
			\cdot \left |D^s J_\ee u_\ee \right | \; dx
			\\
			& \le \left | \frac{\gamma}{2} \right |
			\| D^{s-1} (\p_x u_\ee)^2 \|_{L^2(\ci)}
			\| D^s J_\ee u_\ee \|_{L^2(\ci)}
			\\
			& \lesssim \|(\p_x u_\ee)^2 \|_{H^{s-1}(\ci)}
			\| J_\ee u_\ee \|_{H^{s-1}(\ci)} 
			\\
			& \lesssim \|(\p_x u_\ee)^2 \|_{H^{s-1}(\ci)} \| u_\ee \|_{H^{s-1}(\ci)} 
		\end{split}
	\end{equation}
	and applying the algebra property yields
	\begin{equation}
		\label{hl2}
		\begin{split}
		\left | - \frac{\gamma}{2} \int_\ci D^{s-2} (\p_x u_\ee)^2 \cdot
		D^s J_\ee u_\ee \; dx \right |
		& \lesssim \| \p_x u_\ee \|_{H^{s-1}(\ci)}^2 \| u_\ee \|_{H^s(\ci)} 
		\\
		& \lesssim \|u_\ee\|_{H^s(\ci)}^3.
	\end{split}
	\end{equation}
	%
	Combining \eqref{burgers_est'}, \eqref{hl1}, and \eqref{hl2}, we obtain
	\eqref{B-diff-ineq}.
	%%%%%%%%%%%%%%%%%%%%%%%%%%%%%%%%%%%
	%  
	%           Lifespan for CH  solution    
	% 
	%%%%%%%%%%%%%%%%%%%%%%%%%%%%%%%%%%%
	%
	%
	%   
	%
	\noindent
	\subsection{  Lifespan estimate of $u_\ee$.} To derive an explicit formula for
	$T_\ee$ we proceed as follows.  Letting  $y(t)=
	\|u_\ee(t)\|_{H^s(\ci)}^2$ inequality  \eqref{B-diff-ineq} takes the
	form
	%
	\begin{equation} 
		\label{energy-y-ineq}
		\frac 12
		y^{-3/2}\frac{dy}{dt}
		\le
		c_s,
		\qquad
		y(0)=y_0=  \|u_0\|_{H^s(\ci)}^2.
	\end{equation}
	%
	Suppose $t$ is non-negative. Integrating  \eqref{energy-y-ineq} from  0  to $t$ gives
	%
	\begin{equation} 
		\label{energy-y-ineq-calc1}
		\frac{1}{\sqrt{y_0}}  - \frac{1}{\sqrt{y(t)}} 
		\le
		c_s t.
	\end{equation}
	%
	%
	Replacing $y(t)$ with   $\|u_\ee(t)\|_{H^s(\ci)}^2$  and solving for  $\|u_\ee(t)\|_{H^s(\ci)}$
	we obtain the formula
	%
	\begin{equation} 
		\label{norm-u(t)-formula}
		\|u_\ee(t)\|_{H^s(\ci)}
		\le
		\frac{ \|u_0\|_{H^s(\ci)}}{1-c_s\|u_0\|_{H^s(\ci)} t}, \quad t\ge
		0.
	\end{equation}
	%
	Now, from \eqref{norm-u(t)-formula} we see that  $\|u_\ee(t)\|_{H^s(\ci)}$ is finite  if 
	%
	\begin{equation*} 
		\label{Lifespan-calc1}
		c_s    \|u_0\|_{H^s(\ci)} t<1,
	\end{equation*}
	%
	or
	%
	\begin{equation} 
		t
		<
		\frac{1}{ c_s \|u_0\|_{H^s(\ci)}}.
	\end{equation}
	%
	Similarly, if $t$ is negative, then 
	\begin{equation} 
		\label{norm-u(t)-formula-prime}
		\|u_\ee(t)\|_{H^s(\ci)}
		\le
		\frac{ \|u_0\|_{H^s(\ci)}}{1+c_s\|u_0\|_{H^s(\ci)} t}, \quad t < 0.
	\end{equation}
	from which it follows that $\|u_\ee(t)\|_{H^s(\ci)}$ is finite  if 
	%
	\begin{equation} 
		t
		>
		 \frac{-1}{ c_s \|u_0\|_{H^s(\ci)}}.
	\end{equation}
	Therefore, the  solution  $u_\ee(t)$ to the mollified CH Cauchy
	problem exists for $|t| <T_0$, where
	%
	\begin{equation} 
		\label{CH-Lifespan}
		T_0
		=
		\frac{1}{ c_s \|u_0\|_{H^s(\ci)}}.
	\end{equation}
	%
	%%%%%%%%%%%%%%%%%%%%%%%%%%%%%%%%%%%
	%  
	%            Norm of   
	% 
	%%%%%%%%%%%%%%%%%%%%%%%%%%%%%%%%%%%
	%
	%
	%   
	%
	\noindent
	\subsection{Size of the solution estimate} If we choose  $T=\frac12 T_0$, that is
	%
	\begin{equation} 
		\label{T-def}
		T
		=
		\frac{1}{2 c_s \|u_0\|_{H^s(\ci)}},
	\end{equation}
	%
	then for $|t| \le T$, estimates \eqref{norm-u(t)-formula} and
	\eqref{norm-u(t)-formula-prime} imply 
	%
	\begin{equation*} 
		\label{u(t)-u(0)-bound}
		\|u_\ee(t)\|_{H^s(\ci)}
		\le
		\frac{ \|u_0\|_{H^s(\ci)}}{1-(c_s\|u_0\|_{H^s(\ci)})/(2 c_s \|u_0\|_{H^s(\ci)})},
	\end{equation*}
	%
	or 
	%
	\begin{equation} 
		\|u_\ee(t)\|_{H^s(\ci)}
		\le
		  2 \|u_0\|_{H^s(\ci)},
		\quad 
		|t| \le T.
	\end{equation}
	%
	Thus we have obtained a lower bound for $T_\ee$ and an upper bound for
	$\|u_\ee(t)\|_{H^s(\ci)}$ independent of $\ee\in (0, 1]$. The following
	lemma summarizes these results and provides an estimate for the
	$H^{s-1}(\ci)$ norm of $\p_t u_\ee(t)$:
	%
	%
	\begin{lemma}
		\label{hr_wp}
		Let  $u_0(x) \in  H^s(\ci)$, $s >3/2$. Then for any $\ee\in (0, 1]$
		the i.v.p. for the mollified HR equation 
		%
		\begin{equation} 
			\label{hr-moli-2}
			\partial_t  u_\ee 
			=
			-\gamma (J_\ee u_\ee \partial_x  J_\ee  u_\ee) - \p_x (1-\p_x^2)^{-1} \left
			[\frac{3-\gamma}{2}(u_\ee)^2 + \frac{\gamma}{2}(\p_x u_\ee)^2
			\right ], 
		\end{equation} 
		%
		\begin{equation} 
			\label{burgers-moli-data-2} 
			u_\ee(x, 0) = u_0 (x),
		\end{equation}
		%
		has a unique solution $u_\ee( t)\in C([-T, T]; H^s(\ci))$. 
		In particular,
		%
		\begin{equation} 
			\label{life-est}
			T
			=
			\frac{1}{2 c_s \|u_0\|_{H^s(\ci)}},
		\end{equation}
		%
		is independent of $\ee$ and
		is a lower bound for the lifespan of $u_\ee( t)$ and
		%
		\begin{equation}
			\label{u-e-Hs-bound}
			\|u_\ee(t)\|_{H^s(\ci)}
			\le
			2 \|u_0 \|_{H^s(\ci)},
			\quad
			|t| \le T.
		\end{equation}
		%
		Furthermore,  $u_\ee( t)\in C^1([T, T]; H^{s-1}(\ci))$ and 
		satisfies
		\begin{equation}
			\label{dt-u-e-Hs-bound}
			\|\p_t u_\ee(t)\|_{H^{s-1}(\ci)}
			\lesssim
			\|u_0 \|_{H^s(\ci)}^2,
			\quad
			|t| \le T.
		\end{equation}
		% 
		Here  $c_s$ is a constant depending only on $s$.
	\end{lemma}
	%
	%
	\subsection{ Proof.}  It suffices to prove  \eqref{dt-u-e-Hs-bound}.
	Using equation \eqref{hr-moli-2}, for any $t\in [-T, T]$ we have
	%
	\begin{equation*}
		\begin{split}
			& \| \partial_t u_\varepsilon(t) \|_{H^{s-1}(\ci)}  
			\\
			& = 
			\| -\gamma (J_\ee u_\ee \partial_x  J_\ee  u_\ee) -
			\p_x (1-\p_x^2)^{-1} \left [\frac{3-\gamma}{2} (u_\ee)^2 +
			\frac{\gamma}{2}(\p_x u_\ee)^2 \right ] \|_{H^{s-1}(\ci)}
			\\
			& \lesssim  
			\| J_\ee u_\ee \partial_x  J_\ee  u_\ee \|_{H^{s-1}(\ci)}
			+ \|\p_x (1-\p_x^2)^{-1} (u_\ee)^2 \|_{H^{s-1}(\ci)}
			\\
			& + \| \p_x (1-\p_x^2)^{-1}(\p_x u_\ee)^2\|_{H^{s-1}(\ci)}.
			\end{split}
		\end{equation*}
		We break this into three parts:
		\begin{equation}
			\label{bixi}
			\begin{split}
				\| J_\ee u_\ee \p_x J_\ee u_\ee \|_{H^{s-1}(\ci)}
				& = 
				\frac{1}{2}\|\p_x[(J_\varepsilon u_\varepsilon
				)^2]\|_{H^{s-1}(\ci)}
				\\
				& \lesssim \|(J_\varepsilon u_\varepsilon )^2\|_{H^s(\ci)}.
			\end{split}
		\end{equation}
		Applying the algebra property of Sobolev spaces and estimate
		\eqref{u-e-Hs-bound} to \eqref{bixi} gives 
		%
		\begin{equation}
			\label{deriv1}
			\begin{split}
				\|J_\ee u_\ee \p_x J_\ee u_\ee  
				\|_{H^{s-1}(\ci)}
				& \lesssim
				\|J_\varepsilon u_\varepsilon \|_{H^s(\ci)}^2
				\\
				&\lesssim
				\| u_\varepsilon \|_{H^s(\ci)}^2
				\\
				&\lesssim
				\|u_0\|_{H^s(\ci)}^2.
			\end{split}
		\end{equation}
		We also have
		\begin{equation*}
			\begin{split}
				\|\p_x (1-\p_x^2)^{-1} (u_\ee)^2\|_{H^{s-1}(\ci)}
				& \le \| (u_\ee)^2\|_{H^{s-1}(\ci)}
				\end{split}
		\end{equation*}
		which by the algebra property and estimate \eqref{u-e-Hs-bound}
		gives
		\begin{equation}
			\begin{split}
				\label{deriv2}
				\|\p_x (1-\p_x^2)^{-1} (u_\ee)^2\|_{H^{s-1}(\ci)}
				& \lesssim \|u_\ee\|^2_{H^s(\ci)} 
				\\
				& \lesssim  \|u_0\|^2_{H^s(\ci)}.
			\end{split}
		\end{equation}
		Similarly,
		\begin{equation}
			\begin{split}
				\label{deriv3}
				\|\p_x (1-\p_x^2)^{-1} (\p_x u_\ee)^2\|_{H^{s-1}(\ci)}
				& \lesssim \|\p_x u_\ee\|^2_{H^{s-1}(\ci)} 
				\\
				& \lesssim  \|u_\ee \|^2_{H^s(\ci)}
				\\
				& \lesssim \|u_0\|^2_{H^s(\ci)}.
			\end{split}
		\end{equation}
		Combining \eqref{deriv1}, \eqref{deriv2}, and \eqref{deriv3}, we
		obtain \eqref{dt-u-e-Hs-bound}. $\qquad \Box$
		%%%%%%%%%%%%%%%%%%%%%%%%
		%
		%     Choosing  a convergent subsequence
		%
		%%%%%%%%%%%%%%%%%%%%%%%%
		\subsection{ Choosing  a convergent subsequence.}
		%
		Next we shall show that  the family $\{ u_\ee\}$ has a convergent subsequence
		whose limit $u$ solves the Hyperelastic i.v.p. 
		Let
		$$
		I= [-T, T].
		$$
		By Lemma \ref{hr_wp} we have 
		%
		\begin{equation}
			\label{C-1-fam}
			\{u_\ee\}\subset C(I, H^s(\ci))\cap C^1(I, H^{s-1}(\ci))
		\end{equation}
		%
		and bounded. Since $I$ is compact, we have  
		%
		\begin{equation}
			\label{Lip-1-fam}
			\{u_\ee\}\subset L^{\infty}(I, H^s(\ci))\cap C^1(I,
			H^{s-1}(\ci)).
		\end{equation}
		%
		Now, by the Riesz Lemma, we can identify $H^s(\rr)$ with
		$(H^s(\rr))^*$, where for $w, \psi \in H^s(\rr)$ the duality is
		defined by 
		\begin{equation*}
			T_w(\psi) = <w, \psi>_{H^s(\rr)}.
		\end{equation*}
		Hence, by the Riesz Representation Theorem it follows that we can
		identify \\ $L^\infty(I, H^s(\ci)) $ with the dual space of $L^1(I,
		H^{s}(\ci)$, where for $v\in L^\infty(I, H^s(\ci)) $ and $ \phi \in
		L^1(I, H^{s}(\ci))$ the duality is defined by  
		%
		\begin{equation}
			T_v(\phi) = \int_I <v (t), \phi (t)>_{H^s(\rr)} dt  = \int_I
			 \int_{\rr}
			 \widehat{v}(\xi, t) \overline{\widehat{\phi}}(\xi, t) \cdot (1
			 + \xi^2)^s \ d \xi dt.
		\end{equation}
		%
		Next, we recall Aloaglu's Theorem:
		\begin{theorem}
			If $X$ is a normed vector space,
			the closed unit ball $B^* = \{f \in X^* : \|f\| \le
			1\}$ in $X^*$ is compact in the $weak^*$ topology.
		\end{theorem}
		Therefore the bounded family $\{u_\ee\}$ is compact 
		in the weak$^*$ topology of \\
		$L^\infty(I, H^s(\ci))$. More precisely,
		there is a sequence  $\{ u_{\ee_n} \}$ converging
		weakly to a $ u\in L^{\infty}(I, H^s(\ci))$;
		that is 
		%
		\begin{equation}
			\label{weak-conv}
			\lim_{n\to \infty} T_{u_{\ee_n}}(\phi)  =  T_u (\phi) 
			\; \;		
			\text{ for all } \;\;  \phi \in L^1(I, H^{s}(\ci)).
		\end{equation}
		%
		In order to show that  $u$ solves the HR i.v.p. we need to 
		obtain a stronger  convergence for  $u_{\ee_n}$ so that 
		we can take the limit in the mollified HR equation.
		In fact we will prove that 
		%
		\begin{equation}
			\label{strong-conv}
			u_{\ee_n}\longrightarrow u
			\quad
			\text{ in } \,\,   C(I, H^{s-\sigma}(\ci)),\ \text{for any} \
			\, 0 < \sigma <
			1.
		\end{equation}
		%
		For this we will need the following interpolation  result:
		%%%%%%%%%%%%%%%%%%%%%%%%%%%
		%
		%
		%                 Interpolation Lemma
		%
		%
		%%%%%%%%%%%%%%%%%%%%%%%%%%%
		\begin{lemma}
			\label{interpolation-lem}
			(Interpolation)     Let  $s > \frac{3}{2}$.
			If $v \in C(I, H^s(\ci)) \cap C^1(I, H^{s-1}(\ci))$
			then $v \in C^\sigma (I, H^{s- \sigma}(\ci))$ for  $0 < \sigma < 1$.
		\end{lemma}
		%
		\subsection{ Proof.}  We have
		\begin{equation*}
			\begin{split}
				& \frac{\|v(t) - v(t')\|^2_{H^{s - \sigma}}}{|t - t'|^{2\sigma}}
				\\
				& = 
				\sum_{\xi \in \zz} (1 + \xi^2)^{s- \sigma} 
				\frac{|\hat{v}(\xi, t) - \hat{v}(\xi, t')|^2}{|t-t'|^{2\sigma}} d\xi\\
				& = \sum_{\xi \in \zz} (1+\xi^2)^s 
				\bigg(\frac{1}{(1+ \xi^2)|t - t'|^2} \bigg)^\sigma |\hat{v}(\xi, t)- \hat{v}(\xi, t')|^2 d\xi\\
				& \leq \sum_{\xi \in \zz}(1+\xi^2)^s \bigg( 1 + \frac{1}{(1+\xi^2)|t-t'|^2} \bigg)
				|\hat{v}(\xi,t) - \hat{v}(\xi,t')|^2 d\xi \\
				& \leq \sum_{\xi \in \zz} (1+ \xi^2)^s |\hat{v}(\xi, t)- \hat{v}(\xi, t')|^2 d\xi
				+ \sum_{\xi \in \zz} (1+ \xi^2)^{s-1} \frac{|\hat{v}(\xi, t) - \hat{v} (\xi, t')|^2}{|t-t'|^2} \\
				& \leq  \sup_t \|v(t)\|_{H^s(\ci)}^2 + \sup_t
				\| \partial_t v(t) \|_{H^{s-1}(\ci)}^2
				\\
				& < \infty.
				\\
			\end{split}
		\end{equation*}
		%
		%
		Next, using this lemma we will show that the family $\{u_\ee\}$ is
		equicontinuous in $C(I, H^{s-\sigma}(\ci))$, $0 < \sigma < 1$. We
		will follow this by proving that there exists a sub-family
		$\{u_{\ee_n} \}$ that is precompact in $C(I,
		H^{s-\sigma}(\ci))$. These two facts, in conjunction with Ascoli's
		Theorem, will yield
		\begin{equation}
			\label{strong-conv2}
			u_{\ee_n} \to u \; \; \text{in} \; \; C(I,H^{s-\sigma}(\ci)),
			\quad
			0 < \sigma < 1.
		\end{equation}
		%%%%%%%%%%%%%%%%%%%%%%
		%
		%
		%       Equicontinuity
		%
		%
		%%%%%%%%%%%%%%%%%%%%%%
		%
		\subsection{  Equicontinuity of $\{u_\ee\}_\ee$  in
		$C(I,H^{s-\sigma}(\ci))$.} Applying  Lemma \ref{interpolation-lem} gives 
		%
		\begin{equation}
			\label{equic-1}
			\sup_{t \neq t'} \frac { \|u_\ee(t) - u_\ee(t') \|_{H^{s -
			\sigma}(\ci)}}{|t - t'|^\sigma} < c<\infty
		\end{equation}
		%
		or
		%
		\begin{equation}
			\label{equic-2}
			\|u_\ee(t) - u_\ee(t') \|_{H^{s - \sigma}(\ci)}< c|t - t'|^\sigma, 
			\text{ for all }  \,\,  t, t'\in I,
		\end{equation}
		%
		which shows that  the family  $\{u_\ee\}$ is equicontinuous in 
		$C(I, H^{s-\sigma}(\ci))$. $\qquad \Box$
		%
		%
		%%%%%%%%%%%%%%%%%%%%%%
		%
		%
		%      PreCompactness
		%
		%
		%%%%%%%%%%%%%%%%%%%%%%%%%%
		%
		%
		%
		%
		%		
		\subsection{ Precompactness of $\{u_\ee(t)\}$ in $H^{s-\sigma}(\ci))$.}
		Now recall that
		\begin{equation}
			\label{compact-1}
			\|u_\ee(t)\|_{H^{s}(\ci)}
			\le
			2 \|u_0 \|_{H^s(\ci)}, \,
			\quad
			t\in I.
		\end{equation}
		%
		By Kondrachov's Theorem, the inclusion $H^s(\ci) \subset H^{s-
		\sigma }(\ci)$ is compact. By \eqref{compact-1},
		it follows that $\{u_\ee(t)\}$ is precompact in $H^{s-\sigma}(\ci)$.
		$\quad \Box$
		%
		%
		%
		%
		We are now in a position to apply Ascoli's Theorem: 
		\begin{theorem}
			\label{Ascoli}
			(Ascoli)  Let $X$ be a Banach space, $I$ be a compact metric space,
			and $C(I,X)$  be the set of continuous functions $f: I\longrightarrow X$.
			Suppose $S \subset C(I,X)$  has the following properties:
			%
			\begin{itemize}
				\item[(1)]   $S$ is  equicontinuous.
				\item[(2)]  For each $x \in M$ that the set $S(x) = \{f(x)\}$  is  precompact in $X$.
			\end{itemize} 
			%
			Then $S$  is  precompact  in  $C(I,X)$.
		\end{theorem}
		Compiling our previous results on equicontinuity and precompactness
		and applying Theorem \ref{Ascoli}, we
		conclude that there exists a subfamily $\left\{ u_{\ee_n} \right\}$
		such that
		\begin{equation}
			\label{strong-conv-of-u_ep}
			u_{\ee_n} \to u \; \; \text{in} \; \; C(I, H^{s-\sigma}(\ci)).
		\end{equation}
		%
		%
		%
		%%%%%%%%%%%%%%%%%%%%%%%%%%%%%%%%%
		%
		%
		%     Verifying that the limit $u$ solves Burgers equation
		%
		%
		%%%%%%%%%%%%%%%%%%%%%%%%%%%%%%%%%
		\subsection{ Verifying that the limit $u$ solves the HR equation.} 
		The following lemma will play a crucial role in our proof of the
		existence of a solution to the HR i.v.p.
		\begin{lemma}
			\label{lem:cc}
			We have
			\begin{equation}
				\begin{split}
					\label{burgers_and_nonlocal_conv}
				&  J_{\varepsilon_n} u_{\varepsilon_n} 
				\cdot J_{\varepsilon_n} \p_x u_{\varepsilon_n} 
				\to  u \partial_x u \; \; 
				\text{in} \; \;
				C(I, H^{s-\sigma-1}(\ci)). 
			\end{split}
			\end{equation}
		\end{lemma}
		%
		\subsection{ Proof.} It is implied by the following propositions:
		\begin{proposition}
			\label{prop:1aa}
			\begin{equation}
				\begin{split}
					 J_{\ee_n} u_{\ee_n} \to  u \ \ \text{in} \ \
					C(I, H^{s-\sigma}(\ci)).
					\label{}
				\end{split}
			\end{equation}
		\end{proposition}
			\subsection{ Proof.} Note that
			\begin{equation}
				\begin{split}
					& \| u -  J_{\ee_n} u_{\ee_n}
					\|_{C(I, H^{s-\sigma}(\ci))}
					\\
					&= \| u -  J_{\ee_n} u_{\ee_n} \pm 
					u_{\ee_n} \|_{C(I, H^{s-\sigma}(\ci))}
					\\
					& = \| u -  u_{\ee_n}
					\|_{C(I,H^{s-\sigma}(\ci))} + \| (I - J_{\ee_n})
					u_{\ee_n} \|_{C(I, H^{s-\sigma}(\ci))}.
					\label{1bb}
				\end{split}
			\end{equation}
			Applying the estimates
			\begin{equation*}
				\begin{split}
					& \|I-J_{\ee_n} \|_{L(H^{s-\sigma}(\ci), H^{s -
					\sigma}(\ci))} = o(1),
					\\
					& \|u_{\ee_n}\|_{H^{s-\sigma}(\ci)} \le 2
					\|u_0\|_{H^{s-\sigma}(\ci)}
				\end{split}
			\end{equation*}
			to \eqref{1bb} gives
			\begin{equation}
				\label{2bb}
				\begin{split}
					\| u -  J_{\ee_n} u_{\ee_n}\|_{H^{s-\sigma}(\ci)}
					\le \left( \| u -  u_{\ee_n}
					\|_{C(I, H^{s-\sigma}(\ci))} + o(1) \cdot \|u_0
					\|_{H^{s-\sigma}(\ci)} \right).
				\end{split}
			\end{equation}
			Letting $\ee_n \to 0$ in \eqref{2bb} and applying
			\eqref{strong-conv-of-u_ep} completes the proof. $\quad \Box$
			%
			%
			\begin{proposition}
				\label{prop:dd}
				\begin{equation}
					\begin{split}
						 J_{\ee_n} \p_x u_{\ee_n} \to  \p_x u \ \
						\text{in} \ \ C(I, H^{s-\sigma-1}(\ci)).
						\label{0dd}
					\end{split}
				\end{equation}
			\end{proposition}
			\subsection{ Proof.} 
			\begin{equation*}
				\begin{split}
					\|\p_x u - J_\ee \p_x u_{\ee_n} \|_{C(I,
					H^{s-\sigma-1}))}  
					& = \|\p_x u - \p_x J_\ee u_{\ee_n} \|_{C(I,
					H^{s-\sigma-1}(\ci))} 
					\\
					& \le \|u - J_\ee u_{\ee_n} \|_{C(I,
					H^{s-\sigma}(\ci))}.
				\end{split}
			\end{equation*}
			Applying Proposition \ref{prop:1aa} completes the proof. $\quad
			\Box$
			%
			This completes the proof of Lemma \ref{lem:cc}. $\quad \Box$
		%
		Note that since $\|\p_x (1-\p_x^2)^{-1}\|_{L(H^s(\ci), H^s(\ci))}
		\le 1$ for all $s \in \rr$, it follows immediately from
		\eqref{strong-conv-of-u_ep} that
		\begin{equation}
			\begin{split}
				& \p_x(1- \p_x^2)^{-1} \left( \frac{3-\gamma}{2}
				(u_{\ee_n})^2
				 + \frac{\gamma}{2} (\p_x u_{\ee_n})^2 \right )
				 \\
				 & \to
				 \p_x(1- \p_x^2)^{-1} \left( \frac{3-\gamma}{2} u^2
				 + \frac{\gamma}{2} (\p_x u)^2 \right ) \ \
				 \text{in} \ \ C(I, H^{s-\sigma-1}(\ci)).
				\label{non-local-convergence}
			\end{split}
		\end{equation}
		Combining \eqref{burgers_and_nonlocal_conv} and
		\eqref{non-local-convergence}, and applying the Sobolev Imbedding
		Theorem, we deduce 
		\begin{equation}
			\begin{split}
				& -\gamma (J_{\ee_n} u_{\ee_n} \cdot J_{\ee_n} \p_x
				u_{\ee_n}) - \p_x(1- \p_x^2)^{-1} \left( \frac{3-\gamma}{2}
				(u_{\ee_n})^2
				 + \frac{\gamma}{2} (\p_x u_{\ee_n})^2 \right )
				 \\
				 \to & -\gamma u \p_x u -
				 \p_x(1- \p_x^2)^{-1} \left( \frac{3-\gamma}{2} u^2
				 + \frac{\gamma}{2} (\p_x u)^2 \right ) \ \
				 \text{in} \ \ C(I, C(\ci)).
				\label{loc-non-loc-tog}
			\end{split}
		\end{equation}
		Furthermore, we note that the convergence  
		%
		\begin{equation}
			\label{weak-conv-2}
			T_{u_{\ee_n}}(\phi)  \longrightarrow  T_u(\phi) \;
			\text{ for all } \;  \phi \in L^1(I, H^{s}(\ci))
		\end{equation}
		%
		can be restated as 
		%
		\begin{equation}
			u_{\ee_n}  \longrightarrow  u
			\quad
			\text{ in }  \,\,
			\mathcal{D}'(I\times \ci).
		\end{equation}
		%
		This implies 
		%
		\begin{equation}
			\label{distib-conv-2}
			\p_tu_{\ee_n}  \longrightarrow  \p_tu
			\quad
			\text{ in }  \,\, \mathcal{D}'(I\times \ci).
		\end{equation}
		%
		Since for all $n$ we have 
		%
		\begin{equation}
			\p_tu_{\ee_n} 
			=
			-\gamma (J_{\varepsilon_n} u_{\varepsilon_n}  \cdot
			J_{\varepsilon_n}\partial_x u_{\varepsilon_n}) - \p_x (1-
			\p_x^2)^{-1} \left
			[\frac{3-\gamma}{2}(u_\ee)^2 + \frac{\gamma}{2}(\p_x u_\ee)^2 \right ] 
		\end{equation}
		%
		by the uniqueness  of the limit in \eqref{loc-non-loc-tog}
		we must have
		%
		\begin{equation}
			\label{1000y}
			\partial_t u =- \gamma u \partial_x u- \p_x (1- \p_x^2)^{-1} \left
			[\frac{3-\gamma}{2}u^2 + \frac{\gamma}{2}(\p_x u)^2 \right ].
		\end{equation}
		%
		Thus we have constructed a solution $u \in L^\infty(I, H^s(\ci))$
		to the HR i.v.p. $\qquad \Box$
		It remains to prove that $u \in C(I, H^s(\ci)).$
		%%%%%%%%%%%%%%%%%%%%%%%%%%
%
%
%Proof that  $u \in C(I, H^s(\ci)) \bigcap C^1(I, H^{s-1}(\ci))$.
%
%
%
%%%%%%%%%%%%%%%%%%%%%%%%%%
\subsection{ Proof that $u \in C(I, H^s(\ci))$.} 
We first outline our strategy. Since \\
$u \in L^\infty(I, H^s(\ci))$, it is a
continuous function from $I$ to $H^s(\ci)$ with respect to the weak
topology on $I$; that is, for $\{t_n\} \subset I$ such that $t_n \to t$, we
have
\begin{equation}
	\begin{split}
		<u(t_n), \ v>_{H^s(\ci)} \ \longrightarrow \
		<u(t), \ v>_{H^s(\ci)}, \quad \forall
		v \in H^s(\ci).
		\label{1ff}
	\end{split}
\end{equation}
Next, note that
\begin{equation}
	\begin{split}
		\|u(t) - u(t_n) \|_{H^s(\ci)}^2
		& = <u(t) - u(t_n), \ u(t) -
		u(t_n)>_{H^s(\ci)}
		\\
		& = \|u(t)\|_{H^s(\ci)}^2 + \|u(t_n)\|_{H^s(\ci)}^2
		\\
		& - <u(t_n), \
		u(t) >_{H^s(\ci)} - <u(t), u(t_n)>_{H^s(\ci)}.
		\label{2ff}
	\end{split}
\end{equation}
Applying \eqref{1ff} and \eqref{2ff}, we see that
\begin{equation}
	\begin{split}
		\lim_{n \to \infty} \|u(t) - u(t_n)\|_{H^s(\ci)}^2 = \left[ \lim_{n
		\to \infty} \|u(t_n)\|_{H^s(\ci)}^2
		\right] - \|u(t)\|_{H^s(\ci)}^2.
		\label{3ff}
	\end{split}
\end{equation}
Hence, by \eqref{3ff}, to prove that $u \in C(I, H^s(\ci))$, it will be
enough to show that the map $t \mapsto \|u(t)\|_{H^s(\ci)}$ is a continuous
function of $t$. However, this will follow from the energy
estimate
		\begin{equation}
			\label{en-est-u}
			\frac{1}{2} \frac{d}{dt} \|u(t)\|_{H^s(\ci)}^2
			\le c_s \|u(t)\|_{H^s(\ci)}^3, \quad |t| \le T
		\end{equation}
		which we now derive. Applying $D^s$ to both sides of
		\eqref{1000y}, multiplying the
		resulting equation by $D^s u$, and integrating for $x\in \ci$, we obtain
		\begin{equation}
			\begin{split}
				\label{bound-int}
				\frac 12
				\frac{d}{dt} \|u \|_{H^s}^2
				=
				&-
				\gamma \int_{\ci}   D^s (u \p_x u) \cdot
				D^s u \  dx
				\\
				&- \frac{3 -\gamma}{2} \int_{\ci}  D^{s-2} \p_x (u^2) 
				\cdot D^s u \ dx
				\\
				&- \frac{\gamma}{2} \int_{\ci}   D^{s-2} \p_x (\p_x u)^2
				\cdot D^s u \ dx.
			\end{split}
		\end{equation}
		First we estimate
	\begin{equation}
		\begin{split}
			\left | - \frac{3 -\gamma}{2} \int_\ci D^{s-2} \p_x (u^2) \cdot
			D^s u \; dx \right |
			& \le \left | \frac{3 -\gamma}{2} \right |
			\int_\ci \left |
			D^{s-2} \p_x (u^2) \cdot D^s u \right | dx 
			\\
			& \le \left | \frac{3 -\gamma}{2} \right |
			\|D^{s-2} \p_x (u^2) \|_{L^2(\ci)} 
			\|D^s u \|_{L^2(\ci)}
			\\
			& \le \left | \frac{3 -\gamma}{2} \right |
			\|D^{s-1} (u^2) \|_{L^2(\ci)} 
			\|D^s u \|_{L^2(\ci)}
			\\
			& = \left | \frac{3 -\gamma}{2} \right |
			\| u^2 \|_{H^{s-1}(\ci)} \| u \|_{H^s(\ci)}
			\\
			& \le
			\left | \frac{3 -\gamma}{2} \right | \| u^2 \|_{H^s(\ci)} \| u
			\|_{H^s(\ci)}.
		\end{split}
	\end{equation}
	%
	%
	Applying the algebra property, we obtain
	%
	\begin{equation}
		\label{hl1-prime}
		\begin{split}
			\left | -\frac{3 -\gamma}{2} \int_\ci D^{s-2} \p_x u^2 \cdot
			D^s u \; dx \right |
			\le c_s' \| u \|_{H^s(\ci)}^3.
		\end{split}
	\end{equation}
	%
	%
	We also have
	\begin{equation}
		\begin{split}
			\left | -\frac{\gamma}{2} \int_\ci D^{s-2} \p_x (\p_x u)^2 \cdot
			D^s u \; dx \right |
			& \le \left | \frac{\gamma}{2} \right |
			\int_\ci \left | D^{s-2} \p_x (\p_x u)^2 
			\cdot D^s u \right | \; dx
			\\
			& \le \left | \frac{\gamma}{2} \right |
			\| D^{s-2} \p_x (\p_x u)^2 \|_{L^2(\ci)}
			\| D^s u \|_{L^2(\ci)}
			\\
			&  \le \left | \frac{\gamma}{2} \right | \|(\p_x u)^2
			\|_{H^{s-1}(\ci)} \| u \|_{H^s(\ci)} 
		\end{split}
	\end{equation}
	and applying the algebra property yields
	\begin{equation}
		\label{hl2-prime}
		\left | -\frac{\gamma}{2} \int_\ci D^{s-2} (\p_x u)^2 \cdot
		D^s u \; dx \right |
		\le c_s'' \|u\|_{H^s(\ci)}^3.
	\end{equation}
	It remains to estimate 
	\begin{equation*}
		- \gamma \int_{\ci} \left [  D^s (u \p_x u) \cdot
		D^s u \right ]  dx
	\end{equation*}
	We have
	\begin{equation} 
	\begin{split}
		\label{B-moli-int-v'}
		-  \gamma \int_{\ci} \left [D^s (u \p_x u) \cdot D^s
		u \right ] \ dx
		= &- \gamma  \int_\ci
		\left [ D^s(u \partial_x u) \cdot   D^s u \right ] \ dx
		\\
		=& - \gamma \int_\ci
		\big[ 
		D^s(u\p_x u)  -  u D^s (\p_xu)
		\big] \cdot
		D^s u   \ dx
		\\
		&
		- \gamma \int_\ci
		u D^s (\p_xu) \cdot
		D^su \ dx.
	\end{split}
\end{equation}
%
%
%
We now estimate \eqref{B-moli-int-v'} in parts. Applying the Cauchy-Schwarz inequality gives
%
\begin{equation*} 
	\begin{split}
		& \Big|
		- \gamma \int_\ci
		\big[ 
		D^s(u\p_x u)  -  u D^s (\p_xu)
		\big] \cdot
		D^s u   \, dx
		\Big|
		\\
		& \le
		|\gamma| \cdot \|
		D^s(u\p_x u)  -  u D^s (\p_xu)
		\|_{L^2(\ci)}
		\|
		D^s u 
		\|_{L^2(\ci)}
		\\
		& =
		|\gamma| \cdot \| D^s(u\p_x u)  -  u D^s (\p_xu)
		\|_{L^2(\ci)}
		\|
		u
		\|_{H^s(\ci)}
			\end{split}
\end{equation*}
Applying \eqref{int1-est-calc3}, we obtain
\begin{equation*}
\begin{split}
		\Big|
		- \gamma \int_\ci
		\big[ 
		D^s(u\p_x u)  -  u D^s (\p_xu)
		\big]
		D^s u   \, dx
		\Big|
		&\le
		 c_s'''   \| \p_x u \|_{L^\infty(\ci)} 
		\| u \|_{H^s(\ci)}^2.
	\end{split}
\end{equation*}
%\label{int1-est-calc2'}
%
Next, we apply Cauchy-Schwartz and the Sobolev Imbedding Theorem to deduce 
	%
	%
	%
	\begin{equation} 
		\label{int1-est-calc5'}
		\begin{split}
			\Big|
			\int_\ci
			\left [u D^s (\p_x u)
			\cdot  D^s u \right ] dx
			\Big|
			& =
			\frac{1}{2} \Big|
			   \int_\ci
			\left [u \p_x\left(D^s u\right)^2 \right ] \ dx
			\Big|
			\\
			& \le
			\frac{1}{2} \int_\ci \Big |
			\left [\p_x u \, (D^s u)^2  \right ] 
			\Big| \ dx
			\\
			& \le
			\frac{1}{2}
			\| \p_x u \|_{L^\infty(\ci)} 
			\| u \|_{H^s(\ci)}^2.
			\\
			& \le c_s'''' \|u\|_{H^s(\ci)}^3.
		\end{split}
	\end{equation}
	%
	%
	%
	Combining \eqref{hl1-prime}, \eqref{hl2-prime},
	and \eqref{int1-est-calc5'}, we obtain \eqref{en-est-u}, as desired.
	\subsection{ Size of the solution}. 
	Letting  $y(t)=  \|u(t)\|_{H^s(\ci)}^2$ inequality \eqref{en-est-u}
	takes the form
	%
	\begin{equation} 
		\label{energy-y-ineq'}
		\frac 12
		y^{-3/2}\frac{dy}{dt}
		\le
		c_s,
		\qquad
		y(0)=y_0=  \|u_0\|_{H^s(\ci)}^2.
	\end{equation}
	%
	Suppose $t$ is non-negative. Then integrating  \eqref{energy-y-ineq'}
	from  0 to $t$ gives
	%
	\begin{equation*} 
		\frac{1}{\sqrt{y_0}}  - \frac{1}{\sqrt{y(t)}} 
		\le 
		c_s t.
	\end{equation*}
	%
	%
	Replacing $y(t)$ with   $\|u(t)\|_{H^s(\ci)}^2$  and solving for  $\|u(t)\|_{H^s(\ci)}$
	we obtain the formula
	%
	\begin{equation} 
		\label{norm-u(t)-formula'}
		\|u(t)\|_{H^s(\ci)}
		\le
		\frac{ \|u_0\|_{H^s(\ci)}}{1-c_s\|u_0\|_{H^s(\ci)} t}.
	\end{equation}
	%
	Now, note that our solution $u$ inherits the common lifespan $T$ of the family
	$\{u_\ee\}$; that is, $u$ has lifespan
	\begin{equation*}
		T
		=
		\frac{1}{2 c_s \|u_0\|_{H^s(\ci)}}.
	\end{equation*}
	Substituting into \eqref{norm-u(t)-formula'} we obtain	
	%
	\begin{equation*} 
		\label{u(t)-u(0)-bound'}
		\|u(t)\|_{H^s(\ci)}
		\le
		\frac{ \|u_0\|_{H^s(\ci)}}{1-(c_s\|u_0\|_{H^s(\ci)})/(2 c_s \|u_0\|_{H^s(\ci)})},
	\end{equation*}
	%
	which simplifies to 
	%
	\begin{equation*}
		\|u(t)\|_{H^s(\ci)}
		\le
		2 \|u_0\|_{H^s(\ci)},
		\quad 
		0\le t \le T.
	\end{equation*}
	Similarly, for negative $t$, we have
	\begin{equation*}
		\|u(t)\|_{H^s(\ci)}
		\le
		2 \|u_0\|_{H^s(\ci)},
		\quad 
		-T \le t < 0.
	\end{equation*}
	Hence,
	\begin{equation}
		\label{uniform_bound_for_u}
		\|u(t)\|_{H^s(\ci)}
		\le
		2 \|u_0\|_{H^s(\ci)},
		\quad 
		|t| \le T.
	\end{equation}
		%
		\subsection{Space of the solution}
	Derivating the left hand side of \eqref{en-est-u} and simplifying, we obtain
	\begin{equation}
		\label{en-est-u-simplified}
	\frac{d}{dt} \|u(t)\|_{H^s(\ci)} \le c_s \|u(t)\|_{H^s(\ci)}^2, \quad |t| \le T.
	\end{equation}
	Since $\|u(t)\|_{H^s(\ci)}$
	is uniformly bounded for $|t| \le T$ by
	\eqref{uniform_bound_for_u}, we conclude from
	\eqref{en-est-u-simplified} that the map $t \mapsto
	\|u(t)\|_{H^s(\ci)}$ is Lipschitz continuous in $t$, for $|t| \le T$.
	Therefore, by \eqref{3ff}, $u \in C(I, H^s(\ci))$. 
	%
	%
	%
	%
	%
	\subsection{Uniqueness.}
	%
	%
	Let $u,\omega \in C(I, H^s(\ci)), \ s > 3/2$ be two solutions to the
	Cauchy-problem \eqref{hyperelastic-rod-equation}-\eqref{init-cond} with
	common initial data. Let $v=u-w$; since
	\begin{align*}
		\p_t u 
		& = - \gamma u \p_x u - D^{-2} \p_x \left[ \frac{3-\gamma}{2} u^2 +
		\frac{\gamma}{2}\left( \p_x u \right)^2 \right]
		\\
		\p_t w & = -\gamma w \p_x w - D^{-2} \p_x \left[
		\frac{3-\gamma}{2} w^2 + \frac{\gamma}{2}(\p_x w)^2 
		\right]
	\end{align*}
	we subtract the two equations to obtain 
	\begin{equation*}
		\begin{split}
			\p_t v
			= -\frac{\gamma}{2} \p_x [v(u + w)] - D^{-2} \p_x \left\{
			\frac{3-\gamma}{2}[v(u+w)] + \frac{\gamma}{2}[\p_x v \cdot \p_x (u+w)]
			\right\}
		\end{split}
	\end{equation*}
	and hence
	\begin{equation}
		\begin{split}
			D^\sigma \p_t v = -\frac{\gamma}{2} D^\sigma \p_x [v(u+w)] - D^{\sigma -2} \p_x
			\left\{ \frac{3-\gamma}{2} [v(u+w)] + \frac{\gamma}{2} [\p_x v
			\cdot \p_x
			(u+w)]
			\right\}.
			\label{1v}
		\end{split}
	\end{equation}
	Multiplying both sides of \eqref{1v} by $D^\sigma v$ and integrating, we obtain
	\begin{equation}
		\begin{split}
			\frac{1}{2} \frac{d}{dt} \|v\|_{H^\sigma(\ci)}^2
			& =  \overbrace{-\frac{\gamma}{2} \int_{\ci} D^\sigma \p_x [v(u+w)] \cdot
			D^\sigma v \ dx}^i
			\\
			& \overbrace{- \frac{3-\gamma}{2} \int_{\ci}  D^{\sigma -2}
			\p_x[v(u+w)] \cdot
			D^\sigma v \ dx}^{ii} 
			\\
			& - \overbrace{\frac{\gamma}{2} \int_{\ci} D^{\sigma -2} \p_x [ \p_x v
			\cdot \p_x (u+w)]\cdot D^\sigma v \ dx }^{iii}.
			\label{2v}
		\end{split}
	\end{equation}
	We will estimate (\hyperref[2v]{ii}) first.
	Applying Cauchy-Schwartz, we have 
	\begin{equation*}
		\begin{split}
			|ii|
			& \le \left | \frac{3-\gamma}{2} \right | \|D^{\sigma -2}
			\p_x [v(u+w)] \cdot D^\sigma
			v  \|_{L^1(\ci)}
			\\
			 & \le  \left | \frac{3-\gamma}{2} \right | \|D^{\sigma -2} \p_x [v(u+w)]
			\|_{L^2(\ci)} \|v\|_{H^\sigma(\ci)}
			\\
			& \lesssim \|v(u+w)\|_{H^{\sigma -1}(\ci)} \|v\|_{H^\sigma(\ci)}
		\end{split}
	\end{equation*}
	which by the algebra property and the Sobolev
	Imbedding Theorem gives
\begin{equation}
		\begin{split}
		|ii| \lesssim \|u+w\|_{H^{\sigma -1}(\ci)} \|v\|_{H^\sigma(\ci)}^2.
			\label{3v}
		\end{split}
	\end{equation}
	To estimate (\hyperref[2v]{iii}) we first apply
	Cauchy-Schwartz and the Sobolev Imbedding Theorem:
	\begin{equation*}
		\begin{split}
		|iii| & \le	\left | \frac{\gamma}{2} \right | \|D^{\sigma -2} \p_x
			[\p_x v \cdot \p_x (u+w)] \cdot D^\sigma v  \|_{L^1(\ci)} 
			\\
			& \le  \left | \frac{\gamma}{2} \right | \|D^{\sigma -2} \p_x
			[\p_x v \cdot \p_x (u+w)] \|_{L^2(\ci)}
			\|v\|_{H^\sigma(\ci)}
			\\
			& \le \left |\frac{\gamma}{2} \right|
			\|[\p_x v \cdot \p_x (u+w)] \|_{H^{\sigma -1}(\ci)}
			\|v\|_{H^\sigma(\ci)}.
		\end{split}
	\end{equation*}
	Restrict $1/2 < \sigma < 1$. Then applying Lemma \ref{impo}, we conclude
	that
	\begin{equation}
		\begin{split}
			|iii|
			& \le C \left | \frac{\gamma}{2} \right |
			\|\p_x(u+w) \|_{H^{\sigma}(\ci)}
			\|\p_x v\|_{H^{\sigma -1}(\ci)} \|v\|_{H^\sigma(\ci)}
			\\
			& \lesssim \|u+w \|_{H^{\sigma + 1}(\ci)}
			\|v\|_{H^\sigma(\ci)}^2.
			\label{3'v}
		\end{split}
	\end{equation}
	It remains to estimate (\hyperref[2v]{i}).
	Proceeding, we rewrite
	\begin{equation}
		\begin{split}
			|i| & =
			\left |
			-\frac{\gamma}{2} \int_{\ci} \left[ D^\sigma \p_x, \ u+w \right]v \cdot
			D^\sigma v \ dx - \frac{\gamma}{2} \int_{\ci} (u+w) D^\sigma
			\p_x v \cdot D^\sigma v\ dx
			\right | 
			\\
			& \le \left |
			-\frac{\gamma}{2} \int_{\ci} \left[ D^\sigma \p_x, \ u+w \right]v \cdot
			D^\sigma v \ dx \right |
			+ \left | \frac{\gamma}{2} \int_{\ci} (u+w) D^\sigma \p_x v
			\cdot D^\sigma v\
			dx \right |.
			\label{4v}
		\end{split}
	\end{equation}
	We now estimate \eqref{4v} in pieces. Observe that by integrating by parts
	and applying Cauchy-Schwartz we have
	\begin{equation}
		\begin{split}
			\left | \frac{\gamma}{2}\int_{\ci} (u+w) D^\sigma \p_x v \cdot
			D^\sigma v \ dx \right |
			& = \left | -\frac{\gamma}{2} \int_{\ci} \p_x (u+w) D^\sigma v
			\cdot D^\sigma v \ dx \right |
			\\
			& \lesssim \|\p_x (u+w) D^\sigma v \|_{L^2(\ci)} \|D^\sigma
			v\|_{L^2(\ci)}
			\\
			& \lesssim \|\p_x (u+w)\|_{L^\infty(\ci)}
			\|v\|_{H^\sigma(\ci)}^2.
			\label{4'v}
		\end{split}
	\end{equation}
	To estimate the remaining piece of \eqref{4v}, we recall first that we
	have the restriction $1/2 < \sigma < 1$. However, this will not prevent
	us from applying Corollary \ref{cor1}; in fact, choosing $\ 3/2 < \rho
	< s,  \ 1/2< \sigma <\min\{1, \ \rho -1 \}$, we obtain
	\begin{equation}
		\begin{split}
			\left | -\frac{\gamma}{2} \int_{\ci} [D^\sigma \p_x, \ u+w] v
			\cdot D^\sigma v \ dx \right |
			& \le \left | \frac{\gamma}{2} \right| \int_{\ci} \left |
			[D^\sigma \p_x, \ u+w] v
			\cdot D^\sigma v \right | dx 
			\\
			& \lesssim \|[D^\sigma \p_x, \ u+w]v\|_{L^2(\ci)}
			\|v\|_{H^\sigma(\ci)} \\
			& \lesssim \|u+w\|_{H^\rho(\ci)} \|v\|_{H^\sigma(\ci)}^2.
			\label{7v}
		\end{split}
	\end{equation}
	Combining \eqref{4'v} and \eqref{7v} and applying the Sobolev Imbedding
	Theorem, we obtain the estimate
	\begin{equation}
		\begin{split}
			|i| \lesssim \|u+w\|_{H^\rho(\ci)} \|v\|_{H^\sigma(\ci)}^2.
			\label{8v}
		\end{split}
	\end{equation}
	Recall \eqref{2v}. Grouping \eqref{3v}, \eqref{3'v}, and \eqref{8v}, and applying
	the Sobolev Imbedding Theorem, we see that 
	\begin{equation}
		\begin{split}
			\frac{1}{2} \frac{d}{dt}
			\|v\|_{H^\sigma(\ci)}^2 \lesssim \|u+w\|_{H^\rho(\ci)}
			\|v\|_{H^\sigma(\ci)}^2.
			\label{9v}
		\end{split}
	\end{equation}
	By Gronwall's inequality, \eqref{9v} gives
	\begin{equation}
		\label{10lv}
		\begin{split}
			\|v\|_{H^\sigma(\ci)}
			& \lesssim e^{\int_0^t \|u+w\|_{H^{\rho}}}
			\|v_0\|_{H^\sigma(\ci)}, \qquad |t| \le T.
		\end{split}
	\end{equation}
	First, note that $v_0 = u_0 - w_0 = 0$; secondly, $\|u + w \|_{H^\rho}
	\le \|u + w \|_{H^s(\ci)} < \infty$ for $|t| \le T$ by
	the triangle inequality and \eqref{u_x-Linfty-Hs}. Hence, from
	\eqref{10lv} we obtain
	\begin{equation*}
		\begin{split}
			\|v\|_{H^\sigma(\ci)}
			& \lesssim \|v_0\|_{H^\sigma(\ci)}, \quad |t| \le T	
			\\
			& = 0.
		\end{split}
	\end{equation*}
	We conclude that solutions to the HR i.v.p. with initial data in
	$H^s(\ci)$ are unique for $s > 3/2$.  $\qquad
	\Box$
	%
	%
	%
	%
	\subsection{Continuous Dependence.}
	Let $\left\{ u_{0, n} \right\}_n \subset H^s(\ci)$ be a uniformly bounded
sequence converging to $u_0$ in $H^s(\ci)$.
Consider solutions $u $, $u^\ee$, $u^\ee_n$, and $u_n$ to the Cauchy-problem
\eqref{hyperelastic-rod-equation}-\eqref{init-cond}
with associated initial data $u_0$, $J_\ee u_0$,
$J_\ee u_{0,n}$, and $u_{0,n}$, respectively, where $J_\ee$ is defined as follows: Pick a function $\widehat{j}(\xi) \in \mathcal{S}(\rr)$ such that
	\begin{equation}
		\label{0u}
		\begin{split}
			& 0 \le \widehat{j}(\xi) \le 1,
			\\
			& \widehat{j}(\xi) = 1 \ \ \text{if} \ \ |\xi| \le 1.
		\end{split}
	\end{equation}
	Since $\sum_{-M}^M \widehat{j}(\ee \xi) e^{i \xi x}$ converges uniformly as $M \to
	\infty$, we can let
	\begin{equation}
		\begin{split}
			j_\ee (x) = \frac{1}{2 \pi}\sum_{\xi \in \zz}
			\widehat{j}(\ee \xi) e^{i \xi x}, \quad \ee > 0
			\label{parseval-def}
		\end{split}
	\end{equation}
	which is equivalent to stating that we can find $\left\{ j_\ee
	\right\} \subset \mathcal{S}(\ci)$ such that
	\begin{equation}
		\begin{split}
			\widehat{j_\ee} = \widehat{j }(\ee \xi), \quad \ee > 0.
			\label{widehat-def}
		\end{split}
	\end{equation}
	We then define $J_\ee$ to be the ``Friedrichs mollifier''
	\begin{equation}
		\label{0'u}
		\begin{split}
			J_\ee f(x) = j_\ee * f(x), \quad \ee>0.
		\end{split}
	\end{equation}
We remark that we have constructed the operator $J_\ee$ in this manner in
order that inequality \eqref{widehat-def} is satisfied; this will prove
crucial later on.
%
Applying
the triangle inequality, we have
\begin{equation*}
	\begin{split}
		\|u - u_n\|_{H^s(\ci)}
		& \le \|u - u^\ee\|_{H^s(\ci)}
		+ \|u^\ee - u^{\ee}_n \|_{H^s(\ci) }
		+  \|u^{\ee}_n - u_n \|_{H^s(\ci)}.
	\end{split}
\end{equation*}
Therefore, to prove continuous dependence, it will be enough to show 
\begin{align}
	& \lim_{\substack{n\to \infty \\ \ee \to 0}} \|u - u^\ee\|_{H^s(\ci)}
	=0,
	\label{enough_to_prove1}
	\\
	& \lim_{\substack{n\to \infty \\ \ee \to 0}} \|u^\ee - u^{\ee}_n
	\|_{H^s(\ci)} = 0,
	\label{enough_to_prove2}
	\\
	& \lim_{\substack{n\to \infty \\ \ee \to 0}}
	\|u^{\ee}_n - u_n \|_{H^s(\ci)} =0
	\label{enough_to_prove3}
\end{align}
where we define 
\medskip
	\begin{equation}
		\label{lim-not}
		\begin{split}
			\lim_{\substack{n\to \infty \\ \ee \to 0}} (\cdot) \doteq \lim_{\ee \to
			\infty} [\lim_{n \to \infty} (\cdot )].
		\end{split}
	\end{equation}
\subsection{ Proof of \eqref{enough_to_prove1}.}
		Consider two solutions $u $ and $u^\ee$ to the Cauchy-problem
	\eqref{hyperelastic-rod-equation}-\eqref{init-cond}
	with associated initial data $u_0$ and
	$J_\ee u_0$, respectively. Set $v= u -u^\ee $. Then $v$ solves the
	Cauchy-problem
	\begin{align}
		\label{4u}
		\p_t v 
		& =  - \gamma (v \p_x v + v \p_x u^\ee + u^\ee \p_x v)  
		\\
		& - D^{-2} \p_x \left\{ \left (\frac{3-\gamma}{2} \right )(v^2 +
		2u^\ee v) + \frac{\gamma}{2}\left[ (\p_x v)^2 + 2 \p_x u^\ee \p_x v \right]
		\right\}, \notag
		\\
		& v(0) = (I- J_\ee)u_0.
		\label{5u}
	\end{align}
	Applying the operator $D^s$ to both sides of \eqref{4u}, multiplying by
	$D^s v$ and integrating, we have
	\begin{equation}
		\begin{split}
			\frac{1}{2}\frac{d}{dt} \|v\|_{H^s(\ci)} = A + B
			\label{6u}
		\end{split}
	\end{equation}
	where
	\begin{equation}
		\begin{split}
			A
			& =  -\gamma \int_{\ci} D^s(v \p_x v) \cdot D^s v \
			dx
			- \frac{3- \gamma}{2} \int_\ci D^{s-2} \p_x (v^2) \cdot D^s v
			\ dx
			\\
			& - \frac{\gamma}{2}\int_\ci D^{s-2} \p_x (\p_x v)^2 \cdot D^s
			v \ dx
			\label{7u}
		\end{split}
	\end{equation}
	and
	\begin{equation}
		\begin{split}
			B 
			= & \ \overbrace{-\gamma \int_\ci D^s (v \p_x u^\ee ) \cdot D^s v \
			 dx}^{(i)} \ \overbrace{-\gamma \int_\ci D^s (u^\ee \p_x v) \cdot D^s v \
			 dx}^{(ii)}
			  \\
			  & \ \overbrace{- \ ( 3- \gamma) \int_\ci D^{s-2} \p_x (u^\ee v) \cdot D^s
			 v \ dx}^{(iii)}
			 \\
			 & \overbrace{-\gamma \int_\ci D^{s-2} \p_x
			(\p_x u^\ee \cdot \p_x v) \cdot D^s v \
			dx}^{(iv)}.
			\label{8u}
		\end{split}
	\end{equation}
	We now provide estimates for $A$ and $B$:
	\subsection{ A.} 
	Recalling the proof of \eqref{en-est-u}, with $u$ replaced by
	$v$ gives 
	\begin{equation}
		\begin{split}
			|A| \lesssim \|v\|_{H^s(\ci)}^3, \quad |t| \le T.
			\label{8'u}
		\end{split}
	\end{equation}
%
	\subsection{ B.} We now estimate in parts:
	%
	%
	%
		%
	%
\subsection{ Estimate of (\hyperref[8u]{i}).} 
We can rewrite
	\begin{equation}
		\begin{split}
			(i)
			= & -\gamma \int_\ci \left[ D^s(v \p_x u^\ee) - v D^s
			\p_x u^\ee \right] \cdot D^s v \ dx
			\\
			& -  \gamma \int_\ci v D^s \p_x u^\ee \cdot D^s v \ dx.
			\label{1wap'}
		\end{split}
	\end{equation}
	Estimating in parts, we have
	\begin{equation}
		\begin{split}
			& |-\gamma \int_\ci \left[ D^s(v \p_x u^\ee) - v D^s
			\p_x u^\ee \right] \cdot D^s v \ dx |
			\\
			& \le |\gamma| \int_\ci |\left[ D^s(v \p_x u^\ee ) - v D^s
			\p_x u^\ee \right] \cdot D^s v| \ dx
			\\
			& \le |\gamma| \cdot \|D^s (v \p_x u^\ee) - v D^s \p_x u^\ee
			\|_{L^2(\ci)} \|v\|_{H^s(\ci)}.
			\label{1wap}
		\end{split}
	\end{equation}
	Applying the Kato-Ponce estimate \eqref{KP-com-est} to \eqref{1wap}, we
	obtain
	\begin{equation*}
		\begin{split}
			& | -\gamma \int_\ci \left[ D^s(v \p_x u^\ee) - v D^s
			\p_x u^\ee \right] \cdot D^s v \ dx |
			\\
			& \le c_s |\gamma| \cdot ( \|D^s v \|_{L^2(\ci)} \|\p_x
			u^\ee\|_{L^\infty(\ci)} + \|\p_x v \|_{L^\infty(\ci)} \|D^{s-1}
			\p_x u^\ee \|_{L^2(\ci)}) \cdot \|v\|_{H^s(\ci)}
		\end{split}
	\end{equation*}
	which by the Sobolev Imbedding Theorem simplifies to
	\begin{equation}
		\begin{split}
			| -\gamma \int_\ci \left[ D^s(v \p_x u^\ee ) - v D^s
			\p_x u^\ee \right] \cdot D^s v \ dx |
			\lesssim \|u^\ee \|_{H^s(\ci)} \|v\|_{H^s(\ci)}^2.
			\label{2wap}
		\end{split}
	\end{equation}
	For the remaining piece of \eqref{1wap'}, we have
	\begin{equation*}
		\begin{split}
			| - \gamma \int_\ci v D^s \p_x u^\ee \cdot D^s v \ dx |
			& \le |\gamma| \int_\ci |v D^s \p_x u^\ee \cdot D^s v | \ dx
			\\
			& \le |\gamma| \cdot \|v\|_{L^\infty(\ci)} \|D^s \p_x u^\ee
			\|_{L^2(\ci)} \|D^s v\|_{L^2(\ci)}
		\end{split}
	\end{equation*}
	which by the Sobolev Imbedding Theorem gives 
	\begin{equation}
		\begin{split}
			| - \gamma \int_\ci u^\ee D^s \p_x v \cdot D^s v \ dx |
			\lesssim \|u^\ee \|_{H^{s+1}(\ci)} \|v\|_{H^{s-1}(\ci)}
			\|v\|_{H^s(\ci)}.
			\label{3wap}
		\end{split}
	\end{equation}
	Combining estimates \eqref{2wap} and \eqref{3wap} we conclude that
	\begin{equation}
		\begin{split}
			|(i)| \lesssim \|u^\ee \|_{H^s(\ci)} \|v\|_{H^s(\ci)}^2 + 
			\|u^\ee \|_{H^{s+1}(\ci)} \|v\|_{H^{s-1}(\ci)}
			\|v\|_{H^s(\ci)}.
			\label{4wap}
		\end{split}
	\end{equation}
%
\subsection{ Estimate of (\hyperref[8u]{ii}).} We can rewrite
	\begin{equation}
		\begin{split}
			(ii)
			= & -\gamma \int_\ci \left[ D^s(u^\ee \p_x v) - u^\ee D^s
			\p_x v \right] \cdot D^s v \ dx
			\\
			& -  \gamma \int_\ci u^\ee D^s \p_x v \cdot D^s v \ dx.
			\label{1wa'}
		\end{split}
	\end{equation}
	Estimating in parts, we have
	\begin{equation}
		\begin{split}
			& |-\gamma \int_\ci \left[ D^s(u^\ee \p_x v) - u^\ee D^s
			\p_x v \right] \cdot D^s v \ dx |
			\\
			& \le |\gamma| \int_\ci |\left[ D^s(u^\ee \p_x v) - u^\ee D^s
			\p_x v \right] \cdot D^s v| \ dx
			\\
			& \le |\gamma| \cdot \|D^s (u^\ee \p_x v) - u^\ee D^s \p_x v
			\|_{L^2(\ci)} \|v\|_{H^s(\ci)}.
			\label{1wa}
		\end{split}
	\end{equation}
	Applying the Kato-Ponce estimate \eqref{KP-com-est} to \eqref{1wa}, we
	obtain
	\begin{equation*}
		\begin{split}
			& | -\gamma \int_\ci \left[ D^s(u^\ee \p_x v) - u^\ee D^s
			\p_x v \right] \cdot D^s v \ dx |
			\\
			& \le c_s |\gamma| \cdot ( \|D^s u^\ee \|_{L^2(\ci)} \|\p_x
			v\|_{L^\infty(\ci)} + \|\p_x u^\ee \|_{L^\infty(\ci)} \|D^{s-1}
			\p_x v \|_{L^2(\ci)}) \cdot \|v\|_{H^s(\ci)}
		\end{split}
	\end{equation*}
	which by the Sobolev Imbedding Theorem simplifies to
	\begin{equation}
		\begin{split}
			| -\gamma \int_\ci \left[ D^s(u^\ee \p_x v) - u^\ee D^s
			\p_x v \right] \cdot D^s v \ dx |
			\lesssim \|u^\ee \|_{H^s(\ci)} \|v\|_{H^s(\ci)}^2.
			\label{2wa}
		\end{split}
	\end{equation}
	For the remaining piece of \eqref{1wa'}, we have
	\begin{equation*}
		\begin{split}
			| - \gamma \int_\ci u^\ee D^s \p_x v \cdot D^s v \ dx |
			& = \left | -\frac{\gamma}{2} \int_\ci u^\ee \p_x (D^s v)^2 \
			dx \right |
			\\
			& = \left | \frac{\gamma}{2} \int_\ci \p_x u^\ee (D^s v)^2 \ dx
			\right |
			\\
			& \le \left | \frac{\gamma}{2} \right | \int_\ci |\p_x u^\ee
			(D^s v)^2 | dx
			\\
			& \le \left | \frac{\gamma}{2} \right | \|\p_x u^\ee
			\|_{L^\infty(\ci)} \|v\|_{H^s(\ci)}^2
		\end{split}
	\end{equation*}
	and applying the Sobolev Imbedding Theorem gives
	\begin{equation}
		\begin{split}
			| - \gamma \int_\ci u^\ee D^s \p_x v \cdot D^s v \ dx |
			\lesssim \|u^\ee \|_{H^s(\ci)} \|v\|_{H^s(\ci)}^2.
			\label{3wa}
		\end{split}
	\end{equation}
	Combining estimates \eqref{2wa} and \eqref{3wa} we conclude that
	\begin{equation}
		\begin{split}
			|(ii)| \lesssim \|u^\ee \|_{H^s(\ci)} \|v\|_{H^s(\ci)}^2.
			\label{4wa}
		\end{split}
	\end{equation}
\subsection{ Estimate of (\hyperref[8u]{iii}).} We have
	\begin{equation}
		\begin{split}
			|(iii)|
			& \le |3-\gamma| \int_{\ci} |D^{s-2} \p_x (u^\ee v) \cdot D^s v
			\ | \ dx
			\\
			& \le |3- \gamma| \cdot  \|D^{s-2}\p_x (u^\ee v)
			\|_{L^2(\ci)} \cdot \|v\|_{H^s(\ci)}
			\\
			& \le |3- \gamma| \cdot  \| u^\ee v \|_{H^{s -1}(\ci)} \cdot \|v\|_{H^s(\ci)}
			\label{12u}
		\end{split}
	\end{equation}
	and applying the algebra property and the Sobolev Imbedding Theorem gives
	\begin{equation}
		\begin{split}
			|(iii)| & \lesssim \|u^\ee\|_{H^{s-1}(\ci)} \|v\|_{H^{s-1}(\ci)}
			\|v\|_{H^s(\ci)}
			\\
			& \lesssim \|u^\ee\|_{H^{s}(\ci)} \|v\|_{H^{s}(\ci)}^2.
			\label{13u}
		\end{split}
	\end{equation}
	%
	%
	%
	%
	\subsection{ Estimate of (\hyperref[8u]{iv}).} We have
	\begin{equation*}
		\begin{split}
			|(iv)|
			& \le |\gamma| \cdot \|D^{s-2} \p_x (\p_x u^\ee \cdot \p_x v)
			\|_{L^2(\ci)} \|D^s v\|_{L^2(\ci)}
			\\
			& \le |\gamma| \cdot \|\p_x u^\ee \cdot \p_x v \|_{H^{s-1}(\ci)}
			\|v\|_{H^s(\ci)}
		\end{split}
	\end{equation*}
	and applying the algebra property gives
	\begin{equation*}
		\begin{split}
			|(iv)|
			& \le |\gamma| \cdot \|\p_x u^\ee \|_{H^{s-1}(\ci)} \|\p_x v
			\|_{H^{s-1}(\ci)} \|v\|_{H^s(\ci)}
			\\
			& \lesssim \|u^\ee\|_{H^s(\ci)} \|v\|_{H^s(\ci)}^2.
		\end{split}
	\end{equation*}
	Hence, collecting our estimates for (\hyperref[8u]{i}),
	(\hyperref[8u]{ii}), (\hyperref[8u]{iii}), and (\hyperref[8u]{iv})
	yields
		\begin{equation}
		\begin{split}
			|B| 
			& \lesssim
			\|u^\ee\|_{H^s(\ci)}
			\|v\|_{H^s(\ci)}^2 + \|u^\ee\|_{H^{s+1}(\ci)}
			\|v\|_{H^{s-1}(\ci)} \|v\|_{H^s(\ci)}.
			\label{14u}
		\end{split}
	\end{equation}
	Combining estimates \eqref{8'u} and \eqref{14u} and recalling
	\eqref{6u}, we obtain
	\begin{equation}
		\begin{split}
			\frac{1}{2}\frac{d}{dt}\|v\|_{H^{s}(\ci)}^2
			& \le c_s(\|v\|_{H^s(\ci)}^3 + \|u^\ee\|_{H^s(\ci)}
			\|v\|_{H^s(\ci)}^2
			\\
			& + \|u^\ee\|_{H^{s+1}(\ci)}
			\|v\|_{H^{s-1}(\ci)} \|v\|_{H^s(\ci)})
			\label{15u}
		\end{split}
	\end{equation}
	where $c_s$ is a constant depending only on $s$.
	Note that the first two terms in the parentheses on the right hand side
	of \eqref{15u} will offer us little trouble;
	it is the third term that requires special care (due to the
	$\|u^\ee\|_{H^{s+1}(\ci)}$ factor, which becomes increasingly large as
	$\ee$ decreases). More precisely:
	%
	%
	%
	\begin{remark}
	\label{lem5r}
	For $r \ge s > 3/2$ and $0 < \ee <<1$, 
	\begin{equation}
		\begin{split}
			\|u^{\ee} (t) \|_{H^r(\ci)} \le C \, \ee^{s-r}
			\label{700r}
		\end{split}
	\end{equation}
	where $C = C(r, \|u_0\|_{H^s(\ci)})$.
\end{remark}	
\subsection{ Proof.} By part (iii) of Theorem
\ref{thm:HR_existence_continous_dependence}, proved in Section
\ref{existence}, we have
\begin{equation}
	\begin{split}
		\|u^\ee(t) \|_{H^r(\ci)}^2
		& \le C' \|u^\ee (0)\|_{H^r(\ci)}^2
		\\
		& = C' \|J_\ee u_0\|_{H^r(\ci)}^2
		\\
		& = C' \sum_{\xi \in \zz} |\widehat{j_\ee} (\xi) \widehat{u_0}(\xi)
		|^2 \cdot (1 + \xi^2)^r
		\label{0qr}
	\end{split}
\end{equation}
Recall how we chose the mollifier $J_\ee$; it will now play a fundamental role. Since
\eqref{widehat-def} holds by construction, \eqref{0qr} gives 
\begin{equation}
	\begin{split}
		\|u^\ee(t) \|_{H^r(\ci)}^2
		& = C' \sum_{\xi \in \zz} |\widehat{j }(\ee \xi)|^2 \cdot (1 +
		\xi^2)^{r-s} \cdot |\widehat{u_0}(\xi)|^2 \cdot (1 + \xi^2)^s
		\\
		& = C'|\widehat{u_0}(0)|^2 +
		C' \sum_{\xi \in \zz \setminus {0}} |\widehat{j }(\ee \xi)|^2 \cdot (1 +
		\xi^2)^{r-s} \cdot |\widehat{u_0}(\xi)|^2 \cdot (1 + \xi^2)^s.
		\label{1qr}
	\end{split}
\end{equation}
Assume $r \ge s$. Since $\widehat{j }(\xi) \in \mathcal{S}(\rr)$, 
\begin{equation}
	\label{schwartz}
	\begin{split}
		|\widehat{j }(\ee \xi)| \le c_r |\ee \xi |^{s-r}, \quad \xi \neq 0.
	\end{split}
\end{equation}
Applying \eqref{schwartz} to \eqref{1qr}, we obtain
\begin{equation}
	\label{calc_ue}
	\begin{split}
		\|u^\ee (t)\|_{H^r(\ci)}^2 
		& \le C' |\widehat{u_0}(0) |^2 + c_r \sum_{\xi \in \zz \setminus
		{0}} |\ee \xi |^{2(s-r)} \cdot (1 + \xi^2)^{r-s}
		|\widehat{u_0}(\xi) |^2 \cdot (1 + \xi^2)^s
		\\
		& \le C' |\widehat{u_0}(0) |^2 + 2^{r-s} c_r \ee^{2(s-r)}
		\sum_{\xi \in \zz \setminus {0}} |\widehat{u_0}(\xi)|^2 \cdot (1 +
		\xi^2)^s
		\\
		& \le C' \|u_0\|_{H^s(\ci)}^2 + 2^{r-s} c_r \ee^{2(s-r)}
		\|u_0\|_{H^s(\ci)}^2
		\\
		& = (C' + 2^{r-s} c_r \ee^{2(s-r)}) \cdot \|u_0\|^2_{H^s(\ci)}.
	\end{split}
\end{equation}
Assuming $0 < \ee <<1$, we conclude from \eqref{calc_ue} that 
\begin{equation*}
	\begin{split}
		\|u^\ee(t)\|_{H^s(\ci)} \le C \ee^{s-r}
	\end{split}
\end{equation*}
where $C = C(r, \|u_0\|_{H^s(\ci)})$. $\qquad \Box$
	We remark that, despite the blowup of $\|u^\ee \|_{H^{s+1}(\ci)}$
	as $\ee$ becomes small, our difficulties would have been
	amplified if we had originally taken $v=w-u$ for some arbitrary
	solution $w$ to the HR
	i.v.p with initial data $w_0 \in H^s(\ci)$, for then we would be dealing with
	\eqref{15u}, with $w$ substituted in for $u^\ee$. However, note that 
	$\|w\|_{H^{s+1}(\ci)}$ might not even be bounded, whereas $\|u^\ee
	\|_{H^{s+1}(\ci)}$ is always bounded, for any $\ee > 0$.
	%
	%
	In light of the blowup of $\|u^\ee \|_{H^{s+1}(\ci)}$,
	our strategy in tackling
	\eqref{15u} will be as follows. First, we will obtain an estimate for
	$\|v\|_{H^\sigma(\ci)}$ for suitably chosen $\sigma < s-1$. Then, we
	will use this estimate to interpolate between $\|v\|_{H^\sigma(\ci)}$
	and $\|v\|_{H^s(\ci)}$,
	yielding an estimate for $\|v\|_{H^{s-1}(\ci)}$ which will allow us to control the growth of
	$\|u^\ee\|_{H^{s+1}(\ci)}$. 
	%
	%
	%
	%
\begin{lemma} 
	\label{lem6r}
	For $\sigma$ such that $1/2 < \sigma < 1$ and $\sigma + 1 \le s$, we have
	\begin{equation}
	\begin{split}
		\|v\|_{H^{\sigma}(\ci)} \le C \cdot o(\ee^{s- \sigma }), \qquad |t| \le T
	\end{split}
\end{equation}
where $C=C(\|u_0\|_{H^s(\ci)})$.
\end{lemma}
%
%
%
\subsection{ Proof.}
Recall that $v$ solves the Cauchy-problem \eqref{4u}-\eqref{5u}.
Applying $D^\sigma$ to both sides of \eqref{4u}, multiplying by
$D^\sigma v$, and integrating, we obtain the
relation
\begin{equation*}
	\begin{split}
		\frac{1}{2}\frac{d}{dt}\|v(t)\|_{H^\sigma(\ci)}^2
		= & - \frac{\gamma}{2}\int_{\ci} D^\sigma
		\p_x \left[ \left( u + u^\ee \right)v
		\right]\cdot D^\sigma v \ dx
		\\
		& - \frac{3-\gamma}{2}\int_{\ci} D^{\sigma
		-2} \p_x \left[ \left( u + u^\ee
		\right)v \right] \cdot D^\sigma v \ dx
		\\
		& - \frac{\gamma}{2}\int_{\ci} D^{\sigma
		-2}
		\p_x \left[ \left( \p_x u + \p_x u^\ee
		\right)\cdot \p_x v \right] \cdot
		D^\sigma v \ dx.
	\end{split}
\end{equation*}
Repeating calculations \eqref{X}-\eqref{12}, with $E$ set to zero,
$u^{\omega,n}$ replaced by $u$, $u_{\omega,n}$ replaced by $u^\ee$, and
$\sigma$ and $\rho$ chosen such that
%
\begin{equation}
	\label{size_of_sigma}
	\begin{split}
	& 1/2 < \sigma < 1,
	\\
	& \sigma + 1 \le \rho \le s 
	\end{split}
\end{equation}
yields
 \begin{equation*}
	\begin{split}
		\frac{1}{2}\frac{d}{dt} \|v\|_{H^\sigma(\ci)}^2
		& \le
		c_s' (\|u^{\ee} + u\|_{H^{\rho}(\ci)} +
		\|\p_x(u^{\ee} + u) \|_{H^\sigma(\ci)})
		\cdot \|v\|_{H^\sigma(\ci)}^2.
	\end{split}
\end{equation*}
\medskip
By the Sobolev Imbedding Theorem, it follows that 
\begin{equation}
	\begin{split}
		\frac{1}{2}\frac{d}{dt} \|v\|_{H^{\sigma}(\ci)}^2
		& \le
		c_s \cdot \|u^{\ee}
		+ u\|_{H^{s}(\ci)}\cdot \|v\|_{H^{\sigma}(\ci)}^2.
		\label{10x}
	\end{split}
\end{equation}
Hence, applying the triangle inequality and
part (iii) of Theorem \ref{thm:HR_existence_continous_dependence} (proved
in Section \ref{existence}) to \eqref{10x} yields
%
\begin{equation}
	\begin{split}
		\label{11x}
		\frac{1}{2}\frac{d}{dt} \|v\|_{H^{\sigma}(\ci)}^2
		& \le
		c_s (\|u^{\ee}(0)\|_{H^{s}(\ci)}
		+ \|u(0)\|_{H^{s}(\ci)})\cdot \|v\|_{H^{\sigma}(\ci)}^2
		\\
		& = c_s (\|J_\ee u_0\|_{H^{s}(\ci)}
		+ \|u_0\|_{H^{s}(\ci)})\cdot \|v\|_{H^{\sigma}(\ci)}^2.
	\end{split}
\end{equation}
We now need the following:
\begin{proposition}
	\label{lem3r}
	For arbitrary $u \in L^2(\ci)$,
	\begin{equation}
		\begin{split}
			\|J_\ee u\|_{H^s(\ci)} \le \|u\|_{H^s(\ci)}.
			\label{lem100u}
		\end{split}
	\end{equation}
\end{proposition}
%
%
%
%
\subsection{ Proof.}
\begin{equation*}
	\begin{split}
		\|J_\ee u\|_{H^s(\ci)} 
		& = \left[\sum_{\xi \in \zz} |\widehat{j_\ee * u}(\xi) |^2
		(1+\xi^2)^s \right ]^{1/2}
		\\
		& = \left [ \sum_{\xi \in \zz} |\widehat{j_\ee} (\xi) \widehat{u}(\xi) |^2
		(1+ \xi^2)^s \right ]^{1/2}
		\\
		& = \left [ \sum_{\xi \in \zz} |\widehat{j}(\ee \xi)
		\widehat{u}(\xi)|^2 ( 1+ \xi^2)^s \right ]^{1/2}
	\end{split}
\end{equation*}
and since $|\widehat{j }(\ee \xi) | \le 1$ by \eqref{0u}, the result
follows. $\qquad \Box$
Using estimate \eqref{11x}, and applying Proposition \ref{lem3r}, 
we obtain the critical estimate 
\begin{equation}
	\begin{split}
		\label{12x}
		\frac{1}{2}\frac{d}{dt} \|v\|_{H^{\sigma}(\ci)}^2
		& \le
	 C \|v\|_{H^{\sigma}(\ci)}^2
\end{split}
\end{equation}
where $C = C(\|u_0\|_{H^s(\ci)})$. Differentiating the left hand side of
\eqref{12x} and simplifying, we obtain
\begin{equation}
	\begin{split}
		\frac{d}{dt}\|v\|_{H^{\sigma}(\ci)} \le C \|v\|_{H^{\sigma}(\ci)}.
		\label{100x}
	\end{split}
\end{equation}
Let $y(t) = \|v\|_{H^{\sigma}(\ci)}$. Then \eqref{100x} gives
\begin{equation*}
	\begin{split}
		\frac{1}{y(t)}\frac{dy}{dt} \le C.
	\end{split}
\end{equation*}
Hence,
\begin{equation*}
	\begin{split}
		\int_0^t \frac{1}{y(\tau)} \frac{dy}{d \tau}
		\le \int_0^t C \ d \tau, \qquad |t| \le T
	\end{split}
\end{equation*}
from which we obtain
\begin{equation}
	\begin{split}
		\ln |y(t) | - \ln |y(0)| \le C t.
		\label{101x}
	\end{split}
\end{equation}
Simplifying \eqref{101x}, we have
\begin{equation*}
	\begin{split}
		\ln \left |\frac{y(t)}{y(0)} \right | \le C t
	\end{split}
\end{equation*}
Since $y(t)$ is non-negative for all $t \in \rr$, this yields the estimate
\begin{equation*}
	\begin{split}
		y(t) \le y(0) e^{C t}, \qquad |t| \le T.
	\end{split}
\end{equation*}
Substituting back in $\|v\|_{H^{\sigma}(\ci)}$ for $y$, we get
\begin{equation}
	\label{conc-lemma}
	\begin{split}
		\|v\|_{H^{\sigma}(\ci)}
		& \le e^{C t}\|v(0)\|_{H^{\sigma}(\ci)}
		\\
		& = e^{C t}\|u(0) - u^\ee(0) \|_{H^{\sigma}(\ci)}
		\\
		& = e^{C t}\|u_0 - J_\ee u_0 \|_{H^{\sigma}(\ci)}.
	\end{split}
\end{equation}
We now require a critical operator norm estimate which will play an
important role later on.
\begin{proposition}
	\label{lem4r}
	For $r \le s$ and $\ee>0$
	\begin{equation}
	\label{0r}
		\begin{split}
			\|I - J_\ee\|_{L(H^s(\ci), H^r(\ci))} \le o(\ee^{s-r}).
		\end{split}
	\end{equation}
\end{proposition}
Using Proposition \ref{lem4r}, we conclude from estimate \eqref{conc-lemma} that
\begin{equation*}
	\begin{split}
		\|v\|_{H^{\sigma}(\ci)} \le C \cdot o(\ee^{s - \sigma}), \qquad |t|
		\le T
	\end{split}
\end{equation*}
where $C=C(\|u_0\|_{H^s(\ci)})$, completing the proof of Lemma \ref{lem6r}.
$\qquad \Box$
%
\subsection{ Proof of Proposition \ref{lem4r}.}
Pick an arbitrary $u \in H^s(\ci)$ such that $\|u\|_{H^s(\ci)} = 1$, and $r, s \in \rr$ such that $r \le s$. Using the fact that
$\widehat{j_\ee}(\xi) = \widehat{j}(\ee \xi)$ by construction, we have 
\begin{equation}
	\begin{split}
		\|u - J_\ee u\|_{H^r(\ci)}^2 
		& = \sum_{\xi \in \zz} |\widehat{u}(\xi) - \widehat{j_\ee * u}(\xi) |^2
		(1+\xi^2)^r
		\\
		& = \sum_{\xi \in \zz} |\widehat{u}(\xi) - \widehat{j_\ee}(\xi)
		\widehat{u}(\xi) |^2 (1+\xi^2)^r
		\\
		& = \sum_{\xi \in \zz} | [1- \widehat{j_\ee}(\xi] \cdot \widehat{u}(\xi) |^2
		(1+\xi^2)^r
		\\
		& = \sum_{\xi \in \zz} | [1- \widehat{j}(\ee \xi)] \cdot \widehat{u}(\xi) |^2
		(1+\xi^2)^r.
		\label{1r}
	\end{split}
\end{equation}
Assume $r \le s$. Then by construction (see \ref{0u}) we have
\begin{equation*}
	\begin{split}
		|1 - \widehat{j } (\xi) | \le |\xi|^{s-r}
	\end{split}
\end{equation*}
for all $\xi \in \rr$; hence
\begin{equation}
	\begin{split}
		|1 - \widehat{ j }(\ee \xi)| \le |\ee \xi |^{s-r}, \quad \forall
		\xi \in \rr, \ \ee > 0.
		\label{2r}
	\end{split}
\end{equation}
Applying \eqref{2r} to \eqref{1r} and recalling that $r \le s$, we obtain
\begin{equation}
	\label{2pr}
	\begin{split}
	\|u - J_\ee u\|_{H^r(\ci)}^2 
	& \le \sum_{\xi \in \zz}  |\ee \xi |^{2(s-r)}
	|\widehat{u}(\xi)|^2 (1 + \xi^2)^r
	\\
	& = \ee^{2(s-r)} \sum_{\xi \in \zz} |\widehat{u}(\xi)|^2  \cdot (\xi^2)^{s-r}
	(1 + \xi^2)^{r-s} (1 + \xi^2)^{s}
	\\
	& \le \ee^{2(s-r)}
	\sum_{\xi \in \zz} |\widehat{u}(\xi)|^2 (1 + \xi^2)^s
	\\
	& =  \ee^{2(s-r)}.
	\end{split}
\end{equation}
Furthermore,
\begin{equation*}
	\begin{split}
		& |[1- \widehat{j_\ee}(\xi)] \cdot \widehat{u}(\xi)|^2 (1 + \xi^2)^r \le
		|\widehat{u}(\xi)|^2 (1 + \xi^2)^r, \quad \ee > 0, \ \text{and}
		\\
		& \sum_{\xi \in \zz} |\widehat{u}(\xi)|^2 (1 + \xi^2)^r < \infty;
	\end{split}
\end{equation*}
therefore, by the dominated convergence theorem for series
\begin{equation}
	\label{o1}
	\begin{split}
		\lim_{\ee \to 0} \|u - J_\ee u \|_{H^r }^2 
		& = \lim_{\ee \to 0} \sum_{\xi \in \zz} |[1-\widehat{j_\ee}(\xi)]
		\widehat{u}(\xi) |^2 (1 + \xi^2)^r
		\\
		& = \lim_{\ee \to 0} \sum_{\xi \in \zz} |[1-\widehat{j}(\ee \xi)]
		\widehat{u}(\xi) |^2 (1 + \xi^2)^r
		\\
		& = \sum_{\xi \in \zz} \lim_{\ee \to 0} |[1-\widehat{j}(\ee \xi)]
		\widehat{u}(\xi) |^2 (1 + \xi^2)^r
		\\
		& = 0.
	\end{split}
\end{equation}
To complete the proof of Proposition \ref{lem4r}, we take note of the following interpolation result:
\begin{remark}
	\label{lem2r}
	For $\sigma < r \le s$ and arbitrary $u \in L^2(\ci)$,
	\begin{equation}
		\begin{split}
			\|u\|_{H^{r}(\ci)} \le
			\|u\|_{H^\sigma(\ci)}^{(r-s)/(\sigma -s)}
			\|u\|_{H^s(\ci)}^{1 - (r-s)/(\sigma -s)}.
			\label{16u}
		\end{split}
	\end{equation}
\end{remark}
%
%
%
%
\subsection{ Proof.} Assuming $u \in L^2(\ci)$ and $\sigma < r \le s$,
we rewrite and apply Holder's inequality:
\begin{equation*}
	\begin{split}
		&\|u\|_{H^{r}(\ci)}^2
		\\
		& = \sum_{\xi \in \zz} |\widehat{u}(\xi)|^2 (1 + \xi^2)^{r}
		\\
		& = \sum_{\xi \in \zz}
		\left [|\widehat{u}(\xi)|^2 (1 + \xi^2)^\sigma \right ]^{(r-s)/(\sigma -s)}
		\cdot \left [ |\widehat{u}(\xi )
		|^2 (1+ \xi^2)^s \right ] ^{1 - (r-s)/(\sigma -s)} 
		\\
		& \le \|\left[ |\widehat{u}(\xi)|^2 (1 + \xi^2)^\sigma
		\right]^{(r-s)/(\sigma -s)} \|_{l^{(\sigma -s)/(r-s)}(\zz)}
		\\
		& \cdot \|\left[ |\widehat{u}(\xi)|^2 (1 + \xi^2)^\sigma
		\right]^{1- (r-s)/(\sigma -s)} \|_{l^{1/[1 -(\sigma -s)/(r-s)]}(\zz)}
		\\
		& = \|v\|_{H^\sigma(\ci)}^{2(r-s)/(\sigma -s)}
		\|v\|_{H^s(\ci)}^{2[1 - (r-s)/(\sigma -s)]}
	\end{split}
\end{equation*}
from which the result follows. 
Assume without loss of generality that $s > 0$. Applying Remark \ref{lem2r}, and estimates \eqref{2pr} and \eqref{o1}, we
see that for $r>0$ 
\begin{equation*}
	\begin{split}
		\|u - J_\ee u \|_{H^r(\ci)}
		& \le \|u - J_\ee u
		\|_{L^2(\ci)}^{(s-r)/s} \|u - J_\ee u \|_{H^s(\ci)}^{1 -
		(s-r)/s}
		\\
		& = \left( \ee^{s} \right)^{(s-r)/s} \cdot o(1)
		\\
		& = o(\ee^{s-r})
	\end{split}
\end{equation*}
Similarly, for $r < 0$
\begin{equation*}
	\begin{split}
		\|u - J_\ee u \|_{H^r(\ci)}^2
		& \le \|u - J_\ee u
		\|_{H^\sigma(\ci)}^{(r-s)/(\sigma - s)} \|u - J_\ee u \|_{H^s(\ci)}^{1 -
		(r-s)/(\sigma -s)}
		\\
		& = \left( \ee^{s-\sigma} \right)^{(r-s)/(\sigma -s)} \cdot o(1)
		\\
		& = o(\ee^{s-r})
	\end{split}
\end{equation*}
Lastly, for the case $r=0$, we note that \eqref{o1} implies $\|u - J_\ee u
\|_{H^r(\ci)} \le o(1)$ for all $r \le s$. Hence, the proof of Proposition
\ref{lem4r} is complete.  $\quad \Box$
%
%
%
%
%
%
%
%
%
%
%
%
%
%
%
%
%
%
%
%
%
%\subsection{ Proof.} By \eqref{uniform_bound_for_u}, we have
%\begin{equation}
%	\begin{split}
%		\|u^\ee(t, \cdot \|_{H^r(\ci)}^2
%		& \le C \|u^\ee (0, \cdot)
%		\|_{H^r(\ci)}^2
%		\\
%		& = \|J_\ee u_0 \|_{H^r(\ci)}^2
%		\\
%		& = \sum_{\xi \in \zz} |\widehat{j_\ee}(\xi) \widehat{u_0}(\xi) |^2
%		\cdot (1 + \xi^2)^r
%		\\
%		& \le \sum_{\xi \in \zz} |[1-\widehat{ j_\ee}(\xi)] \widehat{u_0}(\xi) |^2
%		\cdot (1 + \xi^2)^r
%		\\
%		& + \sum_{\xi \in \zz} |\widehat{j_\ee}(\xi) \widehat{u_0}(\xi) |^2
%		\cdot (1 + \xi^2)^r.
%		\label{1q}
%	\end{split}
%\end{equation}
%Since $J_\ee u_0$ is smooth, by \eqref{uniform_bound_for_u} we have
%\begin{equation*}
%	\begin{split}
%		\sum_{\xi \in \zz} |\widehat{j_\ee}(\xi) \widehat{u_0}(\xi) |^2
%		\cdot (1 + \xi^2)^r
%		= \|J_\ee u_0\|_{H^r(\ci)} = C_r
%	\end{split}
%\end{equation*}
%for all $r \ge 3/2$.
%
%
%
%
%
%
%
%
We now return to analyzing
\eqref{15u}. Applying Remark \ref{lem5r}, Remark \ref{lem2r}, and Lemma
\ref{lem6r}, we have
\begin{equation}
	\begin{split}
		\label{200x}
		\|u^\ee \|_{H^{s+1}(\ci)} \|v \|_{H^{s-1}(\ci)} \|v\|_{H^s(\ci)}
		& \le C''' \ee^{-1} \cdot \|v\|_{H^\sigma(\ci)}^{1/(s-\sigma)}
		\|v\|_{H^s(\ci)}^{2 - 1/(s- \sigma)}
		\\
		& \le C'' \ee^{-1} \cdot o(\ee^{s- \sigma})^{1/(s-\sigma)}
		\|v\|_{H^s(\ci)}^{2- 1/(s-\sigma)}
		\\
		& \le C'' \cdot o(1) \cdot \|v\|_{H^s(\ci)}^{2- 1/(s-\sigma)}
	\end{split}
\end{equation}
where we stress that $C'' = C''(\|u_0\|_{H^s(\ci)})$ does not depend on $\ee$.
Hence, $\|v\|_{H^{s-1}(\ci)}$ has proved to be sufficient to control the
growth of $\|u^\ee \|_{H^{s+1}(\ci)}$. For the remaining terms of
\eqref{15u}, we leave $\|v\|_{H^s(\ci)}^3$ as is, and note that by Remark \ref{lem5r}
\begin{equation}
	\begin{split}
		\|u^\ee\|_{H^s(\ci)} \|v\|_{H^s(\ci)}^2 \le C'
		\cdot \|v\|_{H^s(\ci)}^2
		\label{u-ep-bound}
	\end{split}
\end{equation}
where $C' = C'(\|u_0\|_{H^s(\ci)})$. Hence, applying \eqref{u-ep-bound} and \eqref{200x} to \eqref{15u}, we obtain
\begin{equation}
	\begin{split}
		\frac{1}{2} \frac{d}{dt} \|v\|_{H^s(\ci)}^2 \le C (
		\|v\|_{H^s(\ci)}^3 + \|v\|_{H^s(\ci)}^2 + \ee ^{-1} o(\ee) \|v\|_{H^s(\ci)}^{2-
		1/(s- \sigma)}).
		\label{201x}
	\end{split}
\end{equation}
where $C=C(\|u_0\|_{H^s(\ci)}$ does not depend on $\ee$. We also remark 
that $\|v(t)\|_{H^s(\ci)}$ is uniformly bounded for all $\ee > 0$, since by
the triangle inequality, Proposition \ref{lem3r}, and part (iii) of Theorem
\ref{thm:HR_existence_continous_dependence} we have
\begin{equation*}
	\begin{split}
		\|v(t) \|_{H^s(\ci)}
		& = \|u - u^\ee \|_{H^s(\ci)}
		\\
		& \le \|u \|_{H^s(\ci)} + \|u^\ee \|_{H^s(\ci)}
		\\
		& \le 2( \|u_0\|_{H^s(\ci)} + \|J_\ee u_0\|_{H^s(\ci)})
		\\
		& \le 4 \|u_0\|_{H^s(\ci)}.
	\end{split}
\end{equation*}
Hence, \eqref{201x} gives
\begin{equation}
	\begin{split}
		\lim_{\ee \to 0} \frac{1}{2} \frac{d}{dt} \|v\|_{H^s(\ci)}^2 \le
		 \lim_{\ee \to 0} C (
		\|v\|_{H^s(\ci)}^3 + \|v\|_{H^s(\ci)}^2).
		\label{202x}
	\end{split}
\end{equation}
Differentiating the left hand side of
\eqref{202x} and simplifying, it follows that
\begin{equation*}
	\begin{split}
		\lim_{\ee \to 0}\frac{d}{dt} \|v\|_{H^s(\ci)} \le
		\lim_{\ee \to 0} C (\|v\|_{H^s(\ci)}^2 +
		\|v\|_{H^s(\ci)}).
	\end{split}
\end{equation*}
Letting $y = \|v\|_{H^s(\ci)}$ and rearranging, we obtain
\begin{equation*}
	\begin{split}
		\lim_{\ee \to 0} \ \frac{1}{y(y+1)} \frac{dy}{dt} \le C	
	\end{split}
\end{equation*}
which can be rewritten as
\begin{equation*}
	\begin{split}
		\lim_{\ee \to 0}
		\left( \frac{1}{y} - \frac{1}{y+1} \right)\frac{dy}{dt} \le C
	\end{split}
\end{equation*}
implying
\begin{equation}
	\label{est-int'}
	\begin{split}
		\lim_{\ee \to 0} \left [
\int_0^t \frac{1}{y} \frac{dy}{d \tau} \ d \tau
		- \int_0^t \frac{1}{y+1} \frac{dy}{d \tau} \ d \tau \right ]
		\le \int_0^t C \ d \tau, \quad |t| \le T.
	\end{split}
\end{equation}
Hence \eqref{est-int'} gives 
\begin{equation}
	\begin{split}
	\lim_{\ee \to 0} 
	\left [ \ln \left | \frac{y(t)}{y(0)}
	\cdot \frac{y(0) + 1}{y(t) + 1} \right | \right ] \le C t.
		\label{301''qx}
	\end{split}
\end{equation}
Exponentiating both sides of \eqref{301''qx}, and noting that $f(x) = e^x$
is a continuous function on $\rr$, we must have
\begin{equation*}
	\begin{split}
		\lim_{\ee \to 0}  \
		\left | \frac{y(t)}{y(0)} \cdot \frac{y(0) + 1}{y(t) + 1} \right | \le e^{C t}.
	\end{split}
\end{equation*}
Rearranging, and recalling that $y(t) = \|v(t)\|_{H^s(\ci)} \ge 0$, we obtain
\begin{equation*}
	\begin{split}
		\lim_{\ee \to 0} \frac{y(t)}{y(t) + 1}
		\le \lim_{\ee \to 0} \frac{e^{C t} \cdot y(0)}{y(0) + 1} \le
		\lim_{\ee \to 0} e^{C t} \cdot y(0).
	\end{split}
\end{equation*}
Substituting back in $\|v(t)\|_{H^s(\ci)}$ for $y(t)$ gives
\begin{equation}
	\begin{split}
		\lim_{\ee \to 0}	\frac{\|v(t)\|_{H^s(\ci)}}{\|v(t)\|_{H^s(\ci)} + 1}  \le
		\lim_{\ee \to 0} e^{C t} \cdot \|v(0)\|_{H^s(\ci)}.
		\label{303'qx}
	\end{split}
\end{equation}
Since 
\begin{equation*}
	\label{303''qx}
	\begin{split}
		\|v(0)\|_{H^s(\ci)} = \|u_0 - J_\ee u_0 \|_{H^s(\ci)} \le
		\|u_0\|_{H^s(\ci)} \cdot o(1)
	\end{split}
\end{equation*}
by Proposition \ref{lem4r}, we conclude from \eqref{303'qx} that
\begin{equation}
	\label{304qx}
	\begin{split}
		\lim_{\ee \to 0} \|v(t)\|_{H^s(\ci)} = \lim_{\ee \to 0}
		\|u^\ee(t) - u(t)\|_{H^s(\ci)}= 0, \qquad |t| \le T,
	\end{split}
\end{equation}
and since the family $\left\{ u^\ee - u \right\}_\ee$ does not depend on $n$,
the proof of \eqref{enough_to_prove1} is complete. 
%
%
%
%
%
%
%
%
\subsection{ Proof of \eqref{enough_to_prove2}.} 
Let $v = u^\ee_n - u^\ee$. Then $v$ solves the Cauchy problem
\begin{align}
		\label{4qu}
		\p_t v 
		& =  -\gamma (v \p_x v + v \p_x u^\ee + u^\ee \p_x v)  
		\\
		& - D^{-2} \p_x \left\{ \left (\frac{3-\gamma}{2} \right )(v^2 +
		2u^\ee v) + \frac{\gamma}{2}\left[ (\p_x v)^2 + 2 \p_x u^\ee \p_x v \right]
		\right\}, \notag
		\\
		& v(0) =J_\ee(u_{0,n} - u_0).
		\label{5qu}
	\end{align}
Applying the operator $D^s$ to both sides of \eqref{4qu}, multiplying by
	$D^s$ and integrating, we have
	\begin{equation}
		\begin{split}
			\frac{1}{2}\frac{d}{dt} \|v\|_{H^s(\ci)} = A + B
			\label{6qu}
		\end{split}
	\end{equation}
	where
	\begin{equation}
		\begin{split}
			A
			& =  -\gamma \int_{\ci} D^s(v \p_x v) \cdot D^s v \
			dx
			- \frac{3- \gamma}{2} \int_\ci D^{s-2} \p_x (v^2) \cdot D^s v
			\ dx
			\\
			& - \frac{\gamma}{2}\int_\ci D^{s-2} \p_x (\p_x v)^2 \cdot D^s
			v \ dx
			\label{7qu}
		\end{split}
	\end{equation}
	and
	\begin{equation}
		\begin{split}
			B 
			 = &  \overbrace{-\gamma \int_\ci D^s (u^\ee \p_x v) \cdot D^s v \
			 dx}^{(i)}
			 \ \overbrace{-\gamma \int_\ci D^s (v \p_x u^\ee ) \cdot D^s v \
			 dx}^{(ii)}
			 \\
			  & \overbrace{- \ ( 3- \gamma) \int_\ci D^{s-2} \p_x (u^\ee v) \cdot D^s
			 v \ dx}^{(iii)}
			 \\
			 & \overbrace{-\gamma \int_\ci D^{s-2} \p_x
			(\p_x u^\ee \cdot \p_x v) \cdot D^s v \
			dx}^{(iv)}.
			\label{8qu}
		\end{split}
	\end{equation}
	Estimating as in \eqref{8'u}-\eqref{14u}, we obtain
	\begin{equation}
		\begin{split}
			\frac{1}{2}\frac{d}{dt}\|v\|_{H^{s}(\ci)}^2
			& \le c_s(\|v\|_{H^s(\ci)}^3 + \|u^\ee\|_{H^s(\ci)}
			\|v\|_{H^s(\ci)}^2
			\\
			& + \|u^\ee\|_{H^{s+1}(\ci)}
			\|v\|_{H^{s-1}(\ci)} \|v\|_{H^s(\ci)}).
			\label{15qu}
		\end{split}
	\end{equation}
	We now aim to control the growth of $\|u^\ee\|_{H^{s+1}(\ci)}$ by
	$\|v\|_{H^{s-1}(\ci)}$. To do so, we will need an estimate for
	$\|v\|_{H^{s-1}(\ci)}$, which we will obtain through the following lemma:
%
%
%
%
\begin{lemma}
	\label{lem:left}
	For $\sigma$ such that $1/2 < \sigma < 1$ and $\sigma + 1 \le s$, we have
	\begin{equation}
	\label{lem6rq}
	\begin{split}
		\|v\|_{H^{\sigma}(\ci)} = 
		\|u^\ee_n - u^\ee\|_{H^\sigma(\ci)}
		\le C \cdot o(\ee^{s- \sigma }) + \|u_0 - u_{0,n} \|_{H^s(\ci)}, \qquad |t| \le T
	\end{split}
\end{equation}
where $C=C(\|u_0\|_{H^s(\ci)})$.
\end{lemma}
%
%
%
\subsection{ Proof.}
Repeating calculations \eqref{X}-\eqref{12}, with $E$ set to zero, $u^{\omega,n}$
replaced by $u^\ee_n$, $u_{\omega,n}$ replaced by $u^\ee$, and $\sigma$ and $\rho$ chosen such that
\begin{equation}
	\label{size_of_sigma'}
	\begin{split}
	& 1/2 < \sigma < 1,
	\\
	& \sigma + 1 \le \rho \le s 
	\end{split}
\end{equation}
yields
 \begin{equation*}
	\begin{split}
		\frac{1}{2}\frac{d}{dt} \|v\|_{H^\sigma(\ci)}^2
		& \le
		C'' (\|u^{\ee}_n + u^\ee \|_{H^{\rho}(\ci)} +
		\|\p_x(u^{\ee}_n + u^\ee) \|_{H^\sigma(\ci)})
		\cdot \|v\|_{H^\sigma(\ci)}^2.
	\end{split}
\end{equation*}
\medskip
It follows that 
\begin{equation}
	\begin{split}
		\frac{1}{2}\frac{d}{dt} \|v\|_{H^{\sigma}(\ci)}^2
		& \le
		C'' \cdot \|u^{\ee}_n
		+ u^\ee\|_{H^{s}(\ci)}\cdot \|v\|_{H^{\sigma}(\ci)}^2.
		\label{10qx}
	\end{split}
\end{equation}
Applying the triangle inequality and
part (iii) of Theorem \ref{thm:HR_existence_continous_dependence} (proved in
Section \ref{existence})
to \eqref{10qx} yields
%
\begin{equation}
	\begin{split}
		\label{11qx}
		\frac{1}{2}\frac{d}{dt} \|v\|_{H^{\sigma}(\ci)}^2
		& \le
		C' (\|u^{\ee}_n(0)\|_{H^{s}(\ci)}
		+ \|u^\ee(0)\|_{H^{s}(\ci)})\cdot \|v\|_{H^{\sigma}(\ci)}^2
		\\
		& = C' (\|J_\ee u_{0,n}\|_{H^{s}(\ci)}
		+ \|J_\ee u_0\|_{H^{s}(\ci)})\cdot \|v\|_{H^{\sigma}(\ci)}^2.
	\end{split}
\end{equation}
Note that the family $\left\{ u_{0,n} \right\}_n$ is uniformly bounded in
$H^s(\ci)$. Hence, applying Lemma \ref{lem3r} to \eqref{11qx} we obtain the critical estimate 
\begin{equation}
	\begin{split}
		\label{12qx}
		\frac{1}{2}\frac{d}{dt} \|v\|_{H^{\sigma}(\ci)}^2
		& \le
	C \|v\|_{H^{\sigma}(\ci)}^2
\end{split}
\end{equation}
with $C = C(\|u_0\|_{H^s(\ci)}, \ R)$, where
\begin{equation}
	\label{r-def}
	R = \inf \left\{ R' \in \rr:\ \{u_{0,n}\} \subset B_{H^s(\ci)}(R',0)
	\right\}.
\end{equation}
Differentiating the left hand side of \eqref{12qx} and simplifying, we
obtain
\begin{equation}
	\begin{split}
		\frac{d}{dt}\|v\|_{H^{\sigma}(\ci)} \le C \|v\|_{H^{\sigma}(\ci)}.
		\label{100qx}
	\end{split}
\end{equation}
Let $y(t) = \|v\|_{H^{\sigma}(\ci)}$. Then \eqref{100qx} gives
\begin{equation*}
	\begin{split}
		\frac{1}{y(t)}\frac{dy}{dt} \le C.
	\end{split}
\end{equation*}
Hence,
\begin{equation*}
	\begin{split}
		\int_0^t \frac{1}{y(\tau)} \frac{dy}{d \tau}
		\le \int_0^t C \ d \tau, \qquad |t| \le T
	\end{split}
\end{equation*}
from which we obtain
\begin{equation}
	\begin{split}
		\ln |y(t) | - \ln |y(0)| \le C t.
		\label{101qx}
	\end{split}
\end{equation}
Simplifying \eqref{101qx}, we have
\begin{equation*}
	\begin{split}
		\ln \left |\frac{y(t)}{y(0)} \right | \le C t
	\end{split}
\end{equation*}
which yields the estimate
\begin{equation*}
	\begin{split}
		y(t) \le y(0) e^{C t}, \qquad |t| \le T.
	\end{split}
\end{equation*}
Substituting back in $\|v\|_{H^{\sigma}(\ci)}$ for $y$, we get
\begin{equation*}
	\begin{split}
		\|v\|_{H^{\sigma}(\ci)}
		& \le e^{C t}\|v(0)\|_{H^{\sigma}(\ci)}
		\\
		& = e^{C t}\|u^\ee(0) - u^\ee_n(0) \|_{H^{\sigma}(\ci)}.
	\end{split}
\end{equation*}
To conclude the proof, we apply the following:
\begin{proposition}
		\label{lem11r}
	For $r \le s$,
	\begin{equation}
		\begin{split}
			\|u^\ee(0) - u_n^\ee (0) \|_{H^r(\ci)} \le C
			\cdot o(\ee^{s-r}) + \|u_0 - u_{0,n} \|_{H^s(\ci)}
			\label{3w}
		\end{split}
	\end{equation}
	where $C=C(\|u_0\|_{H^s(\ci)})$ does not depend on $n$.
\end{proposition}
%
%
Recalling \eqref{r-def}, we deduce by Proposition \ref{lem11r}
\begin{equation*}
	\begin{split}
		\|v\|_{H^{\sigma}(\ci)} \le C \cdot o(\ee^{s - \sigma}) + \|u_0 -
		u_{0,n} \|_{H^s(\ci)} \qquad |t| \le T
	\end{split}
\end{equation*}
where $C=C(\|u_0\|_{H^s(\ci)}), \ R)$ does not depend
on $n$, completing the proof of Lemma \ref{lem:left}. $\qquad \Box$
%
%
\subsection{ Proof of Proposition \ref{lem11r}.} We write
\medskip
\begin{equation}
	\begin{split}
		\|u^\ee(0) - u_n^\ee (0) \|_{H^r(\ci)} 
		& = \|J_\ee u_0 - J_\ee u_{0,n} \|_{H^r(\ci)}
		\\
		& \le \|J_\ee u_0 - u_0 \|_{H^r(\ci)} + \| u_0 - u_{0,n}
		\|_{H^r(\ci)}
		\\
		& + \|u_{0,n} - J_\ee u_{0,n} \|_{H^r(\ci)}
		\\
		& \le \|I - J_\ee\|_{L(H^s(\ci), H^r(\ci))} \|u_0\|_{H^s(\ci)}
		\\
		& +
		\|u_0 - u_{0,n} \|_{H^r(\ci)} + 
		\|I - J_\ee\|_{L(H^s(\ci), H^r(\ci))} \|u_{0,n}\|_{H^s(\ci)}.
		\label{4w}
	\end{split}
\end{equation}
Applying Proposition \ref{lem4r} to \eqref{4w}, and recalling that the family
$\left\{ u_{0,n} \right\}_n$ belongs to a bounded subset of
$H^s(\ci)$, we have
\medskip
\begin{equation}
	\label{finito}
	\begin{split}
		\|u^\ee(0) - u_n^\ee (0) \|_{H^r(\ci)} 
		& \le
		C' \cdot o(\ee^{s-r}) \cdot \|u_0\|_{H^s(\ci)}
		 \\
		 & + \|u_0 - u_{0,n} \|_{H^r(\ci)} + C' \cdot o (\ee^{s-r}) \cdot
		 \|u_{0,n}\|_{H^s(\ci)}
		 \\
		 & \le
		 C' \cdot o(\ee^{s-r}) \cdot \|u_0\|_{H^s(\ci)}
		 \\
		 & + \|u_0 - u_{0,n} \|_{H^s(\ci)} + C' \cdot o (\ee^{s-r}) \cdot
		 R
	\end{split}
\end{equation}
where $R$ is defined as in \ref{r-def}. The result follows immediately from
\eqref{finito}. $\qquad \Box$
We are now prepared to interpolate. Recall \eqref{15qu}. Applying Remark \ref{lem5r}, Remark \ref{lem2r}, and
Proposition \ref{lem11r} gives
\begin{equation*}
	\begin{split}
		& \|u^\ee \|_{H^{s+1}(\ci)} \|v\|_{H^{s-1}(\ci)} \|v\|_{H^s
		(\ci)}
		\\
		&\le C' \ee^{-1} \cdot \|v\|_{H^\sigma(\ci)}^{1/(s-\sigma)}
		\|v\|_{H^s(\ci)}^{2 - 1/(s- \sigma)}
		\\
		& \le C' \ee^{-1} \cdot \Big [C \cdot o(\ee^{s- \sigma}) + \|u_0 -
		u_{0,n}\|_{H^s(\ci)} \Big ]^{1/(s-\sigma)}
		\cdot \|v\|_{H^s(\ci)}^{2- 1/(s-\sigma)}
	\end{split}
\end{equation*}
from which we obtain
\begin{equation}
	\begin{split}
		\label{200qx}
		\|u^\ee\|_{H^{s+1}(\ci)} \|v\|_{H^{s-1}(\ci)} \|v \|_{H^s(\ci)}
		& \lesssim  o(1) + \ee^{-1}
		\|u_0-u_{0,n}\|_{H^s(\ci)}^{1/(s-\sigma)}\|v\|_{H^s(\ci)}^{2- 1/(s-\sigma)}.
	\end{split}
\end{equation}
We wish to control the growth of the second term of the
right hand side of \eqref{200qx}.
First, note that the triangle inequality, part (iii) of Theorem
\ref{thm:HR_existence_continous_dependence} and Proposition \ref{lem3r} imply
\begin{equation}
	\begin{split}
		\|v\|_{H^s(\ci)} & = \|u^\ee_n - u^\ee \|_{H^s(\ci)} 
		\\
		& \le \|u^\ee_n\|_{H^s(\ci)} + \|u^\ee \|_{H^s(\ci)}  
		\\
		& \le 2\left[  \|J_\ee u_{0,n}\|_{H^s(\ci)} + \|J_\ee u_0 \|_{H^s(\ci)} 
		 \right]
		\\
		& \le 2 \left[ \|u_{0,n} \|_{H^s(\ci)} + \|u_0 \|_{H^s(\ci)} 
		\right], \qquad |t| \le T
		\label{growth_v}
	\end{split}
\end{equation}
and since $\{u_{0,n}\}_n$ belongs to a bounded subset of
$H^s(\ci)$, we see from \eqref{growth_v} that $\|v \|_{H^s(\ci)}$ is
uniformly bounded in $n$ \emph{and} $\ee$.  Secondly, since $\|u_0 -
u_{0,n} \|_{H^s(\ci)} \to 0$ uniformly in $n$, then for any given $\ee$ we
can chose a family $\{N_j\} $ such that
\begin{equation}
	\begin{split}
		\|u_0 - u_{0,n} \|_{H^s(\ci)} \lesssim
		\frac{\ee^{(s-\sigma)}}{2^{j(s -\sigma)}}, \quad n >
		N_j.
		\label{uniform_n}
	\end{split}
\end{equation}
Thirdly, by Remark \ref{lem5r}, we have 
\begin{equation}
	\label{u-ee-bound}
	\|u^\ee \|_{H^s(\ci)} \le C(\|u_0\|_{H^s(\ci)}), \quad \forall \ee > 0.
\end{equation}
Applying \eqref{200qx} to \eqref{15qu} in light of 
\eqref{growth_v}, \eqref{uniform_n}, and \eqref{u-ee-bound}, we obtain
\begin{equation*}
		\begin{split}
			\lim_{n \to \infty }
			\frac{1}{2}\frac{d}{dt}\|v\|_{H^{s}(\ci)}^2
			& \le
			C \lim_{n \to \infty} \Big [\|v\|_{H^s(\ci)}^3 +
			\|v\|_{H^s(\ci)}^2 + o(1)\Big ]
		\end{split}
	\end{equation*}
	for every $\ee > 0$, where $C = C(\|u_0\|_{H^s(\ci)}, \ R)$ with
	$R$ defined as in \eqref{r-def}; hence we have
\begin{equation}
		\begin{split}
			\lim_{\substack{n \to \infty \\ \ee \to 0} }
			\frac{1}{2}\frac{d}{dt}\|v\|_{H^{s}(\ci)}^2
			& \le C
			\lim_{\substack{n \to \infty \\ \ee \to 0}}
			\Big [\|v\|_{H^s(\ci)}^3 + 
			\|v\|_{H^s(\ci)}^2 \Big ].
			\label{15qx}
		\end{split}
	\end{equation}
	We differentiate the left hand side of \eqref{15qx} and obtain
\begin{equation*}
	\begin{split}
		\lim_{\substack{n \to \infty \\ \ee \to 0}}\frac{d}{dt}
		\|v\|_{H^s(\ci)} \le C
		\lim_{\substack{n \to \infty \\ \ee \to 0}} \left [\|v\|_{H^s(\ci)}^2 +
		\|v\|_{H^s(\ci)} \right ].
	\end{split}
\end{equation*}
Letting $y = \|v\|_{H^s(\ci)}$ and rearranging gives
\begin{equation*}
	\begin{split}
		\lim_{\substack{n \to \infty \\ \ee \to 0} } \ \frac{1}{y(y+1)} \frac{dy}{dt}
		\le	C
	\end{split}
\end{equation*}
which can be rewritten as
\begin{equation*}
	\begin{split}
		\lim_{\substack{n \to \infty \\ \ee \to 0} }
		\left( \frac{1}{y} - \frac{1}{y+1} \right)\frac{dy}{dt} \le C 
	\end{split}
\end{equation*}
implying
\begin{equation}
	\label{est-int}
	\begin{split}
		\lim_{\substack{n \to \infty \\ \ee \to 0} } \left [
\int_0^t \frac{1}{y} \frac{dy}{d \tau} \ d \tau
		- \int_0^t \frac{1}{y+1} \frac{dy}{d \tau} \ d \tau \right ]
		\le \int_0^t C \ d \tau, \quad |t| \le T.
	\end{split}
\end{equation}
Recalling that $y(t) = \|v(t)\|_{H^s(\ci)} > 0$, \eqref{est-int} gives 
\begin{equation}
	\begin{split}
	\lim_{\substack{n \to \infty \\ \ee \to 0} }
	\left [ \ln \left ( \frac{y(t)}{y(0)}
	\cdot \frac{y(0) + 1}{y(t) + 1} \right ) \right ] \le C t.
		\label{301'qx}
	\end{split}
\end{equation}
Exponentiating both sides of \eqref{301'qx}, and noting that $f(x) = e^x$
is a continuous function on $\rr$, we must have
\begin{equation*}
	\begin{split}
		\lim_{\substack{n \to \infty \\ \ee \to 0} } \
		\frac{y(t)}{y(0)} \cdot \frac{y(0) + 1}{y(t) + 1} \le e^{C t}.
	\end{split}
\end{equation*}
Rearranging, we obtain
\begin{equation*}
	\begin{split}
		\lim_{\substack{n \to \infty \\ \ee \to 0}} \frac{y(t)}{y(t) + 1}
		\le \lim_{\substack{n \to \infty \\ \ee \to 0}} \frac{e^{C t} \cdot y(0)}{y(0) + 1} \le
		\lim_{\substack{n \to \infty \\ \ee \to 0}} e^{C t} \cdot y(0).
	\end{split}
\end{equation*}
Substituting back in $\|v(t)\|_{H^s(\ci)}$ for $y(t)$ gives
\begin{equation}
	\begin{split}
		\lim_{\substack{n \to \infty \\ \ee \to 0}}	\frac{\|v(t)\|_{H^s(\ci)}}{\|v(t)\|_{H^s(\ci)} + 1}  \le
		\lim_{\substack{n \to \infty \\ \ee \to 0}} e^{C t} \cdot \|v(0)\|_{H^s(\ci)}.
		\label{303qx}
	\end{split}
\end{equation}
Since by Proposition \ref{lem3r} 
\begin{equation*}
	\begin{split}
	\lim_{\substack{n \to \infty \\ \ee \to 0} }
	\|v(0)\|_{H^s(\ci)}
	& = \lim_{\substack{n \to \infty \\ \ee \to 0} }
	\|J_\ee u_{0,n} - J_\ee u_0 \|_{H^s(\ci)} 
	\\
	& \le \lim_{n \to \infty } \|u_{0,n} - u_0 \|_{H^s(\ci)}
	\\
	& = 0
	\end{split}
\end{equation*}
we deduce from \eqref{303qx} that
\begin{equation*}
	\begin{split}
		\lim_{\substack{n \to \infty \\ \ee \to 0}} \|v(t)\|_{H^s(\ci)} = 0, \qquad |t| \le T
	\end{split}
\end{equation*}
completing the proof of \eqref{enough_to_prove2}. $\quad \Box$
%
%
%
\subsection{ Proof of \eqref{enough_to_prove3}.} 
Let $v = u_n - u^\ee_n$. Then $v$ solves the Cauchy problem
\begin{align}
		\label{a4qu}
		\p_t v 
		& =  -\gamma (v \p_x v + v \p_x u^\ee_n + u^\ee_n \p_x v)  
		\\
		& - D^{-2} \p_x \left\{ \left (\frac{3-\gamma}{2} \right )(v^2 +
		2u^\ee_n v) + \frac{\gamma}{2}\left[ (\p_x v)^2 + 2 \p_x u^\ee_n \p_x v \right]
		\right\}, \notag
		\\
		& v(0) = (I- J_\ee)u_{0,n}.
		\label{a5qu}
	\end{align}
Applying the operator $D^s$ to both sides of \eqref{a4qu}, multiplying by
	$D^s$ and integrating, we have
	\begin{equation}
		\begin{split}
			\frac{1}{2}\frac{d}{dt} \|v\|_{H^s(\ci)} = A + B
			\label{a6qu}
		\end{split}
	\end{equation}
	where
	\begin{equation}
		\begin{split}
			A
			& =  -\gamma \int_{\ci} D^s(v \p_x v) \cdot D^s v \
			dx
			- \frac{3- \gamma}{2} \int_\ci D^{s-2} \p_x (v^2) \cdot D^s v
			\ dx
			\\
			& - \frac{\gamma}{2}\int_\ci D^{s-2} \p_x (\p_x v)^2 \cdot D^s
			v \ dx
			\label{a7qu}
		\end{split}
	\end{equation}
	and
	\begin{equation}
		\begin{split}
			B 
			 = &  \overbrace{-\gamma \int_\ci D^s (u^\ee_n \p_x v) \cdot D^s v \
			 dx}^{(i)}
			 \ \overbrace{-\gamma \int_\ci D^s (v \p_x u^\ee_n ) \cdot D^s v \
			 dx}^{(ii)}
			 \\
			  & \overbrace{- \ ( 3- \gamma) \int_\ci D^{s-2} \p_x (u^\ee_n v) \cdot D^s
			 v \ dx}^{(iii)}
			 \\
			 & \overbrace{-\gamma \int_\ci D^{s-2} \p_x
			(\p_x u^\ee_n \cdot \p_x v) \cdot D^s v \
			dx}^{(iv)}.
			\label{a8qu}
		\end{split}
	\end{equation}
	Estimating as in \eqref{8'u}-\eqref{14u}, we obtain
	\begin{equation}
		\begin{split}
			\frac{1}{2}\frac{d}{dt}\|v\|_{H^{s}(\ci)}^2
			& \le c_s(\|v\|_{H^s(\ci)}^3 + \|u^\ee_n\|_{H^s(\ci)}
			\|v\|_{H^s(\ci)}^2
			\\
			& + \|u^\ee_n\|_{H^{s+1}(\ci)}
			\|v\|_{H^{s-1}(\ci)} \|v\|_{H^s(\ci)}).
			\label{a15qu}
		\end{split}
	\end{equation}
	Note that the first two terms in parentheses on the right hand side
	of \eqref{a15qu} will offer us little trouble;
	it is the third term that requires special care (due to the
	$\|u^\ee_n\|_{H^{s+1}(\ci)}$ factor, which becomes increasingly large as
	$\ee$ decreases). More precisely:
	%
	%
	%
	\begin{remark}
	\label{lem5r'}
	For $r \ge s > 3/2$ and $0 < \ee <<1$ 
	\begin{equation}
		\begin{split}
			\|u^\ee_n (t, \cdot) \|_{H^r(\ci)} \le C \, \ee^{s-r}
			\label{700r'}
		\end{split}
	\end{equation}
	for all $n \in \mathbb{N}$, with $C = C(r, R)$, where $R$ is defined as
	in \eqref{r-def}.
\end{remark}
\subsection{ Proof.} By part (iii) of Theorem
\ref{thm:HR_existence_continous_dependence}, proved in Section
\ref{existence}, we have
\begin{equation}
	\begin{split}
		\|u^\ee_n \|_{H^r(\ci)}^2
		& \le C' \|u^\ee_n (0)\|_{H^r(\ci)}^2
		\\
		& = C' \|J_\ee u_{0,n}\|_{H^r(\ci)}^2
		\\
		& = C' \sum_{\xi \in \zz} |\widehat{j_\ee} (\xi) \widehat{u_{0,n}}(\xi)
		|^2 \cdot (1 + \xi^2)^r
		\\
		& = C' \sum_{\xi \in \zz} |\widehat{j }(\ee \xi)|^2 \cdot (1 +
		\xi^2)^{r-s} \cdot |\widehat{u_{0,n}}(\xi)|^2 \cdot (1 + \xi^2)^s
		\\
		& = C'|\widehat{u_{0,n}}(0)|^2 +
		C' \sum_{\xi \in \zz \setminus {0}} |\widehat{j }(\ee \xi)|^2 \cdot (1 +
		\xi^2)^{r-s} \cdot |\widehat{u_{0,n}}(\xi)|^2 \cdot (1 + \xi^2)^s.
		\label{1qr'}
	\end{split}
\end{equation}
Assume $r \ge s$. Since $\widehat{j }(\xi) \in \mathcal{S}(\rr)$, 
\begin{equation}
	\label{schwartz'}
	\begin{split}
		|\widehat{j }(\ee \xi)| \le c_r |\ee \xi |^{s-r}, \quad \xi \neq 0.
	\end{split}
\end{equation}
Applying \eqref{schwartz'} to \eqref{1qr'}, we obtain
\begin{equation}
	\label{calc_ue'}
	\begin{split}
		\|u^\ee_n \|_{H^r(\ci)}^2 
		& \le C' |\widehat{u_{0,n}}(0) |^2 + c_r \sum_{\xi \in \zz \setminus
		{0}} |\ee \xi |^{2(s-r)} \cdot (1 + \xi^2)^{r-s}
		|\widehat{u_{0,n}}(\xi) |^2 \cdot (1 + \xi^2)^s
		\\
		& \le C' |\widehat{u_{0,n}}(0) |^2 + 2^{r-s} c_r \ee^{2(s-r)}
		\sum_{\xi \in \zz \setminus {0}} |\widehat{u_{0,n}}(\xi)|^2 \cdot (1 +
		\xi^2)^s
		\\
		& \le C' \|u_{0,n}\|_{H^s(\ci)}^2 + 2^{r-s} c_r \ee^{2(s-r)}
		\|u_{0,n}\|_{H^s(\ci)}^2
		\\
		& = (C' + 2^{r-s} c_r \ee^{2(s-r)}) \cdot \|u_{0,n}\|^2_{H^s(\ci)}.
	\end{split}
\end{equation}
Assuming $0 < \ee <<1$, and noting that
\begin{equation*}
	\begin{split}
		\|u_{0,n}\|_{H^s(\ci)} \le R, \quad \forall n \in \mathbb{N}
	\end{split}
\end{equation*}
where $R$ is defined as in \eqref{r-def},
we conclude from \eqref{calc_ue'} that 
\begin{equation*}
	\begin{split}
		\|u^\ee_n\|_{H^s(\ci)} \le C \ee^{s-r}
	\end{split}
\end{equation*}
where $C = C(r, R)$. $\qquad \Box$
%
In light of Remark \ref{lem5r'}, we now aim to control the growth of
$\|u^\ee_n\|_{H^{s+1}(\ci)}$ by $\|v\|_{H^{s-1}(\ci)}$. As before, we will
first obtain an estimate for $\|v\|_{H^\sigma(\ci)}$ for suitably chosen
$\sigma < s-1$. Then, we will use this estimate to interpolate between
$\|v\|_{H^\sigma(\ci)}$ and $\|v\|_{H^s(\ci)}$, yielding an estimate for
$\|v\|_{H^{s-1}(\ci)}$ which will allow us to control the growth of
$\|u^\ee_n\|_{H^{s+1}(\ci)}$. 
%
%
%
%
\begin{proposition}
	\label{prop:180}
If $\sigma$ is chosen appropriately in the range $1/2 < \sigma < 1$ and
$\sigma + 1 < s$, then for all $n \in \mathbb{N}$ 
	\begin{equation}
	\label{alem6rq}
	\begin{split}
		\|v\|_{H^{\sigma}(\ci)} = 
		\|u_n - u^\ee_n\|_{H^\sigma(\ci)}
		\le C \cdot o(\ee^{s- \sigma }), \qquad |t| \le T
	\end{split}
\end{equation}
with $C = C(R)$, where $R$ is defined as in \eqref{r-def}.
\end{proposition}
%
%
%
\subsection{ Proof.}
Recall that $v$ solves the Cauchy-problem \eqref{a4qu}-\eqref{a5qu}.
Applying $D^\sigma$ to both sides of \eqref{a4qu}, multiplying by
$D^\sigma v$, and integrating, we obtain the
relation
\begin{equation*}
	\begin{split}
		\frac{1}{2}\frac{d}{dt}\|v(t)\|_{H^\sigma(\ci)}^2
		= & - \frac{\gamma}{2}\int_{\ci} D^\sigma
		\p_x \left[ \left( u_n + u^\ee_n \right)v
		\right]\cdot D^\sigma v \ dx
		\\
		& - \frac{3-\gamma}{2}\int_{\ci} D^{\sigma
		-2} \p_x \left[ \left( u_n + u^\ee_n
		\right)v \right] \cdot D^\sigma v \ dx
		\\
		& - \frac{\gamma}{2}\int_{\ci} D^{\sigma
		-2}
		\p_x \left[ \left( \p_x u_n + \p_x u^\ee_n
		\right)\cdot \p_x v \right] \cdot
		D^\sigma v \ dx.
	\end{split}
\end{equation*}
Repeating calculations \eqref{X}-\eqref{12}, with $E$ set to zero,
$u^{\omega,n}$ replaced by $u$, $u_{\omega,n}$ replaced by $u^\ee$, and
$\sigma$ and $\rho$ chosen such that
%
\begin{equation}
	\begin{split}
	& 1/2 < \sigma < 1,
	\\
	& \sigma + 1 \le \rho \le s 
	\end{split}
\end{equation}
yields
 \begin{equation*}
	\begin{split}
		\frac{1}{2}\frac{d}{dt} \|v\|_{H^\sigma(\ci)}^2
		& \le
		C'' (\|u_n + u^\ee_n \|_{H^{\rho}(\ci)} +
		\|\p_x(u_n + u^\ee_n) \|_{H^\sigma(\ci)})
		\cdot \|v\|_{H^\sigma(\ci)}^2.
	\end{split}
\end{equation*}
\medskip
It follows that 
\begin{equation}
	\begin{split}
		\frac{1}{2}\frac{d}{dt} \|v\|_{H^{\sigma}(\ci)}^2
		& \le
		C'' \cdot \|u_n
		+ u^\ee_n\|_{H^{s}(\ci)}\cdot \|v\|_{H^{\sigma}(\ci)}^2.
		\label{a10qx}
	\end{split}
\end{equation}
Applying the triangle inequality and
part (iii) of Theorem \ref{thm:HR_existence_continous_dependence} (proved in
Section \ref{existence})
to \eqref{a10qx} yields
%
\begin{equation}
	\begin{split}
		\label{a11qx}
		\frac{1}{2}\frac{d}{dt} \|v\|_{H^{\sigma}(\ci)}^2
		& \le
		C' (\|u_n(0)\|_{H^{s}(\ci)}
		+ \|u^\ee_n(0)\|_{H^{s}(\ci)})\cdot \|v\|_{H^{\sigma}(\ci)}^2
		\\
		& = C' (\|u_{0,n}\|_{H^{s}(\ci)}
		+ \|J_\ee u_{0,n}\|_{H^{s}(\ci)})\cdot \|v\|_{H^{\sigma}(\ci)}^2.
	\end{split}
\end{equation}
Note that the family $\left\{ u_{0,n} \right\}_n$ is uniformly bounded in
$H^s(\ci)$. Hence, applying Proposition \ref{lem3r} to \eqref{a11qx} we obtain the critical estimate 
\begin{equation}
	\begin{split}
		\label{a12qx}
		\frac{1}{2}\frac{d}{dt} \|v\|_{H^{\sigma}(\ci)}^2
		& \le
	C \|v\|_{H^{\sigma}(\ci)}^2
\end{split}
\end{equation}
with $C = C(R)$, where $R$ is defined as in \eqref{r-def}. Note that $C$
does not depend on $n$ or $\ee$. Differentiating
the left hand side of \eqref{a12qx} and simplifying, we obtain
\begin{equation}
	\begin{split}
		\frac{d}{dt}\|v\|_{H^{\sigma}(\ci)} \le C \|v\|_{H^{\sigma}(\ci)}.
		\label{a100qx}
	\end{split}
\end{equation}
Let $y(t) = \|v\|_{H^{\sigma}(\ci)}$. Then \eqref{a100qx} gives
\begin{equation*}
	\begin{split}
		\frac{1}{y(t)}\frac{dy}{dt} \le C.
	\end{split}
\end{equation*}
Hence,
\begin{equation*}
	\begin{split}
		\int_0^t \frac{1}{y(\tau)} \frac{dy}{d \tau}
		\le \int_0^t C \ d \tau, \qquad |t| \le T
	\end{split}
\end{equation*}
from which we obtain
\begin{equation}
	\begin{split}
		\ln |y(t) | - \ln |y(0)| \le C t.
		\label{a101qx}
	\end{split}
\end{equation}
Simplifying \eqref{a101qx}, we have
\begin{equation*}
	\begin{split}
		\ln \left |\frac{y(t)}{y(0)} \right | \le C t
	\end{split}
\end{equation*}
which yields the estimate
\begin{equation*}
	\begin{split}
		y(t) \le y(0) e^{C t}, \qquad |t| \le T.
	\end{split}
\end{equation*}
Substituting back in $\|v\|_{H^{\sigma}(\ci)}$ for $y$, we get
\begin{equation}
	\label{vsig-est}
	\begin{split}
		\|v\|_{H^{\sigma}(\ci)}
		& \le e^{C t}\|v(0)\|_{H^{\sigma}(\ci)}
		\\
		& = e^{C t}\|u_n(0) - u^\ee_n(0) \|_{H^{\sigma}(\ci)}
		\\
		& = e^{C t}\|u_{0,n} - J_\ee u_{0,n}\|_{H^{\sigma}(\ci)}.
	\end{split}
\end{equation}
Applying Proposition \ref{lem4r} to \eqref{vsig-est}, we obtain 
\begin{equation}
	\label{almost}
	\begin{split}
		\|v\|_{H^\sigma (\ci)} \le e^{Ct} \|u_{0,n}\|_{H^\sigma(\ci)} \cdot
		o(\ee^{s-\sigma})
	\end{split}
\end{equation}
and since $\|u_{0,n}\|_{H^s(\ci)} \le R$ for all $n \in \mathbb{N}$, where
$R$ is defined as in \eqref{r-def}, we conclude from estimate \eqref{almost} that
\begin{equation*}
	\begin{split}
		\|v\|_{H^\sigma(\ci)} \le C(R) \cdot o(\ee^{s-\sigma})
	\end{split}
\end{equation*}
completing the proof. $\quad \Box$
We are now prepared to interpolate. Recall \eqref{a15qu}. Applying Remark
\ref{lem2r}, Remark \ref{lem5r'}, and
Proposition \ref{prop:180} gives
\begin{equation*}
	\begin{split}
		& \|u^\ee_n \|_{H^{s+1}(\ci)} \|v\|_{H^{s-1}(\ci)} \|v\|_{H^s
		(\ci)}
		\\
		&\le C'' \ee^{-1} \cdot \|v\|_{H^\sigma(\ci)}^{1/(s-\sigma)}
		\|v\|_{H^s(\ci)}^{2 - 1/(s- \sigma)}
		\\
		& \le C'' \ee^{-1} \cdot \Big [C' \cdot o(\ee^{s- \sigma})\Big ]^{1/(s-\sigma)}
		\cdot \|v\|_{H^s(\ci)}^{2- 1/(s-\sigma)}
	\end{split}
\end{equation*}
from which we obtain
\begin{equation}
	\begin{split}
		\label{a200qx}
		\|u^\ee_n\|_{H^{s+1}(\ci)} \|v\|_{H^{s-1}(\ci)} \|v \|_{H^s(\ci)}
		& \le  C \cdot o(1) \cdot \|v\|_{H^s(\ci)}^{2- 1/(s-\sigma)}.
	\end{split}
\end{equation}
where $C=C(R)$ does not depend on $\ee$ or $n$. We wish to control the growth of the right hand side of \eqref{a200qx}.
First, note that the triangle inequality, part (iii) of Theorem
\ref{thm:HR_existence_continous_dependence}, and Proposition \ref{lem3r} imply
\begin{equation}
	\begin{split}
		\|v\|_{H^s(\ci)} & = \|u_n - u^\ee_n \|_{H^s(\ci)} 
		\\
		& \le \|u_n \|_{H^s(\ci)} + \|u^\ee_n\|_{H^s(\ci)}
		\\
		& \le 2\left[ \|u_{0,n} \|_{H^s(\ci)} + \|J_\ee u_{0,n}
		\|_{H^s(\ci)} \right]
		\\
		& \le 4 \|u_{0,n} \|_{H^s(\ci)}, \qquad |t| \le T
		\label{agrowth_v}
	\end{split}
\end{equation}
and since $\{u_{0,n}\}_n$ belongs to a bounded subset of
$H^s(\ci)$, we see from \eqref{agrowth_v} that $\|v \|_{H^s(\ci)}$ is
uniformly bounded in $n$ \emph{and} $\ee$.  Secondly, by Remark \ref{lem5r'}, we have 
\begin{equation}
	\label{au-ee-bound}
	\|u^\ee_n \|_{H^s(\ci)} \le C(R), \ \ \text{for all} \ \ 0 < \ee <<1, \ n \in
	\mathbb{N}.
\end{equation}
Applying \eqref{a200qx}, \eqref{agrowth_v}, and \eqref{au-ee-bound}
to \eqref{a15qu}, it follows that 
\begin{equation*}
	\label{lim-est-in}
		\begin{split}
			\lim_{n \to \infty }
			\frac{1}{2}\frac{d}{dt}\|v\|_{H^{s}(\ci)}^2
			& \le
			C \lim_{n \to \infty} \Big [\|v\|_{H^s(\ci)}^3 +
			\|v\|_{H^s(\ci)}^2 + o(1)\Big ]
		\end{split}
	\end{equation*}
	for every $0 < \ee <<1$, where $C = C(\|u_0\|_{H^s(\ci)}, \ R)$.
	Therefore
	\begin{equation}
		\begin{split}
			\lim_{\substack{n \to \infty \\ \ee \to 0} }
			\frac{1}{2}\frac{d}{dt}\|v\|_{H^{s}(\ci)}^2
			& \le C
			\lim_{\substack{n \to \infty \\ \ee \to 0}}
			\Big [\|v\|_{H^s(\ci)}^3 + 
			\|v\|_{H^s(\ci)}^2 \Big ].
			\label{a15qx}
		\end{split}
	\end{equation}
	We differentiate the left hand side of \eqref{a15qx} and obtain
\begin{equation*}
	\begin{split}
		\lim_{\substack{n \to \infty \\ \ee \to 0}}\frac{d}{dt}
		\|v\|_{H^s(\ci)} \le C
		\lim_{\substack{n \to \infty \\ \ee \to 0}} \left [\|v\|_{H^s(\ci)}^2 +
		\|v\|_{H^s(\ci)} \right ].
	\end{split}
\end{equation*}
Letting $y = \|v\|_{H^s(\ci)}$ and rearranging gives
\begin{equation*}
	\begin{split}
		\lim_{\substack{n \to \infty \\ \ee \to 0} } \ \frac{1}{y(y+1)} \frac{dy}{dt}
		\le	C
	\end{split}
\end{equation*}
which can be rewritten as
\begin{equation*}
	\begin{split}
		\lim_{\substack{n \to \infty \\ \ee \to 0} }
		\left( \frac{1}{y} - \frac{1}{y+1} \right)\frac{dy}{dt} \le C 
	\end{split}
\end{equation*}
implying
\begin{equation}
	\label{aest-int}
	\begin{split}
		\lim_{\substack{n \to \infty \\ \ee \to 0} } \left [
\int_0^t \frac{1}{y} \frac{dy}{d \tau} \ d \tau
		- \int_0^t \frac{1}{y+1} \frac{dy}{d \tau} \ d \tau \right ]
		\le \int_0^t C \ d \tau, \quad |t| \le T.
	\end{split}
\end{equation}
Hence \eqref{aest-int} gives 
\begin{equation}
	\begin{split}
	\lim_{\substack{n \to \infty \\ \ee \to 0} }	\left [ \ln \left | \frac{y(t)}{y(0)}
	\cdot \frac{y(0) + 1}{y(t) + 1} \right | \right ] \le C t.
		\label{20b}
	\end{split}
\end{equation}
Exponentiating both sides of \eqref{20b}, and noting that $f(x) = e^x$
is a continuous function on $\rr$, we must have
\begin{equation*}
	\begin{split}
		\lim_{\substack{n \to \infty \\ \ee \to 0} }	
		\left |
		\frac{y(t)}{y(0)} \cdot \frac{y(0) + 1}{y(t) + 1} \right | \le e^{C t}.
	\end{split}
\end{equation*}
Recalling that $y(t) = \|v(t)\|_{H^s(\ci)} \ge 0$, we obtain
\begin{equation*}
	\begin{split}
		\lim_{\substack{n \to \infty \\ \ee \to 0} }	
		\frac{y(t)}{y(0)} \cdot \frac{y(0) + 1}{y(t) + 1} \le e^{C t}.
	\end{split}
\end{equation*}
Rearranging, it follows that 
\begin{equation*}
	\begin{split}
		\lim_{\substack{n \to \infty \\ \ee \to 0}} \frac{y(t)}{y(t) + 1}
		\le \lim_{\substack{n \to \infty \\ \ee \to 0}} \frac{e^{C t} \cdot y(0)}{y(0) + 1} \le
		\lim_{\substack{n \to \infty \\ \ee \to 0}} e^{C t} \cdot y(0).
	\end{split}
\end{equation*}
Substituting back in $\|v(t)\|_{H^s(\ci)}$ for $y(t)$ gives
\begin{equation}
	\begin{split}
		\lim_{\substack{n \to \infty \\ \ee \to 0}}	\frac{\|v(t)\|_{H^s(\ci)}}{\|v(t)\|_{H^s(\ci)} + 1}  \le
		\lim_{\substack{n \to \infty \\ \ee \to 0}} e^{C t} \cdot \|v(0)\|_{H^s(\ci)}.
		\label{a303qx}
	\end{split}
\end{equation}
Since by Proposition \ref{lem4r} 
\begin{equation*}
	\begin{split}
	\lim_{\substack{n \to \infty \\ \ee \to 0} }
	\|v(0)\|_{H^s(\ci)}
	& = \lim_{\substack{n \to \infty \\ \ee \to 0} }
	\|u_{0,n} - J_\ee u_{0,n} \|_{H^s(\ci)} 
	\\
	& \le  \lim_{\substack{n \to \infty \\ \ee \to 0}}
	\left [ \|u_{0,n}\|_{H^s(\ci)} \cdot o(1) \right ]
	\\
	& = \lim_{\ee \to 0} \left [ \|u_0\|_{H^s(\ci)} \cdot o(1) \right ]
	\\
	& = 0
	\end{split}
\end{equation*}
we deduce from \eqref{a303qx} that
\begin{equation*}
	\begin{split}
		\lim_{\substack{n \to \infty \\ \ee \to 0}} \|v(t)\|_{H^s(\ci)} = 0, \qquad |t| \le T
	\end{split}
\end{equation*}
completing the proof of \eqref{enough_to_prove3}. $\quad \Box$
%
%
%
\section{Extending Well-Posedness for HR to the Non-Periodic Case}
\label{sec:defs}
The method will be analogous to that of the periodic case, with two major
modifications. First, we must choose a different mollifier $J_\ee$ in the
proof of continuous dependence. Pick a
function $j(x) \in \mathcal{S}(\rr)$ such that
\begin{equation*}
		\begin{split}
			& 0 \le \widehat{j}(\xi) \le 1,
			\\
			& \widehat{j}(\xi) = 1 \ \ \text{if} \ \ |\xi| \le 1.
		\end{split}
	\end{equation*}
Letting
\begin{equation*}
	\begin{split}
		j_\ee(x) = \frac{1}{\ee} j \left (\frac{x}{\ee} \right )
	\end{split}
\end{equation*}
it can be verified that 
		\begin{equation*}
		\begin{split}
			\widehat{j_\ee}(\xi) = \widehat{j }(\ee \xi), \quad \ee > 0.
		\end{split}
	\end{equation*}
We then define $J_\ee$ to be the ``Friedrichs mollifier''
	\begin{equation*}
		\begin{split}
			J_\ee f(x) = j_\ee * f(x), \quad \ee>0.
		\end{split}
	\end{equation*}
Given this construction, the proofs of Remark \ref{lem5r}, Remark
\ref{lem5r'}, and Proposition \ref{lem4r} for the non-periodic case will be
analogous to those in the periodic case.
Secondly, in the proof of existence, we will have difficulties in arranging
that the solutions $\{u_\ee\}$ to the mollified HR i.v.p. converge in $C(I,
H^{s- \sigma}(\rr))$, $0 < \sigma < 1$ to a candidate solution $u$ of the HR
i.v.p. We will get around this by considering the family $\left\{ \varphi
u_\ee \right\}$ instead.
%
%
%
%
We divide our work into three parts:
\subsection{Existence.}
Mirroring the argument in the periodic case, we see that the bounded
family $\{u_\ee\}$ is compact in the weak* topology of $L^\infty(I,
H^{s}(\rr))$. More precisely, there is a sequence  $\{ u_{\ee_n} \}$
converging weak* to a $ u\in L^{\infty}(I, H^s(\rr))$; that is 
		%
		\begin{equation*}
			\label{hhweak-conv}
			\lim_{n\to \infty} T_{u_{\ee_n}}(\varphi)  =  T_u (\varphi) 
			\; \;		
			\text{ for all } \;\;  \varphi \in L^1(I, H^{s}(\rr))
		\end{equation*}
		where
		\begin{equation}
			T_v(\varphi) = \int_I <v (t), \varphi (t)>_{H^s(\rr)} dt  = \int_I
			 \int_\rr
			 \widehat{v}(\xi, t) \bar{\widehat{\varphi}} (\xi, t) \cdot (1 +
			 \xi^2)^s \ d \xi \; dt.
		\end{equation}
		%
		Similarly, $\left\{ \p_x u_{\ee_n} \right\}$ is compact in the
		weak* topology of $L^\infty(I, H^{s-1}(\rr))$ and converges weak*
		to $\p_x u$. Hence, for any $k \in \mathbb{N}$, we have
		\begin{align}
			\label{base-weak}
				& (u_{\ee_n})^k \xrightarrow{\text{weak*}} u^k \ \
				\text{on} \ \
				L^\infty(I, H^s(\rr)),
				\\
				\label{base-weak-2}
				& (\p_x u_{\ee_n})^k \xrightarrow{\text{weak*}} (\p_x u)^k
				\ \ \text{on} \ \
				L^\infty(I, H^{s-1}(\rr)). 
		\end{align}
		In order to show that $u$ solves the HR i.v.p., it would
		suffice to obtain a stronger convergence for  $u_{\ee_n}$ so that 
		we could take the limit in the mollified HR equation. However,
		this is difficult, and unnecessary. Rather, our approach will be to
		show that for any pseudo-differential operator
		$P \in \Psi^0$ and arbitrary $\vp \in S(\rr)$, $k \in
		\mathbb{N}$, $0< \sigma < 1$, we have
		%
		%
			\begin{align}
			\label{hhstrong-conv}
			& \varphi P [(u_{\ee_n})^k] \longrightarrow \varphi P [u^k]  
			\quad
			\text{ in } \,\,   C(I, H^{s-\sigma}(\rr)), \ \,
			\\
			\label{hhstrong-conv-next}
			& \varphi P [(\p_x u_{\ee_n})^k] \longrightarrow \varphi P
			[(\p_x u)^k]  
			\quad
			\text{ in } \,\,   C(I, H^{s-\sigma -1}(\rr)), \ \ 
		\end{align}
		%
		which will then be applied to a rewritten version of the HR
		i.v.p. Our focus will be on proving \eqref{hhstrong-conv}; since the proof of
		\eqref{hhstrong-conv-next} is similar, we will omit the
		details. First, we will need the following
		interpolation result:
		%%%%%%%%%%%%%%%%%%%%%%%%%%%
		%
		%
		%                 Interpolation Lemma
		%
		%
		%%%%%%%%%%%%%%%%%%%%%%%%%%%
		\begin{lemma}
			\label{hhinterpolation-lem}
			(Interpolation)     Let  $s > \frac{3}{2}$.
			If $v \in C(I, H^s(\rr)) \cap C^1(I, H^{s-1}(\rr))$
			then $v \in C^\sigma (I, H^{s- \sigma}(\rr))$ for  $0 < \sigma < 1$.
		\end{lemma}
		%
		\subsection{Proof} It is analogous to the proof in the periodic case.
		$\quad \Box$
		Fix $k \in \mathbb{N}$. Using Lemma \ref{hhinterpolation-lem}, we
		will show that the family
		\begin{equation*}
			\begin{split}
			 \{\varphi P[(u_\ee)^k]\}_\ee
		\end{split}
	\end{equation*}
		is equicontinuous in $C(I, H^{s-\sigma}(\rr))$ 
		for $0 < \sigma < 1$ and $\varphi = \varphi(x) \in \mathcal{S}(\rr)$.
		We will follow this by proving that
		there exists a sub-family $\{\varphi P[(u_{\ee_n}(t))^k]\}_n$
		that is precompact in $H^{s-\sigma}(\rr)$ for $\sigma > 0$. 
		These two facts, in conjunction with Ascoli's Theorem, will
		yield
		\begin{equation*}
			\label{hhstrong-conv2}
			\varphi P[(u_\ee)^k] \to \tilde{u}
			\; \; \text{in} \; \; C(I,H^{s-\sigma}(\rr))
		\end{equation*}
		for $0 < \sigma < 1$.
		We will then show that $\tilde{u} = \varphi P[u^k]$, from which it will
		follow that
		\begin{equation*}
			\label{hhphiplus}
			\begin{split}
				\varphi P[(u_\ee)^k] \to \varphi P[u^k]
				\; \; \text{in} \; \; C(I,H^{s-\sigma}(\rr)).
			\end{split}
		\end{equation*}
		%%%%%%%%%%%%%%%%%%%%%%
		%
		%
		%       Equicontinuity
		%
		%
		%%%%%%%%%%%%%%%%%%%%%%
		%
		\subsection{  Equicontinuity of $\{ \varphi P [(u_\ee)^k]\}_\ee$  in $C(I,
		H^{s-\sigma}(\rr)$}).
		%
		%
		Since $\varphi \in \mathcal{S}(\rr)$, the map $u \mapsto \vp u$
		is a bounded linear function on $H^s(\rr)$, for arbitrary $s \in
		\rr$, where  
		\begin{equation}
			\begin{split}
				\|\varphi u\|_{H^s(\rr)} \le C(s, \varphi)
				\|u\|_{H^s(\rr)}, \quad \forall s\in \rr.
				\label{hhschwartz-estimate}
			\end{split}
		\end{equation}
		Furthermore, $$P: H^s(\rr) \to H^s(\rr)$$ is bounded and linear,
		with 
		\begin{equation}
			\label{operator-normaa}
			\|P\|_{L(H^s(\rr), H^s(\rr))} \le 1.
		\end{equation}
		Hence, the map 
		\begin{equation}
			\label{the-map}
			\begin{split}
			& T: H^s(\rr) \to H^s(\rr),
			\\
			& T(u) = \vp P u 
		\end{split}
	\end{equation}
	is bounded and linear, with 
	\begin{equation}
		\begin{split}
			\|T\|_{L(H^s(\rr), H^s(\rr))} \le C(s, \vp).
			\label{op-norm-product}
		\end{split}
	\end{equation}
	Therefore, applying Lemma
		\ref{hhinterpolation-lem} gives 
		%
		\begin{equation*}
			\begin{split}
			\label{hhequic-1}
			& \sup_{t \neq t'} \frac {\| \varphi P [(u_\ee(t))^k] - \varphi
			P [(u_\ee(t'))^k] \|_{H^{s -
			\sigma  }(\rr)}}{|t - t'|}
			\\
			& \le \sup_{t \neq t'}  \frac { \|\vp P \|_{L(H^{s-\sigma}(\rr),
			H^{s-\sigma}(\rr))} \cdot \|   [u_\ee(t)]^k  - 
			[u_\ee(t')]^k \|_{H^{s -
			\sigma }(\rr)}}{|t - t'|}
			\\
			& \le C(s, \vp) \cdot \sup_{t \neq t'}  \frac { \|   [u_\ee(t)]^k  - 
			[u_\ee(t')]^k \|_{H^{s -
			\sigma }(\rr)}}{|t - t'|}
			\\
			&< c
		\end{split}
		\end{equation*}
		%
		or
		%
		\begin{equation*}
			\label{hhequic-2}
			\|\varphi P [(u_\ee(t))^k] - \varphi
			P [(u_\ee(t'))^k \|_{H^{s - \sigma }(\rr)}< c|t -
			t'|, 
			\text{ for all }  \,\,  t, t'\in I,
		\end{equation*}
		%
		which shows that  the family  $\{\varphi P [(u_\ee)^k]\}_\ee$ is
		equicontinuous in $C(I, H^{s-\sigma }(\rr))$.  $\quad \Box$
		%
		%
		%%%%%%%%%%%%%%%%%%%%%%
		%
		%
		%      PreCompactness
		%
		%
		%%%%%%%%%%%%%%%%%%%%%%%%%%
		%
		%
		%
		%
		%		
		\subsection{Precompactness of $\{\varphi P [(u_\ee(t))^k]\}_\ee$ in
		$H^{s-\sigma  }(\rr)$}
		Applying the algebra property of Sobolev
		Spaces, and recalling \eqref{the-map}-\eqref{op-norm-product}, we have
		\begin{equation}
			\begin{split}
			\label{hhcompact-1}
			 \|\varphi P [(u_\ee(t))^k]\|_{H^{s}(\rr)}
			& \le  C(s, \vp) \cdot \|[u_\ee(t)]^k\|_{H^{s}(\rr)}
			\\
			& \le C(s, \vp) \cdot \|u_\ee(t)\|^k_{H^{s}(\rr)}.
			\end{split}
		\end{equation}
		%
		Letting $|t| \le T$, we now apply Lemma \ref{hr_wp} to
		\eqref{hhcompact-1} to obtain
		\begin{equation*}
			\begin{split}
			\|\varphi P [(u_\ee(t))^k]\|_{H^{s}(\rr)}
			\le 2^k C(s, \vp) \cdot  \|u_0 \|^k_{H^s(\rr)} < \infty.
			\end{split}
		\end{equation*}
		Therefore, by Reillich's Theorem, the family $\left\{
		\varphi P [(u_\ee(t))^k] \right\}_\ee$ is
		precompact in $H^{s- \sigma }(\rr)$ for all $\sigma > 0$ and $|t| \le T$. $\quad
		\Box$ 
		Hence, compiling our previous results on equicontinuity and precompactness
		and applying Ascoli's Theorem, we
		conclude that we can find $\tilde{u}$ and a subfamily 
		\\ $\left\{
		\varphi P [(u_{\ee_n})^k]
		\right\}_n$ such that
		\begin{equation}
			\label{hhstrong-conv-of-u_ep}
			\varphi P [(u_{\ee_n})^k] \to \tilde{u}
			\; \; \text{in} \; \; C(I, H^{s-\sigma}(\rr)).
		\end{equation}
		%
		%
		We would now like to find out what $\tilde{u}$ is:
		%
		%
		%
		\begin{lemma}
			\label{hhlem:crit-conv}
			For arbitrary $k \in \mathbb{N}$,
			\begin{equation}
				\begin{split}
					\varphi P [(u_{\ee_n})^k] \xrightarrow{weak^*}
					\varphi P [u^k] \ \ \text{on} \ \ L^\infty(I,
					H^{s-\sigma}(\rr)).
					\label{hhcrit-conv-est}
				\end{split}
			\end{equation}
		\end{lemma}
		\subsection{ Proof.} 
		Fix $k \in \mathbb{N}$ and recall that the operators 
		\begin{equation*}
			\begin{split}
			 & T_\varphi: H^s(\rr) \to H^s(\rr)\\
			 & T_\varphi u = \varphi u
		\end{split}
	\end{equation*}
and 
\begin{equation*}
	\begin{split}
		P:H^s(\rr) \to H^s(\rr)
	\end{split}
\end{equation*}
	are continuous; therefore 
	\begin{equation*}
		\begin{split}
			T_\vp P: H^s(\rr) \to H^s(\rr)
		\end{split}
	\end{equation*}
	continuously. Hence, its adjoint  $(T_\varphi P)^*$
	exists and
		\begin{equation*}
			(T_\varphi P)^*: H^s(\rr) \to H^s(\rr) 
		\end{equation*}
		continuously. Therefore, applying \eqref{base-weak}, we conclude that
		\begin{equation}
			\label{widpseudo}
			\begin{split}
				& \int_I <\varphi P[u^k] - \varphi
				P [(u_{\ee_n})^k],\  f>_{H^{s-\sigma }(\rr)} dt
				\\
				&= \int_I <u^k - 
				 (u_{\ee_n})^k, \ (T_\vp P)^* f>_{H^{s-\sigma }(\rr)} \to 0
			\end{split}
		\end{equation}
		completing the proof. $\quad \Box$
		%
		%
		Now, recalling \eqref{hhstrong-conv-of-u_ep} and applying Lemma
		\ref{hhlem:crit-conv}, we obtain
			\begin{equation}
			\begin{split}
				\vp P [(u_{\ee_n})^k] \to \vp P [u^k] \ \ \text{in}  \ \ C(I,
				H^{s-\sigma}(\rr))
				\label{hhvp_u_ep_conv}
			\end{split}
		\end{equation}
		for arbitrary $k \in \mathbb{N}$.  Using precisely the same
		strategy we used to prove \eqref{hhvp_u_ep_conv} (applied now to
		the family $\{ \vp P [(\p_x u_{\ee})^k] \}_\ee$), one can also show
	\begin{equation}
			\begin{split}
			\vp P [ (\p_x u_{\ee_n})^k] \to \vp P [(\p_x u)^k] \ \ \text{in}  \ \ C(I,
				H^{s-\sigma -1 }(\rr)).
			\end{split}
		\end{equation}
		We summarize our result below:
		\begin{theorem}
		\label{hhthm:crit1}
		Let $P \in \Psi^0$ be a pseudo-differential operator. Then for
		arbitrary $k \in \mathbb{N}$, 
			\begin{equation}
			\begin{split}
				& \vp P [(u_{\ee_n})^k] \to \vp P [u^k] \ \ \text{in}  \ \ C(I,
				H^{s-\sigma }(\rr)),
				\\
				& 
				\vp P [(\p_x u_{\ee_n})^k] \to \vp P [(\p_x u)^k] \ \
				\text{in}  \ \ C(I,
				H^{s-\sigma -1}(\rr)).
				\label{hhdx_vp_u_ep_conv}
			\end{split}
		\end{equation}
	\end{theorem}
		\subsection{ Verifying that the weak* limit $u$ solves the HR equation.} 
		We recall the mollified HR i.v.p
		\begin{align}
			& \p_t u_{\ee_n}  = -\gamma(J_{\ee_n} u_{\ee_n} \cdot \p_x
			J_{\ee_n} u_{\ee_n})
			\label{hh1gr}
			\\
			& u(x,0) = u_0(x).
			\label{hh2gr}
		\end{align}
		Multiplying both sides of \eqref{hh1gr} by $\varphi$ and rewriting,
		we obtain
		\begin{equation}
			\label{hh3}
			\begin{split}
				\p_t(u_{\ee_n} \varphi) = -\gamma \vp (J_{\ee_n} u_{\ee_n} \cdot
				J_{\ee_n} \p_x u_{\ee_n}).
			\end{split}
		\end{equation}
		The following lemma will play a crucial role in our proof of the
		existence of a solution to the HR i.v.p.
		\begin{lemma}
			\label{hhlem:cc}
			For $\vp \in \mathcal{S}(\rr)$ such that
			$\vp^\frac{1}{2} \in \mathcal{S}(\rr)$, we have
			\begin{equation}
				\begin{split}
					\label{hhburgers_and_nonlocal_conv}
				& \vp (J_{\varepsilon_n} u_{\varepsilon_n} 
				\cdot J_{\varepsilon_n}\partial_x u_{\varepsilon_n}) 
				\to \vp u \partial_x u \; \; 
				\text{in} \; \;
				C(I, H^{s-\sigma-1}(\rr)). 
			\end{split}
			\end{equation}
		\end{lemma}
		%
		\subsection{ Proof.} We will need a couple of propositions:
		\begin{proposition}
			For arbitrary $\vp \in \mathcal{S}(\rr)$
			\label{hhprop:1aa}
			\begin{equation}
				\begin{split}
					\vp J_{\ee_n} u_{\ee_n} \to \vp u \ \ \text{in} \ \
					C(I, H^{s-\sigma}(\rr)).
					\label{hh}
				\end{split}
			\end{equation}
		\end{proposition}
			\subsection{ Proof.} Note that
			\begin{equation}
				\begin{split}
					& \|\vp u - \vp J_{\ee_n} u_{\ee_n}
					\|_{C(I, H^{s-\sigma}(\rr))}
					\\
					&= \|\vp u - \vp J_{\ee_n} u_{\ee_n} \pm \vp
					u_{\ee_n} \|_{C(I, H^{s-\sigma}(\rr))}
					\\
					& = \|\vp u - \vp u_{\ee_n}
					\|_{C(I, H^{s-\sigma}(\rr))} + \|\vp (I - J_{\ee_n})
					u_{\ee_n} \|_{C(I, H^{s-\sigma}(\rr))}.
					\label{hh1bb}
				\end{split}
			\end{equation}
			Applying \eqref{hhschwartz-estimate} and the estimates
			\begin{equation*}
				\begin{split}
					& \|I-J_{\ee_n} \|_{L(H^{s-\sigma}(\rr), H^{s -
					\sigma}(\rr))} = o(1),
					\\
					& \|u_{\ee_n}\|_{H^{s-\sigma}(\rr)} \le 2
					\|u_0\|_{H^{s-\sigma}(\rr)}
				\end{split}
			\end{equation*}
			to \eqref{hh1bb} gives
			\begin{equation}
				\label{hh2bb}
				\begin{split}
					\|\vp u - \vp J_{\ee_n} u_{\ee_n}\|_{H^{s-\sigma}(\rr)}
					\le \left( \|\vp u - \vp u_{\ee_n}
					\|_{C(I, H^{s-\sigma}(\rr))} + C(s, \vp) \cdot o(1) \cdot \|u_0
					\|_{H^{s-\sigma}(\rr)} \right).
				\end{split}
			\end{equation}
			Letting $\ee \to 0$ in \eqref{hh2bb} and applying Theorem
			\ref{hhthm:crit1} completes the proof. $\quad \Box$
			%
			%
			\begin{proposition}
				\label{hhprop:dd}
				For arbitrary $ \vp \in \mathcal{S}(\rr)$,
				\begin{equation}
					\begin{split}
						\vp J_{\ee_n} \p_x u_{\ee_n} \to \vp u \ \
						\text{in} \ \ C(I, H^{s-\sigma-1}(\rr)).
						\label{hh0dd}
					\end{split}
				\end{equation}
			\end{proposition}
			\subsection{ Proof.} The result follows from Theorem \ref{hhthm:crit1}.
			The proof is nearly identical to that of
			Proposition \ref{hhprop:1aa}, with $s-1$ substituted for $s$
			and $\p_x u_{\ee_n}$ substituted for $u_{\ee_n}$. $\quad \Box$
			%
			%
			We now have enough tools to prove Lemma \ref{hhlem:cc}. Restrict the
			choice of $\vp$ such that $\vp^\frac{1}{2} \in S(\rr)$
			(Such Schwartz functions exist; as an example, take the square
			of the Gaussian). Using this fact, and applying Proposition
			\ref{hhprop:1aa} and Proposition \ref{hhprop:dd}, we conclude that
			\begin{equation*}
				\begin{split}
					\vp J_{\ee_n} u_{\ee_n} \p_x J_{\ee_n} u_{\ee_n} 
					& = \vp^\frac{1}{2} J_{\ee_n} u_{\ee_n} \cdot
					\vp^\frac{1}{2} \p_x J_{\ee_n} u_{\ee_n}
					\\
					& \to \vp^\frac{1}{2} u \cdot \vp^\frac{1}{2} \p_x u = \vp
					u \p_x u
				\end{split}
			\end{equation*}
			completing the proof of Lemma \ref{hhlem:cc}. $\quad \Box$
%
%
%
%
By Theorem \ref{hhthm:crit1} it follows immediately that
		\begin{equation}
			\begin{split}
				& \vp \p_x(1- \p_x^2)^{-1} \left( \frac{3-\gamma}{2}
				(u_{\ee_n})^2
				 + \frac{\gamma}{2} (\p_x u_{\ee_n})^2 \right )
				 \\
				 & \to
				 \vp \p_x(1- \p_x^2)^{-1} \left( \frac{3-\gamma}{2} u^2
				 + \frac{\gamma}{2} (\p_x u)^2 \right ) \ \
				 \text{in} \ \ C(I, H^{s-\sigma-1}(\rr)).
				\label{llnon-local-convergence}
			\end{split}
		\end{equation}
		Combining \eqref{hhburgers_and_nonlocal_conv} and
		\eqref{llnon-local-convergence}, and applying the Sobolev Imbedding
		Theorem, we deduce 
		\begin{equation}
			\begin{split}
				& -\gamma \vp (J_{\ee_n} u_{\ee_n} \cdot J_{\ee_n} \p_x
				u_{\ee_n}) -
				\vp \p_x(1- \p_x^2)^{-1} \left( \frac{3-\gamma}{2}
				(u_{\ee_n})^2
				 + \frac{\gamma}{2} (\p_x u_{\ee_n})^2 \right )
				 \\
				 \to & -\gamma \vp u \p_x u -
				 \vp \p_x(1- \p_x^2)^{-1} \left( \frac{3-\gamma}{2} u^2
				 + \frac{\gamma}{2} (\p_x u)^2 \right ) \ \
				 \text{in} \ \ C(I, C(\rr)).
				\label{llloc-non-loc-tog}
			\end{split}
		\end{equation}
		%
		Next, we note that the convergence  
		%
		\begin{equation}
			\label{hhweak-conv-2}
			T_{\vp u_{\ee_n}}(f)  \longrightarrow  T_{\vp u} (f) \;
			\text{ for all } \;  f \in L^1(I, H^{-s}(\rr))
		\end{equation}
		%
		can be restated as 
		%
		\begin{equation}
			\vp u_{\ee_n}  \longrightarrow  \vp u
			\quad
			\text{ in }  \,\,
			\mathcal{D}'(I\times \rr).
		\end{equation}
		%
		This implies 
		%
		\begin{equation}
			\label{hhdistib-conv-2}
			\p_t(\vp u_{\ee_n})  \longrightarrow  \p_t (\vp u)
			\quad
			\text{ in }  \,\, \mathcal{D}'(I\times \rr).
		\end{equation}
		%
		Since for all $n$ we have 
		%
		\begin{equation}
			\begin{split}
			 \p_t (\vp u_{\ee_n})
			 = & -\gamma \vp
			(J_{\varepsilon_n} u_{\varepsilon_n}  \cdot
			J_{\varepsilon_n}\partial_x u_{\varepsilon_n})
			\\
			& -
			\vp \p_x(1- \p_x^2)^{-1} \left( \frac{3-\gamma}{2} (u_{\ee_n})^2
			 + \frac{\gamma}{2} (\p_x u_{\ee_n})^2 \right )
		 \end{split}
		\end{equation}
		%
		it follows from \eqref{hhdistib-conv-2} and the uniqueness of the
		limit in \eqref{llloc-non-loc-tog} that
		\begin{equation}
			\begin{split}
			 \p_t (\vp u)
			 = & -\gamma \vp
			u \p_x u - \vp \p_x(1- \p_x^2)^{-1} \left( \frac{3-\gamma}{2} u^2
			 + \frac{\gamma}{2} (\p_x u)^2 \right )
			\label{hhadone}
			\end{split}
		\end{equation}
		Further restricting $\vp \in \mathcal{S}(\rr)$ to be nonzero in
		$\rr$, we
		can divide both sides of \eqref{hhadone} by $\vp$ to obtain
		\begin{equation}
			\label{hh2yy}
			\begin{split}
			 \p_t  u
			 = & -\gamma
			u \p_x u - \p_x(1- \p_x^2)^{-1} \left( \frac{3-\gamma}{2} u^2
			 + \frac{\gamma}{2} (\p_x u)^2 \right ).
			\end{split}
		\end{equation}
		Thus we have constructed a solution $u \in L^\infty(I, H^s(\rr))$
		to the HR i.v.p. 
\subsection{Uniqueness.} The proof is analogous to that in the periodic case.
\subsection{Continuous Dependence.} The proof is analogous to the proof in
the periodic case, with one important caveat. Recall the introduction to Section
\ref{sec:defs} in the appendix; specifically, how we defined the operator
$J_\ee$. By construction, the proofs of Remark \ref{lem5r}, Remark
\ref{lem5r'}, and Proposition \ref{lem4r} for the non-periodic case will be
analogous to those in the periodic case. Hence, how we
construct the mollifier $J_\ee$ plays a critical role in the proofs of
well-posedness for the HR i.v.p. in both the periodic and non-periodic cases. %
%

 %%
\documentclass[12pt,reqno]{amsart}
\usepackage{amssymb}
\usepackage{appendix}
\usepackage{showkeys}
\usepackage{tikz}
\usepackage[showonlyrefs=true]{mathtools} %amsmath extension package
\usepackage{cancel}  %for cancelling terms explicity on pdf
\usepackage[normalem]{ulem} %for striking out non-math text
\usepackage{yhmath}   %makes fourier transform look nicer, among other things
\usepackage{framed}  %for framing remarks, theorems, etc.
\usepackage{enumerate} %to change enumerate symbols
\usepackage[margin=2.5cm]{geometry}  %page layout
%\setcounter{tocdepth}{1} %must come before secnumdepth--else, pain
%\setcounter{secnumdepth}{1} %number only sections, not subsections
%\usepackage[pdftex]{graphicx} %for importing pictures into latex--pdf compilation
\numberwithin{equation}{section}  %eliminate need for keeping track of counters
\numberwithin{figure}{section}
\setlength{\parindent}{0in} %no indentation of paragraphs after section title
\renewcommand{\baselinestretch}{1.1} %increases vert spacing of text
%
\usepackage{hyperref}
\hypersetup{colorlinks=true,
linkcolor=blue,
citecolor=blue,
urlcolor=blue,
}
%
\DeclareMathOperator{\sgn}{sgn}
%
\newcommand{\ds}{\displaystyle}
\newcommand{\ts}{\textstyle}
\newcommand{\nin}{\noindent}
\newcommand{\rr}{\mathbb{R}}
\newcommand{\nn}{\mathbb{N}}
\newcommand{\zz}{\mathbb{Z}}
\newcommand{\cc}{\mathbb{C}}
\newcommand{\ci}{\mathbb{T}}
\newcommand{\zzdot}{\dot{\zz}}
\newcommand{\wh}{\widehat}
\newcommand{\p}{\partial}
\newcommand{\ee}{\varepsilon}
\newcommand{\vp}{\varphi}
\newcommand{\wt}{\widetilde}
%
%
%
%
\newtheorem{theorem}{Theorem}[section]
\newtheorem{lemma}[theorem]{Lemma}
\newtheorem{corollary}[theorem]{Corollary}
\newtheorem{claim}[theorem]{Claim}
\newtheorem{prop}[theorem]{Proposition}
\newtheorem{proposition}[theorem]{Proposition}
\newtheorem{no}[theorem]{Notation}
\newtheorem{definition}[theorem]{Definition}
\newtheorem{remark}[theorem]{Remark}
\newtheorem{examp}{Example}[section]
\newtheorem {exercise}[theorem]{Exercise}
%
%\makeatletter \renewenvironment{proof}[1][\proofname] {\par\pushQED{\qed}\normalfont\topsep6\p@\@plus6\p@\relax\trivlist\item[\hskip\labelsep\bfseries#1\@addpunct{.}]\ignorespaces}{\popQED\endtrivlist\@endpefalse} \makeatother%
%makes proof environment bold instead of italic
\newcommand{\uol}{u^\omega_\lambda}
\newcommand{\lbar}{\bar{l}}
\renewcommand{\l}{\lambda}
\newcommand{\R}{\mathbb R}
\newcommand{\RR}{\mathcal R}
\newcommand{\al}{\alpha}
\newcommand{\ve}{q}
\newcommand{\tg}{{tan}}
\newcommand{\m}{q}
\newcommand{\N}{N}
\newcommand{\ta}{{\tilde{a}}}
\newcommand{\tb}{{\tilde{b}}}
\newcommand{\tc}{{\tilde{c}}}
\newcommand{\tS}{{\tilde S}}
\newcommand{\tP}{{\tilde P}}
\newcommand{\tu}{{\tilde{u}}}
\newcommand{\tw}{{\tilde{w}}}
\newcommand{\tA}{{\tilde{A}}}
\newcommand{\tX}{{\tilde{X}}}
\newcommand{\tphi}{{\tilde{\phi}}}
\synctex=1
\begin{document}
\title[H\"older Continuity of the Data to Solution Map for HR]{H\"older Continuity of the Data to Solution Map for HR in the
Weak Topology}
\author{David Karapetyan}
\address{Department of Mathematics  \\
    University  of Notre Dame\\
        Notre Dame, IN 46556 }
        \date{\today}
        %
        \maketitle
        %
        %
        %
        %
        %
        %        
        %
        \section{Introduction}
%
%
%
We consider the hyperelastic rod (HR) Cauchy problem
\begin{gather}
 \p_t u =  -\gamma u \p_x u -
 \p_{x} D^{-2} \left[ \frac{3-\gamma}{2}u^2 +
\frac{\gamma}{2} \left( \p_x u \right)^2
\right],
\label{hyperelastic-rod-equation}
\\
 u(x,0) = u_0(x), \; \; x \in \rr, \; \; t \in \rr
\label{init-cond}
\end{gather}
%
%
where 
\begin{equation*}
	D^{m} = (1 - \p_x^2)^{m/2}, \quad m \in \rr
\end{equation*}
and  $\gamma$  is a  nonzero constant. The HR equation was first
derived by Dai in \cite{Dai_1998_Model-equations} as a one-dimensional 
model for finite-length and
small-amplitude axial deformation waves in thin cylindrical
rods composed of a compressible Mooney-Rivlin
material. The derivation relied upon a reductive perturbation technique, 
and took into account the nonlinear dispersion of pulses propagating 
along a rod. It was assumed that each cross-section of the rod is 
subject to a stretching and rotation in space. The solution $u(x,t)$ to the 
HR equation represents the radial stretch relative
to a pre-stressed state, while $\gamma$ is a fixed constant depending upon 
the pre-stress and the material used in
the rod, with values ranging from $- 29.4760$ to $3.4174$.
%
\\
\\
The well-posedness of the HR equation has been studied by several authors. 
In Yin \cite{Yin_2003_On-the-Cauchy-p} and Zhou 
\cite{Zhou_2005_Local-well-pose}, a proof of local well-posedness in Sobolev 
spaces $H^s$,  $s > 3/2$, is described  on the line and the circle, respectively. 
Their approach is to rewrite the HR equation   
in its non-local form, and then to verify the conditions needed to apply 
Kato's semi-group theory \cite{Kato_1975_Quasi-linear-eq}. Using an alternative,
Galerkin type method outlined in Taylor \cite{Taylor_1991_Pseudodifferent}, local
well-posedness in Sobolev spaces $H^s$,  $s > 3/2$ on the line and the circle is
established in \cite{Karapetyan:2010fk}. 
For details on how this is done for CH ($\gamma =1$) on the line, see Rodriguez-Blanco 
\cite{Rodriguez-Blanco_2001_On-the-Cauchy-p}. Blow-up criteria 
is also investigated in \cite{Yin_2003_On-the-Cauchy-p} and 
\cite{Zhou_2005_Local-well-pose}, as well as by Constantin and Strauss 
\cite{Constantin_2000_Stability-of-a-}. 
\\
\\
Setting $\gamma = 0$ gives the celebrated 
BBM equation, which was proposed by 
Benjamin, Bona, and Mahony 
\cite{Benjamin_1972_Model-equations} as a model for 
the unidirectional evolution of long waves.
Solitary-wave solutions to this 
equation are global and orbitally stable (see Benjamin 
\cite{Benjamin_1972_The-stability-o}, 
\cite{Benjamin_1972_Model-equations}, and 
\cite{Constantin_2000_Stability-of-a-}).
For more general $\gamma$, the existence of global 
solutions to HR on the line with constant $H^1$ energy
was proved recently by Mustafa \cite{Mustafa_2007_Global-conserva}
using the approach developed by Bressan and 
Constantin in \cite{Bressan_2007_Global-conserva}. Using a vanishing 
viscosity argument, Coclite, 
Holden, and Karlsen \cite{Coclite_2005_Global-weak-sol}
established existence of a strongly continuous semigroup of global 
weak solutions of HR on the line for initial data in $H^1$.
Bendahmane, Coclite, and Karlsen 
\cite{Bendahmane_2006_Hsp-1-perturbat} extended this result to traveling 
wave solutions that are supersonic solitary shockwaves.
For more information on the existence of global solutions to the HR
equation, see Holden and Raynaud \cite{Holden_2007_Global-conserva}
and \cite{Yin_2003_On-the-Cauchy-p}. 
\\
\\
There is a variety of traveling wave solutions to the HR equation that can be 
obtained using various combinations of peaks, cusps, compactons, 
fractal-like waves, and plateaus (see Lenells 
\cite{Lenells_2006_Traveling-waves}). Orbital stability of solitary wave 
solutions was proved in \cite{Constantin_2000_Stability-of-a-}.
Solitary shock wave formation was 
analyzed in Dai and Huo \cite{Dai_2000_Solitary-shock-} using traveling 
wave solutions of the HR equation to derive a system of ordinary differential 
equations, with a vertical singular line in the phase plane corresponding with the 
formation of shock waves. Head-on collisions between two solitary 
waves was investigated in the work of Hui-Hui Dai, 
Shiqiang Dai, and Huo \cite{Dai_2000_Head-on-collisi} using a reductive 
perturbation method coupled with the technique of strained coordinates. 
\\
\\
In this work we study the continuity properties of the data-to-solution map for
the HR 
equation, expanding upon the work in \cite{Karapetyan:2010fk}, in which it was
shown that the data-to-solution map is not better than continuous in Sobolev
spaces $H^{s}$ on the line
and the circle. More precisely, following \cite{Chen:2011fk} we show the
following result:
%
%
\begin{theorem}
For $\gamma \neq 0$, the
data to solution map for HR is H\"older continuous from $B_{H^{s}}(R)$ (in
the topology of $H^{r}$) to $C([0, T], H^{r})$, where $T = T(R)$, for $s >
3/2$, $-1 \le r < s$. More
precisely, decompose the set $\Omega = \left\{ (s, r) \in \rr^{2}  \right\}$
into the pieces
  %
  %
  \begin{equation*}
  \begin{split}
    & \Omega_{1} = \left\{ (s, \ r):  \ s < 3/2 \right\}
    \\
    & \Omega_{2} = \left\{ (s, \ r):
     \ s>3/2, \ r < -1  \right\}
    \\
    & \Omega_{3} = \left\{ (s, \ r):
     \ s>3/2, \ r > s  \right\}
    \\
    & \Omega_{4} = \left\{ (s, \ r):
     \ s>3/2, \ -1 \le r \le s-1, \ s + r \ge 2  \right\}
    \\
    & \Omega_{5} = \left\{ (s, \ r):
     \ s>3/2, \ -1 \le r < 2-s \right\}
    \\
    & \Omega_{6} = \left\{ (s, \ r):
    \  s>3/2, \  s-1 < r < s  \right\}.
    \end{split}
\end{equation*}
  %
  %
\label{thm:main-thm}
\end{theorem}
%
\begin{center}
\begin{tikzpicture}[scale=2]
% Draw thin grid lines with color 40% gray + 60% white

% Draw x and y axis lines
\draw [->] (0,0) -- (3,0) node [below] {$s$};
\draw [->] (0,-2) -- (0,3) node [left] {$r$};
\draw [->, dashed] (0,0) -- (3,3);
\draw [->, dashed] (0,-1) -- (3,2);
\draw [->, dashed] (0,2) -- (3,-1);
\draw [->, dashed] (0,-1) -- (3,-1);
\draw [->, dashed] (3/2,-2) -- (3/2, 3);
\fill[color=green, fill opacity=0.3] (1.5, 0.5) -- (3,2) -- (3,0) -- (3,-1);
\fill[color=red, fill opacity=0.3] (1.5, 0.5) -- (1.5,1.5) -- (3,3) -- (3,2);
\fill[color=yellow, fill opacity=0.3] (0, -2) -- (1.5, -2) -- (1.5, 3) -- (0, 3);
\fill[color=blue, fill opacity=0.3] (1.5, 0.5) -- (1.5, -1) -- (3, -1);
\fill[color=brown, fill opacity=0.3] (1.5, 1.5) -- (3, 3) -- (1.5, 3);
\fill[color=cyan, fill opacity=0.3] (1.5, -1) -- (3, -1) -- (3, -2) -- (1.5, -2);


\foreach \x/\xtext in {1, 2}
    \draw[shift={(\x,0)}]  node[below] {$\xtext$};
\foreach \y/\ytext in {-1, 1, 2}
    \draw[shift={(0,\y)}]  node[left] {$\ytext$};
    \draw (1,0.5) node {$\Omega_{1}$};
    \draw (2,2.5) node {$\Omega_{3}$};
    \draw (2,1.5) node {$\Omega_{6}$};
    \draw (2,-1.5) node {$\Omega_{2}$};
    \draw (2,0.5) node {$\Omega_{4}$};
    \draw (2,-0.5) node {$\Omega_{5}$};
\end{tikzpicture}
\end{center}
%
%
Then for two initial data $u_{0}, v_{0} \in B_{H^{s}}(R)$, there exist unique
corresponding solutions $u(x,t), v(x,t)$ for $0 \le t \le T= T(R)$ to the
HR equation \eqref{hyperelastic-rod-equation} which satisfy 
%
%
\begin{equation*}
\begin{split}
  \| u(t) - v(t) \|_{H^{r}} \le C \| u_{0} - v_{0} \|^{\alpha(s, r, \gamma)},
  \quad 0
  \le t \le T
\end{split}
\end{equation*}
%
%
where 
%
%
\begin{equation*}
\begin{split}
\alpha = 
\begin{cases}
   1, \quad & (s,r, \gamma) \in \Omega_{4} 
  \\
   2(s-1)/(2-r),  \quad & (s, r, \gamma) \in \Omega_{5}
  \\
   s-r, \quad & (s, r, \gamma) \in \Omega_{6}.
\end{cases}
\end{split}
\end{equation*}
%

%
%%%%%%%%%%%%%%%%%%%%%%%%%%%%%%%%%%%%%%%%%%%%%%%%%%%%%
%
%
%                Main theorem
%
%
%%%%%%%%%%%%%%%%%%%%%%%%%%%%%%%%%%%%%%%%%%%%%%%%%%%%%
%
%
We remark that this result improves upon the H\"older continuity result in
\cite{Chen:2011fk} (there it is for $0 \le r < s$) by a full degree in $r$. 
%
%
%
%
\section{Proof of Theorem~\ref{thm:main-thm}}
%
%
The proof in the periodic case is analogous to that in the non-periodic case.
Hence, we restrict our attention to the non-periodic case. 

%
%
\subsection{Region $\Omega_{1}$} 
\label{ssec:reg-2}
This is open. In particular, well-posedness of HR must be proved or disproved
below $3/2$ before we can even consider discussing H\"older continuity in weak
topologies.
\subsection{Region $\Omega_{2}$} 
\label{ssec:reg-6}
Open. The lower bound on $r$ comes from Lemma~\ref{cor1}. Perhaps the lemma can
be strengthened.
\subsection{Region $\Omega_{3}$} 
\label{ssec:reg-7}
It makes little sense to talk about this region, 
i.e.\ we assume a priori that our initial
data is in $H^{s}$ (i.e.\ it may not even be in $H^{r}$ if $r > s$). If it is in
$H^{r}$, then we are back where we started.
\subsection{Region $\Omega_{4}$} 
\label{ssec:reg-m-imp}
%
%
Let $u_{0}(x), v_{0}(x)
\in B_{H^{s}}(R)$, $s > 3/2$ be two initial datum. Then from
the well-posedness theory for HR \cite{Karapetyan:2010fk}, we
know that there exists unique corresponding solutions $u, v \in C(I,
B_{H^{s}}(2R))$ to \eqref{hyperelastic-rod-equation}.
Set $v=u-w$. Then $v$ solves the Cauchy-problem
%
%
\begin{align}
	\label{uniqueness-exp}
& \p_t v
=  -\frac{\gamma}{2} \p_x [v(u + w)] 
\\
\notag
& \phantom{\p_t v = }- D^{-2} \p_x \left\{
\frac{3-\gamma}{2}[v(u+w)] + \frac{\gamma}{2}[\p_x v \cdot \p_x (u+w)]
\right\},
\\
& v(x,0) = u_{0}(x) - v_{0}(x).
\label{uniqueness-init-data}
\end{align}
%
%
%
%
Applying $D^r$ to both sides of \eqref{uniqueness-exp}, then 
multiplying both sides by $D^r v$ and integrating, we obtain
%
%
\begin{equation}
\begin{split}
 \frac{1}{2} \frac{d}{dt} \|v\|_{H^r}^2
 = & -\frac{\gamma}{2} \int_{\rr} D^r \p_x [v(u+w)] \cdot
D^r v \ dx
\\
& - \frac{3-\gamma}{2} \int_{\rr}  D^{r -2}
\p_x[v(u+w)] \cdot
D^r v \ dx  
\\
& - \frac{\gamma}{2} \int_{\rr} D^{r 
-2} \p_x [ \p_x v
\cdot \p_x (u+w)]\cdot D^r v \ dx.
\label{2v}
\end{split}
\end{equation}
%
%
We now estimate \eqref{2v} in parts.

\subsection{Estimate of Integral 1} Note that
%
%
\begin{equation}
\begin{split}
& \left |  -\frac{\gamma}{2} \int_{\rr} D^r \p_x [v(u+w)] \cdot
D^r v \ dx \right |
\\
& =
\left |
-\frac{\gamma}{2} \int_{\rr} \left[ D^r \p_x, \ u+w \right]v \cdot
D^r v \ dx - \frac{\gamma}{2} \int_{\rr} (u+w) D^r
\p_x v \cdot D^r v\ dx
\right | \\
& \le \left |
-\frac{\gamma}{2} \int_{\rr} \left[ D^r \p_x, \ u+w \right]v \cdot
D^r v \ dx \right |
+ \left | \frac{\gamma}{2} \int_{\rr} (u+w) D^r \p_x v
\cdot D^r v\
dx \right |.
\label{4v}
\end{split}
\end{equation}
%
%
Observe that integrating by parts gives
%
%
\begin{equation}
\begin{split}
\left | \frac{\gamma}{2}\int_{\rr} (u+w) D^r \p_x v \cdot
D^r v \ dx \right |
\lesssim \|\p_x (u+w)\|_{L^\infty}
\|v\|_{H^r}^2.
\label{4'v}
\end{split}
\end{equation}
%
%
%
%
To estimate the remaining integral of \eqref{4v}, we shall need the following
following result taken from \cite{Himonas_2009_Non-uniform-dep-per}:
%
\begin{lemma}
\label{cor1}
If $s > 3/2$ and $-1 \le r  \le s -1$, then
%
%
\begin{equation}
\begin{split}
\|[D^r \p_x ,f]g\|_{L^2} \le C \|f\|_{H^s} \|g\|_{H^r}.
\label{15}
\end{split}
\end{equation}
%
%
\end{lemma}
%
%
Set $s > 3/2$ and $-1 \le r \le s -1$. An application of 
Cauchy-Schwartz and Lemma~\ref{cor1} then yields 
%
%
\begin{equation}
\begin{split}
 \left | -\frac{\gamma}{2} \int_{\rr} [D^r \p_x, \ u+w] v
\cdot D^r v \ dx \right |
& \lesssim \|u+w\|_{H^s} 
\|v\|_{H^r}^2.
\label{7v}
\end{split}
\end{equation}
%
%
Combining \eqref{4'v} and \eqref{7v} and applying the Sobolev Imbedding 
Theorem, we obtain the estimate
%
%
\begin{equation}
\begin{split}
\left |  -\frac{\gamma}{2} \int_{\rr} D^r \p_x [v(u+w)] \cdot
D^r v \ dx \right |
 \lesssim \|u+w\|_{H^s} \|v\|_{H^r}^2, \quad s > 3/2, \ -1 \le r \le s-1.
\label{8v}
\end{split}
\end{equation}
%
%

\subsection{Estimate of Integral 2} We shall need the following.
%
%
%%%%%%%%%%%%%%%%%%%%%%%%%%%%%%%%%%%%%%%%%%%%%%%%%%%%%
%
%
%                frac deriv est
%
%
%%%%%%%%%%%%%%%%%%%%%%%%%%%%%%%%%%%%%%%%%%%%%%%%%%%%%
%
%
\begin{lemma}
For $s > 3/2$, $r \le s$, $s + r \ge 2$, we have
%
%
\begin{equation*}
\begin{split}
  \| fg \|_{H^{r-1}} \le \| f \|_{H^{r-1}} \| g \|_{H^{s}}.
\end{split}
\end{equation*}
%
%
\label{lem:frac-deriv}
\end{lemma}
%
%
%
%
%
%
Applying Cauchy-Schwartz and Lemma~\ref{lem:frac-deriv}, we obtain
%
%
%
\begin{equation*}
\begin{split}
\left | - \frac{3-\gamma}{2} \int_{\rr}  D^{r -2}
\p_x[v(u+w)] \cdot
D^r v \ dx  \right |
 & \lesssim \|u+w\|_{H^{r -1}} \|v\|_{H^r}^2
\end{split}
\end{equation*}
%
%
which implies
\begin{equation}
\begin{split}
\left | - \frac{3-\gamma}{2} \int_{\rr}  D^{r -2}
\p_x[v(u+w)] \cdot
D^r v \ dx  \right |
 & \lesssim \|u+w\|_{H^{s}} \|v\|_{H^r}^2
 \label{3v}
\end{split}
\end{equation}
%
for $s > 3/2, \ r \le s, \ \text{and} \ s + r \ge 2$.
%
%
\subsection{Estimate of Integral 3} We first apply
Cauchy-Schwartz to obtain
%
%
\begin{equation*}
\begin{split}
\left | - \frac{\gamma}{2} \int_{\rr} D^{r 
-2} \p_x [ \p_x v
\cdot \p_x (u+w)]\cdot D^r v \ dx \right | 
 \lesssim 
\|[\p_x v \cdot \p_x (u+w)] \|_{H^{r -1}}
\|v\|_{H^r}.
\end{split}
\end{equation*}
%
We now need the following result.
%
%
%
\begin{lemma}
\label{impo}
If  $s > 3/2$, $r \le s$, and $s + r \ge 2$,  then
%
%
\begin{equation}
\begin{split}
  \|f_{x}g_{x}\|_{H^{r - 1}} \le C \|f\|_{H^{r}}
\|g\|_{H^{s}}.
\label{11}
\end{split}
\end{equation}
%
%
\end{lemma}
%
Applying the lemma, we conclude that
%
\begin{equation}
\begin{split}
\left | - \frac{3-\gamma}{2} \int_{\rr}  D^{r -2}
\p_x[v(u+w)] \cdot
D^r v \ dx  \right |
 \lesssim \|u+w \|_{H^{s}}
\|v\|_{H^r}^2
\label{3'v}
\end{split}
\end{equation}
%
%
for $s > 3/2, \ r \le s, \ \text{and} \ s + r \ge 2$.
%
%
%
%
Grouping \eqref{8v}, \eqref{3v}, and \eqref{3'v}, and 
applying
the Sobolev Imbedding Theorem, we obtain
%
%
\begin{equation}
\begin{split}
\frac{1}{2} \frac{d}{dt}
\|v\|_{H^r}^2
& \lesssim \|u+w\|_{H^s}
\|v\|_{H^r}^2, \quad | t | < T
\\
& \le 2R \| v \|_{H^{r}}^{2}.
\label{9v}
\end{split}
\end{equation}
%
%
%
%
%
Letting $y(t) = \| v \|^{2}_{H^{r}}$, we obtain
%
%
%
\begin{equation*}
\begin{split}
  \frac{dy}{dt} \le cy
\end{split}
\end{equation*}
%
where $c = c(s, r, R)$. 
This admits the solution
%
%
\begin{equation*}
\begin{split}
  y(t) \le y(0) e^{ct}, \quad | t | < T
\end{split}
\end{equation*}
%
%
which implies
%
%
\begin{equation*}
\begin{split}
  y(t) \le y(0) e^{cT}.
\end{split}
\end{equation*}
%
%
Substituting back in for $y$, we see that
%
%
\begin{equation*}
\begin{split}
  \| v \|_{H^{r}}^{2} \le \| v(0) \|^{2}_{H^{r}} e^{cT}
\end{split}
\end{equation*}
%
%
or
%
%
\begin{equation}
  \label{lip-ineq}
\begin{split}
  & \| u(t) - w(t) \|_{H^{r}} \le C \| u_{0} - w_{0} \|_{H^{r}}, 
  \\
  & \text{for} \ | t | < T,
  \ s > 3/2, \ -1 \le r \le s-1, \ s + r \ge 2.
\end{split}
\end{equation}
%
Hence, in region $\Omega_{4}$, the data to solution map is locally Lipschitz from
$B_{H^{s}(R)}$ (in the $H^{r}$
topology) to $C([-T, T], H^{r})$, with Lipschitz constant $C = C(s, r, R)$.
%
%
%
%
%
%
%
%
%
%
\subsection{Region $\Omega_{5}$} 
\label{ssec:case-4}
%
%
Note that   $-1 \le 2-s \le s-1$ for $s>3/2$ and $s + (2-s) = 2$.
Hence, applying \eqref{lip-ineq}, we bound 
%
%
%
%
\begin{equation}
  \label{fgh}
\begin{split}
  \| v(t) \|_{H^{r}}
  & \le \|v(t) \|_{H^{2-s}}
  \\
  & \lesssim \|v(0) \|_{H^{2-s}}.
  \end{split}
\end{equation}
%
We need the following interpolation
result. 
%
%
%%%%%%%%%%%%%%%%%%%%%%%%%%%%%%%%%%%%%%%%%%%%%%%%%%%%%
%
%
%                interp
%
%
%%%%%%%%%%%%%%%%%%%%%%%%%%%%%%%%%%%%%%%%%%%%%%%%%%%%%
%
%
\begin{lemma}
  For $m_{1} < m < m_{2}$,
  %
  %
  \begin{equation*}
  \begin{split}
    \| f \|_{H^{m}} \le \| f \|_{H^{m_{1}}}^{(m_{2}-m)/(m_{2} - m_{1})} \| f
    \|_{H^{m_{2}}}^{(m -m_{1})/(m_{2} - m_{1})}.
  \end{split}
  \end{equation*}
  %
  %
  %
  %
   %
  %
\label{lem:interp}
\end{lemma}
%
Applying the lemma with $m_{1} =r$, $m = 2-s$, and $m_{2} = s$, we bound \eqref{fgh} by
%
%
\begin{equation*}
\begin{split}
  \| v(0) \|_{H^{r}}^{\frac{2(s-1)}{2-r}} \|v(0)
  \|_{H^{s}}^{\frac{2-s-r}{s-r}}
  & = \| u_{0} - w_{0} \|_{H^{r}}^{\frac{2(s-1)}{2-r}} \|u_{0} - w_{0}
  \|_{H^{s}}^{\frac{2-s-r}{s-r}}
  \\
  & \lesssim \| u_{0} - w_{0} \|_{H^{r}}^{\frac{2(s-1)}{2-r}}.
\end{split}
\end{equation*}
%
We conclude that
%
%
\begin{equation*}
\begin{split}
  \| u(t) - w(t) \|_{H^{r}} \lesssim \|u_{0} - w_{0} \|_{H^{r}}^{\frac{2(s-1)}{2-r}}.
\end{split}
\end{equation*}
%
%
%
%
\subsection{Region $\Omega_{6}$} 
\label{ssec:case-2}
%
%
Applying Lemma~\ref{lem:interp} and the fact that 
%
%
\begin{equation*}
\begin{split}
  \|v\|_{H^{s}} = \|u - w \|_{H^{s}} \le 4R
\end{split}
\end{equation*}
%
%
we obtain
%
%
\begin{equation}
  \label{pre-lip-ap}
\begin{split}
  \| v(t) \|_{r} & \lesssim \| v(t) \|_{H^{s-1}}^{s-r} \|v(t) \|_{H^{s}}^{1-s+r}
  \\
  & \simeq \| v(t) \|_{H^{s-1}}^{s-r}.
\end{split}
\end{equation}
%
%
Notice that $-1 \le s-1 \le s-1$, and $s + (s-1) = 2s-1 \ge 2$ for $s >3/2$.
Hence, we may apply \eqref{lip-ineq} to bound \eqref{pre-lip-ap} by
%
%
\begin{equation*}
\begin{split}
  C \|v(0) \|_{H^{s-1}}^{s-r}
  & \lesssim \|v(0) \|_{H^{r}}^{s-r} 
  = \|u_{0} - w_{0}\|_{H^{r}}^{s-r}.
\end{split}
\end{equation*}
%
%
This completes the proof of Theorem~\ref{thm:main-thm}. \qed
%
%
%%%%%%%%%%%%%%%%%%%%%%%%%%%%%%%%%%%%%%%%%%%%%%%%%%%%%
%
%
%				Optimality
%
%
%%%%%%%%%%%%%%%%%%%%%%%%%%%%%%%%%%%%%%%%%%%%%%%%%%%%%
%
%
\section{Optimality} 
\label{sec:optimality}
\subsection{Burgers} 
\label{ssec:burgers-opt}

We first consider the Burgers initial value problem
%
%
\begin{gather}
    \label{burgers}
u_{t} + uu_{x} = 0,
\\
\label{burgers-init}
u(x,0) = u_{0}(x)
\end{gather}
%
%
To solve this, we apply the method of characteristics. That is, we seek a curve
$s \mapsto (x(s), t(s))$ such that
%
%
\begin{equation*}
\begin{split}
\frac{d}{ds} u(x(s), t(s)) = 0.
\end{split}
\end{equation*}
%
Formally differentiating the left hand side, we obtain
%
%
\begin{equation*}
\begin{split}
\frac{du}{dx} \frac{dx}{ds} + \frac{du}{dt} \frac{dt}{ds}.
\end{split}
\end{equation*}
%
%
Setting
%
%
\begin{gather}
    \label{char-ode-space}
    \frac{dx}{ds} = u,
    \\
    \label{char-ode-time}
    \frac{dt}{ds}=1
\end{gather}
and recalling the Burgers equation \eqref{burgers}, we see that
%
%
\begin{equation*}
\begin{split}
\frac{d}{ds} u(x(s), t(s)) = 0.
\end{split}
\end{equation*}
%
%
Hence, $u$ is constant along the characteristic curve $(x(s), t(s))$ given by
the characteristic ode's \eqref{char-ode-space}-\eqref{char-ode-time}. Solving
them, we obtain
\begin{gather}
    \label{0j}
    x(s) = c_{1} + \int_{0}^{s} u(x(s'), t(s'))ds'
    \\
    t(s) = s + c_{2}.
\end{gather}
We remark that by the inverse function theorem, \eqref{0j} admits a solution
$u(x(s), t(s))$ for $s$ in some bounded interval in $\rr$.  By construction, this $u$ will be constant along the characteristic curve $(x(s), t(s))$. Hence, we obtain
%
%
\begin{gather}
    \label{1j}
    x(s) = c_{1} + su
    \\
    \label{2j}
    t(s) = s + c_{2}.
\end{gather}
%
%
Setting $s = -c_{2}$, it follows that 
%
%
%
\begin{equation*}
\begin{split}
u(x(-c_{2}), t(-c_{2})) = u(c_{1} - c_{2}u, 0 ) = u_{0}(c_{1} - c_{2} u).
\end{split}
\end{equation*}
%
%
and so
%
%
%
%
\begin{equation*}
\begin{split}
u(x,t) = u_{0}(x - tu).
\end{split}
\end{equation*}
%
%
We shall prove the following.
%
%
%%%%%%%%%%%%%%%%%%%%%%%%%%%%%%%%%%%%%%%%%%%%%%%%%%%%%
%
%
%			solution burgers 
%
%
%%%%%%%%%%%%%%%%%%%%%%%%%%%%%%%%%%%%%%%%%%%%%%%%%%%%%
%
%
\begin{lemma}
%
Let $u_{0}^{\lambda}(x) = (\lambda +
x_{+}^{\alpha + 1}) \vp(x)$ be a family of initial data indexed by $\lambda \in
[-1, 1]$, where $\alpha > 0$, $x_{+} \doteq \max\{0, x\}$ and $\vp$ is a smooth cutoff
function equal to the identity in $[-2, 2]$ and with support in $[-4,4]$. Then the associated solutions $u^{\lambda}(x,t)$ take the form
\begin{equation}
    \label{u-lam-explicit-form}
    \begin{split}
        u^{\lambda}(x,t) = \lambda + (x - \lambda t)^{\alpha + 1}_{+} p(t(x- \lambda t)^{\alpha}_{+}), \quad | x | \le 1, \quad | \lambda | \le 1
    \end{split}
\end{equation}
%
%
for sufficiently small $| t |$, where $p(z)$ is a power series in $z$ with $p(0) =1$ and a positive radius of convergence.
\label{lem:sol-burg}
\end{lemma}
%
%
%
%
\begin{proof}(Kato)
Since $u_{0}^{\lambda}$ is uniformly bounded in $\lambda$ in $H^{s}$, the
$u^{\lambda}$ have a common lifespan $T$ and are uniformly bounded (via an
energy estimate--see HR). Therefore, choosing $T$ sufficiently small, we have
%
%
\begin{equation}
    \label{burg-sol-cont}
\begin{split}
    | t u^{\lambda}(x,t) | \le 1, \quad x \in \rr, \ | t | \le T, \ | \lambda | \le 1.
\end{split}
\end{equation}
%
%
%
%
Set
%
%
\begin{equation}
    \label{burg-sol-not}
\begin{split}
y(x,t) = x - t u^{\lambda}(x,t).
\end{split}
\end{equation}
%
%
Then by \eqref{burg-sol-cont}, we have
%
%
\begin{equation*}
\begin{split}
    | y(x,t) | & \le | x | + | tu^{\lambda} |
    \\
    & \le 2, \quad | x | \le 1, \ | t | \le T, | \lambda | \le 1.
\end{split}
\end{equation*}
%
%
Hence, $\vp(y) =1$ and so
%
%
\begin{equation}
    \label{burg-sol-redux}
\begin{split}
u^{\lambda}(x,t) 
& = u_{0}^{\lambda}(x - tu)
\\
& = [\lambda + (x - tu^{\lambda})_{+}^{\alpha + 1}] \vp(x - tu^{\lambda})
\\
& = [\lambda + (x - tu^{\lambda})_{+}^{\alpha + 1}].
\end{split}
\end{equation}
%
%
Multiplying both sides by $t$, we get
%
%
\begin{equation*}
\begin{split}
t u^{\lambda}(x,t) = t \lambda + t (x - tu^{\lambda})_{+}^{\alpha + 1}
\end{split}
\end{equation*}
%
%
which implies
%
%
\begin{equation*}
\begin{split}
x - tu^{\lambda} = x - t \lambda - t(x - tu^{\lambda})_{+}^{\alpha + 1}.
\end{split}
\end{equation*}
%
%
Set 
\begin{equation}
    \label{y-for-u}
    y(x,t) = x - tu^{\lambda}. 
\end{equation}
%
%
%
%
Then
%
%
\begin{equation*}
\begin{split}
y = x - t \lambda - t y_{+}^{\alpha + 1}
\end{split}
\end{equation*}
%
%
or
%
\begin{equation*}
\begin{split}
y + ty_{+}^{\alpha + 1} = x - t \lambda
\end{split}
\end{equation*}
%
%
which implies
%
%
\begin{equation}
    \label{y-equation}
\begin{split}
[1 + ty_{+}^{\alpha + 1}]y_{+} = (x - t \lambda)_{+}.
\end{split}
\end{equation}
%
%
Solving this equation for $y$ and then substituting back into \eqref{y-for-u} completes the proof.
\end{proof}
\begin{framed}
    \textbf{I spent $7+$ hours, but couldn't figure out how to solve
\eqref{y-equation} in order to get \eqref{u-lam-explicit-form}. I go about
proving Lemma~\ref{lem:sol-burg} in slightly different fashion; see the next
section}.
\end{framed}
%
%
\begin{proof}(David)
Notice from \eqref{burg-sol-redux} that
%
%
\begin{equation*}
\begin{split}
u^{\lambda}(x, 0) = \lambda + x_{+}^{\alpha +1}.
\end{split}
\end{equation*}
%
%
Furthermore, \eqref{burg-sol-redux} implies that
%
%
%
%
\begin{equation*}
\begin{split}
u^{\lambda} - \lambda
& = (x - tu^{\lambda})^{\alpha + 1}
\\
& = [x - t(u^{\lambda} \pm \lambda)]^{\alpha + 1}_{+}
\\
& = [x - t(u^{\lambda - \lambda}) - t \lambda]_{+}^{\alpha + 1}.
\end{split}
\end{equation*}
%
%
Set $y = u^{\lambda} - \lambda$. Then
%
%
\begin{equation*}
\begin{split}
y = [x - t \lambda - ty]_{+}^{\alpha + 1}
\end{split}
\end{equation*}
%
%
or
%
%
\begin{equation*}
\begin{split}
y^{1/(\alpha + 1)} = [x - t \lambda - ty]_{+}.
\end{split}
\end{equation*}
%
%
Therefore, $y(t \lambda, t) = 0$, and so $$u^{\lambda}(t \lambda, t) = \lambda.$$
%
%
Hence, it is natural to guess that $u^{\lambda}(x,t) = \lambda + (x - \lambda
t)^{\alpha + 1}_{+} p(z)$ for some power series $p(z)$, where $z = z(x,t)$ and
$p(z(x,0)) =1$. We now show that this is indeed the case. Substituting our ansatz
into \eqref{burg-sol-redux}, we see that
%
%
\begin{equation*}
\begin{split}
(x - \lambda t)^{\alpha + 1}_{+} p(z) = \{x - t[\lambda + (x - \lambda
    t)_{+}^{\alpha +1} p(z)]\} 
\end{split}
\end{equation*}
%
%
which implies
%
%
\begin{equation*}
\begin{split}
x - \lambda t - (t + 1)(x - \lambda t)_{+}^{\alpha + 1} p(z) = 0.
\end{split}
\end{equation*}
%
%
Assume $x - \lambda t \neq 0$. Dividing through by $x - \lambda t$ and simplifying, we obtain
%
%
\begin{equation*}
\begin{split}
p(z) =\frac{1}{(t+1)(x - \lambda t)^{\alpha}}.
\end{split}
\end{equation*}
%
%
%
\begin{framed}
    \textbf{This function is NOT analytic at $x = \lambda t$. 
    Hence, I am very confused by Kato's claim that we can arrange that $p(z)$ be a
    power series with $p(0)=1$ and a positive radius of convergence. I feel that
    I have just shown that this is impossible. I will accept it for the time
    being.}
\end{framed}
%
%
%
\end{proof}
%
%
Applying Lemma~\ref{lem:sol-burg}, we have
%
%
\begin{equation*}
\begin{split}
u^{\lambda} - u^{0} = \lambda + (x - \lambda t)^{\alpha + 1}_{+}p(t(x- \lambda t)^{\alpha}_{+}) - (x)_{+}^{\alpha + 1}p(tx_{+}^{\alpha}) + \ \text{positive higher order terms}.
\end{split}
\end{equation*}
%
%
Furthermore, for integer $s \ge 2$ (we need to have well-posedness) and real $\alpha > (s-1)$
%
%
\begin{equation*}
\begin{split}
\frac{d^{s}}{dx^{s}}(u^{\lambda} - u^{0}) = (\alpha + 1)(\alpha) \cdots (\alpha + 1 -s)[(x - \lambda t)^{\alpha + 1 -s}_{+} - (x)_{+}^{\alpha + 1 -s}] + \ \text{positive higher order terms}.
\end{split}
\end{equation*}
%
%
Hence
%
%
\begin{equation*}
\begin{split}
\| u^{\lambda} - u^{0} \|_{H^{s}}^{2} 
& \ge \| \frac{d^{s}}{dx^{s}}(u^{\lambda} - u^{0}) \|_{L^{2}}^{2}
\\
& \ge \int_{-1}^{1} | \frac{d^{s}}{dx^{s}}(u^{\lambda} - u^{0}) |^{2} dx
\\
& \ge \int_{-1}^{1} \{(\alpha + 1)(\alpha) \cdots (\alpha + 1 -s   )[(x - \lambda t)_{+}^{\alpha + 1 -s} - x_{+}^{\alpha + 1 -s}]\}^{2} dx
\\
& \simeq \int_{0}^{\lambda t} (x^{\alpha + 1 -s})^{2} dx + \int_{\lambda t}^{1} [(x - \lambda t)^{\alpha +1 -s} -x^{\alpha + 1 -s}]^{2} dx
\\
& \ge \int_{0}^{\lambda t} x^{2(\alpha + 1 -s)} dx 
\\
& \simeq  | \lambda t |^{2 \alpha + 3 -2s}
\end{split}
\end{equation*}
%
%
and so
\begin{equation}
    \begin{split}
    \label{holder-lb}
\| u^{\lambda} - u^{0} \|_{H^{s}} & \ge c_{\alpha, s} | \lambda t |^{\alpha -s + 3/2}
\\
&  = c_{\alpha, s} | \lambda t |^{3/4}, \quad \alpha = s - 3/4.
\end{split}
\end{equation}
But
%
%
\begin{equation}
    \label{init-lb}
\begin{split}
    \| u^{\lambda}(0) - u^{0}(0) \|_{H^{r}} 
    & \simeq |\lambda |.
\end{split}
\end{equation}
%
Hence, from \eqref{holder-lb} and \eqref{init-lb},
we see that for any fixed constant $c > 0$ there exists sufficiently small $T = T(c) >
0$ such that for all $|t| \le T$
%
\begin{equation*}
\| u^{\lambda} - u^{0} \|_{H^{s}} \ge c \| u^{\lambda}(0) - u^{0}(0) \|_{H^{r}}.
\end{equation*}
Therefore, the Burgers initial value problem on the line with initial data
$u_{0} \in H^{s}, s \ge 2$ is not H\"older continuous for any exponent.
%
%
%%%%%%%%%%%%%%%%%%%%%%%%%%%%%%%%%%%%%%%%%%%%%%%%%%%%%
%
%
%				Optimality HR
%
%
%%%%%%%%%%%%%%%%%%%%%%%%%%%%%%%%%%%%%%%%%%%%%%%%%%%%%
%
%
\section{Optimality for HR} 
\label{sec:op-hr}
We recall the hyperelastic-rod (HR) ivp
\begin{gather}
    \label{hr}
    u_{t} + \frac{\gamma}{2}(u^{2})_{x} + P_{x} = 0
    \\
    \label{hr-data}
    u(x,0) = u_{0}(x)
\end{gather}
where
%
%
\begin{equation*}
\begin{split}
P(x,t) \doteq \frac{1}{2}e^{-| x |} * \left [\frac{3 - \gamma}{2}
    u^{2}(x,t) + \frac{\gamma}{2} u_{2}^{2}(x,t) \right ].
\end{split}
\end{equation*}
%
\begin{framed}
We remark that $e^{-| x |}$ is a weak solution to the ode $(1 + \p_{x}^{2})u =
2\delta$. Taking Fourier transforms of both sides, it follows that
$\widehat{e^{-| \cdot |}}(\xi) = 2/(1 + \xi^{2})$. Hence,
%
%
\begin{equation*}
\begin{split}
\frac{1}{2} e^{-| x |} * f = (1 - \p_{x}^2)f
\end{split}
\end{equation*}
%
%
Therefore, the form of HR which we all know and love is equivalent to
\eqref{hyperelastic-rod-equation}.
\end{framed}
%
Following Bressan \cite{Bressan_2007_Global-conserva}, we let $\xi \in \rr$ be an energy variable, and for fixed $t \in \rr$ define the map $t \mapsto y(\xi, t)$ implicitly by 
%
%
\begin{equation}
\label{potent-def}
\begin{split}
    \int_{0}^{y(\xi, t)} [1 + u^{2}(x,t)]dx = \xi.
\end{split}
\end{equation}
%
%
Consider also the initial value problem
%
%
\begin{gather}
    \label{en-eq}
\frac{dy}{dt}(\xi, t) = u(y(\xi, t), t),
\\
\label{en-data}
y(\xi, 0) = \xi
\end{gather}
%
%
and define
\begin{gather}
    \label{var-1}
    w(\xi, t) \doteq \gamma u(y(\xi, t), t)
    \\
    \label{var-2}
    v(\xi, t) \doteq u_{x}(y(\xi, t), t)
    \\
    \label{var-3}
    q(\xi, t) \doteq y_{\xi}(\xi, t)
\end{gather}
where we adopt the notation $$u_{x}(y(\xi, t),t) \doteq \frac{d}{dz}u(z,t) \big
|_{z = y(\xi, t)}$$ for the remainder of the paper. Using ivp
\eqref{en-eq}-\eqref{en-data}, we will rewrite the HR ivp
\eqref{hr}-\eqref{hr-data} as an ode system for the variables \eqref{var-1},
\eqref{var-2}, and \eqref{var-3}. Proceeding, we note that by the chain rule
%
%
\begin{equation}
\label{w-deriv}
\begin{split}
\frac{d}{dt}w(\xi, t)
& = \gamma u_{y} y_{t}(y(\xi, t), t) + u_{t}(y(\xi, t), )
\\
& = \gamma u u_{x}(y(\xi, t ), t) + u_{t}(y(\xi, t))
\\
& = -P_{x}(y(\xi, t), t).
\end{split}
\end{equation}
%
%
Now, we want to get $P_{x}(y(\xi, t), t)$ in "nice" form as we will be estimating later. Notice that
%
%
\begin{equation*}
\begin{split}
P(x,t) = \frac{1}{2} \int_{\rr}e^{-| x - x_{1} |} \left [\frac{3 - \gamma}{2}
u^{2}(x_{1}, t) + \frac{\gamma}{2} u_{x}^{2}(x_{1}, t) \right ]
\end{split}
\end{equation*}
%
%
and so 
%
%
\begin{equation}
\label{yuu}
\begin{split}
P(y(\xi, t), t) = \frac{1}{2} \int_{\rr} e^{-| y(\xi, t) - x_{1} |} \left [ \frac{3 - \gamma}{2} u^{2} + \frac{\gamma}{2}u_{x}^{2} \right ] (x_{1}, t) d x_{1}.
\end{split}
\end{equation}
%
%
Furthermore
%
%
\begin{equation}
\begin{split}
\label{p}
P_{x}(x,t)
& = \frac{1}{2}\p_{x} e^{-| x |}* \left [ \frac{3 - \gamma}{2} u^{2} + \frac{\gamma}{2} u_{x}^{2} \right ] 
\\
& = \frac{1}{2} \int_{\rr} \sgn(x - x_{1}) e^{-| x - x_{1} |} \left [ \frac{3 - \gamma}{2} u^{2}(x_{1}, t) + \frac{\gamma}{2} u_{x}^{2}(x_{1}, t) \right ] 
\\
& =\frac{1}{2} \left ( \int_{x}^{\infty} - \int_{-\infty}^{x} \right )
e^{-| x - x_{1} |} \left [ \frac{3 - \gamma}{2} u^{2}(x_{1}, t) +
\frac{\gamma}{2} u_{x}^{2}(x_{1}, t) \right ] dx_{1}
\end{split}
\end{equation}
%
%
and so%
%
\begin{equation}
\label{p-deriv}
\begin{split}
P_{x}(y(\xi, t), t)
& = \frac{1}{2} \left ( \int_{y(\xi, t)}^{\infty} - \int_{-\infty}^{y(\xi, t)} \right ) e^{-| y(\xi, t) - x_{1} |} \left [ \frac{3 - \gamma}{2} u^{2}(x_{1}, t) +
\frac{\gamma}{2} u_{x}^{2}(x_{1}, t) \right ] dx_{1}.
\end{split}
\end{equation}
%
%
Adopt the notation 
%
%
\begin{equation*}
\begin{split}
q(\xi, t) \doteq y_{\xi}(\xi, t).
\end{split}
\end{equation*}
%
%
Then applying the change of variable $x_{1} = y(\xi_{1}, t)$, we obtain
%
%
\begin{equation*}
\begin{split}
\eqref{yuu} & = \frac{1}{2} \int_{\rr} e^{-| y(\xi, t) - y(\xi_{1}, t) |} \left [
\frac{3 -\gamma}{2} w^{2}q + \frac{\gamma}{2} v^{2} q  \right ](\xi_{1}, t) d \xi_{1}
\\
& = \frac{1}{2} \int_{\rr} e^{-| \int_{\xi_{1}}^{\xi} q(\lambda, t) d \lambda |} \left [
\frac{3 -\gamma}{2} w^{2}q + \frac{\gamma}{2} v^{2} q  \right ](\xi_{1}, t) d \xi_{1}
\\
& \doteq Q = Q(v, w, q)(\xi, t).
\end{split}
\end{equation*}
%
%
Similarly
%
%
\begin{equation}
\label{R-def}
\begin{split}
\eqref{p-deriv} & = \frac{1}{2} \left ( \int_{\xi}^{\infty} -
\int_{-\infty}^{\xi} \right ) e^{-| \int_{\xi_{1}}^{\xi} q(\lambda, t) d \lambda 
|} \left [ \frac{3 - \gamma}{2} w^{2} q + \frac{\gamma}{2} v^{2} q \right ] (\xi_{1}, t) d \xi_{1}
\\
& \doteq R = R(v, w, q)(\xi, t). 
\end{split}
\end{equation}
%
Substituting this into \eqref{w-deriv}, we see that
%
%
\begin{equation}
\label{w-ode}
\begin{split}
\frac{d}{dt}w(\xi, t) = -R(w, v, q)(\xi, t).
\end{split}
\end{equation}
%
%
Now, we find analogues of \eqref{w-ode} for $v$ and $q$. Observe that by the
chain rule 
%
\begin{equation}
\label{lkk}
\begin{split}
\frac{d}{dt}v(\xi, t)
&  = \frac{du_{x}}{dy}(y(\xi, t), t) \frac{dy}{dt}(\xi, t) + u_{xt}(y(\xi, t), t)
\\
& =  (u_{xx}u + u_{xt})(y(\xi, t), t).
\end{split}
\end{equation}
%
Differentiating \eqref{hyperelastic-rod-equation} formally with respsect to $x$,
and using the fact that
%
%
\begin{equation*}
\begin{split}
P_{xx}
&  = \frac{1}{2} \p_{x}^{2}e^{-| x |}* \left [ \frac{3 - \gamma}{2}u^{2} + \frac{\gamma}{2} u_{x}^{2} \right ] 
\\
& = \frac{1}{2} \left [ e^{-| x |} - 2 \delta \right ] * \left [ \frac{3 - \gamma}{2}u^{2} + \frac{\gamma}{2} u_{x}^{2} \right ] 
\\
& = \frac{1}{2} e^{-| x |} * \left [ \frac{3 - \gamma}{2}u^{2} +
\frac{\gamma}{2} u_{x}^{2} \right ] - \left [ \frac{3 -
\gamma}{2}u^{2} + \frac{\gamma}{2} u_{x}^{2} \right ] 
\\
& = P - \left [ \frac{3 -
\gamma}{2}u^{2} + \frac{\gamma}{2} u_{x}^{2} \right ] 
\end{split}
\end{equation*}
%
%
we obtain
%
%
\begin{equation*}
\begin{split}
u_{xt} + u u_{xx} + u_{x}^{2} + P - \left [ \frac{3 -
\gamma}{2}u^{2} + \frac{\gamma}{2} u_{x}^{2} \right ]  = 0.
\end{split}
\end{equation*}
%
%
Substituting this into \eqref{lkk}, we get
%
%
\begin{equation*}
\begin{split}
\frac{d}{dt}v(\xi, t)
& = \left \{ u_{xx} u + \big[-uu_{xx} - u_{x}^{2} - P + \left ( \frac{3 - \gamma}{2} u^{2} + \frac{\gamma}{2}u_{x}^{2} \right ) ] \right \}(y(\xi, t), t)
\\
& = \left \{- P  + \frac{3 - \gamma}{2} u^{2} + \frac{\gamma-2}{2} u_{x}^{2} \right \}(y(\xi, t), t)
\\
& = \left \{-Q + \frac{3- \gamma}{2 \gamma^{2}}w^{2} + \frac{\gamma-2}{2}
v^{2} \right \}(\xi, t).
\end{split}
\end{equation*}
%
Lastly,
%
%
%
%
\begin{equation*}
\begin{split}
\frac{dq}{dt}(\xi, t)
& = \frac{d}{d\xi}\frac{d}{dt}y(\xi, t)
\\
& = \frac{d}{d \xi}u(y(\xi, t), t)
\\
& = \frac{du}{dy}(y(\xi,t),t)
\\
& = u_{x}(y(\xi, t), t)q(\xi, t)
\\
& = \frac{vq}{\gamma}(\xi,t).
\end{split}
\end{equation*}
%
%
Next, note that $y(\xi, 0) = 0$, and so
%
%
\begin{equation*}
\begin{split}
w(\xi, 0)
& = \gamma u(y(\xi, 0), 0)
\\
& = \gamma u(\xi, 0)
\\
& = \gamma u_{0}(\xi).
\end{split}
\end{equation*}
%
%


Also, 
%
%
\begin{equation*}
\begin{split}
v(\xi, 0)
& = u_{x}(y(\xi, 0), 0)
\\
& = u_{x}(\xi, 0)
\\
& = u_{0}'(\xi)
\end{split}
\end{equation*}
%
%
and
%
%
\begin{equation*}
\begin{split}
q(\xi,0)
& = y_{\xi}(\xi, 0)
\\
& = 1.
\end{split}
\end{equation*}
%
%
Hence, we are interested in solving the ode Cauchy-problem
%
%
%
%
\begin{gather}
\label{ode-system}
\frac{d}{dt}(w, v, q) = \left ( -R, -Q + \frac{3 - \gamma}{2 \gamma^{2}}w^{2} + \frac{\gamma -2}{2}v^{2}, \frac{vq}{\gamma} \right ) 
\\
\label{ode-system-init}
(w, v, q)(0) = (\gamma u_{0}(\xi), u_{0}'(\xi), 1).
\end{gather}
%
Denoting
%
%
\begin{equation*}
\begin{split}
  Y = H^{1} \times L^{\infty} \cap L^{2} \times L^{\infty},
\end{split}
\end{equation*}
%
%
we recall the following.
%
\begin{proposition}[Metric Space ODE Theorem]
	\label{prop:ode-thm}
  Let $X$ be a topological vector space over $\rr$
  with topology induced by a metric $d$, $E \subset X$ open, and $(-a, a)$ an
	open interval in $\rr$. Suppose $f: (-a, a) \times E \to X$ satisfies the
	inequality
	%
	%
	\begin{equation}
		\label{stronger-ode}
		\begin{split}
      d[f(t, x), f(t, y)] \le c d(x, y), \qquad \forall t \in (-a, a),
			\qquad \forall x, y \in E
		\end{split}
	\end{equation}
	%
  Then for given $\vp \in E$, there exists sufficiently small $h > 0$ and a
  unique differentiable map $u: (-h, h) \to E$ which for all $t \in (-h, h)$
  satisfies 
	%
	\begin{gather}
    \label{ode-thm-eq}
			u'(t) = f(t, u(t)),
			\\
      \label{ode-thm-init-data}
			u(0) = \vp.
	\end{gather}
\end{proposition}
We will obtain the following result as a corollary. 
%
%
%%%%%%%%%%%%%%%%%%%%%%%%%%%%%%%%%%%%%%%%%%%%%%%%%%%%%
%
%
%				Existence of Solution to ODE system
%
%
%%%%%%%%%%%%%%%%%%%%%%%%%%%%%%%%%%%%%%%%%%%%%%%%%%%%%
%
%
\begin{theorem}
  Assume $u_{0} \in H^{1} \cap C^{1}$. Then for sufficiently small $T > 0$
the Cauchy problem \eqref{ode-system}-\eqref{ode-system-init} has a unique
solution $(w, v, q) \in C^{1}([0, T], Y)$.
\label{thm:ode-sys-sol}
\end{theorem}
%
%
\begin{proof}
  Due to \eqref{prop:ode-thm}, it will
  be will be enough to show that the map $$(w, v, q) \mapsto
\left ( -R, -Q + \frac{3 - \gamma}{2 \gamma^{2}}w^{2} + \frac{\gamma
-2}{2}v^{2}, \frac{vw}{\gamma} \right ) \doteq F(w, v, q)$$ is Lipschitz on
an open subset $U$ of $Y$. 
%
Using the notation $Q_{1} \doteq Q(w_{1}, v_{1}, q_{1})$ and $Q_{2} \doteq
Q(w_{2}, v_{2}, q_{2})$, we have
%
%
\begin{equation}
  \label{lip-diff}
\begin{split}
  & (w_{2}, v_{2}, q_{2}) - (w_{1}, v_{1}, q_{1})
  \\
  & = \left( -(R_{2} - R_{1}),
  -(Q_{2} - Q_{1}) + \frac{3 - \gamma}{2 \gamma^{2}}(w_{2}^{2} -
  w_{1}^{2}) + \frac{\gamma -2}{2} (v_{2}^{2} - v_{1}^{2}),
  \frac{1}{\gamma}(v_{2} w_{2} - v_{1} w_{1}) \right).
\end{split}
\end{equation}
%
Let $r>0$, $w_{1}, w_{2} \in B_{C^{1} \cap L^{2}}(r)$, $v_{1}, v_{2} \in B_{L^{\infty} \cap L^{2}}(r)$, and
$q_{1}, q_{2} \in L^{\infty}(r)$, where $0 < c < q_{1} < r$ and $0 < c < q_{2} <r$.
Note that $L^{\infty} \cap L^{2}$ is an algebra, since
%
%
\begin{equation*}
\begin{split}
  \| fg \|_{L^{\infty} \cap L^{2}} & = \| fg \|_{L^{\infty}} + \| fg \|_{L^{2}}
  \\
  & \le \| f \|_{L^{\infty}} \| g \|_{L^{\infty}} + \| f \|_{L^{2}}\| g \|_{L^{\infty}}
  \\
  & \le \| f \|_{L^{\infty} \cap L^{2}} \| g \|_{L^{\infty} \cap L^{2}}.
\end{split}
\end{equation*}
%
%
Hence
%
%
%
\begin{equation}
  \label{1aa}
\begin{split}
  \| v_{2}w_{2} - v_{1}w_{1} \|_{L^{\infty} \cap L^{2}}
  & = \| v_{2}w_{2} \pm v_{2}w_{1} - v_{1}w_{1} \|_{L^{\infty} \cap L^{2}}
  \\
  & \le \| v_{2}(w_{2} - w_{1}) \|_{L^{\infty} \cap L^{2}}  
  + \| w_{1}(v_{2} - v_{1}) \|_{L^{\infty} \cap L^{2}}  
  \\
  & \lesssim_{r} \| w_{2} - w_{1} \|_{L^{\infty} \cap L^{2}} +
  \| v_{2} - v_{1} \|_{L^{\infty} \cap L^{2}}
\end{split}
\end{equation}
%
%
and
%
\begin{equation}
  \label{2aa}
\begin{split}
  \| w^{2}_{2} - w_{1}^{2} \|_{L^{\infty} \cap L^{2}} 
  & = \| (w_{2} - w_{1})(w_{2} + w_{1}) \|_{L^{\infty} \cap L^{2}}
    \\
    & \lesssim_{r} \| w_{2} - w_{1} \|_{L^{2} \cap L^{\infty} \cap L^{2}}.
\end{split}
\end{equation}
%
Similarly
%
%
\begin{equation}
  \label{3aa}
\begin{split}
\| v^{2}_{2} - v_{1}^{2} \|_{L^{\infty} \cap L^{2}} 
& \lesssim_{r}\| v_{2} - v_{1} \|_{L^{\infty} \cap L^{2}}.
\end{split}
\end{equation}
%
%
It remains to estimate $(R_{2} - R_{1})$ and $(Q_{2} - Q_{1})$ in $H^{1}$
and $L^{\infty} \cap L^{2}$, respectively. First, note that
%
%
\begin{equation}
\begin{split}
  \| Q_{2} - Q_{1} \|_{L^{\infty}}
  & \le \| \frac{3 - \gamma}{2} (w_{2}^{2} q_{2} - w_{1}^{2} q_{1}) +
  \frac{\gamma}{2}(v_{2}^{2}q_{2} - v_{1}^{2}q_{1}) \|_{L^{1}}
\end{split}
\end{equation}
%
%
Now
%
%
\begin{equation*}
\begin{split}
  w_{2}^{2} q_{2} - w_{1}^{2} q_{1}
  & = w_{2}^{2} q_{2} - w_{1}^{2} q_{2} + w_{1}^{2} q_{2} - w_{1}^{2} q_{1}
  \\
  & = (w_{2} - w_{1})(w_{2} + w_{1})q_{2} + (q_{2} - q_{1})w_{1}^{2}.
\end{split}
\end{equation*}
%
%
Similarly
%
%
\begin{equation*}
\begin{split}
  v_{2}^{2}q_{2} - v_{1}^{2} q_{1} = (v_{2} - v_{1})(v_{2}  + v_{1})q_{2} + (q_{2} - q_{1})v_{1}^{2}.
\end{split}
\end{equation*}
%
%
Hence, applying H\'older, we have 
\begin{equation}
  \label{q-init-bound}
  \begin{split}
  & \| \frac{3 - \gamma}{2} (w_{2}^{2} q_{2} - w_{1}^{2} q_{1}) +
  \frac{\gamma}{2}(v_{2}^{2}q_{2} - v_{1}^{2}q_{1}) \|_{L^{1}}
  \\
  & \lesssim \| q_{2} \|_{L^{\infty}} \| (w_{2} - w_{1})(w_{2} + w_{1})
  + (v_{2} - v_{1})(v_{2} + v_{1}) \|_{L^{1}}
  + \| q_{2} - q_{1} \|_{L^{\infty}} \| w_{1}^{2} + v_{1}^{2}\|_{L^{1}} 
  \\
  & \le \| q_{2} \|_{L^{\infty}} \| (w_{2} - w_{1})(w_{2} + w_{1})
  + (v_{2} - v_{1})(v_{2} + v_{1}) \|_{L^{1}}
  + \| q_{2} - q_{1} \|_{L^{\infty}} (\| w_{1}\|_{L^{2}}^{2} + \|
  v_{1}\|_{L^{2}}^{2}) 
  \\
  & \lesssim_{r} \| v_{2} - v_{1} \|_{L^{2}} + \| q_{2} - q_{1} \|_{L^{\infty}}
\end{split}
\end{equation}
where the last step follows from Cauchy-Schwartz and the a priori bounds
on $w_{i}, v_{i}, q_{i}$, $i \in \{1,2\}$. Therefore, 
%
%
\begin{equation}
  \label{pi}
\begin{split}
  \| Q_{2} - Q_{1} \|_{L^{\infty}} \lesssim_{r} \| v_{2} - v_{1} \|_{L^{2}} + \| q_{2} - q_{1} \|_{L^{\infty}}.
\end{split}
\end{equation}
%
%
Also, note that
%
%
%
%
\begin{equation}
  \label{I-II-split}
\begin{split}
  & Q_{2} - Q_{1}
  \\
  & = \frac{1}{2} \int_{\rr} \left\{ e^{-| \int_{\xi_{1}}^{\xi}
  q_{2}(\lambda) d \lambda|}\left[ \frac{3 - \gamma}{2} w_{2}^{2} q_{2} +
    \frac{\gamma}{2} v_{2}^{2} q_{2} \right] - e^{-| \int_{\xi_{1}}^{\xi}
    q_{1}(\lambda) d \lambda |} \left[ \frac{3 - \gamma}{2} w_{1}^{2}
      q_{1} + \frac{\gamma}{2} v_{1}^{2} q_{1} \right] \right\} d \xi_{1}
      \\
      & = I + II
\end{split}
\end{equation}
%
%
where
\begin{gather*}
  I = \int_{\rr} e^{-| \int_{\xi_{1}}^{\xi} q_{1}(\lambda) |} \left[ \frac{3-
  \gamma}{2}(w_{2}^{2} q_{2} - w_{1}^{2} q_{1}) +
  \frac{\gamma}{2}(v_{2}^{2} q_{2} - v_{1}^{2} q_{1}) \right]
  \\
  II = \int_{\rr} f\left[ \frac{3-\gamma}{2}w_{2}^{2}q_{2} + \frac{\gamma}{2}
    v_{2}^{2} q_{2} \right]d \xi_{1}, \quad
    f = e^{-| \int_{\xi_{1}}^{\xi} q_{2}(\lambda) d \lambda |} -
      e^{- | \int_{\xi_{1}}^{\xi} q_{1}(\lambda) d \lambda |}.
\end{gather*}
Observe that
%
\begin{equation}
  \label{q-l2-pre}
\begin{split}
  \| I \|_{L^{2}}^{2} & = \frac{1}{4} \int_{\rr} | \int_{\rr} e^{-| \int_{\xi_{1}}^{\xi}q_{1}(\lambda) d \lambda |}  \left[ \frac{3 - \gamma}{2}(w_{2}^{2} q_{2} -
  w_{1}^{2}q_{1}) + \frac{\gamma}{2}(v_{2}^{2} q_{2} - v_{1}^{2} q_{1}) \right]  d
  \xi_{1} |^{2} d \xi.
\end{split}
\end{equation}
%
%We shall first bound the exponential by analyzing it's argument. More precisley, recall that
%%
%%
%\begin{equation*}
%\begin{split}
  %\int_{0}^{y(\xi, t)} \left[ 1 + u(x, t) \right]^{2} dx = \xi.
%\end{split}
%\end{equation*}
%%
%%
%Differentiating both sides with respect to $\xi$ gives
%%
%%
%\begin{equation*}
%\begin{split}
  %y_{\xi} = \frac{1}{1 + y^{2}}.
%\end{split}
%\end{equation*}
%%%
%%%
%Fix $t$ and assume without loss of generality that $\xi \ge \xi_{1}$. If $y(s) \le 1$ for all $s \in [\xi_{1}, \xi]$, then
%%
%%
%\begin{equation*}
%\begin{split}
  %| y(\xi) - y(\xi_{1}) | 
  %& = | \int_{\xi_{1}}^{\xi} y_{s} ds|
  %\\
  %& = | \int_{\xi_{1}}^{\xi} \frac{1}{1 + y^{2}} ds |
  %\\
  %& \ge \frac{1}{2}| \xi - \xi_{1} |.
%\end{split}
%\end{equation*}
%%%
%%%
%Otherwise, since $y(s)$ is continuous and increasing, there exists $\xi_{1} \le \xi^{*} \le \xi$ such that $y(\xi^{*}) \ge 1$. Then
%%%
%%%
%\begin{equation*}
%\begin{split}
  %| y(\xi) - y(\xi_{1}) | & = | \int_{\left\{ s \in [\xi_{1}, \xi^{*}]: y \le 1
  %\right\}} \frac{1}{1 + y^{2}} ds + 
  %\int_{\left\{ s \in [\xi_{1}, \xi^{*}]: y \ge 1    
  %\right\}}  \frac{1}{1 + y^{2}} ds | 
  %\\
  %& \ge | \int_{\left \{ s \in [\xi_{1}, \xi^{*}]: y \le 1 \right\}}
  %\frac{1}{1 + y^{2}} ds + 
  %\int_{\left\{ s \in [\xi_{1}, \xi^{*}]: 1 \le y \le 2 \right\}} \frac{1}{1 +
  %y^{2}} ds \\
  %& \ge \frac{1}{2} | \xi^{*} - \xi_{1} | + \frac{1}{5} | \xi - \xi^{*} |
  %\\
  %& \ge \frac{2}{5} | \xi - \xi_{1} |.
%\end{split}
%\end{equation*}
%%
%%
%%
Since $0 < c_{1} < q_{1}$, we bound \eqref{q-l2-pre} by
%
%
\begin{equation*}
\begin{split}
  & \frac{1}{4} \int_{\rr} \left( \int_{\rr} | e^{-c_{1} | \xi - \xi_{1} |} \left[ \frac{3 - \gamma}{2}(w_{2}^{2} q_{2} - w_{1}^{2}q_{1}) + \frac{\gamma}{2}(v_{2}^{2} q_{2} - v_{1}^{2} q_{1}) \right] | d \xi_{1} \right)^{2} d \xi
  \\
  & = \frac{1}{4} \| e^{-c_{1} | \cdot |} * \left[ \frac{3 - \gamma}{2}(w_{2}^{2} q_{2} - w_{1}^{2}q_{1}) + \frac{\gamma}{2}(v_{2}^{2} q_{2} - v_{1}^{2} q_{1}) \right]  \|_{L^{2}}^{2}.
\end{split}
\end{equation*}
%
Applying Young's inequality, we obtain the bound
%
%
\begin{equation*}
\begin{split}
  \| e^{-c_{1} | x |    } \|_{L^{2}}^{2} \|  \frac{3 -
  \gamma}{2}(w_{2}^{2} q_{2} - w_{1}^{2}q_{1}) + \frac{\gamma}{2}(v_{2}^{2} q_{2} -
  v_{1}^{2} q_{1})\|_{L^{1}}^{2}
  & \lesssim \|  \frac{3 -
  \gamma}{2}(w_{2}^{2} q_{2} - w_{1}^{2}q_{1}) + \frac{\gamma}{2}(v_{2}^{2} q_{2} -
  v_{1}^{2} q_{1})\|_{L^{1}}^{2}
  \\
  & \lesssim_{r}
  (\| v_{2} - v_{1} \|_{L^{2}} + \| q_{2} - q_{1} \|_{L^{\infty}})^{2}
\end{split}
\end{equation*}
%
%
where the last step follows from \eqref{q-init-bound}. Therefore
%
%
\begin{equation}
  \label{I-est}
  \| I \|_{L^{2}} \lesssim_{r}
  \| v_{2} - v_{1} \|_{L^{2}} + \| q_{2} - q_{1} \|_{L^{\infty}}
\end{equation}
To bound $II$, we first note that
%
%
\begin{equation*}
\begin{split}
  f & = e^{- | \int_{\xi_{1}}^{\xi} q_{1}(\lambda) d \lambda|} \left[
    e^{| \int_{\xi_{1}}^{\xi} q_{1}(\lambda) |}e^{-|
      \int_{\xi_{1}}^{\xi}q_{2}(\lambda) d \lambda |} - 1
    \right]
    \\
    & = e^{- | \int_{\xi_{1}}^{\xi} q_{1}(\lambda) d \lambda |} \left(
    \sum_{n = 1}^{\infty} \frac{z^{n}}{n!}
    \right)
\end{split}
\end{equation*}
%
%
where
%
%
\begin{equation*}
\begin{split}
  z = | \int_{\xi_{1}}^{\xi} q_{1}(\lambda)d \lambda | - | \int_{\xi_{1}}^{\xi} q_{2}(\lambda) d \lambda |.
\end{split}
\end{equation*}
%
%
By the reverse triangle inequality
%
%
\begin{equation*}
\begin{split}
| z | & \le | \int_{\xi_{1}}^{\xi} q_{1}(\lambda)d \lambda  -  \int_{\xi_{1}}^{\xi} q_{2}(\lambda) d \lambda |
\\
& = | \int_{\xi_{1}}^{\xi} [q_{1} -  q_{2}](\lambda) d \lambda |
\\
& \le \| q_{1} - q_{2} \|_{L^{\infty}} | \xi - \xi_{1} |.
\end{split}
\end{equation*}
%
%
Due to the fact
that $0 < c_{1} <q_{1}$, we also have
%
%
\begin{equation*}
\begin{split}
  | \int_{\xi_{1}}^{\xi} q_{1}(\lambda) d \lambda | \ge c_{1} | \xi - \xi_{1} |.
\end{split}
\end{equation*}
%
%
Hence
%
%
%
\begin{equation*}
\begin{split}
  | f | &  \le e^{-c_{1} | \xi - \xi_{1} |} \sum_{n = 1}^{\infty} \frac{\left[ \| q_{2} - q_{1} \|_{L^{\infty}}| \xi - \xi_{1} | \right]^{n}}{n!}
  \\
  & \le e^{-c_{1} | \xi - \xi_{1} |} \| q_{2} - q_{1} \|_{L^{\infty}} \sum_{n =
  1}^{\infty} \frac{(\| q_{2} - q_{1} \|_{L^{\infty}})^{n-1}| \xi - \xi_{1} |^{n}
    }{n!}
\\
&  \le e^{-c_{1} | \xi - \xi_{1} |} \| q_{2} - q_{1} \|_{L^{\infty}} \sum_{n =
1}^{\infty} \frac{(2c_{2} )^{n-1}| \xi - \xi_{1} |^{n}
    }{n!}
    \\
    & \le e^{-c_{1}| \xi - \xi_{1} |} \| q_{2} - q_{1} \|_{L^{\infty}} e^{2c_{2}| \xi - \xi_{1} |}
    \\
    & = e^{-2c_{1} + c_{2}| \xi - \xi_{1} |} \| q_{2} - q_{1} \|_{L^{\infty}}.
\end{split}
\end{equation*}
%
%
Choose $c_{1}, c_{2}$ such that $-2c_{1} + c_{2} = -1/4$. Then we obtain
%
%
\begin{equation*}
\begin{split}
  | f | \le e^{-\frac{1}{4} | \xi - \xi_{1} |} \| q_{2} - q_{1} \|_{L^{\infty}}
\end{split}
\end{equation*}
%
%
and so
%
%
%
%
\begin{equation*}
\begin{split}
  \| II \|_{L^{2}}^{2}
  & = \| q_{2} - q_{1} \|_{L^{\infty}}^{2} \int_{\rr} |\int_{\rr} f\left[ \frac{3-\gamma}{2}w_{2}^{2}q_{2} + \frac{\gamma}{2}
    v_{2}^{2} q_{2} \right]d \xi_{1} | ^{2} d \xi  
    \\
    & \le \| q_{2} - q_{1} \|_{L^{\infty}}^{2} \int_{\rr} \left (\int_{\rr} e^{-\frac{1}{4}| \xi - \xi_{1} |} |\left[ \frac{3-\gamma}{2}w_{2}^{2}q_{2} + \frac{\gamma}{2}
    v_{2}^{2} q_{2} \right] | d \xi_{1} \right ) ^{2} d \xi 
    \\
    & = \| q_{2} - q_{1} \|_{L^{\infty}}^{2} \| e^{-\frac{1}{4} | \cdot |} * \left[ \frac{3 - \gamma}{2}w_{2}^{2} q_{2}
    + \frac{\gamma}{2}v_{2}^{2} q_{2}  \right]  \|_{L^{2}}^{2}.
  \end{split}
\end{equation*}
%
%
Applying Young's inequality, we bound this by
%
%
%
%
\begin{equation*}
\begin{split}
\| e^{-\frac{1}{4} | \cdot |} * \left[ \frac{3 - \gamma}{2}w_{2}^{2} q_{2}
    + \frac{\gamma}{2}v_{2}^{2} q_{2}  \right]  \|_{L^{2}}^{2}
    & \le \| e^{-\frac{1}{4}| x |} \|_{L^{2}}^{2} \| \frac{3 - \gamma}{2}w_{2}^{2} q_{2}
    + \frac{\gamma}{2}v_{2}^{2} q_{2}\|_{L^{1}}^{2}
    \\
    & \lesssim \| q_{2} \|_{L^{2}}^{2} \| \frac{3 - \gamma}{2} w_{2}^{2} + \frac{\gamma}{2} v_{2}^{2} \|_{L^{1}}
    \\
    & \lesssim_{r} 1 
\end{split}
\end{equation*}
%
since $w_{2}, v_{2}, q_{2} \in B_{L^{2}}(r)$.
%
Therefore
%
%
\begin{equation}
  \label{II-est}
\begin{split}
  \| II \|_{L^{2}} \lesssim_{r} \| q_{2} - q_{1} \|_{L^{\infty}}.
\end{split}
\end{equation}
%
%
Combining \eqref{I-est}, \eqref{II-est} and recalling \eqref{I-II-split}, we obtain
%
%
%
%
\begin{equation}
  \label{yi}
\begin{split}
  \| Q_{2} - Q_{1} \|_{L^{2}} \lesssim_{r} \| v_{2} - v_{1} \|_{L^{2}} + \| q_{2} - q_{1} \|_{L^{\infty}}
\end{split}
\end{equation}
%
%
Combining \eqref{pi} and \eqref{yi}, we see that
%
%
\begin{equation}
  \label{Q-diff-fin-est}
\begin{split}
  \| Q_{2} - Q_{1} \|_{L^{2} \cap L^{\infty}} \lesssim_{r} \| v_{2} - v_{1}
  \|_{L^{2}} + \| q_{2} - q_{1} \|_{L^{\infty}}
\end{split}
\end{equation}
%
%
Next, we turn our attention to the term 
$(R_{2} - R_{1})$. Recall that if
%
%
\begin{equation*}
\begin{split}
F(\xi) = \int_{a(\xi)}^{b(\xi)} f(\xi, \tau) d \tau
\end{split}
\end{equation*}
%
%
with $a(\xi), b(\xi)$ continuous over some region $[\xi_{0}, \xi_{1}]$ and $f, f_{\xi}$ continuous over $[\xi_{0}, \xi_{1}] \times [\tau_{0}, \tau_{1}]$, then by the Leibniz integral rule, we have
%
%
\begin{equation*}
\begin{split}
F'(\xi) = b'(\xi) f(b(\xi)) - a'(\xi) f(a(\xi)) + \int_{a(\xi)}^{b(\xi)} f_{\xi}(\xi, \tau) d \tau.
\end{split}
\end{equation*}
%
%
Recall the definition of $R$ in \eqref{R-def}, and let $f(\xi, \xi_{1}, t)$ denote
its integrand. Then
%
%
\begin{equation*}
\begin{split}
\frac{d}{d \xi} \int_{\xi}^{\infty}f(\xi, \xi_{1}, t) d \xi_{1}
& = -f(\xi, \xi, t) + \int_{\xi}^{\infty} f_{\xi}(\xi, \xi_{1}, t) d \xi_{1}.
\end{split}
\end{equation*}
%
%
Similarly
\begin{equation*}
\begin{split}
\frac{d}{d \xi} \int_{-\infty}^{\xi}f(\xi, \xi_{1}, t) d \xi_{1}
& = f(\xi, \xi, t) + \int_{-\infty}^{\xi} f_{\xi}(\xi, \xi_{1}, t) d \xi_{1}.
\end{split}
\end{equation*}
Note that
\begin{gather*}
    f(\xi, \xi, t) = \left [ \frac{3 - \gamma}{2}w^{2}q + \frac{\gamma}{2} v^{2} q \right ](\xi, t)
\end{gather*}
and
%
%
\begin{equation}
\label{oii}
\begin{split}
f_{\xi}(\xi, \xi_{1}, t) = \sgn \left [ \int_{\xi_{1}}^{\xi} q(\lambda, t) d \lambda \right ]q(\xi, t) \left [ \frac{3- \gamma}{2} w^{2} q + \frac{\gamma}{2} v^{2}q \right ] (\xi_{1}, t).
\end{split}
\end{equation}
%
%
Now, from \eqref{potent-def}, it follows that $y(\xi, t)$ is an increasing function of $\xi$. Therefore, 
%
%
\begin{equation*}
\begin{split}
\sgn \left [ \int_{\xi_1}^{\xi} q(\lambda, t) d \lambda \right ]  = \sgn(\xi - \xi_1)
\end{split}
\end{equation*}
%
%
which in conjunction with \eqref{oii} gives
%
%
\begin{equation*}
\begin{split}
\int_{\xi}^{\infty} f_{\xi}(\xi, \xi_{1}, t) d \xi_{1} - \int_{-\infty}^{\xi} f_{\xi}(\xi, \xi_{1}, t) d \xi_{1}
& = q(\xi, t) \int_{-\infty}^{\infty} \left [ \frac{3- \gamma}{2} w^{2} q + \frac{\gamma}{2} v^{2}q \right ] (\xi_{1}, t).
\\
& = q Q(\xi, t).
\end{split}
\end{equation*}
%
%
Therefore,
%
%
\begin{equation*}
\begin{split}
R_{\xi}(\xi, t) = \frac{1}{2} \left [ \frac{3- \gamma}{2} w^{2} + \frac{\gamma}{2} v^{2} + Q \right ]q(\xi, t)
\end{split}
\end{equation*}
%
and so
%
%
\begin{equation*}
\begin{split}
  (R_{2})_{\xi} - (R_{1})_{\xi} 
  & \simeq \left[ \frac{3 - \gamma}{2} w_{2}^{2} +
  \frac{\gamma}{2}v_{2}^{2} + Q_{2}  \right]q_{2} - \left[ \frac{3 - \gamma}{2}
  w_{1}^{2} + \frac{\gamma}{2}v_{1}^{2} + Q_{1}  \right]q_{1}  
  \\
  & = \left[ \frac{3 - \gamma}{2} (w_{2}^{2} - w_{1}^{2}) +
  \frac{\gamma}{2}(v_{2}^{2} - v_{1}^{2}) + Q_{2} - Q_{1}  \right]q_{1} + \left[
  \frac{3 + \gamma}{2} w_{2}^{2} + \frac{\gamma}{2}v_{2}^{2} + Q_{2}
  \right](q_{2} - q_{1}).  
\end{split}
\end{equation*}
%
By H\"older and the algebra property of $L^{\infty}$, we have the estimates
%
%
%
\begin{equation*}
\begin{split}
  \| (w_{2}^{2} - w_{1}^{2})q_{1} \|_{L^{2}}
  & = \| (w_{2} - w_{1})(w_{2} + w_{1})q \|_{L^{2}} \le \| w_{2} - w_{1} \|_{L^{2}} 
  \| w_{2} + w_{1} \|_{L^{\infty}} \| q_{2} \|_{L^{\infty}}
  \\
  & \lesssim_{r} \| w_{2} - w_{1} \|_{L^{2}}
\end{split}
\end{equation*}
%
%
and 
%
\begin{equation*}
\begin{split}
  \| (v_{2}^{2} - v_{1}^{2})q_{1} \|_{L^{2}}
  & = \| (v_{2} - v_{1})(v_{2} + v_{1})q \|_{L^{2}}
  \\
  & \le \| v_{2} - v_{1} \|_{L^{2}} 
  \| v_{2} + v_{1} \|_{L^{\infty}} \| q_{2} \|_{L^{\infty}}
  \\
  & \lesssim_{r} \| v_{2} - v_{1} \|_{L^{2}}.
\end{split}
\end{equation*}
Applying H\"older and \eqref{yi}, we also have
%
%
%
\begin{equation*}
\begin{split}
  \| (Q_{2} - Q_{1})q_{1} \|_{L^{2}} 
  & \le \| Q_{2} - Q_{1} \|_{L^{2}} \| q_{1} \|_{L^{\infty}}
  \\
  & \le (\| v_{2} - v_{1} \|_{L^{2}} + \| q_{2} - q_{1} \|_{L^{\infty}})\| q_{1} \|_{L^{\infty}}
  \\
  & \lesssim_{r} \| v_{2} - v_{1} \|_{L^{2}} + \| q_{2} - q_{1} \|_{L^{\infty}}.
\end{split}
\end{equation*}
%
%
Lastly,
%
\begin{equation*}
\begin{split}
 \| \left[
  \frac{3 + \gamma}{2} w_{2}^{2} + \frac{\gamma}{2}v_{2}^{2} + Q_{2}
  \right](q_{2} - q_{1}) \|_{L^{\infty}}
  & \lesssim_{r} \| q_{2} - q_{1} \|_{L^{\infty}} 
\end{split}
\end{equation*}
%
%
Hence
%
%
\begin{equation*}
\begin{split}
  \| (R_{2})_{\xi} - (R_{1})_{\xi} \|_{L^{\infty}} \lesssim_{r} \| w_{2} - w_{1} \|_{L^{\infty}} + \| v_{2} - v_{1} \|_{L^{\infty}}
+ \| q_{2} - q_{1} \|_{L^{\infty}}.
\end{split}
\end{equation*}
%
%
which implies
\begin{equation}
  \label{R-diff-fin-est}
\begin{split}
  \| R_{2} - R_{1} \|_{H^{1}} \lesssim_{r} \| w_{2} - w_{1} \|_{L^{\infty}} + \| v_{2} - v_{1} \|_{L^{\infty}}
+ \| q_{2} - q_{1} \|_{L^{\infty}}.
\end{split}
\end{equation}
%
%
Recalling \eqref{lip-diff}, combining estimates \eqref{1aa}-\eqref{3aa},
\eqref{Q-diff-fin-est}, and \eqref{R-diff-fin-est}, and applying the estimates
$\| f \|_{L^{\infty}} \lesssim \| f \|_{H^{1}}$ and $\| f \|_{L^{\infty}} \le \| f
\|_{L^{2} \cap L^{\infty}}$, we conclude that
%
%
%
\begin{equation*}
\begin{split}
  \|(w_{2}, v_{2}, q_{2}) - (w_{2}, v_{2}, q_{2})\|_{Y} \lesssim_{r} \| w_{2} - w_{1} \|_{H^{1}} + \| v_{2} - v_{1} \|_{L^{2} \cap L^{\infty}}  + \| q_{2} - q_{1} \|_{L^{\infty}}
\end{split}
\end{equation*}
%
%
ending the proof.
%
\end{proof} 
%
\section{Proofs of Lemmas} 
\label{sec:pf-lemmas}
%
%
%
\begin{proof}[Proof of Lemma~\ref{lem:frac-deriv}]
For the non-periodic case we have
%
%
\begin{equation*}
\begin{split}
  \| fg\|_{H^{r-1}}^{2}
  & = \int_{\rr} (1 + \xi^{2})^{r-1}| \int_{\rr}
  \wh{f}(\eta) \wh{g}( \xi - \eta) d \eta |^{2} d \xi
  \\
  & \le \int_{\rr} (1 + \xi^{2})^{r-1}\left [ \int_{\rr}
  | \wh{f}(\eta) |  | \wh{g}(\xi - \eta) | 
  d \eta \right ]^{2} d \xi
  \\
  & = \int_{\rr}  (1 + \xi^{2})^{r-1}\left [ \int_{\rr}
  | \wh{f}(\eta) |  | \wh{g}(\xi - \eta) | (1 +
  \eta^{2})^{\frac{1-s}{2}} (1 + \eta^{2})^{\frac{s-1}{2}}
  d \eta \right ]^{2} d \xi
  \\
  & \le \int_{\rr}  (1 + \xi^{2})^{r-1}\left [ \int_{\rr} 
  | \wh{f}(\eta) | | \wh{g}(\xi - \eta) | (1 +
  \eta^{2})^{\frac{1-s}{2}} (1 + \eta^{2})^{\frac{s}{2}}
  d \eta \right ]^{2} d \xi.
\end{split}
\end{equation*}
%
Applying Cauchy Schwartz in $\eta$, we bound this by
%
%
%
\begin{equation}
  \label{np-key-term}
\begin{split}
  \| f \|_{H^{s}}^{2} \int_{\rr}  (1 + \xi^{2})^{r-1}\int_{\rr} \frac{|
  \wh{g}(\xi - \eta) |^{2}}{(1 + \eta^{2})^{s-1}} d \eta d \xi.
  \end{split}
\end{equation}
%
%
Applying a change of variable, we get
%
\begin{equation*}
\begin{split}
  \| f \|_{H^{s}}^{2} \int_{\rr} (1 + \xi^{2})^{r-1} \int_{\rr}
\frac{| \wh{g}(\eta) |^{2}}{[1 + (\xi - \eta)^{2}]^{s-1}} d \eta d \xi
  \end{split}
\end{equation*}
which by Fubini is equal to
%
%
\begin{equation}
  \label{int-pre-calc-lem}
\begin{split}
  \|f \|_{H^{s}}^{2} \int_{\rr} | \wh{g}(\eta) |^{2} \int_{\rr} \frac{1}{\left[
  1 + (\xi - \eta)^{2} \right]^{s-1} (1 + \xi^{2})^{1-r}} d \xi d \eta.
\end{split}
\end{equation}
%
%
We now need the following lemma, whose proof is provided in the appendix.
%
%
\begin{lemma}
	\label{lem:calc}
 %
 Fix $p, q > 0$ such that $p +q >1$, and let $r =\min\left\{p, q, p+q-1
 \right\}$. Adopt the notation
 $\langle x - \alpha \rangle  \doteq 1 + | x - \alpha |$. Then 
 %
 \begin{enumerate}[(I)]
   \item
For $\alpha=\beta \ \text{or} \ p \neq 1 \ \text{or} \ q \neq 1$
 \begin{equation*}
\begin{split}
  & \int_{\rr} \frac{1}{\langle x - \alpha \rangle ^{p} \langle x -
  \beta \rangle
  ^{q}} d x
  \le \frac{c_{p,q}}{\langle \alpha - \beta \rangle ^{r}}, 
  \end{split}
\end{equation*}
  \item
    \begin{equation*}
  \int_{\rr} \frac{1}{\langle x - \alpha \rangle  \langle x - \beta
  \rangle} d x
  \le  \frac{4 \log \langle \alpha - \beta \rangle}{\langle \alpha - \beta
  \rangle}, \quad \alpha \neq \beta.
\end{equation*}
\end{enumerate}
\end{lemma}
To be able to apply the lemma to the integral term in \eqref{int-pre-calc-lem}, 
we must first check
that its conditions are met. Let $ s = 3/2 + \ee$, $r = 1- \delta$, $\ee > 0$, $
\delta \ge 0$ and observe that
%
%
\begin{equation*}
\begin{split}
2(s-1) + 2(1-r)
& = 2(s-r)
\\
& = 2[3/2 + \ee - (1 - \delta)]
\\
& = 2(1/2 + \ee + \delta)
\\
& = 1 + 2 \ee + 2 \delta > 1.
\end{split}
\end{equation*}
%
%
Furthermore, $2(s-1), 2(1-r) > 0$. Hence, Lemma~\ref{lem:calc} is applicable. 
Lastly, observe that 
%
%
\begin{equation*}
\begin{split}
  \min\left\{ 2(s-1), 2(1-r), 2(s-1) + 2(1-r) -1 \right\}
  & = \min\left\{ 1 + 2 \ee, 2 \delta, 2(\ee + \delta) \right\}
  \\
  & = \min\left\{ 1 + 2 \ee, 2 \delta\right\}
  \\
  & = 2 \delta, \quad \delta \le 1/2 + \ee.
\end{split}
\end{equation*}
%
%
Hence, the integral term of \eqref{int-pre-calc-lem} is bounded by
\begin{equation*}
\begin{split}
  C_{s,r} \int_{\rr}  | \wh{g}(\eta) |^{2} \int_{\rr} \frac{1}{\left[ 1
  + 2 |\eta| \right]^{2 \delta}} d \xi d \eta 
  & \lesssim
  \int_{\rr}  | \wh{g}(\eta) |^{2} \int_{\rr} \frac{1}{\left[ 1
  + |\eta| \right]^{2 \delta}} d \xi d \eta  
  \\
  & \le \int_{\rr}  | \wh{g}(\eta) |^{2} \int_{\rr} \frac{1}{\left[ 1
  + \eta^{2} \right]^{\delta}} d \xi d \eta  
  \\
  & \le \| w \|_{H^{-\delta}}^{2}
  \\
  & = \| w \|_{H^{r-1}}^{2}.
\end{split}
\end{equation*}
%
Note that this bound is valid for all $0 \le \delta \le 1/2 + \ee$, $\ee >
0$. Recasting this in terms of $r$ and $s$, we have 
$$1-r \le 1/2 + s - 3/2, \quad r \le 1, \ s > 3/2,$$ which is equivalent to 
$$s + r \ge 2,  \quad  r \le 1, \ s > 3/2.$$ Therefore, 
%
%
%
%
\begin{equation}
  \label{yhh}
\begin{split}
  \| f g \|_{H^{r-1}} \lesssim \| f \|_{H^{s}} \| g \|_{H^{r-1}},
  \quad s + r \ge 2, \ s > 3/2, \ r \le 1.
\end{split}
\end{equation}
We will now establish this bound for $r=s$ where $s > 3/2$, and then interpolate to obtain
bounds for the whole range $1 \le r \le s$, where again $s > 3/2$. We shall need the following. 
%
%
\begin{lemma}[Algebra Property]
  \label{lem:alg-prop}
If  $s>1/2$ then there is $c_s>0$ such that 
%
%
%
\begin{equation} \label{KP-com-est}
  \| fg\|_{H^{s}} \le c_s \| f \|_{H^{s}} \| g \|_{H^{s}}
\end{equation}
%
%
%
\end{lemma}
%
%
%
%
%
Hence,
%
%
\begin{equation}
  \label{pre-interp-1}
\begin{split}
  \| f g \|_{H^{s-1}}
  & \lesssim   \|f  \|_{H^{s-1}} \| g \|_{H^{s-1}}, \quad s >3/2
  \\
  & \le \| f \|_{H^{s}} \| g \|_{H^{s-1}}.
\end{split}
\end{equation}
%
%
%
%
%
%
%
We now wish to interpolate to obtain estimates for the whole range $1 \le r \le s$.
We need the following.
%
%
%%%%%%%%%%%%%%%%%%%%%%%%%%%%%%%%%%%%%%%%%%%%%%%%%%%%%
%
%
%                
%
%
%%%%%%%%%%%%%%%%%%%%%%%%%%%%%%%%%%%%%%%%%%%%%%%%%%%%%
%
%
\begin{proposition}[Sobolev Interpolation]
  For fixed $k \le q, m \le s$ suppose that \\ $T: H^{k} \to H^{m}$ continuously
and $T: H^{q} \to H^{s}$. Then\\ $T: H^{\theta q + (1 - \theta)k} \to H^{\theta
s + (1 - \theta) m}$ continuously for all $\theta \in [0,1)$.
\label{prop:sob-interp}
\end{proposition}
%
To apply Proposition~\ref{prop:sob-interp}, we note that \eqref{yhh}
and \eqref{pre-interp-1} imply
%
%
\begin{equation*}
\begin{split}
  \| f g \|_{H^{r-1}} \lesssim \| g \|_{H^{r-1}}, \quad
  r=1, \  r =s, \ \| f \|_{H^{s}} =1.
\end{split}
\end{equation*}
%
%
That is, for fixed $f \in H^{s}$ with $\| f \|_{H^{s}} =1$, the map $g \mapsto
Tg = fg$ is continuous from $L^{2}$ to $L^{2}$ and from $H^{s-1}$ to
$H^{s-1}$. Therefore, by Proposition~\ref{prop:sob-interp}, it is continuous from
$H^{\theta (s-1) }$ to $H^{\theta (s-1)}$ for all $\theta \in
[0, 1)$. Setting $\theta = (r-1)/(s-1)$, $ 1 \le r \le s$, we obtain that $T$ is
continuous from $H^{r-1}$ to $H^{r-1}$; that is
%
%
\begin{equation*}
\begin{split}
  \| f g \|_{H^{r-1}} \lesssim \| g \|_{H^{r-1}}, \quad 1 \le r \le s, \
  \| f \|_{H^{s}} =1
\end{split}
\end{equation*}
which for general $f \in H^{s}$ implies 
%
\begin{equation}
  \label{hhh}
\begin{split}
  \| f g \|_{H^{r-1}} \lesssim \|f \|_{H^{s}}
  \| g \|_{H^{r-1}}, \quad 1 \le r \le s. 
\end{split}
\end{equation}
%
Combining \eqref{yhh} and \eqref{hhh} completes the proof in the non-periodic
case. For the periodic case, we note that the analogue of the integral term in \eqref{np-key-term} is
%
%
%
\begin{equation*}
\begin{split}
  \sum_{n \in \zz}   (1 + n^{2})^{r-1}\int_{\ci} \frac{| \wh{g}(n - \eta)
  |^{2}}{(1 + \eta^{2})^{s-1}} d \eta. 
\end{split}
\end{equation*}
%
%
The remainder of the proof is analogous to that in the non-periodic case.
\end{proof}
%
\begin{proof}[Proof of Lemma~\ref{impo}]
We have
%
%
\begin{equation*}
\begin{split}
  \| f_{x} g_{x} \|_{H^{r-1}}^{2}
  & = \int_{\rr} (1 + \xi^{2})^{r-1}| \int_{\rr} 
  \eta \wh{f}(\eta) (\xi - \eta) \wh{f}(\xi - \eta) d \eta |^{2} d \xi
  \\
  & \le \int_{\rr} (1 + \xi^{2})^{r-1}\left [ \int_{\rr} |
  \eta| | \wh{f}(\eta) | | \xi - \eta |  | \wh{g}(\xi - \eta) | 
  d \eta \right ]^{2} d \xi
  \\
  & = \int_{\rr}  (1 + \xi^{2})^{r-1}\left [ \int_{\rr} |
  \eta| | \wh{f}(\eta) | | \xi - \eta |  | \wh{g}(\xi - \eta) | (1 +
  \eta^{2})^{\frac{1-s}{2}} (1 + \eta^{2})^{\frac{s-1}{2}}
  d \eta \right ]^{2} d \xi
  \\
  & \le \int_{\rr}  (1 + \xi^{2})^{r-1}\left [ \int_{\rr}
  | \wh{f}(\eta) | |
  \xi -\eta|  | \wh{g}(\xi - \eta) | (1 +
  \eta^{2})^{\frac{1-s}{2}} (1 + \eta^{2})^{\frac{s}{2}}
  d \eta \right ]^{2} d \xi.
\end{split}
\end{equation*}
%
Applying Cauchy Schwartz in $\eta$, we bound this by
%
%
%
\begin{equation}
  \label{np-key-term-d}
\begin{split}
  \| f \|_{H^{s}}^{2} \int_{\rr}  (1 + \xi^{2})^{r-1}\int_{\rr} \frac{(\xi -
  \eta)^{2} | \wh{g}(\xi - \eta) |^{2}}{(1 + \eta^{2})^{s-1}} d \eta d \xi.
  \end{split}
\end{equation}
%
%
Applying a change of variable, we get
%
\begin{equation*}
\begin{split}
  \| f \|_{H^{s}}^{2} \int_{\rr} (1 + \xi^{2})^{r-1} \int_{\rr} \frac{\eta^{2}
  | \wh{g}(\eta) |^{2}}{[1 + (\xi - \eta)^{2}]^{s-1}} d \eta d \xi
  \end{split}
\end{equation*}
which by Fubini is equal to
%
%
\begin{equation*}
\begin{split}
  \int_{\rr} \eta^{2} | \wh{g}(\eta) |^{2} \int_{\rr} \frac{1}{\left[ 1 + (\xi -
  \eta)^{2} \right]^{s-1} (1 + \xi^{2})^{1-r}} d \xi d \eta.
\end{split}
\end{equation*}
%
Using the inequality 
\begin{equation}
  \label{simp-ineq}
  1 + a^{2} \ge (1 + a)^{2}/4
\end{equation}
and a change of
variable, we bound this by
%
%
%
\begin{equation}
  \label{pre-int-lem-calc}
\begin{split}
  C_{s,r} \int_{\rr} \eta^{2} | \wh{g}(\eta) |^{2} \int_{\rr} \frac{1}{\left[ 1
  + |\xi - \eta| \right]^{2(s-1)} (1 + |\xi|)^{2(1-r)}} d \xi d \eta.
\end{split}
\end{equation}
%
%
To be able to apply the lemma to \eqref{pre-int-lem-calc}, we must first check
that its conditions are met. Let $ s = 3/2 + \ee$, $r = 1- \delta$, $\ee > 0$, $
\delta \ge 0$ and observe that
%
%
\begin{equation*}
\begin{split}
2(s-1) + 2(1-r)
& = 2(s-r)
\\
& = 2[3/2 + \ee - (1 - \delta)]
\\
& = 2(1/2 + \ee + \delta)
\\
& = 1 + 2 \ee + 2 \delta > 1.
\end{split}
\end{equation*}
%
%
Furthermore, $2(s-1), 2(1-r) > 0$. Hence, Lemma~\ref{lem:calc} is applicable. 
Lastly, observe that 
%
%
\begin{equation*}
\begin{split}
  \min\left\{ 2(s-1), 2(1-r), 2(s-1) + 2(1-r) -1 \right\}
  & = \min\left\{ 1 + 2 \ee, 2 \delta, 2(\ee + \delta) \right\}
  \\
  & = \min\left\{ 1 + 2 \ee, 2 \delta\right\}
  \\
  & = 2 \delta, \quad \delta \le 1/2 + \ee.
\end{split}
\end{equation*}
%
%
Hence, \eqref{pre-int-lem-calc} is bounded by
\begin{equation*}
\begin{split}
  C_{s,r} \int_{\rr} \eta^{2} | \wh{g}(\eta) |^{2} \int_{\rr} \frac{1}{\left[ 1
  + 2 |\eta| \right]^{2 \delta}} d \xi d \eta 
  & \lesssim
  \int_{\rr} \eta^{2} | \wh{g}(\eta) |^{2} \int_{\rr} \frac{1}{\left[ 1
  + |\eta| \right]^{2 \delta}} d \xi d \eta  
  \\
  & \le \int_{\rr} \eta^{2} | \wh{g}(\eta) |^{2} \int_{\rr} \frac{1}{\left[ 1
  + \eta^{2} \right]^{\delta}} d \xi d \eta  
  \\
  & \le \| w \|_{H^{1- \delta}}^{2}
  \\
  & = \| w \|_{H^{r}}^{2}.
\end{split}
\end{equation*}
%
Note that this bound is valid for all $0 \le \delta \le 1/2 + \ee$, $\ee >
0$. Recasting this in terms of $r$ and $s$, we have 
$$1-r \le 1/2 + s - 3/2, \quad r \le 1, \ s > 3/2,$$ which is equivalent to 
$$s + r \ge 2,  \quad  r \le 1, \ s > 3/2.$$ Therefore, 
%
%
%
%
\begin{equation}
  \label{r-lt-1-range}
\begin{split}
  \| f_{x} g_{x} \|_{H^{r-1}} \lesssim \| f \|_{H^{s}} \| g \|_{H^{r}},
  \quad s + r \ge 2, \ s > 3/2, \ r \le 1.
\end{split}
\end{equation}
%
%
We will now establish this bound for $r=s$ where $s > 3/2$, and then interpolate
to obtain bounds for the whole range $1 \le r \le s$, where again $s > 3/2$.
Applying Lemma~\ref{lem:alg-prop}, we obtain
%
%
%
\begin{equation}
  \label{pre-interp-2}
\begin{split}
  \| f_{x} g_{x} \|_{H^{s-1}}
  & \lesssim   \|f_{x}  \|_{H^{s-1}} \| g \|_{H^{s-1}}, \quad s >3/2
  \\
  & \le \| f \|_{H^{s}} \| g \|_{H^{s}}.
\end{split}
\end{equation}
%
%
%
%
%
%
%
%
To apply Proposition~\ref{prop:sob-interp}, we note that \eqref{r-lt-1-range}
and \eqref{pre-interp-2} imply
%
%
\begin{equation*}
\begin{split}
  \| f_{x} g_{x} \|_{H^{r-1}} \lesssim \| g \|_{H^{r}}, \quad
  r=1, \  r =s, \ \| f \|_{H^{s}} =1.
\end{split}
\end{equation*}
%
%
That is, for fixed $f \in H^{s}$ with $\| f \|_{H^{s}} =1$, the map $g \mapsto
Tg = f_{x} g_{x}$ is continuous from $H^{1}$ to $L^{2}$ and from $H^{s}$ to
$H^{s-1}$. Therefore, by Proposition~\ref{prop:sob-interp}, it is continuous from
$H^{\theta s + 1 - \theta}$ to $H^{\theta(s-1)}$ for all $\theta \in
(0, 1)$. Setting $\theta = (r-1)/(s-1)$, $ 1 \le r \le s$, we obtain that $T$ is continuous from $H^{r}$ to $H^{r-1}$; that is
%
%
\begin{equation*}
\begin{split}
  \| f_{x} g_{x} \|_{H^{r-1}} \lesssim \| g \|_{H^{r}}, \quad 1 \le r \le s, \
  \| f \|_{H^{s}} =1
\end{split}
\end{equation*}
which for general $f \in H^{s}$ implies 
%
\begin{equation}
  \label{r-g-1-range}
\begin{split}
  \| f_{x} g_{x} \|_{H^{r-1}} \lesssim \|f \|_{H^{s}}
  \| g \|_{H^{r}}, \quad 1 \le r \le s. 
\end{split}
\end{equation}
%
Combining \eqref{r-lt-1-range} and \eqref{r-g-1-range} completes the proof in
the non-periodic case. For the periodic case, we note that the analogue of the
integral term in \eqref{np-key-term-d} is
%
%
%
\begin{equation*}
\begin{split}
  \sum_{n \in \zz}   (1 + n^{2})^{r-1}\int_{\ci} \frac{(n - \eta)^{2}| \wh{g}(n
  - \eta) |^{2}}{(1 + \eta^{2})^{s-1}} d \eta. 
\end{split}
\end{equation*}
%
%
The remainder of the proof is analogous to that in the non-periodic case.
\end{proof}
%Following Klainerman \cite{Klainerman:fk}, we introduce the following.  
%\begin{definition}
  %Let $T_{z}$ be a family of linear operators indexed by $z \in D$. We say the
  %$T_{z}$ are an analytic family of operators if
  %\begin{enumerate}
    %\item{$T_{z}$ maps simple functions into measurable functions}
    %\item{ The map $z \mapsto T_{z}$ is analytic in the interior of the strip
      %%
      %%
      %\begin{equation*}
      %\begin{split}
        %0 \le \text{Re}\, z \le 1
      %\end{split}
      %\end{equation*}
      %%
      %%
      %and bounded and continuous on the boundary.}
  %\end{enumerate}
%\end{definition}
%%
%%
%%%%%%%%%%%%%%%%%%%%%%%%%%%%%%%%%%%%%%%%%%%%%%%%%%%%%%
%%
%%
%%                stein-interp
%%
%%
%%%%%%%%%%%%%%%%%%%%%%%%%%%%%%%%%%%%%%%%%%%%%%%%%%%%%%
%%
%%
%\begin{lemma}[Stein Complex Interpolation]
  %Let $T_{z}$ be an analytic family of operators and assume there are positive
  %constants $M_{0}, M_{1}$ such that, for every $b \in \rr$
  %%
  %%
  %\begin{equation*}
  %\begin{split}
    %\| T_{ib} \|_{L^{q_{0}}} \le M_{0} \| f \|_{L^{p_{0}}}, \quad \|
    %T_{1 + ib} f \|_{L^{q_{1}}} \le M \| f \|_{L^{p_{1}}}
  %\end{split}
  %\end{equation*}
  %%
  %%
  %with $1 \le q_{0}, p_{0}, q_{1}, p_{1} \le \infty$. Then, for $z = a + ib \in
  %D$, $T_{z}$ extends to a bounded operator from $L^{p}$ to $L^{q}$ and
  %%
  %%
  %\begin{equation*}
  %\begin{split}
    %\| T_{z} f \|_{L^{q}} \le M_{0}^{1-a} M_{1}^{a} \| f \|_{L^{p}}
  %\end{split}
  %\end{equation*}
  %%
  %%
  %where
  %%
  %%
  %\begin{equation*}
  %\begin{split}
    %\frac{1}{p} = \frac{1-a}{p_{0}} + \frac{a}{p_{1}}, \quad \frac{1}{q} =
    %\frac{1-a}{q_{0}} + \frac{a}{q_{1}}.
  %\end{split}
  %\end{equation*}
  %%
  %%
%\label{lem:stein}
%\end{lemma}
%%
%%
%%
%To apply the lemma, we need some preliminaries. First, recall that for $r \in
%\rr$
%%
%%
%\begin{equation*}
%\begin{split}
  %(1 - \p_x^{2})^{r}h(x) 
  %\overset{\text{it.}}{=} & \int_{\rr} e^{ix
  %\xi}(1 + \xi^{2})^{r} \wh{h}(\xi) d \xi
  %\\
   %\overset{\text{abs. conv.}}{=}  & \lim_{j \to \infty} \frac{1}{2 \pi} \int_{\rr}
  %\int_{\rr} e^{i(x-y)}
  %\chi( \xi/j) (1 + \xi^{2})^{r}dy d \xi
%\end{split}
%\end{equation*}
%%
%%
%where $\chi(\xi)$ is a cutoff function symmetric about the origin. 
%%we fix $g \in H^{s}$ with $\| g\|_{H^{s}} = 1$
%%
%%
%%
%%
%
%
\begin{proof}[Proof of Lemma~\ref{lem:interp}]
 We have
 %
 %
 \begin{equation}
   \label{pre-hold}
 \begin{split}
   \| f \|_{H^{s}}^{2}
   & = \int_{\rr} (1 + \xi^{2})^{s} | \wh{f}(\xi) |^{2} d \xi
   \\
   & = \int_{\rr} | \wh{f}(\xi) | (1 + \xi^{2})^{s_{1}/p} | \wh{f}(\xi) |
   (1 + \xi^{2})^{s_{2}/q} 
 \end{split}
 \end{equation}
 %
 %
 where
 %
 %
 \begin{equation*}
 \begin{split}
   s_{1}/p + s_{2}/q =s, \quad 1/p + 1/q =1.
 \end{split}
 \end{equation*}
 %
 %
This implies 
 %
 %
 \begin{equation*}
 \begin{split}
   1/p = (s_{2} -s)/(s_{2} -s_{1}), \quad 1/q = (s -s_{1})/(s_{2} -s_{1}). 
 \end{split}
 \end{equation*}
 %
 %
 Applying H\"older to the right hand side of \eqref{pre-hold}, we obtain the
 bound
 %
 %
 \begin{equation*}
 \begin{split}
   \| f \|_{H^{s}}^{s}
   & \le  \left[ \int_{\rr} (1 + \xi^{2})^{s_{1}} | \wh{f}(\xi) |^{2} d \xi
   \right]^{(s_{2} - s)/(s_{2} -s_{1})} \left[ \int_{\rr} (1 + \xi^{2})^{s_{1}}
   | \wh{f}(\xi) |^{2} d \xi \right]^{(s - s_{1})/(s_{2} -s_{1})}
   \\
   & = \| f \|_{H^{s_{1}}}^{2 (s_{2} - s)/(s_{2} -s_{1})}
   \| f \|_{H^{s_{2}}}^{2 (s - s_{1})/(s_{2} -s_{1})}.
 \end{split}
 \end{equation*}
 %
 %
 Taking square roots of both sides completes the proof in the non-periodic case.
 The proof in the periodic case is analogous to the non-periodic proof (i.e.\
 integrals are replaced with sums in the proof).
  \end{proof}
\begin{proof}[Proof of Lemma~\ref{lem:calc}]
%
By the change of variable $x \mapsto x + (\alpha + \beta)/2$, we have
%
%
\begin{equation}
  \label{rry}
	\begin{split}
    \int_{\rr} \frac{1}{\langle x - \alpha \rangle^{p} \langle  x -
    \beta
    \rangle^{q}}d x
    & = \int_{\rr} \frac{1}{\langle x - (\alpha - \beta)/2  \rangle^{p}
    \langle  x + (\alpha - \beta)/2 \rangle^{q}} d x
    \\
    & \simeq \int_{\rr} \frac{1}{\langle x - (\alpha - \beta)  \rangle^{p}
    \langle  x + (\alpha - \beta) \rangle^{q}} d x.
  \\
  & = \int_{\rr} \frac{1}{\langle a - x \rangle ^{p} \langle a + x \rangle
  ^{q}} d x, \quad a = \alpha - \beta
\end{split}
\end{equation}
%
which for $a =0$ reduces to 
%
%
\begin{equation*}
\begin{split}
  \int_{\rr} \frac{1}{\langle x \rangle ^{p+q}} d x 
  & = 2 \int_{0}^{\infty} \frac{1}{(1 + x)^{p+q}} d x
  \\
  & = \frac{2}{p+q -1}
  \\
  & = \frac{2}{(p+q -1)\langle a \rangle}.
\end{split}
\end{equation*}
%
%
By symmetry, we now assume without loss of generality that $a > 0$ and split 
%
%
\begin{equation*}
\begin{split}
\int_{\rr} \frac{1}{\langle a + x \rangle ^{p} \langle a - x \rangle
  ^{q}} d x
  & = \int_{-2a}^{2a}
  \frac{1}{\langle a + x \rangle ^{p} \langle a - x \rangle
  ^{q}} d x
  \\
  & + \int_{| x | \ge 2a} 
\frac{1}{\langle a + x \rangle ^{p} \langle a - x \rangle
  ^{q}} d x
  \\
  & = I + II.
\end{split}
\end{equation*}
%
%
If $p=1$ and $q=1$, then 
%
%
\begin{equation*}
\begin{split}
  I
  & \le \sup_{-2a \le x \le 2a} \frac{1}{\langle a - x \rangle
} \int_{-2a}^{2a} \frac{1}{\langle a + x \rangle} d x
  \\
  & = \frac{1}{\langle a \rangle} \int_{-2a}^{2a} \frac{1}{(1 + | a -
  x
  |)} d x
  \\
  & = \frac{4}{\langle a \rangle} \int_{0}^{a} \frac{1}{(1 + a -
  x)} d x.
\end{split}
\end{equation*}
%
%
Integrating, we obtain
%
%
\begin{equation*}
 I
 \le 
 \frac{4 \log \langle a \rangle}{\langle a \rangle}, \qquad p =1, \ q =1.
\end{equation*}
Otherwise, assume that $q \neq 1$. Then
\begin{equation*}
\begin{split}
  I
  & \le \sup_{-2a \le x \le 2a} \frac{1}{\langle a + x \rangle
  ^{p}} \int_{-2a}^{2a} \frac{1}{\langle a - x \rangle ^{q}} d x
  \\
  & = \frac{1}{\langle a \rangle ^{p}} \int_{-2a}^{2a} \frac{1}{(1 + | a -
  x
  |)^{q}} d x
  \\
  & = \frac{4}{\langle a \rangle ^{p}} \int_{0}^{a} \frac{1}{(1 + a -
  x)^{q}} d x.
\end{split}
\end{equation*}
Evaluating the integral, we obtain
\begin{equation*}
  I \le \frac{4}{|q-1| \langle a \rangle ^{p +q -1}}, \qquad q \neq 1.
\end{equation*}
%
%
A similar computation yields
\begin{equation*}
  I \le \frac{4}{|q-1| \langle a \rangle ^{p +q -1}}, \qquad p \neq 1.
\end{equation*}
%
%
Also
%
%
\begin{equation*}
\begin{split}
  II 
  & \simeq \int_{x \ge 2a} \frac{1}{\langle a - x \rangle ^{p} \langle a
  + x \rangle ^{q} d x}
  \\ 
  & = \int_{x \ge 2a} \frac{1}{(1 + x - a)^{p} (1 + x +
  a)^{q}} d x
  \\
  & \le \int_{x \ge 2a} \frac{1}{(1 + x -a)^{p+q}} d x
  \\
  & = \frac{1}{[p + q-1] \langle a \rangle ^{p+q -1}}, \qquad p + q > 1.
\end{split}
\end{equation*}
%
%
Collecting our estimates for $I$ and $II$ we see that for 
$p, q > 0$ such that $p +q >1$, and $r =\min\left\{p, q, p+q-1
 \right\}$, we have 
%
\begin{align*}
  \int_{\rr} \frac{1}{\langle a - x \rangle ^{p} \langle a + x \rangle
  ^{q}} d x
  \le \frac{c_{p,q}}{\langle a \rangle ^{r}}, \qquad & a = 0 \ \text{or} \
  p \neq 1 \ \text{or} \ q \neq 1
  \\
   \int_{\rr} \frac{1}{\langle a - x \rangle  \langle a + x \rangle
} d x
  \le  \frac{4 \log \langle a \rangle}{\langle a \rangle}, \qquad & a \neq 0.
  \label{est-2}
\end{align*}
By symmetry, the second inequality also holds for $a < 0$. Recalling
\eqref{rry}, the proof is complete.
\end{proof}

  \section{Proof of Proposition~\ref{prop:sob-interp}}
\label{ssec:pf-sob-interp}
The proof follows immediately from the following two lemmas and the continuous
embedding $H^{s'} \subset H^{s}$, $s < s'$.
%
%
%%%%%%%%%%%%%%%%%%%%%%%%%%%%%%%%%%%%%%%%%%%%%%%%%%%%%
%
%
%                
%
%
%%%%%%%%%%%%%%%%%%%%%%%%%%%%%%%%%%%%%%%%%%%%%%%%%%%%%
%
%
\begin{lemma}
\label{lem:rest-bound}
Let $X, Y, U, V$ be Banach spaces with $U \subset X, V \subset Y$ continuously,
and suppose $T$ is a bounded linear operator from $X$ to $Y$. If $T$ maps $U$
to $V$, then $T$ is bounded from $U$ to $V$.
\end{lemma}
%
%
%
%%%%%%%%%%%%%%%%%%%%%%%%%%%%%%%%%%%%%%%%%%%%%%%%%%%%%
%
%
%               Interpolation spaces 
%
%
%%%%%%%%%%%%%%%%%%%%%%%%%%%%%%%%%%%%%%%%%%%%%%%%%%%%%
%
%
\begin{lemma}
\label{lem:interp-spaces}
For fixed $k \le q, \ m \le s$ suppose that $T: H^{k} \to H^{m}$
continuously and $T: H^{q} \to H^{s}$. Then
$T: H^{\theta q + (1 - \theta)k} \to H^{\theta s +
(1 - \theta) m}$ for all $\theta \in [0,1]$.
%
%
%
\end{lemma}
%
%
%
Hence, to complete the proof of Proposition~\ref{prop:sob-interp},
it will be enough to prove these lemmas.
%
\begin{proof}[Proof of Lemma~\ref{lem:rest-bound}]
We first recall the following.
\begin{definition}
Let $X$ and $Y$ be Banach spaces, and $T: X \to Y$ be linear. We say that $T$ is
\emph{closed} if 
%
%
%\begin{equation*}
%\begin{split}
  %\Gamma(T) = \left\{ (x, Tx) \in X \times Y \right\}
%\end{split}
%\end{equation*}
%%
%%
%is a closed subspace of $X \times Y$ in the product topology. That is,
%%
%%
%\begin{equation*}
%\begin{split}
  %\| x -x_{n} \|_{X} + \| Tx - Tx_{n} \|_{Y} \to 0.
%\end{split}
%\end{equation*}
%%
%%
%
$x_{n} \to x$ and $Tx_{n} \to y$ imply $y = Tx$. 
\end{definition}

%
%
%%%%%%%%%%%%%%%%%%%%%%%%%%%%%%%%%%%%%%%%%%%%%%%%%%%%%
%
%
%                closed graph theorem
%
%
%%%%%%%%%%%%%%%%%%%%%%%%%%%%%%%%%%%%%%%%%%%%%%%%%%%%%
%
%
\begin{lemma}[Closed Graph Theorem]
\label{lem:closed-graph}
If $X$ and $Y$ are Banach, and $T: X \to Y$ is a closed linear map, then $T$ is
bounded.
\end{lemma}
%
%
Proceeding with the proof of Lemma~\ref{lem:rest-bound},
suppose $x_{n} \to x$ in $U$; that is
%
%
\begin{equation}
  \label{1hh}
\begin{split}
  \| x - x_{n} \|_{U} \to 0.
\end{split}
\end{equation}
%
%
Since $U \subset X$
continuously and
$T: X \to Y$ is bounded, we have
%
%
\begin{equation}
  \label{2hh}
\begin{split}
  \| Tx - Tx_{n} \|_{Y} = \| T(x - x_{n}) \|_{Y} \lesssim \| x -
  x_{n} \|_{X} \le  \| x - x_{n} \|_{U} \to 0.
\end{split}
\end{equation}
%
%
Hence, from \eqref{2hh}, the continuous embedding $V \subset Y$, 
and the uniqueness of the limit,
it follows that if $Tx_{n} \to v$ in $V$, then $v = Tx$. 
Therefore, $T$ is closed from
$U$ to $V$. Applying the closed graph theorem concludes the proof. 
\end{proof}
%
%
%
\begin{proof}[Proof of Lemma~\ref{lem:interp-spaces}]
  We refer the reader to \cite{Taylor:1995kx}, pp.322-324.
\end{proof}
%
%
For additional reading on interpolation spaces, please consult
\cite{Taylor:1995kx}, \cite{bergh1976interpolation}, and \cite{Bennett:1988ys}.
%
%
\providecommand{\bysame}{\leavevmode\hbox to3em{\hrulefill}\thinspace}
\providecommand{\MR}{\relax\ifhmode\unskip\space\fi MR }
% \MRhref is called by the amsart/book/proc definition of \MR.
\providecommand{\MRhref}[2]{%
  \href{http://www.ams.org/mathscinet-getitem?mr=#1}{#2}
}
\providecommand{\href}[2]{#2}
\begin{thebibliography}{HKM09}

\bibitem[BBM72]{Benjamin_1972_Model-equations}
T.~B. Benjamin, J.~L. Bona, and J.~J. Mahony, \emph{Model equations for long
  waves in nonlinear dispersive systems}, Philos. Trans. Roy. Soc. London Ser.
  A \textbf{272} (1972), no.~1220, 47--78.

\bibitem[BC07]{Bressan_2007_Global-conserva}
A.~Bressan and A.~Constantin, \emph{Global conservative solutions of the
  camassa-holm equation}, Arch. Ration. Mech. Anal. \textbf{183} (2007), no.~2,
  215--239.

\bibitem[BCK06]{Bendahmane_2006_Hsp-1-perturbat}
M.~Bendahmane, G.~M. Coclite, and K.~H. Karlsen, \emph{$h^1$-perturbations of
  smooth solutions for a weakly dissipative hyperelastic-rod wave equation},
  Mediterr. J. Math. \textbf{3} (2006), no.~3-4, 419--432.

\bibitem[Ben72]{Benjamin_1972_The-stability-o}
T.~B. Benjamin, \emph{The stability of solitary waves}, Proc. Roy. Soc.
  (London) Ser. A \textbf{328} (1972), 153--183.

\bibitem[BL76]{bergh1976interpolation}
J.~Bergh and J.~L{\"o}fstr{\"o}m, \emph{Interpolation spaces: an introduction},
  vol. 223, Springer-Verlag, 1976.

\bibitem[BS88]{Bennett:1988ys}
C.~Bennett and R.~Sharpley, \emph{Interpolation of operators}, vol. 129,
  Academic Pr, 1988.

\bibitem[Che11]{Chen:2011fk}
R.~M. Chen, \emph{The h{\"o}lder continuity of the solution map to the
  $b$-family equation in weak topology}, 2011.

\bibitem[CHK05]{Coclite_2005_Global-weak-sol}
G.~M. Coclite, H.~Holden, and K.~H. Karlsen, \emph{Global weak solutions to a
  generalized hyperelastic-rod wave equation}, SIAM J. Math. Anal. \textbf{37}
  (2005), no.~4, 1044--1069 (electronic).

\bibitem[CS00]{Constantin_2000_Stability-of-a-}
A.~Constantin and W.~A. Strauss, \emph{Stability of a class of solitary waves
  in compressible elastic rods}, Phys. Lett. A \textbf{270} (2000), no.~3-4,
  140--148.

\bibitem[Dai98]{Dai_1998_Model-equations}
H.-H. Dai, \emph{Model equations for nonlinear dispersive waves in a
  compressible mooney-rivlin rod}, Acta Mech. \textbf{127} (1998), no.~1-4,
  193--207.

\bibitem[DDH00]{Dai_2000_Head-on-collisi}
H.-H. Dai, S.~Dai, and Y.~Huo, \emph{Head-on collision between two solitary
  waves in a compressible mooney-rivlin elastic rod}, Wave Motion \textbf{32}
  (2000), no.~2, 93--111.

\bibitem[DH00]{Dai_2000_Solitary-shock-}
H.-H. Dai and Y.~Huo, \emph{Solitary shock waves and other travelling waves in
  a general compressible hyperelastic rod}, R. Soc. Lond. Proc. Ser. A Math.
  Phys. Eng. Sci. \textbf{456} (2000), no.~1994, 331--363.

\bibitem[HKM09]{Himonas_2009_Non-uniform-dep-per}
A.~Himonas, C.~E. Kenig, and G.~Misiolek, \emph{Non-uniform dependence for the
  periodic ch equation.}, To appear in Communications in Partial Differential
  Equations (2009).

\bibitem[HR07]{Holden_2007_Global-conserva}
H.~Holden and X.~Raynaud, \emph{Global conservative solutions of the
  generalized hyperelastic-rod wave equation}, J. Differential Equations
  \textbf{233} (2007), no.~2, 448--484.

\bibitem[Kar10]{Karapetyan:2010fk}
D.~Karapetyan, \emph{Non-uniform dependence and well-posedness for the
  hyperelastic rod equation}, J. Differential Equations \textbf{249} (2010),
  no.~4, 796--826. \MR{2652154 (2011g:35352)}

\bibitem[Kat75]{Kato_1975_Quasi-linear-eq}
T.~Kato, \emph{Quasi-linear equations of evolution, with applications to
  partial differential equations}, 1975, pp.~25--70. Lecture Notes in Math.,
  Vol. 448.

\bibitem[Len06]{Lenells_2006_Traveling-waves}
J.~Lenells, \emph{Traveling waves in compressible elastic rods}, Discrete
  Contin. Dyn. Syst. Ser. B \textbf{6} (2006), no.~1, 151--167 (electronic).

\bibitem[Mus07]{Mustafa_2007_Global-conserva}
O.~G. Mustafa, \emph{Global conservative solutions of the hyperelastic rod
  equation}, Int. Math. Res. Not. IMRN (2007), no.~13, Art. ID rnm040, 26.

\bibitem[RB01]{Rodriguez-Blanco_2001_On-the-Cauchy-p}
G.~Rodr\'iguez-Blanco, \emph{On the cauchy problem for the camassa-holm
  equation}, Nonlinear Anal. \textbf{46} (2001), no.~3, Ser. A: Theory Methods,
  309--327.

\bibitem[Tay91]{Taylor_1991_Pseudodifferent}
M.~E. Taylor, \emph{Pseudodifferential operators and nonlinear pde}, vol. 100,
  1991.

\bibitem[Tay95]{Taylor:1995kx}
M.~E. Taylor, \emph{Partial differential equations i, basic theory}, Applied
  Mathematical Sciences, vol. 115, Springer-Verlag, New York, 1995.

\bibitem[Yin03]{Yin_2003_On-the-Cauchy-p}
Z.~Yin, \emph{On the cauchy problem for a nonlinearly dispersive wave
  equation}, J. Nonlinear Math. Phys. \textbf{10} (2003), no.~1, 10--15.

\bibitem[Zho05]{Zhou_2005_Local-well-pose}
Y.~Zhou, \emph{Local well-posedness and blow-up criteria of solutions for a rod
  equation}, Math. Nachr. \textbf{278} (2005), no.~14, 1726--1739.

\end{thebibliography}
%
%
%
%
%\bibliographystyle{amsalpha-custom}
%\bibliography{/Users/davidkarapetyan/math/bib-files/references}

\end{document}

\appendix
\chapter{Calculus on Banach Spaces}
\chapter{Calculus in Metric Spaces}
\section{Introduction}
%
%
%
%%%%%%%%%%%%%%%%%%%%%%%%%%%%%%%%%%%%%%%%%%%%%%%%%%%%%
%
%
%				Notation
%
%
%%%%%%%%%%%%%%%%%%%%%%%%%%%%%%%%%%%%%%%%%%%%%%%%%%%%%
%
%
%
%


Our goal will be to prove the following.
%
%
%%%%%%%%%%%%%%%%%%%%%%%%%%%%%%%%%%%%%%%%%%%%%%%%%%%%%
%
%
%				Existence Theorem
%
%
%%%%%%%%%%%%%%%%%%%%%%%%%%%%%%%%%%%%%%%%%%%%%%%%%%%%%
%
%
\begin{theorem}[Metric Space ODE Theorem]
	\label{ode-thm}
  Let $X$ be a topological vector space over $\rr$
  with topology induced by a metric $d$, $E \subset X$ open, $\vp \in E$, and $(-a, a)$ an
	open interval in $\rr$. If $f: (-a, a) \times E \to X$ satisfies the
	inequality
	%
	%
	\begin{equation}
		\label{stronger-ode}
		\begin{split}
      d[f(t, x), f(t, y)] \le c d(x, y), \qquad \forall t \in (-a, a),
			\qquad \forall x, y \in E,
		\end{split}
	\end{equation}
	%
  then for sufficiently small $h > 0$ there exists a unique
	differentiable map $u: (-h, h) \to E$ such that for all $t \in (-h, h)$
	%
	%
	\begin{gather}
    \label{ode-thm-eq}
			u'(t) = f(t, u(t)),
			\\
      \label{ode-thm-init-data}
			u(0) = \vp.
	\end{gather}
\end{theorem}
%
%
Notice that \autoref{ode-thm} is a 
generalization of the following classical result.
%
%
%%%%%%%%%%%%%%%%%%%%%%%%%%%%%%%%%%%%%%%%%%%%%%%%%%%%%
%
%
%				 Picard-Lindelof
%
%
%%%%%%%%%%%%%%%%%%%%%%%%%%%%%%%%%%%%%%%%%%%%%%%%%%%%%
%
%
\begin{theorem}[Picard-Lindel\"{o}f]
	Let $f(t, x)$ be a continuous function on $(- a, a) \times (- b,
	b)$ such that
	%
	%
	\begin{equation*}
		\begin{split}
			| f(t, x) - f(t, y) | \le c| x - y |, \quad \forall t \in (-a, a),
			\quad \forall x,y \in (- b, b).
		\end{split}
	\end{equation*}
	%
	%
	Then for any $\xi \in (-b, b)$ there exists an interval $(-h, h)
	\subset (-a, a)$ such that the initial value problem
	%
	%
	\begin{gather}
			\frac{dx}{dt} = f(t, x(t)),
			\\
			x(0) = \xi 
	\end{gather}
	%
	%
	has a unique solution $x(t)$ for all $t \in (-h, h)$.
	%
	%
	\end{theorem}
To prove \autoref{ode-thm}, we will first define
integration over Banach space. (To extend the theory of integration
to arbitrary complete topological vector spaces with topology induced by a
metric, one need
only replace norms with metrics in the definitions. We leave this to the
reader.) Then we will rewrite
\eqref{ode-thm-eq} as an integral equation. A fixed point argument will complete
the proof. These notes are heavily influenced by a
number of excellent books, among them Dieudonn{\'e}
\cite{Dieudonne_1969_Foundations-of-}, Yosida \cite{Yosida:1980fk},
Folland \cite{Folland_1999_Real-analysis}, Jost
\cite{Jost-1998-Postmodern-analysis}, Rudin \cite{Rudin:1976uq}, and Knapp
\cite{Knapp:2005rm}-\cite{Knapp:2005yg}.
%
%
%%%%%%%%%%%%%%%%%%%%%%%%%%%%%%%%%%%%%%%%%%%%%%%%%%%%%
%
%
%				 Differentiaion in Banach Spaces
%
%
%%%%%%%%%%%%%%%%%%%%%%%%%%%%%%%%%%%%%%%%%%%%%%%%%%%%%
%
%
%%%%%%%%%%%%%%%%%%%%%%%%%%%%%%%%%%%%%%%%%%%%%%%%%%%%%
%
%
%				 Integration in Banach Spaces
%
%
%%%%%%%%%%%%%%%%%%%%%%%%%%%%%%%%%%%%%%%%%%%%%%%%%%%%%
%
%
\section{Integration in Banach Spaces}
\label{sec:int-banach}
%
%
%%%%%%%%%%%%%%%%%%%%%%%%%%%%%%%%%%%%%%%%%%%%%%%%%%%%%
%
%
%				Definition of step functions
%
%
%
%
%
%
%%%%%%%%%%%%%%%%%%%%%%%%%%%%%%%%%%%%%%%%%%%%%%%%%%%%%
%
%
%				Definition of step function
%
%
%%%%%%%%%%%%%%%%%%%%%%%%%%%%%%%%%%%%%%%%%%%%%%%%%%%%%
%
%
\begin{definition}
	A mapping $f:[a,b] \to X$ is a \emph{step-function} if there exists a disjoint
	partitioning $\left\{ (t_{j}, t_{j+1}) \right\}$ of $[a,b]$ such that
	%
	%
	\begin{equation}
		\label{step-function}
		\begin{split}
			f(t)=\sum_{j=0}^{n-1} {\alpha_j} \chi_{(t_{j}, t_{j+1})}.
		\end{split}
	\end{equation}
	%
	%
\end{definition}
%
%
\begin{framed}
\begin{example}
	The map $f:[-1,1] \to L^2(\rr)$ given by 
	%
	%
	\begin{equation*}
		\begin{split}
			f(t) = 
			\begin{cases}
				 e^{-x^2},  \quad & -1 \le t\le0 \\
				 e^{-2x^2},  \quad & \phantom - 0 < t \le 1 \\
				 0,  \quad & \phantom - \text{otherwise}
			\end{cases}
		\end{split}
	\end{equation*}
	%
	%
	is a step-function.
	\end{example}
\end{framed}
%
%
\begin{definition}[Banach Space Integration for Step Functions]
	Let $X$ be a Banach space, $f: [a,b] \to
	X$ a step function as in \eqref{step-function}. 
	Then 
	%
	\begin{equation*}
		\begin{split}
      \int_a^b f(t) dt \doteq \sum_{j=0}^{n-1} \alpha_j (t_{j+1} - t_{j}).
    \end{split}
	\end{equation*}
	%
	%
%
\end{definition}
%
%
Note that this definition is independent of the partitioning we use for $f$, due
to the following lemma.
%
%
%%%%%%%%%%%%%%%%%%%%%%%%%%%%%%%%%%%%%%%%%%%%%%%%%%%%%
%
%
%                Indpendence of Partioning
%
%
%%%%%%%%%%%%%%%%%%%%%%%%%%%%%%%%%%%%%%%%%%%%%%%%%%%%%
%
%
\begin{lemma}
For step functions $f, g$, we have
%
%
\begin{equation*}
\begin{split}
  \int_{a}^{b} f(t) dt - \int_{a}^{b} g(t) dt = \int_{a}^{b}\left[ f(t) - g(t) \right]dt.
\end{split}
\end{equation*}
%
%
\label{lem:indep-part}
\end{lemma}
%
%
\begin{proof}
%
Let
%
\begin{equation*}
\begin{split}
  & f = \sum_{j=1}^{m}\alpha_{j} \chi_{(t_{j}, t_{j+1})}
  \\
  & g = \sum_{k =1}^{n}
  \beta_{k} \chi_{(s_{k}, s_{k+1})}
\end{split}
\end{equation*}
%
and assume without loss of generality that $n \ge m$. 
Then
%
%
\begin{equation*}
\begin{split}
  f(t) - g(t) = \sum_{p=1}^{m+n} \gamma_{p} \chi_{(y_{p}, y_{p+1})} 
\end{split}
\end{equation*}
%
%
where
\begin{equation*}
  \begin{split}
    \gamma_{p} \chi_{(y_{p}, y_{p+1}) } = 
    \begin{cases}
      \alpha_{p} \chi_{(t_{p}, t_{p+1})}
, \quad & 1 \le p \le m \\
-\beta_{p-m} \chi_{(s_{p-m}, s_{p-m+1})}, \quad & m < p \le m+n.
\end{cases}
\end{split}
\end{equation*}
Then
%
%
%
\begin{equation*}
\begin{split}
  \int_{a}^{b}\left[ f(t) - g(t) \right]dt
  & = \sum_{p=1}^{m+n}
  \gamma_{p}(y_{p+1} - y_{p})
  \\
  & = \sum_{p=1}^{m} \alpha_{p}(t_{p+1} - t_{p}) +
  \sum_{p=m+1}^{m+n}-\beta_{p-m}(s_{p-m+1} - s_{p-m})
  \\
  & = \sum_{j=1}^{m}\alpha_{j}(t_{j+1} - t_{j}) -
  \sum_{k=1}^{n}\beta_{k}(s_{k+1} - s_{k})
  \\
  & = \int_{a}^{b}f(t)dt - \int_{a}^{b} g(t) dt.
\end{split}
\end{equation*}
%
%
%
%
\end{proof}
%
We are now interested in extending integration to a larger class of functions. We will need the
following. %
%
\begin{lemma}
	\label{lem:dense}
	Let $X$ be a Banach space, and
	suppose the function $f:[a,b] \to X$ is continuous. Then
	there exists a sequence of step functions $\left\{ f_n \right\}$
	such that $\|f - f_n\|_{L^\infty( [a,b], X)} \to 0$. 
\end{lemma}
%
%
\begin{proof}
  Fix $n \in \mathbb{N}$. Since $f$ is continuous, for every $t \in
\left[ a,b \right]$ there exists an open interval $V_t \subset [a,b]$ containing
$t$ such that
$\|f(x) - f(y)) \|_X \le 1/n$ for all $x, y \in V_t$. Note that $\left\{ V_t
\right\}_{t \in \left[ a,b \right]}$ is an open cover of $[a,b]$. We assume
without loss of generality that this cover is disjoint (since for any two
intervals $A$ and $B$ which overlap, we can replace $A$ with $A \setminus
B$ and still remain with an open cover of $[a,b]$). 
Due to the compactness of $[a,b]$, we can extract
a finite subcover $\left\{ (t_j, t_{j +1}) \right\}$ of $[a,b]$. This subcover
will be disjoint, since the original covering of $[a,b]$ was assumed to be
disjoint. For each interval $(t_j, t_{j +1})$
choose any $t_{j}^* \in (t_{j}, t_{j +1})$, and define the step-function
%
%
\begin{equation*}
	\begin{split}
		f_n(t) =
		\begin{cases}
      f(t_{j}^{*}), \quad & t \in (t_{j},t_{j+1}) 
		\\
		0 & \text{otherwise}.
	\end{cases}
	\end{split}
\end{equation*}
%
Fix $t \in [a,b]$. By construction, $t \in (t_{j}, t_{j+1})$ for some $j$.
Furthermore, since $\{(t_{j}, t_{j+1}) \} \subset \{V_{t}\}_{t \in [a,b]}$, we must have $(t_{j}, t_{j+1}) = V_{t}$. Hence
%
\begin{equation*}
	\begin{split}
		\|f(t)-f_n(t)\|_X
    & = \| f(t) - f(t_{j}^{*}) + f(t_{j}^{*}) - f_{n}(t)  \|_{X}
    \\
    & \le \| f(t) - f(t_{j}^{*}) \|_{X} + \| f(t_{j}^{*}) - f_{n}(t) \|_{X}
    \\
    & = \| f(t) - f(t_{j}^{*}) \|_{X}
    \\
    & \le \frac{1}{n}.
	\end{split}
\end{equation*}
%
%
Since $t$ was chosen arbitrarily, the proof is complete. 
\end{proof}
%
\begin{framed}
\begin{remark}
  \label{rem:unif}
If $f_{n}: [a,b] \to X$ is any sequence of step functions converging pointwise
in $t$ to $f$ in $X$, it converges \emph{uniformly} in $t$, due to the
compactness of $[a,b]$. 
\end{remark}
\end{framed}
%
\begin{definition}[Banach Space Integration for Continuous Functions]
  \label{def:banach}
  Let $X$ be a Banach space, $f: [a,b] \to X$ continuous, and $f_{n} \to f$ in
  $L^{\infty}([a,b], X)$. Then 
  %
  %
  \begin{equation*}
  \begin{split}
    \int_{a}^{b} f(t) dt \doteq \lim_{n \to \infty} \int_{a}^{b} f_{n}(t) dt
  \end{split}
  \end{equation*}
  %
  %
%
where the convergence is in the topology of $X$. 
\end{definition}
We now show that this
definition for Banach space integration is
well-defined. More precisely, we will first show that $\int_{a}^{b}
f_{n}(t) dt$ is Cauchy in $X$. Then, we will show the convergence is independent
of the step functions chosen. Proceeding, we first prove the following lemma.
%
%
%
%
%%%%%%%%%%%%%%%%%%%%%%%%%%%%%%%%%%%%%%%%%%%%%%%%%%%%%
%
%
%                tri inequality
%
%
%%%%%%%%%%%%%%%%%%%%%%%%%%%%%%%%%%%%%%%%%%%%%%%%%%%%%
%
%
\begin{lemma}
  For a step function $f: [a,b] \to X$, we have
  %
  %
  \begin{equation*}
  \begin{split}
    \| \int_{a}^{b} f(t) dt \|_{X} \le \int_{a}^{b} \| f(t) \|_{X} dt.
  \end{split}
  \end{equation*}
  %
  %
\label{lem:tri-ineq-int}
\end{lemma}
%
%
%
%
\begin{proof}
%
%
\begin{equation*}
\begin{split}
  \| \int_{a}^{b} f(t) dt \|_{X} &
  = \| \sum_{j=1}^{m}\alpha_{j} (t_{j+1} - t_{j})
  \|_{X}
  \\
  & \le  \sum_{j=1}^{m} \| \alpha_{j} \|_{X} (t_{j+1} - t_{j})
  \\
  & = \int_{a}^{b} \sum_{j=1}^{m} \chi_{[t_{j}, t_{j+1}]} \| \alpha_{j} \|_{X} dt
  \\
  & = \int_{a}^{b} \| \sum_{j=1}^{m} \alpha_{j} \chi_{[t_{j}, t_{j+1}]}(t)
  \|_{X}dt \quad \text{(the intervals $[t_{j}, t_{j+1}]$ are disjoint)} 
  \\
  & = \int_{a}^{b} \| f(t) \|_{X} dt
\end{split}
\end{equation*}
%
which completes the proof.
%
\end{proof}
%
%
%
%
%%%%%%%%%%%%%%%%%%%%%%%%%%%%%%%%%%%%%%%%%%%%%%%%%%%%%
%
%
%                uniqueness
%
%
%%%%%%%%%%%%%%%%%%%%%%%%%%%%%%%%%%%%%%%%%%%%%%%%%%%%%
%
Definition \ref{def:banach} is then well-defined due to the following two
lemmas.
%
%
%
%
%%%%%%%%%%%%%%%%%%%%%%%%%%%%%%%%%%%%%%%%%%%%%%%%%%%%%
%
%
%                conv
%
%
%%%%%%%%%%%%%%%%%%%%%%%%%%%%%%%%%%%%%%%%%%%%%%%%%%%%%
%
%
\begin{lemma}[Convergence] Let $f_{n}: [a,b] \to X$ be a sequence of simple
  functions converging to $f$ in $L^{\infty}([a,b], X)$. Then
  %
  %
  \begin{equation*}
  \begin{split}
    \int_{a}^{b}f_{n}(t) dt
  \end{split}
  \end{equation*}
  %
  %
  is Cauchy in $X$. 
\label{lem:conv}
\end{lemma}
%
%
%
%
\begin{proof}
  \begin{equation}
    \begin{split}
  \| \int_{a}^{b} f_{n}(t) dt - \int_{a}^{b} f_{m}(t) dt \|_{X}
  & = \| \int_{a}^{b}\left[ f_{n}(t) - f_{m}(t) \right]dt \|_{X}, \quad \text{Lemma
  \ref{lem:indep-part}}
  \\
  & \le \int_{a}^{b} \| f_{n}(t) - f_{m}(t) \|_{X} dt,
  \quad \phantom{m} \text{Lemma \ref{lem:tri-ineq-int}}
  \\
  & \le \int_{a}^{b} (\| f(t) - f_{n}(t) \|_{X} + \| f(t) - f_{m}(t) \|_{X})dt.
\end{split}
\end{equation}
%
Fix $\ee$. By Remark \ref{rem:unif}, we see that there exists $N$ such
that for $m, n > N$
%
%
\begin{equation*}
\begin{split}
& \| f(t) - f_{n}(t) \|_{X} \le \ee, \quad \forall t \in [a, b].
\\
& \| f(t) - f_{m}(t) \|_{X} \le \ee, \quad \forall t \in [a, b].
\end{split}
\end{equation*}
Hence,
%
%
\begin{equation*}
\begin{split}
\| \int_{a}^{b} f_{n}(t) dt - \int_{a}^{b} f_{m}(t) dt \|_{X} \le 2(b-a) \ee,
\quad m,n > N.
\end{split}
\end{equation*}
%
Since $\ee$ can be chosen arbitrarily small, the proof is complete.
\end{proof}
%
%
\begin{lemma}[Independence]
  Let $f_{n}(t)$, $g_{n}(t) : [a,b] \to X$
  be two step functions converging to $f$ in $L^{\infty}([a,b], X)$.
  If
  %
  %
  \begin{equation*}
  \begin{split}
    & \int_{a}^{b} f_{n}(t) dt \xrightarrow{X} F
    \\
    & \int_{a}^{b} g_{n}(t) dt \xrightarrow{X} G
  \end{split}
  \end{equation*}
  %
  %
  then
  %
  %
  \begin{equation*}
  \begin{split}
    F =G.
  \end{split}
  \end{equation*}
  %
  %
\label{lem:well-def}
\end{lemma}
%
%
%
%
%
%
\begin{proof}
%
%
\begin{equation*}
\begin{split}
  \| F - G \|_{X} & = \| \int_{a}^{b} f(t) dt - \int_{a}^{b} g(t) dt \|_{X}
  \\
  & = \| \int_{a}^{b}\left[ f(t) - g(t) \right]dt \|_{X}, \quad \text{Lemma
  \ref{lem:indep-part}}
  \\
  & \le \int_{a}^{b} \| f(t) - g(t) \|_{X} dt,
  \quad \phantom{m} \text{Lemma \ref{lem:tri-ineq-int}}
  \\
  & \le \int_{a}^{b} (\| f(t) - f_{n}(t) \|_{X} + \| f(t) - g_{n}(t) \|_{X})dt.
\end{split}
\end{equation*}
%
%
Fix $\ee$. By Remark \ref{rem:unif}, we see that there exists $N$ such
that for $n > N$
%
%
\begin{equation*}
\begin{split}
& \| f(t) - f_{n}(t) \|_{X} \le \ee, \quad \forall t \in [a, b]
\\
& \| f(t) - g_{n}(t) \|_{X} \le \ee, \quad \forall t \in [a, b].
\end{split}
\end{equation*}
%
%
Hence
%
%
\begin{equation*}
\begin{split}
\| F - G \|_{X} \le 2(b-a) \ee.
\end{split}
\end{equation*}
%
%
Since $\ee$ can be chosen arbitrarily small, we see that we must have $F = G$.
\end{proof}
%
%
We conclude the section by proving the fundamental theorem of calculus, 
thereby establishing that Banach space integration and
differentiation are inverse operations. We shall need the following lemma.
%
%
%
%
%%%%%%%%%%%%%%%%%%%%%%%%%%%%%%%%%%%%%%%%%%%%%%%%%%%%%
%
%
%                int prop
%
%
%%%%%%%%%%%%%%%%%%%%%%%%%%%%%%%%%%%%%%%%%%%%%%%%%%%%%
%
%
\begin{lemma}
  Let $X$ be a Banach space, and suppose $f: [a,b] \to X$ is continuous. Then for
  $a \le b \le c$ 
%
%
\begin{equation*}
\begin{split}
  \int_{a}^{c} f(t) dt = \int_{a}^{b} f(t) dt + \int_{b}^{c} f(t) dt.
\end{split}
\end{equation*}
%
%
\label{lem:int-splitting}
\end{lemma}
%
%
%
%
\begin{proof}
  %
  Since $\| f - f_{n} \|_{L^{\infty}\left( [a,b], X \right)} \to 0$ for some sequence
  $f_{n}$ of simple functions, we have
  %
  \begin{equation*}
  \begin{split}
    \int_{a}^{c} f(t) dt = \lim_{n \to \infty} \int_{a}^{c} f_{n}(t) dt.
  \end{split}
  \end{equation*}
  %
  %
  Recall that the value of $\int_{a}^{c}f_{n}(t) dt$ does not depend upon the
  partioning we use for $f_{n}(t)$. Partition $f_{n}$ such that
  \begin{equation}
    \begin{split}
  & f_{n}(t) = f_{n, [a,b]} + f_{n, (b, c]},  
    \\
    & f_{n, [a,b]}(t) =0, \quad b< t \le c, \quad f_{n, (b, c]}(t) = 0, \quad
    a \le t \le b.
\end{split}
\end{equation}
Then
  %
  %
  \begin{equation*}
  \begin{split}
    \int_{a}^{c} f_{n}(t) dt & = \int_{a}^{c} f_{n, [a,b]}(t)dt  +
    \int_{a}^{c} f_{n, (b,c]}(t)dt 
    \\
    & = \int_{a}^{b} f_{n}(t)dt  +
    \int_{b}^{c} f_{n}(t)dt.
  \end{split}
  \end{equation*}
  %
  %
Hence 
 %
 %
 \begin{equation*}
 \begin{split}
   \int_{a}^{c} f(t) dt 
   & = \lim_{n \to \infty} \int_{a}^{c} f_{n}(t) dt
   \\
   &= \lim_{n \to \infty} \left ( \int_{a}^{b} f_{n}(t)dt  +
    \int_{b}^{c} f_{n}(t)dt \right )
    \\
    & =\int_{a}^{b} f(t) dt + \int_{b}^{c} f(t) dt
 \end{split}
 \end{equation*}
 %
 %
which concludes the proof.
\end{proof}
%
%
%%%%%%%%%%%%%%%%%%%%%%%%%%%%%%%%%%%%%%%%%%%%%%%%%%%%%
%
%
%                FTC
%
%
%%%%%%%%%%%%%%%%%%%%%%%%%%%%%%%%%%%%%%%%%%%%%%%%%%%%%
%
%
\begin{lemma}[Fundamental Theorem of Calculus]
Let $f \in C\left( [a,b], X \right)$, where $X$ is a Banach space. For $a \le x
\le b$, let
%
%
\begin{equation*}
\begin{split}
  F(t) = \int_{a}^{t}f(s)ds.
\end{split}
\end{equation*}
%
%
Then $F$ is differentiable at all $t_{0} \in [a,b]$, and
%
%
\begin{equation}
  \label{fund-thm-calc-diff}
\begin{split}
  F'(t_{0})= f(t_{0}).
\end{split}
\end{equation}
\label{lem:fund-thm-calc}
\end{lemma}
%
%
%
%
\begin{proof}
We apply the definition of Banach space
differentiation given in \eqref{diff-limit-simp} and see that
%
%
\begin{equation*}
\begin{split}
   \| \frac{F(t_{0} + h) - F(t_{0})}{h}  - f(t_{0}) \|_{X}
  & = \| \frac{\int_{a}^{t_{0}+h}f(s)ds - \int_{a}^{t_{0}}f(s)ds}{h} -
  f(t_{0}) \|_{X}
  \\
  & = \| \int_{a}^{t_{0}+h}\frac{f(s)}{h}ds  -
  \int_{t_{0}}^{t_{0} + h} \frac{f(t_{0})}{h} ds \|_{X}
  \\
  & = \frac{1}{h} \| \int_{t_{0}}^{t_{0} + h}[f(s) - f(t_{0})]ds \|_{X}, \qquad
  \text{Lemma \ref{lem:int-splitting}}
  \\
  & \le \frac{1}{h} \int_{t_{0}}^{t_{0} + h} \| f(s) - f(t_{0})
  \|_{L^{\infty}\left( [t_{0}, t_{0} + h], X \right)}
  \\
  & = \| f(s) - f(t_{0})
  \|_{L^{\infty}\left( [t_{0}, t_{0} + h], X \right)}
  \\
  & = \sup_{0 \le k \le 1} \| f(t_{0} + kh) - f(t_{0})
  \|_{X}.
\end{split}
\end{equation*}
%
%
Note that since $f$ is continuous on $[a,b]$, and $[a,b]$ is compact, it follows
that $f$ is uniformly continuous on $[a,b]$. Hence, for fixed $\delta > 0$,
there exists $\ee_{\delta} > 0$ such that for $h < \ee_{\delta}$, we have
%
%
\begin{equation*}
\begin{split}
  \| f(t_{0} + kh) - f(t_{0}) \|_{X} < \delta, \ \text{for all } \  0 \le k \le 1.
\end{split}
\end{equation*}
%
%
Hence
%
%
%
\begin{equation*}
\begin{split}
  \sup_{0 \le k \le 1} \| f(t_{0} + kh) - f(t_{0}) \|_{X} < \delta, \quad h <
  \ee_{\delta}.
\end{split}
\end{equation*}
%
%
Since $\delta$ can be chosen arbitrarily small, we conclude that
%
\begin{equation*}
\begin{split}
  \lim_{h \to 0} \sup_{0 \le k \le 1} \| f(t_{0} + kh) - f(t_{0})
  \|_{X} = 0,
\end{split}
\end{equation*}
%
%
concluding the proof.
\end{proof}
%
%
%
%
%
%
%
%
%%%%%%%%%%%%%%%%%%%%%%%%%%%%%%%%%%%%%%%%%%%%%%%%%%%%%
%
%
%				 Contraction Mapping Theorem
%
%
%%%%%%%%%%%%%%%%%%%%%%%%%%%%%%%%%%%%%%%%%%%%%%%%%%%%%
%
%
\section{Contraction Mapping Lemma}
\begin{definition}
  Let $\left( X, d \right)$ be a metric space. A mapping $T: X \to X$ is called a
\emph{contraction} if there exists an $\alpha$, $0 \le \alpha <1$ such that
%
%
\begin{equation*}
	\begin{split}
		d(Tx, Ty) \le \alpha d(x,y), \qquad \forall x, y \in X.
	\end{split}
\end{equation*}
%
\end{definition}
%
\begin{framed}
\begin{remark}
	Observe that a contraction mapping is always continuous.
\end{remark}
\end{framed}
%
%
\begin{framed}
\begin{example}
	The function $T(x) = 0.1x^2$ defines a contraction mapping in the set 
	$X = [-4, 4]$ equipped with the metric $d(x,y) = |x-y|$. 
\end{example}
\end{framed}
	%
	%
	%
	%
	%%%%%%%%%%%%%%%%%%%%%%%%%%%%%%%%%%%%%%%%%%%%%%%%%%%%%
	%
	%
	%				Banach Space Fixed Point Theorem 
	%
	%
	%%%%%%%%%%%%%%%%%%%%%%%%%%%%%%%%%%%%%%%%%%%%%%%%%%%%%
	%
	%
	\begin{lemma}[Contraction Mapping Lemma]
		\label{lem:fixed-point}
	Let $(X,d)$ be a complete metric space, and $T: X \to X$ a contraction
	mapping. Then $T$ has a unique fixed point in $X$. That is, there is a unique
	point $x^* \in X$ such that $Tx^* = x^*$. Furthermore, if $x_0$ is any point
  in $X$, and we define the sequence $x_{n+1} = Tx_n$, then $x_n \xrightarrow{X} x^*$ as $n
	\to \infty$.
	\end{lemma}
	%
	%
  \begin{proof} First we show uniqueness. If $x^*$ and $x^{**}$ are two fixed
	points, then
	%
	%
	\begin{equation*}
		\begin{split}
			d(x^*, x^{**}) = d(Tx^*, Tx^{**}) \le \alpha d(x^*, x^{**}) \implies d(x^*,
			x^{**}) = 0 \implies x^* = x^{**}.
		\end{split}
	\end{equation*}
	%
	%
To prove existence, we observe that since $X$ is complete it suffices to show
that $x_n$ is Cauchy. A repeated application of the
contraction inequality gives
%
%
\begin{equation*}
	\begin{split}
		d\left( x_{n+1},x_n \right)
		& = d\left( Tx_n, Tx_{n-1} \right)
		\\
		& \le \alpha d\left( x_n, x_{n-1} \right)
		\\
		& \le \alpha^2 d\left( x_{n-1}, x_{n-2} \right)
		\\
		& \cdots
		\\
		& \le \alpha^n d\left( x_1, x_0 \right).
	\end{split}
\end{equation*}
%
%
Hence
%
%
\begin{equation*}
\begin{split}
  d\left( x_{n+k},x_n \right)
  & \le (\alpha^{n } +\alpha^{n+1} + \cdots +
  \alpha^{n+k-2} + \alpha^{n+k-1})d(x_{1}, x_{0}) 
  \\
  & = \alpha^{n}(1 + \alpha + \cdots + \alpha^{k-2} + \alpha^{k-1})
  \\
  & \le \alpha^{n}\left( \frac{1}{1 - \alpha} \right)
  \\
  & \to 0 \ \text{as} \ n \to \infty
\end{split}
\end{equation*}
%
%
since $0 \le \alpha < 1$. 
\end{proof}
%
%%%%%%%%%%%%%%%%%%%%%%%%%%%%%%%%%%%%%%%%%%%%%%%%%%%%%
%
%
%				 Proof of ODE Theorem
%
%
%%%%%%%%%%%%%%%%%%%%%%%%%%%%%%%%%%%%%%%%%%%%%%%%%%%%%
%
%
\section{Proof of Metric Space ODE Theorem}
	%
	%
	Since $f(t)$ is Lipschitz continuous on $E$ for $t \in (-a, a)$, it is
	continuous on $E$. Hence, using the theory of metric space integration
  developed earlier and the fundamental theorem of calculus,
  one can check that the initial value problem
  \eqref{ode-thm-eq}-\eqref{ode-thm-init-data} is equivalent to the
	integral equation
	%
	%
	\begin{equation*}
		\begin{split}
			u(t) = \vp + \int_0^t f(s, u(s) ) \ ds.
		\end{split}
	\end{equation*}
	%
	%
	%
	%
	Let $[-h, h] \subset [-a/2, a/2]$ be a closed interval, with $t \in [-h, h]$,
	and $r$ chosen such that
	$$B_r(\vp) \doteq \left\{ u \in X: d(\vp, u) \le r \right\}$$
	is a subset of $E$. Define 
	$$V_h = \left\{ \text{all maps} \; \;  v: [-h, h]
    \to B_r(\vp)\right\}.$$ Then $(V_h, \mathfrak{d})$ is a complete metric space
    where the metric $\mathfrak{d}$ is defined by
    $$\mathfrak{d}(v_{1}, v_{2}) = \sup_{s \in [-h,
    h]} d[v_1(s),v_2(s)].$$ 
    We shall also use the notation
    %
    %
    \begin{equation*}
    \begin{split}
    ||| v ||| = \sup_{s \in [-h,
    h]} d[v(s),0]. 
    \end{split}
    \end{equation*}
    %
    %
    Let $T$ be a map
	acting on $V_h$ via the relation $$Tv(t) = \vp + \int_0^t f(s, v(s) ) \ ds.$$
	Recalling \autoref{lem:fixed-point}, we see that to complete the proof it will be enough to show that $T$ is a contraction on
	$V_h$, for suitably small $h >0$. First, note that for $v \in V_h$, we have
	%
	%
\begin{equation}
	\label{cont-map-into}
	\begin{split}
		| | | Tv - \vp| | |
		& = | | | \int_0^t f(s, v(s) ) \ ds | | |
		\\
    & \le  \int_0^{|t|} | | | f(s, v(s) ) | | | ds
    \end{split}
    \end{equation}
To estimate the integrand, we recall that $f(t, x)$ is continuous in the time
variable in the interval $(-a, a)$. Hence, for $t \in [-a/2, a/2]$ and fixed $x
\in E$, we have
%
%
\begin{equation*}
\begin{split}
d(f(t, x), 0) \le M
\end{split}
\end{equation*}
%
%
for some constant $M > 0$. Using this and the Lipschitz continuity of $f(t, x)$
in the spatial variable on $E$, we see that
%
%
\begin{equation*}
\begin{split}
| | | f(s, v(s) ) | | | 
& \le | | | f(s, v(s)) - f(s, v(0)) ||| + ||| f(s, v(0)) |||
\\
& \le c \sup_{s \in [-h, h]}  d(v(s), v(0)) +  M
\\
& \le 2cr + M.
\end{split}
\end{equation*}
%
%
Substituting this relation into \eqref{cont-map-into}, we see that
\begin{equation*}
  \begin{split}
| | | Tv - \vp| | |
& \le 
\int_0^{|t|} [2cr + M ] ds
    \\
    & \le  |t| [ 2cr + M]
		\\
    & \le  h [ 2cr + M].
\end{split}
\end{equation*}
%
%
%
Choosing $h \le 1/[2cr + M]$, it follows that $T: V_h \to V_h$. 
%Since $[-h, h] \subset (-a, a)$, we can find a compact set $C$ such that $(-h,
%h) \subset C \subset (-a, a)$. Since $f$ is a continuous function of two
%variables, $f( (C \times K) )$ is compact, and so \eqref{cont-map-into} gives
%the estimate
%%
%%
%\begin{equation*}
%	\begin{split}
%		| | | Tv | | | \le \|u_0\|_X + Mh < \infty
%	\end{split}
%\end{equation*}
%
%
Next, note that for any two points $v_1, v_2 \in
V_h$, we have
%
%
\begin{equation}
	\label{cont-part-1}
	\begin{split}
		| | | Tv_2 - Tv_1 | | | 
		& = | | | \int_0^t \left[ f(s, v_2(s) ) - f(s, v_1 (s) ) \right]ds | | |
		\\
    & \le \int_0^{|t|} | | | f(s, v_2(s) ) - f(s, v_1 (s) ) | | | ds
		\\
    & = \int_0^{|t|} \sup_{s \in [-h, h]} d[f(s, v_1(s) ), f(s, v_2 (s) )] \ ds
		\\
    & \le c \int_0^{|t|}  \sup_{s \in [-h, h]} d[v_1(s), v_2 (s)] \
		ds
		\end{split}
\end{equation}
%
%
where the last step follows from \eqref{stronger-ode}. Hence
\begin{equation*}
	\begin{split}
		| | | Tv_2 - Tv_1 | | | 
		& \le c h \sup_{s \in [-h, h]} d[v_1(s), v_2 (s)]
		\\
		& = ch |||v_2 - v_1|||.
	\end{split}
\end{equation*}
%
%
Restricting $h \le 1/(2c)$, we obtain
\begin{equation*}
	\begin{split}
		| | | Tv_2 - Tv_1 | | | & \le \frac{1}{2} | | | v_2 - v_1 | | |. 
	\end{split}
\end{equation*}
Hence, for $h \le \min\left\{1/(2c), 1/[2cr + M] \right\}$, $T$ is a
contraction on $V_h$. This completes the proof. \qed
%
%%%%%%%%%%%%%%%%%%%%%%%%%%%%%%%%%%%%%%%%%%%%%%%%%%%%%
%
%
%                Dependence on Params
%
%
%%%%%%%%%%%%%%%%%%%%%%%%%%%%%%%%%%%%%%%%%%%%%%%%%%%%%
%
%
\section{Regularity for Solutions to ODEs} 
\label{sec:dep-param}
%
%
Following Taylor \cite{Taylor:1995kx}, let $X \subset \rr$ be an open subset,
$U \subset C^{k}(X)$ be an open subset, where $k \in \mathbb{N}\cup \omega$, and
$F: U \to C^{k}(X)$. We endow $C^{k}(X), k \in \mathbb{N}$ with the metric
induced by its seminorms
%
%
\begin{equation*}
\begin{split}
d(f, g) \doteq \sum_{k \in \mathbb{N}} \sum_{j \in \mathbb{N}} 2^{-j -k} \frac{|
f-g |_{j, k}}{1 + | f-g |_{j, k}}
\end{split}
\end{equation*}
%
where
%
%
\begin{gather*}
|h|_{j,k}  = \sup_{x \in K_j} \sum_{k \in \mathbb{N}} |d^{k}h(x)|,
\\
\{ K_{j} \} \ \text{a family of compact subsets of} \ X \  \text{covering} \ X
\end{gather*}
%
%
%
and equip the set of real analytic functions $C^{\omega}(X)$ with the
metric
%
%
\begin{equation*}
\begin{split}
\rho(f,g) \doteq \sup_{x \in X} |f(x) - g(x)|.
\end{split}
\end{equation*}
%
%
Consider the non-autonomous system
%
%
\begin{gather}
  \label{aa}
\frac{dy}{dt} = F(y(x,t), t),
\\
y(0)= y_{0}(x) \in C^{k}, \quad x \in \ci \ \text{or} \ \rr,t \in \rr.
\label{bb}
\end{gather}
%
Letting $z = (y(x,t), t)$, we get
%
%
\begin{equation*}
\begin{split}
\frac{dz}{dt}  
& = \left (\frac{dy}{dt}, 1 \right )
\\
& = \left( F(y(t), t), 1 \right)
\\
& = \left( F(z), 1 \right)
\\
& \doteq G(z).
\end{split}
\end{equation*}
%
%
Hence, without loss of generality, we may restrict our attention to the
autonomous system
%
%
\begin{gather}
\frac{dy}{dt} = F\left( y(x,t) \right)
\label{ode-eq}
\\
y(0) = y_{0}(x), \quad x \in \ci \ \text{or} \ \rr
\label{ode-init-data}
\end{gather}
%
%
%
for which we shall prove the following.
%
%
%%%%%%%%%%%%%%%%%%%%%%%%%%%%%%%%%%%%%%%%%%%%%%%%%%%%%
%
%
%                dep on init cond
%
%
%%%%%%%%%%%%%%%%%%%%%%%%%%%%%%%%%%%%%%%%%%%%%%%%%%%%%
%
%
%
\begin{corollary}[Regularity With Respect to Parameters]
  \label{cor:reg-param}
  Let $U \subset C^{k}$ be an open subset, $ k \in \mathbb{N} \cup \omega$.  
Consider the ode
\begin{gather}
  \label{ode-param-naut}
\frac{dy}{dt} = F \left [ y(x, \tau, t), \tau \right]
\\
y(0) = y_{0}(x), \quad \tau, x \in \ci \ \text{or} \ \rr, t \in \rr
\label{ode-param-init-naut}
\end{gather}
%
%
where $F: U \times \rr  \to C^{k}$ is Lipschitz. If $y_{0} \in C^{k}$, then
there exists an 
open set $I \subset \rr$
such that the ivp \eqref{ode-param-naut}-\eqref{ode-param-init-naut}
admits a unique solution $y(x, \tau, t)$ for $t \in I$. In addition,
this solution is $C^{k}$ in $x$ and $\tau$.
%
\end{corollary}
%
\begin{proof}
  Without loss of generality, we restrict our attention to the homogeneous ivp
\begin{gather}
  \label{ode-param}
\frac{dy}{dt} = F \left [ y(x, \tau, t) \right]
\\
y(0) = y_{0}(x) \in C^{k}, \quad \tau, x \in \ci \ \text{or} \
\rr, t \in \rr.
\label{ode-param-init}
\end{gather}
  %
  The idea will be to view the couple
  $(x, \tau)$ as the spatial component of the ode
  \eqref{ode-param}-\eqref{ode-param-init}.
  To make this rigorous, let $z = (x, \tau)$. Then
  \eqref{ode-param}-\eqref{ode-param-init} is equivalent to the system
  %
  %
  \begin{gather*}
    \frac{d}{dt}(y, z) = \left( F[y(z,t)], \vec{0} \right), \quad \vec 0 =
    (0, 0)
  \\
  (y, z)(0) = \left ( y_{0}(x), \tau \right ).
  \end{gather*}
  %
  %
  Letting $Y = (y,z)$, and $G: U \times \left\{ \vec{0} \right\} 
  \to C^{k} \times \left\{ \vec{0} \right\}$ be defined
  by $G(u, \vec{0}) = \left( F(u), \vec{0} \right)$, we obtain
  %
  %
  \begin{gather*}
      \frac{dY}{dt} = G\left[ Y(z,t) \right]
      \\
      Y(0) = Y_{0}(z)
  \end{gather*}
%
where $Y_{0}(z) = Y_{0}(x, \tau)$ is a constant function of $\tau$. 
Applying Theorem \ref{ode-thm},  we obtain a solution $Y(z,t)$ which is
$C^{k}$ in $z$ locally in time. Since $z = (x, \tau)$, this implies that $Y(z,t)$ is
$C^{k}$ in $x$ and $\tau$, concluding the proof. 
%
\end{proof}
%
%
We are now prepared to prove the following. 
\begin{theorem}
  Let $U \subset C^{k}$ be an open subset, $k \in \mathbb{N} \cup \omega$. Suppose $G: U \times \rr
  \to C^{k}$
  is Lipschitz. If $y_{0} \in C^{k}$, then there exists an open set $I \subset
  \rr$
such that the ivp 
\begin{gather}
\frac{dy}{dt} = G\left( y(x,t), t \right)
\\
y(0) = y_{0}(x), \quad x \in \ci \ \text{or} \ \rr,t \in \rr
\end{gather}
%
%
admits a unique solution $y(x,t)$ for $t \in I$. In addition,
this solution is $C^{k}$ in $x$ and $t$.
\label{thm:reg-result}
\end{theorem}
%
%
\begin{proof}
Without loss of generality, we restrict our attention to the homogeneous ivp
\begin{gather}
\frac{dy}{dt} = G\left( y(x,t) \right)
\label{cc}
\\
y(0) = y_{0}(x), \quad x \in \ci \ \text{or} \ \rr,t \in \rr.
\label{dd}
\end{gather}

  We choose $F: U \times \rr \to C^{k}$ in \eqref{ode-param-naut}
  such that it is $C^{k}$ and satisfies 
  %
  %
  \begin{equation}
    \label{hjj}
  \begin{split}
    F\left[ y, \tau \right] = \tau F \left[ y, 1 \right].
  \end{split}
  \end{equation}
  %
  Note that \eqref{hjj} does not violate our assumption of $k$-differentiability
  for $F$, since
%
%
\begin{equation*}
\begin{split}
  DF\left[ y, \tau \right]
  & = \left( D_{y}(y, \tau) F, D_{\tau} F[y, \tau] \right)
  \\
  & = (  D_{y} \{\tau F[y, 1] \}, D_{\tau} \{F[y, 1] \} ) 
  \\
  & = ( \tau D_{y} F[y, 1], F[y, 1]).
\end{split}
\end{equation*}
%
%
Then if $y(x, t, \tau)$ is a solution to
\eqref{ode-param-naut}-\eqref{ode-param-init-naut}, we have $y(x, \tau, t ) =
y(x, 1, \tau t)$. To see this, we compute
%
%
\begin{equation*}
\begin{split}
\frac{d}{dt}\left[ y(x, 1, \tau t) \right]
& = \tau y'(x, 1, \tau t)
\\
& = \tau F\left( 1, y(x, 1, \tau t) \right)
\\
& = F(\tau, y(x, 1, \tau t)).
\end{split}
\end{equation*}
%
%
That is, $y(x, 1, \tau t)$ is a solution
\eqref{ode-param-naut}-\eqref{ode-param-init-naut}. By uniqueness, $y(x, \tau, t) =
y(x, 1, \tau t)$. But formally, by the chain rule, we have
%
%
\begin{equation}
  \label{chain-comp}
\begin{split}
\frac{d^{k}y}{d \tau^{k}}(x, \tau, t) 
& = \frac{d}{d \tau^{k}}\left[ y(x, 1, \tau t) \right]
\\
& = \tau^{k} \frac{d^{k}y}{dt^{k}}(x, 1, \tau t) + \dots
\end{split}
\end{equation}
%
%
By Corollary \ref{cor:reg-param}, $y$ is $C^{k}$ in $\tau$. Hence, from
\eqref{chain-comp}, we conclude that $y$ is $C^{k}$ in $t$. $C^{k}$
differentiability in $x$ follows from Corollary \ref{cor:reg-param}.
Running an induction, one obtains the corresponding result
for the case $k = \infty$. The case $k = \omega$ is identical to the case
$k \in \mathbb{N}$. Lastly, for any given $G$,
we can find $F$ satisfying \eqref{hjj} and such that
$G(\cdot) = F(\cdot, 1)$. This concludes the proof.
%
\end{proof}
%
%
\section{Differentiation in Banach Spaces}
We remark that the theory presented below can be extended to arbitrary
topological vector spaces with a topology induced by a metric (in particular,
Fr\'echet spaces) by replacing norms with metrics in what follows.
\begin{definition}
	\label{def:diff}
	Let $X,Y$ be Banach spaces over the real numbers, $U \subset X$ open,
  and consider the map $f: U \to Y$.
	Then $f$ is differentiable at $x_0 \in U$ if there
	is a continuous linear map $Df(x_0): X \to Y$ such that
	%
	%
	%
	%
	\begin{equation}
		\label{diff-limit}
		\begin{split}
			\lim_{h \to 0} \frac{\|f(x_0+ h) - f(x_0) -
			Df(x_0)(h) \|_Y}{\|h\|_{X}} = 0.
		\end{split}
	\end{equation}
	%
	%
	This map, which we call the \emph{total derivative} of $f$ at $t_0$, is 
	unique. If $Df(x_0)$ exists for all $x_0 \in U$,
	then we say that $f$ is
	\emph{differentiable} in $U$. If $f$ is differentiable in $U$, and 
	$\|Df(x_0 + h) - Df(x_0) \|_Y \to 0$ as $\|h\|_{X} \to 0$ for all $x_0 \in U$,
	then we say that $f$ is \emph{continuously differentiable in $U$}. If $X, Y$
  are Banach spaces over the complex numbers, we say that $f$ is
  \emph{holomorphic} (or \emph{continuously holomorphic}) in $U$.
\end{definition}
	%
  %
  %
  \begin{framed}
  %
  %
  \begin{remark}
    Equivalently, $f$ is differentiable at $x_{0}$ is there exists a continuous
    linear map $Df(x_{0}): X \to Y$ such that
    %
    %
    \begin{equation*}
    \begin{split}
      f(x_{0} + h) = f(x_{0}) + Df(x_{0}) \circ h + o(\| h \|_{X}).
    \end{split}
    \end{equation*}
    %
    To see this, suppose
    %
    %
    \begin{equation*}
    \begin{split}
      \lim_{h \to \infty} \frac{\| f(x_{0} + h) - f(x_{0}) - Df(x_{0}) \circ h
      \|_{Y}}{\| h \|_{X}} = 0.
    \end{split}
    \end{equation*}
    %
    %
    Then we can find $\delta > 0$ such that
    %
    %
    \begin{equation*}
    \begin{split}
    \frac{\| f(x_{0} + h) - f(x_{0}) - Df(x_{0}) \circ h
      \|_{Y}}{\| h \|_{X}}  < \ee
    \end{split}
    \end{equation*}
    %
    %
    for all $h \in X$ with $\| h \|_{X} < \delta$. Hence
    %
    %
    \begin{equation*}
    \begin{split}
      \| f(x_{0} + h) - f(x_{0}) - Df(x_{0}) \circ h \|_{Y} < \ee \| h
      \|_{X}.
    \end{split}
    \end{equation*}
    %
    %
    This implies
    %
    %
    \begin{equation*}
    \begin{split}
      f(x_{0} + h) - f(x_{0}) - Df(x_{0}) \circ h = g, \quad g \in Y, \ \| g
      \|_{Y} < \ee \| h \|_{X}.
    \end{split}
    \end{equation*}
    %
    %
    Hence, 
    %
    %
    \begin{equation*}
    \begin{split}
      f(x_{0} + h) 
      & = f(x_{0}) + Df(x_{0}) \circ h + g
      \\
      & =  f(x_{0}) + Df(x_{0})\circ h + o(\| h \|_{X}).
    \end{split}
    \end{equation*}
    %
    %
    For the reverse direction, if
    %
    %
    \begin{equation*}
    \begin{split}
      f(x_{0} + h) = f(x_{0}) + Df(x_{0}) \circ h + o(\| h \|_{X})
    \end{split}
    \end{equation*}
    %
    %
    then
    %
    %
    \begin{equation*}
    \begin{split}
      \| f(x_{0} + h) - f(x_{0}) - Df(x_{0}) \circ h \|_{Y} = o(\| h
      \|_{X})
    \end{split}
    \end{equation*}
    %
    %
    which implies
    %
    %
    \begin{equation*}
    \begin{split}
      \lim_{h \to \infty} \frac{\| f(x_{0} + h) - f(x_{0}) - Df(x_{0}) \circ h
      \|_{Y}}{\| h \|_{X}}  = \lim_{h \to \infty} \frac{o(\| h \|_{X})}{\| h
      \|_{X}} = 0.
    \end{split}
    \end{equation*}
    %
    %
    %
  \label{rem:equiv-def}
  \end{remark}
  %
  %
  \end{framed}
  %
  %
  \begin{framed}
	\begin{remark}
		\label{rem:usual-diff}
		Consider the map $f: \rr \to \rr$. Then \autoref{def:diff} 
		coincides with the usual definitions of
		differentiability and continuous differentiability for maps
		from $\rr $ to $\rr$.
		To see this, assume $f$ is differentiable at $x_0$.
		Let us first verify that the map  $D(x_0): \rr ~\to~\rr$
		defined by $Df(x_0)(x) =
		xf'(x_0)$ is the total derivative of $f$. Assume without loss of generality
		that $h \to 0^+$. Then
		%
		%
		\begin{equation}
			\label{2}
			\begin{split}
				 \lim_{h \to 0^+} \frac{| f( x_0 + h) - f(x_0) -
				Df(x_0)(h) |}{|h|}
				 & = \lim_{h\to 0^+} \left |\frac{f(x_0+h) -
				f(x_0)}{h} - \frac{Df(x_0)(h)}{h}  \right |
				\\
				 & =\lim_{h \to 0^+} \left |\frac{f(x_0+h) - f(x_0)}{h} -
				f'(x_0) \right | = 0.
			\end{split}
		\end{equation}
		%
		%
%
%
For any $a \in \rr$, define the map $T_{a} \in L(\rr , \rr)$ by
$T_{a}(x) = ax$. Then the map $a \mapsto T_a$ is an isometric
isomorphism from $\rr$ to $L( \rr, \rr)$. Hence, 
we may identify $f'(x_0)$ with $T_{f'(x_0)} = Df(x_0)$,
which by \eqref{2} is the total derivative of $f$.  Using this identification,
it follows that $f$ is continuously
differentiable in $\rr$ if and only if $f'$ is a continuous function of
$x_0$, for all $x_0 \in \rr$. \qed
%
\end{remark}
\end{framed}
%
%
In general, for an arbitrary Banach space $X$, open interval $(a,b) \subset
\rr$, and map $f:(a,b) \to X$, we can identify $Df(x_0)$ with an
element of $X$ via the following.
%
%
%%%%%%%%%%%%%%%%%%%%%%%%%%%%%%%%%%%%%%%%%%%%%%%%%%%%%
%
%
%			Lemma Isometry	
%
%
%%%%%%%%%%%%%%%%%%%%%%%%%%%%%%%%%%%%%%%%%%%%%%%%%%%%%
%
%
\begin{lemma}
	\label{lem:isometry} Let $(a,b) \subset \rr$ be an open interval, $X$ a Banach
	space, with $x \in X$. Define the map $T_x \in L\left ( \rr , X \right )$ by
	$T_x(t_0) = x t_0$. Then the map $x \mapsto T_x$ is an
	isometric isomorphism from
	$X$ to $L(\rr , X)$. 
\end{lemma}
%
%
\begin{proof} Note that 
%
%
\begin{equation*}
	\begin{split}
		| | | T_x | | |
		& = \sup_{|t_0| = 1} \| T_x (t_0) \|_X
		= \| x t_0\|_X
		= \|x\|_X.
	\end{split}
\end{equation*}
%
%
Hence, the map $x \mapsto T_x$ is an isometry from $X$ into $L(\rr,
X)$. It remains to show that it is onto. Let $V \in L( \rr, X)$. Then
by linearity
%
%
\begin{equation*}
	\begin{split}
		V(t_0) = V(1)t_0. 
	\end{split}
\end{equation*}
%
%
Hence, $V = T_{V(1)}$, completing the proof. 
\end{proof}
%
%
Applying the lemma, we see that if $Df(t_0)$ exists for a map $f: (a,b) \to X$,
then it can be viewed as an
element of $X$. Similarly, if $Df(t_0)$ exists for all $t_0 \in (a,b)$, then
we may view the map $t_0 \to Df(t_0)$ as an
element of $L( \rr, X)$. Hence, for a
map $f:(a,b) \to X$, we see that the following is an equivalent
reformulation of \autoref{def:diff}. 
\begin{definition}
	\label{def:diff-simp}
	Let $X$ be Banach space, $(a,b) \subset \rr$ an open interval, and
	consider the map $f: (a,b) \to X$.
	Then $f$ is \emph{differentiable at $t_0$} if there exists a map
	$f': \rr \to X$ such that 
	%
	%
	%
	%
	\begin{equation}
		\label{diff-limit-simp}
		\begin{split}
			\lim_{h \to 0} \| \frac{f(t_0+ h) - f(t_0) 
			 }{h} - f'(t_0) \|_X = 0.
		\end{split}
	\end{equation}
	%
	%
	We call $f'(t_0)$ the \emph{derivative of $f$ at $t_0$}.
	If \eqref{diff-limit-simp}
	holds for all $t_0 \in (a,b)$, then we say $f$ is \emph{differentiable in
	$(a,b)$}, and call $f'$ the
	\emph{derivative of $f$ in $(a,b)$}.  
\end{definition}
\begin{framed}
\begin{example}
Let $f: (a,b) \to \rr^2$ be defined by $f(t) = (1, t^2)$. Then $Df(1)$ maps
$(a,b)$ to $\rr$ by the relation $Df(1)(t) = (0, 2t) = t[0, 2]$. Hence, $Df(1)$ is the $1
\times 2$ matrix $[0,2]$, which we may also view as the point $(0,2) \in \rr^2$.
That is, $f'(1) = (0,2)$. Similarly, $f'(t) = (0,2t)$.
\end{example}
\end{framed}

Lastly, we include the following.
%
%
\begin{proposition}
    (Chain Rule) Let $X,Y,Z$ be Banach spaces, $f: X \to Y$ and $g: Y \to Z$
    continuously differentiable maps. Then for all $x_0 \in X$ we have
    %
    \begin{equation*} (g \circ f)' (x_0) = g'(f(x_0)) \circ (f'(x_0)).
		\end{equation*} 
    %
	\end{proposition}
  \begin{proof} Since $f$ and $g$ are continuously differentiable, we can write
			\begin{equation*}
				f(x_0 + s) = f(x_0) + f'(x_0)(s) + o_{1}(s)
			\end{equation*}
			and
      \begin{equation*}
				g(x_0 + t) = g(x_0) + g'(x_0)(t)+ o_{2}(t).
			\end{equation*}
						Hence
			\begin{equation}
        \label{pre-order}
				\begin{split}
					h(x_0 + s) &= g(f(x_0 +s)
					\\
					&= g(f(x_0) + f'(x_0)(s) + o_{1}(s))
					\\
					&= g(f(x_0)) + g'(f(x_0)) \circ [f'(x_0)(s) + o_{1}(s)] 
					+ o_{2}(f'(x_0)(s) + o_{1}(s))
					\\
					&= g(f(x_0)) + g'(f(x_0)) \circ f'(x_0)(s) +
					g'(f(x_0)) \circ o_{1}(s) + o_{2}(f'(x_0)(s) + o_{1}(s)).
				\end{split}
			\end{equation}
      Since $f'(x_0)$
			and $g'(f(x_0))$ are continuous linear operators, we have
			\begin{equation*}
				\begin{split}
         &  f'(x_0)(s)
           = o_{3}(s)
					\\
					& g'(f(x_0))(t) = o_{4}(t).
				\end{split}
			\end{equation*}
      Therefore,
      %
      %
      \begin{equation*}
        \begin{split}
          g'(f(x_0)) \circ o_{1}(s)
          & = o_{4}(o_{1}(s))
          \\
          & = o_{5}(s)
        \end{split}
      \end{equation*}
      %
      %
      and
      %
      %
      \begin{equation*}
      \begin{split}
      o_{2}(f'(x_0)(s) + o_{1}(s))
      & = o_{2}(o_{3}(s) + o_{1}(s))
      \\
      & =o_{6}(s).
      \end{split}
      \end{equation*}
      %
      %
			Recalling \eqref{pre-int} and substituting, we conclude that
			\begin{equation*}
				\begin{split}
				h(x_0 + s) = g(f(x_0)) + g'(f(x_0))(f'(x_0)(s)) +
				o(s)
        \end{split}
        \end{equation*}
			 completing the proof. 
   \end{proof}
%
%
%
%
%


 % If you have appendices, add them here.
 % Begin each one with \chapter{title} as before- the \appendix command takes
 % care of renaming chapter headings and creates a new page in the Table of
 % Contents for them.
 % \include{appendix-one}

\backmatter              % Place for bibliography and index
\bibliographystyle{nddiss2e}
%\bibliographystyle{amsalpha-custom} 
\bibliography{/Users/davidkarapetyan/math/bib-files/references}

\end{document}

%%
\endinput
%%
