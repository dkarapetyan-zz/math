%
%
\chapter{H\"older Continuity for HR in the Weak Topology}
%
%

%
%
%
%
\section{Proof of H\"older Continuity}
%
%
We note that the only significant difference between the proof of H\"older
continuity in the periodic and non-periodic cases is in the proof of Lemma~\ref{lem:frac-deriv}, which we address in Section~\ref{sec:pf-lemmas}. Hence, we focus our
attention on the proof of H\"older continuity in the periodic case. 
%
%
\subsection{Region $\Omega_{1}$} 
\label{ssec:reg-m-imp}
%
%
Let $u_{0}(x), v_{0}(x)
\in B_{H^{s}}(R)$, $s > 3/2$ be two initial data. Then from
the well-posedness theory for HR \cite{Karapetyan:2010fk}, we
know that there exists unique corresponding solutions $u, v \in C(I,
B_{H^{s}}(2R))$ to HR.
Set $v=u-w$. Then $v$ solves the Cauchy-problem
%
%
\begin{align}
\label{uniqueness-exp}
& \p_t v
=  -\frac{\gamma}{2} \p_x [v(u + w)] 
\\
\notag
& \phantom{\p_t v = }-\p_x (1 - \p_{x}^{2})^{-1} \left\{
\frac{3-\gamma}{2}[v(u+w)] + \frac{\gamma}{2}[\p_x v \cdot \p_x (u+w)]
\right\},
\\
& v(x,0) = u_{0}(x) - v_{0}(x).
\label{uniqueness-init-data}
\end{align}
%
%
%
Let
\begin{equation*}
D^{m} \doteq (1 - \p_x^2)^{m/2}, \quad m \in \rr.
\end{equation*}
%
Applying $D^r$ to both sides of \eqref{uniqueness-exp}, then 
multiplying both sides by $D^r v$ and integrating, we obtain
%
%
\begin{equation}
\begin{split}
\frac{1}{2} \frac{d}{dt} \|v\|_{H^r}^2
= & -\frac{\gamma}{2} \int_{\ci} D^r \p_x [v(u+w)] \cdot
D^r v \ dx
\\
& - \frac{3-\gamma}{2} \int_{\ci}  D^{r -2}
\p_x[v(u+w)] \cdot
D^r v \ dx  
\\
& - \frac{\gamma}{2} \int_{\ci} D^{r 
-2} \p_x [ \p_x v
\cdot \p_x (u+w)]\cdot D^r v \ dx.
\label{2v-iu}
\end{split}
\end{equation}
We now estimate \eqref{2v-iu} in parts.
%
\subsubsection{Estimate of Integral 1} 
Note that
%
%
\begin{equation}
\begin{split}
& \left |  -\frac{\gamma}{2} \int_{\ci} D^r \p_x [v(u+w)] \cdot
D^r v \ dx \right |
\\
& =
\left |
-\frac{\gamma}{2} \int_{\ci} \left[ D^r \p_x, \ u+w \right]v \cdot
D^r v \ dx - \frac{\gamma}{2} \int_{\ci} (u+w) D^r
\p_x v \cdot D^r v\ dx
\right | \\
& \lesssim \left |
\int_{\ci} \left[ D^r \p_x, \ u+w \right]v \cdot
D^r v \ dx \right |
+ \left | \int_{\ci} (u+w) D^r \p_x v
\cdot D^r v\
dx \right |.
\label{4v-iu}
\end{split}
\end{equation}
%
%
Observe that integrating by parts gives
%
%
\begin{equation}
\begin{split}
\left | \int_{\ci} (u+w) D^r \p_x v \cdot
D^r v \ dx \right |
\le \|\p_x (u+w)\|_{L^\infty}
\|v\|_{H^r}^2.
\label{4'v-iu}
\end{split}
\end{equation}
%
%
%
%
To estimate the remaining integral of \eqref{4v-iu}, we shall need the following
following result taken from \cite{Himonas:2010p1187}:
%
\begin{lemma}
\label{cor1-iu}
If $s > 3/2$ and $-1 \le r  \le s -1$, then
%
%
\begin{equation*}
\begin{split}
\|[D^r \p_x ,f]g\|_{L^2} \lesssim \|f\|_{H^s} \|g\|_{H^r}.
\end{split}
\end{equation*}
%
%
\end{lemma}
%
%
An application of 
Cauchy-Schwartz and Lemma~\ref{cor1-iu} then yields 
%
%
\begin{equation}
\begin{split}
\left | \int_{\ci} [D^r \p_x, \ u+w] v
\cdot D^r v \ dx \right |
& \lesssim \|u+w\|_{H^s} 
\|v\|_{H^r}^2.
\label{120v}
\end{split}
\end{equation}
%
%
Combining \eqref{4'v-iu} and \eqref{120v} and applying the Sobolev embedding 
theorem, we obtain the estimate
%
%
\begin{equation}
\begin{split}
\left |  -\frac{\gamma}{2} \int_{\ci} D^r \p_x [v(u+w)] \cdot
D^r v \ dx \right |
\lesssim \|u+w\|_{H^s} \|v\|_{H^r}^2, \quad s > 3/2, \ -1 \le r \le s-1.
\label{20v}
\end{split}
\end{equation}
%
%
\subsubsection{Estimate of Integral 2} We shall need the following.
%
%
%%%%%%%%%%%%%%%%%%%%%%%%%%%%%%%%%%%%%%%%%%%%%%%%%%%%%
%
%
%                frac deriv est
%
%
%%%%%%%%%%%%%%%%%%%%%%%%%%%%%%%%%%%%%%%%%%%%%%%%%%%%%
%
%
\begin{lemma}
For $s > 3/2$, $r \le s$, $s + r \ge 2$, we have
%
%
\begin{equation*}
\begin{split}
\| fg \|_{H^{r-1}} \lesssim \| f \|_{H^{r-1}} \| g \|_{H^{s-1}}.
\end{split}
\end{equation*}
%
%
\label{lem:frac-deriv}
\end{lemma}
%
%
%
%
%
%
Applying Cauchy-Schwartz and Lemma~\ref{lem:frac-deriv}, we obtain
%
%
%
\begin{equation*}
\begin{split}
\left | - \frac{3-\gamma}{2} \int_{\ci}  D^{r -2}
\p_x[v(u+w)] \cdot
D^r v \ dx  \right |
& \lesssim \|u+w\|_{H^{r -1}} \|v\|_{H^r}^2
\end{split}
\end{equation*}
%
%
which implies
\begin{equation}
\begin{split}
\left | - \frac{3-\gamma}{2} \int_{\ci}  D^{r -2}
\p_x[v(u+w)] \cdot
D^r v \ dx  \right |
& \lesssim \|u+w\|_{H^{s}} \|v\|_{H^r}^2
\label{3v-iu}
\end{split}
\end{equation}
%
for $s > 3/2, \ r \le s, \ \text{and} \ s + r \ge 2$.
%
%
\subsubsection{Estimate of Integral 3} We first apply
Cauchy-Schwartz to obtain
%
%
\begin{equation*}
\begin{split}
\left | - \frac{\gamma}{2} \int_{\ci} D^{r 
-2} \p_x [ \p_x v
\cdot \p_x (u+w)]\cdot D^r v \ dx \right | 
\lesssim 
\|[\p_x v \cdot \p_x (u+w)] \|_{H^{r -1}}
\|v\|_{H^r}.
\end{split}
\end{equation*}
%
%
Applying Lemma~\ref{lem:frac-deriv} and the inequality $\| f_{x}
\|_{H^{m-1}} \le \| f \|_{H^{m}}$,  we conclude that
%
\begin{equation}
\begin{split}
\left | - \frac{\gamma}{2} \int_{\ci} D^{r 
-2} \p_x [ \p_x v
\cdot \p_x (u+w)]\cdot D^r v \ dx \right | 
\lesssim \|u+w \|_{H^{s}}
\|v\|_{H^r}^2
\label{3'v-iu}
\end{split}
\end{equation}
%
%
for $s > 3/2, \ r \le s, \ \text{and} \ s + r \ge 2$.
%
%
%
%
Grouping \eqref{20v}, \eqref{3v-iu}, and \eqref{3'v-iu}, we obtain
%
%
\begin{equation*}
\begin{split}
\frac{1}{2} \frac{d}{dt}
\|v\|_{H^r}^2
& \lesssim \|u+w\|_{H^s}
\|v\|_{H^r}^2, \quad | t | < T
\\
& \le 4R \| v \|_{H^{r}}^{2}.
\label{9v-iu}
\end{split}
\end{equation*}
%
%
%
%
%
Letting $y(t) = \| v \|^{2}_{H^{r}}$, we obtain
%
%
%
\begin{equation*}
\begin{split}
\frac{dy}{dt} \le cy
\end{split}
\end{equation*}
%
where $c = c(s, r, R) > 0$. Hence
%
%
\begin{equation*}
\begin{split}
y(t) \le y(0) e^{ct}, \quad | t | < T
\end{split}
\end{equation*}
%
%
which implies
%
%
\begin{equation*}
\begin{split}
y(t) \le y(0) e^{cT}.
\end{split}
\end{equation*}
%
%
Substituting back in for $y$, we see that
%
%
\begin{equation*}
\begin{split}
\| v \|_{H^{r}}^{2} \le \| v(0) \|^{2}_{H^{r}} e^{cT}
\end{split}
\end{equation*}
%
%
or
%
%
\begin{equation}
\label{lip-ineq}
\begin{split}
& \| u(t) - w(t) \|_{H^{r}} \le C \| u_{0} - w_{0} \|_{H^{r}}, 
\\
& \text{for} \ | t | < T,
\ s > 3/2, \ -1 \le r \le s-1, \ s + r \ge 2.
\end{split}
\end{equation}
%
Hence, in region $\Omega_{1}$, the data to solution map is locally Lipschitz from
$B_{H^{s}}(R)$ (measured with the $H^{r}$
norm) to $C([-T, T], H^{r})$, with Lipschitz constant $C = C(s, r, R)$.
%
%
%
%
%
%
%
%
%
%
\subsection{Region $\Omega_{2}$} 
\label{ssec:case-4}
%
We have the estimate
\begin{equation}
\label{fgh}
\begin{split}
\| u(t) - w(t) \|_{H^{r}}
& < \|u(t) - w(t) \|_{H^{2-s}}.
\end{split}
\end{equation}
%
We see that \eqref{lip-ineq} is valid for $r = 2-s$, $3/2 < s \le 3$.
Hence, applying \eqref{lip-ineq} to \eqref{fgh}, we obtain 
%
%
%
%
\begin{equation*}
\begin{split}
\| u(t) - w(t) \|_{H^{r}}
\lesssim \|u_{0} - w_{0} \|_{H^{2-s}}.
\end{split}
\end{equation*}
%
%
%
%
%
We need the following interpolation
result. 
%
%
%%%%%%%%%%%%%%%%%%%%%%%%%%%%%%%%%%%%%%%%%%%%%%%%%%%%%
%
%
%                interp
%
%
%%%%%%%%%%%%%%%%%%%%%%%%%%%%%%%%%%%%%%%%%%%%%%%%%%%%%
%
%
\begin{lemma}
For $m_{1} < m < m_{2}$,
%
%
\begin{equation*}
\begin{split}
\| f \|_{H^{m}} \le \| f \|_{H^{m_{1}}}^{(m_{2}-m)/(m_{2} - m_{1})} \| f
\|_{H^{m_{2}}}^{(m -m_{1})/(m_{2} - m_{1})}.
\end{split}
\end{equation*}
%
%
%
%
%
%
\label{lem:interp}
\end{lemma}
%
Applying the lemma with $m_{1} =r$, $m = 2-s$, and $m_{2} = s$ (notice
$m_{2} > m$ for $s > 1$), we bound 
%
%
\begin{equation*}
\begin{split}
\| u_{0} - w_{0} \|_{H^{2-s}} 
& \le \| u_{0} - w_{0} \|_{H^{r}}^{\frac{2(s-1)}{s-r}} \| u_{0} - w_{0}
\|_{H^{s}}^{\frac{2-s-r}{s-r}}
\\
&  \lesssim \| u_{0} - w_{0} \|_{H^{r}}^{\frac{2(s-1)}{s-r}}.
\end{split}
\end{equation*}
%
We conclude that
%
%
\begin{equation*}
\begin{split}
\| u(t) - w(t) \|_{H^{r}} \lesssim \|u_{0} - w_{0} \|_{H^{r}}^{\frac{2(s-1)}{s-r}}.
\end{split}
\end{equation*}
%
%
%
%
\subsection{Region $\Omega_{3}$} 
\label{ssec:case-2}
%
%
Applying Lemma~\ref{lem:interp} with $m_{1} = s-1$, $m =r$ and $m_{2} = s$, and 
using the estimate
%
%
\begin{equation*}
\begin{split}
\|u - w \|_{H^{s}} \le 4R
\end{split}
\end{equation*}
%
%
we obtain
%
%
\begin{equation}
\label{pre-lip-ap}
\begin{split}
\| u(t) - w(t) \|_{H^{r}} & \lesssim \| u(t) - w(t) \|_{H^{s-1}}^{s-r} \|u(t)
- w(t)\|_{H^{s}}^{1-s+r}
\\
& \simeq \| u(t) - w(t) \|_{H^{s-1}}^{s-r}.
\end{split}
\end{equation}
%
%
We see that \eqref{lip-ineq} is valid for  $r = s-1$, $s \ge 3/2$. Hence,
applying \eqref{lip-ineq} to \eqref{pre-lip-ap} gives
%
%
\begin{equation*}
\begin{split}
\| u(t) - w(t) \|_{H^{r}} & \lesssim  \|u_{0} - w_{0}\|_{H^{s-1}}^{s-r} 
\\
& \le
\|u_{0} - w_{0}\|_{H^{r}}^{s-r}.
\end{split}
\end{equation*}
%
This completes the proof of Theorem~\ref{thm:main-thm}. \qed
%
%
%%%%%%%%%%%%%%%%%%%%%%%%%%%%%%%%%%%%%%%%%%%%%%%%%%%%%
%
%
%		Optimality
%
%
%%%%%%%%%%%%%%%%%%%%%%%%%%%%%%%%%%%%%%%%%%%%%%%%%%%%%
%
%
%
%
\section{Proofs of Lemmas} 
\label{sec:pf-lemmas}
%
%
%
\begin{proof}[Proof of Lemma~\ref{lem:frac-deriv}]
For the periodic case we have
%
%
\begin{equation*}
\begin{split}
\| fg\|_{H^{r-1}}^{2}
& \le  \sum_{n \in \zz}  (1 + n^{2})^{r-1}\left [ \sum_{k \in \zz}
| \wh{f}(k) |  | \wh{g}(n - k) | (1 +
k^{2})^{\frac{1-s}{2}} (1 + k^{2})^{\frac{s-1}{2}}
\right ]^{2}.
\end{split}
\end{equation*}
%
Applying Cauchy Schwartz in $k$, we bound this by
%
%
%
\begin{equation*}
\label{np-key-term-iu}
\begin{split}
\| f \|_{H^{s-1}}^{2} \sum_{n \in \zz}  (1 + n^{2})^{r-1}\sum_{k \in \zz} \frac{|
\wh{g}(n - k) |^{2}}{(1 + k^{2})^{s-1}}.
\end{split}
\end{equation*}
%
But by change of variables and Fubini
%
\begin{equation}
\label{opp}
\begin{split}
\sum_{n \in \zz}  (1 + n^{2})^{r-1}\sum_{k \in \zz} \frac{|
\wh{g}(n - k) |^{2}}{(1 + k^{2})^{s-1}}
& = \sum_{k \in \zz}| \wh{g}(k) |^{2} \sum_{n \in \zz}  
\frac{1}{(1 + n^{2})^{s-1}[1 + (n - k)^{2}]^{1-r}}.  
\end{split}
\end{equation}
%
Without loss of generality, we assume $k \ge 0$ and write 
\begin{equation*}
\begin{split}
&  \sum_{n \in \zz}  
\frac{1}{(1 + n^{2})^{s-1}[1 + (n - k)^{2}]^{1-r}}  
\\
& = 
\sum_{0 \le n \le 2k} \frac{1}{(1 + n^{2})^{s-1}[1 + (n - k)^{2}]^{1-r}} 
+ \sum_{n > 2k} \frac{1}{(1 + n^{2})^{s-1}[1 + (n - k)^{2}]^{1-r}}
\\
& + \sum_{n \ge 0} \frac{1}{(1 + n^{2})^{s-1}[1 + (n + k)^{2}]^{1-r}} 
\\
& \doteq I + II + III.
\end{split} 
\end{equation*}
%
We have the estimate
%
%
\begin{equation}
\label{est-tem}
\begin{split}
II 
& \le \sup_{n > 2k} \frac{1}{\left[ 1 + (n-k)^{2} \right]^{1-r}}
\sum_{n > 2k} \frac{1}{(1 + n^{2})^{s-1}} 
\\
& \lesssim (1 + k^{2})^{r-1}, \quad
s > 3/2.
\end{split}
\end{equation}
Similarly
%
%
\begin{equation*}
\begin{split}
III \lesssim (1 + k^{2})^{r-1}, \quad s > 3/2.
\end{split}
\end{equation*}
%
%
%
To estimate $I$, we assume without loss of generality that $k$ is even and write
%
%
\begin{equation*}
\begin{split}
&  I = \sum_{0 \le n \le k/2} \frac{1}{(1 + n^{2})^{s-1}[1 + (n - k)^{2}]^{1-r}} 
+ \sum_{k/2 < n \le 3k/2} \frac{1}{(1 + n^{2})^{s-1}[1 + (n - k)^{2}]^{1-r}} 
\\
& + \sum_{3k/2 < n \le 2k} \frac{1}{(1 + n^{2})^{s-1}[1 + (n - k)^{2}]^{1-r}} 
\\
& \doteq i + ii + iii.
\end{split} 
\end{equation*}
Hence, estimating as in \eqref{est-tem}, we have
%
%
\begin{equation*}
\begin{split}
i, iii \lesssim (1 + k^{2})^{r-1}, \quad
s > 3/2
\end{split}
\end{equation*}
%
and
%
%
\begin{equation*}
\begin{split}
ii & \le \sup_{k/2 \le n \le 3k/2} \frac{1}{\left( 1 + n^{2} \right)^{s-1}}
\sum_{k/2 \le n \le 3k/2} \frac{1}{[1 + (n-k)^{2}]^{1-r}} \\
& \lesssim \frac{1}{(1 + k^{2})^{s-1}}, \quad r \le 1/2.
\end{split}
\end{equation*}
%
%
Therefore, 
%
%
%
\begin{equation*}
\begin{split}
I + II + III & \lesssim (1 + k^{2})^{1-s} + (1 + k^{2})^{r-1}, \quad r \le 1/2, \ s > 3/2
\\
& \lesssim  (1 + k^{2})^{r-1}, \quad r -1 \ge 1-s.
\end{split}
\end{equation*}
%
Applying this estimate to \eqref{opp} and recalling \eqref{np-key-term-iu},
we obtain
%
%
%
%
\begin{equation}
\label{yhh-iu}
\begin{split}
\| f g \|_{H^{r-1}} \lesssim \| f \|_{H^{s-1}} \| g \|_{H^{r-1}},
\quad s > 3/2, \ r \le 1/2, \ s + r \ge 2.
\end{split}
\end{equation}
We now need the following result taken from Taylor \cite{Taylor:1995kx}.
%
%
%%%%%%%%%%%%%%%%%%%%%%%%%%%%%%%%%%%%%%%%%%%%%%%%%%%%%
%
%
%                
%
%
%%%%%%%%%%%%%%%%%%%%%%%%%%%%%%%%%%%%%%%%%%%%%%%%%%%%%
%
%
\begin{lemma}[Sobolev Interpolation]
For fixed $j \le k, m \le n$ suppose that \\ $T: H^{j} \to H^{m}$ continuously
and $T: H^{k} \to H^{n}$. Then\\ $T: H^{\theta j + (1 - \theta)k} \to H^{\theta
m + (1 - \theta) n}$ continuously for all $\theta \in (0,1]$.
\label{prop:sob-interp-iu}
\end{lemma}
%
To apply Lemma~\ref{prop:sob-interp-iu}, we note that \eqref{yhh-iu}
and the algebra property of the Sobolev space $H^{t}$, $t > 1/2$ imply that for $s > 3/2$
%
%
\begin{equation*}
\begin{split}
\| f g \|_{H^{r-1}} \lesssim \| g \|_{H^{r-1}}, \  \text{where} \ 
r=1/2 \ \text{or} \  r =s, \ \| f \|_{H^{s-1}} =1.
\end{split}
\end{equation*}
%
%
That is, for fixed $f \in H^{s-1}$ with $\| f \|_{H^{s-1}} =1$, the map $g \mapsto
Tg = fg$ is linear continuous from $H^{-1/2}$ to $H^{-1/2}$ and from $H^{s-1}$ to
$H^{s-1}$. Therefore, by Lemma~\ref{prop:sob-interp-iu}, it is continuous from
$H^{\theta (s-1) + (1 - \theta)(-1/2) }$ to $H^{\theta (s-1) + (1 - \theta)(-1/2) }$ for all $\theta \in
[0, 1)$. Setting $\theta = (r-1/2)/(s-1/2)$, $ 1/2 \le r < s$, we obtain that $T$ is
continuous from $H^{r-1}$ to $H^{r-1}$. Since $T$ is also linear from $H^{r-1}$
to $H^{r-1}$, we see that 
%
%
\begin{equation*}
\begin{split}
\| f g \|_{H^{r-1}} \lesssim \| g \|_{H^{r-1}}, \quad 1/2 \le r \le s, \ s > 3/2, \ \| f \|_{H^{s-1}} =1
\end{split}
\end{equation*}
and so for general $f \in H^{s-1}$ we have 
%
\begin{equation}
\label{hhh-iu}
\begin{split}
\| f g \|_{H^{r-1}} \lesssim \|f \|_{H^{s-1}}
\| g \|_{H^{r-1}}, \quad 1/2 \le r \le s, \ s > 3/2. 
\end{split}
\end{equation}
%
Combining \eqref{yhh-iu} and \eqref{hhh-iu} completes the proof in the periodic
case. For the non-periodic case we have
%
%
\begin{equation*}
\begin{split}
\| fg\|_{H^{r-1}}^{2}
\le \int_{\rr}  (1 + \xi^{2})^{r-1}\left [ \int_{\rr}
| \wh{f}(\eta) |  | \wh{g}(\xi - \eta) | (1 +
\eta^{2})^{\frac{1-s}{2}} (1 + \eta^{2})^{\frac{s-1}{2}}
d \eta \right ]^{2} d \xi.
\end{split}
\end{equation*}
%
Applying Cauchy Schwartz in $\eta$, we bound this by
%
%
%
\begin{equation*}
\begin{split}
\| f \|_{H^{s-1}}^{2} \int_{\rr}  (1 + \xi^{2})^{r-1}\int_{\rr} \frac{|
\wh{g}(\xi - \eta) |^{2}}{(1 + \eta^{2})^{s-1}} d \eta d \xi.
\end{split}
\end{equation*}
%
We now wish to bound the integral term. Applying a change of variable, we see it
is equal to
%
\begin{equation*}
\begin{split}
\int_{\rr} (1 + \xi^{2})^{r-1} \int_{\rr}
\frac{| \wh{g}(\eta) |^{2}}{[1 + (\xi - \eta)^{2}]^{s-1}} d \eta d \xi
\end{split}
\end{equation*}
which by Fubini is equal to
%
%
\begin{equation}
\label{int-pre-calc-lem-iu}
\begin{split}
& \int_{\rr} | \wh{g}(\eta) |^{2} \int_{\rr} \frac{1}{\left[
1 + (\xi - \eta)^{2} \right]^{s-1} (1 + \xi^{2})^{1-r}} d \xi d \eta
\\
& \lesssim \int_{\rr} | \wh{g}(\eta) |^{2} \int_{\rr} \frac{1}{\left[
1 + |\xi - \eta| \right]^{2(s-1)} (1 + |\xi|)^{2(1-r)}} d \xi d \eta.
\end{split}
\end{equation}
%
%
We now need the following lemma. 
%
%
\begin{lemma}
\label{lem:calc-iu}
%
Fix $p, q > 0$ such that $p +q >1$, and let $r =\min\left\{p - \ee_{q}, q -
\ee_{p}, p+q-1 \right\}$, where $\ee_{j} > 0$ is arbitrarily small for $j = 1$
and $\ee_{j} = 0$ for $j \neq 1$. Adopt the notation
$\langle x - \alpha \rangle  \doteq 1 + | x - \alpha |$. Then 
%
\begin{equation*}
\begin{split}
& \int_{\rr} \frac{1}{\langle x - \alpha \rangle ^{p} \langle x -
\beta \rangle
^{q}} d x
\le \frac{c_{r}}{\langle \alpha - \beta \rangle ^{r}}. 
\end{split}
\end{equation*}
\end{lemma}
To be able to apply the lemma to the integral term in \eqref{int-pre-calc-lem-iu}, 
we must first check
that its conditions are met. Let $ s = 3/2 + \ee$, $r = 1- \delta$, $\ee > 0$, $
\delta \ge 0$ and observe that
%
%
\begin{equation*}
\begin{split}
2(s-1) + 2(1-r)
& = 2(s-r)
\\
& = 2[3/2 + \ee - (1 - \delta)]
\\
& = 2(1/2 + \ee + \delta)
\\
& = 1 + 2 \ee + 2 \delta > 1.
\end{split}
\end{equation*}
%
%
Furthermore, $2(s-1), 2(1-r) > 0$. Hence, Lemma~\ref{lem:calc-iu} is applicable. 
Note that since $s > 3/2$, we see that $2(s-1) \neq 1$. However, it is possible that $2(1-r) =1$; hence we must now separate the cases $r \neq 1/2$ and $r = 1/2$. Suppose $r \neq 1/2$. Then 
%
%
\begin{equation*}
\begin{split}
\min\left\{ 2(s-1), 2(1-r), 2(s-1) + 2(1-r) -1 \right\}
& = \min\left\{ 1 + 2 \ee, 2 \delta, 2\ee + 2 \delta \right\}
\\
& = \min\left\{ 1 + 2 \ee, 2 \delta\right\}
\\
& = 2 \delta, \quad \delta \le 1/2 + \ee.
\end{split}
\end{equation*}
%
If $r = 1/2$, then since $s > 3/2$, we can choose $\eta > 0$ sufficiently small
such that
%
%
\begin{equation*}
\begin{split}
\min\left\{ 2(s-1) -\eta , 2(1-r), 2(1-r) + 2(s-1) - 1  \right\}
& = 1 
\\
& = 2(1 -r)
\\
& = 2\delta.
\end{split}
\end{equation*}
%
Hence, for $0 \le \delta \le 1/2 + \ee$, $\ee >
0$, \eqref{int-pre-calc-lem-iu} is bounded by
\begin{equation*}
\begin{split}
C_{s,r} \int_{\rr}  | \wh{g}(\eta) |^{2} \int_{\rr} \frac{1}{\left( 1
+ |\eta| \right )^{2 \delta}} d \xi d \eta 
& \lesssim
\| g \|_{H^{-\delta}}^{2}
\\
& = \| g \|_{H^{r-1}}^{2}.
\end{split}
\end{equation*}
%
Our restriction on $\delta$ is equivalent to the restriction 
$$1-r \le 1/2 + s - 3/2, \quad r \le 1, \ s > 3/2,$$ or
$$s + r \ge 2,  \quad  r \le 1, \ s > 3/2.$$ Therefore, 
%
%
%
%
\begin{equation*}
\begin{split}
\| f g \|_{H^{r-1}} \lesssim \| f \|_{H^{s-1}} \| g \|_{H^{r-1}},
\quad s + r \ge 2, \ s > 3/2, \ r \le 1.
\end{split}
\end{equation*}
%
%
The remainder of the proof is analogous to that in the periodic case.
\end{proof}
%
%
%
\begin{proof}[Proof of Lemma~\ref{lem:calc-iu}]
%
By the change of variable $x \mapsto x/2 + (\alpha + \beta)/2$, we have
%
%
\begin{equation}
\label{rur}
\begin{split}
\int_{\rr} \frac{1}{\langle x - \alpha \rangle^{p} \langle  x -
\beta
\rangle^{q}}d x
& \simeq \int_{\rr} \frac{1}{\langle x/2 - (\alpha - \beta)/2  \rangle^{p}
\langle  x/2 + (\alpha - \beta)/2 \rangle^{q}} d x
\\
& \lesssim \int_{\rr} \frac{1}{\langle x - (\alpha - \beta)  \rangle^{p}
\langle  x + (\alpha - \beta) \rangle^{q}} d x
\\
& = \int_{\rr} \frac{1}{\langle a - x \rangle ^{p} \langle a + x \rangle
^{q}} d x, \quad a = \alpha - \beta
\end{split}
\end{equation}
%
which for $a =0$ reduces to 
%
%
\begin{equation*}
\begin{split}
\int_{\rr} \frac{1}{\langle x \rangle ^{p+q}} d x 
& = 2 \int_{0}^{\infty} \frac{1}{(1 + x)^{p+q}} d x
\\
& = \frac{2}{p+q -1}.
\end{split}
\end{equation*}
%
%
We now handle the case $a \neq 0$. Note that by the change of variable $x \mapsto
-x$ we may restrict our attention to the case  $a > 0$ without loss of
generality. Split
%
%
\begin{equation*}
\begin{split}
\int_{\rr} \frac{1}{\langle a + x \rangle ^{p} \langle a - x \rangle
^{q}} d x
& = \int_{-2a}^{2a}
\frac{1}{\langle a + x \rangle ^{p} \langle a - x \rangle
^{q}} d x
\\
& + \int_{| x | \ge 2a} 
\frac{1}{\langle a + x \rangle ^{p} \langle a - x \rangle
^{q}} d x
\\
& = I + II.
\end{split}
\end{equation*}
%
%
Then
\begin{equation*}
\begin{split}
I 
& = \int_{0}^{2a}
\frac{1}{\langle a + x \rangle ^{p} \langle a - x \rangle
^{q}} d x + \int_{-2a}^{0}
\frac{1}{\langle a + x \rangle ^{p} \langle a - x \rangle
^{q}} d x.
\end{split}
\end{equation*}
We bound the first term by
\begin{equation*}
\begin{split}
\sup_{0 \le x \le 2a} \frac{1}{\langle a + x \rangle
^{p}} \int_{0}^{2a} \frac{1}{\langle a - x \rangle ^{q}} d x
& = \frac{1}{\langle a \rangle ^{p}} \int_{0}^{2a} \frac{1}{(1 + | a -
x
|)^{q}} d x  
\\
& = \frac{2}{\langle a \rangle ^{p}} \int_{0}^{a} \frac{1}{(1 + a -
x)^{q}} d x
\\
& \lesssim
\begin{cases}
1/{\langle a \rangle ^{p}} \left| 1 - 1/{(1 +
a)^{q -1}} \right|, \quad & q \neq 1
\\
\log(1+a)/{\langle a \rangle^{p} }, \quad & q =1.
\end{cases}
\end{split}
\end{equation*}
%
But
%
%
\begin{equation*}
\begin{split}
\frac{1}{\langle a \rangle ^{p}}\left| 1 - \frac{1}{(1 +
a)^{q -1}} \right|
& \lesssim
\begin{cases}
1/{\langle a \rangle^{p} }, \quad & q > 1
\\
1/{\langle a \rangle ^{p + q -1}}, \quad & q < 1
\end{cases}
\end{split}
\end{equation*}
%
%
and
%
%
\begin{equation*}
\begin{split}
\frac{\log(1 + a)}{\langle a \rangle^{p} } \le  \frac{c_{\ee}}{\langle a
\rangle ^{p - \ee}} \ \text{for any} \ \ee > 0.
\end{split}
\end{equation*}
%
For the second term, we bound by
%
%
\begin{equation*}
\begin{split}
& \sup_{-2a \le x \le 0} \frac{1}{\langle a - x \rangle
^{q}} \int_{-2a}^{0} \frac{1}{\langle a + x \rangle ^{p}} d x
\\
& = \frac{1}{\langle a \rangle ^{q}} \int_{-2a}^{0} \frac{1}{(1 + | a +
x
|)^{p}} d x 
\\
& = \frac{2}{\langle a \rangle ^{q}} \int_{-a}^{0} \frac{1}{(1 + a +
x)^{p}} d x
\\
& \lesssim
\begin{cases}
1/{\langle a \rangle ^{q}} \left| 1 - 1/{(1 +
a)^{p -1}} \right|, \quad & p \neq 1
\\
\log(1+a)/{\langle a \rangle^{q} }, \quad & p =1.
\end{cases}
\end{split}
\end{equation*}
%
But
%
%
\begin{equation*}
\begin{split}
\frac{1}{\langle a \rangle ^{q}}\left| 1 - \frac{1}{(1 +
a)^{p -1}} \right|
& \lesssim
\begin{cases}
1/{\langle a \rangle ^{q}}, \quad & p > 1
\\
1/{\langle a \rangle ^{p + q -1}}, \quad & p < 1
\end{cases}
\end{split}
\end{equation*}
%
%
and
%
%
\begin{equation*}
\begin{split}
\frac{\log(1 + a)}{\langle a \rangle^{q} } \le  \frac{c_{\ee}}{\langle a
\rangle ^{q - \ee}} \ \text{for any} \ \ee > 0.
\end{split}
\end{equation*}
%
%
%
Therefore,
\begin{equation*}
I \le  \frac{c_{p,q, \ee}}{\langle a \rangle ^{\min\left\{ p-\ee_{q}, q -\ee_{p}, p + q-1 \right\}}}.
\end{equation*}
%
%
Also
%
%
\begin{equation*}
\begin{split}
II 
& = \int_{x \ge 2a} \frac{1}{(1 + x - a)^{p} (1 + x +
a)^{q}} d x
\\
& \le \int_{x \ge 2a} \frac{1}{(1 + x -a)^{p+q}} d x
\\
& \simeq \frac{1}{\langle a \rangle^{p+q -1}}, \qquad p + q > 1.
\end{split}
\end{equation*}
%
%
Collecting our estimates for $I$ and $II$ we see that for 
$p, q > 0$ such that $p +q >1$, and $r =\min\left\{p -\ee_{q}, q - \ee_{p}, p+q-1
\right\}$, we have
\begin{align*}
\int_{\rr} \frac{1}{\langle a - x \rangle ^{p} \langle a + x \rangle
^{q}} d x
\le \frac{c_{r}}{\langle a \rangle ^{r}}.
\label{est-2}
\end{align*}
Recalling \eqref{rur}, the proof is complete.
\end{proof}
%
%
