\section{Well-Posedness for HR in the Periodic Case}
%
%
%
%
We will now prove well-posedness for the periodic case, after which we will
provide the necessary details to extend the argument to the non-periodic case.
\subsection{Existence.}
\label{existence}
Here we will prove the existence of a solution to the HR i.v.p. and inequalities
\eqref{Life-span-est} and \eqref{u_x-Linfty-Hs}.  We begin by mollifying the HR equation, so that we may apply the following ODE
theorem: 
%
\begin{theorem}
	\label{ode_theorem}
	Let  $Y$  be a Banach space, $X\subset Y$ be an open subset,
	$I' \subset \rr$, and $f: I' \times X\to Y$ a continuously differentiable
	map.  Then for any $t_{0} \in I'$ and $x_{0} \in X$ there exists an
	open ball $I \subset I'$ and a unique differentiable mapping $u:I
	\to Y$ such that for all $t \in I$,  $u'(t) = f(t, u)$
	and $u(t_{0}) = x_{0}.$
\end{theorem}
%
To see why we cannot apply the Banach Space ODE Theorem to the HR equation as is,
we use a counterexample. Let $u=x^{-1/2} \chi_{[0,1]}$ and $s=0$. Then $u \in H^s$ but
$u\p_x u \notin H^s$. Hence, returning to the general case, we see that the
HR equation as is can not be thought as an ODE on the space $H^s$.  To
deal with this problem we will replace the i.v.p \eqref{hr}--\eqref{hr-data} by  
\begin{equation}
	\label{hr-moli}
	\p_t  u_\ee =
	-\gamma J_\ee u_\ee \partial_x  J_\ee  u_\ee - \p_x (1-\p_x^2)^{-1} 
	\left [\frac{3-\gamma}{2}u^2 + \frac{\gamma}{2}(\p_x u)^2 \right ],
\end{equation} 
%
\begin{equation} 
	\label{hr-moli-data} 
	u_\ee(x, 0) = u_0 (x),
\end{equation}
%
where $J_\ee$ is defined as follows: Pick a non-negative $j(x) \in
\mathcal{S}(\rr)$ and let
\begin{equation*}
	\begin{split}
		j_\ee(x) = \frac{1}{\ee}j\left( \frac{x}{\ee} \right).
	\end{split}
\end{equation*}
	We then define $J_\ee$ to be the ``Friedrichs mollifier''
	\begin{equation}
		\begin{split}
			J_\ee f(x) = j_\ee * f(x), \quad \ee>0.
		\end{split}
	\end{equation}
%
%
Notice that the right hand side of \eqref{hr-moli} is a map from $H^s(\ci)$
to $H^s(\ci)$.  In order to apply the ODE Theorem, we will also need to
show that it is a continuously differentiable map:
%
%
%
\begin{lemma}
	Let $f_\ee:H^s(\ci) \to H^s(\ci)$ be given by 
	\begin{equation}
		\label{f_ep}
		f_{\ee}(u) = -\gamma  J_\varepsilon u \partial_x J_\varepsilon u
		- \p_x (1-\p_x^2)^{-1} \left
		[\frac{3-\gamma}{2}u^2 + \frac{\gamma}{2}(\p_x u)^2 \right ].
	\end{equation}
	Then $f_\ee$  is a continuously differentiable map.
\end{lemma}
%
%
\subsection{ Proof.} We explicitly calculate the derivative of $f_\ee$ at an
arbitrary $w \in H^s(\ci)$:
\begin{equation*}
	\begin{split}
		[Df_{\ee}(u)](w)
		=
		& -\gamma (J_\varepsilon w \cdot \partial_x J_\varepsilon u +
		J_\varepsilon u \cdot \partial_x J_\varepsilon w)
		\\
		& - (1-\p_x ^2)^{-1}
		\p_x \left [(3-\gamma)w u + \gamma\p_x w \p_x u \right ].
	\end{split}
\end{equation*}
Let $w_n \xrightarrow{H^s(\ci)} w$. Then it is easy to check that
%
\begin{equation}
	\begin{split}
		& -\gamma (J_\varepsilon w_n \cdot \partial_x J_\varepsilon u 
		+ J_\varepsilon u \cdot \partial_x J_\varepsilon w_n)
		+ (1-\p_x ^2)^{-1}
		\p_x \left [(3-\gamma)w_n u + \gamma\p_x w_n \p_x u \right ]
		\\
		& \xrightarrow{H^s(\ci)} 
		 -\gamma (J_\varepsilon w \cdot \partial_x J_\varepsilon u 
		+ J_\varepsilon u \cdot \partial_x J_\varepsilon w) + (1-\p_x ^2)^{-1}
		\p_x \left [(3-\gamma)w u + \gamma\p_x w \p_x u \right ].
	\end{split}
\end{equation}
This concludes the proof. $\quad \square$
Hence, by Theorem \ref{ode_theorem}, for each $\ee > 0$ there exists a
unique solution $u_\ee \in C(I, H^s(\ci))$ satisfying the Cauchy-problem
\eqref{hr-moli}-\eqref{hr-moli-data}. Next, we analyze the size and
lifespan of the family $\{u_\ee\}$ of solutions.
%%%%%%%%%%%%%%%%%%%%%%%%
%
%     Estimates  for Life-span and Sobolev norm of $u_\ee$
%
%%%%%%%%%%%%%%%%%%%%%%%%
%
%
\subsection{ Estimates  for Life-span and Sobolev norm of $u_\ee$.}
%
We will show that there is a lower bound  $T$
for $T_\ee$, which is  independent of $\ee\in(0, 1]$.
This is based on the following differential
inequality for the solution $u_\ee$:
%
\begin{equation} 
	\label{B-diff-ineq}
	\frac 12
	\frac{d}{dt}
	\|u_\ee(t)\|_{H^{s}(\ci)}^2
	\le
	c_s
	\|u_\ee(t)\|_{H^{s}(\ci)}^3,
	\quad
	|t| \le T_\ee.
\end{equation}
%
%
We will prove this inequality  by
following the approach used for quasilinear symmetric
hyperbolic systems in Taylor \cite{Taylor_1991_Pseudodifferent}. In what follows we will suppress the
$t$ parameter for the sake of clarity.
%
For any $s\in \ci$ let   $D^s=(1-\p_x^2)^{s/2}$ be the  operator
defined by 
%
$$ \widehat{D^s f}(\xi) \doteq (1 + \xi^2)^{s/2} \widehat{f}(\xi), $$
%
where $ \widehat{f}$ is the Fourier transform
%
$$ \widehat{f}(\xi) =  \frac{1}{2\pi}\int_{\ci} e^{-i \xi x} f(x) \ dx.  $$
%
Applying the operator $D^s$ to  both sides of  \eqref{hr-moli},
then  multiplying the resulting equation by $D^s J_\ee u_\ee$
and integrating it for $x\in\ci$ gives
%
\begin{equation} 
	\begin{split}
		\label{B-moli-int}
		\frac 12
		\frac{d}{dt} \|u_\ee \|_{H^s}^2
		=
		&-
		\gamma \int_{\ci}  D^s(J_\ee u_\ee \partial_x J_\ee u_\ee) \cdot
		D^s J_\ee u_\ee  \  dx
		\\
		&- \frac{3 -\gamma}{2} \int_{\ci} D^{s-2} \p_x (u_{\ee}^2) 
		\cdot D^s J_\ee u_{\ee} \ dx
		\\
		&- \frac{\gamma}{2} \int_{\ci}  D^{s-2} \p_x (\p_x u_\ee)^2
		\cdot D^s J_\ee u_\ee  \ dx.
	\end{split}
\end{equation}
%
We will estimate the right hand side of \eqref{B-moli-int} in parts. In
what follows next we use the fact that  $D^s$ and $J_\ee$ commute and
that  $J_\ee$ satisfies the properties 
%
\begin{equation} 
	\label{J-e-inner-prod-property}
	(J_\ee f, g)_{L^2(\ci)}=( f, J_\ee g)_{L^2(\ci)}
\end{equation}
%
and
%
\begin{equation} 
	\label{Je-u-Hs}
	\| J_\ee u \|_{H^s(\ci)}
	\le
	\|  u \|_{H^s(\ci)}.
\end{equation}
%
%%%%%%%%%%%% Burgers term energy estimate %%%%%%%%%%%%
%
%
%
\noindent
Letting 
%
\begin{equation} 
	\label{v-Je-ue}
	v=J_\ee u_\ee
\end{equation}
%
%
we have
%
\begin{equation} 
	\begin{split}
		\label{B-moli-int-v}
		& -  \gamma \int_{\ci}   D^s (J_{\ee} u_{\ee} \p_x J_\ee u_\ee)
		 \cdot D^s
		J_{\ee}u_\ee \ dx  
		\\
		& = - \gamma \int_\ci
		 D^s(v \partial_x v) \cdot   D^s v \ dx
		\\
		& = - \gamma \int_\ci
		\left [ 
		D^s(v\p_x v)  -  v D^s (\p_xv)
		\right ] 
		D^s v \ dx - \gamma \int_\ci
		v D^s (\p_xv)
		D^s v \ dx.
	\end{split}
\end{equation}
%
%
%
We now estimate \eqref{B-moli-int-v} in parts. Applying the Cauchy-Schwarz inequality gives
%
\begin{equation} 
	\label{int1-est-calc2}
	\begin{split}
		& \Big|
		- \gamma \int_\ci
		\big[ 
		D^s(v\p_x v)  -  v D^s (\p_xv)
		\big]
		D^s v   \, dx
		\Big|
		\\
		& \le
		|\gamma| \cdot \|
		D^s(v\p_x v)  -  v D^s (\p_xv)
		\|_{L^2(\ci)}
		\|
		D^s v 
		\|_{L^2(\ci)}
		\\
		&\le
		|\gamma| \cdot \|
		D^s(v\p_x v)  -  v D^s (\p_xv)
		\|_{L^2(\ci)}
		\|
		v
		\|_{H^s(\ci)}
		\\
		&\le c_s \| \p_x v \|_{L^\infty(\ci)} 
		\| v \|_{H^s(\ci)}^2,
	\end{split}
\end{equation}
%
where the last step follows from 
%
\begin{equation} 
	\label{int1-est-calc3}
	\| D^s(v\p_x v)  -  v D^s (\p_xv) \|_{L^2(\ci)}
	\le
	2 c_s^{\prime}    \| \p_x v \|_{L^\infty(\ci)} 
	\| v \|_{H^s(\ci)},
\end{equation}
which we prove below by using the following Kato-Ponce commutator 
estimate:  
\begin{lemma} 
	\label{KP-lemma}
	[Kato-Ponce]
	If  $s>0$ then there is $c_s^{\prime}>0$ such that 
	%
	\begin{equation} 
		\label{KP-com-est}
		\| D^{s} \big(fg) -  f D^s g\|_{L^2(\ci)}
		\le
		c_s^{\prime}\big(
		\| D^{s}f \|_{L^2(\ci)}    \| g \|_{L^\infty(\ci)} 
		+
		\| \p_xf \|_{L^\infty(\ci)}    \| D^{s-1}g \|_{L^2(\ci)}   
		\big).
	\end{equation}
	%
	\end{lemma}
	%
	%
	In fact, applying  this estimate with $f=v$ and $g=\p_xv$ gives 
	%
	\begin{equation} 
		\label{int1-est-calc4}
		\begin{split}
			& \| D^s(v\p_x v)  -  v D^s (\p_xv) \|_{L^2(\ci)}
			\\
			& \le
			{c_s}^\prime \big(
			\| D^{s}v \|_{L^2(\ci)}    \| \p_x v \|_{L^\infty(\ci)} 
			+
			\| \p_xv \|_{L^\infty(\ci)}    \| D^{s-1}\p_x v \|_{L^2(\ci)}   
			\big)
			\\
			& \le
			2{c_s}^\prime    \| \p_x v \|_{L^\infty(\ci)} 
			\| v \|_{H^s(\ci)}, 
		\end{split}
	\end{equation}
	%
	which  is the desired estimate  \eqref{int1-est-calc3}.
	Next, we have
	%
	%
	%
	\begin{equation} 
		\label{int1-est-calc5}
		\begin{split}
			\Big|
			-\gamma \int_\ci
			v D^s (\p_x v)
			\cdot  D^s v \ dx
			\Big|
			& =
			\left | \frac{\gamma}{2} \right | \cdot \Big|
			\int_\ci
			v \p_x\left(D^s v\right)^2  dx
			\Big|
			\\
			& =
			\left | \frac{\gamma}{2} \right | \cdot \Big | \int_\ci
			\p_x v \, (D^s v)^2 \ dx
			\Big|
			\\
			& \le
			\left | \frac{\gamma}{2} \right |  \cdot \int_\ci
			\Big | \p_x v \, (D^s v)^2   
			\Big| \ dx
			\\
			& \lesssim
			\| \p_x v \|_{L^\infty(\ci)} 
			\| v \|_{H^s(\ci)}^2.
		\end{split}
	\end{equation}
	%
	%
	%
	Combining inequalities  \eqref{int1-est-calc2} and
	\eqref{int1-est-calc5} and applying the Sobolev Imbedding Theorem, we
	have
	%
	\begin{equation} 
		\label{burgers_est'}
		\begin{split}
			\Big|
			-\gamma \int_\ci
			D^s(v \partial_x v) \cdot   D^s v \, dx  
			\Big|
			&\le
			{c_s}^\prime
			\| \p_x v \|_{L^\infty(\ci)} 
			\|  v \|_{H^s(\ci)}^2
			\\
			& \le {c_s}^\prime \| v \|_{C^1(\ci)} \| v \|_{H^s(\ci)}^2
			\\
			& \le {c_s}^{\prime \prime} \| v \|_{H^s(\ci)}^3
			\\
			& \le {c_s}^{\prime \prime} \| u_\ee \|_{H^s(\ci)}^3.
		\end{split}
	\end{equation}
	%
	Next we estimate
	\begin{equation}
		\begin{split}
			\left | - \frac{3 -\gamma}{2} \int_\ci D^{s-2} \p_x u_\ee^2 \cdot
			D^s J_\ee u_\ee \; dx \right |
			& \le \left | \frac{3- \gamma}{2} \right | \int_\ci \left |
			D^{s-2} \p_x u_\ee^2 \cdot D^s J_\ee u_\ee \; dx \right | 
			\\
			& \le \left | \frac{3- \gamma}{2} \right |
			\|D^{s-2} \p_x u_\ee^2 \|_{L^2(\ci)} 
			\|D^s J_\ee u_\ee \|_{L^2(\ci)}
			\\
			& \le \left | \frac{3- \gamma}{2} \right |
			\|D^{s-1} u_\ee^2 \|_{L^2(\ci)} 
			\|D^s u_\ee \|_{L^2(\ci)}
			\\
			& \lesssim \| u_\ee^2 \|_{H^s(\ci)} \| u_\ee \|_{H^s(\ci)}.
		\end{split}
	\end{equation}
	%
	%
	Applying the algebra property, we obtain
	%
	\begin{equation}
		\label{hl1}
		\begin{split}
			\left | - \frac{3 -\gamma}{2} \int_\ci D^{s-2} \p_x u_\ee^2 \cdot
			D^s J_\ee u_\ee \; dx \right |
			\lesssim \| u_\ee \|_{H^s(\ci)}^3.
		\end{split}
	\end{equation}
	%
	%
	We also have
	\begin{equation}
		\begin{split}
			\left |- \frac{\gamma}{2} \int_\ci D^{s-2} \p_x (\p_x u_\ee)^2 \cdot
			D^s J_\ee u_\ee \; dx \right |
			& \le \left | \frac{\gamma}{2} \right | \int_\ci \left | D^{s-2} \p_x (\p_x u_\ee)^2 \right |
			\cdot \left |D^s J_\ee u_\ee \right | \; dx
			\\
			& \le \left | \frac{\gamma}{2} \right |
			\| D^{s-1} (\p_x u_\ee)^2 \|_{L^2(\ci)}
			\| D^s J_\ee u_\ee \|_{L^2(\ci)}
			\\
			& \lesssim \|(\p_x u_\ee)^2 \|_{H^{s-1}(\ci)}
			\| J_\ee u_\ee \|_{H^{s-1}(\ci)} 
			\\
			& \lesssim \|(\p_x u_\ee)^2 \|_{H^{s-1}(\ci)} \| u_\ee \|_{H^{s-1}(\ci)} 
		\end{split}
	\end{equation}
	and applying the algebra property yields
	\begin{equation}
		\label{hl2}
		\begin{split}
		\left | - \frac{\gamma}{2} \int_\ci D^{s-2} (\p_x u_\ee)^2 \cdot
		D^s J_\ee u_\ee \; dx \right |
		& \lesssim \| \p_x u_\ee \|_{H^{s-1}(\ci)}^2 \| u_\ee \|_{H^s(\ci)} 
		\\
		& \lesssim \|u_\ee\|_{H^s(\ci)}^3.
	\end{split}
	\end{equation}
	%
	Combining \eqref{burgers_est'}, \eqref{hl1}, and \eqref{hl2}, we obtain
	\eqref{B-diff-ineq}.
	%%%%%%%%%%%%%%%%%%%%%%%%%%%%%%%%%%%
	%  
	%           Lifespan for CH  solution    
	% 
	%%%%%%%%%%%%%%%%%%%%%%%%%%%%%%%%%%%
	%
	%
	%   
	%
	\noindent
	\subsection{  Lifespan estimate of $u_\ee$.} To derive an explicit formula for
	$T_\ee$ we proceed as follows.  Letting  $y(t)=
	\|u_\ee(t)\|_{H^s(\ci)}^2$ inequality  \eqref{B-diff-ineq} takes the
	form
	%
	\begin{equation} 
		\label{energy-y-ineq}
		\frac 12
		y^{-3/2}\frac{dy}{dt}
		\le
		c_s,
		\qquad
		y(0)=y_0=  \|u_0\|_{H^s(\ci)}^2.
	\end{equation}
	%
	Suppose $t$ is non-negative. Integrating  \eqref{energy-y-ineq} from  0  to $t$ gives
	%
	\begin{equation} 
		\label{energy-y-ineq-calc1}
		\frac{1}{\sqrt{y_0}}  - \frac{1}{\sqrt{y(t)}} 
		\le
		c_s t.
	\end{equation}
	%
	%
	Replacing $y(t)$ with   $\|u_\ee(t)\|_{H^s(\ci)}^2$  and solving for  $\|u_\ee(t)\|_{H^s(\ci)}$
	we obtain the formula
	%
	\begin{equation} 
		\label{norm-u(t)-formula}
		\|u_\ee(t)\|_{H^s(\ci)}
		\le
		\frac{ \|u_0\|_{H^s(\ci)}}{1-c_s\|u_0\|_{H^s(\ci)} t}, \quad t\ge
		0.
	\end{equation}
	%
	Now, from \eqref{norm-u(t)-formula} we see that  $\|u_\ee(t)\|_{H^s(\ci)}$ is finite  if 
	%
	\begin{equation*} 
		\label{Lifespan-calc1}
		c_s    \|u_0\|_{H^s(\ci)} t<1,
	\end{equation*}
	%
	or
	%
	\begin{equation} 
		t
		<
		\frac{1}{ c_s \|u_0\|_{H^s(\ci)}}.
	\end{equation}
	%
	Similarly, if $t$ is negative, then 
	\begin{equation} 
		\label{norm-u(t)-formula-prime}
		\|u_\ee(t)\|_{H^s(\ci)}
		\le
		\frac{ \|u_0\|_{H^s(\ci)}}{1+c_s\|u_0\|_{H^s(\ci)} t}, \quad t < 0.
	\end{equation}
	from which it follows that $\|u_\ee(t)\|_{H^s(\ci)}$ is finite  if 
	%
	\begin{equation} 
		t
		>
		 \frac{-1}{ c_s \|u_0\|_{H^s(\ci)}}.
	\end{equation}
	Therefore, the  solution  $u_\ee(t)$ to the mollified CH Cauchy
	problem exists for $|t| <T_0$, where
	%
	\begin{equation} 
		\label{CH-Lifespan}
		T_0
		=
		\frac{1}{ c_s \|u_0\|_{H^s(\ci)}}.
	\end{equation}
	%
	%%%%%%%%%%%%%%%%%%%%%%%%%%%%%%%%%%%
	%  
	%            Norm of   
	% 
	%%%%%%%%%%%%%%%%%%%%%%%%%%%%%%%%%%%
	%
	%
	%   
	%
	\noindent
	\subsection{Size of the solution estimate} If we choose  $T=\frac12 T_0$, that is
	%
	\begin{equation} 
		\label{T-def}
		T
		=
		\frac{1}{2 c_s \|u_0\|_{H^s(\ci)}},
	\end{equation}
	%
	then for $|t| \le T$, estimates \eqref{norm-u(t)-formula} and
	\eqref{norm-u(t)-formula-prime} imply 
	%
	\begin{equation*} 
		\label{u(t)-u(0)-bound}
		\|u_\ee(t)\|_{H^s(\ci)}
		\le
		\frac{ \|u_0\|_{H^s(\ci)}}{1-(c_s\|u_0\|_{H^s(\ci)})/(2 c_s \|u_0\|_{H^s(\ci)})},
	\end{equation*}
	%
	or 
	%
	\begin{equation} 
		\|u_\ee(t)\|_{H^s(\ci)}
		\le
		  2 \|u_0\|_{H^s(\ci)},
		\quad 
		|t| \le T.
	\end{equation}
	%
	Thus we have obtained a lower bound for $T_\ee$ and an upper bound for
	$\|u_\ee(t)\|_{H^s(\ci)}$ independent of $\ee\in (0, 1]$. The following
	lemma summarizes these results and provides an estimate for the
	$H^{s-1}(\ci)$ norm of $\p_t u_\ee(t)$:
	%
	%
	\begin{lemma}
		\label{hr_wp}
		Let  $u_0(x) \in  H^s(\ci)$, $s >3/2$. Then for any $\ee\in (0, 1]$
		the i.v.p. for the mollified HR equation 
		%
		\begin{equation} 
			\label{hr-moli-2}
			\partial_t  u_\ee 
			=
			-\gamma (J_\ee u_\ee \partial_x  J_\ee  u_\ee) - \p_x (1-\p_x^2)^{-1} \left
			[\frac{3-\gamma}{2}(u_\ee)^2 + \frac{\gamma}{2}(\p_x u_\ee)^2
			\right ], 
		\end{equation} 
		%
		\begin{equation} 
			\label{burgers-moli-data-2} 
			u_\ee(x, 0) = u_0 (x),
		\end{equation}
		%
		has a unique solution $u_\ee( t)\in C([-T, T]; H^s(\ci))$. 
		In particular,
		%
		\begin{equation} 
			\label{life-est}
			T
			=
			\frac{1}{2 c_s \|u_0\|_{H^s(\ci)}},
		\end{equation}
		%
		is independent of $\ee$ and
		is a lower bound for the lifespan of $u_\ee( t)$ and
		%
		\begin{equation}
			\label{u-e-Hs-bound}
			\|u_\ee(t)\|_{H^s(\ci)}
			\le
			2 \|u_0 \|_{H^s(\ci)},
			\quad
			|t| \le T.
		\end{equation}
		%
		Furthermore,  $u_\ee( t)\in C^1([T, T]; H^{s-1}(\ci))$ and 
		satisfies
		\begin{equation}
			\label{dt-u-e-Hs-bound}
			\|\p_t u_\ee(t)\|_{H^{s-1}(\ci)}
			\lesssim
			\|u_0 \|_{H^s(\ci)}^2,
			\quad
			|t| \le T.
		\end{equation}
		% 
		Here  $c_s$ is a constant depending only on $s$.
	\end{lemma}
	%
	%
	\subsection{ Proof.}  It suffices to prove  \eqref{dt-u-e-Hs-bound}.
	Using equation \eqref{hr-moli-2}, for any $t\in [-T, T]$ we have
	%
	\begin{equation*}
		\begin{split}
			& \| \partial_t u_\varepsilon(t) \|_{H^{s-1}(\ci)}  
			\\
			& = 
			\| -\gamma (J_\ee u_\ee \partial_x  J_\ee  u_\ee) -
			\p_x (1-\p_x^2)^{-1} \left [\frac{3-\gamma}{2} (u_\ee)^2 +
			\frac{\gamma}{2}(\p_x u_\ee)^2 \right ] \|_{H^{s-1}(\ci)}
			\\
			& \lesssim  
			\| J_\ee u_\ee \partial_x  J_\ee  u_\ee \|_{H^{s-1}(\ci)}
			+ \|\p_x (1-\p_x^2)^{-1} (u_\ee)^2 \|_{H^{s-1}(\ci)}
			\\
			& + \| \p_x (1-\p_x^2)^{-1}(\p_x u_\ee)^2\|_{H^{s-1}(\ci)}.
			\end{split}
		\end{equation*}
		We break this into three parts:
		\begin{equation}
			\label{bixi}
			\begin{split}
				\| J_\ee u_\ee \p_x J_\ee u_\ee \|_{H^{s-1}(\ci)}
				& = 
				\frac{1}{2}\|\p_x[(J_\varepsilon u_\varepsilon
				)^2]\|_{H^{s-1}(\ci)}
				\\
				& \lesssim \|(J_\varepsilon u_\varepsilon )^2\|_{H^s(\ci)}.
			\end{split}
		\end{equation}
		Applying the algebra property of Sobolev spaces and estimate
		\eqref{u-e-Hs-bound} to \eqref{bixi} gives 
		%
		\begin{equation}
			\label{deriv1}
			\begin{split}
				\|J_\ee u_\ee \p_x J_\ee u_\ee  
				\|_{H^{s-1}(\ci)}
				& \lesssim
				\|J_\varepsilon u_\varepsilon \|_{H^s(\ci)}^2
				\\
				&\lesssim
				\| u_\varepsilon \|_{H^s(\ci)}^2
				\\
				&\lesssim
				\|u_0\|_{H^s(\ci)}^2.
			\end{split}
		\end{equation}
		We also have
		\begin{equation*}
			\begin{split}
				\|\p_x (1-\p_x^2)^{-1} (u_\ee)^2\|_{H^{s-1}(\ci)}
				& \le \| (u_\ee)^2\|_{H^{s-1}(\ci)}
				\end{split}
		\end{equation*}
		which by the algebra property and estimate \eqref{u-e-Hs-bound}
		gives
		\begin{equation}
			\begin{split}
				\label{deriv2}
				\|\p_x (1-\p_x^2)^{-1} (u_\ee)^2\|_{H^{s-1}(\ci)}
				& \lesssim \|u_\ee\|^2_{H^s(\ci)} 
				\\
				& \lesssim  \|u_0\|^2_{H^s(\ci)}.
			\end{split}
		\end{equation}
		Similarly,
		\begin{equation}
			\begin{split}
				\label{deriv3}
				\|\p_x (1-\p_x^2)^{-1} (\p_x u_\ee)^2\|_{H^{s-1}(\ci)}
				& \lesssim \|\p_x u_\ee\|^2_{H^{s-1}(\ci)} 
				\\
				& \lesssim  \|u_\ee \|^2_{H^s(\ci)}
				\\
				& \lesssim \|u_0\|^2_{H^s(\ci)}.
			\end{split}
		\end{equation}
		Combining \eqref{deriv1}, \eqref{deriv2}, and \eqref{deriv3}, we
		obtain \eqref{dt-u-e-Hs-bound}. $\qquad \Box$
		%%%%%%%%%%%%%%%%%%%%%%%%
		%
		%     Choosing  a convergent subsequence
		%
		%%%%%%%%%%%%%%%%%%%%%%%%
		\subsection{ Choosing  a convergent subsequence.}
		%
		Next we shall show that  the family $\{ u_\ee\}$ has a convergent subsequence
		whose limit $u$ solves the Hyperelastic i.v.p. 
		Let
		$$
		I= [-T, T].
		$$
		By Lemma \ref{hr_wp} we have 
		%
		\begin{equation}
			\label{C-1-fam}
			\{u_\ee\}\subset C(I, H^s(\ci))\cap C^1(I, H^{s-1}(\ci))
		\end{equation}
		%
		and bounded. Since $I$ is compact, we have  
		%
		\begin{equation}
			\label{Lip-1-fam}
			\{u_\ee\}\subset L^{\infty}(I, H^s(\ci))\cap C^1(I,
			H^{s-1}(\ci)).
		\end{equation}
		%
		Now, by the Riesz Lemma, we can identify $H^s(\rr)$ with
		$(H^s(\rr))^*$, where for $w, \psi \in H^s(\rr)$ the duality is
		defined by 
		\begin{equation*}
			T_w(\psi) = <w, \psi>_{H^s(\rr)}.
		\end{equation*}
		Hence, by the Riesz Representation Theorem it follows that we can
		identify \\ $L^\infty(I, H^s(\ci)) $ with the dual space of $L^1(I,
		H^{s}(\ci)$, where for $v\in L^\infty(I, H^s(\ci)) $ and $ \phi \in
		L^1(I, H^{s}(\ci))$ the duality is defined by  
		%
		\begin{equation}
			T_v(\phi) = \int_I <v (t), \phi (t)>_{H^s(\rr)} dt  = \int_I
			 \int_{\rr}
			 \widehat{v}(\xi, t) \overline{\widehat{\phi}}(\xi, t) \cdot (1
			 + \xi^2)^s \ d \xi dt.
		\end{equation}
		%
		Next, we recall Aloaglu's Theorem:
		\begin{theorem}
			If $X$ is a normed vector space,
			the closed unit ball $B^* = \{f \in X^* : \|f\| \le
			1\}$ in $X^*$ is compact in the $weak^*$ topology.
		\end{theorem}
		Therefore the bounded family $\{u_\ee\}$ is compact 
		in the weak$^*$ topology of \\
		$L^\infty(I, H^s(\ci))$. More precisely,
		there is a sequence  $\{ u_{\ee_n} \}$ converging
		weakly to a $ u\in L^{\infty}(I, H^s(\ci))$;
		that is 
		%
		\begin{equation}
			\label{weak-conv}
			\lim_{n\to \infty} T_{u_{\ee_n}}(\phi)  =  T_u (\phi) 
			\; \;		
			\text{ for all } \;\;  \phi \in L^1(I, H^{s}(\ci)).
		\end{equation}
		%
		In order to show that  $u$ solves the HR i.v.p. we need to 
		obtain a stronger  convergence for  $u_{\ee_n}$ so that 
		we can take the limit in the mollified HR equation.
		In fact we will prove that 
		%
		\begin{equation}
			\label{strong-conv}
			u_{\ee_n}\longrightarrow u
			\quad
			\text{ in } \,\,   C(I, H^{s-\sigma}(\ci)),\ \text{for any} \
			\, 0 < \sigma <
			1.
		\end{equation}
		%
		For this we will need the following interpolation  result:
		%%%%%%%%%%%%%%%%%%%%%%%%%%%
		%
		%
		%                 Interpolation Lemma
		%
		%
		%%%%%%%%%%%%%%%%%%%%%%%%%%%
		\begin{lemma}
			\label{interpolation-lem}
			(Interpolation)     Let  $s > \frac{3}{2}$.
			If $v \in C(I, H^s(\ci)) \cap C^1(I, H^{s-1}(\ci))$
			then $v \in C^\sigma (I, H^{s- \sigma}(\ci))$ for  $0 < \sigma < 1$.
		\end{lemma}
		%
		\subsection{ Proof.}  We have
		\begin{equation*}
			\begin{split}
				& \frac{\|v(t) - v(t')\|^2_{H^{s - \sigma}}}{|t - t'|^{2\sigma}}
				\\
				& = 
				\sum_{\xi \in \zz} (1 + \xi^2)^{s- \sigma} 
				\frac{|\hat{v}(\xi, t) - \hat{v}(\xi, t')|^2}{|t-t'|^{2\sigma}} d\xi\\
				& = \sum_{\xi \in \zz} (1+\xi^2)^s 
				\bigg(\frac{1}{(1+ \xi^2)|t - t'|^2} \bigg)^\sigma |\hat{v}(\xi, t)- \hat{v}(\xi, t')|^2 d\xi\\
				& \leq \sum_{\xi \in \zz}(1+\xi^2)^s \bigg( 1 + \frac{1}{(1+\xi^2)|t-t'|^2} \bigg)
				|\hat{v}(\xi,t) - \hat{v}(\xi,t')|^2 d\xi \\
				& \leq \sum_{\xi \in \zz} (1+ \xi^2)^s |\hat{v}(\xi, t)- \hat{v}(\xi, t')|^2 d\xi
				+ \sum_{\xi \in \zz} (1+ \xi^2)^{s-1} \frac{|\hat{v}(\xi, t) - \hat{v} (\xi, t')|^2}{|t-t'|^2} \\
				& \leq  \sup_t \|v(t)\|_{H^s(\ci)}^2 + \sup_t
				\| \partial_t v(t) \|_{H^{s-1}(\ci)}^2
				\\
				& < \infty.
				\\
			\end{split}
		\end{equation*}
		%
		%
		Next, using this lemma we will show that the family $\{u_\ee\}$ is
		equicontinuous in $C(I, H^{s-\sigma}(\ci))$, $0 < \sigma < 1$. We
		will follow this by proving that there exists a sub-family
		$\{u_{\ee_n} \}$ that is precompact in $C(I,
		H^{s-\sigma}(\ci))$. These two facts, in conjunction with Ascoli's
		Theorem, will yield
		\begin{equation}
			\label{strong-conv2}
			u_{\ee_n} \to u \; \; \text{in} \; \; C(I,H^{s-\sigma}(\ci)),
			\quad
			0 < \sigma < 1.
		\end{equation}
		%%%%%%%%%%%%%%%%%%%%%%
		%
		%
		%       Equicontinuity
		%
		%
		%%%%%%%%%%%%%%%%%%%%%%
		%
		\subsection{  Equicontinuity of $\{u_\ee\}_\ee$  in
		$C(I,H^{s-\sigma}(\ci))$.} Applying  Lemma \ref{interpolation-lem} gives 
		%
		\begin{equation}
			\label{equic-1}
			\sup_{t \neq t'} \frac { \|u_\ee(t) - u_\ee(t') \|_{H^{s -
			\sigma}(\ci)}}{|t - t'|^\sigma} < c<\infty
		\end{equation}
		%
		or
		%
		\begin{equation}
			\label{equic-2}
			\|u_\ee(t) - u_\ee(t') \|_{H^{s - \sigma}(\ci)}< c|t - t'|^\sigma, 
			\text{ for all }  \,\,  t, t'\in I,
		\end{equation}
		%
		which shows that  the family  $\{u_\ee\}$ is equicontinuous in 
		$C(I, H^{s-\sigma}(\ci))$. $\qquad \Box$
		%
		%
		%%%%%%%%%%%%%%%%%%%%%%
		%
		%
		%      PreCompactness
		%
		%
		%%%%%%%%%%%%%%%%%%%%%%%%%%
		%
		%
		%
		%
		%		
		\subsection{ Precompactness of $\{u_\ee(t)\}$ in $H^{s-\sigma}(\ci))$.}
		Now recall that
		\begin{equation}
			\label{compact-1}
			\|u_\ee(t)\|_{H^{s}(\ci)}
			\le
			2 \|u_0 \|_{H^s(\ci)}, \,
			\quad
			t\in I.
		\end{equation}
		%
		By Kondrachov's Theorem, the inclusion $H^s(\ci) \subset H^{s-
		\sigma }(\ci)$ is compact. By \eqref{compact-1},
		it follows that $\{u_\ee(t)\}$ is precompact in $H^{s-\sigma}(\ci)$.
		$\quad \Box$
		%
		%
		%
		%
		We are now in a position to apply Ascoli's Theorem: 
		\begin{theorem}
			\label{Ascoli}
			(Ascoli)  Let $X$ be a Banach space, $I$ be a compact metric space,
			and $C(I,X)$  be the set of continuous functions $f: I\longrightarrow X$.
			Suppose $S \subset C(I,X)$  has the following properties:
			%
			\begin{itemize}
				\item[(1)]   $S$ is  equicontinuous.
				\item[(2)]  For each $x \in M$ that the set $S(x) = \{f(x)\}$  is  precompact in $X$.
			\end{itemize} 
			%
			Then $S$  is  precompact  in  $C(I,X)$.
		\end{theorem}
		Compiling our previous results on equicontinuity and precompactness
		and applying Theorem \ref{Ascoli}, we
		conclude that there exists a subfamily $\left\{ u_{\ee_n} \right\}$
		such that
		\begin{equation}
			\label{strong-conv-of-u_ep}
			u_{\ee_n} \to u \; \; \text{in} \; \; C(I, H^{s-\sigma}(\ci)).
		\end{equation}
		%
		%
		%
		%%%%%%%%%%%%%%%%%%%%%%%%%%%%%%%%%
		%
		%
		%     Verifying that the limit $u$ solves Burgers equation
		%
		%
		%%%%%%%%%%%%%%%%%%%%%%%%%%%%%%%%%
		\subsection{ Verifying that the limit $u$ solves the HR equation.} 
		The following lemma will play a crucial role in our proof of the
		existence of a solution to the HR i.v.p.
		\begin{lemma}
			\label{lem:cc}
			We have
			\begin{equation}
				\begin{split}
					\label{burgers_and_nonlocal_conv}
				&  J_{\varepsilon_n} u_{\varepsilon_n} 
				\cdot J_{\varepsilon_n} \p_x u_{\varepsilon_n} 
				\to  u \partial_x u \; \; 
				\text{in} \; \;
				C(I, H^{s-\sigma-1}(\ci)). 
			\end{split}
			\end{equation}
		\end{lemma}
		%
		\subsection{ Proof.} It is implied by the following propositions:
		\begin{proposition}
			\label{prop:1aa}
			\begin{equation}
				\begin{split}
					 J_{\ee_n} u_{\ee_n} \to  u \ \ \text{in} \ \
					C(I, H^{s-\sigma}(\ci)).
					\label{}
				\end{split}
			\end{equation}
		\end{proposition}
			\subsection{ Proof.} Note that
			\begin{equation}
				\begin{split}
					& \| u -  J_{\ee_n} u_{\ee_n}
					\|_{C(I, H^{s-\sigma}(\ci))}
					\\
					&= \| u -  J_{\ee_n} u_{\ee_n} \pm 
					u_{\ee_n} \|_{C(I, H^{s-\sigma}(\ci))}
					\\
					& = \| u -  u_{\ee_n}
					\|_{C(I,H^{s-\sigma}(\ci))} + \| (I - J_{\ee_n})
					u_{\ee_n} \|_{C(I, H^{s-\sigma}(\ci))}.
					\label{1bb}
				\end{split}
			\end{equation}
			Applying the estimates
			\begin{equation*}
				\begin{split}
					& \|I-J_{\ee_n} \|_{L(H^{s-\sigma}(\ci), H^{s -
					\sigma}(\ci))} = o(1),
					\\
					& \|u_{\ee_n}\|_{H^{s-\sigma}(\ci)} \le 2
					\|u_0\|_{H^{s-\sigma}(\ci)}
				\end{split}
			\end{equation*}
			to \eqref{1bb} gives
			\begin{equation}
				\label{2bb}
				\begin{split}
					\| u -  J_{\ee_n} u_{\ee_n}\|_{H^{s-\sigma}(\ci)}
					\le \left( \| u -  u_{\ee_n}
					\|_{C(I, H^{s-\sigma}(\ci))} + o(1) \cdot \|u_0
					\|_{H^{s-\sigma}(\ci)} \right).
				\end{split}
			\end{equation}
			Letting $\ee_n \to 0$ in \eqref{2bb} and applying
			\eqref{strong-conv-of-u_ep} completes the proof. $\quad \Box$
			%
			%
			\begin{proposition}
				\label{prop:dd}
				\begin{equation}
					\begin{split}
						 J_{\ee_n} \p_x u_{\ee_n} \to  \p_x u \ \
						\text{in} \ \ C(I, H^{s-\sigma-1}(\ci)).
						\label{0dd}
					\end{split}
				\end{equation}
			\end{proposition}
			\subsection{ Proof.} 
			\begin{equation*}
				\begin{split}
					\|\p_x u - J_\ee \p_x u_{\ee_n} \|_{C(I,
					H^{s-\sigma-1}))}  
					& = \|\p_x u - \p_x J_\ee u_{\ee_n} \|_{C(I,
					H^{s-\sigma-1}(\ci))} 
					\\
					& \le \|u - J_\ee u_{\ee_n} \|_{C(I,
					H^{s-\sigma}(\ci))}.
				\end{split}
			\end{equation*}
			Applying Proposition \ref{prop:1aa} completes the proof. $\quad
			\Box$
			%
			This completes the proof of Lemma \ref{lem:cc}. $\quad \Box$
		%
		Note that since $\|\p_x (1-\p_x^2)^{-1}\|_{L(H^s(\ci), H^s(\ci))}
		\le 1$ for all $s \in \rr$, it follows immediately from
		\eqref{strong-conv-of-u_ep} that
		\begin{equation}
			\begin{split}
				& \p_x(1- \p_x^2)^{-1} \left( \frac{3-\gamma}{2}
				(u_{\ee_n})^2
				 + \frac{\gamma}{2} (\p_x u_{\ee_n})^2 \right )
				 \\
				 & \to
				 \p_x(1- \p_x^2)^{-1} \left( \frac{3-\gamma}{2} u^2
				 + \frac{\gamma}{2} (\p_x u)^2 \right ) \ \
				 \text{in} \ \ C(I, H^{s-\sigma-1}(\ci)).
				\label{non-local-convergence}
			\end{split}
		\end{equation}
		Combining \eqref{burgers_and_nonlocal_conv} and
		\eqref{non-local-convergence}, and applying the Sobolev Imbedding
		Theorem, we deduce 
		\begin{equation}
			\begin{split}
				& -\gamma (J_{\ee_n} u_{\ee_n} \cdot J_{\ee_n} \p_x
				u_{\ee_n}) - \p_x(1- \p_x^2)^{-1} \left( \frac{3-\gamma}{2}
				(u_{\ee_n})^2
				 + \frac{\gamma}{2} (\p_x u_{\ee_n})^2 \right )
				 \\
				 \to & -\gamma u \p_x u -
				 \p_x(1- \p_x^2)^{-1} \left( \frac{3-\gamma}{2} u^2
				 + \frac{\gamma}{2} (\p_x u)^2 \right ) \ \
				 \text{in} \ \ C(I, C(\ci)).
				\label{loc-non-loc-tog}
			\end{split}
		\end{equation}
		Furthermore, we note that the convergence  
		%
		\begin{equation}
			\label{weak-conv-2}
			T_{u_{\ee_n}}(\phi)  \longrightarrow  T_u(\phi) \;
			\text{ for all } \;  \phi \in L^1(I, H^{s}(\ci))
		\end{equation}
		%
		can be restated as 
		%
		\begin{equation}
			u_{\ee_n}  \longrightarrow  u
			\quad
			\text{ in }  \,\,
			\mathcal{D}'(I\times \ci).
		\end{equation}
		%
		This implies 
		%
		\begin{equation}
			\label{distib-conv-2}
			\p_tu_{\ee_n}  \longrightarrow  \p_tu
			\quad
			\text{ in }  \,\, \mathcal{D}'(I\times \ci).
		\end{equation}
		%
		Since for all $n$ we have 
		%
		\begin{equation}
			\p_tu_{\ee_n} 
			=
			-\gamma (J_{\varepsilon_n} u_{\varepsilon_n}  \cdot
			J_{\varepsilon_n}\partial_x u_{\varepsilon_n}) - \p_x (1-
			\p_x^2)^{-1} \left
			[\frac{3-\gamma}{2}(u_\ee)^2 + \frac{\gamma}{2}(\p_x u_\ee)^2 \right ] 
		\end{equation}
		%
		by the uniqueness  of the limit in \eqref{loc-non-loc-tog}
		we must have
		%
		\begin{equation}
			\label{1000y}
			\partial_t u =- \gamma u \partial_x u- \p_x (1- \p_x^2)^{-1} \left
			[\frac{3-\gamma}{2}u^2 + \frac{\gamma}{2}(\p_x u)^2 \right ].
		\end{equation}
		%
		Thus we have constructed a solution $u \in L^\infty(I, H^s(\ci))$
		to the HR i.v.p. $\qquad \Box$
		It remains to prove that $u \in C(I, H^s(\ci)).$
		%%%%%%%%%%%%%%%%%%%%%%%%%%
%
%
%Proof that  $u \in C(I, H^s(\ci)) \bigcap C^1(I, H^{s-1}(\ci))$.
%
%
%
%%%%%%%%%%%%%%%%%%%%%%%%%%
\subsection{ Proof that $u \in C(I, H^s(\ci))$.} 
We first outline our strategy. Since \\
$u \in L^\infty(I, H^s(\ci))$, it is a
continuous function from $I$ to $H^s(\ci)$ with respect to the weak
topology on $I$; that is, for $\{t_n\} \subset I$ such that $t_n \to t$, we
have
\begin{equation}
	\begin{split}
		<u(t_n), \ v>_{H^s(\ci)} \ \longrightarrow \
		<u(t), \ v>_{H^s(\ci)}, \quad \forall
		v \in H^s(\ci).
		\label{1ff}
	\end{split}
\end{equation}
Next, note that
\begin{equation}
	\begin{split}
		\|u(t) - u(t_n) \|_{H^s(\ci)}^2
		& = <u(t) - u(t_n), \ u(t) -
		u(t_n)>_{H^s(\ci)}
		\\
		& = \|u(t)\|_{H^s(\ci)}^2 + \|u(t_n)\|_{H^s(\ci)}^2
		\\
		& - <u(t_n), \
		u(t) >_{H^s(\ci)} - <u(t), u(t_n)>_{H^s(\ci)}.
		\label{2ff}
	\end{split}
\end{equation}
Applying \eqref{1ff} and \eqref{2ff}, we see that
\begin{equation}
	\begin{split}
		\lim_{n \to \infty} \|u(t) - u(t_n)\|_{H^s(\ci)}^2 = \left[ \lim_{n
		\to \infty} \|u(t_n)\|_{H^s(\ci)}^2
		\right] - \|u(t)\|_{H^s(\ci)}^2.
		\label{3ff}
	\end{split}
\end{equation}
Hence, by \eqref{3ff}, to prove that $u \in C(I, H^s(\ci))$, it will be
enough to show that the map $t \mapsto \|u(t)\|_{H^s(\ci)}$ is a continuous
function of $t$. However, this will follow from the energy
estimate
		\begin{equation}
			\label{en-est-u}
			\frac{1}{2} \frac{d}{dt} \|u(t)\|_{H^s(\ci)}^2
			\le c_s \|u(t)\|_{H^s(\ci)}^3, \quad |t| \le T
		\end{equation}
		which we now derive. Applying $D^s$ to both sides of
		\eqref{1000y}, multiplying the
		resulting equation by $D^s u$, and integrating for $x\in \ci$, we obtain
		\begin{equation}
			\begin{split}
				\label{bound-int}
				\frac 12
				\frac{d}{dt} \|u \|_{H^s}^2
				=
				&-
				\gamma \int_{\ci}   D^s (u \p_x u) \cdot
				D^s u \  dx
				\\
				&- \frac{3 -\gamma}{2} \int_{\ci}  D^{s-2} \p_x (u^2) 
				\cdot D^s u \ dx
				\\
				&- \frac{\gamma}{2} \int_{\ci}   D^{s-2} \p_x (\p_x u)^2
				\cdot D^s u \ dx.
			\end{split}
		\end{equation}
		First we estimate
	\begin{equation}
		\begin{split}
			\left | - \frac{3 -\gamma}{2} \int_\ci D^{s-2} \p_x (u^2) \cdot
			D^s u \; dx \right |
			& \le \left | \frac{3 -\gamma}{2} \right |
			\int_\ci \left |
			D^{s-2} \p_x (u^2) \cdot D^s u \right | dx 
			\\
			& \le \left | \frac{3 -\gamma}{2} \right |
			\|D^{s-2} \p_x (u^2) \|_{L^2(\ci)} 
			\|D^s u \|_{L^2(\ci)}
			\\
			& \le \left | \frac{3 -\gamma}{2} \right |
			\|D^{s-1} (u^2) \|_{L^2(\ci)} 
			\|D^s u \|_{L^2(\ci)}
			\\
			& = \left | \frac{3 -\gamma}{2} \right |
			\| u^2 \|_{H^{s-1}(\ci)} \| u \|_{H^s(\ci)}
			\\
			& \le
			\left | \frac{3 -\gamma}{2} \right | \| u^2 \|_{H^s(\ci)} \| u
			\|_{H^s(\ci)}.
		\end{split}
	\end{equation}
	%
	%
	Applying the algebra property, we obtain
	%
	\begin{equation}
		\label{hl1-prime}
		\begin{split}
			\left | -\frac{3 -\gamma}{2} \int_\ci D^{s-2} \p_x u^2 \cdot
			D^s u \; dx \right |
			\le c_s' \| u \|_{H^s(\ci)}^3.
		\end{split}
	\end{equation}
	%
	%
	We also have
	\begin{equation}
		\begin{split}
			\left | -\frac{\gamma}{2} \int_\ci D^{s-2} \p_x (\p_x u)^2 \cdot
			D^s u \; dx \right |
			& \le \left | \frac{\gamma}{2} \right |
			\int_\ci \left | D^{s-2} \p_x (\p_x u)^2 
			\cdot D^s u \right | \; dx
			\\
			& \le \left | \frac{\gamma}{2} \right |
			\| D^{s-2} \p_x (\p_x u)^2 \|_{L^2(\ci)}
			\| D^s u \|_{L^2(\ci)}
			\\
			&  \le \left | \frac{\gamma}{2} \right | \|(\p_x u)^2
			\|_{H^{s-1}(\ci)} \| u \|_{H^s(\ci)} 
		\end{split}
	\end{equation}
	and applying the algebra property yields
	\begin{equation}
		\label{hl2-prime}
		\left | -\frac{\gamma}{2} \int_\ci D^{s-2} (\p_x u)^2 \cdot
		D^s u \; dx \right |
		\le c_s'' \|u\|_{H^s(\ci)}^3.
	\end{equation}
	It remains to estimate 
	\begin{equation*}
		- \gamma \int_{\ci} \left [  D^s (u \p_x u) \cdot
		D^s u \right ]  dx
	\end{equation*}
	We have
	\begin{equation} 
	\begin{split}
		\label{B-moli-int-v'}
		-  \gamma \int_{\ci} \left [D^s (u \p_x u) \cdot D^s
		u \right ] \ dx
		= &- \gamma  \int_\ci
		\left [ D^s(u \partial_x u) \cdot   D^s u \right ] \ dx
		\\
		=& - \gamma \int_\ci
		\big[ 
		D^s(u\p_x u)  -  u D^s (\p_xu)
		\big] \cdot
		D^s u   \ dx
		\\
		&
		- \gamma \int_\ci
		u D^s (\p_xu) \cdot
		D^su \ dx.
	\end{split}
\end{equation}
%
%
%
We now estimate \eqref{B-moli-int-v'} in parts. Applying the Cauchy-Schwarz inequality gives
%
\begin{equation*} 
	\begin{split}
		& \Big|
		- \gamma \int_\ci
		\big[ 
		D^s(u\p_x u)  -  u D^s (\p_xu)
		\big] \cdot
		D^s u   \, dx
		\Big|
		\\
		& \le
		|\gamma| \cdot \|
		D^s(u\p_x u)  -  u D^s (\p_xu)
		\|_{L^2(\ci)}
		\|
		D^s u 
		\|_{L^2(\ci)}
		\\
		& =
		|\gamma| \cdot \| D^s(u\p_x u)  -  u D^s (\p_xu)
		\|_{L^2(\ci)}
		\|
		u
		\|_{H^s(\ci)}
			\end{split}
\end{equation*}
Applying \eqref{int1-est-calc3}, we obtain
\begin{equation*}
\begin{split}
		\Big|
		- \gamma \int_\ci
		\big[ 
		D^s(u\p_x u)  -  u D^s (\p_xu)
		\big]
		D^s u   \, dx
		\Big|
		&\le
		 c_s'''   \| \p_x u \|_{L^\infty(\ci)} 
		\| u \|_{H^s(\ci)}^2.
	\end{split}
\end{equation*}
%\label{int1-est-calc2'}
%
Next, we apply Cauchy-Schwartz and the Sobolev Imbedding Theorem to deduce 
	%
	%
	%
	\begin{equation} 
		\label{int1-est-calc5'}
		\begin{split}
			\Big|
			\int_\ci
			\left [u D^s (\p_x u)
			\cdot  D^s u \right ] dx
			\Big|
			& =
			\frac{1}{2} \Big|
			   \int_\ci
			\left [u \p_x\left(D^s u\right)^2 \right ] \ dx
			\Big|
			\\
			& \le
			\frac{1}{2} \int_\ci \Big |
			\left [\p_x u \, (D^s u)^2  \right ] 
			\Big| \ dx
			\\
			& \le
			\frac{1}{2}
			\| \p_x u \|_{L^\infty(\ci)} 
			\| u \|_{H^s(\ci)}^2.
			\\
			& \le c_s'''' \|u\|_{H^s(\ci)}^3.
		\end{split}
	\end{equation}
	%
	%
	%
	Combining \eqref{hl1-prime}, \eqref{hl2-prime},
	and \eqref{int1-est-calc5'}, we obtain \eqref{en-est-u}, as desired.
	\subsection{ Size of the solution}. 
	Letting  $y(t)=  \|u(t)\|_{H^s(\ci)}^2$ inequality \eqref{en-est-u}
	takes the form
	%
	\begin{equation} 
		\label{energy-y-ineq'}
		\frac 12
		y^{-3/2}\frac{dy}{dt}
		\le
		c_s,
		\qquad
		y(0)=y_0=  \|u_0\|_{H^s(\ci)}^2.
	\end{equation}
	%
	Suppose $t$ is non-negative. Then integrating  \eqref{energy-y-ineq'}
	from  0 to $t$ gives
	%
	\begin{equation*} 
		\frac{1}{\sqrt{y_0}}  - \frac{1}{\sqrt{y(t)}} 
		\le 
		c_s t.
	\end{equation*}
	%
	%
	Replacing $y(t)$ with   $\|u(t)\|_{H^s(\ci)}^2$  and solving for  $\|u(t)\|_{H^s(\ci)}$
	we obtain the formula
	%
	\begin{equation} 
		\label{norm-u(t)-formula'}
		\|u(t)\|_{H^s(\ci)}
		\le
		\frac{ \|u_0\|_{H^s(\ci)}}{1-c_s\|u_0\|_{H^s(\ci)} t}.
	\end{equation}
	%
	Now, note that our solution $u$ inherits the common lifespan $T$ of the family
	$\{u_\ee\}$; that is, $u$ has lifespan
	\begin{equation*}
		T
		=
		\frac{1}{2 c_s \|u_0\|_{H^s(\ci)}}.
	\end{equation*}
	Substituting into \eqref{norm-u(t)-formula'} we obtain	
	%
	\begin{equation*} 
		\label{u(t)-u(0)-bound'}
		\|u(t)\|_{H^s(\ci)}
		\le
		\frac{ \|u_0\|_{H^s(\ci)}}{1-(c_s\|u_0\|_{H^s(\ci)})/(2 c_s \|u_0\|_{H^s(\ci)})},
	\end{equation*}
	%
	which simplifies to 
	%
	\begin{equation*}
		\|u(t)\|_{H^s(\ci)}
		\le
		2 \|u_0\|_{H^s(\ci)},
		\quad 
		0\le t \le T.
	\end{equation*}
	Similarly, for negative $t$, we have
	\begin{equation*}
		\|u(t)\|_{H^s(\ci)}
		\le
		2 \|u_0\|_{H^s(\ci)},
		\quad 
		-T \le t < 0.
	\end{equation*}
	Hence,
	\begin{equation}
		\label{uniform_bound_for_u}
		\|u(t)\|_{H^s(\ci)}
		\le
		2 \|u_0\|_{H^s(\ci)},
		\quad 
		|t| \le T.
	\end{equation}
		%
		\subsection{Space of the solution}
	Derivating the left hand side of \eqref{en-est-u} and simplifying, we obtain
	\begin{equation}
		\label{en-est-u-simplified}
	\frac{d}{dt} \|u(t)\|_{H^s(\ci)} \le c_s \|u(t)\|_{H^s(\ci)}^2, \quad |t| \le T.
	\end{equation}
	Since $\|u(t)\|_{H^s(\ci)}$
	is uniformly bounded for $|t| \le T$ by
	\eqref{uniform_bound_for_u}, we conclude from
	\eqref{en-est-u-simplified} that the map $t \mapsto
	\|u(t)\|_{H^s(\ci)}$ is Lipschitz continuous in $t$, for $|t| \le T$.
	Therefore, by \eqref{3ff}, $u \in C(I, H^s(\ci))$. 
	%
	%
	%
	%
	%
	\subsection{Uniqueness.}
	%
	%
	Let $u,\omega \in C(I, H^s(\ci)), \ s > 3/2$ be two solutions to the
	Cauchy-problem \eqref{hyperelastic-rod-equation}-\eqref{init-cond} with
	common initial data. Let $v=u-w$; since
	\begin{align*}
		\p_t u 
		& = - \gamma u \p_x u - D^{-2} \p_x \left[ \frac{3-\gamma}{2} u^2 +
		\frac{\gamma}{2}\left( \p_x u \right)^2 \right]
		\\
		\p_t w & = -\gamma w \p_x w - D^{-2} \p_x \left[
		\frac{3-\gamma}{2} w^2 + \frac{\gamma}{2}(\p_x w)^2 
		\right]
	\end{align*}
	we subtract the two equations to obtain 
	\begin{equation*}
		\begin{split}
			\p_t v
			= -\frac{\gamma}{2} \p_x [v(u + w)] - D^{-2} \p_x \left\{
			\frac{3-\gamma}{2}[v(u+w)] + \frac{\gamma}{2}[\p_x v \cdot \p_x (u+w)]
			\right\}
		\end{split}
	\end{equation*}
	and hence
	\begin{equation}
		\begin{split}
			D^\sigma \p_t v = -\frac{\gamma}{2} D^\sigma \p_x [v(u+w)] - D^{\sigma -2} \p_x
			\left\{ \frac{3-\gamma}{2} [v(u+w)] + \frac{\gamma}{2} [\p_x v
			\cdot \p_x
			(u+w)]
			\right\}.
			\label{1v}
		\end{split}
	\end{equation}
	Multiplying both sides of \eqref{1v} by $D^\sigma v$ and integrating, we obtain
	\begin{equation}
		\begin{split}
			\frac{1}{2} \frac{d}{dt} \|v\|_{H^\sigma(\ci)}^2
			& =  \overbrace{-\frac{\gamma}{2} \int_{\ci} D^\sigma \p_x [v(u+w)] \cdot
			D^\sigma v \ dx}^i
			\\
			& \overbrace{- \frac{3-\gamma}{2} \int_{\ci}  D^{\sigma -2}
			\p_x[v(u+w)] \cdot
			D^\sigma v \ dx}^{ii} 
			\\
			& - \overbrace{\frac{\gamma}{2} \int_{\ci} D^{\sigma -2} \p_x [ \p_x v
			\cdot \p_x (u+w)]\cdot D^\sigma v \ dx }^{iii}.
			\label{2v}
		\end{split}
	\end{equation}
	We will estimate (\hyperref[2v]{ii}) first.
	Applying Cauchy-Schwartz, we have 
	\begin{equation*}
		\begin{split}
			|ii|
			& \le \left | \frac{3-\gamma}{2} \right | \|D^{\sigma -2}
			\p_x [v(u+w)] \cdot D^\sigma
			v  \|_{L^1(\ci)}
			\\
			 & \le  \left | \frac{3-\gamma}{2} \right | \|D^{\sigma -2} \p_x [v(u+w)]
			\|_{L^2(\ci)} \|v\|_{H^\sigma(\ci)}
			\\
			& \lesssim \|v(u+w)\|_{H^{\sigma -1}(\ci)} \|v\|_{H^\sigma(\ci)}
		\end{split}
	\end{equation*}
	which by the algebra property and the Sobolev
	Imbedding Theorem gives
\begin{equation}
		\begin{split}
		|ii| \lesssim \|u+w\|_{H^{\sigma -1}(\ci)} \|v\|_{H^\sigma(\ci)}^2.
			\label{3v}
		\end{split}
	\end{equation}
	To estimate (\hyperref[2v]{iii}) we first apply
	Cauchy-Schwartz and the Sobolev Imbedding Theorem:
	\begin{equation*}
		\begin{split}
		|iii| & \le	\left | \frac{\gamma}{2} \right | \|D^{\sigma -2} \p_x
			[\p_x v \cdot \p_x (u+w)] \cdot D^\sigma v  \|_{L^1(\ci)} 
			\\
			& \le  \left | \frac{\gamma}{2} \right | \|D^{\sigma -2} \p_x
			[\p_x v \cdot \p_x (u+w)] \|_{L^2(\ci)}
			\|v\|_{H^\sigma(\ci)}
			\\
			& \le \left |\frac{\gamma}{2} \right|
			\|[\p_x v \cdot \p_x (u+w)] \|_{H^{\sigma -1}(\ci)}
			\|v\|_{H^\sigma(\ci)}.
		\end{split}
	\end{equation*}
	Restrict $1/2 < \sigma < 1$. Then applying Lemma \ref{impo}, we conclude
	that
	\begin{equation}
		\begin{split}
			|iii|
			& \le C \left | \frac{\gamma}{2} \right |
			\|\p_x(u+w) \|_{H^{\sigma}(\ci)}
			\|\p_x v\|_{H^{\sigma -1}(\ci)} \|v\|_{H^\sigma(\ci)}
			\\
			& \lesssim \|u+w \|_{H^{\sigma + 1}(\ci)}
			\|v\|_{H^\sigma(\ci)}^2.
			\label{3'v}
		\end{split}
	\end{equation}
	It remains to estimate (\hyperref[2v]{i}).
	Proceeding, we rewrite
	\begin{equation}
		\begin{split}
			|i| & =
			\left |
			-\frac{\gamma}{2} \int_{\ci} \left[ D^\sigma \p_x, \ u+w \right]v \cdot
			D^\sigma v \ dx - \frac{\gamma}{2} \int_{\ci} (u+w) D^\sigma
			\p_x v \cdot D^\sigma v\ dx
			\right | 
			\\
			& \le \left |
			-\frac{\gamma}{2} \int_{\ci} \left[ D^\sigma \p_x, \ u+w \right]v \cdot
			D^\sigma v \ dx \right |
			+ \left | \frac{\gamma}{2} \int_{\ci} (u+w) D^\sigma \p_x v
			\cdot D^\sigma v\
			dx \right |.
			\label{4v}
		\end{split}
	\end{equation}
	We now estimate \eqref{4v} in pieces. Observe that by integrating by parts
	and applying Cauchy-Schwartz we have
	\begin{equation}
		\begin{split}
			\left | \frac{\gamma}{2}\int_{\ci} (u+w) D^\sigma \p_x v \cdot
			D^\sigma v \ dx \right |
			& = \left | -\frac{\gamma}{2} \int_{\ci} \p_x (u+w) D^\sigma v
			\cdot D^\sigma v \ dx \right |
			\\
			& \lesssim \|\p_x (u+w) D^\sigma v \|_{L^2(\ci)} \|D^\sigma
			v\|_{L^2(\ci)}
			\\
			& \lesssim \|\p_x (u+w)\|_{L^\infty(\ci)}
			\|v\|_{H^\sigma(\ci)}^2.
			\label{4'v}
		\end{split}
	\end{equation}
	To estimate the remaining piece of \eqref{4v}, we recall first that we
	have the restriction $1/2 < \sigma < 1$. However, this will not prevent
	us from applying Corollary \ref{cor1}; in fact, choosing $\ 3/2 < \rho
	< s,  \ 1/2< \sigma <\min\{1, \ \rho -1 \}$, we obtain
	\begin{equation}
		\begin{split}
			\left | -\frac{\gamma}{2} \int_{\ci} [D^\sigma \p_x, \ u+w] v
			\cdot D^\sigma v \ dx \right |
			& \le \left | \frac{\gamma}{2} \right| \int_{\ci} \left |
			[D^\sigma \p_x, \ u+w] v
			\cdot D^\sigma v \right | dx 
			\\
			& \lesssim \|[D^\sigma \p_x, \ u+w]v\|_{L^2(\ci)}
			\|v\|_{H^\sigma(\ci)} \\
			& \lesssim \|u+w\|_{H^\rho(\ci)} \|v\|_{H^\sigma(\ci)}^2.
			\label{7v}
		\end{split}
	\end{equation}
	Combining \eqref{4'v} and \eqref{7v} and applying the Sobolev Imbedding
	Theorem, we obtain the estimate
	\begin{equation}
		\begin{split}
			|i| \lesssim \|u+w\|_{H^\rho(\ci)} \|v\|_{H^\sigma(\ci)}^2.
			\label{8v}
		\end{split}
	\end{equation}
	Recall \eqref{2v}. Grouping \eqref{3v}, \eqref{3'v}, and \eqref{8v}, and applying
	the Sobolev Imbedding Theorem, we see that 
	\begin{equation}
		\begin{split}
			\frac{1}{2} \frac{d}{dt}
			\|v\|_{H^\sigma(\ci)}^2 \lesssim \|u+w\|_{H^\rho(\ci)}
			\|v\|_{H^\sigma(\ci)}^2.
			\label{9v}
		\end{split}
	\end{equation}
	By Gronwall's inequality, \eqref{9v} gives
	\begin{equation}
		\label{10lv}
		\begin{split}
			\|v\|_{H^\sigma(\ci)}
			& \lesssim e^{\int_0^t \|u+w\|_{H^{\rho}}}
			\|v_0\|_{H^\sigma(\ci)}, \qquad |t| \le T.
		\end{split}
	\end{equation}
	First, note that $v_0 = u_0 - w_0 = 0$; secondly, $\|u + w \|_{H^\rho}
	\le \|u + w \|_{H^s(\ci)} < \infty$ for $|t| \le T$ by
	the triangle inequality and \eqref{u_x-Linfty-Hs}. Hence, from
	\eqref{10lv} we obtain
	\begin{equation*}
		\begin{split}
			\|v\|_{H^\sigma(\ci)}
			& \lesssim \|v_0\|_{H^\sigma(\ci)}, \quad |t| \le T	
			\\
			& = 0.
		\end{split}
	\end{equation*}
	We conclude that solutions to the HR i.v.p. with initial data in
	$H^s(\ci)$ are unique for $s > 3/2$.  $\qquad
	\Box$
	%
	%
	%
	%
	\subsection{Continuous Dependence.}
	Let $\left\{ u_{0, n} \right\}_n \subset H^s(\ci)$ be a uniformly bounded
sequence converging to $u_0$ in $H^s(\ci)$.
Consider solutions $u $, $u^\ee$, $u^\ee_n$, and $u_n$ to the Cauchy-problem
\eqref{hyperelastic-rod-equation}-\eqref{init-cond}
with associated initial data $u_0$, $J_\ee u_0$,
$J_\ee u_{0,n}$, and $u_{0,n}$, respectively, where $J_\ee$ is defined as follows: Pick a function $\widehat{j}(\xi) \in \mathcal{S}(\rr)$ such that
	\begin{equation}
		\label{0u}
		\begin{split}
			& 0 \le \widehat{j}(\xi) \le 1,
			\\
			& \widehat{j}(\xi) = 1 \ \ \text{if} \ \ |\xi| \le 1.
		\end{split}
	\end{equation}
	Since $\sum_{-M}^M \widehat{j}(\ee \xi) e^{i \xi x}$ converges uniformly as $M \to
	\infty$, we can let
	\begin{equation}
		\begin{split}
			j_\ee (x) = \frac{1}{2 \pi}\sum_{\xi \in \zz}
			\widehat{j}(\ee \xi) e^{i \xi x}, \quad \ee > 0
			\label{parseval-def}
		\end{split}
	\end{equation}
	which is equivalent to stating that we can find $\left\{ j_\ee
	\right\} \subset \mathcal{S}(\ci)$ such that
	\begin{equation}
		\begin{split}
			\widehat{j_\ee} = \widehat{j }(\ee \xi), \quad \ee > 0.
			\label{widehat-def}
		\end{split}
	\end{equation}
	We then define $J_\ee$ to be the ``Friedrichs mollifier''
	\begin{equation}
		\label{0'u}
		\begin{split}
			J_\ee f(x) = j_\ee * f(x), \quad \ee>0.
		\end{split}
	\end{equation}
We remark that we have constructed the operator $J_\ee$ in this manner in
order that inequality \eqref{widehat-def} is satisfied; this will prove
crucial later on.
%
Applying
the triangle inequality, we have
\begin{equation*}
	\begin{split}
		\|u - u_n\|_{H^s(\ci)}
		& \le \|u - u^\ee\|_{H^s(\ci)}
		+ \|u^\ee - u^{\ee}_n \|_{H^s(\ci) }
		+  \|u^{\ee}_n - u_n \|_{H^s(\ci)}.
	\end{split}
\end{equation*}
Therefore, to prove continuous dependence, it will be enough to show 
\begin{align}
	& \lim_{\substack{n\to \infty \\ \ee \to 0}} \|u - u^\ee\|_{H^s(\ci)}
	=0,
	\label{enough_to_prove1}
	\\
	& \lim_{\substack{n\to \infty \\ \ee \to 0}} \|u^\ee - u^{\ee}_n
	\|_{H^s(\ci)} = 0,
	\label{enough_to_prove2}
	\\
	& \lim_{\substack{n\to \infty \\ \ee \to 0}}
	\|u^{\ee}_n - u_n \|_{H^s(\ci)} =0
	\label{enough_to_prove3}
\end{align}
where we define 
\medskip
	\begin{equation}
		\label{lim-not}
		\begin{split}
			\lim_{\substack{n\to \infty \\ \ee \to 0}} (\cdot) \doteq \lim_{\ee \to
			\infty} [\lim_{n \to \infty} (\cdot )].
		\end{split}
	\end{equation}
\subsection{ Proof of \eqref{enough_to_prove1}.}
		Consider two solutions $u $ and $u^\ee$ to the Cauchy-problem
	\eqref{hyperelastic-rod-equation}-\eqref{init-cond}
	with associated initial data $u_0$ and
	$J_\ee u_0$, respectively. Set $v= u -u^\ee $. Then $v$ solves the
	Cauchy-problem
	\begin{align}
		\label{4u}
		\p_t v 
		& =  - \gamma (v \p_x v + v \p_x u^\ee + u^\ee \p_x v)  
		\\
		& - D^{-2} \p_x \left\{ \left (\frac{3-\gamma}{2} \right )(v^2 +
		2u^\ee v) + \frac{\gamma}{2}\left[ (\p_x v)^2 + 2 \p_x u^\ee \p_x v \right]
		\right\}, \notag
		\\
		& v(0) = (I- J_\ee)u_0.
		\label{5u}
	\end{align}
	Applying the operator $D^s$ to both sides of \eqref{4u}, multiplying by
	$D^s v$ and integrating, we have
	\begin{equation}
		\begin{split}
			\frac{1}{2}\frac{d}{dt} \|v\|_{H^s(\ci)} = A + B
			\label{6u}
		\end{split}
	\end{equation}
	where
	\begin{equation}
		\begin{split}
			A
			& =  -\gamma \int_{\ci} D^s(v \p_x v) \cdot D^s v \
			dx
			- \frac{3- \gamma}{2} \int_\ci D^{s-2} \p_x (v^2) \cdot D^s v
			\ dx
			\\
			& - \frac{\gamma}{2}\int_\ci D^{s-2} \p_x (\p_x v)^2 \cdot D^s
			v \ dx
			\label{7u}
		\end{split}
	\end{equation}
	and
	\begin{equation}
		\begin{split}
			B 
			= & \ \overbrace{-\gamma \int_\ci D^s (v \p_x u^\ee ) \cdot D^s v \
			 dx}^{(i)} \ \overbrace{-\gamma \int_\ci D^s (u^\ee \p_x v) \cdot D^s v \
			 dx}^{(ii)}
			  \\
			  & \ \overbrace{- \ ( 3- \gamma) \int_\ci D^{s-2} \p_x (u^\ee v) \cdot D^s
			 v \ dx}^{(iii)}
			 \\
			 & \overbrace{-\gamma \int_\ci D^{s-2} \p_x
			(\p_x u^\ee \cdot \p_x v) \cdot D^s v \
			dx}^{(iv)}.
			\label{8u}
		\end{split}
	\end{equation}
	We now provide estimates for $A$ and $B$:
	\subsection{ A.} 
	Recalling the proof of \eqref{en-est-u}, with $u$ replaced by
	$v$ gives 
	\begin{equation}
		\begin{split}
			|A| \lesssim \|v\|_{H^s(\ci)}^3, \quad |t| \le T.
			\label{8'u}
		\end{split}
	\end{equation}
%
	\subsection{ B.} We now estimate in parts:
	%
	%
	%
		%
	%
\subsection{ Estimate of (\hyperref[8u]{i}).} 
We can rewrite
	\begin{equation}
		\begin{split}
			(i)
			= & -\gamma \int_\ci \left[ D^s(v \p_x u^\ee) - v D^s
			\p_x u^\ee \right] \cdot D^s v \ dx
			\\
			& -  \gamma \int_\ci v D^s \p_x u^\ee \cdot D^s v \ dx.
			\label{1wap'}
		\end{split}
	\end{equation}
	Estimating in parts, we have
	\begin{equation}
		\begin{split}
			& |-\gamma \int_\ci \left[ D^s(v \p_x u^\ee) - v D^s
			\p_x u^\ee \right] \cdot D^s v \ dx |
			\\
			& \le |\gamma| \int_\ci |\left[ D^s(v \p_x u^\ee ) - v D^s
			\p_x u^\ee \right] \cdot D^s v| \ dx
			\\
			& \le |\gamma| \cdot \|D^s (v \p_x u^\ee) - v D^s \p_x u^\ee
			\|_{L^2(\ci)} \|v\|_{H^s(\ci)}.
			\label{1wap}
		\end{split}
	\end{equation}
	Applying the Kato-Ponce estimate \eqref{KP-com-est} to \eqref{1wap}, we
	obtain
	\begin{equation*}
		\begin{split}
			& | -\gamma \int_\ci \left[ D^s(v \p_x u^\ee) - v D^s
			\p_x u^\ee \right] \cdot D^s v \ dx |
			\\
			& \le c_s |\gamma| \cdot ( \|D^s v \|_{L^2(\ci)} \|\p_x
			u^\ee\|_{L^\infty(\ci)} + \|\p_x v \|_{L^\infty(\ci)} \|D^{s-1}
			\p_x u^\ee \|_{L^2(\ci)}) \cdot \|v\|_{H^s(\ci)}
		\end{split}
	\end{equation*}
	which by the Sobolev Imbedding Theorem simplifies to
	\begin{equation}
		\begin{split}
			| -\gamma \int_\ci \left[ D^s(v \p_x u^\ee ) - v D^s
			\p_x u^\ee \right] \cdot D^s v \ dx |
			\lesssim \|u^\ee \|_{H^s(\ci)} \|v\|_{H^s(\ci)}^2.
			\label{2wap}
		\end{split}
	\end{equation}
	For the remaining piece of \eqref{1wap'}, we have
	\begin{equation*}
		\begin{split}
			| - \gamma \int_\ci v D^s \p_x u^\ee \cdot D^s v \ dx |
			& \le |\gamma| \int_\ci |v D^s \p_x u^\ee \cdot D^s v | \ dx
			\\
			& \le |\gamma| \cdot \|v\|_{L^\infty(\ci)} \|D^s \p_x u^\ee
			\|_{L^2(\ci)} \|D^s v\|_{L^2(\ci)}
		\end{split}
	\end{equation*}
	which by the Sobolev Imbedding Theorem gives 
	\begin{equation}
		\begin{split}
			| - \gamma \int_\ci u^\ee D^s \p_x v \cdot D^s v \ dx |
			\lesssim \|u^\ee \|_{H^{s+1}(\ci)} \|v\|_{H^{s-1}(\ci)}
			\|v\|_{H^s(\ci)}.
			\label{3wap}
		\end{split}
	\end{equation}
	Combining estimates \eqref{2wap} and \eqref{3wap} we conclude that
	\begin{equation}
		\begin{split}
			|(i)| \lesssim \|u^\ee \|_{H^s(\ci)} \|v\|_{H^s(\ci)}^2 + 
			\|u^\ee \|_{H^{s+1}(\ci)} \|v\|_{H^{s-1}(\ci)}
			\|v\|_{H^s(\ci)}.
			\label{4wap}
		\end{split}
	\end{equation}
%
\subsection{ Estimate of (\hyperref[8u]{ii}).} We can rewrite
	\begin{equation}
		\begin{split}
			(ii)
			= & -\gamma \int_\ci \left[ D^s(u^\ee \p_x v) - u^\ee D^s
			\p_x v \right] \cdot D^s v \ dx
			\\
			& -  \gamma \int_\ci u^\ee D^s \p_x v \cdot D^s v \ dx.
			\label{1wa'}
		\end{split}
	\end{equation}
	Estimating in parts, we have
	\begin{equation}
		\begin{split}
			& |-\gamma \int_\ci \left[ D^s(u^\ee \p_x v) - u^\ee D^s
			\p_x v \right] \cdot D^s v \ dx |
			\\
			& \le |\gamma| \int_\ci |\left[ D^s(u^\ee \p_x v) - u^\ee D^s
			\p_x v \right] \cdot D^s v| \ dx
			\\
			& \le |\gamma| \cdot \|D^s (u^\ee \p_x v) - u^\ee D^s \p_x v
			\|_{L^2(\ci)} \|v\|_{H^s(\ci)}.
			\label{1wa}
		\end{split}
	\end{equation}
	Applying the Kato-Ponce estimate \eqref{KP-com-est} to \eqref{1wa}, we
	obtain
	\begin{equation*}
		\begin{split}
			& | -\gamma \int_\ci \left[ D^s(u^\ee \p_x v) - u^\ee D^s
			\p_x v \right] \cdot D^s v \ dx |
			\\
			& \le c_s |\gamma| \cdot ( \|D^s u^\ee \|_{L^2(\ci)} \|\p_x
			v\|_{L^\infty(\ci)} + \|\p_x u^\ee \|_{L^\infty(\ci)} \|D^{s-1}
			\p_x v \|_{L^2(\ci)}) \cdot \|v\|_{H^s(\ci)}
		\end{split}
	\end{equation*}
	which by the Sobolev Imbedding Theorem simplifies to
	\begin{equation}
		\begin{split}
			| -\gamma \int_\ci \left[ D^s(u^\ee \p_x v) - u^\ee D^s
			\p_x v \right] \cdot D^s v \ dx |
			\lesssim \|u^\ee \|_{H^s(\ci)} \|v\|_{H^s(\ci)}^2.
			\label{2wa}
		\end{split}
	\end{equation}
	For the remaining piece of \eqref{1wa'}, we have
	\begin{equation*}
		\begin{split}
			| - \gamma \int_\ci u^\ee D^s \p_x v \cdot D^s v \ dx |
			& = \left | -\frac{\gamma}{2} \int_\ci u^\ee \p_x (D^s v)^2 \
			dx \right |
			\\
			& = \left | \frac{\gamma}{2} \int_\ci \p_x u^\ee (D^s v)^2 \ dx
			\right |
			\\
			& \le \left | \frac{\gamma}{2} \right | \int_\ci |\p_x u^\ee
			(D^s v)^2 | dx
			\\
			& \le \left | \frac{\gamma}{2} \right | \|\p_x u^\ee
			\|_{L^\infty(\ci)} \|v\|_{H^s(\ci)}^2
		\end{split}
	\end{equation*}
	and applying the Sobolev Imbedding Theorem gives
	\begin{equation}
		\begin{split}
			| - \gamma \int_\ci u^\ee D^s \p_x v \cdot D^s v \ dx |
			\lesssim \|u^\ee \|_{H^s(\ci)} \|v\|_{H^s(\ci)}^2.
			\label{3wa}
		\end{split}
	\end{equation}
	Combining estimates \eqref{2wa} and \eqref{3wa} we conclude that
	\begin{equation}
		\begin{split}
			|(ii)| \lesssim \|u^\ee \|_{H^s(\ci)} \|v\|_{H^s(\ci)}^2.
			\label{4wa}
		\end{split}
	\end{equation}
\subsection{ Estimate of (\hyperref[8u]{iii}).} We have
	\begin{equation}
		\begin{split}
			|(iii)|
			& \le |3-\gamma| \int_{\ci} |D^{s-2} \p_x (u^\ee v) \cdot D^s v
			\ | \ dx
			\\
			& \le |3- \gamma| \cdot  \|D^{s-2}\p_x (u^\ee v)
			\|_{L^2(\ci)} \cdot \|v\|_{H^s(\ci)}
			\\
			& \le |3- \gamma| \cdot  \| u^\ee v \|_{H^{s -1}(\ci)} \cdot \|v\|_{H^s(\ci)}
			\label{12u}
		\end{split}
	\end{equation}
	and applying the algebra property and the Sobolev Imbedding Theorem gives
	\begin{equation}
		\begin{split}
			|(iii)| & \lesssim \|u^\ee\|_{H^{s-1}(\ci)} \|v\|_{H^{s-1}(\ci)}
			\|v\|_{H^s(\ci)}
			\\
			& \lesssim \|u^\ee\|_{H^{s}(\ci)} \|v\|_{H^{s}(\ci)}^2.
			\label{13u}
		\end{split}
	\end{equation}
	%
	%
	%
	%
	\subsection{ Estimate of (\hyperref[8u]{iv}).} We have
	\begin{equation*}
		\begin{split}
			|(iv)|
			& \le |\gamma| \cdot \|D^{s-2} \p_x (\p_x u^\ee \cdot \p_x v)
			\|_{L^2(\ci)} \|D^s v\|_{L^2(\ci)}
			\\
			& \le |\gamma| \cdot \|\p_x u^\ee \cdot \p_x v \|_{H^{s-1}(\ci)}
			\|v\|_{H^s(\ci)}
		\end{split}
	\end{equation*}
	and applying the algebra property gives
	\begin{equation*}
		\begin{split}
			|(iv)|
			& \le |\gamma| \cdot \|\p_x u^\ee \|_{H^{s-1}(\ci)} \|\p_x v
			\|_{H^{s-1}(\ci)} \|v\|_{H^s(\ci)}
			\\
			& \lesssim \|u^\ee\|_{H^s(\ci)} \|v\|_{H^s(\ci)}^2.
		\end{split}
	\end{equation*}
	Hence, collecting our estimates for (\hyperref[8u]{i}),
	(\hyperref[8u]{ii}), (\hyperref[8u]{iii}), and (\hyperref[8u]{iv})
	yields
		\begin{equation}
		\begin{split}
			|B| 
			& \lesssim
			\|u^\ee\|_{H^s(\ci)}
			\|v\|_{H^s(\ci)}^2 + \|u^\ee\|_{H^{s+1}(\ci)}
			\|v\|_{H^{s-1}(\ci)} \|v\|_{H^s(\ci)}.
			\label{14u}
		\end{split}
	\end{equation}
	Combining estimates \eqref{8'u} and \eqref{14u} and recalling
	\eqref{6u}, we obtain
	\begin{equation}
		\begin{split}
			\frac{1}{2}\frac{d}{dt}\|v\|_{H^{s}(\ci)}^2
			& \le c_s(\|v\|_{H^s(\ci)}^3 + \|u^\ee\|_{H^s(\ci)}
			\|v\|_{H^s(\ci)}^2
			\\
			& + \|u^\ee\|_{H^{s+1}(\ci)}
			\|v\|_{H^{s-1}(\ci)} \|v\|_{H^s(\ci)})
			\label{15u}
		\end{split}
	\end{equation}
	where $c_s$ is a constant depending only on $s$.
	Note that the first two terms in the parentheses on the right hand side
	of \eqref{15u} will offer us little trouble;
	it is the third term that requires special care (due to the
	$\|u^\ee\|_{H^{s+1}(\ci)}$ factor, which becomes increasingly large as
	$\ee$ decreases). More precisely:
	%
	%
	%
	\begin{remark}
	\label{lem5r}
	For $r \ge s > 3/2$ and $0 < \ee <<1$, 
	\begin{equation}
		\begin{split}
			\|u^{\ee} (t) \|_{H^r(\ci)} \le C \, \ee^{s-r}
			\label{700r}
		\end{split}
	\end{equation}
	where $C = C(r, \|u_0\|_{H^s(\ci)})$.
\end{remark}	
\subsection{ Proof.} By part (iii) of Theorem
\ref{thm:HR_existence_continous_dependence}, proved in Section
\ref{existence}, we have
\begin{equation}
	\begin{split}
		\|u^\ee(t) \|_{H^r(\ci)}^2
		& \le C' \|u^\ee (0)\|_{H^r(\ci)}^2
		\\
		& = C' \|J_\ee u_0\|_{H^r(\ci)}^2
		\\
		& = C' \sum_{\xi \in \zz} |\widehat{j_\ee} (\xi) \widehat{u_0}(\xi)
		|^2 \cdot (1 + \xi^2)^r
		\label{0qr}
	\end{split}
\end{equation}
Recall how we chose the mollifier $J_\ee$; it will now play a fundamental role. Since
\eqref{widehat-def} holds by construction, \eqref{0qr} gives 
\begin{equation}
	\begin{split}
		\|u^\ee(t) \|_{H^r(\ci)}^2
		& = C' \sum_{\xi \in \zz} |\widehat{j }(\ee \xi)|^2 \cdot (1 +
		\xi^2)^{r-s} \cdot |\widehat{u_0}(\xi)|^2 \cdot (1 + \xi^2)^s
		\\
		& = C'|\widehat{u_0}(0)|^2 +
		C' \sum_{\xi \in \zz \setminus {0}} |\widehat{j }(\ee \xi)|^2 \cdot (1 +
		\xi^2)^{r-s} \cdot |\widehat{u_0}(\xi)|^2 \cdot (1 + \xi^2)^s.
		\label{1qr}
	\end{split}
\end{equation}
Assume $r \ge s$. Since $\widehat{j }(\xi) \in \mathcal{S}(\rr)$, 
\begin{equation}
	\label{schwartz}
	\begin{split}
		|\widehat{j }(\ee \xi)| \le c_r |\ee \xi |^{s-r}, \quad \xi \neq 0.
	\end{split}
\end{equation}
Applying \eqref{schwartz} to \eqref{1qr}, we obtain
\begin{equation}
	\label{calc_ue}
	\begin{split}
		\|u^\ee (t)\|_{H^r(\ci)}^2 
		& \le C' |\widehat{u_0}(0) |^2 + c_r \sum_{\xi \in \zz \setminus
		{0}} |\ee \xi |^{2(s-r)} \cdot (1 + \xi^2)^{r-s}
		|\widehat{u_0}(\xi) |^2 \cdot (1 + \xi^2)^s
		\\
		& \le C' |\widehat{u_0}(0) |^2 + 2^{r-s} c_r \ee^{2(s-r)}
		\sum_{\xi \in \zz \setminus {0}} |\widehat{u_0}(\xi)|^2 \cdot (1 +
		\xi^2)^s
		\\
		& \le C' \|u_0\|_{H^s(\ci)}^2 + 2^{r-s} c_r \ee^{2(s-r)}
		\|u_0\|_{H^s(\ci)}^2
		\\
		& = (C' + 2^{r-s} c_r \ee^{2(s-r)}) \cdot \|u_0\|^2_{H^s(\ci)}.
	\end{split}
\end{equation}
Assuming $0 < \ee <<1$, we conclude from \eqref{calc_ue} that 
\begin{equation*}
	\begin{split}
		\|u^\ee(t)\|_{H^s(\ci)} \le C \ee^{s-r}
	\end{split}
\end{equation*}
where $C = C(r, \|u_0\|_{H^s(\ci)})$. $\qquad \Box$
	We remark that, despite the blowup of $\|u^\ee \|_{H^{s+1}(\ci)}$
	as $\ee$ becomes small, our difficulties would have been
	amplified if we had originally taken $v=w-u$ for some arbitrary
	solution $w$ to the HR
	i.v.p with initial data $w_0 \in H^s(\ci)$, for then we would be dealing with
	\eqref{15u}, with $w$ substituted in for $u^\ee$. However, note that 
	$\|w\|_{H^{s+1}(\ci)}$ might not even be bounded, whereas $\|u^\ee
	\|_{H^{s+1}(\ci)}$ is always bounded, for any $\ee > 0$.
	%
	%
	In light of the blowup of $\|u^\ee \|_{H^{s+1}(\ci)}$,
	our strategy in tackling
	\eqref{15u} will be as follows. First, we will obtain an estimate for
	$\|v\|_{H^\sigma(\ci)}$ for suitably chosen $\sigma < s-1$. Then, we
	will use this estimate to interpolate between $\|v\|_{H^\sigma(\ci)}$
	and $\|v\|_{H^s(\ci)}$,
	yielding an estimate for $\|v\|_{H^{s-1}(\ci)}$ which will allow us to control the growth of
	$\|u^\ee\|_{H^{s+1}(\ci)}$. 
	%
	%
	%
	%
\begin{lemma} 
	\label{lem6r}
	For $\sigma$ such that $1/2 < \sigma < 1$ and $\sigma + 1 \le s$, we have
	\begin{equation}
	\begin{split}
		\|v\|_{H^{\sigma}(\ci)} \le C \cdot o(\ee^{s- \sigma }), \qquad |t| \le T
	\end{split}
\end{equation}
where $C=C(\|u_0\|_{H^s(\ci)})$.
\end{lemma}
%
%
%
\subsection{ Proof.}
Recall that $v$ solves the Cauchy-problem \eqref{4u}-\eqref{5u}.
Applying $D^\sigma$ to both sides of \eqref{4u}, multiplying by
$D^\sigma v$, and integrating, we obtain the
relation
\begin{equation*}
	\begin{split}
		\frac{1}{2}\frac{d}{dt}\|v(t)\|_{H^\sigma(\ci)}^2
		= & - \frac{\gamma}{2}\int_{\ci} D^\sigma
		\p_x \left[ \left( u + u^\ee \right)v
		\right]\cdot D^\sigma v \ dx
		\\
		& - \frac{3-\gamma}{2}\int_{\ci} D^{\sigma
		-2} \p_x \left[ \left( u + u^\ee
		\right)v \right] \cdot D^\sigma v \ dx
		\\
		& - \frac{\gamma}{2}\int_{\ci} D^{\sigma
		-2}
		\p_x \left[ \left( \p_x u + \p_x u^\ee
		\right)\cdot \p_x v \right] \cdot
		D^\sigma v \ dx.
	\end{split}
\end{equation*}
Repeating calculations \eqref{X}-\eqref{12}, with $E$ set to zero,
$u^{\omega,n}$ replaced by $u$, $u_{\omega,n}$ replaced by $u^\ee$, and
$\sigma$ and $\rho$ chosen such that
%
\begin{equation}
	\label{size_of_sigma}
	\begin{split}
	& 1/2 < \sigma < 1,
	\\
	& \sigma + 1 \le \rho \le s 
	\end{split}
\end{equation}
yields
 \begin{equation*}
	\begin{split}
		\frac{1}{2}\frac{d}{dt} \|v\|_{H^\sigma(\ci)}^2
		& \le
		c_s' (\|u^{\ee} + u\|_{H^{\rho}(\ci)} +
		\|\p_x(u^{\ee} + u) \|_{H^\sigma(\ci)})
		\cdot \|v\|_{H^\sigma(\ci)}^2.
	\end{split}
\end{equation*}
\medskip
By the Sobolev Imbedding Theorem, it follows that 
\begin{equation}
	\begin{split}
		\frac{1}{2}\frac{d}{dt} \|v\|_{H^{\sigma}(\ci)}^2
		& \le
		c_s \cdot \|u^{\ee}
		+ u\|_{H^{s}(\ci)}\cdot \|v\|_{H^{\sigma}(\ci)}^2.
		\label{10x}
	\end{split}
\end{equation}
Hence, applying the triangle inequality and
part (iii) of Theorem \ref{thm:HR_existence_continous_dependence} (proved
in Section \ref{existence}) to \eqref{10x} yields
%
\begin{equation}
	\begin{split}
		\label{11x}
		\frac{1}{2}\frac{d}{dt} \|v\|_{H^{\sigma}(\ci)}^2
		& \le
		c_s (\|u^{\ee}(0)\|_{H^{s}(\ci)}
		+ \|u(0)\|_{H^{s}(\ci)})\cdot \|v\|_{H^{\sigma}(\ci)}^2
		\\
		& = c_s (\|J_\ee u_0\|_{H^{s}(\ci)}
		+ \|u_0\|_{H^{s}(\ci)})\cdot \|v\|_{H^{\sigma}(\ci)}^2.
	\end{split}
\end{equation}
We now need the following:
\begin{proposition}
	\label{lem3r}
	For arbitrary $u \in L^2(\ci)$,
	\begin{equation}
		\begin{split}
			\|J_\ee u\|_{H^s(\ci)} \le \|u\|_{H^s(\ci)}.
			\label{lem100u}
		\end{split}
	\end{equation}
\end{proposition}
%
%
%
%
\subsection{ Proof.}
\begin{equation*}
	\begin{split}
		\|J_\ee u\|_{H^s(\ci)} 
		& = \left[\sum_{\xi \in \zz} |\widehat{j_\ee * u}(\xi) |^2
		(1+\xi^2)^s \right ]^{1/2}
		\\
		& = \left [ \sum_{\xi \in \zz} |\widehat{j_\ee} (\xi) \widehat{u}(\xi) |^2
		(1+ \xi^2)^s \right ]^{1/2}
		\\
		& = \left [ \sum_{\xi \in \zz} |\widehat{j}(\ee \xi)
		\widehat{u}(\xi)|^2 ( 1+ \xi^2)^s \right ]^{1/2}
	\end{split}
\end{equation*}
and since $|\widehat{j }(\ee \xi) | \le 1$ by \eqref{0u}, the result
follows. $\qquad \Box$
Using estimate \eqref{11x}, and applying Proposition \ref{lem3r}, 
we obtain the critical estimate 
\begin{equation}
	\begin{split}
		\label{12x}
		\frac{1}{2}\frac{d}{dt} \|v\|_{H^{\sigma}(\ci)}^2
		& \le
	 C \|v\|_{H^{\sigma}(\ci)}^2
\end{split}
\end{equation}
where $C = C(\|u_0\|_{H^s(\ci)})$. Differentiating the left hand side of
\eqref{12x} and simplifying, we obtain
\begin{equation}
	\begin{split}
		\frac{d}{dt}\|v\|_{H^{\sigma}(\ci)} \le C \|v\|_{H^{\sigma}(\ci)}.
		\label{100x}
	\end{split}
\end{equation}
Let $y(t) = \|v\|_{H^{\sigma}(\ci)}$. Then \eqref{100x} gives
\begin{equation*}
	\begin{split}
		\frac{1}{y(t)}\frac{dy}{dt} \le C.
	\end{split}
\end{equation*}
Hence,
\begin{equation*}
	\begin{split}
		\int_0^t \frac{1}{y(\tau)} \frac{dy}{d \tau}
		\le \int_0^t C \ d \tau, \qquad |t| \le T
	\end{split}
\end{equation*}
from which we obtain
\begin{equation}
	\begin{split}
		\ln |y(t) | - \ln |y(0)| \le C t.
		\label{101x}
	\end{split}
\end{equation}
Simplifying \eqref{101x}, we have
\begin{equation*}
	\begin{split}
		\ln \left |\frac{y(t)}{y(0)} \right | \le C t
	\end{split}
\end{equation*}
Since $y(t)$ is non-negative for all $t \in \rr$, this yields the estimate
\begin{equation*}
	\begin{split}
		y(t) \le y(0) e^{C t}, \qquad |t| \le T.
	\end{split}
\end{equation*}
Substituting back in $\|v\|_{H^{\sigma}(\ci)}$ for $y$, we get
\begin{equation}
	\label{conc-lemma}
	\begin{split}
		\|v\|_{H^{\sigma}(\ci)}
		& \le e^{C t}\|v(0)\|_{H^{\sigma}(\ci)}
		\\
		& = e^{C t}\|u(0) - u^\ee(0) \|_{H^{\sigma}(\ci)}
		\\
		& = e^{C t}\|u_0 - J_\ee u_0 \|_{H^{\sigma}(\ci)}.
	\end{split}
\end{equation}
We now require a critical operator norm estimate which will play an
important role later on.
\begin{proposition}
	\label{lem4r}
	For $r \le s$ and $\ee>0$
	\begin{equation}
	\label{0r}
		\begin{split}
			\|I - J_\ee\|_{L(H^s(\ci), H^r(\ci))} \le o(\ee^{s-r}).
		\end{split}
	\end{equation}
\end{proposition}
Using Proposition \ref{lem4r}, we conclude from estimate \eqref{conc-lemma} that
\begin{equation*}
	\begin{split}
		\|v\|_{H^{\sigma}(\ci)} \le C \cdot o(\ee^{s - \sigma}), \qquad |t|
		\le T
	\end{split}
\end{equation*}
where $C=C(\|u_0\|_{H^s(\ci)})$, completing the proof of Lemma \ref{lem6r}.
$\qquad \Box$
%
\subsection{ Proof of Proposition \ref{lem4r}.}
Pick an arbitrary $u \in H^s(\ci)$ such that $\|u\|_{H^s(\ci)} = 1$, and $r, s \in \rr$ such that $r \le s$. Using the fact that
$\widehat{j_\ee}(\xi) = \widehat{j}(\ee \xi)$ by construction, we have 
\begin{equation}
	\begin{split}
		\|u - J_\ee u\|_{H^r(\ci)}^2 
		& = \sum_{\xi \in \zz} |\widehat{u}(\xi) - \widehat{j_\ee * u}(\xi) |^2
		(1+\xi^2)^r
		\\
		& = \sum_{\xi \in \zz} |\widehat{u}(\xi) - \widehat{j_\ee}(\xi)
		\widehat{u}(\xi) |^2 (1+\xi^2)^r
		\\
		& = \sum_{\xi \in \zz} | [1- \widehat{j_\ee}(\xi] \cdot \widehat{u}(\xi) |^2
		(1+\xi^2)^r
		\\
		& = \sum_{\xi \in \zz} | [1- \widehat{j}(\ee \xi)] \cdot \widehat{u}(\xi) |^2
		(1+\xi^2)^r.
		\label{1r}
	\end{split}
\end{equation}
Assume $r \le s$. Then by construction (see \ref{0u}) we have
\begin{equation*}
	\begin{split}
		|1 - \widehat{j } (\xi) | \le |\xi|^{s-r}
	\end{split}
\end{equation*}
for all $\xi \in \rr$; hence
\begin{equation}
	\begin{split}
		|1 - \widehat{ j }(\ee \xi)| \le |\ee \xi |^{s-r}, \quad \forall
		\xi \in \rr, \ \ee > 0.
		\label{2r}
	\end{split}
\end{equation}
Applying \eqref{2r} to \eqref{1r} and recalling that $r \le s$, we obtain
\begin{equation}
	\label{2pr}
	\begin{split}
	\|u - J_\ee u\|_{H^r(\ci)}^2 
	& \le \sum_{\xi \in \zz}  |\ee \xi |^{2(s-r)}
	|\widehat{u}(\xi)|^2 (1 + \xi^2)^r
	\\
	& = \ee^{2(s-r)} \sum_{\xi \in \zz} |\widehat{u}(\xi)|^2  \cdot (\xi^2)^{s-r}
	(1 + \xi^2)^{r-s} (1 + \xi^2)^{s}
	\\
	& \le \ee^{2(s-r)}
	\sum_{\xi \in \zz} |\widehat{u}(\xi)|^2 (1 + \xi^2)^s
	\\
	& =  \ee^{2(s-r)}.
	\end{split}
\end{equation}
Furthermore,
\begin{equation*}
	\begin{split}
		& |[1- \widehat{j_\ee}(\xi)] \cdot \widehat{u}(\xi)|^2 (1 + \xi^2)^r \le
		|\widehat{u}(\xi)|^2 (1 + \xi^2)^r, \quad \ee > 0, \ \text{and}
		\\
		& \sum_{\xi \in \zz} |\widehat{u}(\xi)|^2 (1 + \xi^2)^r < \infty;
	\end{split}
\end{equation*}
therefore, by the dominated convergence theorem for series
\begin{equation}
	\label{o1}
	\begin{split}
		\lim_{\ee \to 0} \|u - J_\ee u \|_{H^r }^2 
		& = \lim_{\ee \to 0} \sum_{\xi \in \zz} |[1-\widehat{j_\ee}(\xi)]
		\widehat{u}(\xi) |^2 (1 + \xi^2)^r
		\\
		& = \lim_{\ee \to 0} \sum_{\xi \in \zz} |[1-\widehat{j}(\ee \xi)]
		\widehat{u}(\xi) |^2 (1 + \xi^2)^r
		\\
		& = \sum_{\xi \in \zz} \lim_{\ee \to 0} |[1-\widehat{j}(\ee \xi)]
		\widehat{u}(\xi) |^2 (1 + \xi^2)^r
		\\
		& = 0.
	\end{split}
\end{equation}
To complete the proof of Proposition \ref{lem4r}, we take note of the following interpolation result:
\begin{remark}
	\label{lem2r}
	For $\sigma < r \le s$ and arbitrary $u \in L^2(\ci)$,
	\begin{equation}
		\begin{split}
			\|u\|_{H^{r}(\ci)} \le
			\|u\|_{H^\sigma(\ci)}^{(r-s)/(\sigma -s)}
			\|u\|_{H^s(\ci)}^{1 - (r-s)/(\sigma -s)}.
			\label{16u}
		\end{split}
	\end{equation}
\end{remark}
%
%
%
%
\subsection{ Proof.} Assuming $u \in L^2(\ci)$ and $\sigma < r \le s$,
we rewrite and apply Holder's inequality:
\begin{equation*}
	\begin{split}
		&\|u\|_{H^{r}(\ci)}^2
		\\
		& = \sum_{\xi \in \zz} |\widehat{u}(\xi)|^2 (1 + \xi^2)^{r}
		\\
		& = \sum_{\xi \in \zz}
		\left [|\widehat{u}(\xi)|^2 (1 + \xi^2)^\sigma \right ]^{(r-s)/(\sigma -s)}
		\cdot \left [ |\widehat{u}(\xi )
		|^2 (1+ \xi^2)^s \right ] ^{1 - (r-s)/(\sigma -s)} 
		\\
		& \le \|\left[ |\widehat{u}(\xi)|^2 (1 + \xi^2)^\sigma
		\right]^{(r-s)/(\sigma -s)} \|_{l^{(\sigma -s)/(r-s)}(\zz)}
		\\
		& \cdot \|\left[ |\widehat{u}(\xi)|^2 (1 + \xi^2)^\sigma
		\right]^{1- (r-s)/(\sigma -s)} \|_{l^{1/[1 -(\sigma -s)/(r-s)]}(\zz)}
		\\
		& = \|v\|_{H^\sigma(\ci)}^{2(r-s)/(\sigma -s)}
		\|v\|_{H^s(\ci)}^{2[1 - (r-s)/(\sigma -s)]}
	\end{split}
\end{equation*}
from which the result follows. 
Assume without loss of generality that $s > 0$. Applying Remark \ref{lem2r}, and estimates \eqref{2pr} and \eqref{o1}, we
see that for $r>0$ 
\begin{equation*}
	\begin{split}
		\|u - J_\ee u \|_{H^r(\ci)}
		& \le \|u - J_\ee u
		\|_{L^2(\ci)}^{(s-r)/s} \|u - J_\ee u \|_{H^s(\ci)}^{1 -
		(s-r)/s}
		\\
		& = \left( \ee^{s} \right)^{(s-r)/s} \cdot o(1)
		\\
		& = o(\ee^{s-r})
	\end{split}
\end{equation*}
Similarly, for $r < 0$
\begin{equation*}
	\begin{split}
		\|u - J_\ee u \|_{H^r(\ci)}^2
		& \le \|u - J_\ee u
		\|_{H^\sigma(\ci)}^{(r-s)/(\sigma - s)} \|u - J_\ee u \|_{H^s(\ci)}^{1 -
		(r-s)/(\sigma -s)}
		\\
		& = \left( \ee^{s-\sigma} \right)^{(r-s)/(\sigma -s)} \cdot o(1)
		\\
		& = o(\ee^{s-r})
	\end{split}
\end{equation*}
Lastly, for the case $r=0$, we note that \eqref{o1} implies $\|u - J_\ee u
\|_{H^r(\ci)} \le o(1)$ for all $r \le s$. Hence, the proof of Proposition
\ref{lem4r} is complete.  $\quad \Box$
%
%
%
%
%
%
%
%
%
%
%
%
%
%
%
%
%
%
%
%
%
%\subsection{ Proof.} By \eqref{uniform_bound_for_u}, we have
%\begin{equation}
%	\begin{split}
%		\|u^\ee(t, \cdot \|_{H^r(\ci)}^2
%		& \le C \|u^\ee (0, \cdot)
%		\|_{H^r(\ci)}^2
%		\\
%		& = \|J_\ee u_0 \|_{H^r(\ci)}^2
%		\\
%		& = \sum_{\xi \in \zz} |\widehat{j_\ee}(\xi) \widehat{u_0}(\xi) |^2
%		\cdot (1 + \xi^2)^r
%		\\
%		& \le \sum_{\xi \in \zz} |[1-\widehat{ j_\ee}(\xi)] \widehat{u_0}(\xi) |^2
%		\cdot (1 + \xi^2)^r
%		\\
%		& + \sum_{\xi \in \zz} |\widehat{j_\ee}(\xi) \widehat{u_0}(\xi) |^2
%		\cdot (1 + \xi^2)^r.
%		\label{1q}
%	\end{split}
%\end{equation}
%Since $J_\ee u_0$ is smooth, by \eqref{uniform_bound_for_u} we have
%\begin{equation*}
%	\begin{split}
%		\sum_{\xi \in \zz} |\widehat{j_\ee}(\xi) \widehat{u_0}(\xi) |^2
%		\cdot (1 + \xi^2)^r
%		= \|J_\ee u_0\|_{H^r(\ci)} = C_r
%	\end{split}
%\end{equation*}
%for all $r \ge 3/2$.
%
%
%
%
%
%
%
%
We now return to analyzing
\eqref{15u}. Applying Remark \ref{lem5r}, Remark \ref{lem2r}, and Lemma
\ref{lem6r}, we have
\begin{equation}
	\begin{split}
		\label{200x}
		\|u^\ee \|_{H^{s+1}(\ci)} \|v \|_{H^{s-1}(\ci)} \|v\|_{H^s(\ci)}
		& \le C''' \ee^{-1} \cdot \|v\|_{H^\sigma(\ci)}^{1/(s-\sigma)}
		\|v\|_{H^s(\ci)}^{2 - 1/(s- \sigma)}
		\\
		& \le C'' \ee^{-1} \cdot o(\ee^{s- \sigma})^{1/(s-\sigma)}
		\|v\|_{H^s(\ci)}^{2- 1/(s-\sigma)}
		\\
		& \le C'' \cdot o(1) \cdot \|v\|_{H^s(\ci)}^{2- 1/(s-\sigma)}
	\end{split}
\end{equation}
where we stress that $C'' = C''(\|u_0\|_{H^s(\ci)})$ does not depend on $\ee$.
Hence, $\|v\|_{H^{s-1}(\ci)}$ has proved to be sufficient to control the
growth of $\|u^\ee \|_{H^{s+1}(\ci)}$. For the remaining terms of
\eqref{15u}, we leave $\|v\|_{H^s(\ci)}^3$ as is, and note that by Remark \ref{lem5r}
\begin{equation}
	\begin{split}
		\|u^\ee\|_{H^s(\ci)} \|v\|_{H^s(\ci)}^2 \le C'
		\cdot \|v\|_{H^s(\ci)}^2
		\label{u-ep-bound}
	\end{split}
\end{equation}
where $C' = C'(\|u_0\|_{H^s(\ci)})$. Hence, applying \eqref{u-ep-bound} and \eqref{200x} to \eqref{15u}, we obtain
\begin{equation}
	\begin{split}
		\frac{1}{2} \frac{d}{dt} \|v\|_{H^s(\ci)}^2 \le C (
		\|v\|_{H^s(\ci)}^3 + \|v\|_{H^s(\ci)}^2 + \ee ^{-1} o(\ee) \|v\|_{H^s(\ci)}^{2-
		1/(s- \sigma)}).
		\label{201x}
	\end{split}
\end{equation}
where $C=C(\|u_0\|_{H^s(\ci)}$ does not depend on $\ee$. We also remark 
that $\|v(t)\|_{H^s(\ci)}$ is uniformly bounded for all $\ee > 0$, since by
the triangle inequality, Proposition \ref{lem3r}, and part (iii) of Theorem
\ref{thm:HR_existence_continous_dependence} we have
\begin{equation*}
	\begin{split}
		\|v(t) \|_{H^s(\ci)}
		& = \|u - u^\ee \|_{H^s(\ci)}
		\\
		& \le \|u \|_{H^s(\ci)} + \|u^\ee \|_{H^s(\ci)}
		\\
		& \le 2( \|u_0\|_{H^s(\ci)} + \|J_\ee u_0\|_{H^s(\ci)})
		\\
		& \le 4 \|u_0\|_{H^s(\ci)}.
	\end{split}
\end{equation*}
Hence, \eqref{201x} gives
\begin{equation}
	\begin{split}
		\lim_{\ee \to 0} \frac{1}{2} \frac{d}{dt} \|v\|_{H^s(\ci)}^2 \le
		 \lim_{\ee \to 0} C (
		\|v\|_{H^s(\ci)}^3 + \|v\|_{H^s(\ci)}^2).
		\label{202x}
	\end{split}
\end{equation}
Differentiating the left hand side of
\eqref{202x} and simplifying, it follows that
\begin{equation*}
	\begin{split}
		\lim_{\ee \to 0}\frac{d}{dt} \|v\|_{H^s(\ci)} \le
		\lim_{\ee \to 0} C (\|v\|_{H^s(\ci)}^2 +
		\|v\|_{H^s(\ci)}).
	\end{split}
\end{equation*}
Letting $y = \|v\|_{H^s(\ci)}$ and rearranging, we obtain
\begin{equation*}
	\begin{split}
		\lim_{\ee \to 0} \ \frac{1}{y(y+1)} \frac{dy}{dt} \le C	
	\end{split}
\end{equation*}
which can be rewritten as
\begin{equation*}
	\begin{split}
		\lim_{\ee \to 0}
		\left( \frac{1}{y} - \frac{1}{y+1} \right)\frac{dy}{dt} \le C
	\end{split}
\end{equation*}
implying
\begin{equation}
	\label{est-int'}
	\begin{split}
		\lim_{\ee \to 0} \left [
\int_0^t \frac{1}{y} \frac{dy}{d \tau} \ d \tau
		- \int_0^t \frac{1}{y+1} \frac{dy}{d \tau} \ d \tau \right ]
		\le \int_0^t C \ d \tau, \quad |t| \le T.
	\end{split}
\end{equation}
Hence \eqref{est-int'} gives 
\begin{equation}
	\begin{split}
	\lim_{\ee \to 0} 
	\left [ \ln \left | \frac{y(t)}{y(0)}
	\cdot \frac{y(0) + 1}{y(t) + 1} \right | \right ] \le C t.
		\label{301''qx}
	\end{split}
\end{equation}
Exponentiating both sides of \eqref{301''qx}, and noting that $f(x) = e^x$
is a continuous function on $\rr$, we must have
\begin{equation*}
	\begin{split}
		\lim_{\ee \to 0}  \
		\left | \frac{y(t)}{y(0)} \cdot \frac{y(0) + 1}{y(t) + 1} \right | \le e^{C t}.
	\end{split}
\end{equation*}
Rearranging, and recalling that $y(t) = \|v(t)\|_{H^s(\ci)} \ge 0$, we obtain
\begin{equation*}
	\begin{split}
		\lim_{\ee \to 0} \frac{y(t)}{y(t) + 1}
		\le \lim_{\ee \to 0} \frac{e^{C t} \cdot y(0)}{y(0) + 1} \le
		\lim_{\ee \to 0} e^{C t} \cdot y(0).
	\end{split}
\end{equation*}
Substituting back in $\|v(t)\|_{H^s(\ci)}$ for $y(t)$ gives
\begin{equation}
	\begin{split}
		\lim_{\ee \to 0}	\frac{\|v(t)\|_{H^s(\ci)}}{\|v(t)\|_{H^s(\ci)} + 1}  \le
		\lim_{\ee \to 0} e^{C t} \cdot \|v(0)\|_{H^s(\ci)}.
		\label{303'qx}
	\end{split}
\end{equation}
Since 
\begin{equation*}
	\label{303''qx}
	\begin{split}
		\|v(0)\|_{H^s(\ci)} = \|u_0 - J_\ee u_0 \|_{H^s(\ci)} \le
		\|u_0\|_{H^s(\ci)} \cdot o(1)
	\end{split}
\end{equation*}
by Proposition \ref{lem4r}, we conclude from \eqref{303'qx} that
\begin{equation}
	\label{304qx}
	\begin{split}
		\lim_{\ee \to 0} \|v(t)\|_{H^s(\ci)} = \lim_{\ee \to 0}
		\|u^\ee(t) - u(t)\|_{H^s(\ci)}= 0, \qquad |t| \le T,
	\end{split}
\end{equation}
and since the family $\left\{ u^\ee - u \right\}_\ee$ does not depend on $n$,
the proof of \eqref{enough_to_prove1} is complete. 
%
%
%
%
%
%
%
%
\subsection{ Proof of \eqref{enough_to_prove2}.} 
Let $v = u^\ee_n - u^\ee$. Then $v$ solves the Cauchy problem
\begin{align}
		\label{4qu}
		\p_t v 
		& =  -\gamma (v \p_x v + v \p_x u^\ee + u^\ee \p_x v)  
		\\
		& - D^{-2} \p_x \left\{ \left (\frac{3-\gamma}{2} \right )(v^2 +
		2u^\ee v) + \frac{\gamma}{2}\left[ (\p_x v)^2 + 2 \p_x u^\ee \p_x v \right]
		\right\}, \notag
		\\
		& v(0) =J_\ee(u_{0,n} - u_0).
		\label{5qu}
	\end{align}
Applying the operator $D^s$ to both sides of \eqref{4qu}, multiplying by
	$D^s$ and integrating, we have
	\begin{equation}
		\begin{split}
			\frac{1}{2}\frac{d}{dt} \|v\|_{H^s(\ci)} = A + B
			\label{6qu}
		\end{split}
	\end{equation}
	where
	\begin{equation}
		\begin{split}
			A
			& =  -\gamma \int_{\ci} D^s(v \p_x v) \cdot D^s v \
			dx
			- \frac{3- \gamma}{2} \int_\ci D^{s-2} \p_x (v^2) \cdot D^s v
			\ dx
			\\
			& - \frac{\gamma}{2}\int_\ci D^{s-2} \p_x (\p_x v)^2 \cdot D^s
			v \ dx
			\label{7qu}
		\end{split}
	\end{equation}
	and
	\begin{equation}
		\begin{split}
			B 
			 = &  \overbrace{-\gamma \int_\ci D^s (u^\ee \p_x v) \cdot D^s v \
			 dx}^{(i)}
			 \ \overbrace{-\gamma \int_\ci D^s (v \p_x u^\ee ) \cdot D^s v \
			 dx}^{(ii)}
			 \\
			  & \overbrace{- \ ( 3- \gamma) \int_\ci D^{s-2} \p_x (u^\ee v) \cdot D^s
			 v \ dx}^{(iii)}
			 \\
			 & \overbrace{-\gamma \int_\ci D^{s-2} \p_x
			(\p_x u^\ee \cdot \p_x v) \cdot D^s v \
			dx}^{(iv)}.
			\label{8qu}
		\end{split}
	\end{equation}
	Estimating as in \eqref{8'u}-\eqref{14u}, we obtain
	\begin{equation}
		\begin{split}
			\frac{1}{2}\frac{d}{dt}\|v\|_{H^{s}(\ci)}^2
			& \le c_s(\|v\|_{H^s(\ci)}^3 + \|u^\ee\|_{H^s(\ci)}
			\|v\|_{H^s(\ci)}^2
			\\
			& + \|u^\ee\|_{H^{s+1}(\ci)}
			\|v\|_{H^{s-1}(\ci)} \|v\|_{H^s(\ci)}).
			\label{15qu}
		\end{split}
	\end{equation}
	We now aim to control the growth of $\|u^\ee\|_{H^{s+1}(\ci)}$ by
	$\|v\|_{H^{s-1}(\ci)}$. To do so, we will need an estimate for
	$\|v\|_{H^{s-1}(\ci)}$, which we will obtain through the following lemma:
%
%
%
%
\begin{lemma}
	\label{lem:left}
	For $\sigma$ such that $1/2 < \sigma < 1$ and $\sigma + 1 \le s$, we have
	\begin{equation}
	\label{lem6rq}
	\begin{split}
		\|v\|_{H^{\sigma}(\ci)} = 
		\|u^\ee_n - u^\ee\|_{H^\sigma(\ci)}
		\le C \cdot o(\ee^{s- \sigma }) + \|u_0 - u_{0,n} \|_{H^s(\ci)}, \qquad |t| \le T
	\end{split}
\end{equation}
where $C=C(\|u_0\|_{H^s(\ci)})$.
\end{lemma}
%
%
%
\subsection{ Proof.}
Repeating calculations \eqref{X}-\eqref{12}, with $E$ set to zero, $u^{\omega,n}$
replaced by $u^\ee_n$, $u_{\omega,n}$ replaced by $u^\ee$, and $\sigma$ and $\rho$ chosen such that
\begin{equation}
	\label{size_of_sigma'}
	\begin{split}
	& 1/2 < \sigma < 1,
	\\
	& \sigma + 1 \le \rho \le s 
	\end{split}
\end{equation}
yields
 \begin{equation*}
	\begin{split}
		\frac{1}{2}\frac{d}{dt} \|v\|_{H^\sigma(\ci)}^2
		& \le
		C'' (\|u^{\ee}_n + u^\ee \|_{H^{\rho}(\ci)} +
		\|\p_x(u^{\ee}_n + u^\ee) \|_{H^\sigma(\ci)})
		\cdot \|v\|_{H^\sigma(\ci)}^2.
	\end{split}
\end{equation*}
\medskip
It follows that 
\begin{equation}
	\begin{split}
		\frac{1}{2}\frac{d}{dt} \|v\|_{H^{\sigma}(\ci)}^2
		& \le
		C'' \cdot \|u^{\ee}_n
		+ u^\ee\|_{H^{s}(\ci)}\cdot \|v\|_{H^{\sigma}(\ci)}^2.
		\label{10qx}
	\end{split}
\end{equation}
Applying the triangle inequality and
part (iii) of Theorem \ref{thm:HR_existence_continous_dependence} (proved in
Section \ref{existence})
to \eqref{10qx} yields
%
\begin{equation}
	\begin{split}
		\label{11qx}
		\frac{1}{2}\frac{d}{dt} \|v\|_{H^{\sigma}(\ci)}^2
		& \le
		C' (\|u^{\ee}_n(0)\|_{H^{s}(\ci)}
		+ \|u^\ee(0)\|_{H^{s}(\ci)})\cdot \|v\|_{H^{\sigma}(\ci)}^2
		\\
		& = C' (\|J_\ee u_{0,n}\|_{H^{s}(\ci)}
		+ \|J_\ee u_0\|_{H^{s}(\ci)})\cdot \|v\|_{H^{\sigma}(\ci)}^2.
	\end{split}
\end{equation}
Note that the family $\left\{ u_{0,n} \right\}_n$ is uniformly bounded in
$H^s(\ci)$. Hence, applying Lemma \ref{lem3r} to \eqref{11qx} we obtain the critical estimate 
\begin{equation}
	\begin{split}
		\label{12qx}
		\frac{1}{2}\frac{d}{dt} \|v\|_{H^{\sigma}(\ci)}^2
		& \le
	C \|v\|_{H^{\sigma}(\ci)}^2
\end{split}
\end{equation}
with $C = C(\|u_0\|_{H^s(\ci)}, \ R)$, where
\begin{equation}
	\label{r-def}
	R = \inf \left\{ R' \in \rr:\ \{u_{0,n}\} \subset B_{H^s(\ci)}(R',0)
	\right\}.
\end{equation}
Differentiating the left hand side of \eqref{12qx} and simplifying, we
obtain
\begin{equation}
	\begin{split}
		\frac{d}{dt}\|v\|_{H^{\sigma}(\ci)} \le C \|v\|_{H^{\sigma}(\ci)}.
		\label{100qx}
	\end{split}
\end{equation}
Let $y(t) = \|v\|_{H^{\sigma}(\ci)}$. Then \eqref{100qx} gives
\begin{equation*}
	\begin{split}
		\frac{1}{y(t)}\frac{dy}{dt} \le C.
	\end{split}
\end{equation*}
Hence,
\begin{equation*}
	\begin{split}
		\int_0^t \frac{1}{y(\tau)} \frac{dy}{d \tau}
		\le \int_0^t C \ d \tau, \qquad |t| \le T
	\end{split}
\end{equation*}
from which we obtain
\begin{equation}
	\begin{split}
		\ln |y(t) | - \ln |y(0)| \le C t.
		\label{101qx}
	\end{split}
\end{equation}
Simplifying \eqref{101qx}, we have
\begin{equation*}
	\begin{split}
		\ln \left |\frac{y(t)}{y(0)} \right | \le C t
	\end{split}
\end{equation*}
which yields the estimate
\begin{equation*}
	\begin{split}
		y(t) \le y(0) e^{C t}, \qquad |t| \le T.
	\end{split}
\end{equation*}
Substituting back in $\|v\|_{H^{\sigma}(\ci)}$ for $y$, we get
\begin{equation*}
	\begin{split}
		\|v\|_{H^{\sigma}(\ci)}
		& \le e^{C t}\|v(0)\|_{H^{\sigma}(\ci)}
		\\
		& = e^{C t}\|u^\ee(0) - u^\ee_n(0) \|_{H^{\sigma}(\ci)}.
	\end{split}
\end{equation*}
To conclude the proof, we apply the following:
\begin{proposition}
		\label{lem11r}
	For $r \le s$,
	\begin{equation}
		\begin{split}
			\|u^\ee(0) - u_n^\ee (0) \|_{H^r(\ci)} \le C
			\cdot o(\ee^{s-r}) + \|u_0 - u_{0,n} \|_{H^s(\ci)}
			\label{3w}
		\end{split}
	\end{equation}
	where $C=C(\|u_0\|_{H^s(\ci)})$ does not depend on $n$.
\end{proposition}
%
%
Recalling \eqref{r-def}, we deduce by Proposition \ref{lem11r}
\begin{equation*}
	\begin{split}
		\|v\|_{H^{\sigma}(\ci)} \le C \cdot o(\ee^{s - \sigma}) + \|u_0 -
		u_{0,n} \|_{H^s(\ci)} \qquad |t| \le T
	\end{split}
\end{equation*}
where $C=C(\|u_0\|_{H^s(\ci)}), \ R)$ does not depend
on $n$, completing the proof of Lemma \ref{lem:left}. $\qquad \Box$
%
%
\subsection{ Proof of Proposition \ref{lem11r}.} We write
\medskip
\begin{equation}
	\begin{split}
		\|u^\ee(0) - u_n^\ee (0) \|_{H^r(\ci)} 
		& = \|J_\ee u_0 - J_\ee u_{0,n} \|_{H^r(\ci)}
		\\
		& \le \|J_\ee u_0 - u_0 \|_{H^r(\ci)} + \| u_0 - u_{0,n}
		\|_{H^r(\ci)}
		\\
		& + \|u_{0,n} - J_\ee u_{0,n} \|_{H^r(\ci)}
		\\
		& \le \|I - J_\ee\|_{L(H^s(\ci), H^r(\ci))} \|u_0\|_{H^s(\ci)}
		\\
		& +
		\|u_0 - u_{0,n} \|_{H^r(\ci)} + 
		\|I - J_\ee\|_{L(H^s(\ci), H^r(\ci))} \|u_{0,n}\|_{H^s(\ci)}.
		\label{4w}
	\end{split}
\end{equation}
Applying Proposition \ref{lem4r} to \eqref{4w}, and recalling that the family
$\left\{ u_{0,n} \right\}_n$ belongs to a bounded subset of
$H^s(\ci)$, we have
\medskip
\begin{equation}
	\label{finito}
	\begin{split}
		\|u^\ee(0) - u_n^\ee (0) \|_{H^r(\ci)} 
		& \le
		C' \cdot o(\ee^{s-r}) \cdot \|u_0\|_{H^s(\ci)}
		 \\
		 & + \|u_0 - u_{0,n} \|_{H^r(\ci)} + C' \cdot o (\ee^{s-r}) \cdot
		 \|u_{0,n}\|_{H^s(\ci)}
		 \\
		 & \le
		 C' \cdot o(\ee^{s-r}) \cdot \|u_0\|_{H^s(\ci)}
		 \\
		 & + \|u_0 - u_{0,n} \|_{H^s(\ci)} + C' \cdot o (\ee^{s-r}) \cdot
		 R
	\end{split}
\end{equation}
where $R$ is defined as in \ref{r-def}. The result follows immediately from
\eqref{finito}. $\qquad \Box$
We are now prepared to interpolate. Recall \eqref{15qu}. Applying Remark \ref{lem5r}, Remark \ref{lem2r}, and
Proposition \ref{lem11r} gives
\begin{equation*}
	\begin{split}
		& \|u^\ee \|_{H^{s+1}(\ci)} \|v\|_{H^{s-1}(\ci)} \|v\|_{H^s
		(\ci)}
		\\
		&\le C' \ee^{-1} \cdot \|v\|_{H^\sigma(\ci)}^{1/(s-\sigma)}
		\|v\|_{H^s(\ci)}^{2 - 1/(s- \sigma)}
		\\
		& \le C' \ee^{-1} \cdot \Big [C \cdot o(\ee^{s- \sigma}) + \|u_0 -
		u_{0,n}\|_{H^s(\ci)} \Big ]^{1/(s-\sigma)}
		\cdot \|v\|_{H^s(\ci)}^{2- 1/(s-\sigma)}
	\end{split}
\end{equation*}
from which we obtain
\begin{equation}
	\begin{split}
		\label{200qx}
		\|u^\ee\|_{H^{s+1}(\ci)} \|v\|_{H^{s-1}(\ci)} \|v \|_{H^s(\ci)}
		& \lesssim  o(1) + \ee^{-1}
		\|u_0-u_{0,n}\|_{H^s(\ci)}^{1/(s-\sigma)}\|v\|_{H^s(\ci)}^{2- 1/(s-\sigma)}.
	\end{split}
\end{equation}
We wish to control the growth of the second term of the
right hand side of \eqref{200qx}.
First, note that the triangle inequality, part (iii) of Theorem
\ref{thm:HR_existence_continous_dependence} and Proposition \ref{lem3r} imply
\begin{equation}
	\begin{split}
		\|v\|_{H^s(\ci)} & = \|u^\ee_n - u^\ee \|_{H^s(\ci)} 
		\\
		& \le \|u^\ee_n\|_{H^s(\ci)} + \|u^\ee \|_{H^s(\ci)}  
		\\
		& \le 2\left[  \|J_\ee u_{0,n}\|_{H^s(\ci)} + \|J_\ee u_0 \|_{H^s(\ci)} 
		 \right]
		\\
		& \le 2 \left[ \|u_{0,n} \|_{H^s(\ci)} + \|u_0 \|_{H^s(\ci)} 
		\right], \qquad |t| \le T
		\label{growth_v}
	\end{split}
\end{equation}
and since $\{u_{0,n}\}_n$ belongs to a bounded subset of
$H^s(\ci)$, we see from \eqref{growth_v} that $\|v \|_{H^s(\ci)}$ is
uniformly bounded in $n$ \emph{and} $\ee$.  Secondly, since $\|u_0 -
u_{0,n} \|_{H^s(\ci)} \to 0$ uniformly in $n$, then for any given $\ee$ we
can chose a family $\{N_j\} $ such that
\begin{equation}
	\begin{split}
		\|u_0 - u_{0,n} \|_{H^s(\ci)} \lesssim
		\frac{\ee^{(s-\sigma)}}{2^{j(s -\sigma)}}, \quad n >
		N_j.
		\label{uniform_n}
	\end{split}
\end{equation}
Thirdly, by Remark \ref{lem5r}, we have 
\begin{equation}
	\label{u-ee-bound}
	\|u^\ee \|_{H^s(\ci)} \le C(\|u_0\|_{H^s(\ci)}), \quad \forall \ee > 0.
\end{equation}
Applying \eqref{200qx} to \eqref{15qu} in light of 
\eqref{growth_v}, \eqref{uniform_n}, and \eqref{u-ee-bound}, we obtain
\begin{equation*}
		\begin{split}
			\lim_{n \to \infty }
			\frac{1}{2}\frac{d}{dt}\|v\|_{H^{s}(\ci)}^2
			& \le
			C \lim_{n \to \infty} \Big [\|v\|_{H^s(\ci)}^3 +
			\|v\|_{H^s(\ci)}^2 + o(1)\Big ]
		\end{split}
	\end{equation*}
	for every $\ee > 0$, where $C = C(\|u_0\|_{H^s(\ci)}, \ R)$ with
	$R$ defined as in \eqref{r-def}; hence we have
\begin{equation}
		\begin{split}
			\lim_{\substack{n \to \infty \\ \ee \to 0} }
			\frac{1}{2}\frac{d}{dt}\|v\|_{H^{s}(\ci)}^2
			& \le C
			\lim_{\substack{n \to \infty \\ \ee \to 0}}
			\Big [\|v\|_{H^s(\ci)}^3 + 
			\|v\|_{H^s(\ci)}^2 \Big ].
			\label{15qx}
		\end{split}
	\end{equation}
	We differentiate the left hand side of \eqref{15qx} and obtain
\begin{equation*}
	\begin{split}
		\lim_{\substack{n \to \infty \\ \ee \to 0}}\frac{d}{dt}
		\|v\|_{H^s(\ci)} \le C
		\lim_{\substack{n \to \infty \\ \ee \to 0}} \left [\|v\|_{H^s(\ci)}^2 +
		\|v\|_{H^s(\ci)} \right ].
	\end{split}
\end{equation*}
Letting $y = \|v\|_{H^s(\ci)}$ and rearranging gives
\begin{equation*}
	\begin{split}
		\lim_{\substack{n \to \infty \\ \ee \to 0} } \ \frac{1}{y(y+1)} \frac{dy}{dt}
		\le	C
	\end{split}
\end{equation*}
which can be rewritten as
\begin{equation*}
	\begin{split}
		\lim_{\substack{n \to \infty \\ \ee \to 0} }
		\left( \frac{1}{y} - \frac{1}{y+1} \right)\frac{dy}{dt} \le C 
	\end{split}
\end{equation*}
implying
\begin{equation}
	\label{est-int}
	\begin{split}
		\lim_{\substack{n \to \infty \\ \ee \to 0} } \left [
\int_0^t \frac{1}{y} \frac{dy}{d \tau} \ d \tau
		- \int_0^t \frac{1}{y+1} \frac{dy}{d \tau} \ d \tau \right ]
		\le \int_0^t C \ d \tau, \quad |t| \le T.
	\end{split}
\end{equation}
Recalling that $y(t) = \|v(t)\|_{H^s(\ci)} > 0$, \eqref{est-int} gives 
\begin{equation}
	\begin{split}
	\lim_{\substack{n \to \infty \\ \ee \to 0} }
	\left [ \ln \left ( \frac{y(t)}{y(0)}
	\cdot \frac{y(0) + 1}{y(t) + 1} \right ) \right ] \le C t.
		\label{301'qx}
	\end{split}
\end{equation}
Exponentiating both sides of \eqref{301'qx}, and noting that $f(x) = e^x$
is a continuous function on $\rr$, we must have
\begin{equation*}
	\begin{split}
		\lim_{\substack{n \to \infty \\ \ee \to 0} } \
		\frac{y(t)}{y(0)} \cdot \frac{y(0) + 1}{y(t) + 1} \le e^{C t}.
	\end{split}
\end{equation*}
Rearranging, we obtain
\begin{equation*}
	\begin{split}
		\lim_{\substack{n \to \infty \\ \ee \to 0}} \frac{y(t)}{y(t) + 1}
		\le \lim_{\substack{n \to \infty \\ \ee \to 0}} \frac{e^{C t} \cdot y(0)}{y(0) + 1} \le
		\lim_{\substack{n \to \infty \\ \ee \to 0}} e^{C t} \cdot y(0).
	\end{split}
\end{equation*}
Substituting back in $\|v(t)\|_{H^s(\ci)}$ for $y(t)$ gives
\begin{equation}
	\begin{split}
		\lim_{\substack{n \to \infty \\ \ee \to 0}}	\frac{\|v(t)\|_{H^s(\ci)}}{\|v(t)\|_{H^s(\ci)} + 1}  \le
		\lim_{\substack{n \to \infty \\ \ee \to 0}} e^{C t} \cdot \|v(0)\|_{H^s(\ci)}.
		\label{303qx}
	\end{split}
\end{equation}
Since by Proposition \ref{lem3r} 
\begin{equation*}
	\begin{split}
	\lim_{\substack{n \to \infty \\ \ee \to 0} }
	\|v(0)\|_{H^s(\ci)}
	& = \lim_{\substack{n \to \infty \\ \ee \to 0} }
	\|J_\ee u_{0,n} - J_\ee u_0 \|_{H^s(\ci)} 
	\\
	& \le \lim_{n \to \infty } \|u_{0,n} - u_0 \|_{H^s(\ci)}
	\\
	& = 0
	\end{split}
\end{equation*}
we deduce from \eqref{303qx} that
\begin{equation*}
	\begin{split}
		\lim_{\substack{n \to \infty \\ \ee \to 0}} \|v(t)\|_{H^s(\ci)} = 0, \qquad |t| \le T
	\end{split}
\end{equation*}
completing the proof of \eqref{enough_to_prove2}. $\quad \Box$
%
%
%
\subsection{ Proof of \eqref{enough_to_prove3}.} 
Let $v = u_n - u^\ee_n$. Then $v$ solves the Cauchy problem
\begin{align}
		\label{a4qu}
		\p_t v 
		& =  -\gamma (v \p_x v + v \p_x u^\ee_n + u^\ee_n \p_x v)  
		\\
		& - D^{-2} \p_x \left\{ \left (\frac{3-\gamma}{2} \right )(v^2 +
		2u^\ee_n v) + \frac{\gamma}{2}\left[ (\p_x v)^2 + 2 \p_x u^\ee_n \p_x v \right]
		\right\}, \notag
		\\
		& v(0) = (I- J_\ee)u_{0,n}.
		\label{a5qu}
	\end{align}
Applying the operator $D^s$ to both sides of \eqref{a4qu}, multiplying by
	$D^s$ and integrating, we have
	\begin{equation}
		\begin{split}
			\frac{1}{2}\frac{d}{dt} \|v\|_{H^s(\ci)} = A + B
			\label{a6qu}
		\end{split}
	\end{equation}
	where
	\begin{equation}
		\begin{split}
			A
			& =  -\gamma \int_{\ci} D^s(v \p_x v) \cdot D^s v \
			dx
			- \frac{3- \gamma}{2} \int_\ci D^{s-2} \p_x (v^2) \cdot D^s v
			\ dx
			\\
			& - \frac{\gamma}{2}\int_\ci D^{s-2} \p_x (\p_x v)^2 \cdot D^s
			v \ dx
			\label{a7qu}
		\end{split}
	\end{equation}
	and
	\begin{equation}
		\begin{split}
			B 
			 = &  \overbrace{-\gamma \int_\ci D^s (u^\ee_n \p_x v) \cdot D^s v \
			 dx}^{(i)}
			 \ \overbrace{-\gamma \int_\ci D^s (v \p_x u^\ee_n ) \cdot D^s v \
			 dx}^{(ii)}
			 \\
			  & \overbrace{- \ ( 3- \gamma) \int_\ci D^{s-2} \p_x (u^\ee_n v) \cdot D^s
			 v \ dx}^{(iii)}
			 \\
			 & \overbrace{-\gamma \int_\ci D^{s-2} \p_x
			(\p_x u^\ee_n \cdot \p_x v) \cdot D^s v \
			dx}^{(iv)}.
			\label{a8qu}
		\end{split}
	\end{equation}
	Estimating as in \eqref{8'u}-\eqref{14u}, we obtain
	\begin{equation}
		\begin{split}
			\frac{1}{2}\frac{d}{dt}\|v\|_{H^{s}(\ci)}^2
			& \le c_s(\|v\|_{H^s(\ci)}^3 + \|u^\ee_n\|_{H^s(\ci)}
			\|v\|_{H^s(\ci)}^2
			\\
			& + \|u^\ee_n\|_{H^{s+1}(\ci)}
			\|v\|_{H^{s-1}(\ci)} \|v\|_{H^s(\ci)}).
			\label{a15qu}
		\end{split}
	\end{equation}
	Note that the first two terms in parentheses on the right hand side
	of \eqref{a15qu} will offer us little trouble;
	it is the third term that requires special care (due to the
	$\|u^\ee_n\|_{H^{s+1}(\ci)}$ factor, which becomes increasingly large as
	$\ee$ decreases). More precisely:
	%
	%
	%
	\begin{remark}
	\label{lem5r'}
	For $r \ge s > 3/2$ and $0 < \ee <<1$ 
	\begin{equation}
		\begin{split}
			\|u^\ee_n (t, \cdot) \|_{H^r(\ci)} \le C \, \ee^{s-r}
			\label{700r'}
		\end{split}
	\end{equation}
	for all $n \in \mathbb{N}$, with $C = C(r, R)$, where $R$ is defined as
	in \eqref{r-def}.
\end{remark}
\subsection{ Proof.} By part (iii) of Theorem
\ref{thm:HR_existence_continous_dependence}, proved in Section
\ref{existence}, we have
\begin{equation}
	\begin{split}
		\|u^\ee_n \|_{H^r(\ci)}^2
		& \le C' \|u^\ee_n (0)\|_{H^r(\ci)}^2
		\\
		& = C' \|J_\ee u_{0,n}\|_{H^r(\ci)}^2
		\\
		& = C' \sum_{\xi \in \zz} |\widehat{j_\ee} (\xi) \widehat{u_{0,n}}(\xi)
		|^2 \cdot (1 + \xi^2)^r
		\\
		& = C' \sum_{\xi \in \zz} |\widehat{j }(\ee \xi)|^2 \cdot (1 +
		\xi^2)^{r-s} \cdot |\widehat{u_{0,n}}(\xi)|^2 \cdot (1 + \xi^2)^s
		\\
		& = C'|\widehat{u_{0,n}}(0)|^2 +
		C' \sum_{\xi \in \zz \setminus {0}} |\widehat{j }(\ee \xi)|^2 \cdot (1 +
		\xi^2)^{r-s} \cdot |\widehat{u_{0,n}}(\xi)|^2 \cdot (1 + \xi^2)^s.
		\label{1qr'}
	\end{split}
\end{equation}
Assume $r \ge s$. Since $\widehat{j }(\xi) \in \mathcal{S}(\rr)$, 
\begin{equation}
	\label{schwartz'}
	\begin{split}
		|\widehat{j }(\ee \xi)| \le c_r |\ee \xi |^{s-r}, \quad \xi \neq 0.
	\end{split}
\end{equation}
Applying \eqref{schwartz'} to \eqref{1qr'}, we obtain
\begin{equation}
	\label{calc_ue'}
	\begin{split}
		\|u^\ee_n \|_{H^r(\ci)}^2 
		& \le C' |\widehat{u_{0,n}}(0) |^2 + c_r \sum_{\xi \in \zz \setminus
		{0}} |\ee \xi |^{2(s-r)} \cdot (1 + \xi^2)^{r-s}
		|\widehat{u_{0,n}}(\xi) |^2 \cdot (1 + \xi^2)^s
		\\
		& \le C' |\widehat{u_{0,n}}(0) |^2 + 2^{r-s} c_r \ee^{2(s-r)}
		\sum_{\xi \in \zz \setminus {0}} |\widehat{u_{0,n}}(\xi)|^2 \cdot (1 +
		\xi^2)^s
		\\
		& \le C' \|u_{0,n}\|_{H^s(\ci)}^2 + 2^{r-s} c_r \ee^{2(s-r)}
		\|u_{0,n}\|_{H^s(\ci)}^2
		\\
		& = (C' + 2^{r-s} c_r \ee^{2(s-r)}) \cdot \|u_{0,n}\|^2_{H^s(\ci)}.
	\end{split}
\end{equation}
Assuming $0 < \ee <<1$, and noting that
\begin{equation*}
	\begin{split}
		\|u_{0,n}\|_{H^s(\ci)} \le R, \quad \forall n \in \mathbb{N}
	\end{split}
\end{equation*}
where $R$ is defined as in \eqref{r-def},
we conclude from \eqref{calc_ue'} that 
\begin{equation*}
	\begin{split}
		\|u^\ee_n\|_{H^s(\ci)} \le C \ee^{s-r}
	\end{split}
\end{equation*}
where $C = C(r, R)$. $\qquad \Box$
%
In light of Remark \ref{lem5r'}, we now aim to control the growth of
$\|u^\ee_n\|_{H^{s+1}(\ci)}$ by $\|v\|_{H^{s-1}(\ci)}$. As before, we will
first obtain an estimate for $\|v\|_{H^\sigma(\ci)}$ for suitably chosen
$\sigma < s-1$. Then, we will use this estimate to interpolate between
$\|v\|_{H^\sigma(\ci)}$ and $\|v\|_{H^s(\ci)}$, yielding an estimate for
$\|v\|_{H^{s-1}(\ci)}$ which will allow us to control the growth of
$\|u^\ee_n\|_{H^{s+1}(\ci)}$. 
%
%
%
%
\begin{proposition}
	\label{prop:180}
If $\sigma$ is chosen appropriately in the range $1/2 < \sigma < 1$ and
$\sigma + 1 < s$, then for all $n \in \mathbb{N}$ 
	\begin{equation}
	\label{alem6rq}
	\begin{split}
		\|v\|_{H^{\sigma}(\ci)} = 
		\|u_n - u^\ee_n\|_{H^\sigma(\ci)}
		\le C \cdot o(\ee^{s- \sigma }), \qquad |t| \le T
	\end{split}
\end{equation}
with $C = C(R)$, where $R$ is defined as in \eqref{r-def}.
\end{proposition}
%
%
%
\subsection{ Proof.}
Recall that $v$ solves the Cauchy-problem \eqref{a4qu}-\eqref{a5qu}.
Applying $D^\sigma$ to both sides of \eqref{a4qu}, multiplying by
$D^\sigma v$, and integrating, we obtain the
relation
\begin{equation*}
	\begin{split}
		\frac{1}{2}\frac{d}{dt}\|v(t)\|_{H^\sigma(\ci)}^2
		= & - \frac{\gamma}{2}\int_{\ci} D^\sigma
		\p_x \left[ \left( u_n + u^\ee_n \right)v
		\right]\cdot D^\sigma v \ dx
		\\
		& - \frac{3-\gamma}{2}\int_{\ci} D^{\sigma
		-2} \p_x \left[ \left( u_n + u^\ee_n
		\right)v \right] \cdot D^\sigma v \ dx
		\\
		& - \frac{\gamma}{2}\int_{\ci} D^{\sigma
		-2}
		\p_x \left[ \left( \p_x u_n + \p_x u^\ee_n
		\right)\cdot \p_x v \right] \cdot
		D^\sigma v \ dx.
	\end{split}
\end{equation*}
Repeating calculations \eqref{X}-\eqref{12}, with $E$ set to zero,
$u^{\omega,n}$ replaced by $u$, $u_{\omega,n}$ replaced by $u^\ee$, and
$\sigma$ and $\rho$ chosen such that
%
\begin{equation}
	\begin{split}
	& 1/2 < \sigma < 1,
	\\
	& \sigma + 1 \le \rho \le s 
	\end{split}
\end{equation}
yields
 \begin{equation*}
	\begin{split}
		\frac{1}{2}\frac{d}{dt} \|v\|_{H^\sigma(\ci)}^2
		& \le
		C'' (\|u_n + u^\ee_n \|_{H^{\rho}(\ci)} +
		\|\p_x(u_n + u^\ee_n) \|_{H^\sigma(\ci)})
		\cdot \|v\|_{H^\sigma(\ci)}^2.
	\end{split}
\end{equation*}
\medskip
It follows that 
\begin{equation}
	\begin{split}
		\frac{1}{2}\frac{d}{dt} \|v\|_{H^{\sigma}(\ci)}^2
		& \le
		C'' \cdot \|u_n
		+ u^\ee_n\|_{H^{s}(\ci)}\cdot \|v\|_{H^{\sigma}(\ci)}^2.
		\label{a10qx}
	\end{split}
\end{equation}
Applying the triangle inequality and
part (iii) of Theorem \ref{thm:HR_existence_continous_dependence} (proved in
Section \ref{existence})
to \eqref{a10qx} yields
%
\begin{equation}
	\begin{split}
		\label{a11qx}
		\frac{1}{2}\frac{d}{dt} \|v\|_{H^{\sigma}(\ci)}^2
		& \le
		C' (\|u_n(0)\|_{H^{s}(\ci)}
		+ \|u^\ee_n(0)\|_{H^{s}(\ci)})\cdot \|v\|_{H^{\sigma}(\ci)}^2
		\\
		& = C' (\|u_{0,n}\|_{H^{s}(\ci)}
		+ \|J_\ee u_{0,n}\|_{H^{s}(\ci)})\cdot \|v\|_{H^{\sigma}(\ci)}^2.
	\end{split}
\end{equation}
Note that the family $\left\{ u_{0,n} \right\}_n$ is uniformly bounded in
$H^s(\ci)$. Hence, applying Proposition \ref{lem3r} to \eqref{a11qx} we obtain the critical estimate 
\begin{equation}
	\begin{split}
		\label{a12qx}
		\frac{1}{2}\frac{d}{dt} \|v\|_{H^{\sigma}(\ci)}^2
		& \le
	C \|v\|_{H^{\sigma}(\ci)}^2
\end{split}
\end{equation}
with $C = C(R)$, where $R$ is defined as in \eqref{r-def}. Note that $C$
does not depend on $n$ or $\ee$. Differentiating
the left hand side of \eqref{a12qx} and simplifying, we obtain
\begin{equation}
	\begin{split}
		\frac{d}{dt}\|v\|_{H^{\sigma}(\ci)} \le C \|v\|_{H^{\sigma}(\ci)}.
		\label{a100qx}
	\end{split}
\end{equation}
Let $y(t) = \|v\|_{H^{\sigma}(\ci)}$. Then \eqref{a100qx} gives
\begin{equation*}
	\begin{split}
		\frac{1}{y(t)}\frac{dy}{dt} \le C.
	\end{split}
\end{equation*}
Hence,
\begin{equation*}
	\begin{split}
		\int_0^t \frac{1}{y(\tau)} \frac{dy}{d \tau}
		\le \int_0^t C \ d \tau, \qquad |t| \le T
	\end{split}
\end{equation*}
from which we obtain
\begin{equation}
	\begin{split}
		\ln |y(t) | - \ln |y(0)| \le C t.
		\label{a101qx}
	\end{split}
\end{equation}
Simplifying \eqref{a101qx}, we have
\begin{equation*}
	\begin{split}
		\ln \left |\frac{y(t)}{y(0)} \right | \le C t
	\end{split}
\end{equation*}
which yields the estimate
\begin{equation*}
	\begin{split}
		y(t) \le y(0) e^{C t}, \qquad |t| \le T.
	\end{split}
\end{equation*}
Substituting back in $\|v\|_{H^{\sigma}(\ci)}$ for $y$, we get
\begin{equation}
	\label{vsig-est}
	\begin{split}
		\|v\|_{H^{\sigma}(\ci)}
		& \le e^{C t}\|v(0)\|_{H^{\sigma}(\ci)}
		\\
		& = e^{C t}\|u_n(0) - u^\ee_n(0) \|_{H^{\sigma}(\ci)}
		\\
		& = e^{C t}\|u_{0,n} - J_\ee u_{0,n}\|_{H^{\sigma}(\ci)}.
	\end{split}
\end{equation}
Applying Proposition \ref{lem4r} to \eqref{vsig-est}, we obtain 
\begin{equation}
	\label{almost}
	\begin{split}
		\|v\|_{H^\sigma (\ci)} \le e^{Ct} \|u_{0,n}\|_{H^\sigma(\ci)} \cdot
		o(\ee^{s-\sigma})
	\end{split}
\end{equation}
and since $\|u_{0,n}\|_{H^s(\ci)} \le R$ for all $n \in \mathbb{N}$, where
$R$ is defined as in \eqref{r-def}, we conclude from estimate \eqref{almost} that
\begin{equation*}
	\begin{split}
		\|v\|_{H^\sigma(\ci)} \le C(R) \cdot o(\ee^{s-\sigma})
	\end{split}
\end{equation*}
completing the proof. $\quad \Box$
We are now prepared to interpolate. Recall \eqref{a15qu}. Applying Remark
\ref{lem2r}, Remark \ref{lem5r'}, and
Proposition \ref{prop:180} gives
\begin{equation*}
	\begin{split}
		& \|u^\ee_n \|_{H^{s+1}(\ci)} \|v\|_{H^{s-1}(\ci)} \|v\|_{H^s
		(\ci)}
		\\
		&\le C'' \ee^{-1} \cdot \|v\|_{H^\sigma(\ci)}^{1/(s-\sigma)}
		\|v\|_{H^s(\ci)}^{2 - 1/(s- \sigma)}
		\\
		& \le C'' \ee^{-1} \cdot \Big [C' \cdot o(\ee^{s- \sigma})\Big ]^{1/(s-\sigma)}
		\cdot \|v\|_{H^s(\ci)}^{2- 1/(s-\sigma)}
	\end{split}
\end{equation*}
from which we obtain
\begin{equation}
	\begin{split}
		\label{a200qx}
		\|u^\ee_n\|_{H^{s+1}(\ci)} \|v\|_{H^{s-1}(\ci)} \|v \|_{H^s(\ci)}
		& \le  C \cdot o(1) \cdot \|v\|_{H^s(\ci)}^{2- 1/(s-\sigma)}.
	\end{split}
\end{equation}
where $C=C(R)$ does not depend on $\ee$ or $n$. We wish to control the growth of the right hand side of \eqref{a200qx}.
First, note that the triangle inequality, part (iii) of Theorem
\ref{thm:HR_existence_continous_dependence}, and Proposition \ref{lem3r} imply
\begin{equation}
	\begin{split}
		\|v\|_{H^s(\ci)} & = \|u_n - u^\ee_n \|_{H^s(\ci)} 
		\\
		& \le \|u_n \|_{H^s(\ci)} + \|u^\ee_n\|_{H^s(\ci)}
		\\
		& \le 2\left[ \|u_{0,n} \|_{H^s(\ci)} + \|J_\ee u_{0,n}
		\|_{H^s(\ci)} \right]
		\\
		& \le 4 \|u_{0,n} \|_{H^s(\ci)}, \qquad |t| \le T
		\label{agrowth_v}
	\end{split}
\end{equation}
and since $\{u_{0,n}\}_n$ belongs to a bounded subset of
$H^s(\ci)$, we see from \eqref{agrowth_v} that $\|v \|_{H^s(\ci)}$ is
uniformly bounded in $n$ \emph{and} $\ee$.  Secondly, by Remark \ref{lem5r'}, we have 
\begin{equation}
	\label{au-ee-bound}
	\|u^\ee_n \|_{H^s(\ci)} \le C(R), \ \ \text{for all} \ \ 0 < \ee <<1, \ n \in
	\mathbb{N}.
\end{equation}
Applying \eqref{a200qx}, \eqref{agrowth_v}, and \eqref{au-ee-bound}
to \eqref{a15qu}, it follows that 
\begin{equation*}
	\label{lim-est-in}
		\begin{split}
			\lim_{n \to \infty }
			\frac{1}{2}\frac{d}{dt}\|v\|_{H^{s}(\ci)}^2
			& \le
			C \lim_{n \to \infty} \Big [\|v\|_{H^s(\ci)}^3 +
			\|v\|_{H^s(\ci)}^2 + o(1)\Big ]
		\end{split}
	\end{equation*}
	for every $0 < \ee <<1$, where $C = C(\|u_0\|_{H^s(\ci)}, \ R)$.
	Therefore
	\begin{equation}
		\begin{split}
			\lim_{\substack{n \to \infty \\ \ee \to 0} }
			\frac{1}{2}\frac{d}{dt}\|v\|_{H^{s}(\ci)}^2
			& \le C
			\lim_{\substack{n \to \infty \\ \ee \to 0}}
			\Big [\|v\|_{H^s(\ci)}^3 + 
			\|v\|_{H^s(\ci)}^2 \Big ].
			\label{a15qx}
		\end{split}
	\end{equation}
	We differentiate the left hand side of \eqref{a15qx} and obtain
\begin{equation*}
	\begin{split}
		\lim_{\substack{n \to \infty \\ \ee \to 0}}\frac{d}{dt}
		\|v\|_{H^s(\ci)} \le C
		\lim_{\substack{n \to \infty \\ \ee \to 0}} \left [\|v\|_{H^s(\ci)}^2 +
		\|v\|_{H^s(\ci)} \right ].
	\end{split}
\end{equation*}
Letting $y = \|v\|_{H^s(\ci)}$ and rearranging gives
\begin{equation*}
	\begin{split}
		\lim_{\substack{n \to \infty \\ \ee \to 0} } \ \frac{1}{y(y+1)} \frac{dy}{dt}
		\le	C
	\end{split}
\end{equation*}
which can be rewritten as
\begin{equation*}
	\begin{split}
		\lim_{\substack{n \to \infty \\ \ee \to 0} }
		\left( \frac{1}{y} - \frac{1}{y+1} \right)\frac{dy}{dt} \le C 
	\end{split}
\end{equation*}
implying
\begin{equation}
	\label{aest-int}
	\begin{split}
		\lim_{\substack{n \to \infty \\ \ee \to 0} } \left [
\int_0^t \frac{1}{y} \frac{dy}{d \tau} \ d \tau
		- \int_0^t \frac{1}{y+1} \frac{dy}{d \tau} \ d \tau \right ]
		\le \int_0^t C \ d \tau, \quad |t| \le T.
	\end{split}
\end{equation}
Hence \eqref{aest-int} gives 
\begin{equation}
	\begin{split}
	\lim_{\substack{n \to \infty \\ \ee \to 0} }	\left [ \ln \left | \frac{y(t)}{y(0)}
	\cdot \frac{y(0) + 1}{y(t) + 1} \right | \right ] \le C t.
		\label{20b}
	\end{split}
\end{equation}
Exponentiating both sides of \eqref{20b}, and noting that $f(x) = e^x$
is a continuous function on $\rr$, we must have
\begin{equation*}
	\begin{split}
		\lim_{\substack{n \to \infty \\ \ee \to 0} }	
		\left |
		\frac{y(t)}{y(0)} \cdot \frac{y(0) + 1}{y(t) + 1} \right | \le e^{C t}.
	\end{split}
\end{equation*}
Recalling that $y(t) = \|v(t)\|_{H^s(\ci)} \ge 0$, we obtain
\begin{equation*}
	\begin{split}
		\lim_{\substack{n \to \infty \\ \ee \to 0} }	
		\frac{y(t)}{y(0)} \cdot \frac{y(0) + 1}{y(t) + 1} \le e^{C t}.
	\end{split}
\end{equation*}
Rearranging, it follows that 
\begin{equation*}
	\begin{split}
		\lim_{\substack{n \to \infty \\ \ee \to 0}} \frac{y(t)}{y(t) + 1}
		\le \lim_{\substack{n \to \infty \\ \ee \to 0}} \frac{e^{C t} \cdot y(0)}{y(0) + 1} \le
		\lim_{\substack{n \to \infty \\ \ee \to 0}} e^{C t} \cdot y(0).
	\end{split}
\end{equation*}
Substituting back in $\|v(t)\|_{H^s(\ci)}$ for $y(t)$ gives
\begin{equation}
	\begin{split}
		\lim_{\substack{n \to \infty \\ \ee \to 0}}	\frac{\|v(t)\|_{H^s(\ci)}}{\|v(t)\|_{H^s(\ci)} + 1}  \le
		\lim_{\substack{n \to \infty \\ \ee \to 0}} e^{C t} \cdot \|v(0)\|_{H^s(\ci)}.
		\label{a303qx}
	\end{split}
\end{equation}
Since by Proposition \ref{lem4r} 
\begin{equation*}
	\begin{split}
	\lim_{\substack{n \to \infty \\ \ee \to 0} }
	\|v(0)\|_{H^s(\ci)}
	& = \lim_{\substack{n \to \infty \\ \ee \to 0} }
	\|u_{0,n} - J_\ee u_{0,n} \|_{H^s(\ci)} 
	\\
	& \le  \lim_{\substack{n \to \infty \\ \ee \to 0}}
	\left [ \|u_{0,n}\|_{H^s(\ci)} \cdot o(1) \right ]
	\\
	& = \lim_{\ee \to 0} \left [ \|u_0\|_{H^s(\ci)} \cdot o(1) \right ]
	\\
	& = 0
	\end{split}
\end{equation*}
we deduce from \eqref{a303qx} that
\begin{equation*}
	\begin{split}
		\lim_{\substack{n \to \infty \\ \ee \to 0}} \|v(t)\|_{H^s(\ci)} = 0, \qquad |t| \le T
	\end{split}
\end{equation*}
completing the proof of \eqref{enough_to_prove3}. $\quad \Box$
%
%
%
\section{Extending Well-Posedness for HR to the Non-Periodic Case}
\label{sec:defs}
The method will be analogous to that of the periodic case, with two major
modifications. First, we must choose a different mollifier $J_\ee$ in the
proof of continuous dependence. Pick a
function $j(x) \in \mathcal{S}(\rr)$ such that
\begin{equation*}
		\begin{split}
			& 0 \le \widehat{j}(\xi) \le 1,
			\\
			& \widehat{j}(\xi) = 1 \ \ \text{if} \ \ |\xi| \le 1.
		\end{split}
	\end{equation*}
Letting
\begin{equation*}
	\begin{split}
		j_\ee(x) = \frac{1}{\ee} j \left (\frac{x}{\ee} \right )
	\end{split}
\end{equation*}
it can be verified that 
		\begin{equation*}
		\begin{split}
			\widehat{j_\ee}(\xi) = \widehat{j }(\ee \xi), \quad \ee > 0.
		\end{split}
	\end{equation*}
We then define $J_\ee$ to be the ``Friedrichs mollifier''
	\begin{equation*}
		\begin{split}
			J_\ee f(x) = j_\ee * f(x), \quad \ee>0.
		\end{split}
	\end{equation*}
Given this construction, the proofs of Remark \ref{lem5r}, Remark
\ref{lem5r'}, and Proposition \ref{lem4r} for the non-periodic case will be
analogous to those in the periodic case.
Secondly, in the proof of existence, we will have difficulties in arranging
that the solutions $\{u_\ee\}$ to the mollified HR i.v.p. converge in $C(I,
H^{s- \sigma}(\rr))$, $0 < \sigma < 1$ to a candidate solution $u$ of the HR
i.v.p. We will get around this by considering the family $\left\{ \varphi
u_\ee \right\}$ instead.
%
%
%
%
We divide our work into three parts:
\subsection{Existence.}
Mirroring the argument in the periodic case, we see that the bounded
family $\{u_\ee\}$ is compact in the weak* topology of $L^\infty(I,
H^{s}(\rr))$. More precisely, there is a sequence  $\{ u_{\ee_n} \}$
converging weak* to a $ u\in L^{\infty}(I, H^s(\rr))$; that is 
		%
		\begin{equation*}
			\label{hhweak-conv}
			\lim_{n\to \infty} T_{u_{\ee_n}}(\varphi)  =  T_u (\varphi) 
			\; \;		
			\text{ for all } \;\;  \varphi \in L^1(I, H^{s}(\rr))
		\end{equation*}
		where
		\begin{equation}
			T_v(\varphi) = \int_I <v (t), \varphi (t)>_{H^s(\rr)} dt  = \int_I
			 \int_\rr
			 \widehat{v}(\xi, t) \bar{\widehat{\varphi}} (\xi, t) \cdot (1 +
			 \xi^2)^s \ d \xi \; dt.
		\end{equation}
		%
		Similarly, $\left\{ \p_x u_{\ee_n} \right\}$ is compact in the
		weak* topology of $L^\infty(I, H^{s-1}(\rr))$ and converges weak*
		to $\p_x u$. Hence, for any $k \in \mathbb{N}$, we have
		\begin{align}
			\label{base-weak}
				& (u_{\ee_n})^k \xrightarrow{\text{weak*}} u^k \ \
				\text{on} \ \
				L^\infty(I, H^s(\rr)),
				\\
				\label{base-weak-2}
				& (\p_x u_{\ee_n})^k \xrightarrow{\text{weak*}} (\p_x u)^k
				\ \ \text{on} \ \
				L^\infty(I, H^{s-1}(\rr)). 
		\end{align}
		In order to show that $u$ solves the HR i.v.p., it would
		suffice to obtain a stronger convergence for  $u_{\ee_n}$ so that 
		we could take the limit in the mollified HR equation. However,
		this is difficult, and unnecessary. Rather, our approach will be to
		show that for any pseudo-differential operator
		$P \in \Psi^0$ and arbitrary $\vp \in S(\rr)$, $k \in
		\mathbb{N}$, $0< \sigma < 1$, we have
		%
		%
			\begin{align}
			\label{hhstrong-conv}
			& \varphi P [(u_{\ee_n})^k] \longrightarrow \varphi P [u^k]  
			\quad
			\text{ in } \,\,   C(I, H^{s-\sigma}(\rr)), \ \,
			\\
			\label{hhstrong-conv-next}
			& \varphi P [(\p_x u_{\ee_n})^k] \longrightarrow \varphi P
			[(\p_x u)^k]  
			\quad
			\text{ in } \,\,   C(I, H^{s-\sigma -1}(\rr)), \ \ 
		\end{align}
		%
		which will then be applied to a rewritten version of the HR
		i.v.p. Our focus will be on proving \eqref{hhstrong-conv}; since the proof of
		\eqref{hhstrong-conv-next} is similar, we will omit the
		details. First, we will need the following
		interpolation result:
		%%%%%%%%%%%%%%%%%%%%%%%%%%%
		%
		%
		%                 Interpolation Lemma
		%
		%
		%%%%%%%%%%%%%%%%%%%%%%%%%%%
		\begin{lemma}
			\label{hhinterpolation-lem}
			(Interpolation)     Let  $s > \frac{3}{2}$.
			If $v \in C(I, H^s(\rr)) \cap C^1(I, H^{s-1}(\rr))$
			then $v \in C^\sigma (I, H^{s- \sigma}(\rr))$ for  $0 < \sigma < 1$.
		\end{lemma}
		%
		\subsection{Proof} It is analogous to the proof in the periodic case.
		$\quad \Box$
		Fix $k \in \mathbb{N}$. Using Lemma \ref{hhinterpolation-lem}, we
		will show that the family
		\begin{equation*}
			\begin{split}
			 \{\varphi P[(u_\ee)^k]\}_\ee
		\end{split}
	\end{equation*}
		is equicontinuous in $C(I, H^{s-\sigma}(\rr))$ 
		for $0 < \sigma < 1$ and $\varphi = \varphi(x) \in \mathcal{S}(\rr)$.
		We will follow this by proving that
		there exists a sub-family $\{\varphi P[(u_{\ee_n}(t))^k]\}_n$
		that is precompact in $H^{s-\sigma}(\rr)$ for $\sigma > 0$. 
		These two facts, in conjunction with Ascoli's Theorem, will
		yield
		\begin{equation*}
			\label{hhstrong-conv2}
			\varphi P[(u_\ee)^k] \to \tilde{u}
			\; \; \text{in} \; \; C(I,H^{s-\sigma}(\rr))
		\end{equation*}
		for $0 < \sigma < 1$.
		We will then show that $\tilde{u} = \varphi P[u^k]$, from which it will
		follow that
		\begin{equation*}
			\label{hhphiplus}
			\begin{split}
				\varphi P[(u_\ee)^k] \to \varphi P[u^k]
				\; \; \text{in} \; \; C(I,H^{s-\sigma}(\rr)).
			\end{split}
		\end{equation*}
		%%%%%%%%%%%%%%%%%%%%%%
		%
		%
		%       Equicontinuity
		%
		%
		%%%%%%%%%%%%%%%%%%%%%%
		%
		\subsection{  Equicontinuity of $\{ \varphi P [(u_\ee)^k]\}_\ee$  in $C(I,
		H^{s-\sigma}(\rr)$}).
		%
		%
		Since $\varphi \in \mathcal{S}(\rr)$, the map $u \mapsto \vp u$
		is a bounded linear function on $H^s(\rr)$, for arbitrary $s \in
		\rr$, where  
		\begin{equation}
			\begin{split}
				\|\varphi u\|_{H^s(\rr)} \le C(s, \varphi)
				\|u\|_{H^s(\rr)}, \quad \forall s\in \rr.
				\label{hhschwartz-estimate}
			\end{split}
		\end{equation}
		Furthermore, $$P: H^s(\rr) \to H^s(\rr)$$ is bounded and linear,
		with 
		\begin{equation}
			\label{operator-normaa}
			\|P\|_{L(H^s(\rr), H^s(\rr))} \le 1.
		\end{equation}
		Hence, the map 
		\begin{equation}
			\label{the-map}
			\begin{split}
			& T: H^s(\rr) \to H^s(\rr),
			\\
			& T(u) = \vp P u 
		\end{split}
	\end{equation}
	is bounded and linear, with 
	\begin{equation}
		\begin{split}
			\|T\|_{L(H^s(\rr), H^s(\rr))} \le C(s, \vp).
			\label{op-norm-product}
		\end{split}
	\end{equation}
	Therefore, applying Lemma
		\ref{hhinterpolation-lem} gives 
		%
		\begin{equation*}
			\begin{split}
			\label{hhequic-1}
			& \sup_{t \neq t'} \frac {\| \varphi P [(u_\ee(t))^k] - \varphi
			P [(u_\ee(t'))^k] \|_{H^{s -
			\sigma  }(\rr)}}{|t - t'|}
			\\
			& \le \sup_{t \neq t'}  \frac { \|\vp P \|_{L(H^{s-\sigma}(\rr),
			H^{s-\sigma}(\rr))} \cdot \|   [u_\ee(t)]^k  - 
			[u_\ee(t')]^k \|_{H^{s -
			\sigma }(\rr)}}{|t - t'|}
			\\
			& \le C(s, \vp) \cdot \sup_{t \neq t'}  \frac { \|   [u_\ee(t)]^k  - 
			[u_\ee(t')]^k \|_{H^{s -
			\sigma }(\rr)}}{|t - t'|}
			\\
			&< c
		\end{split}
		\end{equation*}
		%
		or
		%
		\begin{equation*}
			\label{hhequic-2}
			\|\varphi P [(u_\ee(t))^k] - \varphi
			P [(u_\ee(t'))^k \|_{H^{s - \sigma }(\rr)}< c|t -
			t'|, 
			\text{ for all }  \,\,  t, t'\in I,
		\end{equation*}
		%
		which shows that  the family  $\{\varphi P [(u_\ee)^k]\}_\ee$ is
		equicontinuous in $C(I, H^{s-\sigma }(\rr))$.  $\quad \Box$
		%
		%
		%%%%%%%%%%%%%%%%%%%%%%
		%
		%
		%      PreCompactness
		%
		%
		%%%%%%%%%%%%%%%%%%%%%%%%%%
		%
		%
		%
		%
		%		
		\subsection{Precompactness of $\{\varphi P [(u_\ee(t))^k]\}_\ee$ in
		$H^{s-\sigma  }(\rr)$}
		Applying the algebra property of Sobolev
		Spaces, and recalling \eqref{the-map}-\eqref{op-norm-product}, we have
		\begin{equation}
			\begin{split}
			\label{hhcompact-1}
			 \|\varphi P [(u_\ee(t))^k]\|_{H^{s}(\rr)}
			& \le  C(s, \vp) \cdot \|[u_\ee(t)]^k\|_{H^{s}(\rr)}
			\\
			& \le C(s, \vp) \cdot \|u_\ee(t)\|^k_{H^{s}(\rr)}.
			\end{split}
		\end{equation}
		%
		Letting $|t| \le T$, we now apply Lemma \ref{hr_wp} to
		\eqref{hhcompact-1} to obtain
		\begin{equation*}
			\begin{split}
			\|\varphi P [(u_\ee(t))^k]\|_{H^{s}(\rr)}
			\le 2^k C(s, \vp) \cdot  \|u_0 \|^k_{H^s(\rr)} < \infty.
			\end{split}
		\end{equation*}
		Therefore, by Reillich's Theorem, the family $\left\{
		\varphi P [(u_\ee(t))^k] \right\}_\ee$ is
		precompact in $H^{s- \sigma }(\rr)$ for all $\sigma > 0$ and $|t| \le T$. $\quad
		\Box$ 
		Hence, compiling our previous results on equicontinuity and precompactness
		and applying Ascoli's Theorem, we
		conclude that we can find $\tilde{u}$ and a subfamily 
		\\ $\left\{
		\varphi P [(u_{\ee_n})^k]
		\right\}_n$ such that
		\begin{equation}
			\label{hhstrong-conv-of-u_ep}
			\varphi P [(u_{\ee_n})^k] \to \tilde{u}
			\; \; \text{in} \; \; C(I, H^{s-\sigma}(\rr)).
		\end{equation}
		%
		%
		We would now like to find out what $\tilde{u}$ is:
		%
		%
		%
		\begin{lemma}
			\label{hhlem:crit-conv}
			For arbitrary $k \in \mathbb{N}$,
			\begin{equation}
				\begin{split}
					\varphi P [(u_{\ee_n})^k] \xrightarrow{weak^*}
					\varphi P [u^k] \ \ \text{on} \ \ L^\infty(I,
					H^{s-\sigma}(\rr)).
					\label{hhcrit-conv-est}
				\end{split}
			\end{equation}
		\end{lemma}
		\subsection{ Proof.} 
		Fix $k \in \mathbb{N}$ and recall that the operators 
		\begin{equation*}
			\begin{split}
			 & T_\varphi: H^s(\rr) \to H^s(\rr)\\
			 & T_\varphi u = \varphi u
		\end{split}
	\end{equation*}
and 
\begin{equation*}
	\begin{split}
		P:H^s(\rr) \to H^s(\rr)
	\end{split}
\end{equation*}
	are continuous; therefore 
	\begin{equation*}
		\begin{split}
			T_\vp P: H^s(\rr) \to H^s(\rr)
		\end{split}
	\end{equation*}
	continuously. Hence, its adjoint  $(T_\varphi P)^*$
	exists and
		\begin{equation*}
			(T_\varphi P)^*: H^s(\rr) \to H^s(\rr) 
		\end{equation*}
		continuously. Therefore, applying \eqref{base-weak}, we conclude that
		\begin{equation}
			\label{widpseudo}
			\begin{split}
				& \int_I <\varphi P[u^k] - \varphi
				P [(u_{\ee_n})^k],\  f>_{H^{s-\sigma }(\rr)} dt
				\\
				&= \int_I <u^k - 
				 (u_{\ee_n})^k, \ (T_\vp P)^* f>_{H^{s-\sigma }(\rr)} \to 0
			\end{split}
		\end{equation}
		completing the proof. $\quad \Box$
		%
		%
		Now, recalling \eqref{hhstrong-conv-of-u_ep} and applying Lemma
		\ref{hhlem:crit-conv}, we obtain
			\begin{equation}
			\begin{split}
				\vp P [(u_{\ee_n})^k] \to \vp P [u^k] \ \ \text{in}  \ \ C(I,
				H^{s-\sigma}(\rr))
				\label{hhvp_u_ep_conv}
			\end{split}
		\end{equation}
		for arbitrary $k \in \mathbb{N}$.  Using precisely the same
		strategy we used to prove \eqref{hhvp_u_ep_conv} (applied now to
		the family $\{ \vp P [(\p_x u_{\ee})^k] \}_\ee$), one can also show
	\begin{equation}
			\begin{split}
			\vp P [ (\p_x u_{\ee_n})^k] \to \vp P [(\p_x u)^k] \ \ \text{in}  \ \ C(I,
				H^{s-\sigma -1 }(\rr)).
			\end{split}
		\end{equation}
		We summarize our result below:
		\begin{theorem}
		\label{hhthm:crit1}
		Let $P \in \Psi^0$ be a pseudo-differential operator. Then for
		arbitrary $k \in \mathbb{N}$, 
			\begin{equation}
			\begin{split}
				& \vp P [(u_{\ee_n})^k] \to \vp P [u^k] \ \ \text{in}  \ \ C(I,
				H^{s-\sigma }(\rr)),
				\\
				& 
				\vp P [(\p_x u_{\ee_n})^k] \to \vp P [(\p_x u)^k] \ \
				\text{in}  \ \ C(I,
				H^{s-\sigma -1}(\rr)).
				\label{hhdx_vp_u_ep_conv}
			\end{split}
		\end{equation}
	\end{theorem}
		\subsection{ Verifying that the weak* limit $u$ solves the HR equation.} 
		We recall the mollified HR i.v.p
		\begin{align}
			& \p_t u_{\ee_n}  = -\gamma(J_{\ee_n} u_{\ee_n} \cdot \p_x
			J_{\ee_n} u_{\ee_n})
			\label{hh1gr}
			\\
			& u(x,0) = u_0(x).
			\label{hh2gr}
		\end{align}
		Multiplying both sides of \eqref{hh1gr} by $\varphi$ and rewriting,
		we obtain
		\begin{equation}
			\label{hh3}
			\begin{split}
				\p_t(u_{\ee_n} \varphi) = -\gamma \vp (J_{\ee_n} u_{\ee_n} \cdot
				J_{\ee_n} \p_x u_{\ee_n}).
			\end{split}
		\end{equation}
		The following lemma will play a crucial role in our proof of the
		existence of a solution to the HR i.v.p.
		\begin{lemma}
			\label{hhlem:cc}
			For $\vp \in \mathcal{S}(\rr)$ such that
			$\vp^\frac{1}{2} \in \mathcal{S}(\rr)$, we have
			\begin{equation}
				\begin{split}
					\label{hhburgers_and_nonlocal_conv}
				& \vp (J_{\varepsilon_n} u_{\varepsilon_n} 
				\cdot J_{\varepsilon_n}\partial_x u_{\varepsilon_n}) 
				\to \vp u \partial_x u \; \; 
				\text{in} \; \;
				C(I, H^{s-\sigma-1}(\rr)). 
			\end{split}
			\end{equation}
		\end{lemma}
		%
		\subsection{ Proof.} We will need a couple of propositions:
		\begin{proposition}
			For arbitrary $\vp \in \mathcal{S}(\rr)$
			\label{hhprop:1aa}
			\begin{equation}
				\begin{split}
					\vp J_{\ee_n} u_{\ee_n} \to \vp u \ \ \text{in} \ \
					C(I, H^{s-\sigma}(\rr)).
					\label{hh}
				\end{split}
			\end{equation}
		\end{proposition}
			\subsection{ Proof.} Note that
			\begin{equation}
				\begin{split}
					& \|\vp u - \vp J_{\ee_n} u_{\ee_n}
					\|_{C(I, H^{s-\sigma}(\rr))}
					\\
					&= \|\vp u - \vp J_{\ee_n} u_{\ee_n} \pm \vp
					u_{\ee_n} \|_{C(I, H^{s-\sigma}(\rr))}
					\\
					& = \|\vp u - \vp u_{\ee_n}
					\|_{C(I, H^{s-\sigma}(\rr))} + \|\vp (I - J_{\ee_n})
					u_{\ee_n} \|_{C(I, H^{s-\sigma}(\rr))}.
					\label{hh1bb}
				\end{split}
			\end{equation}
			Applying \eqref{hhschwartz-estimate} and the estimates
			\begin{equation*}
				\begin{split}
					& \|I-J_{\ee_n} \|_{L(H^{s-\sigma}(\rr), H^{s -
					\sigma}(\rr))} = o(1),
					\\
					& \|u_{\ee_n}\|_{H^{s-\sigma}(\rr)} \le 2
					\|u_0\|_{H^{s-\sigma}(\rr)}
				\end{split}
			\end{equation*}
			to \eqref{hh1bb} gives
			\begin{equation}
				\label{hh2bb}
				\begin{split}
					\|\vp u - \vp J_{\ee_n} u_{\ee_n}\|_{H^{s-\sigma}(\rr)}
					\le \left( \|\vp u - \vp u_{\ee_n}
					\|_{C(I, H^{s-\sigma}(\rr))} + C(s, \vp) \cdot o(1) \cdot \|u_0
					\|_{H^{s-\sigma}(\rr)} \right).
				\end{split}
			\end{equation}
			Letting $\ee \to 0$ in \eqref{hh2bb} and applying Theorem
			\ref{hhthm:crit1} completes the proof. $\quad \Box$
			%
			%
			\begin{proposition}
				\label{hhprop:dd}
				For arbitrary $ \vp \in \mathcal{S}(\rr)$,
				\begin{equation}
					\begin{split}
						\vp J_{\ee_n} \p_x u_{\ee_n} \to \vp u \ \
						\text{in} \ \ C(I, H^{s-\sigma-1}(\rr)).
						\label{hh0dd}
					\end{split}
				\end{equation}
			\end{proposition}
			\subsection{ Proof.} The result follows from Theorem \ref{hhthm:crit1}.
			The proof is nearly identical to that of
			Proposition \ref{hhprop:1aa}, with $s-1$ substituted for $s$
			and $\p_x u_{\ee_n}$ substituted for $u_{\ee_n}$. $\quad \Box$
			%
			%
			We now have enough tools to prove Lemma \ref{hhlem:cc}. Restrict the
			choice of $\vp$ such that $\vp^\frac{1}{2} \in S(\rr)$
			(Such Schwartz functions exist; as an example, take the square
			of the Gaussian). Using this fact, and applying Proposition
			\ref{hhprop:1aa} and Proposition \ref{hhprop:dd}, we conclude that
			\begin{equation*}
				\begin{split}
					\vp J_{\ee_n} u_{\ee_n} \p_x J_{\ee_n} u_{\ee_n} 
					& = \vp^\frac{1}{2} J_{\ee_n} u_{\ee_n} \cdot
					\vp^\frac{1}{2} \p_x J_{\ee_n} u_{\ee_n}
					\\
					& \to \vp^\frac{1}{2} u \cdot \vp^\frac{1}{2} \p_x u = \vp
					u \p_x u
				\end{split}
			\end{equation*}
			completing the proof of Lemma \ref{hhlem:cc}. $\quad \Box$
%
%
%
%
By Theorem \ref{hhthm:crit1} it follows immediately that
		\begin{equation}
			\begin{split}
				& \vp \p_x(1- \p_x^2)^{-1} \left( \frac{3-\gamma}{2}
				(u_{\ee_n})^2
				 + \frac{\gamma}{2} (\p_x u_{\ee_n})^2 \right )
				 \\
				 & \to
				 \vp \p_x(1- \p_x^2)^{-1} \left( \frac{3-\gamma}{2} u^2
				 + \frac{\gamma}{2} (\p_x u)^2 \right ) \ \
				 \text{in} \ \ C(I, H^{s-\sigma-1}(\rr)).
				\label{non-localii-convergence}
			\end{split}
		\end{equation}
		Combining \eqref{hhburgers_and_nonlocal_conv} and
		\eqref{non-localii-convergence}, and applying the Sobolev Imbedding
		Theorem, we deduce 
		\begin{equation}
			\begin{split}
				& -\gamma \vp (J_{\ee_n} u_{\ee_n} \cdot J_{\ee_n} \p_x
				u_{\ee_n}) -
				\vp \p_x(1- \p_x^2)^{-1} \left( \frac{3-\gamma}{2}
				(u_{\ee_n})^2
				 + \frac{\gamma}{2} (\p_x u_{\ee_n})^2 \right )
				 \\
				 \to & -\gamma \vp u \p_x u -
				 \vp \p_x(1- \p_x^2)^{-1} \left( \frac{3-\gamma}{2} u^2
				 + \frac{\gamma}{2} (\p_x u)^2 \right ) \ \
				 \text{in} \ \ C(I, C(\rr)).
				\label{llloc-non-loc-tog}
			\end{split}
		\end{equation}
		%
		Next, we note that the convergence  
		%
		\begin{equation}
			\label{hhweak-conv-2}
			T_{\vp u_{\ee_n}}(f)  \longrightarrow  T_{\vp u} (f) \;
			\text{ for all } \;  f \in L^1(I, H^{-s}(\rr))
		\end{equation}
		%
		can be restated as 
		%
		\begin{equation}
			\vp u_{\ee_n}  \longrightarrow  \vp u
			\quad
			\text{ in }  \,\,
			\mathcal{D}'(I\times \rr).
		\end{equation}
		%
		This implies 
		%
		\begin{equation}
			\label{hhdistib-conv-2}
			\p_t(\vp u_{\ee_n})  \longrightarrow  \p_t (\vp u)
			\quad
			\text{ in }  \,\, \mathcal{D}'(I\times \rr).
		\end{equation}
		%
		Since for all $n$ we have 
		%
		\begin{equation}
			\begin{split}
			 \p_t (\vp u_{\ee_n})
			 = & -\gamma \vp
			(J_{\varepsilon_n} u_{\varepsilon_n}  \cdot
			J_{\varepsilon_n}\partial_x u_{\varepsilon_n})
			\\
			& -
			\vp \p_x(1- \p_x^2)^{-1} \left( \frac{3-\gamma}{2} (u_{\ee_n})^2
			 + \frac{\gamma}{2} (\p_x u_{\ee_n})^2 \right )
		 \end{split}
		\end{equation}
		%
		it follows from \eqref{hhdistib-conv-2} and the uniqueness of the
		limit in \eqref{llloc-non-loc-tog} that
		\begin{equation}
			\begin{split}
			 \p_t (\vp u)
			 = & -\gamma \vp
			u \p_x u - \vp \p_x(1- \p_x^2)^{-1} \left( \frac{3-\gamma}{2} u^2
			 + \frac{\gamma}{2} (\p_x u)^2 \right )
			\label{hhadone}
			\end{split}
		\end{equation}
		Further restricting $\vp \in \mathcal{S}(\rr)$ to be nonzero in
		$\rr$, we
		can divide both sides of \eqref{hhadone} by $\vp$ to obtain
		\begin{equation}
			\label{hh2yy}
			\begin{split}
			 \p_t  u
			 = & -\gamma
			u \p_x u - \p_x(1- \p_x^2)^{-1} \left( \frac{3-\gamma}{2} u^2
			 + \frac{\gamma}{2} (\p_x u)^2 \right ).
			\end{split}
		\end{equation}
		Thus we have constructed a solution $u \in L^\infty(I, H^s(\rr))$
		to the HR i.v.p. 
\subsection{Uniqueness.} The proof is analogous to that in the periodic case.
\subsection{Continuous Dependence.} The proof is analogous to the proof in
the periodic case, with one important caveat. Recall the introduction to Section
\ref{sec:defs} in the appendix; specifically, how we defined the operator
$J_\ee$. By construction, the proofs of Remark \ref{lem5r}, Remark
\ref{lem5r'}, and Proposition \ref{lem4r} for the non-periodic case will be
analogous to those in the periodic case. Hence, how we
construct the mollifier $J_\ee$ plays a critical role in the proofs of
well-posedness for the HR i.v.p. in both the periodic and non-periodic cases. %
%
