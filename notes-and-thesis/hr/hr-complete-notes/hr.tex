\chapter{Non-Uniform Dependence for the
Hyperelastic Rod Equation}
%\begin{abstract}
%The solution map for the Hyperelastic Rod equation is not uniformly continuous
	%from any bounded set of $H^s$ into $C([-T, T]; H^s)$
	%for $s>3/2$ in the periodic case and for $s>1$ in the non-periodic case.
	%The proof is based on the method of approximate solutions.
%\end{abstract}
%\maketitle
%\markboth{Non-Uniform Dependence for the Hyperelastic Rod Equation}{David Karapetyan}
%\parindent0in
%\parskip0.1in
%\end{titlepage}
%%%%%%%%%%%%%%%%%%%%%%%%
%
%      introduction
%
%%%%%%%%%%%%%%%%%%%%%%%%
\section{Introduction}
%
We consider the periodic initial value problem for
the hyperelastic-rod (HR)  equation
\begin{equation}
	\label{hr}
	\p_t u
	-
	\p_t \p_x^2 u
	+
	3u\p_x u
	=
	\gamma \big (
	2\p_x u \p_x^2 u
	+
	u \p_x^3 u
	\big ),
\end{equation}
\begin{equation}
	\label{hr-data} u(x, 0) = u_0 (x),
	\quad x  \in \ci, \text{  or  } \rr \quad t \in \rr,
\end{equation}
where $\gamma$ is a constant. Equation \ref{hr} was first
derived by Dai in \cite{Dai_1998_Model-equations} as a model for finite-length and
small-amplitude axial deformation waves in thin cylindrical
rods composed of a compressible isotropic hyperelastic
material. Local well-posedness and blow-up criteria for
solutions were established by  Zhou in \cite{Liu_2008_Blow-up-phenome}. Orbital
stability of a class of solitary waves for this equation was
proved in \cite{Constantin_2000_Stability-of-a-} by Constantin and Strauss.
\\
\\
Motivated by the work of Himonas and Kenig \cite{Himonas:2009fk} and
Himonas and Misiolek \cite{Himonas:2005kx} we prove that 
the solution map for the HR equation is not uniformly
continuous in both the periodic and non-periodic case.
%
\begin{theorem}
	\label{hr-non-unif-dependence}
	For $s>3/2$ in the periodic case and for $s>1$ in
	the non-periodic case, the flow map $u_0 \to u(t)$ of the
	Cauchy-problem \eqref{hr}-\eqref{hr-data} is not uniformly continuous
	from any bounded set of $H^s$ into $C([-T, T]; H^s)$.
	More precisely, in both cases there exist two sequences of solutions $u_n(t)$
	and $v_n(t)$ in $C([-T, T]; H^s)$ such that
	%
	%
	\begin{equation}
		\label{h-s-bdd}
		\| u_n(t)  \|_{H^s}
		+
		\| v_n(t)  \|_{H^s}
		\lesssim
		1,
	\end{equation}
	%
	\begin{equation}
		\label{zero-limit-at-0}
		\lim_{n\to\infty}
		\|
		u_n(0)
		-
		v_n(0)
		\|_{H^s}
		=
		0,
	\end{equation}
	%
	%
	and
	%
	%
	\begin{equation}
		\label{bdd-away-from-0}
		\liminf_{n\to\infty}
		\|
		u_n(t)
		-
		v_n(t)
	\|_{H^s}
		\gtrsim
		\sin ( \gamma t),
		\quad
		|t|\le T.
	\end{equation}
	%
	%
	\end{theorem}
	For $\gamma \neq 3$ this result was proved by Olson in
  \cite{Olson_2006_The-initial-val} using
	traveling wave solutions. In this paper
	we will eliminate the restriction on $\gamma$ using approximate solutions to the HR
	equation. In Section 1 we introduce the approximate
	solutions we will be using, and derive a functional representation of
	the error in Section 2. In Section 3 we find an upper bound for the
	error. This leads to a crucial lemma in Section 4 that gives a
	decaying bound for the difference of approximate and actual solutions.
	In Section 5 the lemma is used in an interpolation which allows us to 
	conclude the non-uniform dependence on initial
	data of solutions to the HR equation, independent of the choice of 
	$\gamma$.
\section{The non-periodic case}
We consider the Cauchy problem for the Hyperelastic Rod Equation (HR)
\begin{equation}
	\begin{split}
		\p_t u + \gamma u \p_x u + \p_x \left( 1 - \p_x^2
		\right)^{-1}  \left[ \frac{3-\gamma}{2}u^2 +
		\frac{\gamma}{2} \left( \p_x u \right)^2
		\right] = 0,
		\label{apple1'}
	\end{split}
\end{equation}
%
\begin{equation}
	\begin{split}
		u(x,0) = u_0(x), \; \; x \in \rr, \; \; t \in \rr. 
		\label{apple2'}
	\end{split}
\end{equation}
Our approximate solutions $u^{\omega, \lambda} = u^{\omega,
\lambda}(x,t)$ to \eqref{apple1'}-\eqref{apple2'} will
consist of a low frequency and a high frequency part,
i.e.
\begin{equation}
	\label{apple1}
	u^{\omega,\lambda} = u_\ell + u^h.
\end{equation}
The high frequency part is given by 
\begin{equation}
	\begin{split}
		u^h = u^{h,\omega,\lambda}(x,t) =
		\lambda^{-\frac{\delta}{2} -s}
		\phi \left (\frac{x}{\lambda^\delta}\right )
		\cos(\lambda x - \gamma \omega t)
	\end{split}
\end{equation}
where $\phi$ is a $C^\infty$ cutoff function such that
\begin{equation*}
	\phi = 
	\begin{cases}
		1, &\text{if $|x|<1$;} \\
		0, &\text{if $|x| \ge 2$.} 
	\end{cases}
\end{equation*}
By Theorem \ref{thm:HR_existence_continous_dependence} ,
we let the low frequency part $u_\ell = u_{l,
\omega, \lambda}(x,t)$ be the unique solution to the HR equation
\begin{equation}
	\p_t u_\ell + \gamma u_\ell \p_x u_\ell + \p_x (1- \p_x^2)^{-1}  \left[
	\frac{3- \gamma}{2}(u_\ell)^2 + \frac{\gamma}{2}\left( \p_x u_\ell
	\right)^2 \right] = 0
	\label{apple1*}
\end{equation}
with initial data
\begin{equation}
	u_\ell(x,0) = \omega \lambda^{-1} \tilde{\phi} \left(
	\frac{x}{\lambda^{\delta}}
	\right), \quad x \in \rr, \quad t \in \rr
	\label{apple1**}
\end{equation}
where $\tilde{\phi}$ is a $C^{\infty}_0(\rr)$ function such that
\begin{equation}
	\label{apple1***}
	\tilde{\phi}(x) = 1 \; \;  \text{if} \; \;
	x \in \text{supp} \; \phi.
\end{equation}
Let $\Lambda^{-1} = \p_x (1 - \p_x^2)^{-1} $. Substituting the
approximate solution $u^{\omega, \lambda} = u_\ell + u^h$ into the HR
equation, and recalling that $u_\ell$ is a solution to the
HR equation, we obtain
\begin{equation}
	\begin{split}
		E 
		& = \p_t u^h + \gamma u_\ell \p_x u^h + \gamma u^h \p_x u_\ell +
		\gamma u^h \p_x u^h
		\\
		& + \Lambda^{-1} \left\{ \frac{3-\gamma}{2}\left[ \left( u^h
		\right)^2 + 2u_\ell u^h
		\right]+ \frac{\gamma}{2}\left[ \left( \p_x u^h \right)^2 + 2
		\p_x u_\ell \p_x u^h\right] \right\}.
		\label{apple2star}
	\end{split}
\end{equation}
Using a straightforward calculation of derivatives, and
noting that $\tilde{\phi} (x) = 1$ for $x \in \text{supp} \;
\phi$,
we deduce
\begin{equation}
	\begin{split}
		\p_t u^h + \gamma u_\ell \p_x u^h 
		& = \gamma \lambda\left[ u_\ell(x,0) - u_\ell(x,t)
		\right]\lambda^{-\frac{\delta}{2}-s} \phi\left(
		\frac{x}{\lambda^\delta}
		\right) \sin(\lambda x - \gamma \omega t)
		\\
		& + \gamma u_\ell(x,t) \cdot \lambda^{-\frac{3\delta}{2}-s}
		\phi'\left( \frac{x}{\lambda^\delta} \right)\cos\left( \lambda
		x - \gamma \omega t
		\right).
						 \label{apple5}
					 \end{split}
				 \end{equation}
				 Therefore, applying \eqref{apple5} to \eqref{apple2star}, we see that the error
				 $E$ of our approximate solution is given by
				 \begin{equation*}
					 E=E_1 + E_2 + \dots + E_8
				 \end{equation*}
				 where
				 \begin{equation}
					 \label{all_errors_together}
					 \begin{split}
						  E_1 & = \gamma \lambda \left[ u_\ell(x,0) - u_\ell(x,t)
						 \right] \lambda^{-\frac{\delta}{2}-s} \phi\left(
						 \frac{x}{\lambda^ \delta}
						 \right)\sin(\lambda x - \gamma \omega t)
						 \\
						 E_2 & = \gamma u_\ell(x,t) \cdot \lambda^{-\frac{3\delta}{2}-s}
						 \phi'\left( \frac{x}{\lambda^\delta} \right)\cos\left( \lambda
						 x - \gamma \omega t
						 \right)
						 \\
						 E_3 & = \gamma u^h \p_x u_\ell, \; \; E_4 = \gamma u^h \p_x u^h
						 \\
						 E_5 & = \Lambda^{-1}\left[ \frac{3-\gamma}{2} \left(
						 u^h \right)^2 
						 \right], \; \; E_6 = \Lambda^{-1}
						 \left[ (3- \gamma)u_\ell u^h \right]
						 \\
						 E_7 & = \Lambda^{-1} \left[ \frac{\gamma}{2} \left(
						 \p_x u^h \right)^2 \right ], \; \;
						 E_8 = \Lambda^{-1} \left[ \gamma \p_x u_\ell \p_x u^h \right]
						 .
						 \end{split}
				 \end{equation}
				 %
				 %
				 %
				 \subsection{An Upper Bound in $H^1$ For the Error of the Approximate
				 Solutions}
         For estimating the $H^1(\ci)$ norm of $u^h$, we need the
				 following result, whose proof can be found in
         \cite{Himonas:2009fk}:
				  \begin{lemma}
					 \label{applea}
					 Let $\psi \in S(\rr)$, $\alpha \in \rr$. Then for $s \ge 0$ we have
					 \begin{equation}
						 \begin{split}
							 \lim_{\lambda \to \infty} \lambda^{-\frac{\delta}{2}-s}
							 \|\psi \left( \frac{x}{\lambda^\delta} \right)\cos(\lambda
							 x - \alpha) \|_{H^s(\rr)} = \frac{1}{\sqrt
							 2}\|\psi\|_{L^2(\rr)}.
							 \label{apple6}
						 \end{split}
					 \end{equation}
					 Relation \eqref{apple6} remains true if $\cos$ is
					 replaced by $\sin$.
				 \end{lemma}
				 Next, we provide an upper bound for the $H^1(\ci)$ norm of
				 $u_l$; details can again be found in
         \cite{Himonas:2009fk}:
				 %
				\begin{lemma}
					\label{appleb}
					Let $0<\delta<2$, with $\omega$ belonging to a bounded
					subset of $\rr$. Then the initial value problem
					\eqref{apple1*}-\eqref{apple1**} has a unique solution
					$u_\ell \in C\left( [0,T], H^s(\rr) \right)$ for all $s
					\ge 0$, which 
					satisfies
					\begin{equation}
						\label{apple10'}
						\|u_\ell(t)\|_{H^s(\rr)} \le c_s \lambda^{-1 +
						\frac{\delta}{2}}, \quad |t| \le T.
					\end{equation}
				\end{lemma}
								We will also need the following:
\begin{lemma}
	\label{applec}
	For any $f,g \in L^2(\rr)$,
	\begin{equation*}
		\|fg\|_{H^1(\rr)} \le \sqrt{2} \|f\|_{C^1(\rr)} \|g\|_{H^1(\rr)}.
	\end{equation*}
\end{lemma}
%
We are now prepared to estimate the $H^1$ norms of each $E_i$.
\subsection{Estimating the $H^1$ norm of $\hyperref[all_errors_together]{E_1}$.} We have
\begin{equation*}
	\begin{split}
		\|E_1\|_{H^1(\rr)}
		& = \| \gamma \lambda \left[ u_\ell(x,0) - u_\ell(x,t) \right]
		\lambda^{-\frac{\delta}{2}-s} \phi\left( \frac{x}{\lambda^\delta}
		\right ) \sin (\lambda x - \gamma \omega t )\|_{H^1(\rr)}
		\\
		& = |\gamma| \lambda^{1 -\frac{\delta}{2} -s } \|\left[ u_\ell(x,0) - u_\ell(x,t)
		\right] \phi\left( \frac{x}{\lambda^\delta} \right )
		\sin\left( \lambda x - \gamma \omega t
		\right) \|_{H^1(\rr)}.
	\end{split}
\end{equation*}
Applying Lemma \ref{applec}, we obtain
\begin{equation}
	\begin{split}
		\|E_1\|_{H^1(\rr)} \le |\gamma| \lambda^{1 - \frac{\delta}{2} -s } \|\phi
		\left( \frac{x}{\lambda^\delta} \right) \sin (\lambda x - \gamma \omega t)
		\|_{C^1(\rr)} \|[u_\ell (x,0) - u_\ell (x,t) ] \|_{H^1(\rr)}.
		\label{apple14}
	\end{split}
\end{equation}
We now estimate the right hand side of \eqref{apple14} in pieces. For the first piece, we
have
\begin{equation*}
	\begin{split}
		& \|\phi \left( \frac{x}{\lambda^\delta} \right) \sin (\lambda x - \gamma \omega t)
		\|_{C^1(\rr)} 
		\\
		&
		\le \|\phi \left( \frac{x}{\lambda^\delta} \right) \|_{L^\infty(\rr)} + \lambda
		\|\phi\left( \frac{x}{\lambda^\delta} \right)\|_{L^\infty(\rr)} +
		\lambda^{-\delta} \|\phi'\left( \frac{x}{\lambda^\delta}
		\right)\|_{L^\infty(\rr)}
	\end{split}
\end{equation*}
which gives
\begin{equation}
	\begin{split}
		\|\phi\left( \frac{x}{\lambda^\delta} \right) \sin(\lambda x - \gamma \omega t)
		\|_{C^1(\rr)}
		\lesssim \lambda.
		\label{apple15}
	\end{split}
\end{equation}
For the next piece of \eqref{apple14}, we observe that the fundamental theorem
of calculus gives
\begin{equation*}
	u_\ell(x,t) - u_\ell(x,0) = \int_0^t \p_\tau u_\ell(x,\tau) \; d \tau.
\end{equation*}
Hence
\begin{equation}
	\begin{split}
		\|u_\ell(x,t) - u_\ell(x,0)\|_{H^1(\rr)}
		& = \left \| \int_0^t \p_\tau
		u_\ell(x,\tau) \; d \tau \right \|_{H^1(\rr)}
		\\
		& \le  \int_0^t \|\p_\tau u_\ell (x,\tau) \|_{H^1(\rr)} \; d \tau.
		\label{apple100}
	\end{split}
\end{equation}
We want to estimate the right hand side of \eqref{apple100}. Recalling
\eqref{apple1'}, we have
\begin{equation}
	\label{apple101}
	\begin{split}
		\|\p_\tau u_\ell(x,\tau) \|_{H^1(\rr)}
		& =  \|-\gamma u_\ell \p_x u_\ell + \Lambda^{-1} \left[
		\frac{3-\gamma}{2}(u_\ell)^2 + \frac{\gamma}{2} \left( \p_x u_\ell \right)^2
		\right] \|_{H^1(\rr)}
		\\
		& \le \|\gamma u_\ell \p_x u_\ell \|_{H^1(\rr)} + \|\Lambda^{-1} \left[
		\frac{3-\gamma}{2} (u_\ell)^2 + \frac{\gamma}{2} \left( \p_x u_\ell \right)^2
		\right] \|_{H^1(\rr)}.
	\end{split}
\end{equation}
Applying the algebra property of Sobolev spaces, we obtain
\begin{equation*}
	\begin{split}
		\|\gamma u_\ell \p_x u_\ell \|_{H^1(\rr)} &
		= \|\gamma \p_x (u_\ell)^2 \|_{H^1(\rr)}
		\\
		& \le |\gamma| \cdot \| (u_\ell)^2 \|_{H^2(\rr)}
		\\
		& \lesssim \|u_\ell\|_{H^2(\rr)}^2
	\end{split}
\end{equation*}
which by \eqref{apple10'} reduces to
\begin{equation}
	\begin{split}
		\|\gamma u_\ell \p_x u_\ell \|_{H^1(\rr)} \lesssim \lambda^{-2 + \delta}.
		\label{apple102}
	\end{split}
\end{equation}
Next, note that, for any $u \in L^2(\rr)$, we
have
\begin{equation}
	\begin{split}
		\|\Lambda^{-1} u \|_{H^1(\rr)} 
		\le \|u\|_{L^2(\rr)}.
		\label{apple27}
	\end{split}
\end{equation}
Hence, applying \eqref{apple27}, the algebra property of Sobolev spaces,
and the Sobolev Imbedding Theorem, we obtain
\begin{equation*}
	\begin{split}
		\|\Lambda^{-1} \left[ \frac{3-\gamma}{2}u^2 +
		\frac{\gamma}{2}\left( \p_x u \right)^2 \right] \|_{H^1(\rr)}
		& \lesssim \|u\|_{H^2(\rr)}^2
	\end{split}
\end{equation*}
which by \eqref{apple10'} reduces to 
\begin{equation}
	\begin{split}
		\|\Lambda^{-1} \left[ \frac{3-\gamma}{2}u^2 +
		\frac{\gamma}{2}\left( \p_x u \right)^2 \right] \|_{H^1(\rr)}
		\lesssim \lambda^{-2 + \delta}.
		\label{apple104}
	\end{split}
\end{equation}
Substituting \eqref{apple102} and \eqref{apple104} into the right hand side of
\eqref{apple101}, and recalling \eqref{apple100}, we obtain
\begin{equation}
	\begin{split}
		\|u_\ell(x,t) - u_\ell(x,0)\|_{H^1(\rr)} \lesssim \lambda^{-2 + \delta}.
		\label{apple105}
	\end{split}
\end{equation}
By estimates \eqref{apple14}, \eqref{apple15}, and \eqref{apple105}, we conclude that 
\begin{equation}
	\begin{split}
		\|E_1\|
		& \lesssim \lambda^{\frac{\delta}{2}-s}.
		\label{apple106}
	\end{split}
\end{equation}
%
\subsection{Estimating the $H^1$ norm of $\hyperref[all_errors_together]{E_2}$.} Applying Lemma \ref{applec}, we have
\begin{equation}
	\begin{split}
		\|E_2\|_{H^1(\rr)} 
		& = \gamma \lambda^{-\frac{3 \delta}{2} -s } \|u_\ell(x,t) \cdot
		\phi'\left( \frac{x}{\lambda^\delta} \right) \cos (\lambda x - \gamma \omega t)
		\|_{H^1(\rr)}
		\\
		& \le c_s \gamma \lambda^{\frac{-3 \delta}{2} -s } \|u_\ell(x,t) \|_{H^1(\rr)}
		\|\phi'\left( \frac{x}{\lambda^\delta} \right )
		\cos(\lambda x - \gamma \omega t 
		\|_{C^1(\rr)}.
		\label{apple18}
	\end{split}
\end{equation}
We note that
\begin{equation*}
	\begin{split}
		& \|\phi'\left( \frac{x}{\lambda^\delta} \right) \cos(\lambda x - \gamma \omega t)
		\|_{C^1(\rr)}
		\\
		& \le \|\phi' \left( \frac{x}{\lambda^\delta} \right)\|_{L^\infty(\rr)} +
		\lambda \|\phi'\left( \frac{x}{\lambda^\delta} \right)\|_{L^\infty(\rr)}
		+ \lambda^{-\delta} \|\phi''\left( \frac{x}{\lambda^\delta} \right)
		\|_{L^\infty(\rr)}
	\end{split}
\end{equation*}
which gives
\begin{equation}
	\begin{split}
		\|\phi'\left( \frac{x}{\lambda^\delta} \right) \cos(\lambda x - \gamma \omega t)
		\|_{C^1(\rr)} \lesssim \lambda.
		\label{apple19}
	\end{split}
\end{equation}
Applying estimates \eqref{apple19} and \eqref{apple10'} to \eqref{apple18}, we obtain
\begin{equation*}
	\begin{split}
	\label{apple20}
	\|E_2\|_{H^1(\rr)} \lesssim \lambda^{-\delta -s }.
\end{split}
\end{equation*}
%
%
%
%
\subsection{Estimating the $H^1$ norm of $\hyperref[all_errors_together]{E_3}$.} 
By Lemma \ref{applec}, we deduce
\begin{equation}
	\begin{split}
		\|\gamma u^h \p_x u_\ell \| \le \sqrt{2} |\gamma| \cdot \|u^h\|_{C^1(\rr)}
		\|u_\ell\|_{H^1(\rr)}.
		\label{apple21}
	\end{split}
\end{equation}
Now, note that
\begin{equation}
	\begin{split}
		\|u^h\|_{L^\infty(\rr)} 
		& = \lambda^{-\frac{\delta}{2} -s } \|\phi\left( \frac{x}{\lambda^\delta}
		\right) \cos \left( \lambda x - \gamma \omega t \right) \|_{L^\infty(\rr)}
		\\
		& \lesssim \lambda^{-\frac{\delta}{2} -s }.
		\label{apple22}
	\end{split}
\end{equation}
and 
\begin{equation}
	\begin{split}
		& \|\p_x u^h \|_{L^\infty(\rr)}
		\\
		& = \lambda^{-\frac{\delta}{2}-s} \|\phi\left(
		\frac{x}{\lambda^\delta}
		\right) \cdot -\lambda \sin(\lambda x - \gamma \omega t) + \lambda^{-\delta}
		\phi'\left( \frac{x}{\lambda^\delta}\right) \cos(\lambda x - \gamma \omega
		t) \|_{L^\infty(\rr)}
		\\
		& \lesssim \lambda^{1 - \frac{\delta}{2} -s }.
		\label{apple23}
	\end{split}
\end{equation}
Therefore, from \eqref{apple22} and \eqref{apple23} it follows that
\begin{equation}
	\begin{split}
		\|u^h\|_{C^1(\rr)} \lesssim \lambda^{-\frac{\delta}{2} -s } + \lambda^{1
		-\frac{\delta}{2} -s}
		\approx \lambda^{1- \frac{\delta}{2} -s}.
		\label{apple24}
	\end{split}
\end{equation}
Substituting estimates \eqref{apple24} and  \eqref{apple10'} into \eqref{apple21} we obtain
\begin{equation}
	\begin{split}
		\|\gamma u^h \p_x u_\ell \|_{H^1(\rr)} \lesssim \lambda^{-s}.
		\label{apple24'}
	\end{split}
\end{equation}
\subsection{Estimating the $H^1$ norm of $\hyperref[all_errors_together]{E_4}$.} Applying Lemma \ref{applec} we have
\begin{equation}
	\begin{split}
		\|\gamma u^h \p_x u^h\|_{H^1(\rr)} \lesssim \|u^h\|_{C^1(\rr)}
		\|u^h\|_{H^1(\rr)}.
		\label{apple25}
	\end{split}
\end{equation}
Substituting in \eqref{apple24}, and recalling that $\|u^h\|_{H^1(\rr)} \approx 1$ by Lemma
\ref{applea}, we obtain
\begin{equation}
	\begin{split}
		\|u^h \p_x u^h \|_{H^1(\rr)} \lesssim \lambda^{1-\frac{\delta}{2}-s}.
		\label{apple26}
	\end{split}
\end{equation}
%
%
\subsection{Estimating the $H^1$ norm of $\hyperref[all_errors_together]{E_5}$.}
Applying \eqref{apple27}, we obtain
\begin{equation}
	\begin{split}
		\|E_5\|_{H^1(\rr)}
		& = \|\Lambda^{-1}\left[ \frac{3-\gamma}{2}(u^h)^2
		\right]\|_{H^1(\rr)}
		\\
		& \lesssim \|u^h\|_{L^\infty(\rr)} \|u^h\|_{L^2(\rr)}.
		\label{apple28}
	\end{split}
\end{equation}
Substituting \eqref{apple6} and \eqref{apple22} into \eqref{apple28}, we conclude that
\begin{equation}
	\begin{split}
		\|E_5\|_{H^1(\rr)} \lesssim \lambda^{-\frac{\delta}{2}-s}.
		\label{apple29}
	\end{split}
\end{equation}
%
%
%
%
\subsection{Estimating the $H^1$ norm of $\hyperref[all_errors_together]{E_6}$.} Applying \eqref{apple27}, we obtain
\begin{equation}
	\begin{split}
		\|E_6\|_{H^1(\rr)} 
		& = \|\Lambda^{-1} \left[ (3 -\gamma) u_\ell u^h \right]\|_{H^1(\rr)}
		\\
		& \lesssim \|u_\ell\|_{L^2(\rr)} \|u^h\|_{L^\infty(\rr)}.
		\label{apple30}
	\end{split}
\end{equation}
which by Lemma \ref{appleb} and \eqref{apple22} reduces to
\begin{equation}
	\begin{split}
		\|E_6\|_{H^1(\rr)} \lesssim \lambda^{-1-s}.
		\label{apple31}
	\end{split}
\end{equation}
%
%
%
%
\subsection{Estimating the $H^1$ norm of $\hyperref[all_errors_together]{E_7}$.} Applying \eqref{apple27}, we obtain
\begin{equation}
	\begin{split}
		\|E_7\|_{H^1(\rr)} 
		& = \|\Lambda^{-1} \left[ \frac{\gamma}{2}\left( \p_x u \right)^2
		\right]\|_{H^1(\rr)}
		\\
		& \lesssim  \|\p_x u^h\|_{L^\infty(\rr)} \|u^h\|_{H^1(\rr)}.
		\label{apple32}
	\end{split}
\end{equation}
which by Lemma \ref{applea} and \eqref{apple23} reduces to
\begin{equation}
	\begin{split}
		\|E_7\|_{H^1(\rr)} \lesssim \lambda^{1-\frac{\delta}{2}-s}.
		\label{apple33'}
	\end{split}
\end{equation}
%
%
%
%
\subsection{Estimating the $H^1$ norm of $\hyperref[all_errors_together]{E_8}$.} Applying \eqref{apple27}, we have
\begin{equation}
	\begin{split}
		\|E_8\|_{H^1(\rr)}
		& = \|\Lambda^{-1}\left[ \gamma \p_x u_\ell \p_x u^h \right]\|_{H^1(\rr)}
		\\
		& \lesssim \|u_\ell\|_{H^2(\rr)} \|\p_x u^h\|_{L^\infty(\rr)}.
		\label{apple33}
	\end{split}
\end{equation}
which by Lemma \ref{appleb} and \eqref{apple23} reduces to
\begin{equation}
	\begin{split}
		\|E_8\|_{H^1(\rr)} \lesssim \lambda^{-s}.
		\label{apple34}
	\end{split}
\end{equation}
Collecting all our estimates for the $E_i$ and recalling that we have assumed
$1<\delta<2$, we obtain
\begin{equation*}
	\begin{split}
		\|E\|_{H^1(\rr)}
		 \lesssim \lambda^{\frac{\delta}{2} -s }, \qquad \lambda >>1.
	\end{split}
\end{equation*}
We now summarize our result:
%
%
\begin{proposition}
	Let $1<\delta<2$. Then for $s > 1$, bounded $\omega$, and
	$\lambda >>1$ we are assured the decay of the error $E$ of the
	approximate solutions to the HR equation; specifically
	\begin{equation}
		\begin{split}
			\|E(t)\|_{H^1(\rr)} \lesssim \lambda^{-r_s}
			\label{apple35}
		\end{split}
	\end{equation}
	where
	\begin{equation}
		\begin{split}
			r_s = s - \frac{\delta}{2}.   
			\label{appler_s}
		\end{split}
	\end{equation}
\end{proposition}
%
%
%
%
%
\subsection{Estimating the $H^1$ norm of the difference 
between approximate and actual
	solutions.}
	We wish now to estimate the difference between approximate and actual solutions to
	the HR i.v.p with common initial data $u_0 \in H^s$. Let
	$u_{\omega,\lambda}(x,t)$ be the unique solution to the HR equation
	with initial data $u^{\omega,\lambda}(x,0)$; that is,
	$u_{\omega,\lambda}$ solves the initial value problem
	\begin{align}
		& \p_t u_{\omega,\lambda} + \gamma u_{\omega,\lambda} \p_x u_{\omega,\lambda} + \Lambda^{-1} \left[
		\frac{3- \gamma}{2}\left( u_{\omega,\lambda} \right)^2 + \frac{\gamma}{2}\left(
		\p_x u_{\omega,\lambda} \right)^2
		\right], \; \; x\in \rr, \; \; t \in \rr,
		\label{apple50}
		\\
		& u_{\omega,\lambda} = u^{\omega,\lambda}(x,0)=\omega \lambda^{-1}
		\tilde{\phi} \left( \frac{x}{\lambda^\delta} \right)
		+ \lambda^{-\frac{\delta}{2} -s}
		\phi\left( \frac{x}{\lambda^\delta} \right) \cos(\lambda x).
		\label{apple41}
	\end{align}
	%
%
%
Letting $v = u^{\omega,\lambda} - u_{\omega,\lambda}$, we will now prove the following critical lemma:
\begin{lemma}
	\label{applelem:bound_for_difference-of-approx-and-actual-soln}
	For $s > 1$ and $1<\delta<2$ we have 
			\begin{equation} 
				\|
				v(t)
				\|_{H^1(\rr)}
				\doteq
				\label{applediffer-H1-est} 
				\|
				u^{\omega,\lambda}(t) 
				- 
				u_{\omega,\lambda}(t)
				\|_{H^1(\rr)}
				\lesssim 
				n^{-r_s}, 
				\quad
				|t| \le T
			\end{equation}
			where $r_s$ is defined as in \eqref{appler_s}.
			%
			\end{lemma}
      \begin{proof} Our plan will be to calculate an energy-estimate for $v$.
			To do so, we must first calculate $\p_t v$. Subtracting $\p_t
			u_{\omega, \lambda}$ from $\p_t u^{\omega,\lambda}$ and
			recalling that $u_{\omega,t}$ is a solution to the HR Cauchy
			problem \eqref{apple50}-\eqref{apple41},
			and $u_{\omega,\lambda}$ is an approximate solution, we obtain
			\begin{equation*}
				\begin{split}
					\p_t v 
					& = E + \gamma(v \p_x v - v \p_x u^{\omega,\lambda} - u^{\omega,\lambda} \p_x v) 
					\\
					& + \p_x \left( 1 - \p_x^2 \right)^{-1}  \left[ \frac{3-
					\gamma}{2}v^2 + \frac{\gamma}{2}\left( \p_x v \right)^2 - \left(
					3 - \gamma \right)u^{\omega,\lambda} v -
					\gamma \p_x u^{\omega,\lambda} \p_x v \right].
				\end{split}
			\end{equation*}
			It follows immediately that
		\begin{equation}
			\label{applev-dtv-pseudo-functional-equality}
			\begin{split}
			v(1-\p_x^2)\p_t v &= v(1- \p_x^2)E + v\gamma(1- \p_x^2)(v\p_x v 
			- v\p_x u^{\omega,\lambda} -
			u^{\omega,\lambda} \p_x v)
			\\
			&+ v\p_x \left[ \frac{3-\gamma}{2}v^2 + \frac{\gamma}{2}(\p_x v)^2 -
			(3-\gamma)u^{\omega,\lambda} v - \gamma \p_x u^{\omega,\lambda} \p_x v \right].
		\end{split}
	\end{equation}
	Applying the equality $v\p_t v = v(1-\p_x^2) \p_t v + v\p_x^2 \p_t v$ to
	\eqref{applev-dtv-pseudo-functional-equality}, we obtain
	\begin{equation*}
		\begin{split}
		v \p_t v &= v(1- \p_x^2)E + v\gamma(1- \p_x^2)(v\p_x v - v\p_x u^{\omega,\lambda} -
			u^{\omega,\lambda} \p_x v)
			\\
			&+ v\p_x \left[ \frac{3-\gamma}{2}v^2 + \frac{\gamma}{2}(\p_x v)^2 -
			(3-\gamma)u^{\omega,\lambda} v - \gamma \p_x u^{\omega,\lambda} \p_x v
			\right] + v\p_x^2 \p_t v.
		\end{split}
	\end{equation*}
	Hence
	\begin{equation}
		\label{appleenergy-est}
		\begin{split}
			&\frac{1}{2} \frac{d}{dt} \|v\|_{H^1(\rr)}^2  
			\\
		& =  \int_{\rr} \left[ v(1-\p_x^2)E \right]dx
		- \gamma \int_{\rr} \left[ v(1-\p_x^2)(v\p_x u^{\omega,\lambda} + u^{\omega,\lambda} \p_x v) \right]dx
		\\
		&- \int_{\rr}\left[ \left( 3-\gamma \right)v \p_x\left( u^{\omega,\lambda}v \right) + \gamma v
		\p_x \left( \p_x u^{\omega,\lambda} \p_x v \right)\right]dx
		\\
		&+  \int_{\rr}
		\left[ \gamma v \left( 1-\p_x^2 \right)\left( v \p_x v \right) + v
		\p_x \left( \frac{3-\gamma}{2} v^2 + \frac{\gamma}{2}\left( \p_x v \right)^2
		\right) \right . +  v \p_x^2 \p_t v + \p_x v \p_t \p_x v\bigg]dx.
	\end{split}
\end{equation}
We compute the last term of \eqref{appleenergy-est} first:
\begin{equation*}
	\begin{split}
	&  \int_{\rr} \bigg[ \gamma v \left( 1-\p_x^2 \right)(v\p_x v) + v \p_x\left(
	\frac{3-\gamma}{2}v^2 + \frac{\gamma}{2}\left( \p_x v \right )^2 \right)
	+ v\p_x^2 \p_t v + \p_x v \p_t \p_x v
	\bigg]dx
	\\
	& =  \int_{\rr} \left[ 3v^2 \p_x v - \gamma v^2 \p_x^3 v - 2 \gamma v \p_x v \p_x^2
	v + v \p_x^2 \p_t v + \p_x v \p_t \p_x v \right]dx
	\\
	&=  \int_{\rr} \left[ \p_x (v^3) - \gamma \p_x (v^2 \p_x^2 v) + \p_x\left( v \p_t
	\p_x v
	\right) \right]dx
	\\
	& = 0.
\end{split}
\end{equation*}
Therefore			
\begin{equation}
	\label{appleenergy-estimate-simplified}
	\begin{split}
		\frac{1}{2}	\frac{d}{dt} \|v(t)\|_{H^1(\rr)}^2
		& =  \int_{\rr}
		 v\left( 1-\p_x^2
	\right)E \; dx
	\\
	&-  \gamma  \int_{\rr}  v\left( 1-\p_x^2 \right)\left( v \p_x u^{\omega,\lambda}
	+ u^{\omega,\lambda} \p_x v
	\right) \; dx
	\\
	& -  \int_{\rr} \left[ \left( 3-\gamma \right)v \p_x \left( u^{\omega,\lambda}v \right) + \gamma v
	\p_x \left( \p_x u^{\omega,\lambda} \p_x v \right)\right]dx.
\end{split}
\end{equation}
We now estimate the right-hand-side of \eqref{appleenergy-estimate-simplified}:
%
\begin{equation}
	\begin{split}
		\label{applefirst_piece}
	\left |\int_{\rr} \left [v (1- \p_x^2)E \right ] dx \right |
	& \lesssim
	\left( \|v\|_{L^2(\rr)}
	\|E\|_{L^2(\rr)} + \|\p_x v \|_{L^2(\rr)}
		\|\p_xE\|_{L^2(\rr)}\right)
	\\
	&
	\lesssim
	\|v\|_{H^1(\rr)} \|E\|_{H^1(\rr)}.
\end{split}
\end{equation}
For the second piece of \eqref{appleenergy-estimate-simplified} we have
\begin{equation}
	\begin{split}
		\label{applesecond-piece}
		& -\gamma \int_{\rr} v\left( 1-\p_x^2 \right)\left( v \p_x u^{\omega,\lambda} +
		u^{\omega,\lambda} \p_x v
		\right) \; dx
		\\
		& = -\gamma \int_{\rr} \left[ v^2 \p_x u^{\omega,\lambda} - v \p_x^2\left( v \p_x u^{\omega,\lambda}
		\right) \right]dx
		\\
		&   -\gamma \int_{\rr}\left[ vu^{\omega,\lambda} \p_x v - v \p_x^2\left( u^{\omega,\lambda} \p_x v \right)
		\right]dx.
	\end{split}
\end{equation}
We estimate the first term of \eqref{applesecond-piece} in parts:
\begin{equation*}
	\begin{split}
		\left | -\gamma \int_{\rr} \left[ v^2 \p_x u^{\omega,\lambda} \right]dx \right |
		& \lesssim \|\p_x u^{\omega,\lambda} \|_{L^\infty(\rr)} \|v\|_{H^1(\rr)}^2
	\end{split}
\end{equation*}
and by the product rule, Cauchy-Schwartz, and the Sobolev Imbedding Theorem,
we also have 
\begin{equation*}
	\begin{split}
		\left |-  \gamma \int_{\rr} \left [-v \p_x^2 \left(v \p_x u^{\omega,\lambda}
		\right) \right ]  dx \right |
		& \lesssim \left ( \|v\|_{L^2(\rr)} \|\p_x v\|_{L^2(\rr)} \|\p_x^2
		u^{\omega,\lambda} \|_{L^\infty(\rr)} \right .
		\\
		&+ \|\p_x v \|_{L^2(\rr)}^2 \|\p_x u^{\omega,\lambda}\|_{L^\infty(\rr)} \left )
		\right .
		\\
		& \lesssim \left( \|\p_x u^{\omega,\lambda} \|_{L^\infty(\rr)}+ \|\p_x^2
		u^{\omega,\lambda} \|_{L^\infty(\rr)} \right )\|v\|_{H^1(\rr)}^2 .
	\end{split}
\end{equation*}
Hence,
\begin{equation}
	\label{applepart1}
	\begin{split}
		& \left | - \gamma \int_{\rr} \left[ v^2 \p_x u^{\omega,\lambda} - v\p_x^2 \left( v \p_x u^{\omega,\lambda} \right)
		\right]dx \right |
		 \\
		 &  \lesssim  \left( \|\p_x u^{\omega,\lambda} \|_{L^\infty (\rr)} + \| \p_x^2 u^{\omega,\lambda}
		\|_{L^\infty(\rr)}
		\right)
		\|v \|_{H^1 (\rr)}^2.
	\end{split}
\end{equation}
Next, we estimate the second term of \eqref{applesecond-piece} in parts. For the first part, we apply Cauchy-Schwartz to obtain:
\begin{equation}
	\label{applefirst-part}
	\begin{split}
		\left |  -\gamma \int_{\rr} \left[ v u^{\omega,\lambda} \p_x v \right] dx \right |
		& \lesssim \|u^{\omega,\lambda}\|_{L^\infty(\rr)} \|v\|_{H^1(\rr)}^2
	\end{split}
\end{equation}
For the second part, we use integration by parts and the product rule to
obtain
\begin{equation}
	\label{appleboo}
	\begin{split}
		 \left | -\gamma \int_{\rr} 
		 -v \p_x^2 \big ( u^{\omega,\lambda} \p_x v \big )
		\; dx \right | 
		& \simeq \left | \int_{\rr} \left[ 
		\p_x\left( u^{\omega,\lambda}\left( \p_x v
		\right)^2 \right) + \p_x u^{\omega,\lambda}\left( \p_x v \right)^2
		\right]dx \right |.
	\end{split}
\end{equation}
Since $u^{\omega,\lambda}$ is compactly supported on $\rr$, \eqref{appleboo}
gives
\begin{equation}
	\label{applesecond-part}
	\begin{split}
		 \left | -\gamma \int_{\rr} \left [-v \p_x^2 \big ( u^{\omega,\lambda} \p_x v \big )\right
		] dx \right | 
		& \lesssim \|\p_x u^{\omega,\lambda} \|_{L^\infty(\rr)} \|v \|_{H^1(\rr)}^2.
	\end{split}
\end{equation}
Grouping \eqref{applefirst-part} and \eqref{applesecond-part} we obtain
\begin{equation}
	\label{applepart2}
	\left |  -\gamma \int_{\rr} \left[ v u^{\omega,\lambda} \p_x v - v \p_x^2\left( u^{\omega,\lambda} \p_x v \right)
	\right]dx \right | \lesssim \left( \|u^{\omega,\lambda}\|_{L^\infty(\rr)} + \|\p_x u^{\omega,\lambda}
	\|_{L^\infty(\rr)}
	\right)\|v\|_{H^1(\rr)}^2.
\end{equation}
which, combined with \eqref{applepart1} yields an estimate for
\eqref{applesecond-piece}:
\begin{equation}
	\begin{split}
		\label{applesecond-piece-final}
		\left | -\gamma \int_{\rr}
		\left[ v\left( 1-\p_x^2 \right)\left( v \p_x u^{\omega,\lambda} + u^{\omega,\lambda} \p_x v
		\right) \right] dx \right |
		&\lesssim \left( \|u^{\omega,\lambda}\|_{L^\infty(\rr)}\| + \|\p_x u^{\omega,\lambda}
		\|_{L^\infty(\rr)} \right . 
		\\
		& + \|\p_x^2 u^{\omega,\lambda} \|_{L^\infty(\rr)}
		\big )\|v\|_{H^1(\rr)}^2.
	\end{split}
\end{equation}
We now estimate the final piece of the right-hand-side of
\eqref{appleenergy-estimate-simplified}, i.e.
\begin{equation}
	\label{applelast_piece}
	-\int_{\rr} \left[ \left( 3 -\gamma \right)v \p_x \left( u^{\omega,\lambda} v \right) + \gamma
	v \p_x \left( \p_x u^{\omega,\lambda} \p_x v \right)\right]dx.
\end{equation}
We will estimate in parts:
\begin{equation}
	\begin{split}
		\label{applelast_piece_part1}
		\left | -\int_{\rr}  \left( 3- \gamma \right)v \p_x\left( u^{\omega,\lambda} v \right)
		 dx \right | 
		& \lesssim \|\p_x v \|_{L^2(\rr)} \|u^{\omega,\lambda} v \|_{L^2(\rr)}
		\\
		& \lesssim \|u^{\omega,\lambda}\|_{L^\infty(\rr)} \|v\|_{H^1(\rr)}^2
	\end{split}
\end{equation}
and
\begin{equation}
	\begin{split}
		\label{applelast_piece_part2}
		\left | -\int_{\rr}  \gamma v \p_x \left( \p_x u^{\omega,\lambda} \p_x v
		\right) dx  \right | 
		& \lesssim \|\p_x v \|_{L^2(\rr)} \| \p_x u^{\omega,\lambda} \p_x v \|_{L^2(\rr)}
		\\
		& \lesssim \|\p_x u^{\omega,\lambda} \|_{L^\infty(\rr)} \|v \|_{H^1(\rr)}^2.
	\end{split}
\end{equation}
Using \eqref{applelast_piece_part1} and \eqref{applelast_piece_part2}, we now have the
following estimate for \eqref{applelast_piece}:
\begin{equation}
	\begin{split}
	\label{applelast_piece_final}
	& \left | -\int_{\rr} \left[ \left( 3-\gamma \right)v
	\p_x \left( u^{\omega,\lambda} v \right) + \gamma
	v \p_x \left( \p_x u^{\omega,\lambda} \p_x v \right)\right]dx \right |
	\\
	& \lesssim \big(
	\|u^{\omega,\lambda}\|_{L^\infty(\rr)}
	 + \|\p_x u^{\omega,\lambda} \|_{L^\infty(\rr)} \big)
	\|v\|_{H^1(\rr)}^2.
\end{split}
\end{equation}
Combining \eqref{applefirst_piece}, \eqref{applesecond-piece-final},
and \eqref{applelast_piece_final}, we can
simplify \eqref{appleenergy-estimate-simplified} to obtain
\begin{equation}
	\begin{split}
		\label{appleenergy-estimate-best}
		\frac{d}{dt} \|v(t)\|_{H^1(\rr)}^2
		& \lesssim \left( \|u^{\omega,\lambda}\|_{L^\infty(\rr)} + \|
		\p_x u^{\omega,\lambda} \|_{L^\infty(\rr)} + \|\p_x^2 u^{\omega,\lambda} \|_{L^\infty (\rr)} \right)
		\|v\|_{H^1(\rr)}^2 
		\\
		&+ \|v\|_{H^1(\rr)} \|E\|_{H^1(\rr)}.
	\end{split}
\end{equation}
Now, observe that
\begin{equation}
	\begin{split}
		\p_x^2 u^h 
		& = \lambda^{-\frac{\delta}{2}-s} \Big[ - \lambda^2 \phi\left(
		\frac{x}{\lambda^\delta} \right ) \cos(\lambda x - \gamma \omega t) \\
		& - 2\lambda^{1 -\delta } \phi'\left( \frac{x}{\lambda^\delta}
		\right )
		\sin(\lambda x - \gamma \omega t ) + \lambda^{-2\delta} \phi''\left(
		\frac{x}{\lambda^\delta} \right) \cos (\lambda x - \gamma \omega t) \Big].
		\label{apple51}
	\end{split}
\end{equation}
Since the $\lambda^2$ term dominates inside the brackets, \eqref{apple51} yields
\begin{equation}
	\begin{split}
		\|\p_x^2 u^h \|_{L^\infty(\rr)} \lesssim
		\lambda^{2-\frac{\delta}{2}-s}.
		\label{apple52}
	\end{split}
\end{equation}
Combining \eqref{apple22}, \eqref{apple23}, and \eqref{apple52}, we obtain
\begin{equation}
	\begin{split}
		\|u^h\|_{L^\infty(\rr)} + \|\p_x u^h\|_{L^\infty(\rr)} + \|\p_x^2
		u^h\|_{L^\infty(\rr)} \lesssim \lambda^{-\left(
		\frac{\delta}{2} + s -2 \right)}.
		\label{apple53}
	\end{split}
\end{equation}
Furthermore, we have
\begin{equation}
	\begin{split}
		\|u_\ell\|_{L^\infty(\rr)} + \|\p_x u_\ell \|_{L^\infty(\rr)} + \|\p_x^2
		u_\ell\|_{L^\infty(\rr)} \le c_s \|u_\ell\|_{H^3(\rr)}.
		\label{apple54}
	\end{split}
\end{equation}
Applying Lemma \ref{appleb} to \eqref{apple54}, we see that
\begin{equation}
	\begin{split}
		\|u_\ell\|_{L^\infty(\rr)} + \|\p_x u_\ell \|_{L^\infty(\rr)} + \|\p_x^2
		u_\ell\|_{L^\infty(\rr)} \lesssim \lambda^{-(1 - \frac{\delta}{2})},
		\quad |t| \le T.
		\label{apple55}
	\end{split}
\end{equation}
Combining \eqref{apple53} and \eqref{apple55}, we obtain
\begin{equation}
	\begin{split}
		\|u^{\omega,\lambda}\|_{L^\infty(\rr)} + \|\p_x u^{\omega,\lambda}\|_{L^\infty(\rr)} + \|\p_x^2
		u^{\omega,\lambda}\|_{L^\infty(\rr)}
		& \lesssim \lambda^{-\rho_s},
		\label{apple56}
	\end{split}
\end{equation}
where
\begin{equation}
	\begin{split}
		\rho_s = \text{min} \left\{ \frac{\delta}{2} + s -2, \; 1-
		\frac{\delta}{2} \right\}.
		\label{apple57}
	\end{split}
\end{equation}
Note that for $s>1$, we can assure $\rho_s > 0$
by choosing a suitable $1<\delta<2$.
Substituting \eqref{apple35} and \eqref{apple56} into \eqref{appleenergy-estimate-best},
we get
\begin{equation}
	\label{apple58}
	\frac{d}{dt} \|v(t)\|_{H^(\rr)}^2 \lesssim \lambda^{-\rho_s}
	\|v\|_{H^1(\rr)}^2 + \lambda^{-r_s}
	\|v \|_{H^1(\rr)}
\end{equation}
where we recall the definition of $r_s$ from \eqref{appler_s}. By Gronwall's Inequality, we conclude that for $s>1$ and
suitably chosen $1<\delta<2$, we are assured the
decay of $\|v(t)\|_{H^1(\rr)}$, i.e. 
\begin{equation}
	\label{appleen-est-fin!}
	\|v(t)\|_{H^1(\rr)} 
	\lesssim
	\lambda^{-r_s}, \quad |t| \le T,\quad \lambda>>1 . 
\end{equation}
This completes the proof. 
\end{proof}
%
%
%
\subsection{Non-Uniform Dependence for $s > 1$.}
Let $u_{\pm 1,\lambda}$ be solutions to the HR i.v.p. with common initial data $u^{\pm 1,
n}(0)$, respectively.
We wish to show that the $H^s$ norm of the difference of $u_{\pm 1,
n}$ and the associated approximate solution $u^{\pm 1,\lambda}$
decays as $n \to \infty$. In order for \eqref{appleen-est-fin!} to hold,
we assume $s > 1$; recalling Theorem \ref{thm:HR_existence_continous_dependence}
, we
have
\begin{equation}
	\begin{split}
		\|u_{\pm 1,\lambda} (t) \|_{H^{2s-1}(\rr)}
		& \le 2 \|u^{\pm 1,\lambda}(0) \|_{H^{2s-1}(\rr)}, \qquad
		|t| \le T.
		\label{apple60}
	\end{split}
\end{equation}
Furthermore, recalling \eqref{apple1}-\eqref{apple1***}, we have 
\begin{equation}
	\begin{split}
		& \|u^{\pm 1, \lambda}(t)\|_{H^{2s-1}(\rr)}
		\\
		& \le \|u_{\ell, \pm \omega, \lambda}\|_{H^{2s-1}(\rr)} +
		 \| \lambda^{-\frac{\delta}{2} -s} \phi \left(
		\frac{x}{\lambda^\delta} \right) \cos(\lambda x \mp \gamma \omega t)
		\|_{H^{2s-1}(\rr)}
		\\
		& = \|u_{\ell, \pm \omega, \lambda}\|_{H^{2s-1}(\rr)}
		+
		\lambda^{s-1} \cdot
		\lambda^{-\frac{\delta}{2}-(2s-1)} \|\phi \left(
		\frac{x}{\lambda^\delta} \right) \cos(\lambda x \mp \gamma \omega t)
		\|_{H^{2s-1}(\rr)}.
		\label{apple61}
	\end{split}
\end{equation}
Applying Lemma \ref{applea} and Lemma \ref{appleb} to \eqref{apple61}, we obtain
\begin{equation}
	\begin{split}
		\|u^{\pm 1, \lambda}(t) \|_{H^{2s-1}(\rr)}
		& \lesssim \lambda^{s-1}.
		\label{apple62}
	\end{split}
\end{equation}
Hence, using \eqref{apple60}, \eqref{apple62}, and the triangle inequality, we deduce
\begin{equation}
	\begin{split}
		\|u^{\pm 1, \lambda}(t) - u_{\pm 1, \lambda}(t) \|_{H^{2s-1}(\rr)}
		\lesssim \lambda^{s-1}.
		\label{apple63}
	\end{split}
\end{equation}
Furthermore, by Lemma
\ref{applelem:bound_for_difference-of-approx-and-actual-soln}, we have
\begin{equation}
	\begin{split}
		\|u^{\pm 1, \lambda}(t) - u_{\pm 1, \lambda} \|_{H^1(\rr)} \lesssim
		\lambda^{-r_s}.
		\label{apple64}
	\end{split}
\end{equation}
		We now wish to interpolate in order to obtain an estimate of the $H^s (\rr)$
		norm of the difference of the approximate and actual solutions:
		\begin{lemma}
			\label{apple403}
			For all $\psi \in L^2(\rr)$,
			\begin{equation*}
				\|\psi \|_{H^s (\rr)}^2 \leq  \| \psi \|_{H^1 (\rr)} \| \psi
				\|_{H^{2s-1}}. 
			\end{equation*}
		\end{lemma}
			Hence, using Lemma \ref{apple403} to interpolate between estimates
			\eqref{apple63} and \eqref{apple64}, we obtain
			\begin{equation}
				\begin{split}
					\|u^{\pm 1, \lambda}(t) - u_{\pm 1, \lambda}(t)
					\|_{H^s(\rr)}
					\lesssim \lambda^{\frac{\delta -2}{4}}.
					\label{apple65}
				\end{split}
			\end{equation}
			Next, we will use estimate \eqref{apple65} to prove non-uniform
			dependence when $s > 1$.
%%%%%%%%%%%%% Behavior at time  t = 0  %%%%%%%%%%%% 
%
\subsection{Behavior at time $t=0$.}  We have
%
%
\begin{equation}
	\begin{split}
		\|u_{1,\lambda}(0) - u_{-1,\lambda}(0) \|_{H^s(\rr)} 
		& = \|u^{1,\lambda}(0) - u^{-1,\lambda}(0) \|_{H^s(\rr)}
		\\
		& = 2 \lambda^{-1} \| \tilde{\phi}\left( \frac{x}{\lambda^\delta}
		\right) \|_{H^s(\rr)}.
		\label{apple}
	\end{split}
\end{equation}
Applying the estimate
\begin{equation}
	\begin{split}
		\|\tilde{\phi}\left( \frac{x}{\lambda^\delta}
		\right)\|_{H^{k}(\rr)} \le
		\lambda^{\frac{\delta}{2}}\|\tilde{\phi}\|_{H^{k}(\rr)},
		\qquad k\ge 0,
	\end{split}
\end{equation}
and recalling that $1<\delta<2$, we conclude that
\begin{equation}
	\begin{split}
		\|u_{1,\lambda}(0) - u_{-1,\lambda}(0) \| \le 2
		\lambda^{\frac{\delta}{2}-1} \|\tilde{\phi} \|_{H^s(\rr)} \to 0
		\; \; \text{as} \; \; \lambda \to \infty.
		\label{apple70}
	\end{split}
\end{equation}
%
%
%%%%%%%%%%%%%% Behavior at time  t >0  %%%%%%%%%%%% 
%  
%
\subsection{Behavior at time  $t>0$.}  Using the reverse triangle inequality, we have
%
%
%
\begin{equation} 
	\label{appleHR-slns-differ-t-pos}
	\begin{split}
		\|
		u_{1,\lambda}(t)
		-
		u_{- 1,\lambda}(t)
		\|_{H^s(\rr)}
		&
		\ge
		\|
		u^{1,\lambda}(t)
		-
		u^{- 1,\lambda}(t)
		\|_{H^s(\rr)}
		\\
		&
		-
		\|
		u^{1,\lambda}(t)
		-
		u_{1,\lambda}(t)
		\|_{H^s(\rr)}
		\\
		&
		-
		\|
		-u^{-1,\lambda}(t)
		+
		u_{-1,\lambda}(t)
		\|_{H^s(\rr)} .
	\end{split}
\end{equation}
%
%
Using estimate \eqref{apple65} for the last two terms 
in \eqref{appleHR-slns-differ-t-pos} we obtain
%
%
%
\begin{equation} 
	\label{appleHR-slns-differ-t-pos-est}
	\|
	u_{1,\lambda}(t)
	-
	u_{- 1,\lambda}(t)
	\|_{H^s(\rr)}
	\ge
	\|
	u^{1,\lambda}(t)
	-
	u^{- 1,\lambda}(t)
	\|_{H^s(\rr)}
	-
	c \lambda^{\frac{\delta - 2}{4}}
\end{equation}
where c is a positive, non-zero constant. Letting $\lambda$ go to $\infty$ in
\eqref{appleHR-slns-differ-t-pos-est}
yields
%
\begin{equation} 
	\label{appleHR-slns-to-ap-est}
	\liminf_{n\to\infty}
	\|
	u_{1,\lambda}(t)
	-
	u_{- 1,\lambda}(t)
	\|_{H^s(\rr)}
	\ge
	\liminf_{n\to\infty}
	\|
	u^{1,\lambda}(t)
	-
	u^{- 1,\lambda}(t)
	\|_{H^s(\rr)}.
\end{equation}
%
%
Hence, by \eqref{appleHR-slns-to-ap-est}, we see we have reduced the problem of
analyzing the growth of the difference of actual solutions to the more
manageable problem of analyzing the growth of the associated approximate
solutions. Using the identity 
$$
\cos \alpha -\cos \beta
=
-2
\sin(\frac{\alpha + \beta}{2})
\sin(\frac{\alpha - \beta}{2})
$$
gives
\begin{equation}
	\label{apple80}
	\begin{split}
u^{1,\lambda}(t)
-
u^{- 1,\lambda}(t)
=
u_{\ell,1,\lambda}(t) - u_{\ell,-1,\lambda}(t) + 2\lambda^{-\frac{\delta}{2}-s}
\phi\left( \frac{x}{\lambda^\delta} \right)\sin(\lambda x) \sin(\gamma t).
\end{split}
\end{equation}
Now, by Lemma \ref{appleb}, we have
\begin{equation*}
	\begin{split}
	\|u_{\ell,-1,\lambda}(t) - u_{\ell,1,\lambda}(t)\|_{H^s(\rr)} \lesssim
	\lambda^{-1 + \frac{\delta}{2}};
	\end{split}
\end{equation*}
hence applying the reverse triangle inequality to \eqref{apple80}, we obtain
\begin{equation} 
	\label{apple90}
	\begin{split}
	& \|
	u^{1,\lambda}(t)
	-
	u^{- 1,\lambda}(t)
	\|_{H^s(\rr)}
	\\
	& \ge 2 \lambda^{-\frac{\delta}{2}-s} \|\phi\left(
	\frac{x}{\lambda^\delta} \right) \sin(\lambda x) \|_{H^s(\rr)} |\sin \gamma t|
	\\
	& - \|u_{\ell,-1,\lambda}(t) - u_{\ell,1,\lambda}(t)\|_{H^s(\rr)} 
	\\
	& \gtrsim \lambda^{-\frac{\delta}{2}-s} \|\phi\left(
	\frac{x}{\lambda^\delta} \right ) \sin(\lambda x) \|_{H^s(\rr)} |\sin \gamma t| -
	\lambda^{-1 + \frac{\delta}{2}}.
\end{split}
\end{equation}
%
%
Letting $\lambda$ go to $\infty$, Lemma \ref{applea}
with \eqref{apple90}  gives
%
%
\begin{equation} 
	\label{apple91}
	\liminf_{\lambda \to\infty}
	\|
	u^{1,\lambda}(t)
	-
	u^{- 1,\lambda}(t)
	\|_{H^s(\rr)}
	\gtrsim
	|\sin \gamma t|.
\end{equation}
Combining \eqref{appleHR-slns-to-ap-est} with \eqref{apple91}, we see that
\begin{equation}
	\begin{split}
		\liminf_{\lambda \to \infty} \|u_{1,\lambda}(t) -
		u_{-1,\lambda}(t) \|_{H^s(\rr)} \gtrsim |\sin
		\gamma t |, \qquad |t| \le T,
		\label{apple92}
	\end{split}
\end{equation}
proving \eqref{bdd-away-from-0}. Furthermore, a computation analogous
to that in \eqref{apple60}-\eqref{apple62} yields
\begin{equation}
	\label{apple93}
	\begin{split}
		\|u_{\pm 1, \lambda} (t) \|_{H^{s}(\rr)}
		& \lesssim 1.
	\end{split}
\end{equation}
Collecting \eqref{apple70}, \eqref{apple92}, and \eqref{apple93}, we
conclude that we have proven Theorem \ref{hr-non-unif-dependence} for the
non-periodic case.
	%
	%%%%%%%%%%%%%%%%%%%%%%%%%%%%%%%%%%
	%
	%
	%
	%             Proof of Theorem in the Periodic case
	%
	%
	%
	%%%%%%%%%%%%%%%%%%%%%%%%%%%%%%%%%%
	%
	\section{Proof of Non-Uniform Dependence in the Periodic case} 
	%
	%
	%
	%
	We will consider approximate solutions of form
	\begin{equation}
		\label{approx-solutions-form}
		u^{\omega,n}(x,t) = \omega n^{-1} + n^{-s} \cos \left( nx - \gamma \omega t
		\right) 
	\end{equation}
	with integer valued $n >>1$, bounded $\omega \in \rr$, and a fixed constant $\gamma \in \rr$. We rewrite 
	the HR i.v.p as
	\begin{align}
			 \p_t u + \gamma u \p_x u \ + & \ \p_x (1 - \p_x^2)^{-1} 
			 \left[ \frac{3 - \gamma}{2}u^2 +
			\frac{\gamma}{2}(\p_x u)^2 \right] = 0,
			\label{hyperelastic-rod-equation}
			\\
			& {u(x,0) = u_0(x)},
			\label{init-cond}
			\end{align}
	and aim to substitute our approximate solution into the left hand side 
	in order to obtain a functional representation of its error. Hence, some 
	preliminary calculations are necessary. We omit the superscripts $w,n$ for clarity: 
	\begin{equation}
		\begin{split}
			 \p_t u
			 & = \gamma \omega n^{-s} \sin\left( nx - \gamma \omega t \right),
			\\
			 \p_x u
			 & = -n^{-s+1} \sin(nx - \gamma \omega t),
			\\
			\gamma u \p_x u
			& = - \gamma \omega n^{-s} \sin\left( nx - \gamma
			\omega t \right) - \frac{\gamma}{2}n^{-2s+1}\sin\left( 2\left( nx - \gamma
			\omega t
			\right) \right),
			\\
			u^2 
			& = \omega^2 n^{-2} + 2\omega n^{-s -1} \cos \left( nx - \gamma \omega t
			\right) + n^{-2s} \cos^2\left( nx - \gamma \omega t \right),
			\\
			(\p_x u)^2 
			& = n^{-2s+2} \sin^2\left( nx - \gamma \omega t \right).
			\label{calculation of functional representation of error}
		\end{split}
	\end{equation}
	Using these relations, we obtain
	\begin{equation}
		\begin{split}
			\p_t u + \gamma u \p_x u
			& + \p_x(1- \p_x^2)^{-1} \left[
			\frac{3-\gamma}{2}u^2 + \frac{\gamma}{2}(\p_x u)^2 \right]
			\\
			& = \cancel{\gamma \omega n^{-s} \sin(nx - \gamma \omega t)} -
			\cancel{\gamma \omega n^{-s}\sin\left( nx - \gamma \omega t \right)}
			\\
			& -
			\frac{\gamma}{2}n^{-2s+1}\sin\left( 2\left( nx - \gamma \omega t \right)
			\right)
			\\
			& + \p_x \left( 1-\p_x^2 \right)^{-1}\bigg[ \frac{3-\gamma}{2} \bigg (
			\omega^2 n^{-2} + 2 \omega n^{-s -1}\cos( nx - \gamma \omega t )
			\\
			& + n^{-2s}\cos^2\left( nx - \gamma \omega t \right) \bigg ) + \frac{\gamma}{2}
			n^{-2s+2}\sin^2\left( nx - \gamma \omega t \right)
			\bigg]
			\\
			& \doteq E.
			\label{functional-representation-of-error}
		\end{split}
	\end{equation}
	Since $\frac{(3-\gamma)}{2}w^2 n^{-2}$ is a constant, it's derivative vanishes;
	hence we can rewrite the error $E$ as
	\begin{equation}
		\begin{split}
			E= E_1 + E_2 + E_3 + E_4
			\label{57}
		\end{split}
	\end{equation}
	where
	\begin{align}
		\label{90*}
			& E_1 =
			- \frac{\gamma}{2}n^{-2s+1}\sin\left[ 2\left( nx - \gamma 
			\omega t \right)
			\right],
			\\
			\label{90**}
			& E_2 = \p_x \left( 1-\p_x^2 \right)^{-1}\bigg[ \frac{3-\gamma}{2} \bigg (
			2 \omega n^{-s -1}\cos( nx - \gamma \omega t )
			\bigg )
			\bigg ],
			\\
			\label{90***}
			& E_3 = \p_x \left( 1-\p_x^2 \right)^{-1}\bigg[ 
			\frac{3-\gamma}{2} \bigg (
			 n^{-2s}\cos^2\left( nx - \gamma \omega t \right) \bigg )
			\bigg ],
			\\
			& E_4 = \frac{\gamma}{2}
			n^{-2s+2}\sin^2\left( nx - \gamma \omega t \right).
			\label{90}
	\end{align}
%
%
%
\noindent
\subsection{Estimate for the  Error of the Approximate Solutions.}
%
%
First, we will need the following two lemmas:
%
%
%
	 \begin{proposition}
		 \label{1n}
		 For nonzero $k \in \rr$, we have
		 \begin{equation}
			 \begin{split}
				 \|\sin(k(nx-c))\|_{H^\sigma(\ci)} \simeq n^\sigma.
				 \label{1m}
			 \end{split}
		 \end{equation}
		The same result holds with $\sin$ replaced by $\cos$.
	\end{proposition}
		%
    \begin{proof} The Fourier transform of $\psi_n(x) = \sin[k(nx-c)]$
		is
		\begin{equation*}
			\begin{split}
				\widehat{\psi_n}(\xi)
				& = \int_0^{2\pi} e^{-ix \xi} \sin [k(nx-c)]
				\ dx
				\\
				& = \int_0^{2\pi} e^{-ix \xi} \left( \frac{e^{ik(nx-c)} -
				e^{-ik(nx-c)}}{2i} \right) \ dx
				\\
				& = \frac{1}{2i} \int_0^{2\pi} e^{i[x(kn- \xi)-kc]} \ dx
				- \frac{1}{2i}\int_0^{2\pi} e^{-i[x(kn+\xi) - kc]} \ dx.
			\end{split}
		\end{equation*}
		Therefore
		\begin{equation*}
			\begin{split}
				\widehat{\psi_n}(\xi) =
				\begin{cases}
					- i \pi e^{-ikc}, \qquad & \xi = kn\\
					i \pi e^{ikc}, \qquad & \xi = -kn\\
					0,  \qquad & \xi \neq \pm kn.
				\end{cases}
			\end{split}
		\end{equation*}
		Hence
		\begin{equation*}
			\begin{split}
				\|\psi_n\|_{H^\sigma(\ci)}^2
				& = \sum_{\xi \in \zz}
				(1+\xi^2)^{\sigma} \widehat{\psi_n}(\xi) 
				\\
				& = i \pi (1+k^2 n^2)^{\sigma} (e^{ikc} - e^{-ikc})
				\\
				& \simeq n^{2 \sigma}, \qquad n>>1
			\end{split}
		\end{equation*}
		from which we obtain \eqref{1m}. Furthermore, since
		\begin{equation*}
			\begin{split}
				\cos[k(nx-c)]
				&= \sin[k(nx-c)- \pi/2] \\
				& = \sin\{k[nx - (c + \pi/2k)]\},
			\end{split}
		\end{equation*}
		we conclude \eqref{1m} holds when $\sin$ is replaced by $\cos$. This
    completes the proof. 
  \end{proof}
	%	
		\begin{proposition}
			\label{2n}
			For nonzero $k \in \rr$, $n >>1$,
			\begin{equation}
				\label{2m}
				\begin{split}
					\|\sin^2[k(nx-c)] \|_{H^\sigma(\ci)} \simeq
					\begin{cases}
					1, \qquad & \sigma \le 0
					\\
					n^\sigma,\qquad &\sigma > 0.
				\end{cases}
				\end{split}
			\end{equation}
			The same result holds with $\sin$ replaced by $\cos$.
		\end{proposition}
		\begin{proof} The Fourier transform of $\psi_n(x) = \sin^2[k(nx-c)]$
		is
		\begin{equation*}
			\begin{split}
				\widehat{\psi_n}(\xi) 
				& = \int_0^{2\pi} e^{-ix \xi} \sin^2[k(nx-c)] \ dx
				\\
				& = \int_0^{2\pi} e^{-ix \xi} \left( \frac{e^{ik(nx-c)} -
				e^{-ik(nx-c)}}{2i} \right)^2 \ dx
				\\
				& = -\frac{1}{4} \int_0^{2\pi} e^{-ix \xi} (e^{2ik(nx-c)} +
				e^{-2ik(nx-c)} -2) \ dx
				\\
				& = -\frac{1}{4} \int_0^{2\pi} e^{ix(2kn - \xi) - 2ikc} \
				dx - \frac{1}{4} \int_0^{2\pi} e^{-ix(2kn + \xi) + 2ikc} \ dx
				\\
				& + \frac{1}{2} \int_0^{2\pi} e^{-ix \xi} \ dx.
			\end{split}
		\end{equation*}
	Hence, for nonzero $k \in \rr$ and $n >>1$ we have
	\begin{equation*}
		\begin{split}
			\widehat{\psi_n}(\xi) = 
			\begin{cases}
				-\frac{\pi}{2}e^{-2ikc}, \qquad & \xi=2kn
				\\
				-\frac{\pi}{2}e^{2ikc}, \qquad & \xi = -2kn
				\\
				\pi, \qquad & \xi = 0
				\\
				0, \qquad & \xi \neq 0, \ \pm 2kn.
			\end{cases}
		\end{split}
	\end{equation*}
	which gives
	\begin{equation*}
		\begin{split}
			\|\psi_n\|_{H^\sigma(\ci)}^2 
			& = \sum_{\xi \in \zz} (1+ \xi^2)^\sigma \widehat{\psi_n}(\xi)
			\\
			& = -\frac{\pi}{2}(1+4k^2n^2)^\sigma (e^{2ikc} + e^{-2ikc}) +
			\pi
			\\
			& \simeq 
			\begin{cases}
				1,  \qquad & \sigma \le 0 
				\\
				n^{2 \sigma}, &  \sigma > 0
			\end{cases}
		\end{split}
	\end{equation*}
	from which \eqref{2m} follows. Since $\sin^2[k(nx-c)] = 1-
	\cos^2[k(nx-c)]$, \eqref{2m} holds with $\sin$ replaced by
	$\cos$. 
\end{proof}
	%
	\subsection{An Estimate for $\hyperref[90*]{E_1}$.}
	We apply Proposition \ref{1n} to obtain
	\begin{equation}
		\label{85}
		\begin{split}
			\|E_1\|_{H^\sigma(\ci)}
			& = 
			\left\| - \frac{\gamma}{2}n^{-2s+1}\sin\left( 2\left(
			nx - \gamma \omega t \right)\right )
			\right\|_{H^\sigma(\ci)}
			\\
			& \lesssim
			n^{-2s + \sigma + 1}
		\end{split}
	\end{equation}
	\subsection{An Estimate for $\hyperref[90**]{E_2}$.}
	We will need the following:
	%
	\begin{remark}
		\label{lem:operator-norm-lemma}
		For $u \in L^2(\ci)$ and arbitrary $k$, we have
		\begin{equation}
			\begin{split}
				\|\p_x (1 -\p_x^2)^{-1}u \|_{H^{k}(\ci))} \le
				\|u\|_{H^{k-1}(\ci)}.
				\label{operator norm of pseudo-diff operator we use}
			\end{split}
		\end{equation}
	\end{remark}
	\begin{proof} Let $u \in L^2(\ci)$. Then
	\begin{equation*}
		\begin{split}
			\|\p_x \left( 1- \p_x^2 \right)^{-1} u \|_{H^{k}(\ci)}
			& = \sum_{\xi \in \zz}  \left[ \xi\left( 1+\xi^2 \right)^{-1} \right]^2
			\cdot \left( 1 + \xi^2 \right)^{k} \cdot |\hat{u}(\xi)|^2 
			\\
			& \le \sum_{\xi \in \zz}  \left( 1+ \xi^2 \right)^{k-1} \cdot 
			|\hat{u}(\xi)|^2 
			\\
			& \le \|u\|_{H^{k-1}(\ci)}.
			\qquad \Box
		\end{split}
	\end{equation*}
	Applying Remark \ref{lem:operator-norm-lemma}, we obtain
		\begin{equation}
		\begin{split}
			 \|E_2\|_{H^\sigma(\ci)} & = \left \|\p_x(1-\p_x^2)^{-1}
			\left[ \frac{3-\gamma}{2}
			\left( 2 \omega n^{-s -1}\cos( nx - \gamma \omega t )
			\right) \right] \right \|_{H^\sigma(\ci)}
			\\
			& \le \left |\frac{3-\gamma}{2}\right |
			\left \|2 \omega n^{-s -1}\cos( nx - \gamma \omega t )
			\right \|_{H^{\sigma -1 }(\ci)}
			\label{non-local_term_first_piece_without_constant}
		\end{split}
	\end{equation}
	which by Proposition \ref{1n} gives
		\begin{equation}
			\label{3.10}
		\begin{split}
			\|E_2\|_{H^\sigma(\ci)}
			& \lesssim n^{-s + \sigma -2}.
		\end{split}
	\end{equation}
	%
  This completes the proof. 
\end{proof}
\subsection{An Estimate for $\hyperref[90]{E_3}$.}
Applying Remark \ref{lem:operator-norm-lemma}, we obtain
		\begin{equation}
		\begin{split}
			\|E_3\|_{H^\sigma(\ci)} & = \left \|\p_x(1-\p_x^2)^{-1}
			\left[ \frac{3-\gamma}{2}\left( n^{-2s}\cos^2\left( nx - \gamma 
			\omega
			t\right)\right) \right] \right \|_{H^\sigma(\ci)}
			\\
			& \le \left |\frac{3-\gamma}{2}\right |
			\left \| n^{-2s}\cos^2\left( nx - \gamma \omega t \right)
			\right \|_{H^{\sigma -1 }(\ci)}
			\end{split}
	\end{equation}
which by Proposition \ref{2n} gives
		\begin{equation}
			\label{yuoo}
		\begin{split}
			\|E_3\|_{H^\sigma(\ci)}
			& \lesssim 
			\begin{cases}
				n^{-2s}, \qquad & \sigma \le 1  \\
				n^{-2s +\sigma -1}, \qquad & \sigma
				> 1.
			\end{cases}
		\end{split}
	\end{equation}
	\subsection{An Estimate for $\hyperref[90]{E_4}$.}
	We now apply Remark \ref{lem:operator-norm-lemma} and Proposition 
	\ref{2n} to obtain
	\begin{equation*}
		\begin{split}
			\|E_4\|_{H^\sigma(\ci)}
			& = \Big \|\p_x \left( 1 - \p_x^2 \right)^{-1} 
			\frac{\gamma}{2}n^{-2s+2}\sin^2\left(
			nx - \gamma \omega t \right) \Big \|_{H^\sigma(\ci)}
			\\
			& \leq \left | \frac{\gamma}{2} \right |  \| 
			n^{-2s+2}\sin^2\left( nx - \gamma \omega t
			\right)\|_{H^{\sigma -1}(\ci)}
			\end{split}
	\end{equation*}
	which by Proposition \ref{2n} gives
	\begin{equation}
			\label{estimate_for_second_piece_of_non_local_final}
		\begin{split}
			\|E_4\|_{H^\sigma(\ci)} \lesssim 
			\begin{cases}
				n^{-2s+2}, \qquad & \sigma \le 1
				\\
				n^{-2s+ \sigma + 1}, \qquad & \sigma >1.
			\end{cases}
		\end{split}
	\end{equation}
	Grouping estimates \eqref{85}, \eqref{3.10}, \eqref{yuoo}, and
	\eqref{estimate_for_second_piece_of_non_local_final} we see that for bounded $\omega$ and $n >> 1$
	we have the following upper bound for the $H^\sigma(\ci)$ error
	of our approximate solutions:
	\begin{equation*}
		\begin{split}
			\|E(t)\|_{H^{\sigma}(\ci)} \lesssim 
			  \begin{cases}
				  n^{-s-1} + n^{-2s +2}, \qquad & \sigma \le 1
				  \\
				  n^{-s + \sigma - 2} + n^{-2s + \sigma + 1}, \qquad &
				  \sigma > 1.
			  \end{cases}
		\end{split}
	\end{equation*}
		Noting that $\|E(t)\|_{H^\sigma(\ci)}$ blows up as $\sigma$ 
		increases,
	regardless of the choice of $s$, we restrict our attention to the case $\sigma \le 1$
	and obtain the following:
	  \begin{lemma}
		  \label{lem:error_of_approx_solution}
		  Let $u^{\omega,n}$ be an approximate solution to the HR i.v.p., 
		  with $\sigma \le 1$,  $\omega$ bounded, and $n >> 1$.
		  Then for the error $E$ we have
		  \begin{equation}
			  \begin{split}
				  \|E(t)\|_{H^\sigma(\ci)} \lesssim n^{-r_s}
				  \label{total-error-approx-solution}
			  \end{split}
		  \end{equation}
		  where
		  \begin{equation}
			  \begin{split}
			r_s = 
			\begin{cases}
				2(s-1)   & \text{if} \quad s \le 3,\\  
				s+1  & \text{if} \quad s > 3. \\
			\end{cases}
			\label{r_s-definition}
			  \end{split}
		  \end{equation}
	  \end{lemma}
	 % 
	 %%%%%%%%%%%%%%%%%%%%%%%%%%%%%%%%%
	 %
	 %
	 %
	 %   Proof of  Theorem in periodic case for s between 3/2 and 2
	 %
	 %
	 %
	 %%%%%%%%%%%%%%%%%%%%%%%%%%%%%%%%%%%
	 \subsection{A Critical Lemma.}
	We wish now to estimate the difference between approximate and actual solutions to
	the HR i.v.p with common initial data $u_0 \in H^s$. To do so, we must first
	establish the existence and lifespan of solutions $u(x,t)$ with initial data $u_0$.
	We have the following:
\begin{theorem}
	\label{thm:HR_existence_continous_dependence}
For  $s>3/2$  the following  results  hold:
%
\noindent
(i) If $u_0\in H^s(\ci)$  then  there exists a unique solution to
the Cauchy problem  \eqref{hr}--\eqref{hr-data} 
  in $C([-T, T]; H^s(\ci))$, where the life-span  $T$ depends on the size
  of the initial data $u_0$, that is
  $T=T(\|u_0 \|_{H^s(\ci)})$.
  \noindent
(ii)
 The flow  map $u_0 \to u(t)$  is continuous from
 bounded sets of $H^s(\ci)$ into $C([-T, T]; H^s(\ci))$.
%
 \noindent
\\
(iii)  The  lifespan $T$ satisfies the lower bound estimate 
%
     \begin{equation}
   \label{Life-span-est}
T
\ge
\frac{1}{2c_s}
\frac{1}{\|
u_0
  \|_{H^s(\ci)}},
   \end{equation}
   %
and the solution $u$ satisfies the estimate
%
     \begin{equation}
   \label{u_x-Linfty-Hs}
\|
u(t)
  \|_ {H^s(\ci))}
  \le
  2
  \|
u_0
  \|_{H^s(\ci)},\,\, |t|\le T.
   \end{equation}
   %
 \end{theorem}
%
A proof of these results is provided in the appendix.
%
%
%
Let $v=u^{\omega,n} -
u_{\omega,n}$, where $u_{\omega,n}$ denotes a solution to
the Cauchy-problem \eqref{hyperelastic-rod-equation}-\eqref{init-cond} with
initial data $u_0(x) = u^{\omega,n}(x,0)$. We are now prepared to prove the following critical lemma:
\begin{lemma}
	\label{lem:bound_for_difference-of-approx-actual-soln}
	If \ $s > 3/2 $ and $\sigma = 1/2 + \ee$ for an appropriately
	chosen $\ee = \ee(s) > 0$, then 
			\begin{equation} 
				\|
				v(t)
				\|_{H^\sigma(\ci)}
				\doteq
				\label{differ-Hsigma-est} 
				\|
				u^{\omega, n}(t) 
				- 
				u_{\omega, n}(t)
				\|_{H^\sigma(\ci)}
				\lesssim 
				n^{-r_s}, 
				\quad
				|t| \le T.
			\end{equation}
			\end{lemma}
      \begin{proof} Recall the HR Cauchy problem
\begin{align}
	\label{1.1}
	&\p_t u  = -\gamma u \p_x u - \p_x\left( 1-\p_x^2
		\right)^{-1}\left[ \frac{3-\gamma}{2}
		u^2 + \frac{\gamma}{2}\left( \p_x
		u
		\right)^2 \right],
		\\
		\label{1.2}
		& u(x,0) = u_0.
\end{align}
and its approximate solutions of form
	\begin{equation}
		\label{approx-solns-form}
		u^{\omega,n}(x,t) = \omega n^{-1} + n^{-s} \cos \left( nx - \gamma \omega t
		\right) 
	\end{equation}
	with integer valued $n >>1$, bounded $\omega \in \rr$, and a fixed 
	constant $\gamma \in \rr$.	Then the approximate solution 
	$u^{\omega,n}$ satisfies the equation
\begin{equation}
	\begin{split}
		\label{1.4}
		\p_t u^{\omega,n} = E - \gamma u^{\omega,n} \p_x u^{\omega,n} 
		- \p_x\left( 1-\p_x^2 \right)^{-1} \left[
		\frac{3-\gamma}{2} \left( u^{\omega,n} \right)^2 +
		\frac{\gamma}{2}\left( \p_x u^{\omega,n} \right)^2 \right].
	\end{split}
\end{equation}
Let $u_{\omega,n}$ denote the solution to the i.v.p
\begin{align}
	\label{1.5}
	&\p_t u_{\omega,n}  = -\gamma u_{\omega,n} \p_x u_{\omega,n} - 
	\p_x\left( 1-\p_x^2
		\right)^{-1}\left[ \frac{3-\gamma}{2}\left(
		u_{\omega,n} \right)^2 + \frac{\gamma}{2}\left( \p_x
		u_{\omega,n}
		\right)^2 \right],
		\\
		& u_{\omega,n}(x,0) = u^{\omega,n}(x,0).
\end{align}
Subtracting \eqref{1.5} from \eqref{1.4}, we see that the
difference $v = u^{\omega,n} - u_{\omega,n}$ satisfies the i.v.p
\begin{align}
		\label{1.7}
		& \p_t v  =  E - \frac{\gamma}{2} \p_x
		\left[ \left( u^{\omega,n} + u_{\omega,n} \right)v \right]
		 \notag
		\\
		& - \p_x(1-\p_x^2)^{-1} \left[
		\frac{3-\gamma}{2} \left( u^{\omega,n} + u_{\omega,n}
		\right) v +
		\frac{\gamma}{2}\left( \p_x u^{\omega,n} +
		\p_x u_{\omega,n}
		\right) \p_x v
		\right], 
		\\
		& v(x,0)=0.
\end{align}
Applying $D^\sigma$ to both sides of \eqref{1.7}, multiplying by
$D^\sigma v$, and integrating, we obtain the
relation
\begin{equation}
	\begin{split}
		\frac{1}{2}\frac{d}{dt}\|v(t)\|_{H^\sigma(\ci)}^2
		& = \int_{\ci} D^\sigma E \cdot D^\sigma
		v \ dx
		\\
		& - \frac{\gamma}{2}\int_{\ci} D^\sigma
		\p_x \left[ \left( u^{\omega,n} + u_{\omega,n} \right)v
		\right]\cdot D^\sigma v \ dx
		\\
		& - \frac{3-\gamma}{2}\int_{\ci} D^{\sigma
		-2} \p_x \left[ \left( u^{\omega,n} + u_{\omega,n}
		\right)v \right] \cdot D^\sigma v \ dx
		\\
		& - \frac{\gamma}{2}\int_{\ci} D^{\sigma
		-2}
		\p_x \left[ \left( \p_x u^{\omega,n} + \p_x u_{\omega,n}
		\right)\cdot \p_x v \right] \cdot
		D^\sigma v \ dx.
		\label{X}
	\end{split}
\end{equation}
We now estimate each term of the right hand side
of \eqref{X}.
\textbf{Estimate for Term 1.} Applying Cauchy-Schwartz, we obtain
\begin{equation}
	\begin{split}
	\left |\int_{\ci} D^\sigma E \cdot D^\sigma v \ dx \right |
		& \le \|D^\sigma E \cdot D^\sigma v \|_{L^1(\ci)}
		\\
		& \le \|E\|_{H^\sigma(\ci)} \|v\|_{H^\sigma(\ci)}.
		\label{est_for_1}
	\end{split}
\end{equation}
%
\textbf{Estimate for Term 2.} We can rewrite
\begin{equation}
	\begin{split}
		-\frac{\gamma}{2} \int_{\ci} D^\sigma \p_x \left[ \left( u^{\omega,n} + u_{\omega,n}
		\right)v \right] \cdot D^\sigma v \ dx
		 = & -\frac{\gamma}{2}\int_{\ci} \left[ D^\sigma \p_x , u^{\omega,n} + u_{\omega,n}
		\right]v \cdot D^\sigma v \ dx
		\\
		& - \frac{\gamma}{2} \int_{\ci} (u^{\omega,n} + u_{\omega,n})
		D^\sigma \p_x v \cdot
		D^\sigma v \ dx.
		\label{est_for_2}
	\end{split}
\end{equation}
We now estimate \eqref{est_for_2} in parts. For the first term, we have
\begin{equation}
	\begin{split}
		\left | \frac{\gamma}{2} \int_{\ci} (u^{\omega,n} + u_{\omega,n})
		D^\sigma \p_x v \cdot
		D^\sigma v \ dx \right |
		& = \bigg | \frac{\gamma}{4}\int_{\ci} (u^{\omega,n} +
		u_{\omega,n}) \cdot \p_x (D^\sigma v)^2 \ dx \bigg |
		\\
		& = \left | -\frac{\gamma}{4} \int_{\ci} \p_x(u^{\omega,n} + u_{\omega,n}) \cdot
		(D^\sigma v)^2  \ dx \right |
		\\
		& \lesssim \|\p_x(u^{\omega,n} + u_{\omega,n}) \|_{L^\infty(\ci)}
		\|v\|_{H^\sigma(\ci)}^2.
		\label{2'}
	\end{split}
\end{equation}
To deal with the remaining term, we will need the following commutator
estimate:
\begin{theorem}
	\label{thm10}
	Assume $1<p<\infty$, $m \ge 0$. Consider a pseudo-differential operator $P
	\in \Psi^m$; then for $\rho >n/p + 1$, $s \ge 0$, and $s+m \le \rho$ we
	have
	\begin{equation}
		\begin{split}
			\|[P,f]v\|_{H^{s,p}} \le C \|f\|_{H^{\rho,p}}
			\|v\|_{H^{s+m-1,p}}.
			\label{5}
		\end{split}
	\end{equation}
\end{theorem}
%
A proof of \eqref{5} can be found in \cite{Taylor_2003_Commutator-esti}. We also have the following
corollary:
\begin{corollary}
	\label{cor1}
If $\rho > 3/2$ and $0 \le \sigma + 1 \le \rho$, then
\begin{equation}
	\begin{split}
		\|[D^\sigma \p_x ,f]v\|_{L^2(\ci)} \le C \|f\|_{H^\rho} \|v\|_{H^\sigma}.
		\label{15}
	\end{split}
\end{equation}
\end{corollary}
\begin{proof} It follows from Theorem \ref{thm10} by setting $n=1$, $p=2$,
$s=0$, and noting  that $D^\sigma \p_x \in
\Psi^{\sigma + 1}$. 
\end{proof}
Let $\sigma = 1/2 + \ee$ and $\rho = 3/2 + \ee$, where $\ee > 0$ is
arbitrarily small. Then
applying Corollary \ref{cor1}, we obtain
\begin{equation}
	\begin{split}
		\|[D^\sigma \p_x, u^{\omega,n} + u_{\omega,n}]v\|_{L^2(\ci)} \le C \|u^{\omega,n} + u_{\omega,n}
		\|_{H^{\rho}(\ci)} \|v\|_{H^\sigma(\ci)}.
		\label{6}
	\end{split}
\end{equation}
Applying Cauchy-Schwartz and estimate \eqref{6} gives
\begin{equation}
	\begin{split}
		\left | -\frac{\gamma}{2} \int_{\ci} [D^\sigma \p_x , u^{\omega,n} + u_{\omega,n}]v
		\cdot D^\sigma v \ dx \right | \lesssim \|u^{\omega,n} +
		u_{\omega,n}\|_{H^{\rho}(\ci)} \|v\|_{H^\sigma(\ci)}^2.
		\label{7}
	\end{split}
\end{equation}
Combining estimates \eqref{2'} and \eqref{7} we conclude that
\begin{equation}
	\begin{split}
		& \left | -\frac{\gamma}{2} \int_{\ci} D^\sigma \p_x \left[ \left( u^{\omega,n} + u_{\omega,n}
		\right)v \right]  \cdot D^\sigma v \ dx \right |
		\\
		& \lesssim (\|u^{\omega,n} + u_{\omega,n}\|_{H^{\rho}(\ci)} 
		 + \|\p_x u^{\omega,n} +
		\p_x u_{\omega,n}\|_{L^\infty(\rr)} ) \cdot \|v\|_{H^\sigma(\ci)}^2.
		\label{8}
	\end{split}
\end{equation}
%
\subsection{Estimate for Term 3.} Using Cauchy-Schwartz, and recalling that
$\sigma = 1/2 + \ee$,  we obtain
\begin{equation}
	\begin{split}
		\bigg | -\frac{3-\gamma}{2} \int_{\ci} D^{\sigma -2} \p_x \left[
		(u^{\omega,n} + u_{\omega,n})v \right]
		& \cdot D^\sigma v \ dx \bigg |
		\\
		& \lesssim
		\|D^{\sigma -2 } \p_x [(u^{\omega,n} + u_{\omega,n})v] \cdot D^\sigma v
		\|_{L^1(\ci)}
		\\
		& \lesssim \|D^{\sigma -2 } \p_x [(u^{\omega,n} +
		u_{\omega,n})v\|_{L^2(\ci)} \cdot \|D^\sigma v \|_{L^2(\ci)}
		\\
		& \lesssim \|(u^{\omega,n} + u_{\omega,n})v \|_{H^{\sigma -1 }(\ci)}
		\|v\|_{H^\sigma(\ci)}
		\\
		& \lesssim \|(u^{\omega,n} + u_{\omega,n})v \|_{L^2(\ci)} \|v\|_{H^\sigma(\ci)}
		\\
		& \lesssim \|u^{\omega,n} + u_{\omega,n} \|_{L^\infty(\ci)} \|v\|_{H^\sigma(\ci)}^2.
		\label{9}
	\end{split}
\end{equation}
%
\subsection{Estimate for Term 4.}
We will need the following:
\begin{lemma}
	\label{impo}
	For $1/2 < \sigma < 1 $, $f,g \in \mathcal{S'}$,
	\begin{equation}
		\begin{split}
			\|fg\|_{H^{\sigma - 1}} \le C \|f\|_{H^{\sigma}}
			\cdot \|g\|_{H^{\sigma -1}}.
			\label{11}
		\end{split}
	\end{equation}
\end{lemma}
%
	Hence, applying Cauchy-Schwartz and Lemma \ref{impo}, we obtain
	\begin{equation}
		\begin{split}
			& \left | -\frac{\gamma}{2} \int_{\ci} D^{\sigma -2 } \p_x \left[
			\left( \p_x u^{\omega,n} + \p_x u_{\omega,n} \right) \cdot \p_x v
			\right] \cdot D^\sigma v \ dx \right |
			\\
			& \lesssim \|D^{\sigma -2} \p_x [(\p_x u^{\omega,n} + \p_x
			u_{\omega,n}) \cdot \p_x v]\|_{L^2(\ci)} \cdot \|D^\sigma v
			\|_{L^2(\ci)}
			\\
			& \lesssim \|(\p_x u^{\omega,n} + \p_x u_{\omega,n}) \cdot \p_x v
			\|_{H^{\sigma -1}(\ci)} \| \cdot \|v\|_{H^\sigma (\ci)} 
			\\
			& \lesssim \|\p_x u^{\omega,n} + 
			\p_x u_{\omega,n}
			\|_{H^\sigma(\ci)} \|v\|_{H^\sigma(\ci)}^2.
			\label{12}
		\end{split}
	\end{equation}
Collecting estimates \eqref{est_for_1}, \eqref{8}, \eqref{9}, and
\eqref{12}, and applying the Sobolev Imbedding Theorem, we deduce
\begin{equation}
	\begin{split}
		\frac{1}{2}\frac{d}{dt} \|v\|_{H^\sigma(\ci)}^2
		& \lesssim
		(\|u^{\omega,n} + u_{\omega,n}\|_{H^{\rho}(\ci)} +
		\|\p_x(u^{\omega,n} + u_{\omega,n}) \|_{H^\sigma(\ci)})
		\cdot \|v\|_{H^\sigma(\ci)}^2
		\\
		& + \|E\|_{H^\sigma(\ci)}
		\|v\|_{H^\sigma(\ci)}.
		\label{10}
	\end{split}
\end{equation}
We now estimate the right hand side of \eqref{10} in parts. Applying
Proposition \ref{1n} and Theorem \ref{thm:HR_existence_continous_dependence} 
gives
\begin{equation}
	\begin{split}
		 \|u^{\omega,n} + u_{\omega,n} \|_{H^\rho(\ci)}
		 & \le
		\|u^{\omega,n}\|_{H^\rho(\ci)} + \|u_{\omega,n}\|_{H^\rho(\ci)}
		\\
		& \lesssim n^{\rho -s} + \|u^{\omega,n}(0)\|_{H^\rho(\ci)}
		\\
		& \lesssim n^{\rho -s}
		\label{3r}
	\end{split}
\end{equation}
and
\begin{equation}
	\begin{split}
		\|\p_x(u^{\omega,n} + u_{\omega,n}) \|_{H^\sigma(\ci)} 
		& \le \|u^{\omega,n} + u_{\omega,n}\|_{H^{\sigma + 1}(\ci)}
		\\
		& = \|u^{\omega,n} + u_{\omega,n}\|_{H^{\rho}(\ci)}
		\\
		& \le \|u^{\omega,n} \|_{H^{\rho}(\ci)} + \|u_{\omega,n}
		\|_{H^{\rho}(\ci)}
		\\
		& \lesssim n^{\rho -s} + \|u^{\omega,n}(0)\|_{H^{\rho}(\ci)}
		\\
		& \lesssim n^{\rho -s}.
		\label{4r}
	\end{split}
\end{equation}
Applying Lemma \ref{lem:error_of_approx_solution}, and
substituting \eqref{total-error-approx-solution}, \eqref{3r},
and \eqref{4r} into \eqref{10}, we obtain
\begin{equation}
	\begin{split}
		\frac{1}{2}\frac{d}{dt}\|v\|_{H^\sigma(\ci)}^2 \lesssim n^{\rho - s}
		\|v\|_{H^\sigma(\ci)}^2 + n^{-r_s}\|v\|_{H^\sigma(\ci)}.
		\label{200r}
	\end{split}
\end{equation}
Setting $y(t) = \|v(t)\|_{H^\sigma(\ci)}$  in \eqref{200r}, derivating the
left-hand-side, and dividing through by $y(t)$ yields
\begin{equation}
	\label{sub8}
	\frac{d}{dt} y(t) \le cn^{\rho - s}y + cn^{-r_s}, \quad c\in \rr^{+}.
\end{equation}
We multiply both sides of \eqref{sub8} by the integrating factor $e^{-tcn^{\rho - s}}$,
which gives
\begin{equation*}
	e^{-tcn^{\rho - s}}\frac{d}{dt} y(t) \le cn^{\rho - s} e^{-tcn^{\rho - s}}y
	+ cn^{-r_s}e^{-tcn^{\rho - s}}.
\end{equation*}
It follows that
\begin{equation*}
	\frac{d}{dt}\left (e^{-tcn^{\rho - s}} y \right ) \le cn^{-r_s}e^{-tcn^{\rho - s}} .
\end{equation*}
Hence,
\begin{equation*}
	\begin{split}
	\int_0^t  \frac{d}{d \tau} \left[ e^{-\tau cn^{\rho - s}} y(\tau)
	\right] \; d \tau
	& \le \int_0^t  c n^{-r_s} e^{-\tau cn^{\rho - s}}  \; d \tau
	\\
	& \le \int_0^t cn^{-r_s} \; d \tau
\end{split}
\end{equation*}
from which we obtain
\begin{equation}
	\label{almost8}
	e^{-tcn^{\rho - s}} y(t) - y(0) \le ctn^{-r_s}.
\end{equation}
Noting that $y(0)=0$, we can simplify \eqref{almost8} to obtain
\begin{equation}
	\label{gronwall-ineq8}
	y(t) \le ctn^{-r_s} e^{tcn^{\rho - s}}.
\end{equation}
Substituting back in $\|v(t)\|_{H^\sigma(\ci)}$ for $y$, we see that we are assured the
decay of $\|v(t)\|_{H^\sigma(\ci)}$ only when $s \ge \rho$. Recall that
we previously set $\sigma = 1/2 + \ee$, $\rho = 3/2 + \ee$, where $\ee$ was
chosen to be arbitrarily small. Hence, we conclude that for $s>3/2$
\begin{equation}
	\label{en-est-fin!8}
	\|v(t)\|_{H^\sigma(\ci)} 
	\lesssim
	n^{-r_s}, \quad |t| \le T,\quad n>>1 . 
\end{equation}
This concludes the proof of Lemma
\ref{lem:bound_for_difference-of-approx-actual-soln}.
\end{proof}
%
%
%
%
\subsection{Non-Uniform Dependence for $3/2<s<2$.}
%
%
%
Let $u_{\pm 1, n}$ be solutions to the HR i.v.p. with common initial data $u^{\pm 1,
n}(0)$, respectively.
We wish to show that the $H^s$ norm of the difference of $u_{\pm 1,
n}$ and the associated approximate solution $u^{\pm 1, n}$
decays as $n \to \infty$. Due to Lemma
\ref{lem:bound_for_difference-of-approx-actual-soln} we assume
$s > 3/2 $ and $\sigma = 1/2 + \ee$ for an appropriately
chosen $\ee= \ee(s) > 0$. Then by Proposition \ref{1n} and
Theorem \ref{thm:HR_existence_continous_dependence}
we obtain
\begin{equation}
	\begin{split}
		\|u_{\pm 1, n} (t) \|_{H^{2s - \sigma}(\ci)}
		& \le 2 \|u^{\pm 1, n}(0) \|_{H^{2s - \sigma}(\ci)}
		\\
		& \lesssim n^{s- \sigma}.
			\label{final-est-Hk-norm-sol}
	\end{split}
\end{equation}
Furthermore, by Proposition \ref{1n}, we have
\begin{equation}
	\begin{split}
		\|u^{\pm 1, n} (t) \|_{H^{2s - \sigma} (\ci)}
		& = \|\pm n^{-1} + n^{-s} \cos(nx \mp \gamma \omega t) \|_{H^{2s - \sigma}(\ci)}
		\\
		& \le \| \pm n^{-1} \|_{H^{2s - \sigma}(\ci)} +
		\|n^{-s} \cos(nx \mp \gamma \omega
		t) \|_{H^{2s - \sigma}(\ci)}
		\\
		& \lesssim n^{-1} + n^{s-\sigma}
		\\
		& \lesssim n^{s-\sigma}.
		\label{4}
	\end{split}
\end{equation}
		Therefore, \eqref{final-est-Hk-norm-sol}, \eqref{4}, and the triangle
		inequality yield
		\begin{equation}
			\begin{split}
				\|u^{\pm 1, n} (t) - u_{\pm 1, n}(t)\|_{H^{2s - \sigma}(\ci)}
				\lesssim n^{s-\sigma}.
				\label{5h}
			\end{split}
		\end{equation}
		Recalling
		Lemma \ref{lem:bound_for_difference-of-approx-actual-soln}, we also
		have
		\begin{equation}
			\begin{split}
				\|u^{\pm 1, n}(t) - u_{\pm 1, n} (t) \|_{H^\sigma (\ci)} 
				\lesssim n^{-r_s}
				\label{6h}.
			\end{split}
		\end{equation}
		We now wish to interpolate in order to obtain an estimate of the $H^s (\ci)$
		norm of the difference of the approximate and actual solutions:
		\begin{lemma}
			\label{7h}
			For all $\psi \in L^2(\ci)$,
			\begin{equation*}
				\|\psi \|_{H^s (\ci)} \leq \left( \| \psi \|_{H^k (\ci)} \| \psi
				\|_{H^{2s - k}} \right)^{\frac{1}{2}}.
			\end{equation*}
		\end{lemma}
		\begin{proof}
			\begin{equation*}
				\begin{split}
					\|\psi\|^2_{H^s (\ci)} & = \sum_{\xi \in \zz} (1 + 
					\xi^2)^s |\hat{\psi}(\xi) |^2
				\\
				& = \sum_{\xi \in \zz} (1 + \xi^2)^{s - 
				\frac{k}{2}}|\hat{\psi}(\xi)| \cdot
				( 1 + \xi^2)^{\frac{k}{2}}|\hat{\psi}(\xi)|
				\\
				& \le \|\psi\|_{H^k(\ci)}
				\|\psi\|_{H^{2s - k}(\ci)}
			\end{split}
			\end{equation*}
			from which we obtain 
		\begin{equation*}
				\|\psi \|_{H^s (\ci)} \leq \left( \| \psi \|_{H^k (\ci)} \| \psi
				\|_{H^{2s - k}} \right)^{\frac{1}{2}}. \qquad \Box
			\end{equation*}
			Applying Lemma \ref{7h}, and estimates \eqref{5h} and \eqref{6h}
			yields
			\begin{equation}
				\label{comp-200}
				\begin{split}
					\|u^{\pm 1,n}(t) - u_{\pm 1, n}(t) \|_{H^s (\ci)}
					& \le ( \| u^{\pm 1,n}(t)
					- u_{\pm 1, n}(t) \|_{H^\sigma (\ci)}
					\\
					& \cdot \| u^{\pm 1,n}(t)
					- u_{\pm 1, n}(t)\|_{H^{2s - \sigma}(\ci)} )^{\frac{1}{2}}
					\\
					& \lesssim (n^{-r_s} \cdot n^{s-\sigma})^{\frac{1}{2}}.
				\end{split}
			\end{equation}
			Recalling \eqref{r_s-definition}, we see that for $s \le 3$,
			\eqref{comp-200} reduces to 
			\begin{equation}
				\begin{split}
					\|u^{\pm 1,n}(t) - u_{\pm 1, n}(t) \|_{H^s (\ci)}
					& \lesssim (n^{-2(s-1)} \cdot n^{s-\sigma})^{\frac{1}{2}}
					\\
					& \lesssim n^{(2-s - \sigma)/2}
					\label{8h}
				\end{split}
			\end{equation}
			and for $s > 3$ reduces to 
			\begin{equation}
				\begin{split}
					\|u^{\pm 1,n}(t) - u_{\pm 1, n}(t) \|_{H^s (\ci)}
					& \lesssim \left( n^{-\left( s+1 \right)} \cdot
					n^{s-\sigma}
					\right)^{\frac{1}{2}}
					\\
					& \lesssim n^{(-1-\sigma)/2}.
					\label{9h}
				\end{split}
			\end{equation}
			Since $s > 3/2 $ by assumption, we now recall \eqref{8h},
			\eqref{9h}, and the
			relation $\sigma = 1/2 + \ee(s)$ to obtain 
			\begin{equation}
				\begin{split}
					\|u^{\pm 1,n}(t) - u_{\pm 1, n}(t) \|_{H^s (\ci)} \lesssim
					n^{-\ee(s)/2}.
					\label{10h}
				\end{split}
			\end{equation}
      which concludes the proof. 
    \end{proof}
%
%
%%%%%%%%%%%%%% Behavior at time  t = 0  %%%%%%%%%%%% 
%  
%
\textbf{Behavior at time $t=0$.}  We have
%
%
\begin{equation} 
	\label{HR-slns-differ-t-0} 
	\|
	u_{1, n}(0)
	-
	u_{-1, n}(0)
	\|_{H^s(\ci)}
	=
	\|
	2   n^{-1}
	\|_{H^s(\ci)}
	\simeq
	n^{-1}
	\longrightarrow 
	0
	\,\,
	\text{as}
	\,\,
	n \to \infty.
\end{equation}
%
%
%%%%%%%%%%%%%% Behavior at time  t >0  %%%%%%%%%%%% 
%  
%
\textbf{Behavior at time  $t>0$.}  Using the reverse triangle inequality, we 
have
%
%
%
\begin{equation} 
	\label{HR-slns-differ-t-pos}
	\begin{split}
		\|
		u_{1, n}(t)
		-
		u_{- 1, n}(t)
		\|_{H^s(\ci)}
		&
		\ge
		\|
		u^{1, n}(t)
		-
		u^{- 1, n}(t)
		\|_{H^s(\ci)}
		\\
		&
		-
		\|
		u^{1, n}(t)
		-
		u_{1, n}(t)
		\|_{H^s(\ci)}
		\\
		&
		-
		\|
		-u^{-1, n}(t)
		+
		u_{-1, n}(t)
		\|_{H^s(\ci)} .
	\end{split}
\end{equation}
%
%
Using estimate \eqref{10h} for the last two terms 
in \eqref{HR-slns-differ-t-pos} we obtain
%
%
%
\begin{equation} 
	\label{HR-slns-differ-t-pos-est}
	\|
	u_{1, n}(t)
	-
	u_{- 1, n}(t)
	\|_{H^s(\ci)}
	\ge
	\|
	u^{1, n}(t)
	-
	u^{- 1, n}(t)
	\|_{H^s(\ci)}
	-
	c n^{- \ee(s)/2}
\end{equation}
where $c \in \rr^+$. Letting $n$ go to $\infty$ in
\eqref{HR-slns-differ-t-pos-est}
yields
%
\begin{equation} 
	\label{HR-slns-to-ap-est}
	\liminf_{n\to\infty}
	\|
	u_{1, n}(t)
	-
	u_{- 1, n}(t)
	\|_{H^s(\ci)}
	\ge
	\liminf_{n\to\infty}
	\|
	u^{1, n}(t)
	-
	u^{- 1, n}(t)
	\|_{H^s(\ci)}.
\end{equation}
%
%
Hence, by \eqref{HR-slns-to-ap-est}, we see that in order to find a lower bound for
the difference of the unknown solution sequences it is
sufficient to find a lower bound for the difference of the
associated approximate solutions. We set out to do so; using the identity 
$$
\cos \alpha -\cos \beta
=
-2
\sin(\frac{\alpha + \beta}{2})
\sin(\frac{\alpha - \beta}{2})
$$
we obtain
$$
u^{1, n}(t)
-
u^{- 1, n}(t)
=
2
n^{-1}
+
2
n^{-  s}
\sin (n x) \sin (\gamma t).
$$
Therefore
%
%
\begin{equation} 
	\label{B--ap-below-est-1}
	\|
	u^{1, n}(t)
	-
	u^{- 1, n}(t)
	\|_{H^s(\ci)}
	\ge
	2
	n^{  -  s}
	\|
	\sin (n x)
	\|_{H^s(\ci)}
	|\sin \gamma t|.
\end{equation}
%
%
Letting $n$ go to $\infty$, Proposition \ref{1n}
and \eqref{B--ap-below-est-1} imply
%
%
\begin{equation} 
	\label{HR-ap-below-est}
	\liminf_{n\to\infty}
	\|
	u^{1, n}(t)
	-
	u^{- 1, n}(t)
	\|_{H^s(\ci)}
	\gtrsim
	|\sin \gamma t|.
\end{equation}
%
%
Combining  \eqref{HR-slns-to-ap-est} and  \eqref{HR-ap-below-est}
yields
%
%
\begin{equation} 
	\label{HR-slns-below-est-fin}
	\liminf_{n\to\infty}
	\|
	u_{1, n}(t)
	-
	u_{- 1, n}(t)
	\|_{H^s(\ci)}
	\gtrsim
	|\sin \gamma t|
\end{equation}
%
%
proving \eqref{bdd-away-from-0}. Furthermore, by Proposition \ref{1n} and
Theorem \ref{thm:HR_existence_continous_dependence}  we obtain
\begin{equation}
	\begin{split}
		\label{solutions-are-small}
		\|u_{\pm 1, n} (t) \|_{H^{s}(\ci)}
		& \lesssim \|u^{\pm 1, n}(0) \|_{H^{s}(\ci)}
		\\
		& \lesssim 1
	\end{split}
\end{equation}
Collecting  \eqref{HR-slns-differ-t-0}, \eqref{HR-slns-below-est-fin}, and
\eqref{solutions-are-small}, we conclude that we have proven Theorem
\ref{hr-non-unif-dependence} for the periodic case.
%
%
%
%
%
%	
	 %%%%%%%%%%%%%%%%%%%%%%%%%%%%%%%%%
	 %
	 %
	 %
	 %   Proof of  Theorem in periodic case for s greater or equal to 2
	 %
	 %
	 %
	 %%%%%%%%%%%%%%%%%%%%%%%%%%%%%%%%%%%
	\section{A Proof of Non-Uniform Dependence In the Periodic Case For $s\ge 2$}
	In this situation we use the following initial value problem
	satisfied  by the difference
	$
	v=
		u^{\omega, n}(t) 
				- 
				u_{\omega, n}(t)
				$:
				%
				%
				%
	\begin{align}
	\label{v-eq}
					\p_t v 
					& = E + \gamma(v \p_x v - v \p_x u^{\omega,n} - u^{\omega,n} \p_x v) 
					\\
					& + \p_x\left( 1 - \p_x^2 \right)^{-1} \left[ \frac{3-
					\gamma}{2}v^2 + \frac{\gamma}{2}\left( \p_x v \right)^2 - \left(
					3 - \gamma \right)u^{\omega,n} v -
					\gamma \p_x u^{\omega,n} \p_x v \right],
					\notag
					\\
					v(0) & =0
					\label{v-data}
			\end{align}
							%
	%
	%
and prove the following key result:
%
\begin{lemma}
	\label{lem:bound_for_difference-of-approx-and-actual-soln}
If $s \ge 2$ then
			\begin{equation} 
				\|
				v(t)
				\|_{H^1(\ci)}
				=
				\label{differ-H1-est} 
				\|
				u^{\omega, n}(t) 
				- 
				u_{\omega, n}(t)
				\|_{H^1(\ci)}
				\lesssim 
				n^{-r_s}, 
				\quad
				|t| \le T.
			\end{equation}
			%
			\end{lemma}
%
			\subsection{{Proof.}} 
Straightforward computations give		
\begin{equation}
	\label{energy-estimate-simplified}
	\begin{split}
		\frac{d}{dt} \|v(t)\|_{H^1(\ci)}^2
		& = 2 \int_{\ci}
		 v\left( 1-\p_x^2
	\right)E \; dx
	\\
	&+ 2 \gamma  \int_{\ci}  v\left( 1-\p_x^2 \right)\left( v \p_x u^{\omega,n}
	- u^{\omega,n} \p_x v
	\right) \; dx
	\\
	& - 2 \int_{\ci} \left[ \left( 3-\gamma \right)v \p_x \left( u^{\omega,n}v \right) + \gamma v
	\p_x \left( \p_x u^{\omega,n} \p_x v \right)\right] \; dx.
\end{split}
\end{equation}
Applying Cauchy-Schwartz and the inequality $ab + cd \le (a^2 +
c^2)^{\frac{1}{2}}(b^2 + d^2)^{\frac{1}{2}}$ for $a,b,c,d \in \rr$, we
obtain
\begin{equation}
	\begin{split}
		\label{energy-estimate-best}
		\frac{d}{dt} \|v(t)\|_{H^1(\ci)}^2
		& \lesssim \left( \|u^{\omega,n}\|_{L^\infty(\ci)} + \|
		\p_x u^{\omega,n} \|_{L^\infty(\ci)} + \|\p_x^2 u^{\omega,n} \|_{L^\infty (\ci)} \right)
		\|v\|_{H^1(\ci)}^2 
		\\
		&+ \|v\|_{H^1(\ci)} \|E\|_{H^1(\ci)}.
	\end{split}
\end{equation}
Since we have
\begin{equation}
	\label{L-infty-error}
	\begin{split}
		\|u^{\omega,n} \|_{L^\infty(\ci)} + \|\p_x u^{\omega,n} \|_{L^\infty(\ci)}
		+ \|\p_x^2 u^{\omega,n} \|_{L^\infty(\ci)}
		& \lesssim 
		n^{-s} + n^{-s+1} + n^{-s + 2} \\
		& \lesssim n^{-s + 2}  ,
	\end{split}
\end{equation}
substituting \eqref{L-infty-error} and \eqref{total-error-approx-solution} into
\eqref{energy-estimate-best} gives
\begin{equation}
	\label{en-est-fin!}
	\frac{d}{dt} \|v(t)\|_{H^1(\ci)}^2 \lesssim n^{-s+2} \|v\|_{H^1(\ci)}^2 + n^{-r_s}
	\|v \|_{H^1(\ci)}
\end{equation}
where $r_s$ is defined in \eqref{r_s-definition}.
Applying Gronwall's inequality completes the proof. $\Box$
%
Next, note that Proposition \ref{1n}, Theorem
\ref{thm:HR_existence_continous_dependence}, and the triangle inequality
yield
\begin{equation}
	\begin{split}
		\|u^{\pm 1, n} (t) - u_{\pm 1, n}(t)\|_{H^{2s - 1}(\ci)}
		\lesssim n^{s-1}.
		\label{5hprimus}
	\end{split}
\end{equation}
Hence, interpolating and applying Lemma
\ref{lem:bound_for_difference-of-approx-and-actual-soln} and
\eqref{5hprimus}, we obtain
		%
			\begin{equation*}
				\begin{split}
					\|u^{\pm 1,n}(t) - u_{\pm 1, n}(t) \|_{H^s (\ci)}
					& \le ( \| u^{\pm 1,n}(t)
					- u_{\pm 1, n}(t) \|_{H^1 (\ci)}
					\\
					& \cdot \| u^{\pm 1,n}(t)
					- u_{\pm 1, n}(t)\|_{H^{2s-1}(\ci)} )^{\frac{1}{2}}
					\\
					& \lesssim (n^{-r_s} \cdot n^{s-1})^{\frac{1}{2}}.
				\end{split}
			\end{equation*}
			Recalling \eqref{r_s-definition}, we see that for $s \ge 2$ this reduces to
			\begin{equation}
				\begin{split}
					\|u^{\pm 1,n}(t) - u_{\pm 1, n}(t) \|_{H^s (\ci)} \lesssim
					n^{-\frac{1}{2}}.
					\label{10v}
				\end{split}
			\end{equation}
%
The rest of the proof is the same as in the case $s>3/2$.
%
%
%	
%
%
%
%
%
%

