\chapter{Calculus in Metric Spaces}
\section{Introduction}
%
%
%
%%%%%%%%%%%%%%%%%%%%%%%%%%%%%%%%%%%%%%%%%%%%%%%%%%%%%
%
%
%				Notation
%
%
%%%%%%%%%%%%%%%%%%%%%%%%%%%%%%%%%%%%%%%%%%%%%%%%%%%%%
%
%
%
%
Our goal will be to prove the following.
%
%
%%%%%%%%%%%%%%%%%%%%%%%%%%%%%%%%%%%%%%%%%%%%%%%%%%%%%
%
%
%				Existence Theorem
%
%
%%%%%%%%%%%%%%%%%%%%%%%%%%%%%%%%%%%%%%%%%%%%%%%%%%%%%
%
%
\begin{theorem}[Metric Space ODE Theorem]
	\label{ode-thm}
  Let $X$ be a topological vector space over $\rr$
  with topology induced by a metric $d$, $E \subset X$ open, and $(-a, a)$ an
	open interval in $\rr$. Suppose $f: (-a, a) \times E \to X$ satisfies the
	inequality
	%
	%
	\begin{equation}
		\label{stronger-ode}
		\begin{split}
      d[f(t, x), f(t, y)] \le c d(x, y), \qquad \forall t \in (-a, a),
			\qquad \forall x, y \in E.
		\end{split}
	\end{equation}
	%
  and let $B_{r}$ be a ball of radius $r$ contained in $E$. Then there exists a
  sufficiently small $h = h(r) > 0$ such that for each $\vp \in B_{r}$, there
  exists a unique differentiable map $u: (-h, h) \to E$ which for all $t \in
  (-h, h)$ satisfies
	%
	%
	\begin{gather}
    \label{ode-thm-eq}
			u'(t) = f(t, u(t)),
			\\
      \label{ode-thm-init-data}
			u(0) = \vp.
	\end{gather}
\end{theorem}
%
%
Notice that \autoref{ode-thm} is a 
generalization of the following classical result.
%
%
%%%%%%%%%%%%%%%%%%%%%%%%%%%%%%%%%%%%%%%%%%%%%%%%%%%%%
%
%
%				 Picard-Lindelof
%
%
%%%%%%%%%%%%%%%%%%%%%%%%%%%%%%%%%%%%%%%%%%%%%%%%%%%%%
%
%
\begin{theorem}[Picard-Lindel\"{o}f]
	Let $f(t, x)$ be a continuous function on $(- a, a) \times (- b,
	b)$ such that
	%
	%
	\begin{equation*}
		\begin{split}
			| f(t, x) - f(t, y) | \le c| x - y |, \quad \forall t \in (-a, a),
			\quad \forall x,y \in (- b, b).
		\end{split}
	\end{equation*}
	%
	%
	Then for any $\xi \in (-b, b)$ there exists an interval $(-h, h)
	\subset (-a, a)$ such that the initial value problem
	%
	%
	\begin{gather}
			\frac{dx}{dt} = f(t, x(t)),
			\\
			x(0) = \xi 
	\end{gather}
	%
	%
	has a unique solution $x(t)$ for all $t \in (-h, h)$.
	%
	%
	\end{theorem}
To prove \autoref{ode-thm}, we will first define
integration over Banach space. (To extend the theory of integration
to arbitrary complete topological vector spaces with topology induced by a
metric, one need
only replace norms with metrics in the definitions. We leave this to the
reader.) Then we will rewrite
\eqref{ode-thm-eq} as an integral equation. A fixed point argument will complete
the proof. These notes are heavily influenced by a
number of excellent books, among them Dieudonn{\'e}
\cite{Dieudonne_1969_Foundations-of-}, Yosida \cite{Yosida:1980fk},
Folland \cite{Folland_1999_Real-analysis}, Jost
\cite{Jost-1998-Postmodern-analysis}, Rudin \cite{Rudin:1976uq}, and Knapp
\cite{Knapp:2005rm}-\cite{Knapp:2005yg}.
%
%
%%%%%%%%%%%%%%%%%%%%%%%%%%%%%%%%%%%%%%%%%%%%%%%%%%%%%
%
%
%				 Differentiaion in Banach Spaces
%
%
%%%%%%%%%%%%%%%%%%%%%%%%%%%%%%%%%%%%%%%%%%%%%%%%%%%%%
%
%
%%%%%%%%%%%%%%%%%%%%%%%%%%%%%%%%%%%%%%%%%%%%%%%%%%%%%
%
%
%				 Integration in Banach Spaces
%
%
%%%%%%%%%%%%%%%%%%%%%%%%%%%%%%%%%%%%%%%%%%%%%%%%%%%%%
%
%
\section{Integration in Banach Spaces}
\label{sec:int-banach}
%
%
%%%%%%%%%%%%%%%%%%%%%%%%%%%%%%%%%%%%%%%%%%%%%%%%%%%%%
%
%
%				Definition of step functions
%
%
%
%
%
%
%%%%%%%%%%%%%%%%%%%%%%%%%%%%%%%%%%%%%%%%%%%%%%%%%%%%%
%
%
%				Definition of step function
%
%
%%%%%%%%%%%%%%%%%%%%%%%%%%%%%%%%%%%%%%%%%%%%%%%%%%%%%
%
%
\begin{definition}
	A mapping $f:[a,b] \to X$ is a \emph{step-function} if there exists a disjoint
	partitioning $\left\{ (t_{j}, t_{j+1}) \right\}$ of $[a,b]$ such that
	%
	%
	\begin{equation}
		\label{step-function}
		\begin{split}
			f(t)=\sum_{j=0}^{n-1} {\alpha_j} \chi_{(t_{j}, t_{j+1})}.
		\end{split}
	\end{equation}
	%
	%
\end{definition}
%
%
\begin{framed}
\begin{example}
	The map $f:[-1,1] \to L^2(\rr)$ given by 
	%
	%
	\begin{equation*}
		\begin{split}
			f(t) = 
			\begin{cases}
				 e^{-x^2},  \quad & -1 \le t\le0 \\
				 e^{-2x^2},  \quad & \phantom - 0 < t \le 1 \\
				 0,  \quad & \phantom - \text{otherwise}
			\end{cases}
		\end{split}
	\end{equation*}
	%
	%
	is a step-function.
	\end{example}
\end{framed}
%
%
\begin{definition}[Banach Space Integration for Step Functions]
	Let $X$ be a Banach space, $f: [a,b] \to
	X$ a step function as in \eqref{step-function}. 
	Then 
	%
	\begin{equation*}
		\begin{split}
      \int_a^b f(t) dt \doteq \sum_{j=0}^{n-1} \alpha_j (t_{j+1} - t_{j}).
    \end{split}
	\end{equation*}
	%
	%
%
\end{definition}
%
%
Note that this definition is independent of the partitioning we use for $f$, due
to the following lemma.
%
%
%%%%%%%%%%%%%%%%%%%%%%%%%%%%%%%%%%%%%%%%%%%%%%%%%%%%%
%
%
%                Indpendence of Partioning
%
%
%%%%%%%%%%%%%%%%%%%%%%%%%%%%%%%%%%%%%%%%%%%%%%%%%%%%%
%
%
\begin{lemma}
For step functions $f, g$, we have
%
%
\begin{equation*}
\begin{split}
  \int_{a}^{b} f(t) dt - \int_{a}^{b} g(t) dt = \int_{a}^{b}\left[ f(t) - g(t) \right]dt.
\end{split}
\end{equation*}
%
%
\label{lem:indep-part}
\end{lemma}
%
%
\begin{proof}
%
Let
%
\begin{equation*}
\begin{split}
  & f = \sum_{j=1}^{m}\alpha_{j} \chi_{(t_{j}, t_{j+1})}
  \\
  & g = \sum_{k =1}^{n}
  \beta_{k} \chi_{(s_{k}, s_{k+1})}
\end{split}
\end{equation*}
%
and assume without loss of generality that $n \ge m$. 
Then
%
%
\begin{equation*}
\begin{split}
  f(t) - g(t) = \sum_{p=1}^{m+n} \gamma_{p} \chi_{(y_{p}, y_{p+1})} 
\end{split}
\end{equation*}
%
%
where
\begin{equation*}
  \begin{split}
    \gamma_{p} \chi_{(y_{p}, y_{p+1}) } = 
    \begin{cases}
      \alpha_{p} \chi_{(t_{p}, t_{p+1})}
, \quad & 1 \le p \le m \\
-\beta_{p-m} \chi_{(s_{p-m}, s_{p-m+1})}, \quad & m < p \le m+n.
\end{cases}
\end{split}
\end{equation*}
Then
%
%
%
\begin{equation*}
\begin{split}
  \int_{a}^{b}\left[ f(t) - g(t) \right]dt
  & = \sum_{p=1}^{m+n}
  \gamma_{p}(y_{p+1} - y_{p})
  \\
  & = \sum_{p=1}^{m} \alpha_{p}(t_{p+1} - t_{p}) +
  \sum_{p=m+1}^{m+n}-\beta_{p-m}(s_{p-m+1} - s_{p-m})
  \\
  & = \sum_{j=1}^{m}\alpha_{j}(t_{j+1} - t_{j}) -
  \sum_{k=1}^{n}\beta_{k}(s_{k+1} - s_{k})
  \\
  & = \int_{a}^{b}f(t)dt - \int_{a}^{b} g(t) dt.
\end{split}
\end{equation*}
%
%
%
%
\end{proof}
%
We are now interested in extending integration to a larger class of functions. We will need the
following. %
%
\begin{lemma}
	\label{lem:dense}
	Let $X$ be a Banach space, and
	suppose the function $f:[a,b] \to X$ is continuous. Then
	there exists a sequence of step functions $\left\{ f_n \right\}$
	such that $\|f - f_n\|_{L^\infty( [a,b], X)} \to 0$. 
\end{lemma}
%
%
\begin{proof}
  Fix $n \in \mathbb{N}$. Since $f$ is continuous, for every $t \in
\left[ a,b \right]$ there exists an open interval $V_t \subset [a,b]$ containing
$t$ such that
$\|f(x) - f(y)) \|_X \le 1/n$ for all $x, y \in V_t$. Note that $\left\{ V_t
\right\}_{t \in \left[ a,b \right]}$ is an open cover of $[a,b]$. We assume
without loss of generality that this cover is disjoint (since for any two
intervals $A$ and $B$ which overlap, we can replace $A$ with $A \setminus
B$ and still remain with an open cover of $[a,b]$). 
Due to the compactness of $[a,b]$, we can extract
a finite subcover $\left\{ (t_j, t_{j +1}) \right\}$ of $[a,b]$. This subcover
will be disjoint, since the original covering of $[a,b]$ was assumed to be
disjoint. For each interval $(t_j, t_{j +1})$
choose any $t_{j}^* \in (t_{j}, t_{j +1})$, and define the step-function
%
%
\begin{equation*}
	\begin{split}
		f_n(t) =
		\begin{cases}
      f(t_{j}^{*}), \quad & t \in (t_{j},t_{j+1}) 
		\\
		0 & \text{otherwise}.
	\end{cases}
	\end{split}
\end{equation*}
%
Fix $t \in [a,b]$. By construction, $t \in (t_{j}, t_{j+1})$ for some $j$.
Furthermore, since $\{(t_{j}, t_{j+1}) \} \subset \{V_{t}\}_{t \in [a,b]}$, we must have $(t_{j}, t_{j+1}) = V_{t}$. Hence
%
\begin{equation*}
	\begin{split}
		\|f(t)-f_n(t)\|_X
    & = \| f(t) - f(t_{j}^{*}) + f(t_{j}^{*}) - f_{n}(t)  \|_{X}
    \\
    & \le \| f(t) - f(t_{j}^{*}) \|_{X} + \| f(t_{j}^{*}) - f_{n}(t) \|_{X}
    \\
    & = \| f(t) - f(t_{j}^{*}) \|_{X}
    \\
    & \le \frac{1}{n}.
	\end{split}
\end{equation*}
%
%
Since $t$ was chosen arbitrarily, the proof is complete. 
\end{proof}
%
\begin{framed}
\begin{remark}
  \label{rem:unif}
If $f_{n}: [a,b] \to X$ is any sequence of step functions converging pointwise
in $t$ to $f$ in $X$, it converges \emph{uniformly} in $t$, due to the
compactness of $[a,b]$. 
\end{remark}
\end{framed}
%
\begin{definition}[Banach Space Integration for Continuous Functions]
  \label{def:banach}
  Let $X$ be a Banach space, $f: [a,b] \to X$ continuous, and $f_{n} \to f$ in
  $L^{\infty}([a,b], X)$. Then 
  %
  %
  \begin{equation*}
  \begin{split}
    \int_{a}^{b} f(t) dt \doteq \lim_{n \to \infty} \int_{a}^{b} f_{n}(t) dt
  \end{split}
  \end{equation*}
  %
  %
%
where the convergence is in the topology of $X$. 
\end{definition}
We now show that this
definition for Banach space integration is
well-defined. More precisely, we will first show that $\int_{a}^{b}
f_{n}(t) dt$ is Cauchy in $X$. Then, we will show the convergence is independent
of the step functions chosen. Proceeding, we first prove the following lemma.
%
%
%
%
%%%%%%%%%%%%%%%%%%%%%%%%%%%%%%%%%%%%%%%%%%%%%%%%%%%%%
%
%
%                tri inequality
%
%
%%%%%%%%%%%%%%%%%%%%%%%%%%%%%%%%%%%%%%%%%%%%%%%%%%%%%
%
%
\begin{lemma}
  For a step function $f: [a,b] \to X$, we have
  %
  %
  \begin{equation*}
  \begin{split}
    \| \int_{a}^{b} f(t) dt \|_{X} \le \int_{a}^{b} \| f(t) \|_{X} dt.
  \end{split}
  \end{equation*}
  %
  %
\label{lem:tri-ineq-int}
\end{lemma}
%
%
%
%
\begin{proof}
%
%
\begin{equation*}
\begin{split}
  \| \int_{a}^{b} f(t) dt \|_{X} &
  = \| \sum_{j=1}^{m}\alpha_{j} (t_{j+1} - t_{j})
  \|_{X}
  \\
  & \le  \sum_{j=1}^{m} \| \alpha_{j} \|_{X} (t_{j+1} - t_{j})
  \\
  & = \int_{a}^{b} \sum_{j=1}^{m} \chi_{[t_{j}, t_{j+1}]} \| \alpha_{j} \|_{X} dt
  \\
  & = \int_{a}^{b} \| \sum_{j=1}^{m} \alpha_{j} \chi_{[t_{j}, t_{j+1}]}(t)
  \|_{X}dt \quad \text{(the intervals $[t_{j}, t_{j+1}]$ are disjoint)} 
  \\
  & = \int_{a}^{b} \| f(t) \|_{X} dt
\end{split}
\end{equation*}
%
which completes the proof.
%
\end{proof}
%
%
%
%
%%%%%%%%%%%%%%%%%%%%%%%%%%%%%%%%%%%%%%%%%%%%%%%%%%%%%
%
%
%                uniqueness
%
%
%%%%%%%%%%%%%%%%%%%%%%%%%%%%%%%%%%%%%%%%%%%%%%%%%%%%%
%
Definition~\ref{def:banach} is then well-defined due to the following two
lemmas.
%
%
%
%
%%%%%%%%%%%%%%%%%%%%%%%%%%%%%%%%%%%%%%%%%%%%%%%%%%%%%
%
%
%                conv
%
%
%%%%%%%%%%%%%%%%%%%%%%%%%%%%%%%%%%%%%%%%%%%%%%%%%%%%%
%
%
\begin{lemma}[Convergence] Let $f_{n}: [a,b] \to X$ be a sequence of simple
  functions converging to $f$ in $L^{\infty}([a,b], X)$. Then
  %
  %
  \begin{equation*}
  \begin{split}
    \int_{a}^{b}f_{n}(t) dt
  \end{split}
  \end{equation*}
  %
  %
  is Cauchy in $X$. 
\label{lem:conv}
\end{lemma}
%
%
%
%
\begin{proof}
  \begin{equation}
    \begin{split}
  \| \int_{a}^{b} f_{n}(t) dt - \int_{a}^{b} f_{m}(t) dt \|_{X}
  & = \| \int_{a}^{b}\left[ f_{n}(t) - f_{m}(t) \right]dt \|_{X}, \quad \text{Lemma~\ref{lem:indep-part}}
  \\
  & \le \int_{a}^{b} \| f_{n}(t) - f_{m}(t) \|_{X} dt,
  \quad \phantom{m} \text{Lemma~\ref{lem:tri-ineq-int}}
  \\
  & \le \int_{a}^{b} (\| f(t) - f_{n}(t) \|_{X} + \| f(t) - f_{m}(t) \|_{X})dt.
\end{split}
\end{equation}
%
Fix $\ee$. By Remark~\ref{rem:unif}, we see that there exists $N$ such
that for $m, n > N$
%
%
\begin{equation*}
\begin{split}
& \| f(t) - f_{n}(t) \|_{X} \le \ee, \quad \forall t \in [a, b].
\\
& \| f(t) - f_{m}(t) \|_{X} \le \ee, \quad \forall t \in [a, b].
\end{split}
\end{equation*}
Hence,
%
%
\begin{equation*}
\begin{split}
\| \int_{a}^{b} f_{n}(t) dt - \int_{a}^{b} f_{m}(t) dt \|_{X} \le 2(b-a) \ee,
\quad m,n > N.
\end{split}
\end{equation*}
%
Since $\ee$ can be chosen arbitrarily small, the proof is complete.
\end{proof}
%
%
\begin{lemma}[Independence]
  Let $f_{n}(t)$, $g_{n}(t) : [a,b] \to X$
  be two step functions converging to $f$ in $L^{\infty}([a,b], X)$.
  If
  %
  %
  \begin{equation*}
  \begin{split}
    & \int_{a}^{b} f_{n}(t) dt \xrightarrow{X} F
    \\
    & \int_{a}^{b} g_{n}(t) dt \xrightarrow{X} G
  \end{split}
  \end{equation*}
  %
  %
  then
  %
  %
  \begin{equation*}
  \begin{split}
    F =G.
  \end{split}
  \end{equation*}
  %
  %
\label{lem:well-def}
\end{lemma}
%
%
%
%
%
%
\begin{proof}
%
%
\begin{equation*}
\begin{split}
  \| F - G \|_{X} & = \| \int_{a}^{b} f(t) dt - \int_{a}^{b} g(t) dt \|_{X}
  \\
  & = \| \int_{a}^{b}\left[ f(t) - g(t) \right]dt \|_{X}, \quad \text{Lemma~\ref{lem:indep-part}}
  \\
  & \le \int_{a}^{b} \| f(t) - g(t) \|_{X} dt,
  \quad \phantom{m} \text{Lemma~\ref{lem:tri-ineq-int}}
  \\
  & \le \int_{a}^{b} (\| f(t) - f_{n}(t) \|_{X} + \| f(t) - g_{n}(t) \|_{X})dt.
\end{split}
\end{equation*}
%
%
Fix $\ee$. By Remark~\ref{rem:unif}, we see that there exists $N$ such
that for $n > N$
%
%
\begin{equation*}
\begin{split}
& \| f(t) - f_{n}(t) \|_{X} \le \ee, \quad \forall t \in [a, b]
\\
& \| f(t) - g_{n}(t) \|_{X} \le \ee, \quad \forall t \in [a, b].
\end{split}
\end{equation*}
%
%
Hence
%
%
\begin{equation*}
\begin{split}
\| F - G \|_{X} \le 2(b-a) \ee.
\end{split}
\end{equation*}
%
%
Since $\ee$ can be chosen arbitrarily small, we see that we must have $F = G$.
\end{proof}
%
%
We conclude the section by proving the fundamental theorem of calculus, 
thereby establishing that Banach space integration and
differentiation are inverse operations. We shall need the following lemma.
%
%
%
%
%%%%%%%%%%%%%%%%%%%%%%%%%%%%%%%%%%%%%%%%%%%%%%%%%%%%%
%
%
%                int prop
%
%
%%%%%%%%%%%%%%%%%%%%%%%%%%%%%%%%%%%%%%%%%%%%%%%%%%%%%
%
%
\begin{lemma}
  Let $X$ be a Banach space, and suppose $f: [a,b] \to X$ is continuous. Then for
  $a \le b \le c$ 
%
%
\begin{equation*}
\begin{split}
  \int_{a}^{c} f(t) dt = \int_{a}^{b} f(t) dt + \int_{b}^{c} f(t) dt.
\end{split}
\end{equation*}
%
%
\label{lem:int-splitting}
\end{lemma}
%
%
%
%
\begin{proof}
  %
  Since $\| f - f_{n} \|_{L^{\infty}\left( [a,b], X \right)} \to 0$ for some sequence
  $f_{n}$ of simple functions, we have
  %
  \begin{equation*}
  \begin{split}
    \int_{a}^{c} f(t) dt = \lim_{n \to \infty} \int_{a}^{c} f_{n}(t) dt.
  \end{split}
  \end{equation*}
  %
  %
  Recall that the value of $\int_{a}^{c}f_{n}(t) dt$ does not depend upon the
  partitioning we use for $f_{n}(t)$. Partition $f_{n}$ such that
  \begin{equation}
    \begin{split}
  & f_{n}(t) = f_{n, [a,b]} + f_{n, (b, c]},  
    \\
    & f_{n, [a,b]}(t) =0, \quad b< t \le c, \quad f_{n, (b, c]}(t) = 0, \quad
    a \le t \le b.
\end{split}
\end{equation}
Then
  %
  %
  \begin{equation*}
  \begin{split}
    \int_{a}^{c} f_{n}(t) dt & = \int_{a}^{c} f_{n, [a,b]}(t)dt  +
    \int_{a}^{c} f_{n, (b,c]}(t)dt 
    \\
    & = \int_{a}^{b} f_{n}(t)dt  +
    \int_{b}^{c} f_{n}(t)dt.
  \end{split}
  \end{equation*}
  %
  %
Hence 
 %
 %
 \begin{equation*}
 \begin{split}
   \int_{a}^{c} f(t) dt 
   & = \lim_{n \to \infty} \int_{a}^{c} f_{n}(t) dt
   \\
   &= \lim_{n \to \infty} \left ( \int_{a}^{b} f_{n}(t)dt  +
    \int_{b}^{c} f_{n}(t)dt \right )
    \\
    & =\int_{a}^{b} f(t) dt + \int_{b}^{c} f(t) dt
 \end{split}
 \end{equation*}
 %
 %
which concludes the proof.
\end{proof}
%
%
%%%%%%%%%%%%%%%%%%%%%%%%%%%%%%%%%%%%%%%%%%%%%%%%%%%%%
%
%
%                FTC
%
%
%%%%%%%%%%%%%%%%%%%%%%%%%%%%%%%%%%%%%%%%%%%%%%%%%%%%%
%
%
\begin{lemma}[Fundamental Theorem of Calculus]
Let $f \in C\left( [a,b], X \right)$, where $X$ is a Banach space. For $a \le x
\le b$, let
%
%
\begin{equation*}
\begin{split}
  F(t) = \int_{a}^{t}f(s)ds.
\end{split}
\end{equation*}
%
%
Then $F$ is differentiable at all $t_{0} \in [a,b]$, and
%
%
\begin{equation}
  \label{fund-thm-calc-diff}
\begin{split}
  F'(t_{0})= f(t_{0}).
\end{split}
\end{equation}
\label{lem:fund-thm-calc}
\end{lemma}
%
%
%
%
\begin{proof}
We apply the definition of Banach space
differentiation given in \eqref{diff-limit-simp} and see that
%
%
\begin{equation*}
\begin{split}
   \| \frac{F(t_{0} + h) - F(t_{0})}{h}  - f(t_{0}) \|_{X}
  & = \| \frac{\int_{a}^{t_{0}+h}f(s)ds - \int_{a}^{t_{0}}f(s)ds}{h} -
  f(t_{0}) \|_{X}
  \\
  & = \| \int_{a}^{t_{0}+h}\frac{f(s)}{h}ds  -
  \int_{t_{0}}^{t_{0} + h} \frac{f(t_{0})}{h} ds \|_{X}
  \\
  & = \frac{1}{h} \| \int_{t_{0}}^{t_{0} + h}[f(s) - f(t_{0})]ds \|_{X}, \qquad
  \text{Lemma~\ref{lem:int-splitting}}
  \\
  & \le \frac{1}{h} \int_{t_{0}}^{t_{0} + h} \| f(s) - f(t_{0})
  \|_{L^{\infty}\left( [t_{0}, t_{0} + h], X \right)}
  \\
  & = \| f(s) - f(t_{0})
  \|_{L^{\infty}\left( [t_{0}, t_{0} + h], X \right)}
  \\
  & = \sup_{0 \le k \le 1} \| f(t_{0} + kh) - f(t_{0})
  \|_{X}.
\end{split}
\end{equation*}
%
%
Note that since $f$ is continuous on $[a,b]$, and $[a,b]$ is compact, it follows
that $f$ is uniformly continuous on $[a,b]$. Hence, for fixed $\delta > 0$,
there exists $\ee_{\delta} > 0$ such that for $h < \ee_{\delta}$, we have
%
%
\begin{equation*}
\begin{split}
  \| f(t_{0} + kh) - f(t_{0}) \|_{X} < \delta, \ \text{for all } \  0 \le k \le 1.
\end{split}
\end{equation*}
%
%
Hence
%
%
%
\begin{equation*}
\begin{split}
  \sup_{0 \le k \le 1} \| f(t_{0} + kh) - f(t_{0}) \|_{X} < \delta, \quad h <
  \ee_{\delta}.
\end{split}
\end{equation*}
%
%
Since $\delta$ can be chosen arbitrarily small, we conclude that
%
\begin{equation*}
\begin{split}
  \lim_{h \to 0} \sup_{0 \le k \le 1} \| f(t_{0} + kh) - f(t_{0})
  \|_{X} = 0,
\end{split}
\end{equation*}
%
%
concluding the proof.
\end{proof}
%
%
%
%
%
%
%
%
%%%%%%%%%%%%%%%%%%%%%%%%%%%%%%%%%%%%%%%%%%%%%%%%%%%%%
%
%
%				 Contraction Mapping Theorem
%
%
%%%%%%%%%%%%%%%%%%%%%%%%%%%%%%%%%%%%%%%%%%%%%%%%%%%%%
%
%
\section{Contraction Mapping Lemma}
\begin{definition}
  Let $\left( X, d \right)$ be a metric space. A mapping $T: X \to X$ is called a
\emph{contraction} if there exists an $\alpha$, $0 \le \alpha <1$ such that
%
%
\begin{equation*}
	\begin{split}
		d(Tx, Ty) \le \alpha d(x,y), \qquad \forall x, y \in X.
	\end{split}
\end{equation*}
%
\end{definition}
%
\begin{framed}
\begin{remark}
	Observe that a contraction mapping is always continuous.
\end{remark}
\end{framed}
%
%
\begin{framed}
\begin{example}
	The function $T(x) = 0.1x^2$ defines a contraction mapping in the set 
	$X = [-4, 4]$ equipped with the metric $d(x,y) = |x-y|$. 
\end{example}
\end{framed}
	%
	%
	%
	%
	%%%%%%%%%%%%%%%%%%%%%%%%%%%%%%%%%%%%%%%%%%%%%%%%%%%%%
	%
	%
	%				Banach Space Fixed Point Theorem 
	%
	%
	%%%%%%%%%%%%%%%%%%%%%%%%%%%%%%%%%%%%%%%%%%%%%%%%%%%%%
	%
	%
	\begin{lemma}[Contraction Mapping Lemma]
		\label{lem:fixed-point}
	Let $(X,d)$ be a complete metric space, and $T: X \to X$ a contraction
	mapping. Then $T$ has a unique fixed point in $X$. That is, there is a unique
	point $x^* \in X$ such that $Tx^* = x^*$. Furthermore, if $x_0$ is any point
  in $X$, and we define the sequence $x_{n+1} = Tx_n$, then $x_n \xrightarrow{X} x^*$ as $n
	\to \infty$.
	\end{lemma}
	%
	%
  \begin{proof} First we show uniqueness. If $x^*$ and $x^{**}$ are two fixed
	points, then
	%
	%
	\begin{equation*}
		\begin{split}
			d(x^*, x^{**}) = d(Tx^*, Tx^{**}) \le \alpha d(x^*, x^{**}) \implies d(x^*,
			x^{**}) = 0 \implies x^* = x^{**}.
		\end{split}
	\end{equation*}
	%
	%
To prove existence, we observe that since $X$ is complete it suffices to show
that $x_n$ is Cauchy. A repeated application of the
contraction inequality gives
%
%
\begin{equation*}
	\begin{split}
		d\left( x_{n+1},x_n \right)
		& = d\left( Tx_n, Tx_{n-1} \right)
		\\
		& \le \alpha d\left( x_n, x_{n-1} \right)
		\\
		& \le \alpha^2 d\left( x_{n-1}, x_{n-2} \right)
		\\
		& \cdots
		\\
		& \le \alpha^n d\left( x_1, x_0 \right).
	\end{split}
\end{equation*}
%
%
Hence
%
%
\begin{equation*}
\begin{split}
  d\left( x_{n+k},x_n \right)
  & \le (\alpha^{n } +\alpha^{n+1} + \cdots +
  \alpha^{n+k-2} + \alpha^{n+k-1})d(x_{1}, x_{0}) 
  \\
  & = \alpha^{n}(1 + \alpha + \cdots + \alpha^{k-2} + \alpha^{k-1})
  \\
  & \le \alpha^{n}\left( \frac{1}{1 - \alpha} \right)
  \\
  & \to 0 \ \text{as} \ n \to \infty
\end{split}
\end{equation*}
%
%
since $0 \le \alpha < 1$. 
\end{proof}
%
%%%%%%%%%%%%%%%%%%%%%%%%%%%%%%%%%%%%%%%%%%%%%%%%%%%%%
%
%
%				 Proof of ODE Theorem
%
%
%%%%%%%%%%%%%%%%%%%%%%%%%%%%%%%%%%%%%%%%%%%%%%%%%%%%%
%
%
\section{Proof of Metric Space ODE Theorem}
	%
	%
	Since $f(t)$ is Lipschitz continuous on $E$ for $t \in (-a, a)$, it is
	continuous on $E$. Hence, using the theory of metric space integration
  developed earlier and the fundamental theorem of calculus,
  one can check that the initial value problem
  \eqref{ode-thm-eq}-\eqref{ode-thm-init-data} is equivalent to the
	integral equation
	%
	%
	\begin{equation*}
		\begin{split}
			u(t) = \vp + \int_0^t f(s, u(s) ) \ ds.
		\end{split}
	\end{equation*}
	%
	%
	%
	%
	Let $[-h, h] \subset [-a/2, a/2]$ be a closed interval, with $t \in [-h, h]$,
	and $r$ chosen such that
	$$B_r(\vp) \doteq \left\{ \psi \in X: d(\vp, \psi) \le r \right\}$$
	is a subset of $E$. Define 
	$$V_h = \left\{ \text{all maps} \; \;  v: [-h, h]
    \to B_r(\vp)\right\}.$$ Then $(V_h, \mathfrak{d})$ is a complete metric space
    where the metric $\mathfrak{d}$ is defined by
    $$\mathfrak{d}(v_{1}, v_{2}) = \sup_{s \in [-h,
    h]} d[v_1(s),v_2(s)].$$ 
    We shall also use the notation
    %
    %
    \begin{equation*}
    \begin{split}
    ||| v ||| = \sup_{s \in [-h,
    h]} d[v(s),0]. 
    \end{split}
    \end{equation*}
    %
    %
    Let $T$ be a map
	acting on $V_h$ via the relation $$Tv(t) = \vp + \int_0^t f(s, v(s) ) \ ds.$$
	Recalling \autoref{lem:fixed-point}, we see that to complete the proof it will be enough to show that $T$ is a contraction on
	$V_h$, for suitably small $h >0$. First, note that for $v \in V_h$, we have
	%
	%
\begin{equation}
	\label{cont-map-into}
	\begin{split}
		| | | Tv - \vp| | |
		& = | | | \int_0^t f(s, v(s) ) \ ds | | |
		\\
    & \le  \int_0^{|t|} | | | f(s, v(s) ) | | | ds
    \end{split}
    \end{equation}
To estimate the integrand, we recall that $f(t, x)$ is continuous in the time
variable in the interval $(-a, a)$. Hence, for $t \in [-a/2, a/2]$ and fixed $x
\in E$, we have
%
%
\begin{equation*}
\begin{split}
d(f(t, x), 0) \le M
\end{split}
\end{equation*}
%
%
for some constant $M > 0$. Using this and the Lipschitz continuity of $f(t, x)$
in the spatial variable on $E$, we see that
%
%
\begin{equation*}
\begin{split}
| | | f(s, v(s) ) | | | 
& \le | | | f(s, v(s)) - f(s, v(0)) ||| + ||| f(s, v(0)) |||
\\
& \le c \sup_{s \in [-h, h]}  d(v(s), v(0)) +  M
\\
& \le 2cr + M.
\end{split}
\end{equation*}
%
%
Substituting this relation into \eqref{cont-map-into}, we see that
\begin{equation*}
  \begin{split}
| | | Tv - \vp| | |
& \le 
\int_0^{|t|} [2cr + M ] ds
    \\
    & \le  |t| [ 2cr + M]
		\\
    & \le  h [ 2cr + M].
\end{split}
\end{equation*}
%
%
%
Choosing $h \le 1/[2cr + M]$, it follows that $T: V_h \to V_h$. 
%Since $[-h, h] \subset (-a, a)$, we can find a compact set $C$ such that $(-h,
%h) \subset C \subset (-a, a)$. Since $f$ is a continuous function of two
%variables, $f( (C \times K) )$ is compact, and so \eqref{cont-map-into} gives
%the estimate
%%
%%
%\begin{equation*}
%	\begin{split}
%		| | | Tv | | | \le \|u_0\|_X + Mh < \infty
%	\end{split}
%\end{equation*}
%
%
Next, note that for any two points $v_1, v_2 \in
V_h$, we have
%
%
\begin{equation}
	\label{cont-part-1}
	\begin{split}
		| | | Tv_2 - Tv_1 | | | 
		& = | | | \int_0^t \left[ f(s, v_2(s) ) - f(s, v_1 (s) ) \right]ds | | |
		\\
    & \le \int_0^{|t|} | | | f(s, v_2(s) ) - f(s, v_1 (s) ) | | | ds
		\\
    & = \int_0^{|t|} \sup_{s \in [-h, h]} d[f(s, v_1(s) ), f(s, v_2 (s) )] \ ds
		\\
    & \le c \int_0^{|t|}  \sup_{s \in [-h, h]} d[v_1(s), v_2 (s)] \
		ds
		\end{split}
\end{equation}
%
%
where the last step follows from \eqref{stronger-ode}. Hence
\begin{equation*}
	\begin{split}
		| | | Tv_2 - Tv_1 | | | 
		& \le c h \sup_{s \in [-h, h]} d[v_1(s), v_2 (s)]
		\\
		& = ch |||v_2 - v_1|||.
	\end{split}
\end{equation*}
%
%
Restricting $h \le 1/(2c)$, we obtain
\begin{equation*}
	\begin{split}
		| | | Tv_2 - Tv_1 | | | & \le \frac{1}{2} | | | v_2 - v_1 | | |. 
	\end{split}
\end{equation*}
Hence, for $h \le \min\left\{1/(2c), 1/[2cr + M] \right\}$, $T$ is a
contraction on $V_h$. This completes the proof. \qed
%
\section{Differentiation in Banach Spaces}
We remark that the theory presented below can be extended to arbitrary
topological vector spaces with a topology induced by a metric (in particular,
Fr\'echet spaces) by replacing norms with metrics in what follows.
\begin{definition}
	\label{def:diff}
	Let $X,Y$ be Banach spaces over the real numbers, $U \subset X$ open,
  and consider the map $f: U \to Y$.
  Then $f$ is differentiable at $x_0 \in U$ at $x_{0} \in U$ if there
	is a continuous linear map $Df(x_0): X \to Y$ such that
	%
	%
	%
	%
	\begin{equation}
		\label{diff-limit}
		\begin{split}
			\lim_{h \to 0} \frac{\|f(x_0+ h) - f(x_0) -
			Df(x_0)(h) \|_Y}{\|h\|_{X}} = 0.
		\end{split}
	\end{equation}
	%
	%
	This map, which we call the \emph{total derivative} of $f$ at $t_0$, is 
	unique. If $Df(x_0)$ exists for all $x_0 \in U$,
	then we say that $f$ is
    \emph{differentiable} in $U$. If $f$ is differentiable in $U$, and 
	$\|Df(x_0 + h) - Df(x_0) \|_Y \to 0$ as $\|h\|_{X} \to 0$ for all $x_0 \in U$,
    then we say that $f$ is \emph{continuously differentiable in $U$}, or $C^{1}$ in $U$. If $X, Y$
  are Banach spaces over the complex numbers, we say that $f$ is
  \emph{holomorphic} (or \emph{continuously holomorphic}) in $U$.
\end{definition}
	%
  %
  %
  \begin{framed}
  %
  %
  \begin{remark}
    Equivalently, $f$ is differentiable at $x_{0}$ if there exists a continuous
    linear map $Df(x_{0}): X \to Y$ such that
    %
    %
    \begin{equation*}
    \begin{split}
      f(x_{0} + h) = f(x_{0}) + Df(x_{0}) \circ h + o(h).
    \end{split}
    \end{equation*}
    %
    To see this, suppose
    %
    %
    \begin{equation*}
    \begin{split}
      \lim_{h \to \infty} \frac{\| f(x_{0} + h) - f(x_{0}) - Df(x_{0}) \circ h
      \|_{Y}}{\| h \|_{X}} = 0.
    \end{split}
    \end{equation*}
    %
    %
    Then we can find $\delta > 0$ such that
    %
    %
    \begin{equation*}
    \begin{split}
    \frac{\| f(x_{0} + h) - f(x_{0}) - Df(x_{0}) \circ h
      \|_{Y}}{\| h \|_{X}}  < \ee
    \end{split}
    \end{equation*}
    %
    %
    for all $h \in X$ with $\| h \|_{X} < \delta$. Hence
    %
    %
    \begin{equation*}
    \begin{split}
      \| f(x_{0} + h) - f(x_{0}) - Df(x_{0}) \circ h \|_{Y} < \ee \| h
      \|_{X}.
    \end{split}
    \end{equation*}
    %
    %
    This implies
    %
    %
    \begin{equation*}
    \begin{split}
      f(x_{0} + h) - f(x_{0}) - Df(x_{0}) \circ h = g, \quad g \in Y, \ \| g
      \|_{Y} < \ee \| h \|_{X}.
    \end{split}
    \end{equation*}
    %
    %
    Hence, 
    %
    %
    \begin{equation*}
    \begin{split}
      f(x_{0} + h) 
      & = f(x_{0}) + Df(x_{0}) \circ h + g
      \\
      & =  f(x_{0}) + Df(x_{0})\circ h + o(\| h \|_{X}).
    \end{split}
    \end{equation*}
    %
    %
    For the reverse direction, if
    %
    %
    \begin{equation*}
    \begin{split}
      f(x_{0} + h) = f(x_{0}) + Df(x_{0}) \circ h + o(\| h \|_{X})
    \end{split}
    \end{equation*}
    %
    %
    then
    %
    %
    \begin{equation*}
    \begin{split}
      \| f(x_{0} + h) - f(x_{0}) - Df(x_{0}) \circ h \|_{Y} = o(\| h
      \|_{X})
    \end{split}
    \end{equation*}
    %
    %
    which implies
    %
    %
    \begin{equation*}
    \begin{split}
      \lim_{h \to \infty} \frac{\| f(x_{0} + h) - f(x_{0}) - Df(x_{0}) \circ h
      \|_{Y}}{\| h \|_{X}}  = \lim_{h \to \infty} \frac{o(\| h \|_{X})}{\| h
      \|_{X}} = 0.
    \end{split}
    \end{equation*}
    %
    %
    %
  \label{rem:equiv-def}
  \end{remark}
  %
  %
  \end{framed}
  %
  %
%
\begin{framed}
%
%
\begin{remark}
    Observe that if $F$ is differentiable at $x_{0}$, then it is continuous at $x_{0}$.
\label{rem:deriv-imp-cont}
\end{remark}
%
%
\end{framed}
\begin{framed}
    \begin{exercise}
        Compute the total
        derivative of the function $f:L^{\infty} \to L^{\infty}$ at $u_{0}$,
        where $f(u) = u^{2}$. 
    \end{exercise}
    Answer:       %
    %
    \begin{equation*}
    \begin{split}
    \frac{\| (u_{0} + h)^{2} - u_{0}^{2} - 2u_{0}h \|_{L^{\infty}}}{\| h
        \|_{L^{\infty}}}
        &=         \frac{\| h^{2} \|_{L^{\infty}}}{\| h \|_{L^{\infty}}}
        \\
        & \le \| h \|_{L^{\infty}} 
        \\
        & \to 0.
    \end{split}
    \end{equation*}
    %
    %
    Therefore, $D_{u}(f(u))(u_{0}) = 2u_{0}I$. 
\end{framed}
Recall that if $F$ is continuously differentiable in $U$, then $DF : U \to L(U,
X)$ is a continuous map. Since $X$ is Banach, it follows that $L(U, X)$ is also
Banach under the operator norm, hence it makes sense to talk about derivatives
of $DF$ at a point. More precisely, if $DF$ is differentiable at $x_{0}$, then
we denote its derivative at $x_{0}$ by $D^{2}F(x_{0})$ and say that 
is $F$ is $2$-times differentiable at $x_{0}$.
Observe then that $D^{2}F(x_{0}) \in L(U, L(U,X))$. If $DF$ is continuously
differentiable in $U$, then we say that $F$ is $C^{2}$ in $U$, and observe that
$D^{2}F : U \to L(U, L(U, X))$ continuously. In general, we say $F$ $k$-times
differentiable at $x_{0}$ if $D^{k-1}F$ is differentiable at $x_{0}$, and
observe that $D^{k}F(x_{0}) \in L(U, L(U, L(U, \dots)))$. If $D^{k-1}F$ is
continuously differentiable in $U$, then we say $F$ is $C^{k}$ in $U$, and
observe that $D^{k}F : U \in L(U, L(U, L(U, \dots)))$ continuously.

Notice that our definition of $D^{k}F$ becomes more and more complicated as $k$
increases. However, there are special cases in which our definition of the
Banach space derivative becomes easier to deal with. More precisely, for an
arbitrary Banach space $X$, open interval $(a,b) \subset \rr$, and map $f:(a,b)
\to X$, we can identify $Df(x_0)$ with an element of $X$ via the following.
%
%
%%%%%%%%%%%%%%%%%%%%%%%%%%%%%%%%%%%%%%%%%%%%%%%%%%%%%
%
%
%			Lemma Isometry	
%
%
%%%%%%%%%%%%%%%%%%%%%%%%%%%%%%%%%%%%%%%%%%%%%%%%%%%%%
%
%
\begin{lemma}
	\label{lem:isometry} Let $(a,b) \subset \rr$ be an open interval, $X$ a Banach
	space, with $x \in X$. Define the map $T_x \in L\left ( \rr , X \right )$ by
	$T_x(t_0) = x t_0$. Then the map $x \mapsto T_x$ is an
	isometric isomorphism from
	$X$ to $L(\rr , X)$. 
\end{lemma}
%
%
\begin{proof} Note that 
%
%
\begin{equation*}
	\begin{split}
		| | | T_x | | |
		& = \sup_{|t_0| = 1} \| T_x (t_0) \|_X
		= \| x t_0\|_X
		= \|x\|_X.
	\end{split}
\end{equation*}
%
%
Hence, the map $x \mapsto T_x$ is an isometry from $X$ into $L(\rr,
X)$. It remains to show that it is onto. Let $V \in L( \rr, X)$. Then
by linearity
%
%
\begin{equation*}
	\begin{split}
		V(t_0) = V(1)t_0. 
	\end{split}
\end{equation*}
%
%
Hence, $V = T_{V(1)}$, completing the proof. 
\end{proof}
%
%
Applying the lemma, we see that if $Df(t_0)$ exists for a map $f: (a,b) \to X$,
then it can be viewed as an
element of $X$. Similarly, if $Df(t_0)$ exists for all $t_0 \in (a,b)$, then
we may view the map $t_0 \to Df(t_0)$ as an
element of $L( \rr, X)$. Hence, for a
map $f:(a,b) \to X$, we see that the following is an equivalent
reformulation of \autoref{def:diff}. 
\begin{definition}
	\label{def:diff-simp}
	Let $X$ be Banach space, $(a,b) \subset \rr$ an open interval, and
	consider the map $f: (a,b) \to X$.
	Then $f$ is \emph{differentiable at $t_0$} if there exists a map
	$f': \rr \to X$ such that 
	%
	%
	%
	%
	\begin{equation}
		\label{diff-limit-simp}
		\begin{split}
			\lim_{h \to 0} \| \frac{f(t_0+ h) - f(t_0) 
			 }{h} - f'(t_0) \|_X = 0.
		\end{split}
	\end{equation}
	%
	%
	We call $f'(t_0)$ the \emph{derivative of $f$ at $t_0$}.
	If \eqref{diff-limit-simp}
	holds for all $t_0 \in (a,b)$, then we say $f$ is \emph{differentiable in
	$(a,b)$}, and call $f'$ the
	\emph{derivative of $f$ in $(a,b)$}.  
\end{definition}
\begin{framed}
\begin{example}
Let $f: (a,b) \to \rr^2$ be defined by $f(t) = (1, t^2)$. Then $Df$ maps the
interval $(a,b)$ to $L((a,b), \rr^{2})$ by the relation $Df(t) = (0, 2t)$, and
$Df(t)(h) = (0,2th)$. Hence, we may view $Df(t)$ as the point $(0,2t) \in
\rr^2$, and view its action on $(a,b)$ as nothing more than standard scalar
multiplication in $\rr^{2}$. That is, $f'(t) = (0,2t)$, and $f'(t)(s) = s(0,
2t)$.
\end{example}
\end{framed}

Lastly, we include the following.
%
%
\begin{lemma}[Chain Rule]
    Let $X,Y,Z$ be Banach spaces, $f: X \to Y$ and $g: Y \to Z$
    continuously differentiable maps. Then for all $x_0 \in X$ we have
    %
    \begin{equation*} (g \circ f)' (x_0) = g'(f(x_0)) \circ (f'(x_0)).
		\end{equation*} 
    %
	\end{lemma}
  \begin{proof} Since $f$ and $g$ are continuously differentiable, we can write
			\begin{equation*}
				f(x_0 + s) = f(x_0) + f'(x_0)(s) + o_{1}(s)
			\end{equation*}
			and
      \begin{equation*}
				g(x_0 + t) = g(x_0) + g'(x_0)(t)+ o_{2}(t).
			\end{equation*}
						Hence
			\begin{equation}
        \label{pre-order}
				\begin{split}
					h(x_0 + s) &= g(f(x_0 +s)
					\\
					&= g(f(x_0) + f'(x_0)(s) + o_{1}(s))
					\\
					&= g(f(x_0)) + g'(f(x_0)) \circ [f'(x_0)(s) + o_{1}(s)] 
					+ o_{2}(f'(x_0)(s) + o_{1}(s))
					\\
					&= g(f(x_0)) + g'(f(x_0)) \circ f'(x_0)(s) +
					g'(f(x_0)) \circ o_{1}(s) + o_{2}(f'(x_0)(s) + o_{1}(s)).
				\end{split}
			\end{equation}
      Since $f'(x_0)$
			and $g'(f(x_0))$ are continuous linear operators, we have
			\begin{equation*}
				\begin{split}
         &  f'(x_0)(s)
           = o_{3}(s)
					\\
					& g'(f(x_0))(t) = o_{4}(t).
				\end{split}
			\end{equation*}
      Therefore,
      %
      %
      \begin{equation*}
        \begin{split}
          g'(f(x_0)) \circ o_{1}(s)
          & = o_{4}(o_{1}(s))
          \\
          & = o_{5}(s)
        \end{split}
      \end{equation*}
      %
      %
      and
      %
      %
      \begin{equation*}
      \begin{split}
      o_{2}(f'(x_0)(s) + o_{1}(s))
      & = o_{2}(o_{3}(s) + o_{1}(s))
      \\
      & =o_{6}(s).
      \end{split}
      \end{equation*}
      %
      %
			Recalling \eqref{pre-int} and substituting, we conclude that
			\begin{equation*}
				\begin{split}
				h(x_0 + s) = g(f(x_0)) + g'(f(x_0))(f'(x_0)(s)) +
				o(s)
        \end{split}
        \end{equation*}
			 completing the proof. 
   \end{proof}
%
%
%%%%%%%%%%%%%%%%%%%%%%%%%%%%%%%%%%%%%%%%%%%%%%%%%%%%%
%
%
%				product rule
%
%
%%%%%%%%%%%%%%%%%%%%%%%%%%%%%%%%%%%%%%%%%%%%%%%%%%%%%
%
%
   %\begin{lemma}[Product Rule]
%Let $X$, $Y$, $Z$ be Banach spaces, and $B: X \times Y \to Z$ is a continuous
%bilinear operator, where $X \times Y$ is Banach under the usual product
%norm $\| (f,g) \|_{X \times Y} = \| f \|_{X} + \| g \|_{Y}$.  Then $B$ is differentiable in $X \times Y$, with derivative at $(x_{0}, y_{0})$ given
%by the bilinear map $DB(x_{0}, y_{0}): X \times Y \to Z$ defined by
%%
%%
%\begin{equation}
    %\label{ansatz}
%\begin{split}
    %DB(x_{0},y_{0})(x,y) = B(x, y_{0}) + B(x_{0}, y)
%\end{split}
%\end{equation}
%%
%%
%\label{lem:product-rule}
%\end{lemma}
%%
%%
%%
%%
%\begin{proof}
%%
%It will be enough to verify that \eqref{ansatz} is the derivative of $B$.
%Using the bilinearity of $B$, we see that 
%%
%\begin{equation*}
%\begin{split}
     %\frac{\| B(x_{0} + h, y_{0} + k) - B(x_{0}, y_{0})  - B(h, y_{0}) - B(x_{0}, k)\|_{Z}}{\| (h, k) \|_{X \times Y}}
    %& = \frac{\| B(h, k)\|_{Z}}{\| (h, k) \|_{X \times Y}}
    %\\
    %& \le \frac{\| h \|_{X} \| k \|_{Y} ||| B |||}{\| (h, k) \|_{X \times Y}}
    %\\
    %& \simeq \frac{\| h \|_{X} \| k \|_{Y}}{\| (h, k) \|_{X \times Y}}
%\end{split}
%\end{equation*}
%%
%which goes to $0$ as $\| (h, k) \|_{X \times Y}$ goes to $0$. This concludes the proof.
%%
%\end{proof}
%%
%%
%\begin{framed}
    %\begin{example} Let $u \in L^{\infty}$. Then multiplication is a continuous bilinear mapping taking $L^{\infty} \times L^{\infty}$ to $L^{\infty}$. Hence
        %%
        %%
        %\begin{equation*}
        %\begin{split}
            %D(u^{2})(h, k) 
        %\end{split}
        %\end{equation*}
        %%
        %%
        %<++>
    %\end{example}
%\end{framed}
%<++>
%%%%%%%%%%%%%%%%%%%%%%%%%%%%%%%%%%%%%%%%%%%%%%%%%%%%%
%
%
%                Dependence on Params
%
%
%%%%%%%%%%%%%%%%%%%%%%%%%%%%%%%%%%%%%%%%%%%%%%%%%%%%%
%
%
%
\section{Regularity of the Data-to-Solution Map for ODEs} 
\label{sec:dep-param-smooth}
%
%
Following Taylor \cite{Taylor:1995kx}, let $X$ be a Banach space, $U \subset X$ an open subset, and $F: U \to X$. Consider the non-autonomous system
%
%
\begin{gather}
  \label{aa-sm}
\frac{dy}{dt} = F(y(t), t),
\\
y(0)= y_{0} \in X, \quad t \in \rr.
\label{bb-sm}
\end{gather}
%
Letting $z = (y(t), t)$, we get
%
%
\begin{equation*}
\begin{split}
\frac{dz}{dt}  
& = \left (\frac{dy}{dt}, 1 \right )
\\
& = \left( F(y(t), t), 1 \right)
\\
& = \left( F(z), 1 \right)
\\
& \doteq G(z).
\end{split}
\end{equation*}
%
%
Hence, without loss of generality, we may restrict our attention to the
autonomous system
%
%
\begin{gather}
\frac{dy}{dt} = F\left( y(t) \right)
\label{ode-eq-sm}
\\
y(0) = y_{0}, \quad t \in \rr
\label{ode-init-data-sm}
\end{gather}
%
%
%
for which we shall prove the following.
%
%
%%%%%%%%%%%%%%%%%%%%%%%%%%%%%%%%%%%%%%%%%%%%%%%%%%%%%
%
%
%                dep on init cond
%
%
%%%%%%%%%%%%%%%%%%%%%%%%%%%%%%%%%%%%%%%%%%%%%%%%%%%%%
%
%
\begin{proposition}
  Let $U \subset X$ be an open subset, and $F: U  \to X$.
  If $F$ is Lipschitz, then the flow map $\vp \mapsto u(t, \vp) \doteq
  S_{t}(\vp)$ is Lipschitz. If $F$ is $k$ times differentiable, where $k \ge 1$
  is an integer, then the flow map is $k$ times differentiable.
  \label{prop:reg-result}
\end{proposition}
%
%
%
%
\begin{proof}
  We divide our work into cases.
  \subsection*{Case $k = 0$.} The ivp \eqref{ode-eq-sm}-\eqref{ode-init-data-sm} is equivalent to 
  %
  %
  \begin{equation*}
  \begin{split}
  u(t, \vp) = \vp + \int_{0}^{t} F[u(\tau, \vp)] d \tau
  \end{split}
  \end{equation*}
  %
  and so
  %
  %
  %
  \begin{equation*}
    \begin{split}
  u(t, \vp) - u(t, \psi) = \vp - \psi + \int_{0}^{t} \{F[u(\tau, \vp)] - F[u(\tau, \psi)]\} d \tau.
  \end{split}
  \end{equation*}
  %
 Recall that Theorem~\ref{ode-thm} tells us that for any $\vp \in B_{r} \subset U$ there exists a unique solution $u(t, \vp)$ for $t \in [-h, h]$, where $h = h(r)$. Therefore, applying the triangle inequality and the Lipschitz condition, we obtain the estimate
  \begin{equation*}
  \begin{split}
  \sup_{t \in [-h, h]} \|u(t, \vp) - u(t, \psi)\|_{L^{\infty}}
  & \le \|\vp -
  \psi\|_{L^{\infty}} +  h \sup_{t \in [-h, h]} \|F[u(t, \vp)] - F[u(t, \psi)] \|_{L^{\infty}}
  \\
  & \le \|\vp - \psi\|_{L^{\infty}} +  ch \sup_{t \in [-h, h]} \|[u(t, \vp)] -
  [u(t, \psi)] \|_{L^{\infty}}.
  \end{split}
  \end{equation*}
Restricting $h < 1/c$ and rearranging terms, we obtain
%
%
\begin{equation*}
\begin{split}
\sup_{t \in [-h, h]} \| u(t, \vp) - u(t, \psi) \|_{L^{\infty}} \le \frac{1}{1-ch} \| \vp - \psi \|_{L^{\infty}}
\end{split}
\end{equation*}
%
%
concluding the proof.
 \subsection*{Case $k=1$} 
  \label{ssec:case-k}
 % 
 %
Formally differentiate
\eqref{ode-eq-sm}-\eqref{ode-init-data-sm} with respect to $\vp$ and obtain
%
%
%
\begin{gather}
  W' = DF(y(\vp,t)) \circ W
  \label{ode-lin}
  \\
  W(0) = I
  \label{ode-lin-init-data}
\end{gather}
%
%
where $W \doteq D_{\vp} y(\vp,t)$ and $y$ is the unique solution to
\eqref{ode-eq-sm}-\eqref{ode-init-data-sm}. Let $w(\vp,t) = D_{\vp}y(\vp,t) \circ h$, where $h \in U$ is
arbitrary. Then
from \eqref{ode-lin}-\eqref{ode-lin-init-data} we
obtain
%
%
\begin{gather}
  w' = DF(y(\vp, t)) \circ w 
  \label{ode-lin-h}
  \\
  w(0) = h.
  \label{ode-lin-init-data-h}
\end{gather}
%
%
Since $F$ is $C^{1}$, $DF(y(\vp, t)): U \to X$ continuously. Since $Df(y(\vp,
t))$ is also linear, it is bounded, and hence Lipschitz on $U$. Therefore, by
the Banach Space ODE theorem, there exists a unique solution $w = w(\vp,t)$ to
\eqref{ode-lin-h}-\eqref{ode-lin-init-data-h} for $t$ in some open subset of
$\rr$. We now show that this solution is in fact $D_{\vp} y(\vp,t)$ as follows.
Let $z(\vp,t) = y(\vp + h, t) - y(\vp,t)$. Note that
%
%
\begin{equation}
  \label{uhh}
\begin{split}
\frac{dz}{dt} 
& = F\left[ y(\vp + h, t) \right] - F\left[ y(\vp,t) \right]
\\
& = \int_{0}^{1} \frac{d}{ds} F\left[ sy(\vp+h, t) + (1-s)y(\vp,t) \right]ds
\\
& = \int_{0}^{1} \frac{d}{ds} F\left[ sz(\vp,t) + y(\vp,t) \right] ds
\\
& = F[y(\vp,t) +z(\vp, t)] - F[y(\vp, t)]
\end{split}
\end{equation}
%
%
which by the definition of the Banach space derivative implies
%
%
\begin{equation*}
\begin{split}
  \frac{dz}{dt} = DF(y) \circ z + R_{1}(y) \circ z, \quad R_{1}(y) \circ z = o(\|
  z(t) \|_{X})
\end{split}
\end{equation*}
%
%
and so
%
%
\begin{equation*}
\begin{split}
  \frac{d}{dt}(z-w) = DF(y) \circ (z-w) + R_{1}(y) \circ z.
\end{split}
\end{equation*}
%
%
Noting that %
%
\begin{equation*}
\begin{split}
  (z-w)(0) & = 0
\end{split}
\end{equation*}
%
%
we now seek to analyze the the inhomogeneous ode
%
%
\begin{gather}
  \frac{d}{dt}(z-w)  = DF(y) \circ (z-w) + R_{1}(y) \circ z
\label{inhom-ode}
  \\
  (z-w)(0) = 0.
  \label{inhom-ode-init}
\end{gather}
%
Since $DF(y)$ and $R_{1}(y)$ are Lipschitz on $U$, the Banach Space ODE theorem
guarantees a unique solution to this equation for $t$ in a sufficiently small
interval. Taking norms of both sides, and using the fact that $DF(y)$ is
Lipschitz, we obtain
%
%
%
%
\begin{gather}
    \label{yt}
    \frac{d}{dt} \| z - w \|_{X} \lesssim  \| z - w \|_{X}  + o(\|z\|_{X}),
    \\
    (z-w)(0) = 0.
    \label{ytt}
\end{gather}
%
We now bound $\| z \|_{X}$. From the
second line of \eqref{uhh}, we get
%
%
\begin{equation*}
\begin{split}
    \frac{d}{dt} \| z(t) \|_{X}
& \le \| \int_{0}^{1} \frac{d}{ds} F\left[ sy(\vp+h, t) + (1-s)y(\vp,t) \right]ds
\|_{X}
\\
& \le \| \int_{0}^{1} DF\left[  sy(\vp+h, t) + (1-s)y(\vp,t) \right] \circ z ds
\|_{X}
\\
& \le \| z(t) \|_{X} | | | DF | | |_{L(U, X)}
\\
& = M \| z(t) \|_{X}
\end{split}
\end{equation*}
%
where we emphasize that $M$ does not depend on $h$, $z$, or $w$.
Hence, setting $\vp(t) =  \| z(t) \|_{X}$, we wish to solve
%
%
\begin{gather*}
\frac{d \vp}{dt} = M \vp
\\
\vp(0) = \| h \|_{X}
\end{gather*}
%
%
which gives
%
%
\begin{equation*}
\begin{split}
  \vp(t) & = \vp(0)e^{Mt}
  \\
  & = \| h \|_{X} e^{Mt}
  \\
  & \lesssim \| h \|_{X}, \quad 0 \le t < T.
\end{split}
\end{equation*}
%
%
Hence,
%
%
\begin{equation*}
\begin{split}
  \| z(t) \|_{X}  \lesssim  \| h \|_{X}  \ \text{for all } \ t \in I.
\end{split}
\end{equation*}
Applying this relation to the Cauchy problem \eqref{yt}-\eqref{ytt}, we obtain
%
\begin{gather*}
    \frac{d}{dt} \| z - w \|_{X} \lesssim  \| z - w \|_{X}  + o(\| h \|_{X}),
    \\
    (z-w)(0) = 0
\end{gather*}
which by Gronwall's inequality gives
%
%
%
%
%
%
%
\begin{equation*}
\begin{split}
  \| z(t) - w(t) \|_{X} 
  = o(\| h \|_{X}),
\end{split}
\end{equation*}
%
%
that is
%
%
\begin{equation*}
\begin{split}
  \| y(\vp + h, t) - y(\vp,t) - w(\vp,t) \|_{X} = o(\| h \|_{X}).
\end{split}
\end{equation*}
%
%
Hence, by the definition of the Banach space derivative, $D_{\vp}y(\vp,t)$
exists and is equal to $w(\vp,t)$. The proof for the case $k=1$ is complete.
\subsection*{Case $k >1$} 
\label{ssec:case-kg1}
For finite $k$, formally differentiating
\eqref{ode-eq-sm}-\eqref{ode-init-data-sm}
$k$ times with respect to $\vp$, we obtain
\begin{gather*}
  W' = DG(y) \circ W
  \\
  W(0) = I
\end{gather*}
where $G \doteq D^{k-1}F$, $W \doteq D_{\vp}^{k} y$. Hence, the proof of
$k$-time differentiability in $x$ reduces to the proof in the case $k=1$, which
completes the proof for finite $k$. If $k = \infty$, we run an induction on $k$ to
obtain the corresponding result.
\end{proof}
%
%
%
\section{Time Regularity for Solutions to ODEs with $C^{k}$ Data} 
\label{sec:dep-param}
%
%
Following Taylor \cite{Taylor:1995kx}, let $X \subset \rr$ be an open subset,
$U \subset C^{k}(X)$ be an open subset, where $k \in \mathbb{N}\cup \omega$, and
$F: U \to C^{k}(X)$. We endow $C^{k}(X), k \in \mathbb{N}$ with the metric
induced by its seminorms
%
%
\begin{equation*}
\begin{split}
d(f, g) \doteq \sum_{k \in \mathbb{N}} \sum_{j \in \mathbb{N}} 2^{-j -k} \frac{|
f-g |_{j, k}}{1 + | f-g |_{j, k}}
\end{split}
\end{equation*}
%
where
%
%
\begin{gather*}
|h|_{j,k}  = \sup_{x \in K_j} \sum_{k \in \mathbb{N}} |d^{k}h(x)|,
\\
\{ K_{j} \} \ \text{a family of compact subsets of} \ X \  \text{satisfying} \ \bigcup_{j} K_{j} = X 
\end{gather*}
%
%
%
and equip the set of real analytic functions $C^{\omega}(X)$ with the
metric
%
%
\begin{equation*}
\begin{split}
\rho(f,g) \doteq \sup_{x \in X} |f(x) - g(x)|.
\end{split}
\end{equation*}
%
%
Consider the non-autonomous system
%
%
\begin{gather}
  \label{aa}
\frac{dy}{dt} = F(y(x,t), t),
\\
y(0)= y_{0}(x) \in C^{k}, \quad x \in \ci \ \text{or} \ \rr,t \in \rr.
\label{bb}
\end{gather}
%
Letting $z = (y(x,t), t)$, we get
%
%
\begin{equation*}
\begin{split}
\frac{dz}{dt}  
& = \left (\frac{dy}{dt}, 1 \right )
\\
& = \left( F(y(t), t), 1 \right)
\\
& = \left( F(z), 1 \right)
\\
& \doteq G(z).
\end{split}
\end{equation*}
%
%
Hence, without loss of generality, we may restrict our attention to the
autonomous system
%
%
\begin{gather}
\frac{dy}{dt} = F\left( y(x,t) \right)
\label{ode-eq}
\\
y(0) = y_{0}(x), \quad x \in \ci \ \text{or} \ \rr
\label{ode-init-data}
\end{gather}
%
%
%
for which we shall prove the following.
%
%
%%%%%%%%%%%%%%%%%%%%%%%%%%%%%%%%%%%%%%%%%%%%%%%%%%%%%
%
%
%                dep on init cond
%
%
%%%%%%%%%%%%%%%%%%%%%%%%%%%%%%%%%%%%%%%%%%%%%%%%%%%%%
%
%
%
\begin{corollary}[Regularity With Respect to Parameters]
  \label{cor:reg-param}
  Let $U \subset C^{k}$ be an open subset, $ k \in \mathbb{N} \cup \omega$.  
Consider the ode
\begin{gather}
  \label{ode-param-naut}
\frac{dy}{dt} = F \left [ y(x, \tau, t), \tau \right]
\\
y(0) = y_{0}(x), \quad \tau, x \in \ci \ \text{or} \ \rr, t \in \rr
\label{ode-param-init-naut}
\end{gather}
%
%
where $F: U \times \rr  \to C^{k}$ is Lipschitz. If $y_{0} \in C^{k}$, then
there exists an 
open set $I \subset \rr$
such that the ivp \eqref{ode-param-naut}-\eqref{ode-param-init-naut}
admits a unique solution $y(x, \tau, t)$ for $t \in I$. In addition,
this solution is $C^{k}$ in $x$ and $\tau$.
%
\end{corollary}
%
\begin{proof}
  Without loss of generality, we restrict our attention to the homogeneous ivp
\begin{gather}
  \label{ode-param}
\frac{dy}{dt} = F \left [ y(x, \tau, t) \right]
\\
y(0) = y_{0}(x) \in C^{k}, \quad \tau, x \in \ci \ \text{or} \
\rr, t \in \rr.
\label{ode-param-init}
\end{gather}
  %
  The idea will be to view the couple
  $(x, \tau)$ as the spatial component of the ode
  \eqref{ode-param}-\eqref{ode-param-init}.
  To make this rigorous, let $z = (x, \tau)$. Then
  \eqref{ode-param}-\eqref{ode-param-init} is equivalent to the system
  %
  %
  \begin{gather*}
    \frac{d}{dt}(y, z) = \left( F[y(z,t)], \vec{0} \right), \quad \vec 0 =
    (0, 0)
  \\
  (y, z)(0) = \left ( y_{0}(x), \tau \right ).
  \end{gather*}
  %
  %
  Letting $Y = (y,z)$, and $G: U \times \left\{ \vec{0} \right\} 
  \to C^{k} \times \left\{ \vec{0} \right\}$ be defined
  by $G(u, \vec{0}) = \left( F(u), \vec{0} \right)$, we obtain
  %
  %
  \begin{gather*}
      \frac{dY}{dt} = G\left[ Y(z,t) \right]
      \\
      Y(0) = Y_{0}(z)
  \end{gather*}
%
where $Y_{0}(z) = Y_{0}(x, \tau)$ is a constant function of $\tau$. 
Applying Theorem~\ref{ode-thm},  we obtain a solution $Y(z,t)$ which is
$C^{k}$ in $z$ locally in time. Since $z = (x, \tau)$, this implies that $Y(z,t)$ is
$C^{k}$ in $x$ and $\tau$, concluding the proof. 
%
\end{proof}
%
%
We are now prepared to prove the following. 
\begin{theorem}
  Let $U \subset C^{k}$ be an open subset, $k \in \mathbb{N} \cup \omega$. Suppose $G: U \times \rr
  \to C^{k}$
  is Lipschitz. If $y_{0} \in C^{k}$, then there exists an open set $I \subset
  \rr$
such that the ivp 
\begin{gather}
\frac{dy}{dt} = G\left( y(x,t), t \right)
\\
y(0) = y_{0}(x), \quad x \in \ci \ \text{or} \ \rr,t \in \rr
\end{gather}
%
%
admits a unique solution $y(x,t)$ for $t \in I$. In addition,
this solution is $C^{k}$ in $x$ and $t$.
\label{thm:reg-result}
\end{theorem}
%
%
\begin{proof}
Without loss of generality, we restrict our attention to the homogeneous ivp
\begin{gather}
\frac{dy}{dt} = G\left( y(x,t) \right)
\label{cc}
\\
y(0) = y_{0}(x), \quad x \in \ci \ \text{or} \ \rr,t \in \rr.
\label{dd}
\end{gather}

  We choose $F: U \times \rr \to C^{k}$ in \eqref{ode-param-naut}
  such that it is $C^{k}$ and satisfies 
  %
  %
  \begin{equation}
    \label{hjj}
  \begin{split}
    F\left[ y, \tau \right] = \tau F \left[ y, 1 \right].
  \end{split}
  \end{equation}
  %
  Note that \eqref{hjj} does not violate our assumption of $k$-differentiability
  for $F$, since
%
%
\begin{equation*}
\begin{split}
  DF\left[ y, \tau \right]
  & = \left( D_{y}(y, \tau) F, D_{\tau} F[y, \tau] \right)
  \\
  & = (  D_{y} \{\tau F[y, 1] \}, D_{\tau} \{F[y, 1] \} ) 
  \\
  & = ( \tau D_{y} F[y, 1], F[y, 1]).
\end{split}
\end{equation*}
%
%
Then if $y(x, t, \tau)$ is a solution to
\eqref{ode-param-naut}-\eqref{ode-param-init-naut}, we have $y(x, \tau, t ) =
y(x, 1, \tau t)$. To see this, we compute
%
%
\begin{equation*}
\begin{split}
\frac{d}{dt}\left[ y(x, 1, \tau t) \right]
& = \tau y'(x, 1, \tau t)
\\
& = \tau F\left( 1, y(x, 1, \tau t) \right)
\\
& = F(\tau, y(x, 1, \tau t)).
\end{split}
\end{equation*}
%
%
That is, $y(x, 1, \tau t)$ is a solution
\eqref{ode-param-naut}-\eqref{ode-param-init-naut}. By uniqueness, $y(x, \tau, t) =
y(x, 1, \tau t)$. But formally, by the chain rule, we have
%
%
\begin{equation}
  \label{chain-comp}
\begin{split}
\frac{d^{k}y}{d \tau^{k}}(x, \tau, t) 
& = \frac{d}{d \tau^{k}}\left[ y(x, 1, \tau t) \right]
\\
& = \tau^{k} \frac{d^{k}y}{dt^{k}}(x, 1, \tau t) + \dots
\end{split}
\end{equation}
%
%
By Corollary~\ref{cor:reg-param}, $y$ is $C^{k}$ in $\tau$. Hence, from
\eqref{chain-comp}, we conclude that $y$ is $C^{k}$ in $t$. $C^{k}$
differentiability in $x$ follows from Corollary~\ref{cor:reg-param}.
Running an induction, one obtains the corresponding result
for the case $k = \infty$. The case $k = \omega$ is identical to the case
$k \in \mathbb{N}$. Lastly, for any given $G$,
we can find $F$ satisfying \eqref{hjj} and such that
$G(\cdot) = F(\cdot, 1)$. This concludes the proof.
%
\end{proof}

\section{The BBM Equation} 
Setting $\gamma = 0$ in the HR equation, we obtain the Benjamin, Bona,
Mahoney (BBM) equation. That is
%
%
\begin{gather*}
    \p_{t} u = F(u(t))
    \\
     u(x,0) = u_0(x), \; \; x \in \rr, \; \; t \in \rr
\end{gather*}
where
%
%
\begin{equation*}
\begin{split}
    F(u(t)) \doteq -\frac{3}{2} \p_{x}(1 - \p_{x}^{2})^{-1} u^{2}.
\end{split}
\end{equation*}
%
Observe that $F$ maps open sets in $C^{k}$ to $C^{k}$,  $k \in \mathbb N \cup
\{\infty \}$, and is hence an ODE on the space of $C^{k}$ functions. Following
\cite{Bona_2009_Sharp-well-pose}, we note that the BBM is also an ODE on
$H^{s}$, $s \ge 0$ due to the following bilinear estimate.
%
%
%%%%%%%%%%%%%%%%%%%%%%%%%%%%%%%%%%%%%%%%%%%%%%%%%%%%%
%
%
%				bbm bilinear estimate
%
%
%%%%%%%%%%%%%%%%%%%%%%%%%%%%%%%%%%%%%%%%%%%%%%%%%%%%%
%
%
\begin{lemma}
For $s \ge 0$
%
%
\begin{equation*}
\begin{split}
    \| \p_{x}(1 - \p_{x}^{2})^{-1} (uv) \|_{H^{s}} \lesssim \| u \|_{H^{s}} \| v \|_{H^{s}}.
\end{split}
\end{equation*}
%
%
\label{lem:bbm-bilin-est}
\end{lemma}
%
%
%
\begin{proof}
    We diverge slightly from the proof in \cite{Bona_2009_Sharp-well-pose}. Observe that
        \begin{equation*}
        \begin{split}
        \| \p_{x}(1 - \p_{x}^{2})^{-1} (uv) \|_{H^{s}}
        & = \left [ \int_{\rr} \frac{\xi^{2}}{(1 + \xi^{2})^{2}} (1 + |\xi|)^{2s} | \wh{u} * \wh{v}(\xi) |^{2} d \xi \right ]^{1/2}
        \\
        & = \left [ \int_{\rr} \frac{\xi^{2}}{(1 + \xi^{2})^{2}} | (1 + |\xi|)^{s}\wh{u} * \wh{v}(\xi) |^{2} d \xi \right ]^{1/2}
        \\
        & \le \| \xi^{2}/(1 + \xi^{2})^{2} \|_{L^{2}}^{1/2} \| (1 + |\xi|)^{s} \wh{u} * \wh{v}(\xi) \|_{L^{\infty}}
        \\
        & \simeq \| (1 + |\xi|)^{s} \wh{u} * \wh{v}(\xi) \|_{L^{\infty}}.
            \end{split}
\end{equation*}
%
Applying Peetre's inequality (valid only for $s \ge 0$), followed by Young's inequality and Parseval, we bound this by
%
%
%
%
\begin{equation*}
\begin{split}
    \| (1 + |\cdot| )^{s}\wh{u}(\cdot) * (1 + |\cdot| )^{s}\wh{v}(\cdot)(\xi) \|_{L^{\infty}} 
    & \lesssim \| (1 + |\xi| )^{s}\wh{u}(\xi)\|_{L^{2}} \| (1 + |\xi| )^{s}\wh{v}(\xi) \|_{L^{2}}
    \\
    &  = \| u \|_{H^{s}} \| v \|_{H^{s}}
\end{split}
\end{equation*}
%
%
which concludes the proof.
%
\end{proof}
%
%
%
%
%
%
By the ODE theory already presented, for $u_{0}(x) \in C^{k}$, the BBM
admits a unique solution $u(x,t)$ which is $C^{k}$ in both the space and time
variables for $0 \le t < T$, where $T$ > $0$ is sufficiently small. Furthermore, by the same
theory, we see that for $u_{0}(x) \in H^{s}$, $s \ge 0$, the BBM admits a unique
solution $u(x,t) \in C^{1}([0, T], H^{s})$ for sufficiently small $T > 0$. In
both cases, we claim that the data to solution map is smooth. To prove this, by
the ODE theory it will suffice to show that $F$ is smooth. Adopt
the notation
\begin{gather*}
    F(u) = -\frac{3}{2}Lf(u), \ \text{where} \ 
    \\
    \Lambda \doteq \p_{x}(1 - \p_{x}^{2})^{-1}, \ f(u) = u^{2}.
\end{gather*}
Note that $L$ is a bounded linear operator on both $C^{k}$ and $H^{s}$, and
hence $DL(f) = L$ for $f$ in $C^{k}$ or $H^{s}$. Applying the chain rule, we see that for $h \in H^{s}$ or $h \in C^{k}$
we get
%
%
\begin{equation*}
\begin{split}
DF(u) & = -\frac{3}{2}D \Lambda (f(u))Df(u)\\
& = -\frac{3}{2} \Lambda (2uI_{1})
\\
& = -3 \Lambda (uI_{1})
\end{split}
\end{equation*}
%
%
where $\Lambda (uI_{1}) \in L(H^{s}, H^{s})$ via the relation
$\Lambda (uI_{1})(h_{1}) = \Lambda (u h_{1})$ and as a consequence of 
Lemma~\ref{lem:bbm-bilin-est}. 
%
We now compute the second derivative of $F$ at $u_{0}$. Applying the chain rule, we have
%
%
\begin{equation*}
\begin{split}
    D^{2}_{u}F(u) 
    & = D_{u} \left[ \Lambda (u I_{1}) \right](u_{0})
    \\
    & = (D \Lambda)(u_{0}) \circ \left[ D_{u}(u I_{1}) \right](u_{0})
    \\
    & = \Lambda \circ I_{2}I_{1}
\end{split}
\end{equation*}
%
%
where $\Lambda \circ I_{2} I_{1}: H^{s} \to L(H^{s}, H^{s}))$ via the
relation
%
%
\begin{equation*}
\begin{split}
    \Lambda \circ I_{2} I_{1} (h_{2}) = \Lambda \circ h_{2} I_{1}.
\end{split}
\end{equation*}
%
%
For the higher order derivatives, a simple induction completes the proof.
%We now run an induction. Assume $F$ is $k$-times differentiable. Then by the
%Leibniz rule
%%
%%
%\begin{equation*}
%\begin{split}
    %D^{k+1}F(u)(h_{1}, h_{2}, \dots, h_{k+1})
    %& = -3 D^{k} L(uI)(h_{1}, h_{2}, h_{3}, \dots, h_{k+1})
    %\\
    %& = -3 \sum_{j = 0}^{k} {k \choose j} D^{j}u D^{k - j}F(u)
    %\\
    %& = 2 \left[ u D^{k}F(u) + \sum_{j=1}^{k} {k \choose j} I D^{k-j}F(u)
        %\right].
%\end{split}
%\end{equation*}
%%
%which closes the inductive loop.
We conclude that $F$ is smooth. Therefore, the
data-to-solution map for the BBM equation is smooth for $C^{k}$ and $H^{s}$
initial data, $s \ge 0$. Hence, the method of proof developed to prove
non-uniform dependence for HR does not apply to the BBM.
%
%
\subsection*{Acknowledgements.} The author thanks his advisor Alex Himonas for 
his guidance and many helpful remarks. The author also thanks the Department of 
Mathematics at the University of Notre Dame and the Arthur J. Schmitt Foundation for 
supporting his doctoral 
studies.

%
%

