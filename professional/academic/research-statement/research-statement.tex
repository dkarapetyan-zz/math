%
\documentclass[12pt,reqno]{amsart}
\usepackage{amssymb}
\usepackage{appendix}
%\usepackage[T1]{fontenc}%needed in order to use Polish Characters !!!
%\usepackage{ae,aecompl}%to make it all look better using T1
\usepackage{fix-cm}
\usepackage{cancel}  %for cancelling terms explicity on pdf
\usepackage{yhmath}   %makes fourier transform look nicer, among other things
\usepackage{framed}  %for framing remarks, theorems, etc.
\usepackage{enumerate} %to change enumerate symbols
\usepackage[margin=2.5cm]{geometry}  %page layout
\setcounter{tocdepth}{1} %must come before secnumdepth--else, pain
\setcounter{secnumdepth}{1} %number only sections, not subsections
%\usepackage[pdftex]{graphicx} %for importing pictures into latex--pdf compilation
%\numberwithin{figure}{section}
\setlength{\parindent}{0in} %no indentation of paragraphs after section title
\renewcommand{\baselinestretch}{1.1} %increases vert spacing of text
%
\usepackage{hyperref}
\hypersetup{colorlinks=true,
linkcolor=blue,
citecolor=blue,
urlcolor=blue,
}
\usepackage{cleveref} %must be last loaded package to work properly
%
%
\newcommand{\ds}{\displaystyle}
\newcommand{\ts}{\textstyle}
\newcommand{\nin}{\noindent}
\newcommand{\rr}{\mathbb{R}}
\newcommand{\nn}{\mathbb{N}}
\newcommand{\zz}{\mathbb{Z}}
\newcommand{\cc}{\mathbb{C}}
\newcommand{\ci}{\mathbb{T}}
\newcommand{\zzdot}{\dot{\zz}}
\newcommand{\wh}{\widehat}
\newcommand{\p}{\partial}
\newcommand{\ee}{\varepsilon}
\newcommand{\vp}{\varphi}
\newcommand{\wt}{\widetilde}
%
%
%
%
\newtheorem{theorem}{Theorem}
\newtheorem{lemma}[theorem]{Lemma}
\newtheorem{corollary}[theorem]{Corollary}
\newtheorem{claim}[theorem]{Claim}
\newtheorem{prop}[theorem]{Proposition}
\newtheorem{proposition}[theorem]{Proposition}
\newtheorem{no}[theorem]{Notation}
\newtheorem{definition}[theorem]{Definition}
\newtheorem{remark}[theorem]{Remark}
\newtheorem{examp}{Example}[section]
\newtheorem {exercise}[theorem] {Exercise}
%
%\makeatletter \renewenvironment{proof}[1][\proofname] {\par\pushQED{\qed}\normalfont\topsep6\p@\@plus6\p@\relax\trivlist\item[\hskip\labelsep\bfseries#1\@addpunct{.}]\ignorespaces}{\popQED\endtrivlist\@endpefalse} \makeatother%
%%makes proof environment bold instead of italic
\newcommand{\uol}{u^\omega_\lambda}
\newcommand{\lbar}{\bar{l}}
\renewcommand{\l}{\lambda}
\newcommand{\R}{\mathbb R}
\newcommand{\RR}{\mathcal R}
\newcommand{\al}{\alpha}
\newcommand{\ve}{q}
\newcommand{\tg}{{tan}}
\newcommand{\m}{q}
\newcommand{\N}{N}
\newcommand{\ta}{{\tilde{a}}}
\newcommand{\tb}{{\tilde{b}}}
\newcommand{\tc}{{\tilde{c}}}
\newcommand{\tS}{{\tilde S}}
\newcommand{\tP}{{\tilde P}}
\newcommand{\tu}{{\tilde{u}}}
\newcommand{\tw}{{\tilde{w}}}
\newcommand{\tA}{{\tilde{A}}}
\newcommand{\tX}{{\tilde{X}}}
\newcommand{\tphi}{{\tilde{\phi}}}
\synctex=1
%
%
%
\begin{document}
\title{Research Statement} 
\author{{\it David Karapetyan}\\
    \/ University of Notre Dame}
\maketitle


\parindent0in
\parskip0.1in
%%%%%%%%%%%%%%%%%%%%%%%%
%
%      introduction
%
%%%%%%%%%%%%%%%%%%%%%%%%
%
\setcounter{section}{0}
%
%
%%%%%%%%%%%%%%%%%%%%%%%%%%%%%%%%%%%%%%%%%%%%%%%%%%%%%
%
%
%				Current Research
%
%
%%%%%%%%%%%%%%%%%%%%%%%%%%%%%%%%%%%%%%%%%%%%%%%%%%%%%
%
%
\subsection{Introduction} 
\label{ssec:cur-res}
The main focus of my research so far has been well-posedness and \\ ill-posedness
for the hyperelastic rod equation (HR)
\begin{align}
\label{hr}
& \p_t u
-
\p_t \p_x^2 u
+
3u\p_x u
=
\gamma \big (
2\p_x u \p_x^2 u
+
u \p_x^3 u
\big ),
\\
\label{hr-data} 
& u(x, 0) = u_0 (x),
\quad x  \in \ci \ \text{or} \ \rr, \ t \in \rr.
\end{align}
The HR equation was first derived by Dai in \cite{Dai_1998_Model-equations} as a
one-dimensional model for finite-length and small-amplitude axial deformation
waves in thin cylindrical rods composed of a compressible Mooney-Rivlin
material. The derivation relied upon a reductive perturbation technique, and
took into account the nonlinear dispersion of pulses propagating along a rod. It
was assumed that each cross-section of the rod is subject to a stretching and
rotation in space. The solution $u(x,t)$ to the HR equation represents the
radial stretch relative to a pre-stressed state, while $\gamma$ is a fixed
constant depending upon the pre-stress and the material used in the rod, with
values ranging from $- 29.4760$ to $3.4174$.

Writing HR in its nonlocal form
$$
u_{t} + \gamma u u_{x}+ \p_{x} (1 - \p_{x}^{2})^{-1} \left [ \frac{3 - \gamma}{2}u^{2} + \frac{\gamma}{2} u_{x}^{2} \right ]=0
$$
we see that the HR equation can be thought of as a 
weakly dispersive perturbation of
Burgers equation, unlike the Korteweg-de Vries (KDV) equation
\begin{align}
     u_{t} + u u_{x} + u_{xxx}=0,
\end{align}
which is a dispersive perturbation of Burgers equation.
%
Furthermore, unlike KDV,
the HR equation admits traveling wave solutions which are not smooth. In fact,
it admits a wide variety of traveling wave solutions, some of which break in
finite time. A complete classification can be found in the work of Lenells
\cite{Lenells_2006_Traveling-waves}.  These traveling wave solutions are
obtained using various combinations of peaks, cusps, compactons, fractal-like
waves, and plateaus. Orbital stability of solitary wave solutions was proved in
Constantin and Strauss \cite{Constantin_2000_Stability-of-a-}.  Solitary shock wave formation was
analyzed in Dai and Huo \cite{Dai_2000_Solitary-shock-} using traveling wave
solutions of the HR equation to derive a system of ordinary differential
equations, with a vertical singular line in the phase plane corresponding with
the formation of shock waves. Head-on collisions between two solitary waves was
investigated in the work of Hui-Hui Dai, Shiqiang Dai, and Huo
\cite{Dai_2000_Head-on-collisi} using a reductive perturbation method coupled
with the technique of strained coordinates. 

Among the most interesting of the breaking
traveling wave solutions are the ``peakons" and
``cuspons". The primary difference between the peakons and cuspons is that the
peakons have bounded derivative, whereas the cuspons do not. Visually, the
peakons are crest shaped, whereas the cuspons are much steeper. 

We note that the HR equation only admits peakons for certain $\gamma$. 
For example, it does not admit peakons for $\gamma = 0$. In fact, this is a
special equation known as the BBM, proposed by 
Benjamin, Bona, and Mahony 
\cite{Benjamin_1972_Model-equations} as a model for 
the unidirectional evolution of long waves. This equation has only smooth
traveling wave solutions. Furthermore, 
its solitary-wave solutions are global and orbitally stable (see Benjamin 
\cite{Benjamin_1972_The-stability-o}, 
\cite{Benjamin_1972_Model-equations}, and 
\cite{Constantin_2000_Stability-of-a-}).
On the other hand, if we set $\gamma =1$, we obtain the Camassa-Holm equation, a celebrated bi-Hamiltonian
model for the evolution of shallow water waves in a channel, with the notable
property that it does possess traveling wave solutions which break in
finite time (including peakons and cuspons). The CH equation has an extensive
literature. For more information, see Himonas-Kenig \cite{Himonas:2010ch} and the references
therein.


%
%
\subsection{Results Obtained on HR} 
\label{ssec:weak-disp}
%
%%%%%%%%%%%%%%%%%%%%%%%%
%
%
%    Theorem:  hr-non-unif-dependence
%
%
%%%%%%%%%%%%%%%%%%%%%
%
%
%
%
%
I recently published a paper on the HR equation entitled 
{\it ``Non-uniform dependence and well-posedness for the hyperelastic rod equation''}. 
In it, it is proved that the dependence of
solutions on initial data is not uniformly continuous in Sobolev spaces
$H^s$, $s>3/2$.  More precisely: 
\begin{theorem}
\label{hr-non-unif-dependence}
Let $\gamma$ be a nonzero constant. Then 
the data-to-solution map $u_{0} \mapsto u$ of the Cauchy-problem
for the HR equation
\eqref{hr}-\eqref{hr-data}
is not uniformly continuous
from any bounded subset of  $H^s$ into $C([-T, T], H^s)$
for $s>3/2$ on the line and the circle.
%
\end{theorem}
%
This extended a result proved by Olson \cite{Olson_2006_Non-uniform-dep} in the
periodic case (for $s\ge 2$ and $\gamma \ne 3$)  to  $s>3/2$ (the entire
well-posedness range) for HR\@. Furthermore, motivated by the work of Himonas
and Kenig~\cite{Himonas:2009fk} and Himonas, Kenig, and
Misiolek~\cite{Himonas:2010ch}, I established non-uniform dependence on initial
data in the non-periodic case.  The method of proof involved choosing
approximate solutions to the HR equation
such that the size of the difference between approximate and actual solutions
with identical initial data was negligible. Hence, to understand the degree of
dependence, it sufficed to focus on the behavior of the approximate solutions
(which are simple in form), rather than on the behavior of the actual solutions.
In order for the method to go through, it was necessary to prove well-posedness estimates
for the size of the actual solutions to the HR equation, as well a lower bound
for their lifespan. This was needed in order to obtain an upper bound for the
size of the difference of approximate and actual solutions. More precisely, 
the following was proved: 
%
\begin{theorem}
\label{thm:HR_existence_continuous_dependence}
If   $s>3/2$,  then we have:

(i) If $u_0\in H^s$ then there exists a unique solution to
the Cauchy problem  \eqref{hr}--\eqref{hr-data} in \\ $C([-T, T], H^s)$, where 
the lifespan $T$ depends on the size
of the initial data $u_0$. Moreover, 
the lifespan $T$ satisfies the lower bound estimate 
%
%
%
\begin{equation}
\label{Life-span-est}
T
\ge
\frac{1}{2c_s \|u_0\|_{H^s}}.
\end{equation}
%

(ii)
The data-to-solution map $u_0 \mapsto u$ is continuous from
bounded sets of $H^s$ into \\ $C([-T, T], H^s)$,
and the solution $u$ satisfies the estimate
%
%
%
\begin{equation}
\label{u_x-Linfty-Hs}
\|
u(t)
\|_ {H^s}
\le
2
\|
u_0
\|_{H^s}, \ \ |t|\le T.
\end{equation}
%
%
%
\end{theorem}

Recall that when  $\gamma=0$ the HR equation becomes the BBM
equation. Bona and Tzvetkov \cite{Bona_2009_Sharp-well-pose} have recently
proved  that this equation  is globally well-posed in  Sobolev spaces $H^s$ for
$s \ge 0$, and that its data-to-solution map is smooth. Hence, non-uniform
dependence for HR cannot extend to the case $\gamma = 0$. 

Motivated by the failure of non-uniform continuity for HR for $\gamma \neq 0$ in
the $H^{s}$ topology and the work of Chen \cite{Chen:2011fk}, I recently also
proved the following:
%
%
\begin{theorem}
For $\gamma \neq 0$, the
data-to-solution map for HR is H\"older continuous from $B_{H^{s}}(R)$ (measured in the $H^{r}$ norm)
to $C([0, T], H^{r})$, where $T = T(R)$, for $s >
3/2$, $-1 \le r < s$. 
%
More precisely, for two initial data $u_{0}, v_{0} \in B_{H^{s}}(R)$, $s > 3/2$, there exist unique
corresponding solutions $u(x,t), v(x,t)$ for $0 \le t \le T= T(R)$ to the
HR equation which satisfy 
%
%
\begin{equation*}
\begin{split}
  \| u(t) - v(t) \|_{H^{r}} \le C \| u_{0} - v_{0} \|_{H^{r}}^{\alpha(s, r)},
  \quad 0
  \le t \le T
\end{split}
\end{equation*}
%
%
where  
%
%
\begin{equation*}
\begin{split}
\alpha = 
\begin{cases}
    1,  \quad  & -1 \le r \le s-1, \ s + r \ge 2  
  \\
  2(s-1)/(s-r), \quad  &  -1 \le r < 2-s 
  \\
  s-r, \quad  & s-1 < r < s. 
\end{cases}
\end{split}
\end{equation*}
  %
\label{thm:main-thm}
\end{theorem}
This result was proved in \cite{Chen:2011fk} for the $b$-family in the
non-periodic case.
%
%
\subsection{Future Research} 
\label{ssec:fut-res}
 I am interested in strengthening the well-posedness
result for HR from failure of uniform continuity of the data-to-solution map
when $s<3/2$ to failure of continuity for sufficiently rough initial data, that
is, $s<3/2$. More generally, I am interested in proving ill-posedness in the
sense of Hadamard (failure of existence, uniqueness, or
continuity for the data-to-solution map). In order to obtain ill-posedness
results for HR, I have begun researching a series of other equations. Since HR is a perturbation of the
inviscid Burgers equation
$
    u_{t} + u u_{x} = 0,
$
and Burgers is well-posed for initial data in Sobolev space $H^{s}$, $s > 3/2$,
it is natural to ask whether Burgers is ill-posed for $s \le 3/2$. I intend to
pursue this question in the future, in order to shed-light on ill-posedness
issues for equations involving a perturbation of Burgers (such as HR).

Another approach to tackling the ill-posedness question for HR is to better
understand the framework given in the work of De Lelllis, Kappeler, and Topalov
\cite{Lellis_2007_Low-regularity-} for constructing low-regularity solutions to
the Camassa-Holm, and then applying these techniques to the HR equation. More
precisely, for initial data $u_{0} \in H^{s}$, $s =1$ or $s > 3/2$, the HR
well-posedness theory states that there exist corresponding classical solutions
$u(x,t) \in H^{s}$. However, what occurs in the region $1 < s \le 3/2$ is
unclear. De Lelllis et al.\ addressed this gap for the Camassa-Holm in the
periodic case. I am of the hope that understanding their work will give greater
insight into proving ill-posedness for HR in Sobolev spaces with index $s \le
3/2$. 

Also, in collaboration with Dan Geba and Alex Himonas,
I am investigating failure of the continuity of the data-to-solution map for
initial data in $H^{s} \times H^{s-2}$ for the Boussinesq equation
\begin{align}
   & u_{tt} - u_{xx} + u_{xxxx} + (u^{2})_{xx} = 0,
  \\
   & u(x,0) = u_{0}(x) \in H^{s}, \  u_{t}(x,0) = u_{1}(x) \in H^{s-2},
  \quad x \in \ci \ \text{or} \ \rr, \ t \in \rr. 
\end{align} 
Farah and Scialom \cite{Farah:2010ys}  have shown that this equation is well-posed on the circle
when $s>-1/4$. On the line, well-posedness was proved 
by Farah \cite{Farah:2009uq} when $s>-1/4$. Kishimoto and Tsugawa
\cite{Kishimoto:2010ly} then improved this result to $s>-1/2$.  I  plan to
investigate whether the Sobolev exponents 
$s=-1/4$ (in the circle case) and  $s=-1/2$ (in the line case) 
are optimal concerning well-posedness in Sobolev spaces.
In the non-periodic case, it is known  \cite{Farah:2009uq} that
the data-to-solution map is not $C^2$.
I would like to explore failure of continuity
for the data-to-solution map for both the periodic
and non-periodic cases. One possible approach is motivated by Birnir
\cite{Birnir:1996uq}. That is, to prove the failure of continuity of the
data-to-solution map, I am attempting to find a sequence of traveling wave
solutions $u_{n}(x,t)$ to the Boussinesq with corresponding initial data
$(u_{0,n}, u_{1,n})$ in a sufficiently rough product of Sobolev spaces $H^{s} \times
H^{s-2}$ such that $(u_{0,n}, u_{1,n}) \to (u_{0}, u_{1})$ in
$H^{s} \times H^{s-2}$ but $u_{n}(x,t) \not \to u(x,t)$ in $H^{s}$. To get clues as to
what traveling wave solutions might help prove failure of continuity, I
have been using MATLAB to numerically solve various ODEs with adjustable
parameters, in addition to running other numerical experimentations in MATLAB
and C++. 

Finally, I am involved in a project of Alex Himonas and Gerson Petronilho about
Gevrey and analytic regularity of the NLS equation.  While attending the {\em
    ``VI Workshop on Geometric Analysis of PDEs and Several Complex Variables"}
\/in Sao Paulo, Brazil last August, I had the opportunity to explore ways for
constructing non-analytic in time solutions of the NLS equation when the initial
data are analytic. 
\providecommand{\bysame}{\leavevmode\hbox to3em{\hrulefill}\thinspace}
\providecommand{\MR}{\relax\ifhmode\unskip\space\fi MR }
% \MRhref is called by the amsart/book/proc definition of \MR.
\providecommand{\MRhref}[2]{%
  \href{http://www.ams.org/mathscinet-getitem?mr=#1}{#2}
}
\providecommand{\href}[2]{#2}
\begin{thebibliography}{dLKT07}

\bibitem[BBM72]{Benjamin_1972_Model-equations}
T.~B. Benjamin, J.~L. Bona, and J.~J. Mahony, \emph{{Model equations for long
  waves in nonlinear dispersive systems}}, Philosophical Transactions of the
  Royal Society of London. Series A. Mathematical and Physical Sciences
  \textbf{272} (1972), no.~1220, 47--78.

\bibitem[Ben72]{Benjamin_1972_The-stability-o}
T.~B. Benjamin, \emph{{The stability of solitary waves}}, Proc. Roy. Soc.
  (London) Ser. A \textbf{328} (1972), 153--183.

\bibitem[BPS96]{Birnir:1996uq}
B.~Birnir, G.~Ponce, and N.~Svanstedt, \emph{{The local ill-posedness of the
  modified KdV equation}}, Annales de l'Institut Henri Poincar{\'e}. Analyse
  Non Lin{\'e}aire \textbf{13} (1996), no.~4, 529--535.

\bibitem[BT09]{Bona_2009_Sharp-well-pose}
J.~L. Bona and N.~Tzvetkov, \emph{{Sharp well-posedness results for the BBM
  equation}}, Discrete Contin. Dyn. Syst. \textbf{23} (2009), no.~4,
  1241--1252.

\bibitem[Che11]{Chen:2011fk}
R.M. Chen, Y. Liu, and P.Z. Zhang, \emph{{The H{\"o}lder Continuity of the Solution Map to the
  b-family Equation in weak topology}}, submitted.

\bibitem[CS00]{Constantin_2000_Stability-of-a-}
A.~Constantin and W.~A. Strauss, \emph{{Stability of a class of solitary waves
  in compressible elastic rods}}, Phys. Lett. A \textbf{270} (2000), no.~3-4,
  140--148.

\bibitem[Dai98]{Dai_1998_Model-equations}
H.-H. Dai, \emph{{Model equations for nonlinear dispersive waves in a
  compressible Mooney-Rivlin rod}}, Acta Mech. \textbf{127} (1998), no.~1-4,
  193--207.

\bibitem[DDH00]{Dai_2000_Head-on-collisi}
H.-H. Dai, S.~Dai, and Y.~Huo, \emph{{Head-on collision between two solitary
  waves in a compressible Mooney-Rivlin elastic rod}}, Wave Motion \textbf{32}
  (2000), no.~2, 93--111.

\bibitem[DH00]{Dai_2000_Solitary-shock-}
H.-H. Dai and Y.~Huo, \emph{{Solitary shock waves and other travelling waves in
  a general compressible hyperelastic rod}}, R. Soc. Lond. Proc. Ser. A Math.
  Phys. Eng. Sci. \textbf{456} (2000), no.~1994, 331--363.

\bibitem[dLKT07]{Lellis_2007_Low-regularity-}
C.~de~Lellis, T.~Kappeler, and P.~Topalov, \emph{{Low-regularity solutions of
  the periodic Camassa-Holm equation}}, Communications in Partial Differential
  Equations \textbf{32} (2007), no.~1-3, 87--126.

\bibitem[Far09]{Farah:2009uq}
L.~G. Farah, \emph{{Local solutions in Sobolev spaces with negative indices for
  the ``good'' Boussinesq equation}}, Communications in Partial Differential
  Equations \textbf{34} (2009), no.~1-3, 52--73.

\bibitem[FS10]{Farah:2010ys}
L.~G. Farah and M.~Scialom, \emph{{On the periodic ``good'' Boussinesq
  equation}}, Proceedings of the American Mathematical Society \textbf{138}
  (2010), no.~3, 953--964.

\bibitem[HK09]{Himonas:2009fk}
A.~Himonas and C.~E. Kenig, \emph{{Non-uniform dependence on initial data for
  the CH equation on the line}}, Differential and Integral Equations. An
  International Journal for Theory \& Applications \textbf{22} (2009),
  no.~3-4, 201--224.

\bibitem[HKM10]{Himonas:2010ch}
A. Himonas, C.~Kenig, and G.~Misiolek, \emph{{Non-uniform dependence for
  the periodic CH equation}}, Communications in Partial Differential Equations
  \textbf{35} (2010), no.~6, 1145--1162.
  
  \bibitem[KT10]{Kishimoto:2010ly}
N.~Kishimoto and K.~Tsugawa, \emph{Local well-posedness for quadratic nonlinear
  {S}chr{\"o}dinger equations and the ``good'' {B}oussinesq equation},
  Differential Integral Equations \textbf{23} (2010), no.~5-6, 463--493.
  

\bibitem[Len06]{Lenells_2006_Traveling-waves}
J.~Lenells, \emph{{Traveling waves in compressible elastic rods}}, Discrete
  Contin. Dyn. Syst. Ser. B \textbf{6} (2006), no.~1, 151--167 (electronic).

\bibitem[Ols06]{Olson_2006_Non-uniform-dep}
E.~Olson, \emph{{Non-uniform dependence on initial data for a family of
  non-linear evolution equations}}, Differential and Integral Equations. An
  International Journal for Theory \& Applications \textbf{19} (2006),
  no.~10, 1081--1102.

\end{thebibliography}
%\bibliography{/Users/davidkarapetyan/math/bib-files/references-papers}
%\bibliographystyle{amsalpha-custom}
\end{document}

