%
\documentclass[12pt,reqno]{amsart}
\usepackage{amssymb}
\usepackage{appendix}
\usepackage{fix-cm}
\usepackage{cancel}  %for cancelling terms explicity on pdf
\usepackage{yhmath}   %makes fourier transform look nicer, among other things
\usepackage{framed}  %for framing remarks, theorems, etc.
\usepackage{enumerate} %to change enumerate symbols
\usepackage[margin=2.5cm]{geometry}  %page layout
\setcounter{tocdepth}{1} %must come before secnumdepth--else, pain
\setcounter{secnumdepth}{1} %number only sections, not subsections
%\usepackage[pdftex]{graphicx} %for importing pictures into latex--pdf compilation
%\numberwithin{figure}{section}
\setlength{\parindent}{0in} %no indentation of paragraphs after section title
\renewcommand{\baselinestretch}{1.1} %increases vert spacing of text
%
\usepackage{hyperref}
\hypersetup{colorlinks=true,
linkcolor=blue,
citecolor=blue,
urlcolor=blue,
}
\usepackage{cleveref} %must be last loaded package to work properly
%
%
\newcommand{\ds}{\displaystyle}
\newcommand{\ts}{\textstyle}
\newcommand{\nin}{\noindent}
\newcommand{\rr}{\mathbb{R}}
\newcommand{\nn}{\mathbb{N}}
\newcommand{\zz}{\mathbb{Z}}
\newcommand{\cc}{\mathbb{C}}
\newcommand{\ci}{\mathbb{T}}
\newcommand{\zzdot}{\dot{\zz}}
\newcommand{\wh}{\widehat}
\newcommand{\p}{\partial}
\newcommand{\ee}{\varepsilon}
\newcommand{\vp}{\varphi}
\newcommand{\wt}{\widetilde}
%
%
%
%
\newtheorem{theorem}{Theorem}
\newtheorem{lemma}[theorem]{Lemma}
\newtheorem{corollary}[theorem]{Corollary}
\newtheorem{claim}[theorem]{Claim}
\newtheorem{prop}[theorem]{Proposition}
\newtheorem{proposition}[theorem]{Proposition}
\newtheorem{no}[theorem]{Notation}
\newtheorem{definition}[theorem]{Definition}
\newtheorem{remark}[theorem]{Remark}
\newtheorem{examp}{Example}[section]
\newtheorem {exercise}[theorem] {Exercise}
%
%\makeatletter \renewenvironment{proof}[1][\proofname] {\par\pushQED{\qed}\normalfont\topsep6\p@\@plus6\p@\relax\trivlist\item[\hskip\labelsep\bfseries#1\@addpunct{.}]\ignorespaces}{\popQED\endtrivlist\@endpefalse} \makeatother%
%%makes proof environment bold instead of italic
\newcommand{\uol}{u^\omega_\lambda}
\newcommand{\lbar}{\bar{l}}
\renewcommand{\l}{\lambda}
\newcommand{\R}{\mathbb R}
\newcommand{\RR}{\mathcal R}
\newcommand{\al}{\alpha}
\newcommand{\ve}{q}
\newcommand{\tg}{{tan}}
\newcommand{\m}{q}
\newcommand{\N}{N}
\newcommand{\ta}{{\tilde{a}}}
\newcommand{\tb}{{\tilde{b}}}
\newcommand{\tc}{{\tilde{c}}}
\newcommand{\tS}{{\tilde S}}
\newcommand{\tP}{{\tilde P}}
\newcommand{\tu}{{\tilde{u}}}
\newcommand{\tw}{{\tilde{w}}}
\newcommand{\tA}{{\tilde{A}}}
\newcommand{\tX}{{\tilde{X}}}
\newcommand{\tphi}{{\tilde{\phi}}}
\synctex=1
%
%
%
\begin{document}
\title{Research Statement} 
\author{{\it David Karapetyan}\\
    \/ University of Notre Dame}
\maketitle


\parindent0in
\parskip0.1in
%%%%%%%%%%%%%%%%%%%%%%%%
%
%      introduction
%
%%%%%%%%%%%%%%%%%%%%%%%%
%
\setcounter{section}{0}
%
%
%%%%%%%%%%%%%%%%%%%%%%%%%%%%%%%%%%%%%%%%%%%%%%%%%%%%%
%
%
%				Current Research
%
%
%%%%%%%%%%%%%%%%%%%%%%%%%%%%%%%%%%%%%%%%%%%%%%%%%%%%%
%
%
\subsection{Introduction} 
\label{ssec:cur-res}
The main focus of my research so far has been well-posedness and ill-posedness
for the hyperelastic rod equation (HR)
\begin{gather}
\label{hr}
\p_t u
-
\p_t \p_x^2 u
+
3u\p_x u
=
\gamma \big (
2\p_x u \p_x^2 u
+
u \p_x^3 u
\big )
\\
\label{hr-data} u(x, 0) = u_0 (x),
\quad x  \in \ci, \ \ \text{or} \ \ \rr \quad t \in \rr,
\end{gather}
The HR equation was first derived by Dai in \cite{Dai_1998_Model-equations} as a
one-dimensional model for finite-length and small-amplitude axial deformation
waves in thin cylindrical rods composed of a compressible Mooney-Rivlin
material. The derivation relied upon a reductive perturbation technique, and
took into account the nonlinear dispersion of pulses propagating along a rod. It
was assumed that each cross-section of the rod is subject to a stretching and
rotation in space. The solution $u(x,t)$ to the HR equation represents the
radial stretch relative to a pre-stressed state, while $\gamma$ is a fixed
constant depending upon the pre-stress and the material used in the rod, with
values ranging from $- 29.4760$ to $3.4174$.

By observation, we see that the HR equation can be seen as a perturbation of
Burgers, much like the Korteweg-de Vries (KDV) equation.  However, unlike KDV,
the HR equation admits traveling wave solutions which are not smooth. In fact,
it admits a wide variety of traveling wave solutions, some of which break in
finite time. A complete classification can be found in the work of Lenells
\cite{Lenells_2006_Traveling-waves}.  These traveling wave solutions are
obtained using various combinations of peaks, cusps, compactons, fractal-like
waves, and plateaus. Orbital stability of solitary wave solutions was proved in
\cite{Constantin_2000_Stability-of-a-}.  Solitary shock wave formation was
analyzed in Dai and Huo \cite{Dai_2000_Solitary-shock-} using traveling wave
solutions of the HR equation to derive a system of ordinary differential
equations, with a vertical singular line in the phase plane corresponding with
the formation of shock waves. Head-on collisions between two solitary waves was
investigated in the work of Hui-Hui Dai, Shiqiang Dai, and Huo
\cite{Dai_2000_Head-on-collisi} using a reductive perturbation method coupled
with the technique of strained coordinates. 

Among the most interesting of the breaking
traveling wave solutions are the "peakons" and
"cuspons". These primary difference between the peakons and cuspons is that the
peakons have bounded derivative, whereas the cuspons do not. Visually, the
peakons are crest shaped, whereas the cuspons are much steeper. 

We note that the HR equation only admits peakons for certain $\gamma$. 
For example, it does not admit peakons for $\gamma = 0$. In fact, this is a
special equation known as the BBM, proposed by 
Benjamin, Bona, and Mahony 
\cite{Benjamin_1972_Model-equations} as a model for 
the unidirectional evolution of long waves. This equation has only smooth
traveling wave solutions. Furthermore, 
its solitary-wave solutions are global and orbitally stable (see Benjamin 
\cite{Benjamin_1972_The-stability-o}, 
\cite{Benjamin_1972_Model-equations}, and 
\cite{Constantin_2000_Stability-of-a-}).
Conversely, if we set $\gamma =1$, we obtain the Camassa-Holm equation, a celebrated bi-Hamiltonian
model for the evolution of shallow water waves in a channel, with the notable
property that it does possess traveling wave solutions which break in
finite time (including peakons and cuspons). It has an extensive literature.
%
%
\subsection{Results Obtained on HR} 
\label{ssec:weak-disp}
%
%%%%%%%%%%%%%%%%%%%%%%%%
%
%
%    Theorem:  hr-non-unif-dependence
%
%
%%%%%%%%%%%%%%%%%%%%%
%
%
%
%
%
I recently published a paper on the HR equation entitled 
{\it ``Non-Uniform Dependence  and well-posedness
for the
Hyperelastic Rod Equation''}. In it, I prove
that 
the dependence of solutions on initial data is not uniformly 
continuous in Sobolev spaces $H^s(\ci)$, $s>3/2$.
More precisely, I proved the following.
\begin{theorem}
\label{hr-non-unif-dependence}
Let $\gamma$ be a nonzero constant. Then 
the data-to-solution map $u(0) \mapsto u(t)$ of the Cauchy-problem
for the HR equation
\eqref{hr}-\eqref{hr-data}
is not uniformly continuous
from any bounded subset of  $H^s$ into $C([-T, T], H^s)$
for $s>1$ on the line  and for $s>3/2$ on the circle.
%
\end{theorem}
This extended a result proved by Olson 
\cite{Olson_2006_Non-uniform-dep} in the periodic
case (for $s\ge 2$ and $\gamma \ne 3$)  to  $s>3/2$ (the entire well-posedness
range) for HR\@. Furthermore, motivated by the work of Himonas and
Kenig~\cite{Himonas:2009fk}, \cite{Himonas:2009fk}, and 
Himonas, Kenig, and Misiolek~\cite{Himonas_2009_Non-uniform-dep-per},
I established non-uniform dependence on initial data in the non-periodic case.
The method of proof involved choosing approximate solutions to the HR equation
such that the size of the difference between approximate and actual solutions
with identical initial data was negligible. Hence, to understand the degree of
dependence, it sufficed to focus on the behavior of the approximate solutions
(which are simple in form), rather than on the behavior of the actual solutions.
In order for the method to go through, I also proved well-posedness estimates
for the size of the actual solutions to the HR equation, as well a lower bound
for their lifespan. This was needed in order to obtain an upper bound for the
size of the difference of approximate and actual solutions. More precisely, I
showed the following.
\begin{theorem}
\label{thm:HR_existence_continuous_dependence}
If   $s>3/2$  then we have:

(i) If $u_0\in H^s$  then  there exists a unique solution to
the Cauchy problem  \eqref{hr}--\eqref{hr-data} in $C([-T, T], H^s)$, where 
the lifespan  $T$ depends on the size
of the initial data $u_0$. Moreover, 
the  lifespan $T$ satisfies the lower bound estimate 
%
%
%
\begin{equation}
\label{Life-span-est}
T
\ge
\frac{1}{2c_s \|u_0\|_{H^s}}.
\end{equation}
%

(ii)
The flow map $u_0 \mapsto u(t)$ is continuous from
bounded sets of $H^s$ into \\ $C([-T, T], H^s)$,
and the solution $u$ satisfies the estimate
%
%
%
\begin{equation}
\label{u_x-Linfty-Hs}
\|
u(t)
\|_ {H^s}
\le
2
\|
u_0
\|_{H^s}, \ \ |t|\le T.
\end{equation}
%
%
%
\end{theorem}

Recall that when  $\gamma=0$ the HR equation becomes the BBM
equation.  Bona and Tzvetkov \cite{Bona_2009_Sharp-well-pose} have recently
proved  that this equation  is globally well-posed in  Sobolev spaces $H^s$, if
$s \ge 0$, and that its data-to-solution map is smooth. Hence, the failure of
non-uniform continuity for HR does not apply in this case. 

Motivated by the failure of non-uniform continuity for HR for $\gamma \neq 0$ in
the $H^{s}$ topology and the work of Chen \cite{Chen:2011fk}, I recently also proved the following.
%
%
\begin{theorem}
For $\gamma \neq 0$, the
data to solution map for HR is H\"older continuous from $B_{H^{s}}(R)$ (in
the topology of $H^{r}$) to $C([0, T], H^{r})$, where $T = T(R)$, for $s >
3/2$, $-1 \le r < s$. More
precisely, decompose the set $\Omega = \left\{ (s, r) \in \rr^{2}  \right\}$
into the pieces
  %
  %
  \begin{equation*}
  \begin{split}
    & \Omega_{1} = \left\{ (s, \ r):  \ s < 3/2 \right\}
    \\
    & \Omega_{2} = \left\{ (s, \ r):
     \ s>3/2, \ r < -1  \right\}
    \\
    & \Omega_{3} = \left\{ (s, \ r):
     \ s>3/2, \ r > s  \right\}
    \\
    & \Omega_{4} = \left\{ (s, \ r):
     \ s>3/2, \ -1 \le r \le s-1, \ s + r \ge 2  \right\}
    \\
    & \Omega_{5} = \left\{ (s, \ r):
     \ s>3/2, \ -1 \le r < 2-s \right\}
    \\
    & \Omega_{6} = \left\{ (s, \ r):
    \  s>3/2, \  s-1 < r < s  \right\}.
    \end{split}
\end{equation*}
  %
  %
\label{thm:main-thm}
\end{theorem}
%
%
This result improves upon the H\"older continuity result in
\cite{Chen:2011fk} (there it is for $0 \le r < s$) by a full degree in $r$. 
\subsection{Future Research} 
\label{ssec:fut-res}
I am currently
interested in strengthening the ill-posedness result for HR from failure of
uniform continuity of the data-to-solution map to failure of continuity for
sufficiently rough initial data. More generally, I am interested in disproving
well-posedness in the sense of Hadamard (failure of existence, uniqueness, or
continuity for the data-to-solution map). In order to obtain ill-posedness
results for HR, I have begun researching a series of other equations. Since HR is a perturbation of the
inviscid Burgers initial value problem 
\begin{gather*}
    u_{t} + u u_{x} = 0
    \\
    u(x, 0) = u_{0}(x), \quad x \in \ci \ \text{or} \rr, \ t \in \rr,
\end{gather*}
and Burgers is well-posed for initial data in Sobolev space $H^{s}$, $s > 3/2$,
it seems natural to ask whether Burgers is ill-posed for $s \le 3/2$. I intend to
pursue this question in the future, in order to shed-light on ill-posedness
issues for equations involving a perturbation of Burgers (such as HR).

Another approach to tackling the ill-posedness question for HR
is to better understand the framework given in the work De Lelllis,
Kappeler, and Topalov \cite{Lellis_2007_Low-regularity-} for constructing
low-regularity solutions to the Camassa-Holm, and then applying these techniques
to the HR equation. More precisely, for initial data $u_{0} \in H^{s}$, $s =1$
or $s > 3/2$, the HR well-posedness theory states that there exist corresponding
classical solutions $u(x,t) \in H^{s}$. However, what occurs in the regime $1 <
s \le 3/2$ is unclear. De Lelllis et al.\ addressed this gap for the
Camassa-Holm. I am of the hope that understanding their work will give greater
insight into proving ill-posedness for HR in Sobolev spaces with index $s \le
3/2$. 

I am also interested in proving failure of the continuity of the data-to-solution map for
initial data in $H^{s} \times H^{s-2}, s < -1/2$ for the Boussinesq 
\begin{gather}
   u_{tt} - u_{xx} + u_{xxxx} + (u^{2})_{xx} = 0,
  \\
   u(x,0) = u_{0}(x) \in H^{s}, \  u_{t}(x,0) = u_{1}(x) \in H^{s-2},
  \quad t \in \rr, \ x \in \ci \ \text{or} \ \rr.
\end{gather} 
equation. I am using three
different approaches to gain insight into the problem. The first is motivated by
\cite{Bejenaru-Tao-2006-Sharp-well-posedness-and-ill-posedness}. More
specifically, the approach runs as follows:
\begin{enumerate}[I.]
  \item
    Prove well-posedness in $H^{s} \times H^{s-2}, s \ge -1/2$.
 \item
   Show that the well-posedness implies that for a solution $u(x,t)$ with
   initial data $u_{0} \in H^{s} \times H^{s-2}$, we have the following
   decomposition (local in $t$) 
   $$u(x,t) = \sum_{n} A_{n},$$ where the $A_{n}$ are non-linear functions of
   the initial data. 
 \item 
   Choose an appropriate sequence of initial datum $u_{0,n}(x) \in H^{-1/2}
   \times H^{-3/2}$ such that
   we have failure of continuity for some iterate $A_{k}$ for $s < -1/2$.
   This will imply
   failure of the data-to-solution map for $s < -1/2$.
   \end{enumerate}
%		Bibliography		
%
The second approach is motivated by \cite{Himonas:2005kx}. That is,
to prove the failure of continuity of the data-to-solution map, I am attempting
to find a sequence of traveling wave solutions $u_{n}(x,t)$ to the Boussinesq with
corresponding initial data $u_{0,n}$ in a sufficiently rough Sobolev space such 
$u_{0,n} \to u_{0}$ in $H^{s}$ but $v_{n}(x,t) \not \to u(x,t)$ in $H^{s}$. 

Lastly, the third approach involves numerical simulation in Matlab. More
precisely, to get clues as to what which traveling wave solutions might allow me
to prove failure of continuity, I have been numerically solving various ODEs
with adjustable parameters. 
\providecommand{\bysame}{\leavevmode\hbox to3em{\hrulefill}\thinspace}
\providecommand{\MR}{\relax\ifhmode\unskip\space\fi MR }
% \MRhref is called by the amsart/book/proc definition of \MR.
\providecommand{\MRhref}[2]{%
  \href{http://www.ams.org/mathscinet-getitem?mr=#1}{#2}
}
\providecommand{\href}[2]{#2}
\begin{thebibliography}{dLKT07}

\bibitem[BBM72]{Benjamin_1972_Model-equations}
T.~B. Benjamin, J.~L. Bona, and J.~J. Mahony, \emph{Model equations for long
  waves in nonlinear dispersive systems}, Philos. Trans. Roy. Soc. London Ser.
  A \textbf{272} (1972), no.~1220, 47--78.

\bibitem[Ben72]{Benjamin_1972_The-stability-o}
T.~B. Benjamin, \emph{The stability of solitary waves}, Proc. Roy. Soc.
  (London) Ser. A \textbf{328} (1972), 153--183.

\bibitem[BT06]{Bejenaru-Tao-2006-Sharp-well-posedness-and-ill-posedness}
I.~Bejenaru and T.~Tao, \emph{Sharp well-posedness and ill-posedness results
  for a quadratic non-linear schr{\"o}dinger equation}, J. Funct. Anal.
  \textbf{233} (2006), no.~1, 228--259.

\bibitem[BT09]{Bona_2009_Sharp-well-pose}
J.~L. Bona and N.~Tzvetkov, \emph{Sharp well-posedness results for the bbm
  equation}, Discrete Contin. Dyn. Syst. \textbf{23} (2009), no.~4, 1241--1252.

\bibitem[Che11]{Chen:2011fk}
R.~M. Chen, \emph{The h{\"o}lder continuity of the solution map to the
  $b$-family equation in weak topology}, 2011.

\bibitem[CS00]{Constantin_2000_Stability-of-a-}
A.~Constantin and W.~A. Strauss, \emph{Stability of a class of solitary waves
  in compressible elastic rods}, Phys. Lett. A \textbf{270} (2000), no.~3-4,
  140--148.

\bibitem[Dai98]{Dai_1998_Model-equations}
H.-H. Dai, \emph{Model equations for nonlinear dispersive waves in a
  compressible mooney-rivlin rod}, Acta Mech. \textbf{127} (1998), no.~1-4,
  193--207.

\bibitem[DDH00]{Dai_2000_Head-on-collisi}
H.-H. Dai, S.~Dai, and Y.~Huo, \emph{Head-on collision between two solitary
  waves in a compressible mooney-rivlin elastic rod}, Wave Motion \textbf{32}
  (2000), no.~2, 93--111.

\bibitem[DH00]{Dai_2000_Solitary-shock-}
H.-H. Dai and Y.~Huo, \emph{Solitary shock waves and other travelling waves in
  a general compressible hyperelastic rod}, R. Soc. Lond. Proc. Ser. A Math.
  Phys. Eng. Sci. \textbf{456} (2000), no.~1994, 331--363.

\bibitem[dLKT07]{Lellis_2007_Low-regularity-}
C.~de~Lellis, T.~Kappeler, and P.~Topalov, \emph{Low-regularity solutions of
  the periodic camassa-holm equation}, Comm. Partial Differential Equations
  \textbf{32} (2007), no.~1-3, 87--126.

\bibitem[HK09]{Himonas:2009fk}
A.~Himonas and C.~E. Kenig, \emph{Non-uniform dependence on initial data for
  the ch equation on the line}, Differential Integral Equations \textbf{22}
  (2009), no.~3-4, 201--224.

\bibitem[HKM09]{Himonas_2009_Non-uniform-dep-per}
A.~Himonas, C.~E. Kenig, and G.~Misiolek, \emph{Non-uniform dependence for the
  periodic ch equation.}, To appear in Communications in Partial Differential
  Equations (2009).

\bibitem[HM05]{Himonas:2005kx}
A.~Himonas and G.~Misiolek, \emph{High-frequency smooth solutions and
  well-posedness of the {C}amassa-{H}olm equation}, Int. Math. Res. Not.
  (2005), no.~51, 3135--3151. \MR{2187502 (2006i:35315)}

\bibitem[Len06]{Lenells_2006_Traveling-waves}
J.~Lenells, \emph{Traveling waves in compressible elastic rods}, Discrete
  Contin. Dyn. Syst. Ser. B \textbf{6} (2006), no.~1, 151--167 (electronic).

\bibitem[Ols06]{Olson_2006_Non-uniform-dep}
E.~Olson, \emph{Non-uniform dependence on initial data for a family of
  non-linear evolution equations}, Differential Integral Equations \textbf{19}
  (2006), no.~10, 1081--1102.

\end{thebibliography}
%\bibliography{/Users/davidkarapetyan/math/bib-files/references}
%\bibliographystyle{amsalpha-custom}
\end{document}

