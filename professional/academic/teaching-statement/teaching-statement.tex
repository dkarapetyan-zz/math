\documentclass[12pt,oneside]{amsart}
\pagestyle{plain}
\synctex=1
\usepackage[margin=2.5cm]{geometry}  
\usepackage{hyperref}
\setcounter{secnumdepth}{1} %number only sections, not subsections
\renewcommand{\baselinestretch}{1.1} %increases vert spacing of text

\begin{document}
\title{Teaching Statement}
\author{{\it David Karapetyan}\\
\/ University of Notre Dame}
\date{}
\maketitle
\thispagestyle{empty} %don't want general page numbering
\pagestyle{empty} %don't want page number for book cover
\subsection{Teaching Philosophy} The keys to being a great teacher are
flexibility and passion. Students may not know as much about mathematics as
their teachers, but their instincts are just as good as their teachers'
when it comes to sizing up people. If you don't have passion for what
you're teaching, nor the desire to accommodate your students' needs, why should
they have passion for what you're teaching?  Why should they be willing to
accommodate your goals for the
course, to engage you, to dance with you? Students are looking to learn. If they
weren't willing, they wouldn't come to class. It's the teacher's job to
deliver.

Preparation is crucial, but not enough. There are many teachers who go into
their lectures extremely prepared, yet lose their classes within the first week.
A lecture is like a musical performance. You have to prepare to have any chance
of excelling, but once you're in front of an audience, an excellent
performance depends largely on feel. Suppose that during the course of a
lecture, a student is confused about material you thought would be easy to
grasp. If you have anticipated this, and rattle off a
sterile, canned response, you may eliminate the confusion, but you will create a
disconnect between yourself and the student. The other students will pick up on
this, and immediately lose faith in you and the course. There is no empathy in
the canned response, and no fire. The students will wonder: how can mathematics
be worth learning if our teacher cares so little if someone is confused about a
mathematical topic?   

\subsection{Teaching Experience} I have taught freshman calculus twice at Notre
Dame, and showed tremendous improvement the second time around, scoring in the
top 10\% in all meaningful categories, in particular teaching clarity,
preparation, and ability to promote mastery of the subject matter. I attribute
this to the fact that I was experimenting with various
teaching styles my first time around, and that the experience helped me settle
on an approach that was right for me. Both classes were part of large,
multi-section courses, with universal exams and grading policies. I prepared my
own lectures following the rubric set out by the course chair, collaborated in
creating exams, helped keep the course website up to date, and maintained the
online WebCT framework used to administer quizzes and keep track of quiz scores.

\subsection{Teaching Tools and Techniques} My teaching method is largely
organic, relying heavily upon the fruits of preparation. At the beginning of
each lecture, I pass out a typed activity sheet provided by the chair to all
instructors. The activity sheet gives an outline of the day's lecture, with
plenty of strategically positioned blank space that the students can use during
the lecture to jot down notes. The activities also provide the students the
opportunity to collaborate with each other during class, and to collaborate with
students from other sections. I write out my lecture on one of these activity
sheets, then commit it to memory. If I need to, I consult my activity sheet
during the lecture. I do so sparingly, since I believe that rigid adherence to a
set of notes results in a stale classroom atmosphere.  If a student is confused
about a particular problem I am working out on the board, I tell them that we're
going to shake and bake, that the pain they feel from not understanding is my
pain, that mathematics is the queen of all sciences, the path to great power,
and that they must understand, that they have the ability to understand, that I
will do everything possible to help them understand. The great teacher is a
leader, first and foremost, inspiring confidence in his students via preparation
and passion. Everything else is just mere lecturing.

Computer-automated management of multi-section courses is the way of the future.
Course websites, if coded correctly, require minimal effort to maintain and are
the ideal means of distributing course materials to students. My definition of
``course materials'' includes homework assignments, lecture
handouts, and interactive media, such as quizzes and exams which are
automatically graded by the course server once they are submitted by the
student. By eliminating the time lost by tallying scores by hand or organizing
paper files, computers afford a professor more time to interact with students.
While teaching calculus these past semesters, I was heavily involved
with maintaining the course website, and was the sole professor in charge of
WebCT, a web framework used by all course lecturers to both administer quizzes
to students and record their quiz grades.

During the semester, I use whatever means necessary, technological
or otherwise, to keep track of student progress. In particular, I submit a
course evaluation to students after the first midterm. Their suggestions and
commentary allow me to improve as a teacher by
adapting my instruction to their needs. Each class has a unique personality,
unlike that of any other. Hence, a good teacher must be willing to grow over the
course of a semester. His approach must be malleable, shaped by the needs of his
current students. If the students sense that you are such a teacher from the first
day of class forward, they are yours. By reciprocity, from that moment on their
passion for mathematics and knowledge of the subject matter is shaped by your own. 
\end{document}
