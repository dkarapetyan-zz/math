 \documentstyle[12pt,nd]{letter}
  \textheight 9.0in
%\textwidth 6.3in
%\oddsidemargin .5in
%\evensidemargin .5in 
\pagestyle{empty}

\begin{document}
%\calligraphy
%\noheadings

%%%%%%%%%%%%%%%%%%%%%%%%%%%%%%%%%%%%%%%%%%%%%%%%%%%%%%%%%%%%%%%%%%%%%%%%%%%%%%%%%
%
%  REMARK: the following five lines should be deleted if you want the letterhead
%          to also appear on the subsequent pages rat his  than just the first.
%
%%%%%%%%%%%%%%%%%%%%%%%%%%%%%%%%%%%%%%%%%%%%%%%%%%%%%%%%%%%%%%%%%%%%%%%%%%%%%%%%%

\headheight 0mm
\noheadings
\vspace*{-.75in}
\notredame
\vspace*{.75in}
\vskip.25in  

\signature{Alex Himonas, Professor\\
E-mail:himonas.1@nd.edu
\\Phone: (574) 631-7583}

\vskip-0.4in
\mailto{{\bf \Large  Recommendation Letter for David Karapetyan}}


This letter  is in support of the application of my Ph.D. student
{\bf  David Karapetyan}  for a  faculty position in your department. 
David   has made a great progress during his  stay here at Notre Dame in terms
of both doing and teaching mathematics. His  research is excellent and his
teaching   reached a very high level of success 
(spring 2011 composite  score   4.6 out of 5.0  in Calculus for Business,
one of the most challenging first year courses to teach at Notre Dame).


David's  research is focused on the study of well-posedness for nonlinear 
evolution equations. His  first result, which has been  published in 
the Journal of Differential Equations, 
 is about the initial value problem for 
the hyperelastic rod equation 
%
\begin{equation}
\label{eq:HRE}
\partial_t u-\partial_t\partial_x^2u+3u\partial_x u=\gamma\left(2\partial_x u\partial_x^2 u+u\partial_x^3 u\right),
\qquad  \gamma\neq 0.
\end{equation}
%
In both the periodic and non-periodic case,
David   showed that the solutions to this equation do {\bf not} depend uniformly continuously on initial data in the  Sobolev space $H^s$ when $s>3/2$.
This extends a result  of  Himonas and Kenig (non-periodic),
and Himonas, Kenig and  Misio\l ek (periodic)  for   the Camassa-Holm 
equation (that is,   when  $\gamma = 1$ in which case  equation (\ref{eq:HRE}) is integrable).
Considering  that  the  hyperelastic rod equation  is well posed 
in $H^s$ for $s>3/2$   with continuous dependence on initial data,
David's result demonstrates that continuous dependence
is optimal  for this equation, unlike for the case of other 
evolution equations (say KdV).
The proof  is based on the idea of approximate solutions 
and  uses  delicate commutator and multiplier estimates
for estimating the error between actual and approximate
solutions. Furthermore, it required  life-span and solution size 
estimates in terms of of the $H^s$-size of the initial data,
which  were not available in the existing proofs 
of well-posedness available in the literature.
David, derived these  estimates by reproving well-posedness 
following a Galerkin type approximation scheme.
Furthermore, equipped with a  significant amount of analytic tools 
and  using the insight he gained thus far about the hyperelastic rod equation 
he investigated H\"older continuity of its $H^s$ solutions
in  weaker  $H^r$-norms, $r<s$. 
In a recent preprint, motivated by the work of  Chen,  Liu and Zhang
for the b-family, he proved that the data-to-solution map for  the hyperelastic rod equation 
is  H\"older continuous in $H^r$ topology.
Now, in this research  direction,  David is interested in finding out
what happens  when the initial data are in Sobolev spaces 
with exponent less than 3/2.  For the Camassa-Holm equation,
 continuity of the solution map fails and this is proved by 
 using peakon-antipeakon interactions.
 One may expect  failure of continuity for 
 the hyperelastic rod equation too.
  However, one must find a new way for proving this since
this equation has no peakon solutions.
This is a challenging problem which demonstrates David's
motivation as  young researcher.

%

Also, in collaboration with Dan Geba and myself,
he is studying the Cauchy  problem for 
 the Boussinesq equation
\begin{equation}
\label{buss}
   u_{tt} - u_{xx} + u_{xxxx} + (u^{2})_{xx} = 0,
   \end{equation}
   %
  for initial data $u(x,0) \in H^{s}$ and  $u_{t}(x,0)\in H^{s-2}$.
  It is known that this equation is well-posed on the circle
when $s>-1/4$ (Farah and Scialom) and on the line 
when  $s>-1/2$  (Kishimoto and Tsugawa).
Investigating  well-posedness properties for
the Boussinesq equation below the  
the Sobolev exponents 
$s=-1/4$ (periodic) and  $s=-1/2$ (non-periodic) 
is a very interesting problem
that expands David's experience 
to other important partial differential equations. 
Finally,  David is part of a project on the  analytic theory
of dispersive equations with  Gerson Petronilho and myself.
Currently, he is trying to construct non-analytic in time solutions of the 
NLS equation when the initial data are analytic. 


David is a very active member of the department of mathematics
 and the  group of partial differential equations.
 He attends and gives talks in our partial differential equations, 
complex analysis and differential  geometry seminars as well as 
national and international conferences.
He has been the most involved and productive students 
in the topics classes I have offered the last four years
on nonlinear evolution equations.
He has gone far and beyond of what was discussed 
in class and produced extensive private notes 
on the techniques  developped over the last 20 years by Bourgain, 
 Kenig, Ponce, Vega Colliander, Keel, Staffiliani, Takaoka, Tao and many other mathematicians for the study of KdV the NLS and other related equations.



 

Concerning teaching,  David  is doing beautifully.
It is fun to watch him  leading students to learning.
He knows how to communicates mathematics in an interesting
 way and how to involve students in a discussion. 
He is a gifted teacher who plans his classes carefully,
knows all his students very well,  and patiently helps
each one of them reach her/his full learning potential.
In addition,  during  the last two years David served
as the  computer technology expert supporting
the teaching of business calculus.
He assumed full  responsibility of the on line quiz 
delivered via concourse and maintained 
an elaborate webpage for this course.
He was able to handle technology  issues for 
the  500 students in this multi-section course 
with an amazing effectiveness. 




In summary, David   is a promising young scholar.
He is highly motivated, personable, likeable, and enthusiastic 
about  research and  teaching.  
He is ready to take full advantage of the opportunities 
offered and have a great influence in the life of a mathematics department. 
 Therefore, I recommend him in the {\bf strongest possible terms} in your department.
Please let me know if you need any additional information.






\closing{Sincerely yours,}
\end{document}
