%
\documentclass[12pt,reqno]{article}
\usepackage{enumerate} %to change enumerate symbols
\usepackage[margin=1.49cm]{geometry}  %page layout
\setcounter{secnumdepth}{1} %number only sections, not subsections
\renewcommand{\baselinestretch}{1.1} %increases vert spacing of text
%
\pagestyle{empty}
\usepackage{hyperref}
\hypersetup{colorlinks=true,
linkcolor=blue,
citecolor=blue,
urlcolor=blue,
}
\synctex=1
%
%
%
\begin{document}
\title{Shaheen Award Personal Statement} 
\author{{David Karapetyan}\\
    \/ University of Notre Dame}
    \date{}
\maketitle
%
%
\thispagestyle{empty}
\subsection{Research} 
\label{ssec:cur-res}
The main focus of my research so far has been in the field of mathematical analysis and partial differential equations, specifically \textbf{non-linear evolution equations}. These are used to model a wide variety of physical phenomena, such as the evolution of water waves in a channel (with applications to how waves evolve at the beach, or predicting future weather patterns), the quantum state of a physical system (for example, the probability density distribution of electrons in a field), and the stochastic movement of stock option prices. Recently, following work of my advisor Alex A. Himonas and his collaborators Carlos E. Kenig and Gerard Misio{\l}ek, I have proved results on sharp well-posedness and ill-posedness
for both smooth and rough initial data for the \textbf{hyperelastic rod equation (HR)}, a very important water wave equation which was one of the first models for breaking water waves. We say that an equation is \textit{well-posed} if the following hold:
\begin{enumerate}[I.]
  \item{}For given initial data $u_{0}$, a corresponding solution $u$ exists
    for some period of time.
  \item{} The solution is unique.
  \item{} The solution depends continuously on the initial data. That is, suppose we are given initial data $u_{0}$ with associated solution $u$. If one changes the initial data slightly (say, to $\tilde{u_{0}}$), then the associated solution $\tilde{u}$ should differ only a little from $u$. 
\end{enumerate}
If all three criteria are satisfied and the solution exists for all time, we say that our equation is \textit{globally well-posed}. If an equation is well-posed, but the dependence of solutions on initial data is no better than continuous, then we say that the well-posedness is \textit{sharp}. If an equation violates one of the three criteria above, we say that it is \textit{ill-posed}. The question of well-posedness is an extremely important one in the study of partial differential equations. For example, if an equation does not admit a solution (or admits a solution that is not unique) then it is of little use in modeling physical phenomena. If continuous dependence is violated, either the equation is a poor model, or the physical phenomenon that is being modeled is highly unstable with respect to initial data.

We remark that the HR equation is a toy model for the \textbf{Euler equations},
the fundamental model for the motion of inviscid fluids.
For  viscous  fluids the corresponding model is  the  3-D {\bf Navier-Stokes equations}.  
One of the most famous open problem in mathematics is the question of whether or not the 3-D Navier-Stokes equations are globally well-posed for smooth initial data. It is a \textbf{Millennium Prize Problem}, carrying with it a cash prize of \$$1,000,000$ for the correct answer. I hope that a better understanding of well-posedness and ill-posedness questions for the HR equation will help illuminate the way to tackling similar questions for the Euler and the Navier-Stokes equations. 


In collaboration with Dan Geba and Alex Himonas, I have also recently proved
the failure of continuity (for sufficiently rough initial data) for the
data-to-solution map of the ``good" Boussinesq equation, a water wave equation
with behavior quite different from that of HR\@. The Boussinesq equation is a
\textit{dispersive} equation. That is, solutions to the Boussinesq smooth
out, or disperse, over time. This is not the case with HR\@. Furthermore, unlike
HR, the Boussinesq does not admit traveling wave solutions which break. Our
results improved substantially upon those in the literature, and we hope to
extend them further in the near future. 

For a list of my publications and more information, please visit \href{http://davidkarapetyan.com}{http://davidkarapetyan.com}.
\subsection{Teaching} 
\label{ssec:teaching}
I have taught freshman calculus twice at Notre
Dame, and showed tremendous improvement the second time around, scoring in the
top 10\% in all meaningful categories, in particular teaching clarity,
preparation, and ability to promote mastery of the subject matter. I attribute
this to the fact that I was experimenting with various
teaching styles my first time around, and that the experience helped me settle
on an approach that was right for me. Both classes were part of large,
multi-section courses, with universal exams and grading policies. I prepared my
own lectures following the rubric set out by the course chair, collaborated in
creating exams, helped keep the course website up to date, and maintained the
online WebCT framework used to administer quizzes and keep track of quiz scores.

My teaching method is largely
organic, relying heavily upon the fruits of preparation. At the beginning of
each lecture, I pass out a typed activity sheet provided by the chair to all
instructors. The activity sheet gives an outline of the day's lecture, with
plenty of strategically positioned blank space that the students can use during
the lecture to jot down notes. The activities also provide the students the
opportunity to collaborate with each other during class, and to collaborate with
students from other sections. I write out my lecture on one of these activity
sheets, then commit it to memory. If I need to, I consult my activity sheet
during the lecture. I do so sparingly, since I believe that rigid adherence to a
set of notes results in a stale classroom atmosphere. A great teacher has to prepare before class---however, they also have to be able to adapt their teaching approach to their students' needs on the fly. 

I believe that computer-automated management of multi-section courses is the
way of the future.
Course websites, if coded correctly, require minimal effort to maintain and are
the ideal means of distributing course materials to students. My definition of
``course materials'' includes homework assignments, lecture
handouts, and interactive media, such as quizzes and exams which are
automatically graded by the course server once they are submitted by the
student. By eliminating the time lost by tallying scores by hand or organizing
paper files, computers afford a professor more time to interact with students.
While teaching calculus these past semesters, I was heavily involved
with maintaining the course website, and was the sole professor in charge of
WebCT, a web framework used by all course lecturers to both administer quizzes
to students and record their quiz grades.

During the semester, I use whatever means necessary, technological
or otherwise, to keep track of student progress. In particular, I submit a
course evaluation to students after the first midterm. Their suggestions and
commentary allow me to improve as a teacher by
adapting my instruction to their needs. Each class has a unique personality,
unlike that of any other. Hence, a good teacher must be willing to grow over the
course of a semester. His approach must be malleable, shaped by the needs of his
current students. If the students sense that you are such a teacher from the first
day of class forward, they are yours. By reciprocity, from that moment on their
passion for mathematics and knowledge of the subject matter is shaped by your own. 

\end{document}

