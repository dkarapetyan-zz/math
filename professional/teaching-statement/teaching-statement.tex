\documentclass[12pt,oneside]{amsart}
\pagestyle{plain}
\synctex=1
\usepackage[margin=2.5cm]{geometry}  %page laIt
\usepackage{hyperref}
\setcounter{secnumdepth}{1} %number only sections, not subsections
\renewcommand{\baselinestretch}{1.1} %increases vert spacing of text

\begin{document}
\title{Teaching Statement}
\author{{\it David Karapetyan}\\
University of Notre Dame}
\date{}
\maketitle
\thispagestyle{empty} %don't want general page numbering
\pagestyle{empty} %don't want page number for book cover
\subsection{Teaching Philosophy} The keys to being a great teacher are
flexibility and passion. My students may not know as much about mathematics as
I do, but their instincts are as as good as mine when it comes to sizing up
people. If I don't have the passion for what I'm teaching, nor the desire
to accommodate my students' needs, why should they have passion for what I
am teaching? Why should they be willing to accommodate my goals for the
course, to engage me, to dance with me? Students are looking to learn. If they
weren't willing, they wouldn't be sitting in my classroom. It is my job to
deliver.

Preparation is crucial, but not enough. There are many teachers who go into
their lectures extremely prepared, yet lose their classes within the first week.
A lecture is like a musical performance. I have to prepare to have any chance
of excelling, but once I am in a room in front of an audience, a excellent
performance depends largely on feel. Suppose that during the course of a
lecture, a student is confused about material I thought would be easy to
grasp. If I have anticipated even this remote contingency, and rattle off a
sterile, canned response, I may eliminate the confusion, but I will create a
disconnect between myself and the student. The other students will pick up on
this, and immediately lose faith in  and the course. There is no empathy in
the canned response, and no fire. How can mathematics be worth learning if I
care so little if someone is confused about a mathematical topic?   

\subsection{Teaching Experience} I have taught freshman calculus twice at Notre
Dame, and showed tremendous improvement the second time around, scoring in the
top 10\% in all meaningful categories, in particular teaching clarity,
preparation, and ability to promote mastery of the subject matter. I attribute
this to the fact that I was experimenting with various
teaching styles my first time around, and that the experience helped me settle
on an approach that was right for me. Both were large, multi-section courses,
with universal exams and grading policies. I prepared my own lectures following
the rubric set out by the course chair, collaborated in the creating of exams,
helped keep the course website up to date, and maintained the online WebCT
framework used to administer quizzes and keep track of quiz scores.

\subsection{Teaching Tools and Techniques} My teaching method is largely
organic, relying heavily upon the fruits of my preparation. At the beginning of
each lecture, I pass out a typed activity sheet provided by the chair to all
instructors. The activity sheet gives an outline of the day's lecture, with
plenty of strategically positioned blank space that the students can use during
the lecture to jot down notes. The activities also provide the students the
opportunity to collaborate with each other during class, and to collaborate with
students from other sections.  I write out my lecture on one of these activity
sheets, then commit it to memory. If I need to, I consult my activity sheet
during the lecture. I do so sparingly, since I believe that rigid adherence to a
set of notes results in a stale classroom atmosphere.  If a student is confused
about a particular problem I am working out on the board, I tell them that we're
going to shake and bake, that the pain they feel from not understanding is my
pain, that mathematics is the queen of all sciences, the path to understanding
and power, and that they must understand, that they have the ability to
understand, that I will do everything possible to hep them understand. These
kids have been told since elementary school that mathematics is difficult, that
they don't have what it takes to grasp difficult material. Sadly, the U.S.
secondary school education system is largely broken.  Hence, when a student is
confused, and is looking to me for help, I don't rattle off the prepared
response. I have a responsibility to give them the education that they were
cheated out of in the past. I have the theory, the backbone of the lecture
organized in my head. Now, it is time to riff, to improvise, as any good jazz
musician would. I go to the chalkboard, and from the way I write the student
knows that I am treating the moment as a small revolution against years of poor
instruction and neglect. They are then willing to engage and ask questions as I
write on the blackboard. The great teacher is a leader, first and foremost,
inspiring confidence in his students via preparation and passion.  Everything
else is just mere lecturing.

I am a huge technophile, and believe that computer-automated management of
multi-section courses is the way of the future. Course websites, if coded
correctly, require minimal effort to maintain and are the ideal means of
distributing course materials to students. My definition of ``course materials''
is broad, and includes homework assignments, lecture handouts, and even
interactive media, such as quizzes and exams which are automatically graded by
the course server once they are submitted by the student. By eliminating the
time lost by tallying scores by hand or organizing paper files,
computers afford a professor more time to interact with students. While teaching
Calculus these past couple of semesters, I was heavily involved with maintaining the
course website, and was the sole professor in charge of WebCT, a web framework
used by all course lecturers to both administer quizzes to students and 
record their quiz grades. I believe homework assignments should also be
administered and graded via a similar computerized system, as this would be very
efficient. For example, using a
script or a built in software option, one can immediately find the lowest
homework scores on a particular assignment. This would take much longer if done
by hand. 

During the course of the semester, I use whatever means necessary, technological
or otherwise, to keep track of the students' progress. In particular, I submit a
course evaluation to them after the first midterm. Their suggestions and
commentary allow me to improve as a teacher over the course of the semester by
adapting my instruction to their needs. Each class has a unique personality,
unlike that of any other. Hence, a good teacher must be willing to grow over the
course of a semester. His approach must be malleable, shaped by the needs of his
current students.  If the students sense that I am such a teacher from the first
day of class forward, they are mine. By reciprocity, from that moment on their
passion for mathematics and knowledge of the subject matter is shaped by my own. 
\end{document}
