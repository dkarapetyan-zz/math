\documentclass[12pt]{article}

\textheight= 9.1in
\textwidth= 6.5in

%\hoffset = -1.1in
%\voffset= -0.75in
%\voffset= -1.25in
%\voffset= -.75in
%\voffset=-1.40in
\voffset=-.8in
\hoffset = -.55in

\parindent= 0.0in
\pagestyle{empty}

\usepackage{amssymb}
\usepackage{amsmath}
\usepackage{amsfonts}






\newcommand\kreis[1]{\ensuremath{\mathbin{\settowidth{\dimen6}{\mbox{$\bigcirc$}}%
              \makebox[0pt][l]{$\bigcirc$}\makebox[\dimen6]{#1}}}}



\begin{document}

% Times fonts /usr/local/ndvips/PStfms/
%\font\reg = ptmr at 11pt
%\font\bld = ptmb at 11pt
%\font\emph = ptmri at 11pt
%\font\bigbld = ptmb at 13pt

% For Testing, use regular fonts to view with xdvi
\newcommand{\bld}{\bf}
\newcommand{\reg}{\normalsize}
%% my fix 8/13/97
%\newcommand{\emph}{\em}
%%
\newcommand{\bigbld}{\LARGE\bf}



\centerline{\bf \large STATEMENT OF TEACHING}
\vspace{0.2in}
\centerline{ JENNIFER M.\ GORSKY}
\vspace{0.3in}
\reg






\noindent
{\bf Teaching Philosophy.}  
``Oh!  I'm terrible at math!"  Most often this is the response I receive
upon saying I am a mathematician.  Are that many people
really terrible at mathematics, or have they simply had terrible luck
with mathematics instructors?  If the latter, then what constitutes a
successful mathematics instructor?  From my experience as a student and
educator, I believe it is the product of two fundamental skills: 
engagement and communication.

\quad The ability to engage and inspire one's
audience is essential.  Mathematics is an active sport, and is rarely
learned by passively watching someone else.  As with any sport, practice
and active participation yield better performance.  Just as coaches can
draw their players into practice through exercises, drills and pep talks, a
teacher can draw students into a lecture through probing questions,
specialized handouts, and, most importantly, exuding excitement about the
subject.   Additionally, an effective mathematics instructor must have
the ability to present  material in an understandable, unintimidating
manner.  Communicating complicated concepts in a clear way is crucial to a
student's comprehension, and hence success.  Of course, this clarity is a
function of having well-organized lectures, but equally important is having
an intuition into which areas will be problematic for students and
addressing those areas with care. 
 
\quad Possessing these two abilities has not only made teaching mathematics
a pleasure for me, but it has made learning mathematics fun and exciting
for my students.  



\vspace{0.1in}


{\bf Classroom Experience.}  I have instructed freshman-level calculus four
times at Notre Dame with great success.  This is clear from my teaching evaluations, where in the
Spring of 2003 I received an overall score of 
3.95 out of
4.00 on student evaluations, exceeding 96\% of all other instructors'
teaching scores for 100 level (freshman/sophomore) mathematics courses at
the University of Notre Dame. Three of these four classes were large,
multi-section courses with universal exams, and grading policies.  
I prepared my own lectures, collaborated in the creation of exams and
administered quizzes through the web-based software WebCT.  The
students consisted of business and social science majors, and the courses
emphasized business applications. I also taught a small, individually
organized class for pre-professional and science students.  Here I was
solely responsible for lectures, exams and grading.  In addition to
instructing courses, I have led several tutorial sections at Notre
Dame, where I explained concepts, answered homework questions, and
developed and graded quizzes.   



\vspace{0.1in}


{\bf Classroom Methods and Techniques.}  My teaching methods center
around providing an enjoyable and stimulating place for my students to
experience the beauty of mathematics.
For example, at the
beginning of each lecture, I pass out a typed activity sheet on colored
paper that outlines the day's lecture, leaving spaces for the students to
fill in answers to questions and solutions to problems.  These activities
allow students to spend less time taking notes and more time listening to
and participating in the lecture.  They are better able to digest the
material being presented, and therefore learn more during class.   When
appropriate, I also bring in physical aids (i.e., ``toys") to make the
material more concrete and more memorable for the students.  For instance,
in order to illustrate the notion of a saddle point, I pass out  Pringles;
to teach surfaces in $\mathbb{R}^3$, I use a Tinkertoy
construction
set.



 \newpage

{\bf Jennifer M.\ Gorsky, page 2}



\vskip 0.3in

\quad Activities and toys aside, I bring into the classroom both an
enthusiasm for teaching and an excitement about the subject of
mathematics that students find particularly motivating.  My
students' education and mathematical development are my top priority.  The
above-mentioned methods are some of the ways in which I try to create an
atmosphere that cultivates learning and enjoyment.  I have adopted most of
these methods by observing what inspires and reaches students.  

\quad To
supplement my own observations, I distribute an evaluation during the third
week of classes.  This gives students the opportunity to express which
methods they would like to see used more or less frequently, and what they
feel are the strengths and weaknesses of the instructor.  Passing out
evaluations early in the semester, as opposed to at
the end, gives me the chance to meet the needs of my current students.  

\quad Teaching is not simply a job for me.  It is an opportunity to inspire,
to challenge and to communicate the intricacies of this amazing and complex
subject, mathematics.



\end{document}


