%%%%%%%%%%%%%%%%%%%%%%%%%%%%%%%%%%%%%%%%%%%%%%%%%%%%%%%%%%%%%%%%%%%%%%%%
%%%%%%%%%%%%%%%%%%%%%% Simple LaTeX CV Template %%%%%%%%%%%%%%%%%%%%%%%%
%%%%%%%%%%%%%%%%%%%%%%%%%%%%%%%%%%%%%%%%%%%%%%%%%%%%%%%%%%%%%%%%%%%%%%%%

%%%%%%%%%%%%%%%%%%%%%%%%%%%%%%%%%%%%%%%%%%%%%%%%%%%%%%%%%%%%%%%%%%%%%%%%
%% NOTE: If you find that it says                                     %%
%%                                                                    %%
%%                           1 of ??                                  %%
%%                                                                    %%
%% at the bottom of your first page, this means that the AUX file     %%
%% was not available when you ran LaTeX on this source. Simply RERUN  %% 
%% LaTeX to get the ``??'' replaced with the number of the last page  %% 
%% of the document. The AUX file will be generated on the first run   %%
%% of LaTeX and used on the second run to fill in all of the          %%
%% references.                                                        %%
%%%%%%%%%%%%%%%%%%%%%%%%%%%%%%%%%%%%%%%%%%%%%%%%%%%%%%%%%%%%%%%%%%%%%%%%

%%%%%%%%%%%%%%%%%%%%%%%%%%%% Document Setup %%%%%%%%%%%%%%%%%%%%%%%%%%%%

% Don't like 10pt? Try 11pt or 12pt
\documentclass[10pt]{article}

% This is a helpful package that puts math inside length specifications
\usepackage{calc}
\usepackage{multicol}
% Layout: Puts the section titles on left side of page
\reversemarginpar
\synctex=1
%
%         PAPER SIZE, PAGE NUMBER, AND DOCUMENT LAYOUT NOTES:
%
% The next \usepackage line changes the layout for CV style section
% headings as marginal notes. It also sets up the paper size as either
% letter or A4. By default, letter was used. If A4 paper is desired,
% comment out the letterpaper lines and uncomment the a4paper lines.
%
% As you can see, the margin widths and section title widths can be
% easily adjusted.
%
% ALSO: Notice that the includefoot option can be commented OUT in order
% to put the PAGE NUMBER *IN* the bottom margin. This will make the
% effective text area larger.
%
% IF YOU WISH TO REMOVE THE ``of LASTPAGE'' next to each page number,
% see the note about the +LP and -LP lines below. Comment out the +LP
% and uncomment the -LP.
%
% IF YOU WISH TO REMOVE PAGE NUMBERS, be sure that the includefoot line
% is uncommented and ALSO uncomment the \pagestyle{empty} a few lines
% below.
%

%% Use these lines for letter-sized paper
\usepackage[paper=letterpaper,
            %includefoot, % Uncomment to put page number above margin
            marginparwidth=1.2in,     % Length of section titles
            marginparsep=.05in,       % Space between titles and text
            margin=1in,               % 1 inch margins
            includemp]{geometry}

%% Use these lines for A4-sized paper
%\usepackage[paper=a4paper,
%            %includefoot, % Uncomment to put page number above margin
%            marginparwidth=30.5mm,    % Length of section titles
%            marginparsep=1.5mm,       % Space between titles and text
%            margin=25mm,              % 25mm margins
%            includemp]{geometry}

%% More layout: Get rid of indenting throughout entire document
\setlength{\parindent}{0in}

%% This gives us fun enumeration environments. compactitem will be nice.
\usepackage{paralist}

%% Reference the last page in the page number
%
% NOTE: comment the +LP line and uncomment the -LP line to have page
%       numbers without the ``of ##'' last page reference)
%
% NOTE: uncomment the \pagestyle{empty} line to get rid of all page
%       numbers (make sure includefoot is commented out above)
%
\usepackage{fancyhdr,lastpage}
\pagestyle{fancy}
\pagestyle{empty}      % Uncomment this to get rid of page numbers
\fancyhf{}\renewcommand{\headrulewidth}{0pt}
\fancyfootoffset{\marginparsep+\marginparwidth}
\newlength{\footpageshift}
\setlength{\footpageshift}
          {0.5\textwidth+0.5\marginparsep+0.5\marginparwidth-2in}
\lfoot{\hspace{\footpageshift}%
       \parbox{4in}{\, \hfill %
                    \arabic{page} of \protect\pageref*{LastPage} % +LP
%                    \arabic{page}                               % -LP
                    \hfill \,}}

% Finally, give us PDF bookmarks
\usepackage{color,hyperref}
\definecolor{darkblue}{rgb}{0.0,0.0,0.3}
\hypersetup{colorlinks,breaklinks,
            linkcolor=darkblue,urlcolor=darkblue,
            anchorcolor=darkblue,citecolor=darkblue}

%%%%%%%%%%%%%%%%%%%%%%%% End Document Setup %%%%%%%%%%%%%%%%%%%%%%%%%%%%


%%%%%%%%%%%%%%%%%%%%%%%%%%% Helper Commands %%%%%%%%%%%%%%%%%%%%%%%%%%%%

% The title (name) with a horizontal rule under it
%
% Usage: \makeheading{name}
%
% Place at top of document. It should be the first thing.
\newcommand{\makeheading}[1]%
        {\hspace*{-\marginparsep minus \marginparwidth}%
         \begin{minipage}[t]{\textwidth+\marginparwidth+\marginparsep}%
                {\large \bfseries #1}\\[-0.15\baselineskip]%
                 \rule{\columnwidth}{1pt}%
         \end{minipage}}

% The section headings
%
% Usage: \section{section name}
%
% Follow this section IMMEDIATELY with the first line of the section
% text. Do not put whitespace in between. That is, do this:
%
%       \section{My Information}
%       Here is my information.
%
% and NOT this:
%
%       \section{My Information}
%
%       Here is my information.
%
% Otherwise the top of the section header will not line up with the top
% of the section. Of course, using a single comment character (%) on
% empty lines allows for the function of the first example with the
% readability of the second example.
\renewcommand{\section}[2]%
        {\pagebreak[2]\vspace{1.3\baselineskip}%
         \phantomsection\addcontentsline{toc}{section}{#1}%
         \hspace{0in}%
         \marginpar{
         \raggedright \scshape #1}#2}

% An itemize-style list with lots of space between items
\newenvironment{outerlist}[1][\enskip\textbullet]%
        {\begin{itemize}[#1]}{\end{itemize}%
         \vspace{-.6\baselineskip}}

% An environment IDENTICAL to outerlist that has better pre-list spacing
% when used as the first thing in a \section 
\newenvironment{lonelist}[1][\enskip\textbullet]%
        {\vspace{-\baselineskip}\begin{list}{#1}{%
        \setlength{\partopsep}{0pt}%
        \setlength{\topsep}{0pt}}}
        {\end{list}\vspace{-.6\baselineskip}}

% An itemize-style list with little space between items
\newenvironment{innerlist}[1][\enskip\textbullet]%
        {\begin{compactitem}[#1]}{\end{compactitem}}

% To add some paragraph space between lines.
% This also tells LaTeX to preferably break a page on one of these gaps
% if there is a needed pagebreak nearby.
\newcommand{\blankline}{\quad\pagebreak[2]}

%%%%%%%%%%%%%%%%%%%%%%%% End Helper Commands %%%%%%%%%%%%%%%%%%%%%%%%%%%

%%%%%%%%%%%%%%%%%%%%%%%%% Begin CV Document %%%%%%%%%%%%%%%%%%%%%%%%%%%%

\begin{document}
\makeheading{David Karapetyan}

\section{Contact Information}
%
% NOTE: Mind where the & separators and \\ breaks are in the following
%       table.
%
% ALSO: \rcollength is the width of the right column of the table 
%       (adjust it to your liking; default is 1.85in).
%
\newlength{\rcollength}\setlength{\rcollength}{2.50in}%
%
\begin{tabular}[t]{@{}p{\textwidth-\rcollength}p{\rcollength}}
\href{http://math.nd.edu/}{University of Notre Dame} & \textit{Voice:} 574-631-5706 \\
\href{http://math.nd.edu/}{Department of Mathematics}	& \textit{Fax:} 574-631-6579 \\
255 Hurley Hall    & \textit{E-mail:} 
\href{mailto:dkarapet@nd.edu}{dkarapet@nd.edu}\\
Notre Dame, IN 46556   & \textit{Website:}
\href{http://davidkarapetyan.com}{http://davidkarapetyan.com}     \\
\end{tabular}

\section{Citizenship}
%
USA

\section{Research Interests}
%
Partial differential equations. Applied mathematics. Numerical analysis.
%

\section{Education}
%
\href{http://nd.edu/}{\textbf{University of Notre Dame}}, 
Notre Dame, IN 
\begin{innerlist}
%
\item Ph.D. (candidate),
	\href{http://math.berkeley.edu/}{Mathematics}, May 2012  
\end{innerlist} 
%
\blankline

\href{http://berkeley.edu/}{\textbf{University of California, Berkeley}}, 
Berkeley, CA 
	\begin{innerlist}	
\item B.S., 
	\href{http://math.berkeley.edu/}
	{Mathematics}, May 2004
\item        \href{http://english.berkeley.edu/}
	{ B.S., English}, May 2004
\end{innerlist}
%
%
\section{Awards} 
%
\href{http://berkeley.edu}
{\textbf{University of California, Berkeley }}
\begin{innerlist}
\item \href{http://students.berkeley.edu/finaid/undergraduates/
	types_regents.htm}{Regents Scholarship}, 2000-2004
\end{innerlist}

\blankline

\href{http://nd.edu}{\textbf{University of Notre Dame}}
\begin{innerlist}
\item \href{http://graduateschool.nd.edu/admissions/financial-support/prestigious-fellowships/presidential-fellowships-arthur-j-schmitt-fellowships}
           {Arthur J. Schmitt Presidential Fellowship}, 2007-2011
\end{innerlist}


\section{Publications}
%
\textit{Non-Uniform Dependence and well-posedness
for the
Hyperelastic Rod Equation.},\\ J. Differential Equations \textbf{249} (2010),
796-826.

%
\section{Teaching Experience}
\href{http://nd.edu/}{\textbf{University of Notre Dame}}, 
Notre Dame, Indiana
\begin{outerlist}

\item[] Instructor \hfill \textbf{January 2010-May 2010}
	\\ \textit{Elements of Calculus II for Business}
  \hfill \textbf{January 2011-May 2011}
 
%
\begin{innerlist}
\item Lectured three times a week
\item Wrote and graded exams 
\item Held office hours 
\end{innerlist}
%
\item[] Teaching and Technology Assistant
	\hfill \textbf{August 2008-December 2009}\\
 \textit{Elements of Calculus I 
	for Business} \hfill \textbf{January 2011-May 2011}
	\\
	\textit{Elements of Calculus II 
	for Business} 
	\begin{innerlist}
	\item Maintained online framework for quizzes, homework assignments, 
		and course materials
	\item Improved course webpage layout and online quiz infrastructure
	\item Substituted for course lectures when needed
	\item Assisted with lecturer office hours when needed
	\end{innerlist}
\end{outerlist}

\blankline

\section{Invited Talks}
\textit{PDE, Complex Analysis and Differential Geometry Seminar} \\
``Non-Uniform Dependence and Well-posedness for the Hyperelastic Rod Equation'' \\
February 9, 2010, University of 
Notre Dame\\

\textit{AIMS conference session \\ 
Evolutionary Partial Differential Equations: Theory and Applications} \\
``Sharp Well-posedness for the Hyperelastic Rod equation''\\ 
May 25-28, 2010, Dresden University of Technology, Germany  \\

\textit{AMS Central Sectional Meeting \\ Nonlinear Evolution Equations} \\
``On the Cauchy problem for the Hyperelastic Rod equation'' \\
November 5-7, 2010, University of Notre Dame \\

\textit{Nonlinear Evolution Equations and Wave Phenomena: Computation and Theory} \\
``Well-posedness of Periodic NLS in Analytic Spaces'' \\
April 04-07, 2011, University of Georgia \\

\section{Other Professional Activities}
\textit{Topics in Analysis} \\
``Four Classical PDEs'' \\
September 20-December 2, 2008, University of Notre Dame \\

\textit{Mathematics Teaching Seminar} \\
``Taylor Approximations'' \\
April 7, 2009, University of Notre Dame \\

\textit{The 4th Symposium on Analysis \& PDEs} \\
May 26-29, 2009, Purdue University\\

\textit{Notre Dame Analysis Seminar } \\
``Pseudodifferential and Fourier Integral Operators'' \\
June 15-August 3, 2009, University of 
Notre Dame \\


\textit{Notre Dame Graduate Student Seminar} \\
``On the Uniqueness of Solutions to the Burgers Equation in Sobolev Spaces'' \\
November 23, 2009, University of Notre Dame\\

\textit{AMS Joint Mathematics Meetings} \\
January 13-16, 2010, San Francisco, California \\

\textit{Mathematical Problems in Hydrodynamics} \\
June 14-25, 2010, Universit\'e de Cergy-Pontoise, France \\

\section{Professional Memberships}
American Mathematical Society  
\vspace{1em}

\section{Service to Notre Dame}
\textit{Graduate Student Union representative for the Department of 
Mathematics} \\
January 2009-present

\section{Technical \\ Skills} 
%
Extensive knowledge of PC hardware. Expert proficiency in 
Unix/Linux/FreeBSD, Mac OS X, Windows XP/Vista/7, Bash, C, C++, HTML, 
JavaScript, Matlab, Ruby, \TeX{}, \LaTeX{}, B\textsc{ib}\TeX{}, Vim \\
%
%
\newpage
\section{References}
\vspace{-2.35em}
\begin{multicols}{2}
	{ 
Professor Alex Himonas (\textit{advisor}) \\
Department of Mathematics \\
University of Notre Dame  \\	
274 Hurley \\
Notre Dame, IN 46556-4618	\\
574-631-7538  \\
Alex.A.Himonas.1@nd.edu \\

Professor Yongtao Zhang \\
Department of Mathematics \\
University of Notre Dame \\
242 Hayes-Healy Hall \\
Notre Dame, IN 46556-4618 \\
574-631-6079 \\
zhang.103@nd.edu \\

Professor Gerard Misiolek \\
Department of Mathematics \\
University of Notre Dame \\
218 Hayes-Healy Hall \\
Notre Dame, IN 46556-4618 \\
574-631-4179 \\
Gerard.Misiolek.1@nd.edu \\
\blankline \\
\blankline \\
\blankline \\
\blankline \\
\blankline \\
\blankline \\
\blankline \\

}
\end{multicols}

\end{document}


