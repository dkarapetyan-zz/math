%
\documentclass[12pt,reqno]{amsart}
\usepackage{amssymb}
\usepackage{appendix}
\usepackage{fix-cm}
\usepackage{cancel}  %for cancelling terms explicity on pdf
\usepackage{yhmath}   %makes fourier transform look nicer, among other things
\usepackage{framed}  %for framing remarks, theorems, etc.
\usepackage{enumerate} %to change enumerate symbols
\usepackage[margin=2.5cm]{geometry}  %page layout
\setcounter{tocdepth}{1} %must come before secnumdepth--else, pain
\setcounter{secnumdepth}{1} %number only sections, not subsections
%\usepackage[pdftex]{graphicx} %for importing pictures into latex--pdf compilation
%\numberwithin{figure}{section}
\setlength{\parindent}{0in} %no indentation of paragraphs after section title
\renewcommand{\baselinestretch}{1.1} %increases vert spacing of text
%
\usepackage{hyperref}
\hypersetup{colorlinks=true,
linkcolor=blue,
citecolor=blue,
urlcolor=blue,
}
\usepackage{cleveref} %must be last loaded package to work properly
%
%
\newcommand{\ds}{\displaystyle}
\newcommand{\ts}{\textstyle}
\newcommand{\nin}{\noindent}
\newcommand{\rr}{\mathbb{R}}
\newcommand{\nn}{\mathbb{N}}
\newcommand{\zz}{\mathbb{Z}}
\newcommand{\cc}{\mathbb{C}}
\newcommand{\ci}{\mathbb{T}}
\newcommand{\zzdot}{\dot{\zz}}
\newcommand{\wh}{\widehat}
\newcommand{\p}{\partial}
\newcommand{\ee}{\varepsilon}
\newcommand{\vp}{\varphi}
\newcommand{\wt}{\widetilde}
%
%
%
%
\newtheorem{theorem}{Theorem}
\newtheorem{lemma}[theorem]{Lemma}
\newtheorem{corollary}[theorem]{Corollary}
\newtheorem{claim}[theorem]{Claim}
\newtheorem{prop}[theorem]{Proposition}
\newtheorem{proposition}[theorem]{Proposition}
\newtheorem{no}[theorem]{Notation}
\newtheorem{definition}[theorem]{Definition}
\newtheorem{remark}[theorem]{Remark}
\newtheorem{examp}{Example}[section]
\newtheorem {exercise}[theorem] {Exercise}
%
\makeatletter \renewenvironment{proof}[1][\proofname] {\par\pushQED{\qed}\normalfont\topsep6\p@\@plus6\p@\relax\trivlist\item[\hskip\labelsep\bfseries#1\@addpunct{.}]\ignorespaces}{\popQED\endtrivlist\@endpefalse} \makeatother%
%makes proof environment bold instead of italic
\newcommand{\uol}{u^\omega_\lambda}
\newcommand{\lbar}{\bar{l}}
\renewcommand{\l}{\lambda}
\newcommand{\R}{\mathbb R}
\newcommand{\RR}{\mathcal R}
\newcommand{\al}{\alpha}
\newcommand{\ve}{q}
\newcommand{\tg}{{tan}}
\newcommand{\m}{q}
\newcommand{\N}{N}
\newcommand{\ta}{{\tilde{a}}}
\newcommand{\tb}{{\tilde{b}}}
\newcommand{\tc}{{\tilde{c}}}
\newcommand{\tS}{{\tilde S}}
\newcommand{\tP}{{\tilde P}}
\newcommand{\tu}{{\tilde{u}}}
\newcommand{\tw}{{\tilde{w}}}
\newcommand{\tA}{{\tilde{A}}}
\newcommand{\tX}{{\tilde{X}}}
\newcommand{\tphi}{{\tilde{\phi}}}
\synctex=1
%
%
%
\begin{document}
\title{Research Statement} 
\author{{\it David Karapetyan}\\
University of Notre Dame}
\maketitle


\parindent0in
\parskip0.1in
%%%%%%%%%%%%%%%%%%%%%%%%
%
%      introduction
%
%%%%%%%%%%%%%%%%%%%%%%%%
%
\setcounter{section}{0}

The main focus of my research so far has been well-posedness and ill-posedness
for the hyperelastic rod (HR) equation
%
%
\begin{gather}
\label{hr}
\p_t u
-
\p_t \p_x^2 u
+
3u\p_x u
=
\gamma \big (
2\p_x u \p_x^2 u
+
u \p_x^3 u
\big ), \quad \gamma \neq 0
\\
\label{hr-data} u(x, 0) = u_0 (x),
\quad x  \in \ci, \text{  or  } \rr \quad t \in \rr,
\end{gather}
%
%
the Korteweg-–de Vries equation (KDV) 
\begin{gather}
  u_{t} + 6u_{xx} + u u_{x} = 0,
  \\
  u(x,0) = u_{0} \in H^{s}, \quad t \in \rr, \ x \in \ci \ \text{or} \ \rr
\end{gather}
%
%
the cubic nonlinear Schr\"odinger
equation (NLS)
\begin{gather}
  iu_{t} + u_{xx} + | u |^{2}u = 0,
\\
u(x,0) = u_{0} \in H^{s}, \quad t \in \rr, \ x \in \ci \ \text{or} \ \rr
\end{gather}
%
%
%
%%%%%%%%%%%%%%%%%%%%%%%%%%%%%%%%%%%%%%%%%%%%%%%%%%%%%
and the Boussinesq
equation
\begin{gather}
   u_{tt} - u_{xx} + u_{xxxx} + (u^{2})_{xx} = 0,
  \\
   u(x,0) = u_{0}(x) \in H^{s}, \  u_{t}(x,0) = u_{1}(x) \in H^{s-2},
  \quad t \in \rr, \ x \in \ci \ \text{or} \ \rr.
\end{gather}
I recently published a paper on the HR equation entitled 
{\it ``Non-Uniform Dependence  and well-posedness
for the
Hyperelastic Rod Equation''}. In it, I prove
that 
the dependence of solutions on initial data is not uniformly 
continuous in Sobolev spaces $H^s(\ci)$, $s>3/2$.
This extended a result proved by Olson 
\cite{Olson_2006_Non-uniform-dep} in the periodic
case (for $s\ge 2$ and $\gamma \ne 3$)  to  $s>3/2$ (the entire well-posedness
range) for HR. Furthermore, motivated by the work of Himonas and
Kenig~\cite{Himonas:2009fk}, \cite{Himonas:2009fk}, and 
Himonas, Kenig, and Misiolek~\cite{Himonas_2009_Non-uniform-dep-per},
I established non-uniform dependence on initial data
in the non-periodic case.
The method of proof involved choosing
approximate solutions to the HR equation such that the size of the difference between approximate and actual solutions with 
identical initial data was negligible. Hence, to understand the degree of 
dependence, it sufficed to focus on the behavior of the approximate 
solutions (which are simple in form), rather than on the behavior of the 
actual solutions. In order for the method to go through, I also proved 
well-posedness estimates for the size of the 
actual solutions to the HR equation, as well a 
lower bound for their lifespan. This was needed in order to obtain an upper
bound for the size of the difference of approximate and actual solutions. I am
currently interested in strengthening the ill-posedness result for HR from
failure of uniform continuity of the data-to-solution map to failure of
continuity for sufficiently rough initial data. More generally, I am interested
in disproving well-posedness in the sense of Hadamard (failure of existence,
uniqueness, or continuity for the data-to-solution map). 

I also have recent work (joint with Alex Himonas and Gerson Petronilho) on the
well-posedness of periodic NLS in analytic spaces. Using a suitable modification
of the Bourgain spaces \cite{Bourgain-Fourier-transfo}, we show that for analytic initial data $u_{0}(x)$ with
complex analytic extension to a symmetric strip about the real axis with width
$\delta$, there exists a unique local
solution $u(x,t)$ to the NLS which for $| t | < T$
is analytic in the space variable. In fact, this solution admits admits a
complex analytic extension to the same strip as the initial data. Furthermore,
the data-to-solution map is locally Lipschitz continuous from the space of
initial data to the space of solutions. Motivated by
\cite{Gorsky-The-Cauchy-prob}, I have also recently shown that for analytic
initial data $u_{0}$, the associated solution $u(x,t)$ is $G^{2}$ in time. I am
currently investigating whether the time regularity of the solution can be improved. 

To better understand the smoothing effect for NLS coming from the diffusive term
$\p_{x}^{2}u$, I
recently investigated KDV and NLS type equations with higher order
dispersion and diffusivity, respectively. More precisely, I have proved
existence, uniqueness, and Lipschitz continuity for initial data in
$\dot{H}^{s}, s \ge (1-m)/4$ for the higher order KDV

\begin{gather}
	\label{mmKDV-eq}
	\p_t u + \p_x^{m} u + \lambda u \p_x u = 0,
	\\
	\label{mmKDV-init-data}
  u(x,0) = u_0(x) \in \dot{H}^{s}(\ci), \quad x \in \ci, \ t \in \rr
\end{gather}
%
%
where $m \in \{3, 5, 7,\dots \}$ and $\lambda \in \{-1, 1\}$, 
%
and for initial data $H^{s}, s \ge 0$ for the higher order NLS
%
%
\begin{gather}
	\label{nmNLS-eq}
	i \p_t u + \p_x^{m} u + \lambda |u|^2 u =0,
		\\
		\label{nmNLS-init-data}
		u(x,0) = \vp(x) \in H^s(\ci), \ \ t \in \rr, \ \ x \in \ci
\end{gather}
%
%
where $m \in \{2,4,6,\dots\}$ and $\lambda \in \{-1, 1\}$. If we strengthen our
notion of well-posedness to require uniform continuity of the data-to-solution
map, then my well-posedness result for the higher order NLS is sharp. That is,
using an an argument
similar to that found in \cite{Burq_Gerad_Tzvetkov-An-instability-}, I have
shown that the data-to-solution map for the higher oder NLS is not uniformly
continuous for initial data in $H^{s}, s \le 0$. 
%
%
%
I am also working on proving failure of the continuity of the data-to-solution map for
initial data in $H^{s} \times H^{s-2}, s < -1/2$ for the Boussinesq equation. I am using three
different approaches to gain insight into the problem. The first is motivated by
\cite{Bejenaru-Tao-2006-Sharp-well-posedness-and-ill-posedness}. More
specifically, the approach runs as follows:
\begin{enumerate}[I.]
  \item
    Prove well-posedness in $H^{s} \times H^{s-2}, s \ge -1/2$.
 \item
   Show that the well-posedness implies that for a solution $u(x,t)$ with
   initial data $u_{0} \in H^{s} \times H^{s-2}$, we have the following
   decomposition (local in $t$) 
   $$u(x,t) = \sum_{n} A_{n},$$ where the $A_{n}$ are non-linear functions of
   the initial data. 
 \item 
   Choose an appropriate sequence of initial datum $u_{0,n}(x) \in H^{-1/2}
   \times H^{-3/2}$ such that
   we have failure of continuity for some iterate $A_{k}$ for $s < -1/2$.
   This will imply
   failure of the data-to-solution map for $s < -1/2$.
   \end{enumerate}
%		Bibliography		
%
The second approach is motivated by \cite{Himonas:2005kx}. That is,
to prove the failure of continuity of the data-to-solution map, I am attempting
to find a sequence of traveling wave solutions $u_{n}(x,t)$ to the Boussinesq with
corresponding initial data $u_{0,n}$ in a sufficiently rough Sobolev space such 
$u_{0,n} \to u_{0}$ in $H^{s}$ but $v_{n}(x,t) \not \to u(x,t)$ in $H^{s}$. 

Lastly, the third approach involves numerical simulation in Matlab. More
precisely, to get clues as to what which traveling wave solutions might allow me
to prove failure of continuity, I have been numerically solving various ODEs
with adjustable parameters. 

\providecommand{\bysame}{\leavevmode\hbox to3em{\hrulefill}\thinspace}
\providecommand{\MR}{\relax\ifhmode\unskip\space\fi MR }
% \MRhref is called by the amsart/book/proc definition of \MR.
\providecommand{\MRhref}[2]{%
  \href{http://www.ams.org/mathscinet-getitem?mr=#1}{#2}
}
\providecommand{\href}[2]{#2}
\begin{thebibliography}{HKM09}

\bibitem[BGT02]{Burq_Gerad_Tzvetkov-An-instability-}
N.~Burq, P.~G{\'e}rad, and N.~Tzvetkov, \emph{An instability property of the
  nonlinear schr{\"o}dinger equation on {\$}s\^{}d{\$}}, Math. Res. Lett.
  \textbf{9} (2002), no.~2-3, 323--335.

\bibitem[Bou93]{Bourgain-Fourier-transfo}
J.~Bourgain, \emph{Fourier transform restriction phenomena for certain lattice
  subsets and applications to nonlinear evolution equations. ii. the
  kdv-equation}, Geom. Funct. Anal. \textbf{3} (1993), no.~3, 209--262.

\bibitem[BT06]{Bejenaru-Tao-2006-Sharp-well-posedness-and-ill-posedness}
I.~Bejenaru and T.~Tao, \emph{Sharp well-posedness and ill-posedness results
  for a quadratic non-linear schr{\"o}dinger equation}, J. Funct. Anal.
  \textbf{233} (2006), no.~1, 228--259.

\bibitem[Gor04]{Gorsky-The-Cauchy-prob}
J.~M. Gorsky, \emph{The cauchy problem for a modified camassa-holm equation
  with analytic initial data}, Differential Integral Equations \textbf{17}
  (2004), no.~11-12, 1233--1254.

\bibitem[HK09]{Himonas:2009fk}
A.~A. Himonas and C.~Kenig, \emph{Non-uniform dependence on initial data for
  the ch equation on the line}, Differential Integral Equations \textbf{22}
  (2009), no.~3-4, 201--224.

\bibitem[HKM09]{Himonas_2009_Non-uniform-dep-per}
A.~Himonas, C.~E. Kenig, and G.~Misiolek, \emph{Non-uniform dependence for the
  periodic ch equation.}, To appear in Communications in Partial Differential
  Equations (2009).

\bibitem[HM05]{Himonas:2005kx}
A.~A. Himonas and G.~Misiolek, \emph{High-frequency smooth solutions and
  well-posedness of the {C}amassa-{H}olm equation}, Int. Math. Res. Not.
  (2005), no.~51, 3135--3151. \MR{2187502 (2006i:35315)}

\bibitem[Ols06]{Olson_2006_Non-uniform-dep}
E.~Olson, \emph{Non-uniform dependence on initial data for a family of
  non-linear evolution equations}, Differential Integral Equations \textbf{19}
  (2006), no.~10, 1081--1102.

\end{thebibliography}
%\bibliography{/Users/davidkarapetyan/math/bib-files/references.bib}
%\bibliographystyle{amsalpha-custom}
\end{document}

