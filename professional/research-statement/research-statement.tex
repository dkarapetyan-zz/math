%
\documentclass[12pt,reqno]{amsart}
\usepackage{amssymb}
\usepackage{appendix}
\usepackage{cancel}  %for cancelling terms explicity on pdf
\usepackage{yhmath}   %makes fourier transform look nicer, among other things
\usepackage{framed}  %for framing remarks, theorems, etc.
\usepackage{enumerate} %to change enumerate symbols
\usepackage[margin=2.5cm]{geometry}  %page layout
\setcounter{tocdepth}{1} %must come before secnumdepth--else, pain
\setcounter{secnumdepth}{1} %number only sections, not subsections
%\usepackage[pdftex]{graphicx} %for importing pictures into latex--pdf compilation
%\numberwithin{figure}{section}
\setlength{\parindent}{0in} %no indentation of paragraphs after section title
\renewcommand{\baselinestretch}{1.1} %increases vert spacing of text
%
\usepackage{hyperref}
\hypersetup{colorlinks=true,
linkcolor=blue,
citecolor=blue,
urlcolor=blue,
}
\usepackage{cleveref} %must be last loaded package to work properly
%
%
\newcommand{\ds}{\displaystyle}
\newcommand{\ts}{\textstyle}
\newcommand{\nin}{\noindent}
\newcommand{\rr}{\mathbb{R}}
\newcommand{\nn}{\mathbb{N}}
\newcommand{\zz}{\mathbb{Z}}
\newcommand{\cc}{\mathbb{C}}
\newcommand{\ci}{\mathbb{T}}
\newcommand{\zzdot}{\dot{\zz}}
\newcommand{\wh}{\widehat}
\newcommand{\p}{\partial}
\newcommand{\ee}{\varepsilon}
\newcommand{\vp}{\varphi}
\newcommand{\wt}{\widetilde}
%
%
%
%
\newtheorem{theorem}{Theorem}
\newtheorem{lemma}[theorem]{Lemma}
\newtheorem{corollary}[theorem]{Corollary}
\newtheorem{claim}[theorem]{Claim}
\newtheorem{prop}[theorem]{Proposition}
\newtheorem{proposition}[theorem]{Proposition}
\newtheorem{no}[theorem]{Notation}
\newtheorem{definition}[theorem]{Definition}
\newtheorem{remark}[theorem]{Remark}
\newtheorem{examp}{Example}[section]
\newtheorem {exercise}[theorem] {Exercise}
%
\makeatletter \renewenvironment{proof}[1][\proofname] {\par\pushQED{\qed}\normalfont\topsep6\p@\@plus6\p@\relax\trivlist\item[\hskip\labelsep\bfseries#1\@addpunct{.}]\ignorespaces}{\popQED\endtrivlist\@endpefalse} \makeatother%
%makes proof environment bold instead of italic
\newcommand{\uol}{u^\omega_\lambda}
\newcommand{\lbar}{\bar{l}}
\renewcommand{\l}{\lambda}
\newcommand{\R}{\mathbb R}
\newcommand{\RR}{\mathcal R}
\newcommand{\al}{\alpha}
\newcommand{\ve}{q}
\newcommand{\tg}{{tan}}
\newcommand{\m}{q}
\newcommand{\N}{N}
\newcommand{\ta}{{\tilde{a}}}
\newcommand{\tb}{{\tilde{b}}}
\newcommand{\tc}{{\tilde{c}}}
\newcommand{\tS}{{\tilde S}}
\newcommand{\tP}{{\tilde P}}
\newcommand{\tu}{{\tilde{u}}}
\newcommand{\tw}{{\tilde{w}}}
\newcommand{\tA}{{\tilde{A}}}
\newcommand{\tX}{{\tilde{X}}}
\newcommand{\tphi}{{\tilde{\phi}}}
\synctex=1
%
%
%
\begin{document}
\title{Research Statement} 
\author{{\it David Karapetyan}\\
University of Notre Dame}
\maketitle


\markboth{Non-Uniform Dependence for the Hyperelastic Rod Equation}{David 
Karapetyan}
\parindent0in
\parskip0.1in
%%%%%%%%%%%%%%%%%%%%%%%%
%
%      introduction
%
%%%%%%%%%%%%%%%%%%%%%%%%
%
\setcounter{section}{0}
\subsection{Well-Posedness and Ill-Posedness for \\``Burgers Type'' Nonlinear Evolution Equations}
I recently considered the initial value problem for
the hyperelastic rod (HR) equation
%
%
\begin{gather}
\label{hr}
\p_t u
-
\p_t \p_x^2 u
+
3u\p_x u
=
\gamma \big (
2\p_x u \p_x^2 u
+
u \p_x^3 u
\big ),
\\
\label{hr-data} u(x, 0) = u_0 (x),
\quad x  \in \ci, \text{  or  } \rr \quad t \in \rr,
\end{gather}
%
%
where $\gamma$  is a  nonzero constant,
and proved that the dependence of solutions on initial data is not uniformly 
continuous in Sobolev spaces $H^s(\ci)$, $s>3/2$.
This extended a result proved by Olson 
\cite{Olson_2006_Non-uniform-dep} in the periodic
case (for $s\ge 2$ and $\gamma \ne 3$)  to  $s>3/2$ (the entire well-posedness
range) for HR. Furthermore, motivated by the work of Himonas and
Kenig~\cite{Himonas:2009fk}
I established non-uniform dependence on initial data
in the non-periodic case. More precisely, I showed the following. 
%
%
\begin{theorem}
\label{hr-non-unif-dependence}
Let $\gamma$ be a nonzero constant. Then 
the data-to-solution map $u(0) \mapsto u(t)$ of the Cauchy-problem
for the HR equation
\eqref{hr}-\eqref{hr-data}
is not uniformly continuous
from any bounded subset of  $H^s$ into $C([-T, T], H^s)$
for $s>3/2$ on the line and circle. %
\end{theorem}
%
\begin{framed}
I still need to fix the argument for the real line.
\end{framed}
%
My approach for proving \autoref{hr-non-unif-dependence}  
mirrors that in Himonas and Kenig~\cite{Himonas:2009fk} and 
Himonas, Kenig, and Misiolek~\cite{Himonas_2009_Non-uniform-dep-per}.
That is, I choose 
approximate solutions to the HR equation such that the size of the difference between approximate and actual solutions with 
identical initial data is negligible. Hence, to understand the degree of 
dependence, it sufficed to focus on the behavior of the approximate 
solutions (which are simple in form), rather than on the behavior of the 
actual solutions. In order for the method to go through, I needed 
well-posedness estimates for the size of the 
actual solutions to the HR equation, as well a 
lower bound for their lifespan. This was needed in order to obtain an upper 
bound for the size of the difference of approximate and actual solutions. 
More precisely, I needed the following well-posedness result  with estimates,  
stated in both the  periodic and non-periodic case.
%%%%%%%%%%%%%%%%%%%%%%%%
%
%            wp of theorem in R and T
%
%%%%%%%%%%%%%%%%%%%%%%%%
%
%
%
%
\begin{theorem}
\label{thm:HR_existence_continuous_dependence}
If   $s>3/2$  then we have:

(i) If $u_0\in H^s$  then  there exists a unique solution to
the Cauchy problem  \eqref{hr}--\eqref{hr-data} in $C([-T, T], H^s)$, where 
the lifespan  $T$ depends on the size
of the initial data $u_0$. Moreover, 
the  lifespan $T$ satisfies the lower bound estimate 
%
%
%
\begin{equation}
\label{Life-span-est}
T
\ge
\frac{1}{2c_s \|u_0\|_{H^s}}.
\end{equation}
%

(ii)
The flow map $u_0 \mapsto u(t)$ is continuous from
bounded sets of $H^s$ into \\ $C([-T, T], H^s)$,
and the solution $u$ satisfies the estimate
%
%
%
\begin{equation}
\label{u_x-Linfty-Hs}
\|
u(t)
\|_ {H^s}
\le
2
\|
u_0
\|_{H^s}, \ \ |t|\le T.
\end{equation}
%
%
%
\end{theorem}
%
%
Proofs of 
existence, uniqueness, and continuous dependence for HR
have been outlined in~\cite{Yin_2003_On-the-Cauchy-p} and~\cite{Zhou_2005_Local-well-pose} for the line and circle, 
respectively. Both outlines rely upon an application of Kato's semi-group 
method. However, I was not able to find estimates  
\eqref{Life-span-est} and \eqref{u_x-Linfty-Hs}  in the literature.
Hence, I also provided a proof of local well-posedness of HR,
including  estimates \eqref{Life-span-est} and \eqref{u_x-Linfty-Hs},
which are key ingredients in the proof of non-uniform dependence, by 
following an alternative approach used for nonlinear hyperbolic equations
in Taylor~\cite{Taylor_1991_Pseudodifferent}.

I am currently investigating ill-posedness for the HR equation. In particular, I
am interested in showing that the data-to-solution map fails to be continuous
for sufficiently rough initial data. 

\subsection{Well-Posedness for Dispersive and Dissipative \\
Nonlinear Evolution Equations} 
I am also interested in proving well-posedness for a series of unexplored dispersive equations modeled after the Korteweg–de Vries equation (KDV) 
\begin{gather*}
  u_{t} + 6u_{xx} + u u_{x} = 0,
  \\
  u(x,0) = u_{0} \in H^{s}, \quad t \in \rr, \ x \in \ci \ \text{or} \ \rr
\end{gather*}
%
%
and the cubic nonlinear Schr\"odinger
equation (NLS)
\begin{gather*}
  iu_{t} + u_{xx} + | u |^{2}u = 0,
\\
u(x,0) = u_{0} \in H^{s}, \quad t \in \rr, \ x \in \ci \ \text{or} \ \rr
\end{gather*}

Specifically, I am interested in the relationship between the degree of
dispersion (or dissipation) and the lowest Sobolev index for which local and
global well-posedness holds.
%
%%%%%%%%%%%%%%%%%%%%%%%%%%%%%%%%%%%%%%%%%%%%%%%%%%%%%
My research on the KDV has led me to become interested in the Boussinesq
equation
\begin{equation*}
  \begin{split}
  & u_{tt} - u_{xx} + u_{xxxx} + (u^{2})_{xx} = 0,
  \\
  & u(x,0) = u_{0} \in H^{s} \\
  & u_{t}(x,0) = u_{1} \in H^{s-2},
  \quad t \in \rr, \ x \in \ci \ \text{or} \ \rr.
\end{split}
\end{equation*}
\begin{framed}
Alex, how much can I reveal about the current work we are doing on
this equation?
\end{framed}
%
\subsection{Numerics} 
 \newpage
 \begin{framed}
   Would like to mention how I am using numerics to construct the necessary traveling
wave solutions to prove ill-posedness for Boussinesq. Don't know if I should at
this point, since I don't yet have a result.
\end{framed}
%
%		Bibliography		
%
%
%%%%%%%%%%%%%%%%%%%%%%%%%%%%%%%%%%%%%%%%%%%%%%%%%%%%%
%
% \bib, bibdiv, biblist are defined by the amsrefs package.
%
\providecommand{\bysame}{\leavevmode\hbox to3em{\hrulefill}\thinspace}
\providecommand{\MR}{\relax\ifhmode\unskip\space\fi MR }
% \MRhref is called by the amsart/book/proc definition of \MR.
\providecommand{\MRhref}[2]{%
  \href{http://www.ams.org/mathscinet-getitem?mr=#1}{#2}
}
\providecommand{\href}[2]{#2}
\begin{thebibliography}{HKM09}

\bibitem[HK09]{Himonas:2009fk}
A.~Himonas and C.~Kenig, \emph{Non-uniform dependence on initial data for
  the ch equation on the line}, Differential Integral Equations \textbf{22}
  (2009), no.~3-4, 201--224.

\bibitem[HKM09]{Himonas_2009_Non-uniform-dep-per}
A.~Himonas, C.~E. Kenig, and G.~Misiolek, \emph{Non-uniform dependence for the
  periodic ch equation.}, To appear in Communications in Partial Differential
  Equations (2009).

\bibitem[Ols06]{Olson_2006_Non-uniform-dep}
E.~Olson, \emph{Non-uniform dependence on initial data for a family of
  non-linear evolution equations}, Differential Integral Equations \textbf{19}
  (2006), no.~10, 1081--1102.

\bibitem[Tay91]{Taylor_1991_Pseudodifferent}
M.~E. Taylor, \emph{Pseudodifferential operators and nonlinear pde}, vol. 100,
  1991.

\bibitem[Yin03]{Yin_2003_On-the-Cauchy-p}
Z.~Yin, \emph{On the cauchy problem for a nonlinearly dispersive wave
  equation}, J. Nonlinear Math. Phys. \textbf{10} (2003), no.~1, 10--15.

\bibitem[Zho05]{Zhou_2005_Local-well-pose}
Y.~Zhou, \emph{Local well-posedness and blow-up criteria of solutions for a rod
  equation}, Math. Nachr. \textbf{278} (2005), no.~14, 1726--1739.

\end{thebibliography}


%\bibliography{/Users/davidkarapetyan/math/bib-files/references.bib}
%\bibliographystyle{amsalpha-custom}
\end{document}

