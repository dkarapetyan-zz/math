%
\documentclass[12pt,reqno]{amsart}
\usepackage{amscd}
\usepackage{amsfonts}
\usepackage{amsmath}
\usepackage{amssymb}
\usepackage{amsthm}
\usepackage{appendix}
\usepackage{fancyhdr}
\usepackage{latexsym}
\usepackage[shortalphabetic, initials, msc-links]{amsrefs} 
\usepackage{cancel}
\usepackage{amsxtra}
\synctex=1
\hypersetup{colorlinks=true,
    linkcolor=blue,
    citecolor=blue,
    urlcolor=blue,
}
%%%%%%%%%%%%%%%%%%%%%%
%%%%%%%%%%%%%%%%%%%%%%
%%%%%%%%%%%%%%%%%%%%%%%%%%%%
%  for importing pictures  %
%%%%%%%%%%%%%%%%%%%%%%%%%%%%
% \usepackage{graphicx}
%% page setup %%
\setlength{\textheight}{20.8truecm}
\setlength{\textwidth}{14.8truecm}
\marginparwidth  0truecm
\oddsidemargin   01truecm
\evensidemargin  01truecm
\marginparsep    0truecm
\renewcommand{\baselinestretch}{1.1}
\renewcommand{\qedsymbol}{\ensuremath{\square}}
%% new commands %%
\newcommand{\tf}{\tilde{f}}
\newcommand{\ti}{\tilde}
\newcommand{\bigno}{\bigskip\noindent}
\newcommand{\ds}{\displaystyle}
\newcommand{\medno}{\medskip\noindent}
\newcommand{\smallno}{\smallskip\noindent}
\newcommand{\nin}{\noindent}
\newcommand{\ts}{\textstyle}
\newcommand{\rr}{\mathbb{R}}
\newcommand{\p}{\partial}
\newcommand{\zz}{\mathbb{Z}}
\newcommand{\cc}{\mathbb{C}}
\newcommand{\ci}{\mathbb{T}}
\newcommand{\ee}{\varepsilon}
\newcommand{\vp}{\varphi}
\def\autorefer #1\par{\noindent\hangindent=\parindent\hangafter=1 #1\par}
%% equation numbers %%
\renewcommand{\theequation}{\thesection.\arabic{equation}}
%% new environments %%
%\swapnumbers
\theoremstyle{plain}  % default
\newtheorem{theorem}{Theorem}
\newtheorem{proposition}{Proposition}
\newtheorem{lemma}{Lemma}
\newtheorem{corollary}{Corollary}
\newtheorem{claim}{Claim}
\newtheorem{remark}{Remark}
\newtheorem{conjecture}[subsection]{conjecture}
%
\theoremstyle{definition}
\newtheorem{definition}{Definition}
\def\makeautorefname#1#2{\expandafter\def\csname#1autorefname\endcsname{#2}}
\makeautorefname{equation}{Equation}
\makeautorefname{footnote}{footnote}
\makeautorefname{item}{item}
\makeautorefname{figure}{Figure}
\makeautorefname{table}{Table}
\makeautorefname{part}{Part}
\makeautorefname{appendix}{Appendix}
\makeautorefname{chapter}{Chapter}
\makeautorefname{section}{Section}
\makeautorefname{subsection}{Section}
\makeautorefname{subsubsection}{Section}
\makeautorefname{paragraph}{Paragraph}
\makeautorefname{subparagraph}{Paragraph}
\makeautorefname{theorem}{Theorem}
\makeautorefname{theo}{Theorem}
\makeautorefname{thm}{Theorem}
\makeautorefname{addendum}{Addendum}
\makeautorefname{addend}{Addendum}
\makeautorefname{add}{Addendum}
\makeautorefname{maintheorem}{Main theorem}
\makeautorefname{mainthm}{Main theorem}
\makeautorefname{corollary}{Corollary}
\makeautorefname{corol}{Corollary}
\makeautorefname{coro}{Corollary}
\makeautorefname{cor}{Corollary}
\makeautorefname{lemma}{Lemma}
\makeautorefname{lemm}{Lemma}
\makeautorefname{lem}{Lemma}
\makeautorefname{sublemma}{Sublemma}
\makeautorefname{sublem}{Sublemma}
\makeautorefname{subl}{Sublemma}
\makeautorefname{proposition}{Proposition}
\makeautorefname{proposit}{Proposition}
\makeautorefname{propos}{Proposition}
\makeautorefname{propo}{Proposition}
\makeautorefname{prop}{Proposition}
\makeautorefname{proposition}{Proposition}
\makeautorefname{property}{Property}
\makeautorefname{proper}{Property}
\makeautorefname{scholium}{Scholium}
\makeautorefname{step}{Step}
\makeautorefname{conjecture}{Conjecture}
\makeautorefname{conject}{Conjecture}
\makeautorefname{conj}{Conjecture}
\makeautorefname{question}{Question}
\makeautorefname{questn}{Question}
\makeautorefname{quest}{Question}
\makeautorefname{ques}{Question}
\makeautorefname{qn}{Question}
\makeautorefname{definition}{Definition}
\makeautorefname{defin}{Definition}
\makeautorefname{defi}{Definition}
\makeautorefname{def}{Definition}
\makeautorefname{dfn}{Definition}
\makeautorefname{notation}{Notation}
\makeautorefname{nota}{Notation}
\makeautorefname{notn}{Notation}
\makeautorefname{remark}{Remark}
\makeautorefname{rema}{Remark}
\makeautorefname{rem}{Remark}
\makeautorefname{rmk}{Remark}
\makeautorefname{rk}{Remark}
\makeautorefname{remarks}{Remarks}
\makeautorefname{rems}{Remarks}
\makeautorefname{rmks}{Remarks}
\makeautorefname{rks}{Remarks}
\makeautorefname{example}{Example}
\makeautorefname{examp}{Example}
\makeautorefname{exmp}{Example}
\makeautorefname{exam}{Example}
\makeautorefname{exa}{Example}
\makeautorefname{algorithm}{Algorith}
\makeautorefname{algo}{Algorith}
\makeautorefname{alg}{Algorith}
\makeautorefname{axiom}{Axiom}
\makeautorefname{axi}{Axiom}
\makeautorefname{ax}{Axiom}
\makeautorefname{case}{Case}
\makeautorefname{claim}{Claim}
\makeautorefname{clm}{Claim}
\makeautorefname{assumption}{Assumption}
\makeautorefname{assumpt}{Assumption}
\makeautorefname{conclusion}{Conclusion}
\makeautorefname{concl}{Conclusion}
\makeautorefname{conc}{Conclusion}
\makeautorefname{condition}{Condition}
\makeautorefname{condit}{Condition}
\makeautorefname{cond}{Condition}
\makeautorefname{construction}{Construction}
\makeautorefname{construct}{Construction}
\makeautorefname{const}{Construction}
\makeautorefname{cons}{Construction}
\makeautorefname{criterion}{Criterion}
\makeautorefname{criter}{Criterion}
\makeautorefname{crit}{Criterion}
\makeautorefname{exercise}{Exercise}
\makeautorefname{exer}{Exercise}
\makeautorefname{exe}{Exercise}
\makeautorefname{problem}{Problem}
\makeautorefname{problm}{Problem}
\makeautorefname{probm}{Problem}
\makeautorefname{prob}{Problem}
\makeautorefname{solution}{Solution}
\makeautorefname{soln}{Solution}
\makeautorefname{sol}{Solution}
\makeautorefname{summary}{Summary}
\makeautorefname{summ}{Summary}
\makeautorefname{sum}{Summary}
\makeautorefname{operation}{Operation}
\makeautorefname{oper}{Operation}
\makeautorefname{observation}{Observation}
\makeautorefname{observn}{Observation}
\makeautorefname{obser}{Observation}
\makeautorefname{obs}{Observation}
\makeautorefname{ob}{Observation}
\makeautorefname{convention}{Convention}
\makeautorefname{convent}{Convention}
\makeautorefname{conv}{Convention}
\makeautorefname{cvn}{Convention}
\makeautorefname{warning}{Warning}
\makeautorefname{warn}{Warning}
\makeautorefname{note}{Note}
\makeautorefname{fact}{Fact}
%
\begin{document}
%
%
%
%
\title{Research Summary} 
\author{{\it David Karapetyan}\\
University of Notre Dame}
\maketitle
%\begin{abstract}
%It is shown that the solution map for the hyperelastic rod equation is not 
%uniformly continuous on bounded sets of Sobolev spaces with exponent 
%greater than 3/2 in the periodic case and greater than 1 in the 
%non-periodic case. The proof is based on the method of approximate 
%solutions and well-posedness estimates for the solution and its lifespan.%
%\end{abstract}




%\maketitle
\markboth{Non-Uniform Dependence for the Hyperelastic Rod Equation}{David 
Karapetyan}
\parindent0in
\parskip0.1in
%%%%%%%%%%%%%%%%%%%%%%%%
%
%      introduction
%
%%%%%%%%%%%%%%%%%%%%%%%%
\setcounter{equation}{0}
%
I recently considered the initial value problem for
the hyperelastic rod (HR) equation
%
%
\begin{equation}
\label{hr}
\p_t u
-
\p_t \p_x^2 u
+
3u\p_x u
=
\gamma \big (
2\p_x u \p_x^2 u
+
u \p_x^3 u
\big ),
\end{equation}
%
%
%
%
\begin{equation}
\label{hr-data} u(x, 0) = u_0 (x),
\quad x  \in \ci, \text{  or  } \rr \quad t \in \rr,
\end{equation}
%
%
where  $\gamma$  is a  nonzero constant,
and proved that the dependence of solutions on initial data is not uniformly 
continuous in Sobolev spaces $H^s(\ci)$, $s>3/2$.
This extended a result proved by Olson 
\cite{Olson_2006_Non-uniform-dep} in the periodic
case (for $s\ge 2$ and $\gamma \ne 3$)  to  $s>3/2$ (the entire 
well-posedness range)
for HR. Furthermore,  motivated by the work of
Himonas  and Kenig \cite{Himonas_2009_Non-uniform-dep},
I established non-uniform dependence on initial data
in the non-periodic case. More precisely, I showed the following. 
%
%
\begin{theorem}
\label{hr-non-unif-dependence}
Let $\gamma$ be a nonzero constant. Then 
the data-to-solution map $u(0) \mapsto u(t)$ of the Cauchy-problem
for the HR equation
\eqref{hr}-\eqref{hr-data}
is not uniformly continuous
from any bounded subset of  $H^s$ into $C([-T, T], H^s)$
for $s>1$ on the line  and for $s>3/2$ on the circle.
%
\end{theorem}
%
%
My approach  for proving \autoref{hr-non-unif-dependence}  
mirrors that in Himonas and Kenig \cite{Himonas_2009_Non-uniform-dep} and 
Himonas, Kenig, and Misiolek \cite{Himonas_2009_Non-uniform-dep-per}.
That is, I choose 
approximate solutions to the HR equation such that the size of the difference between approximate and actual solutions with 
identical initial data is negligible. Hence, to understand the degree of 
dependence, it sufficed to focus on the behavior of the approximate 
solutions (which are simple in form), rather than on the behavior of the 
actual solutions. In order for the method to go through, I needed 
well-posedness estimates for the size of the 
actual solutions to the HR equation, as well a 
lower bound for their lifespan. This was needed in order to obtain an upper 
bound for the size of the difference of approximate and actual solutions. 
More precisely, I needed the following well-posedness result  with estimates,  
stated in both the  periodic and non-periodic case.
%%%%%%%%%%%%%%%%%%%%%%%%
%
%            wp of theorem in R and T
%
%%%%%%%%%%%%%%%%%%%%%%%%
%
%
%
%
\begin{theorem}
\label{thm:HR_existence_continuous_dependence}
If   $s>3/2$  then we have:

(i) If $u_0\in H^s$  then  there exists a unique solution to
the Cauchy problem  \eqref{hr}--\eqref{hr-data} in $C([-T, T], H^s)$, where 
the lifespan  $T$ depends on the size
of the initial data $u_0$. Moreover, 
the  lifespan $T$ satisfies the lower bound estimate 
%
%
%
\begin{equation}
\label{Life-span-est}
T
\ge
\frac{1}{2c_s \|u_0\|_{H^s}}.
\end{equation}
%

(ii)
The flow map $u_0 \mapsto u(t)$ is continuous from
bounded sets of $H^s$ into \\ $C([-T, T], H^s)$,
and the solution $u$ satisfies the estimate
%
%
%
\begin{equation}
\label{u_x-Linfty-Hs}
\|
u(t)
\|_ {H^s}
\le
2
\|
u_0
\|_{H^s}, \ \ |t|\le T.
\end{equation}
%
%
%
\end{theorem}
%
%
Proofs of 
existence, uniqueness, and continuous dependence for HR
have been outlined in \cite{Yin_2003_On-the-Cauchy-p} and 
\cite{Zhou_2005_Local-well-pose} for the line and circle, 
respectively. Both outlines rely upon an application of Kato's semi-group 
method. However, I was not able to find estimates  
\eqref{Life-span-est} and \eqref{u_x-Linfty-Hs}  in the literature.
Hence, I also provided a proof of local well-posedness of HR,
including  estimates \eqref{Life-span-est} and \eqref{u_x-Linfty-Hs},
which are key ingredients in the proof of non-uniform dependence, by 
following an alternative approach used for nonlinear hyperbolic equations
in Taylor \cite{Taylor_1991_Pseudodifferent}.

I have since moved into studying a series of unexplored dispersive equations
modeled after the KDV and NLS. Specifically, I am interested in the relationship
between the degree of dispersion and the lowest Sobolev index for which local
and global well-posedness holds.
%
%%%%%%%%%%%%%%%%%%%%%%%%%%%%%%%%%%%%%%%%%%%%%%%%%%%%%
%
%
%		Bibliography		
%
%
%%%%%%%%%%%%%%%%%%%%%%%%%%%%%%%%%%%%%%%%%%%%%%%%%%%%%
%
% \bib, bibdiv, biblist are defined by the amsrefs package.
%
%\bibliography{/Users/davidkarapetyan/math/bib-files/references.bib}
% \bib, bibdiv, biblist are defined by the amsrefs package.
\begin{bibdiv}
\begin{biblist}

\bib{Himonas_2009_Non-uniform-dep}{article}{
      author={Himonas, Alex},
      author={Kenig, Carlos~E.},
       title={Non-uniform dependence on initial data for the ch equation on the
  line},
        date={2009},
     journal={Differential Integral Equations},
      volume={22},
      number={3-4},
       pages={201\ndash 224},
}

\bib{Himonas_2009_Non-uniform-dep-per}{article}{
      author={Himonas, Alex},
      author={Kenig, Carlos~E.},
      author={Misio{\l}ek, G.},
       title={Non-uniform dependence for the periodic ch equation.},
        date={2009},
     journal={To appear in Communications in Partial Differential Equations},
}

\bib{Olson_2006_Non-uniform-dep}{article}{
      author={Olson, Erika},
       title={Non-uniform dependence on initial data for a family of non-linear
  evolution equations},
        date={2006},
     journal={Differential Integral Equations},
      volume={19},
      number={10},
       pages={1081\ndash 1102},
}

\bib{Taylor_1991_Pseudodifferent}{book}{
      author={Taylor, Michael~E.},
       title={Pseudodifferential operators and nonlinear {PDE}},
      series={Progress in Mathematics},
   publisher={Birkh{\"a}user Boston Inc.},
     address={Boston, MA},
        date={1991},
      volume={100},
        ISBN={0-8176-3595-5},
}

\bib{Yin_2003_On-the-Cauchy-p}{article}{
      author={Yin, Zhaoyang},
       title={On the cauchy problem for a nonlinearly dispersive wave
  equation},
        date={2003},
     journal={J. Nonlinear Math. Phys.},
      volume={10},
      number={1},
       pages={10\ndash 15},
}

\bib{Zhou_2005_Local-well-pose}{article}{
      author={Zhou, Yong},
       title={Local well-posedness and blow-up criteria of solutions for a rod
  equation},
        date={2005},
     journal={Math. Nachr.},
      volume={278},
      number={14},
       pages={1726\ndash 1739},
}

\end{biblist}
\end{bibdiv}
\end{document}

