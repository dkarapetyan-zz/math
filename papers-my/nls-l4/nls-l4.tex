%
\documentclass[12pt,reqno]{amsart}
\usepackage{amscd}
\usepackage{amsfonts}
\usepackage{amsmath}
\usepackage{amssymb}
\usepackage{amsthm}
\usepackage{appendix}
\usepackage{fancyhdr}
\usepackage{latexsym}
\usepackage{cancel}
\usepackage{amsxtra}
\synctex=1
\usepackage[colorlinks=true, pdfstartview=fitv, linkcolor=blue,
citecolor=blue, urlcolor=blue]{hyperref}
\input epsf
\input texdraw
\input txdtools.tex
\input xy
\xyoption{all}
%\setcounter{secnumdepth}{1} 
%%%%%%%%%%%%%%%%%%%%%%
\usepackage{color}
\definecolor{red}{rgb}{1.00, 0.00, 0.00}
\definecolor{darkgreen}{rgb}{0.00, 1.00, 0.00}
\definecolor{blue}{rgb}{0.00, 0.00, 1.00}
\definecolor{cyan}{rgb}{0.00, 1.00, 1.00}
\definecolor{magenta}{rgb}{1.00, 0.00, 1.00}
\definecolor{deepskyblue}{rgb}{0.00, 0.75, 1.00}
\definecolor{darkgreen}{rgb}{0.00, 0.39, 0.00}
\definecolor{springgreen}{rgb}{0.00, 1.00, 0.50}
\definecolor{darkorange}{rgb}{1.00, 0.55, 0.00}
\definecolor{orangered}{rgb}{1.00, 0.27, 0.00}
\definecolor{deeppink}{rgb}{1.00, 0.08, 0.57}
\definecolor{darkviolet}{rgb}{0.58, 0.00, 0.82}
\definecolor{saddlebrown}{rgb}{0.54, 0.27, 0.07}
\definecolor{black}{rgb}{0.00, 0.00, 0.00}
\definecolor{dark-magenta}{rgb}{.5,0,.5}
\definecolor{myblack}{rgb}{0,0,0}
\definecolor{darkgray}{gray}{0.5}
\definecolor{lightgray}{gray}{0.75}
%%%%%%%%%%%%%%%%%%%%%%
%%%%%%%%%%%%%%%%%%%%%%%%%%%%
%  for importing pictures  %
%%%%%%%%%%%%%%%%%%%%%%%%%%%%
\usepackage[pdftex]{graphicx}
\usepackage{epstopdf}
% \usepackage{graphicx}
%% page setup %%
\setlength{\textheight}{20.8truecm}
\setlength{\textwidth}{14.8truecm}
\marginparwidth  0truecm
\oddsidemargin   01truecm
\evensidemargin  01truecm
\marginparsep    0truecm
\renewcommand{\baselinestretch}{1.1}
%% new commands %%
\newcommand{\tf}{\tilde{f}}
\newcommand{\ti}{\tilde}
\newcommand{\wh}{\widehat}
\newcommand{\bigno}{\bigskip\noindent}
\newcommand{\ds}{\displaystyle}
\newcommand{\medno}{\medskip\noindent}
\newcommand{\smallno}{\smallskip\noindent}
\newcommand{\nin}{\noindent}
\newcommand{\ts}{\textstyle}
\newcommand{\rr}{\mathbb{R}}
\newcommand{\p}{\partial}
\newcommand{\zz}{\mathbb{Z}}
\newcommand{\zzdot}{\dot{\zz}}
\newcommand{\cc}{\mathbb{C}}
\newcommand{\ci}{\mathbb{T}}
\newcommand{\ee}{\varepsilon}
\newcommand{\vp}{\varphi}
\def\autorefer #1\par{\noindent\hangindent=\parindent\hangafter=1 #1\par}
%% equation numbers %%
\renewcommand{\theequation}{\thesection.\arabic{equation}}
%% new environments %%
%\swapnumbers
\theoremstyle{plain}  % default
\newtheorem{theorem}{Theorem}
\newtheorem{proposition}{Proposition}
\newtheorem{lemma}{Lemma}
\newtheorem{corollary}{Corollary}
\newtheorem{claim}{Claim}
\newtheorem{remark}{Remark}
\newtheorem{conjecture}[subsection]{conjecture}
\newtheorem{definition}{Definition}
\def\makeautorefname#1#2{\expandafter\def\csname#1autorefname\endcsname{#2}}
\makeautorefname{equation}{Equation}
\makeautorefname{footnote}{footnote}
\makeautorefname{item}{item}
\makeautorefname{figure}{Figure}
\makeautorefname{table}{Table}
\makeautorefname{part}{Part}
\makeautorefname{appendix}{Appendix}
\makeautorefname{chapter}{Chapter}
\makeautorefname{section}{Section}
\makeautorefname{subsection}{Section}
\makeautorefname{subsubsection}{Section}
\makeautorefname{paragraph}{Paragraph}
\makeautorefname{subparagraph}{Paragraph}
\makeautorefname{theorem}{Theorem}
\makeautorefname{theo}{Theorem}
\makeautorefname{thm}{Theorem}
\makeautorefname{addendum}{Addendum}
\makeautorefname{addend}{Addendum}
\makeautorefname{add}{Addendum}
\makeautorefname{maintheorem}{Main theorem}
\makeautorefname{mainthm}{Main theorem}
\makeautorefname{corollary}{Corollary}
\makeautorefname{corol}{Corollary}
\makeautorefname{coro}{Corollary}
\makeautorefname{cor}{Corollary}
\makeautorefname{lemma}{Lemma}
\makeautorefname{lemm}{Lemma}
\makeautorefname{lem}{Lemma}
\makeautorefname{sublemma}{Sublemma}
\makeautorefname{sublem}{Sublemma}
\makeautorefname{subl}{Sublemma}
\makeautorefname{proposition}{Proposition}
\makeautorefname{proposit}{Proposition}
\makeautorefname{propos}{Proposition}
\makeautorefname{propo}{Proposition}
\makeautorefname{prop}{Proposition}
\makeautorefname{proposition}{Proposition}
\makeautorefname{property}{Property}
\makeautorefname{proper}{Property}
\makeautorefname{scholium}{Scholium}
\makeautorefname{step}{Step}
\makeautorefname{conjecture}{Conjecture}
\makeautorefname{conject}{Conjecture}
\makeautorefname{conj}{Conjecture}
\makeautorefname{question}{Question}
\makeautorefname{questn}{Question}
\makeautorefname{quest}{Question}
\makeautorefname{ques}{Question}
\makeautorefname{qn}{Question}
\makeautorefname{definition}{Definition}
\makeautorefname{defin}{Definition}
\makeautorefname{defi}{Definition}
\makeautorefname{def}{Definition}
\makeautorefname{dfn}{Definition}
\makeautorefname{notation}{Notation}
\makeautorefname{nota}{Notation}
\makeautorefname{notn}{Notation}
\makeautorefname{remark}{Remark}
\makeautorefname{rema}{Remark}
\makeautorefname{rem}{Remark}
\makeautorefname{rmk}{Remark}
\makeautorefname{rk}{Remark}
\makeautorefname{remarks}{Remarks}
\makeautorefname{rems}{Remarks}
\makeautorefname{rmks}{Remarks}
\makeautorefname{rks}{Remarks}
\makeautorefname{example}{Example}
\makeautorefname{examp}{Example}
\makeautorefname{exmp}{Example}
\makeautorefname{exam}{Example}
\makeautorefname{exa}{Example}
\makeautorefname{algorithm}{Algorith}
\makeautorefname{algo}{Algorith}
\makeautorefname{alg}{Algorith}
\makeautorefname{axiom}{Axiom}
\makeautorefname{axi}{Axiom}
\makeautorefname{ax}{Axiom}
\makeautorefname{case}{Case}
\makeautorefname{claim}{Claim}
\makeautorefname{clm}{Claim}
\makeautorefname{assumption}{Assumption}
\makeautorefname{assumpt}{Assumption}
\makeautorefname{conclusion}{Conclusion}
\makeautorefname{concl}{Conclusion}
\makeautorefname{conc}{Conclusion}
\makeautorefname{condition}{Condition}
\makeautorefname{condit}{Condition}
\makeautorefname{cond}{Condition}
\makeautorefname{construction}{Construction}
\makeautorefname{construct}{Construction}
\makeautorefname{const}{Construction}
\makeautorefname{cons}{Construction}
\makeautorefname{criterion}{Criterion}
\makeautorefname{criter}{Criterion}
\makeautorefname{crit}{Criterion}
\makeautorefname{exercise}{Exercise}
\makeautorefname{exer}{Exercise}
\makeautorefname{exe}{Exercise}
\makeautorefname{problem}{Problem}
\makeautorefname{problm}{Problem}
\makeautorefname{probm}{Problem}
\makeautorefname{prob}{Problem}
\makeautorefname{solution}{Solution}
\makeautorefname{soln}{Solution}
\makeautorefname{sol}{Solution}
\makeautorefname{summary}{Summary}
\makeautorefname{summ}{Summary}
\makeautorefname{sum}{Summary}
\makeautorefname{operation}{Operation}
\makeautorefname{oper}{Operation}
\makeautorefname{observation}{Observation}
\makeautorefname{observn}{Observation}
\makeautorefname{obser}{Observation}
\makeautorefname{obs}{Observation}
\makeautorefname{ob}{Observation}
\makeautorefname{convention}{Convention}
\makeautorefname{convent}{Convention}
\makeautorefname{conv}{Convention}
\makeautorefname{cvn}{Convention}
\makeautorefname{warning}{Warning}
\makeautorefname{warn}{Warning}
\makeautorefname{note}{Note}
\makeautorefname{fact}{Fact}
%
\begin{document}
%\begin{titlepage}
\title{Well-Posedness for the Cubic Nonlinear Schr\"{o}dinger Equation }
\author{David Karapetyan}
\address{Department of Mathematics  \\
         University  of Notre Dame\\
		          Notre Dame, IN 46556 }
				  \date{02/28/10}
				  %
				  \maketitle
				  %
				  %
				  \parindent0in
				  \parskip0.1in
				  %
				  %\end{titlepage}
				  %
				  %
				  %
				  \section{Introduction}
				  \setcounter{equation}{0}
				  We consider the cubic nonlinear Schr\"{o}dinger (NLS) 
				  initial value problem (ivp)
%
%
\begin{gather}
	\label{NLS-eq}
		i \p_t u = - \p_x^2 u - |u|^{2} u,
		\\
		\label{NLS-init-data}
		u(x,0) = \vp(x), \ \ x \in \ci, \ \ t \in \rr.
\end{gather}
%
%
and prove the following result.
%
%
%
%
%%%%%%%%%%%%%%%%%%%%%%%%%%%%%%%%%%%%%%%%%%%%%%%%%%%%%
%
%
%	Main Result				
%
%
%%%%%%%%%%%%%%%%%%%%%%%%%%%%%%%%%%%%%%%%%%%%%%%%%%%%%
%
%
\begin{theorem}
	\label{thm:main}
	For any $s \ge 0$, the initial value problem 
	\eqref{NLS-eq}-\eqref{NLS-init-data} is locally well-posed for 
	initial data $\vp(x) \in H^s(\ci)$.
%
%
\end{theorem} 
%
%
%
%
%%%%%%%%%%%%%%%%%%%%%%%%%%%%%%%%%%%%%%%%%%%%%%%%%%%%%
%
%
%				Outline
%
%
%%%%%%%%%%%%%%%%%%%%%%%%%%%%%%%%%%%%%%%%%%%%%%%%%%%%%
%
%
\section{Outline of the Proof of \autoref{thm:main}}
\setcounter{equation}{1}
%
%
%
%
%
We first derive a weak formulation of the NLS ivp. Without loss of 
generality we let $\ci = [-2B, 2B]$, and use
the following notation for the Fourier transform
%
%
%
%
\begin{equation*}
	\begin{split}
		\widehat{f}(n) = \int_{\ci} e^{-ix n} f(x) \ dx.
	\end{split}
\end{equation*}
Assume 
$u(x,t)$ is a classical solution of \eqref{NLS-eq}-\eqref{NLS-init-data}.
Then $u(\cdot ,   t): \rr \to C^2(\ci)$ and is $C^1$ in time; hence, applying 
the Fourier transform to the NLS ivp in the space variable we obtain 
%
%
\begin{gather*}
	\p_t \widehat{u}(n, t) = -i n^2 \widehat{u}(n, t) + i  
	\widehat{|u|^{2} u} (n, t),
	\\
	\widehat{u} (n,0) = \widehat{\vp}(n)
\end{gather*}
%
%
which is a globally well-defined relation in $t$ 
and $n$. Multiplication by the integrating factor $e^{itn^2}$ then yields
%%
%%
\begin{equation*}
	\begin{split}
		\left[ e^{it n^2} \widehat{u}(n) \right]_t = i
		 e^{it n^2} \widehat{|u|^{2} u} (n, t).	
	\end{split}
\end{equation*}
%
%
Integrating from $0$ to $t$, we obtain
%
%
\begin{equation*}
	\begin{split}
		\wh{u}(n, t) = \wh{\vp}(n) e^{-it n^2} + i  
		\int_0^t e^{i(t' - t)n^2} \wh{|u|^{2} u}(n, t') \ 
		dt'
	\end{split}
\end{equation*}
%
%
which by Fourier inversion yields 
%
%
\begin{equation}
	\label{NLS-integral-form}
	\begin{split}
		u(x,t) & = \sum_{n \in \zz} \wh{\vp}(n) e^{i\left( xn - t n^2 
		\right)} 
		 + i \sum_{n \in \zz} \int_0^t e^{i\left[ xn + \left( t' - t 
		\right)n^2 \right]} \wh{|u|^2 u}(n, t') \ dt'.
	\end{split}
\end{equation}
%
%
Here we remark that \eqref{NLS-integral-form} is a weaker 
restatement of the Cauchy-problem \eqref{NLS-eq}-\eqref{NLS-init-data}, 
since by construction any classical solution of the NLS ivp is a solution to 
\eqref{NLS-integral-form}. This motivates the following.



\begin{definition}
	%
	%
	We say $u(x,t)$ is a \emph{local solution} of the NLS ivp
\eqref{NLS-eq}-\eqref{NLS-init-data} if it satisfies the weak formulation 
\eqref{NLS-integral-form} for $t \in I \subset \rr$. If $I = \rr$, then we say 
$u$ is a \emph{global solution}. 
%
%
\end{definition}
%
%
Restrict $t \in [0, 2\delta]$, where $\delta = \pi/m_0$ for some $m_0 \in 
\mathbb{N}$. We now derive an integral 
equation global in $t$ and equivalent to \eqref{NLS-integral-form} for $t 
\in [0, \delta]$. Let $\psi_1(t)$ be a cutoff function symmetric about the 
origin such that $\psi_1(t) = 1$ for $|t| \le \delta$ and $\text{supp} \, \psi_1 
= 
[-2\delta, 2\delta ]$. Multiplying the right hand side of expression 
$\eqref{NLS-integral-form}$ by $\psi_1(t)$, we obtain
%
%
\begin{equation}
	\begin{split}
		\label{cutoff-int-eq}
		u(x, t)
		& = \frac{1}{2 \pi} \psi_1(t) \sum_{n \in \zz} e^{i(xn - tn^{2
		})} \widehat{\vp}(n) 
		\\
		& + \frac{i }{2 \pi} \psi_1(t) \int_0^t \sum_{n \in \zz} 
		e^{i\left[ xn + (t - t')n^2 \right]} \wh{w}(n, t') \ dt'
	\end{split}
\end{equation}
%
%
where $w(x, t) \doteq |u|^2 u$. Periodically extend $u(x, t)$ in the time 
variable, and assume a priori that $u \in 
L^4(\ci^2)$. 
\\
\hrule
Claim: Therefore, by Fourier inversion (makes sense if we view $w$ as a 
distribution, which we are permitted to do since $w \in L^{4/3}$) and 
Fubini-Tonelli (does not make sense to me--what guarantee 
do we have that $\wh{w} \in \ell^1$? \textbf{these questions have been resolved 
in mNLS document--in process of copying them over})
\\
\hrule
%
\begin{equation}
	\label{prim-int-form}
	\begin{split}
		& \frac{i }{2 \pi} \psi_1(t) \int_0^t \sum_{n \in \zz} 
		e^{i\left[ xn + (t - t')n^2 \right]} \wh{w}(n, t') \ dt'
		\\
		& = \frac{i }{4 \pi^2} \psi_1(t)
		\int_0^t 
		e^{it'(\tau - n^2)} \ dt' 
		\sum_{n, \tau \in \zz} e^{i(xn + tn^2)} \wh{w}(n, \tau) .
	\end{split}
\end{equation}
%%
%%
Since
%
%
\begin{equation*}
	\begin{split}
		\int_0^t e^{it'(\tau - n^2)} \ dt' = -i \frac{e^{it(\tau - n^{2  
		})}- 1}{\tau - n^2}
	\end{split}
\end{equation*}
%%
%%
expression \eqref{prim-int-form} gives
%
%
%
\begin{equation*}
	\begin{split}
		& \frac{i }{2 \pi} \psi_1(t) \int_0^t \sum_{n \in \zz} 
		e^{i\left[ xn + (t - t')n^2 \right]} \wh{w}(n, t') \ dt'
		\\
		& = \frac{1}{4 \pi^2} \psi_1(t) \sum_{n, \tau \in \zz} 
		e^{i(xn + tn^2)} \frac{e^{it(\tau - n^2)}- 1}{\tau - n^2} 
		\wh{w}(n, \tau)  
		\end{split}
\end{equation*}
%
%
which we substitute
into \eqref{cutoff-int-eq} to obtain
%
%
\begin{equation}
	\begin{split}
		\label{cutoff-int-eq-2}
		u(x, t)
		& = \frac{1}{2 \pi} \psi_1(t) \sum_{n \in \zz} e^{i(xn - tn^{2  
		})} \widehat{\vp}(n) 
		\\
		& + \frac{1}{4 \pi^2} \psi_1(t) \sum_{n, \tau \in \zz} 
		e^{i(xn + tn^2)} \frac{e^{it(\tau - n^2)}- 1}{\tau - n^2} \wh{w}(n, 
		\tau). 
	\end{split}
\end{equation}
%
%
Next, we localize near the singular curve $\tau = n^2$. Let $\psi_2(t)$ be a cutoff function symmetric about the origin such that 
$\psi_2(t) = 1$ for $|t| \le B$ and $\text{supp} \, \psi_2 = 
[-2B, 2 B]$.
Multiplying the summand of the second term of \eqref{cutoff-int-eq-2} by $1 \pm \psi_2(\tau - n^2)$ and rearranging 
terms, we have
%
%
\begin{equation*}
	\begin{split}
		 u(x, t)
		& = \frac{1}{2 \pi} \psi_1(t) \sum_{n \in \zz} e^{i(xn + tn^{2 
		})} \widehat{\vp}(n) 
		\\
		& + \frac{1}{4 \pi^2} \psi_1(t) \sum_{n \in \zz} e^{ixn} \sum_{\tau \in \zz} 
		e^{it \tau} \frac{ 1 - \psi_2(\tau - n^2) 
		}{\tau - n^2} \wh{w}(n, \tau) 
		\\
		& - \frac{1}{4 \pi^2} \psi_1(t) \sum_{n \in \zz} e^{i(xn + 
		tn^2)}
		\sum_{\tau \in \zz} \frac{1- \psi_2(\tau - n^2)}{\tau - n^2} \wh{w}(n, \tau) \ 
		d\tau
		\\
		& + \frac{1}{4 \pi^2} \psi_1(t) \sum_{n \in \zz}
		e^{i(xn + tn^2)}
		\sum_{\tau \in \zz} 
		\frac{\psi_2(\tau - n^2)\left[ e^{it(\tau - n^2)}-1 
		\right]}{\tau - n^2} \wh{w}(n, \tau) 
	\end{split}
\end{equation*}
%
%
which by a power series expansion of $[e^{it(\tau - n^2)}-1]$ simplifies  to
%
%
\begin{align}
	\label{main-int-expression-1}
	& u(x, t) \notag
		\\
		& = \frac{1}{2 \pi} \psi_1(t) \sum_{n \in \zz} e^{i(xn + tn^{2 
		j})} \widehat{\vp}(n) 
		\\
		\label{main-int-expression-2}
		& + \frac{1}{4 \pi^2} \psi_1(t) \sum_{n \in \zz} e^{ixn} \sum_{\tau \in \zz} 
		e^{it \tau} \frac{ 1 - \psi_2(\tau - n^2) 
		}{\tau - n^2} \wh{w}(n, \tau) 
		\\
		\label{main-int-expression-3}
		& - \frac{1}{4 \pi^2} \psi_1(t) \sum_{n \in \zz} e^{i(xn + 
		tn^2)}
		\sum_{\tau \in \zz} \frac{1- \psi_2(\tau - n^2)}{\tau - n^2} \wh{w}(n, \tau) 
		\\
		\label{main-int-expression-4}
		& + \frac{1}{8 B \pi^2} \psi_1(t) \sum_{k \ge 1} \frac{i^k (2Bt)^k}{k!}
		\sum_{n \in \zz} e^{i(xn + tn^2 )}
		\sum_{\tau \in \zz}	\psi_2 (\tau - n^2) \left(\frac{\tau - 
		n^2}{2B} \right)^{k -1} \wh{w}(n, \tau) 
		\\
		& \doteq T(u). \notag
\end{align}
where $T = T(\vp)$. Note that 
\eqref{main-int-expression-1}-\eqref{main-int-expression-4} is a global 
relation in $t$; furthermore, the fixed point solution $Tu=u$ gives rise to a 
local solution of the NLS ivp by simply restricting the time 
variable to 
the $[0, \delta]$ interval. Hence, we focus our attention on 
\eqref{main-int-expression-1}-\eqref{main-int-expression-4}. In 
Section 3 we will show that for initial data $\vp \in H^s(\ci)$, $T$ is 
a contraction on $B_M 
\subset L^4(\ci^2)$, where $B_M$ is the ball centered at 
the origin of radius $M = M(\vp) > 0$, by estimating the $L^4(\ci^2)$
norm of 
\eqref{main-int-expression-1}-\eqref{main-int-expression-4}. The Picard 
fixed point theorem and time restriction will
then yield a unique local solution to the NLS ivp in the time interval
$I = [0, \delta]$. Continuous 
dependence will follow from estimates used to establish the contraction 
mapping.  In Section 4 a similar argument for handling well-posedness for 
\eqref{NLS-integral-form} in $H^s(\ci)$, $s > 0$ will be discussed.
%In the 
%appendix we establish necessary conditions under which a local solution to 
%\eqref{NLS-integral-form} can be viewed as a distributional solution of 
%\eqref{NLS-eq}-\eqref{NLS-init-data}.
%
%
%
%
%%%%%%%%%%%%%%%%%%%%%%%%%%%%%%%%%%%%%%%%%%%%%%%%%%%%%
%
%
%	Case \ell^2			
%
%
%%%%%%%%%%%%%%%%%%%%%%%%%%%%%%%%%%%%%%%%%%%%%%%%%%%%%
%
%
\section{The case $\vp \in L^2(\ci)$}
\label{sec:s=0}
%
%
%
%
%%%%%%%%%%%%%%%%%%%%%%%%%%%%%%%%%%%%%%%%%%%%%%%%%%%%%
%
%
%		Estimation of Integral Equality Part 1		
%
%
%%%%%%%%%%%%%%%%%%%%%%%%%%%%%%%%%%%%%%%%%%%%%%%%%%%%%
%
%
\vskip0.1in
{\bf Estimate for \eqref{main-int-expression-1}.}
Letting $f(x,t) = \psi(t) \sum_{n \in \zz} e^{i(xn + tn^2)} 
\wh{\vp}(n)$, we have $\wh{f}(n,t) = \psi(t) \wh{\vp}(n) e^{itn^2}$,
from which we obtain
%
%
\begin{equation}
	\label{fourier-trans-calc}
	\begin{split}
		\wh{f}(n, \tau)
		& = \wh{\vp}(n) \int_\rr e^{-it( \tau - n^2}) 
		\wh{\psi_1}(\tau) \ d\tau
		= \wh{\psi_1}(\tau - n^2) \wh{\vp}(n).
	\end{split}
\end{equation}
%
%
We now require the following multiplier estimate, whose proof can be found in 
\cite{Bourgain-Fourier-transfo-1}. %
%
%%%%%%%%%%%%%%%%%%%%%%%%%%%%%%%%%%%%%%%%%%%%%%%%%%%%%
%
%
%			Fourier Multiplier Estimate	
%
%
%%%%%%%%%%%%%%%%%%%%%%%%%%%%%%%%%%%%%%%%%%%%%%%%%%%%%
%
%
\begin{lemma}[Bourgain]
	\label{lem:four-mult-est}
	Let $(x, t) \in \ci^2$ and let $(n, \tau) \in \zz^2$ 
	be the dual variables. Then for $h \in L^2(\ci^2)$,
%
\begin{equation}
	\label{four-mult-est}
	\begin{split}
		\|h\|_{L^4(\ci^2)} \le c \|(1 + |\tau - 
		n^2|)^{3/8} \, \wh{h}(n, \tau) \|_{\ell^2(\zz^2)}.
	\end{split}
\end{equation}
%
%
\end{lemma}
%
%
%
%
%
%
%
Applying \autoref{lem:four-mult-est} and \eqref{schwartz-bound} to
\eqref{fourier-trans-calc}, we obtain
%
%
\begin{equation}
	\label{main-int-est-part-1}
	\begin{split}
		\|\eqref{main-int-expression-1}\|_{L^4(\ci^2)} 
		& \le c  \|(1 + |\tau - 
		n^2|)^{3/8} \wh{\psi}(\tau - n^2) 		\wh{\vp}(n) \|_{\ell^2(\zz^2)}
		\\
		& \le c \|(1 + |\xi|)  \wh{\psi_1}(\xi)\|_{L^\infty(\zz)} 
		\|\wh{\vp} \|_{\ell^2(\zz)}
		\\
		& \lesssim \|\vp \|_{L^2(\ci)}.
	\end{split}
\end{equation}
where the last step follows from Plancharel and the estimate
%
%
\begin{equation}
	\label{schwartz-bound}
	\begin{split}
		\|\left( 1 + |\lambda| \right)^k \wh{\psi_1}(\lambda) 
		\|_{L^\infty(\zz)} \le c_k , \qquad k\ge0
	\end{split}
\end{equation}
%
%
whose proof is provided in the appendix.
%
%
\vskip0.1in
%
%
%
%
%
%
\vskip0.1in
{\bf Estimate for \eqref{main-int-expression-2}.}
We have
%
%
\begin{equation}
	\label{1ag}
	\begin{split}
		\|\eqref{main-int-expression-2}\|_{L^4(\ci^2)} 
		& \lesssim \| \sum_{n \in \zz} e^{ixn} \sum_{\tau \in
		\zz} e^{it \tau} \frac{1- \psi_2\left( \tau - n^2 \right)}{\tau - 
		n^2} \wh{w}\left( n, \tau \right) \|_{L^4(\ci^2)}
		\\
		& \lesssim  \|\left( 1 + |\tau - n^2| \right)^{3/8}
		\frac{1- \psi_2\left( \tau - n^2 \right)}{\tau - 
		n^2} \wh{w}(n, \tau) \|_{\ell^2(\zz^2)}
	\end{split}
\end{equation}
%
%
where the last step follows from \autoref{lem:four-mult-est}. Applying the 
estimate
%
%
\begin{equation}
	\label{2ag}
	\begin{split}
		\left[ 1 - \psi_2\left( \tau - n^2 \right) \right]
		\left( \tau - n^2 \right)^{-1}
		& \le \left( 1 + 1 /B \right)\left( 1 + |\tau - n^2| 
		\right)^{-1}
		\\
		& = \left( 1 + 1/B \right)\left( 1 + B \right)^{-k }
		\left( 1 + B \right)^k \left( 1 + |\tau - n^2| \right)^{-1} 
				\\
		& \le \left( 1 + 1/B \right)\left( 1 + B \right)^{-k 
		} \left( 1 + |\tau - n^2| \right)^{k-1}
		\\
		& = \frac{\left( 1 + B \right)^{1-k }}{B} \left( 1 + |\tau - 
		n^2|
		\right)^{k - 1}
		\\
		& \le 2^{k -1} B^{-k } \left( 1 + |\tau - n^2| \right)^{k - 1}, 
		\qquad k \le 1
	\end{split}
\end{equation}
%
%
with $k = 1/4$ to \eqref{1ag}, we obtain 
%
%
\begin{equation}
	\label{3ag}
	\begin{split}
		\|\eqref{main-int-expression-2}\|_{L^4(\ci^2)}
		& \lesssim  
		B^{-1/4} \|\left( 1 + |\tau - n^2| \right)^{-3/8} \wh{w} (n, 
		\tau) \|_{\ell^2(\zz^2)}.
	\end{split}
\end{equation}
%
%
We now need the following dual estimate of 
\autoref{lem:four-mult-est}, whose proof is provided in the appendix:
%
%
\begin{corollary}
	\label{cor:four-mult-est-dual}
Let $(n, \tau) \in \zz^2$ and let  $(x, t) \in \ci^2$
be the dual variables. Then for $h \in L^2(\ci^2)$
%
%
\begin{equation}
	\label{four-mult-est-dual}
	\|(1 + |\tau - 
	n^2|)^{-3/8} \wh{h}(n, \tau) \|_{\ell^2(\zz^2)} \le 
	c \|h\|_{L^{4/3}(\ci^2)}.
\end{equation}
%
%
\end{corollary}
%%
%%
Applying \autoref{cor:four-mult-est-dual} to \eqref{3ag}, we conclude that 
%
%
\begin{equation}
	\label{main-int-est-2}
	\begin{split}
		\|\eqref{main-int-expression-2}\|_{L^4(\ci^2)} 
		& \le c B^{-1/4} \|w\|_{L^{4/3}(\ci^2)}
		\\
		& = c B^{-1/4} \|u^3\|_{L^{4/3}(\ci^2)}
		\\
		& = c B^{-1/4} \|u\|^3_{L^4(\ci^2)}.
	\end{split}
\end{equation}
%%
%%

\vskip0.1in
{\bf Estimate for \eqref{main-int-expression-3}.} 
Letting $$f(x,t) = \psi(t) \sum_{n \in \zz} e^{i\left( xn + tn^2 \right)} 
\sum_{\lambda \in \zz} \frac{1 - \psi\left( \tau - n^2 \right)}{\tau - n^2} 
\wh{w} \left( n, \tau \right),$$ we have
%
%
\begin{equation*}
	\begin{split}
		& \wh{f^x}(n, t) = \psi(t) e^{itn^2} \sum_{\lambda \in \zz} 
		\frac{1 - \psi\left( \lambda - n^2 \right)}{\lambda - n^2} 
		\wh{w}(n, \lambda)
	\end{split}
\end{equation*}
and
\begin{equation*}
	\begin{split}
		 \wh{f}\left( n, \tau \right)
		 & = \int_\ci e^{-it\left( \tau - n^2 
		\right)} \psi(t) \sum_{\lambda \in \zz} \frac{1 - \psi\left( 
		\lambda - n^2 
		\right)}{\lambda - n^2} \wh{w}(n, \lambda)
		\\
		& = \wh{\psi_1}\left( \tau - n^2 \right) \sum_{\lambda \in \zz}
		\frac{1 - \psi\left( 
		\lambda - n^2 
		\right)}{\lambda - n^2} \wh{w}(n, \lambda).
	\end{split}
\end{equation*}
%
%
Therefore, applying \autoref{lem:four-mult-est} and 
\eqref{schwartz-bound} gives 
%
%
\begin{equation}
	\label{4hh}
	\begin{split}
		\|f\|_{L^4(\ci^2)}
		& \lesssim \|\left( 1 + | \tau - n^2| \right)^{3/8} 
		\wh{\psi_1}\left( \tau - n^2 \right) \sum_{\lambda \in \zz} 
		\frac{1 - \psi_2\left( \lambda - n^2 \right)}{\lambda - n^2} 
		\wh{w}\left( n, \lambda \right) 
		\|_{\ell^2\left( \zz^2 \right)}
		\\
		& \lesssim \|\sum_{\lambda \in \zz}
		\frac{1 - \psi_2\left( \lambda - n^2 \right)}{\lambda - n^2} \wh{w} 
		\left( n, \lambda \right) \|_{\ell^2\left( \zz \right)}.
	\end{split}
\end{equation}
%
%
Next, note 
that 
%
%
\begin{equation}
	\label{apply-ortho}
	\begin{split}
		\|\sum_{\lambda \in \zz}
		\frac{1 - \psi_2\left( \lambda - n^2 \right)}{\lambda - n^2} \wh{w} 
		\left( n, \lambda \right) \|_{\ell^2\left( \zz \right)}
		& = \left( \sum_{n \in \zz}  | \sum_{\lambda \in \zz}
		\frac{1 - \psi_2\left( \lambda - n^2 \right)}{\lambda - n^2} \wh{w} 
		\left( n, \lambda \right) |^2 \right )^{1/2}
		\\
		& \lesssim \left( \sum_{n \in \zz} \sum_{\lambda \in \zz} |
		\frac{1 - \psi_2\left( \lambda - n^2 \right)}{\lambda - n^2} 
		\wh{w}\left( n, \lambda \right) |^2 
		\right)^{1/2} 
	\end{split}
\end{equation}
%
%
\hrule
Claim [Bourgain]: where the last step follows from orthogonality. 
\\
Preliminary Explanation: I understand now how Bourgain does this; the work is in 
my mNLS document. I am in the process of copying it over to this document.
\\
\hrule
%
%
But by \eqref{2ag} and \autoref{cor:four-mult-est-dual}, we have
\begin{equation*}
	\begin{split}
		\left( \sum_{n, \tau \in \zz} |
		\frac{1 - \psi_2\left( \lambda - n^2 \right)}{\lambda - n^2} 
		\wh{w}\left( n, \lambda \right) |^2 
		\right)^{1/2}
		& = \|\frac{1 - \psi_2 \left( \lambda - n^2 
		\right)}{\lambda - n^2} \wh{w}\left( n, \lambda \right) 
		\|_{\ell^2(\zz^2)}
		\\
		& \lesssim B^{-5/8} \|\left( 1 + |\tau - n^2| \right)^{-3/8} 
		\wh{w}\left( n, \lambda \right) \|_{\ell^2(\zz^2)}
		\\
		& \lesssim B^{-5/8} \|w\|_{L^{4/3}(\ci^2)}
		\\
		& = B^{-5/8} \|u\|_{L^4(\ci^2)}^3.
	\end{split}
\end{equation*}
%
%
Substituting back into \eqref{4hh}, and letting $c$ absorb all superficial 
constants, we obtain
%
%
\begin{equation}
	\label{main-int-3-est}
	\begin{split}
		\|f\|_{L^4(\ci^2)} \le c \delta B^{-5/8} 
		\|u\|_{L^4(\ci^2)}^3.
	\end{split}
\end{equation}
%
%
{\bf Estimate for \eqref{main-int-expression-4}.}
Noting that
%
%
\begin{equation*}
	\begin{split}
		|\sum_{k \ge 1} \frac{i^k (2Bt)^k}{k!}| = |e^{2iBt} - 1| \le 2,
	\end{split}
\end{equation*}
%
%
we obtain
%
%
\begin{equation}
	\label{10aa}
	\begin{split}
		\|\eqref{main-int-expression-4}\|_{L^4(\ci^2)} 
		\le 2 \sup_k \|f \|_{L^4(\ci^2)}
	\end{split}
\end{equation}
%
%
where $$f(x,t) = \psi_1(t) \sum_{n \in \zz} e^{i\left( xn + tn^2 \right)} 
		\sum_{\tau \in \zz} \psi_2\left( \tau - n^2 
		\right)\left (\frac{ \tau - n^2}{2B}\right)^{k - 1} \wh{w}\left( n, \tau 
		\right).$$
%
%
Calculating the Fourier transform of $f(x, t)$, we have
\begin{equation*}
	\begin{split}
		& \wh{f^x}(n, t) = \psi_1(t) e^{itn^2} \sum_{\tau \in \zz} 
		\psi_2\left (\frac{ \tau - n^2}{2B}\right)^{k - 1}
		\wh{w}\left( n, \tau \right)
	\end{split}
\end{equation*}
which gives
\begin{equation*}
	\begin{split}
		\wh{f}(n, \lambda)
		& = \int_\ci e^{-it\left( \lambda - n^2 \right)} 
		\psi_1(t) \ dt \sum_{\tau \in \zz} \psi_2
		\left (\frac{ \tau - n^2}{2B}\right)^{k - 1}
\wh{w}\left( n, \tau 
		\right)
		\\
		&  = \wh{\psi_1}\left( \lambda - n^2 \right) \sum_{\tau \in \zz} 
		\psi_2\left( \tau - n^2 \right)
		\left (\frac{ \tau - n^2}{2B}\right)^{k - 1}
		\wh{w} \left(n, \tau \right).
	\end{split}
\end{equation*}
%
%
Therefore, applying \autoref{lem:four-mult-est} and 
\eqref{schwartz-bound} to \eqref{10aa} yields
%
%
\begin{equation}
	\label{main-int-4-est-prelim}
	\begin{split}
		& \|f\|_{L^4(\ci^2)} 
		\\
		& \le c \sup_{k \ge 1} \|\left( 1 + |\tau - n^2| 
		\right)^{3/8} \wh{\psi_1}\left( \tau - n^2 \right) \sum_{\lambda 
		\in \zz} \psi_2\left( \lambda - n^2 \right)\left( \frac{\lambda - 
		n^2}{2B} 
		\right)^{k - 1} \wh{w}\left( n, \lambda \right) \|_{\ell^2(\zz^2)}
		\\
		& \le c \delta \| \sum_{\lambda \in \zz} 
		\psi_2\left( \lambda - n^2 \right) \wh{w}\left( n, \lambda \right)\|_{\ell^2(\zz)}
		\\
		& = c \delta \left( \sum_{n \in \zz} | 
		\sum_{\lambda \in \zz} \psi_2\left( \lambda - n^2 \right) 
		\wh{w}\left( n, \lambda \right)|^2\right)^{1/2}
		\\
		& \lesssim \delta \left( \sum_{n \in \zz} \wh{w}(n, 
		n^2) \right)^{1/2}
	\end{split}
\end{equation}
%
%
\hrule
Claim [Bourgain]: where the last step follows by the definition of 
$\psi_2$. 
\\
Preliminary Explanation: Again, resolved in the mNLS document. It follows from a 
duality argument, coupled with the first trilinear estimate proved there. \\
\hrule
Applying \autoref{cor:four-mult-est-dual}, and letting $c$ absorb all 
superficial constants, we conclude
that 
%
%
\begin{equation}
	\label{main-int-4-est}
	\begin{split}
		\|f\|_{L^4(\ci^2)}
		& \le c \delta B \|w\|_{L^{4/3}(\ci^2)}
		\\
		& \le c \delta B \|u^3\|_{L^{4/3}(\ci^2)}
		\\
		& \le c \delta B |u\|_{L^4(\ci^2)}^3.
	\end{split}
\end{equation}
%
%
Collecting estimates \eqref{main-int-est-part-1},\eqref{main-int-est-2}, 
\eqref{main-int-3-est}, and \eqref{main-int-4-est}, we obtain
%
%
\begin{equation}
	\label{1gh}
	\begin{split}
		\|Tu\|_{L^4(\ci^2)} \le c\left( \|\vp\|_{L^2(\ci)}
		+ B^{-1/4} \|u\|_{L^4(\ci^2)}^3 + \delta B \|u\|_{L^4(\ci^2)}^3 \right).
	\end{split}
\end{equation}
%
%
Let $B_M = \left\{ u \in L^4(\ci^2): \|u\|_{L^4(\ci^2)} \le 2c \|\vp 
\|_{L^2(\ci)} \right\}$. Setting $B = \delta^{-1/2}$ and restricting $u \in B_M$, we 
obtain %
%
\begin{equation*}
	\begin{split}
		\|Tu\|_{L^4(\ci^2)}
		& \le c \left[ \|\vp\|_{L^2(\ci)} + \left( \delta^{1/8} 
		+ \delta^{1/2} \right) M^3 \right]
		\\
		& \le c \left[ \|\vp\|_{L^2(\ci)} + 2 \delta^{1/2} M^3 \right]
		\\
		& \le c \left[ \|\vp\|_{L^2(\ci)} + 16c^3 \delta^{1/2} \|\vp\|_{L^2(\ci)}^3 
		\right].
	\end{split}
\end{equation*}
 %
%
Recalling that  $\delta = \pi/n, \ n \in \mathbb{N}$ and choosing $n$ large enough 
such that $\delta < 1/(10^6 c^6 \|\vp\|_{L^2(\ci)}^4)$ then gives
%
%
\begin{equation}
	\label{ball-to-ball}
	\begin{split}
		\|Tu\|_{L^4(\ci^2)} < 2c\|\vp\|_{L^2(\ci)} = M, \qquad u \in B_M.
	\end{split}
\end{equation}
%
%
Hence, $T: B_M \to B_M$. Furthermore, from our definition of $T$ in 
\eqref{main-int-expression-1}-\eqref{main-int-expression-4}, we obtain
%
%
\begin{equation*}
	\begin{split}
		Tu - Tv = \eqref{main-int-expression-2} + 
		\eqref{main-int-expression-3} + \eqref{main-int-expression-4}
	\end{split}
\end{equation*}
%
%
where now $w = u|u|^2 - v|v|^2$. By the triangle inequality and generalized
H\"{o}lder
%
%
\begin{equation}
	\label{gen-holder}
	\begin{split}
		\|u |u|^2 - v |v|^2\|_{4/3}
		& = \| |u|^2\left( u -v \right) + v\left( |u|^2 - |v|^2 
		\right)\|_{4/3}
		\\
		& \le \|u^2\left( u -v \right)\|_{4/3} + \|v\left( |u| + |v| 
		\right)\left( |u| - |v| \right) \|_{4/3}
		\\
		& \le \|u\|_4^2 \|u -v \|_4 + \|v\|_4  \| |u| + |v| \|_4 
		\| |u| - |v |\|_4 
		\\
		& \le \|u \|_4^2 \|u -v\|_4 + \|v\|_4\left( \|u\|_4 + \|v\|_4 
		\right) \|u -v \|_4
		\\
		& \le 2 \|u -v \|_4 \left(  \|u\|_4 + \|v\|_4 \right)^2.
	\end{split}
\end{equation}
%
%
Substituting $w = u|u|^2 - v|v|^2$ into the first line of \eqref{main-int-est-2}, 
\eqref{main-int-3-est}, and \eqref{main-int-4-est}, and estimating using  
\eqref{gen-holder}, we conclude that
%
%
\begin{equation*}
	\begin{split}
		\|Tu - Tv\|_{L^4(\ci^2)}
		& \le 2c\left( \delta B + B^{-1/4} 
		\right)\left( \|u\|_{L^4(\ci^2)} +
		\|v\|_{L^4(\ci^2)}\right)^2 \|u -v \|_{L^4(\ci^2)}
		\\
		& \le 8 c M^2(\delta B + B^{-1/4}) \|u - v\|_{L^4(\ci^2)}
		\\
		& \le 16cM^2 \delta^{1/2} \|u-v\|_{L^4(\ci^2)}
		\\
		& < 64 \cancel{c^3} \cancel{\|\vp\|_{L^2(\ci)}^2}
		\times \frac{1}{10^3 \cancel{c^3} \cancel{\|\vp\|_{L^2(\ci)}^2}}
		\|u-v\|_{L^4(\ci^2)}
	\end{split}
\end{equation*}
%
%
which yields the estimate
%
%
\begin{equation}
	\label{contract-est}
	\begin{split}
		\|Tu - Tv\|_{L^4(\ci^2)} < \frac{1}{2} \|u -v \|_{L^4(\ci^2)}.
	\end{split}
\end{equation}
%
%
By \eqref{ball-to-ball} and \eqref{contract-est}, we conclude that
$T = T(\vp)$ is a contraction on $B_M \subset L^4(\ci^2)$. 
%
%

%
%
\newpage



\begin{appendices}

\section{Proofs of Elementary Estimates}

{\bf Proof of \eqref{schwartz-bound}.} We first observe that
\begin{equation*}
	\begin{split}
		\wh{\psi_1}(0) = \int_\ci \psi_1 \ dt \le 2 \pi 
	\end{split}
\end{equation*}
%
%
while for $k \ge 0$, we use integration by parts to express the 
the Fourier transform of the periodic extension of $\psi_1$ 
in the following form
%
%
\begin{equation*}
	\begin{split}
		\wh{\psi_1}(\lambda) 
		& = \int_\ci e^{-i \lambda t} \psi_1 (t) \ dt
		= \frac{1}{\left( i \lambda \right)^k } \int_\ci e^{-i \lambda t} 
		\p_t^k \psi_1(t) \ dt
	\end{split}
\end{equation*}
%
%
which in conjunction with the estimate $\left( 1 + |\lambda|  \right)^{k } \le |2 \lambda|^k$ for $\lambda \in \mathbb{N}$ 
gives
%
%
\begin{equation*}
	\begin{split}
		\left( 1 + |\lambda| \right)^k |\wh{\psi_1}(\lambda)|
		& \le |2 \lambda|^k | \int_\ci e^{-i \lambda t} \psi_1(t) \ dt |
		\\
		& \le 2 ^k | \int_\ci e^{-i \lambda t} \p_t^k \psi_1(t) \ dt|
		\\
		& \le 2^k \int_\ci | \p_t^k \psi(t) | \ dt 
		\\
		& \le 2^k \cdot 2 \pi  \|\p_t^k \psi_1(t) \|_{L^\infty(\ci)}
		\\
		& = c_k.  \qquad \qed
	\end{split}
\end{equation*}
%
%
%
%
\vskip0.1in
{\bf Proof of \autoref{cor:four-mult-est-dual}.}
By duality,
%
%
\begin{equation}
	\label{1g}
	\begin{split}
		\|f\|_{L^{\frac{4}{3}}(\ci^2)}
		& = \sup_{g \in L^4(\ci^2)} 
		\frac{|\ \int_{\ci^2} f \bar{g} \ dx dt|}{\|g\|_{L^4(\ci^2)}}
		\\
		& \ge \frac{|\ \int_{\ci^2} f \bar{g} \ dx 
		dt|}{\|g\|_{L^4(\ci^2)}}, \qquad g \in L^4(\ci^2)
		\\
		&  =   \frac{|\sum_{m,n} \wh{f}(m,n) 
		\bar{\wh{g}}\left( m,n \right)| }{ 4 \pi^2 \|g\|_{L^4(\ci^2)}}
	\end{split}
\end{equation}
%
%
where the last step follows from Parseval's theorem. Choose $g$ such that
%
%
\begin{equation*}
	\begin{split}
		\wh{g}(m,n) = \left( 1 + |n - m^2| \right)^{-3/4} \wh{f}(m,n).
	\end{split}
\end{equation*}
%
%
Then by \autoref{lem:four-mult-est},
%
%
\begin{equation*}
	\begin{split}
		\|g\|_{L^4(\ci^2)}
		& \le \|\left( 1 + |n-m^2| \right)^{-3/8} 
		\wh{g}(m,n) \|_{\ell^2(\zz^2)}
		\\
		& = \|\left( 1 + |n-m^2| \right)^{-3/8} 
		\left( 1 + |n - m^2| \right)^{-3/4} \wh{f}(m,n)  \|_{\ell^2(\zz^2)}
		\\
		& \lesssim \|f\|_{\ell^2(\ci^2)} < \infty.
	\end{split}
\end{equation*}
%
%
Therefore, substituting into \eqref{1g}, we obtain
%
%
\begin{equation*}
	\begin{split}
		\|f\|_{L^{4/3}(\ci^2)}
		& \ge \frac{\sum_{m,n} |\wh{f}(m,n)|^2 \left( 
		1 + |n-m^2| \right)^{-3/4}}{4 \pi^2 \|g\|_{L^4(\ci^2)}}
		\\
		& \ge \frac{\sum_{m,n} |\wh{f}(m,n)|^2 \left( 
		1 + |n-m^2| \right)^{-3/4}}{4 \pi^2 c \|\left( 1 + |n - m^2| 
		\right)^{3/8} \wh{g}(n,m) \|_{\ell^2(\zz^2)}}
		\\
		& = \frac{ \sum_{m,n} |\wh{f}(m,n)|^2 \left( 1 + |n - m| 
		\right)^{-3/4}}{4 \pi^2 c \left( \sum_{n,m} |\wh{f}(n,m)|^2 \left( 
		1 + |n-m^2| \right)^{-3/4} \right)^{1/2}}
		\\
		& \simeq \|\left( 1 + |n-m^2| \right)^{-3/8} \wh{f}(m,n) \|_{\ell^2(\zz^2)}
	\end{split}
\end{equation*}
%
%
completing the proof. \qquad \qedsymbol
\vskip0.1in
%
\section{An Alternate Well-Posedness Theorem}
%
%
%%%%%%%%%%%%%%%%%%%%%%%%%%%%%%%%%%%%%%%%%%%%%%%%%%%%%
%
%
%			Alternate WP Theorem	
%
%
%%%%%%%%%%%%%%%%%%%%%%%%%%%%%%%%%%%%%%%%%%%%%%%%%%%%%
%
%
%
%
{\bf Remarks.} We have introduced the spaces $Y_s$ in part because well-posedness
in $H^s(\ci)$ for the mNLS becomes problematic as $s$ becomes small. 
On the other hand, for $s > 1/2$, well-posedness in $H^s(\ci)$ is a direct 
consequence of the algebra property of Sobolev spaces and the fact that the operator 
$e^{it \p_x^2}$ isometrically preserves Sobolev spaces. Stated more 
precisely, we have the following result:
%
%
\begin{proposition}
	Let $B_R \doteq \{f \in H^s : \|f\|_{H^s} < R \}$.
	Then the generalized mNLS ivp
\begin{gather}
	\label{general-mNLS-eq}
		i \p_t v = - \p_x^2 u - \lambda |u|^{\alpha -1} u, \ \ \alpha > 
		1, \lambda > 1
		\\
		\label{general-mNLS-init-data}
		u(x,0) = \vp(x), \ \ t \in \rr, \ \ x \in \ci \ \text{or} \ \rr
\end{gather}
	is locally well-posed in $H^s$ for $s > 1/2$ for 
	sufficiently small initial data $\vp \in B_R$, where the lifespan $T$ 
	satisfies 
%
%
\begin{equation*}
	\begin{split}
		T < 1/c
	\end{split}
\end{equation*}
%
%
for some constant $c = c(s, \lambda, \alpha, R, \vp)$.
\end{proposition}
%
%
{\bf Proof.} We will only provide a proof on the circle; the case on 
the line is nearly identical. The key ingredient
will be to establish that $L$ is a 
contraction on $C([-T, T], B_R)$. For the sake of clarity, we let $H^s_x 
= H^s_x(\ci)$. Let $e^{it \p_x^2}: \mathcal{E}'(\ci) \to 
\mathcal{E}'(\ci)$ be an operator defined by  
%
%
\begin{equation}
	\label{unit-op}
	\begin{split}
		e^{it \p_x^2} f(x) = \left[ e^{(-1)^j i t n^2} \wh{f}(n)
		\right]^{\vee} = 
		\sum_{n \in \zz} e^{i(nx + (-1)^j it n^2)} \wh{f}(n).
	\end{split}
\end{equation}
%
%
First, note that $e^{it \p_x^2}$ 
is unitary on $H^s(\ci)$; that is
%
%
\begin{equation}
	\label{unitary-op}
	\begin{split}
		\|e^{it \p_x^2} f \|_{H^s_x} & = \sum_{n \in \zz} |e^{(-1)^j it n^2} 
		\wh{f}(n)|^2 (1 + n^2)^s  
		\\
		& = \sum_{n \in \zz} |\wh{f}(n)|^2 (1 + n^2)^s 
		\\
		& = \|f\|_{H^s_x}.
	\end{split}
\end{equation}

Rewriting \eqref{general-mNLS-eq}-\eqref{general-mNLS-init-data} in its 
integral form
%
%
\begin{equation}
	\label{mNLS-int-form-with-op}
	\begin{split}
		u(x,t) = e^{it \p_x^2} \vp + i \lambda \int_0^t e^{i(t - 
		t')\p_x^2} |u|^{\alpha -1} u(x, t') \ dt' 
	\end{split}
\end{equation}
%
%
and applying the triangle inequality, Minkowski's inequality, and 
\eqref{unitary-op}, we obtain
%
%
\begin{equation}
	\label{bound-for-L}
	\begin{split}
		& \|Lu\|_{L^\infty_t[-T, T] H^s_x}
		\\
		& \le \|e^{t \p_x^2}
		\vp\|_{L^\infty_t[-T, T] H^s_x} + \|i \lambda \int_0^t e^{i(t - 
		t')\p_x^2} |u|^{\alpha -1} u(x, t') \ dt' \|_{L^\infty_t[-T, T] 
		H^s_x} 
		\\
		& \le \|\vp\|_{H^s_x} + |\lambda| \int_0^T \|e^{i(t 
		-t')\p_x^2} |u|^{\alpha -1} u \|_{L^\infty_t[-T, T] H^s_x} \ 
		dt'
		\\
		& = \|\vp\|_{H^s} + T |\lambda| \|u^\alpha \|_{L^\infty_t[-T, T] H^s_x}.
	\end{split}
\end{equation}
%
%
We now need the following lemma, whose proof can be found in Taylor 
\cite{Taylor_1991_Pseudodifferent}:
%
%
%
\begin{lemma}
	\label{lem:algebra-prop}
	The Sobolev space $H^s$ is an algebra for $s>1/2$. More precisely, 
%
%
\begin{equation}
	\label{algebra-prop}
	\begin{split}
		\|fg\|_{H^s} \le c_s \|f\|_{H^s} \|g\|_{H^s}.
	\end{split}
\end{equation}
%
%
%
\end{lemma}
%
%
Applying \autoref{lem:algebra-prop} to estimate \eqref{bound-for-L} gives
%
%
%
%
\begin{equation}
	\label{Tu-space-bound}
	\begin{split}
		\|Lu\|_{L^\infty_t[-T, T] H^s_x}
		& \le \|\vp\|_{H^s_x} + Tc_s | \lambda| \|u\|_{L^\infty_t[-T, T] 
		H^s_x}^\alpha
	\end{split}
\end{equation}
%
%
and since $u \in C([-T, T], B_R)$ a priori, it follows
that for sufficiently small $\vp$ and $T = T(s, \lambda, \alpha, R, \vp)$ we must 
have $Lu \in L^\infty([-T, T], B_R)$. To improve the regularity of 
$Lu$, let $\{t_n\} \subset [-T, T]$ and suppose that $t_n \to t \in [-T, 
T]$. Then
%
%
\begin{equation}
	\label{befo-dom}
	\begin{split}
		& \lim_{n \to \infty} \|Lu(\cdot, t) - Lu(\cdot, t_n)\|_{H^s_x} 
		\\
		& = \lim_{n \to \infty} \| \left \{ i \lambda \int_0^{t - t_n} e^{i(t  
		- t') \p_x^2} \left [|u|^{\alpha -1}u( \cdot, t') \right ]
		\ dt'\right \} \|_{H^s_x}
		\\
		& \le |\lambda|
		\lim_{n \to \infty}  \left \{  \int_0^{t - t_n} \| e^{i(t  
		- t') \p_x^2} \left [ |u|^{\alpha -1}u( \cdot, t') \right ]  
		\|_{H^s_x} \ dt' \right \}
		\\
		& = |\lambda|
		\lim_{n \to \infty}  \left[  \int_\rr \chi_{[0, t-t_n]}
		\| u^{\alpha}(\cdot, t') \|_{H^s_x} \ dt' \right ]
	\end{split}
\end{equation}
%
%
where the last step follows from \eqref{unitary-op}. 
By the algebra property and our a priori
assumption $u \in C([-T, T], B_R )$, we have 
%
%
\begin{equation*}
	\begin{split}
		\chi_{[0,t-t_n]}	\|u^\alpha(\cdot, t')\|_{H^s_x}
		\lesssim \chi_{[0,t-t_n]} 	\|u(\cdot, t')\|_{H^s_x}^\alpha 
		\le \chi_{[0,T]}\|u\|^\alpha_{L^\infty_t(\rr) H^s_x} \in 
		L^1_t(\rr).
		\end{split}
\end{equation*}
%
%
%
%
Hence, applying dominated 
convergence to \eqref{befo-dom}, we may pass the limit inside the integral,  
giving
%
\begin{equation*}
	\begin{split}
		\lim_{n \to \infty} \|Lu(\cdot, t)  - Lu(\cdot, t_n)\|_{H^s_x} 
		\le |\lambda|
		\int_\rr \lim_{n \to \infty} \left [ \chi_{[0, t-t_n]}
		\| u^{\alpha}(\cdot, t') \|_{H^s_x} \ dt' \right ]
		= 0
	\end{split}
\end{equation*}
%
%
which implies $Lu \in C([-T, T], B_R)$. Furthermore, for 
$u, v \in C([-T, T], B_R)$, we have
%
%
\begin{equation*}
	\begin{split}
		& \|Lu-Lv\|_{L^\infty_t[-T, T] H^s_x}
		\\
		& = \|i \lambda \int_0^t e^{i(t 
		-t')\p_x^2} (|u|^{\alpha - 1}u -|v|^{\alpha -1} v ) \ dt'
		\|_{L^\infty_t[-T, T] H^s_x}
		\\
		& \le |\lambda| \int_0^T \||u^{\alpha-1 }| u - | v^{\alpha - 1}| v
		\|_{L^\infty_t[-T, T] H^s_x} \ dt'
		\\
		& = |\lambda| T \cdot \|(u-v)(|u^{\alpha -1}| + |v^{\alpha -1}|) 
		+ |u^{\alpha -1}|v 
		+ u |v^{\alpha -1}| \|_{L^\infty_t[-T, T] H^s_x}
	\end{split}
\end{equation*}
%
%
which by the triangle inequality and algebra property simplifies to
%
%
\begin{equation}
	\label{L-contract}
	\begin{split}
		& \|Lu-Lv\|_{L^\infty_t[-T, T] H^s_x}
		\\
		& \le  T c_s |\lambda| \cdot \big [ \|u-v\|_{L^\infty_t[-T, T]
		H^s_x}(\|u\|^{\alpha -1}_{L^\infty_t[-T, T] H^s_x} +
		\|v\|^{\alpha -1}_{L^\infty_t[-T, T] H^s_x})
		\\
		& + \|u\|^{\alpha-1}_
		{L^\infty_t[-T, T] H^s_x} \|v\|_{L^\infty_t[-T, T] H^s_x}
		+ \|u\|_
		{L^\infty_t[-T, T] H^s_x} \|v\|^{\alpha -1}_
		{L^\infty_t[-T, T] H^s_x} \big ]
		\\
		& \le T c_s |\lambda| \cdot \left[  2R^{\alpha -1} 
		\|u -v\|_{L^\infty_t[-T, T] H^s_x} + 2R^{\alpha} \right]
		\\
		& \le T c' \|u -v \|_{L^\infty_t[-T, T] 
		H^s_x}
	\end{split}
\end{equation}
%
%
where $c' = c'(s, \lambda, \alpha, R)$ is a constant.
Since it was established earlier that $T = T(s, \lambda, \alpha, R, \vp)$,  
we conclude that for $$T < 1/c,  \qquad c = c(s, \lambda, \alpha, R, 
\vp),$$ $L$ is a 
contraction on $C([-T, T], B_R)$. \qquad \qedsymbol

\end{appendices}

%\nocite{*}
%\bibliographystyle{custom}
%\bibliography{/Users/davidkarapetyan/Documents/Math/schrodinger}
\begin{thebibliography}{10}
\newcommand{\enquote}[1]{``#1''}

\bibitem{Bourgain-Fourier-transfo-1}
\textsc{J.~Bourgain}.
\newblock \enquote{Fourier transform restriction phenomena for certain lattice
  subsets and applications to nonlinear evolution equations. {I}.
  {S}chr{\"o}dinger equations.}
\newblock \emph{Geom. Funct. Anal.}, \textbf{3} (1993), no.~2, 107--156.

\bibitem{Bourgain-Fourier-transfo}
---{}---{}---.
\newblock \enquote{Fourier transform restriction phenomena for certain lattice
  subsets and applications to nonlinear evolution equations. {II}. {T}he
  {K}d{V}-equation.}
\newblock \emph{Geom. Funct. Anal.}, \textbf{3} (1993), no.~3, 209--262.

\bibitem{Colliander_Keel_Staffilani_Takaoka_Tao-A-refined-globa}
\textsc{J.~Colliander, M.~Keel, G.~Staffilani, H.~Takaoka, and T.~Tao}.
\newblock \enquote{A refined global well-posedness result for {S}chr{\"o}dinger
  equations with derivative.}
\newblock \emph{SIAM J. Math. Anal.}, \textbf{34} (2002), no.~1, 64--86
  (electronic).

\bibitem{Colliander_Keel_Staffilani_Takaoka_Tao-Multilinear-est}
---{}---{}---.
\newblock \enquote{Multilinear estimates for periodic {K}d{V} equations, and
  applications.}
\newblock \emph{J. Funct. Anal.}, \textbf{211} (2004), no.~1, 173--218.

\bibitem{Folland_1999_Real-analysis}
\textsc{G.~B. Folland}.
\newblock \emph{Real analysis}.
\newblock Pure and Applied Mathematics (New York). John Wiley \& Sons Inc., New
  York, second edn., 1999.
\newblock Modern techniques and their applications, A Wiley-Interscience
  Publication.

\bibitem{Gorsky_2007_Well-posedness-}
\textsc{J.~Gorsky and A.~Himonas}.
\newblock \enquote{Well-posedness for a class of nonlinear dispersive
  equations.}
\newblock \emph{Dyn. Contin. Discrete Impuls. Syst. Ser. A Math. Anal.},
  \textbf{14} (2007), no. Advances in Dynamical Systems, suppl. S2, 85--90.

\bibitem{Himonas_Misioek-The-Cauchy-prob}
\textsc{A.~Himonas and G.~Misio{\l}ek}.
\newblock \enquote{The {C}auchy problem for a shallow water type equation.}
\newblock \emph{Comm. Partial Differential Equations}, \textbf{23} (1998), no.
  1-2, 123--139.

\bibitem{Himonas_Misiolek-A-priori-estima}
\textsc{A.~A. Himonas and G.~Misiolek}.
\newblock \enquote{A priori estimates for {S}chr{\"o}dinger type multipliers.}
\newblock \emph{Illinois J. Math.}, \textbf{45} (2001), no.~2, 631--640.

\bibitem{Schlotthauer-Hannah-Well-posedness-}
\textsc{H.~Schlotthauer-Hannah}.
\newblock \emph{Well-posedness and Regularity for a Higher Order Periodic mKdV
  Equation}.
\newblock Dissertation, University of Notre Dame, Notre Dame, Indiana, April
  2007.

\bibitem{Tao-Multilinear-wei}
\textsc{T.~Tao}.
\newblock \enquote{Multilinear weighted convolution of {$L^2$}-functions, and
  applications to nonlinear dispersive equations.}
\newblock \emph{Amer. J. Math.}, \textbf{123} (2001), no.~5, 839--908.

\bibitem{Taylor_1991_Pseudodifferent}
\textsc{M.~E. Taylor}.
\newblock \emph{Pseudodifferential operators and nonlinear {PDE}}, vol. 100 of
  \emph{Progress in Mathematics}.
\newblock Birkh{\"a}user Boston Inc., Boston, MA, 1991.

\bibitem{Tzvetkov_2006_Ill-posedness-i}
\textsc{N.~Tzvetkov}.
\newblock \enquote{Ill-posedness issues for nonlinear dispersive equations.}
\newblock In \emph{Lectures on nonlinear dispersive equations}, vol.~27 of
  \emph{GAKUTO Internat. Ser. Math. Sci. Appl.}, pp. 63--103. Gakk\=otosho,
  Tokyo, 2006.

\end{thebibliography}
				  \end{document}


