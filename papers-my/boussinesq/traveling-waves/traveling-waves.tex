%
%
\documentclass[12pt,reqno]{amsart}
\usepackage{amsmath}
\usepackage{amssymb}
\usepackage{cancel}  %for cancelling terms explicity on pdf
\usepackage{yhmath}   %makes fourier transform look nicer, among other things
\usepackage{framed}  %for framing remarks, theorems, etc.
\usepackage{enumerate} %to change enumerate symbols
\usepackage[margin=2.5cm]{geometry}  %page layout
\usepackage{graphicx}
\usepackage{subfig}
\numberwithin{equation}{section}  %eliminate need for keeping track of counters
\setlength{\parindent}{0in} %no indentation of paragraphs after section title
\renewcommand{\baselinestretch}{1.1} %increases vert spacing of text
%
\usepackage{hyperref}
\hypersetup{colorlinks=true,
linkcolor=blue,
citecolor=blue,
urlcolor=blue,
}
%\usepackage[alphabetic, initials, msc-links]{amsrefs} %for the bibliography;
%%uses cite pkg. Must be loaded after hyperref, otherwise doesn't work properly
%%(conflicts with cref in particular)
%
%
\newcommand{\ds}{\displaystyle}
\newcommand{\ts}{\textstyle}
\newcommand{\nin}{\noindent}
\newcommand{\rr}{\mathbb{R}}
\newcommand{\nn}{\mathbb{N}}
\newcommand{\zz}{\mathbb{Z}}
\newcommand{\cc}{\mathbb{C}}
\newcommand{\ci}{\mathbb{T}}
\newcommand{\zzdot}{\dot{\zz}}
\newcommand{\wh}{\widehat}
\newcommand{\p}{\partial}
\newcommand{\ee}{\varepsilon}
\newcommand{\vp}{\varphi}
\newcommand{\wt}{\widetilde}
%
%
%
%
\newtheorem{theorem}{Theorem}[section]
\newtheorem{lemma}[theorem]{Lemma}
\newtheorem{corollary}[theorem]{Corollary}
\newtheorem{claim}[theorem]{Claim}
\newtheorem{prop}[theorem]{Proposition}
\newtheorem{proposition}[theorem]{Proposition}
\newtheorem{no}[theorem]{Notation}
\newtheorem{definition}[theorem]{Definition}
\newtheorem{remark}[theorem]{Remark}
\newtheorem{examp}{Example}[section]
\newtheorem{exercise}[theorem]{Exercise}
%
\makeatletter \renewenvironment{proof}[1][\proofname]
{\par\pushQED{\qed}\normalfont\topsep6\p@\@plus6\p@\relax\trivlist\item[\hskip\labelsep\bfseries#1\@addpunct{.}]\ignorespaces}{\popQED\endtrivlist\@endpefalse}
\makeatother%
%makes proof environment bold instead of italic
%\makeatletter
%\renewcommand\subsubsection{\@startsection{subsubsection}{3}{\z@}%
%{-3.25ex\@plus -1ex \@minus -.2ex}%
%{1.5ex \@plus .2ex}%
%{\normalfont\normalsize \bfseries}}
%\makeatother
%makes subsubsubsection bold instead of italic
%
\def\sgn{\operatorname{sgn}}
\def\sech{\operatorname{sech}}
\newcommand{\uol}{u^\omega_\lambda}
\newcommand{\lbar}{\bar{l}}
\renewcommand{\l}{\lambda}
\newcommand{\R}{\mathbb{R}}
\newcommand{\RR}{\mathcal{R}}
\newcommand{\al}{\alpha}
\newcommand{\ve}{q}
\newcommand{\tg}{{tan}}
\newcommand{\m}{q}
\newcommand{\N}{N}
\newcommand{\ta}{{\tilde{a}}}
\newcommand{\tb}{{\tilde{b}}}
\newcommand{\tc}{{\tilde{c}}}
\newcommand{\tS}{{\tilde{S}}}
\newcommand{\tP}{{\tilde{P}}}
\newcommand{\tu}{{\tilde{u}}}
\newcommand{\tw}{{\tilde{w}}}
\newcommand{\tA}{{\tilde{A}}}
\newcommand{\tX}{{\tilde{X}}}
\newcommand{\tphi}{{\tilde{\phi}}}
\begin{document}
\title{Traveling Wave Solutions for $B_{4}$ and Boussinesq}
\author{Dan-Andrei Geba, Alexandrou Himonas, and David Karapetyan}
\address{Department of Mathematics, University of Rochester, Rochester, NY 14627}
\address{Department of Mathematics, University of Notre Dame, Notre Dame, IN 46556}
\address{Department of Mathematics, University of Notre Dame, Notre Dame, IN 46556}
\date{06/03/2011}
%
%
\subjclass[2000]{35B30, 35Q55, 35Q72}
\keywords{local well-posedness; ill-posedness.}
\maketitle
%
%
%%%%%%%%%%%%%%%%%%%%%%%%%%%%%%%%%%%%%%%%%%%%%%%%%%%%%
%
%
%                Traveling waves for B4
%
%
%%%%%%%%%%%%%%%%%%%%%%%%%%%%%%%%%%%%%%%%%%%%%%%%%%%%%
%
%
\section{Traveling Waves for the $B_{4}$ Equation} 
\label{sec:trav-wave}
We are looking for functions $f(x)$ such that
%
%
\begin{equation}
  \label{ansatz}
\begin{split}
u(x,t) = f(x-ct)
\end{split}
\end{equation}
%
%
satisfies the $B_{4}$ equation
%
%
\begin{equation}
  \label{bous-eqn}
\begin{split}
  u_{tt} + u_{xxxx} + (u^{2})_{xx} = 0.
\end{split}
\end{equation}
%
%
Substituting \eqref{ansatz} into \eqref{bous-eqn} gives
%
%
\begin{equation}
\begin{split}
  c^{2} f'' + f'''' + (f^{2})'' = 0
\end{split}
\label{post-sub}
\end{equation}
%
%
where $$f^{(k)} = f^{(k)}(x-ct) = \frac{d^{k}f}{dx^{k}} \Big |_{x-ct}.$$ 
Integrating \eqref{post-sub} twice with repspect to $x$, we obtain
%
%
\begin{equation}
  \label{pre-const-elim}
\begin{split}
  f'' + c^{2} f + f^{2} + ax +b = 0, \quad a,b \in \rr.
\end{split}
\end{equation}
%
%
%
%
Assuming $a = 0$, we obtain
%
\begin{equation}
  \label{2nd-order-ode-pre}
\begin{split}
  f'' = - f^{2} -c^{2} f -b.
\end{split}
\end{equation}
%
%
Multiplying both sides of \eqref{2nd-order-ode-pre} by $f'$ and integrating then
gives 
%
\begin{equation}
  \label{first-order-ode}
\begin{split}
  (f')^{2} = -\frac{2}{3} \left [ f^{3} + \frac{3}{2} c^{2} f^{2} + 3bf +d
  \right ].
\end{split}
\end{equation}
%
%
We now restrict our attention to solutions of \eqref{first-order-ode} where
we attain a local maximum $M$ and local minimum $m$. Then $b = b(m, M)$ and $d =
d(m, M)$ are 
completely determined by the system
%
%
\begin{equation*}
\begin{split}
  & 0 = -\frac{2}{3} \left [ M^{3} + \frac{3}{2} c^{2} M^{2} + 3bM + d
  \right ]
  \\
  & 0 = -\frac{2}{3} \left [ m^{3} + \frac{3}{2} c^{2} m^{2} + 3bm + d
  \right ]
\end{split}
\end{equation*}
%
giving
%
%
\begin{equation}
  \label{b-val}
\begin{split}
  & b  = -\frac{1}{3}\left( m^{2} + mM + M^{2} \right) - \frac{1}{2}c^{2}\left( m + M
  \right)
  \\
  & d = -M\left( m^{2} + mM + \frac{3}{2}c^{2} m \right).
\end{split}
\end{equation}
%
%
%
Furthermore, recall that for a cubic polynomial $g(x) = x^{3} +bx^{2} +cx +d$
with roots $r_{1}, r_{2}, r_{3}$, we have $r_{1} + r_{2} + r_{3} = -b$.
%
%
\begin{framed}
%
%
\begin{remark}
\label{rem:fact-pf}
For a proof, expand 
$g(x) = (x-a)(x-b)(x-c)$.
\end{remark}
%
%
\end{framed}
%
%
Combining this information with the fundamental theorem of algebra, 
we see that \eqref{first-order-ode} has the factorization
%
%
%
%
\begin{equation}
\begin{split}
  \left ( f'\right )^{2}
  = \frac{2}{3} \left (f-m\right )
  \left( M-f\right )
  \left [ f + m + M + \frac{3}{2}c^{2} \right ]
\end{split}
\label{ode-fact}
\end{equation}
%
Note that any solution to \eqref{ode-fact} is a solution to
\eqref{2nd-order-ode}. Using this fact, and setting $M > m = 0$, we obtain the following result.
%
%
%
%%%%%%%%%%%%%%%%%%%%%%%%%%%%%%%%%%%%%%%%%%%%%%%%%%%%%
%
%
%                
%
%
%%%%%%%%%%%%%%%%%%%%%%%%%%%%%%%%%%%%%%%%%%%%%%%%%%%%%
%
%
\begin{lemma}
  For fixed $M > m = 0$ and $c \in \rr$,  there exists a positive
  number $\ell = \ell(m, M)$ and an even $2\ell$-periodic analytic function $f =
  f(x)$ which solves the initial value problem
\begin{align}
  \label{2nd-order-ode}
  & f'' = -f^{2} - c^{2} f + \frac{1}{3} M^{2} + \frac{1}{2}c^{2} M
\\
\label{2nd-order-ode-data}
& f(0)=0, \quad f'(0) = 0.
\end{align}
Furthermore, the function
  $f(x)$ satisfies
  %
  %
  \begin{equation}
    \label{f-bound}
  \begin{split}
  0 \le f(x) \le M
  \end{split}
  \end{equation}
  %
  %
  and the function $u(x,t) = f(x-t)$ is a traveling wave solution of the
  $B_{4}$ equation. 
\label{lem:ode-solution}
\end{lemma}
%
%
%
\begin{proof}
By the Picard-Lindel\"of theorem, the Cauchy problem
\begin{gather}
  \label{ode-cauchy-left}
   f'
  = -\left [ \frac{2}{3} f
  \left( M -f \right )
  \left ( f +M + \frac{3}{2}c^{2}\right ) \right
  ]^{1/2},
  \\
\label{ode-cauchy-data-left}
f(0) = M-\ee
\end{gather}
admits a unique non-trivial solution in the domain $\eta \le f^{-1}(M- \ee) - \eta$.
From the form of \eqref{ode-cauchy-left}, we see that this solution extends to a
nontrivial solution of \eqref{ode-cauchy-left} in the domain $0 \le x \le \ell$,
where $\ell = f^{-1}(M)$. Reflecting this solution across the $y$-axis, and then
taking the periodic extension of resulting $y$-symmetric graph, we obtain 
obtain a non-trivial $2\ell$-periodic
solution to the Cauchy problem
\begin{gather}
  (f')^{2}
  = \frac{2}{3} f
  \left( M -f \right )
  \left [ f +M + \frac{3}{2}c^{2}  \right ],
  \\
   f(0) =0
\end{gather}
which solves 
\eqref{2nd-order-ode}-\eqref{2nd-order-ode-data}. Analyticity follows from the
Cauchy-Kovalevskaya theorem (the right hand side of \eqref{2nd-order-ode} is an
analytic function of $f$). The
estimate \eqref{f-bound} follows by construction.
%
%
\end{proof}
%
We now provide some graphs of solution to the ODE ivp above with $y(0) = 0$,
$y'(0) =1$. 
%\newpage
%\begin{figure}[!h]
  %\vspace{-50mm}
  %\subfloat{
  %\hspace{-50mm}
%\centering
%\includegraphics[scale=0.7]{c3b1}
%}
%\subfloat{
%\hspace{-50mm}
%\centering
%\includegraphics[scale=0.7]{c3b3}
%}
%\end{figure}
%\begin{figure}
  %\vspace{-150mm}
  %\subfloat{
  %\hspace{-50mm}
%\centering
%\includegraphics[scale=0.7]{c7b3}
%}
%\subfloat{
%\hspace{-50mm}
%\centering
%\includegraphics[scale=0.7]{c13b3}
%}
%\end{figure}

%\begin{figure}[ht]
  %\begin{minipage}[h]{4.0\linewidth}
    %\vspace{-40mm}
    %\hspace{-45mm}
%\includegraphics[scale=0.7]{c3b1}
%\hspace{-50mm}
%\includegraphics[scale=0.7]{c3b3}
%\end{minipage}
%\end{figure}
\begin{figure}[!ht]
  \begin{center}
\subfloat{\includegraphics[scale=0.4]{c3b1}} 
\subfloat{\includegraphics[scale=0.4]{c3b3}}
\\
\vspace{-40mm}
\subfloat{\includegraphics[scale=0.4]{c7b3}}
\subfloat{\includegraphics[scale=0.4]{c13b3}}
\end{center}
\end{figure}



%
%
%
%%%%%%%%%%%%%%%%%%%%%%%%%%%%%%%%%%%%%%%%%%%%%%%%%%%%%
%
%
%                Period Estimate
%
%
%%%%%%%%%%%%%%%%%%%%%%%%%%%%%%%%%%%%%%%%%%%%%%%%%%%%%
%
%
\newpage
\begin{lemma}
  The period $2 \ell$ of the solution to the Cauchy problem
  \eqref{2nd-order-ode}-\eqref{2nd-order-ode-data} described in Lemma
  \ref{lem:ode-solution} satisfies
  %
  %
  \begin{equation}
  \begin{split}
    \ell \sim M^{-1/2}.
  \end{split}
  \label{period-est}
  \end{equation}
  %
  %
  where $\ell$ varies continuously with respect to $M$.
  %
  %
\label{lem:period-est}
\end{lemma}
%
%
%
%
\begin{proof}
%
For $0 < x < \ell$, we have $f'(x) > 0$. Hence, the function $f = f(x)$ has an
inverse $x = x(f)$, and so by the fundamental theorem of calculus 
%
%
\begin{equation*}
\begin{split}
\ell & = x(M) - x(0)
\\
& = \int_{0}^{M} \frac{dx}{df} df
\\
& = \int_{0}^{M} \frac{1}{ \left\{ \frac{2}{3}f(M-f)
\left[ f + M + \frac{3}{2} c^{2} \right] \right\}^{1/2}} df.
\end{split}
\end{equation*}
%
Bounding from above, we have
%
%
\begin{equation*}
\begin{split}
\ell
& \le \frac{\sqrt{\frac{3}{2}}}{M^{1/2}} \int_{0}^{M} \frac{1}{f^{1/2}(M-f)^{1/2}}df
\\
& = \frac{\sqrt{\frac{3}{2}}}
{M^{1/2}} 
\\
& \times \left[ \int_{0}^{\frac{M}{2}} \frac{1}{f^{1/2}(M-f)^{1/2}}df 
+ \int_{\frac{M}{2}}^{M} \frac{1}{f^{1/2}(M-f)^{1/2}}df
\right]
\\
& = \frac{\sqrt{\frac{3}{2}}}
{M^{1/2}} 
\\
& \times \frac{1}{\left( \frac{M}{2} \right)^{1/2}} \left[ \int_{0}^{\frac{M}{2}} \frac{1}{f^{1/2}}df 
+ \int_{\frac{M}{2}}^{M} \frac{1}{(M-f)^{1/2}}df
\right]
\\
& = \frac{4 \sqrt{\frac{3}{2}}}
{M^{1/2}} 
\\
& \le 2 \sqrt{6} M ^{-1/2}.
\end{split}
\end{equation*}
%
%
Bounding from below, we have
%
%
\begin{equation*}
\begin{split}
\ell
& \ge  \frac{1}{ \left\{ \frac{2}{3}M^{2}
\left( 2M 
\right) \right\}^{1/2}} \int_{0}^{M} df
\\
& = \frac{1}{\left\{ \frac{2}{3}
\left( 2M 
\right) \right\}^{1/2}} 
\\
& = \frac{\sqrt{3}}{2} M^{-1/2}
\end{split}
\end{equation*}
%
The continuity of $\ell = \ell(M)$ follows immediately follows from the fact
that $d\ell /dM$is bounded, which can be shown using the estimates above and
the Leibniz integral rule. This completes the proof.
\end{proof}
%
%
%
%%%%%%%%%%%%%%%%%%%%%%%%%%%%%%%%%%%%%%%%%%%%%%%%%%%%%
%
%
%                Bound on derivatives
%
%
%%%%%%%%%%%%%%%%%%%%%%%%%%%%%%%%%%%%%%%%%%%%%%%%%%%%%
%
%
\begin{lemma}
  For $M \ge 1$ and $k \in \left\{ 1,2,3,\dots \right\}$, we have
%
%
\begin{equation}
\begin{split}
  | f^{(k)} | \le c_{k} M^{1 + k/2}.
\end{split}
\label{eqn:deriv-bound}
\end{equation}
%
%
\label{lem:deriv-bound}
\end{lemma}
%
%
%
\begin{proof}
The proof will be by induction. Note that 
%
%
\begin{equation*}
\begin{split}
  | f' |
  & \lesssim \left [ M^{2}(2M + \frac{3}{2}c^{2}) \right]^{1/2}
  \\
  & \lesssim M^{1 + 1/2}
\end{split}
\end{equation*}
%
and
%
%
\begin{equation*}
\begin{split}
| f'' |
& = | -f^{2} - c^{2} f - \frac{1}{3} M^{2} - \frac{1}{2}c^{2} M
 |
\\
& \le | f |^{2} + c^{2} | f | + \frac{1}{3}M^{2} + \frac{1}{2}c^{2} M
\\
& \le M ^{2} +  c^{2} M  + \frac{1}{3}M^{2} + \frac{1}{2}c^{2} M
\\
& \lesssim M^{1 + 2/2}
\end{split}
\end{equation*}
%
and
%
%
\begin{equation*}
\begin{split}
| f''' | 
& = 2 |f+1| | f' |
\\
& \lesssim (M+1)M^{3/2}
\\
& \lesssim M^{1 + 3/2}.
\end{split}
\end{equation*}
%
%
%
%
Assume $| f^{(k)} | \le c_{k} M^{1 + \frac{k}{2}}$ for $1 \le k < n, \ n \ge 3$.
Using \eqref{2nd-order-ode}, we have
%
%
\begin{equation*}
\begin{split}
  f^{(n)}
  & =
  (f'')^{(n-2)}
  \\
  & = \left( -f^{2} - c^{2} f + \frac{1}{3} M^{2} + \frac{1}{2}c^{2} M
 \right)^{(n-2)}
  \\
  & = -(f^{2})^{(n-2)} - c^{2}f^{(n-2)}.
\end{split}
\end{equation*}
%
%
Hence, by the Leibniz rule and our inductive assumption, we have
%
%
\begin{equation*}
\begin{split}
  | f^{(n)} |
  & \le  \sum_{p=0}^{n-2} \binom{n-2}{p} | f^{(n-2-p)} | |
  f^{(p)} | + c^{2} | f^{(n-2)} |
  \\
  & \le  \sum_{p=0}^{n-2} \binom{n-2}{p} c_{n-2-p}M^{1 +
  \frac{n-2-p}{2}}c_{p}M^{1 + \frac{p}{2}} + c_{n-2} M^{1 + \frac{n-2}{2}}
  \\
  & = M^{1 + \frac{n}{2}} \sum_{p=0}^{n-2} \binom{n-2}{p} c_{n-2-p}c_{p}
  + c_{n-2} M^{\frac{n}{2}}
  \\
  & \le C M^{1 + n/2}, \qquad M \ge 1
\end{split}
\end{equation*}
%
%
concluding the proof.
\end{proof}
%
%
%
%%%%%%%%%%%%%%%%%%%%%%%%%%%%%%%%%%%%%%%%%%%%%%%%%%%%%
%
%
%                
%
%
%%%%%%%%%%%%%%%%%%%%%%%%%%%%%%%%%%%%%%%%%%%%%%%%%%%%%
%
%
\begin{corollary}
For $M \ge 1$,
%
%
\begin{equation*}
\begin{split}
  \| f^{(k)} \|_{L^{2}(-\ell, \ell)} \le c_{k} M^{1/2 + k/2}.
\end{split}
\end{equation*}
%
%
\label{cor:sob-norm-bound}
\end{corollary}
%
%
%
%
\begin{proof}
  Applying Lemma \ref{lem:deriv-bound}, we have
  %
  %
  \begin{equation*}
  \begin{split}
    \| f^{(k)} \|_{L^{2}(-\ell, \ell)}
    & \le 2 \ell \| f^{(k)} \|_{L^{\infty}(-\ell, \ell)}
    \\
    & =  c_{k} \ell M^{1 + k/2}
    \\
    & \sim c_{k} M^{-1/2} M^{1 + k/2}
    \\
    & = c_{k} M^{1/2 + k/2}
  \end{split}
  \end{equation*}
  %
  %
  concluding the proof.
\end{proof}
%
%
Recall that for fixed $M > 0$, Lemma \ref{lem:period-est} gives us a
solution $f_{M}$ to the Cauchy problem
\eqref{2nd-order-ode}-\eqref{2nd-order-ode-data} with period $2 \ell_{M} \sim
M^{-1/2}$. From the continuity of $\ell$ with respect to $M$, it follows that for $n \in \mathbb{N}$ we have 
%
%
\begin{equation*}
\begin{split}
  \ell_{M} = \frac{2\pi}{n} \quad \text{for some } M \sim \sqrt{n}.
\end{split}
\end{equation*}
%
%
%This fact will prove crucial in proving non-uniform dependence. Note that the
%$B_{4}$ equation admits the scaling
%%
%%
%\begin{equation}
%\begin{split}
  %u_{\lambda}(x,t) = \lambda u(\lambda x, \lambda^{2} t).
%\end{split}
%\label{scaling}
%\end{equation}
%%
%%
%Let 
%%
%%
%\begin{align}
  %& u_{n} = f_{n}(x-t),
  %\\
  %& v_{n} = c_{n}f_{n}(c_{n} x - c_{n}^{2}t). 
%\label{cand-solns}
%\end{align}
%%
%where
%%
%%
%\begin{equation*}
%\begin{split}
  %c_{n} = 1- 1/n.
%\end{split}
%\end{equation*}
%%
%%
%%
%\subsection{Boundedness}
%%We first define the Fourier transform of a $2 \ell$
%%periodic function $f(x)$. First note
%%that the set
%%%
%%%
%%\begin{equation*}
%%\begin{split}
  %%\left\{ e^{\pi i kx/\ell} \right\}_{k \in \zz}
%%\end{split}
%%\end{equation*}
%%%
%%%
%%is a basis for $L^{2}(-\ell, \ell)$. Therefore, we have
%%%
%%%
%%\begin{equation*}
%%\begin{split}
  %%f(x) = \sum_{k \in \zz} c_{k} e^{ \pi ikx / \ell}.
%%\end{split}
%%\end{equation*}
%%%
%%%
%%To find $c_{k}$, we multiply both sides by $e^{- \pi imx / \ell}$, integrate,
%%and apply Fubini to obtain
%%%
%%%
%%\begin{equation*}
%%\begin{split}
  %%\int_{-\ell}^{\ell} e^{- \pi imx / \ell}f(x) dx
  %%& = \sum_{k \in \zz}
  %%\int_{-\ell}^{\ell} c_{k}e^{- \pi i x(m-k)}
  %%\\
  %%& = c_{m} 2 \ell.
%%\end{split}
%%\end{equation*}
%%%
%%%
%%Hence, 
%%%
%%%
%%\begin{equation*}
%%\begin{split}
  %%\wh{f}(k) \doteq  c_{k} = \frac{1}{2 \ell} \int_{-\ell}^{\ell} e^{- \pi ikx/
  %%\ell}f(x) dx.
%%\end{split}
%%\end{equation*}
%%
%%
%Using change of variables, we note that 
%%
%%
%\begin{equation*}
%\begin{split}
  %\wh{f_{n}(a \cdot - b t)}(\xi)
  %& = \int_{-\pi}^{\pi} e^{-ix \xi} f_{n}(ax - bt)dx
  %\\
  %& = \int_{-\pi - \frac{b}{a}t}^{\pi - \frac{b}{a}t} e^{-i \xi(x +
  %\frac{b}{a}t)} f_{n}(ax) dx
  %\\
  %& = \int_{-\pi}^{\pi } e^{-i \xi(x +
  %\frac{b}{a}t)} f_{n}(ax) dx
  %\\
  %& = e^{-i\xi bt/a} \int_{-\pi}^{\pi} e^{-ix \xi} f_{n}(ax) dx
%\end{split}
%\end{equation*}
%
%
%
%
%
%
%
%
%%%%%%%%%%%%%%%%%%%%%%%%%%%%%%%%%%%%%%%%%%%%%%%%%%%%%
%
%
%                Solitons
%
%
%%%%%%%%%%%%%%%%%%%%%%%%%%%%%%%%%%%%%%%%%%%%%%%%%%%%%
%
%
\section{Solitons for the Boussinesq} 
\label{sec:soliton}
We are looking for functions $f(x)$ such that
%
%
\begin{equation}
  \label{ansatz-bous}
\begin{split}
u(x,t) = f(x-ct)
\end{split}
\end{equation}
%
%
satisfies the Boussinesq equation
%
%
\begin{equation}
  \label{bous-eqn*}
\begin{split}
  u_{tt} -u_{xx} + u_{xxxx} + (u^{2})_{xx} = 0.
\end{split}
\end{equation}
%
%
The analog of \eqref{first-order-ode} in this case is
%
%
\begin{equation}
  \label{hh}
\begin{split}
  (f')^{2} = -\frac{2}{3} \left [ f^{3} + \frac{3}{2}(c^{2}-1)f^{2} + 3bf +d \right ]
\end{split}
\end{equation}
%
We may rewrite \eqref{hh} as
%
%
\begin{equation}
  \label{first-order-rewritten}
\begin{split}
  \frac{(f')^{2}}{f} = -\frac{2}{3}\left[ f^{2} + \frac{3}{2}(c^{2}-1)f+ 3b -
  \frac{3d}{2f}\right].
\end{split}
\end{equation}
%
%
Next, note that for $f$ to be a soliton, we must have $|f(x)|, |f'(x)|,
|f''(x)| \to 0$ as $|x| \to \infty$. Hence, we assume this. 
Then by L'H\^opital's rule
%
%
\begin{equation*}
\begin{split}
  \lim_{|x| \to \infty} \frac{(f')^{2}}{f} = \lim_{|x| \to \infty} \frac{2f'
  f''}{f'} = \lim_{|x| \to \infty} 2f'' = 0.
\end{split}
\end{equation*}
%
%
Coupling this with our assumption that $| f(x) |, | f'(x) | \to 0$ as $|x| \to
\infty$, we see that we must have $b =d=0$. Hence, from 
\eqref{first-order-ode} we obtain 
%
%
\begin{equation}
  \label{bous-ode-simp}
\begin{split}
  (f')^{2} = -\frac{2}{3} f^{2} \left[ f - \frac{3}{2}(1-c^{2})  \right].
\end{split}
\end{equation}
%
%
For $f$ to be a real valued solution to \eqref{bous-ode-simp} in some domain
$D$, we must have $$f(x) \le \frac{3}{2}(1-c^{2}) \ \text{for all} \  x \in D.$$ 
%
%
Then
%
%
\begin{equation*}
\begin{split}
  f' = \sqrt{\frac{2}{3}}f \left[ \frac{3}{2}(1 - c^{2}) -f \right]^{1/2}
\end{split}
\end{equation*}
%
%
or
%
%
\begin{equation*}
\begin{split}
\frac{df}{ f \left[ \frac{3}{2}(1 - c^{2}) -f \right]^{1/2}} = \sqrt{\frac{2}{3}}
dx.
\end{split}
\end{equation*}
%
%
Integrating both sides, and using the formula
%
%
\begin{equation}
  \label{key-int-form}
\begin{split}
  \int \frac{dy}{y \sqrt{a + by}} = -\frac{2}{ \sqrt{a}} \tanh^{-1}
  \frac{\sqrt{a+ by}}{ \sqrt{a}}
\end{split}
\end{equation}
%
%
we obtain
%
%
\begin{equation*}
\begin{split}
  -\frac{2}{\sqrt{\frac{3}{2}(1-c^{2})}} \tanh^{-1}
  \frac{\sqrt{\frac{3}{2}(1-c^{2}) - f}}{\sqrt{\frac{3}{2}(1-c^{2})}} =
  \sqrt{\frac{2}{3}}x + A, \quad A \in \rr
\end{split}
\end{equation*}
%
%
or
\begin{equation*}
\begin{split}
  \tanh^{-1}
  \frac{\sqrt{\frac{3}{2}(1-c^{2}) - f}}{\sqrt{\frac{3}{2}(1-c^{2})}} =
  - \frac{\sqrt{(1-c^{2})}}{2}x + B, \quad B \in \rr
\end{split}
\end{equation*}
%
Assume that
%
%
\begin{equation*}
\begin{split}
  f(x_{0}) = \frac{3}{2}(1-c^{2})
\end{split}
\end{equation*}
%
%
for fixed $x_{0}$. Then
%
%
\begin{equation*}
\begin{split}
  0 = \tanh^{-1}(0) = - \frac{\sqrt{(1-c^{2})}}{2}x_{0} + B
\end{split}
\end{equation*}
%
%
or
%
%
\begin{equation*}
\begin{split}
B = \frac{\sqrt{(1-c^{2})}}{2}x_{0}.
\end{split}
\end{equation*}
%
%
Therefore,
%
%
\begin{equation*}
\begin{split}
  \frac{\sqrt{\frac{3}{2}(1-c^{2}) - f}}{\sqrt{\frac{3}{2}(1-c^{2})}} =
  \tanh \left [- \frac{\sqrt{(1-c^{2})}}{2}(x -x_{0}) \right ] 
\end{split}
\end{equation*}
%
%
or
%
%
%
\begin{equation*}
\begin{split}
  f = \frac{3}{2}(1-c^{2}) \left \{ 1 - \tanh^{2} \left [- \frac{\sqrt{(1-c^{2})}}{2}(x -x_{0})
  \right ] \right \}
  \end{split}
\end{equation*}
%
%
Using the identities
%
%
\begin{equation*}
\begin{split}
  & 1 - \tanh^{2}x = \sech^{2}x,
  \\
  & \sech^{2}(-x) = \sech^{2}x
\end{split}
\end{equation*}
%
we see that
%
%
\begin{equation*}
\begin{split}
  f(x) = \frac{3}{2}(1-c^{2}) \sech^{2} \left [\frac{\sqrt{(1-c^{2})}}{2}(x -x_{0})
  \right ]. 
  \end{split}
\end{equation*}
%
%
Hence, %
%
\begin{equation*}
\begin{split}
  u_{c}(x,t) = f(x - ct) = \frac{3}{2}(1-c^{2}) \sech^{2} \left
  [\frac{\sqrt{(1-c^{2})}}{2}(x -ct -x_{0})
  \right ]
\end{split}
\end{equation*}
%
%
is a traveling soliton solution to the Boussinesq equation.
%

%%%%%%%%%%%%%%%%%%%%%%%%%%%%%%%%%%%%%%%%%%%%%%%%%%%%%
%
%
%                B4 no Solitons
%
%
%%%%%%%%%%%%%%%%%%%%%%%%%%%%%%%%%%%%%%%%%%%%%%%%%%%%%
%
%
\section{Non-Existence of Solitons for $B_{4}$} 
\label{sec:B4-soliton-fail}
%
%
First note that the
analog of \eqref{bous-ode-simp} for the $B_{4}$ ivp
\begin{equation}
  \label{b4}
\begin{split}
  u_{tt} + u_{xxxx} + (u^{2})_{xx} = 0
\end{split}
\end{equation}
is
\begin{equation}
  \label{b4-ode-simp}
\begin{split}
  (f')^{2} = -\frac{2}{3} f^{2} \left [ f + \frac{3}{2}c^{2}\right ]. 
\end{split}
\end{equation}
For $f(x)$ to be a real valued solution to \eqref{b4-ode-simp} in some domain
$D \subset \rr$, we must have $f(x)~\le~-\frac{3}{2}c^{2}$ for all
$x \in D \subset \rr$. Then 
%
\begin{equation*}
\begin{split}
  f' = \sqrt{\frac{2}{3}}f \left[ -\frac{3}{2}c^{2} -f \right]^{1/2}
\end{split}
\end{equation*}
%
%
or
%
%
\begin{equation*}
\begin{split}
\frac{df}{ f \left[ -\frac{3}{2}c^{2} -f \right]^{1/2}} = \sqrt{\frac{2}{3}}
dx
\end{split}
\end{equation*}
%
Integrating both sides and applying the formula
%
%
\begin{equation*}
\begin{split}
  \int \frac{dx}{x(a-x)^{1/2}} = \frac{2}{(-a)^{1/2}} \tan^{-1}\left[
  \frac{(a-x)^{1/2}}{(-a)^{1/2}}
  \right], \quad a < 0,\ a -x >0
\end{split}
\end{equation*}
%
%
we obtain
%
%
\begin{equation*}
\begin{split}
  \frac{2\sqrt{\frac{2}{3}}}{c} \tan^{-1}\left[ 2
  \sqrt{\frac{2}{3}}(-\frac{3}{2} c^{2} -f)^{1/2}
  \right] = \sqrt{\frac{2}{3}} + A
\end{split}
\end{equation*}
%
%
or
%
%
\begin{equation*}
\begin{split}
  (-\frac{3}{2} c^{2}-f)^{1/2} = \frac{1}{2} \sqrt{\frac{3}{2}} \tan\left[
  \frac{c}{2}(x- x_{0})
  \right].
\end{split}
\end{equation*}
%
%
or
%
%
\begin{equation*}
\begin{split}
  f(x) = -\frac{3}{8} \tan^{2}\left[ \frac{c}{2}(x- x_{0}) \right] - \frac{3}{2}c^{2}.
\end{split}
\end{equation*}
%
%
%
%
which blows up as $$ x \to x_{0} + \frac{\pi}{c}.$$ Hence, $f(x)$ cannot be a
soliton.
%
%
\end{document}
