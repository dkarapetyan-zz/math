\documentclass[12pt,reqno]{amsart}
\usepackage{amsmath}
\usepackage{amssymb}
\usepackage{cancel}  %for cancelling terms explicity on pdf
\usepackage{yhmath}   %makes fourier transform look nicer, among other things
\usepackage{framed}  %for framing remarks, theorems, etc.
\usepackage{enumerate} %to change enumerate symbols
\usepackage[margin=2.5cm]{geometry}  %page layout
\numberwithin{equation}{section}  %eliminate need for keeping track of counters
\setlength{\parindent}{0in} %no indentation of paragraphs after section title
\renewcommand{\baselinestretch}{1.1} %increases vert spacing of text
%
\usepackage{hyperref}
\hypersetup{colorlinks=true,
linkcolor=blue,
citecolor=blue,
urlcolor=blue,
}
\usepackage[alphabetic, initials, msc-links]{amsrefs} %for the bibliography;
%uses cite pkg. Must be loaded after hyperref, otherwise doesn't work properly
%(conflicts with cref in particular)
\usepackage{cleveref} %must be last loaded package to work properly
\renewcommand{\cref}{\Cref}
%
%
\newcommand{\ds}{\displaystyle}
\newcommand{\ts}{\textstyle}
\newcommand{\nin}{\noindent}
\newcommand{\rr}{\mathbb{R}}
\newcommand{\nn}{\mathbb{N}}
\newcommand{\zz}{\mathbb{Z}}
\newcommand{\cc}{\mathbb{C}}
\newcommand{\ci}{\mathbb{T}}
\newcommand{\zzdot}{\dot{\zz}}
\newcommand{\wh}{\widehat}
\newcommand{\p}{\partial}
\newcommand{\ee}{\varepsilon}
\newcommand{\vp}{\varphi}
\newcommand{\wt}{\widetilde}
%
%
%
%
\newtheorem{theorem}{Theorem}[section]
\newtheorem{lemma}[theorem]{Lemma}
\newtheorem{corollary}[theorem]{Corollary}
\newtheorem{claim}[theorem]{Claim}
\newtheorem{prop}[theorem]{Proposition}
\newtheorem{proposition}[theorem]{Proposition}
\newtheorem{no}[theorem]{Notation}
\newtheorem{definition}[theorem]{Definition}
\newtheorem{remark}[theorem]{Remark}
\newtheorem{examp}{Example}[section]
\newtheorem{exercise}[theorem]{Exercise}
%
\makeatletter \renewenvironment{proof}[1][\proofname]
{\par\pushQED{\qed}\normalfont\topsep6\p@\@plus6\p@\relax\trivlist\item[\hskip\labelsep\bfseries#1\@addpunct{.}]\ignorespaces}{\popQED\endtrivlist\@endpefalse}
\makeatother%
%makes proof environment bold instead of italic
%\makeatletter
%\renewcommand\subsubsection{\@startsection{subsubsection}{3}{\z@}%
%{-3.25ex\@plus -1ex \@minus -.2ex}%
%{1.5ex \@plus .2ex}%
%{\normalfont\normalsize \bfseries}}
%\makeatother
%makes subsubsubsection bold instead of italic
%
\def\sgn{\operatorname{sgn}}
\newcommand{\uol}{u^\omega_\lambda}
\newcommand{\lbar}{\bar{l}}
\renewcommand{\l}{\lambda}
\newcommand{\R}{\mathbb{R}}
\newcommand{\RR}{\mathcal{R}}
\newcommand{\al}{\alpha}
\newcommand{\ve}{q}
\newcommand{\tg}{{tan}}
\newcommand{\m}{q}
\newcommand{\N}{N}
\newcommand{\ta}{{\tilde{a}}}
\newcommand{\tb}{{\tilde{b}}}
\newcommand{\tc}{{\tilde{c}}}
\newcommand{\tS}{{\tilde{S}}}
\newcommand{\tP}{{\tilde{P}}}
\newcommand{\tu}{{\tilde{u}}}
\newcommand{\tw}{{\tilde{w}}}
\newcommand{\tA}{{\tilde{A}}}
\newcommand{\tX}{{\tilde{X}}}
\newcommand{\tphi}{{\tilde{\phi}}}
\synctex=1
\begin{document}
\title{A Modified Boussinesq equation}
\author{Dan-Andrei Geba, Alexandrou Himonas, and David Karapetyan}
\address{Department of Mathematics, University of Rochester, Rochester, NY 14627}
\address{Department of Mathematics, University of Notre Dame, Notre Dame, IN 46556}
\address{Department of Mathematics, University of Notre Dame, Notre Dame, IN 46556}
\date{04/15/2011}
%
%
\subjclass[2000]{35B30, 35Q55, 35Q72}
\keywords{local well-posedness; ill-posedness.}
\maketitle
\tableofcontents
%
%
\section{Introduction}
%
%
%
We consider the initial value problem (ivp) for the fourth order modified Boussinesq
($B_4$) equation 
\begin{align}
  & u_{tt} + u_{xxxx} + (u^2)_{xx} = 0, \quad x \in \rr \ \text{or} \
  \ci, \ t \in \rr
  \label{eqn:mb-2}
  \\
  & u(x,0) = u_{0}(x) \in H^{s}, \quad \p_t u(x, 0) = u_1(x) \in H^{s-2}.
  \label{eqn:mb-init-data-2}
  \end{align}
%
%
We shall work in the periodic case first, and later generalize our results to
the non-periodic case. The $B_4$ ivp
\eqref{eqn:mb-2}-\eqref{eqn:mb-init-data-2} can be rewritten in the integral form
%
%
%
\begin{equation}
  \begin{split}
    u(x,t)
    & = \frac{1}{2\pi}\sum_{n \in \zz} e^{inx} \wh{u_{0}}(n) \frac{e^{in^{2}t} + e^{-in^{2}t}}{2} 
    \\
    & + \frac{1}{2 \pi}\sum_{n \in \zz} e^{inx}
    \wh{u_{1}}(n)\frac{e^{in^{2}t} - e^{-in^{2}t}}{2 i n^{2}} 
    \\
    & + \frac{1}{4 i \pi}\sum_{n \in \zz} e^{inx}
    \int_{0}^{t}[e^{in^{2}(t-t')}-e^{-in^{2}(t-t')}]
    \wh{u^{2}}(n, t') dt'.
  \end{split}
  \label{eqn:integral-form}
\end{equation}
%
where
%
%
\begin{equation*}
\begin{split}
  \frac{e^{in^{2}t} - e^{-in^{2}t}}{2 i n^{2}} \vert_{n=0} \doteq t 
\end{split}
\end{equation*}
%
%
We now localize in time. Let $\psi(t)$ be a cutoff function symmetric about the 
origin such that $\psi(t) = 1$ for $|t| \le 1$ and $\text{supp} \, \psi 
= [-2, 2 ]$.
Define $\psi_{\delta}(t) = \psi(t/\delta)$.  Multiplying both sides of expression
$\eqref{eqn:integral-form}$ by $\psi_{\delta}(t)$, we obtain
%
%
\begin{align}
  & \psi_{\delta}u(x,t) 
  \label{1hh}
    \\
    & = \frac{1}{2 \pi}\psi_{\delta}(t)
    \sum_{n \in \zz} e^{inx} \wh{u_{0}}(n) \frac{e^{in^{2}t} + e^{-in^{2}t}}{2} 
    \\
    \label{2hh}
    & + \frac{1}{2 \pi}\psi_{\delta}(t) \sum_{n \in \zz} e^{inx}
    \wh{u_{1}}(n)\frac{e^{in^{2}t} - e^{-in^{2}t}}{2 i n^{2}} 
    \\
    \label{3hh}
    & + \frac{1}{4 i \pi}\psi_{\delta}(t) \sum_{n \in \zz} e^{inx}
    \int_{0}^{t}e^{in^{2}(t-t')}
    \wh{w}(n, t') dt'
    \\
    \label{4hh}
    & - \frac{1}{4 i \pi}\psi_{\delta}(t) \sum_{n \in \zz} e^{inx}
    \int_{0}^{t}e^{-in^{2}(t-t')}
    \wh{w}(n, t') dt'
  \end{align}
where $$w(x,t) = \frac{1}{2\pi} \sum_{n \in \zz}
e^{inx} \wh{u^{2}}(n,t).$$  Next, note that for $0 < \delta \le 1$, any $u$
solving
\begin{align}
  & u(x,t) 
    \\
    & = \frac{1}{2 \pi}\psi(t)
    \sum_{n \in \zz} e^{inx} \wh{u_{0}}(n) \frac{e^{in^{2}t} + e^{-in^{2}t}}{2} 
    \\
    & + \frac{1}{2 \pi}\psi(t) \sum_{n \in \zz} e^{inx}
    \wh{u_{1}}(n)\frac{e^{in^{2}t} - e^{-in^{2}t}}{2 i n^{2}} 
    \\
    \label{term-3}
    & + \frac{1}{4 i \pi} \psi_{\delta}(t) \sum_{n \in \zz} e^{inx}
    \int_{0}^{t}e^{in^{2}(t-t')}
    \wh{w}(n, t') dt'
    \\
    \label{term-4}
    & - \frac{1}{4 i \pi} \psi_{\delta} (t) \sum_{n \in \zz} e^{inx}
    \int_{0}^{t}e^{-in^{2}(t-t')}
    \wh{w}(n, t') dt'
  \end{align}
  solves \eqref{1hh}-\eqref{4hh} for $| t | \le \delta$. Furthermore, replacing
  $w$ with $w_{\delta} \doteq \psi_{2 \delta} w$ yields expressions equivalent
  to \eqref{term-3} and \eqref{term-4}, respectively. 
  Following the decomposition method of Bourgain, and neglecting $\pi$ related
  constants, the localized integral equation above is equivalent to 
%
%
\begin{align}
  & u(x,t)
  \label{main1-rel-term-0}
  \\
  \label{main1-rel-term-1}
  & = \psi(t) \sum_{n \in \zz} e^{inx} \wh{u_{0}}(n) \frac{e^{in^{2}t} + e^{-in^{2}t}}{2} 
  \\
  \label{main1-rel-term-2}
  & + \psi(t) \sum_{n \in \zz} e^{inx}
  \wh{u_{1}}(n)\frac{e^{in^{2}t} - e^{-in^{2}t}}{2 i n^{2}} 
  \\
  \label{main1-rel-term-3}
  & +  \psi_{\delta}(t)\sum_{\alpha =  \pm 1} \sum_{n\in \zz} \int_\rr e^{ixn}  
  e^{it \tau} \frac{1 - \psi(\tau -  \alpha n^{2}) 
}{\tau -  \alpha n^{2}} \wh{w_{\delta}}(n, \tau) \ d \tau
  \\
  \label{main1-rel-term-4}
  & + \psi_{\delta}(t) \sum_{\alpha =  \pm 1} \sum_{n\in \zz} \int_\rr e^{i(xn + 
  t \alpha n^{2})}
  \frac{1- \psi(\tau -  \alpha n^{2})}{\tau -  \alpha n^{2}} \wh{w_{\delta}}(n, \tau) \ d \tau
  \\
  \label{main1-rel-term-4.5}
  & +  \psi_{\delta}(t) \sum_{\alpha =  \pm 1}  \sum_{k \ge 1} \frac{i^k t^k}{k!}
  \sum_{n \in \zz} \int_\rr e^{i(xn + t \alpha n^{2} )}
  \psi(\tau -  \alpha n^{2}) (\tau -  \alpha n^{2})^{k-1} \wh{w_{\delta}}(n, \tau)
  \\
  \label{main1-rel-term-5}
  & \doteq Tu, \quad T=T_{u_0, u_1, \psi, \delta}.
\end{align}
%
%
%
%
%
%
%
%
%
%
%
%
We now introduce the following spaces. 
%
%
\begin{definition}
  Let $\mathcal{Y}$ be the space of functions $F(\cdot)$ such that
  \begin{enumerate}[(I)]
   \item{$F: \ci \times \rr \to \cc$}.
   \item{$F(x, \cdot) \in \mathcal{S}(\rr)$ for each $x \in \ci$}.
   \item{$F(\cdot, t) \in C^{\infty}(\ci)$for each $t \in \rr$}.
  \end{enumerate}
  For $s, b \in \rr$, $X_{s,b}$ denotes the completion of $\mathcal{Y}$ with
  respect to the norm
  %
  %
  \begin{equation}
  \begin{split}
    \|F\|_{X_{s,b}} = \left( \sum_{n \in \zz} (1 + |n|)^{2s} \int_{\rr}
    (1 + | | \tau | - n^{2} |)^{2b} |\wh{F}(n, \tau)|^{2} d \tau\right)^{1/2}.
  \end{split}
  \label{eqn:bous-norm}
  \end{equation}
  %
  \begin{framed}
    %
    %
    \begin{remark}
    Note that the norm here is different than the one used to for the KdV. In
    the KdV case, $b$ has to be equal to $1/2$. Furthermore, to get the embedding
    of the lemma below, one has to add an extra term, thus producing the
    $Y^{s}$ norm of CKSTT\@.
    \label{rem:alternate-space}
    \end{remark}
    %
  \end{framed}
    %
  %
  %
\end{definition}
%
The $X_{s,b}$ spaces have the following important embedding, whose proof is
provided in the appendix.
%
%
%%%%%%%%%%%%%%%%%%%%%%%%%%%%%%%%%%%%%%%%%%%%%%%%%%%%%
%
%
%               Embedding 
%
%
%%%%%%%%%%%%%%%%%%%%%%%%%%%%%%%%%%%%%%%%%%%%%%%%%%%%%
%
%
\begin{lemma}[Lemma 2.3 in~\cite{Farah:2009uq}]
  Let $b > 1/2$. Then $X_{s, b} \subset C(\rr, H^s)$ continuously. That is,
  there exists $c>0$ depending only on $b$ such that
%
%
\begin{equation*}
\begin{split}
  \| u \|_{C(\rr, H^s)} \doteq \sup_{t \in \rr} \| u(t) \|_{H^s} 
  \le c \| u \|_{X_{s,b}}.
\end{split}
\end{equation*}
%
\label{lem:embedding}
\end{lemma}
%
\begin{definition}
  Equip $H^{s} \times H^{s-2}$ with the 
  topology defined by the norm $\|(u_{0}, u_{1})\|_{H^{s} \times H^{s-2}}
  = \|u_{0}\|_{H^{s}} + \|u_{1}\|_{H^{s-2}}$.
   We say that the $B_{4}$ ivp
  \eqref{eqn:mb-2}-\eqref{eqn:mb-init-data-2} is
	\emph{locally well posed} in
  $H^s \times H^{s-2}$ if 
	\begin{enumerate}
    \item Given $(u_{0}, u_{1}) \in H^{s} \times H^{s-2}$
      there exists $\delta>0$ depending on $(u_{0}, u_{1})$
      such that the Cauchy problem
      $\psi_{\delta} u = Tu$ has a solution $u \in C([-\delta,
      \delta], H^s) \cap X_{s,b}$ for $ |t| \le \delta$.
    \item The solution is unique in $C([-\delta, \delta], H^{s}) \cap
      X_{s,b}$.
    \item
      The data to solution map $(u_0, u_{1}) \mapsto u(x,t)$ is continuous. That
      is, given a sequence $\{(u_{0,n}, u_{1,n} ) \} \in H^{s} \times H^{s-2}$
      such that $$\|(u_{0}, u_{1})
      - (u_{0,n}, v_{1,n}) \|_{H^{s} \times
      H^{s-2}} \to 0,$$ with corresponding solutions $u_{n} \in
      C([-\delta_{n},
      \delta_{n}])$ and $u \in C([-\delta_{\infty}, \delta_{\infty}])$
      then there exists $0 < \delta \le \inf\left\{
      \delta_{n}, \delta_{\infty} \right\}$ such that $\psi_{\delta}u_{n} =
      Tu_{n}, \psi_{\delta}u = Tu$ and 
      $$\sup_{t \in [-\delta, \delta]}
      \|u(\cdot, t) - u_{n}(\cdot, t) \|_{H^s} \to 0.$$
  \end{enumerate}
	Otherwise, we say that the $B_{4}$ ivp is \emph{ill-posed}.
\end{definition}
%

%
We are now prepared to state the main result.
%
%
%
%
%%%%%%%%%%%%%%%%%%%%%%%%%%%%%%%%%%%%%%%%%%%%%%%%%%%%%
%
%
%	Main Result				
%
%
%%%%%%%%%%%%%%%%%%%%%%%%%%%%%%%%%%%%%%%%%%%%%%%%%%%%%
%
%
\begin{theorem}
\label{thm:main}
The initial value problem 
\eqref{eqn:mb-2}-\eqref{eqn:mb-init-data-2} is locally well-posed in $H^s$ for
$s >
-1/4$ and ill-posed for $s < -2 \ ?$ in both the periodic and non-periodic cases.
%
%
\end{theorem} 
%
%
%
%
%%%%%%%%%%%%%%%%%%%%%%%%%%%%%%%%%%%%%%%%%%%%%%%%%%%%%
%
%
%                Proof of Thm
%
%
%%%%%%%%%%%%%%%%%%%%%%%%%%%%%%%%%%%%%%%%%%%%%%%%%%%%%
%
%
%
%
%
%
\section{Well-Posedness for $s > -1/4$ in the Periodic Case}
%
%
%
%
\subsection{Proof of Existence and Uniqueness}
\label{sec:proof-b4-per-case}
%
%
%%%%%%%%%%%%%%%%%%%%%%%%%%%%%%%%%%%%%%%%%%%%%%%%%%%%%
%
%% Contraction Proposition
%				 
%%%%%%%%%%%%%%%%%%%%%%%%%%%%%%%%%%%%%%%%%%%%%%%%%%%%%%
%
The following bilinear
estimate is crucial to the proof.
%
%
%%%%%%%%%%%%%%%%%%%%%%%%%%%%%%%%%%%%%%%%%%%%%%%%%%%%%
%
%
%				Proposition
%
%
%%%%%%%%%%%%%%%%%%%%%%%%%%%%%%%%%%%%%%%%%%%%%%%%%%%%%
%
%
\begin{proposition}[Theorem 1.1 in~\cite{Farah:2009uq}]
\label{prop:bilinear-est}
	%
	%
	If $a > 1/4$, $b > 1/2$, and $s \ge -a/2$, 
  then 
	\begin{equation}
    \| w_{fg} \|_{X_{s,-a}}
		    \le c_{a} \|f\|_{X_{s,b}} \|g\|_{X_{s,b}}
	\end{equation}
  where $w_{fg}(x,t)$ = $fg (x,t)$.
%
%
%
%
\end{proposition}
%
Estimating the $X_{s,b}$ norms of the linear terms of the $B_{4}$ localized
integral ivp, and applying \cref{prop:bilinear-est} when estimating the
$X_{s,b}$ norm of the non-linearity,
one can show the following.
%
\begin{proposition}
\label{prop:contraction}
%
For $1/2 < b \le 1$, $s \ge -1/4 + 2(b -1/2)$, $0 < \delta \le 1$, we have
%
%%
\begin{equation*}
	\begin{split}
    \|Tu\|_{X_{s,b}} \le c \left( \|u_0 \|_{H^s(\ci)} + \|u_1 \|_{H^{s-2}(\ci)}
    + \delta^{b-1/2} \|u\|_{X_{s,b}}^2 
		\right)
	\end{split}
\end{equation*}
%
where $c = c_{\psi, b} > 1$.  
%%
\end{proposition}

%%
%
We will now use \cref{prop:contraction} and the bilinear estimate
to prove local well-posedness for the 
$B_4$ ivp. Suppose
%
%%
\begin{equation*}
	\begin{split}
    \|(u_0, u_{1})\|_{H^s(\ci) \times H^{s-2}(\ci)} \le r.
  \end{split}
\end{equation*}
%
%%
Then $$\|u\|_{X_{s,b}} \le 2rc$$ implies
%
%%
\begin{equation*}
	\begin{split}
		\|Tu \|_{X_{s,b}} 
    & \le c \left[ r + \delta^{b - 1/2} \left( 
		2rc \right)^2 \right].
	\end{split}
\end{equation*}
%
Choosing 
%
%
\begin{equation}
  \label{delta-suf-small}
\begin{split}
  \delta \le \left (\frac{1}{8rc^{2}} \right )^{\frac{1}{b - 1/2}}
\end{split}
\end{equation}
%
%
we obtain 
%%
%
%
%
\begin{equation*}
\begin{split}
\|Tu \|_{X_{s,b}} 
    & \le \frac{3}{2}rc.
  \end{split}
\end{equation*}
%
%
Hence, $T=T_{u_0, u_1, \psi, \delta}$ maps the ball $B_{X_{s,b}}(2rc)$ into
itself. Next, note that 
%
%%
\begin{equation*}
	\begin{split}
    Tu - Tv = \eqref{main1-rel-term-3} + \eqref{main1-rel-term-4} +
    \eqref{main1-rel-term-5} 
  \end{split}
  \label{eqn:integral-form-dif}
\end{equation*}
%
%%
where now $w_{\delta}(x,t) =\psi_{2 \delta}(u^{2} - v^{2})$. Rewriting
%
%%
\begin{equation*}
	\begin{split}
	u^2 - v^2
		& = (u-v)(u+v)
		\end{split}
\end{equation*}
%
%%
and repeating earlier arguments, we obtain
%
%%
%%
\begin{equation}
	\label{20a}
	\begin{split}
		\|Tu - Tv \|_{X_{s,b}}  
    & \le c \delta^{b - 1/2}\|u -v\|_{X_{s,b}} \|u + v \|_{X_{s,b}}
		\\
    & \le c \delta^{b -1/2} \|u -v\|_{X_{s,b}} (\|u\|_{X_{s,b}}+ \|v \|_{X_{s,b}}).
	\end{split}
\end{equation}
%
%%
If $$ u, v \in B_{X_{s,b}} \left (2rc \right )$$ then choosing $\delta$ as in
\eqref{delta-suf-small}, we obtain
%
%%
\begin{equation}
	\label{21a}
	\begin{split}
		\|Tu - Tv \|_{X_{s,b}}
    & \le c \delta^{b-1/2} \|u -v \|_{X_{s,b}} \left( 2rc + 
		2rc \right)
		\\
		& = \frac{1}{2} \|u -v \|_{X_{s,b}}. 
	\end{split}
\end{equation}
%
%%
We conclude that $T$ is a contraction on the ball $B_{X_{s,b}}(2rc)$.
A Picard iteration then yields a unique 
$u \in X_{s,b}$ satisfying $u = Tu$. Applying
\cref{lem:embedding}, it follows that $u(x,t) \subset C( [-\delta, \delta], H^s)
\cap X_{s,b}$ is a unique
solution of the $B_{4}$ ivp \eqref{eqn:mb-2}-\eqref{eqn:mb-init-data-2} for $t
\in [-\delta, \delta]$.
%
%
%
%
%
%
\subsection{Proof of Lipschitz Continuity} 
\label{sec:lip-continuity}
%
%
%
%
Let $$(u_0, u_1), (v_0, v_1)  \subset
B_{H^{s} \times H^{s-2}} \left (r \right ).$$ Then for $\delta$ sufficiently
small (i.e.\ satisfying \eqref{delta-suf-small}), there exist $u, v \in
X_{s,b}$ such that $u =
T_{u_0, u_1}u$, $v = T_{v_0, v_1} v$, and so
%
%
\begin{align}
  \notag
    & T_{u_0, u_1}(u) - T_{v_0, v_1}(v)
		\\
    & = \psi(t) \sum_{n \in \zz} e^{inx} \wh{u_{0} - v_{0}}(n) \frac{e^{in^{2}t} + e^{-in^{2}t}}{2} 
\label{main1-rel-term-1g}
  \\
  & + \psi(t) \sum_{n \in \zz} e^{inx}
  \wh{u_{1} - v_{1}}(n)\frac{e^{in^{2}t} - e^{-in^{2}t}}{2 i n^{2}} 
\label{main1-rel-term-2g}
  \\
  & + \psi_{\delta}(t) \sum_{\alpha =  \pm 1} \sum_{n\in \zz} \int_\rr e^{ixn}  
  e^{it \tau} \frac{1 - \psi(\tau -  \alpha n^{2}) 
}{\tau -  \alpha n^{2}} \wh{w_{\delta}}(n, \tau) \ d \tau
\label{main1-rel-term-3g}
  \\
  & + \psi_{\delta}(t) \sum_{\alpha =  \pm 1} \sum_{n\in \zz} \int_\rr e^{i(xn + 
  t \alpha n^{2})}
  \frac{1- \psi(\tau -  \alpha n^{2})}{\tau -  \alpha n^{2}} \wh{w_{\delta}}(n, \tau) \ d \tau
\label{main1-rel-term-4g}
  \\
  & + \psi_{\delta}(t) \sum_{\alpha =  \pm 1}  \sum_{k \ge 1} \frac{i^k t^k}{k!}
  \sum_{n \in \zz} \int_\rr e^{i(xn + t \alpha n^{2} )}
  \psi(\tau -  \alpha n^{2}) (\tau -  \alpha n^{2})^{k-1} \wh{w_{\delta}}(n, \tau)
  \label{main1-rel-term-5g}
\end{align}
%
where now $w_{\delta} = \psi_{2 \delta}(u^{2} - v^{2})$. One can show 
%
\begin{equation}
	\label{gen-2a}
	\begin{split}
    & \| \eqref{main1-rel-term-1g}\|_{X_{s,b}}
		\le c \|u_0 -v_0\|_{H^s},
    \\
    & \| \eqref{main1-rel-term-2g}\|_{X_{s,b}}
    \le c \|u_1 -v_1\|_{H^{s-2}}.
	\end{split}
\end{equation}
%
%
%
%
Therefore, from \eqref{21a} and \eqref{gen-2a}, we obtain
%
%
\begin{equation*}
	\begin{split}
    \|u -v \|_{X_{s,b}}
    & = \|T_{u_0, v_0}(u) - T_{u_1, v_1}(v) \|_{X_{s,b}}
    \\
    & \le
    c \left( \|u_0 -v_0 \|_{H^s\left( \ci \right)} +\|u_1 -v_1
        \|_{H^{s-2}\left( \ci \right)} \right )
        + \frac{1}{2} \|u -v \|_{X_{s,b}}
  \end{split}
\end{equation*}
%
%
which implies
%
%
\begin{equation*}
	\begin{split}
		\frac{1}{2} \|u-v\|_{X_{s,b}} \le
    c \left( \|u_0 -v_0 \|_{H^s\left( \ci \right)} +\|u_1 -v_1
        \|_{H^{s-2}\left( \ci \right)} \right )
      \end{split}
\end{equation*}
%
%
or
%
%
\begin{equation}
	\begin{split}
		\|u -v \|_{X_{s,b}} \le 2 c \left( \|u_0 -v_0 \|_{H^s\left( \ci \right)} +\|u_1 -v_1
        \|_{H^{s-2}\left( \ci \right)} \right ).
	\end{split}
  \label{pre-lem-estimate}
\end{equation}
%
%
Applying \cref{lem:embedding} to \cref{pre-lem-estimate}, it follows that
for $(u_0, u_1), (v_0, v_1)  \subset
B_{H^{s} \times H^{s-2}} \left (r \right )$, the
associated solutions $u, v \in C([-\delta, \delta], H^{s}(\ci))$ satisfy the estimate%
%
%
	 %
	 %
	 \begin{equation*}
		 \begin{split}
       \sup_{t \in [-\delta, \delta]} \|u(\cdot, t) -v(\cdot, t) \|_{H^s(\ci)} \le
      2 c \left( \|u_0 -v_0 \|_{H^s\left( \ci \right)} +\|u_1 -v_1
        \|_{H^{s-2}\left( \ci \right)} \right ).
		 \end{split}
	 \end{equation*}
	 %
	 %
Hence, the data to solution map is Lipschitz continuous from $B_{H^{s}
\times H^{s-2}} \left (r \right )$ to $C([-\delta, \delta],
H^{s}(\ci))$, where $\delta = \delta(r)$. This
concludes the proof of well-posedness for the $B_4$ ivp
\eqref{eqn:mb-2}-\eqref{eqn:mb-init-data-2}. \qed %
%%%%%%%%%%%%%%%%%%%%%%%%%%%%%%%%%%%%%%%%%%%%%%%%%%%%%
%
%
%                Proof of Bilinear Estimate B4 Per
%
%
%%%%%%%%%%%%%%%%%%%%%%%%%%%%%%%%%%%%%%%%%%%%%%%%%%%%%
%
%
\section{Well-Posedness for $s > -1/4$ in the Non-Periodic Case} 
\label{sec:non-periodic-case}
%
%
\begin{definition}
  Let $S(\rr^{2})$ denote the space of Schwartz functions on
  $\rr^{2}$.  For $s, b \in \rr$, $\mathcal{X}_{s,b}$
  denotes the completion of $S(\rr^{2})$ with
  respect to the norm
  %
  %
  \begin{equation}
  \begin{split}
    \|F\|_{\mathcal{X}_{s,b}} = \left( \int_{\rr} \int_{\rr} (1 + \xi^{2})^{s}
    (1 + | | \tau | - \xi^{2} |) \wh{F}(n, \tau) d \tau d \xi \right)^{1/2}.
  \end{split}
  \label{eqn:bous-norm-real}
  \end{equation}
  %
  %
  %
  %
\end{definition}
%
These spaces have the following important embedding.
\begin{lemma}
  Let $b > 1/2$. Then $\mathcal{X}_{s, b} \subset C(\rr, H^s)$ continuously. That is,
  there exists $c>0$ depending only on $b$ such that
%
%
\begin{equation*}
\begin{split}
  \| u \|_{C(\rr, H^s)} \doteq \sup_{t \in \rr} \| u(t) \|_{H^s} 
  \le c \| u \|_{\mathcal{X}_{s,b}}.
\end{split}
\end{equation*}
%
\label{lem:real-embedding}
\end{lemma}

%
Hence, to complete the proof of well-posedness in the non-periodic case, we need
only establish the following bilinear estimate. All other arguments are
analogous to those in the periodic case.
%
\begin{proposition}[Theorem 1.1 in Farah nonperiodic]
\label{prop:bilin-est-real}
If $b > 1/2$, $a > 1/4$, and $s \ge -a/2$, 
  then there exists $c > 0$ depending only on $a$, $b$, and $s$ such that
  %
  %
  \begin{equation*}
  \begin{split}
    \| uv \|_{\mathcal{X}_{s,-a}} \le c \| u \|_{\mathcal{X}_{s,b}} \| v \|_{\mathcal{X}_{s,b}}.
  \end{split}
  \end{equation*}
  %
  %
\end{proposition}
%
%
%%%%%%%%%%%%%%%%%%%%%%%%%%%%%%%%%%%%%%%%%%%%%%
%
%
%
%Ill-Posedness
%
%
%
%%%%%%%%%%%%%%%%%%%%%%%%%%%%
%
%
\section{Ill-Posedness for $s < -2 ?$ in the Periodic Case}
Our motivation will be the work of Bejanaru and Tao
\cite{Bejenaru-Tao-2006-Sharp-well-posedness-and-ill-posedness}. 
%
\begin{definition}
  For $f =f(x)= (f_{1}(x), f_{2}(x)), u = u(x,t), v = v(x,t)$ formally define 
%
%
\begin{equation*}
\begin{split}
  L(f)
  \doteq \frac{1}{2 \pi} \psi(t) \sum_{n \in \zz} e^{inx}
  \wh{f_{1}}(n) \frac{e^{in^{2}t} + e^{-in^{2}t}}{2} 
  & + \psi(t) \sum_{n \in \zz} e^{inx}
  \wh{f_{2}}(n)\frac{e^{in^{2}t} - e^{-in^{2}t}}{2 i n^{2}} 
\end{split}
\end{equation*}
%
%
and
%
%
\begin{equation*}
\begin{split}
N(u, v)
& \doteq \frac{1}{4 \pi i} \psi_{\delta}(t) \sum_{n \in \zz} e^{inx}
    \int_{0}^{t}[e^{in^{2}(t-t')}-e^{-in^{2}(t-t')}]
    \wh{uv}(n, t') dt'.
\end{split}
\end{equation*}
%
%
Then we let $A_{n}: H^{s} \times H^{s-2} \to C([0, \delta], H^{s}), \ n = 1, 2, \dots$ be the
recursively defined maps
%
%
\begin{equation*}
\begin{split}
  & A_{1}(f) \doteq L(f),
  \\
  & A_{n}(f) \doteq \sum_{j, k \in \mathbb{N}: j + k = n} N\left[
  A_{j}(f), A_{k}(f) \right], \quad n > 1.
\end{split}
\end{equation*}
\end{definition}
%
%
%
For example, 
%
%
\begin{equation*}
\begin{split}
  & A_{2}(f_{N}) = N(A_{1}(f_{N}), A_{1}(f_{N}))
  \\
  & A_{3}(f_{N}) = N(A_{1}(f_{N}), A_{2}(f_{N})) + N(A_{2}(f_{N}), A_{2}(f_{N}))
  \\
  & A_{4}(f_{N})= N(A_{1}(f_{N}), A_{3}(f_{N})) + N(A_{2}(f_{N}), A_{2}(f_{N}))
  + N(A_{3}(f_{N}), A_{1}(f_{N}))
\end{split}
\end{equation*}
%
%
and so on. The key tool in our proof of ill-posedness is that for initial data
in $H^{s} \times H^{s-2}$, $s > -1/4$, solutions to the
$B_{4}$ ivp can be asymptotically expanded as the sum of the iterates defined
above. More precisely, we have the following.
%%
\begin{lemma}
  \label{lem:qwp-awp}
  For $f \in B_{H^{s} \times H^{s-2}}(r)$, $s > -1/4$ and $\delta=\delta(r)$
sufficiently small, we have the absolutely convergent
(in $C([0, \delta], H^{s})$) power series expansion
%
%
\begin{equation}
  \label{power-series-soln}
\begin{split}
  u[f] = \sum_{n=1}^{\infty} A_{n}(f).
\end{split}
\end{equation}
%
%
\label{lem:analytic-wp}
\end{lemma}
%
%
From this, it follows that we have failure of continuity at $\vec{0} =
(0, 0)$, due to the
following.
%
%
%
%
%
%%%%%%%%%%%%%%%%%%%%%%%%%%%%%%%%%%%%%%%%%%%%%%%%%%%%%
%
%
%                Failure of Continuity at 0
%
%
%%%%%%%%%%%%%%%%%%%%%%%%%%%%%%%%%%%%%%%%%%%%%%%%%%%%%
%
%
\begin{theorem}
  Let $N$ be a positive integer, and $s < -2$. For initial data $$f_{N, s, \ee}(x) =  \left ( \frac{N^{-s-\ee}e^{iNx}}{2
  \pi}, 0 \right ) \in
  H^{s + \ee} \times H^{s+ \ee-2},$$ we have
  %
  %
  \begin{equation*}
  \begin{split}
    & f_{N,s,\ee} \to 0 \ \text{in} \ H^{s} \times H^{s-2}, \
    \text{but} 
    \\
    & A_{2}[f_{N,s,\ee}] \to \infty \ \text{in} \ C([0, \delta], H^{s}).
  \end{split}
  \end{equation*}
  %
  %
\label{thm:ill-pos}
\end{theorem}
%
%
%
%
%
%
%
%
%
%
%
\begin{proof}
  The method of proof will be to bound $A_{2}(f_{N,s,\ee})$ from below by a
  constant depending on $N$, and then take $N \to \infty$. Expanding the $A_{2}$
  iterate, we obtain
%
%
\begin{equation*}
\begin{split}
  & \| A_{2}(f_{N,s,\ee}) \|_{C([0, \delta], H^{s})} 
  \\
  & =  \| N[A_{1}(f_{N,s,\ee}), A_{1}(f_{N,s,\ee})] \|_{C([0, \delta],
  H^{s})} 
  \\
  & = \sup_{0 \le t \le \delta} \| \psi_{\delta}(t) (1 + | n |)^{s}
  \frac{1}{4 i \pi}
  \int_{0}^{t} \left( e^{in^{2}(t-t')} - e^{-in^{2}(t-t')} \right)
  \wh{[L(f_{N,s,\ee})]^{2}}(n, t') dt' \|_{\ell^{2}_{n}}
  \\
  & = \sup_{0 \le t \le \delta} \| (1 + | n |)^{s} \frac{1}{4 i \pi} 
  \int_{0}^{t} \sum_{n_{1}}
  \\
  & \left( e^{in^{2}(t-t')} - e^{-in^{2}(t-t')} \right)
  \wh{L(f_{N,s,\ee})}(n - n_{1}, t')\wh{L(f_{N,s,\ee})}(n_{1}, t') dt'
  \|_{\ell^{2}_{n}}
  \\
  &= \sup_{0 \le t \le \delta} \frac{1}{16 i \pi^{2}}\| (1 + | n |)^{s}
  \int_{0}^{t} \sum_{n_{1}} \left( e^{in^{2}(t-t')} - e^{-in^{2}(t-t')} \right)
  \\
  & \times \left[ \frac{\wh{f_{N,1,s,\ee}}(n - n_{1})\left( e^{i(n - n_{1})^{2}t'} +
  e^{-i(n - n_{1})^{2}t'} \right)}{2} \right ]
  \\
  & \times \left[ \frac{\wh{f_{N,1,s,\ee}}(n_{1})\left( e^{in_{1}^{2}t'} +
  e^{-in_{1}^{2}t'} \right)}{2}  \right]
  dt' \|_{\ell^{2}_{n}}
\end{split}
\end{equation*}
%
which is equal to
\begin{equation}
  \label{pre-loc}
\begin{split}
& \sup_{0 \le t \le \delta} \frac{1}{8 \pi^{2}}\| (1 + | n |)^{s}
\int_{0}^{t} \sum_{n_{1}} \sin[n^{2}(t-t')] \cos[(n - n_{1})^{2}t']
\cos(n_{1}^{2}t') \wh{f_{N,1,s,\ee}}(n - n_{1})\wh{f_{N,1,s,\ee}}(n_{1}) dt'
\|_{\ell_{2}}.
\end{split}
\end{equation}
Using the inequality $\| a_{n} \|_{\ell^{2}} \ge
|a_{2N}|$ and the fact that
%
%
\begin{equation*}
\begin{split}
  (1 + 2N)^{s} \ge 2^{s}(1 + N)^{s} \ge
  4^{s} N^{s}, \qquad s \le 0
\end{split}
\end{equation*}
%
%
we bound \eqref{pre-loc} below by
%
%
\begin{equation}
  \label{yut}
\begin{split}
  & \frac{4^{s}}{8\pi^{2}} N^{s} \sup_{0 \le t \le \delta} 
  | \int_{0}^{t} \sum_{n_{1}} \sin[4N^{2}(t-t')] \cos[(2N -n_{1})^{2}t']
\cos(n_{1}^{2}t') \wh{f_{N,1,s,\ee}}(2N - n_{1})\wh{f_{N,1,s,\ee}}(n_{1}) dt' |.
\end{split}
\end{equation}
%
%
%
%
\begin{framed}
  I have found a way to get rid of the $N^{s}$ decay in front of \eqref{yut}
  by choosing slightly different initial data. However, per your instructions, I
  have decided to just give you a document that is a copy of what we discussed
  in your office. 
\end{framed}
%
%
%
%
Note that $f_{N, 1, s, \ee} \in C^{\infty}$. In particular
%
%
\begin{equation}
  \label{ill-pos-ce}
\begin{split}
  \wh{f_{N,1, s, \ee}}(n)
  & = N^{-s-\ee} \int_{\ci} e^{-ix(n - N)} dx  
  \\ 
  & = 
  \begin{cases}
     N^{-s-\ee},  \quad  & n = N
    \\
     0, \quad  & \text{otherwise}.
  \end{cases}
\end{split}
\end{equation}
%
Hence
%
%
\begin{equation}
  \label{ill-pos-ced}
\begin{split}
  \| f_{N,1, s, \ee} \|_{H^{s}}
  & = N^{-s -\ee} \left( \sum_{n \in \zz} (1 + | n |)^{2s} |
  \wh{f_{N,1, s, \ee}}(n) |^{2} \right)^{1/2}
  \\
  & \le N^{-\ee}.
\end{split}
\end{equation}
%
\begin{framed}
  \begin{remark}
    The mistake here is that I am using the ill-posedness index $s$ in \eqref{ill-pos-ce} and
    \eqref{ill-pos-ced} in both the definition of the initial data and when
    taking its Sobolev norm, respectively. Substituting $\eta$ for $s$ fixes the
    contradiction. However, we must have $\eta > -1/4$, since $B_{4}$ is
    well-posed for this regime. We don't even know if solutions exist for
    $B_{4}$ for initial data $s \le -1/4$.
  \end{remark}
\end{framed}
%
Hence, 
\begin{equation*}
  \begin{split}
    & (f_{N,1,s, \ee}, 0) \in B_{H^{s} \times H^{s-2}}(1)
    \\
    & (f_{N,1, s, \ee}, 0) \to 0 \ \text{in} \ H^{s} \times H^{s-2}.
  \end{split}
\end{equation*}
Note that from the well-posedness theory for the $B_{4}$ equation,
the associated solutions $u_{N}$ have common lifespan $\delta$. 
\begin{framed}
  \begin{remark}
  This is true only if $s > -1/4$. But we are trying to prove ill-posedness for
  index $s \le -1/4$. Contradiction.
  Again, it was a mistake to use $s$ when defining the
  initial data--it would have been better to use $\eta$, where $\eta > -1/4$. 
  \end{remark}
\end{framed}
Substituting
\eqref{ill-pos-ce} into \eqref{yut}, we obtain
%
%
\begin{equation}
  \label{rxx}
\begin{split}
  r^{2}N^{-s -2 \ee} \sup_{0 \le t \le \delta} 
  | \int_{0}^{t} \sin[4N^{2}(t-t')] [\cos(N^{2}t')]^{2} dt' |.
\end{split}
\end{equation}
%
Let $t = N^{-2}$, where we restrict $N \ge
\delta^{-1/2}$. Then \eqref{rxx} is bounded below by
%
%
%
%
\begin{equation*}
\begin{split}
\frac{1}{4} r^{2}N^{-s -2 \ee}  
\int_{0}^{N^{-2}} \sin[4N^{2}(N^{-2}-t')] dt'
\end{split}
\end{equation*}
%
%
which evaluates to
%
%
\begin{equation*}
\begin{split}
\frac{1}{16} r^{2} N^{-s -2 -2 \ee}  
[1 - \cos 4]  \to \infty, \qquad s < -2 -2\ee
\end{split}
\end{equation*}
%
%
completing the proof.  
\end{proof}
%
%
%
\begin{framed}
%
%
\begin{remark}
  Note the $N^{-2}$ decay coming from the integral in \eqref{rxx}. This is
  the main impediment to proving ill-posedness using the BT method.
  If we could show that $B_{4}$ is well-posed for $s \ge -1$ (by
  well-posed, I mean we still have bilinear estimates on the non-linearity in
  some Banach space), I
  could make the method go through (in essence, by choosing initial data as
  in the case above but with amplitude $N^{1}$. Since I have a product of
  $f_{N,1,s,\ee}$ with $f_{N,1,s,\ee}$ in \eqref{yut}, we would obtain a factor of
  $N^{2}$, which would offset the decay
  factor of $N^{-2}$ coming from the trigonometric terms of the integrand found
  in \eqref{rxx}. This is precisely what BT do in the qNLS proof of
  ill-posedness. However, they have the better well-posedness result of $s \ge
  -1$ for qNLS.
\label{rem:extra-decay}
\end{remark}
%
%
\end{framed}
%
%
%
%%%%%%%%%%%%%%%%%%%%%%%%%%%%%%%%%%%%%%%%%%%%%%%%%%%%%%%%%
%
%
%
%Quadratic NLS
%
%
%%%%%%%%%%%%%%%%%%%%%%%%%%%%%%%%%
%
%
\section{Quadratic NLS: An Equation for which the BT method Works}
%
  We consider the quadratic non-linear Schr\"odinger equation (qNLS)
\begin{equation}\label{nls-quad}
\begin{split}
& i u_t + u_{xx} = u^2\\
& u(x, 0) = f \in H^s(\R) \\
\end{split}
\end{equation}
which has the localized integral formulation 
\begin{equation}\label{uln-2}
 u = L(f) + N(u,u)
 \end{equation}
where $L$ is the linear operator
\begin{equation}\label{Ldef}
  L(f)(t) := \eta(t) e^{it\partial_{xx}} f
\end{equation}
and $N$ is the bilinear operator
\begin{equation}\label{Ndef}
\begin{split}
N(u,v)(t) &:=
\eta(t) e^{it\partial_{xx}} \int_\R a(s) e^{-is\partial_{xx}}(u(s)v(s))\ ds\\
&+ \int_\R a(t-s) e^{i(t-s)\partial_{xx}}(u(s) v(s))\ ds
\end{split}
\end{equation}
%
where $\eta: \R \to \R$ is a smooth bump function such that $\eta(t) = 1$ for
$|t| \leq 1$ and $\eta(t) = 0$ for $|t| > 2$, and $a(t) := \frac{1}{2}
\sgn(t)\eta(t/5)$.  
%
Bejanaru and Tao proved the following well-posedness result.
\begin{theorem}[Local well-posedness in $H^{-1}(\R)$]\label{lwp}
 %
 %
 %
Let $r > 0$ be any radius, and let $B_r$ be the ball
$$ B_r := B_{H^{-1}(\R)}(0,r) := \{ u_0 \in H^{-1}(\R): \| u_0 \|_{H^{-1}(\R)} < r \}.$$
Then there exists a time $T > 0$ (in fact we obtain $T = \max(1, c r^{-1/2} )$ for some absolute constant $c>0$)
and a map $f \mapsto u[f]$ which is continuous from $B_r$ to $C([0,T],
H^{-1}(\rr))$, such that the
restriction of this map to $B_r \cap H^s(\R)$ (with the $H^s(\R)$ topology) maps continuously 
to $C([0,T], H^s(\rr))$ for any $s \geq -1$.  Furthermore, if $f$ lies in
a smooth space, say $B_r \cap H^3(\R)$, then $u[f]$ lies in $C([0, T], H^3) \cap
C^1([0,T], H^1(\rr))$ and
solves the equation \eqref{nls-quad} in the classical sense.   
\end{theorem}
%
%
Furthermore, as in the case of $B_{4}$, solutions to the qNLS have an asymptotic
expansion in terms of iterates of the non-linearity. More precisely, they show
the following. 
%
\begin{lemma}
  \label{lem:qnls-asymp}
  For $f \in B_{H^{s}}(r)$, $s \ge -1$ and $\delta=\delta(r)$
sufficiently small, we have the absolutely convergent
(in $C([0, \delta], H^{s})$) power series expansion
%
%
\begin{equation}
  \label{qnlspower-series-soln}
\begin{split}
  u[f] = \sum_{n=1}^{\infty} A_{n}(f)
\end{split}
\end{equation}
%
%
\end{lemma}
They then invoke this well-posedness result to prove ill-posedness for $s <
-1$ as follows.
%
%
%
%
\subsection{Proof of Ill-Posedness for $s < -1$} 
%
%
%
Fix $s < -1$; we may rescale the lifespan $\delta$ to equal $1$.  Suppose for contradiction
that the solution map $f \mapsto u[f]$ is continuous on $B_r$ (with the
$H^{-1}_x(\R)$ topology) to $C([0, 1], H^{-1}(\rr))$ 
(with the $C([0,1], H^{s}(\rr)$ topology). From 
\cref{lem:qnls-asymp}, we conclude that the quadratic operator
$$ A_2: f \mapsto N_2(Lf, Lf)$$ 
restricted to $[0,1] \times \R$, is continuous from $B_r$ (with the $H^{-1}_x(\R)$ topology) 
to $C([0,1], H^{s}(\rr))$.  In particular, this implies that
$$ \sup_{0 \leq t\leq 1}\| A_2(f_N)(y) \|_{H^{s}(\R)} \to 0$$
whenever $f_N$ is a sequence of functions in $B_r$ which goes to zero in $H^{-1}_x(\R)$ norm.
The left-hand side can be expanded by \eqref{Ldef}, \eqref{Ndef} as
$$ \sup_{0 \leq t \leq 1} 
\| \int_0^t \exp(i(t-t')\partial_{xx})((\exp(it' \partial_{xx}) f_N)^2)\ dt' \|_{H^{s}(\R)}$$
which after taking Fourier transforms becomes
$$\sup_{0 \leq t \leq 1} 
\| \langle \xi \rangle^{s}
\int_0^t \int_\R \exp(-i(t-t')\xi^2) \exp(it' (\xi_1^2 + (\xi-\xi_1)^2) \hat f_N(\xi_1) \hat f_N(\xi-\xi_1)\ d\xi_1 dt'
\|_{L^2_\xi(\R)}.$$
Now let $N > 100$ be a large parameter, and set 
$$ \hat f_N := r N 1_{[-10,10]}(|\xi|-N) / 1000.$$
Then $f_N \in B_r$, and $\|f_N\|_{H^{-1}_x(\R)} \to 0$ as $N \to \infty$.  Thus,
from the well-posedness result \cref{lwp} and the fact that
\cref{lem:qnls-asymp}
holds for
the qNLS, we see that for continuity of the data to solution map to hold for $s < -1$ we must have
\begin{equation}\label{sin}
\sup_{0 \leq t \leq 1} \| \langle \xi \rangle^{s}
\int_0^t \int_\R \exp(-i(t-t')\xi^2) \exp(it' (\xi_1^2 + (\xi-\xi_1)^2) \hat f_N(\xi_1) \hat f_N(\xi-\xi_1)\ d\xi_1 dt'
\|_{L^2_\xi(\R)} \to 0
\end{equation}
as $N \to \infty$.
Now set $t := 1/100N^2$ and localize to the region where $-1 \leq \xi \leq 1$.  One can verify that
$$ \Re( \exp(-i(t-t')\xi^2) \exp(it' (\xi_1^2 + (\xi-\xi_1)^2) ) > 1/2$$
whenever $0 \leq t' \leq t$ and $\xi_1$ lives in the support of $f_{N}$, hence we obtain
$$ 
\| \langle \xi \rangle^{s}
\int_0^t \int_\R \exp(-i(t-t')\xi^2) \exp(it' (\xi_1^2 + (\xi-\xi_1)^2) \hat f_N(\xi_1) \hat f_N(\xi-\xi_1)\ d\xi_1 dt'
\|_{L^2_\xi(\R)} \geq c r^2$$
for some $c > 0$.  But this contradicts \eqref{sin}. \qed
%%%%%%%%%%%%%%%%%%%%%%%%%%%%%%%%%%%%%%%%%%%%%%%%%%%%%
%
%
%
\begin{framed}
%
%
\begin{remark}
  The problem with $B_{4}$ is that unlike the quadratic
  Schr{\"o}dinger case, a factor of $N^{-\alpha}$, where $\alpha > 0 $ depends on
  the choice of initial data, keeps popping up. This is extra decay than that
  found in the qNLS case, with the effect that it becomes more
  difficult to show that the norm of the iterates for $B_{4}$ do not go to $0$ as $N
  \to \infty$. To compensate for this extra decay, a better well-posedness
  result is required--this allows us to choose initial data with a larger
  amplitude (i.e. $N$ to a larger power). 
\end{remark}
%
%
\end{framed}
%
%
%
%                Ill Posedness improvement
%
%
%%%%%%%%%%%%%%%%%%%%%%%%%%%%%%%%%%%%%%%%%%%%%%%%%%%%%
%
%
%
%%%%%%%%%%%%%%%%%%%%%%%%%%%%%%%%%%%%%%%%%%%%%%%%%%%%%
%
%
%                Ill-pos Line
%
%
%%%%%%%%%%%%%%%%%%%%%%%%%%%%%%%%%%%%%%%%%%%%%%%%%%%%%
%
%
%
% \bib, bibdiv, biblist are defined by the amsrefs package.
\begin{bibdiv}
\begin{biblist}

\bib{Bejenaru-Tao-2006-Sharp-well-posedness-and-ill-posedness}{article}{
      author={Bejenaru, Ioan},
      author={Tao, Terence},
       title={Sharp well-posedness and ill-posedness results for a quadratic
  non-linear schr{\"o}dinger equation},
        date={2006},
     journal={J. Funct. Anal.},
      volume={233},
      number={1},
       pages={228\ndash 259},
  url={http://www.ams.org.proxy.library.nd.edu/mathscinet-getitem?mr=2204680},
}

\bib{Farah:2009uq}{article}{
      author={Farah, Luiz~Gustavo},
       title={Local solutions in sobolev spaces with negative indices for the
  ``good'' boussinesq equation},
        date={2009},
     journal={Comm. Partial Differential Equations},
      volume={34},
      number={1-3},
       pages={52\ndash 73},
  url={http://www.ams.org.proxy.library.nd.edu/mathscinet-getitem?mr=2512853},
}

\bib{Ginibre:1996fk}{article}{
      author={Ginibre, Jean},
       title={Le probl{\`e}me de cauchy pour des edp semi-lin{\'e}aires
  p{\'e}riodiques en variables d'espace (d'apr{\`e}s bourgain)},
        date={1996},
     journal={Ast{\'e}risque},
      number={237},
       pages={Exp.\ No.\ 796, 4, 163\ndash 187},
  url={http://www.ams.org.proxy.library.nd.edu/mathscinet-getitem?mr=1423623},
        note={S{{\'e}}minaire Bourbaki, Vol. 1994/95},
}

\bib{Ginibre:1997fk}{article}{
      author={Ginibre, J},
      author={Tsutsumi, Y},
      author={Velo, G},
       title={On the cauchy problem for the zakharov system},
        date={1997},
     journal={J. Funct. Anal.},
      volume={151},
      number={2},
       pages={384\ndash 436},
  url={http://www.ams.org.proxy.library.nd.edu/mathscinet-getitem?mr=1491547},
}

\bib{Kenig:1996aa}{article}{
      author={Kenig, Carlos~E},
      author={Ponce, Gustavo},
      author={Vega, Luis},
       title={A bilinear estimate with applications to the kdv equation},
        date={1996},
     journal={J. Amer. Math. Soc.},
      volume={9},
      number={2},
       pages={573\ndash 603},
  url={http://www.ams.org.proxy.library.nd.edu/mathscinet-getitem?mr=1329387},
}

\bib{Kenig-Ponce-Vega-1996-Quadratic-forms-for-the-1-D-semilinear}{article}{
      author={Kenig, Carlos~E},
      author={Ponce, Gustavo},
      author={Vega, Luis},
       title={Quadratic forms for the {\$}1{\$}-d semilinear schr{\"o}dinger
  equation},
        date={1996},
     journal={Trans. Amer. Math. Soc.},
      volume={348},
      number={8},
       pages={3323\ndash 3353},
  url={http://www.ams.org.proxy.library.nd.edu/mathscinet-getitem?mr=1357398},
}

\end{biblist}
\end{bibdiv}
%
% \bib, bibdiv, biblist are defined by the amsrefs package.
%\bibliography{/Users/davidkarapetyan/math/bib-files/references.bib}
%
%\nocite{*}
\end{document}
