%
\documentclass[12pt,reqno]{amsart}
\usepackage{amssymb}
\usepackage{amsmath} 
\usepackage{cancel}  %for cancelling terms explicity on pdf
\usepackage{yhmath}   %makes fourier transform look nicer, among other things
\usepackage{framed}  %for framing remarks, theorems, etc.
\usepackage{enumerate} %to change enumerate symbols
\usepackage{hyperref}
%\usepackage{showkeys}  %shows source equation labels on the pdf
\usepackage[margin=3cm]{geometry}  %page layout
%\usepackage[pdftex]{graphicx} %for importing pictures into latex--pdf compilation
%\setcounter{secnumdepth}{1} %number only sections, not subsections
\hypersetup{colorlinks=true,
linkcolor=blue,
citecolor=blue,
urlcolor=blue,
}
\synctex=1
\numberwithin{equation}{section}  %eliminate need for keeping track of counters
\numberwithin{figure}{section}
\setlength{\parindent}{0in} %no indentation of paragraphs after section title
\renewcommand{\baselinestretch}{1.1} %increases vert spacing of text
%
%
\newcommand{\ds}{\displaystyle}
\newcommand{\ts}{\textstyle}
\newcommand{\nin}{\noindent}
\newcommand{\rr}{\mathbb{R}}
\newcommand{\nn}{\mathbb{N}}
\newcommand{\zz}{\mathbb{Z}}
\newcommand{\cc}{\mathbb{C}}
\newcommand{\ci}{\mathbb{T}}
\newcommand{\zzdot}{\dot{\zz}}
\newcommand{\wh}{\widehat}
\newcommand{\p}{\partial}
\newcommand{\ee}{\varepsilon}
\newcommand{\vp}{\varphi}
\newcommand{\wt}{\widetilde}
%
%
\theoremstyle{plain}  
\newtheorem{theorem}{Theorem}
\newtheorem{proposition}{Proposition}
\newtheorem{lemma}{Lemma}
\newtheorem{corollary}{Corollary}
\newtheorem{claim}{Claim}
\newtheorem{conjecture}[subsection]{conjecture}
%
\theoremstyle{definition}
\newtheorem{definition}{Definition}
%
\theoremstyle{remark}
\newtheorem{remark}{Remark}
%
%
%\newtheorem{theorem}{Theorem}[section]
%\newtheorem{lemma}[theorem]{Lemma}
%\newtheorem{corollary}[theorem]{Corollary}
%\newtheorem{claim}[theorem]{Claim}
%\newtheorem{prop}[theorem]{Proposition}
%\newtheorem{proposition}[theorem]{Proposition}
%\newtheorem{no}[theorem]{Notation}
%\newtheorem{definition}[theorem]{Definition}
%\newtheorem{remark}[theorem]{Remark}
%\newtheorem{examp}{Example}[section]
%\newtheorem {exercise}[theorem] {Exercise}
%
\def\makeautorefname#1#2{\expandafter\def\csname#1autorefname\endcsname{#2}}
\makeautorefname{equation}{Equation}
\makeautorefname{footnote}{footnote}
\makeautorefname{item}{item}
\makeautorefname{figure}{Figure}
\makeautorefname{table}{Table}
\makeautorefname{part}{Part}
\makeautorefname{appendix}{Appendix}
\makeautorefname{chapter}{Chapter}
\makeautorefname{section}{Section}
\makeautorefname{subsection}{Section}
\makeautorefname{subsubsection}{Section}
\makeautorefname{paragraph}{Paragraph}
\makeautorefname{subparagraph}{Paragraph}
\makeautorefname{theorem}{Theorem}
\makeautorefname{theo}{Theorem}
\makeautorefname{thm}{Theorem}
\makeautorefname{addendum}{Addendum}
\makeautorefname{add}{Addendum}
\makeautorefname{maintheorem}{Main theorem}
\makeautorefname{corollary}{Corollary}
\makeautorefname{lemma}{Lemma}
\makeautorefname{sublemma}{Sublemma}
\makeautorefname{proposition}{Proposition}
\makeautorefname{property}{Property}
\makeautorefname{scholium}{Scholium}
\makeautorefname{step}{Step}
\makeautorefname{conjecture}{Conjecture}
\makeautorefname{question}{Question}
\makeautorefname{definition}{Definition}
\makeautorefname{notation}{Notation}
\makeautorefname{remark}{Remark}
\makeautorefname{remarks}{Remarks}
\makeautorefname{example}{Example}
\makeautorefname{algorithm}{Algorithm}
\makeautorefname{axiom}{Axiom}
\makeautorefname{case}{Case}
\makeautorefname{claim}{Claim}
\makeautorefname{assumption}{Assumption}
\makeautorefname{conclusion}{Conclusion}
\makeautorefname{condition}{Condition}
\makeautorefname{construction}{Construction}
\makeautorefname{criterion}{Criterion}
\makeautorefname{exercise}{Exercise}
\makeautorefname{problem}{Problem}
\makeautorefname{solution}{Solution}
\makeautorefname{summary}{Summary}
\makeautorefname{operation}{Operation}
\makeautorefname{observation}{Observation}
\makeautorefname{convention}{Convention}
\makeautorefname{warning}{Warning}
\makeautorefname{note}{Note}
\makeautorefname{fact}{Fact}
%
%


\newcommand{\uol}{u^\omega_\lambda}
\newcommand{\lbar}{\bar{l}}
\renewcommand{\l}{\lambda}
\newcommand{\R}{\mathbb R}
\newcommand{\RR}{\mathcal R}
\newcommand{\al}{\alpha}
\newcommand{\ve}{q}
\newcommand{\tg}{{tan}}
\newcommand{\m}{q}
\newcommand{\N}{N}
\newcommand{\ta}{{\tilde{a}}}
\newcommand{\tb}{{\tilde{b}}}
\newcommand{\tc}{{\tilde{c}}}
\newcommand{\tS}{{\tilde S}}
\newcommand{\tP}{{\tilde P}}
\newcommand{\tu}{{\tilde{u}}}
\newcommand{\tw}{{\tilde{w}}}
\newcommand{\tA}{{\tilde{A}}}
\newcommand{\tX}{{\tilde{X}}}
\newcommand{\tphi}{{\tilde{\phi}}}


\begin{document}
\title{$1$-D ``good'' Boussinesq equation}

\author{Dan-Andrei Geba, Alexandrou Himonas, and David Karapetyan}

\address{Department of Mathematics, University of Rochester, Rochester, NY 14627}
\address{Department of Mathematics, University of Notre Dame, Notre Dame, IN 46556}
\address{Department of Mathematics, University of Notre Dame, Notre Dame, IN 46556}
\date{}

%\begin{abstract}
%\end{abstract}

\subjclass[2000]{35B30, 35Q55, 35Q72}
\keywords{local well-posedness; ill-posedness.}

\maketitle
%
%
\section{Introduction}
%
We consider the initial value problem (ivp) for the fourth order modified Boussinesq
($B_4$) equation 
\begin{gather}
  u_{tt}   + u_{xxxx} + (u^2)_{xx} = 0, \quad x \in \rr \text{ or }
  \ci,\text{ } t \in \rr
  \label{eqn:mb-2}
  \\
  u(x,0) = u_{0}(x), \quad \p_t u(x, 0) = u_1(x), 
  \label{eqn:mb-init-data-2}
  \\
  \notag
  (u_0, u_1) \in
  H^{s}\times
  H^{s-2}
\end{gather}
%
%
We shall work in the periodic case first, and later generalize our results to
the non-periodic case. We first rewrite the $B_4$ ivp
\eqref{eqn:mb-2}-\eqref{eqn:mb-init-data-2} in integral form. 
Consider
the linear $B_4$ ivp
\begin{gather}
  u_{tt} + u_{xxxx} = 0,
  \label{lin-mb}
  \\
  u(x, 0)=u_{0}(x), \quad u_{t}(x,0) = u_{1}(x).
  \label{lin-mb-init-data-1}
\end{gather}
Taking the spatial Fourier transform of the linear $B_{4}$ ivp
\eqref{lin-mb}-\eqref{lin-mb-init-data-1} yields
%
%
\begin{gather}
  \wh{u_{tt}} + n^{4} \wh{u} = 0
  \label{four-trans-lin-mb}
  \\
  \wh{u}(n, 0) = \wh{u_{0}}(n), \quad \wh{u_{t}}(n, 0) = \wh{u_{1}}(n)
  \label{four-trans-lin-mb-data}
\end{gather}
For $n=0$, this admits the unique solution
%
%
\begin{equation*}
\begin{split}
  \wh{u}(0,t) = \wh{u_{0}}(0) + \wh{u_{1}}(0)t
\end{split}
\end{equation*}
%
%
For the case $n \neq 0$, we substitute the ansatz $e^{\lambda t}$ into
\eqref{four-trans-lin-mb} to obtain the characteristic equation
%
%
\begin{equation*}
\begin{split}
  \lambda^{2} + n^{4} = 0
\end{split}
\end{equation*}
%
%
which gives 
%
%
\begin{equation*}
\begin{split}
  \lambda = \pm in^{2}.
\end{split}
\end{equation*}
%
Since
\eqref{lin-mb} is second order, and $e^{in^{2}t}$ and $e^{-in^{2}t}$ are
linearly independent solutions to \eqref{lin-mb}, it follows that $\left\{
e^{in^{2}t}, e^{-in^{2}t}
\right\}$ is a basis for all solutions of \eqref{lin-mb}. Therefore, the general
solution of \eqref{lin-mb} takes the form
%
%
\begin{equation}
  \label{explicit-homog-soln}
\begin{split}
  \wh{u}(n,t) = c_{1}e^{in^{2}t} + c_{2}e^{-in^{2}t}
\end{split}
\end{equation}
%
%
which, in conjunction with initial data 
\eqref{four-trans-lin-mb-data}, implies
%
%
\begin{gather*}
   c_{1} + c_{2} = \wh{u_{0}}(n)
  \\
   in^{2}c_{1} - in^{2}c_{2} = \wh{u_{1}}(n).
\end{gather*}
%
%
Solving for $c_{1}$ and $c_{2}$, we obtain
%
%
\begin{gather*}
  c_{1} = \frac{1}{2} \wh{u_{0}}(n) + \frac{1}{2in^{2}}\wh{u_{1}}(n),
  \\
  c_{2} = \frac{1}{2} \wh{u_{0}}(n) - \frac{1}{2in^{2}}\wh{u_{1}}(n).
\end{gather*}
%
%
Substituting into \eqref{explicit-homog-soln}, we obtain the unique solution for
$n \neq 0$ to
ivp \eqref{four-trans-lin-mb}-\eqref{four-trans-lin-mb-data}
%
%
\begin{equation}
  \label{lin-mb-four-soln}
\begin{split}
  \wh{u}(n, t) = \wh{u_{0}}(n) \frac{e^{in^{2}t} + e^{-in^{2}t}}{2} +
  \wh{u_{1}}(n)\frac{e^{in^{2}t} - e^{-in^{2}}t}{2 i n^{2}}.
\end{split}
\end{equation}
%
%
Notice that 
\begin{equation}
  \label{remov-sing}
\begin{split}
  \frac{e^{in^{2}t} - e^{-in^{2}t}}{2 i n^{2}} \Big |_{n =0} = t
\end{split}
\end{equation}
%
%
since $n=0$ is a removable singularity. Furthermore, 
%
%
\begin{equation*}
\begin{split}
  \frac{e^{in^{2}t} + e^{-in^{2}t}}{2} \Big |_{n = 0} =1.
\end{split}
\end{equation*}
%
%
Therefore, \eqref{lin-mb-four-soln} is the unique solution to ivp
\eqref{four-trans-lin-mb}-\eqref{four-trans-lin-mb-data} for all $n \in \zz$.  
Hence, taking the inverse spatial Fourier transform of \eqref{lin-mb-four-soln},
we obtain the unique solution to ivp \eqref{lin-mb}-\eqref{lin-mb-init-data-1}
%
%
\begin{equation*}
\begin{split}
  u(x,t)
  & = \frac{1}{ 2 \pi}
  \sum_{n \in \zz}\wh{\vp}(n) \frac{e^{in^{2}t} + e^{-in^{2}t}}{2}
  \\
  & + \frac{1}{2 \pi} \sum_{n \in \zz}
  \wh{\vp}(n) \frac{e^{in^{2}t} - e^{-in^{2}t}}{2i n^{2}}.
\end{split}\end{equation*}
%
%
%
%
%
Turning our attention now to the $B_{4}$ ivp
\eqref{eqn:mb-2}-\eqref{eqn:mb-init-data-2} and taking the spatial Fourier
transform yields 
%
%
\begin{gather}
  \wh{u_{tt}} + n^{4} \wh{u} = -\wh{(u^{2})_{xx}}
  \label{four-trans-mb}
  \\
  \wh{u}(n, 0) = \wh{u_{0}}(n), \quad \wh{u_{t}}(n, 0) = \wh{u_{1}}(n)
  \label{four-trans-mb-data}
\end{gather}
which for fixed $t$ is a second order ODE in $n$. 
We now need the following.
%
%
%%%%%%%%%%%%%%%%%%%%%%%%%%%%%%%%%%%%%%%%%%%%%%%%%%%%%
%
%
%                General Solution NonHomog 2nd Order Eqn
%
%
%%%%%%%%%%%%%%%%%%%%%%%%%%%%%%%%%%%%%%%%%%%%%%%%%%%%%
%
%
\begin{lemma}[Variation of Parameters]
\label{lem:nonhomog-ode-soln}
The general solution of the $2$nd order nonhomogeneous ODE 
%
%
\begin{equation}
  \label{2nd-order-ode}
\begin{split}
y'' + p(t)y' + q(t)y = g(t)
\end{split}
\end{equation}
%
%
can be written in the form
%
%
\begin{equation*}
\begin{split}
  y = c_{1}y_{1}(t) + c_{2}y_{2}(t) + y_{p}(t),
\end{split}
\end{equation*}
%
%
where $y_{1}$ and $y_{2}$ are linearly independent solutions to the
corresponding homogeneous equation (i.e. $g(t) = 0$), $c_{1}$ and $c_{2}$ are
arbitrary constants, and $y_{p}$ is some specific solution of the nonhomogeneous
equation. Furthermore, one such $y_{p}$ is given by
%
%
\begin{equation}
  \label{2nd-order-ansatz}
\begin{split}
  y_{p} = y_{1}v_{1} + y_{2} v_{2}
\end{split}
\end{equation}
%
%
where $v_{1}$ and $v_{2}$ are solutions to the system
\begin{gather}
  \label{cancel-rel-1}
  y_{1} v_{1}' + y_{2} v_{2}' = 0
  \\
  \label{cancel-rel-2}
  y_{1}' v_{1}' + y_{2}' v_{2}' = g(t).
\end{gather}
\end{lemma}
%
{\bf Proof.} Substituting the ansatz \eqref{2nd-order-ansatz} into
the left hand side of \eqref{2nd-order-ode}, we obtain the expression
%
%
%
%
\begin{equation*}
\begin{split}
  (y_{1}v_{1} + y_{2}v_{2})'' + p(t)(y_{1}v_{1} + y_{2}v_{2})' +
  q(t)(y_{1}v_{1} + y_{2}v_{2}) 
\end{split}
\end{equation*}
%
%
or
%
%
\begin{equation*}
  \begin{split}
    & y_{1}'' v_{1} + 2y_{1}'v_{1}' + y_{1}v_{1}'' + y_{2}''v_{2} + 2y_{2}' v_{2}'
  + y_{2} v_{2}'' + p(t)y_{1}'v_{1} + p(t)y_{1}v_{1}'
  \\
  & + p(t)y_{2}'v_{2} + p(t)y_{2}v_{2}' + q(t)y_{1}v_{1} + q(t)y_{2}v_{2} =g
\end{split}
\end{equation*}
%
Collecting terms, this can be rewritten as
%
%
%
%
\begin{equation*}
\begin{split}
  & (y_{1}v_{1}'' + y_{1}' v_{1}') + (y_{2}v_{2}'' + y_{2}' v_{2}') + y_{1}''
  v_{1} + y_{2}'' v_{2} + (y_{1}'v_{1}' + y_{2}' v_{2}')
  \\
  & + p(t)\left(
  y_{1}'v_{1} + y_{2}' v_{2} + y_{1}v_{1}' + y_{2}v_{2}'
  \right) + q(t)\left( y_{1}v_{1} + y_{2}v_{2} \right)
  \end{split}
\end{equation*}
%
or
%
%
%
%
\begin{equation*}
\begin{split}
  & \cancel{(y_{1}v_{1}' + y_{2}v_{2}')'} + y_{1}''
  v_{1} + y_{2}'' v_{2} + \overbrace{(y_{1}'v_{1}' + y_{2}' v_{2}')}^{g}
  \\
  & + p(t)\left(
  y_{1}'v_{1} + y_{2}' v_{2} + y_{1}v_{1}' + y_{2}v_{2}'
  \right) + q(t)\left( y_{1}v_{1} + y_{2}v_{2} \right)
  \end{split}
\end{equation*}
%
%
or
%
%
\begin{equation*}
\begin{split}
  g + \cancel{v_{1}\left[ y_{1}'' + p(t)y_{1}' + q(t)y_{1} \right]} +
  \cancel{v_{2}\left[ y_{2}'' + p(t)y_{2}' + q(t)y_{2}
  \right]}
\end{split}
\end{equation*}
%
%
where the last cancellation is due to the fact that $y_{1}$ and $y_{2}$ are
solutions to the corresponding homogeneous equation of \eqref{2nd-order-ode}
(i.e. $g(t) = 0$). This concludes the proof. \qquad \qedsymbol
%
\\
\\
%
Rewriting the system \eqref{cancel-rel-1}-\eqref{cancel-rel-2}
as
  \begin{equation*}
  \begin{bmatrix}
    y_{1} & y_{2} \\
    y_{1}' & y_{2}'
  \end{bmatrix}
  \begin{bmatrix}
    v_{1}'
    \\
    v_{2}'
  \end{bmatrix}=
  \begin{bmatrix}
  0 \\
  g
  \end{bmatrix}
\end{equation*}
  we obtain 
  \begin{equation*}
\begin{bmatrix}
  v_{1}'
  \\
  v_{2}'
\end{bmatrix}=
\begin{bmatrix}
  -\frac{y_{2}g}{y_{1}y_{2}' - y_{1}' y_{2}} \\
  \frac{y_{1}g}{y_{1}y_{2}' - y_{1}' y_{2}}.
\end{bmatrix}
\end{equation*}
Integrating from $0$ to $t$, and setting $v_{1}(0) = v_{2}(0) = 0$, we see that
one particular solution is
%
%
\begin{equation*}
\begin{split}
\begin{bmatrix}
  v_{1}
  \\
  v_{2}
\end{bmatrix}=
\begin{bmatrix}
 -\int_{0}^{t} \frac{y_{2}g}{y_{1}y_{2}' - y_{1}' y_{2}} dt' \\
  \int_{0}^{t}\frac{y_{1}g}{y_{1}y_{2}' - y_{1}' y_{2}}dt'.
\end{bmatrix}
\end{split}
\end{equation*}
Therefore,
%
%
\begin{equation*}
\begin{split}
  y_{p} =  \int_{0}^{t}
  \frac{ y_{2}(t)y_{1}(t') - y_{1}(t)y_{2}(t')}{y_{1}(t')y_{2}'(t') -
  y_{1}'(t') y_{2}(t')}g \ dt'.
\end{split}
\end{equation*}
%
%
%
%
%
Applying \autoref{lem:nonhomog-ode-soln}, it follows that 
the unique solution to ivp
\eqref{four-trans-mb}-\eqref{four-trans-mb-data} is given by
%
%
\begin{equation*}
\begin{split}
\wh{u}(n, t)
& = \wh{u_{0}}(n) \frac{e^{in^{2}t} + e^{-in^{2}t}}{2} +
  \wh{u_{1}}(n)\frac{e^{in^{2}t} - e^{-in^{2}t}}{2 i n^{2}}
  \\
  & -
  \int_{0}^{t}\frac{e^{in^{2}(t-t')}-e^{-in^{2}(t-t')}}{2in^{2}}
  \wh{(u^{2})_{xx}}(n, t') dt'.
\end{split}
\end{equation*}
%
Hence, taking the inverse Fourier transform we obtain
%
\begin{equation}
  \begin{split}
    u(x,t)
    & = \frac{1}{2\pi}\sum_{n \in \zz} e^{inx} \wh{u_{0}}(n) \frac{e^{in^{2}t} + e^{-in^{2}t}}{2} 
    \\
    & + \frac{1}{2 \pi}\sum_{n \in \zz} e^{inx}
    \wh{u_{1}}(n)\frac{e^{in^{2}t} - e^{-in^{2}t}}{2 i n^{2}} 
    \\
    & - \frac{1}{2 \pi}\sum_{n \in \zz} e^{inx}
    \int_{0}^{t}\frac{e^{in^{2}(t-t')}-e^{-in^{2}(t-t')}}{2 i n^{2}}
    \wh{(u^{2})_{xx}}(n, t') dt'.
  \end{split}
  \label{eqn:pre-integral-form}
\end{equation}
%
Note that \eqref{remov-sing} implies
%
%
\begin{equation*}
\begin{split}
  \frac{e^{in^{2}(t - t')} - e^{-in^{2}(t-t')}}{2 i n^{2}} \Big |_{n=0} = t-t'
\end{split}
\end{equation*}
%
%
But 
%
%
\begin{equation*}
\begin{split}
  \wh{(u^{2})_{xx}}(0, t') = (-n^{2} \wh{u^{2}})(0, t') = 0.
\end{split}
\end{equation*}
%
%
Hence, \eqref{eqn:pre-integral-form} simplifies to
%
\begin{equation}
  \begin{split}
    u(x,t)
    & = \frac{1}{2\pi}\sum_{n \in \zz} e^{inx} \wh{u_{0}}(n) \frac{e^{in^{2}t} + e^{-in^{2}t}}{2} 
    \\
    & + \frac{1}{2 \pi}\sum_{n \in \zz} e^{inx}
    \wh{u_{1}}(n)\frac{e^{in^{2}t} - e^{-in^{2}}t}{2 i n^{2}} 
    \\
    & + \frac{1}{4 i \pi}\sum_{n \in \zz} e^{inx}
    \int_{0}^{t}[e^{in^{2}(t-t')}-e^{-in^{2}(t-t')}]
    \wh{u^{2}}(n, t') dt'.
  \end{split}
  \label{eqn:integral-form}
\end{equation}

%
\begin{framed}
  \begin{remark}
For an alternative technique for formulating the $B_{4}$ ivp in integral form,
see the appendix.
\end{remark}
\end{framed}
Hence, we have rewritten the $B_{4}$ ivp
\eqref{eqn:mb-2}-\eqref{eqn:mb-init-data-2} in integral form, which we will now
localize in time. 
Let $\psi(t)$ be a cutoff function symmetric about the 
origin such that $\psi(t) = 1$ for $|t| \le 1/2$ and $\text{supp} \, \psi 
= [-1, 1 ]$.
Define $\psi_{\delta}(t) = \psi(2t/\delta)$.  Multiplying both sides of expression
$\eqref{eqn:integral-form}$ by $\psi_{\delta}(t)$, we obtain
%
%
\begin{align}
  & \psi_{\delta}u(x,t) 
    \\
    & = \frac{1}{2 \pi}\psi_{\delta}(t)
    \sum_{n \in \zz} e^{inx} \wh{u_{0}}(n) \frac{e^{in^{2}t} + e^{-in^{2}t}}{2} 
    \\
    & + \frac{1}{2 \pi}\psi_{\delta}(t) \sum_{n \in \zz} e^{inx}
    \wh{u_{1}}(n)\frac{e^{in^{2}t} - e^{-in^{2}}t}{2 i n^{2}} 
    \\
    \label{term-3}
    & + \frac{1}{4 i \pi}\psi_{\delta}(t) \sum_{n \in \zz} e^{inx}
    \int_{0}^{t}e^{in^{2}(t-t')}
    \wh{w}(n, t') dt'
    \\
    \label{term-4}
    & - \frac{1}{4 i \pi}\psi_{\delta}(t) \sum_{n \in \zz} e^{inx}
    \int_{0}^{t}e^{-in^{2}(t-t')}
    \wh{w}(n, t') dt'
  \end{align}
where $$w(x,t) = \frac{1}{2\pi} \sum_{n \in \zz}
e^{inx} \wh{u^{2}}(n,t).$$
Since \eqref{term-3} is a \emph{global} relation in $t$, using Fourier inversion
we can rewrite it as
%
%
\begin{equation*}
\begin{split}
  & \frac{1}{8 i \pi^{2} } \psi_{\delta}(t) \sum_{n \in \zz} e^{inx} e^{in^{2}t}
  \int_{0}^{t} \int_{\rr} e^{it'(\tau - n^{2})} \wh{w}(n, \tau) d \tau dt'
\end{split}
\end{equation*}
%
%
which by Fubini and integration is equal to
%
%
\begin{equation*}
\begin{split}
  -\frac{1}{8 \pi^{2}} \psi_{\delta} (t) \sum_{n \in \zz} \int_{\rr} e^{ixn}
  e^{in^{2}t} \frac{e^{it(\tau - n^{2})} -1}{ \tau - n^{2}}\wh{w}(n, \tau) d \tau.
\end{split}
\end{equation*}
%
Next, we localize near the singular curve $\tau =  n^2$.  Multiplying the
integrand by $1 + \psi(\tau -
n^m) - \psi(\tau -
n^m) $ and
rearranging terms, we get
%
%
\begin{equation*}
	\begin{split}
	& - \frac{1}{8 \pi^2} \psi_{\delta}(t) \sum_{n \in \zz} \int_\rr e^{ixn}  
		e^{it \tau} \frac{ 1 - \psi(\tau - n^{2}) 
		}{\tau - n^{2}} \wh{w}(n, \tau) \ d \tau
		\\
		& + \frac{1}{8 \pi^2} \psi_{\delta}(t) \sum_{n \in \zz} \int _\rr e^{i(xn + 
		t n^{2})}
		 \frac{1- \psi(\tau - n^{2})}{\tau - n^{2}} \wh{w}(n, \tau) \ d \tau
		\\
		& - \frac{1}{8 \pi^2} \psi_{\delta}(t) \sum_{n \in \zz} \int_\rr
		e^{i(xn + t n^{2})}
		\frac{\psi(\tau - n^{2})\left[ e^{it(\tau - n^{2})}-1 
		\right]}{\tau - n^{2}} \wh{w}(n, \tau) \ d \tau
	\end{split}
\end{equation*}
%
%
which by a power series expansion of $[e^{it(\tau - n^{2})}-1]$ simplifies  
to
%
%
\begin{align}
		\label{main-int-expression'-2}
		& -\frac{1}{8 \pi^2} \psi_{\delta}(t) \sum_{n\in \zz} \int_\rr e^{ixn}  
		e^{it \tau} \frac{ 1 - \psi(\tau -  n^{2}) 
		}{\tau -  n^{2}} \wh{w}(n, \tau) \ d \tau
		\\
		\label{main-int-expression'-3}
		& + \frac{1}{8 \pi^2} \psi_{\delta}(t) \sum_{n\in \zz} \int_\rr e^{i(xn + 
		t n^{2})}
		 \frac{1- \psi(\tau -  n^{2})}{\tau -  n^{2}} \wh{w}(n, \tau) \ d \tau
		\\
		\label{main-int-expression'-4}
		& - \frac{1}{8 \pi^2} \psi_{\delta}(t) \sum_{k \ge 1} \frac{i^k t^k}{k!}
		\sum_{n \in \zz} \int_\rr e^{i(xn + t n^{2} )}
		\psi(\tau -  n^{2}) (\tau -  n^{2})^{k-1} \wh{w}(n, \tau).
\end{align}
%
Similarly, \eqref{term-4} can be rewritten as
%
\begin{align}
		\label{main-int-expression''-2}
		& \frac{1}{8 \pi^2} \psi_{\delta}(t) \sum_{n\in \zz} \int_\rr e^{ixn}  
		e^{it \tau} \frac{ 1 - \psi(\tau +  n^m) 
		}{\tau +  n^m} \wh{w}(n, \tau) \ d \tau
		\\
		\label{main-int-expression''-3}
		&  - \frac{1}{8 \pi^2} \psi_{\delta}(t) \sum_{n\in \zz} \int_\rr e^{i(xn - 
		t n^2)}
		 \frac{1- \psi(\tau +  n^m)}{\tau +  n^m} \wh{w}(n, \tau) \ d \tau
		\\
		\label{main-int-expression''-4}
		& + \frac{1}{8 \pi^2} \psi_{\delta}(t) \sum_{k \ge 1} \frac{i^k t^k}{k!}
		\sum_{n \in \zz} \int_\rr e^{i(xn - t n^2 )}
		\psi(\tau +  n^2) (\tau +  n^m)^{k-1} \wh{w}(n, \tau).
\end{align}
%
%
Therefore, neglecting $\pi$ related constants, we have
%
%
\begin{align}
  & \psi_{\delta} u(x,t)
  \notag
  \\
  \label{main-rel-term-1}
  & = \psi_{\delta}(t) \sum_{n \in \zz} e^{inx} \wh{u_{0}}(n) \frac{e^{in^{2}t} + e^{-in^{2}t}}{2} 
  \\
  \label{main-rel-term-2}
  & + \psi_{\delta}(t) \sum_{n \in \zz} e^{inx}
  \wh{u_{1}}(n)\frac{e^{in^{2}t} - e^{-in^{2}}t}{2 i n^{2}} 
  \\
  \label{main-rel-term-3}
  & + \psi_{\delta}(t) \sum_{a = \pm 1} \sum_{n\in \zz} \int_\rr e^{ixn}  
  e^{it \tau} \frac{ 1 - \psi(\tau -  an^{2}) 
  }{\tau -  an^{2}} \wh{w}(n, \tau) \ d \tau
  \\
  \label{main-rel-term-4}
  & + \psi_{\delta}(t) \sum_{a = \pm 1} \sum_{n\in \zz} \int_\rr e^{i(xn + 
  t an^{2})}
  \frac{1- \psi(\tau -  an^{2})}{\tau -  an^{2}} \wh{w}(n, \tau) \ d \tau
  \\
  & + \psi_{\delta}(t) \sum_{a = \pm 1}  \sum_{k \ge 1} \frac{i^k t^k}{k!}
  \sum_{n \in \zz} \int_\rr e^{i(xn + t an^{2} )}
  \psi(\tau -  an^{2}) (\tau -  an^{2})^{k-1} \wh{w}(n, \tau)
  \\
  \label{main-rel-term-5}
  & \doteq Tu
\end{align}
%
%
where $T=T_{u_0, u_1, \psi, \delta}$.
\begin{definition}
  Equip $H^{s} \times H^{s-2}$ with the 
  topology defined by the norm $\|(f_0, f_1)\|_{H^{s} \times H^{s-2}}
  = \|f_0\|_{H^{s}} + \|f_1\|_{H^{s-2}}$.
  and denote $B_{H^{s} \times H^{s-2}}(R) \doteq \left\{ f: \| f \|_{H^{s} \times
  H^{s-2}} < R
  \right\}$. We say that the $B_{4}$ ivp
  \eqref{eqn:mb-2}-\eqref{eqn:mb-init-data-2} is
	\emph{locally well posed} in
  $H^s \times H^{s-2}$ if 
	\begin{enumerate}
    \item For every $(u_{0}, u_{1}) \in B_{H^{s} \times H^{s-2}}(R)$
      there exists $\delta>0$ depending on $R$ such that the Cauchy problem
      $\psi_{\delta} u = Tu$ has a unique solution $u \in C([-\delta,
      \delta], H^s)$ for $ |t| \le \delta$.
    \item
      The flow map $(u_0, u_{1}) \mapsto u(t)$ is uniformly continuous from
      $B_{H^{s} \times H^{s-2}}(R)$ 
      to $C(\left[ -\delta, \delta \right], H^s)$. That is, if
      $\{(u_{0,n}, u_{1,n} ) \}, \{(v_{0,n}, v_{1,n} )\}
      \subset B_{H^{s} \times H^{s-2}}(R)$ such that $\|(u_{0,n}, u_{1,n}) -
      (v_{0,n}, v_{1,n}) \|_{H^{s} \times H^{s-2}} \to 0$, then 
      $\sup_{t \in [-\delta, \delta]}
      \|u_{n}(\cdot, t) - v_{n}(\cdot, t) \|_{H^s} \to 0$.
  \end{enumerate}
	Otherwise, we say that the $B_{4}$ ivp is \emph{ill-posed}.
\end{definition}
%
%
\begin{framed}
\begin{remark}
  If $$\sup_{t \in [-\delta, \delta]}\|u(\cdot, t) - v(\cdot, t)
  \|_{H^{s}} \le c \left( \|u_{0} - v_0 \|_{H^{s}} + \|u_{1} - v_1 \|_{H^{s-2}}
  \right),$$ we
  say the flow map is \emph{Lipschitz} from $B_{H^{s} \times H^{s-2}}(R)$ 
  to $C(\left[ -\delta, \delta \right], H^s)$. 
%
%
Note that this implies uniform continuity of the flow map from $B_{H^{s}
\times H^{s-2}}(R)$ to $C(\left[ -\delta, \delta \right], H^s)$.
\label{rem:lipschitz-cont}
\end{remark}
\end{framed}
%
%
We now introduce the following spaces. 
%
%
\begin{definition}
  Let $\mathcal{Y}$ be the space of functions $F(\cdot)$ such that
  \begin{enumerate}[(i)]
   \item{$F: \ci \times \rr \to \cc$ }.
   \item{ $F(x, \cdot) \in \mathcal{S}(\rr)$ for each $x \in \ci$}.
   \item{ $F(\cdot, t) \in C^{\infty}(\ci)$for each $t \in \rr$}.
  \end{enumerate}
  For $s, b \in \rr$, $X_{s,b}$ denotes the completion of $\mathcal{Y}$ with
  respect to the norm
  %
  %
  \begin{equation}
  \begin{split}
    \|F\|_{X_{s,b}} = \left( \sum_{n \in \zz} (1 + |n|)^{2s} \int_{\rr}
    (1 + | | \tau | - n^{2} |)^{2b} |\wh{F}(n, \tau)|^{2} d \tau\right)^{1/2}.
  \end{split}
  \label{eqn:bous-norm}
  \end{equation}
  %
  \begin{framed}
    %
    %
    \begin{remark}
    Note that the norm here is different than the one used to for the KdV. In
    the KdV case, $b$ has to be equal to $1/2$. And, for getting the embedding
    of the lemma below, one has to add an extra term, thus producing the
    $Y^{s}$ norm of CKSTT.
    \label{rem:alternate-space}
    \end{remark}
    %
  \end{framed}
    %
  %
  %
\end{definition}
%
The $X_{s,b}$ spaces have the following important embedding, whose proof is
provided in the appendix.
%
%
%%%%%%%%%%%%%%%%%%%%%%%%%%%%%%%%%%%%%%%%%%%%%%%%%%%%%
%
%
%               Embedding 
%
%
%%%%%%%%%%%%%%%%%%%%%%%%%%%%%%%%%%%%%%%%%%%%%%%%%%%%%
%
%
\begin{lemma}[Lemma 2.3 in \cite{Farah:2009uq}]
  Let $b > 1/2$. Then $X_{s, b} \subset C(\rr, H^s)$ continuously. That is, there exists $c>0$ depending only on $b$ such that
%
%
\begin{equation*}
\begin{split}
  \| u \|_{C(\rr, H^s) } \doteq \sup_{t \in \rr} \| u(t) \|_{H^s } 
  \le c \| u \|_{X_{s,b}}.
\end{split}
\end{equation*}
%
\label{lem:embedding}
\end{lemma}
%
%
We are now prepared to state the main result of this paper.
%
%
%
%
%%%%%%%%%%%%%%%%%%%%%%%%%%%%%%%%%%%%%%%%%%%%%%%%%%%%%
%
%
%	Main Result				
%
%
%%%%%%%%%%%%%%%%%%%%%%%%%%%%%%%%%%%%%%%%%%%%%%%%%%%%%
%
%
\begin{theorem}
\label{thm:main}
The initial value problem 
\eqref{eqn:mb-2}-\eqref{eqn:mb-init-data-2} is locally well-posed in $H^s$ for
$s >
-1/4$ and ill-posed for $s < -1/4$ in both the periodic and non-periodic cases.
%
%
\end{theorem} 
%
%
%
%
%%%%%%%%%%%%%%%%%%%%%%%%%%%%%%%%%%%%%%%%%%%%%%%%%%%%%
%
%
%                Proof of Thm
%
%
%%%%%%%%%%%%%%%%%%%%%%%%%%%%%%%%%%%%%%%%%%%%%%%%%%%%%
%
%
\section{Proof of \autoref{thm:main}} 
\label{sec:pf-main}
%
%
\subsection{The Periodic Case} 
\label{ssec:periodic}
To prove well-posedness for the $B_4$ ivp we we will 
show that for initial data $\vp \in B_{H^{s}(\ci) \times H^{s-2}(\ci)}(R)$, $T$ is a contraction on
$B_{X^{s}}(M_{R})$, where $M_{R}$ is a constant depending on $R$, 
by estimating the $X_{s,b}$
norm of \eqref{main-rel-term-1}-\eqref{main-rel-term-5}. The 
Picard fixed point theorem will
then yield a unique $u \in X^{s}$ satisfying $u = Tu$.
An application of
\autoref{lem:embedding} will then imply the existence of a unique
$u \in C([-\delta, \delta], H^s(\ci))$ solving $\psi_{\delta} u = Tu$ for $| t | \le \delta$.
Lipschitz continuity of the flow map (and hence, uniform
continuity) will follow from estimates used to establish the contraction
mapping. 
%
%
%
%
%
%
%
%
%
\subsubsection{Estimate for \eqref{main-rel-term-1}}
\label{sssec:est-init-term-1}
We have
%
%
\begin{equation*}
  \begin{split}
    \wh{\eqref{main-rel-term-1}}
    = \frac{\wh{\psi}(\tau -
    n^{2})\wh{u_{0}}(n)}{2} + \frac{\wh{\psi}(\tau +
    n^{2})\wh{u_{0}}(n)}{2}.
  \end{split}
\end{equation*}
%
%
Hence, substituting and applying the inequality 
%
%
\begin{equation}
  \label{square-ineq}
\begin{split}
(a + b)^{2} \le 4(a^{2} +
b^{2}),\ a, b \in \rr,
\end{split}
\end{equation}
%
%
we have

%
%
\begin{align}
  & \| \eqref{main-rel-term-1} \|_{X^{s}}^{2} 
    \notag
    \\
    & = \sum_{n \in \zz}(1 + |n|)^{2s} \int_{\rr}\left( 1 + | | \tau
    |-n^{2} | \right) | \frac{\wh{\psi}(\tau - n^{2})\wh{u_{0}(n)}}{2 } +
    \frac{\wh{\psi}(\tau + n^{2})\wh{u_{0}}(n)}{2 } |^{2} d \tau
    \notag
    \\
    & \le \sum_{n \in \zz} \left( 1 + |n| \right)^{2s} | \wh{u_{0}}(n)
    |^{2} \int_{\rr} | \wh{\psi}(\tau - n^{2}) |^{2}\left( 1 + | | \tau | -
    n^{2} | \right) d \tau
    \label{u-0-norm-comp-1}
    \\
    & + \sum_{n \in \zz} \left( 1 + |n| \right)^{2s} | \wh{u_{0}}(n)
    |^{2} \int_{\rr} | \wh{\psi}(\tau + n^{2}) |^{2}\left( 1 + | | \tau | -
    n^{2} | \right) d \tau.
    \label{u-0-norm-comp-3}
\end{align}
%
%
Noting that
\begin{equation}
  \begin{split}
    | | \tau | - n^{2} | \le \min\left\{ | \tau - n^{2} |, | \tau + n^{2} |
    \right\},
  \end{split}
  \label{eqn:norm-key-ineq}
\end{equation}
%
we bound \eqref{u-0-norm-comp-1} by
%
%
\begin{equation*}
  \begin{split}
    & \sum_{n \in \zz} \left( 1 + |n| \right)^{2s} | \wh{u_{0}}(n)
    |^{2} \int_{\rr} | \wh{\psi}(\tau - n^{2}) |^{2}\left( 1 +  | \tau  -
    n^{2} | \right) d \tau
    \\
    & = \sum_{n \in \zz} \left( 1 + |n| \right)^{2s} | \wh{u_{0}}(n)
    |^{2} \int_{\rr} | \wh{\psi}(\tau') |^{2}\left( 1 +  | \tau'| \right) d \tau
    \\
    & \le c_{\psi} \sum_{n \in \zz} \left( 1 + |n| \right)^{2s} | \wh{u_{0}}(n)
    |^{2}
    \\
    & = c_{\psi} \| u_{0} \|_{H^{s}}^{2}
  \end{split}
\end{equation*}
%
%
where $c_{\psi}$ is a constant depending only on $\psi$. The
term \eqref{u-0-norm-comp-3} is bounded in similar fashion. Therefore, 
$\|\eqref{main-rel-term-1}\|_{X^{s}}^{2} \le c_{\psi}
\|u_{0}\|_{H^s}^2$. Taking square roots of both sides gives
%
%
\begin{equation}
  \begin{split}
    \|\eqref{main-rel-term-1}\|_{X^{s}} \le c_{\psi}
    \|u_{0}\|_{H^s}.
  \end{split}
  \label{eqn:u-0-fin-est}
\end{equation}
%
%
%
%
\subsubsection{Estimate for \eqref{main-rel-term-2}}
\label{sssec:estimate-init-term-2}
We have
%
%
\begin{equation*}
  \begin{split}
    \wh{\psi(t)S_{t}u_{1}}^{x}(n, t)
    & = \psi(t) \wh{u_{1}}(n) \frac{e^{in^2 t} - e^{-in^{2}t}}{2i n^{2}}
    \\
    & = \frac{\psi(t) \wh{u_{1}}(n)e^{in^{2}t}}{2i n^{2}} - \frac{\psi(t)
    \wh{u_{1}}(n)e^{-in^{2}t}}{2i n^{2}}  
  \end{split}
\end{equation*}
%
%
and
%
%
\begin{equation*}
  \begin{split}
    \wh{\psi(t) S_{t}u_{1}}(n, \tau) = \frac{\wh{\psi}(\tau -
    n^{2})\wh{u_{1}}(n)}{2i n^{2}} - \frac{\wh{\psi}(\tau + n^{2})\wh{u_{1}}(n)}{2i
    n^{2}}.
  \end{split}
\end{equation*}
%
Note that 
%
\begin{equation*}
  \begin{split}
    \wh{\psi(t)S_{t}u_{1}}^{x}(0, t)
    & = \psi(t) \wh{u_{1}}(0) t
      \end{split}
\end{equation*}
and so 
%
%
\begin{equation*}
  \begin{split}
    \wh{\psi(t) S_{t}u_{1}}(0, \tau) = i \frac{d}{d \tau} \wh{\psi}(\tau)
    \wh{u_{1}}(0).
  \end{split}
\end{equation*}
%
Hence, substituting and applying \eqref{square-ineq}, we have
%
%
\begin{equation}
  \begin{split}
    \| \eqref{main-rel-term-2} \|_{X^{s}}^{2} 
    & = \sum_{n \in \zzdot}(1 + |n|)^{2s} \int_{\rr}\left( 1 + | | \tau
    |-n^{2} | \right) | \frac{\wh{\psi}(\tau - n^{2})\wh{u_{1}(n)}}{2i
    n^{2}} -
    \frac{\wh{\psi}(\tau + n^{2})\wh{u_{1}}(n)}{2i n^{2}} |^{2} d \tau
    \\
    & + |\wh{u_{1}}(0)|^{2} \int_{\rr} (1 + | \tau |) | i \frac{d }{d \tau}
    \wh{\psi}(\tau)|^{2} d \tau
    \\
    & \le \sum_{n \in \dot{\zz}} n^{-4} \left( 1 + |n| \right)^{2s} | \wh{u_{1}}(n)
    |^{2} \int_{\rr} | \wh{\psi}(\tau - n^{2}) |^{2}\left( 1 + | | \tau | -
    n^{2} | \right) d \tau
    \\
    & + \sum_{n \in \dot{\zz}} n^{-4} \left( 1 + |n| \right)^{2s} | \wh{u_{1}}(n)
    |^{2} \int_{\rr} | \wh{\psi}(\tau + n^{2}) |^{2}\left( 1 + | | \tau | -
    n^{2} | \right) d \tau
    \\
    & + |\wh{u_{1}}(0)|^{2} \int_{\rr} (1 + | \tau |) |\frac{d }{d \tau}
    \wh{\psi}(\tau)|^2 d \tau.
\end{split}
\label{u-1-norm-comp-pre}
\end{equation}
%
%
Applying the inequality
%
%
\begin{equation*}
\begin{split}
  \frac{(1 + |n|)^{2s}}{ n^{4}} \le \frac{(1 + |n|)^{2s}}{\frac{1}{16}(1 +
  |n|)^{4}} = 16 (1 + | n |)^{2(s-2)},  \quad s \in \rr, \quad n \ge 1
\end{split}
\end{equation*}
%
to \eqref{u-1-norm-comp-pre} gives
%
\begin{equation}
  \begin{split}
    \|  \eqref{main-rel-term-2}\|_{X^{s}}^{2} 
    & \lesssim \sum_{n \in \dot{\zz}} \left( 1 + |n| \right)^{2(s-2)} | \wh{u_{1}}(n)
    |^{2} \int_{\rr} | \wh{\psi}(\tau - n^{2}) |^{2}\left( 1 + | | \tau | -
    n^{2} | \right) d \tau
    \\
    & + \sum_{n \in \dot{\zz}} \left( 1 + |n| \right)^{2(s-2)} | \wh{u_{1}}(n)
    |^{2} \int_{\rr} | \wh{\psi}(\tau + n^{2}) |^{2}\left( 1 + | | \tau | -
    n^{2} | \right) d \tau
    \\
    & + |\wh{u_{1}}(0)|^{2} \int_{\rr} (1 + | \tau |) |\frac{d }{d \tau}
    \wh{\psi}(\tau)|^2 d \tau.
\end{split}
\label{u-1-norm-comp}
\end{equation}
%
%
Applying \eqref{eqn:norm-key-ineq},
we bound the first term of
\eqref{u-1-norm-comp} by
%
%
%
\begin{equation*}
  \begin{split}
    & \sum_{n \in \dot{\zz}} \left( 1 + |n| \right)^{2(s-2)} | \wh{u_{1}}(n)
    |^{2} \int_{\rr} | \wh{\psi}(\tau - n^{2}) |^{2}\left( 1 +  | \tau  -
    n^{2} | \right) d \tau
    \\
    & = \sum_{n \in \dot{\zz}} \left( 1 + |n| \right)^{2(s-2)} | \wh{u_{1}}(n)
    |^{2} \int_{\rr} | \wh{\psi}(\tau') |^{2}\left( 1 +  | \tau'| \right) d \tau
    \\
    & \le c_{\psi} \| u_{1} \|_{H^{s-2}}^{2}. 
  \end{split}
\end{equation*}
%
%
Applying
\eqref{eqn:norm-key-ineq} again, the
second term of \eqref{u-1-norm-comp} is bounded by
\begin{equation*}
  \begin{split}
    & \sum_{n \in \dot{\zz}} \left( 1 + |n| \right)^{2(s-2)} | \wh{u_{1}}(n)
    |^{2} \int_{\rr} | \wh{\psi}(\tau + n^{2}) |^{2}\left( 1 +  | \tau  -
    n^{2} | \right) d \tau
    \\
    & = \sum_{n \in \dot{\zz}} \left( 1 + |n| \right)^{2(s-2)} | \wh{u_{1}}(n)
    |^{2} \int_{\rr} | \wh{\psi}(\tau') |^{2}\left( 1 +  | \tau'| \right) d \tau
    \\
    & \le c_{\psi} \| u_{1} \|_{H^{s-2}}^{2}
  \end{split}
\end{equation*}
while the third term is bounded by  
%
%
\begin{equation*}
\begin{split}
  c_{\psi} \| u_{1} \|_{H^{s-2}}^{2}.
\end{split}
\end{equation*}
%
%
Therefore, 
$\|\eqref{main-rel-term-2}|_{X^{s}}^{2} \le c_{\psi}
\|u_{1}\|_{H^{s-2}}^2$ and
taking square roots of both sides gives
%
%
\begin{equation*}
  \begin{split}
    \|\eqref{main-rel-term-2}|_{X^{s}} \le c_{\psi}
    \|u_{1}\|_{H^{s-2}}.
  \end{split}
\end{equation*}
%
%
%
%
\subsection{Estimate for \eqref{main-rel-term-3}.}
We now need the following lemma, whose proof is provided in the appendix.
%
%
%%%%%%%%%%%%%%%%%%%%%%%%%%%%%%%%%%%%%%%%%%%%%%%%%%%%%
%
%
%			Schwartz Multiplier	
%
%
%%%%%%%%%%%%%%%%%%%%%%%%%%%%%%%%%%%%%%%%%%%%%%%%%%%%%
%
%
\begin{lemma}
\label{lem:schwartz-mult}
	For $\psi \in S(\rr)$,
%
%
\begin{equation}
	\label{schwartz-mult}
	\begin{split}
    \|\psi f \|_{X_{s,b}} \le c_{\psi} \|f \|_{X_{s,b}}.
	\end{split}
\end{equation}
%
%
\end{lemma}
%
%
Hence,
%
%
\begin{equation}
  \label{yu}
	\begin{split}
		\|\eqref{main-rel-term-3}\|_{X_{s,b}} 
    & \le c_{\psi, \delta}
    \sum_{a = \pm 1} \| \sum_{n \in \zz} e^{ixn} \int_\rr 
		e^{it \tau} \frac{ 1 - a\psi (\tau - an^{m} ) 
		}{\tau - an^{m}} \wh{w}(n, \tau) \ 
		d \tau\|_{X_{s,b}}.
			\end{split}
\end{equation}
%
But
%
%
\begin{equation}
\label{main-int2-est-X-s-part}
\begin{split}
  & \| \sum_{n \in \zz} e^{ixn} \int_\rr 
		e^{it \tau} \frac{ 1 - a\psi (\tau - an^{m} ) 
		}{\tau - an^{m}} \wh{w}(n, \tau) \ 
		d \tau\|_{X_{s,b}}
		\\
    & = \sum_{a = \pm 1}\left( \sum_{n \in \zz} \left (1 + |n| \right )^{2s} \int_\rr
    (1 + |  |\tau| - n^{m}|)^{2b} \left | \frac{1 - a\psi(\tau - an^{2 
		})}{\tau - an^{m}} 
		\wh{w}(n, \tau) \right |^2 \ d 
		\tau \right)^{1/2}
		\\
    & \le \sum_{a = \pm 1}
    \left( \sum_{n \in \zz} \left (1 + |n| \right )^{2s} \int_{| \tau - an^{m }| \ge 1}
    (1 + | |\tau| - n^{m}|)^{2b} \frac{|\wh{w}(n, \tau)|^2 }{|\tau - an^{m }|^2} 
		\ d 
		\tau \right)^{1/2}
  \end{split}
\end{equation}
Applying the inequality
\begin{equation}
	\label{one-plus-ineq}
	\begin{split}
		& \frac{1}{|j|} \le \frac{2}{1 + |j| } , 
		\qquad |j| \ge 1
   	\end{split}
\end{equation}
and \eqref{eqn:norm-key-ineq} we bound this by
\begin{equation}
  \label{pre-bi-est}
  \begin{split}
		& 4 \left( \sum_{n \in 
		\zz} \left (1 + |n| \right )^{2s} \int_\rr
		\frac{|\wh{w}(n, \tau) |^2}{(1+ |  |\tau| - 
    n^{m}|)^{2(1-b)}} 
		 \ d \tau 
		\right)^{1/2}
\end{split}
\end{equation}
%
%
%
%
%
%
We now need the following bilinear estimate, whose proof we leave for later.
%
%
%%%%%%%%%%%%%%%%%%%%%%%%%%%%%%%%%%%%%%%%%%%%%%%%%%%%%
%
%
%				Proposition
%
%
%%%%%%%%%%%%%%%%%%%%%%%%%%%%%%%%%%%%%%%%%%%%%%%%%%%%%
%
%
\begin{proposition}[Theorem 1.1 in \cite{Farah:2009uq}]
\label{prop:bilinear-est}
	%
	%
	If $a > 1/4$, $b > 1/2$, and $s \ge -a/2$, 
  then 
	\begin{equation}
		\left( \sum_{n \in \zz} \left (1 + |n| \right )^{2s} \int_\rr
		\frac{|\wh{w_{fg}}(n, \tau) |^2}{\left (1+ |\tau - 
    n^{m}| \right )^{2a}} 
		 \ d \tau 
		\right)^{1/2}
		\lesssim \|f\|_{X_{s,b}} \|g\|_{X_{s,b}}
	\end{equation}
	where $w_{fg}(x,t)$ = $fg (x,t)$.
%
%
%
%
\end{proposition}
%
Applying \autoref{prop:bilinear-est} to \eqref{pre-bi-est}, we conclude that
%
%
%
%
\begin{equation}
	\label{main-int2-est}
	\begin{split}
		\|\eqref{main-rel-term-3}\|_{X_{s,b}} \le c_{\psi, \delta}\|f\|_{X_{s,b}}
    \|g\|_{X_{s,b}}, \qquad 1/2 < b < 3/4.
	\end{split}
\end{equation}
%
%
%
%
%
%
%
%
%
%
%
\subsection{Estimate for \eqref{main-rel-term-4}.}
Letting $$f(x,t) = \sum_{a = \pm 1} \psi_{\delta}(t) \sum_{n \in \zz} e^{i\left( xn +
atn^{m} \right)} 
\int_\rr \frac{1 - \psi\left( \lambda - an^{m} \right)}{\lambda - an^{m}} 
\wh{w} \left( n, \lambda \right) \ d \lambda,$$ we have
%
%
\begin{equation*}
	\begin{split}
		& \wh{f^x}(n, t) = \psi_{\delta}(t) e^{aitn^{m}} \int_\rr
		\frac{1 - \psi\left( \lambda - an^{m} \right)}{\lambda - an^{m}} 
		\wh{w}(n, \lambda) \ d \lambda
	\end{split}
\end{equation*}
and
\begin{equation*}
	\begin{split}
		 \wh{f}\left( n, \tau \right)
		 & = \int_\rr e^{-it\left( \tau - an^{m} 
		\right)} \psi_{\delta}(t) \int_\rr \frac{1 - a\psi\left( 
		\lambda - an^{m} 
		\right)}{\lambda - an^{m}} \wh{w}(n, \lambda) \ d \lambda d \tau
		\\
    & = \wh{\psi_{\delta}}\left( \tau - an^{m} \right) \int_\rr 
		\frac{1 - a\psi\left( 
		\lambda - an^{m} 
		\right)}{\lambda - an^{m}} \wh{w}(n, \lambda) \ d \lambda.
	\end{split}
\end{equation*}
Therefore,
%
%
\begin{equation}
  \label{iu}
	\begin{split}
		& \| \eqref{main-rel-term-4} \|_{X_{s,b}} 
		\\
    & \lesssim \sum_{a = \pm 1} \left( \sum_{n \in \zz} \left (1 + |n| \right
    )^{2s} \int_\rr \left( 1 + | |\tau| - n^{m } \right )^{2b} | | \wh{\psi_{\delta}}\left(
    \tau - an^{m } \right) |^2 \ d \tau \right.
		\\
		& \times \left . |
		\int_\rr \frac{1 - \psi\left( \lambda - an^{m } \right)}{\lambda -
		an^{m }} \wh{w}(n, \lambda) \ d \lambda |^2  \right)^{1/2}.
  \end{split}
\end{equation}
Applying \eqref{eqn:norm-key-ineq}, we have the bound
%
%
\begin{equation*}
\begin{split}
& \sum_{a = \pm 1} 
\int_\rr \left( 1 + | |\tau| - n^{m } \right )^{2b} | | \wh{\psi_{\delta}}\left(
    \tau - an^{m } \right) |^2 \ d \tau 
    \\
    & \le  \sum_{a = \pm 1} 
    \int_\rr \left( 1 + |\tau - an^{m } | \right )^{2b} | | \wh{\psi_{\delta}}\left(
    \tau - an^{m } \right) |^2 \ d \tau 
    \\
    & = c_{\psi, \delta, b}.
\end{split}
\end{equation*}
%
Hence, \eqref{iu} is bounded by
%
\begin{equation}
  \begin{split}
    & c_{\psi, \delta, b} \sum_{a = \pm 1}\left( \sum_{n \in \zz} \left (1 + |n| \right )^{2s} | \int_\rr
		\frac{1 - \psi\left( \lambda - an^{m } \right)}{\lambda - an^{m }}
		\wh{w}(n, \lambda) \ d\lambda |^2 \right)^{1/2}
		\\
		& \simeq \sum_{a = \pm 1} \left( \sum_{n \in \zz} \left (1 + |n| \right )^{2s}  \left ( \int_\rr
		\frac{1 - \psi\left( \lambda - an^{m } \right)}{|\lambda - an^{m }|}
		|\wh{w}(n, \lambda) | \ d\lambda \right )^2 \right)^{1/2}
		\\
		& \le \sum_{a = \pm 1} \left( \sum_{n \in \zz} \left (1 + |n| \right )^{2s}  \left ( \int_{| \lambda - 
		an^{m } | \ge 1}
		\frac{|\wh{w}(n, \lambda) | }{|\lambda - an^{m }|}
		\ d\lambda \right )^2 \right)^{1/2}.
  \end{split}
\end{equation}
Applying \eqref{one-plus-ineq} then \eqref{eqn:norm-key-ineq} we bound this by
\begin{equation}
  \label{fh}
\begin{split}
    4 \left( \sum_{n \in \zz} \left (1 + |n| \right )^{2s}  \left ( \int_\rr
    \frac{|\wh{w}(n, \lambda)| }{1 + |\lambda - n^{m }|}
    \ d\lambda \right )^2 \right)^{1/2}.
	\end{split}
\end{equation}
%
%
%
%%
We now need the following corollary to
\autoref{prop:bilinear-est}.
%
%
%%%%%%%%%%%%%%%%%%%%%%%%%%%%%%%%%%%%%%%%%%%%%%%%%%%%%
%
%
%				Second bilinear Estimate 
%
%
%%%%%%%%%%%%%%%%%%%%%%%%%%%%%%%%%%%%%%%%%%%%%%%%%%%%%
%
%
\begin{corollary}
\label{cor:bilinear-estimate2}
	If $a > 1/4$, $b > 1/2$, and $s \ge -a/2$, 
  then 
%
%
\begin{equation}
	\label{bilinear-estimate2}
	\begin{split}
		\left( \sum_{n \in \zz} \left (1 + |n| \right )^{2s}  \left ( \int_\rr 
    \frac{|\wh{w_{fgh}}(n, \tau) |}{(1 + | |\tau| - n^{m } |)^{2a}}
		 \ d\tau \right)^2  \right)^{1/2} \lesssim \|f\|_{X_{s,b}} \|g\|_{X_{s,b}}.
	\end{split}
\end{equation}
\end{corollary}
Applying \autoref{cor:bilinear-estimate2} to \eqref{fh}, we
conclude that
%
%
\begin{equation}
	\label{main-int3-est}
	\begin{split}
		\|\eqref{main-rel-term-4}\|_{X_{s,b}} 
    \le c_{\psi, \delta, b} \|f\|_{X_{s,b}} \|g\|_{X_{s,b}}. 
	\end{split}
\end{equation}
%
%
%
\subsection{Estimate for \eqref{main-rel-term-5}.}
Note that
%
%
\begin{equation}
	\label{1n}
	\begin{split}
    \eqref{main-rel-term-5} \simeq \sum_{a = \pm 1}\sum_{k \ge 1}
		\frac{i^k}{k!}g_k(x,t)
	\end{split}
\end{equation}
%
%
where 
%
%
\begin{equation*}
	\begin{split}
		& g_k(x,t) = t^k \psi_{\delta}(t) \sum_{n \in \zz} e^{i\left( xn + tan^{m}
		\right)} h_k(n),
		\\
		& h_k(n) = \int_\rr \psi \left( \tau - an^{m } \right) \cdot \left(
		\tau - an^{m } \right)^{k -1} \wh{w}(n, \tau) \ d \tau.
	\end{split}
\end{equation*}
%
%
Hence
%
%
\begin{equation*}
	\begin{split}
		\wh{g_k^x}(n, t) = t^{k} \psi_{\delta}(t) e^{i t an^{m }} h_k(n)
	\end{split}
\end{equation*}
%
%
which gives
%
%
\begin{equation*}
	\begin{split}
		\wh{g_k}(n, \tau)
		& = h_k(n) \int_\rr e^{-it\left( \tau - an^{m } \right)}
		t^{k}\psi_{\delta}(t) \ dt
		\\
		& = h_k(n) \wh{t^{k}\psi_{\delta}(t)} \left( \tau - an^{m } \right).
	\end{split}
\end{equation*}
%
%
Applying this to \eqref{1n} and using Minkowski's inequality, we obtain
%
%
\begin{equation}
	\label{2n}
	\begin{split}
		\|\eqref{main-rel-term-5}\|_{X_{s,b}} 
    & \lesssim \sum_{a = \pm 1} \left( \sum_{n \in \zz} \left (1 + |n| \right
    )^{2s} \int_\rr \left( 1 + | |\tau| - an^{m } | \right)^{2b}
    | \wh{\sum_{k \ge 1} \frac{i^k}{k!}g_k(x,t)} |^2 \ d \tau
		\right)^{1/2}
		\\
		& \le \sum_{a = \pm 1}\sum_{k \ge 1} \frac{1}{k!}\left( \sum_{n \in \zz} \left (1 + |n| \right )^{2s}
    \int_\rr \left( 1 + | |\tau| - an^{m } | \right)^{2b} | \wh{g_k}(n, \tau) |^2 \
		d \tau \right)^{1/2}
		\\
		& = \sum_{a = \pm 1}\sum_{k \ge 1} \frac{1}{k!} \left( \sum_{n \in \zz} \left (1 + |n| \right )^{2s}
    \int_\rr \left( 1 + | |\tau| - an^{m } | \right)^{2b} | h_k(n) \wh{t^k
		\psi_{\delta}(t)} \left( \tau - an^{m } \right) |^2 \ d \tau \right)^{1/2}
		\\
		& = \sum_{a = \pm 1}\sum_{k \ge 1} \frac{1}{k!} \left( \sum_{n \in \zz} \left (1 + |n| \right )^{2s} |
    h_k(n) |^2 \int_\rr \left( 1 + | |\tau| - an^{m } | \right)^{2b} | \wh{t^k
		\psi_{\delta}(t)} \left( \tau - an^{m } \right) |^2 \ d \tau \right)^{1/2}.
	\end{split}
\end{equation}
%
%
Applying \eqref{eqn:norm-key-ineq} and the change
of variable $\tau - an^{m } = \tau'$
gives
%
%
\begin{equation}
	\label{3n}
	\begin{split}
		& \int_\rr \left( 1 + | |\tau| - an^{m } | \right)^{2b} | \wh{t^{k}
		\psi_{\delta}(t)}\left( \tau - an^{m } \right) |^2 \ d \tau
    \\
    & \le \int_\rr \left( 1 + | \tau - an^{m } | \right)^{2b} | \wh{t^{k}
		\psi_{\delta}(t)}\left( \tau - an^{m } \right) |^2 \ d \tau
		\\
    & = \int_\rr \left( 1 + |\tau'| \right)^{2b} | \wh{t^k \psi_{\delta}(t)}(\tau') |^2 \
		d \tau'
		\\
    & \le \int_\rr \left( 1 + |\tau'| \right)^{2b} | \wh{t^k \psi_{\delta}(t)}(\tau')
		|^2 \ d \tau'
		\\
    & \lesssim \int_\rr \left( 1 + | \tau' | \right)^{2b}| \wh{t^{k}
		\psi_{\delta}(t)}(\tau') |^2 \ d \tau'
		\\
    & \le \|t^k \psi_{\delta}(t) \|_{H^{[b] + 1}(\rr)}^2
	\end{split}
\end{equation}
%
where $[b]$ denotes the least integer of $b$. Note that
%
%
\begin{equation}
	\label{4n}
	\begin{split}
    & \|t^k \psi_{\delta} \|_{H^{[b] +1}(\rr)}
		\\
    & = \|t^k \psi_{\delta}\|_{L^2(\rr)} + \|\p_t (t^k \psi_{\delta} )
    \|_{L^2(\rr)} 
    \\
    & + \| \p_{t}^{2} (t^{k} \psi_{\delta}) \|_{L^{2}(\rr)} + \ldots + \|
    \p_{t}^{[b] + 1} (t^{k} \psi_{\delta})\|_{L^{2}}
    \\
    & \le c_{\psi, \delta} + k c'_{\psi, \delta} + k (k -1) c_{\psi, \delta}'' + \ldots +
    k(k-1) \cdots (k - [b]) c_{\psi}^{[b] + 1}
    \\
    & \lesssim c_{\psi, \delta} k(k-1) \cdots (k - [b]).
	\end{split}
\end{equation}
%
%
Hence, applying \eqref{3n} and \eqref{4n} to \eqref{2n}, we obtain
%
%%
\begin{equation}
	\label{5n}
	\begin{split}
		\|\eqref{main-rel-term-5} \|_{X_{s,b}}
		& \lesssim \sum_{a = \pm 1}
    c_{\psi, \delta}\sum_{k \ge [b] +1} \frac{1}{(k-[b] - 1)!} \left( \sum_{n \in \zz} \left (1 + |n| \right )^{2s} | h_k(n) |^2 
		\right)^{1/2}
		\\
		& \lesssim  \sum_{a = \pm 1} \sum_{k \ge [b] +1} \frac{1}{(k-[b] - 1)!}
    \times \sup_{k \ge [b] + 1} \left( \sum_{n \in \zz} \left (1 + |n| \right )^{2s} | 
		h_k(n) |^2 \right)^{1/2}
		\\
		& = \sum_{a = \pm 1}\sum_{k \ge [b] +1} \frac{1}{(k-[b] - 1)!}
    \\
    & \times \sup_{k \ge [b] + 1} 
		\left( \sum_{n \in \zz} \left (1 + |n| \right )^{2s} |\int_\rr 
		\psi\left( \tau - an^{m } \right) \cdot \left( \tau - an^{m } 
    \right)^{k -1} \wh{w}(n, \tau) \ d \tau|^{2} \right)^{1/2}.
    \end{split}
\end{equation}
%
%%
Recall that $0 \le \psi \le 1, \text{ supp} \, \psi \subset [-1,1 ]$. 
This implies $| \psi\left( \tau - an^{m } \right) \cdot \left( \tau - an^{m } \right)^{k 
-1} | \le \chi_{| \tau - an^{m } | \le 1}$ for all $k \ge 1$. Hence, we bound the
right hand side of \eqref{5n} by
%
%%
\begin{equation*}
	\begin{split}
    & \sum_{a = \pm 1}
    \sum_{k \ge [b] +1} \frac{1}{(k-[b] - 1)!}
    \times \left( \sum_{n \in \zz} (1 + | n |)^{2s}| 
		\int_{| \tau - an^{m}  |\le 1}  \wh{w}(n, \tau) \ d \tau |^2 
		\right)^{1/2}
    \\
    & = e \sum_{a = \pm 1} \left( \sum_{n \in \zz} (1 + | n |)^{2s}| 
		\int_{| \tau - an^{m}  |\le 1}  \wh{w}(n, \tau) \ d \tau |^2 
		\right)^{1/2}
    \\
    & \le e \sum_{a = \pm 1}
\left[ \sum_{n \in \zz} (1 + | n |)^{2s}\left (  
		\int_{| \tau - an^{m}  |\le 1} | \wh{w}(n, \tau) | \ d \tau \right ) ^2 
		\right]^{1/2}
	\end{split}
\end{equation*}
%
%%
which by the inequality
%
%%
\begin{equation*}
	\begin{split}
		1 \le 
		\frac{2}{1 + | j |}, \qquad | j | \le 1
	\end{split}
\end{equation*}
%
%%
and \eqref{eqn:norm-key-ineq}
is bounded by 
%
%%
\begin{equation}
\label{main-int4-est-X-s-part}
	\begin{split}
		& 4e \left[ \sum_{n \in \zz} (1 + | n |)^{2s}\left ( \int_\rr
		\frac{|\wh{w}(n, \tau)|}{1 + | |\tau| - n^{m } |} \ d \tau \right ) ^2 
		\right]^{1/2}.
  \end{split}
\end{equation}
%
%%
Applying \autoref{cor:bilinear-estimate2} to \eqref{main-int4-est-X-s-part}, we
conclude that
%
%
\begin{equation}
\label{main-int4-est}
	\begin{split}
    \|\eqref{main-rel-term-5}\|_{X_{s,b}} \le c_{\psi, \delta} \|u\|_{X_{s,b}}^{3}.
	\end{split}
\end{equation}
%
%
Collecting estimates \eqref{main-int2-est}, 
\eqref{main-int3-est}, and \eqref{main-int4-est}, and recalling 
\eqref{main-rel-term-2}-\eqref{main-rel-term-5}, we obtain the following. 
%
%
\begin{proposition}
\label{prop:contraction}
For $s > -1/4$ and  $1/2 < b \le \min \{ 3/4, 1 + 2s \}$, we have 
%
%%
\begin{equation*}
	\begin{split}
    \|Tu\|_{X^{s}} \le c
    \left( \|u_0 \|_{H^s(\ci)} + \|u_1 \|_{H^{s-2}(\ci)}
    + \|u\|_{X^{s}}^2 
		\right)
	\end{split}
\end{equation*}
%
where $c = c_{\psi, \delta, b}$.  
%%
\end{proposition}

%
\subsubsection{Proof of Existence and Uniqueness}
\label{sssec:proof-b4-per-case}
%
%
%%%%%%%%%%%%%%%%%%%%%%%%%%%%%%%%%%%%%%%%%%%%%%%%%%%%%
%
%% Contraction Proposition
%				 
%%%%%%%%%%%%%%%%%%%%%%%%%%%%%%%%%%%%%%%%%%%%%%%%%%%%%%
%%
%%
%
We will now use \autoref{prop:contraction} to prove local well-posedness for the 
$B_4$ ivp. For given $u_0, u_1$, we may choose $\delta$ sufficiently small such
that 
%
%%
\begin{equation*}
	\begin{split}
    \|u_0\|_{H^s(\ci)} \le \frac{3}{32c^2}, \quad \|u_1\|_{H^{s-2}(\ci)} \le \frac{3}{32c^2}.
	\end{split}
\end{equation*}
%
%%
Then $$\|u\|_{X^{s}} \le \frac{1}{4c}$$ implies
%
%%
\begin{equation*}
	\begin{split}
		\|Tu \|_{X^{s}} 
		& \le c \left[ \frac{3}{32c^2} + \frac{3}{32c^2}+ \left( 
		\frac{1}{4c} \right)^2 \right]
		=  \frac{1}{4c}.
	\end{split}
\end{equation*}
%
%%
Hence, $T=T_{u_0, u_1, \psi, \delta}$ maps the ball $B_{X_{s,b}}(\frac{1}{4c} )$ into
itself. Next, note that 
%
%%
\begin{equation*}
	\begin{split}
    Tu - Tv = \eqref{main-rel-term-3} + \eqref{main-rel-term-4} +
    \eqref{main-rel-term-5} 
  \end{split}
  \label{eqn:integral-form-dif}
\end{equation*}
%
%%
where now $w(x,t) =$. Rewriting
%
%%
\begin{equation*}
	\begin{split}
	\p_x^2 (u^2 - v^2)	
		& = \p_x^2[(u-v)(u+v)]
		\end{split}
\end{equation*}
%
%%
and repeating earlier arguments, we obtain
%
%%
%%
\begin{equation}
	\label{20a}
	\begin{split}
		\|Tu - Tv \|_{X^{s}}  
		& \le c \|u -v\|_{X^{s}} \|u + v \|_{X^{s}}
		\\
		& \le c \|u -v\|_{X^{s}} (\|u\|_{X^{s}}+ \|v \|_{X^{s}}).
	\end{split}
\end{equation}
%
%%
If $$ u, v \in B_{X_{s,b}} \left (\frac{1}{4c} \right )$$ then
%
%%
\begin{equation}
	\label{21a}
	\begin{split}
		\|Tu - Tv \|_{X^{s}}
		& \le c \|u -v \|_{X^{s}} \left( \frac{1}{4c} + 
		\frac{1}{4c} \right)
		\\
		& = \frac{1}{2} \|u -v \|_{X^{s}}. 
	\end{split}
\end{equation}
%
%%
We conclude that $T = T_{\vp}$ is a contraction on the ball $B(0, 
\frac{1}{4c}) \subset X^{s}$. A Picard iteration then yields a unique 
$u \in X^{s}$ satisfying $u = Tu$. Applying
\autoref{lem:embedding}, it follows that $u(x,t) \subset C( [-\delta, \delta], H^s)$ is a unique
solution of the B4 ivp \eqref{eqn:mb-2}-\eqref{eqn:mb-init-data-2} for $t
\in [-\delta, \delta]$.
%
%
\subsubsection{Proof of Lipschitz Continuity in the Periodic Case} 
\label{sssec:lip-continuity}
%
%
%
%
Let $(u_0, u_1), (v_0, v_1) \subset \in H^{s}(\ci) \times H^{s-2}(\ci)  $
be given. Choose $\delta$ sufficiently small
such that $$(u_0, u_1), (v_0, v_1)  \subset
B_{H^{s} \times H^{s-2}} \left (\frac{15}{64c^{3}} \right ).$$ where $c$ is
the constant in \autoref{prop:contraction}. Then there exist $u, v \in
X_{s,b}$ such that $u =
T_{u_0, u_1}u$, $v = T_{v_0, v_1} v$, and so
%
%
\begin{align}
  \notag
    & T_{u_0, u_1}(u) - T_{v_0, v_1}(v)
		\\
    & = \psi_{\delta}(t) \sum_{n \in \zz} e^{inx} \wh{u_{0} - v_{0} }(n) \frac{e^{in^{2}t} + e^{-in^{2}t}}{2} 
\label{main-rel-term-1g}
  \\
  & + \psi_{\delta}(t) \sum_{n \in \zz} e^{inx}
  \wh{u_{1} - v_{1} }(n)\frac{e^{in^{2}t} - e^{-in^{2}}t}{2 i n^{2}} 
\label{main-rel-term-2g}
  \\
  & + \psi_{\delta}(t) \sum_{a = \pm 1} \sum_{n\in \zz} \int_\rr e^{ixn}  
  e^{it \tau} \frac{ 1 - \psi(\tau -  an^{2}) 
  }{\tau -  an^{2}} \wh{w}(n, \tau) \ d \tau
\label{main-rel-term-3g}
  \\
  & + \psi_{\delta}(t) \sum_{a = \pm 1} \sum_{n\in \zz} \int_\rr e^{i(xn + 
  t an^{2})}
  \frac{1- \psi(\tau -  an^{2})}{\tau -  an^{2}} \wh{w}(n, \tau) \ d \tau
\label{main-rel-term-4g}
  \\
  & + \psi_{\delta}(t) \sum_{a = \pm 1}  \sum_{k \ge 1} \frac{i^k t^k}{k!}
  \sum_{n \in \zz} \int_\rr e^{i(xn + t an^{2} )}
  \psi(\tau -  an^{2}) (\tau -  an^{2})^{k-1} \wh{w}(n, \tau)
  \label{main-rel-term-5g}
\end{align}
%
where now $w = u^{2} - v^{2}$. Using arguments similar to those in 
\autoref{sssec:est-init-term-1}-\autoref{sssec:estimate-init-term-2}
we obtain
%
%
\begin{equation}
	\label{gen-2a}
	\begin{split}
    & \| \eqref{main-rel-term-1g}\|_{X^{s}}
		\le c \|u_0 -v_0\|_{H^s},
    \\
    & \| \eqref{main-rel-term-2g}\|_{X^{s}}
    \le c \|u_1 -v_1\|_{H^{s-2}}.
	\end{split}
\end{equation}
%
%
%
%
Therefore, from \eqref{21a} and \eqref{gen-2a}, we obtain
%
%
\begin{equation*}
	\begin{split}
    \|u -v \|_{X^{s}}
    & = \|T_{u_0, v_0}(u) - T_{u_1, v_1}(v) \|_{X^{s}}
    \\
    & \le
    c \left( \|u_0 -v_0 \|_{H^s\left( \ci \right)} +\|u_1 -v_1
        \|_{H^{s-2}\left( \ci \right)} \right )
        + \frac{1}{2} \|u -v \|_{X^{s}}
  \end{split}
\end{equation*}
%
%
which implies
%
%
\begin{equation*}
	\begin{split}
		\frac{1}{2} \|u-v\|_{X^{s}} \le
    c \left( \|u_0 -v_0 \|_{H^s\left( \ci \right)} +\|u_1 -v_1
        \|_{H^{s-2}\left( \ci \right)} \right )
      \end{split}
\end{equation*}
%
%
or
%
%
\begin{equation}
	\begin{split}
		\|u -v \|_{X^{s}} \le 2 c \left( \|u_0 -v_0 \|_{H^s\left( \ci \right)} +\|u_1 -v_1
        \|_{H^{s-2}\left( \ci \right)} \right ).
	\end{split}
  \label{pre-lem-estimate}
\end{equation}
%
%
Applying  \autoref{lem:embedding} to \eqref{pre-lem-estimate}, it follows that
for $(u_0, u_1), (v_0, v_1)  \subset
B_{H^{s} \times H^{s-2}} \left (\frac{15}{64c^{3}} \right )$, the
associated solutions $u, v \in C([-\delta, \delta], H^{s}(\ci))$ satisfy the estimate%
%
%
	 %
	 %
	 \begin{equation*}
		 \begin{split}
       \sup_{t \in [-\delta, \delta]} \|u(\cdot, t) -v(\cdot, t) \|_{H^s(\ci)} \le
      2 c \left( \|u_0 -v_0 \|_{H^s\left( \ci \right)} +\|u_1 -v_1
        \|_{H^{s-2}\left( \ci \right)} \right ).
		 \end{split}
	 \end{equation*}
	 %
	 %
Hence, the flow map of the $B_4$ ivp is Lipschitz continuous from $B_{H^{s}
\times H^{s-2}} \left (\frac{15}{64c^{3}} \right )$ to $C([-\delta, \delta],
H^{s}(\ci))$. This
concludes the proof of well-posedness for the $B_4$ ivp
\eqref{eqn:mb-2}-\eqref{eqn:mb-init-data-2}. \qquad \qedsymbol

%
%%%%%%%%%%%%%%%%%%%%%%%%%%%%%%%%%%%%%%%%%%%%%%%%%%%%%
%
%
%                The non-periodic Case
%
%
%%%%%%%%%%%%%%%%%%%%%%%%%%%%%%%%%%%%%%%%%%%%%%%%%%%%%
%
%
%
%%%%%%%%%%%%%%%%%%%%%%%%%%%%%%%%%%%%%%%%%%%%%%%%%%%%%
%
%
%                Proof of Bilinear Estimate B4 Per
%
%
%%%%%%%%%%%%%%%%%%%%%%%%%%%%%%%%%%%%%%%%%%%%%%%%%%%%%
%
%
\subsubsection{Proof of \autoref{prop:bilinear-est}} 
\label{sssec:proof-bilin-est}
Since $\|f\|_{X_{s,-a'}} \le \|f\|_{X_{s, -a}}$ for $0 < a < a'$, we may assume
$1/4 < a \le b$ and $b > 1/2$ without loss of generality.
By duality, it suffices to show that for $s \ge -a/2$, 
%
%%
\begin{equation}
	\label{duality-est}
	\begin{split}
	|	\sum_{n}  (1 + |n|)^{s}
		\int_{\rr} \phi(n, \tau) \wh{uv}(n, \tau)(1 
    + | |\tau| - n^{2} |^{-a}) d \tau | \lesssim \|u\|_{X^{s}}
    \|v\|_{X^{s}}
    \|\phi \|_{L^{2}_{n, \tau}}.
	\end{split}
\end{equation}
Note first that $|\wh{uv}(n, \tau) |  = | \wh{u} *  \wh{v} 
(n, \tau)|$. From this it follows that
%
%
\begin{equation}
	\label{non-lin-rep}
	\begin{split}
		| \wh{uv}(n, \tau)|
    & = | \sum_{n_{1} }  \int
    \wh{u}\left( n_1,  \tau_1 \right) \wh{v}\left( n - n_1 , \tau - \tau_1   
\right) d \tau_1 |
\\
& \le  \sum_{n_{1} }  \int
    |\wh{u}\left( n_1,  \tau_1 \right)| |\wh{v}\left( n - n_1 , \tau - \tau_1   
\right)| d \tau_1 
\\
& = \sum_{n_{1} } \int \frac{c_u\left( n_1, \tau_1 
\right)}{\langle n_1 \rangle ^s \langle |\tau_1| - n_1^{2} | \rangle ^{b}}
\\
& \times \frac{c_{v}\left( n - n_1, \tau - \tau_1 \right)}{\langle n -
n_1 \rangle ^s\ \langle |\tau - \tau_1 | -  (n - n_1)^{2} \rangle^{b}}
  \ d \tau_1 
\end{split}
\end{equation}
%
%
where for clarity of notation we have introduced 
%
%
%
\begin{equation*}
\begin{split}
\langle k \rangle \doteq 1 + |k|
\end{split}
\end{equation*}
%
%
and
%
\begin{equation*}
	\begin{split}
		c_h(n, \tau) =
			\langle n \rangle ^s \langle |\tau| - n^{2} \rangle ^{b} | \wh{h}\left( n, \tau \right) |.
	\end{split}
\end{equation*}
%
%
From our work above, it follows that 
%
%
\begin{equation}
	\label{convo-est-starting-pnt}
	\begin{split}
		 & \langle n \rangle^s \langle \tau - n^{2} \rangle^{-a} | \wh{uv}\left( 
		n, \tau \right) |
		\\
		& \le \langle |\tau| - n^{2} \rangle^{-a}
		\sum_{n_{1}} \int \frac{\langle n \rangle^{s}}{\langle n_1 \rangle^s
    \langle n - n_1 \rangle^s} 
		\times \frac{c_f(n_1, \tau_1)}{\langle |\tau_1| - n_1^{2} \rangle ^{b}}
		\\
		& \times
		\frac{c_g(n - n_1, \tau - \tau_1 )}{\langle |\tau - \tau_1| - (n - n_1)^{2}
    \rangle^{b}}\ d \tau_1.
	\end{split}
\end{equation}
%
%
Hence, 
%
%
\begin{equation}
  \label{pre-fubini-int-form}
	\begin{split}
    |\text{lhs of} \ \eqref{duality-est}|
	& \lesssim \sum_{n} \int_{\tau} \phi(n, \tau) \langle n \rangle^s 
  \sum_{n_{1}}
  \int_{\tau_{1}} c_f(n_1, \tau_1)
		c_g(n - n_1, \tau - \tau_1 )
		\\
    & \times \frac{\langle n \rangle ^{s}}{\langle n_{1} \rangle ^{s} \langle
    n-n_{1} \rangle ^{s}} \times \frac{1}{\langle |\tau| - n^{2} \rangle
    ^{a}\langle |\tau_{1}|-n_{1}^{2} \rangle ^{b}\langle | \tau -
    \tau_{1}|-(n - n_{1})^{2}
    \rangle ^{b}} d \tau_1 d \tau.
	\end{split}
\end{equation}
%
Let $A \subset \rr^{2} \times \zz^{2}$, and $\chi_{A}(\tau, \tau_{1}, n, n_{1})$
be its
characteristic function. Then by Cauchy-Schwartz in
$\tau_{1}, \xi_{1}$
\begin{equation*}
	\begin{split}
    & \sum_{n} \int_{\tau}   \sum_{n_{1}}
    \int_{\tau_{1}} \chi_{A}
    \phi(n, \tau) \langle n \rangle^s \langle \tau - n^{2} \rangle^{-a}
  c_f(n_1, \tau_1)
		c_g(n - n_1, \tau - \tau_1 )
		\\
    & \times \frac{\langle n \rangle ^{s}}{\langle n_{1} \rangle ^{s} \langle
    n-n_{1} \rangle ^{s}} \times \frac{1}{\langle |\tau| - n^{2} \rangle
    ^{a}\langle |\tau_{1}|-n_{1}^{2} \rangle ^{b}\langle | \tau -
    \tau_{1}|-(n - n_{1})^{2}
    \rangle ^{b}} d \tau_1 d \tau.
	\end{split}
\end{equation*}
%
is bounded by 
%
%
\begin{equation}
	\label{10g}
	\begin{split}
    & \sum_{n} \int_{\tau} \phi(n, \tau) \langle | \tau | - n^{2} \rangle
    ^{-a} \langle n \rangle ^{s}
    \\
    & \times \left( \sum_{n_{1}} \int_{\tau_{1}}
    \frac{\chi_{A}}{\langle n_{1} \rangle ^{2s} \langle n-n_{1} \rangle ^{2s} \langle |
    \tau_{1} | - n_{1}^{2}\rangle  \langle | \tau - \tau_{1} | -
    (n - n_{1})^{2} \rangle } d \tau_{1} \right)^{1/2}
    \\
    & \times \left( \sum_{n_{1}} \int_{\tau_{1}} c_{u}^{2}(n, \tau_{1})
    c_{v}^{2}(n - n_{1}, \tau - \tau_{1}) d \tau_{1} \right)^{1/2} d \tau
  \end{split}
\end{equation}
%
%
Applying Cauchy-Schwartz again, \eqref{10g} is bounded by
%
%
\begin{equation*}
  \begin{split}
    & \|\left( \sum_{n_{1} }\int_{\tau_{1} } c_{u}^{2}(n_1, \tau_1)
  c_{v}^{2} (n - n_1, \tau - \tau_{1} ) d \tau_1  \right)^{1/2} \|_{L^{2}(\zz \times
		\rr)}
		\\
    & \times  \|\phi(n, \tau) \langle | \tau | - n^{2} \rangle ^{-a} \langle n
    \rangle ^{s}
		\\
    & \times \left( \sum_{n_{1}} \int_{\tau_{1}} \frac{\chi_{A}}{ \langle n_{1}
    \rangle ^{2s} \langle n-n_{1} \rangle ^{2s} \langle | \tau_{1}|-n_{1}^{2}
    \rangle \langle  |\tau -
    \tau_{1} | -(n - n_{1})^{2}
    \rangle } d \tau_1 \right)^{1/2} \|_{L^2(\zz \times \rr)}
		\\
    & = \|u\|_{X^{s}} \|v\|_{X^{s}} \label{holder-term}
     \|\phi(n, \tau)     \\
    & \times \left( \langle | \tau | - n^{2} \rangle ^{-2a} \langle n
    \rangle ^{2s}
    \sum_{n_{1}} \int_{\tau_{1}} \frac{\chi_{A}}{ \langle n_{1} \rangle ^{2s} \langle
n-n_{1} \rangle ^{2s}  \langle | \tau_{1}|-n_{1}^{2} \rangle \langle  |\tau -
    \tau_{1} | -(n - n_{1})^{2}
    \rangle } d \tau_1 \right)^{1/2} \|_{L^2(\zz \times \rr)}.
  \end{split}
\end{equation*}
%
Applying H{\"o}lder, we bound this by 
%
%
\begin{equation}
  \label{integral-bound-1st-form-per}
	\begin{split}
    & \|u\|_{X^{s}} \|v\|_{X^{s}} \| \phi \|_{L^{2}_{n, \tau}}
    \\
    & \times \|\left( \langle | \tau | - n^{2} \rangle ^{-2a} \langle n
    \rangle ^{2s}
    \sum_{n_{1}} \int_{\tau_{1}} \frac{\chi_{A}}{ \langle n_{1} \rangle ^{2s} \langle
n-n_{1} \rangle ^{2s} \langle | \tau_{1}|-n_{1}^{2} \rangle \langle  |\tau -
    \tau_{1} | -(n - n_{1})^{2}
    \rangle  } d \tau_1 \right)^{1/2} \|_{L^\infty_{n, \tau}}
	\end{split}
\end{equation}
%
%
Let us now return to the right hand side of \eqref{pre-fubini-int-form}.
Then by the change of variable $\lambda =
\tau - \tau_{1}$, we obtain
\begin{equation*}
	\begin{split}
    & \sum_{n} \int_{\tau}   \sum_{n_{1}}
    \int_{\tau_{1}} \chi_{A}
    \phi(n, \tau) \langle n \rangle^s \langle \tau - n^{2} \rangle^{-a}
  c_f(n_1, \tau_1)
		c_g(n - n_1, \lambda )
		\\
    & \times \frac{\langle n \rangle ^{s}}{\langle n_{1} \rangle ^{s} \langle
    n-n_{1} \rangle ^{s}} \times \frac{1}{\langle |\tau| - n^{2} \rangle
    ^{a}\langle |\tau_{1}|-n_{1}^{2} \rangle ^{b}\langle | \tau -
    \tau_{1}|-(n - n_{1})^{2}
    \rangle ^{b}} d \tau_1 d \tau
    \\
    & = \sum_{n} \int_{\tau}   \sum_{n_{1}}
    \int_{\lambda} \chi^{*}_{A}
    \phi(n, \tau) \langle n \rangle^s \langle \tau - n^{2} \rangle^{-a}
  c_f(n_1, \tau - \lambda)
		c_g(n - n_1, \lambda )
		\\
    & \times \frac{\langle n \rangle ^{s}}{\langle n_{1} \rangle ^{s} \langle
    n-n_{1} \rangle ^{s}} \times \frac{1}{\langle |\tau| - n^{2} \rangle
    ^{a}\langle |\tau - \lambda|-n_{1}^{2} \rangle ^{b}\langle |
    \lambda|-(n - n_{1})^{2}
    \rangle ^{b}} d \lambda  d \tau
	\end{split}
\end{equation*}
where 
%
%
\begin{equation}
  \label{change-of-var}
\begin{split}
  \chi^{*}_{A}(\tau, \lambda, n, n_{1}) =
  \chi_{A}(\tau, \tau - \lambda, n, n_{1}).
\end{split}
\end{equation}
%
%
Cauchy-Schwartz in
$\lambda, n$ then gives the bound
%
%
%
\begin{equation}
	\label{10g*}
	\begin{split}
    & \sum_{n} \int_{\tau} \phi(n, \tau) \langle | \tau | - n^{2} \rangle
    ^{-a} \langle n \rangle ^{s}
    \\
    & \times \left( \sum_{n_{1}} \int_{\lambda}
    \frac{\chi^{*}_{A}}{\langle n_{1} \rangle ^{2s} \langle n-n_{1} \rangle ^{2s} \langle |
    \tau - \lambda | - n_{1}^{2}\rangle  \langle | \lambda | -
    (n - n_{1})^{2} \rangle } d \lambda \right)^{1/2}
    \\
    & \times \left( \sum_{n_{1}} \int_{\lambda} c_{u}^{2}(n_{1}, \tau - \lambda)
    c_{v}^{2}(n - n_{1}, \lambda) d \lambda \right)^{1/2} d \tau
  \end{split}
\end{equation}
%
%
Applying Cauchy-Schwartz again, \eqref{10g*} is bounded by
%
%
\begin{equation*}
  \begin{split}
    & \|\left( \sum_{n_{1} }\int_{\tau_{1} } c_{u}^{2}(n_1, \tau - \lambda)
  c_{v}^{2} (n - n_1, \lambda ) d \tau_1  \right)^{1/2} \|_{L^{2}(\zz \times
		\rr)}
		\\
    & \times  \|\phi(n, \tau) \langle | \tau | - n^{2} \rangle ^{-a} \langle n
    \rangle ^{s}
		\\
    & \times \left( \sum_{n_{1}} \int_{\tau_{1}} \frac{\chi^{*}_{A}}{ \langle n_{1}
    \rangle ^{2s} \langle n-n_{1} \rangle ^{2s} \langle | \tau - \lambda|-n_{1}^{2}
    \rangle \langle  |\lambda | -(n - n_{1})^{2}
    \rangle } d \tau_1 \right)^{1/2} \|_{L^2(\zz \times \rr)}
		\\
    & = \|u\|_{X^{s}} \|v\|_{X^{s}} \label{holder-term*}
    \left \{ \sum_{n} \int_{\tau} |\phi(n, \tau)|^{2} \right .
    \\
    & \left. \times \sum_{n_{1}} \int_{\lambda} \frac{\chi^{*}_{A}
    \langle n \rangle ^{2s}
    }{ \langle n_{1} \rangle^{2s} \langle | \tau | - n^{2}
    \rangle ^{2a}  \langle
n-n_{1} \rangle ^{2s}  \langle | \tau - \lambda|-n_{1}^{2}
\rangle \langle  | \lambda | -(n - n_{1})^{2}
    \rangle } d \lambda d \tau \right \}^{1/2}.
  \end{split}
\end{equation*}
%
%
Applying Fubini and H{\"o}lder to the last term gives the bound
%
%
\begin{equation}
  \label{integral-bound-1st-form-per*}
	\begin{split}
    & \|u\|_{X^{s}} \|v\|_{X^{s}} \| \phi \|_{L^{2}_{n, \tau}}
    \\
    & \times \left( \sum_{n_{1}} \int_{\tau} \frac{\chi^{*}_{A}
    \langle n \rangle ^{2s}
    }{ \langle n_{1} \rangle^{2s} \langle | \tau | - n^{2}
    \rangle ^{2a}  \langle
n-n_{1} \rangle ^{2s}  \langle | \tau - \lambda|-n_{1}^{2}
\rangle \langle  | \lambda | -(n - n_{1})^{2}
    \rangle } d \tau  \right)^{1/2} \|_{L^\infty_{n, \lambda}}.
	\end{split}
\end{equation}
Again, we return to the right hand side of \eqref{pre-fubini-int-form}.
We seek to bound
\begin{equation*}
\begin{split}
  & \sum_{n} \int_{\tau}  \sum_{n_{1} }
  \int_{\tau_{1}} \chi_{A} \phi(n, \tau)
    c_f(n_1, \tau_1)
		c_g(n - n_1, \tau - \tau_1 )
		\\
    & \times \frac{\langle n \rangle ^{s}}{\langle n_{1} \rangle ^{s} \langle
    n-n_{1} \rangle ^{s}} \times \frac{1}{ \langle \tau - n^{2} \rangle^{a}
\langle |\tau| - n^{2} \rangle
    ^{b}\langle |\tau_{1}|-n_{1}^{2} \rangle ^{b}\langle | \tau|-n_{2}^{2}
    \rangle ^{b}} d \tau_1 d \tau 
   \end{split}
\end{equation*}
in a different manner than before. First, we apply 
Fubini, then Cauchy-Schwartz in $n_{1}, \tau_{1}$ to obtain the bound
%
%
\begin{equation*}
\begin{split}
  & \left[ \sum_{n_{1}} \int_{\tau_{1}} c_{f}^{2}(n_{1}, \tau_{1}) d \tau_{1}
  \right]^{1/2}
  \\
  & \times \left \{ \sum_{n_{1}} \int_{\tau_{1}}   
 \left[
 \sum_{n} \int_{\tau}
   \frac{\langle n \rangle ^{s}}{\langle n_{1} \rangle ^{s} \langle
   n - n_{1}\rangle ^{s}} \times \frac{\chi_{A} |\phi(n, \tau)| c_{g}(n -
   n_{1}, \tau - \tau_{1})
}{\langle | \tau | - n^{2} \rangle
  ^{a} \langle | \tau_{1} | - n_{1}^{2} \rangle ^{b} \langle | \tau -
  \tau_{1} | - (n - n_{1}^{2}) \rangle ^{b}} d \tau 
  \right]^{2} d \tau_{1} \right \}^{1/2}
  \\
  & = \| f \|_{X^{s}}
  \\
  & \times \left \{ \sum_{n_{1}} \int_{\tau_{1}}   
 \left[
 \sum_{n} \int_{\tau}
   \frac{\langle n \rangle ^{s}}{\langle n_{1} \rangle ^{s} \langle
   n - n_{1}\rangle ^{s}} \times \frac{\chi_{A}|\phi(n, \tau)| c_{g}(n -
   n_{1}, \tau - \tau_{1})
}{\langle | \tau | - n^{2} \rangle
  ^{a} \langle | \tau_{1} | - n_{1}^{2} \rangle ^{b} \langle | \tau -
  \tau_{1} | - (n - n_{1}^{2}) \rangle ^{b}} d \tau 
  \right]^{2} d \tau_{1}  \right \}^{1/2}
\end{split}
\end{equation*}
%
Applying Cauchy-Schwartz in $\tau, n$, we bound the last line by 
%
%
\begin{equation*}
\begin{split}
  & \left \{ \sum_{n_{1}} \int_{\tau_{1}}   
  \left [ \sum_{n} \int_{\tau}
  | \phi(n, \tau)|^{2} c_{g}^{2}(n - n_{1}, \tau - \tau_{1}) d \tau  
    \right ] \right . 
   \\
   & \left. \times \left [ \sum_{n} \int_{\tau} \frac{\langle n \rangle
   ^{2s}}{\langle n_{1} \rangle ^{2s} \langle n - n_{1}\rangle ^{2s}}
   \times \frac{\chi_{A}}{\langle | \tau | - n^{2} \rangle ^{2a} \langle | \tau_{1} |
   - n_{1}^{2} \rangle  \langle | \tau - \tau_{1} | - (n - n_{1}^{2})
   \rangle } d \tau  \right ] \right \}^{1/2}d \tau_{1} 
\end{split}
\end{equation*}
%
%
which by Holder is bounded by 
%
%
%
\begin{equation}
  \label{integral-bound-2nd-form-per}
\begin{split}
  & \| \sum_{n} \int_{\rr} \frac{\langle n \rangle ^{2s}}{\langle n_{1} \rangle ^{2s} \langle
  n - n_{1}\rangle ^{2s}}  \times \frac{\chi_{A}}{\langle | \tau | - n^{2} \rangle
  ^{2a} \langle | \tau_{1} | - n_{1}^{2} \rangle  \langle | \tau -
  \tau_{1} | - (n - n_{1}^{2}) \rangle } d \tau 
  \|_{L^{\infty}_{n_{1}, \tau_{1}}}^{1/2}
  \\
  & \times \|\phi\|_{L^{2}} \| g \|_{X^{s}}.
\end{split}
\end{equation}
%
%
Now consider the family $\{A_{j}\}_{1}^{k}, A_{j} \subset \rr^{2} \times
\zz^{2}$ with
$$\bigcup_{1}^{k} A_{j}= \rr^{2} \times
\zz^{2}.$$ From \eqref{integral-bound-1st-form-per},
\eqref{integral-bound-1st-form-per*},
\eqref{integral-bound-2nd-form-per}, and our preceding argumentation,
we see that the proof of \autoref{prop:bilin-est-real} reduces to showing that
either 
%
%
%
%
\begin{equation}
  \label{key-sup-estimate-per-1}
  \begin{split}
     \| \langle | \tau | - n^{2} \rangle ^{-2a} \langle n
    \rangle ^{2s}
    \sum_{n_{1}}
    \int_{\tau_{1}} \frac{\chi_{A_{j}}}{ \langle n_{1} \rangle ^{2s} \langle
n-n_{1} \rangle ^{2s} \langle | \tau_{1}|-n_{1}^{2} \rangle  \langle  |\tau -
    \tau_{1} | -(n - n_{1})^{2}
    \rangle  } d \tau_1  \|_{L^\infty_{n, \tau}} < \infty.
  \end{split}
\end{equation}
%
or
%%
\begin{equation}
  \label{key-sup-estimate-per-2}
\begin{split}
  & \| \frac{1}{\langle n_{1} \rangle ^{2s}
  \langle | \tau_{1} | - n_{1}^{2} \rangle
  ^{2a}} \sum_{n} \int_{\tau} \frac{\langle n \rangle ^{2s}}{\langle
  n - n_{1}\rangle ^{2s}}  \times \frac{\chi_{A_{j}}}{\langle | \tau | - n^{2} \rangle  \langle | \tau -
  \tau_{1} | - (n - n_{1}^{2}) \rangle } d \tau 
  \|_{L^{\infty}_{n_{1}, \tau_{1}}}
\end{split}
\end{equation}
or
%
%
\begin{equation}
  \label{key-sup-estimate-per-3}
\begin{split}
  \| \sum_{n_{1}} \int_{\tau} \frac{\chi^{*}_{A_{j}}
    \langle n \rangle ^{2s}
    }{ \langle n_{1} \rangle^{2s} \langle | \tau | - n^{2}
    \rangle ^{2a}  \langle
n-n_{1} \rangle ^{2s}  \langle | \tau - \lambda|-n_{1}^{2}
\rangle \langle  | \lambda | -(n - n_{1})^{2}
\rangle } d \tau  \|_{L^{\infty}_{n, \lambda}}
\end{split}
\end{equation}
%
%
for each $j \in \left\{ 1,\dots,k \right\}$. 
By the triangle inequality and the fact that 
%
%
\begin{equation*}
\begin{split}
& | \tau | =
\begin{cases}
  - \tau, \quad & \tau < 0, 
\\
\tau, \quad & \tau > 0
\end{cases}
\end{split}
\end{equation*}
%
%
it follows that the proof of \autoref{prop:bilinear-est} reduces to showing that
for any $j$, either 
%
%
\begin{equation}
  \label{sup-est-gen-per-1}
  \begin{split}
    \| \frac{ \langle n
    \rangle ^{2s}}{\langle \sigma \rangle ^{2a}}
    \sum_{n_{1}} \int_{\tau_{1}} \frac{\chi_{A_{j}}}{ \langle n_{1} \rangle ^{2s} \langle n-n_{1} \rangle ^{2s} 
    \langle \sigma_{1} \rangle \langle  \sigma_{2} \rangle }
    d \tau_1  \|_{L^{\infty}_{n, \tau}} < \infty
  \end{split}
\end{equation}
%
%
or 
\begin{equation}
  \label{sup-est-gen-per-2}
\begin{split}
  & \| \frac{1}{\langle n_{1} \rangle ^{2s}
  \langle \sigma_{1} \rangle
  ^{2a}} \sum_{n} \int_{\tau} \frac{\langle n \rangle ^{2s}}{\langle
  n - n_{1}\rangle ^{2s}}  \times \frac{\chi_{A_{j}}}{\langle
  \sigma \rangle  \langle \sigma_{2} \rangle } d \tau 
  \|_{L^{\infty}_{n_{1}, \tau_{1}}} < \infty
\end{split}
\end{equation}
%
or
\begin{equation}
  \label{sup-est-gen-per-3}
\begin{split}
  \| \sum_{n_{1}} \int_{\tau} \frac{\chi^{*}_{A_{j}}
    \langle n \rangle ^{2s}
    }{ \langle n_{1} \rangle^{2s} \langle
    n-n_{1} \rangle ^{2s} \langle \sigma^{*}  
    \rangle ^{2a}
    \langle \sigma_{1}^{*} \rangle
    \langle  \sigma_{2}^{*} \rangle  } d \tau  \|_{L^{\infty}_{n, \lambda}}
\end{split}
\end{equation}
%
%
where we consider cases
\begin{enumerate}[(i)]
    \item $ \sigma=\tau+n^2,\quad \sigma_1=\tau_1+n_1^2,\quad \sigma_2=\tau -
      \tau_1+(n - n_1)^2$,
\label{it-1}
    \item $ \sigma=\tau-n^2,\quad \sigma_1=\tau_1-n_1^2,\quad \sigma_2=\tau - \tau_1+(n - n_1)^2$,
\label{it-2}
    \item  $\sigma=\tau+n^2,\quad \sigma_1=\tau_1-n_1^2,\quad \sigma_2=\tau - \tau_1+(n - n_1)^2$,
      \label{it-3}
    \item $\sigma=\tau-n^2,\quad \sigma_1=\tau_1+n_1^2,\quad \sigma_2=\tau - \tau_1-(n - n_1)^2$,
\label{it-4}
    \item $\sigma=\tau+n^2,\quad \sigma_1=\tau_1+n_1^2,\quad \sigma_2=\tau - \tau_1-(n - n_1)^2$,
\label{it-5}
    \item $\sigma=\tau-n^2,\quad \sigma_1=\tau_1-n_1^2,\quad \sigma_2=\tau - \tau_1-(n - n_1)^2$.
\label{it-6}
\end{enumerate}
%
for \eqref{sup-est-gen-per-1} and \eqref{sup-est-gen-per-2}, and cases
%
\begin{enumerate}[(i)]
\item $ \sigma^{*}=\tau+n^2,\quad \sigma^{*}_1=\tau - \lambda+n_1^2,\quad
  \sigma^{*}_2=\lambda+(n - n_1)^2$, \label{it-1-star} \item $
  \sigma^{*}=\tau-n^2,\quad \sigma^{*}_1=\tau - \lambda-n_1^2,\quad
  \sigma^{*}_2=\lambda+(n - n_1)^2$, \label{it-2-star} \item
  $\sigma^{*}=\tau+n^2,\quad \sigma^{*}_1=\tau - \lambda-n_1^2,\quad
  \sigma^{*}_2=\lambda+(n - n_1)^2$, \label{it-3-star} \item
  $\sigma^{*}=\tau-n^2,\quad \sigma^{*}_1=\tau - \lambda+n_1^2,\quad
  \sigma^{*}_2=\lambda-(n - n_1)^2$, \label{it-4-star} \item
  $\sigma^{*}=\tau+n^2,\quad \sigma^{*}_1=\tau - \lambda+n_1^2,\quad
  \sigma^{*}_2=\lambda-(n - n_1)^2$, \label{it-5-star} \item
  $\sigma^{*}=\tau-n^2,\quad \sigma^{*}_1=\tau - \lambda-n_1^2,\quad
  \sigma^{*}_2= \lambda-(n - n_1)^2$.  \label{it-6-star}
  \end{enumerate}
for \eqref{sup-est-gen-per-3}.
%
%
\begin{framed}
\begin{remark}
Note that the cases $\sigma=\tau+n^2,\quad \sigma_1=\tau_1-n_1^2,\quad
\sigma_2=\tau - \tau_1-(n - n_1)^2$ and $\sigma=\tau-n^2,\quad
\sigma_1=\tau_1+n_1^2,\quad \sigma_2=\tau - \tau_1+(n - n_1)^2$ cannot occur, since
$\tau_1< 0, \tau-\tau_1< 0$ implies $\tau<0$ and $\tau_1\geq 0, \tau-\tau_1\geq
0$ implies $\tau\geq 0$. An analogous argument holds for $\sigma^{*},
\sigma_{1}^{*}$ and $\sigma_{2}^{*}$.
\end{remark}
\end{framed}
%
Observe that the transformation $(n, \tau, n_{1}, \tau_{1}) \mapsto -(n, \tau,
n_{1}, \tau_{1})$ reduces \eqref{it-3} to \eqref{it-4}, \eqref{it-2} to
\eqref{it-5}, and \eqref{it-1} to \eqref{it-6}. Furthermore, the change of
variables $\tau_{2} = \tau - \tau_{1}, n_{2} = n - n_{1}$, and the
transformation $(n, \tau, n_{2}, \tau_{2}) \mapsto - (n, \tau, n_{2},
\tau_{2})$ reduces \eqref{it-5} to \eqref{it-4}. Since $L^{2}$ is invariant
under change of variables and reflections, we may without loss of generality
restrict our attention to cases \eqref{it-4} and \eqref{it-6}.
 \subsubsection{Case \eqref{it-6}} 
\label{sssec:case-it-6}
Let 
%
%
\begin{align*}
A_1&=\{(n, n_1, \tau, \tau_1)\in A: n=0\},\\
A_2&=\{(n, n_1, \tau, \tau_1)\in A: n_1 = n \},\\
A_3&=\{(n, n_1, \tau, \tau_1)\in A: n_1=0 \},\\
A_4&=\{(n, n_1, \tau, \tau_1)\in A: n \neq 0, n_1 \neq 0 \text{ and } n_1 \neq n \}.
\end{align*} 
%
%
%
We seek to bound
\begin{equation*}
\begin{split}
  & \frac{1}{\langle n_{1} \rangle ^{2s}
  \langle \tau_{1} - n_{1}^{2} \rangle
  ^{2a}} \sum_{n } \int_{\rr} \frac{\langle n \rangle ^{2s}}{\langle
  n - n_{1}\rangle ^{2s}}  \times \frac{\chi_{A_{1}}}{\langle
  \tau - n^{2}  \rangle  \langle \tau - \tau_{1} - (n - n_{1})^{2}
  \rangle } d \tau 
\end{split}
\end{equation*}
which is equal to 
%
\begin{equation}
  \label{case-1-term-1-reduc}
\begin{split}
  & \frac{1}{\langle n_{1} \rangle ^{4s}
  \langle \tau_{1} - n_{1}^{2} \rangle
  ^{2a}} \int_{\rr} \frac{1}{\langle
  \tau  \rangle  \langle \tau - \tau_{1} - n_{1}^{2}
  \rangle } d \tau.
\end{split}
\end{equation}
%
Following Ginibre, Tsutsumi, Velo~\cite{Ginibre:1997fk}, Kenig, Ponce, Vega~\cite{Kenig:1996aa}, and others,
we now need the following Calculus lemma.
%
%%%%%%%%%%%%%%%%%%%%%%%%%%%%%%%%%%%%%%%%%%%%%%%%%%%%%
%
%
%				 Calculus Lemma
%
%
%%%%%%%%%%%%%%%%%%%%%%%%%%%%%%%%%%%%%%%%%%%%%%%%%%%%%
%
%
\begin{lemma}
	\label{lem:calc}
 %
 %
 For $r > 1/2$
\begin{equation*}
  \int_{\rr} \frac{1} {\langle  \theta \rangle^{r} \langle  a - \theta
  \rangle^{r}} d \theta \leq\frac{\log 2} {\langle a \rangle^{r}}.
\end{equation*}
 %
 %
 \end{lemma}

%
By \autoref{lem:calc}, \eqref{case-1-term-1-reduc} is bounded by
%
%
\begin{equation*}
\begin{split}
  & \frac{c}{\langle n_{1} \rangle ^{4s} \langle \tau_{1} - n_{1}^{2} \rangle
  ^{2a}
  \langle \tau_{1} + n_{1}^{2} \rangle
  }
\end{split}
\end{equation*}
%
%
which for $a \le b$ is bounded by
%
%
\begin{equation*}
\begin{split}
& \frac{c}{\langle n_{1} \rangle ^{4s} (\langle \tau_{1} - n_{1}^{2} \rangle
\langle \tau_{1} + n_{1}^{2} \rangle)^{2a}
  }
\end{split}
\end{equation*}
%
%
Applying the estimate (see appendix for the proof)
%
%
\begin{equation}
  \label{simp-est-lower-bound}
\begin{split}
[1 + |x-a|][1 + |x+a|] \ge 1 + 2|a|, \quad x \in \rr
\end{split}
\end{equation}
%
%
this is bounded by 
%
%
\begin{equation*}
\begin{split}
  \frac{C}{(1 + 2|n_{1}  | ) ^{4(s + a)}} < \infty, \qquad s > 1/4 -a.
\end{split}
\end{equation*}
%
\begin{framed}
\begin{remark}
This result is optimal for region $A_{1}$. To illustrate this, we now estimate
%
%
\begin{equation*}
  \begin{split}
     \frac{ \langle n
    \rangle ^{2s}}{\langle \tau - n^{2} \rangle ^{2a}}
    \sum_{n_{1}} \int_{\rr} \frac{\chi_{A_{1}}}{ \langle n_{1} \rangle ^{2s} \langle n-n_{1} \rangle ^{2s} 
    \langle \tau_{1} - n_{1}^{2} \rangle \langle  \tau - \tau_{1} -
    (n - n_{1})^{2} \rangle }
    d \tau_1 
  \end{split}
\end{equation*}
which reduces to 
\begin{equation}
  \label{pathological-equality-case-1}
  \begin{split}
    \frac{ 1}{\langle \tau \rangle ^{2a}}
    \sum_{n_{1}} \int_{\rr} \frac{\chi_{A_{1}}}{ \langle n_{1} \rangle ^{4s}
    \langle \tau_{1} - n_{1}^{2} \rangle \langle  \tau - \tau_{1} -
    n_{1}^{2} \rangle }
    d \tau_1.
  \end{split}
\end{equation}
%
%
%
%
Setting $\tau = 0$, we obtain
%
%
%
\begin{equation*}
\begin{split}
   \sum_{n_{1}} \langle & n_{1}\rangle ^{-4s} \int_{\rr} \frac{1}{\langle
   \tau_{1} - n_{1}^{2} \rangle ^{4b}}d \tau_{1}
   \\
   & = \sum_{n_{1}} \langle
  n_{1}\rangle ^{-4s} \int_{\rr} \frac{1}{\langle
   \tau' \rangle ^{4b}}d \tau'
   \\
   & \simeq \sum_{n_{1}} \langle n_{1} \rangle ^{-4s}, \quad b > 1/4
   \\
   & < \infty, \quad s > 1/4.
\end{split}
\end{equation*}
%
Hence, we cannot hope to bound \eqref{pathological-equality-case-1} for $s \le
1/4$ using this splitting. Instead, consider now
%
%
\begin{equation*}
\begin{split}
  \sum_{n_{1}} \int_{\rr} \frac{\chi^{*}_{A_{1}}}{\langle n_{1} \rangle
  ^{4s} \langle \tau - n^{2} \rangle ^{2a} \langle \tau_{1} - n_{1}^{2} \rangle
   \langle \tau - \tau_{1} - (n - n_{1})^{2} \rangle }
\end{split}
\end{equation*}
%
%
which reduces to 
\begin{equation*}
\begin{split}
  \sum_{n_{1}} \int_{\rr} \frac{\chi^{*}_{A_{1}}}{\langle n_{1} \rangle
  ^{4s} \langle \tau \rangle ^{2a} \langle \tau_{1} - n_{1}^{2} \rangle
   \langle \tau - \tau_{1} - n_{1}^{2} \rangle }
\end{split}
\end{equation*}
Applying \autoref{lem:calc} and \eqref{simp-est-lower-bound}, this is bounded by
%
\begin{equation*}
\begin{split}
  & \sum_{n_{1}} \frac{\chi^{*}_{A_{1}}}{\langle n_{1} \rangle ^{4s} \langle
  \tau_{1} - n_{1}^{2} \rangle  \langle \tau_{1} + n_{1}^{2} \rangle
  ^{2a}}
  \\
  & \le \sum_{n_{1}} \frac{\chi^{*}_{A_{1}}}{\langle n_{1} \rangle ^{4s}
  (\langle
  \tau_{1} - n_{1}^{2} \rangle \langle \tau_{1} + n_{1}^{2} \rangle )
  ^{2a}}, \qquad a \le b
  \\
  & C \sum_{n_{1}}  \frac{\chi_{A_{1}}^{*}}{(1 + 2| n_{1} | )
  ^{4(s + a )}}
  \\
  & < \infty, \qquad s > 1/4 - a.
\end{split}
\end{equation*}
%
%
%
\end{remark}
\end{framed}
%
Applying \autoref{lem:calc}, we now bound 
\begin{equation*}
  \begin{split}
    & \frac{ \langle n
    \rangle ^{2s}}{\langle \tau - n^{2} \rangle ^{2a}}
    \sum_{n_{1}} \int_{\rr} \frac{\chi_{A_{2}}}{ \langle n_{1} \rangle ^{2s} \langle n-n_{1} \rangle ^{2s} 
    \langle \tau_{1} - n_{1}^{2} \rangle \langle  \tau - \tau_{1} -
    (n - n_{1})^{2} \rangle }
    d \tau_1 
    \\
    & = 
   \langle \tau -n^{2} \rangle ^{-2a}\int_{\rr} \frac{1}{\langle \tau_{1} -
  n^{2} \rangle \langle
  \tau - \tau_{1}\rangle }d \tau_{1}
  \\
  & \lesssim 
  \langle \tau - n^{2} \rangle ^{-2a-2b} 
  \\
  & < \infty.
\end{split}
\end{equation*}
%
%
Similarly, we bound
%
%
\begin{equation}
\begin{split}
  & \frac{ \langle n
    \rangle ^{2s}}{\langle \tau - n^{2} \rangle ^{2a}}
    \sum_{n_{1}} \int_{\rr} \frac{\chi_{A_{3}}}{ \langle n_{1} \rangle ^{2s} \langle n-n_{1} \rangle ^{2s} 
    \langle \tau_{1} - n_{1}^{2} \rangle \langle  \tau - \tau_{1} -
    (n - n_{1})^{2} \rangle }
    d \tau_1 
    \\
  & = \langle \tau - n^{2} \rangle ^{-2a}
  \int_{\rr} \frac{1}{ \langle \tau_{1} \rangle  \langle \tau -
  \tau_{1} - n^{2} \rangle}
d \tau_1 
\\
  & \lesssim   \langle \tau - n^{2} \rangle ^{-2a-2b} 
  \\
  & < \infty.
	\end{split}
\end{equation}
%
%
We now 
partition $ A_{4}$ into two parts
\begin{align*}
A_{4,1}&=\{(n, n_1, \tau, \tau_1)\in A_3: |\tau_1-n_1^2|\leq|\tau-n^2|\},\\
A_{4,2}&=\{(n, n_1, \tau, \tau_1)\in A_3: |\tau-n^2|\leq|\tau_1-n_1^2| \}.
\end{align*} 
Furthermore, by the symmetry of the convolution, we may assume without loss of
generality that
$$|(\tau-\tau_1)-(n-n_1)^2|\leq|\tau_1-n_1^2|\}.$$
Then in region $A_{4,1}$
\begin{equation}
\begin{split}
  | \tau - n^{2} |
  & \ge \frac{1}{3}\left[ | \tau_{1} - n_{1}^{2} | + | \tau -
  \tau_{1} - (n - n_{1})^{2}
  | + | \tau - n^{2} | \right]
  \\
  & \ge \frac{1}{3} | - n_{1}^{2} - (n - n_{1})^{2} + n^{2} |
  \\
  & = \frac{2}{3} | n_{1} | | n - n_{1} |
  \\
  & \gtrsim | n_{1} |. 
\end{split}
\label{smoothing-per-4-1}
\end{equation}
%
%
Hence, applying apply \autoref{lem:calc} and \eqref{smoothing-per-4-1}
we obtain
%
%
%
%
\begin{equation}
  \label{region-a41}
\begin{split}
& \langle \tau - n^{2}  \rangle ^{-2a} \langle n
    \rangle ^{2s}
    \sum_{n_{1}} \int_{\rr} \frac{\chi_{A_{4,1}}}{ \langle n_{1} \rangle ^{2s} \langle n-n_{1} \rangle ^{2s} 
\langle \tau_{1} - n_{1}^{2}  \rangle \langle  \tau - \tau_{1} - (n -
n_{1})^{2}  \rangle}
d \tau_1 
\\
& \lesssim \langle \tau - n^{2} \rangle ^{-2a} \langle n \rangle ^{2s}
\sum_{n_{1} \in
\zz}  \frac{\chi_{A_{4,1}}}{\langle n_{1} \rangle ^{2s} \langle n - n_{1} \rangle
^{2s} \langle \tau - n^{2} - 2n_{1}^{2} + 2nn_{1}  \rangle }
\\
& \lesssim 
\sum_{n_{1} \in
\zz}  \frac{\langle n_1 \rangle ^{-2s} \langle n - n_{1} \rangle ^{-2s}}{\langle
n \rangle ^{-2s} \langle n_{1} \rangle
^{2a}} \times \frac{\chi_{A_{4,1}}}{\langle \tau - n^{2} - 2n_{1}^{2} + 2nn_{1}
\rangle }.
\end{split}
\end{equation}
%
%
We now need the following. 
%
%
%%%%%%%%%%%%%%%%%%%%%%%%%%%%%%%%%%%%%%%%%%%%%%%%%%%%%
%
%
%                Integer Bound
%
%
%%%%%%%%%%%%%%%%%%%%%%%%%%%%%%%%%%%%%%%%%%%%%%%%%%%%%
%
%
\begin{lemma}
  Let $n, n_1 \in \zz$ such that $n_{1} \neq 0$ and $n_{1} \neq n$.
  Then
  %
  %
  \begin{equation*}
  \begin{split}
    | n | \le | n - n_{1} | | n_{1} |.
  \end{split}
  \end{equation*}
  %
  %
\label{lem:integer-bound}
\end{lemma}
%
Hence,
%
\begin{equation}
  \label{growth-term-per}
\begin{split}
  \frac{\langle n \rangle ^{2s} \chi_{A_{4}}}{\langle n_{1} \rangle ^{2s} \langle n -
  n_{1} \rangle ^{2s}} \le \langle n_{1} \rangle ^{\gamma(s)},
  \quad 
  \gamma(s) = 
  \begin{cases} 0, \quad & s \ge 0
    \\
    4|s|, \quad & s < 0.
  \end{cases}
\end{split}
\end{equation}
%
%
%
%
Since $a \ge 0$, it follows from \eqref{growth-term-per} that 
%
\begin{equation}
  \label{growth-term-control-per}
  \frac{\langle n_1 \rangle ^{-2s} \langle n - n_{1} \rangle
  ^{-2s}\chi_{A_{4}}}{\langle
n \rangle ^{-2s} \langle n_{1} \rangle
^{2a}} \le 1, \quad s \ge -a/2
\end{equation}
%
%
which we use to bound the right hand side of \eqref{region-a41} by
%
%
\begin{equation*}
\begin{split}
\sum_{n_{1} \in
\zz} 
\frac{1}{\langle \tau - n^{2} - 2n_{1}^{2} + 2nn_{1}  \rangle }
\end{split}
\end{equation*}
%
%
%
which is finite for $b > 1/4$, due to the following lemma, which can be found in
Kenig, Ponce, and Vega
\cite{Kenig-Ponce-Vega-1996-Quadratic-forms-for-the-1-D-semilinear}.
\begin{lemma}
  \label{lem:sum-estimate}
If $\gamma>1/2$, then
\begin{equation}\label{CI2}
\sup_{(n,\tau)\in \zz \times \rr}\sum_{n_1\in \zz}\frac{1}{(1+|\tau\pm
n_1(n-n_1)|)^{\gamma}}<\infty. 
\end{equation}
\end{lemma}
%
Working now in region $A_{4,2}$, we seek to bound 
\begin{equation}
  \label{region-4-2}
\begin{split}
  &  \frac{1}{\langle n_{1} \rangle ^{2s}
  \langle \tau_{1} - n_{1}^{2} \rangle
  ^{2a}} \sum_{n} \int_{\rr} \frac{\langle n \rangle ^{2s}}{\langle
  n - n_{1}\rangle ^{2s}}  \times \frac{\chi_{A_{4,2}}}{\langle
  \tau - n^{2} \rangle  \langle \tau - \tau_{1} - (n - n_{1})^{2} \rangle } d \tau 
\end{split}
\end{equation}
%
%
Note that in region $A_{4,2}$
\begin{equation}
  \label{smoothing-per-4-2}
\begin{split}
  | \tau_{1} - n_{1}^{2} |
  & \ge \frac{1}{3}\left[ | \tau_{1} - n_{1}^{2} | + | \tau -
  \tau_{1} - (n - n_{1})^{2}
  | + | \tau - n^{2} | \right]
  \\
  & \ge \frac{1}{3} | - n_{1}^{2} - (n - n_{1})^{2} + n^{2} |
  \\
  & = \frac{2}{3} | n_{1} | | n - n_{1} |
  \\
  & \gtrsim | n_{1} |.
\end{split}
\end{equation}
Hence, applying
\autoref{lem:calc}, \eqref{growth-term-control-per}, and
\eqref{smoothing-per-4-2}, we bound \eqref{region-4-2} by
%
%
\begin{equation*}
\begin{split}
&  \frac{c}{\langle n_{1} \rangle ^{2s}
  \langle \tau_{1} - n_{1}^{2} \rangle
  ^{2a}} \sum_{n} \frac{\langle n \rangle ^{2s}}{\langle
  n - n_{1}\rangle ^{2s}}  \times \frac{\chi_{A_{4,2}}}{\langle
  \tau_{1} - 2nn_{1} + n_{1}^{2} \rangle } 
  \\
  & \lesssim 
  \sum_{n} \frac{\chi_{A_{4,2}}}{\langle
  \tau_{1} - 2nn_{1} + n_{1}^{2} \rangle },
  \quad  s \ge -a/2
  \end{split}
\end{equation*}
%
%
Since the right hand side is bounded for $b > 1/4$ by \autoref{lem:sum-estimate}, this
completes the proof for case \eqref{it-6}.
\subsubsection{Case \eqref{it-4}} 
\label{sssec:case-it-4}
Let 
%
%
\begin{align*}
B_1&=\{(n, n_1, \tau, \tau_1)\in B: n=0\},\\
B_2&=\{(n, n_1, \tau, \tau_1)\in B: n_1 = 0 \},\\
B_3&=\{(n, n_1, \tau, \tau_1)\in B: n \neq 0, n_1 \neq 0 \}.
\end{align*} 
%
%
We seek to bound

\begin{equation*}
\begin{split}
  \sum_{n_{1}} \int_{\rr} \frac{\chi^{*}_{B_{3}}
    \langle n \rangle ^{2s}
    }{ \langle n_{1} \rangle^{2s} \langle
    n-n_{1} \rangle ^{2s} \langle \tau - n^{2}    \rangle ^{2a}
    \langle \tau - \lambda + n_{1}^{2} \rangle
    \langle  \lambda + n_{1}^{2} \rangle  } d \tau  
\end{split}
\end{equation*}
which is equal to
\begin{equation*}
\begin{split}
  \sum_{n_{1}} \int_{\rr} \frac{1}
    { \langle n_{1} \rangle^{4s} \langle \tau    \rangle ^{2a}
    \langle \tau - \lambda + n_{1}^{2} \rangle
    \langle  \lambda + n_{1}^{2} \rangle  } d \tau  
\end{split}
\end{equation*}
which by \eqref{growth-term-per} and \autoref{lem:calc} is bounded by
%
%
\begin{equation*}
\begin{split}
  \sum_{n_{1}} \frac{1}
  { \langle n_{1} \rangle^{4s} \langle n_{1}^{2} - \lambda   \rangle ^{2a}
  \langle n_{1}^{2} + \lambda \rangle
     } d \tau  .
\end{split}
\end{equation*}
%
%
Applying estimate \eqref{simp-est-lower-bound}, this is bounded by
\begin{equation*}
\begin{split}
  & \sum_{n_{1}} \frac{c}
  { ( 1 + 2 |n_{1} | )^{4(s+a)}} 
  \\
  & < \infty, \qquad s > 1/4 -a 
\end{split}
\end{equation*}

\begin{framed}
  \begin{remark}
    This result cannot be improved. For pedagogical purposes, we now estimate 
%
%
\begin{equation}
  \label{pathological-equality}
\begin{split}
   \langle \tau \rangle ^{-2a} \sum_{n_{1}} \langle
  n_{1}\rangle ^{-4s} \int_{\rr} \frac{\chi_{B_{1}}}{\langle \tau_{1} + n_{1}^{2} \rangle \langle
  \tau - \tau_{1} - n_{1}^{2}\rangle }d \tau_{1}.
\end{split}
\end{equation}
Applying \autoref{lem:calc}, we obtain the bound
%
%
\begin{equation*}
\begin{split}
  c  \langle \tau \rangle
  ^{-2a-2b} \sum_{n_{1}} \langle n_{1} \rangle
  ^{-4s} 
  & \lesssim \sum_{n_{1}} \langle n_{1} \rangle ^{-4s}
  \\
  & < \infty, \quad s > \frac{1}{4}.
\end{split}
\end{equation*}
%
%
Note that we cannot improve our lower bound for $s$ using this splitting.
To see this, we set $\tau = 0$
in the right hand side of \eqref{pathological-equality} and obtain
%
%
%
\begin{equation*}
\begin{split}
   \sum_{n_{1}} \langle
  & n_{1}\rangle ^{-4s} \int_{\rr} \frac{1}{\langle \tau_{1} + n_{1}^{2} \rangle \langle
   \tau_{1} + n_{1}^{2}\rangle }d \tau_{1} 
   \\
   & = \sum_{n_{1}} \langle
  n_{1}\rangle ^{-4s} \int_{\rr} \frac{1}{\langle
   \tau' \rangle ^{4b}}d \tau'
   \\
   & \simeq \sum_{n_{1}} \langle n_{1} \rangle ^{-4s}, \quad b > 1/4
   \\
   & < \infty, \quad s > 1/4.
\end{split}
\end{equation*}
%
%
%
Hence, we now try to bound
\begin{equation}
\begin{split}
  & \frac{1}{\langle n_{1} \rangle ^{2s}
  \langle \tau_{1} + n_{1}^{2} \rangle
  ^{2a}} \sum_{n } \int_{\rr} \frac{\langle n \rangle ^{2s}}{\langle
  n - n_{1}\rangle ^{2s}}  \times \frac{\chi_{B_{1}}}{\langle
  \tau - n^{2}  \rangle  \langle \tau - \tau_{1} - (n - n_{1})^{2}
  \rangle } d \tau 
\end{split}
\end{equation}
instead, and look to obtain a better result. This is equal to 
%
\begin{equation*}
\begin{split}
  & \frac{1}{\langle n_{1} \rangle ^{4s}
  \langle \tau_{1} + n_{1}^{2} \rangle
  ^{2a}} \int_{\rr} \frac{1}{\langle
  \tau  \rangle  \langle \tau - \tau_{1} - n_{1}^{2}
  \rangle } d \tau
\end{split}
\end{equation*}
%
%
which by \autoref{lem:calc} is bounded by
%
%
\begin{equation*}
\begin{split}
  & \frac{c}{\langle n_{1} \rangle ^{4s}
  \langle \tau_{1} + n_{1}^{2} \rangle
  ^{2a + 2b}}
\\
& < \infty, \qquad s \ge 0. 
\end{split}
\end{equation*}
%
%
We remark that the case $n=0$ is easy to estimate for the Boussinesq. This is
because  $$\frac{n^{2}}{n^{2} + n^{4}} |_{n=0} = 0$$ while
$$\frac{n^{2}}{n^{2}} |_{n=0} = 1.$$ 
Hence, the existence of the extra $n = 0$ term 
for the $B_{4}$ equation requires some extra work to be done, compared to the
Boussinesq equation. 
\end{remark}
\end{framed}
Applying \autoref{lem:calc}, we now bound 
\begin{equation*}
  \begin{split}
    & \frac{ \langle n
    \rangle ^{2s}}{\langle \tau - n^{2} \rangle ^{2a}}
    \sum_{n_{1}} \int_{\rr} \frac{\chi_{B_{2}}}{ \langle n_{1} \rangle ^{2s} \langle n-n_{1} \rangle ^{2s} 
    \langle \tau_{1} + n_{1}^{2} \rangle \langle  \tau - \tau_{1} -
    (n - n_{1})^{2} \rangle }
    d \tau_1 
    \\
    & = 
   \langle \tau -n^{2} \rangle ^{-2a}\int_{\rr} \frac{1}{\langle \tau_{1} +
  n^{2} \rangle \langle
  \tau - \tau_{1}\rangle }d \tau_{1}
  \\
  & \lesssim 
  \langle \tau - n^{2} \rangle ^{-2a-2b} 
  \\
  & < \infty.
\end{split}
\end{equation*}
%
%
We now 
partition $ B_{3}$ into three parts
\begin{align*}
B_{3,1}&=\{(n, n_1, \tau, \tau_1)\in B_3:
|\tau-\tau_1-(n-n_1)^2|, |\tau_1+n_1^2| \le |\tau-n^2|\},\\
B_{3,2}&=\{(n, n_1, \tau, \tau_1)\in B_3:
|\tau-\tau_1-(n-n_1)^2|, |\tau-n^2| \le |\tau_1+n_1^2|\},\\
B_{3,3}&=\{(n, n_1, \tau, \tau_1)\in B_3: |\tau_{1}+n_{1}^2|, | \tau - n^{2} | \le |  \tau - \tau_{1} -
(n - n_{1})^{2} |\}.
\end{align*} 
Then in region $B_{3,1}$
\begin{equation}
\begin{split}
  | \tau - n^{2} |
  & \ge \frac{1}{3}\left[ | \tau_{1} + n_{1}^{2} | + | \tau -
  \tau_{1} - (n - n_{1})^{2}
  | + | \tau - n^{2} | \right]
  \\
  & \ge \frac{1}{3} |  n_{1}^{2} - (n - n_{1})^{2} + n^{2} |
  \\
  & = \frac{2}{3} | n_{1} | | n |
  \\
  & \gtrsim | n_{1} |.
\end{split}
\label{smoothing-per-3-1-case-6}
\end{equation}
%
%
Estimating first in region
$B_{3,1}$, we apply \autoref{lem:calc} and \eqref{smoothing-per-3-1-case-6}
to obtain
%
%
%
%
\begin{equation}
  \label{region-a31-case-6}
\begin{split}
& \langle \tau - n^{2}  \rangle ^{-2a} \langle n
    \rangle ^{2s}
    \sum_{n_{1}} \int_{\rr} \frac{\chi_{B_{3,1}}}{ \langle n_{1} \rangle ^{2s} \langle n-n_{1} \rangle ^{2s} 
\langle \tau_{1} - n_{1}^{2}  \rangle \langle  \tau - \tau_{1} - (n -
n_{1})^{2}  \rangle}
d \tau_1 
\\
& \lesssim \langle \tau - n^{2} \rangle ^{-2a} \langle n \rangle ^{2s}
\sum_{n_{1} \in
\zz}  \frac{\chi_{B_{3,1}}}{\langle n_{1} \rangle ^{2s} \langle n - n_{1} \rangle
^{2s} \langle \tau - n^{2} - 2n_{1}^{2} + 2nn_{1}  \rangle }
\\
& \lesssim 
\sum_{n_{1}}  \frac{\langle n_1 \rangle ^{-2s} \langle n - n_{1} \rangle ^{-2s}}{\langle
n \rangle ^{-2s} \langle n_{1} \rangle
^{2a}} \times \frac{\chi_{B_{3,1}}}{\langle \tau - n^{2} - 2n_{1}^{2} + 2nn_{1}
\rangle }.
\end{split}
\end{equation}
%
%
Since $a \ge 0$, it follows from \eqref{growth-term-control-per} that the right
hand side is bounded by
%
%
%
%
\begin{equation*}
\begin{split}
  \sum_{n_{1}}   \frac{\chi_{B_{3,1}}}{\langle \tau - n^{2} - 2n_{1}^{2} + 2nn_{1}
\rangle }
\end{split}
\end{equation*}
%
%
%
%
%
%
which is finite for $b > 1/4$, due to \autoref{lem:sum-estimate}. 
Working now in region $B_{3,2}$ (ATTENTION: Farah estimates differently here. His
way is convoluted, and unnecessary; see page 960 of Farah periodic), we seek to estimate 
\begin{equation}
  \label{region-B-3-split-3}
\begin{split}
  &  \frac{1}{\langle n_{1} \rangle ^{2s}
  \langle \tau_{1} + n_{1}^{2} \rangle
  ^{2a}} \sum_{n} \int_{\rr} \frac{\langle n \rangle ^{2s}}{\langle
  n - n_{1}\rangle ^{2s}}  \times \frac{\chi_{B_{3,2}}}{\langle
  \tau - n^{2} \rangle  \langle \tau - \tau_{1} - (n - n_{1})^{2} \rangle
  } d \tau.
\end{split}
\end{equation}
%
Note that in region $B_{3,2}$
\begin{equation}
  \label{smoothing-per-3-2-case-6}
\begin{split}
  | \tau_{1} + n_{1}^{2} |
  & \ge \frac{1}{3}\left[ | \tau_{1} + n_{1}^{2} | + | \tau -
  \tau_{1} - (n - n_{1})^{2}
  | + | \tau - n^{2} | \right]
  \\
  & \ge \frac{1}{3} |  n_{1}^{2} - (n - n_{1})^{2} + n^{2} |
  \\
  & = \frac{2}{3} | n_{1} | | n |
  \\
  & \gtrsim | n_{1} |.
\end{split}
\end{equation}
%
Hence, applying
\autoref{lem:calc}, \eqref{growth-term-control-per}, and
\eqref{smoothing-per-3-2-case-6} to \eqref{region-B-3-split-3}, we obtain the bound
%
%
\begin{equation*}
  \begin{split}
    &  \frac{c}{\langle n_{1} \rangle ^{2s}
    \langle \tau_{1} + n_{1}^{2} \rangle
    ^{2a}} \sum_{n} \frac{\langle n \rangle ^{2s}}{\langle
    n - n_{1}\rangle ^{2s}}  \times \frac{\chi_{B_{3,2}}}{\langle
    \tau_{1} - 2nn_{1} + n_{1}^{2} \rangle } 
    \\
    & \lesssim 
    \sum_{n} \frac{\chi_{B_{3,2}}}{\langle
    \tau_{1} - 2nn_{1} + n_{1}^{2} \rangle },
    \quad s \ge -a/2.
  \end{split}
\end{equation*}
%
%
%
It remains to show that 
%
%
%
\begin{equation}
  \label{sum-bound}
\begin{split}
\sum_{n_{1} } \frac{\chi_{B_{3,2}}}{\langle \tau - n^{2} + 2nn_{1}
\rangle } < c, \qquad b > 1/2
\end{split}
\end{equation}
%
%
where $c$ does not depend on the choice of $\tau$ or $n$. 
%
%
To see this, note that if $\tau - n^{2} = 0$, the above follows easily, since
$n \neq 0$ in $B_{3,2}$.
Hence, assuming $\tau - n^{2} \neq 0$, we use the fact that $| \tau - n^{2} |
\gtrsim | n_{1} |$ and $n \neq 0$ in $B_{3,2}$ to obtain 
%
%
\begin{equation*}
\begin{split}
\sum_{n_{1}} \frac{\chi_{B_{3,2}}}{\langle \tau - n^{2} + 2nn_{1}
\rangle }
& = \sum_{n_{1}} \frac{\chi_{B_{3,2}}}{(1 + | \tau - n^{2} +
2nn_{1} |)}
\\
& \le \sum_{n_{1}} \frac{\chi_{B_{3,2}}}{1 + | \tau - n^{2} +
2nn_{1} |}, \quad b \ge 1/2
\\
& \le \sum_{n_{1}} \frac{\chi_{B_{3,2}}}{1 + | \tau - n^{2}
| | 1 + \frac{2nn_{1}}{\tau - n^{2}} |}
\\
& \lesssim \sum_{n_{1}} \frac{\chi_{B_{3,2}}}{1 + |n_{1}|
| 1 + \frac{2nn_{1}}{\tau - n^{2}} |}
\\
& \le \sup_{a \in \rr} \sum_{n_{1}} \frac{\chi_{B_{3,2}}}{1 + |n_{1}|
| 1 + an_{1}| }
\\
& = \sum_{n_{1}} \frac{\chi_{B_{3,2}}}{1 + |n_{1}|}
\\
& < c, \qquad b > 1/2.
\end{split}
\end{equation*}
%
%
It remains to handle region $B_{3,3}$. It will be enough to bound
%
%
\begin{equation}
  \label{region-B-3-star-split}
\begin{split}
   \sum_{n_{1}} \int_{\rr} \frac{\chi^{*}_{B_{3,3}}
    \langle n \rangle ^{2s}
    }{ \langle n_{1} \rangle^{2s} \langle  \tau  - n^{2}
    \rangle ^{2a}  \langle
n-n_{1} \rangle ^{2s}  \langle  \tau - \lambda+n_{1}^{2}
\rangle \langle   \lambda  -(n - n_{1})^{2}
\rangle } d \tau.
\end{split}
\end{equation}
%
Due to the presence of $\chi^{*}_{B_{3,3}}$ factor, we have the restriction
%
%
\begin{equation*}
\begin{split}
& |\tau - \lambda +n_{1}^2|, | \tau - n^{2} | \le |  \lambda -
(n - n_{1})^{2} | \text{ and }  n \neq 0, n_1 \neq 0.
\end{split}
\end{equation*}
%
It follows that
\begin{equation}
  \label{smoothing-per-3-3-case-6}
\begin{split}
  | \lambda - (n - n_{1})^{2} |
  & \ge \frac{1}{3}\left[ | \tau - \lambda + n_{1}^{2} | + | \lambda - (n - n_{1})^{2}
  | + | \tau - n^{2} | \right]
  \\
  & \ge \frac{1}{3} |  n_{1}^{2} - (n - n_{1})^{2} + n^{2} |
  \\
  & = \frac{2}{3} | n_{1} | | n |
  \\
  & \gtrsim | n_{1} |.
\end{split}
\end{equation}
Hence, applying
\eqref{growth-term-control-per}, \autoref{lem:calc}, and
\eqref{smoothing-per-3-3-case-6}, we bound \eqref{region-B-3-star-split} by
%
%
\begin{equation*}
\begin{split}
   & \sum_{n_{1}} \int_{\rr} \frac{\chi^{*}_{B_{3,3}}
    }{ \langle  \tau  - n^{2}
    \rangle ^{2a}   \langle  \tau - \lambda+n_{1}^{2}
\rangle } d \tau, \quad s \ge -a/2.
\\
& \lesssim  \sum_{n_{1}} \frac{\chi_{B_{3,3}^{*}}}{\langle n_{1}^{2} +
n^{2} - \lambda \rangle^{2a }}
\end{split}
\end{equation*}
which is bounded for $a > 1/4$ by
\autoref{lem:sum-estimate}. This completes the proof of
\autoref{prop:bilinear-est}. \qquad \qedsymbol
%
%
\subsection{The Non-Periodic Case} 
\label{ssec:non-periodic-case}
We now introduce the following spaces. 
%
%
\begin{definition}
  Let $S(\rr^{2})$ denote the space of Schwartz functions on
  $\rr^{2}$.  For $s, b \in \rr$, $\mathcal{X}_{s,b}$
  denotes the completion of $S(\rr^{2})$ with
  respect to the norm
  %
  %
  \begin{equation}
  \begin{split}
    \|F\|_{\mathcal{X}_{s,b}} = \left( \sum_{n} (1 + \xi^{2})^{s} \int_{\rr}
    (1 + | | \tau | - \xi^{2} |) \wh{F}(n, \tau) d \tau\right)^{1/2}.
  \end{split}
  \label{eqn:bous-norm-real}
  \end{equation}
  %
  %
  %
  %
\end{definition}
%
%
We need only establish the following bilinear estimate. All other arguments are
analogous to those in the periodic case.
%
\begin{proposition}[Theorem 1.1 in Farah periodic]
\label{prop:bilin-est-real}
If $b > 1/2$, $a > 1/4$, and $s \ge -a/2$, 
  then there exists $c > 0$ depending only on $a$, $b$, and $s$ such that
  %
  %
  \begin{equation*}
  \begin{split}
    \| uv \|_{\mathcal{X}_{s,-a}} \le c \| u \|_{\mathcal{X}_{s,b}} \| v \|_{\mathcal{X}_{s,b}}.
  \end{split}
  \end{equation*}
  %
  %
\end{proposition}


\subsubsection{Proof of \autoref{prop:bilin-est-real}.} 
\label{sssec:bilin-est-real}
Since $\| f \|_{\mathcal{X}_{s,-a}} \le \| f \|_{\mathcal{X}_{s, -a'}}$ for $a \ge a' \ge 0$, we assume
without loss of generality that $1/4 > a \le b$ and $b > 1/2$ throughout. 
By duality, it suffices to show that 
%
%%
\begin{equation}
	\label{duality-est-real}
	\begin{split}
    |	\int_{\rr} \int_{\rr} (1 + |\xi|)^{s}
		\phi(\xi, \tau) \wh{uv}(\xi, \tau)(1 
    + | |\tau| - \xi^{2} |^{-a}) d \tau d \xi | \lesssim \|u\|_{\mathcal{X}_{s,b}}
    \|v\|_{\mathcal{X}_{s,b}}
    \|\phi \|_{L^{2}_{\xi, \tau}}.
	\end{split}
\end{equation}
Note first that $|\wh{uv}(\xi, \tau) |  = | \wh{u} *  \wh{v} 
(\xi, \tau)|$. From this it follows that
%
%
\begin{equation}
	\label{non-lin-rep-real}
	\begin{split}
		| \wh{uv}(\xi, \tau)|
    & = | \sum_{\xi_{1} \in \zz }  \int
    \wh{u}\left( \xi_1,  \tau_1 \right) \wh{v}\left( \xi - \xi_1 , \tau - \tau_1   
\right) d \tau_1 |
\\
& \le  \sum_{\xi_{1} \in \zz }  \int
    |\wh{u}\left( \xi_1,  \tau_1 \right)| |\wh{v}\left( \xi - \xi_1 , \tau - \tau_1   
\right)| d \tau_1 
\\
& = \sum_{\xi_1 \in \zz } \int \frac{c_u\left( \xi_1, \tau_1 
\right)}{\langle \xi_1 \rangle ^s \langle |\tau_1| - \xi_1^{2} | \rangle ^{b}}
\\
& \times \frac{c_{v}\left( \xi - \xi_1, \tau - \tau_1 \right)}{\langle \xi -
\xi_1 \rangle ^s\ \langle |\tau - \tau_1 | -  (\xi - \xi_1)^{2} \rangle^{b}}
  \ d \tau_1 
\end{split}
\end{equation}
%
%
where for clarity of notation we have introduced 
%
%
%
\begin{equation*}
\begin{split}
\langle k \rangle \doteq 1 + |k|
\end{split}
\end{equation*}
%
%
and
%
\begin{equation*}
	\begin{split}
		c_h(\xi, \tau) =
			\langle \xi \rangle ^s \langle |\tau| - \xi^{2} \rangle ^{b} | \wh{h}\left( \xi, \tau \right) |.
	\end{split}
\end{equation*}
%
%
From our work above, it follows that 
%
%
\begin{equation}
	\label{convo-est-starting-pnt-real}
	\begin{split}
		 & \langle \xi \rangle^s \langle \tau - \xi^{2} \rangle^{-a} | \wh{uv}\left( 
		\xi, \tau \right) |
		\\
		& \le \langle |\tau| - \xi^{2} \rangle^{-a}
		\sum_{\xi_1 \in \zz} \int \frac{\langle \xi \rangle^{s}}{\langle \xi_1 \rangle^s
    \langle \xi - \xi_1 \rangle^s} 
		\times \frac{c_f(\xi_1, \tau_1)}{\langle |\tau_1| - \xi_1^{2} \rangle ^{b}}
		\\
		& \times
		\frac{c_g(\xi - \xi_1, \tau - \tau_1 )}{\langle |\tau - \tau_1| - (\xi - \xi_1)^{2}
    \rangle^{b}}\ d \tau_1.
	\end{split}
\end{equation}
%
%
Hence, 
%
%
\begin{equation}
  \label{pre-fubini-int-form-real}
	\begin{split}
    |\text{lhs of} \ \eqref{duality-est-real}|
    & \lesssim \int_{\rr} \int_{\rr}     \int_{\rr}  \int_{\rr} \phi(\xi, \tau)
    c_f(\xi_1, \tau_1)
		c_g(\xi - \xi_1, \tau - \tau_1 )
		\\
    & \times \frac{\langle \xi \rangle ^{s}}{\langle \xi_{1} \rangle ^{s} \langle
    \xi-\xi_{1} \rangle ^{s}} \times \frac{1}{ \langle \tau - \xi^{2} \rangle^{a}
\langle |\tau| - \xi^{2} \rangle
    ^{b}\langle |\tau_{1}|-\xi_{1}^{2} \rangle ^{-b}\langle | \tau|-\xi_{2}^{2}
    \rangle ^{b}} d \tau_1 d \xi_{1} d \tau d \xi.
	\end{split}
\end{equation}
%
%Let $A \subset \rr^{4}$, and $\chi_{A}(\xi, \tau, \xi_{1}, \tau_{1})$ be its
%characteristic function. Then by Cauchy-Schwartz in
%$\tau_{1}, \xi_{1}$, we bound
%%
%%
%%
%\begin{equation*}
%\begin{split}
  %& \int_{\rr} \int_{\rr}     \int_{\rr}  \int_{\rr} \chi_{A} \phi(\xi, \tau)
    %c_f(\xi_1, \tau_1)
		%c_g(\xi - \xi_1, \tau - \tau_1 )
		%\\
    %& \times \frac{\langle \xi \rangle ^{s}}{\langle \xi_{1} \rangle ^{s} \langle
    %\xi-\xi_{1} \rangle ^{s}} \times \frac{1}{ \langle \tau - \xi^{2} \rangle^{a}
%\langle |\tau| - \xi^{2} \rangle
    %^{b}\langle |\tau_{1}|-\xi_{1}^{2} \rangle ^{-b}\langle | \tau|-\xi_{2}^{2}
    %\rangle ^{b}} d \tau_1 d \xi_{1} d \tau d \xi
%\end{split}
%\end{equation*}
%%
%%
%by
%%
%%
%\begin{equation}
	%\label{10g-real}
	%\begin{split}
    %& \int_{\rr} \int_{\rr} \phi(\xi, \tau) \langle | \tau | - \xi^{2} \rangle
    %^{-a} \langle \xi \rangle ^{s}
    %\\
    %& \times \left( \int_{\rr} \int_{\rr}
    %\frac{\chi_{A}}{\langle \xi_{1} \rangle ^{2s} \langle \xi-\xi_{1} \rangle ^{2s} \langle |
    %\tau_{1} | - \xi_{1}^{2}\rangle  \langle | \tau - \tau_{1} | -
    %(\xi - \xi_{1})^{2} \rangle } d \tau_{1} d \xi_{1} \right)^{1/2}
    %\\
    %& \times \left( \int_{\rr} \int_{\rr} c_{u}^{2}(\xi, \tau_{1})
    %c_{v}^{2}(\xi - \xi_{1}, \tau - \tau_{1}) d \tau_{1} d \xi_{1}
    %\right)^{1/2} d \tau d
    %\xi.
  %\end{split}
%\end{equation}
%%
%%
%Applying Cauchy-Schwartz in $\tau, \xi$, \eqref{10g-real} is bounded by
%%
%%
%\begin{equation*}
  %\begin{split}
    %& \|\left( \int_{\rr} \int_{\rr } c_{u}^{2}(\xi_1, \tau_1)
  %c_{v}^{2} (\xi - \xi_1, \tau - \tau_{1} ) d \tau_1 d \xi_{1}  \right)^{1/2} \|_{L^{2}(\zz \times
		%\rr)}
		%\\
    %& \times  \|\phi(\xi, \tau) \langle | \tau | - \xi^{2} \rangle ^{-a} \langle \xi
    %\rangle ^{s}
		%\\
    %& \times \left( \int_{\rr} \int_{\rr} \frac{\chi_{A}}{ \langle \xi_{1}
    %\rangle ^{2s} \langle \xi-\xi_{1} \rangle ^{2s} \langle | \tau_{1}|-\xi_{1}^{2}
    %\rangle \langle  |\tau -
    %\tau_{1} | -(\xi - \xi_{1})^{2}
    %\rangle } d \tau_1 d \xi_{1} \right)^{1/2} \|_{L^2(\zz \times \rr)}
		%\\
    %& = \|u\|_{\mathcal{X}_{s,b}} \|v\|_{\mathcal{X}_{s,b}} \label{holder-term-real}
     %\|\phi(\xi, \tau)     \\
    %& \times \left( \langle | \tau | - \xi^{2} \rangle ^{-2a} \langle \xi
    %\rangle ^{2s}
    %\int_{\rr} \int_{\rr} \frac{\chi_{A}}{ \langle \xi_{1} \rangle ^{2s} \langle
%\xi-\xi_{1} \rangle ^{2s}  \langle | \tau_{1}|-\xi_{1}^{2} \rangle \langle  |\tau -
    %\tau_{1} | -(\xi - \xi_{1})^{2}
    %\rangle } d \tau_1 d \xi_{1} \right)^{1/2} \|_{L^2(\zz \times \rr)}.
  %\end{split}
%\end{equation*}
%%
%Applying H{\"o}lder, we bound this by 
%%
%%
%\begin{equation}
  %\label{integral-bound-1st-form}
	%\begin{split}
    %& \|u\|_{\mathcal{X}_{s,b}} \|v\|_{\mathcal{X}_{s,b}} \| \phi \|_{L^{2}_{\xi, \tau}}
    %\\
    %& \times \|\left( \langle | \tau | - \xi^{2} \rangle ^{-2a} \langle \xi
    %\rangle ^{2s}
    %\sum_{n_{1}} \int_{\rr} \frac{\chi_{A}}{ \langle \xi_{1} \rangle ^{2s} \langle
%\xi-\xi_{1} \rangle ^{2s} \langle | \tau_{1}|-\xi_{1}^{2} \rangle \langle  |\tau -
    %\tau_{1} | -(\xi - \xi_{1})^{2}
    %\rangle  } d \tau_1 \right)^{1/2} \|_{L^\infty_{\xi, \tau}}.
	%\end{split}
%\end{equation}
%%
%%
%%Hence, to complete the proof, it will be enough
%%to show that 
%%%
%%%
%%%
%%%
%%\begin{equation}
  %%\label{key-sup-estimate-real-1}
	%%\begin{split}
		 %%\| \langle | \tau | - \xi^{2} \rangle ^{-2a} \langle \xi
    %%\rangle ^{2s}
%%\sum_{\xi_{1} \in \zz} \int_{\rr} \frac{1}{  \langle \xi_{1} \rangle ^{2s} \langle
%%\xi-\xi_{1} \rangle ^{2s} \langle | \tau_{1}|-\xi_{1}^{2} \rangle  \langle  |\tau -
    %%\tau_{1} | -(\xi - \xi_{1}^{2}
    %%\rangle  } d \tau_1 \|_{L^\infty_{\xi, \tau}} < \infty.
	%%\end{split}
%%\end{equation}
%%
%%
%%By the triangle inequality and the fact that 
%%%
%%%
%%\begin{equation*}
%%\begin{split}
%%& | \tau | =
%%\begin{cases}
  %%- \tau, \quad & \tau < 0, 
%%\\
%%\tau, \quad & \tau > 0
%%\end{cases}
%%\end{split}
%%\end{equation*}
%%%
%%%
%%\eqref{key-sup-estimate-real} will be proved if we can bound the
%%$L^{\infty}_{\tau, n}$ norm of the quantity
%%%
%%%
%%\begin{equation}
  %%\label{sup-est-gen-real}
%%\begin{split}
			%%\langle \sigma \rangle ^{-2a} \langle n
    %%\rangle ^{2s}
%%\sum_{n_{1}} \int_{\rr} \frac{1}{ \langle n_{1} \rangle ^{2s} \langle n-n_{1} \rangle ^{2s} 
%%\langle \sigma_{1} \rangle \langle  \sigma_{2} \rangle }
%%d \tau_1 
	%%\end{split}
%%\end{equation}
%Let us now return to the right hand side of \eqref{pre-fubini-int-form-real}.
%Let $A \subset \rr^{4}$, and $\chi_{A}(\xi, \tau, \xi_{1}, \tau_{1})$ be its
%characteristic function, as before.  We seek to bound
%\begin{equation*}
%\begin{split}
  %& \int_{\rr} \int_{\rr}     \int_{\rr}  \int_{\rr} \chi_{A} \phi(\xi, \tau)
    %c_f(\xi_1, \tau_1)
		%c_g(\xi - \xi_1, \tau - \tau_1 )
		%\\
    %& \times \frac{\langle \xi \rangle ^{s}}{\langle \xi_{1} \rangle ^{s} \langle
    %\xi-\xi_{1} \rangle ^{s}} \times \frac{1}{ \langle \tau - \xi^{2} \rangle^{a}
%\langle |\tau| - \xi^{2} \rangle
    %^{b}\langle |\tau_{1}|-\xi_{1}^{2} \rangle ^{-b}\langle | \tau|-\xi_{2}^{2}
    %\rangle ^{b}} d \tau_1 d \xi_{1} d \tau d \xi
%\end{split}
%\end{equation*}
%in a different manner than before. First, we apply 
%Fubini, then Cauchy-Schwartz in $\xi_{1}, \tau_{1}$ to obtain the bound
%%
%%
%\begin{equation*}
%\begin{split}
  %& \left[ \int_{\rr} \int_{\rr} c_{f}^{2}(\xi_{1}, \tau_{1}) d \tau_{1} d
  %\xi_{1} \right]^{1/2}
  %\\
  %& \times \left \{ \int_{\rr} \int_{\rr}   
 %\left[
  %\int_{\rr} \int_{\rr}
   %\frac{\langle \xi \rangle ^{s}}{\langle \xi_{1} \rangle ^{s} \langle
   %\xi - \xi_{1}\rangle ^{s}} \times \frac{|\phi(\xi, \tau)| c_{g}(\xi -
   %\xi_{1}, \tau - \tau_{1})
%}{\langle | \tau | - \xi^{2} \rangle
  %^{a} \langle | \tau_{1} | - \xi_{1}^{2} \rangle ^{b} \langle | \tau -
  %\tau_{1} | - (\xi - \xi_{1}^{2}) \rangle ^{b}} d \tau d \xi 
  %\right]^{2} \right \}^{1/2}
  %\\
  %& = \| f \|_{X^{s}}
  %\\
  %& \times \left \{ \int_{\rr} \int_{\rr}   
 %\left[
  %\int_{\rr} \int_{\rr}
   %\frac{\langle \xi \rangle ^{s}}{\langle \xi_{1} \rangle ^{s} \langle
   %\xi - \xi_{1}\rangle ^{s}} \times \frac{|\phi(\xi, \tau)| c_{g}(\xi -
   %\xi_{1}, \tau - \tau_{1})
%}{\langle | \tau | - \xi^{2} \rangle
  %^{a} \langle | \tau_{1} | - \xi_{1}^{2} \rangle ^{b} \langle | \tau -
  %\tau_{1} | - (\xi - \xi_{1}^{2}) \rangle ^{b}} d \tau d \xi 
  %\right]^{2} d \tau_{1} d \xi_{1} \right \}^{1/2}
%\end{split}
%\end{equation*}
%%
%Applying Cauchy-Schwartz in $\tau, \xi$, we bound the last line by 
%%
%%
%\begin{equation*}
%\begin{split}
%& \left \{ \int_{\rr} \int_{\rr}   
  %\left [ \int_{\rr} \int_{\rr}
  %| \phi(\xi, \tau)|^{2} c_{g}^{2}(\xi - \xi_{1}, \tau - \tau_{1}) d \tau d \xi 
    %\right ] \right . 
   %\\
   %& \left. \times \left [ \int_{\rr} \int_{\rr} \frac{\langle \xi \rangle
   %^{2s}}{\langle \xi_{1} \rangle ^{2s} \langle \xi - \xi_{1}\rangle ^{2s}}
   %\times \frac{\chi_{A}}{\langle | \tau | - \xi^{2} \rangle ^{2a} \langle | \tau_{1} |
   %- \xi_{1}^{2} \rangle  \langle | \tau - \tau_{1} | - (\xi - \xi_{1}^{2})
   %\rangle } d \tau d \xi \right ] \right \}^{1/2}d \tau_{1} d \xi_{1}
%\end{split}
%\end{equation*}
%%
%%
%which by Holder is bounded by 
%%
%%
%%
%\begin{equation}
  %\label{integral-bound-2nd-form}
%\begin{split}
  %& \| \int_{\rr} \int_{\rr} \frac{\langle \xi \rangle ^{2s}}{\langle \xi_{1} \rangle ^{2s} \langle
  %\xi - \xi_{1}\rangle ^{2s}}  \times \frac{\chi_{A}}{\langle | \tau | - \xi^{2} \rangle
  %^{2a} \langle | \tau_{1} | - \xi_{1}^{2} \rangle  \langle | \tau -
  %\tau_{1} | - (\xi - \xi_{1}^{2}) \rangle } d \tau d \xi
  %\|_{L^{\infty}_{\xi_{1}, \tau_{1}}}^{1/2}
  %\\
  %& \times \|\phi\|_{L^{2}} \| g \|_{X^{s}}
%\end{split}
%\end{equation}
%
%
Now consider the family $\{A_{j}\}_{1}^{k}, A_{j} \subset \rr^{4}$ with
$$\bigcup_{1}^{k} A_{j}= \rr^{4}.$$ 
As in the periodic case, a combination of Cauchy-Schwartz, Fubini, and
H{\"o}lder to estimate the right hand side of \eqref{pre-fubini-int-form-real}
reduces the proof of \autoref{prop:bilin-est-real} to showing that

%From \eqref{integral-bound-1st-form},
%\eqref{integral-bound-2nd-form}, and our preceding argumentation,
%we see that the proof of \autoref{prop:bilin-est-real} reduces to showing that
%either 
%
%
%
%
\begin{equation}
  \label{key-sup-estimate-real}
  \begin{split}
     \| \langle  \frac{\langle \xi
     \rangle ^{2s}}{ | \tau | - \xi^{2} \rangle ^{2a}}\int_{\rr} \int_{\rr} \frac{\chi_{A_{j}}}{ \langle \xi_{1} \rangle ^{2s} \langle
\xi-\xi_{1} \rangle ^{2s}
\langle | \tau_{1}|-\xi_{1}^{2} \rangle  \langle  |\tau -
    \tau_{1} | -(\xi - \xi_{1})^{2}
    \rangle  } d \tau_1 d \xi_{1} \|_{L^\infty_{\xi, \tau}} < \infty.
  \end{split}
\end{equation}
%
or
%%
\begin{equation}
\begin{split}
  & \| \frac{1}{\langle \xi_{1} \rangle ^{2s}
  \langle | \tau_{1} | - \xi_{1}^{2} \rangle
  ^{2a}} \int_{\rr} \int_{\rr} \frac{\langle \xi \rangle ^{2s}}{\langle
  \xi - \xi_{1}\rangle ^{2s}}  \times \frac{\chi_{A_{j}}}{\langle | \tau | - \xi^{2} \rangle  \langle | \tau -
  \tau_{1} | - (\xi - \xi_{1}^{2}) \rangle } d \tau d \xi
  \|_{L^{\infty}_{\xi_{1}, \tau_{1}}} < \infty
\end{split}
\end{equation}
or
\begin{equation}
\begin{split}
  \| \int_{\rr} \int_{\rr} \frac{\chi^{*}_{A}
    \langle \xi \rangle ^{2s}
    }{ \langle \xi_{1} \rangle^{2s} \langle | \tau | - \xi^{2}
    \rangle ^{2a}  \langle
\xi-\xi_{1} \rangle ^{2s}  \langle | \tau - \lambda|-\xi_{1}^{2}
\rangle \langle  | \lambda | -(\xi - \xi_{1})^{2}
\rangle } d \tau d \xi_{1} \|_{L^{\infty}_{\xi, \lambda}} < \infty
\end{split}
\end{equation}

for each $j \in \left\{ 0,1,\dots,k \right\}$. 
By the triangle inequality and the fact that 
%
%
\begin{equation*}
\begin{split}
& | \tau | =
\begin{cases}
  - \tau, \quad & \tau < 0, 
\\
\tau, \quad & \tau > 0
\end{cases}
\end{split}
\end{equation*}
%
%
it follows that the proof of \autoref{prop:bilinear-est} reduces to showing that
for any $j$, either 
%
%
\begin{equation}
  \label{sup-est-gen-1}
  \begin{split}
    \| \frac{ \langle \xi
    \rangle ^{2s}}{\langle \sigma \rangle ^{2a}}
    \int_{\rr} \int_{\rr} \frac{\chi_{A_{j}}}{ \langle \xi_{1} \rangle ^{2s} \langle \xi-\xi_{1} \rangle ^{2s} 
    \langle \sigma_{1} \rangle \langle  \sigma_{2} \rangle }
    d \tau_1 d \xi_{1} \|_{L^{\infty}_{\xi, \tau}} < \infty
  \end{split}
\end{equation}
%
%
or 
\begin{equation}
  \label{sup-est-gen-2}
\begin{split}
  & \| \frac{1}{\langle \xi_{1} \rangle ^{2s}
  \langle \sigma_{1} \rangle
  ^{2a}} \int_{\rr} \int_{\rr} \frac{\langle \xi \rangle ^{2s}}{\langle
  \xi - \xi_{1}\rangle ^{2s}}  \times \frac{\chi_{A_{j}}}{\langle
  \sigma \rangle  \langle \sigma_{2} \rangle } d \tau d \xi
  \|_{L^{\infty}_{\xi_{1}, \tau_{1}}} < \infty
\end{split}
\end{equation}
%
or
\begin{equation}
  \label{sup-est-gen-3}
\begin{split}
  \| \int_{\rr}  \int_{\rr} \frac{\chi^{*}_{A_{j}}
    \langle \xi \rangle ^{2s}
    }{ \langle \xi_{1} \rangle^{2s} \langle
    \xi-\xi_{1} \rangle ^{2s} \langle \sigma^{*}  
    \rangle ^{2a}
    \langle \sigma_{1}^{*} \rangle
    \langle  \sigma_{2}^{*} \rangle  } d \tau d \xi_{1}  \|_{L^{\infty}_{\xi, \lambda}}
    < \infty
\end{split}
\end{equation}
%
for the following cases.
\begin{enumerate}[(i)]
    \item $ \sigma=\tau+\xi^2,\quad \sigma_1=\tau_1+\xi_1^2,\quad \sigma_2=\tau -
      \tau_1+(\xi - \xi_1)^2$,
\label{it-real-1}
    \item $ \sigma=\tau-\xi^2,\quad \sigma_1=\tau_1-\xi_1^2,\quad \sigma_2=\tau - \tau_1+(\xi - \xi_1)^2$,
\label{it-real-2}
    \item  $\sigma=\tau+\xi^2,\quad \sigma_1=\tau_1-\xi_1^2,\quad \sigma_2=\tau - \tau_1+(\xi - \xi_1)^2$,
      \label{it-real-3}
    \item $\sigma=\tau-\xi^2,\quad \sigma_1=\tau_1+\xi_1^2,\quad \sigma_2=\tau - \tau_1-(\xi - \xi_1)^2$,
\label{it-real-4}
    \item $\sigma=\tau+\xi^2,\quad \sigma_1=\tau_1+\xi_1^2,\quad \sigma_2=\tau - \tau_1-(\xi - \xi_1)^2$,
\label{it-real-5}
    \item $\sigma=\tau-\xi^2,\quad \sigma_1=\tau_1-\xi_1^2,\quad \sigma_2=\tau - \tau_1-(\xi - \xi_1)^2$.
\label{it-real-6}
\end{enumerate}
%
for \eqref{sup-est-gen-1} and \eqref{sup-est-gen-2}, and the analogous cases
%
\begin{enumerate}[(i)]
\item $ \sigma^{*}=\tau+\xi^2,\quad \sigma^{*}_1=\tau - \lambda+\xi_1^2,\quad
  \sigma^{*}_2=\lambda+(\xi - \xi_1)^2$, \label{it-1-star-real} \item $
  \sigma^{*}=\tau-\xi^2,\quad \sigma^{*}_1=\tau - \lambda-\xi_{1}^2,\quad
  \sigma^{*}_2=\lambda+(\xi - \xi_1)^2$, \label{it-2-star-real} \item
  $\sigma^{*}=\tau+\xi^2,\quad \sigma^{*}_1=\tau - \lambda-\xi_1^2,\quad
  \sigma^{*}_2=\lambda+(\xi - \xi_1)^2$, \label{it-3-star-real} \item
  $\sigma^{*}=\tau-\xi^2,\quad \sigma^{*}_1=\tau - \lambda+\xi_1^2,\quad
  \sigma^{*}_2=\lambda-(\xi - \xi_1)^2$, \label{it-4-star-real} \item
  $\sigma^{*}=\tau+\xi^2,\quad \sigma^{*}_1=\tau - \lambda+\xi_1^2,\quad
  \sigma^{*}_2=\lambda-(\xi - \xi_1)^2$, \label{it-5-star-real} \item
  $\sigma^{*}=\tau-\xi^2,\quad \sigma^{*}_1=\tau - \lambda-\xi_1^2,\quad
  \sigma^{*}_2= \lambda-(\xi - \xi_1)^2$.  \label{it-6-star-real}
  \end{enumerate}
for \eqref{sup-est-gen-3}.
%
%
\begin{framed}
\begin{remark}
Note that the cases $\sigma=\tau+\xi^2,\quad \sigma_1=\tau_1-\xi_1^2,\quad
\sigma_2=\tau - \tau_1-(\xi - \xi_1)^2$ and $\sigma=\tau-\xi^2,\quad
\sigma_1=\tau_1+\xi_1^2,\quad \sigma_2=\tau - \tau_1+(\xi - \xi_1)^2$ cannot occur, since
$\tau_1< 0, \tau-\tau_1< 0$ implies $\tau<0$ and $\tau_1\geq 0, \tau-\tau_1\geq
0$ implies $\tau\geq 0$. An analogous argument holds for $\sigma^{*},
\sigma_{1}^{*}$ and $\sigma_{2}^{*}$.
\end{remark}
\end{framed}
%
%
Observe that the transformation $(\xi, \tau, \xi_{1}, \tau_{1}) \mapsto -(\xi, \tau,
\xi_{1}, \tau_{1})$ reduces \eqref{it-real-3} to \eqref{it-real-4}, \eqref{it-real-2} to
\eqref{it-real-5}, and \eqref{it-real-1} to \eqref{it-real-6}. Furthermore, the change of
variables $\tau_{2} = \tau - \tau_{1}, \xi_{2} = \xi - \xi_{1}$, and the
transformation $(\xi, \tau, \xi_{2}, \tau_{2}) \mapsto - (\xi, \tau, \xi_{2},
\tau_{2})$ reduces \eqref{it-real-5} to \eqref{it-real-4}. Since $L^{2}$ is invariant
under change of variables and reflections, we may without loss of generality
restrict our attention to cases \eqref{it-real-4} and \eqref{it-real-6}.
 \subsubsection{Case \eqref{it-real-6}} 
\label{sssec:case-it-real-6}
We partition $\rr^{4}$ into the following three sets 
%
%
\begin{equation*}
\begin{split}
  & A_{1} = \left\{ (\xi, \tau, \xi_{1}, \tau_{1}) \subset \rr^{4}: |
  \xi_{1} \le 1 | \text{ or } | \xi - \xi_{1} | \le 1 \right\},
  \\
  & A_{2} = 
  \begin{Bmatrix}
    | \xi_{1} \ge 1 \text{ and } | \xi - \xi_{1} | \ge 1,
    \\
    \langle \tau_{1} - \xi_{1}^{2} \rangle  \le \langle \tau -
  \xi^{2} \rangle
\end{Bmatrix}
  \\
  & A_{3} = 
  \begin{Bmatrix}
    | \xi_{1} | \ge 1 \text{ and } | \xi - \xi_{1} | \ge 1,
    \\
    \langle \tau - \xi^{2} \rangle  \le \langle \tau_{1} - \xi_{1}^{2} \rangle 
  \end{Bmatrix}
\end{split}
\end{equation*}
%
[ATTENTION: Farah splits a bit differently. He in particular does not have a set
like $A_1$-it is split into smaller pieces. This is redundant. Instead,  we follow KPV--makes things cleaner and easier.]
and first seek to bound
%
%
\begin{equation}
  \label{case-1-region-1}
  \begin{split}
    \frac{ \langle \xi
    \rangle ^{2s}}{\langle \tau - \xi^{2} \rangle ^{2a}}
    \int_{\rr} \int_{\rr} \frac{\chi_{A_{1}}}{ \langle \xi_{1} \rangle ^{2s} \langle \xi-\xi_{1} \rangle ^{2s} 
    \langle \tau_{1} - \xi_{1}^{2} \rangle \langle  \tau - \tau_{1} -
    (\xi - \xi_{1})^{2} \rangle }
    d \tau_1 d \xi_{1}.
  \end{split}
\end{equation}

In region $A_{1}$, we note that if $| \xi_{1} | \le 1$
%
%
%
%
\begin{equation*}
\begin{split}
  (1 + | \xi_{1} |)(1 + | \xi - \xi_{1} |)
  & \le (1 + | \xi_{1} |)(1 + | \xi | + \xi_{1})
  \\
  & \le 2 (2 + | \xi |)
  \\
  & \le 4 (1 + | \xi |).
\end{split}
\end{equation*}
%
%
If $| \xi - \xi_{1} |\le 1$, then
%
%
\begin{equation*}
\begin{split}
  (1 + | \xi_{1} |)(1 + | \xi - \xi_{1} |)
  & \le 2 (1 + | \xi_{1} |)
  \\
  & \le 2 (1 + | \xi - \xi_{1} | + | \xi |)
  \\
  & \le 2(2 + | \xi |)
  \\
  & \le 4 (1 + | \xi |).
\end{split}
\end{equation*}
%
%
Hence, in region $A_{1}$, we have the estimate
%
%
\begin{equation}
\begin{split}
  \langle \xi_{1} \rangle \langle \xi - \xi_{1} \rangle  \le 4 \langle \xi \rangle 
\end{split}
\label{splitting-estimate}
\end{equation}
%
%
which we use to bound \eqref{case-1-region-1} by
%
%
%
%
\begin{equation*}
\begin{split}
    \frac{ 1}{\langle \tau - \xi^{2} \rangle ^{2a}}
    \int_{\rr} \int_{\rr} \frac{\chi_{A_{1}}}{  
    \langle \tau_{1} - \xi_{1}^{2} \rangle \langle  \tau - \tau_{1} -
    (\xi - \xi_{1})^{2} \rangle }
    d \tau_1 d \xi_{1}.
\end{split}
\end{equation*}
%
%
Applying \autoref{lem:calc}, this in turn is bounded by
%
%
\begin{equation}
  \label{uniform-bound-region-1}
\begin{split}
  & \frac{c_{1}}{\langle \tau - \xi^{2} \rangle^{2a}} \int_{\rr}
  \frac{\chi_{A_{1}}}{\langle \tau - \xi^{2} + 2 \xi \xi_{1} - 2
  \xi_{1}^{2} \rangle } d \xi_{1}
  \\
  & \le c_{1} \int_{\rr}
  \frac{\chi_{A_{1}}}{\langle \tau - \xi^{2} + 2 \xi \xi_{1} - 2
  \xi_{1}^{2} \rangle } d \xi_{1}, \quad a > 0
  \\
& < c, \quad b > 1/4
\end{split}
\end{equation}
%
%
where the last line follows from a corollary to \autoref{lem:calc}.
%
%
%%%%%%%%%%%%%%%%%%%%%%%%%%%%%%%%%%%%%%%%%%%%%%%%%%%%%
%
%
%                Corollary to calculus lemma
%
%
%%%%%%%%%%%%%%%%%%%%%%%%%%%%%%%%%%%%%%%%%%%%%%%%%%%%%
%
%
\begin{corollary}[Lemma 3.1 in \cite{Farah:2009uq}]
  For $a_{0}, a_{1}, a_{2} \in \rr$ with $a_{2} \neq 0$ and $q > 1/2$
  %
  %
  \begin{equation*}
  \begin{split}
    \int_{\rr} \frac{1}{\langle a_{0} + a_{1}x + a_{2}x^{2} \rangle ^{q}} dx \le c
  \end{split}
  \end{equation*}
  %
  where $c$ is a constant independent of the choice of $a_{0}, a_{1}$, and $a_{2}$.
  %
\label{cor:integral-bound}
\end{corollary}
Note that for \eqref{uniform-bound-region-1}, the
value of $c$ does not depend on the choice of $\tau$ or $\xi$. 
Next we seek to bound
\begin{equation}
  \label{case-1-region-2}
  \begin{split}
    \frac{ \langle \xi
    \rangle ^{2s}}{\langle \tau - \xi^{2} \rangle ^{2a}}
    \int_{\rr} \int_{\rr} \frac{\chi_{A_{2}}}{ \langle \xi_{1} \rangle ^{2s} \langle \xi-\xi_{1} \rangle ^{2s} 
    \langle \tau_{1} - \xi_{1}^{2} \rangle \langle  \tau - \tau_{1} -
    (\xi - \xi_{1})^{2} \rangle }
    d \tau_1 d \xi_{1}.
  \end{split}
\end{equation}
Due to the symmetry of the convolution, we assume without loss of generality that
%
%
\begin{equation*}
\begin{split}
  \langle \tau - \tau_{1} - (\xi - \xi_{1})^{2} \rangle \le \langle
  \tau_{1} - \xi_{1}^{2}\rangle 
\end{split}
\end{equation*}
%
%
Then in region $A_{2}$, we have the estimate
%
%
\begin{equation}
\begin{split}
  | \tau - \xi^{2} |
  & \ge \frac{1}{3}\left[ | \tau_{1} - \xi_{1}^{2} | + | \tau -
  \tau_{1} - (\xi - \xi_{1})^{2}
  | + | \tau - \xi^{2} | \right]
  \\
  & \ge \frac{1}{3} | - \xi_{1}^{2} - (\xi - \xi_{1})^{2} + \xi^{2} |
  \\
  & = \frac{2}{3} | \xi_{1} | | \xi - \xi_{1} |
  \\
  & \gtrsim | \xi_{1} | \quad (| \xi - \xi_{1} |\chi_{A_{2}} \ge 1).
\end{split}
\label{region-2-smoothing}
\end{equation}
%
%
Furthermore, from the inequality
%
%
\begin{equation*}
\begin{split}
  \langle \xi \rangle  \le \langle \xi_{1} \rangle \langle \xi - \xi_{1} \rangle 
\end{split}
\end{equation*}
%
we obtain
%
%
\begin{equation}
  \label{growth-term}
\begin{split}
  \frac{\langle \xi \rangle ^{2s}}{\langle \xi_{1} \rangle ^{2s} \langle \xi -
  \xi_{1} \rangle ^{2s}} \le \langle \xi_{1} \rangle ^{\gamma(s)},
  \quad 
  \gamma(s) = 
  \begin{cases} 0, \quad & s \ge 0
    \\
    4|s|, \quad & s < 0.
  \end{cases}
\end{split}
\end{equation}
Applying \eqref{region-2-smoothing} and \eqref{growth-term}, we obtain 
%
%
\begin{equation*}
\begin{split}
  \eqref{case-1-region-2}
  & \lesssim
  \int_{\rr} \int_{\rr}  \frac{\chi_{A_{2}} \langle \xi_{1} \rangle
  ^{\gamma(s) -2a}}{ 
  \langle \tau_{1} - \xi_{1}^{2} \rangle \langle  \tau - \tau_{1} -
    (\xi - \xi_{1})^{2} \rangle }
    d \tau_1 d \xi_{1}
    \\
    & \le \int_{\rr} \int_{\rr}  \frac{\chi_{A_{2}}}{ 
    \langle \tau_{1} - \xi_{1}^{2} \rangle \langle  \tau - \tau_{1} -
    (\xi - \xi_{1})^{2} \rangle }
    d \tau_1 d \xi_{1}, \quad s \ge -a/2.
  \end{split}
\end{equation*}
%
%
By \autoref{lem:calc} and \autoref{cor:integral-bound}, we bound the right hand
side by
%
%
\begin{equation*}
\begin{split}
  & c_{1} \int_{\rr}  \frac{1}{\langle \tau - \xi^{2} + 2 \xi \xi_{1} - 2
  \xi_{1}^{2} \rangle }d \xi_{1}
  \\
  & \le  c, \quad b > 1/2
\end{split}
\end{equation*}
%
%
where $c$ is independent of the value of $\tau$ and $\xi$. 
%
Lastly, we seek to bound
\begin{equation*}
\begin{split}
  &  \frac{1}{\langle \xi_{1} \rangle ^{2s}
  \langle \tau_{1} - \xi_{1}^{2}  \rangle
  ^{2a}} \int_{\rr} \int_{\rr} \frac{\langle \xi \rangle ^{2s}}{\langle
  \xi - \xi_{1}\rangle ^{2s}}  \times \frac{\chi_{A_{3}}}{\langle
  \tau - \xi^{2} \rangle ^{2a} \langle \tau - \tau_{1} - (\xi -
  \xi_{1})^{2} \rangle } d \tau d \xi.
\end{split}
\end{equation*}
Applying \autoref{lem:calc}, we bound this by
%
%
\begin{equation}
  \label{pre-A-3-bound}
\begin{split}
  &  \frac{1}{\langle \xi_{1} \rangle ^{2s}
  \langle \tau_{1} - \xi_{1}^{2}  \rangle
  ^{2a}} \int_{\rr} \frac{\langle \xi \rangle ^{2s}}{\langle
  \xi - \xi_{1}\rangle ^{2s}}  \times \frac{\chi_{A_{3}}}{\langle
  \tau_{1} + 2 \xi \xi_{1} - \xi_{1}^{2} \rangle } d \xi.
\end{split}
\end{equation}
%
Next, noting that in region $A_{3}$,
%
%
\begin{equation*}
\begin{split}
  | \tau_{1} -\xi_{1}^{2} |
  & \ge \frac{1}{3} \left[ | \tau_{1} - \xi_{1}^{2} | + | \tau - \tau_{1} -
  (\xi - \xi_{1})^{2} | + | \tau - \xi^{2} | \right]
  \\
  & \ge \frac{1}{3} | - \xi_{1}^{2} - (\xi - \xi_{1})^{2} + \xi^{2} |
  \\
  & = \frac{2}{3} | \xi_{1} | | \xi - \xi_{1} |
  \\
  & \gtrsim | \xi_{1} | \quad (| \xi - \xi_{1} |\chi_{A_{2}} \ge 1)
\end{split}
\end{equation*}
%
%
and recalling \eqref{growth-term}, we see that \eqref{pre-A-3-bound} is bounded by
\begin{equation*}
\begin{split}
  &  \int_{\rr} \frac{\chi_{A_{2,2}} \langle \xi_{1} \rangle ^{\gamma(s) -2a}}{\langle
  \tau_{1} + 2 \xi \xi_{1} - \xi_{1}^{2} \rangle } d \xi
\end{split}
\end{equation*}
%
which by the change of variable
%
%
\begin{equation*}
\begin{split}
  & \eta = \tau_{1} - \xi_{1}^{2} + 2 \xi \xi_{1},
  \\
  & d \eta = 2 \xi_{1} d \xi
\end{split}
\end{equation*}
%
%
is equal to
%
%
\begin{equation*}
\begin{split}
  & \frac{1}{2} \langle \xi_{1} \rangle ^{\gamma(s)-2a}  \int_{\rr} 
  \frac{1}{| \xi_{1} |\langle \eta \rangle ^{2a} }d \eta
  \\
  & = \langle \xi_{1} \rangle ^{\gamma(s)-2a-1} \int_{0}^{\infty} \frac{1}{(1 + \eta
  )^{2a}}d \eta
  \\
  & \lesssim 1, \qquad s \ge -\frac{2a+1}{4}.
\end{split}
\end{equation*}
%
%
\subsubsection{Case \eqref{it-real-4}} 
\label{sssec:case-it-real-4}
We partition $\rr^{4}$ into three sets 
%
%
\begin{equation*}
\begin{split}
  & A_{1} = \left\{ (\xi, \tau, \xi_{1}, \tau_{1}) \subset \rr^{4}: |
  \xi_{1} \le 1 | \right \} \\
  & A_{2} = \left\{ (\xi, \tau, \xi_{1}, \tau_{1}) \subset \rr^{4}:|
  \xi_{1} | \ge 1 \text{ and } | \xi| \le 1 \right \}
  \\
  & A_{3} = \left\{ (\xi, \tau, \xi_{1}, \tau_{1}) \subset \rr^{4}:|
  \xi_{1} | \ge 1 \text{ and } | \xi| \ge 1 \right \}
  \end{split}
\end{equation*}
%
%
and first seek to bound
%
%
\begin{equation}
  \label{case-2-region-1}
  \begin{split}
    \frac{ \langle \xi
    \rangle ^{2s}}{\langle \tau - \xi^{2} \rangle ^{2a}}
    \int_{\rr} \int_{\rr} \frac{\chi_{A_{1}}}{ \langle \xi_{1} \rangle ^{2s} \langle \xi-\xi_{1} \rangle ^{2s} 
    \langle \tau_{1} + \xi_{1}^{2} \rangle \langle  \tau - \tau_{1} -
    (\xi - \xi_{1})^{2} \rangle }
    d \tau_1 d \xi_{1}.
  \end{split}
\end{equation}
By inequality \eqref{splitting-estimate}, this is bounded by
%
%
\begin{equation*}
\begin{split}
  \frac{1}{\langle \tau - \xi^{2} \rangle ^{2a}} \int_{\rr} \int_{\rr}
  \frac{\chi_{A_{1}}}{\langle \tau_{1} + \xi_{1}^{2} \rangle  \langle \tau
  - \tau_{1} - ( \xi - \xi_{1})^{2}\rangle } d \tau_{1} d \xi_{1}.
\end{split}
\end{equation*}
%
%
Applying \autoref{lem:calc} then gives the bound
%
%
\begin{equation}
\begin{split}
  \frac{1}{\langle \tau - \xi^{2} \rangle ^{2a}} \int_{\rr}
  \frac{\chi_{A_{1}}}{\langle \tau - \xi^{2} + 2 \xi \xi_{1} \rangle } d
  \xi_{1}.
\end{split}
\label{region-1-case-2-pre-est}
\end{equation}
%
%
Since $| \xi_{1} | \le 2$ in $A_{1}$, this is is bounded by
%
%
\begin{equation*}
\begin{split}
  \int_{| \xi_{1} | \le 2} d \xi_{1} \simeq 1.
\end{split}
\end{equation*}
%
%
Next, we seek to bound
\begin{equation*}
\begin{split}
  &  \frac{1}{\langle \xi_{1} \rangle ^{2s}
  \langle \tau_{1} + \xi_{1}^{2}  \rangle
  ^{2a}} \int_{\rr} \int_{\rr} \frac{\langle \xi \rangle ^{2s}}{\langle
  \xi - \xi_{1}\rangle ^{2s}}  \times \frac{\chi_{A_{2}}}{\langle
  \tau - \xi^{2} \rangle ^{2a} \langle \tau - \tau_{1} - (\xi -
  \xi_{1})^{2} \rangle } d \tau d \xi.
\end{split}
\end{equation*}
Applying \eqref{growth-term}, this is bounded by
\begin{equation*}
\begin{split}
  &  \langle \xi_{1} \rangle ^{\gamma(s)}
   \int_{\rr} \int_{\rr} \frac{\chi_{A_{2}}}{\langle
  \tau - \xi^{2} \rangle ^{2a} \langle \tau - \tau_{1} - (\xi -
  \xi_{1})^{2} \rangle } d \tau d \xi.
\end{split}
\end{equation*}
Applying \autoref{lem:calc} then gives the bound
\begin{equation*}
\begin{split}
  & \langle \xi_{1} \rangle ^{\gamma(s)} \int_{\rr} \frac{\chi_{A_{2,2}} }{\langle
  \tau_{1} + 2 \xi \xi_{1} - \xi_{1}^{2} \rangle } d \xi
\end{split}
\end{equation*}
which by the change of variable
%
%
\begin{equation*}
\begin{split}
  & \eta = \tau_{1} - \xi_{1}^{2} + 2 \xi \xi_{1},
  \\
  & d \eta = 2 \xi_{1} d \xi
\end{split}
\end{equation*}
%
%
is equal to
%
%
\begin{equation*}
\begin{split}
  & \frac{\langle \xi_{1} \rangle^{\gamma(s)}}{2}  \int_{\rr} 
  \frac{1}{| \xi_{1} |\langle \eta \rangle ^{2a} }d \eta
  \\
  & = \langle \xi_{1} \rangle ^{\gamma(s) -1} \int_{0}^{\infty} \frac{1}{(1 + \eta
  )^{2a}}d \eta
  \\
  & \lesssim 1, \qquad s \ge -1/4.
\end{split}
\end{equation*}

%If $| \xi | \ge 1$, then using the change of variable
%%
%%
%\begin{equation*}
%\begin{split}
  %& \eta = \tau - \xi^{2} + 2 \xi \xi_{1}
  %\\
  %& d \eta = 2 \xi d \xi_{1}
%\end{split}
%\end{equation*}
%%
%%
%to \eqref{region-1-case-2-pre-est} gives
%\begin{equation*}
  %\begin{split}
 %\frac{1}{2 \langle \tau - \xi^{2} \rangle^{2a} } \int_{\rr} \frac{1}{| \xi
%|\langle \eta \rangle } d \eta
%& \le  \int_{\rr} \frac{1}{\langle \eta \rangle
%} d \eta
%\\
%& \simeq 1, \qquad b > 1/2.
%\end{split}
%\end{equation*}
%
%
We now split $A_{3}$ into the sets
%
%
%
\begin{align*}
A_{3,1}&=\{(\xi, \xi_1, \tau, \tau_1)\in A_2:
|\tau-\tau_1-(\xi-\xi_1)^2|, |\tau_1+\xi_1^2| \le |\tau-\xi^2|\},\\
A_{3,2}&=\{(\xi, \xi_1, \tau, \tau_1)\in A_2:
|\tau-\tau_1-(\xi-\xi_1)^2|, |\tau-\xi^2| \le |\tau_1+\xi_1^2|\},\\
A_{3,3}&=\{(\xi, \xi_1, \tau, \tau_1)\in A_2: |\tau_{1}+\xi_{1}^2|, | \tau - \xi^{2} | \le |  \tau - \tau_{1} -
(\xi - \xi_{1})^{2} |\}
\end{align*} 
and first seek to estimate
%
%
\begin{equation}
  \label{case-2-region-2}
  \begin{split}
    \frac{ \langle \xi
    \rangle ^{2s}}{\langle \tau - \xi^{2} \rangle ^{2a}}
    \int_{\rr} \int_{\rr} \frac{\chi_{A_{3,1}}}{ \langle \xi_{1} \rangle ^{2s} \langle \xi-\xi_{1} \rangle ^{2s} 
    \langle \tau_{1} + \xi_{1}^{2} \rangle \langle  \tau - \tau_{1} -
    (\xi - \xi_{1})^{2} \rangle }
    d \tau_1 d \xi_{1}.
  \end{split}
\end{equation}
Notice that, unlike case \eqref{it-6}, we may \emph{not} assume without loss of generality
that $|\tau - \tau_{1} - (\xi - \xi_1)^{2} | \le | \tau_{1} + \xi^{2} | $.
Applying \autoref{lem:calc} to \eqref{case-2-region-2} gives the bound
%
%
\begin{equation*}
  \begin{split}
    \frac{ \langle \xi
    \rangle ^{2s}}{\langle \tau - \xi^{2} \rangle ^{2a}}
    \int_{\rr} \frac{\chi_{A_{3,1}}}{ \langle \xi_{1} \rangle ^{2s} \langle \xi-\xi_{1} \rangle ^{2s} 
    \langle \tau - \xi^{2} +2 \xi \xi_{1} \rangle } d \xi_{1}.
  \end{split}
\end{equation*}
%
%
%
But in $A_{3,1}$
%
%
\begin{equation}
\begin{split}
  | \tau - \xi^{2} |
  & \ge \frac{1}{3} \left[ | \tau_{1} + \xi_{1}^{2} | + | \tau - \tau_{1} -
  (\xi - \xi_{1})^{2} | + | \tau - \xi^{2} | \right]
  \\
  & \ge \frac{1}{3} | \xi_{1}^{2} - (\xi - \xi_{1})^{2} + \xi^{2} |
  \\
  & = \frac{2}{3}| \xi | | \xi_{1} |
  \\
  & \gtrsim  | \xi_{1} |.
\end{split}
\label{case-2-region-A2-key-est}
\end{equation}
%
%
Hence, applying \eqref{growth-term} and \eqref{case-2-region-A2-key-est}, we obtain
%
%
%
%
\begin{equation*}
\begin{split}
  \eqref{case-2-region-2}
  & \lesssim 
  \int_{\rr} \int_{\rr} \frac{\chi_{A_{3,1}} \langle \xi_{1}
  \rangle^{\gamma(s) -2a}}{  
    \langle \tau_{1} + \xi_{1}^{2} \rangle \langle  \tau - \tau_{1} -
    (\xi - \xi_{1})^{2} \rangle }
    d \tau_1 d \xi_{1}
    \\
    & \le \int_{\rr} \int_{\rr} \frac{\chi_{A_{3,1}}}{  
    \langle \tau_{1} + \xi_{1}^{2} \rangle \langle  \tau - \tau_{1} -
    (\xi - \xi_{1})^{2} \rangle }
    d \tau_1 d \xi_{1}, \quad s \ge -a/2.
\end{split}
\end{equation*}
By \autoref{lem:calc} and \autoref{cor:integral-bound}, we bound the right hand
side by
%
%
\begin{equation*}
\begin{split}
  & c_{1} \int_{\rr}  \frac{1}{\langle \tau - \xi^{2} + 2 \xi \xi_{1} 
  \rangle }d \xi_{1}
  \\
  & = c_{1} \int_{\rr}  \frac{1}{\left [ \langle \tau - \xi^{2} + 2 \xi \xi_{1} 
  \rangle ^{2} \right ]^{b}}d \xi_{1}
  \\
  & \le  c, \quad b > 1/2
\end{split}
\end{equation*}
where $c$ is independent of the value of $\tau$ and $\xi$. 
Next, we seek to estimate
\begin{equation*}
\begin{split}
  &  \frac{1}{\langle \xi_{1} \rangle ^{2s}
  \langle \tau_{1} + \xi_{1}^{2}  \rangle
  ^{2a}} \int_{\rr} \int_{\rr} \frac{\langle \xi \rangle ^{2s}}{\langle
  \xi - \xi_{1}\rangle ^{2s}}  \times \frac{\chi_{A_{3,2}}}{\langle
  \tau - \xi^{2} \rangle ^{2a} \langle \tau - \tau_{1} - (\xi -
  \xi_{1})^{2} \rangle } d \tau d \xi.
\end{split}
\end{equation*}
Applying \autoref{lem:calc}, we bound this by
%
%
\begin{equation}
  \label{pre-A-2-2}
\begin{split}
  &  \frac{1}{\langle \xi_{1} \rangle ^{2s}
  \langle \tau_{1} + \xi_{1}^{2}  \rangle
  ^{2a}} \int_{\rr} \frac{\langle \xi \rangle ^{2s}}{\langle
  \xi - \xi_{1}\rangle ^{2s}}  \times \frac{\chi_{A_{3,2}}}{\langle
  \tau_{1} + 2 \xi \xi_{1} - \xi_{1}^{2} \rangle } d \xi.
\end{split}
\end{equation}
%
Next, noting that in region $A_{3,2}$,
%
%
\begin{equation}
\begin{split}
  | \tau_{1} + \xi_{1}^{2} |
  & \ge \frac{1}{3} \left[ | \tau_{1} + \xi_{1}^{2} | + | \tau - \tau_{1} -
  (\xi - \xi_{1})^{2} | + | \tau - \xi^{2} | \right]
  \\
  & \ge \frac{1}{3} | \xi_{1}^{2} - (\xi - \xi_{1})^{2} + \xi^{2} |
  \\
  & = \frac{2}{3}| \xi | | \xi_{1} |
  \\
  & \gtrsim  | \xi_{1} |
\end{split}
\label{case-2-region-A-2-2-key-est}
\end{equation}
%
%
and recalling \eqref{growth-term}, we see that \eqref{pre-A-2-2} is bounded by
\begin{equation*}
\begin{split}
  &  \int_{\rr} \frac{\chi_{A_{3,2}} \langle \xi_{1} \rangle ^{\gamma(s) -2a}}{\langle
  \tau_{1} + 2 \xi \xi_{1} - \xi_{1}^{2} \rangle } d \xi
\end{split}
\end{equation*}

which by the change of variable
%
%
\begin{equation*}
\begin{split}
  & \eta = \tau_{1} - \xi_{1}^{2} + 2 \xi \xi_{1},
  \\
  & d \eta = 2 \xi_{1} d \xi
\end{split}
\end{equation*}
%
%
is equal to
%
%
\begin{equation*}
\begin{split}
  & \frac{\xi_{1}^{\gamma(s) -2a}}{2}  \int_{\rr} 
  \frac{1}{| \xi_{1} |\langle \eta \rangle ^{2a} }d \eta
  \\
  & = \langle \xi_{1} \rangle ^{\gamma(s) -2a -1} \int_{0}^{\infty} \frac{1}{(1 + \eta
  )^{2a}}d \eta
  \\
  & \lesssim 1, \qquad s \ge -\frac{2a+1}{4}.
\end{split}
\end{equation*}
It remains to handle region $A_{3,3}$. It will be enough to bound
%
%
\begin{equation}
  \label{region-A-2-3-star-split}
\begin{split}
   \sum_{\xi_{1} \in \zz} \int_{\rr} \frac{\chi^{*}_{A_{3,3}}
    \langle \xi \rangle ^{2s}
    }{ \langle \xi_{1} \rangle^{2s} \langle  \tau  - \xi^{2}
    \rangle ^{2a}  \langle
\xi-\xi_{1} \rangle ^{2s}  \langle  \tau - \lambda+\xi_{1}^{2}
\rangle \langle   \lambda  -(\xi - \xi_{1})^{2}
\rangle } d \tau.
\end{split}
\end{equation}
%
Due to the presence of $\chi^{*}_{A_{3,3}}$ factor, we have the restriction
%
%
\begin{equation*}
\begin{split}
& |\tau - \lambda +\xi_{1}^2|, | \tau - \xi^{2} | \le |  \lambda -
(\xi - \xi_{1})^{2} | \text{ and }  |\xi| \ge 1, |\xi_1| \ge 1.
\end{split}
\end{equation*}
%
It follows that
\begin{equation}
  \label{smoothing-2-3-case-6}
\begin{split}
  | \lambda - (\xi - \xi_{1})^{2} |
  & \ge \frac{1}{3}\left[ | \tau - \lambda + \xi_{1}^{2} | + | \lambda - (\xi - \xi_{1})^{2}
  | + | \tau - \xi^{2} | \right]
  \\
  & \ge \frac{1}{3} |  \xi_{1}^{2} - (\xi - \xi_{1})^{2} + \xi^{2} |
  \\
  & = \frac{2}{3} | \xi_{1} | | \xi |
  \\
  & \gtrsim | \xi_{1} |.
\end{split}
\end{equation}
Hence, applying
\eqref{growth-term}, \autoref{lem:calc}, and
\eqref{smoothing-2-3-case-6}, we bound \eqref{region-A-2-3-star-split} by
%
%
\begin{equation*}
\begin{split}
   & \sum_{\xi_{1} \in \zz} \int_{\rr} \frac{\chi^{*}_{B_{3,3}} \langle
   \xi_{1} \rangle ^{\gamma(s) -2a}
    }{ \langle  \tau  - \xi^{2}
    \rangle ^{2a}   \langle  \tau - \lambda+\xi_{1}^{2}
\rangle } d \tau
\\
& \lesssim  \sum_{\xi_{1} \in \zz} \frac{\chi_{B_{3,3}^{*}}}{\langle \xi_{1}^{2} +
\xi^{2} - \lambda \rangle^{2 a} }, \quad s \ge -a/2
\end{split}
\end{equation*}
which is bounded for $a > 1/4$ by
\autoref{lem:sum-estimate}. This concludes the proof of \autoref{prop:bilin-est-real}. \qquad \qedsymbol
%
%
%%
%%
%\begin{equation}
%\begin{split}
  %|  \eqref{sup-est-gen-real} \chi_{B_{2}}|
  %& =  \langle \tau - \xi^{2} \rangle
  %^{-2a} \langle \xi \rangle ^{2s} \int _{\rr} \int_{\rr} \frac{\chi_{B_{1}}}{
  %\langle \xi_{1} \rangle ^{2s} \langle \xi-\xi_{1}\rangle ^{2s} \langle \tau_{1} - \xi_{1}^{2} \rangle \langle
  %\tau - \tau_{1} - (\xi -\xi_{1})^{2}\rangle }d \tau_{1} d \xi_{1} 
  %\\
  %& \le  \int_{\rr} \int_{\rr} \frac{\chi_{B_{1}}
  %\langle
  %\xi_{1}\rangle ^{\gamma(s) -2b} 
%}{ \langle \tau_{1} - \xi_{1}^{2} \rangle \langle
  %\tau - \tau_{1} - (\xi - \xi_{1})^{2}\rangle }
  %d \tau_{1} d \xi_{1} 
  %\\
  %& \le \int_{\rr} \int_{\rr} \frac{\chi_{B_{1}}
  %}{ \langle \tau_{1} - \xi_{1}^{2} \rangle \langle
  %\tau - \tau_{1} - (\xi - \xi_{1})^{2}\rangle }
  %d \tau_{1} d \xi_{1}, \quad s \ge 0 \text{ or } s \ge -a/2
  %\\
  %& \le \int _{\rr} 
  %\frac{\chi_{B_{1}}}{\langle \tau - \xi^{2} + 2 \xi
  %\xi_{1} - 2 \xi_{1}^{2} \rangle }d \xi_{1}, \quad (\autoref{lem:calc}) 
  %\\
  %& < \infty
%\end{split}
%\end{equation}
\newpage
\appendix
\section{An Alternative Way to Derive the Integral form of the $B_{4}$ ivp.}
\label{ssec:integral-form-deriv}
We will describe an alternative to the method of variation of parameters used in
the paper. 
%
%
\subsection{Reducing to a First Order ODE} 
\label{sssec:first-order-ode}
Taking the spatial Fourier transform of \eqref{lin-mb} yields
the ivp
%
%
\begin{gather*}
  \wh{u_{tt}} + n^{4} \wh{u} = \wh{-u^{2}_{xx}},
  \\
  \wh{u}(n, 0) = \wh{u_{0}}(n), \quad \wh{u_{t}}(n, 0) = \wh{u_{1}}(n)
\end{gather*}
%
%
which we rewrite as 
%
%
\begin{gather}
  \label{eqn:lin-mb-ode}
  y_{tt} + n^{4}y = -f,
  \\
  y(n, 0) = y_{0}(n), \quad y_{t}(n, 0) = y_{1}(n)
\label{eqn:lin-mb-ode-init-data}
\end{gather}
%
%
where
%
%
\begin{gather*}
  \label{not-1}
  y = y(n, t) \doteq \wh{u}(n, t), \quad f = f(n, t) \doteq
  \wh{u^{2}_{xx}}(n,t),
  \\
  \label{not-2}
  y_{0}(n) \doteq \wh{u_{0}}(n), \quad y_{1}(n) = \wh{u_{1}}(n).
\end{gather*}
%
%
Viewing $n$ as fixed, we see that \eqref{eqn:lin-mb-ode} is a second order
linear nonhomogeneous ODE. To solve it, we set 
%
%
\begin{equation*}
  \label{not-3}
\begin{split}
   & v_{1} = y, 
   \\
   & v_{2} = y_{t}
\end{split}
\end{equation*}
%
%
giving
%
%
\begin{equation*}
\begin{split}
  & v_{1}' = v_{2},
  \\
  & v_{2}' = -n^{4}v_{1} - f.
\end{split}
\end{equation*}
%
%
Therefore 
%
%
\begin{equation}
\begin{split}
\frac{d \vec v}{dt} = 
\begin{bmatrix}
0 & 1 \\
-n^{4} & 0
\end{bmatrix}
\begin{bmatrix}
  v_{1}\\
  v_{2}
\end{bmatrix}
-
\begin{bmatrix}
0\\
f
\end{bmatrix}
\doteq A \vec v - \vec f.
\end{split}
\label{eqn:first-order-ode-reduction}
\end{equation}
%
%
Hence, we have reduced solving the $2$nd order ODE \eqref{eqn:lin-mb-ode} to
solving the first order ODE \eqref{eqn:first-order-ode-reduction}. Multiplying
by the integrating factor $e^{-At}$ on both sides of
\eqref{eqn:first-order-ode-reduction}, we obtain
%
%
\begin{equation*}
\begin{split}
  \frac{d}{dt}(e^{-At} \vec v) = -e^{-At} \vec f.
\end{split}
\end{equation*}
%
%
Integrating in time then gives
%
%
\begin{equation*}
\begin{split}
  e^{-At} \vec v(n, t) = \vec v(n, 0) - \int_{0}^{t}e^{-At'} \vec f dt'
\end{split}
\end{equation*}
%
%
or
%
%
\begin{equation}
  \label{ode-vec-soln}
\begin{split}
  \vec v(n, t) = e^{At} \vec v(n, 0) - \int_{0}^{t}e^{A(t - t')} \vec f dt'.
\end{split}
\end{equation}
%
%
We wish to compute $e^{At}$. It is easy to check that $A$ has eigenvalues
$\lambda = \pm in^{2}$, with corresponding eigenvectors 
%
%
\begin{equation*}
\begin{split}
\pm \begin{bmatrix}
1 \\
in^{2}
\end{bmatrix}.
\end{split}
\end{equation*}
%
%
Since $A$ is a square matrix and has no repeated
eigenvalues, it is diagonizable. More precisely,
%
%
%
%
\begin{equation*}
\begin{split}
  A = Q D Q^{-1} = 
  \begin{bmatrix}
  1 & 1
  \\
  in^{2} & -in^{2}
  \end{bmatrix}
  \begin{bmatrix}
    in^{2} & 0 
    \\
    0 & -in^{2}
  \end{bmatrix}
  \begin{bmatrix}
    \frac{1}{2} & \frac{1}{2i n^{2}} \\
    \frac{1}{2} & -\frac{1}{2i n^{2} }
  \end{bmatrix}
\end{split}
\end{equation*}
%
%
where the column vectors of $Q$ are comprised of the eigenvectors of $A$.
Recalling that 
%
%
\begin{equation*}
\begin{split}
  e^{At} \doteq \sum_{n=0}^{\infty} \frac{A^{n}}{n}t^{n}
\end{split}
\end{equation*}
%
%
it is easy to check that 
%
%
\begin{equation*}
\begin{split}
  e^{Q D Q^{-1}} = 
\begin{bmatrix}
  1 & 1
  \\
  in^{2} & -in^{2}
  \end{bmatrix}
  \begin{bmatrix}
    e^{in^{2}} & 0 
    \\
    0 & e^{-in^{2}}
  \end{bmatrix}
  \begin{bmatrix}
    \frac{1}{2} & \frac{1}{2i n^{2}} \\
    \frac{1}{2} & -\frac{1}{2i n^{2} }
  \end{bmatrix}
\end{split}
\end{equation*}
%
%
and 
\begin{equation}
  \label{matrix-expo}
\begin{split}
  e^{Q D Q^{-1}t}
  & = 
\begin{bmatrix}
  1 & 1
  \\
  in^{2} & -in^{2}
  \end{bmatrix}
  \begin{bmatrix}
    e^{in^{2}t} & 0 
    \\
    0 & e^{-in^{2}t}
  \end{bmatrix}
\begin{bmatrix}
    \frac{1}{2} & \frac{1}{2i n^{2}} \\
    \frac{1}{2} & -\frac{1}{2i n^{2} }
  \end{bmatrix}
  \\
  & =
  \begin{bmatrix}
    \frac{1}{2}(e^{in^{2}t} + e^{-in^{2}t}) & \frac{1}{2 i n^{2}} (e^{in^{2}t} -
    e^{-in^{2}t})    \\
    \frac{in^{2}}{2}(e^{in^{2}t} - e^{-in^{2}t}) & \frac{1}{2}(e^{in^{2}t} +
    e^{-in^{2}t})
  \end{bmatrix}.
\end{split}
\end{equation}
%
%
\begin{framed}
\begin{remark}
If a matrix has repeated eigenvalues, it may no longer be diagonizable. However,
any matrix can be written in Jordan canonical form. The above computations
become slightly more complicated in this case. The important observation is that
writing a matrix in Jordan canonical (or, ideally, diagonal) form allows us to
easily compute its exponential. 
\label{rem:jordan-form}
\end{remark}
\end{framed}
%
%
%
\newpage
%
\begin{framed}
\begin{remark}
\label{rem:simpler-comp}
  Since $A$ is a particularly simple matrix, i.e. $A^{2} = -n^{4} I$, one can
  compute its exponential easily without the use of diagonalization (this is
  not true in general). From the above equality, we obtain $A^{n} =
  (-1)^{n/2} n^{2n} I$ for even $n$, and so
  %
  %
  \begin{equation*}
  \begin{split}
    e^{At}
    & = \sum_{n=0}^{\infty} \frac{(-1)^{n}n^{4n}t^{2n}}{(2n)!}I + A
    \sum_{n=0}^{\infty} \frac{(-1)^{n} n^{4n} t^{2n + 1}}{(2n + 1)!} I 
    \\
    & = \cos n^{2}t \, I - \frac{\sin n^{2}t}{n^{2}}A
    \\
    & = 
    \begin{bmatrix}
      \cos n^{2}t &  -\frac{\sin n^{2}t}{n^{2}}
      \\
      - n^{2} \sin n^{2}t & \cos n^{2}t
    \end{bmatrix}
    \\
    & = \text{rhs of }\eqref{matrix-expo}
  \end{split}
  \end{equation*}
\end{remark}
\end{framed}
%
%
Substituting \eqref{matrix-expo} into \eqref{ode-vec-soln} and recalling our notation, we obtain
%
%
\begin{equation*}
\begin{split}
  v_{1}(n, t) = (e^{in^{2}t} + e^{-in^{2}t})v_{1}(n, 0) + \frac{e^{in^{2}} -
  e^{-in^{2}t}}{2 i n^{2}} v_{1}(n, 0) - \int_{0}^{t} \frac{e^{in^{2}(t - t')} -
  e^{-in^{2}(t-t')}}{2 i n^{2}} f dt'
\end{split}
\end{equation*}
%
%
or
\begin{equation*}
\begin{split}
  y(n, t) = (e^{in^{2}t} + e^{-in^{2}t})y_{0} + \frac{e^{in^{2}} -
  e^{-in^{2}t}}{2 i n^{2}} y_{1} - \int_{0}^{t} \frac{e^{in^{2}(t - t')} -
  e^{-in^{2}(t-t')}}{2 i n^{2}} f dt'
\end{split}
\end{equation*}
or
\begin{equation*}
\begin{split}
  \wh{u}(n, t) = (e^{in^{2}t} + e^{-in^{2}t})\wh{u_0} + \frac{e^{in^{2}} -
  e^{-in^{2}t}}{2 i n^{2}} \wh{u_1} - \int_{0}^{t} \frac{e^{in^{2}(t - t')} -
  e^{-in^{2}(t-t')}}{2 i n^{2}} \wh{(u^{2})_{xx}} dt'.
\end{split}
\end{equation*}
%
Taking the inverse Fourier transform then yields \eqref{eqn:integral-form}, as
desired.

\section{Farah's Approach to Reducing the Proof of Well-Posedness to the
Bilinear Estimates} 
\label{ssec:farah-approach-reduction}
The unique solution to ivp
\eqref{four-trans-mb}-\eqref{four-trans-mb-data} is given by
%
%
\begin{equation*}
\begin{split}
\wh{u}(n, t) = \wh{u_{0}}(n) \frac{e^{in^{2}t} + e^{-in^{2}t}}{2} +
  \wh{u_{1}}(n)\frac{e^{in^{2}t} - e^{-in^{2}}t}{2 i n^{2}} +
  \int_{0}^{t}\frac{e^{in^{2}(t-t')}-e^{-in^{2}(t-t')}}{2in^{2}}
  \wh{(u^{2})_{xx}} dt'.
\end{split}
\end{equation*}
%
%
Taking the inverse Fourier transform then gives
%
\begin{equation}
  \begin{split}
    u(x,t) = R_{t}u_{0} + S_{t}u_{1} + \int_{0}^{t} S_{t-t'}
    (u^{2})_{xx} dt'.
  \end{split}
  \label{eqn:integral-formb}
\end{equation}
%
%
Hence, we have rewritten the $B_{4}$ ivp
\eqref{eqn:mb-2}-\eqref{eqn:mb-init-data-2} in integral form, which we will now
localize in time. 
Let $\psi(t)$ be a cutoff function symmetric about the 
origin such that $\psi(t) = 1$ for $|t| \le 1/2$ and $\text{supp} \, \psi 
= [-1, 1 ]$. Define $\psi_{\delta}(t) = \psi(t/T)$, $ 0 < T \le 1$.
Motivated by $\eqref{eqn:integral-formb}$, we now consider the equation
%
%
\begin{equation}
  \begin{split}
    u(x,t)
    & = \psi(t) R_{t} u_{0} + \psi(t) S_{t}u_{1} +
    \psi_{\delta}(t) \int_{0}^{t} S_{t-t'}
    (u^{2})_{xx} dt'
    \\
    & \doteq Tu
  \end{split}
  \label{localized-int-eqn}
\end{equation}
%
%
%
%
%
%
%
\subsection{Estimate for $\psi(t) R_{t}u_{0}$.} 
We have
%
%
\begin{equation*}
  \begin{split}
    \wh{\psi(t)R_{t}u_{0}}^{x}(n, t)
    & = \psi(t) \wh{u_{0}}(n) \frac{e^{in^2 t} + e^{-in^{2}t}}{2}
    \\
    & = \frac{\psi(t) \wh{u_{0}}(n)e^{in^{2}t}}{2} + \frac{\psi(t)
    \wh{u_{0}}(n)e^{-in^{2}t}}{2}  
  \end{split}
\end{equation*}
%
%
and
%
%
\begin{equation*}
  \begin{split}
    \wh{\psi(t) R_{t}u_{0}}(n, \tau) = \frac{\wh{\psi}(\tau -
    n^{2})\wh{u_{0}}(n)}{2} + \frac{\wh{\psi}(\tau + n^{2})\wh{u_{0}}(n)}{2}.
  \end{split}
\end{equation*}
%
%
Hence, substituting and applying \eqref{square-ineq} we have

%
%
\begin{align}
    & \| \psi(t) R_{t}u_{0} \|_{X_{s,b}}^{2} 
    \notag
    \\
    & = \sum_{n \in \zz}(1 + |n|)^{2s} \int_{\rr}\left( 1 + | | \tau
    |-n^{2} | \right)^{2b} | \frac{\wh{\psi}(\tau - n^{2})\wh{u_{0}(n)}}{2} +
    \frac{\wh{\psi}(\tau + n^{2})\wh{u_{0}}(n)}{2} |^{2} d \tau
    \notag
    \\
    & \le \sum_{n \in \zz} \left( 1 + |n| \right)^{2s} | \wh{u_{0}}(n)
    |^{2} \int_{\rr} | \wh{\psi}(\tau - n^{2}) |^{2}\left( 1 + | | \tau | -
    n^{2} | \right)^{2b} d \tau
    \label{u-0-norm-comp-1g}
    \\
    & + \sum_{n \in \zz} \left( 1 + |n| \right)^{2s} | \wh{u_{0}}(n)
    |^{2} \int_{\rr} | \wh{\psi}(\tau + n^{2}) |^{2}\left( 1 + | | \tau | -
    n^{2} | \right)^{2b} d \tau.
    \label{u-0-norm-comp-3g}
\end{align}
%
Applying \eqref{eqn:norm-key-ineq}, we bound \eqref{u-0-norm-comp-1g} by
%
%
\begin{equation*}
  \begin{split}
    & \sum_{n \in \zz} \left( 1 + |n| \right)^{2s} | \wh{u_{0}}(n)
    |^{2} \int_{\rr} | \wh{\psi}(\tau - n^{2}) |^{2}\left( 1 +  | \tau  -
    n^{2} | \right)^{2b} d \tau
    \\
    & = \sum_{n \in \zz} \left( 1 + |n| \right)^{2s} | \wh{u_{0}}(n)
    |^{2} \int_{\rr} | \wh{\psi}(\tau') |^{2}\left( 1 +  | \tau'| \right)^{2b} d \tau
    \\
    & \le c_{\psi} \sum_{n \in \zz} \left( 1 + |n| \right)^{2s} | \wh{u_{0}}(n)
    |^{2}, \quad b \le 1 
    \\
    & = c_{\psi} \| u_{0} \|_{H^{s}}^{2}
  \end{split}
\end{equation*}
%
%
where $c_{\psi}$ is a constant depending only on $\psi$. The
term \eqref{u-0-norm-comp-3} is bounded in similar fashion. Therefore, 
$\|\psi(t) R_{t} u_{0}\|_{X_{s,b}}^{2} \le c_{\psi}
\|u_{0}\|_{H^s}^2$ for $b \le 1$. Taking square roots of both sides gives
%
%
\begin{equation}
  \begin{split}
    \|\psi(t) R_{t} u_{0}\|_{X_{s,b}} \le c_{\psi}
    \|u_{0}\|_{H^s}, \quad b \le 1.
  \end{split}
  \label{eqn:u-0-fin-estg}
\end{equation}
%
%

\subsection{Estimate for $\psi(t) S_{t}u_{1}$.}
We have
%
%
\begin{equation*}
  \begin{split}
    \wh{\psi(t)S_{t}u_{1}}^{x}(n, t)
    & = \psi(t) \wh{u_{1}}(n) \frac{e^{in^2 t} - e^{-in^{2}t}}{2i n^{2}}
    \\
    & = \frac{\psi(t) \wh{u_{1}}(n)e^{in^{2}t}}{2i n^{2}} - \frac{\psi(t)
    \wh{u_{1}}(n)e^{-in^{2}t}}{2i n^{2}}  
  \end{split}
\end{equation*}
%
%
and
%
%
\begin{equation*}
  \begin{split}
    \wh{\psi(t) S_{t}u_{1}}(n, \tau) = \frac{\wh{\psi}(\tau -
    n^{2})\wh{u_{1}}(n)}{2i n^{2}} + \frac{\wh{\psi}(\tau + n^{2})\wh{u_{1}}(n)}{2i
    n^{2}}.
  \end{split}
\end{equation*}
%
Note that 
%
\begin{equation*}
  \begin{split}
    \wh{\psi(t)S_{t}u_{1}}^{x}(0, t)
    & = \psi(t) \wh{u_{1}}(0) t
      \end{split}
\end{equation*}
and so 
%
%
\begin{equation*}
  \begin{split}
    \wh{\psi(t) S_{t}u_{1}}(0, \tau) = i \frac{d}{d \tau} \wh{\psi}(\tau)
    \wh{u_{1}}(0).
  \end{split}
\end{equation*}
%
Hence, substituting and applying \eqref{square-ineq}, we have
%
%
\begin{equation}
  \begin{split}
    \| \psi(t) S_{t}u_{1} \|_{X_{s,b}}^{2} 
    & = \sum_{n \in \zzdot}(1 + |n|)^{2s} \int_{\rr}\left( 1 + | | \tau
    |-n^{2} | \right)^{2b} | \frac{\wh{\psi}(\tau - n^{2})\wh{u_{1}(n)}}{2i
    n^{2}} -
    \frac{\wh{\psi}(\tau + n^{2})\wh{u_{1}}(n)}{2i n^{2}} |^{2} d \tau
    \\
    & + |\wh{u_{1}}(0)|^{2} \int_{\rr} (1 + | \tau |)^{2b} | i \frac{d }{d \tau}
    \wh{\psi}(\tau)|^{2} d \tau
    \\
    & \le \sum_{n \in \dot{\zz}} n^{-4} \left( 1 + |n| \right)^{2s} | \wh{u_{1}}(n)
    |^{2} \int_{\rr} | \wh{\psi}(\tau - n^{2}) |^{2}\left( 1 + | | \tau | -
    n^{2} | \right)^{2b} d \tau
    \\
    & + \sum_{n \in \dot{\zz}} n^{-4} \left( 1 + |n| \right)^{2s} | \wh{u_{1}}(n)
    |^{2} \int_{\rr} | \wh{\psi}(\tau + n^{2}) |^{2}\left( 1 + | | \tau | -
    n^{2} | \right)^{2b} d \tau
    \\
    & + |\wh{u_{1}}(0)|^{2} \int_{\rr} (1 + | \tau |)^{2b} |\frac{d }{d \tau}
    \wh{\psi}(\tau)|^2 d \tau.
\end{split}
\label{u-1-norm-comp-preg}
\end{equation}
%
%
Applying the inequality
%
%
\begin{equation*}
\begin{split}
  \frac{(1 + |n|)^{2s}}{ n^{4}} \le \frac{(1 + |n|)^{2s}}{\frac{1}{16}(1 +
  |n|)^{4}} = 16 (1 + | n |)^{2(s-2)},  \quad s \in \rr, \quad n \ge 1
\end{split}
\end{equation*}
%
to \eqref{u-1-norm-comp-preg} gives
%
\begin{equation}
  \begin{split}
    \| \psi(t) S_{t}u_{1} \|_{X_{s,b}}^{2} 
    & \lesssim \sum_{n \in \dot{\zz}} \left( 1 + |n| \right)^{2(s-2)} | \wh{u_{1}}(n)
    |^{2} \int_{\rr} | \wh{\psi}(\tau - n^{2}) |^{2}\left( 1 + | | \tau | -
    n^{2} | \right)^{2b} d \tau
    \\
    & + \sum_{n \in \dot{\zz}} \left( 1 + |n| \right)^{2(s-2)} | \wh{u_{1}}(n)
    |^{2} \int_{\rr} | \wh{\psi}(\tau + n^{2}) |^{2}\left( 1 + | | \tau | -
    n^{2} | \right)^{2b} d \tau
    \\
    & + |\wh{u_{1}}(0)|^{2} \int_{\rr} (1 + | \tau |)^{2b} |\frac{d }{d \tau}
    \wh{\psi}(\tau)|^2 d \tau.
\end{split}
\label{u-1-norm-compe}
\end{equation}
%
%
Applying \eqref{eqn:norm-key-ineq},
we bound the first term of
\eqref{u-1-norm-compe} by
%
%
%
\begin{equation*}
  \begin{split}
    & \sum_{n \in \dot{\zz}} \left( 1 + |n| \right)^{2(s-2)} | \wh{u_{1}}(n)
    |^{2} \int_{\rr} | \wh{\psi}(\tau - n^{2}) |^{2}\left( 1 +  | \tau  -
    n^{2} | \right)^{2b} d \tau
    \\
    & = \sum_{n \in \dot{\zz}} \left( 1 + |n| \right)^{2(s-2)} | \wh{u_{1}}(n)
    |^{2} \int_{\rr} | \wh{\psi}(\tau') |^{2}\left( 1 +  | \tau'| \right)^{2b} d \tau
    \\
    & \le c_{\psi} \| u_{1} \|_{H^{s-2}}^{2}, \quad b \le 1
  \end{split}
\end{equation*}
%
%
where $c_{\psi, b}$ is a constant depending only on $\psi$. Applying
\eqref{eqn:norm-key-ineq} again, the
second term of \eqref{u-1-norm-compe} is bounded by
\begin{equation*}
  \begin{split}
    & \sum_{n \in \dot{\zz}} \left( 1 + |n| \right)^{2(s-2)} | \wh{u_{1}}(n)
    |^{2} \int_{\rr} | \wh{\psi}(\tau + n^{2}) |^{2}\left( 1 +  | \tau  -
    n^{2} | \right)^{2b} d \tau
    \\
    & = \sum_{n \in \dot{\zz}} \left( 1 + |n| \right)^{2(s-2)} | \wh{u_{1}}(n)
    |^{2} \int_{\rr} | \wh{\psi}(\tau') |^{2}\left( 1 +  | \tau'| \right)^{2b} d \tau
    \\
    & \le c_{\psi} \| u_{1} \|_{H^{s-2}}^{2}, \quad b \le 1
  \end{split}
\end{equation*}
while the third term is bounded by  
%
%
\begin{equation*}
\begin{split}
  c_{\psi} \| u_{1} \|_{H^{s-2}}^{2}, \quad b \le 1.
\end{split}
\end{equation*}
%
%
Therefore, 
$\|\psi(t) S_{t} u_{1}\|_{X_{s,b}}^{2} \le c_{\psi}
\|u_{1}\|_{H^{s-2}}^2$ and
taking square roots of both sides gives
%
%
\begin{equation}
  \begin{split}
    \|\psi(t) S_{t} u_{1}\|_{X_{s,b}} \le c_{\psi}
    \|u_{1}\|_{H^{s-2}}, \quad b \le 1.
  \end{split}
  \label{eqn:u-1-fin-est}
\end{equation}

\subsection{Estimate for $\psi_{\delta}(t) \int_{0}^{t} S_{t-t'} (u^{2})_{xx} dt'$.}
\label{sssec:non-lin-term}
We define the spatial Fourier transform by 
%
%
\begin{equation*}
\begin{split}
  \tilde{f}(n, t) = \int_{\ci} e^{-inx}f(x,t) dx
\end{split}
\end{equation*}
%
%
and the spacetime Fourier transform by
\begin{equation*}
\begin{split}
  \wh{f}(n, \tau) = \int_{\rr} \int_{\ci} e^{-inx-it\tau}f(x,t) dx dt.
\end{split}
\end{equation*}
%
%
Let $f(x,t) \doteq \psi_{\delta}(t) \int_{0}^{t} S_{t-t'} (u^{2})_{xx} dt'$. 
Then
%
%
\begin{equation}
  \begin{split}
    \wt{f}(n, t)
    & = \psi_{\delta} \int_{0}^{t}-n^{2}\wt{u^{2}}(n, t') \left[
    \frac{e^{in^{2}(t-t')} - e^{-in^{2}(t-t')}}{2 i n^{2}}
    \right] dt'
    \\
    & = - \frac{1}{2i} e^{in^{2}t} \int_{0}^{t} \psi_{\delta}(t) \wt{u^{2}}(n, t')
    e^{-in^{2}t'} dt' + 
    \frac{1}{2i} e^{-in^{2}t} \int_{0}^{t} \psi_{\delta}(t) \wt{u^{2}}(n, t')
    e^{in^{2}t'} dt' \\
    & \doteq - e^{in^{2}t} \wt{w_1}(n, t) + e^{-in^{2}t} \wt{w_2}(n, t)
  \end{split}
  \label{space-four-trans}
\end{equation}
%
where
%
%
\begin{gather*}
  w_{1}(x,t) = \frac{1}{4 i \pi} \sum_{n \in \zz} e^{inx} \left[ \int_{0}^{t}
  \psi_{\delta}(t) \wt{u^{2}}(n, t') e^{-in^{2}t'}
  dt'\right],
  \\
  w_{2}(x,t) = \frac{1}{4 i\pi} \sum_{n \in \zz} e^{inx} \left[ \int_{0}^{t} \psi_{\delta}(t) \wt{u^{2}}(n, t') e^{in^{2}t'} dt'
 \right].
\end{gather*}
%
%
%
Notice that \eqref{space-four-trans} is a \emph{global} relation in $t$.
Hence, taking its time Fourier transform gives
%
%
\begin{equation}
  \label{full-fourier-trans-exp}
\begin{split}
  \wh{f}(n, \tau) = -\wh{w_{1}}(n, \tau - n^{2}) + \wh{w_{2}}(n, \tau +
  n^{2}).
\end{split}
\end{equation}
%
%
Therefore, using the definition of the $X_{s,b}$ spaces, and applying
\eqref{square-ineq} gives 
%
%
\begin{equation*}
\begin{split}
  \| f \|_{X_{s,b}}^{2}
  & = \sum_{n \in \zz} (1 + |n|)^{2s} \int_{\rr} (1 + |
  | \tau | - n^{2} |)^{2b} | -\wh{w_{1}}(n, \tau - n^{2}) + \wh{w_{2}}(n, \tau +
  n^{2}) |^{2} d \tau
  \\
  & \le 4 \sum_{n \in \zz} (1 + |n|)^{2s} \int_{\rr} (1 + |
  | \tau | - n^{2} |)^{2b} | \wh{w_{1}}(n, \tau - n^{2}) d \tau
  \\
  & + 4 \sum_{n \in \zz} (1 + |n|)^{2s} \int_{\rr} (1 + |
  | \tau | - n^{2} |)^{2b} | \wh{w_{2}}(n, \tau + n^{2}) d \tau.
\end{split}
\end{equation*}
%
%
Applying a change of variable implies
%
%
%
%
\begin{equation}
\begin{split}
  \| f \|_{X_{s,b}}^{2}
  & \le 4 \sum_{n \in \zz} (1 + |n|)^{2s} \int_{\rr} (1 + |
  | \tau + n^{2} | - n^{2} |)^{2b} | \wh{w_{1}}(n, \tau) |^2 d \tau
  \\
  & + 4 \sum_{n \in \zz} (1 + |n|)^{2s} \int_{\rr} (1 + |
  | \tau - n^{2} | - n^{2} |)^{2b} | \wh{w_{2}}(n, \tau )|^2 d \tau.
\end{split}
\label{comp-pre-lemma}
\end{equation}
%
%
%
Applying  to \eqref{comp-pre-lemma} yields
%
%
\begin{equation}
  \label{pre-smoothing-lem}
\begin{split}
\| f \|_{X_{s,b}}^{2}
  & \le 4 \sum_{j=1}^{2}  \sum_{n \in \zz} (1 + |n|)^{2s} \int_{\rr} (1 + |
  \tau|)^{2b} | \wh{w_{j}}(n, \tau)|^2 d \tau
  \\
  & = 4 \sum_{j=1}^{2} \sum_{n \in \zz} (1 + |n|)^{2s} \|\wt{w_{j}}(n, t)
  \|^{2}_{H_{t}^{b}}
  \\
  & = \sum_{n \in \zz} \| \psi_{\delta}(t) \int_{0}^{t} \wt{u^2}(n, t')
  e^{in^{2}t'}dt'  \|_{H_{t}^{b}}
  + 
  \sum_{n \in \zz} \| \psi_{\delta}(t) \int_{0}^{t} \wt{u^2}(n, t')
  e^{-in^{2}t'}dt'  \|_{H_{t}^{b}}.
\end{split}
\end{equation}
%
We now need the following, whose proof is provided in~\cite{Ginibre:1996fk} and
the appendix.
%
%
%%%%%%%%%%%%%%%%%%%%%%%%%%%%%%%%%%%%%%%%%%%%%%%%%%%%%
%
%
%                Lemma to Reduce to Bilinear Est Form
%
%
%%%%%%%%%%%%%%%%%%%%%%%%%%%%%%%%%%%%%%%%%%%%%%%%%%%%%
%
%
\begin{lemma}[Lemma 2.2 in \cite{Farah:2009uq}]

Let $-1/2 < b' \le 0 \le b \le b' +1$ and $\delta \le 1$. Then
%
%
\begin{equation*}
\begin{split}
  \| \psi_{\delta}(t) \int_{0}^{t} g(t') dt' \|_{H^{b}_{t}} \le c_{\psi, b'}\delta^{1-(b - b')} \| g
  \|_{H_{t}^{b'}}
\end{split}
\end{equation*}
%
%
where $c_{\psi, b'}$ is a constant depending only on $\psi$ and $b'$.
\label{lem:pre-bilin-est}
\end{lemma}
%
%
%
%
%
%
Applying the lemma with $b' = -a$ we bound the right hand side
of \eqref{pre-smoothing-lem} by
%
%
\begin{equation*}
\begin{split}
  & c_{\psi} \delta^{1- (a + b)}\left[ \sum_{n \in \zz} (1 + |n|)^{s} \| \wt{u^{2}}(n, t')
  e^{in^{2}t'} \|_{H_{t}^{-a}}  +
  \sum_{n \in \zz} (1 + |n|)^{s} \| \wt{u^{2}}(n, t')
  e^{-in^{2}t'} \|_{H_{t}^{-a}} \right]
  \\
  & = c_{\psi} \delta^{1-(a + b)}
  \left [ \sum_{n \in \zz} (1 + |n|)^{s} \int_{\rr} (1 + | \tau
  |)^{-a} |\wh{u^{2}}(n, \tau - n^{2})|^{2} d \tau  \right .
  \\
  & + \left .
  \sum_{n \in \zz} (1 + |n|)^{s} \int_{\rr} (1 + | \tau
  |)^{-a} |\wh{u^{2}}(n, \tau + n^{2})|^{2} d \tau  \right ]
  \\
  & = c_{\psi} \delta^{1- (a + b)}
  \left [ \sum_{n \in \zz} (1 + |n|)^{s} \int_{\rr} (1 + | \tau
  + n^{2}
  |)^{-a} |\wh{u^{2}}(n, \tau )|^{2} d \tau  \right .
  \\
  & + \left . 
  \sum_{n \in \zz} (1 + |n|)^{s} \int_{\rr} (1 + | \tau
  - n^{2} |)^{-a} |\wh{u^{2}}(n, \tau )|^{2} d \tau  \right ]
\end{split}
\end{equation*}
%
%
which by \eqref{eqn:norm-key-ineq} is bounded by 
%
%
%
\begin{equation*}
\begin{split}
  & 
  2 c_{\psi} \delta^{1- (a + b)}
\sum_{n \in \zz} (1 + |n|)^{s} \int_{\rr} (1 + | |\tau|
  - n^{2} |)^{-a} 
|\wh{u^{2}}(n, \tau)|^{2} d \tau 
  \\
  & = 2 c_{\psi} \delta^{1- (a + b)}
\|u^{2} \|_{X_{s,-a}}^{2}.
\end{split}
\end{equation*}
%
%
Substituting back in for $f$ and taking square roots gives
%
%
\begin{equation}
\begin{split}
  \|\psi_{\delta}(t) \int_{0}^{t} S_{t-t'} (u^{2})_{xx} dt'\|_{X_{s,b}} \le
  c_{\psi} \delta^{[1- (a + b)]/2}\| u^{2}
  \|_{X_{s,-a}}, \qquad b \le 1-a, \quad 0 \le a < 1/2.
\end{split}
\label{eqn:non-lin-bound}
\end{equation}
%
%
To bound the right hand side, we now require a crucial bilinear
estimate.
%
%
%
%
%
%
Applying \autoref{prop:bilinear-est} to \eqref{eqn:non-lin-bound}, we conclude that
for $-1/4 < s < 0$, $1/2 < b \le 1 + 2s$, and $s \ge 0$, $1/2 < b \le 1$, we have 
%
%
\begin{equation}
\begin{split}
  \|\psi_{\delta}(t) \int_{0}^{t} S_{t-t'} (u^{2})_{xx} dt'\|_{X_{s,b}} \le
  c \| u \|^2_{X_{s,b}}. 
\end{split}
\label{eqn:nonlinear-term-bound}
\end{equation}
%
%
where $c = c_{\psi, s, T}$ for $-1/4 < s < 0$ and $c =c_{\psi, T}$ for $s \ge
0$.  
%
%
\begin{framed}
\begin{remark}
We obtain the restrictions on $s$, $b$, and $c$ as follows. If $-1/4 < s \le 0$, let
$0 < \epsilon < 1/2 + 2s$ and set
$a = -2s$, $b = 1 + 2s - \epsilon$. Then $0 \le a < 1/2$ and $b \le 1-a$,
so \eqref{eqn:non-lin-bound} holds. Furthermore, \autoref{prop:bilinear-est} holds,
and substituting for $a,b$ we get $\delta^{[1-(a + b)]/2} = \delta^{\ee/2}$.
\\
If $ s \ge 0$, choose $1/4 < a < 1/2$, and let $0 < \ee < 1/2 -a$ and $b = 1-a
-\ee$. Then
\eqref{eqn:non-lin-bound} holds. Furthermore, 
\autoref{prop:bilinear-est} holds, and substituting for $a, b$ we get $\delta^{\left[
1- (a + b) \right]/2} = \delta^{\ee/2} < \delta^{1/8}$. 
\end{remark}
\end{framed}
%
%
%
%%%%%%%%%%%%%%%%%%%%%%%%%%%%%%%%%%%%%%%%%%%%%%%%%%%%%
%
%
%                Proof of miscellaneous lemmas
%
%
%%%%%%%%%%%%%%%%%%%%%%%%%%%%%%%%%%%%%%%%%%%%%%%%%%%%%
%
%
\section{Proofs of Lemmas and Estimates} 
\label{sec:pfs-lems-est}
\subsection{Proof of \autoref{lem:embedding}}
%
%
\begin{equation}
  \label{dm}
	\begin{split}
		\lim_{t_{k} \to t} \|u(\cdot, t) - u(\cdot, t_{k})\|_{H^s(\ci)} 
    & = \lim_{t_{k} \to t} \|\psi_{\delta}(t) u(\cdot, t) - \psi_{\delta}(t_{k}) u(\cdot, t_{k})\|_{H^s(\ci)} 
		\\
		& = \lim_{t_{k} \to t} \left[ \sum_{n}\left( 1 + | n |
    \right)^{2s} | \psi_{\delta}(t)  \wh{u}(n, t) - \psi_{\delta}(t_{k}) \wh{ u}(n, t_{k}) |^2 \right]^{1/2}
		\\
		& = \lim_{t_{k} \to t} \left[ \sum_{n} \left( 1 + | n |
    \right)^{2s} | \int_{\rr} (e^{it \tau} - e^{it_{k} \tau})
    \wh{\psi_{\delta} u}(n,
		\tau) d \tau |^2 \right]^{1/2}.
	\end{split}
\end{equation}
First note that
%
%
%
%
\begin{equation*}
\begin{split}
& \lim_{t_{k} \to t}  | \int_{\rr} (e^{it \tau} - e^{it_{k} \tau})
    \wh{\psi_{\delta} u}(n,
		\tau) d \tau |^2 
    \\
    = 
     & \lim_{t_{k} \to t}  \int_{\rr} (e^{it \tau} - e^{it_{k} \tau})
    \wh{\psi_{\delta} u}(n,
    \tau) d \tau \times \lim_{t_{k} \to t} \overline{\int_{\rr} (e^{it \tau} - e^{it_{k} \tau})
    \wh{\psi_{\delta} u}(n,
    \tau) d \tau }  
    \\
    = 
    &  \lim_{t_{k} \to t}  \int_{\rr} (e^{it \tau} - e^{it_{k} \tau})
    \wh{\psi_{\delta} u}(n,
    \tau) d \tau \times \lim_{t_{k} \to t} \int_{\rr} (e^{-it \tau} - e^{-it_{k} \tau})
    \overline{\wh{\psi_{\delta} u}}(n,
    \tau) d \tau.   
    \end{split}
\end{equation*}
%
%
But for fixed $n$ 
%
%
\begin{equation*}
\begin{split}
|(e^{it \tau} - e^{it_{k} \tau})  
    \wh{\psi_{\delta} u}(n, \tau) | \le 2 |\wh{\psi_{\delta} u(n, \tau)} |
\end{split}
\end{equation*}
%
%
and
%
%
%
\begin{equation*}
\begin{split}
  \int_{\rr} |2 \wh{\psi_{\delta} u(n, \tau)} | d \tau < \infty.
\end{split}
\end{equation*}
%
%
Hence, by dominated convergence
%
%
\begin{equation*}
\begin{split}
\lim_{t_{k} \to t}  \int_{\rr} (e^{it \tau} - e^{it_{k} \tau})
    \wh{\psi_{\delta} u}(n,
    \tau) d \tau =  \int_{\rr} \lim_{t_{k} \to t} (e^{it \tau} - e^{it_{k} \tau})
    \wh{\psi_{\delta} u}(n,
    \tau) d \tau = 0. 
\end{split}
\end{equation*}
%
%
Similarly, 
%
%
%
\begin{equation*}
\begin{split}
\lim_{t_{k} \to t} \int_{\rr} (e^{-it \tau} - e^{-it_{k} \tau})
    \overline{\wh{\psi_{\delta} u}}(n,
    \tau) d \tau  =\int_{\rr}  \lim_{t_{k} \to t} (e^{-it \tau} - e^{-it_{k} \tau})
    \overline{\wh{\psi_{\delta} u}}(n,
    \tau) d \tau  = 0.
\end{split}
\end{equation*}
%
%
Hence
%
%
%
\begin{equation}
  \label{gh}
\begin{split}
  \lim_{t_{k} \to t} | \int_{\rr} (e^{it \tau} - e^{it_{k} \tau})
    \wh{\psi_{\delta} u}(n,
		\tau) d \tau |^2 = 0.
\end{split}
\end{equation}
%
%
		Furthermore,
    %
    %
    \begin{equation*}
    \begin{split}
      (1 + | n |)^{2s} | \int_{\rr} \left( e^{it\tau} - e^{it_{k} \tau} \right)
      \wh{\psi_{\delta}u}(n, \tau) d \tau|^{2} \le 4 (1 + | n |)^{2s} \left(
      \int_{\rr} | \wh{\psi_{\delta} u}(n, \tau)  | d \tau
      \right)^{2}
    \end{split}
    \end{equation*}
    %
    %
    and
		%
		%
		\begin{equation*}
			\begin{split}
         \sum_{n}  \left( 1 + | n |
        \right)^{2s} \left ( \int_{\rr} |\wh{\psi_{\delta} u}(n, \tau)| d \tau
        \right )^2  
        & = \|\wh{\psi_{\delta} u}\|_{\ell^{2}_{n}L^{1}_{\tau}}^2
		\\
		& \le \|\psi_{\delta} u \|_{X_{s,b}}^2 
	\end{split}
\end{equation*}
which is bounded by assumption. Therefore, applying dominated convergence and
\eqref{gh}, we
obtain 
%
%
\begin{equation*}
\begin{split}
  \text{rhs of \eqref{dm}} = \left[ \sum_{n} \left( 1 + | n |
    \right)^{2s} \lim_{t_{k} \to t} | \int_{\rr} (e^{it \tau} - e^{it_{k} \tau})
    \wh{\psi_{\delta} u}(n,
		\tau) d \tau |^2 \right]^{1/2} = 0
\end{split}
\end{equation*}
%
%
completing the proof. \qquad \qedsymbol
\subsection{Proof of \autoref{lem:embedding}} 
\label{ssec:embedding-pf}
%
%
\begin{equation*}
\begin{split}
  \| u(t) \|_{H^{s}}
  & = \left( \sum_{n} (1 + n^{2})^{s} |
  \wh{u}(n, t) |^{2} \right)^{1/2}
  \\
  & = \left( \sum_{n} (1 + n^{2})^{s} | \frac{1}{2\pi}
  \int_{\rr}e^{i \tau t} \wh{u}(n, \tau) d \tau |^{2} \right)^{1/2}
  \\
  & \lesssim \left[ \sum_{n} (1 + n^{2})^{s} \left( \int_{\rr} |
  \wh{u}(n, \tau)
  | d \tau \right)^{2} \right]^{1/2}
  \\
  & = \left[ \sum_{n} (1 + n^{2})^{s} \left( \int_{\rr} |
  \wh{u}(n, \tau)
  | (1 + | | \tau | - n^{2} |)^{b} (1 + | | \tau | - n^{2} |)^{-b} d \tau
  \right)^{2} \right]^{1/2}.
\end{split}
\end{equation*}
%
%
Cauchy-Schwartz in the $\tau$ variable then gives the bound
%
%
\begin{equation*}
\begin{split}
  & \left \{\sum_{n} (1 + n^{2})^{s} \left[ \left( \int_{\rr} (1 + | |
  \tau | - n^{2}
  |) | \wh{u}(n, \tau) |^{2} d \tau  \right)^{1/2} 
 \left( \int_{\rr} (1 + | |
  \tau | - n^{2}
  |)^{-1}  d \tau  \right)^{1/2}
  \right]^{2}\right \}^{1/2}
  \\
  & \le c_{b} \| u \|_{X^{s}}
\end{split}
\end{equation*}
%
where the last step follows from the estimate
%
%
%
%
\begin{equation*}
\begin{split}
\int_{\rr} (1 + | |
  \tau | - n^{2}
  |)^{-1}  d \tau 
  & = \int_{\tau \le 0} (1 + 
  |\tau  + n^{2}
  |)^{-1}  d \tau  + \int_{\tau \ge 0} (1 + | 
  \tau - n^{2}
  |)^{-1}  d \tau
  \\
  & = c_{b}.
\end{split}
\end{equation*}
%
%
completing the proof. \qquad \qedsymbol
%
\subsection{Proof of \eqref{eqn:norm-key-ineq}}
\label{ssec:pf-mod-princ}
By the reverse triangle inequality, we have
%
%
\begin{equation*}
\begin{split}
  | \tau | = | \tau + n^{2} - n^{2} | \ge | | \tau + n^{2} | - n^{2} |.
\end{split}
\end{equation*}
%
%
Furthermore, if $\tau - n^{2} < 0$, then
%
%
\begin{equation*}
\begin{split}
  | | \tau - n^{2} | - n^{2} | = | n^{2} - \tau - n^{2} | = | \tau |
\end{split}
\end{equation*}
%
%
while if $\tau - n^{2} > 0$, then
%
%
\begin{equation*}
\begin{split}
  | | \tau - n^{2} | - n^{2} | \le n^{2} \le \tau = |\tau|
\end{split}
\end{equation*}
%
%
completing the proof. \qquad \qedsymbol
%

%
\subsection{Proof of \autoref{lem:calc}}
%
By the change of variable $\theta \mapsto a/2 + x$, we have
%
%
\begin{equation*}
	\begin{split}
		\int_{\rr} \frac{1}{\langle \theta  \rangle \langle  a - \theta  \rangle}d \theta
	= \int_{\rr} \frac{1}{ \langle   a/2 + x  \rangle \langle  a/2 - x  \rangle}d x.
	\end{split}
\end{equation*}
%
%
Hence, it suffices to estimate 
%
%
%
%
\begin{equation*}
\begin{split}
  \int_{\rr} \frac{1}{\langle a + \theta \rangle ^{p} \langle a - \theta \rangle
  ^{q}} d \theta.
\end{split}
\end{equation*}
%
We will handle the case $a = 0$ later.
By symmetry, we may assume without loss of generality that $a > 0$. Furthermore,
we may assume without loss of generality that $p \ge q$. 
We split the integral above 
%
%
\begin{equation*}
\begin{split}
\int_{\rr} \frac{1}{\langle a + \theta \rangle ^{p} \langle a - \theta \rangle
  ^{q}} d \theta
  & = \int_{-2a}^{2a}
  \frac{1}{\langle a + \theta \rangle ^{p} \langle a - \theta \rangle
  ^{q}} d \theta
  \\
  & + \int_{| \theta | \ge 2a} 
\frac{1}{\langle a + \theta \rangle ^{p} \langle a - \theta \rangle
  ^{q}} d \theta
  \\
  & = I + II.
\end{split}
\end{equation*}
%
%
and first estimate
%
%
\begin{equation*}
\begin{split}
  I
  & \le \sup_{-2a \le \theta \le 2a} \frac{1}{\langle a + \theta \rangle
  ^{p}} \int_{-2a}^{2a} \frac{1}{\langle a - \theta \rangle ^{q}} d \theta
  \\
  & = \frac{1}{\langle a \rangle ^{p}} \int_{-2a}^{2a} \frac{1}{(1 + | a -
  \theta
  |)^{q}} d \theta
  \\
  & = \frac{4}{\langle a \rangle ^{p}} \int_{0}^{a} \frac{1}{(1 + a -
  \theta)^{q}} d \theta.
\end{split}
\end{equation*}
%
%
Integrating, we obtain
%
%
\begin{equation*}
 I
 \le 
  \frac{4 \log \langle a \rangle }{\langle a \rangle }, \qquad q =1
\end{equation*}
and
\begin{equation*}
  I \le \frac{4}{|q-1| \langle a \rangle ^{p +q -1}}, \qquad q \neq 1
\end{equation*}
%
%
%
%
Also
%
%
\begin{equation*}
\begin{split}
  II 
  & \simeq \int_{\theta \ge 2a} \frac{1}{\langle a + \theta \rangle ^{p} \langle a
  - \theta \rangle ^{q} d \theta}
  \\ 
  & = \int_{\theta \ge 2a} \frac{1}{(1 + \theta + a)^{p} (1 + \theta -
  a)^{q}} d \theta
  \\
  & \le \int_{\theta \ge 2a} \frac{1}{(1 + \theta -a)^{p+q}} d \theta
  \\
  & = \frac{1}{[(p + q)-1] \langle a \rangle ^{p+q -1}}, \qquad p + q > 1.
\end{split}
\end{equation*}
%
%
Collecting our estimates for $I$ and $II$, we obtain
%
%
\begin{equation*}
\begin{split}
  \int_{\rr} \frac{1}{\langle a + \theta \rangle ^{p} \langle a - \theta \rangle
  ^{q}} d \theta \le \frac{c_{p,q}}{\langle a \rangle ^{r}}, \qquad r =
  \min\left\{ p, q, p+q-1 \right\}.
\end{split}
\end{equation*}
which concludes the proof. \qquad \qedsymbol
%
%
\subsection{Proof of \eqref{simp-est-lower-bound}}
\label{ssec:simp-est-proof}
By symmetry, we may assume $ a \ge 0$, $x \ge 0$ without loss of generality.
Then
%
%
\begin{equation*}
\begin{split}
f(x) = 
\begin{cases}
  f_{1}(x) \doteq (1 + x-a)(1 + x + a), \quad & x \ge a \\
  f_{2}(x) \doteq (1 + a -x)(1 + x + a), \quad & 0 \le x \le a.
\end{cases}
\end{split}
\end{equation*}
%
%
Since $f_{1}'(x) = 2 + 2x$, $f_{1}(x)$ has a critical point at $x=-1$. Since
$f'(x) < 0$ for $x <-1$, and $f'(x) > 0$ for $x > -1$, $x=-1$ is the absolute
minimum of $f_{1}(x)$ for all $x \in \rr$. Since $f$ is increasing for $x > -1$,
it follows that $x=a$ is the absolute minimum of $f_{1}$ in the region $x \ge
a$. Furthermore, we have
%
%
\begin{equation*}
\begin{split}
  f_{1}(a) = 1 + 2 a.
\end{split}
\end{equation*}
%
%
Next, note that $f_{2}'(x) = -2x$, which gives the critical point $x = 0$. Since
$f_{2}' < 0$ for $x>0$ and $f_{2}' > 0$ for $x < 0$, we see that $x=0$ is
the absolute maximum of $f_{2}'$ for  $x \in \rr$. Since $f_{2}$ is decreasing
for $x > 0$, it follows that $x = a$
is the absolute minimum of $f_{2}$ in the region $0 \le x \le a$. Furthermore
%
%
\begin{equation*}
\begin{split}
  f_{2}(a) = 1 + 2 a
\end{split}
\end{equation*}
%
%
completing the proof. \qquad \qedsymbol
%
%
\subsection{Proof of \autoref{lem:pre-bilin-est}}
\label{ssec:pf-pre-bilin-est}
We have
%
%
\begin{equation*}
\begin{split}
  \psi_{\delta}(t) \int_{0}^{t} g(t') dt'
  & = \frac{\psi_{\delta}(t)}{2 \pi} \int_{0}^{t} \int_{\rr} e^{it' \tau}
  \wh{g}(\tau) d \tau dt'
  \\
  & \simeq \psi_{\delta}(t) \int_{\rr} \int_{0}^{t} e^{it' \tau} \wh{g}(\tau) dt' d\tau
  \\
  & = \psi_{\delta}(t)  \int_{\rr} \frac{e^{it \tau} -1}{i \tau}
  \wh{g}(\tau) d \tau
  \\
  & = \psi_{\delta}(t) \int_{| \tau |\delta \le 1} \frac{e^{it\tau}
  -1}{i\tau}\wh{g}(\tau) d \tau + \psi_{\delta}(t) \int_{| \tau |\delta \ge 1} \frac{e^{it\tau}
  -1}{i\tau}\wh{g}(\tau) d \tau
  \\
  & = \psi_{\delta}(t) \sum_{k \ge 1}\frac{t^{k}}{k!} \int_{| \tau |\delta \le 1}
  (i\tau)^{k-1} \wh{g}(\tau) d \tau
  - \psi_{\delta}(t) \int_{| \tau |\delta \ge 1}\frac{\wh{g}(\tau)}{i \tau} d \tau
  \\
  & + \psi_{\delta}(t) \int_{| \tau |\delta \ge 1}
  \frac{e^{it \tau}}{i \tau}\wh{g}(\tau) d \tau
  \\
  & = I + II + III.
\end{split}
\end{equation*}
%
%
Then
%
%
\begin{equation}
  \label{h1-norm}
\begin{split}
  & \|I \|_{H^{b}} = \sum_{k \ge 1} \frac{1}{k!} \| t^{k} \psi_{\delta}(t) \|_{H^{b}}
  | \int_{| \tau |\delta \le 1} (i \tau)^{k-1} \wh{g}(\tau) d \tau |
  \\
  & \le \sum_{k \ge 1} \frac{1}{k!} \|t^{k} \psi_{\delta}(t)\|_{H^{b}} \delta^{1-k}\int_{| \tau |\delta \le
  1} | \wh{g}(\tau) | d \tau.
\end{split}
\end{equation}
%
%
%Applying the estimate 
%%
%%
%\begin{equation}
	%\begin{split}
%\|t^k \psi_{\delta}(t) \|_{H^b(\rr)} 
%& = \left( \int_{\rr}(1 + | \lambda |)^{2b}| t^{k}\psi_{\delta}(\lambda) |^{2}
%\right)^{1/2}
%\\
%& = \left( \int_{\rr} (1 + | \lambda |)^{2b} | \int_{\rr} |e^{-i\lambda
%t}t^{k} \psi_{\delta}(t) dt|^{2} \right)^{1/2}
%\\
%& = \left( \int_{\rr} (1 + | \lambda |)^{2b} | \int_{| t | \le \delta} |e^{-i\lambda
%t}t^{k} \psi_{\delta}(t) dt|^{2} \right)^{1/2}
%\\
%& \le \delta^{k}
%\left( \int_{\rr} (1 + | \lambda |)^{2b}  \int_{| t | \le \delta} |\psi_{\delta}(t)|^{2} dt \right)^{1/2}
%\\
%& = \delta^{k} \| \psi_{\delta} \|_{H^{b}}
%\\
%& = c_{\psi_{\delta}} \delta^{k}
%\end{split}
%\end{equation}
%we bound the right hand side of \eqref{h1-norm} by
%%
%%
%\begin{equation}
  %\label{hu}
%\begin{split}
   %c_{\psi_{\delta}} \sum_{k \ge 1} \frac{1}{k!} \delta \int_{| \tau |\delta \le 1} |
  %\wh{g}(\tau)| d \tau
  %& \le  c_{\psi_{\delta}}e \int_{| \tau |\delta \le 1}| \wh{g}(\tau) | d \tau
  %\\
  %& =  \delta c_{\psi_{\delta}} e \int_{| \tau |\delta \le 1}| \wh{g}(\tau) | d \tau.
%\end{split}
%\end{equation}
%
%
First we estimate
%
%
\begin{equation*}
  \begin{split}
  & \int_{| \tau |\delta \le 1}| \wh{g}(\tau) | d \tau
  \\
  & \le  \| g \|_{H^{b'}} \left( \int_{| \tau | \delta \le 1}  (1 + | \tau
  |)^{-2b'} d \tau \right)^{1/2} 
  \\
  & = \sqrt{2} \| g \|_{H^{b'}} \left( \int_{0 \le \tau \le
  \frac{1}{\delta}}(1 + \tau)^{-2b'} d \tau \right)^{1/2}
  \\
  & = \sqrt{\frac{2}{-2b'+1}} \left[ (1 + \tau)^{-2b'+1} \Big |_{0}^{1/\delta}
  \right]^{1/2} \| g \|_{H^{b'}}, \qquad b'>-1/2
  \\
  & \simeq_{b'} \left[ (1 + 1/\delta)^{-2b'+1} - 1 \right]^{1/2} \| g
  \|_{H^{b'}}
  \\
  & \le \left[ (2/\delta)^{-2b'+1} -1 \right]^{1/2} \| g \|_{H^{b'}}, \qquad \delta \le 1
\end{split}
\end{equation*}
and so 
%
%
\begin{equation}
\label{microl-est}
\begin{split}
\int_{| \tau |\delta \le 1}| \wh{g}(\tau) | d \tau
\lesssim_{b'} \delta^{-1/2 + b'} \| g \|_{H^{b'}}, \qquad -1/2 < b' \le 0 < b \le b' +1.
\end{split}
\end{equation}
%
%
Next, for $k \ge 1$ we consider
%
%
\begin{equation}
  \label{1h}
\begin{split}
  \| t^{k} \psi_{\delta} \|^{2}_{H^{b}} 
  & = \int_{\rr} (1 + | \tau |)^{2b} | \wh{t^{k}\psi_{\delta}(t)}(\tau) |^{2} d \tau
  \\
  & = \int_{\rr} (1 + | \tau |)^{2b} | \int_{\rr} e^{-it \tau} t^{k}
  \psi_{\delta}(t) dt |^{2} d \tau
  \\
  & = \int_{\rr} (1 + | \tau |)^{2b} | \int_{\rr} e^{-it \tau} t^{k}
  \psi(t/\delta) dt |^{2} d \tau
    \end{split}
\end{equation}
%
%
which by the change of variable $s = t/\delta$ is equal to
%
%
%
%
\begin{equation}
  \label{2h}
\begin{split}
  \int_{\rr} (1 + | \xi |)^{2b} | \int_{\rr} e^{-is\delta \xi} (s\delta)^{k} \psi(s) \delta ds
  |^{2} d \xi.
\end{split}
\end{equation}
%
%
Now applying the change of variable $\xi = \lambda/\delta$, this is equal to
%
%
%
\begin{equation}
  \label{3h}
\begin{split}
  & \int_{\rr} \left( 1 + | \lambda/\delta | \right)^{2b} | \int_{\rr} e^{-is\lambda}
  (s\delta)^{k} \psi(s) \delta ds |^{2} \delta^{-1} d \lambda
  \\
  & = \delta^{2k+1}
  \int_{\rr} \left( 1 + | \lambda/\delta | \right)^{2b} | \wh{s^{k}
  \psi(s)}(\lambda )|^{2}  d \lambda
  \\
  & \le \delta^{2k+1}
  \int_{\rr} \left( 1/\delta + | \lambda/\delta | \right)^{2b} | \wh{s^{k}
  \psi(s)}(\lambda )|^{2}  d \lambda, \quad 0 < \delta \le 1
  \\
  & = \delta^{2k +1 -2b} \| s^{k} \psi(s) \|_{H^{b}}.
\end{split}
\end{equation}
%
%
But
%
%
\begin{equation}
	\label{4ng}
	\begin{split}
		\|t^k \psi(t) \|_{H^b(\rr)}^2
    & \le \|t^k \psi(t) \|_{H^1(\rr)}^2, \quad b \le 1
		\\
    & = \left( \|t^k \psi(t)\|_{L^2(\rr)} + \|\p_t \left( t^k \psi(t)
		\right)\|_{L^2(\rr)} \right)^2
		\\
		& \lesssim \|t^{k}\psi(t) \|_{L^2(\rr)}^2 + \|\p_t \left (t^{k}
		\psi(t) \right )\|_{L^2(\rr)}^2
		\\
		& \le \|t^k \psi(t) \|_{L^2(\rr)}^2 + \|t^k \p_t \psi(t)
		\|_{L^2(\rr)}^2 + \|k t^{k -1} \psi(t) \|_{L^2(\rr)}^2
		\\
		& = c_{\psi} + c_{\psi}' + c_{\psi}''k^2 
		\\
    & \lesssim 
    \begin{cases}
      c_{\psi} k^2 \qquad  & k \ge 1
      \\
      c_{\psi}, \qquad & k = 0.
    \end{cases}
	\end{split}
\end{equation}
%
%
%
%
Therefore,
%
%
\begin{equation}
\begin{split}
  \| t^{k}\psi_{\delta}(t) \|_{H^b} \lesssim_{\psi}  k \delta^{k+1/2 -b}.
\end{split}
\label{hb-norm}
\end{equation}
%
%
Substituting estimates \eqref{microl-est} and \eqref{hb-norm} into the right
hand side of \eqref{h1-norm}, we see that
%
%
\begin{equation*}
\begin{split}
  \| I \|_{H^{b}}
  & \lesssim_{\psi, b'} \sum_{k \ge 1}\frac{k}{k!} \delta^{k + 1/2 -b}
  \delta^{1-k} \delta^{-1/2 + b'}
  \\
  & = \sum_{k \ge 1}\frac{1}{(k-1)!}\delta^{1 - (b - b')}
  \\
  & = e \delta^{1 - (b - b')}
\end{split}
\end{equation*}
%
%
and so
%
%
\begin{equation}
\begin{split}
  \| I \|_{H^{b}} \lesssim_{\psi, b'} \delta^{1 - (b - b')}.
\end{split}
\label{rom-1}
\end{equation}
%
%
Next, we estimate
%
%
\begin{equation}
  \label{est-II}
\begin{split}
  \| II \|_{H^{b}}
  & = \| \psi_{\delta}(t) \int_{| \tau |\delta \ge 1} \frac{\wh{g}(\tau)}{i
  \tau} d \tau \|_{H^{b}}
  \\
  & = | \int_{| \tau | \delta \ge 1} \frac{\wh{g}(\tau)}{i \tau} d \tau | \times \| \psi_{\delta}
  \|_{H^{b}}
  \\
  & \le \int_{| \tau |\delta \ge 1} \frac{| \wh{g}(\tau)|}{ |\tau|} d \tau  \times \| \psi_{\delta}
  \|_{H^{b}}.
\end{split}
\end{equation}
%
%
Note that 
%
%
\begin{equation*}
\begin{split}
\int_{| \tau |\delta \ge 1} \frac{| \wh{g}(\tau)|}{ |\tau|} d \tau 
& = \int_{| \tau |\delta \ge 1} \frac{| \wh{g}(\tau)| (1 + | \tau |)^{b'} (1 + | \tau
|)^{-b'}}{ |\tau|} d \tau 
\\
& \le \|g\|_{H^{b'}} \left[ \int_{| \tau | \delta \ge 1} (1 + | \tau |)^{-2b'} | \tau
|^{-2} d \tau \right]^{1/2}.
\end{split}
\end{equation*}
%
%
Applying the inequality 
%
%
\begin{equation}
  \label{sob-term-bound}
\begin{split}
(1 + | \tau |) \le 2 | \tau |, \qquad | \tau |\delta \ge 1, \ 0 < \delta \le 1
\end{split}
\end{equation}
we bound the integral term by
%
%
%
%
\begin{equation*}
\begin{split}
\left[ \int_{| \tau | \delta \ge 1} 4 (1 + | \tau |)^{-2b'-2} | \tau
|^{-2} d \tau \right]^{1/2}
& = \left[ \int_{1/\delta \le  \tau < \infty } 8 (1 +  \tau )^{-2b'-2} 
d \tau \right]^{1/2}
\\
& = c_{b'} \left [ \left( 1 + \frac{1}{\delta} \right)^{-2b' -1} \right ] ^{1/2}, \quad b'> -1/2
\\
& \le c_{b'} \left [ \left (\frac{2}{\delta} \right )^{-2b' -1} \right ]^{1/2}, \quad \delta \le 1
\\
& \simeq_{b'} \delta^{1/2 + b'}.
\end{split}
\end{equation*}
%
It follows that
%
%
\begin{equation}
  \label{est-II-1}
\begin{split}
\int_{| \tau |\delta \ge 1} \frac{| \wh{g}(\tau)|}{ |\tau|} d \tau 
\lesssim_{b'} \delta^{1/2 + b'}.
\end{split}
\end{equation}
%
%
%
Furthermore, from computations \eqref{1h}-\eqref{3h} and \eqref{4ng} (both with
$k=0$), we obtain 
%
\begin{equation}
  \label{est-II-2}
\begin{split}
  \| \psi_{\delta} \|_{H^{b}}
  & = \delta^{1/2 - b} \| \psi(t) \|_{H^{b}}
  \\
  & \simeq_{\psi} \delta^{1/2 -b}.
\end{split}
\end{equation}
%
%
Substituting \eqref{est-II-1} and \eqref{est-II-2} into the right hand side of
\eqref{est-II}, we obtain
%
%
\begin{equation}
  \label{rom-2}
\begin{split}
  \| II \|_{H^{b}} \lesssim_{\psi, b'} \delta^{1 - (b - b')}.
\end{split}
\end{equation}
%
%
Lastly, we estimate $III$. Setting 
%
%
\begin{equation*}
\begin{split}
J(t) = \int_{| \tau |\delta \ge 1}
  \frac{e^{it \tau}}{i \tau}\wh{g}(\tau) d \tau
\end{split}
\end{equation*}
%
%
we have
%
%
\begin{equation*}
\begin{split}
  \| III \|_{H^{b}_{t}} = \| (1 + | \cdot |)^{b} \wh{\psi_{\delta}}(\cdot) *
  \wh{J}(\cdot) (\lambda) \|_{L^{2}_{\lambda}}
  \\
  & = \left[ \int_{\rr} (1 + | \lambda |)^{2b} | \int_{\rr}
  \wh{\psi_{\delta}}(\lambda - \tau) \wh{J}(\tau) d \tau |^{2} \right]^{1/2}
  \\
  & = \left[ \int_{\rr}  | \int_{\rr}
 (1 + | \lambda |)^{b} \wh{\psi_{\delta}}(\lambda - \tau) \wh{J}(\tau) d \tau |^{2} \right]^{1/2}
  \\
  & \le \left[ \int_{\rr}  | \int_{\rr}
  \wh{\psi_{\delta}}(\lambda - \tau) (1 + | \lambda - \tau |)^{b} \wh{J}(\tau)
  (1 + | \tau |)^{b}  d \tau |^{2} \right]^{1/2}
  \\
  & = \left[ \int_{\rr} | \wh{\psi_{\delta}}(1 + | \cdot |)^{b} * \wh{J}(1 + | \cdot
  |)^{b} (\tau) d \tau |^{2}  \right]^{1/2}
\end{split}
\end{equation*}
%
%
which by Young's inequality is bounded by
%
%
\begin{equation*}
\begin{split}
  \| \wh{\psi_{\delta}}(\tau) (1 + | \tau |) \|_{L^{1}} \| \wh{J}(\tau) (1 + | \tau
  |)^{b} \|_{L^{2}}
\end{split}
\end{equation*}
%
%
Rewriting
%
%
\begin{equation*}
\begin{split}
  J(t) 
  & = \int_{| \tau |\delta \ge 1}
  \frac{e^{it \tau}}{i \tau}\wh{g}(\tau) d \tau
  \\
  & = \int_{\rr }
  \frac{e^{it \tau}}{i \tau}\wh{g}(\tau) \chi_{| \tau |\delta \ge 1} d \tau
\end{split}
\end{equation*}
%
%
we see that
%
%
\begin{equation*}
\begin{split}
  \wh{J}(\tau) = \wh{g}(\tau) (i \tau)^{-1} \chi_{| \tau |\delta \ge 1}
\end{split}
\end{equation*}
%
%
and so 
%
%
\begin{equation*}
\begin{split}
\| \wh{J}(\tau) (1 + | \tau
|)^{b} \|_{L^{2}}^{2}
& = \int_{| \tau |\delta \ge 1} (1 + | \tau |)^{2b} | \wh{g}(\tau) |^{2}
\tau^{-2} d \tau
\\
& = \int_{| \tau |\delta \ge 1} (1 + | \tau |)^{2b'} | \wh{g}(\tau) |^{2} (1 + | \tau
|)^{2(b - b')} \tau^{-2} d \tau
\\
& \le \sup_{| \tau |\delta \ge 1} (1 + | \tau |)^{2(b - b')} \tau^{-2}
\|g\|_{H^{b'}}^{2}
\end{split}
\end{equation*}
%
%
Applying \eqref{sob-term-bound} we bound this by
%
%
\begin{equation*}
\begin{split}
  & \sup_{| \tau |\delta \ge 1} 2^{2(b - b')} (| \tau |)^{2(b - b' -1)} 
\|g\|_{H^{b'}}^{2}
\\
& \le \sup_{| \tau |\delta \ge 1} 4 (| \tau |)^{2(b - b' -1)} 
\|g\|_{H^{b'}}^{2}, \qquad -1/2 < b' \le 0 < b 
\\
& \simeq \delta^{2[1 - (b - b') ]} \| g \|_{H^{b'}}^{2}.
\end{split}
\end{equation*}
%
%
Therefore,
%
%
\begin{equation}
  \label{rom-3}
\begin{split}
  \| III \| \lesssim \delta^{1 - (b - b')} \| g \|_{H^{b'}}.
\end{split}
\end{equation}
%
%
Combining \eqref{rom-1}, \eqref{rom-2}, and \eqref{rom-3} concludes the proof.
\qquad \qedsymbol
%
%
%\subsection{Proof of \autoref{lem:embedding}}
%%
%%
%\begin{equation*}
%\begin{split}
  %\| u(t) - u(t') \|_{H^s}^{2}
  %& = \sum_{n} (1 + |n|)^{2s} [\wt{u}(n, t) - \wt{u}(n, t')]
  %\\
  %& = \sum_{n} (1 + |n|)^{2s} \int_{\rr} (e^{it\tau} - e^{it'
  %\tau})\wh{u}(n, \tau) d \tau
  %\\
  %& \le 2 \sum_{n} (1 + |n|)^{2s} \int_{\rr} \wh{u}(n, \tau) d \tau
  %\\
  %& \simeq \sum_{n} (1 + |n|)^{2s} \int_{\rr} (1 + | | \tau | -
  %n^{2} |)^{b}(1 + | | \tau | - n^{2} |)^{-b} | \wh{u}(n, \tau) | d \tau.
%\end{split}
%\end{equation*}
%%
%%
%Applying Cauchy-Schwartz in $\tau$, we bound this by
%%
%%
%\begin{equation*}
%\begin{split}
  %& \sum_{n} (1 + |n|)^{2s} \left[ \int_{\rr} (1 + | | \tau | -
  %n^{2} |) | \wh{u}(n, \tau) |^{2} d \tau \right]^{1/2} \left[ \int_{\rr}
  %(1 + | | \tau | - n^{2} |)^{-1} d \tau \right]^{1/2}
  %\\
  %& = \sum_{n} (1 + |n|)^{2s} \left[ \int_{\rr} (1 + | | \tau | -
  %n^{2} |) | \wh{u}(n, \tau) |^{2} d \tau \right]^{1/2} \left[ \int_{\rr}
  %(1 +  | \tau' | )^{-1} d \tau' \right]^{1/2}
  %\\
  %& = c \sum_{n} (1 + |n|)^{2s} \left[ \int_{\rr} (1 + | | \tau | -
  %n^{2} |) | \wh{u}(n, \tau) |^{2} d \tau \right]^{1/2} \qquad (b > 1/2). 
%\end{split}
%\end{equation*}
%%
%%
%%
%%
%Applying Cauchy-Schwartz in $n$ then gives the bound
%%
%%
%\begin{equation*}
%\begin{split}
  %\sum_{n} (1 + |n|)^{2s} \int_{\rr} (1 + | | \tau | - n^{2}
  %|) \wh{u}(n, \tau) d \tau = \| u \|_{X^{s}}^{2}.
%\end{split}
%\end{equation*}
%%
%%
%An application of dominated convergence completes the proof. \qquad \qedsymbol
%
%
%
%
%\nocite{*}
\bibliography{/Users/davidkarapetyan/math/bib-files/references.bib}
\bibliographystyle{amsalpha}
\end{document}
