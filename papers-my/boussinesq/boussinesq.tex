%
%
\documentclass[12pt,reqno]{amsart}
\usepackage{amsmath}
\usepackage{amssymb}
\usepackage[notref, notcite]{showkeys}
\usepackage{cancel}  %for cancelling terms explicity on pdf
\usepackage{yhmath}   %makes fourier transform look nicer, among other things
\usepackage{framed}  %for framing remarks, theorems, etc.
\usepackage{enumerate} %to change enumerate symbols
\usepackage[margin=2.5cm]{geometry}  %page layout
\numberwithin{equation}{section}  %eliminate need for keeping track of counters
\setlength{\parindent}{0in} %no indentation of paragraphs after section title
\renewcommand{\baselinestretch}{1.1} %increases vert spacing of text
%
\usepackage{hyperref}
\hypersetup{colorlinks=true,
linkcolor=blue,
citecolor=blue,
urlcolor=blue,
}
\synctex=1
%\usepackage[alphabetic, initials, msc-links]{amsrefs} %for the bibliography;
%%uses cite pkg. Must be loaded after hyperref, otherwise doesn't work properly
%%(conflicts with cref in particular)
%
%
\newcommand{\ds}{\displaystyle}
\newcommand{\ts}{\textstyle}
\newcommand{\nin}{\noindent}
\newcommand{\rr}{\mathbb{R}}
\newcommand{\nn}{\mathbb{N}}
\newcommand{\zz}{\mathbb{Z}}
\newcommand{\cc}{\mathbb{C}}
\newcommand{\ci}{\mathbb{T}}
\newcommand{\zzdot}{\dot{\zz}}
\newcommand{\wh}{\widehat}
\newcommand{\p}{\partial}
\newcommand{\ee}{\varepsilon}
\newcommand{\vp}{\varphi}
\newcommand{\wt}{\widetilde}
%
%
%
%
\newtheorem{theorem}{Theorem}[section]
\newtheorem{lemma}[theorem]{Lemma}
\newtheorem{corollary}[theorem]{Corollary}
\newtheorem{claim}[theorem]{Claim}
\newtheorem{prop}[theorem]{Proposition}
\newtheorem{proposition}[theorem]{Proposition}
\newtheorem{no}[theorem]{Notation}
\newtheorem{definition}[theorem]{Definition}
\newtheorem{remark}[theorem]{Remark}
\newtheorem{examp}{Example}[section]
\newtheorem{exercise}[theorem]{Exercise}
%
%makes proof environment bold instead of italic
%\makeatletter
%\renewcommand\subsubsection{\@startsection{subsubsection}{3}{\z@}%
%{-3.25ex\@plus -1ex \@minus -.2ex}%
%{1.5ex \@plus .2ex}%
%{\normalfont\normalsize \bfseries}}
%\makeatother
%makes subsubsubsection bold instead of italic
%
\def\sgn{\operatorname{sgn}}
\newcommand{\uol}{u^\omega_\lambda}
\newcommand{\lbar}{\bar{l}}
\renewcommand{\l}{\lambda}
\newcommand{\R}{\mathbb{R}}
\newcommand{\RR}{\mathcal{R}}
\newcommand{\al}{\alpha}
\newcommand{\ve}{q}
\newcommand{\tg}{{tan}}
\newcommand{\m}{q}
\newcommand{\N}{N}
\newcommand{\ta}{{\tilde{a}}}
\newcommand{\tb}{{\tilde{b}}}
\newcommand{\tc}{{\tilde{c}}}
\newcommand{\tS}{{\tilde{S}}}
\newcommand{\tP}{{\tilde{P}}}
\newcommand{\tu}{{\tilde{u}}}
\newcommand{\tw}{{\tilde{w}}}
\newcommand{\tA}{{\tilde{A}}}
\newcommand{\tX}{{\tilde{X}}}
\newcommand{\tphi}{{\tilde{\phi}}}
\begin{document}
\title{Ill-Posedness for the Boussinesq Equation}
\author{Dan-Andrei Geba, Alexandrou Himonas, and David Karapetyan}
\address{Department of Mathematics, University of Rochester, Rochester, NY 14627}
\address{Department of Mathematics, University of Notre Dame, Notre Dame, IN 46556}
\address{Department of Mathematics, University of Notre Dame, Notre Dame, IN 46556}
\date{\today}
%
%
\maketitle
%
%
%
%
%%%%%%%%%%%%%%%%%%%%%%%%%%%%%%%%%%%%%%%%%%%%%%
%
%
%
%Ill-Posedness
%
%
%
%%%%%%%%%%%%%%%%%%%%%%%%%%%%
%
%
\section{Introduction}
Our motivation will be the work of Bejanaru and Tao
\cite{Bejenaru-Tao-2006-Sharp-well-posedness-and-ill-posedness}. 
%
\begin{definition}
  For $f =f(x)= (f_{1}(x), f_{2}(x)), u = u(x,t), v = v(x,t)$, smooth cutoff $\psi(t)$ equal to the identity on $[0, 1]$, and $\psi_{\delta}(t) = \psi(t/\delta)$, formally define 
%
%
\begin{equation*}
\begin{split}
  L(f)
  \doteq \frac{1}{2 \pi} \psi(t) \sum_{n \in \zz} e^{inx}
  \wh{f_{1}}(n) \frac{e^{i\lambda(n)t} + e^{-i\lambda(n)t}}{2} 
  & + \psi(t) \sum_{n \in \zz} e^{inx}
  \wh{f_{2}}(n)\frac{e^{i\lambda(n)t} - e^{-i\lambda(n)t}}{2 i \lambda(n)} 
\end{split}
\end{equation*}
%
%
and
%
%
\begin{equation*}
\begin{split}
N(u, v)
& \doteq \frac{1}{4 \pi i} \psi_{\delta}(t) \sum_{n \in \zz} e^{inx} \frac{n^{2}}{\lambda(n)}
    \int_{0}^{t}[e^{i\lambda(n)(t-t')}-e^{-i\lambda(n)(t-t')}]
    \wh{uv}(n, t') dt'.
\end{split}
\end{equation*}
%
%
Then we let $A_{n}: H^{s} \times H^{s-2} \to C([0, \delta], H^{s}), \ n = 1, 2, \dots$ be the
recursively defined maps
%
%
\begin{equation*}
\begin{split}
  & A_{1}(f) \doteq L(f),
  \\
  & A_{n}(f) \doteq \sum_{j, k \in \mathbb{N}: j + k = n} N\left[
  A_{j}(f), A_{k}(f) \right], \quad n > 1.
\end{split}
\end{equation*}
\end{definition}
%
%
%
For example, 
%
%
\begin{equation*}
\begin{split}
  & A_{2}(f_{N}) = N(A_{1}(f_{N}), A_{1}(f_{N}))
  \\
  & A_{3}(f_{N}) = N(A_{1}(f_{N}), A_{2}(f_{N})) + N(A_{2}(f_{N}), A_{2}(f_{N}))
  \\
  & A_{4}(f_{N})= N(A_{1}(f_{N}), A_{3}(f_{N})) + N(A_{2}(f_{N}), A_{2}(f_{N}))
  + N(A_{3}(f_{N}), A_{1}(f_{N}))
\end{split}
\end{equation*}
%
%
and so on. We have the following.
%%
\begin{lemma}
  \label{lem:qwp-awp}
  For $\delta > 0$, $f \in B_{H^{s} \times H^{s-2}}(r)$, $s > -1/4$,  and $r = r(\delta)$
sufficiently small, we have the absolutely convergent
(in $C([0,\delta], H^{s})$) power series expansion
%
%
\begin{equation}
  \label{power-series-soln}
\begin{split}
  u[f] = \sum_{n=1}^{\infty} A_{n}(f).
\end{split}
\end{equation}
%
%
\label{lem:analytic-wp}
\end{lemma}
%
%
%
%
%%%%%%%%%%%%%%%%%%%%%%%%%%%%%%%%%%%%%%%%%%%%%%%%%%%%%
%
%
%				Proof of Ill-Posedness
%
%
%%%%%%%%%%%%%%%%%%%%%%%%%%%%%%%%%%%%%%%%%%%%%%%%%%%%%
%
%
\section{Ill-Posedness on the Circle (Failure of Continuity)} 
\label{sec:pf-ill-pos}
Given the machinery established in Dan's notes (following Bejenaru and Tao), we see that to prove failure of continuity of the Boussinesq data to solution map at $\vec{0} =
(0, 0)$, it will be enough to show the
following.
%
%
%
%
%
%%%%%%%%%%%%%%%%%%%%%%%%%%%%%%%%%%%%%%%%%%%%%%%%%%%%%
%
%
%                Failure of Continuity at 0
%
%
%%%%%%%%%%%%%%%%%%%%%%%%%%%%%%%%%%%%%%%%%%%%%%%%%%%%%
%
%
\begin{theorem}
  Let $N$ be a positive integer. Consider initial data $$f_{N}(x) =
  (f_{N,1}(x), f_{N,2}(x)) = \left ( \frac{r N^{-s}}{2 \pi}\left[ e^{iNx} + e^{i(1-N)x} \right], 0
  \right ).$$  Fix $ s > -1/4$ (the qualitative well-posedness index to date, see Farah and Scialom \cite{Farah:2010ys}). Then for $\beta < -s -2$
  %
  %
    \begin{enumerate}[(I)]
      \item{$ f_{N} \in B_{H^{s} \times H^{s}}(cr) \ \text{with associated
    solutions}
    \ u[f_{N}] = \sum_{n=1}^{\infty} A_{n}(f_{N}) \in C([0, \delta],
    H^{s})$}.
    \\
  \item
    $\|f_{N}\|_{H^{\beta} \times H^{\beta-2}} \to 0$ for $\beta < -s$ but 
    $ \displaystyle \frac{\|A_{2}[f_{N}]\|_{C( [-\delta, \delta], H^{\beta})}}{\| f_{N} \|_{H^{\beta} \times H^{\beta -2}}}$ diverges.
\end{enumerate}
  %
  %
\label{thm:ill-pos}
\end{theorem}
%
%
\begin{framed}
  \begin{remark}
    Fix $s = -1/4 + \ee/2$, $\ee > 0$. Taking $-s = 1/4 - \ee$, we see that the theorem implies failure of continuity of the data to solution map at $0$ in $H^{\beta} \times H^{\beta -2}$, where $\beta + 2 < 1/4 - \ee$, or $\beta < -7/4 - \ee$. Since $\ee > 0$ can be chosen arbitrarily small, we obtain ill-posedness for $\beta < -7/4$. 
  \end{remark}
\end{framed}
%
%
\begin{framed}
  \begin{remark}
    In particular, if were able to obtain qualitative well-posedness for $s >-1$, then we obtain ill-posedness for $\beta + 2 < 1 - \ee$, or $ \beta < -1 - \ee$, where $\ee > 0$ can be taken arbitrarily small. 
  \end{remark}
\end{framed}

%
%
%
%
%
%
%
\begin{proof}
  We first bound $A_{2}(f_{N})$ from below.
  Observe that $f_{N} \in C^{\infty}$. In particular
%
%
\begin{equation}
  \label{ill-pos-ce}
\begin{split}
  \wh{f_{N,1}}(n)
  & = r N^{-s}\int_{\ci} [e^{-ix(n - N)} + e^{-ix(n - (1-N))}]  dx  
  \\ 
  & = 
  \begin{cases}
    r N^{-s},  \quad  & n = N \ \text{or} \ n = 1-N
    \\
     0, \quad  & \text{otherwise}.
  \end{cases}
\end{split}
\end{equation}
%
Hence
%
%
\begin{equation}
  \label{ill-pos-ced}
\begin{split}
  \| f_{N,1} \|_{H^{\beta}}
  & = \frac{rN^{-s}}{2 \pi} \left( \sum_{n \in \zz} (1 + | n |)^{2 \beta} |
  \wh{f_{N,1}}(n) |^{2} \right)^{1/2}
  \\
  & \sim rN^{-s + \beta}.
\end{split}
\end{equation}
%
%
Therefore, from \eqref{ill-pos-ced} we see that 
\begin{equation*}
  \begin{split}
    & (f_{N,1}, 0) \in B_{H^{s} \times H^{s-2}}(cr)  
    \\
    & \|(f_{N,1}, 0)\|_{H^{\beta} \times H^{\beta -2}} \to 0 \ \text{for} \ \beta < s.
    \end{split}
\end{equation*}
However, recall that the Boussinesq equation is known to be qualitatively well-posed for $s > -1/4$. Hence, restricting $s > -1/4$, we see that since $f_{N} \in B_{H^{s} \times H^{s-2}}(cr)$, there exist unique associated 
solutions $u[f_{N}]$ which have common lifespan $\delta > 0$ in $H^{s}$ and weaker Sobolev topologies \emph{which does not depend
on $N$}. Choose $r$ sufficiently small such that $\delta =1$.  
Observe that
%
%
\begin{equation*}
\begin{split}
& \| A_{2}(f_{N}) \|_{C([0, 1], H^{\beta})} 
\\
  &  =  \| N[A_{1}(f_{N}), A_{1}(f_{N})] \|_{C([0, 1],
  H^{\beta})} 
  \\
  & = \sup_{0 \le t \le 1} \|  (1 + | n |)^{\beta}
  \frac{1}{4 i \pi} \frac{n^{2}}{\lambda(n)}
  \int_{0}^{t} \left( e^{i\lambda(n)(t-t')} - e^{-i\lambda(n)(t-t')} \right)
  \wh{[L(f_{N})]^{2}}(n, t') dt' \|_{\ell^{2}_{n}}
  \\
  & \gtrsim \sup_{0 \le t \le 1} \| (1 + | n |)^{\beta}
  \int_{0}^{t} \sum_{n_{1}} \left( e^{i\lambda(n)(t-t')} - e^{-i\lambda(n)(t-t')} \right)
  \\
  & \times \wh{f_{N}}(n - n_{1})\left( e^{i\lambda(n - n_{1})t'} +
  e^{-i\lambda(n - n_{1})t'} \right)
  \wh{f_{N}}(n_{1})\left( e^{i \lambda(n_{1})t'} +
  e^{-i \lambda(n_{1})t'} \right) 
  dt' \|_{\ell^{2}_{n}}
  \\
  & \gtrsim | \sup_{0 \le t \le 1} N^{-2s}
\int_{0}^{t} \left( e^{i\lambda(1)(t-t')} - e^{-i\lambda(1)(t-t')} \right)
  \\
  & \times \left( e^{i\lambda(1 - N)t'} +
  e^{-i\lambda(1 - N)t'} \right)
  \left( e^{i \lambda(N)t'} +
  e^{-i \lambda(N)t'} \right) 
  dt' | 
 \end{split}
\end{equation*}
%
%
where the last step follows from the inequality $\| a_{n} \|_{\ell^{2}} \ge | a_{1} |$, the definition of $f_{N}$, and the symmetry of the convolution. Distributing terms, and setting $t=1$, we bound below by 
%
%
\begin{equation}
  \label{pre-int-decay}
\begin{split}
  & N^{-2s} | \int_{0}^{1} \big \{ e^{i \lambda(1)}
  \\
  & \times \left( e^{it'[\lambda(1-N) + \lambda(N) - \lambda(1)]}
 + e^{it'[\lambda(1-N) - \lambda(N) - \lambda(1)]} 
 + e^{it'[\lambda(N) - \lambda(1-N) - \lambda(1)]}
 + e^{it'[-\lambda(1-N) - \lambda(N) - \lambda(1)]}
  \right)
 \\ 
 & - e^{-i \lambda(1)}
  \\
  & \times \left( e^{it'[\lambda(1-N) + \lambda(N) + \lambda(1)]}
 + e^{it'[\lambda(1-N) - \lambda(N) + \lambda(1)]} 
 + e^{it'[\lambda(N) - \lambda(1-N) + \lambda(1)]}
 + e^{it'[-\lambda(1-N) - \lambda(N) + \lambda(1)]}
  \right) \big \} dt'|.
\end{split}
\end{equation}
%
Now, an algebraic computation shows that
%
%
\begin{equation*}
\begin{split}
\lambda(N) - \lambda(1-N)
 & = \sqrt{N^{2} + N^{4}} - \sqrt{(N-1)^{2} + (N-1)^{4}} 
 \\
 & = O(N)
\end{split}
\end{equation*}
%
\begin{framed}
  This little computation is the beef of the argument, and will give us $O(N^{-2})$ less decay of \eqref{pre-int-decay} than what we had in previous versions of our work. See below for more details. 
\end{framed}
%
while
%
%
\begin{equation*}
\begin{split}
  \lambda(N) + \lambda(N-1) = O(N^{2}).
\end{split}
\end{equation*}
%
%
Hence, evaluating the integrand of \eqref{pre-int-decay} from $0$ to $1$, we obtain 
%
%
\begin{equation}
  \label{puy}
\begin{split}
  & \Big | e^{i \lambda(1)}\bigg ( \frac{e^{i[\lambda(1-N) + \lambda(N) - \lambda(1)]} - 1}{i[\lambda(1-N) + \lambda(N) - \lambda(1)]} + 
  \frac{e^{i[\lambda(1-N) - \lambda(N) - \lambda(1)]} - 1}{i[\lambda(1-N) - \lambda(N) + \lambda(1)]}  
  \\
  & +  \frac{e^{i[\lambda(N) - \lambda(1-N) - \lambda(1)]} - 1}{i[\lambda(N) - \lambda(1-N) - \lambda(1)]} + \frac{e^{i[-\lambda(1-N) - \lambda(N) - \lambda(1)]} - 1}{i[-\lambda(1-N) - \lambda(N) - \lambda(1)]} \bigg ) 
  \\
  & + e^{-i \lambda(1)}\bigg ( \frac{e^{-i[-\lambda(1-N) - \lambda(N) - \lambda(1)]} - 1}{i[-\lambda(1-N) - \lambda(N) - \lambda(1)]} + 
  \frac{e^{-i[\lambda(N) - \lambda(1-N) - \lambda(1)]} - 1}{i[\lambda(N) - \lambda(1-N) - \lambda(1)]} 
  \\
  & +  \frac{e^{-i[\lambda(1-N) - \lambda(N) - \lambda(1)]} - 1}{i[\lambda(1-N) - \lambda(N) - \lambda(1)]} + \frac{e^{-i[\lambda(1-N) + \lambda(N) - \lambda(1)]} - 1}{i[\lambda(1-N) + \lambda(N) - \lambda(1)]} \bigg ) 
  \bigg )  \bigg |
  \\
  & \simeq \bigg | \frac{\cos[\lambda(1-N) - \lambda(N)] - \cos \lambda(1)}{\lambda(1-N) - \lambda(N) - \lambda(1)} + \frac{\cos[\lambda(N) - \lambda(1-N)] - \cos \lambda(1)}{\lambda(N) - \lambda(1-N) - \lambda(1)} 
  \\
  & + \frac{\cos[\lambda(1-N) + \lambda(N)] - \cos \lambda(1)}{\lambda(1-N) + \lambda(N) - \lambda(1)} + \frac{\cos[-\lambda(1-N) - \lambda(N)] - \cos \lambda(1)}{-\lambda(1-N) - \lambda(N) - \lambda(1)}
  \bigg | 
  \\
  & \simeq \bigg | a_{N} \bigg ( \frac{1}{\lambda(1-N) - \lambda(N) - \lambda(1)} + \frac{1}{\lambda(N) - \lambda(1-N) - \lambda(1)} \bigg )
  \\
  & + b_{N} \bigg ( \frac{1}{\lambda(1-N) + \lambda(N) - \lambda(1)} + \frac{1}{-\lambda(N) - \lambda(1-N) - \lambda(1)} 
  \bigg ) \bigg | 
\end{split}
\end{equation}
%
%
where
%
%
\begin{equation*}
\begin{split}
  & a_{N} =   \cos[\lambda(1-N) - \lambda(N)] - \cos \lambda(1),
  \\
  & b_{N} =  \cos[\lambda(1-N) + \lambda(N)] - \cos \lambda(1).
\end{split}
\end{equation*}
%
Simplifying \eqref{puy} further, we obtain
%
%
\begin{equation*}
\begin{split}
  & 2 \lambda(1) \bigg |  \frac{a_{N}}{[\lambda(1-N) - \lambda(N) - \lambda(1)][\lambda(N) - \lambda(1-N) - \lambda(1)]} 
  \\
  & +  \frac{b_{N}}{[\lambda(1-N) + \lambda(N) - \lambda(1)][-\lambda(N) - \lambda(1-N) - \lambda(1)]} 
   \bigg |
   \\
   & \simeq | a_{N} O(N^{-2}) + b_{N} O(N^{-4}) | 
   \\
   & \sim | a_{n} O(N^{-2})|, \quad N > >1
\end{split}
\end{equation*}
%
since $\{|b_{N}|\}$ is uniformly bounded. Notice also that $\{|a_{N}|\}$ is divergent. Substituting \eqref{puy} into \eqref{pre-int-decay}, we see that for $N > > 1$,  
%
%
\begin{equation*}
\begin{split}
\|A_{2}[f_{N}]\|_{C( [-\delta, \delta], H^{\beta})}
\gtrsim N^{-2 s -2}|a_{N}| 
\end{split}
\end{equation*}
%
%
and so recalling \eqref{ill-pos-ced}, we get
%
%
\begin{equation*}
\begin{split}
\frac{\|A_{2}[f_{N}]\|_{C([-\delta, \delta], H^{\beta})}}{\| f_{N}
\|_{H^{\beta}}} \gtrsim N^{-s -2 - \beta} | a_{N} |
\end{split}
\end{equation*}
%
%
which diverges for $\beta < -s -2$. This concludes the proof.
%
%
%
%
%
\end{proof}
%
%
%
%
%
%%%%%%%%%%%%%%%%%%%%%%%%%%%%%%%%%%%%%%%%%%%%%%%%%%%%%
%
%
\section{Ill-Posedness on the Line (Failure of Continuity)} 
Given the machinery established in the introduction, we see that to prove failure of continuity of the Boussinesq data to solution map at $\vec{0} =
(0, 0)$, it will be enough to show the
following.
%
%
%
%
%
%%%%%%%%%%%%%%%%%%%%%%%%%%%%%%%%%%%%%%%%%%%%%%%%%%%%%
%
%
%                Failure of Continuity at 0
%
%
%%%%%%%%%%%%%%%%%%%%%%%%%%%%%%%%%%%%%%%%%%%%%%%%%%%%%
%
  \begin{theorem}
  Let $N$ be a positive integer. Consider initial data $$f_{N}(x) \doteq (f_{N,1}(x), f_{N,2}(x)) 
  = (r N^{-s} \frac{e^{ix(N+1)} - e^{ix N} + e^{-ix(N+1)} - e^{-iNx}}{ix} , 0).$$
  Fix $ s > -1/4$ (the qualitative well-posedness index to date, see \cite{Farah:2009uq}). Then for $\beta < -s -2$
  %
  %
    \begin{enumerate}[(I)]
      \item{$ f_{N} \in B_{H^{s} \times H^{s}}(cr) \ \text{with associated
    solutions}
    \ u[f_{N}] = \sum_{n=1}^{\infty} A_{n}(f_{N}) \in C([0, \delta],
    H^{s})$}.
    \\
  \item
    $\|f_{N}\|_{H^{\beta} \times H^{\beta-2}} \to 0$ for $\beta < -s$ but 
    $ \displaystyle \frac{\|A_{2}[f_{N}]\|_{C( [-\delta, \delta], H^{\beta})}}{\| f_{N} \|_{H^{\beta} \times H^{\beta -2}}}$ diverges.
\end{enumerate}
  %
  %
\label{thm:ill-pos-real}
\end{theorem}
  %
  %

%
%
%
\begin{proof}
%
Observe that
%
%
\begin{equation}
\begin{split}
  \wh{f_{N,1}}(\xi) \doteq r N^{-s} \chi_{\{N \le |\xi| \le N + 1\}\}}
\end{split}
\end{equation}
%
and so
%
%
%
\begin{equation*}
\begin{split}
  \| f_{N,1} \|_{H^{-1/2}}
  & \simeq rN^{-s} \left( \int_{N}^{N+1} (1 + \xi)^{-1} d \xi
  \right)^{1/2}
  \\
  & = rN^{-s} \left| \ln (1 + 1/N) \right|^{1/2},
  \\
  & \sim r N^{-s - 1/2}, \ N >>1 \ (\text{by l'Hospital})
\end{split}
\end{equation*}
%
%
and 
%
\begin{equation}
  \label{llh}
\begin{split}
  \| f_{N,1} \|_{H^{\beta}}
  & \sim rN^{-s} \left( \int_{N}^{N+1} (1 + \xi)^{2\beta} d \xi
  \right)^{1/2}
  \\
  & = rN^{-s} \left| \frac{1}{2\beta+1}( (N+2)^{2\beta+1} - (N+1)^{2\beta+1} ) \right|^{1/2}, \quad \beta \neq -1/2
  \\
  & = r N^{-s} \left| {\frac{1}{2\beta+1}} (N+1)^{2 \beta + 1} \left [ \left( \frac{N+2}{N+1}
    \right)^{2\beta + 1} - 1 \right ] \right|^{1/2}  \\
  & \sim r N^{-s + \beta}, \ N >>1 \ (\text{by l'Hospital}).
\end{split}
\end{equation}
%
Therefore, we see that 
\begin{equation*}
  \begin{split}
     & (f_{N,1}, 0) \in B_{H^{s} \times H^{s-2}}(cr) 
\\
    & \|(f_{N,1}, 0)\|_{H^{\beta} \times H^{\beta -2}} \to 0 \ \text{for} \ \beta < s.
    \end{split}
\end{equation*}
However, recall that the non-periodic Boussinesq equation is known to be qualitatively well-posed for $s > -1/4$. Hence, restricting $s > -1/4$, we see that since $f_{N} \in B_{H^{s} \times H^{s-2}}(cr)$, there exist unique associated 
solutions $u[f_{N}]$ which have common lifespan $\delta > 0$ in $H^{s}$ and weaker Sobolev topologies \emph{which does not depend
on $N$}. Choose $r$ sufficiently small such that $\delta =1$.  
Observe that
%
%
\begin{equation}
  \label{pol}
\begin{split}
  & \| A_{2}(f_{N}) \|_{C([0, 1], H^{\beta})} 
  \\
  &  =  \| N[A_{1}(f_{N}), A_{1}(f_{N})] \|_{C([0, 1],
  H^{\beta})} 
  \\
  & = \sup_{0 \le t \le 1} \| (1 + | \xi |)^{\beta}
  \frac{1}{4 i \pi} \frac{\xi^{2}}{\lambda(\xi)}
  \int_{0}^{t} \left( e^{i\lambda(\xi)(t-t')} - e^{-i\lambda(\xi)(t-t')} \right)
  \wh{[L(f_{N})]^{2}}(\xi, t') dt' \|_{L^{2}_{\xi}}
  \\
  & \gtrsim \sup_{0 \le t \le 1} \| 
  \int_{0}^{t} \int_{\xi_{1}} \left( e^{i\lambda(\xi)(t-t')} - e^{-i\lambda(\xi)(t-t')} \right)
  \\
  & \times \wh{f_{N}}(\xi - \xi_{1})\left( e^{i\lambda(\xi - \xi_{1})t'} +
  e^{-i\lambda(\xi - \xi_{1})t'} \right)
  \wh{f_{N}}(\xi_{1})\left( e^{i \lambda(\xi_{1})t'} +
  e^{-i \lambda(\xi_{1})t'} \right) 
  d \xi_{1} dt' \|_{L^{2}_{\xi}}
  \\
  & \ge N^{-2s} \sup_{0 \le t \le 1} 
  \int_{0}^{t} \int_{S} \left( e^{i\lambda(\xi)(t-t')} - e^{-i\lambda(\xi)(t-t')} \right)
  \\
  & \times \left( e^{i\lambda(\xi - \xi_{1})t'} +
  e^{-i\lambda(\xi - \xi_{1})t'} \right)
  \left( e^{i \lambda(\xi_{1})t'} +
  e^{-i \lambda(\xi_{1})t'} \right) 
  d \xi_{1} dt' \|_{L^{2}_{1/2 \le \xi \le 1}}
   \end{split}
\end{equation}
%
%
where the last step follows from the inequality $\| \cdot \|_{L^{2}_{\xi}} \ge \| \cdot \|_{L^{2}_{1/2 \le \xi \le 1}}$, the definition of $f_{N}$, and the fact that 
the condition $N \le | \xi - \xi_{1} | \le N+1$ 
implies $\xi_{1}$ is restricted to the set $S = \left\{ \xi_{1} \in \rr: -1 - N + \xi \le  \xi_{1} \le -N + \xi \ \text{or} \ N + \xi \le \xi_{1} \le N+1 + \xi \right\}$.
Distributing terms, setting $t=1$, and applying Fubini, we obtain the equivalent expression
%
%
\begin{equation}
  \label{pu}
\begin{split}
  & N^{-2s} \| \int_{S}\int_{0}^{1} e^{i \lambda(\xi)}
  \\
  & \times \left( e^{it'[\lambda(\xi - \xi_{1}) + \lambda(\xi_{1}) - \lambda(\xi)]}
 + e^{it'[\lambda(\xi_{1}) - \lambda(\xi - \xi_{1}) - \lambda(\xi)]} 
 + e^{it'[\lambda(\xi - \xi_{1}) - \lambda(\xi) - \lambda(\xi_{1})]}
 + e^{it'[-\lambda(\xi - \xi_{1}) - \lambda(\xi_{1}) - \lambda(\xi)]}
  \right)
 \\ 
 & - e^{-i \lambda(\xi)}
  \\
  & \times \left( e^{it'[\lambda(\xi - \xi_{1}) + \lambda(\xi_{1}) + \lambda(\xi)]}
 + e^{it'[\lambda(\xi_{1}) - \lambda(\xi - \xi_{1}) + \lambda(\xi)]} 
 + e^{it'[\lambda(\xi_{1}) - \lambda(\xi - \xi_{1}) - \lambda(\xi)]}
 + e^{it'[-\lambda(\xi - \xi_{1}) - \lambda(\xi_{1}) + \lambda(\xi)]}
 \right) \|_{L^{2}_{1/2 \le \xi \le 1}}.
\end{split}
\end{equation}
%
%
Evaluating the integrand of \eqref{pu} from $0$ to $1$, we obtain 
%
%
\begin{equation}
  \label{puy-real}
\begin{split}
  & e^{i \lambda(\xi)}\bigg ( \frac{e^{i[\lambda(\xi - \xi_{1}) + \lambda(\xi_{1}) - \lambda(\xi)]} - 1}{i[\lambda(\xi - \xi_{1}) + \lambda(\xi_{1}) - \lambda(\xi)]} + 
  \frac{e^{i[\lambda(\xi - \xi_{1}) - \lambda(\xi_{1}) - \lambda(\xi)]} - 1}{i[\lambda(\xi - \xi_{1}) - \lambda(\xi_{1}) + \lambda(\xi)]}  
  \\
  & +  \frac{e^{i[\lambda(\xi_{1}) - \lambda(\xi - \xi_{1}) - \lambda(\xi)]} - 1}{i[\lambda(\xi_{1}) - \lambda(\xi - \xi_{1}) - \lambda(\xi)]} + \frac{e^{i[-\lambda(\xi - \xi_{1}) - \lambda(\xi_{1}) - \lambda(\xi)]} - 1}{i[-\lambda(\xi - \xi_{1}) - \lambda(\xi_{1}) - \lambda(\xi)]} \bigg ) 
  \\
  & + e^{-i \lambda(\xi)}\bigg ( \frac{e^{-i[-\lambda(\xi - \xi_{1}) - \lambda(\xi_{1}) - \lambda(\xi)]} - 1}{i[-\lambda(\xi - \xi_{1}) - \lambda(\xi_{1}) - \lambda(\xi)]} + 
  \frac{e^{-i[\lambda(\xi_{1}) - \lambda(\xi - \xi_{1}) - \lambda(\xi)]} - 1}{i[\lambda(\xi_{1}) - \lambda(\xi - \xi_{1}) - \lambda(\xi)]} 
  \\
  & +  \frac{e^{-i[\lambda(\xi - \xi_{1}) - \lambda(\xi_{1}) - \lambda(\xi)]} - 1}{i[\lambda(\xi - \xi_{1}) - \lambda(\xi_{1}) - \lambda(\xi)]} + \frac{e^{-i[\lambda(\xi - \xi_{1}) + \lambda(\xi_{1}) - \lambda(\xi)]} - 1}{i[\lambda(\xi - \xi_{1}) + \lambda(\xi_{1}) - \lambda(\xi)]} \bigg ) 
  \bigg )  
  \\
  & \simeq \frac{\cos[\lambda(\xi - \xi_{1}) - \lambda(\xi_{1})] - \cos \lambda(\xi)}{\lambda(\xi - \xi_{1}) - \lambda(\xi_{1}) - \lambda(\xi)} + \frac{\cos[\lambda(\xi_{1}) - \lambda(\xi - \xi_{1})] - \cos \lambda(\xi)}{\lambda(\xi_{1}) - \lambda(\xi - \xi_{1}) - \lambda(\xi)} 
  \\
  & + \frac{\cos[\lambda(\xi - \xi_{1}) + \lambda(\xi_{1})] - \cos \lambda(\xi)}{\lambda(\xi - \xi_{1}) + \lambda(\xi_{1}) - \lambda(\xi)} + \frac{\cos[-\lambda(\xi - \xi_{1}) - \lambda(\xi_{1})] - \cos \lambda(\xi)}{-\lambda(\xi - \xi_{1}) - \lambda(\xi_{1}) - \lambda(\xi)}
  \\
  & \simeq a_{N} \bigg ( \frac{1}{\lambda(\xi - \xi_{1}) - \lambda(\xi_{1}) - \lambda(\xi)} + \frac{1}{\lambda(\xi_{1}) - \lambda(\xi - \xi_{1}) - \lambda(\xi)} \bigg )
  \\
  & + b_{N} \bigg ( \frac{1}{\lambda(\xi - \xi_{1}) + \lambda(\xi_{1}) - \lambda(\xi)} + \frac{1}{-\lambda(\xi_{1}) - \lambda(\xi - \xi_{1}) - \lambda(\xi)} 
  \bigg ) 
\end{split}
\end{equation}
where
%
%
\begin{equation*}
\begin{split}
  & a_{N} =   \cos[\lambda(\xi - \xi_{1}) - \lambda(\xi_{1})] - \cos \lambda(\xi),
  \\
  & b_{N} =  \cos[\lambda(\xi - \xi_{1}) + \lambda(\xi_{1})] - \cos \lambda(\xi).
\end{split}
\end{equation*}
Now, for $1/2 \le \xi \le 1$ and $\xi_{1} \in S$, a calculus computation shows that
%
%
\begin{equation*}
\begin{split}
|\lambda(\xi_{1}) - \lambda(\xi - \xi_{1})|
 & \sim \sqrt{N^{2} + N^{4}} - \sqrt{(N-1)^{2} + (N-1)^{4}} 
 \\
 & = O(N)
\end{split}
\end{equation*}
%
%
while
%
%
\begin{equation*}
\begin{split}
  |\lambda(\xi_{1}) + \lambda(\xi - \xi_{1})| \sim O(N^{2}).
\end{split}
\end{equation*}
Hence, we can simplify \eqref{puy-real} further to obtain
%
%
\begin{equation}
  \label{pre-cont-an}
\begin{split}
  & -2 \lambda(\xi) \bigg (  \frac{a_{N}}{[\lambda(\xi - \xi_{1}) - \lambda(\xi_{1}) - \lambda(\xi)][\lambda(\xi_{1}) - \lambda(\xi - \xi_{1}) - \lambda(\xi)]} 
  \\
  & +  \frac{b_{N}}{[\lambda(\xi - \xi_{1}) + \lambda(\xi_{1}) - \lambda(\xi)][-\lambda(\xi_{1}) - \lambda(\xi - \xi_{1}) - \lambda(\xi)]} 
   \bigg )
   \\
   & = -2 \lambda(\xi) \left [ a_{N} O(N^{-2}) + b_{N} O(N^{-4})  \right ]
\end{split}
\end{equation}
\begin{framed}
In order to discard the $b_{N}O(N^{-4})$ term, it will be enough to obtain 
a positive nonzero lower bound for the first term of $a_{N}$ (the second terms of $a_{N}$ and $b_{N}$ are identical). We shall do this by choosing an appropriate sequence of $N_{k} >>1$. Proceeding, we first observe that
%

%
\begin{equation}
  \label{ty}
\begin{split}
  \lambda(\xi_{1}) - \lambda(\xi - \xi_{1})  
  & = \sqrt{\xi_{1}^{2} + \xi_{1}^{4}} -
  \sqrt{(\xi - \xi_{1}^{2})^{2} + (\xi - \xi_1)^{4}}  
  \\
  & = 
  \frac{\xi_{1}^{2} + \xi_{1}^{4} - (\xi - \xi_{1})^{2} - (\xi - \xi_{1})^{4}}{\sqrt{(\xi - \xi_{1})^{2} + (\xi - \xi_{1})^{4}} + \sqrt{\xi_{1}^{2} + \xi_{1}^{4}}}
  \\
  & = \frac{-\xi^{4} - \xi^{2} + 2 \xi \xi_{1} + 4 \xi_{1}^{3} \xi - 6 \xi_{1}^{2}\xi^{2} + 4 \xi_{1} \xi^{3}}{\sqrt{(\xi - \xi_{1})^{2} + (\xi - \xi_{1})^{4}} + \sqrt{\xi_{1}^{2} + \xi_{1}^{4}}}
\\
  & = 2 \xi_{1} \xi - 3 \xi^{2} + \eta
\end{split}
\end{equation}
%
where $\eta = \eta(\xi_{1}, \xi)$ and $\eta \to 0$ as $N \to \infty$ uniformly in $\xi$, where $\xi \in [1/2, 1]$. 
%
%
Since $\cos(\cdot)$ is a continuous function, and the lower bound for our sequence $N_{k}$ can be chosen arbitrarily large, to control the first term of $a_{N}$ it will be enough to establish a non-zero lower bound for the quantity $\cos[2 \xi_{1} \xi - 3 \xi^{2}]$ independent of $1/2 \le \xi \le 1$
when $N-1 \le \xi_{1} \le N$. Solving the system
%
%
\begin{equation*}
\begin{split}
  &k\pi/8 \le 2N\xi - 3\xi^{2} \le k\pi/4
\\
& k\pi/8 \le 2(N-1)\xi - 3\xi^{2} \le k\pi/4
\end{split}
\end{equation*}
%
where $k >>1$ and $k = 1\mod{16}$, we obtain
%
%
%
\begin{equation*}
\begin{split}
  \frac{k\pi /8 + 2 \xi + 3 \xi^{2}}{2 \xi} \le N \le \frac{k \pi /4 + 3 \xi^{2}}{2 \xi}.
\end{split}
\end{equation*}
%
Since we are varying $1/2 \le \xi \le 1$, we want to ensure our choice of $N = N_{k}$ does not vary with $\xi$. Since $k >>1$, it will suffice to take 
%
%
\begin{equation*}
\begin{split}
\frac{k\pi /8 + 2 (1/2) + 3 (1/2)^{2}}{2 (1/2)} \le N \le \frac{k \pi /4 + 3 (1/2)^{2}}{2 (1/2)}
\end{split}
\end{equation*}
%
%
or
%
%
\begin{equation}
  \label{reg}
\begin{split}
k\pi/8 + 7/4 \le N \le k \pi/4 + 3/4.
\end{split}
\end{equation}
%
%
%
Since $0.5 < \cos(x) < 1$ for $k\pi/8  \le x \le k\pi/4$, it follows that for $k >>1$ and $N = N_{k}$ satisfying \eqref{reg}, we have 

\begin{equation*}
  \eqref{pre-cont-an} \sim -2 \lambda(\xi) a_{N} O(N^{-2})
 \end{equation*}
%
 \end{framed}
%
due to the uniform boundedness of $\{|b_{N}|\}$. Substituting into \eqref{pu}, we see that for $N > > 1$,  
%
%
\begin{equation*}
\begin{split}
\|A_{2}[f_{N}]\|_{C( [-\delta, \delta], H^{\beta})}
\gtrsim N^{-2 s -2} \|  \lambda(\xi) \int_{S}a_{N} d \xi_{1} \|_{L^{2}_{1/2 \le \xi \le 1}}
\end{split}
\end{equation*}
%
where we have a priori chosen $\ee$ sufficiently small.
%
To estimate the integral from below, first observe that 
\begin{equation*}
\begin{split}
  & \cos[\lambda(\xi - \xi_{1}) - \lambda(\xi_{1})] 
  \\
  & = \cos\left[ \sqrt{(\xi - \xi_{1})^{2} + (\xi - \xi_{1})^{4}} - \sqrt{\xi_{1}^{2} + \xi_{1}^{4}} \right] 
  \\
  & = \p_{x} \{ \cos\left[ \sqrt{(\xi - \xi_{1})^{2} + (\xi - \xi_{1})^{4}} - \sqrt{\xi_{1}^{2} + \xi_{1}^{4}} \right] \} 
  \\
  & \times \left\{ [(\xi - \xi_{1})^{2} + (\xi - \xi_{1})^{4}]^{-1/2}[\xi_{1} - \xi + 2(\xi_{1} - \xi)^{3}] - (\xi_{1}^{2} + \xi_{1}^{4})^{-1/2} (\xi_{1} + 2 \xi_{1}^{3}) \right\}^{-1}.
\end{split}
\end{equation*}
%
%
Furthermore, since $\cos[\lambda(\xi - \xi_{1}) - \lambda(\xi_{1})]$ is continuous with respect to $\xi_{1}$ and the interval $[N, N+1]$ is compact, it follows that we can partition $[N, N+1]$ into finitely many disjoint sets $S_{j}$ with $\cup_{j} S_{j} = [N, N+1]$ and such that for fixed $j$ the quantity $\cos[\lambda(\xi - \xi_{1}) - \lambda(\xi_{1})]$ is either positive or negative for all $\xi_{1} \in S_{j}$. Therefore, applying the fundamental theorem of calculus, the triangle inequality, and the fact that, for $1/2 \le \xi \le 1$ and $\xi_{1} \in S$
%
%
\begin{equation*}
\begin{split}
  \left | [(\xi - \xi_{1})^{2} + (\xi - \xi_{1})^{4}]^{-1/2}[\xi_{1} - \xi + 2(\xi_{1} - \xi)^{3}] - (\xi_{1}^{2} + \xi_{1}^{4})^{-1/2} (\xi_{1} + 2 \xi_{1}^{3}) \right | ^{-1} \le O(N^{-1}) 
\end{split}
\end{equation*}
%
%
we obtain the bound
%
%
\begin{equation*}
\begin{split}
  | \int_{S} \cos[\lambda(\xi - \xi_{1}) - \lambda(\xi_{1})] d \xi_{1} |
   & \lesssim O(N^{-1}) \sum_{j} \left | \cos[\lambda(\xi - \xi_{1}) - \lambda(\xi_{1}) ] \} \right | \bigg |_{dS_{j}}
  \\
  & \le O(N^{-1}) \sum_{j} | S_{j} |
  \\
  & = O(N^{-1}).
\end{split}
\end{equation*}
%
%
Hence, recalling that $a_{N} =   \cos[\lambda(\xi - \xi_{1}) - \lambda(\xi_{1})] - \cos \lambda(\xi)$, we see that for $N >>1$
%
%
\begin{equation*}
\begin{split}
\| | \int_{S}a_{N} d \xi_{1} \|_{L^{2}_{1/2 \le \xi \le 1}}
& \gtrsim N^{-2s -2}\| \lambda(\xi) \left [ O(N^{-1}) +  \int_{S} -\cos \lambda(\xi) d \xi_{1} \right ] \|_{L^{2}_{1/2 \le \xi \le 1}}
\\
& = N^{-2s-2} \| \lambda(\xi)\left [ O(N^{-1}) - \cos \lambda(\xi) \right ] \|_{L^{2}_{1/2 \le \xi \le 1}}
\\
& \gtrsim N^{-2s-2}
\end{split}
\end{equation*}
%
and so recalling \eqref{llh}, we get
%
%
\begin{equation*}
\begin{split}
\frac{\|A_{2}[f_{N}]\|_{C([-\delta, \delta], H^{\beta})}}{\| f_{N}
\|_{H^{\beta}}} \gtrsim N^{-s -2 - \beta} 
\end{split}
\end{equation*}
%
%
which for $\beta < -s-2$ blows up as $N \to \infty$. This concludes the proof.
%
%
\end{proof}
%
%
%
%
%
%
%
%
%%%%%%%%%%%%%%%%%%%%%%%%%%%%%%%%%%%%%%%%%%%%%%%%%%%%%%%%%
%
%
%
%Quadratic NLS
%
%
%%%%%%%%%%%%%%%%%%%%%%%%%%%%%%%%%
%
%
\section{Quadratic NLS}
%
  We consider the quadratic non-linear Schr\"odinger equation (qNLS)
\begin{equation}\label{nls-quad}
\begin{split}
& i u_t + u_{xx} = u^2\\
& u(x, 0) = f \in H^s(\R) \\
\end{split}
\end{equation}
which has the localized integral formulation 
\begin{equation}\label{uln-2}
 u = L(f) + N(u,u)
 \end{equation}
where $L$ is the linear operator
\begin{equation}\label{Ldef}
  L(f)(t) := \eta(t) e^{it\partial_{xx}} f
\end{equation}
and $N$ is the bilinear operator
\begin{equation}\label{Ndef}
\begin{split}
N(u,v)(t) &:=
\eta(t) e^{it\partial_{xx}} \int_\R a(s) e^{-is\partial_{xx}}(u(s)v(s))\ ds\\
&+ \int_\R a(t-s) e^{i(t-s)\partial_{xx}}(u(s) v(s))\ ds
\end{split}
\end{equation}
%
where $\eta: \R \to \R$ is a smooth bump function such that $\eta(t) = 1$ for
$|t| \leq 1$ and $\eta(t) = 0$ for $|t| > 2$, and $a(t) := \frac{1}{2}
\sgn(t)\eta(t/5)$.  
%
Bejanaru and Tao proved the following well-posedness result.
\begin{theorem}[Local well-posedness in $H^{-1}(\R)$]\label{lwp}
 %
 %
 %
Let $r > 0$ be any radius, and let $B_r$ be the ball
$$ B_r := B_{H^{-1}(\R)}(0,r) := \{ u_0 \in H^{-1}(\R): \| u_0 \|_{H^{-1}(\R)} < r \}.$$
Then there exists a time $T > 0$ (in fact we obtain $T = \max(1, c r^{-1/2} )$ for some absolute constant $c>0$)
and a map $f \mapsto u[f]$ which is continuous from $B_r$ to $C([0,T],
H^{-1}(\rr))$, such that the
restriction of this map to $B_r \cap H^s(\R)$ (with the $H^s(\R)$ topology) maps continuously 
to $C([0,T], H^s(\rr))$ for any $s \geq -1$.  Furthermore, if $f$ lies in
a smooth space, say $B_r \cap H^3(\R)$, then $u[f]$ lies in $C([0, T], H^3) \cap
C^1([0,T], H^1(\rr))$ and
solves the equation \eqref{nls-quad} in the classical sense.   
\end{theorem}
%
%
Furthermore, as in the case of Boussinesq, solutions to the qNLS have an asymptotic
expansion in terms of iterates of the non-linearity. More precisely, they show
the following. 
%
\begin{lemma}
  \label{lem:qnls-asymp}
  For $f \in B_{H^{s}}(r)$, $s \ge -1$ and $\delta=\delta(r)$
sufficiently small, we have the absolutely convergent
(in $C([-\delta, \delta], H^{s})$) power series expansion
%
%
\begin{equation}
  \label{qnlspower-series-soln}
\begin{split}
  u[f] = \sum_{n=1}^{\infty} A_{n}(f)
\end{split}
\end{equation}
%
%
\end{lemma}
They then invoke this well-posedness result to prove ill-posedness for $s <
-1$ as follows.
%
%
%
%
\subsection{Proof of Ill-Posedness for $s < -1$} 
%
%
%
Fix $s < -1$; we may rescale the lifespan $\delta$ to equal $1$.  Suppose for contradiction
that the solution map $f \mapsto u[f]$ is continuous on $B_r$ (with the
$H^{-1}_x(\R)$ topology) to $C([0, 1], H^{-1}(\rr))$ 
(with the $C([0,1], H^{s}(\rr)$ topology). From 
Lemma~\ref{lem:qnls-asymp}, we conclude that the quadratic operator
$$ A_2: f \mapsto N_2(Lf, Lf)$$ 
restricted to $[0,1] \times \R$, is continuous from $B_r$ (with the $H^{-1}_x(\R)$ topology) 
to $C([0,1], H^{s}(\rr))$.  In particular, this implies that
$$ \sup_{0 \leq t\leq 1}\| A_2(f_N)(y) \|_{H^{s}(\R)} \to 0$$
whenever $f_N$ is a sequence of functions in $B_r$ which goes to zero in $H^{-1}_x(\R)$ norm.
The left-hand side can be expanded by \eqref{Ldef}, \eqref{Ndef} as
$$ \sup_{0 \leq t \leq 1} 
\| \int_0^t \exp(i(t-t')\partial_{xx})((\exp(it' \partial_{xx}) f_N)^2)\ dt' \|_{H^{s}(\R)}$$
which after taking Fourier transforms becomes
$$\sup_{0 \leq t \leq 1} 
\| \langle \xi \rangle^{s}
\int_0^t \int_\R \exp(-i(t-t')\xi^2) \exp(it' (\xi_1^2 + (\xi-\xi_1)^2) \hat f_N(\xi_1) \hat f_N(\xi-\xi_1)\ d\xi_1 dt'
\|_{L^2_\xi(\R)}.$$
Now let $N > 100$ be a large parameter, and set 
$$ \hat f_N := r N 1_{[-10,10]}(|\xi|-N) / 1000.$$
Then $f_N \in B_r$, and $\|f_N\|_{H^{-1}_x(\R)} \to 0$ as $N \to \infty$.  Thus,
from the well-posedness result Theorem~\ref{lwp} and the fact that
Lemma~\ref{lem:qnls-asymp}
holds for
the qNLS, we see that for continuity of the data to solution map to hold for $s < -1$ we must have
\begin{equation}\label{sin}
\sup_{0 \leq t \leq 1} \| \langle \xi \rangle^{s}
\int_0^t \int_\R \exp(-i(t-t')\xi^2) \exp(it' (\xi_1^2 + (\xi-\xi_1)^2) \hat f_N(\xi_1) \hat f_N(\xi-\xi_1)\ d\xi_1 dt'
\|_{L^2_\xi(\R)} \to 0
\end{equation}
as $N \to \infty$.
Now set $t := 1/100N^2$ and localize to the region where $-1 \leq \xi \leq 1$.  One can verify that
$$ \Re( \exp(-i(t-t')\xi^2) \exp(it' (\xi_1^2 + (\xi-\xi_1)^2) )) > 1/2$$
whenever $0 \leq t' \leq t$ and $\xi_1$ lives in the support of $f_{N}$, hence we obtain
$$ 
\| \langle \xi \rangle^{s}
\int_0^t \int_\R \exp(-i(t-t')\xi^2) \exp(it' (\xi_1^2 + (\xi-\xi_1)^2)) \hat f_N(\xi_1) \hat f_N(\xi-\xi_1)\ d\xi_1 dt'
\|_{L^2_\xi(\R)} \geq c r^2$$
for some $c > 0$.  But this contradicts \eqref{sin}. \qed
%
\providecommand{\bysame}{\leavevmode\hbox to3em{\hrulefill}\thinspace}
\providecommand{\MR}{\relax\ifhmode\unskip\space\fi MR }
% \MRhref is called by the amsart/book/proc definition of \MR.
\providecommand{\MRhref}[2]{%
  \href{http://www.ams.org/mathscinet-getitem?mr=#1}{#2}
}
\providecommand{\href}[2]{#2}
\begin{thebibliography}{Far09}

\bibitem[BT06]{Bejenaru-Tao-2006-Sharp-well-posedness-and-ill-posedness}
I.~Bejenaru and T.~Tao, \emph{Sharp well-posedness and ill-posedness results
  for a quadratic non-linear schr{\"o}dinger equation}, J. Funct. Anal.
  \textbf{233} (2006), no.~1, 228--259.

\bibitem[Far09]{Farah:2009uq}
L.~G. Farah, \emph{Local solutions in sobolev spaces with negative indices for
  the ``good'' boussinesq equation}, Comm. Partial Differential Equations
  \textbf{34} (2009), no.~1-3, 52--73.

\bibitem[FS10]{Farah:2010ys}
L.~G. Farah and M.~Scialom, \emph{On the periodic ``good'' boussinesq
  equation}, Proc. Amer. Math. Soc. \textbf{138} (2010), no.~3, 953--964.

\end{thebibliography}
%
%\bibliography{/Users/davidkarapetyan/math/bib-files/references.bib}
%\bibliographystyle{amsalpha-custom}
\end{document}
