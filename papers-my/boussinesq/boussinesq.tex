%
\documentclass[12pt,reqno]{amsart}
\usepackage{amssymb}
\usepackage{mathtools} %upgrade to amsmath--can now do \right and \left on dif lines
\usepackage{cancel}  %for cancelling terms explicity on pdf
\usepackage{yhmath}   %makes fourier transform look nicer, among other things
\usepackage{framed}  %for framing remarks, theorems, etc.
\usepackage[shortalphabetic, initials, msc-links]{amsrefs} %for the bibliography; uses cite pkg
\usepackage{enumerate} %to change enumerate symbols
%\usepackage{showkeys}  %shows source equation labels on the pdf
\usepackage[margin=3cm]{geometry}  %page layout
%\usepackage[pdftex]{graphicx} %for importing pictures into latex--pdf compilation
\setcounter{secnumdepth}{1} %number only sections, not subsections
\hypersetup{colorlinks=true,
linkcolor=blue,
citecolor=blue,
urlcolor=blue,
}
\synctex=1
\numberwithin{equation}{section}  %eliminate need for keeping track of counters
\numberwithin{figure}{section}
\setlength{\parindent}{0in} %no indentation of paragraphs after section title
\renewcommand{\baselinestretch}{1.1} %increases vert spacing of text
%
%
\newcommand{\ds}{\displaystyle}
\newcommand{\ts}{\textstyle}
\newcommand{\nin}{\noindent}
\newcommand{\rr}{\mathbb{R}}
\newcommand{\nn}{\mathbb{N}}
\newcommand{\zz}{\mathbb{Z}}
\newcommand{\cc}{\mathbb{C}}
\newcommand{\ci}{\mathbb{T}}
\newcommand{\zzdot}{\dot{\zz}}
\newcommand{\wh}{\widehat}
\newcommand{\p}{\partial}
\newcommand{\ee}{\varepsilon}
\newcommand{\vp}{\varphi}
%
%
%\theoremstyle{plain}  
%\newtheorem{theorem}{Theorem}
%\newtheorem{proposition}{Proposition}
%\newtheorem{lemma}{Lemma}
%\newtheorem{corollary}{Corollary}
%\newtheorem{claim}{Claim}
%\newtheorem{conjecture}[subsection]{conjecture}
%%
%\theoremstyle{definition}
%\newtheorem{definition}{Definition}
%%
%\theoremstyle{remark}
%\newtheorem{remark}{Remark}
%
%
\newtheorem{theorem}{Theorem}[section]
\newtheorem{lemma}[theorem]{Lemma}
\newtheorem{corollary}[theorem]{Corollary}
\newtheorem{claim}[theorem]{Claim}
\newtheorem{prop}[theorem]{Proposition}
\newtheorem{proposition}[theorem]{Proposition}
\newtheorem{no}[theorem]{Notation}
\newtheorem{definition}[theorem]{Definition}
\newtheorem{remark}[theorem]{Remark}
\newtheorem{examp}{Example}[section]
\newtheorem {exercise}[theorem] {Exercise}
%
\def\makeautorefname#1#2{\expandafter\def\csname#1autorefname\endcsname{#2}}
\makeautorefname{equation}{Equation}
\makeautorefname{footnote}{footnote}
\makeautorefname{item}{item}
\makeautorefname{figure}{Figure}
\makeautorefname{table}{Table}
\makeautorefname{part}{Part}
\makeautorefname{appendix}{Appendix}
\makeautorefname{chapter}{Chapter}
\makeautorefname{section}{Section}
\makeautorefname{subsection}{Section}
\makeautorefname{subsubsection}{Section}
\makeautorefname{paragraph}{Paragraph}
\makeautorefname{subparagraph}{Paragraph}
\makeautorefname{theorem}{Theorem}
\makeautorefname{theo}{Theorem}
\makeautorefname{thm}{Theorem}
\makeautorefname{addendum}{Addendum}
\makeautorefname{add}{Addendum}
\makeautorefname{maintheorem}{Main theorem}
\makeautorefname{corollary}{Corollary}
\makeautorefname{lemma}{Lemma}
\makeautorefname{sublemma}{Sublemma}
\makeautorefname{proposition}{Proposition}
\makeautorefname{property}{Property}
\makeautorefname{scholium}{Scholium}
\makeautorefname{step}{Step}
\makeautorefname{conjecture}{Conjecture}
\makeautorefname{question}{Question}
\makeautorefname{definition}{Definition}
\makeautorefname{notation}{Notation}
\makeautorefname{remark}{Remark}
\makeautorefname{remarks}{Remarks}
\makeautorefname{example}{Example}
\makeautorefname{algorithm}{Algorithm}
\makeautorefname{axiom}{Axiom}
\makeautorefname{case}{Case}
\makeautorefname{claim}{Claim}
\makeautorefname{assumption}{Assumption}
\makeautorefname{conclusion}{Conclusion}
\makeautorefname{condition}{Condition}
\makeautorefname{construction}{Construction}
\makeautorefname{criterion}{Criterion}
\makeautorefname{exercise}{Exercise}
\makeautorefname{problem}{Problem}
\makeautorefname{solution}{Solution}
\makeautorefname{summary}{Summary}
\makeautorefname{operation}{Operation}
\makeautorefname{observation}{Observation}
\makeautorefname{convention}{Convention}
\makeautorefname{warning}{Warning}
\makeautorefname{note}{Note}
\makeautorefname{fact}{Fact}
%
%


\newcommand{\uol}{u^\omega_\lambda}
\newcommand{\lbar}{\bar{l}}
\renewcommand{\l}{\lambda}
\newcommand{\R}{\mathbb R}
\newcommand{\RR}{\mathcal R}
\newcommand{\al}{\alpha}
\newcommand{\ve}{q}
\newcommand{\tg}{{tan}}
\newcommand{\m}{q}
\newcommand{\N}{N}
\newcommand{\ta}{{\tilde{a}}}
\newcommand{\tb}{{\tilde{b}}}
\newcommand{\tc}{{\tilde{c}}}
\newcommand{\tS}{{\tilde S}}
\newcommand{\tP}{{\tilde P}}
\newcommand{\tu}{{\tilde{u}}}
\newcommand{\tw}{{\tilde{w}}}
\newcommand{\tA}{{\tilde{A}}}
\newcommand{\tX}{{\tilde{X}}}
\newcommand{\tphi}{{\tilde{\phi}}}


\begin{document}
\title{$1$-D ``good" Boussinesq equation}

\author{Dan-Andrei Geba, Alexandrou Himonas, and David Karapetyan}

\address{Department of Mathematics, University of Rochester, Rochester, NY 14627}
\address{Department of Mathematics, University of Notre Dame, Notre Dame, IN 46556}
\address{Department of Mathematics, University of Notre Dame, Notre Dame, IN 46556}
\date{}

%\begin{abstract}
%\end{abstract}

\subjclass[2000]{35B30, 35Q55, 35Q72}
\keywords{local well-posedness; ill-posedness.}

\maketitle

({\bf  Yes, there is scaling, see page 2.})

Our object of investigation is the initial value problem for the periodic/non-periodic $1$-D ``good" Boussinesq equation, i.e.,
\begin{equation}
  \aligned
  &u_{tt}-u_{xx}+u_{xxxx}+(u^2)_{xx}\,=\,0, \quad x\in \mathbb{T}\ \text{or} \ \mathbb{R}, \quad t>0,\\
&u(0,x)\,=\,u_0(x),\qquad u_t(0,x)\,=\,u_1(x).\endaligned
\label{main}
\end{equation}

Due to the fact that the leading terms in the linear operator above are $u_{tt}$ and $u_{xxxx}$, morally speaking, one derivative in time is like two derivatives in space. This is why the Sobolev regularity scale for the initial data should be as follows:
\[
u_0\in H^s(\mathbb{T}\ \text{or} \ \mathbb{R}), \qquad u_1\in H^{s-2}(\mathbb{T}\ \text{or} \ \mathbb{R})
\]
Results for these problems are usually formulated with $u_0=\phi \in H^s$ and $u_1=\psi_x$, $\psi\in H^{s-1}$.

Current state of the art in terms of local well-posedness/ill-posedness for the two problems is:
\begin{itemize}
  \item LWP for both problems when $s>-\frac 14$(Farah '09, Farah-Scialom '10), with iteration done in
    the norm
    \[
    \|F\|_{X^{s,b}}\,=\,\|<|\tau|-\sqrt{\xi^2+\xi^4}>^b\,<\xi>^s \tilde{F}\|_{L^2_{\tau,\xi}};
    \]

  \item main IP result is for the non-periodic problem when $s<-2$, as the solution map 
    \[
    S: H^s\times H^{s-1} \to C([0,T]; H^s), \quad
    S(\phi,\psi)\,=\,u
    \]
    is not $C^2$ at zero (Farah '09);

  \item also, for the non-periodic problem, one can not find a space in which to run a contraction argument based on treating the nonlinearity as bilinear for $s<-2$ (see Theorem 1.4 in Farah '09);

  \item finally, and really puzzling\footnote{for other dispersive equations (e.g., KdV, Schrodinger), there is usually a gap of $\frac 14$ between regularities for the two problems}, the crucial bilinear estimate (equation (5) in both papers) fails basically at the same threshold for both problems: $s\leq -\frac 14$ (non-periodic), $s<-\frac{1}{4}$ (periodic).
\end{itemize}

The equation does not have an associated scaling, however one can do a formal scaling analysis by ignoring one of the two linear terms containing spatial derivatives:
\begin{itemize}
  \item for 
    \[
    u_{tt}+u_{xxxx}+(u^2)_{xx}\,=\,0,
    \]
    one has 
    \[
    u_{\lambda}(t,x)\,=\,\frac{1}{\lambda^2}u\left(\frac{t}{\lambda^2}, \frac{x}{\lambda}\right),
    \]
    which leads to $s_c=-\frac 32$;

  \item for 
    \[
    u_{tt}-u_{xx}+(u^2)_{xx}\,=\,0,
    \]
    one has 
    \[
    u_{\lambda}(t,x)\,=\,u\left(\frac{t}{\lambda}, \frac{x}{\lambda}\right),
    \]
    which leads to $s_c=\frac 12$.
\end{itemize}

This might suggest that the current results are not optimal.


%

%
%%%%%%%%%%%%%%%%%%%%%%%%%%%%%%%%%%%%%%%%%%%%%%%%%%%%%
%
%
%             Fourth order Modified Boussinesq  equation
%
%
%%%%%%%%%%%%%%%%%%%%%%%%%%%%%%%%%%%%%%%%%%%%%%%%%%%%%


%
\section{Fourth order Modified Boussinesq  equation}
\label{sec:intro-2}
We consider the initial value problem (ivp) for a modified Boussinesq
equation (MB) equation 
\begin{gather}
  u_{tt}   + u_{xxxx} + (u^2)_{xx} = 0,
  \label{eqn:mb-2}
  \\
  u(x,0) = u_{0}(x), \quad u_{0} \in H^{s}
  \label{eqn:mb-init-data-2}
\end{gather}
and conjecture the following.
%
%
%%%%%%%%%%%%%%%%%%%%%%%%%%%%%%%%%%%%%%%%%%%%%%%%%%%%%
%
%
%                Main Theorem
%
%
%%%%%%%%%%%%%%%%%%%%%%%%%%%%%%%%%%%%%%%%%%%%%%%%%%%%%
%
%
\begin{theorem}
  If $s>s_c$ then 
  then the  i.v.p, for the fourth order modified Boussinesq  equation  is well-posed
  is well-posed 
  \begin{itemize}
    \item In $H^s(\rr)$ if $s > s_c$
    \item In $H^{s}(\ci)$ if $s > s_c + 1/4$,
  \end{itemize}
  and the data-to-solution map is  ?? (continuous?, Lip?, smooth, analytic?). 
  \label{thm:wp-2}
\end{theorem}
%

%
%
\begin{framed}
\begin{remark}
  This equation admits no scaling. To see this, set $u_{a,b}(x,t) =
  u(ax,bt)$ and substitute into \eqref{eqn:mb-2} to obtain
  %
  %
  \begin{equation*}
    \begin{split}
      b^{2}u_{tt} + a^{4}u_{xxxx} + a^{2}(u^{2})_{xx} = 0.
    \end{split}
  \end{equation*}
  %
  %
  Setting $b^{2} = a^{4} = a^{2}$ and seeking a non-trivial solution, we
  find $a=b=1$. 
  \label{rem:scaling}
\end{remark}
\end{framed}
%



\vskip0.1in
\hrule
\vskip0.1in


%%%%%%%%%%%%%%%%%%%%%%%%%%%%
%
%
%           Scaling for B4
%
%
%%%%%%%%%%%%%%%%%%%%%%%%%%%
{ \bf Scaling  for $B_4$.}
%
%
Let $u(x, t)$ be a solution to the B4 equation, that is
%
$$
B_4(u)=
 \partial_t^2u + \partial^4_x u + \partial_x^2(u^2)  = 0
$$
%
We would like to find the constants
$a, b, c$ such that
\[
u_\lambda (x, t) = \lambda^a u(\lambda^b x, \lambda^c t)
\]
is also a solution to B4.  Since 
$$
B_4(u_\lambda)=
\lambda^{a+2c} \partial_t^2u 
+
 \lambda^{a+4b} \partial^4_x u 
 +
  \lambda^{2a+2b}
  \partial_x^2(u^2),  
$$
we see that $u_\lambda$ is a B4 solution only if
$$
a+2c=a+4b=2a+2b,
$$
or
$
c= 2b =a.
$
  Thus
\[
u_\lambda (x, t) = \lambda^{2b} u(\lambda^{b}x,  \lambda^{2b} t).
\]
%
%
Therefore, replacing  $ \lambda^b$ with  $ \lambda$ gives the following scaling:
%
\begin{equation}
\label{DP-scal}
\boxed{
u(x, t) \text{ solution to }  B_4 
 \Longrightarrow 
u_\lambda (x, t) = \lambda^2 u(\lambda x, \lambda^2 t)  \text { is also a solution to }  B_4. 
}
\end{equation}


\hrule
\vskip0.1in
%
Since the scaling conserves data in $\dot{H}^{-3/2}$......
It seems that this equation is ``like KdV".
So one may expect KdV type theorems...
That is, $s_c=-3/4$ on the line and $s_c=-1/2$ on the circle,
if one uses bilinear estimates.
But, Kappeler and collaborators went all the way to $-1$ for KdV.
However KdV is integrable. Is this equation integrable?
Also, people conjecture that the critical index for KdV well-posedness 
in some appropriate sense should be the scaling index which is  $-3/2$.



%
%
%%%%%%%%%%%%%%%%%%%%%%%%%%%%%%%%%%%%%%%%%%%%%%%%%%%%%
%
%
%                The Periodic Case
%
%
%%%%%%%%%%%%%%%%%%%%%%%%%%%%%%%%%%%%%%%%%%%%%%%%%%%%%
%
%
\section{The Periodic Case} 
\label{sec:periodic-case}
We will first rewrite the MB ivp
\eqref{eqn:mb-2}-\eqref{eqn:mb-init-data-2} in integral form. Consider
the linear MB
\begin{gather}
  u_{tt} + u_{xxxx} = 0,
  \label{lin-mb}
  \\
  u(x, 0)=u_{0}(x), \quad u_{t}(x,0) = u_{1}(x).
  \label{lin-mb-init-data-1}
\end{gather}
Taking the spatial Fourier transform yields the ivp
\begin{gather*}
  \wh{u_{tt}^{x}} + n^{4} \wh{u^{x}} = 0,
  \\
  \wh{u^{x}}(\cdot, 0) = \wh{u_{0}}(n), \quad
  \wh{u_{t}^{x}}(\cdot, 0) = \wh{u_{1}}(n)
\end{gather*}
which admits the unique solution
%
%
\begin{equation*}
  \begin{split}
    \wh{u^{x}}(n, t) = \wh{u_{0}}(n) \frac{e^{in^{2}t} + e^{in^{2}t}}{2} + 
    \wh{u_{1}}(n) \frac{e^{in^{2}t} - e^{-in^{2}t}}{2i n^{2}}.
  \end{split}
\end{equation*}
%
%
%
%
\begin{framed}
\begin{remark}
  Note that $$g(n) \doteq \frac{e^{in^{2}t} - e^{-in^{2}t}}{2i n^{2}}$$ has a removable
  singularity at $n=0$. Since $$\lim_{n \to 0} g(n) = 0$$ we may analytically
  extend $g(n)$ to the entire complex plane. 
\label{rem:analytic-extension}
\end{remark}
\end{framed}
%
%
%
Therefore,
%
%
\begin{equation*}
  \begin{split}
    u(x,t) = R_t u_{0} + S_{t}u_{1}
  \end{split}
\end{equation*}
%
is the unique solution to the ivp
\eqref{lin-mb}-\eqref{lin-mb-init-data-1}, where $R_{t}$ and $S_{t}$ are linear operators defined via the relation
%
%
\begin{gather*}
  \wh{R_{t}\vp} = \wh{\vp}(n) \frac{e^{in^{2}t} + e^{-in^{2}t}}{2} , \quad 
  \wh{S_{t}\vp} = \wh{\vp}(n) \frac{e^{in^{2}t} - e^{-in^{2}t}}{2i n^{2}}.
\end{gather*}

%
By Duhamel's principle, it
follows that the MB ivp \eqref{eqn:mb-2}-\eqref{eqn:mb-init-data-2} can
be written in the integral form
%
%
\begin{equation}
  \begin{split}
    u(x,t) = R_{t}u_{0} + S_{t}u_{1} + \int_{0}^{t} S_{t-t'}
    (u^{2})_{xx} dt'.
  \end{split}
  \label{eqn:integral-form}
\end{equation}
%
%
Let $\psi(t)$ be a cutoff function symmetric about the 
origin such that $\psi(t) = 1$ for $|t| \le T$ and $\text{supp} \, \psi 
= [-2T, 2T ]$. Multiplying the right hand side of expression
$\eqref{eqn:integral-form}$ by $\psi(t)$, we obtain
%
%
\begin{equation}
  \begin{split}
    \psi(t) u(x,t)
    & = \psi(t) R_{t} u_{0} + \psi(t) S_{t}u_{1} +
    \psi(t) \int_{0}^{t} S_{t-t'}
    (u^{2})_{xx} dt'
    \\
    & \doteq Tu
  \end{split}
  \label{localized-int-eqn}
\end{equation}
where $T=T_{u_0, u_1}$.We now introduce the following spaces. 
%
%
\begin{definition}
  Let $\mathcal{Y}$ be the space of functions $F(\cdot)$ such that
  \begin{enumerate}[(i)]
   \item{$F: \ci \times \rr \to \cc$ }.
   \item{ $F(x, \cdot) \in S(\rr)$ for each $x \in \ci$}.
   \item{ $F(\cdot, t) \in C^{\infty}(\ci)$for each $t \in \rr$}.
  \end{enumerate}
  For $s, b \in \rr$, $X_{s,b}$ denotes the completion of $\mathcal{Y}$ with
  respect to the norm
  %
  %
  \begin{equation}
  \begin{split}
    \|F\|_{X_{s,b}} = \left( \sum_{n \in \zz} (1 + n^{2})^{s}\| \int_{\rr}
    1 + | | \tau | - n^{2} |^{2b} \wh{F}(n, \tau) d \tau\right)^{1/2}
  \end{split}
  \label{eqn:bous-norm}
  \end{equation}
  %
  %
  %
  %
\end{definition}
%
%
We will 
show that for initial data $\vp \in {H}^s(\ci)$, $T$ is a contraction on $B_M 
\subset {X}_{s,b}$, where $B_M$ is the ball centered at the origin of radius $M = 
M_{\vp}> 0$, by estimating the $X_{s,b}$
norm of \eqref{localized-int-eqn}. The Picard fixed point theorem will
then yield a unique solution to
\eqref{localized-int-eqn}. An application of an embedding lemma
will then imply the existence of a unique, local
solution $u \in C([-T, T], H^s(\ci))$ to the MB ivp. Local Lipschitz continuity of the flow map will follow from estimates used to establish the contraction mapping. %
%
%
%
%
%%%%%%%%%%%%%%%%%%%%%%%%%%%%%%%%%%%%%%%%%%%%%%%%%%%%%
%
%
%                Proof of Main Theorem
%
%
%%%%%%%%%%%%%%%%%%%%%%%%%%%%%%%%%%%%%%%%%%%%%%%%%%%%%
%
%
\section{Proof of Main Theorem} 
\label{sec:pf-main-thm}
%
%

\subsection{Estimate for $\psi(t) R_{t})u_{0}$.} 
We have
%
%
\begin{equation*}
  \begin{split}
    \wh{\psi(t)R_{t}u_{0}}^{x}(n, t)
    & = \psi(t) \wh{u_{0}}(n) \frac{e^{in^2 t} + e^{-in^{2}t}}{2}
    \\
    & = \frac{\psi(t) \wh{u_{0}}(n)e^{in^{2}t}}{2} + \frac{\psi(t)
    \wh{u_{0}}(n)e^{-in^{2}t}}{2}  
  \end{split}
\end{equation*}
%
%
and
%
%
\begin{equation*}
  \begin{split}
    \wh{\psi(t) R_{t}u_{0}}(n, \tau) = \frac{\wh{\psi}(\tau -
    n^{2})\wh{u_{0}}(n)}{2} + \frac{\wh{\psi}(\tau - n^{2})\wh{u_{0}}(n)}{2}.
  \end{split}
\end{equation*}
%
%
Hence,
%
%
\begin{equation}
  \begin{split}
    \| \psi(t) R_{t}u_{0} \|_{X_{s,b}}^{2} 
    & = \sum_{n \in \zz}(1 + n^{2})^{s} \int_{\rr}\left( 1 + | | \tau
    |-n^{2} | \right)^{2b} | \frac{\wh{\psi}(\tau - n^{2})\wh{u_{0}(n)}}{2} +
    \frac{\wh{\psi}(\tau + n^{2})\wh{u_{0}}(n)}{2} |^{2} d \tau
    \\
    & \simeq \sum_{n \in \zz}(1 + n^{2})^{s} | \wh{u_{0}(n)} |^{2} \int_{\rr}
    \left( 1 + | | \tau | - n^{2} | \right)^{2b} | \wh{\psi}(\tau - n^{2}) +
    \wh{\psi}(\tau + n^{2}) |^{2} d \tau
    \\
    & \le \sum_{n \in \zz} \left( 1 + n^{2} \right)^{s} | \wh{u_{0}}(n)
    |^{2} \int_{\rr} | \wh{\psi}(\tau - n^{2}) |^{2}\left( 1 + | | \tau | -
    n^{2} | \right)^{2b} d \tau
    \\
    & + \sum_{n \in \zz} \left( 1 + n^{2} \right)^{s} | \wh{u_{0}}(n)
    |^{2} \int_{\rr} | 2\wh{\psi}(\tau - n^{2})\wh{\psi}(\tau + n^{2}) |^{2}\left( 1 + | | \tau | -
    n^{2} | \right)^{2b} d \tau
    \\
    & + \sum_{n \in \zz} \left( 1 + n^{2} \right)^{s} | \wh{u_{0}}(n)
    |^{2} \int_{\rr} | \wh{\psi}(\tau + n^{2}) |^{2}\left( 1 + | | \tau | -
    n^{2} | \right)^{2b} d \tau.
  \end{split}
  \label{u-0-norm-comp}
\end{equation}
%
Noting that
\begin{equation}
  \begin{split}
    | | \tau | - n^{2} | \le \min\left\{ | \tau - n^{2} |, | \tau + n^{2} | \right\}
  \end{split}
  \label{eqn:norm-key-ineq}
\end{equation}
%
%
and that $\wh{\psi}$ is Schwartz, we bound the first term of
\eqref{u-0-norm-comp}  
%
%
\begin{equation*}
  \begin{split}
    & \sum_{n \in \zz} \left( 1 + n^{2} \right)^{s} | \wh{u_{0}}(n)
    |^{2} \int_{\rr} | \wh{\psi}(\tau - n^{2}) |^{2}\left( 1 + | | \tau | -
    n^{2} | \right)^{2b} d \tau
    \\
    & \le  \sum_{n \in \zz} \left( 1 + n^{2} \right)^{s} | \wh{u_{0}}(n)
    |^{2} \int_{\rr} | \wh{\psi}(\tau - n^{2}) |^{2}\left( 1 +  | \tau  -
    n^{2} | \right)^{2b} d \tau
    \\
    & = \sum_{n \in \zz} \left( 1 + n^{2} \right)^{s} | \wh{u_{0}}(n)
    |^{2} \int_{\rr} | \wh{\psi}(\tau') |^{2}\left( 1 +  | \tau'| \right)^{2b} d \tau
    \\
    & = c_{\psi, b} \sum_{n \in \zz} \left( 1 + n^{2} \right)^{s} | \wh{u_{0}}(n)
    |^{2} 
    \\
    & = c_{\psi, b} \| u_{0} \|_{H^{s}}^{2}
  \end{split}
\end{equation*}
%
%
where $c_{\psi, b}$ is a constant depending only on $\psi$ and $b$. The
remaining terms of \eqref{u-0-norm-comp} are bounded in similar fashion.  
Therefore, 
$\|\psi(t) R_{t} u_{0}\|_{X_{s,b}}^{2} = c_{\psi, b}
\|u_{0}\|_{H^s}^2$ and
taking square roots of both sides gives
%
%
\begin{equation}
  \begin{split}
    \|\psi(t) R_{t} u_{0}\|_{X_{s,b}} = c_{\psi, b}
    \|u_{0}\|_{H^s}.
  \end{split}
  \label{eqn:u-0-fin-est}
\end{equation}
%
%

\subsection{Estimate for $\psi(t) S_{t}u_{1}$.}
We have
%
%
\begin{equation*}
  \begin{split}
    \wh{\psi(t)S_{t}u_{1}}^{x}(n, t)
    & = \psi(t) \wh{u_{1}}(n) \frac{e^{in^2 t} - e^{-in^{2}t}}{2i n^{2}}
    \\
    & = \frac{\psi(t) \wh{u_{1}}(n)e^{in^{2}t}}{2i n^{2}} - \frac{\psi(t)
    \wh{u_{1}}(n)e^{-in^{2}t}}{2i n^{2}}  
  \end{split}
\end{equation*}
%
%
and
%
%
\begin{equation*}
  \begin{split}
    \wh{\psi(t) S_{t}u_{1}}(n, \tau) = \frac{\wh{\psi}(\tau -
    n^{2})\wh{u_{1}}(n)}{2i n^{2}} + \frac{\wh{\psi}(\tau - n^{2})\wh{u_{1}}(n)}{2i
    n^{2}}.
  \end{split}
\end{equation*}
%
%
Hence,
%
%
\begin{equation*}
  \begin{split}
    \| \psi(t) S_{t}u_{1} \|_{X_{s,b}}^{2} 
    & = \sum_{n \in \zz}(1 + n^{2})^{s} \int_{\rr}\left( 1 + | | \tau
    |-n^{2} | \right)^{2b} | \frac{\wh{\psi}(\tau - n^{2})\wh{u_{1}(n)}}{2i
    n^{2}} -
    \frac{\wh{\psi}(\tau + n^{2})\wh{u_{1}}(n)}{2i n^{2}} |^{2} d \tau.
    \end{split}
\end{equation*}
%
Applying \ref{rem:analytic-extension}, this simplifies to
%
%
\begin{equation}
\begin{split}
  & \sum_{n \in \dot{\dot{\zz}}}(1 + n^{2})^{s} \int_{\rr}\left( 1 + | | \tau
    |-n^{2} | \right)^{2b} | \frac{\wh{\psi}(\tau - n^{2})\wh{u_{1}(n)}}{2i
    n^{2}} -
    \frac{\wh{\psi}(\tau + n^{2})\wh{u_{1}}(n)}{2i n^{2}} |^{2} d \tau
    \\
    & \lesssim \sum_{n \in \dot{\zz}}(1 + n^{2})^{s-1} | \wh{u_{1}(n)} |^{2} \int_{\rr}
    \left( 1 + | | \tau | - n^{2} | \right)^{2b} | \wh{\psi}(\tau - n^{2}) -
    \wh{\psi}(\tau + n^{2}) |^{2} d \tau
    \\
    & \le \sum_{n \in \dot{\zz}} \left( 1 + n^{2} \right)^{s-1} | \wh{u_{1}}(n)
    |^{2} \int_{\rr} | \wh{\psi}(\tau - n^{2}) |^{2}\left( 1 + | | \tau | -
    n^{2} | \right)^{2b} d \tau
    \\
    & + \sum_{n \in \dot{\zz}} \left( 1 + n^{2} \right)^{s-1} | \wh{u_{1}}(n)
    |^{2} \int_{\rr} | 2\wh{\psi}(\tau - n^{2})\wh{\psi}(\tau + n^{2}) |^{2}\left( 1 + | | \tau | -
    n^{2} | \right)^{2b} d \tau
    \\
    & + \sum_{n \in \dot{\zz}} \left( 1 + n^{2} \right)^{s-1} | \wh{u_{1}}(n)
    |^{2} \int_{\rr} | \wh{\psi}(\tau + n^{2}) |^{2}\left( 1 + | | \tau | -
    n^{2} | \right)^{2b} d \tau.
\end{split}
\label{u-1-norm-comp}
\end{equation}
%
%
%
Recalling \eqref{eqn:norm-key-ineq} 
and that $\wh{\psi}$ is Schwartz, we bound the first term of
\eqref{u-1-norm-comp}  
%
%
\begin{equation*}
  \begin{split}
    & \sum_{n \in \dot{\zz}} \left( 1 + n^{2} \right)^{s-1} | \wh{u_{1}}(n)
    |^{2} \int_{\rr} | \wh{\psi}(\tau - n^{2}) |^{2}\left( 1 + | | \tau | -
    n^{2} | \right)^{2b} d \tau
    \\
    & \le  \sum_{n \in \dot{\zz}} \left( 1 + n^{2} \right)^{s-1} | \wh{u_{1}}(n)
    |^{2} \int_{\rr} | \wh{\psi}(\tau - n^{2}) |^{2}\left( 1 +  | \tau  -
    n^{2} | \right)^{2b} d \tau
    \\
    & = \sum_{n \in \dot{\zz}} \left( 1 + n^{2} \right)^{s-1} | \wh{u_{1}}(n)
    |^{2} \int_{\rr} | \wh{\psi}(\tau') |^{2}\left( 1 +  | \tau'| \right)^{2b} d \tau
    \\
    & = c_{\psi, b} \sum_{n \in \dot{\zz}} \left( 1 + n^{2} \right)^{s-1} | \wh{u_{1}}(n)
    |^{2} 
    \\
    & \le c_{\psi, b} \| u_{1} \|_{H^{s-1}}^{2}
  \end{split}
\end{equation*}
%
%
where $c_{\psi, b}$ is a constant depending only on $\psi$ and $b$. The
remaining terms of \eqref{u-1-norm-comp} are bounded in similar fashion.  
Therefore, 
$\|\psi(t) S_{t} u_{1}\|_{X_{s,b}}^{2} = c_{\psi, b}
\|u_{1}\|_{H^{s-1}}^2$ and
taking square roots of both sides gives
%
%
\begin{equation}
  \begin{split}
    \|\psi(t) S_{t} u_{1}\|_{X_{s,b}} = c_{\psi, b}
    \|u_{1}\|_{H^{s-1}}.
  \end{split}
  \label{eqn:u-1-fin-est}
\end{equation}

\subsection{Estimate for $\psi(t) \int_{0}^{t} S_{t-t'} (u^{2})_{xx} dt'$.}
Recall that
%
%
\begin{equation*}
  \begin{split}
    \psi(t) \int_{0}^{t} S_{t-t'} (u^{2})_{xx} dt'
    & = \psi(t) \int_{0}^{t}\widetilde{u^{2}}(n, t') \sin[n^{2}(t -t')] dt'
    \\
    & = \frac{\psi(t)}{2i} \int_{0}^{t}\widetilde{u^{2}}(n, t') \left[
    e^{in^{2}(t-t')} - e^{-in^{2}(t-t')}.
    \right] dt'
  \end{split}
\end{equation*}
%
%
This is a \emph{global} relation in $t$. Therefore, applying the inverse Fourier
transform in time gives
\begin{equation*}
  \begin{split}
    & \psi(t) \int_{0}^{t} S_{t-t'} (u^{2})_{xx} dt'
    \\
    & = \frac{\psi(t)}{2i} \int_{\rr} \int_{0}^{t} e^{it'
    \tau}\wh{u^{2}}(n, \tau) \left[
    e^{in^{2}(t-t')} - e^{-in^{2}(t-t')}
    \right] dt'
    \\
    & = \frac{\psi(t)e^{in^{2}t}}{2i} \int_{0}^{t} \int_{\rr} \wh{u^{2}}(n,
    \tau)e^{it'(\tau - n^{2})}d\tau dt' 
    - \frac{\psi(t)e^{-in^{2}t}}{2i} \int_{0}^{t} \int_{\rr} \wh{u^{2}}(n,
    \tau)e^{it'(\tau + n^{2})}d\tau dt' 
  \end{split}
\end{equation*}
%
Applying Fubini and integrating then gives
%
%
%
%
\begin{equation*}
  \begin{split}
    -\frac{\psi(t)e^{in^{2}t}}{2} \int_{\rr}  \frac{\left( e^{it(\tau -
    n^{2})}-1 \right)}{\tau - n^{2}} \wh{u^{2}}(n, \tau)d \tau +
    \frac{\psi(t)e^{-in^{2}t}}{2} \int_{\rr}  \frac{\left( e^{it(\tau +
    n^{2})}-1 \right)}{\tau + n^{2}} \wh{u^{2}}(n, \tau)d \tau.
  \end{split}
\end{equation*}
%
%
We shall only estimate the first term. The estimates for the second term will be
analogous. 
We localize near the singular curve $\tau =  n^2$ 
by multiplying the
integrant of the first term  by $1 + \psi(\tau -
n^2) - \psi(\tau -
n^2) $, which gives 
%
%
\begin{equation*}
  \begin{split}
    & -\frac{\psi(t)e^{in^{2}t}}{2} \int_{\rr}  \frac{\left( e^{it(\tau -
    n^{2})}-1 \right)}{\tau - n^{2}} \wh{u^{2}}(n, \tau)d \tau 
    \\
    & = -\frac{\psi(t)e^{in^{2}t}}{2} \int_{\rr}  \frac{\left( e^{it(\tau -
    n^{2})}-1 \right)}{\tau - n^{2}} \wh{u^{2}}(n, \tau)\left[ 1 - \psi(t) \right] d \tau 
    \\
    & -\frac{\psi(t)e^{in^{2}t}}{2} \int_{\rr}  \frac{\left( e^{it(\tau -
    n^{2})}-1 \right)}{\tau - n^{2}} \wh{u^{2}}(n, \tau)\psi(t) d \tau 
  \end{split}
\end{equation*}
%
%






%%%%%%%%%%%%%%%%%%%%%%%%%%%%%%%%%%%%%%%%%%%%%%%%%%%%%
%
%
%		Estimation of Integral Equality Part 1		
%
%
%%%%%%%%%%%%%%%%%%%%%%%%%%%%%%%%%%%%%%%%%%%%%%%%%%%%%
%
%
%
%



%
%%%%%%%%%%%%%%%%%%%%%%%%%%%%%%%%%%%%%%%%%%%%%%%%%%%%%
%
%
%             Secod order Modified Boussinesq  equation
%
%
%%%%%%%%%%%%%%%%%%%%%%%%%%%%%%%%%%%%%%%%%%%%%%%%%%%%%


%
\section{Second order Modified Boussinesq  equation}
\label{sec:intro}
We consider the initial value problem (ivp) for a modified Boussinesq
equation (MB) equation 
\begin{gather}
  u_{tt} - u_{xx} + (u^2)_{xx} = 0,
  \label{eqn:mb}
  \\
  u(x,0) = u_{0}(x), \quad u_{0} \in H^{s}
  \label{eqn:mb-init-data}
\end{gather}
and conjecture the following.
%
%
%%%%%%%%%%%%%%%%%%%%%%%%%%%%%%%%%%%%%%%%%%%%%%%%%%%%%
%
%
%                Main Theorem
%
%
%%%%%%%%%%%%%%%%%%%%%%%%%%%%%%%%%%%%%%%%%%%%%%%%%%%%%
%
%
\begin{theorem}
  If $s>s_c$ then then the  i.v.p, for the fourth order modified
  Boussinesq  equation is well-posed
  \begin{itemize}
    \item In $H^s(\rr)$ if $s > s_c$
    \item In $H^{s}(\ci)$ if $s > s_c + 1/4$,
  \end{itemize}
  and the data-to-solution map is  ?? (continuous?, Lip?, smooth, analytic?). 
  \label{thm:wp}
\end{theorem}
%

%
%
\begin{framed}
\begin{remark}
  This equation admits some scaling. To see this, set $u_{a,b}(x,t) =
  u(ax,bt)$ and substitute into \eqref{eqn:mb-2} to obtain
  %
  %
  \begin{equation*}
    \begin{split}
      b^{2}u_{tt} - a^{2}u_{xx} + a^{2}(u^{2})_{xx} = 0.
    \end{split}
  \end{equation*}
  %
  %
  Setting $b^{2} = a^{2} = a^{2}$, we
  find $a=b$. Therefore, we have the scaling relation
  $u_{\lambda}(x,t) = u(\lambda x, \lambda t)$. 
  \label{rem:}
\end{remark}
\end{framed}
%
%
Since the scaling conserves data in $\dot{H}^{-3/2}$......
It seems that this equation is ``like KdV".
So one may expect KdV type theorems...
That is, $s_c=-3/4$ on the line and $s_c=-1/2$ on the circle,
if one uses bilinear estimates.
But, Kappeler and collaborators went all the way to $-1$ for KdV.
However KdV is integrable. Is this equation integrable?
Also, people conjecture that the critical index for KdV well-posedness 
in some appropriate sense should be the scaling index which is  $-3/2$.




%\nocite{*}
%\bibliography{/Users/davidkarapetyan/math/bib-files/references.bib}


\end{document}
