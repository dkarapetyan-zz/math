\documentclass{amsart}
%\usepackage{showkeys}
\usepackage{amssymb}
\usepackage{amsmath}
\usepackage{amsfonts}
\newtheorem{theorem}{Theorem}[section]
\newtheorem{lemma}[theorem]{Lemma}
\newtheorem{corollary}[theorem]{Corollary}
\newtheorem{claim}[theorem]{Claim}
\newtheorem{prop}[theorem]{Proposition}
\newtheorem{no}[theorem]{Notation}
\newtheorem{definition}[theorem]{Definition}
\newtheorem{remark}[theorem]{Remark}
\newtheorem{examp}{Example}[section]
\newtheorem {exercise}[theorem] {Exercise}

\newcommand{\uol}{u^\omega_\lambda}
\newcommand{\lbar}{\bar{l}}
\renewcommand{\l}{\lambda}
\newcommand{\R}{\mathbb R}
\newcommand{\RR}{\mathcal R}
\newcommand{\p}{\partial}
\newcommand{\al}{\alpha}
\newcommand{\ve}{q}
\newcommand{\tg}{{tan}}
\newcommand{\m}{q}
\newcommand{\N}{N}
\newcommand{\ta}{{\tilde{a}}}
\newcommand{\tb}{{\tilde{b}}}
\newcommand{\tc}{{\tilde{c}}}
\newcommand{\tS}{{\tilde S}}
\newcommand{\tP}{{\tilde P}}
\newcommand{\tu}{{\tilde{u}}}
\newcommand{\tw}{{\tilde{w}}}
\newcommand{\tA}{{\tilde{A}}}
\newcommand{\tX}{{\tilde{X}}}
\newcommand{\tphi}{{\tilde{\phi}}}
\begin{document}
\title{An ill-posedness result for the Boussinesq equation}

\author{Dan-Andrei Geba, A. Alexandrou Himonas, and David Karapetyan}

\address{Department of Mathematics, University of Rochester, Rochester, NY 14627}
\address{Department of Mathematics, University of Notre Dame, Notre Dame, IN 46556}
\address{Department of Mathematics, University of Notre Dame, Notre Dame, IN 46556}
\date{}

\begin{abstract}
The aim of this paper is to present new ill-posedness results for the nonlinear ``good" Boussinesq equation, which improve upon the ones previously obtained in the literature. In particular, it is proved that the solution map is not continuous in Sobolev spaces $H^s$, for all $s<-7/4$.

\end{abstract}

\subjclass[2000]{35B30, 35Q55}
\keywords{Boussinesq equation, well-posedness, ill-posedness.}

\maketitle

\section{Introduction}

In this article, we consider the Cauchy problem associated with the  nonlinear ``good" Boussinesq equation, i.e.,
\begin{equation}
\left\{
\begin{array}{l}
u_{tt}-u_{xx}+u_{xxxx}+(u^2)_{xx}\,=\,0, \qquad u=u(t,x): \mathbb{R}_+\times M \to \mathbb{C},\\
\\
u(0,x)\,=\,u_0(x),\qquad u_t(0,x)\,=\,u_1(x),\\
\end{array}\right.
\label{main}
\end{equation}
where either $M=\mathbb{R}$ (the non-periodic case) or $M=\mathbb{T}$ (the periodic case). This equation, with $u_{xxxx}$ replaced by $-u_{xxxx}$, was originally introduced by Boussinesq \cite{B72} in connection with propagation of dispersive waves. 

For dispersive equations, one of the most interesting topics to study is their local well-posedness\footnote{As explained in \cite{L93}, the appelative ``good" associated to \eqref{main} has to do, in part, with the fact that, for Boussinesq's original equation, only soliton-type solutions are known, and so one can not study well-posedness except only in very special circumstances.} theory in Sobolev spaces. To gain intuition on this problem for  \eqref{main}, we notice first that the leading terms of the linear operator, $u_{tt}$ and $u_{xxxx}$, tell us that, in principle, one derivative in time ``scales" like two derivatives in the spatial variable. Hence, a correct assumption on the Sobolev regularity of the 
initial data should be in the form of  
\begin{equation}
u_0 \in H^s(\mathbb{M}), \qquad u_1 \in H^{s-2}(\mathbb{M}).
\label{not}
\end{equation}
In certain papers (e.g., \cite{F09} or \cite{FS10}), the more restrictive profile $u_1=\psi_x \in H^{s-2}$, with $\psi \in H^{s-1}$, is considered.

Secondly, the nonlinear Boussinesq equation lacks scaling. However, if one ignores the linear lower order term $u_{xx}$, the resulting equation 
\begin{equation}
u_{tt}+u_{xxxx}+(u^2)_{xx}\,=\,0\label{new}
\end{equation}
is left invariant by the transformation
\[
u_{\lambda}(t,x)\,=\,\frac{1}{\lambda^2}u\left(\frac{t}{\lambda^2}, \frac{x}{\lambda}\right),
\]
which points to the critical Sobolev index $s_c=-3/2$.

Recently, the local well-posedness of \eqref{main} has received considerable interest, with the current state of the art being $s>-1/2$ for the non-periodic case, obtained by Kishimoto and Tsugawa \cite{KT10}, and $s>-3/8$ for the periodic case, due to Oh and Stefanov \cite{OS12}. For more previous results, we refer the reader to the papers of Linares \cite{L93}, Fang and Grillakis \cite{FG96}, Farah \cite{F09}, Farah and Scialom \cite{FS10}, and references therein.

The main goal of this paper is to address the ill-posedness theory for \eqref{main}, which is far less understood. To our knowledge, the only results in this direction are:

1. The bilinear estimate in the Bourgain-type spaces adapted to \eqref{main}, used in proving local well-posedness, fails for $s\leq -1/4$ in both the non-periodic and periodic cases, as shown in  \cite{F09} and \cite{FS10}, respectively;

2. For the non-periodic problem, Farah \cite{F09} demonstrated that the solution map 
\[
S: H^s\times \{\psi_x |\,\psi\in H^{s-1}\}\subset H^s\times H^{s-2} \to C([0,T]; H^s), \quad
S(u_0,u_1)\,=\,u,
\]
is not $C^2$ at zero for $s<-2$.


Our results, on the other hand, are concerned with the failure of continuity for the above solution map, and rely on a rather general technique introduced by Bejenaru and Tao \cite{BT06}, in the context of a quadratic nonlinear Schrodinger equation. This technique has been used also for other dispersive equations: KdV-Burgers (Molinet-Vento \cite{MV09}, \cite{MV10}), Benjamin-Bona-Mahony (Panthee \cite{P11}), Benjamin-Bona-Mahony-Burgers (Banquet \cite{B11}). 


\section{Statement of main results}

\subsection{Framework} Let us now describe the setup in \cite{BT06}, directly formulated for \eqref{main}. The nonlinear ``good" Boussinesq equation can be seen as an abstract semilinear evolution equation, with a bilinear nonlinearity, which can be written in the form 
\begin{equation}
u\,=\,L(f)\,+\,N(u,u),
\label{LN}
\end{equation}
where $f=(u_0, u_1)$ is an initial data lying in a data space $D$ (e.g., $H^s \times H^{s-2}$), the solution $u$ takes values in a solution space $S \subseteq	 C([0,T]; H^s)$, $L: D \to S$ is a linear operator, and $N:S\times S \to S$ is a bilinear form, both of which are densely defined.  

If  $(D,\|\cdot \|_D)$ and $(S,\|\cdot \|_S)$ are a pair of Banach spaces satisfying 
\begin{equation}
\|L(f)\|_S \leq C \|f\|_D,\qquad \|N(u,v)\|_S \leq C \|u\|_S \|v\|_S,
\label{estim}
\end{equation}
where $C>0$ is an absolute constant, the equation $\eqref{LN}$ is called \textbf{quantitatively well-posed}, as a standard contraction argument shows that, for all  $\|f\|_D<\frac{1}{16C^2}$, there exists a unique solution $u$ for $\eqref{LN}$, with $\|u\|_D<\frac{1}{4C}$. In fact, the solution is the sum of an absolutely convergent series in S, 
\begin{equation}
u\,=\,\sum_{n=1}^{\infty} A_n(f), \qquad (\forall) \|f\|_D<\frac{1}{16C^2},
\label{series}
\end{equation}
where the nonlinear maps $A_n: D\to S$ ($n\geq 1$) are defined recursively by
\begin{equation}
A_1(f)\,=\,L(f), \qquad A_n(f)\,=\,\sum_{k=1}^{n-1} N(A_k(f),A_{n-k}(f)), \qquad (\forall)n\geq 2.
\label{An}
\end{equation}


\begin{remark}
In the notation of \cite{F09}, the non-periodic problem is quantitatively well-posed for
\[
D\,=\,H^s\times \{\psi_x |\,\psi\in H^{s-1}\} \subset H^s\times H^{s-2}, \qquad S= C([0,T]; H^s)\cap X^T_{s,b},\]
with the usual norms, where $s>-1/4$ and $b>1/2$. Similar spaces are available for the periodic problem (see \cite{FS10}).
\label{qw}
\end{remark}


\begin{remark}
The local well-posedness results of Kishimoto-Tsugawa \cite{KT10} and Oh-Stefanov \cite{OS12} are obtained by rewriting the Boussinesq equation in the form of a Schrodinger and Schrodinger-like equation, respectively, using \textbf{nonlinear} transformations. For example, \cite{KT10} relies on
\[
v\,=\,u - i(1 - \partial_{xx})^{-1} u_t\]
to transform \eqref{main} into
\[
 i v_{t} \,-\, v_{xx} = \frac{\bar{v}-v}{2}\, -
\,\left[\partial_{xx}\,(1 - \partial_{xx})^{-1}\right] \frac{(v + \bar{v})^{2}}{4}, \quad v(0)\,=\,u_0 - i(1 - \partial_{xx})^{-1} u_1,
\]
which is a quadratic nonlinear Schrodinger equation, followed by a bilinear analysis which is done in modified Bourgain-type spaces for this equation. 

Due to the nonlinear link between the two problems, it is highly nontrivial how to translate this result into a quantitatively well-posed one for \eqref{main}. The same goes for the normal form argument in  \cite{OS12}. We plan to address these questions in a future work.
\label{sw}
\end{remark}

Moreover, Bejenaru-Tao's argument (precisely Proposition 1 in \cite{BT06}) tells us that if the solution series \eqref{series} is continuous as a function of $f$ in  weaker topologies than the ones given by $\| \cdot\|_D$ and $\| \cdot \|_S$, then each of its terms is also continuous in this setup. This is the key fact  that will be used in our proof.

\subsection{Method of proof} A strategy in proving an ill-posedness result based on the previous idea can be formulated as follows:

1. Start with a \textbf{quantitative well-posedness} pair $(D,S)$ for which one has, as described above, a continuous solution map associated to \eqref{LN},
\begin{equation}
f\in (B_1,\|\cdot \|_D)\,\longrightarrow\,u\in (B_2,\|\cdot \|_S),
\end{equation} 
where $B_1\subset D$ and $B_2 \subset S$ are two balls centered at the origin;

2. Choose two weaker norms, $\| \cdot\|_{D'}$ and $\| \cdot\|_{S'}$ related to the regularity for which you try to prove ill-posedness and find an $n\geq 1$ and a sequence $(f_N)_N\subset D$ such that
\begin{equation}
\limsup_{N\to \infty} \|f_N\|_D\,\ll\,1, \qquad \quad \sup_N \frac{\|A_n(f_N)\|_{S'}}{\|f_N\|_{D'}}\,=\,\infty.
\label{ip}
\end{equation}

This will show that 
\begin{equation}
A_n: (B_1,\|\cdot \|_{D'})\,\longrightarrow\,(B_2,\| \cdot\|_{S'}),
\end{equation} 
is not continuous at the origin, which implies that the same is true for the solution map in these coarser  topologies. Moreover, what makes this strategy go through is that, as the time of existence $T=T(\|f\|_D)$ and $\limsup_{N\to \infty} \|f_N\|_D \ll 1$, the $u_N$'s (for $N$ sufficiently large) will all be defined for the same amount of time.


We are now ready to state our main result:

\begin{theorem}
Let the Cauchy problem \eqref{main} be quantitatively well-posed for
\[
D\,=\,H^s \times H^{s-2}(M), \quad S\,\subseteq \,C([0,T]; H^s(M)), \]
where  $M=\mathbb{R}$ or $\mathbb{T}$. Then the solution map
\begin{equation}
S: H^{s'} \times H^{s'-2}(M) \to C([0,T]; H^{s'}(M)), \quad
S(u_0,u_1)\,=\,u,
\label{solmap}
\end{equation}
is not continuous at the origin for any $s'< \min\{-2-s, s\}$.
\label{mainth}
\end{theorem}

Based on Remark \ref{qw}, we obtain immediately the following improvement of the $s'<-2$ ill-posedness result mentioned above:

\begin{corollary}
The solution map \eqref{solmap} is not continuous at the origin for all $s'<-7/4$, in both the periodic and the non-periodic cases.
\end{corollary}

\begin{remark}
Theorem \ref{mainth} suggests that $s=-1$ is potentially the threshold between well-posedness and ill-posedness for \eqref{main}.  In the non-periodic case, this can be corroborated also with the intimate relation between the Boussinesq equation and the quadratic nonlinear Schrodinger equation (as described in Remark \ref{sw}) , and the fact that for 
\[
iv_t\,+\,v_{xx}\,=\,v^2,
\]
$s=-1$ is indeed that threshold, the main result of Bejenaru-Tao \cite{BT06}. 
\end{remark}


\section{Proof of Theorem \ref{mainth}}
We will use the strategy outlined in our method of proof and show that, for a quantitative well-posedness  index $s$, we can construct a sequence $(f_N)_N$ satisfying
\[
\limsup_{N\to \infty} \|f_N\|_{H^s \times H^{s-2}}\,\ll\,1, \qquad \quad \sup_N \frac{\|A_2(f_N)\|_{C([0,T],H^{s'})}}{\|f_N\|_{H^{s'} \times H^{s'-2}}}\,=\,\infty,\]
for any $s'< \min\{-2-s, s\}$. We start by computing the nonlinear map $A_2$.


\subsection{Formulas for $A_2$} In the non-periodic case, using the Fourier transform in the spatial variable and Duhamel's principle, one can derive that the solution of the linear problem
\begin{equation}
\left\{
\begin{array}{l}
u_{tt}-u_{xx}+u_{xxxx}\,=\,F, \quad (t,x)\in \mathbb{R}_+\times\mathbb{R},\\
\\
u(0,x)\,=\,u_0(x),\qquad u_t(0,x)\,=\,u_1(x),\\
\end{array}\right.
\label{hom}
\end{equation}
is given by the formula
\begin{equation}
\hat{u}(t,\xi)\,=\,\cos(t \lambda(\xi)) \hat{u}_0(\xi)\,+\,\frac{\sin(t \lambda(\xi))}{\lambda(\xi)} \hat{u}_1(\xi)\,+\,\int_0^t\,\frac{\sin((t-s) \lambda(\xi))}{\lambda(\xi)} \,\hat{F}(s,\xi)\,ds,
\label{lin}\end{equation}
where $\lambda(\xi)=\sqrt{\xi^2+\xi^4}$. 

Therefore, \eqref{main} can be restated in integral form as
\begin{equation}
\aligned
\hat{u}(t,\xi)\,&=\,\cos(t \lambda(\xi)) \hat{u}_0(\xi)+\frac{\sin(t \lambda(\xi))}{\lambda(\xi)} \hat{u}_1(\xi)-\int_0^t\,\frac{\sin((t-s) \lambda(\xi))}{\lambda(\xi)} \,\widehat{(u^2)_{xx}}(s,\xi)\,ds\\
&=\,\cos(t \lambda(\xi)) \hat{u}_0(\xi)+\frac{\sin(t \lambda(\xi))}{\lambda(\xi)} \hat{u}_1(\xi)+\int_0^t\,\frac{\sin((t-s) \lambda(\xi))}{\lambda(\xi)}\, \xi^2\, \widehat{u^2}(s,\xi)\,ds,\endaligned
\label{mainf}
\end{equation}
which is nothing but the Fourier version of \eqref{LN} with
\begin{equation}
\aligned
\widehat{Lf}(t,\xi)\,=\,\cos(t \lambda(\xi))& \hat{u}_0(\xi)+\frac{\sin(t \lambda(\xi))}{\lambda(\xi)} \hat{u}_1(\xi),\quad f=(u_0,u_1),\\
\widehat{N(u,v)}(t,\xi)&=\,\int_0^t\,\frac{\sin((t-s) \lambda(\xi))}{\lambda(\xi)}\, \xi^2\, \widehat{u\cdot v}(s,\xi)\,ds.\endaligned
\label{lnf}
\end{equation}

Using now \eqref{An}, we can infer that
\[
A_2(f)\,=\,N(A_1(f),A_1(f))\,=\,N(L(f),L(f)),\]
which further implies
\begin{equation}
\aligned
\widehat{A_2(f)}(t,\xi)\,&=\,\int_0^t\,\frac{\sin((t-s) \lambda(\xi))}{\lambda(\xi)}\, \xi^2\, \widehat{(Lf)^2}(s,\xi)\,ds\\
&=\,\int_0^t \int_{\mathbb{R}}\,\frac{\sin((t-s) \lambda(\xi))}{\lambda(\xi)}\, \xi^2\, \widehat{Lf}(s,\xi-\eta)\,\widehat{Lf}(s,\eta)\,d\eta\,ds.\endaligned
\label{A2f}
\end{equation}

For the periodic case, the computations are almost identical, the only difference being that, instead of integrals in $\xi$ or $\eta$, we have to sum up series over integers in $n$ or $m$. As a consequence, we write directly the formula for the coefficients of $A_2(f)$:
\begin{equation}
\widehat{A_2(f)}(t,n)\,=\, n^2 \sum_{m\in\mathbb{Z}} \int_0^t \frac{\sin((t-s) \lambda(n))}{\lambda(n)} \widehat{Lf}(s,n-m)\,\widehat{Lf}(s,m)\,ds,
\label{A2p}\end{equation}
with
\[
\widehat{Lf}(t,m)\,=\,\cos(t \lambda(m)) \hat{u}_0(m)+\frac{\sin(t \lambda(m))}{\lambda(m)} \hat{u}_1(m),\quad f=(u_0,u_1).\]


\subsection{Proof of the non-periodic case} Our choice for the sequence of initial data $(f_N)_{N\geq N_0}$, with $N_0$ a positive arbitrary integer, is given by
\begin{equation}
f_N\,=\,(g_N, 0), \qquad \widehat{g_N}(\xi)\,=\,\frac{r}{N^s} \chi_{\{N \leq |\xi| \leq N + 1\}\}},
\label{fn}
\end{equation}
where $r>0$ is a positive constant and $s$ is a quantitative well-posedness index. A simple computation yields
\begin{equation}
\|f_N\|_{H^{s'}\times H^{s'-2}}\,=\,\|g_N\|_{H^s}\,\simeq\,r\,N^{s'-s},
\label{hsfn}
\end{equation}
which tells us that
\[
\limsup_{N\to \infty} \|f_N\|_{H^{s'} \times H^{s' -2}}\,=\,0, \qquad (\forall)\ s'<s, 
\]
and, in order to have 
\[
\limsup_{N\to \infty} \|f_N\|_{H^s \times H^{s -2}}\,\ll\,1,
\]
one needs to take $r$ sufficiently small. This allows us to normalize the time of existence $T=T(r)$ for all the solutions of size at most $r$.

Hence, choosing $r$ sufficiently small such that $T=1$, plugging the profile \eqref{fn} into \eqref{A2f} and taking into account \eqref{lnf}, we obtain
\begin{equation}
\aligned
&\widehat{A_2(f_N)}(t,\xi)\\
&=\int_0^t \int_{\mathbb{R}} \frac{\sin((t-t') \lambda(\xi))}{\lambda(\xi)} \xi^2 \cos(t' \lambda(\xi-\eta))\cos(t'\lambda(\eta))\widehat{g_N}(\xi-\eta)\widehat{g_N}(\eta)d\eta\,dt'\\
&=\frac{r^2 \xi^2}{N^{2s}\lambda(\xi)}\int_0^t \int_{N \leq |\xi-\eta|, |\eta| \leq N+1}\sin((t-t') \lambda(\xi))\cos(t' \lambda(\xi-\eta))\cos(t'\lambda(\eta))d\eta\,dt'.\endaligned
\label{A2fn}
\end{equation}
On the other hand,
\begin{equation}
\aligned
\|A_2(f_N)\|_{C([0,1], H^{s'}(\mathbb{R}))} &\geq \|A_2(f_N)(t)\|_{H^{s'}(1/4\leq |\xi| \leq 1/2)}\\ &\gtrsim \|A_2(f_N)(t)\|_{L^2(1/4\leq |\xi| \leq 1/2)},\label{hsA2}
\endaligned
\end{equation}
where $0<t=t_N<1$ is an arbitrary time which will be specified later. Thus, we will be working in the regime given by $ |\xi|\sim 1$ and $|\eta|\sim N$, for which simple algebraic manipulations yield
\begin{equation}
\lambda(\xi) \sim 1,\quad \lambda(\xi-\eta)+\lambda(\eta) \sim N^2, \quad \left| \lambda(\xi-\eta)-\lambda(\eta)\right| \sim N.
\label{aprox}
\end{equation}

Relying on the trigonometric identity
\[
\sin a\,\cos b\,\cos c=\frac{\sin (a+b+c)+\sin (a-b-c)+\sin (a+b-c)+\sin (a-b+c)}{4},
\]
we can perform the integration in $t'$ for \eqref{A2fn} to deduce:
\begin{equation}
\aligned
N^{2s}\cdot \widehat{A_2(f_N)}&(t,\xi)\\ 
\simeq  \int_{- N-1+\xi}^{-N} &+\int_{N+\xi}^{N+1} \bigg\{ a(t,\xi,\eta) \left[ \frac{1}{\lambda(\xi - \eta) + \lambda(\eta)- \lambda(\xi)} - \frac{1}{\lambda(\xi - \eta) + \lambda(\eta)+\lambda(\xi)} \right]\\
&+b(t,\xi,\eta) \left[ \frac{1}{-\lambda(\xi - \eta) + \lambda(\eta)-\lambda(\xi)} + \frac{1}{\lambda(\xi - \eta) - \lambda(\eta)-\lambda(\xi)} \right]\bigg\}\,d\eta,
\endaligned
\end{equation}
where
\begin{equation}\aligned
a(t,\xi,\eta)&= \cos (t\lambda(\xi))-\cos[t(\lambda(\xi - \eta) + \lambda(\eta))],\\ b(t,\xi,\eta)&= \cos (t\lambda(\xi))-\cos[t(\lambda(\xi - \eta) - \lambda(\eta))].
\endaligned
\end{equation}

Using \eqref{aprox}, we can estimate the first integrand by
\[
\left|a(t,\xi,\eta) \left[ \frac{1}{\lambda(\xi - \eta) + \lambda(\eta)- \lambda(\xi)} - \frac{1}{\lambda(\xi - \eta) + \lambda(\eta)+\lambda(\xi)} \right]\right|\,\lesssim \frac{1}{N^4}, 
\]
while for the second one, we choose $t=t_N \sim \frac{1}{N}$ such that $b(t_N,\xi) \sim 1$, which further implies:
\[
b(t_N,\xi,\eta) \left[ \frac{1}{-\lambda(\xi - \eta) + \lambda(\eta)-\lambda(\xi)} + \frac{1}{\lambda(\xi - \eta) - \lambda(\eta)-\lambda(\xi)} \right] \sim \frac{1}{N^2}.\]
Both estimates are uniform in $\xi$ and $\eta$.

These allow us, then, to infer that for $N$ sufficiently large,
\[
\widehat{A_2(f_N)}(t_N,\xi) \sim \frac{1}{N^{2s+2}},
\]
uniformly in $\xi$, which in turn yields
\[
\frac{\|A_2(f_N)\|_{C([0,1], H^{s'}(\mathbb{R}))}}{\|f_N\|_{H^{s'}\times H^{s'-2}}}\,\gtrsim\, \frac{1}{N^{s+s'+2}}.\]
Thus
\[
\sup_N\,\frac{\|A_2(f_N)\|_{C([0,1], H^{s'}(\mathbb{R}))}}{\|f_N\|_{H^{s'}\times H^{s'-2}}}\,=\,\infty,\]
if $s'<s$ and $s+s'+2<0$, which finishes the proof.

\subsection{Proof of the periodic case} The analysis is similar to the non-periodic one, as one picks the sequence of initial data $(f_N)_{N\geq 10}$ to be defined as
\begin{equation}
f_N\,=\,(g_N, 0), \qquad \widehat{g_N}(n)\,=\,\frac{r}{N^s} \,\left(\chi_{\{n=N\}}\,+\,\chi_{\{n=1-N\}}\right).
\label{fnp}
\end{equation}
As before, we obtain
\[
\|f_N\|_{H^{s'}\times H^{s'-2}}\,\simeq\,r\,N^{s'-s}, \qquad \widehat{A_2(f_N)}(t,n)\,=\,0, \ (\forall) n \neq 1,\]
and
\[
\widehat{A_2(f_N)}(t,1)\,\sim\,\frac{1}{N^{2s}}\,\int_0^t \sin((t-t') \lambda(1))\cos(t' \lambda(N-1))\cos(t'\lambda(N))\,dt'.
\]

Next, we deduce
\[
\|A_2(f_N)\|_{C([0,1], H^{s'}(\mathbb{T}))} \gtrsim \left|\widehat{A_2(f_N)}(t,1)\right|,
\]
which will be coupled with
\[
\widehat{A_2(f_N)}(t,1)\,\sim\,a(t,N)\cdot O(N^{-2s-4})\,+\,b(t,N)\cdot N^{-2s-2},
\]
where
\[\aligned
a(t,N)&= \cos (t\lambda(1))-\cos[t(\lambda(N-1) + \lambda(N))],\\ b(t,N)&= \cos (t\lambda(1))-\cos[t(\lambda(N) - \lambda(N-1))].
\endaligned
\]

In comparison with the non-periodic case, we have a little bit more flexibility in our choice of $t=t_N$. We can take as in the previous subsection $t_N \sim \frac{1}{N}$, but we can also set $t_N=1$, as the divergence of the sequence $(\cos n)_n$ allows one to identify a subsequence $(N_k)_k\to\infty$ such that 
\[
\liminf_k \left|b(1, N_k)\right|\,>\,0.
\]
Any of these two avenues leads to the same conclusion:
\[
\sup_N\,\frac{\|A_2(f_N)\|_{C([0,1], H^{s'}(\mathbb{T}))}}{\|f_N\|_{H^{s'}\times H^{s'-2}}}\,=\,\infty,\]
for $s'<\min\{s,-2-s\}$, which concludes this argument.

\section*{Acknowledgements}
The first author would like to thank the Department of Mathematics at University of Notre Dame for the hospitality during the summer of 2011, where part of this project was completed. The first author was supported in part by the National Science Foundation Career grant DMS-0747656.

\bibliographystyle{amsplain}
\bibliography{bousbib}


\end{document}
