\documentclass[12pt,reqno]{amsart}
\usepackage{amsmath}
\usepackage{fix-cm} %fix font errors
\usepackage{amssymb}
\usepackage[notcite]{showkeys}
\usepackage{appendix}
\usepackage{cancel}  %for cancelling terms explicity on pdf
\usepackage{yhmath}   %makes fourier transform look nicer, among other things
\usepackage{framed}  %for framing remarks, theorems, etc.
\usepackage{enumerate} %to change enumerate symbols
\usepackage[margin=2.5cm]{geometry}  %page layout
\setcounter{tocdepth}{2} %must come before secnumdepth--else, pain
\setcounter{secnumdepth}{2} %number only sections, not subsections
%\usepackage[pdftex]{graphicx} %for importing pictures into latex--pdf compilation
\numberwithin{equation}{section}  %eliminate need for keeping track of counters
%\numberwithin{figure}{section}
\setlength{\parindent}{0in} %no indentation of paragraphs after section title
\renewcommand{\baselinestretch}{1.1} %increases vert spacing of text
%
\usepackage{hyperref}
\hypersetup{colorlinks=true,
linkcolor=blue,
citecolor=blue,
urlcolor=blue,
}
\usepackage[alphabetic, initials, msc-links]{amsrefs} %for the bibliography; uses cite pkg. Must be loaded after hyperref, otherwise doesn't work properly (conflicts with cref in particular)
\usepackage{cleveref} %must be last loaded package to work properly
\renewcommand{\cref}{\Cref}
%\Crefname{enumi}{}{} %don't write item iv, just iv
%\Crefname{equation}{}{} %don't write Equation (2.1), just (2.1)
\crefformat{equation}{(#2#1#3)} %don't write Equation (2.1), just (2.1)
\Crefformat{equation}{(#2#1#3)} %don't write Equation (2.1), just (2.1)
%
%
\newcommand{\ds}{\displaystyle}
\newcommand{\ts}{\textstyle}
\newcommand{\nin}{\noindent}
\newcommand{\rr}{\mathbb{R}}
\newcommand{\nn}{\mathbb{N}}
\newcommand{\zz}{\mathbb{Z}}
\newcommand{\cc}{\mathbb{C}}
\newcommand{\ci}{\mathbb{T}}
\newcommand{\zzdot}{\dot{\zz}}
\newcommand{\wh}{\widehat}
\newcommand{\p}{\partial}
\newcommand{\ee}{\varepsilon}
\newcommand{\vp}{\varphi}
\newcommand{\wt}{\widetilde}
%
%
%
%
\newtheorem{theorem}{Theorem}[section]
\newtheorem{lemma}[theorem]{Lemma}
\newtheorem{corollary}[theorem]{Corollary}
\newtheorem{claim}[theorem]{Claim}
\newtheorem{prop}[theorem]{Proposition}
\newtheorem{proposition}[theorem]{Proposition}
\newtheorem{no}[theorem]{Notation}
\newtheorem{definition}[theorem]{Definition}
\newtheorem{remark}[theorem]{Remark}
\newtheorem{examp}{Example}[section]
\newtheorem{exercise}[theorem]{Exercise}
%
%makes proof environment bold instead of italic
%\makeatletter
%\renewcommand\subsubsection{\@startsection{subsubsection}{3}{\z@}%
%{-3.25ex\@plus -1ex \@minus -.2ex}%
%{1.5ex \@plus .2ex}%
%{\normalfont\normalsize \bfseries}}
%\makeatother
%makes subsubsubsection bold instead of italic
%
\newcommand{\uol}{u^\omega_\lambda}
\newcommand{\lbar}{\bar{l}}
\renewcommand{\l}{\lambda}
%\renewcommand{\qed}{\qquad \qedsymbol}
\newcommand{\R}{\mathbb{R}}
\newcommand{\RR}{\mathcal{R}}
\newcommand{\al}{\alpha}
\newcommand{\ve}{q}
\newcommand{\tg}{{tan}}
\newcommand{\m}{q}
\newcommand{\N}{N}
\newcommand{\ta}{{\tilde{a}}}
\newcommand{\tb}{{\tilde{b}}}
\newcommand{\tc}{{\tilde{c}}}
\newcommand{\tS}{{\tilde{S}}}
\newcommand{\tP}{{\tilde{P}}}
\newcommand{\tu}{{\tilde{u}}}
\newcommand{\tw}{{\tilde{w}}}
\newcommand{\tA}{{\tilde{A}}}
\newcommand{\tX}{{\tilde{X}}}
\newcommand{\tphi}{{\tilde{\phi}}}
\synctex=1
\frenchspacing %no extra space after periods
\begin{document}
\title{Well-Posedness for the B4 equation}
\author{Dan-Andrei Geba, Alexandrou Himonas, and David Karapetyan}
\address{Department of Mathematics, University of Rochester, Rochester, NY 14627}
\address{Department of Mathematics, University of Notre Dame, Notre Dame, IN 46556}
\address{Department of Mathematics, University of Notre Dame, Notre Dame, IN 46556}
\date{\today}
%
%
\subjclass[2000]{35B30, 35Q55, 35Q72}
\keywords{local well-posedness; ill-posedness.}
\maketitle
\tableofcontents
%
%
\section{Introduction}
%
We consider the initial value problem (ivp) for the B4 
equation 
\begin{gather}
  u_{tt} + u_{xxxx} + (u^2)_{xx} = 0, \quad x \in \rr \ \text{or} \
  \ci, \ t \in \rr
  \label{eqn:mb-2}
  \\
  u(x,0) = u_{0}(x), \quad \p_t u(x, 0) = u_1(x), 
  \label{eqn:mb-init-data-2}
  \\
  \notag
  (u_0, u_1) \in
  H^{s}\times
  H^{s-2}
\end{gather}
%
%
We shall work in the periodic case first, and later generalize our results to
the non-periodic case. Writing the B4 in equivalent integral form, we obtain 
%
\begin{equation}
  \begin{split}
    u(x,t)
    & = \frac{1}{2\pi}\sum_{n \in \zz} e^{inx} \wh{u_{0}}(n) \frac{e^{in^{2}t} + e^{-in^{2}t}}{2} 
    \\
    & + \frac{1}{2 \pi}\sum_{n \in \zz} e^{inx}
    \wh{u_{1}}(n)\frac{e^{in^{2}t} - e^{-in^{2}t}}{2 i n^{2}} 
    \\
    & + \frac{1}{4 i \pi}\sum_{n \in \zz}  e^{inx}
    \int_{0}^{t}[e^{in^{2}(t-t')}-e^{-in^{2}(t-t')}]
    \wh{u^{2}}(n, t') dt'
  \end{split}
  \label{eqn:integral-form}
\end{equation}
%
%
where 
%
%
\begin{equation*}
\begin{split}
  \frac{e^{in^{2}t} - e^{-in^{2}t}}{2 i n^{2}} \vert_{n=0} \doteq t.
\end{split}
\end{equation*}
%
%
Let $\psi(t)$ be a cutoff function symmetric about the 
origin such that $\psi(t) = 1$ for $|t| \le 1$ and $\text{supp} \, \psi 
= [-2, 2 ]$.
Multiplying both sides of expression
$\cref{eqn:integral-form}$ by $\psi(t)$, rearranging terms as in the work of Bourgain \cite{Bourgain:1993ju}, and neglecting constants, we obtain
%
%
%
%
%
\begin{align}
  & u(x,t)
  \label{main1-rel-term-0}
  \\
  \label{main1-rel-term-1}
  & = \psi(t) \sum_{n \in \zz} e^{inx} \wh{u_{0}}(n) \frac{e^{in^{2}t} + e^{-in^{2}t}}{2} 
  \\
  \label{main1-rel-term-2}
  & + \psi(t) \sum_{n \in \zz} e^{inx}
  \wh{u_{1}}(n)\frac{e^{in^{2}t} - e^{-in^{2}t}}{2 i n^{2}} 
  \\
  \label{main1-rel-term-3}
  & +  \psi(t)\sum_{a = \pm 1} \sum_{n\in \zz} \int_\rr e^{ixn}  
  e^{it \tau} \frac{1 - \psi(\tau -  an^{2}) 
}{\tau -  an^{2}} \wh{w}(n, \tau) \ d \tau
  \\
  \label{main1-rel-term-4}
  & + \psi(t) \sum_{a = \pm 1} \sum_{n\in \zz} \int_\rr e^{i(xn + 
  t an^{2})}
  \frac{1- \psi(\tau -  an^{2})}{\tau -  an^{2}} \wh{w}(n, \tau) \ d \tau
  \\
  \label{main1-rel-term-4.5}
  & +  \psi(t) \sum_{a = \pm 1}  \sum_{k \ge 1} \frac{i^k t^k}{k!}
  \sum_{n \in \zz} \int_\rr e^{i(xn + t an^{2} )}
  \psi(\tau -  an^{2}) (\tau -  an^{2})^{k-1} \wh{w}(n, \tau)
  \\
  \label{main1-rel-term-5}
  & \doteq Tu, \quad T=T_{u_0, u_1, \psi}.
\end{align}
%
%
%
%
%
%
%
%
%
%
%
%
We now introduce the following spaces. 
%
%
\begin{definition}
  Let $\mathcal{Y}$ be the space of functions $F(\cdot)$ such that
  \begin{enumerate}[(I)]
   \item{$F: \ci \times \rr \to \cc$}.
   \item{$F(x, \cdot) \in \mathcal{S}(\rr)$ for each $x \in \ci$}.
   \item{$F(\cdot, t) \in C^{\infty}(\ci)$for each $t \in \rr$}.
  \end{enumerate}
  For $s, b \in \rr$, let $X_{s,b}$, $\mathcal{X}_{s,b}$, and $Z_{s,b}$ denote the completion of $\mathcal{Y}$ with
  respect to the norms
  %
  %
  \begin{equation}
  \begin{split}
    \|F\|_{X_{s,b}} = \left( \sum_{n \in \zz} (1 + |n|)^{2s} \int_{\rr}
    (1 + |\tau - n^{2} |)^{2b} \wh{F}(n, \tau)|^{2} d \tau\right)^{1/2},
  \end{split}
  \end{equation}
  %
  \begin{equation}
  \begin{split}
    \|F\|_{\mathcal{X}_{s,b}} = \left( \sum_{n \in \zz} (1 + |n|)^{2s} \int_{\rr}
    (1 + |\tau + n^{2} |)^{2b} \wh{F}(n, \tau)|^{2} d \tau\right)^{1/2},
  \end{split}
  \end{equation}
  %
  %\begin{equation*}
  %\begin{split}
    %\|F\|_{Y_{s,b}} = \|F\|_{X_{s,b}} + \|n^s \wh{F}\|_{\ell^{2}_{n} L^1_\tau },
  %\end{split}
  %\end{equation*}
  %
  and
  %
  %
  \begin{gather*}
    \|F\|_{Z_{s}} = \|P_{| \tau| \ge 2n^{2}}F\|_{X_{s,1/2}} + \| P_{0 \le \tau \le 2n^{2}} u
    \|_{X_{s,1/2}}
    + \| P_{-2n^2 \le \tau \le 0} u  \|_{\mathcal{X}_{s,1/2}}  
  \end{gather*}
\end{definition}
%
where $P_{\Omega} F$ is the Fourier spacetime projection of $F$ onto the subset $\Omega \subset \ci \times \rr$. More precisely, $\wh{P_{\Omega}F} = \chi(n, \tau)\wh{F}(n, \tau)$, where $\chi$ is smooth in $\tau$ with support in $\Omega$.  
%\begin{framed}
%\begin{remark}
  %Observe that $X_{s,b} \subset C(\rr, H^{s})$ continuously for $b>1/2$, and $Y_{s,b} \subset C(\rr, H^{s})$ continuously for $b\ge1/2$. 
  %\end{remark}
  %\end{framed}
%

\begin{framed}
%
%
\begin{remark}
Is $\| \cdot \|_{Z_{s}}$ a norm?
\begin{enumerate}
\item{}
We first check that $\| u \|_{Z^{s}} = 0$ if and only if $u=0$ almost everywhere. 
Clearly $\| 0 \|_{Z_{s}} = 0$, while if $\| u \|_{Z_{s}} = 0$, then $\| P_{| \tau | \ge 2n^{2}}u \|_{X_{s,1/2}}$ and $\|P_{| \tau | < 2n^{2}u}\|_{X_{s-1,0}} = 0$. This implies that $\wh{u}(n, \tau) = 0$ almost everywhere for $| \tau \ge 2n^{2} |$ and $| \tau | < 2n^{2}$. Therefore, $u = 0$ almost everywhere.
\item{}
The triangle inequality also holds. It follows from Minkowski (for $L^{2}$).
\item{}
$Z_{s}$ is also complete. To see this, observe that if $\left \{ u_{n}  \right
\}$ is Cauchy in $Z_{s}$, %
then $\{P_{|\tau| \ge 2n^{2}} u_{n}\}$ is Cauchy in
$X_{s,1/2}$. Using the equality
%
%
\begin{equation*}
\begin{split}
\min(a,b) = | | a-b | - | a+b | |/2
\end{split}
\end{equation*}
%
it follows that $\left \{ P_{0 \le \tau \le 2n^{2}} u_{n} \right \}$ is Cauchy $X_{s,1/2}$. Similarly, $\left \{P_{-2n^{2} \le \tau \le 0} u_{n} \right \}$ is Cauchy in $\mathcal{X}_{s,1/2}$.
But $X_{s,b}, \mathcal{X}_{s,b}$ are Banach for all $s,b \in \rr$. 
%
\end{enumerate}
We conclude that $\| \cdot \|_{Z_{s}}$ is indeed a norm.
\label{rem:}
\end{remark}
%
%

\end{framed}

%%%%%%%%%%%%%%%%%%%%%%%%%%%%%%%%%%%%%%%%%%%%%%%%%%%%%
%
%
%               Embedding 
%
%
%%%%%%%%%%%%%%%%%%%%%%%%%%%%%%%%%%%%%%%%%%%%%%%%%%%%%
%
%
%
\begin{definition}
  Equip $H^{s} \times H^{s-2}$ with the 
  topology defined by the norm $\|(f_0, f_1)\|_{H^{s} \times H^{s-2}}
  = \|f_0\|_{H^{s}} + \|f_1\|_{H^{s-2}}$.
   We say that the B4 ivp
  \cref{eqn:mb-2}-\cref{eqn:mb-init-data-2} is
	\emph{locally well posed} for small data in
  $H^s \times H^{s-2}$ if 
	\begin{enumerate}
    \item Given $(u_{0}, u_{1}) \in H^{s} \times H^{s-2}$
      sufficiently small, the Cauchy problem
      $\psi u = Tu$ has a solution $u \in C([-1,
      1], H^s) \cap Y_{s}$, where $Y_{s} \doteq Y_{s,1/2}$. 
    \item The solution is unique in $C([-1, 1], H^{s}) \cap
      Y_{s}$.
    \item
      The data to solution map $(u_0, u_{1}) \mapsto u(x,t)$ is continuous. That
      is, given a sequence $\{(u_{0,n}, u_{1,n} ) \} \in H^{s} \times H^{s-2}$
      such that $$\|(u_{0}, u_{1})
      - (u_{0,n}, v_{1,n}) \|_{H^{s} \times
      H^{s-2}} \to 0,$$ with corresponding solutions $u_{n} \in
      C([-\delta_{n},
      \delta_{n}])$ and $u \in C([-\delta_{\infty}, \delta_{\infty}])$
      then there exists $0 < \delta \le \inf\left\{
      \delta_{n}, \delta_{\infty} \right\}$ such that $\psi u_{n} =
      Tu_{n}, \psi u = Tu$ and 
      $$\sup_{t \in [-\delta, \delta]}
      \|u(\cdot, t) - u_{n}(\cdot, t) \|_{H^s} \to 0.$$
  \end{enumerate}
\end{definition}
%

%
We are now prepared to state the main result of this paper.
%
%
%
%
%%%%%%%%%%%%%%%%%%%%%%%%%%%%%%%%%%%%%%%%%%%%%%%%%%%%%
%
%
%	Main Result				
%
%
%%%%%%%%%%%%%%%%%%%%%%%%%%%%%%%%%%%%%%%%%%%%%%%%%%%%%
%
%
\begin{theorem}
\label{thm:main}
The initial value problem 
\cref{eqn:mb-2}-\cref{eqn:mb-init-data-2} is locally well-posed for small data in $H^s$ for
$s >
-1/2$ in the periodic case.
%
%
\end{theorem} 
%
%
%
%
%%%%%%%%%%%%%%%%%%%%%%%%%%%%%%%%%%%%%%%%%%%%%%%%%%%%%
%
%
%                Proof of Thm
%
%
%%%%%%%%%%%%%%%%%%%%%%%%%%%%%%%%%%%%%%%%%%%%%%%%%%%%%
%
%
\section{The Periodic Case} 
\label{sec:periodic}
%
What we are interested in is estimating the $X_{s}$ norm of
\cref{main1-rel-term-1}-\cref{main1-rel-term-5} and reducing the construction of a Picard iteration to proving a bilinear estimate.  Obtaining well-posedness from the Picard iteration is standard, and so we shall omit it.
%
%
%
%
%
%
%
%
%
\subsubsection{Estimate for \cref{main1-rel-term-1}}
\label{ssec:est-init-term-1}
We have
%
%
\begin{equation*}
  \begin{split}
    \wh{\cref{main1-rel-term-1}}
    = \frac{\wh{\psi}(\tau -
    n^{2})\wh{u_{0}}(n)}{2} + \frac{\wh{\psi}(\tau +
    n^{2})\wh{u_{0}}(n)}{2}.
  \end{split}
\end{equation*}
%
%
Hence, substituting and applying the inequality 
%
%
\begin{equation}
  \label{square-ineq}
\begin{split}
(a + b)^{2} \le 4(a^{2} +
b^{2}),\ a, b \in \rr
\end{split}
\end{equation}
%
%
gives
%
%
\begin{align*}
  & \| P_{| \tau | \ge 2n^{2}}\cref{main1-rel-term-1} \|_{X_{s,1/2}}^{2} 
    \notag
    \\
    & = \sum_{n \in \zz} \chi_{| \tau | \ge 2n^{2}}(1 + |n|)^{2s} \int_{\rr}\left( 1 + | \tau
    -n^{2} | \right) | \frac{\wh{\psi}(\tau - n^{2})\wh{u_{0}(n)}}{2} +
    \frac{\wh{\psi}(\tau + n^{2})\wh{u_{0}}(n)}{2} |^{2} d \tau
    \notag
    \\
    & \lesssim \sum_{n \in \zz} \left( 1 + |n| \right)^{2s} | \wh{u_{0}}(n)
    |^{2} \int_{\rr} | \wh{\psi}(\tau - n^{2}) |^{2}\left( 1 + |  \tau  -
    n^{2} | \right) d \tau
    \\
    & \lesssim \| u_{0} \|^{2}.
\end{align*}
%
%
%
%
%
%
%
%
%
%
%
Observe that
\begin{align*}
  & \| P_{0 \le \tau \le 2n^{2}}\cref{main1-rel-term-1} \|_{X_{s,1/2}}^{2} 
    \notag
    \\
    & = \sum_{n \in \zz} (1 + |n|)^{2s} \int_{\rr} \chi_{0 \le \tau \le 2n^{2}}(1 + | \tau - n^{2} |)| \frac{\wh{\psi}(\tau - n^{2})\wh{u_{0}(n)}}{2} +
    \frac{\wh{\psi}(\tau + n^{2})\wh{u_{0}}(n)}{2} |^{2} d \tau
    \notag
    \\
    & \lesssim \sum_{n \in \zz} \left( 1 + |n| \right)^{2s} | \wh{u_{0}}(n)
    |^{2} \int_{\rr} \wh{\psi}(\tau - n^{2})d \tau
    \\
    & \lesssim \| u_{0} \|^{2}.
\end{align*}
%
and
\begin{align*}
  & \| P_{-2n^{2} \le \tau \le 0}\cref{main1-rel-term-1} \|_{\mathcal{X}_{s,1/2}}^{2} 
    \notag
    \\
    & = \sum_{n \in \zz} (1 + |n|)^{2s} \int_{\rr} \chi_{-2n^{2} \le \tau \le 0}(1 + | \tau + n^{2} |)| \frac{\wh{\psi}(\tau - n^{2})\wh{u_{0}(n)}}{2} +
    \frac{\wh{\psi}(\tau + n^{2})\wh{u_{0}}(n)}{2} |^{2} d \tau
    \notag
    \\
    & \lesssim \sum_{n \in \zz} \left( 1 + |n| \right)^{2s} | \wh{u_{0}}(n)
    |^{2} \int_{\rr} \wh{\psi}(\tau + n^{2})d \tau
    \\
    & \lesssim \| u_{0} \|^{2}.
\end{align*}

%
Grouping our estimates, and taking square roots, we obtain
%
%
\begin{equation*}
\begin{split}
\| \cref{main1-rel-term-1} \|_{Z_{s}} \lesssim \| u_{0} \|_{H^{s}}.
\end{split}
\end{equation*}
%
%

%
\subsubsection{Estimate for \cref{main1-rel-term-2}}
\label{ssec:est-init-term-2}
We have
%
%
\begin{equation*}
  \begin{split}
    \wh{\psi(t)S_{t}u_{1}}^{x}(n, t)
    & = \psi(t) \wh{u_{1}}(n) \frac{e^{in^2 t} - e^{-in^{2}t}}{2i n^{2}}
    \\
    & = \frac{\psi(t) \wh{u_{1}}(n)e^{in^{2}t}}{2i n^{2}} - \frac{\psi(t)
    \wh{u_{1}}(n)e^{-in^{2}t}}{2i n^{2}}  
  \end{split}
\end{equation*}
%
%
and
%
%
\begin{equation*}
  \begin{split}
    \wh{\psi(t) S_{t}u_{1}}(n, \tau) = \frac{\wh{\psi}(\tau -
    n^{2})\wh{u_{1}}(n)}{2i n^{2}} - \frac{\wh{\psi}(\tau + n^{2})\wh{u_{1}}(n)}{2i
    n^{2}}.
  \end{split}
\end{equation*}
%
Note that 
%
\begin{equation*}
  \begin{split}
    \wh{\psi(t)S_{t}u_{1}}^{x}(0, t)
    & = \psi(t) \wh{u_{1}}(0) t
      \end{split}
\end{equation*}
and so 
%
%
\begin{equation*}
  \begin{split}
    \wh{\psi(t) S_{t}u_{1}}(0, \tau) = i \frac{d}{d \tau} \wh{\psi}(\tau)
    \wh{u_{1}}(0).
  \end{split}
\end{equation*}
%
Hence, substituting and applying \cref{square-ineq}, we have
%
%
\begin{equation*}
  \begin{split}
    \| P_{| \tau | \ge 2n^{2}}\cref{main1-rel-term-2} \|_{X_{s,1/2}}^{2} 
    & = \sum_{n \in \zzdot} \chi_{| \tau | \ge 2n^{2}}(1 + |n|)^{2s} \int_{\rr}\left( 1 + | \tau
    -n^{2} | \right) | \frac{\wh{\psi}(\tau - n^{2})\wh{u_{1}(n)}}{2i
    n^{2}} -
    \frac{\wh{\psi}(\tau + n^{2})\wh{u_{1}}(n)}{2i n^{2}} |^{2} d \tau
    \\
    & + |\wh{u_{1}}(0)|^{2} \int_{\rr} (1 + | \tau |) | i \frac{d}{d \tau}
    \wh{\psi}(\tau)|^{2} d \tau
    \\
    & \lesssim \sum_{n \in \dot{\zz}} n^{-4} \left( 1 + |n| \right)^{2s} | \wh{u_{1}}(n)
    |^{2} \int_{\rr} | \wh{\psi}(\tau - n^{2}) |^{2}\left( 1 + |  \tau -
    n^{2} | \right) d \tau
    \\
    & + |\wh{u_{1}}(0)|^{2} \int_{\rr} (1 + | \tau |) |\frac{d}{d \tau}
    \wh{\psi}(\tau)|^2 d \tau
    \\
    & \lesssim \| u_{1} \|_{H^{s-2}}^{2}.
\end{split}
\end{equation*}
%
Also,
\begin{equation*}
  \begin{split}
    \| P_{0 \le  \tau  \le  2n^{2}}\cref{main1-rel-term-2} \|_{X_{s,1/2}}^{2} 
    & = \sum_{n \in \zzdot} (1 + |n|)^{2s} \int_{\rr}  \chi_{0 \le \tau  \le 2n^{2}}(1 + | \tau - n^{2} |) | \frac{\wh{\psi}(\tau - n^{2})\wh{u_{1}(n)}}{2i
    n^{2}} -
    \frac{\wh{\psi}(\tau + n^{2})\wh{u_{1}}(n)}{2i n^{2}} |^{2} d \tau
    \\
    & + |\wh{u_{1}}(0)|^{2} \int_{\rr} (1 + | \tau|) | i \frac{d}{d \tau}
    \wh{\psi}(\tau)|^{2} d \tau
    \\
    & \lesssim \sum_{n \in \dot{\zz}} n^{-4} \left( 1 + |n| \right)^{2s} | \wh{u_{1}}(n)
    |^{2} \int_{\rr}  (1 + | \tau - n^{2} |)| \wh{\psi}(\tau - n^{2}) |^{2} d \tau
    \\
    & + |\wh{u_{1}}(0)|^{2} \int_{\rr} (1 + | \tau |) |\frac{d}{d \tau}
    \wh{\psi}(\tau)|^2 d \tau
    \\
    & \lesssim \| u_{1} \|_{H^{s-2}}^{2}.
\end{split}
\end{equation*}
%
and
\begin{equation*}
  \begin{split}
    \| P_{0 \le  \tau  \le  2n^{2}}\cref{main1-rel-term-2} \|_{\mathcal{X}_{s,1/2}}^{2} 
    & = \sum_{n \in \zzdot} (1 + |n|)^{2s} \int_{\rr}  \chi_{-2n^{2} \le \tau \le 0}(1 + | \tau + n^{2} |) | \frac{\wh{\psi}(\tau - n^{2})\wh{u_{1}(n)}}{2i
    n^{2}} -
    \frac{\wh{\psi}(\tau + n^{2})\wh{u_{1}}(n)}{2i n^{2}} |^{2} d \tau
    \\
    & + |\wh{u_{1}}(0)|^{2} \int_{\rr} | i \frac{d}{d \tau}
    \wh{\psi}(\tau)|^{2} d \tau
    \\
    & \lesssim \sum_{n \in \dot{\zz}} n^{-4} \left( 1 + |n| \right)^{2s} | \wh{u_{1}}(n)
    |^{2} \int_{\rr}  (1 + | \tau + n^{2} |) | \wh{\psi}(\tau + n^{2}) |^{2} d \tau
    \\
    & + |\wh{u_{1}}(0)|^{2} \int_{\rr} (1 + | \tau |) |\frac{d}{d \tau}
    \wh{\psi}(\tau)|^2 d \tau
    \\
    & \lesssim \| u_{1} \|_{H^{s-2}}^{2}.
\end{split}
\end{equation*}

%
Combining our estimates and taking square roots, it follows that
%
\begin{equation}
  \label{u-1-est}
  \begin{split}
    \|\cref{main1-rel-term-2}|_{Z_{s}} \lesssim 
    \|u_{1}\|_{H^{s-2}}.
  \end{split}
\end{equation}
%
%
%
%
\subsubsection{Estimate for \cref{main1-rel-term-3}.}
%
%
%%%%%%%%%%%%%%%%%%%%%%%%%%%%%%%%%%%%%%%%%%%%%%%%%%%%%
%
%
%			Schwartz Multiplier	
%
%
%%%%%%%%%%%%%%%%%%%%%%%%%%%%%%%%%%%%%%%%%%%%%%%%%%%%%
%
%
%
We now need the following.
%
\begin{lemma}
\label{lem:schwartz-mult}
For $s\in \rr$, we have
%
%
\begin{equation}
	\label{schwartz-mult}
	\begin{split}
    \|\psi f \|_{X_{s,1/2}} \lesssim_{\psi} \|f \|_{X_{s,1/2}}.
	\end{split}
\end{equation}
%
%
\end{lemma}
%
Hence
%
%
\begin{equation}
  \label{yu}
	\begin{split}
		\|P_{| \tau | > 2n^{2}}\cref{main1-rel-term-3}\|_{X_{s,1/2}}^{2} 
    & \lesssim_{\psi} 
    \sum_{a = \pm 1} \| P_{| \tau | > 2n^{2}}\sum_{n \in \zz}  e^{ixn} \int_\rr 
		e^{it \tau} \frac{1 - a\psi (\tau - an^{2} ) 
}{\tau - an^{2}} \wh{w}(n, \tau) \ 
		d \tau\|_{X_{s,1/2}}^{2}
		\\
    & = \sum_{a = \pm 1}\sum_{n \in \zzdot} \chi_{| \tau | \ge 2n^{2}}\left (1 + |n| \right )^{2s} \int_\rr
    (1 + |  \tau - n^{2}|) \left | \frac{1 - a\psi(\tau - an^{2 
})}{\tau - an^{2}} 
     \wh{w}(n, \tau) \right |^2 \ d 
		\tau 
		\\
    & \le \sum_{a = \pm 1}
    \sum_{n \in \zzdot} \chi_{| \tau | \ge 2n^{2}} \left (1 + |n| \right )^{2s} \int_{| \tau - an^{2}| \ge 1}
    (1 + | \tau - n^{2}|) \frac{|  \wh{w}(n, \tau)|^2}{|\tau - an^{2}|^2} 
		\ d 
		\tau 
		\\
		  & \lesssim 
 \sum_{n \in 
		\zzdot} \chi_{| \tau | \ge 2n^{2}} \left (1 + |n| \right )^{2s} \int_\rr (1 +  | \tau - n^{2}|)^{-1} | \wh{w}(n, \tau) |^2 
		 \ d \tau 
		 \\
		 & +  \sum_{n \in 
		\zzdot} \chi_{| \tau | \ge 2n^{2}} \left (1 + |n| \right )^{2(s+1)} \int_\rr (1 +  | \tau + n^{2}|)^{-1} | \wh{w}(n, \tau) |^2 
		 \ d \tau 
		\\
		& \lesssim \| P_{| \tau | \ge 2n^{2}}w \|_{X_{s,0}}
		\\
		& \lesssim \| P_{| \tau | \ge 2n^{2}}u \|^{2}_{X_{s,1/2}}
\end{split}
\end{equation}
%
%
%
where the last step follows from the following bilinear
estimate, whose proof we leave for later.
%
%
%%%%%%%%%%%%%%%%%%%%%%%%%%%%%%%%%%%%%%%%%%%%%%%%%%%%%
%
%
%				Proposition
%
%
%%%%%%%%%%%%%%%%%%%%%%%%%%%%%%%%%%%%%%%%%%%%%%%%%%%%%
%
%
\begin{proposition}
\label{prop:bilinear-est}
	%
	%
	If $s \ge -1/2$ then 
	\begin{equation}
	  \| P_{| \tau | \ge 2n^{2}}(fg) \|_{X_{s,0}}
		    \lesssim \|P_{| \tau | \ge 2n^{2}}f\|_{X_{s,1/2}} \|P_{| \tau | \ge 2n^{2}}g\|_{X_{s,1/2}}.
	\end{equation}
%
%
%
%
\end{proposition}
%
%
Also
%
%
\begin{equation}
	\begin{split}
		\|P_{0 \le  \tau  \le 2n^{2}}\cref{main1-rel-term-3}_{a=1}\|_{X_{s,1/2}}^{2} 
    & \lesssim
    \| P_{| 0 \le \tau \le 2n^{2}} \sum_{n \in \zz} e^{ixn} \int_\rr 
		e^{it \tau} \frac{1 - \psi (\tau - n^{2} ) 
}{\tau - n^{2}} \wh{w}(n, \tau) \ 
		d \tau\|_{X_{s,1/2}}^{2}
		\\
    & = \sum_{n \in \zzdot} \chi_{0 \le \tau \le 2n^{2}}\left (1 + |n| \right )^{2s} \int_\rr (1 + | \tau - n^{2} |)
    \left | \frac{1 - \psi(\tau - n^{2 
})}{\tau - n^{2}} 
     \wh{w}(n, \tau) \right |^2 \ d 
		\tau 
		\\
    & \le 
    \sum_{n \in \zzdot} \chi_{0 \le  \tau  < 2n^{2}} \left (1 + |n| \right )^{2s} \int_{| \tau - n^{2}| \ge 1} (1 + | \tau - n^{2} |)
    \frac{|  \wh{w}(n, \tau)|^2}{|\tau - n^{2}|^2} 
		\ d 
		\tau 
		\\
		& \lesssim \| P_{0 \le  \tau \le 2n^{2}}w \|_{X_{s,-1/2}}
		\\
		& \lesssim \| P_{0 \le  \tau \le 2n^{2}}u \|^{2}_{X_{s,1/2}}
\end{split}
\end{equation}
where the last step follows from the following bilinear
estimate, whose proof we leave for later.
%
%
%%%%%%%%%%%%%%%%%%%%%%%%%%%%%%%%%%%%%%%%%%%%%%%%%%%%%
%
%
%				Proposition
%
%
%%%%%%%%%%%%%%%%%%%%%%%%%%%%%%%%%%%%%%%%%%%%%%%%%%%%%
%
%
\begin{proposition}
\label{prop:bilinear-est2}
	%
	%
	If $s \ge -1/2$ then 
	\begin{equation}
	  \| P_{0 \le  \tau  \le 2n^{2}}(fg) \|_{X_{s,-1/2}}
		    \lesssim \|P_{0 \le  \tau \le 2n^{2}}f\|_{X_{s,1/2}} \|P_{0 \le \tau \le 2n^{2}}g\|_{X_{s,1/2}}.
	\end{equation}
%
%
%
%
\end{proposition}

%
%
Next, we analyze
%
%
\begin{equation}
	\begin{split}
		\|P_{0 \le  \tau  \le 2n^{2}}\cref{main1-rel-term-3}_{a=-1}\|_{X_{s,1/2}}^{2} 
    & \lesssim
    \| P_{| 0 \le \tau \le 2n^{2}} \sum_{n \in \zz} e^{ixn} \int_\rr 
		e^{it \tau} \frac{1 - \psi (\tau + n^{2} ) 
}{\tau + n^{2}} \wh{w}(n, \tau) \ 
		d \tau\|_{X_{s,1/2}}^{2}
		\\
    & = \sum_{n \in \zzdot} \left (1 + |n| \right )^{2s} \int_\rr \chi_{0 \le \tau \le 2n^{2}}(1 + | \tau - n^{2} |)
    \left | \frac{1 - \psi(\tau + n^{2 
})}{\tau + n^{2}} 
     \wh{w}(n, \tau) \right |^2 \ d 
		\tau 
		\\
    & \le 
    \sum_{n \in \zzdot}  \left (1 + |n| \right )^{2s} \int_{| \tau + n^{2}| \ge 1} \chi_{0 \le  \tau  < 2n^{2}}(1 + | \tau - n^{2} |)
    \frac{|  \wh{w}(n, \tau)|^2}{|\tau + n^{2}|^2} 
		\ d 
		\tau 
		\\
& \lesssim
    \sum_{n \in \zzdot}  \left (1 + |n| \right )^{2s} \int \chi_{0 \le  \tau  < 2n^{2}} (1 + | \tau - n^{2} |)
    \frac{|  \wh{w}(n, \tau)|^2}{(1 + |\tau + n^{2}|)^2} d \tau
\\
		& \lesssim \| P_{0 \le  \tau \le 2n^{2}}w \|_{X_{s,-1/2}}
		\\
		& \lesssim \| P_{0 \le  \tau \le 2n^{2}}u \|^{2}_{X_{s,1/2}}
\end{split}
\end{equation}
where the last step follows from \cref{prop:bilinear-est2}. 
Having gone through all the above machinery, observe that we can estimate $\| P_{-2n^{2} \le \tau \le 0} \cref{main1-rel-term-3}_{a = \pm 1}\|_{\mathcal{X}_{s,1/2}}$ in precisely the same way when equipped with the following bilinear estimate, whose proof we leave for later. 
\begin{proposition}
	%
	%
	If $s \ge -1/2$ then 
	\begin{equation}
	  \| P_{-2n^{2} \le \tau \le 0}(fg) \|_{\mathcal{X}_{s,-1/2}}
		    \lesssim \|P_{-2n^{2} \le \tau \le 0}f\|_{\mathcal{X}_{s,1/2}} \|P_{-2n^{2} \le \tau \le 0}g\|_{\mathcal{X}_{s,1/2}}.
	\end{equation}
%
%
%
%
\end{proposition}
Collecting our estimates, we obtain%
\begin{equation}
	\begin{split}
    \|\cref{main1-rel-term-3}\|_{Z_{s}} \lesssim
    \|u\|^{2}_{X_{s,1/2}}.
	\end{split}
\end{equation}
%
%
%
%
%
\subsubsection{Estimate for \cref{main1-rel-term-4}.}
Letting $$f(x,t) = \sum_{a = \pm 1} \psi(t) \sum_{n \in \zz} e^{i\left( xn +
atn^{2} \right)}  
\int_\rr \frac{1 - \psi\left( \lambda - an^{2} \right)}{\lambda - an^{2}} 
\wh{w} \left( n, \lambda \right) \ d \lambda,$$ we have
%
%
\begin{equation*}
	\begin{split}
		& \wh{f^x}(n, t) = 
		\psi(t) e^{aitn^{2}} \int_\rr
		\frac{1 - \psi\left( \lambda - an^{2} \right)}{\lambda - an^{2}} 
		\wh{w}(n, \lambda) \ d \lambda
	\end{split}
\end{equation*}
and
\begin{equation*}
	\begin{split}
		 \wh{f}\left( n, \tau \right)
		 & = \int_\rr e^{-it\left( \tau - an^{2} 
		\right)} \psi(t) \int_\rr \frac{1 - a\psi\left( 
		\lambda - an^{2} 
		\right)}{\lambda - an^{2}} \wh{w}(n, \lambda) \ d \lambda d \tau
		\\
    & = \wh{\psi}\left( \tau - an^{2} \right) \int_\rr 
		\frac{1 - a\psi\left( 
		\lambda - an^{2} 
		\right)}{\lambda - an^{2}} \wh{w}(n, \lambda) \ d \lambda.
	\end{split}
\end{equation*}
Therefore,
%
%
\begin{equation}
  \label{iu}
	\begin{split}
	& \| P_{| \tau | \ge 2n^{2}}\cref{main1-rel-term-4} \|_{X_{s,1/2}}^{2} 
		\\
    & \lesssim 
    \sum_{a = \pm 1} \sum_{n \in \zzdot} \left (1 + |n| \right)^{2s}
    \int_\rr \chi_{|\tau| \ge 2n^{2}}\left( 1 + | \tau - n^{2} \right ) | | \wh{\psi}\left(
    \tau - an^{2} \right) |^2 \ d \tau 
		\\
		& \times |
		\int_\rr \frac{1 - \psi\left( \lambda - an^{2} \right)}{\lambda -
		an^{2}} \wh{w}(n, \lambda) \ d \lambda |^2  
		\\
		& \lesssim 
\sum_{a = \pm 1}\sum_{n \in \zzdot} \left (1 + |n| \right )^{2s} | \int_\rr
		\frac{1 - \psi\left( \lambda - an^{2} \right)}{\lambda - an^{2}}
		\wh{w}(n, \lambda) \ d\lambda |^2 
		\\
    & \simeq  \sum_{a = \pm 1} \sum_{n \in \zzdot} \left (1 + |n| \right )^{2s}  \left ( \int_\rr
		\frac{1 - \psi\left( \lambda - an^{2} \right)}{|\lambda - an^{2}|}
		|\wh{w}(n, \lambda) | \ d\lambda \right )^2
		\\
    & \lesssim \sum_{a = \pm 1} \sum_{n \in \zzdot} \left (1 + |n| \right )^{2s}  \left ( \int
		\frac{(\chi_{A_{1}} + \chi_{A_{2}}+ \chi_{A_{3}})|\wh{w}(n, \lambda) |}{1 + |\lambda - an^{2}|}
		\ d\lambda \right )^2 
  \end{split}
\end{equation}
where $A_{1} = \{(n, \lambda) \in \zz \times \rr: | \tau | \ge n^{2} \}$, $A_{2} = \left \{ (n, \tau) \in \zz \times \rr: 0 \le \tau \le 2n^{2} \right \}$, and $A_{3} = \left \{ (n, \tau) \in \zz \times \rr: -2n^{2} \le \tau \le 0 \right \}$. Observe that in region $A_{1}$, $(1 + | \lambda - an^{2} |) \sim (1 + | \lambda + an^{2} |)$. Also observe that $(1 + | \lambda + n^{2} |) \ge (1 + | \lambda - n^{2} |)$ in region $A_{2}$, and $(1 + | \lambda + n^{2} |) \le (1 + | \lambda - n^{2} |)$ in region $A_{3}$. Therefore,
\begin{equation}
\begin{split}
& \sum_{a = \pm 1} \sum_{n \in \zzdot} \left (1 + |n| \right )^{2s}  \left ( \int
		\frac{(\chi_{A_{1}} + \chi_{A_{2}}+ \chi_{A_{3}})|\wh{w}(n, \lambda) |}{1 + |\lambda - an^{2}|}
		\ d\lambda \right )^2
		\\
		& \lesssim \| P_{| \tau | \ge 2n^	2} \|_{X_{s,1/2}}^{2} + \| P_{0 \le  \tau  \le 2n^	2} \|_{X_{s,1/2}}^{2} + \| P_{-2n^{2} \le  \tau \le 0} \|_{\mathcal{X}_{s,1/2}}^{2}
		\end{split}
		\end{equation}
		via the inequality $(a + b + c)^{2} \le 8(a^{2} + b^{2} +
		c^{2})$ and the following bilinear estimates.
		%
		%
		%%%%%%%%%%%%%%%%%%%%%%%%%%%%%%%%%%%%%%%%%%%%%%%%%%%%%
		%
		%
		%				
		%
		%
		%%%%%%%%%%%%%%%%%%%%%%%%%%%%%%%%%%%%%%%%%%%%%%%%%%%%%
		%
		%
		\begin{proposition}
\label{prop:bilin-1-endpoint}
		For $s \ge -1/2$
		\begin{equation}
		\left [ \sum_{n \in \zz} \left (1 + |n| \right )^{2s}  \left ( \int
		\frac{\chi_{| \tau | \ge 2n^{2}}|\wh{fg}(n, \tau) |}{1 + |\tau - n^{2}|}
		\ d\tau \right )^2 \right ]^{1/2} \lesssim \| P_{| \tau | \ge 2n^{2}} f \|_{X_{s,1/2}}\| P_{| \tau | \ge 2n^{2}} g \|_{X_{s,1/2}}
		\end{equation}
				\end{proposition}
		%
		%
		\begin{proposition}
		\label{prop:bilin-2-endpoint}
		For $s \ge -1/2$
		\begin{equation}
		\left [ \sum_{n \in \zz} \left (1 + |n| \right )^{2s}  \left ( \int
		\frac{\chi_{| 0 \le \tau  \le 2n^{2}}|\wh{fg}(n, \tau) |}{1 + |\tau - n^{2}|}
		\ d\tau \right )^2 \right ]^{1/2} \lesssim \| P_{0 \le \tau \le 2n^{2}} f \|_{X_{s,1/2}}\| P_{0 \le  \tau \le 2n^{2}} g \|_{X_{s,1/2}}
		\end{equation}
		\end{proposition}
\begin{proposition}
		\label{prop:bilin-3-endpoint}
		For $s \ge -1/2$
		\begin{equation}
		\left [ \sum_{n \in \zz} \left (1 + |n| \right )^{2s}  \left ( \int
		\frac{\chi_{-2n^{2} \le  \tau \le 0}|\wh{fg}(n, \tau) |}{1 + |\tau + n^{2}|}
		\ d\tau \right )^2 \right ]^{1/2} \lesssim \| P_{-2n^{2} \le \tau \le 0} f \|_{\mathcal{X}_{s,1/2}}\| P_{-2n^{2} \le  \tau  \le 0} g \|_{\mathcal{X}_{s,1/2}}.
		\end{equation}
		\end{proposition}
Estimating in analogous fashion, we obtain
$$ \| P_{0 \le \tau \le 2n^{2}}\cref{main1-rel-term-4} \|_{X_{s,1/2}}^{2}
\lesssim P_{| \tau | \ge 2n^	2} \|_{X_{s,1/2}}^{2} + \| P_{0 \le  \tau  \le 2n^	2} \|_{X_{s,1/2}}^{2} + \| P_{-2n^{2} \le  \tau \le 0} \|_{\mathcal{X}_{s,1/2}}^{2}
$$
and $$ \| P_{-2n^{2} \le  \tau  \le 0}\cref{main1-rel-term-4} \|_{X_{s,1/2}}
\lesssim P_{| \tau | \ge 2n^	2} \|_{X_{s,1/2}}^{2} + \| P_{0 \le  \tau  \le 2n^	2} \|_{X_{s,1/2}}^{2} + \| P_{-2n^{2} \le  \tau \le 0} \|_{\mathcal{X}_{s,1/2}}^{2}
$$
%
%
Hence, collecting our estimates and taking square roots, we obtain
%
\begin{equation}
  \label{main-int4-est}
	\begin{split}
    \|\cref{main1-rel-term-4}\|_{Z_{s}} \lesssim 
     \|u\|^{2}_{Z_{s}}, \quad   s \ge -1/2.
	\end{split}
\end{equation}

%
%
%
\subsubsection{Estimate for \cref{main1-rel-term-4.5}.}
Note that
%
%
\begin{equation}
	\label{1n}
	\begin{split}
    \cref{main1-rel-term-4.5} \simeq \sum_{a = \pm 1}\sum_{k \ge 1}
		\frac{i^k}{k!}g_k(x,t)
	\end{split}
\end{equation}
%
%
where 
%
%
\begin{equation*}
	\begin{split}
	  & g_k(x,t) = t^k \psi(t) \sum_{n \in \zz}  e^{i\left( xn + ta n^{2}
		\right)} h_k(n),
		\\
		& h_k(n) = \int_\rr \psi \left( \tau - an^{2} \right) \cdot \left(
		\tau - an^{2} \right)^{k -1} \wh{w}(n, \tau) \ d \tau.
	\end{split}
\end{equation*}
%
%
Hence
%
%
\begin{equation*}
	\begin{split}
		\wh{g_k^x}(n, t) = t^{k} \psi(t) e^{i t an^{2}} h_k(n)
	\end{split}
\end{equation*}
%
%
which gives
%
%
\begin{equation*}
	\begin{split}
		\wh{g_k}(n, \tau)
		& = h_k(n) \int_\rr e^{-it\left( \tau - an^{2} \right)}
		t^{k}\psi(t) \ dt
		\\
		& = h_k(n) \wh{t^{k}\psi(t)} \left( \tau - an^{2} \right).
	\end{split}
\end{equation*}
%
%
Applying this to \cref{1n} and using Minkowski's inequality, we obtain
%
%
\begin{equation}
	\label{2n}
	\begin{split}
		& \|P_{| \tau | \ge 2n^{2}}\cref{main1-rel-term-4.5}\|_{X_{s,1/2}} 
    \\
    & \lesssim \sum_{a = \pm 1} \left( \sum_{n \in \zzdot} \left (1 + |n| \right )^{2s}
    \int_\rr \chi_{| \tau | \ge 2n^{2}} \left( 1 + | \tau - an^{2} | \right)^{2}
    | \wh{\sum_{k \ge 1} \frac{i^k}{k!}g_k(x,t)} |^2 \ d \tau
		\right)^{1/2}
		\\
		& \le \sum_{a = \pm 1}\sum_{k \ge 1} \frac{1}{k!}\left( \sum_{n \in \zzdot} \left (1 + |n| \right )^{2s}
    \int_\rr \left( 1 + | \tau - an^{2} | \right)^{2} | \wh{g_k}(n, \tau) |^2 \
		d \tau \right)^{1/2}
		\\
		& = \sum_{a = \pm 1}\sum_{k \ge 1} \frac{1}{k!} \left( \sum_{n \in \zzdot} \left (1 + |n| \right )^{2s}
    \int_\rr \left( 1 + | \tau - an^{2} | \right)^{2} | h_k(n) \wh{t^k
		\psi(t)} \left( \tau - an^{2} \right) |^2 \ d \tau \right)^{1/2}
		\\
		& = \sum_{a = \pm 1}\sum_{k \ge 1} \frac{1}{k!} \left( \sum_{n \in \zzdot} \left (1 + |n| \right )^{2s} |
    h_k(n) |^2 \int_\rr \left( 1 + | \tau - an^{2} | \right)^{2} | \wh{t^k
		\psi(t)} \left( \tau - an^{2} \right) |^2 \ d \tau \right)^{1/2}.
	\end{split}
\end{equation}
%
%
Applying the change
of variable $\tau - an^{2} = \tau'$
gives
%
%
\begin{equation}
	\label{3n}
	\begin{split}
		& \int_\rr \left( 1 + | \tau - an^{2} | \right)^{2} | \wh{t^{k}
		\psi(t)}\left( \tau - an^{2} \right) |^2 \ d \tau
    \\
    & = 2 
    \int_\rr \left( 1 + |\tau'| \right)^{2} | \wh{t^k \psi(t)}(\tau') |^2 \
		d \tau'
		\\
    & \lesssim
    \int_\rr \left( 1 + |\tau'| \right)^{2} | \wh{t^k \psi(t)}(\tau')
		|^2 \ d \tau'
		\\
    & \le \| t^{k} \psi \|_{H^{[b] +
  1}}^{2}
\end{split}
\end{equation}
%
%
where $[b]$ denotes the least integer of $b$. Note that
%
%
\begin{equation}
	\label{4n}
	\begin{split}
    & \|t^k \psi \|_{H^{[b] +1}(\rr)}
		\\
    & = \|t^k \psi\|_{L^2(\rr)} + \|\p_t (t^k \psi )
    \|_{L^2(\rr)} 
    \\
    & + \| \p_{t}^{2} (t^{k} \psi) \|_{L^{2}(\rr)} + \cdots + \|
    \p_{t}^{[b] + 1} (t^{k} \psi)\|_{L^{2}}
    \\
    & \le c_{\psi} + k c'_{\psi} + k (k -1) c_{\psi}'' + \cdots +
    k(k-1) \cdots (k - [b]) c_{\psi}^{[b] + 1}
    \\
    & \lesssim c_{\psi} k(k-1) \cdots (k - [b]).
	\end{split}
\end{equation}
%
%
Hence, applying \cref{3n} and \cref{4n} to \cref{2n}, we obtain
%
%%
\begin{equation}
	\label{5n}
	\begin{split}
		\|P_{| \tau | \ge 2n^{2}}\cref{main1-rel-term-4.5} \|_{X_{s,1/2}}
		& \lesssim \sum_{a = \pm 1}
    \sum_{k \ge [b] +1} \frac{1}{(k-[b] - 1)!} \left( \sum_{n \in \zzdot} \left (1 + |n| \right )^{2s} | h_k(n) |^2 
		\right)^{1/2}
		\\
    & \le \sum_{a = \pm 1} \sum_{k \ge [b] +1} \frac{1}{(k-[b] - 1)!}
    \times \sup_{k \ge [b] + 1} \left( \sum_{n \in \zzdot} \left (1 + |n| \right )^{2s} | 
		h_k(n) |^2 \right)^{1/2}
		\\
    & = \sum_{a = \pm 1}\sum_{k \ge [b] +1} \frac{1}{(k-[b] - 1)!}
    \\
    & \times \sup_{k \ge [b] + 1} 
		\left( \sum_{n \in \zzdot} \left (1 + |n| \right )^{2s} |\int_\rr 
		\psi\left( \tau - an^{2} \right) \cdot \left( \tau - an^{2} 
    \right)^{k -1} \wh{w}(n, \tau) \ d \tau|^{2} \right)^{1/2}.
    \end{split}
\end{equation}
%
%%
Recall that $0 \le \psi \le 1, \text{supp} \, \psi \subset [-2,2 ]$. 
This implies $$| \psi\left( \tau - an^{2} \right) \cdot \left( \tau - an^{2}
\right)^{k -1} | \le 2^{k-1} \chi_{| \tau - an^{2} | \le 1}, \qquad k \ge 1.$$ Hence,
we bound the right hand side of \cref{5n} by
%
%%
\begin{equation*}
	\begin{split}
    & c_{\psi} \sum_{a = \pm 1}
    \sum_{k \ge [b] +1} \frac{2^{k-1}}{(k-[b] - 1)!}
    \times \left( \sum_{n \in \zzdot} (1 + | n |)^{2s}| 
		\int_{| \tau - an^{2}  |\le 1}  \wh{w}(n, \tau) \ d \tau |^2 
		\right)^{1/2}
    \\
    & = \frac{e^{2}}{2^{[b]}} c_{\psi} \sum_{a = \pm 1} \left( \sum_{n \in \zzdot} (1 + | n |)^{2s}| 
		\int_{| \tau - an^{2}  |\le 1}  \wh{w}(n, \tau) \ d \tau |^2 
		\right)^{1/2}
    \\
    & \lesssim \sum_{a = \pm 1}
\left[ \sum_{n \in \zzdot} (1 + | n |)^{2s}\left (  
		\int_{| \tau - an^{2}  |\le 1} | \wh{w}(n, \tau) | \ d \tau \right ) ^2 
		\right]^{1/2}
		\\
		  & \lesssim \left[ \sum_{n \in \zzdot} (1 + | n |)^{2s}\left(
		  \int_\rr \frac{|\wh{w}(n, \tau)|}{1 + | \tau - n^{2} |} \ d
		  \tau \right ) ^2 \right]^{1/2}.
		\\
		& \lesssim \| P_{| \tau | \ge 2n^	2} \|_{X_{s,1/2}}^{2} + \| P_{0 \le  \tau  \le 2n^	2} \|_{X_{s,1/2}}^{2} + \| P_{-2n^{2} \le  \tau \le 0} \|_{\mathcal{X}_{s,1/2}}^{2}
		\end{split}
		\end{equation*}
%
where the last step follows from the 
the triangle inequality and 
\cref{prop:bilin-1-endpoint}-\cref{prop:bilin-3-endpoint}. Hence, 
%
%
\begin{equation}
	\begin{split}
    \|P_{| \tau | \ge 2n^{2}}\cref{main1-rel-term-4.5}\|_{X_{s,1/2}} \lesssim 
    \| P_{| \tau | \ge 2n^	2} \|_{X_{s,1/2}}^{2} + \| P_{0 \le  \tau  \le 2n^	2} \|_{X_{s,1/2}}^{2} + \| P_{-2n^{2} \le  \tau \le 0} \|_{\mathcal{X}_{s,1/2}}^{2}.
	\end{split}
\end{equation}
%
Doing an analogous analysis to the above, we also obtain
\begin{equation}
	\begin{split}
    \|P_{0 \le \tau \le 2n^{2}}\cref{main1-rel-term-4.5}\|_{X_{s,1/2}} \lesssim 
    \| P_{| \tau | \ge 2n^	2} \|_{X_{s,1/2}}^{2} + \| P_{0 \le  \tau  \le 2n^	2} \|_{X_{s,1/2}}^{2} + \| P_{-2n^{2} \le  \tau \le 0} \|_{\mathcal{X}_{s,1/2}}^{2}
	\end{split}
\end{equation}
and
\begin{equation}
	\begin{split}
    \|P_{-2n^{2} \le \tau \le 0}\cref{main1-rel-term-4.5}\|_{X_{s,1/2}} \lesssim 
    \| P_{| \tau | \ge 2n^	2} \|_{X_{s,1/2}}^{2} + \| P_{0 \le  \tau  \le 2n^	2} \|_{X_{s,1/2}}^{2} + \| P_{-2n^{2} \le  \tau \le 0} \|_{\mathcal{X}_{s,1/2}}^{2}
	\end{split}
\end{equation}
%
Collecting these estimates, we obtain 
%
%
\begin{equation*}
\begin{split}
\| \cref{main1-rel-term-4.5} \|_{Z_{s}} \lesssim \| u \|_{Z_{s}}^{2}.
\end{split}
\end{equation*}
%
%
Collecting our estimates for  
\cref{main1-rel-term-1}-\cref{main1-rel-term-5}, we obtain
the following.
%
%
\begin{proposition}
\label{prop:contraction}
%
For $s \ge -1/2$, we have
%
%%
\begin{equation*}
	\begin{split}
    \|Tu\|_{Z_{s}} \le c \left( \|u_0 \|_{H^s(\ci)} + \|u_1 \|_{H^{s-2}(\ci)}
    + \|u\|_{Z_{s}}^2 
		\right).
	\end{split}
\end{equation*}
%
%%
\end{proposition}
%
%
From here, setting up the contraction is standard. Hence, we have reduced the proof of well-posedness to the proof of the bilinear estimates.
%
%
%%%%%%%%%%%%%%%%%%%%%%%%%%%%%%%%%%%%%%%%%%%%%%%%%%%%%
%
%
%                Proof of Bilinear Estimate B4 Per
%
%
%%%%%%%%%%%%%%%%%%%%%%%%%%%%%%%%%%%%%%%%%%%%%%%%%%%%%
%
%
\section{Proof of Periodic Bilinear Estimates} 
\label{sec:proof-bilin-est}
By duality it suffices to show that for $s \ge -1/2$, 
%
%%
\begin{equation}
	\label{duality-est}
	\begin{split}
	|	\sum_{n \in \zzdot}  \langle n \rangle^{s}
	\int_{\rr} \chi_{| \tau | \ge 2n^{2}}\langle \tau - n^{2}  \rangle ^{-1/2}\phi(n, \tau) \wh{uv}(n, \tau) d \tau | \lesssim \|u\|_{\mathcal{X}_{s}}
    \|v\|_{\mathcal{X}_{s}}
    \|\phi \|_{L^{2}(\zzdot \times \rr)}.
	\end{split}
\end{equation}
Note first that $|\wh{uv}(n, \tau) |  = | \wh{u} *  \wh{v} 
(n, \tau)|$. From this it follows that
%
%
\begin{equation}
	\label{non-lin-rep}
	\begin{split}
		| \wh{uv}(n, \tau)|
    & = | \sum_{n_{1}}  \int_{\tau_{1}}
    \wh{u}\left( n_1,  \tau_1 \right) \wh{v}\left( n - n_1 , \tau - \tau_1   
\right) d \tau_1 |
\\
& \le  \sum_{n_{1}}  \int_{\tau_{1}}
    |\wh{u}\left( n_1,  \tau_1 \right)| |\wh{v}\left( n - n_1 , \tau - \tau_1   
\right)| d \tau_1 
\\
& = \sum_{n_{1}} \int_{\tau_{1}} \frac{c_u\left( n_1, \tau_1 
\right)}{\langle n_1 \rangle ^s \langle |\tau_1| - n_1^{2} | \rangle }
\\
& \times \frac{c_{v}\left( n - n_1, \tau - \tau_1 \right)}{\langle n -
n_1 \rangle ^s\ \langle \tau - \tau_{1} -  (n - n_1)^{2} \rangle}
  \ d \tau_1 
\end{split}
\end{equation}
%
%
where for clarity of notation we have introduced 
%
%
%
\begin{equation*}
\begin{split}
\langle k \rangle \doteq 1 + |k|
\end{split}
\end{equation*}
%
%
and
%
\begin{equation*}
	\begin{split}
		c_h(n, \tau) =
		\langle n \rangle ^s \langle \tau - n^{2} \rangle^{-1/2} | \wh{h}\left( n, \tau \right) |.
	\end{split}
\end{equation*}
%
%
From our work above, it follows that 
%
%
\begin{equation}
	\label{convo-est-starting-pnt}
	\begin{split}
		 &  | \wh{uv}\left( 
		n, \tau \right) |
		\\
		& \le  
		\sum_{n_{1}} \int_{\tau_{1}} \frac{\langle n \rangle^{s}}{\langle n_1 \rangle^s
    \langle n - n_1 \rangle^s} 
    \times \frac{c_f(n_1, \tau_1)}{\langle |\tau_1| - n_1^{2} \rangle^{1/2}}
		\\
		& \times
		\frac{c_g(n - n_1, \tau - \tau_1 )}{\langle \tau - \tau_{1} - (n - n_1)
		\rangle^{1/2}}\ d \tau_1.
	\end{split}
\end{equation}
%
%
Hence, 
%
%
\begin{equation}
  \label{pre-fubini-int-form}
	\begin{split}
    |\text{lhs of} \ \cref{duality-est}|
    & \lesssim \sum_{n \in \zzdot} \int_{\tau} \phi(n, \tau)\langle \tau - n^{2} \rangle^{-1/2}  \langle n \rangle^s 
  \sum_{n_{1}}
  \int_{\tau_{1}} c_f(n_1, \tau_1)
		c_g(n - n_1, \tau - \tau_1 )
		\\
    & \times \frac{\langle n \rangle ^{s}}{\langle n_{1} \rangle ^{s} \langle
    n-n_{1} \rangle ^{s}} \times \frac{1}{\langle |\tau_{1}|-n_{1}^{2} \rangle^{1/2}\langle | \tau -
    \tau_{1}|-(n - n_{1})^{2}
    \rangle^{1/2}} d \tau_1 d \tau.
	\end{split}
\end{equation}
%
%
%
Let $A \subset \rr^{2} \times \zzdot \times \zz$, and $\chi_{A}(\tau, \tau_{1}, n, n_{1})$
be its
characteristic function. Then by Cauchy-Schwartz in
$\tau_{1}, \xi_{1}$
\begin{equation*}
	\begin{split}
    & \sum_{n \in \zzdot} \int_{\tau}   \sum_{n_{1}}
    \int_{\tau_{1}} \chi_{A}
    \phi(n, \tau) \langle n \rangle^s \langle \tau - n^{2} \rangle^{-1/2}
  c_f(n_1, \tau_1)
		c_g(n - n_1, \tau - \tau_1 )
		\\
    & \times \frac{\langle n \rangle ^{s}}{\langle n_{1} \rangle ^{s} \langle
    n-n_{1} \rangle ^{s}} \times \frac{1}{\langle |\tau_{1}|-n_{1}^{2} \rangle\langle | \tau -
    \tau_{1}|-(n - n_{1})^{2}
    \rangle} d \tau_1 d \tau.
	\end{split}
\end{equation*}
%
is bounded by 
%
%
\begin{equation}
	\label{10g}
	\begin{split}
    & \sum_{n \in \zzdot} \int_{\tau} \phi(n, \tau) \langle n \rangle ^{s}
    \langle \tau - n^{2} \rangle ^{-1/2}
    \\
    & \times \left( \sum_{n_{1}} \int_{\tau_{1}}
    \frac{\chi_{A}}{\langle n_{1} \rangle ^{2s} \langle n-n_{1} \rangle ^{2s} \langle |
    \tau_{1} | - n_{1}^{2}\rangle^{2}  \langle \tau - \tau_{1} -
    (n - n_{1})^{2} \rangle^{2}} d \tau_{1} \right)^{1/2}
    \\
    & \times \left( \sum_{n_{1}} \int_{\tau_{1}} c_{u}^{2}(n, \tau_{1})
    c_{v}^{2}(n - n_{1}, \tau - \tau_{1}) d \tau_{1} \right)^{1/2} d \tau
  \end{split}
\end{equation}
%
%
Applying Cauchy-Schwartz again, \cref{10g} is bounded by
%
%
\begin{equation*}
  \begin{split}
    & \|\left( \sum_{n_{1}}\int_{\tau_{1}} c_{u}^{2}(n_1, \tau_1)
  c_{v}^{2} (n - n_1, \tau - \tau_{1} ) d \tau_1  \right)^{1/2} \|_{L^{2}(\zzdot \times
		\rr)}
		\\
    & \times  \|\phi(n, \tau)  \langle n
    \rangle ^{s} \langle \tau - n^{2} \rangle ^{-1/2}
		\\
    & \times \left( \sum_{n_{1}} \int_{\tau_{1}} \frac{\chi_{A}}{\langle n_{1}
    \rangle ^{2s} \langle n-n_{1} \rangle ^{2s} \langle | \tau_{1}|-n_{1}^{2}
    \rangle^{2} \langle  |\tau -
    \tau_{1} | -(n - n_{1})^{2}
    \rangle^{2}} d \tau_1 \right)^{1/2} \|_{L^2(\zzdot \times \rr)}
		\\
    & = \|u\|_{X_{s,1/2}} \|v\|_{X_{s,1/2}} \label{holder-term}
     \|\phi(n, \tau) \times \left(  \langle n
     \rangle ^{2s} \langle \tau - n^{2} \rangle^{-2a}  \right .
    \\
    & \times \left . \sum_{n_{1}} \int_{\tau_{1}} \frac{\chi_{A}}{\langle n_{1} \rangle ^{2s} \langle
n-n_{1} \rangle ^{2s}  \langle | \tau_{1}|-n_{1}^{2} \rangle^{2} \langle  |\tau -
    \tau_{1} | -(n - n_{1})^{2}
    \rangle^{2}} d \tau_1 \right)^{1/2} \|_{L^2(\zzdot \times \rr)}.
  \end{split}
\end{equation*}
%
Applying H{\"o}lder, we bound this by 
%
%
\begin{equation}
  \label{integral-bound-1st-form-per}
	\begin{split}
    & \|u\|_{X_{s,1/2}} \|v\|_{X_{s,1/2}} \| \phi \|_{L^{2}_{n, \tau}}
    \|\bigg (  \langle n
    \rangle ^{2s} \langle \tau - n^{2} \rangle ^{-2a}
    \\
    & \times \left. 
    \sum_{n_{1}} \int_{\tau_{1}} \frac{\chi_{A}}{\langle n_{1} \rangle ^{2s} \langle
n-n_{1} \rangle ^{2s} \langle | \tau_{1}|-n_{1}^{2} \rangle^{2} \langle  |\tau -
    \tau_{1} | -(n - n_{1})^{2}
    \rangle^{2}} d \tau_1 \right)^{1/2} \|_{L^\infty_{n \neq 0, \tau}}
	\end{split}
\end{equation}
%
%
We return to the right hand side of \cref{pre-fubini-int-form}.
We seek to bound
\begin{equation*}
\begin{split}
  & \sum_{n \in \zzdot} \int_{\tau}  \sum_{n_{1}}
  \int_{\tau_{1}} \chi_{A} \phi(n, \tau)
    c_f(n_1, \tau_1)
    c_g(n - n_1, \tau - \tau_1 ) \langle \tau - n^{2} \rangle ^{-1/2}
		\\
    & \times \frac{\langle n \rangle ^{s}}{\langle n_{1} \rangle ^{s} \langle
    n-n_{1} \rangle ^{s}} \times \frac{1}{\langle |\tau_{1}| - n_{1}^{2} \rangle
    \langle |\tau - \tau_{1}|-(n - n_{1})^{2} \rangle } d \tau_1 d \tau 
   \end{split}
\end{equation*}
in a different manner than before. First, we apply 
Fubini, then Cauchy-Schwartz in $n_{1}, \tau_{1}$ to obtain the bound
%
%
\begin{equation*}
\begin{split}
  & \left[ \sum_{n_{1}} \int_{\tau_{1}} c_{f}^{2}(n_{1}, \tau_{1}) d \tau_{1}
  \right]^{1/2}
  \\
  & \times \left \{\sum_{n_{1}} \int_{\tau_{1}}   
 \left[
 \sum_{n \in \zzdot} \int_{\tau} \langle \tau - n^{2} \rangle ^{-1/2}
   \frac{\langle n \rangle ^{s}}{\langle n_{1} \rangle ^{s} \langle
   n - n_{1}\rangle ^{s}} \times \frac{\chi_{A} |\phi(n, \tau)| c_{g}(n -
   n_{1}, \tau - \tau_{1})
}{\langle | \tau_{1} | - n_{1}^{2} \rangle \langle | \tau -
  \tau_{1} | - (n - n_{1}^{2}) \rangle} d \tau 
  \right]^{2} d \tau_{1} \right \}^{1/2}
  \\
  & = \| f \|_{X_{s,1/2}}
  \\
  & \times \left \{\sum_{n_{1}} \int_{\tau_{1}}   
 \left[
 \sum_{n \in \zzdot} \int_{\tau} \langle \tau - n^{2} \rangle ^{-1/2}
   \frac{\langle n \rangle ^{s}}{\langle n_{1} \rangle ^{s} \langle
   n - n_{1}\rangle ^{s}} \times \frac{\chi_{A}|\phi(n, \tau)| c_{g}(n -
   n_{1}, \tau - \tau_{1}) 
}{\langle | \tau_{1} | - n_{1}^{2} \rangle \langle | \tau -
  \tau_{1} | - (n - n_{1}^{2}) \rangle} d \tau 
  \right]^{2} d \tau_{1}  \right \}^{1/2}
\end{split}
\end{equation*}
%
Applying Cauchy-Schwartz in $\tau, n$, we bound the last line by 
%
%
\begin{equation*}
\begin{split}
  & \left \{\sum_{n_{1}} \int_{\tau_{1}}   
  \left [ \sum_{n \in \zzdot} \int_{\tau}
  | \phi(n, \tau)|^{2} c_{g}^{2}(n - n_{1}, \tau - \tau_{1}) d \tau  
    \right ] \right . 
   \\
   & \left. \times \left [ \sum_{n \in \zzdot} \int_{\tau} 
\langle \tau - n^{2} \rangle ^{-2a}
     \frac{\langle n \rangle
   ^{2s}}{\langle n_{1} \rangle ^{2s} \langle n - n_{1}\rangle ^{2s}}
   \times \frac{\chi_{A}}{\langle | \tau_{1} |
   - n_{1}^{2} \rangle  \langle \tau - \tau_{1} - (n - n_{1}^{2})
   \rangle} d \tau  \right ] \right \}^{1/2}d \tau_{1} 
\end{split}
\end{equation*}
%
%
which by Holder is bounded by 
%
%
%
\begin{equation}
  \label{integral-bound-2nd-form-per}
\begin{split}
  & \| \sum_{n \in \zzdot} \int_{\rr} 
\langle \tau - n^{2} \rangle ^{-2a}
  \frac{\langle n \rangle ^{2s}}{\langle n_{1} \rangle ^{2s} \langle
  n - n_{1}\rangle ^{2s}}  \times \frac{\chi_{A}}{\langle | \tau_{1} | - n_{1}^{2} \rangle  \langle | \tau -
  \tau_{1} | - (n - n_{1}^{2}) \rangle} d \tau 
  \|_{L^{\infty}_{n_{1}, \tau_{1}}}^{1/2}
  \\
  & \times \|\phi\|_{L^{2}} \| g \|_{X_{s,1/2}}.
\end{split}
\end{equation}
%
%
Now consider the family $\{A_{j}\}_{1}^{k}, A_{j} \subset \rr^{2} \times
\zzdot \times \zz$ with
$$\bigcup_{1}^{k} A_{j}= \rr^{2} \times
\zzdot \times \zz.$$ From 
\cref{integral-bound-2nd-form-per} and our preceding argumentation,
we see that the proof of the bilinear estimate reduces to showing that
either
%%
\begin{equation}
  \label{key-sup-estimate-per-2}
\begin{split}
  & \| \frac{1}{\langle n_{1} \rangle ^{2s}
  } \sum_{n \in \zzdot} \int_{\tau} 
\langle \tau - n^{2} \rangle ^{-2a}
  \frac{\langle n \rangle ^{2s}}{\langle
    n - n_{1}\rangle ^{2s}}  \times \frac{\chi_{A_{j}}}{\langle | \tau_{1} | - n_{1}^{2}
  \rangle  \langle | \tau -
  \tau_{1} | - (n - n_{1}^{2}) \rangle} d \tau 
  \|_{L^{\infty}_{n_{1}, \tau_{1}}}
\end{split}
\end{equation}
%
or
%
%
%
%
\begin{equation}
  \label{key-sup-estimate-per-1}
  \begin{split}
     \|  \langle n
    \rangle ^{2s}
    \sum_{n_{1}}
    \int_{\tau_{1}}\langle \tau - n^{2} \rangle ^{-2a} \frac{\chi_{A_{j}}}{\langle n_{1} \rangle ^{2s} \langle
    n-n_{1} \rangle ^{2s} \langle | \tau_{1}|-n_{1}^{2} \rangle  \langle  |\tau -
    \tau_{1} | -(n - n_{1})^{2}
    \rangle} d \tau_1  \|_{L^\infty_{n \neq 0, \tau}} < \infty.
  \end{split}
\end{equation}
%
%
%
for each $j \in \left\{1,\dots,k \right\}$. 
By the triangle inequality and the fact that 
%
%
\begin{equation*}
\begin{split}
& | \tau | =
\begin{cases}
  - \tau, \quad & \tau < 0, 
\\
\tau, \quad & \tau > 0
\end{cases}
\end{split}
\end{equation*}
%
%
it follows that the proof of \cref{prop:bilinear-est} reduces to showing that
for any $j$, either 
%
%
\begin{equation}
  \label{sup-est-gen-per-1}
  \begin{split}
    \| \langle n
    \rangle ^{2s}
    \sum_{n_{1}} \int_{\tau_{1}} \frac{\chi_{A_{j}}}{\langle n_{1} \rangle ^{2s} \langle n-n_{1} \rangle ^{2s} 
    \langle \sigma \rangle^{2a}   \langle \sigma_{1} \rangle \langle  \sigma_{2}
  \rangle}
    d \tau_1  \|_{L^{\infty}_{n \neq 0, \tau}} < \infty
  \end{split}
\end{equation}
%
%
or 
\begin{equation}
  \label{sup-est-gen-per-2}
\begin{split}
  & \| \frac{1}{\langle n_{1} \rangle ^{2s}
  } \sum_{n \in \zzdot} \int_{\tau} \frac{\langle n \rangle ^{2s}}{\langle
    n - n_{1}\rangle ^{2s}}  \times \frac{\chi_{A_{j}}}{\langle \sigma \rangle^{2a}  \langle
    \sigma_{1} \rangle  \langle \sigma_{2} \rangle} d \tau 
  \|_{L^{\infty}_{n_{1}, \tau_{1}}} < \infty
\end{split}
\end{equation}

%
%
%
where we consider cases
\begin{enumerate}[(I)]
    \item $ \sigma=\tau+n^2,\quad \sigma_1=\tau_1+n_1^2,\quad \sigma_2=\tau -
      \tau_1+(n - n_1)^2$,
\label{it-1}
    \item $ \sigma=\tau-n^2,\quad \sigma_1=\tau_1-n_1^2,\quad \sigma_2=\tau - \tau_1+(n - n_1)^2$,
\label{it-2}
    \item  $\sigma=\tau+n^2,\quad \sigma_1=\tau_1-n_1^2,\quad \sigma_2=\tau - \tau_1+(n - n_1)^2$,
      \label{it-3}
    \item $\sigma=\tau-n^2,\quad \sigma_1=\tau_1+n_1^2,\quad \sigma_2=\tau - \tau_1-(n - n_1)^2$,
\label{it-4}
    \item $\sigma=\tau+n^2,\quad \sigma_1=\tau_1+n_1^2,\quad \sigma_2=\tau - \tau_1-(n - n_1)^2$,
\label{it-5}
    \item $\sigma=\tau-n^2,\quad \sigma_1=\tau_1-n_1^2,\quad \sigma_2=\tau - \tau_1-(n - n_1)^2$.
\label{it-6}
\end{enumerate}
%
%
%
\begin{framed}
\begin{remark}
Note that the cases $\sigma=\tau+n^2,\quad \sigma_1=\tau_1-n_1^2,\quad
\sigma_2=\tau - \tau_1-(n - n_1)^2$ and $\sigma=\tau-n^2,\quad
\sigma_1=\tau_1+n_1^2,\quad \sigma_2=\tau - \tau_1+(n - n_1)^2$ cannot occur, since
$\tau_1< 0, \tau-\tau_1< 0$ implies $\tau<0$ and $\tau_1\geq 0, \tau-\tau_1\geq
0$ implies $\tau\geq 0$. 
\end{remark}
\end{framed}
%
Observe that the transformation $(n, \tau, n_{1}, \tau_{1}) \mapsto -(n, \tau,
n_{1}, \tau_{1})$ reduces \cref{it-3} to \cref{it-4}, \cref{it-2} to
\cref{it-5}, and \cref{it-1} to \cref{it-6}. Furthermore, the change of
variables $\tau_{2} = \tau - \tau_{1}, n_{2} = n - n_{1}$, and the
transformation $(n, \tau, n_{2}, \tau_{2}) \mapsto - (n, \tau, n_{2},
\tau_{2})$ reduces \cref{it-5} to \cref{it-4}. Since $L^{2}$ is invariant
under change of variables and reflections, we may without loss of generality
restrict our attention to cases \cref{it-4} and \cref{it-6}.
 \subsubsection{Case \cref{it-6}} 
\label{ssec:case-it-6}
Let 
%
%
\begin{align*}
  A_1&=\{(n, n_1, \tau, \tau_1)\in A: n = n_{1},  n_{1} \neq 0 \},\\
  A_2&=\{(n, n_1, \tau, \tau_1)\in A: n_1=0, n \neq n_{1} \},\\
  A_3&=\{(n, n_1, \tau, \tau_1)\in A: 0 < | n | <  | n_{1}| \} \\
  A_4&=\{(n, n_1, \tau, \tau_1)\in A: 0 < | n_{1} | <  | n| \}.
\end{align*} 
%
%
%
Following Ginibre, Tsutsumi, Velo~\cite{Ginibre:1997jp}, Kenig, Ponce, Vega~\cite{Kenig:1996yn}, and others,
we now need the following Calculus lemma, whose proof is provided in the
appendix.
%
%%%%%%%%%%%%%%%%%%%%%%%%%%%%%%%%%%%%%%%%%%%%%%%%%%%%%
%
%
%				 Calculus Lemma
%
%
%%%%%%%%%%%%%%%%%%%%%%%%%%%%%%%%%%%%%%%%%%%%%%%%%%%%%
%
%
\begin{lemma}
	\label{lem:calc}
 %
 Fix $p, q > 0$ such that $p +q >1$, and let $r =\min\left\{p, q, p+q-1
 \right\}$. Then 
 %
 \begin{enumerate}[(I)]
   \item
For $\alpha=\beta \ \text{or} \ p \neq 1 \ \text{or} \ q \neq 1$
 \begin{equation*}
\begin{split}
  & \int_{\rr} \frac{1}{\langle x - \alpha \rangle ^{p} \langle x -
  \beta \rangle
  ^{q}} d x
  \le \frac{c_{p,q}}{\langle \alpha - \beta \rangle ^{r}}, 
  \end{split}
\end{equation*}
  \item
    \begin{equation*}
  \int_{\rr} \frac{1}{\langle x - \alpha \rangle  \langle x - \beta
  \rangle} d x
  \le  \frac{4 \log \langle \alpha - \beta \rangle}{\langle \alpha - \beta
  \rangle}, \quad \alpha \neq \beta.
\end{equation*}
\end{enumerate}
\end{lemma}
  %
  %
  %
%
%
%
%
%
%
Applying \cref{lem:calc}, we now bound 
\begin{equation*}
  \begin{split}
    & 
    \sum_{n \in \zzdot} \int_{\tau} \frac{\chi_{A_{1}}\langle n \rangle ^{2s} }{\langle n_{1} \rangle ^{2s} \langle n-n_{1} \rangle ^{2s} \langle \tau - n^{2} \rangle^{2a} 
    \langle \tau_{1} - n_{1}^{2} \rangle \langle  \tau - \tau_{1} -
    (n - n_{1})^{2} \rangle}
    d \tau 
    \\
    & \le 
    \int_{\tau} \frac{1}{\langle \tau -
      n_{1}^{2} \rangle \langle
  \tau - \tau_{1}\rangle}d \tau
  \\
  & \lesssim 
  \langle \tau_{1} - n_{1}^{2} \rangle ^{-1} 
  \\
  & < \infty
\end{split}
\end{equation*}
%
uniformly in $\tau_{1}, n_{1}$. 
%
Similarly, we bound
%
%
\begin{equation}
\begin{split}
  & \langle n
    \rangle ^{2s}
    \sum_{n_{1}} \int_{\tau_{1}} \frac{\chi_{A_{2}}}{\langle n_{1} \rangle ^{2s}
    \langle n-n_{1} \rangle ^{2s} 
    \langle \tau - n^{2} \rangle^{2a} 
    \langle \tau_{1} - n_{1}^{2} \rangle \langle  \tau - \tau_{1} -
    (n - n_{1})^{2} \rangle}
    d \tau_1 
    \\
  & \le 
  \int_{\tau_{1}} \frac{1}{\langle \tau_{1} \rangle  \langle \tau -
  \tau_{1} - n^{2} \rangle}
d \tau_1 
\\
  & \lesssim   \langle \tau - n^{2} \rangle ^{-1} 
  \\
  & < \infty.
	\end{split}
\end{equation}
%

We now observe that
%
%
\begin{equation*}
\begin{split}
  | \tau_{1} - n_{1}^{2} + \tau - \tau_{1} - (n - n_{1})^{2} - (\tau - n^{2}) | = 2| n_{1} || n - n_{1} |
\end{split}
\end{equation*}
%
%
and so by the pigeonhole principle we have $\cup_{1 \le j \le 3} A_{3,j} = A_{3}$, where 
\begin{align*}
  A_{3,1}&=\{(n, n_1, \tau, \tau_1)\in A_3: |\tau-n^{2}|\ge \frac{2}{3} |n_{1}|| n - n_{1} |\},\\
  A_{3,2}&=\{(n, n_1, \tau, \tau_1)\in A_3: |\tau_{1}-n_{1}^2|\ge \frac{2}{3} |n_{1}||n - n_{1}| \}\\
  A_{3,3}&=\{(n, n_1, \tau, \tau_1)\in A_3: |\tau - \tau_{1}-(n - n_{1})^2|\ge \frac{2}{3} |n_{1}||n - n_{1}| \}.
\end{align*} 
%
%
Hence, we obtain
%
%
%
%
%
%
%
%
%
%
%
%
%
%
\begin{equation}
  \label{region-4-1}
\begin{split}
  &  \frac{1}{\langle n_{1} \rangle^{2s}} \sum_{n \in \zzdot} \int_{\tau} \frac{\langle n \rangle ^{2s}}{\langle n - n_{1}\rangle ^{2s}}  \times \frac{\chi_{A_{3,1}}}{\langle \tau - n^{2} \rangle^{2a \pm \eta} \langle
      \tau_{1} - n_{1}^{2} \rangle  \langle \tau - \tau_{1} - (n - n_{1})^{2} \rangle} d \tau
  \\
  &  \lesssim_{\eta} \sum_{n \in \zzdot} \frac{\chi_{A_{3,1}}\langle n - n_{1} \rangle ^{-2s-2a + \eta } \langle n_{1} \rangle ^{-2s-2a + \eta}}{\langle
    n\rangle ^{-2s}}
      \end{split}
\end{equation}
%
uniformly in $\tau$. We now need the following.
%
%
%%%%%%%%%%%%%%%%%%%%%%%%%%%%%%%%%%%%%%%%%%%%%%%%%%%%%
%
%
%				key trick
%
%
%%%%%%%%%%%%%%%%%%%%%%%%%%%%%%%%%%%%%%%%%%%%%%%%%%%%%
%
%
\begin{lemma}
  For integers $n, n_{1} \neq 0$
  %
  %
  \begin{equation*}
  \begin{split}
    | n - n_{1} | \le | n | | n_{1}|.
  \end{split}
  \end{equation*}
  %
  %
  \label{lem:key-trick}
\end{lemma}
%
%
%
%
%
%
%
Adopting the notation
  \begin{equation*}
  \begin{split}
  \gamma(s) = 
  \begin{cases} 0, \quad & s \ge 0
    \\
    4|s|, \quad & s < 0
  \end{cases}
\end{split}
  \end{equation*}
  %
  and applying \cref{lem:key-trick}, and the fact that 
  $0 < | n | < | n_{1} |$ in region $A_{3,1}$, we see that we can bound \eqref{region-4-1} by
  %
  %
  \begin{equation*}
  \begin{split}
    \sum_{n \in \zzdot} \frac{\langle n_{1}\rangle ^{\gamma(s)-4a  + 2 \eta +\ee}}{\langle n \rangle^{2a - \eta + \ee}} < \infty, \quad s \ge -1/2 + \ee/4 + \eta/2, \ a > 1/2 - \ee/2 + \eta/2
  \end{split}
  \end{equation*}
  %
  for $\ee, \eta > 0$, uniformly in $n_{1}$.
  %
  For region $A_{3,2}$, we apply \cref{lem:calc} and \cref{lem:key-trick} to obtain
  \begin{equation}
\begin{split}
  &  \frac{1}{\langle n_{1} \rangle^{2s}} \sum_{n \in \zzdot} \int_{\tau} \frac{\langle n \rangle ^{2s}}{\langle n - n_{1}\rangle ^{2s}}  \times \frac{\chi_{A_{3,1}}}{\langle \tau - n^{2} \rangle^{2a} \langle
      \tau_{1} - n_{1}^{2} \rangle  \langle \tau - \tau_{1} - (n - n_{1})^{2} \rangle} d \tau
      \\
  &  \lesssim \sum_{n \in \zzdot} \frac{\langle n - n_{1} \rangle ^{-2s-1} \langle n_{1} \rangle ^{-2s-1}}{\langle
    n\rangle ^{-2s}}
      \end{split}
\end{equation}
%
uniformly in $n_{1}$, $\tau_{1}$. The remainder of the argument is analogous to that of region $A_{3,1}$. For region $A_{3,3}$, we 
apply \cref{lem:calc} and \cref{lem:key-trick} to obtain
  \begin{equation}
\begin{split}
  &  \frac{1}{\langle n_{1} \rangle^{2s}} \sum_{n \in \zzdot} \int_{\tau} \frac{\langle n \rangle ^{2s}}{\langle n - n_{1}\rangle ^{2s}}  \times \frac{\chi_{A_{3,3}}}{\langle \tau - n^{2} \rangle^{2a} \langle
  \tau_{1} - n_{1}^{2} \rangle  \langle \tau - \tau_{1} - (n - n_{1})^{2} \rangle^{1 \pm \eta}} d \tau
      \\
      &  \lesssim \sum_{n \in \zzdot} \frac{\chi_{A_{3,3}}\langle n - n_{1} \rangle ^{-2s-1 + \eta} \langle n_{1} \rangle ^{-2s-1 + \eta}}{\langle
    n\rangle ^{-2s}}
      \end{split}
\end{equation}
%
and the remainder of the argument is analogous to that of region $A_{3,1}$.
  %
To handle the region $A_{4}$, we apply the pigeonholde principle again and partition it into the pieces

\begin{align*}
  A_{4,1}&=\{(n, n_1, \tau, \tau_1)\in A_4: |\tau-n^{2}|\ge \frac{2}{3} |n_{1}|| n - n_{1} |\},\\
  A_{4,2}&=\{(n, n_1, \tau, \tau_1)\in A_4: |\tau_{1}-n_{1}^2|\ge \frac{2}{3} |n_{1}||n - n_{1}| \}\\
  A_{4,3}&=\{(n, n_1, \tau, \tau_1)\in A_4: |\tau - \tau_{1}-(n - n_{1})^2|\ge \frac{2}{3} |n_{1}||n - n_{1}| \}.
\end{align*} 
  %
Hence, applying \cref{lem:calc}, we obtain 
%
%
%
%
%
%
%
%
%
%
%
%
%
%
\begin{equation*}
  \begin{split}
    & \langle n
    \rangle ^{2s}
    \sum_{n_{1}} \int_{\tau_{1}} \frac{\chi_{A_{4,1}}}{\langle n_{1} \rangle ^{2s} \langle n-n_{1} \rangle ^{2s} \langle \tau - n^{2} \rangle^{2a} 
    \langle \tau_{1} - n_{1}^{2} \rangle \langle  \tau - \tau_{1} -
    (n - n_{1})^{2} \rangle}
    d \tau_1 
    \\
    & \lesssim
    \sum_{n_{1} \in \zz} \frac{\langle n - n_{1} \rangle ^{-2s-2a} \langle n_{1}\rangle ^{-2s -2a}}{\langle
      n\rangle ^{-2s}} 
\end{split}
\end{equation*}
%
uniformly in $\tau$. However, since $0 <  | n_{1} | < | n |$ in region $A_{4}$,we bound this by
%
%
%
\begin{equation*}
\begin{split}
  & \sum_{n_{1} \in \zz} \frac{\chi_{A_{4,1}}}{| n - n_{1}| ^{2s+2a}
   |n_{1}|^{2a}} 
   \\
   & =  \sum_{n_{1} \in \zz} \frac{\chi_{A_{4,1}}}{| n - n_{1}| ^{\ee}
   |n_{1}|^{2a}}, \quad s \ge -1/2 + \ee 
   \\
   & \lesssim \sum_{n_{1} \in \zz} \frac{\chi_{A_{4,1}}}{|n_{1}|^{2a + \ee}} < \infty \quad a > 1/2 - \ee. 
 \end{split}
\end{equation*}
%
%
For region $A_{4,2}$, we apply \cref{lem:calc}, \cref{lem:key-trick}, and the fact that $0 < | n_{1} | < | n |$ in $A_{4}$ to obtain
  \begin{equation}
\begin{split}
  &  \frac{1}{\langle n_{1} \rangle^{2s}} \sum_{n \in \zzdot} \int_{\tau} \frac{\langle n \rangle ^{2s}}{\langle n - n_{1}\rangle ^{2s}}  \times \frac{\chi_{A_{4,2}}}{\langle \tau - n^{2} \rangle^{2a} \langle
      \tau_{1} - n_{1}^{2} \rangle  \langle \tau - \tau_{1} - (n - n_{1})^{2} \rangle} d \tau
      \\
      &  \lesssim \sum_{n \in \zzdot} \frac{\chi_{A_{4,2}}\langle n - n_{1} \rangle ^{-2s-1} \langle n_{1} \rangle ^{-2s-1}}{\langle
    n\rangle ^{-2s}}
    \\
    & \lesssim \sum_{n \in \zzdot} \frac{\chi_{A_{4,2}}}{\langle n \rangle \langle n - n_{1} \rangle^{2s + 1}}, \quad s \ge -1/2,
    \\
    & \lesssim \sum_{n \in \zzdot} \frac{\chi_{A_{4,2}}}{\langle n \rangle^{1 + \ee}}, \quad s \ge -1/2 + \ee
      \end{split}
\end{equation}
%
uniformly in $n_{1}$, $\tau_{1}$. For region $A_{4,3}$, we apply \cref{lem:calc}, \cref{lem:key-trick}, and the fact that $0 < | n_{1} | < | n |$ in $A_{4}$ to obtain
  \begin{equation}
\begin{split}
  &  \frac{1}{\langle n_{1} \rangle^{2s}} \sum_{n \in \zzdot} \int_{\tau} \frac{\langle n \rangle ^{2s}}{\langle n - n_{1}\rangle ^{2s}}  \times \frac{\chi_{A_{4,3}}}{\langle \tau - n^{2} \rangle^{2a} \langle
  \tau_{1} - n_{1}^{2} \rangle  \langle \tau - \tau_{1} - (n - n_{1})^{2} \rangle^{1 + \pm \eta}} d \tau
      \\
      &  \lesssim \sum_{n \in \zzdot} \frac{\chi_{A_{4,3}}\langle n - n_{1} \rangle ^{-2s-1 + \eta} \langle n_{1} \rangle ^{-2s-1 + \eta}}{\langle
    n\rangle ^{-2s}}
      \end{split}
\end{equation}
%
and the remainder of the argument is analogous to that of region $A_{4,2}$.
\subsubsection{Case \cref{it-4}} 
\label{ssec:case-it-4}
Let 
%
%
\begin{align*}
  B_1&=\{(n, n_1, \tau, \tau_1)\in B: n = n_{1},  n_{1} \neq 0 \},\\
  B_2&=\{(n, n_1, \tau, \tau_1)\in B: n_1=0, n \neq n_{1} \},\\
  B_3&=\{(n, n_1, \tau, \tau_1)\in B: 0 < | n | <  | n_{1}| \} \\
  B_4&=\{(n, n_1, \tau, \tau_1)\in B: 0 < | n_{1} | <  | n| \}.
\end{align*} 
%
%
For region $B_{1}$, we apply \cref{lem:calc} to obtain 
\begin{equation*}
  \begin{split}
    & \langle n
    \rangle ^{2s}
    \sum_{n_{1}} \int_{\tau_{1}} \frac{\chi_{B_{1}}}{\langle n_{1} \rangle ^{2s} \langle n-n_{1} \rangle ^{2s} \langle \tau - n^{2} \rangle^{2a} 
    \langle \tau_{1} + n_{1}^{2} \rangle \langle  \tau - \tau_{1} -
    (n - n_{1})^{2} \rangle}
    d \tau_1 
    \\
    & \le
   \int_{\tau_{1}} \frac{1}{\langle \tau_{1} +
  n^{2} \rangle \langle
  \tau - \tau_{1}\rangle}d \tau_{1}
  \\
  & \lesssim 
  \langle \tau - n^{2} \rangle ^{-2} 
  \\
  & < \infty.
\end{split}
\end{equation*}
We also apply \cref{lem:calc} in $B_{2}$ to obtain
\begin{equation*}
  \begin{split}
    & \langle n
    \rangle ^{2s}
    \sum_{n_{1}} \int_{\tau_{1}} \frac{\chi_{B_{2}}}{\langle n_{1} \rangle ^{2s} \langle n-n_{1} \rangle ^{2s} 
    \langle \tau - n^{2} \rangle^{2a}   \langle \tau_{1} + n_{1}^{2} \rangle \langle  \tau - \tau_{1} -
    (n - n_{1})^{2} \rangle}
    d \tau_1 
    \\
    & = 
   \int_{\tau_{1}} \frac{1}{\langle \tau_{1} 
  \rangle \langle
  \tau - \tau_{1} - n^{2}\rangle}d \tau_{1}
  \\
  & \lesssim 
  \langle \tau - n^{2} \rangle ^{-2} 
  \\
  & < \infty.
\end{split}
\end{equation*}
%
We now observe that
%
%
\begin{equation*}
\begin{split}
  | \tau_{1} + n_{1}^{2} + \tau - \tau_{1} - (n - n_{1})^{2} - (\tau - n^{2}) | = 2| n || n_{1}|
\end{split}
\end{equation*}

By the pigeonhole principle, we have $\cup_{1 \le j \le 3} B_{3,j} = B_{3}$, where 
\begin{align*}
  B_{3,1}&=\{(n, n_1, \tau, \tau_1)\in B_3: |\tau-n^{2}|\ge \frac{2}{3} |n|| n_{1} |\},\\
  B_{3,2}&=\{(n, n_1, \tau, \tau_1)\in B_3: |\tau_{1}+n_{1}^2|\ge \frac{2}{3} |n||n_{1}| \}\\
  B_{3,3}&=\{(n, n_1, \tau, \tau_1)\in B_3: |\tau - \tau_{1}-(n - n_{1})^2|\ge \frac{2}{3} |n||n_{1}| \}.
\end{align*} 
%
%
Hence, for region $B_{3,1}$ we apply the pidgeonhole smoothing and \cref{lem:calc} to obtain 
\begin{equation}
\begin{split}
  &  \frac{1}{\langle n_{1} \rangle^{2s}} \sum_{n \in \zzdot} \int_{\tau} \frac{\langle n \rangle ^{2s}}{\langle n - n_{1}\rangle ^{2s}}  \times \frac{\chi_{B_{3,1}}}{\langle \tau - n^{2} \rangle^{2a} \langle
      \tau_{1} - n_{1}^{2} \rangle  \langle \tau - \tau_{1} - (n - n_{1})^{2} \rangle} d \tau
  \\
  &  \lesssim_{\eta} \sum_{n \in \zzdot} \frac{\chi_{B_{3,1}}\langle n - n_{1} \rangle ^{-2s } \langle n_{1} \rangle ^{-2s-2a + \eta}}{\langle
    n\rangle ^{-2s +2a-\eta}}
      \end{split}
\end{equation}
%
uniformly in $\tau$. Applying \cref{lem:key-trick}, and the fact that 
$0 < | n | < | n_{1} |$ in region $B_{3,1}$, we bound this by
%
%
\begin{equation*}
\begin{split}
\sum_{n \in \zzdot} \frac{\chi_{B_{3,1}}\langle n_{1} \rangle ^{\gamma(s)-2a + \eta - \ee}}{\langle
    n\rangle ^{2a-\eta  + \ee}} < \infty, \quad s \ge -1/2 + \ee/4 + \eta/2, \ a > 1/2 - \ee/2 + \eta/2.
\end{split}
\end{equation*}
%
%
For region $B_{3,2}$, we apply the pidgeonhole smoothing and \cref{lem:calc} to obtain 
\begin{equation}
\begin{split}
  &  \frac{1}{\langle n_{1} \rangle^{2s}} \sum_{n \in \zzdot} \int_{\tau} \frac{\langle n \rangle ^{2s}}{\langle n - n_{1}\rangle ^{2s}}  \times \frac{\chi_{B_{3,2}}}{\langle \tau - n^{2} \rangle^{2a} \langle
      \tau_{1} + n_{1}^{2} \rangle  \langle \tau - \tau_{1} - (n - n_{1})^{2} \rangle} d \tau
      \\
      & \lesssim \sum_{n \in \zzdot} \frac{\chi_{B_{3,1}}\langle n - n_{1} \rangle ^{-2s } \langle n_{1} \rangle ^{-2s-2a}}{\langle
    n\rangle ^{-2s +2a}}
  \end{split}
\end{equation}
%
and the rest of the argument is identical to that of region $B_{3,1}$.
\begin{framed}
  Suppose we want to be more clever and apply the KPV summation lemma here. Assume $a=1/2$, the endpoint case (and the ``natural'' value of $a$ for attaining the best well-posedness result). Integrating in $\tau$, applying \cref{lem:calc}, and using the smoothing in region $B_{3,2}$ via the pigeonhole principle, we get the bound (modulo a constant)
%
%
\begin{equation*}
\begin{split}
  \frac{1}{\langle n_{1} \rangle^{2s+1}} \sum_{n \in \zzdot} \frac{\langle n \rangle ^{2s-1}}{\langle n - n_{1}\rangle ^{2s}}  \times \frac{\chi_{B_{3,2}}}{\langle \tau_{1} + 2nn_{1} - n_{1}^{2} \rangle} 
\end{split}
\end{equation*}
%
%
If $s \ge -1/2$, we bound this by
\begin{equation*}
\begin{split}
  c_{\ee}\sum_{n \in \zzdot} \frac{\chi_{B_{3,2}}}{\langle \tau_{1} - n_{1}(  n_{1} - 2n) \rangle^{1-\ee}}
\end{split}
\end{equation*}
which is unbounded. Why has the KPV method failed us? The answer lies in the curvature of the graph of $\tau_{1} = \pm n_{1}^{2}$ versus that of the graph of $\tau_{1} = n_{1}^{3}$ (KDV). For example, if we were to estimate the KDV analogue of the above%
%
\begin{equation*}
\begin{split}
      \frac{1}{\langle n_{1} \rangle^{2s}} \sum_{n \in \zzdot} \int_{\tau} \frac{\langle n \rangle ^{2s}}{\langle n - n_{1}\rangle ^{2s}}  \times \frac{\chi_{B_{3,2}}}{\langle \tau - n^{3} \rangle^{2a} \langle
      \tau_{1} - n_{1}^{3} \rangle  \langle \tau + \tau_{1} - (n - n_{1})^{3} \rangle} d \tau
\end{split}
\end{equation*}
%
%
for $s \ge -1/2$, we get
%
%
\begin{equation*}
\begin{split}
  c_{\ee}\sum_{n \in \zzdot} \frac{\chi_{B_{3,2}}}{\langle \tau_{1} - nn_{1}(n - n_{1})\rangle^{1-\ee}} < \infty,
\end{split}
\end{equation*}
%
(the denominator ``behaves'' like $n^{2}$, see Lemma 5.1 in \cite{Kenig:1996yn}).
%
\end{framed}

For region $B_{3,3}$, we apply the pidgeonhole smoothing and \cref{lem:calc} to obtain 
\begin{equation}
\begin{split}
  &  \frac{1}{\langle n_{1} \rangle^{2s}} \sum_{n \in \zzdot} \int_{\tau} \frac{\langle n \rangle ^{2s}}{\langle n - n_{1}\rangle ^{2s}}  \times \frac{\chi_{B_{3,2}}}{\langle \tau - n^{2} \rangle^{2a} \langle
  \tau_{1} + n_{1}^{2} \rangle  \langle \tau - \tau_{1} - (n - n_{1})^{2} \rangle^{1 \pm \eta}} d \tau
      \\
      & \lesssim_{\eta} \sum_{n \in \zzdot} \frac{\chi_{B_{3,1}}\langle n - n_{1} \rangle ^{-2s } \langle n_{1} \rangle ^{-2s-1 + \eta}}{\langle
    n\rangle ^{-2s +1 - \eta}}
  \end{split}
\end{equation}
%
and the rest of the argument is analogous to that of region $B_{3,1}$.

The arguments for the regions $B_{3,3}$ and $B_{4}$ are analogous to those of regions $A_{3,3}$ and $A_{4}$, respectively. This completes the proof of
\cref{prop:bilinear-est}. \qquad \qedsymbol
%
%
%
%%%%%%%%%%%%%%%%%%%%%%%%%%%%%%%%%%%%%%%%%%%%%%%%%%%%%
%
%
%                Proof of miscellaneous lemmas
%
%
%%%%%%%%%%%%%%%%%%%%%%%%%%%%%%%%%%%%%%%%%%%%%%%%%%%%%
%
\appendix
%
\section{Proofs of Lemmas and Estimates} 
\label{sec:pfs-lems-est}
%
%
%
\begin{proof}[Proof of \cref{lem:calc}]
%
By the change of variable $x \mapsto x + (\alpha + \beta)/2$, we have
%
%
\begin{equation*}
	\begin{split}
    \int_{\rr} \frac{1}{\langle x - \alpha \rangle^{p} \langle  x -
    \beta
    \rangle^{q}}d x.
    & = \int_{\rr} \frac{1}{\langle x - (\alpha - \beta)/2  \rangle^{p}
    \langle  x + (\alpha - \beta)/2 \rangle^{q}} d x.
	\end{split}
\end{equation*}
%
%
Hence, we seek to bound
%
%
%
\begin{equation*}
\begin{split}
  \int_{\rr} \frac{1}{\langle a - x \rangle ^{p} \langle a + x \rangle
  ^{q}} d x
\end{split}
\end{equation*}
%
which for $a =0$ reduces to 
%
%
\begin{equation*}
\begin{split}
  \int_{\rr} \frac{1}{\langle x \rangle ^{p+q}} d x 
  & = 2 \int_{0}^{\infty} \frac{1}{(1 + x)^{p+q}} d x
  \\
  & = \frac{2}{p+q -1}
  \\
  & = \frac{2}{(p+q -1)\langle a \rangle}.
\end{split}
\end{equation*}
%
%
By symmetry, we now assume without loss of generality that $a > 0$ and split 
%
%
\begin{equation*}
\begin{split}
\int_{\rr} \frac{1}{\langle a + x \rangle ^{p} \langle a - x \rangle
  ^{q}} d x
  & = \int_{-2a}^{2a}
  \frac{1}{\langle a + x \rangle ^{p} \langle a - x \rangle
  ^{q}} d x
  \\
  & + \int_{| x | \ge 2a} 
\frac{1}{\langle a + x \rangle ^{p} \langle a - x \rangle
  ^{q}} d x
  \\
  & = I + II.
\end{split}
\end{equation*}
%
%
If $p=1$ and $q=1$, then 
%
%
\begin{equation*}
\begin{split}
  I
  & \le \sup_{-2a \le x \le 2a} \frac{1}{\langle a - x \rangle
} \int_{-2a}^{2a} \frac{1}{\langle a + x \rangle} d x
  \\
  & = \frac{1}{\langle a \rangle} \int_{-2a}^{2a} \frac{1}{(1 + | a -
  x
  |)} d x
  \\
  & = \frac{4}{\langle a \rangle} \int_{0}^{a} \frac{1}{(1 + a -
  x)} d x.
\end{split}
\end{equation*}
%
%
Integrating, we obtain
%
%
\begin{equation*}
 I
 \le 
 \frac{4 \log \langle a \rangle}{\langle a \rangle}, \qquad p =1, \ q =1.
\end{equation*}
Otherwise, assume that $q \neq 1$. Then
\begin{equation*}
\begin{split}
  I
  & \le \sup_{-2a \le x \le 2a} \frac{1}{\langle a + x \rangle
  ^{p}} \int_{-2a}^{2a} \frac{1}{\langle a - x \rangle ^{q}} d x
  \\
  & = \frac{1}{\langle a \rangle ^{p}} \int_{-2a}^{2a} \frac{1}{(1 + | a -
  x
  |)^{q}} d x
  \\
  & = \frac{4}{\langle a \rangle ^{p}} \int_{0}^{a} \frac{1}{(1 + a -
  x)^{q}} d x.
\end{split}
\end{equation*}
Evaluating the integral, we obtain
\begin{equation*}
  I \le \frac{4}{|q-1| \langle a \rangle ^{p +q -1}}, \qquad q \neq 1.
\end{equation*}
%
%
A similar computation yields
\begin{equation*}
  I \le \frac{4}{|q-1| \langle a \rangle ^{p +q -1}}, \qquad p \neq 1.
\end{equation*}
%
%
Also
%
%
\begin{equation*}
\begin{split}
  II 
  & \simeq \int_{x \ge 2a} \frac{1}{\langle a - x \rangle ^{p} \langle a
  + x \rangle ^{q} d x}
  \\ 
  & = \int_{x \ge 2a} \frac{1}{(1 + x - a)^{p} (1 + x +
  a)^{q}} d x
  \\
  & \le \int_{x \ge 2a} \frac{1}{(1 + x -a)^{p+q}} d x
  \\
  & = \frac{1}{[(p + q)-1] \langle a \rangle ^{p+q -1}}, \qquad p + q > 1.
\end{split}
\end{equation*}
%
%
Collecting our estimates for $I$ and $II$ we see that for 
$p, q > 0$ such that $p +q >1$, and $r =\min\left\{p, q, p+q-1
 \right\}$, we have 
%
\begin{align*}
  & \int_{\rr} \frac{1}{\langle a - x \rangle ^{p} \langle a + x \rangle
  ^{q}} d x
  \le \frac{c_{p,q}}{\langle a \rangle ^{r}}, \qquad a = 0 \ \text{or} \
  p \neq 1 \ \text{or} \ q \neq 1
  \\
  & \int_{\rr} \frac{1}{\langle a - x \rangle  \langle a + x \rangle
} d x
  \le  \frac{4 \log \langle a \rangle}{\langle a \rangle}, \qquad a > 0.
  \label{est-2}
\end{align*}
By symmetry, the second inequality also holds for $a < 0$. Substituting $(\alpha -
\beta)$ for $a$ completes the proof.
\end{proof}
%
%
%
%
%
%
%
%%%%%%%%%%%%%%%%%%%%%%%%%%%%%%%%%%%%%%%%%%%%%%%%%%%%%
%
%
%                Scaling
%
%
%%%%%%%%%%%%%%%%%%%%%%%%%%%%%%%%%%%%%%%%%%%%%%%%%%%%%
%
%
\section{Scaling} 
\label{sec:scaling}
Our object of investigation is the initial value problem for the
periodic/non-periodic $1$-D ``good'' B4 equation, i.e.,
\begin{equation}
  \aligned
  & u_{tt}-u_{xx}+u_{xxxx}+ (u^2)_{xx}\,=\,0, \quad x\in \mathbb{T}\ \text{or} \ \mathbb{R}, \quad t>0,\\
&u(0,x)\,=\,u_0(x),\qquad u_t(0,x)\,=\,u_1(x).\endaligned
\label{main}
\end{equation}
Due to the fact that the leading terms in the linear operator above are $u_{tt}$ and $u_{xxxx}$, morally speaking, one derivative in time is like two derivatives in space. This is why the Sobolev regularity scale for the initial data should be as follows:
\[
u_0\in H^s(\mathbb{T}\ \text{or} \ \mathbb{R}), \qquad u_1\in H^{s-2}(\mathbb{T}\ \text{or} \ \mathbb{R})
\]
Results for these problems are usually formulated with $u_0=\phi \in H^s$ and $u_1=\psi_x$, $\psi\in H^{s-1}$.
Current state of the art in terms of local well-posedness/ill-posedness for the two problems is:
\begin{itemize}
  \item LWP for both problems when $s>-\frac 14$ (Farah '09, Farah-Scialom '10), with iteration done in
    the norm
    \[
    \|F\|_{X^{s}}\,=\,\|<\tau-\sqrt{\xi^2+\xi^4}>^b\,<\xi>^s \tilde{F}\|_{L^2_{\tau,\xi}};
    \]
  \item main IP result is for the non-periodic problem when $s<-2$, as the solution map 
    \[
    S: H^s\times H^{s-1} \to C([0,\delta]; H^s), \quad
    S(\phi,\psi)\,=\,u
    \]
    is not $C^2$ at zero (Farah '09);
  \item also, for the non-periodic problem, one can not find a space in which to run a contraction argument based on treating the nonlinearity as bilinear for $s<-2$ (see Theorem 1.4 in Farah '09);
  \item finally, and really puzzling\footnote{for other dispersive equations (e.g., KdV, Schrodinger), there is usually a gap of $\frac 14$ between regularities for the two problems}, the crucial bilinear estimate (equation (5) in both papers) fails basically at the same threshold for both problems: $s\leq -\frac 14$ (non-periodic), $s<-\frac{1}{4}$ (periodic).
\end{itemize}
The equation does not have an associated scaling, however one can do a formal scaling analysis by ignoring one of the two linear terms containing spatial derivatives:
\begin{itemize}
  \item for 
    \[
    u_{tt}+u_{xxxx}+(u^2)_{xx}\,=\,0,
    \]
    one has 
    \[
    u_{\lambda}(t,x)\,=\,\frac{1}{\lambda^2}u\left(\frac{t}{\lambda^2}, \frac{x}{\lambda}\right),
    \]
    which leads to $s_c=-\frac 32$.
\end{itemize}
\begin{proof}
Let $u(x, t)$ be a solution to the B4 equation, that is
%
$$
B_3(u)=
 \partial_t^2u + \partial^4_x u + \partial_x^2(u^2)  = 0
$$
%
We would like to find the constants
$a, b, c$ such that
\[
u_\lambda (x, t) = \lambda^a u(\lambda^b x, \lambda^c t)
\]
is also a solution to B4.  Since 
$$
B_3(u_\lambda)=
\lambda^{a+2c} \partial_t^2u 
+
 \lambda^{a+4b} \partial^4_x u 
 +
  \lambda^{2a+2b}
  \partial_x^2(u^2),  
$$
we see that $u_\lambda$ is a B4 solution only if
$$
a+2c=a+4b=2a+2b,
$$
or
$
c= 2b =a.
$
  Thus
\[
u_\lambda (x, t) = \lambda^{2b} u(\lambda^{b}x,  \lambda^{2b} t).
\]
%
%
Therefore, replacing  $\lambda^b$ with  $ \lambda$ gives the following scaling:
%
\begin{equation}
\label{DP-scal}
\boxed{u(x, t) \ \text{solution to} \  B_3
 \Longrightarrow 
u_\lambda (x, t) = \lambda^2 u(\lambda x, \lambda^2 t)  \ \text{is also a
solution to} \  B_3.}
\end{equation}
\label{rem:scaling}
To find the critical Sobolev index, we compute
%
%
\begin{equation}
\begin{split}
  \| u_{\lambda} \|_{\dot{H}^s(\ci)} 
  & = \lambda^{2} \| u(\lambda x, \lambda^2 t) \|_{\dot{H}^{s}(\ci)}
  \\
  & = \lambda^{2} \left( \int_{\rr} | \xi |^{2s} | \wh{u (\lambda x,
  \lambda^{2} t)}^x (\xi, t)| \right)^{1/2}.
\end{split}
\label{crit-ind-comp}
\end{equation}
%
But
%
%
\begin{equation*}
\begin{split}
  \wh{u(\lambda x, \lambda^{2}t)^x}(\xi, t)
  & = \int_{\rr}e^{-i\xi x}u(\lambda x, \lambda^2 t) dx
  \\
  & = \frac{1}{\lambda} \int_{\rr}e^{-i \frac{n}{\lambda} x'}u(x',
  \lambda^{2} t) dx'
  \\
  & = \frac{1}{\lambda} \wh{u(\cdot, \lambda^{2}t)}(\frac{\xi}{\lambda})
\end{split}
\end{equation*}
%
%
Substituting back into \cref{crit-ind-comp}, we obtain
%
%
\begin{equation*}
\begin{split}
  \| u_{\lambda} \|_{\dot{H}^s(\rr)} 
  & = \lambda^{2} \left( \int_{\rr} | \xi |^{2s} |
  \frac{1}{\lambda}\wh{u(\cdot, \lambda^{2}t)}(\frac{\xi}{\lambda}) |^2 d \xi
  \right)^{1/2}
  \\
  & = \lambda \left( \int_{\rr}| \xi |^{2s} | \wh{u(\cdot,
  \lambda^{2}t)}(\frac{\xi}{\lambda}) |^2 d \xi  \right)^{1/2}
  \\
  & = \lambda \left( \int_{\rr} | \lambda \xi' |^{2s} 
  \wh{u(\cdot, \lambda^{2}t)}(\xi') |^2 \lambda d \xi
  \right)^{1/2}
  \\
  & = \lambda^{s + 3/2} \|u(\cdot, \lambda^{2}t) \|_{\dot{H}^s (\ci)}.
\end{split}
\end{equation*}
%
%
Therefore, $\| u_{\lambda}(0) \|_{\dot{H}^s(\rr)} = \lambda^{s + 3/2} \|
u_{0} \|_{\dot{H}^{s}(\rr)}$. Hence, $s=-3/2$ is the critical Sobolev index.
\end{proof}
%
%
\begin{framed}
\begin{remark}
Since the scaling conserves data in $\dot{H}^{-3/2}$, it
seems that this equation is ``like KdV''.
So one may expect KdV type theorems.
That is, $s_c=-3/4$ on the line and $s_c=-1/2$ on the circle,
if one uses bilinear estimates.
But, Kappeler and collaborators went all the way to $-1$ for KdV.
However KdV is integrable. Is this equation integrable?
Also, people conjecture that the critical index for KdV well-posedness 
in some appropriate sense should be the scaling index which is  $-3/2$.
\label{rem:kdv-like}
\end{remark}
\end{framed}
%
%
\begin{itemize}
  \item for 
    \[
    u_{tt}-u_{xx}+(u^2)_{xx}\,=\,0,
    \]
    one has 
    \[
    u_{\lambda}(t,x)\,=\,u\left(\frac{t}{\lambda}, \frac{x}{\lambda}\right),
    \]
    which leads to $s_c=\frac 12$.
\end{itemize}
\begin{proof}
Let $u(x, t)$ be a solution to the $B_1$ equation, that is
%
$$
B_1(u)=
 \partial_t^2u - \partial^2_x u + \partial_x^2(u^2)  = 0
$$
%
We would like to find the constants
$a, b, c$ such that
\[
u_\lambda (x, t) = \lambda^a u(\lambda^b x, \lambda^c t)
\]
is also a solution to $B_1$.  Since 
$$
B_1(u_\lambda)=
\lambda^{a+2c} \partial_t^2u 
-
 \lambda^{a+2b} \partial^2_x u 
 +
  \lambda^{2a+2b}
  \partial_x^2(u^2),  
$$
we see that $u_\lambda$ is a $B_1$ solution only if
$$
a+2c=a+2b=2a+2b,
$$
or
$
a=0, b=c.
$
  Thus
\[
u_\lambda (x, t) = u(\lambda^{b}x,  \lambda^{b} t).
\]
%
%
Therefore, replacing  $ \lambda^b$ with  $ \lambda$ gives the following scaling:
%
\begin{equation}
\label{B2-scal}
\boxed{u(x, t) \ \text{solution to} \  B_1
 \Longrightarrow 
u_\lambda (x, t) = u(\lambda x, \lambda t) \ \text{is also a
solution to} \  B_1. 
}
\end{equation}
\label{rem:scaling-B2}
To find the critical Sobolev index, we compute
%
%
\begin{equation}
\begin{split}
  \| u_{\lambda} \|_{\dot{H}^s(\ci)} 
  & =  \| u(\lambda x, \lambda t) \|_{\dot{H}^{s}(\ci)}
  \\
  & = \left( \int_{\rr} | \xi |^{2s} | \wh{u(\lambda x,
  \lambda t)}^x (\xi, t)| \right)^{1/2}.
\end{split}
\label{crit-ind-comp-B2}
\end{equation}
%
But
%
%
\begin{equation*}
\begin{split}
  \wh{u(\lambda x, \lambda t)^x}(\xi, t)
  & = \int_{\rr}e^{-i\xi x}u(\lambda x, \lambda t) dx
  \\
  & = \frac{1}{\lambda} \int_{\rr}e^{-i \frac{n}{\lambda} x'}u(x',
  \lambda t) dx'
  \\
  & = \frac{1}{\lambda} \wh{u(\cdot, \lambda t)}(\frac{\xi}{\lambda})
\end{split}
\end{equation*}
%
%
Substituting back into \cref{crit-ind-comp-B2}, we obtain
%
%
\begin{equation*}
\begin{split}
  \| u_{\lambda} \|_{\dot{H}^s(\rr)} 
  & = \left( \int_{\rr} | \xi |^{2s} |
  \frac{1}{\lambda}\wh{u(\cdot, \lambda t)}(\frac{\xi}{\lambda}) |^2 d \xi
  \right)^{1/2}
  \\
  & = \frac{1}{\lambda} \left( \int_{\rr}| \xi |^{2s} | \wh{u(\cdot,
  \lambda t)}(\frac{\xi}{\lambda}) |^2 d \xi  \right)^{1/2}
  \\
  & = \frac{1}{\lambda} \left( \int_{\rr} | \lambda \xi' |^{2s} 
  \wh{u(\cdot, \lambda)}(\xi') |^2 \lambda d \xi
  \right)^{1/2}
  \\
  & = \lambda^{s - 1/2} \|u(\cdot, t) \|_{\dot{H}^s (\rr)}.
\end{split}
\end{equation*}
%
%
Therefore, $\| u_{\lambda(0)} \|_{\dot{H}^s(\rr)} = \lambda^{s - 1/2} \|
u_{0} \|_{\dot{H}^{s}(\rr)}$, and so $s=1/2$ is the critical Sobolev index.
\end{proof}
This might suggest that the current results are not optimal.
%
%
\bibliography{/Users/davidkarapetyan/math/bib-files/references}
%
%\nocite{*}
\end{document}
