
\documentclass{amsart}
\usepackage{showkeys}
\usepackage{framed}
\synctex=1
\usepackage{amssymb}
\usepackage{hyperref}
\hypersetup{colorlinks=true,
linkcolor=blue,
citecolor=blue,
urlcolor=blue,
}
\newtheorem{theorem}{Theorem}[section]
\newtheorem{lemma}[theorem]{Lemma}
\newtheorem{corollary}[theorem]{Corollary}
\newtheorem{claim}[theorem]{Claim}
\newtheorem{prop}[theorem]{Proposition}
\newtheorem{no}[theorem]{Notation}
\newtheorem{definition}[theorem]{Definition}
\newtheorem{remark}[theorem]{Remark}
\newtheorem{examp}{Example}[section]
\newtheorem {exercise}[theorem] {Exercise}

\newcommand{\ds}{\displaystyle}
\newcommand{\ts}{\textstyle}
\newcommand{\nin}{\noindent}
\newcommand{\rr}{\mathbb{R}}
\newcommand{\nn}{\mathbb{N}}
\newcommand{\zz}{\mathbb{Z}}
\newcommand{\cc}{\mathbb{C}}
\newcommand{\ci}{\mathbb{T}}
\newcommand{\zzdot}{\dot{\zz}}
\newcommand{\wh}{\widehat}
\newcommand{\ee}{\varepsilon}
\newcommand{\vp}{\varphi}
\newcommand{\wt}{\widetilde}
\newcommand{\uol}{u^\omega_\lambda}
\newcommand{\lbar}{\bar{l}}
\renewcommand{\l}{\lambda}
\newcommand{\R}{\mathbb R}
\newcommand{\RR}{\mathcal R}
\newcommand{\p}{\partial}
\newcommand{\al}{\alpha}
\newcommand{\ve}{q}
\newcommand{\tg}{{tan}}
\newcommand{\m}{q}
\newcommand{\N}{N}
\newcommand{\ta}{{\tilde{a}}}
\newcommand{\tb}{{\tilde{b}}}
\newcommand{\tc}{{\tilde{c}}}
\newcommand{\tS}{{\tilde S}}
\newcommand{\tP}{{\tilde P}}
\newcommand{\tu}{{\tilde{u}}}
\newcommand{\tw}{{\tilde{w}}}
\newcommand{\tA}{{\tilde{A}}}
\newcommand{\tX}{{\tilde{X}}}
\newcommand{\tphi}{{\tilde{\phi}}}
\begin{document}
\title{Ill-posedness issues for the Boussinesq equation}

\author{Dan-Andrei Geba, Alexandrou Himonas, and David Karapetyan}

\address{Department of Mathematics, University of Rochester, Rochester, NY 14627}
\address{Department of Mathematics, University of Notre Dame, Notre Dame, IN 46556}
\address{Department of Mathematics, University of Notre Dame, Notre Dame, IN 46556}
\date{}

%\begin{abstract}
%\end{abstract}

\subjclass[2000]{35B30, 35Q55, 35Q72}
\keywords{Ill-posedness; Picard iteration method; $X^{s,b}$ spaces.}

\maketitle

\section{Introduction}

\subsection{Formulation}  object of investigation is the Cauchy problem for
the periodic/non-periodic ``good" Boussinesq equation, i.e.,
\begin{equation}
\left\{
\begin{array}{l}
u_{tt}-u_{xx}+u_{xxxx}+(u^2)_{xx}\,=\,0, \quad x\in \mathbb{T}\ \text{or} \ \mathbb{R}, \quad t>0,\\
\\
u(0,x)\,=\,u_0(x),\qquad u_t(0,x)\,=\,u_1(x).\\
\end{array}\right.
\label{main}
\end{equation}
%
%
An alternative way to write \eqref{main} (\cite{Linares:1993ly},
\cite{Fang:1996zr}) is in the form of the following Hamiltonian system
\begin{equation}
\left\{
\begin{array}{l}
u_{t}\,=\,v_{x},\\
\\
v_t\,=\, \left(u-u_{xx}-(u^2)_x\right)_x,\\
\end{array}\right.
\label{sys}
\end{equation}
for which one obtains the conservation of energy, momentum, and mass, i.e.,
\[
\aligned
&E(u,v)\,=\,\int \frac{u^2+u_x^2+v^2}{2} - \frac{u^3}{3}\,dx,\\
M(u,v)\,=&\,\int uv \,dx, \quad N(u)\,=\,\int u \,dx,\quad P(v)\,=\,\int v \,dx.
\endaligned
\]

In order to understand what is the Sobolev theory associated with \eqref{main}, we could do first a formal analysis for \eqref{sys} by assigning $H^{s_1}$ regularity to $u$ and $H^{s_2}$ regularity to $v$. Then, from the first equation of \eqref{sys} we obtain that $u_t$ has $H^{s_2-1}$ regularity, while the main equation \eqref{main} implies that $u_{tt}$ is $H^{s_1-4}$ regular. Hence, by working under the natural assumption that the regularity for $u_t$ should be the average of the ones for $u$, respectively $u_{tt}$, we deduce that
\[
s_2=s_1-1.
\] 
This is the motivation behind the initial data notation used by certain authors (e.g., \cite{Farah:2009uq}, \cite{Farah:2010ys}):
\begin{equation}
u_0=\phi\in H^s(\mathbb{T}\ \text{or} \ \mathbb{R}), \qquad u_1=\psi_x \in H^{s-2}(\mathbb{T}\ \text{or} \ \mathbb{R}).
\label{not}
\end{equation}

A more intuitive explanation for the Sobolev regularity in \eqref{not} is given by the fact that the leading terms in the linear operator associated to \eqref{main} are $u_{tt}$ and $u_{xxxx}$, and so, morally speaking, one derivative in time behaves like two derivatives in the spatial variable. 

\subsection{Scaling and well-posedness theory} We notice immediately that \eqref{main} doesn't have a scaling invariance. However, if one ignores the lower order term $u_{xx}$, the new equation 
\begin{equation}
u_{tt}+u_{xxxx}+(u^2)_{xx}\,=\,0\label{new}
\end{equation}
is invariant under the transformation
\[
u_{\lambda}(t,x)\,=\,\frac{1}{\lambda^2}u\left(\frac{t}{\lambda^2}, \frac{x}{\lambda}\right),
\]
which leads to $s_c=-\frac 32$.

The current state of the art of the LWP theory for \eqref{main}, with Sobolev regularity prescribed by \eqref{not}, is as follows:
\begin{itemize}
\item the non-periodic problem is LWP for $s>-1/2$ (\cite{Kishimoto:2010ly}), which is obtained as a corollary for an improved theory for quadratic NLS. Previously obtained results include $s>-1/4$ (\cite{Farah:2009uq}) and $s\geq 0$ (\cite{Linares:1993ly}).

\item the periodic problem is LWP for $s>-1/4$ (\cite{Farah:2010ys}), with a Picard iteration done in
the norm
\[
\|F\|_{X^{s,b}}\,=\,\|<|\tau|-\sqrt{n^2+n^4}>^b\,<n>^s \tilde{F}(\tau,n)\|_{L^2_{\tau}l^2_n}.
\]
The previous best result was $s\geq 0$ (cite{Fang:1996zr}).
\end{itemize}

\begin{remark}
This picture is in line with the ones for $1D$ quadratic NLS, KdV, and $mKdV$, for which there is a $1/4$ gap in regularity between the non-periodic and the periodic initial value problems. 
\end{remark}

In what concerns IP, the main results for the non-periodic problem (obtained in \cite{Farah:2009uq}) state that:
\begin{itemize}
\item the bilinear estimate in $X^{s,b}$ spaces, used in proving LWP, fails for $s\leq -1/4$;

\item the solution map 
\[
S: H^s\times H^{s-2} \to C([0,T]; H^s), \quad
S(u_0,u_1)\,=\,u,
\]
is not $C^2$ at zero for $s<-2$.
\end{itemize}

For the periodic problem, only the failure of the corresponding bilinear estimate when $s\leq -1/4$ was shown in \cite{Farah:2010ys}.

\subsection{Main goals} Our main objective is to obtain an IP result describing Sobolev exponents for which the solution map is not continuous. Our preliminary computations, which are based on a quite general technique pioneered by Bejenaru and Tao (\cite{Bejenaru-Tao-2006-Sharp-well-posedness-and-ill-posedness}), can be summarized in the following

\begin{theorem}
The solution map 
\[
S: H^s\times H^{s-2} \to C([0,T]; H^s), \quad
S(u_0,u_1)\,=\,u,
\]
associated to 
\begin{equation}
\left\{
\begin{array}{l}
u_{tt}-u_{xx}+u_{xxxx}+(u^2)_{xx}\,=\,0,\\
\\
u(0,x)\,=\,u_0(x),\qquad u_t(0,x)\,=\,u_1(x).\\
\end{array}\right.
\label{chs}
\end{equation}
is not continuous for $s<-7/2$ in the non-periodic case and for $s<-15/4$ in the periodic case.
\label{mth}
\end{theorem}

\section{Description of the Bejenaru-Tao technique}
\subsection{Abstract well-posedness setting} The Boussinesq equation falls under the general category of abstract semilinear evolution equations with a bilinear nonlinearity, which can be written as (\textbf{description at page 234 in \cite{Bejenaru-Tao-2006-Sharp-well-posedness-and-ill-posedness}})
\begin{equation}
u\,=\,L(f)\,+\,N(u,u),
\label{LN}
\end{equation}
where
\begin{itemize}

\item $f$ is the initial data lying in a data space $D$ (e.g. $H^s \times H^{s-2}$);

\item the solution $u$ takes values in a solution space $S$ (e.g. $C^0H^s \cap X^{s,b}$);

\item $L: D \to S$ is a linear operator  and $N:S\times S \to S$ is a bilinear form.  

\end{itemize} 

For $(D,\|\, \|_D)$ and $(S,\|\, \|_S)$ Banach spaces with 
\begin{equation}
\|L(f)\|_S \leq C \|f\|_D,\qquad \|N(u,v)\|_S \leq C \|u\|_S \|v\|_S,
\label{estim}
\end{equation}
where $C>0$ is an absolute constant, we can prove, by using a standard contraction argument, that, for all  $\|f\|_D<\frac{1}{16C^2}$, there exists a unique solution $u$ for $\eqref{LN}$ satisfying $\|u\|_D<\frac{1}{4C}$.  Moreover, the solution is the sum of an absolutely convergent series in S, i.e.,
\begin{equation}
u\,=\,\sum_{n=1}^{\infty} A_n(f), \qquad (\forall) \|f\|_D<\frac{1}{16C^2},
\label{series}
\end{equation}
where the nonlinear maps $A_n: D\to S$, $n\geq 1$, are defined recursively by
\begin{equation}
A_1(f)\,=\,L(f), \qquad A_n(f)\,=\,\sum_{k=1}^{n-1} N(A_k(f),A_{n-k}(f)), (\forall)n\geq 2.
\label{An}
\end{equation}
This is the result contained in \textbf{Theorem 3 (page 235) in \cite{Bejenaru-Tao-2006-Sharp-well-posedness-and-ill-posedness}}.

\subsection{Sufficient conditions for ill-posedness} However, in proving IP, the key fact that we will be using is \textbf{Proposition 1, page 238 in \cite{Bejenaru-Tao-2006-Sharp-well-posedness-and-ill-posedness}}. It basically says that if the solution series \eqref{series} is continuous, as a function of $f$, in  weaker topologies than the ones given by $\| \|_D$ and $\| \|_S$, then each of the $A_n$'s is also continuous in this setup.

Precisely, let $(D,S)$ be a pair of Banach spaces satisfying \eqref{estim} (the term used in \cite{Bejenaru-Tao-2006-Sharp-well-posedness-and-ill-posedness} is \textit{quantitatively well-posed}), for which one has, as described above, a continuous solution map associated to \eqref{LN},
\begin{equation}
f\in (B_1,\|\, \|_D)\,\longrightarrow\,u\in (B_2,\|\, \|_S),
\end{equation} 
where $B_1\subset D$ and $B_2 \subset S$ are two balls centered at the origin. If one endows these balls with coarser norms denoted $\|\, \|_{D'}$, respectively $\| \,\|_{S'}$, and 
\begin{equation}
f\in (B_1,\|\, \|_{D'})\,\longrightarrow\,u\in (B_2,\|\, \|_{S'})
\end{equation} 
is still continuous, then
\begin{equation}
A_n: (B_1,\|\, \|_{D'})\,\longrightarrow\,(B_2,\| \,\|_S)
\end{equation} 
is continuous for all $n\geq 1$.

Therefore, if we want to use the above result in order to prove IP, a strategy would look like this:
\begin{itemize}
\item start with a \textit{well-posedness} pair $(D,S)$, i.e., $D$ is an initial data space for which
there is a solution lying in $S$. In principle, $D$ is an $H^s$ space, with $s$ large enough to ensure LWP, while $S$ is a subset of $C([0,T]; H^s)$. An important comment about the time of existence $T$ will be made below.

\item choose two weaker norms, $\| \,\|_{D'}$ and $\| \,\|_{S'}$, related to the regularity for which you try to prove IP. The obvious choice, which maybe one can refine afterwards, is
\[
\| \,\|_{D'}\,=\,\| \,\|_{H^{s'}}, \qquad \| \,\|_{S'}\,=\,\| \,\|_{C([0,T], H^{s'})},\] 
with $s'<s$.

\item find $n\geq 1$ and a sequence $(f_N)_N\subset D$ such that
\begin{equation}
\lim_{N\to \infty} \|f_N\|_D\,=\,0, \qquad \quad \sup_N \frac{\|A_n(f_N)\|_{S'}}{\|f_N\|_{D'}}\,=\,\infty.
\label{ip}
\end{equation}
\end{itemize} 

\begin{remark}
i) Obviously, the second condition in \eqref{ip} is sufficient to infer that $A_n$ is discontinuous;

ii) \textbf{However, equally important in \eqref{ip} is the first limit, which makes this strategy go through. This happens because the LWP has to happen in the strongest sense, i.e., the time of existence $T=T(\|f\|_D)$, and so $\lim_{N\to \infty} \|f_N\|_D=0$ assures that all the corresponding solutions (or alternatively, for $N$ sufficiently large) $u_N$'s exist for the same amount of time.} This is not precisely stated in \cite{Bejenaru-Tao-2006-Sharp-well-posedness-and-ill-posedness}, being quite formally addressed. 
\label{remip}
\end{remark}

%
%


\section{Precise computations for the Boussinesq equation}

\subsection{The non-periodic case} The first step is to write formulas for $L$ and $N$. We start with the homogeneous problem
\begin{equation}
\left\{
\begin{array}{l}
u_{tt}-u_{xx}+u_{xxxx}\,=\,0, \quad x\in\mathbb{R}, \quad t>0,\\
\\
u(0,x)\,=\,u_0(x),\qquad u_t(0,x)\,=\,u_1(x),\\
\end{array}\right.
\label{hom}
\end{equation}
which, using the Fourier transform in the spatial variable, leads to
\begin{equation}
\hat{u}(t,\xi)\,=\,\cos(t \lambda(\xi)) \hat{u}_0(\xi)\,+\,\frac{\sin(t \lambda(\xi))}{\lambda(\xi)} \hat{u}_1(\xi),
\label{homf}\end{equation}
where $\lambda(\xi)=\sqrt{\xi^2+\xi^4}$.

The inhomogeneous problem 
\begin{equation}
\left\{
\begin{array}{l}
u_{tt}-u_{xx}+u_{xxxx}\,=\,F, \quad x\in\mathbb{R}, \quad t>0,\\
\\
u(0,x)\,=\,0,\qquad u_t(0,x)\,=\,0,\\
\end{array}\right.
\label{ih}
\end{equation}
can then be solved using \eqref{homf} and the Duhamel's principle, which gives us
\begin{equation}
\hat{u}(t,\xi)\,=\,\int_0^t\,\frac{\sin((t-s) \lambda(\xi))}{\lambda(\xi)} \,\hat{F}(s,\xi)\,ds.
\label{ihf}
\end{equation}

Putting together \eqref{homf} and \eqref{ihf}, we obtain that \eqref{main} can be rewritten as
\begin{equation}
\aligned
\hat{u}(t,\xi)\,&=\,\cos(t \lambda(\xi)) \hat{u}_0(\xi)+\frac{\sin(t \lambda(\xi))}{\lambda(\xi)} \hat{u}_1(\xi)-\int_0^t\,\frac{\sin((t-s) \lambda(\xi))}{\lambda(\xi)} \,\widehat{(u^2)_{xx}}(s,\xi)\,ds\\
&=\,\cos(t \lambda(\xi)) \hat{u}_0(\xi)+\frac{\sin(t \lambda(\xi))}{\lambda(\xi)} \hat{u}_1(\xi)+\int_0^t\,\frac{\sin((t-s) \lambda(\xi))}{\lambda(\xi)}\, \xi^2\, \widehat{u^2}(s,\xi)\,ds\endaligned
\label{mainf}
\end{equation}
which is nothing but the Fourier version of \eqref{LN}, with
\begin{equation}
\aligned
\widehat{Lf}(t,\xi)\,=\,\cos(t \lambda(\xi))& \hat{u}_0(\xi)+\frac{\sin(t \lambda(\xi))}{\lambda(\xi)} \hat{u}_1(\xi),\quad f=(u_0,u_1),\\
\widehat{N(u,v)}(t,\xi)&=\,\int_0^t\,\frac{\sin((t-s) \lambda(\xi))}{\lambda(\xi)}\, \xi^2\, \widehat{u\cdot v}(s,\xi)\,ds.\endaligned
\label{lnf}
\end{equation}

\textbf{Our IP result will be proved by showing that $A_2$ is not continuous (as in pages 241-242 in \cite{Bejenaru-Tao-2006-Sharp-well-posedness-and-ill-posedness})}. Based on \eqref{An} and \eqref{lnf}, it follows that:
\begin{equation}
\aligned
A_2(f)(t,x)\,&=\,N(A_1(f),A_1(f))(t,x)\,=\,N(L(f),L(f))(t,x),\\
\widehat{A_2(f)}(t,\xi)\,&=\,\int_0^t\,\frac{\sin((t-s) \lambda(\xi))}{\lambda(\xi)}\, \xi^2\, \widehat{(Lf)^2}(s,\xi)\,ds\\
&=\,\int_0^t \int_{\mathbb{R}}\,\frac{\sin((t-s) \lambda(\xi))}{\lambda(\xi)}\, \xi^2\, \widehat{Lf}(s,\xi-\eta)\,\widehat{Lf}(s,\eta)\,d\eta\,ds.\endaligned
\label{A2f}
\end{equation}

We will choose the sequence of initial data $(f_N)_{N\geq 10}$ that satisfies \eqref{ip} as
\begin{equation}
f_N\,=\,(g_N, 0), \qquad \widehat{g_N}(\xi)\,=\,\frac{1}{N^\alpha}\,\left\{
\begin{array}{l}
1,\quad \text{for} \quad N-1\leq |\xi| \leq N+1,\\
\\
0,\quad \text{otherwise},\\
\end{array}\right.
\label{fn}
\end{equation}
$\alpha$ being a real parameter to be chosen later. A simple computation yields
\begin{equation}
\|f_N\|_{H^\beta\times H^{\beta-2}}\,=\,\|g_N\|_{H^\beta}\,\simeq\,N^{\beta-\alpha}
\label{hsfn}
\end{equation}
Therefore, in order to have 
\[
\lim_{N\to \infty} \|f_N\|_{H^\beta \times H^{\beta -2}}\,=\,0
\]
we need to impose $\beta<\alpha$. This $\beta$ has to be in the well-posedness range obtained in \cite{Kishimoto:2010ly}, i.e., $\beta>-1/2$; hence, it would suffice to take $\alpha >-1/2$. This choice assures, as mentioned in Remark \ref{remip}, that all the solutions corresponding to sequence of initial data $(f_N)_N$ exist for at least an identical amount of time, which can be normalized to be equal to $1$. 

Plugging \eqref{fn} into \eqref{A2f} and using \eqref{lnf}, we obtain
\begin{equation}
\aligned
&\widehat{A_2(f_N)}(t,\xi)\\
&=\int_0^t \int_{\mathbb{R}} \frac{\sin((t-s) \lambda(\xi))}{\lambda(\xi)} \xi^2 \cos(s \lambda(\xi-\eta))\cos(s\lambda(\eta))\widehat{g_N}(\xi-\eta)\widehat{g_N}(\eta)d\eta\,ds\\
&=\frac{\xi^2}{N^{2\alpha}}\int_0^t \int_{N-1\leq |\xi-\eta|, |\eta| \leq N+1}\frac{\sin((t-s) \lambda(\xi))}{\lambda(\xi)}\cos(s \lambda(\xi-\eta))\cos(s\lambda(\eta))d\eta\,ds.\endaligned
\label{A2fn}
\end{equation}
On another hand
\begin{equation}
\aligned
\|A_2(f_N)\|_{C([0,1], H^{s'}(\mathbb{R}))} &\geq \|A_2(f_N)(t_N)\|_{H^{s'}(1/4\leq |\xi| \leq 1/2)}\\ &\gtrsim \|A_2(f_N)(t_N)\|_{L^2(1/4\leq |\xi| \leq 1/2)},\label{hsA2}
\endaligned
\end{equation}
with $t_N=\frac{1}{100 N^2}$ and $s'$ is an arbitrary real number.

Thus, we will be working in the regime given by
\[
0\leq s\leq t_N=\frac{1}{100 N^2}, \quad |\xi|\sim 1, \quad |\eta|\sim N,\]
which implies
\[
0 \leq (t_N-s) \lambda(\xi),\,s \lambda(\xi-\eta),\, s\lambda(\eta) \ll 1. \]
This allows us to infer
\[
0 \leq \sin ((t_N-s) \lambda(\xi)),
\]
\[
1/2 < \min\{\cos(s \lambda(\xi-\eta)), \cos(s\lambda(\eta))\},
\]
for $N$ sufficiently large. 

We can then estimate $\widehat{A_2(f_N)}(t_N,\xi)$ (for $1/4\leq |\xi| \leq 1/2$) as follows
\[
\widehat{A_2(f_N)}(t_N,\xi)\,\gtrsim\,\frac{1}{N^{2\alpha}}\,\int_0^{t_N}\, \sin((t_N-s) \lambda(\xi))\,ds\,=\,\frac{1-\cos(t_N \lambda(\xi))}{\lambda(\xi)\,N^{2\alpha}}\,\gtrsim \frac{1}{N^{2\alpha+4}},\]
which, based on \eqref{hsA2}, yields
\begin{equation}
\|A_2(f_N)\|_{C([0,1], H^{s'}(\mathbb{R}))}\,\gtrsim \frac{1}{N^{2\alpha+4}}.
\end{equation}

Combining this with \eqref{hsfn}, we deduce 
\begin{equation}
\frac{\|A_2(f_N)\|_{C([0,1], H^{s'}(\mathbb{R}))}}{\|f_N\|_{H^s \times H^{s -2}}}\,\gtrsim\, \frac{1}{N^{s+\alpha+4}}
\end{equation}
for $N$ sufficiently large, $(f_N)_{N\geq 10}$ given by \eqref{fn}, and $\alpha>-1/2$. Therefore, for $s<-\alpha-4$ and $s'$ a real arbitrary number, the conclusion is that
\begin{equation}
\lim_{N\to\infty}\,\frac{\|A_2(f_N)\|_{C([0,1], H^{s'}(\mathbb{R}))}}{\|f_N\|_{H^s \times H^{s -2}}}\,=\,0,
\end{equation}
which proves Theorem \ref{mth} in the non-periodic case, based on the Bejenaru-Tao technique.

\subsection{The periodic case} Here, the computations are almost identical, the only difference being that, instead of integrals in $\xi$ or $\eta$, we have to sum up series over integers in $n$ or $m$. Also, as mentioned in the introduction, the best LWP for the periodic problem is for $s>-1/4$ (\cite{Farah:2010ys}), thus forcing $\alpha>-1/4$. 

We start by writing \eqref{mainf} and \eqref{lnf} in the periodic case, simply by replacing $\xi$ with $n\in\mathbb{Z}$, i.e.,
\begin{equation}
\aligned
\hat{u}(t,n)\,=\,\cos(t \lambda(n))& \hat{u}_0(n)+\frac{\sin(t \lambda(n))}{\lambda(n)} \hat{u}_1(n)+\int_0^t\,\frac{\sin((t-s) \lambda(n))}{\lambda(n)}\, n^2\, \widehat{u^2}(s,n)\,ds\\
\widehat{Lf}(t,n)\,&=\,\cos(t \lambda(n)) \hat{u}_0(n)+\frac{\sin(t \lambda(n))}{\lambda(n)} \hat{u}_1(n),\quad f=(u_0,u_1),\\
\widehat{N(u,v)}&(t,n)=\,\int_0^t\,\frac{\sin((t-s) \lambda(n))}{\lambda(n)}\, n^2\, \widehat{u\cdot v}(s,n)\,ds.\endaligned
\end{equation}

The sequence of initial data $(f_N)_{N\geq 10}$ modifies accordingly, 
\begin{equation}
f_N\,=\,(g_N, 0), \qquad \widehat{g_N}(n)\,=\,\frac{1}{N^\alpha}\,\left\{
\begin{array}{l}
1,\quad \text{for} \quad N-1\leq |n| \leq N+1,\\
\\
0,\quad \text{otherwise},\\
\end{array}\right.
\end{equation}
with $\alpha>-1/4$, as discussed above.

Hence, \eqref{A2fn} becomes
\begin{equation}
\aligned
&\widehat{A_2(f_N)}(t,n)\\
&=\frac{n^2}{N^{2\alpha}} \sum_{N-1\leq |n-m|, |m| \leq N+1} \int_0^t \frac{\sin((t-s) \lambda(n))}{\lambda(n)}\cos(s \lambda(n-m))\cos(s\lambda(m))\,ds,
\endaligned
\end{equation}
which, as in the periodic case, leads to 
\[
\widehat{A_2(f_N)}\left(\frac{1}{100N^2},1\right)\,\gtrsim\,\frac{1}{N^{2\alpha+4}},\]
for $N$ sufficiently large. Thus, one obtains
\begin{equation}
\frac{\|A_2(f_N)\|_{C([0,1], H^{s'}(\mathbb{T}))}}{\|f_N\|_{H^s \times H^{s -2}}}\,\gtrsim\, \frac{1}{N^{s+\alpha+4}},
\end{equation}
which finishes the proof.

\appendix
\section{Why We Must Redo the Bilinear Estimates in the
Kishimoto Paper for Boussinesq}
Recall the Boussinesq ivp \eqref{main}. Factoring, we obtain
%
%
\begin{equation*}
\begin{split}
  (i \p_{t} - \p_{x}^{2})(-i \p_{t} - \p_{x}^{2})u - \p_{x}^{2} u +
  \p_{x}^{2}(u^{2}) = 0
\end{split}
\end{equation*}
%
%
which is equivalent to
%
%
\begin{equation*}
\begin{split}
  (i \p_{t} - \p_{x}^{2} + 1)(-i \p_{t} - \p_{x}^{2} +1)u = (1 -
  \p_{x}^{2})u - \p_{x}^{2}(u^{2})
\end{split}
\end{equation*}
%
%
or
%
%
\begin{equation*}
\begin{split}
  \frac{(i \p_{t} - \p_{x}^{2} + 1)(-i \p_{t} - \p_{x}^{2} +1)u}{(1 -
  \p_{x}^{2})} = u - \frac{\p_{x}^{2}}{1 - \p_{x}^{2}}(u^{2}). 
\end{split}
\end{equation*}
%
%
Set
%
%
\begin{equation*}
\begin{split}
  v & = \left( \frac{-i \p_{t} + 1 -\p_{x}^{2}}{1 - \p_{x}^{2}} \right)u
  = u - i(1 - \p_{x}^{2})^{-1} \p_{t}u
\end{split}
\end{equation*}
and
\begin{equation*}
  \begin{split}
    \bar{w} & = \left( \frac{i \p_{t} + 1 -\p_{x}^{2}}{1 - \p_{x}^{2}} \right)u
  = u + i(1 - \p_{x}^{2})^{-1} \p_{t}u.
\end{split}
\end{equation*}
%
Then we obtain the equations
\begin{gather*}
  (i \p_{t} + 1 - \p_{x}^{2})v = \frac{1}{2}(v + \bar{w}) -
  \frac{\p_{x}^{2}}{1 - \p_{x}^{2}} \frac{(v + \bar{w})^{2}}{4}
  \\
  (-i \p_{t} + 1 - \p_{x}^{2})\bar{w}= \frac{1}{2}(v + \bar{w}) -
  \frac{\p_{x}^{2}}{1 - \p_{x}^{2}} \frac{(v + \bar{w})^{2}}{4}
\end{gather*}
%
Since solutions for the Boussinesq are real-valued, we obtain that $v=w$. Hence,
the above system reduces to 
\begin{gather}
  \label{simp-sys}
  (i \p_{t} - \p_{x}^{2})v = \frac{1}{2}(\bar{v} -v) -
  \frac{\p_{x}^{2}}{1 - \p_{x}^{2}} \frac{(v + \bar{v})^{2}}{4}
  \\
  v(x,0) = v_{0}(x) = u_{0} - i(1 - \p_{x}^{2})^{-1}u_{1}.
  \label{simp-sys-init}
\end{gather}
%
%
%%%%%%%%%%%%%%%%%%%%%%%%%%%%%%%%%%%%%%%%%%%%%%%%%%%%%
%
%
%                init-data-lip
%
%
%%%%%%%%%%%%%%%%%%%%%%%%%%%%%%%%%%%%%%%%%%%%%%%%%%%%%
%
%
\begin{lemma}
  The map $$(u_{0}, u_{1}) \mapsto 
  u_{0} - i(1 - \p_{x}^{2})^{-1}u_{1}$$ is Lipschitz from $H^{s} \times
  H^{s-2}$ to $H^{s}$.
\label{lem:lip-init-data}
\end{lemma}
%
%
%
%
\begin{proof}
%
%
\begin{equation*}
\begin{split}
  \| u_{0} - i(1 - \p_{x}^{2})^{-1} u_{1} \|_{H^{s}} 
  & \le \| u_{0} \|_{H^{s}} + \| (1 - \p_{x}^{2})^{-1} u_{1} \|_{H^{s}}
  \\
  & = \| (u_{0}, u_{1}) \|_{H^{s} \times H^{s-2}}.
\end{split}
\end{equation*}
%
%
\end{proof}
%
%
%
%
%%%%%%%%%%%%%%%%%%%%%%%%%%%%%%%%%%%%%%%%%%%%%%%%%%%%%
%
%
%                lip-sol
%
%
%%%%%%%%%%%%%%%%%%%%%%%%%%%%%%%%%%%%%%%%%%%%%%%%%%%%%
%
%
\begin{lemma}
  Let $v_{0,n} \subset B_{H^{s}}(R)$ be a sequence of initial data for the ivp
  \eqref{simp-sys}-\eqref{simp-sys-init}. Then the map $$v \mapsto
  \frac{1}{2}(v + \bar{v})$$ is Lipschitz from $B_{H^{s}}(R)$ to $B_{H^{s}}(2R)$.
\label{lem:lip-sol}
\end{lemma}
%
%
%
%
\begin{proof}
%
%
\begin{equation*}
\begin{split}
  \sup_{0 \le t \le T} \| u(t) \|_{C\left( [0, T], H^{s} \right)}
  & = \sup_{0 \le t \le T} \frac{1}{2} \| v + \bar v \|_{H^{s}}
  \\
  & \le \sup_{0 \le t \le T} \| v \|_{H^{s}}.
\end{split}
\end{equation*}
%
%
\end{proof}
%
%
\subsection{NLS-type WP implies Boussinesq WP} 
\label{ssec:wp-imp-wp}
From Lemma \ref{lem:lip-init-data}, Lemma \ref{lem:lip-sol}, and the fact that
the map
%
%
\begin{equation}
  \label{solution-prop}
\begin{split}
  v \mapsto \frac{1}{2} (v + \bar{v}) 
\end{split}
\end{equation}
%
%
takes solutions of \eqref{simp-sys} to solutions of the Boussinesq, it follows
that well-posedness (in the sense of Hadamard) in $H^{s}$ for the NLS-type ivp
\eqref{simp-sys}-\eqref{simp-sys-init} 
implies well-posedness in $H^{s}$ for the Boussinesq ivp \eqref{main}.
%
%
\begin{framed}
%
%
\begin{remark}
More precisely, Lemma \ref{lem:lip-init-data} and Lemma \ref{lem:lip-sol} yield
propagation of continuity of the data-to-solution map in $H^{s}$, while
\eqref{solution-prop} yields the propagation of existence and uniqueness in
$H^{s}$.
\label{rem:wp-imp-wp}
\end{remark}
%
%
\end{framed}
%
\subsection{Failure of continuity for NLS-type equation does not imply same for Boussinesq} 
\label{ssec:ip-no-ip}
%
Suppose the data-to-solution map for the NLS-type equation is not continuous at
$0$ in $H^{s}$, for some $s \in \rr$. This is equivalent to saying that for
fixed $\ee, \delta > 0$, there exists a sequence $v_{0,n}
\xrightarrow{H^{s}}{0}$ such that
%
%
\begin{equation*}
\begin{split}
\| v_{0,n} \|_{H^{s}} < \ee, \ \text{but} \ \| v_{n} \|_{C([0,T], H^{s})} > \delta.
\end{split}
\end{equation*}
%
%
Suppose $v_{0,n}$ can be written in form
%
%
\begin{equation*}
\begin{split}
v_{0,n} = u_{0,n} - i(1 - \p_{x}^{2})^{-1}u_{1,n}.
\end{split}
\end{equation*}
%
for some $u_{0,n} \in H^{s}, u_{1,n} \in H^{s-2}$. Then 
%
%
\begin{equation*}
\begin{split}
v_{0} = u_{0} - i(1- \p_{x}^{2})^{-1}u_{1}.
\end{split}
\end{equation*}
%
%
We would like to show that
%
%
\begin{equation*}
\begin{split}
\| (u_{0,n}, u_{1,n}) \|_{H^{s} \times H^{s-2}} < \ee \ \text{but} \ \| u \|_{C([0,T], H^{s})} > \delta.
\end{split}
\end{equation*}
%
%
Recall that
%
%
\begin{equation*}
\begin{split}
u = \frac{1}{2}(v + \bar{v}).
\end{split}
\end{equation*}
%
%
Taking $v_{0}$ such that $u_{1,n}=0$, we reduce to showing that
%
%
\begin{equation*}
\begin{split}
\| u_{0,n} \|_{H^{s}} < \ee \ \text{but} \ \| u_{n} \|_{C([0, T], H^{s})} > \delta.
\end{split}
\end{equation*}
%
%
It would be possible to do so if, for example, the maps in Lemma
\ref{lem:lip-init-data} and \ref{lem:lip-sol} were bi-Lipschitz. I don't see a
clear cut way to prove this though.
%
%
\section{The Well-Posedness Spaces for Boussinesq}
\subsection{The non-periodic case} 
\label{ssec:non-per-spaces}
It is enough to consider the integral equation
%
%
\begin{equation*}
\begin{split}
  u(t) = U(t) u_{0} -i \int_{0}^{t} U(t -t')N(u(t')) dt'
\end{split}
\end{equation*}
%
%
where 
%
%
\begin{equation*}
\begin{split}
  & U(t) = e^{-it \p_x^{2}},
  \\
  & N(u) = -u + (u + \bar{u})/2 + \omega^{2}(u + \bar{u})^{2}/4
\end{split}
\end{equation*}
%
%
Partition $\rr^{2}$ as follows.
%
%
\begin{equation*}
\begin{split}
  & P_{1} = \left\{ (\tau, \xi) \in \rr^{2}: | \tau - \xi^{2} | \le \frac{| \xi
  |}{4} \ \text{and} \ | \xi | \ge 1 \right\},
  \\
  & P_{2} = \left\{ (\tau, \xi) \in \rr^{2}: | \tau - \xi^{2} | \ge \frac{| \xi
  |}{4} \ \text{and} \ | \xi | < 1 \right\}.
\end{split}
\end{equation*}
%
%
Set
%
%
\begin{equation*}
\begin{split}
  w_{s} = 
  \begin{cases}
    \langle \xi \rangle ^{s} \langle \tau - \xi^{2} \rangle , \quad & (\tau,
    \xi) \in P_{1},
    \\
    \langle \xi \rangle ^{1/2} \langle \tau - \xi^{2} \rangle^{1/2}, \quad
    & (\tau, \xi) \in P_{2}
  \end{cases}
\end{split}
\end{equation*}
%
%
and let $Z^{s}, Y^{s}$ be the completion of the space of schwartz functions
$S(\rr^{2})$ under the norms
%
%
\begin{equation*}
\begin{split}
  & \| u \|_{Z^{s}} = \| w_{s} \wh{u} \|_{L^{2}_{\tau, \xi}},
  \\
  & \| u \|_{Y^{s}} = \| \int_{\rr} \langle \xi \rangle^{s} | \wh{u} | d \tau
  \|_{L^{2}_{\xi}}.
\end{split}
\end{equation*}
%
%
Our candidate for the well-posedness space (in the Picard iterative sense) is
$Z^{s} \cap L^{\infty}\left( [0, T], H^{s} \right)$. Indeed, we have the
following.
%
%
%
%
%%%%%%%%%%%%%%%%%%%%%%%%%%%%%%%%%%%%%%%%%%%%%%%%%%%%%
%
%
%               init data bound
%
%
%%%%%%%%%%%%%%%%%%%%%%%%%%%%%%%%%%%%%%%%%%%%%%%%%%%%%
%
%
\begin{lemma}[Prop 2.6 in Kishimoto]
  For $u \in Z^{s} \cap L^{\infty}\left( [0,T], H^{s} \right)$
  %
  %
  \begin{equation*}
  \begin{split}
    \| u \|_{Z^{s}} + \| u \|_{L^{\infty}\left( [0,T], H^{s} \right)} \lesssim
    \| u_{0} \|_{H^{s}}
  \end{split}
  \end{equation*}
  %
  %
\label{lem:init-data-b}.
\end{lemma}
%
%
%
%
%%%%%%%%%%%%%%%%%%%%%%%%%%%%%%%%%%%%%%%%%%%%%%%%%%%%%
%
%
%                
%
%
%%%%%%%%%%%%%%%%%%%%%%%%%%%%%%%%%%%%%%%%%%%%%%%%%%%%%
%
%
\begin{lemma}[Prop 2.7 in Kishimoto]
  Let $\vp(t)$ be a smooth cutoff
  function symmetric about the origin, equal to the identity in $[-1, 1]$, and
  supported in $[-2,2]$. Denote $U(t) = e^{it \p_{x}^{2}}$, and let 
  $u = \vp(t) \int_{0}^{t} U(t - t') F(t') dt'$. Then
  %
  %
  \begin{equation*}
  \begin{split}
    \| u \|_{Z^{s}} + \| u \|_{L^{\infty}\left( [0,T], H^{s} \right)} \lesssim
    \| \mathcal{F}^{-1} \langle \tau - \xi^{2} \rangle ^{-1} \wh{F}
    \|_{Z^{s}} + \| F^{-1}\langle \tau - \xi^{2} \rangle^{-1} \wh{F}
    \|_{Y^{s}}.
  \end{split}
  \end{equation*}
  %
  %
  \label{lem:non-lin-to-bilin}
\end{lemma}
%
%
%
%
%%%%%%%%%%%%%%%%%%%%%%%%%%%%%%%%%%%%%%%%%%%%%%%%%%%%%
%
%
%                bilinear estimates
%
%
%%%%%%%%%%%%%%%%%%%%%%%%%%%%%%%%%%%%%%%%%%%%%%%%%%%%%
%
%
\begin{lemma}[Bilinear Estimates, Prop 4.4 in Kishimoto]
For $u = u(x,t)$, $v = v(x,t)$, we have
%
%
\begin{equation*}
\begin{split}
  & \|\mathcal{F}^{-1} \langle \tau - \xi^{2} \rangle \wh{\omega^{2} uv}
  \|_{Z^{s}} \lesssim \| u \|_{Z^{s}} \|v \|_{Z^{s}}
  \\
  & \| \mathcal{F}^{-1} \langle \tau - \xi^{2} \rangle ^{-1} \wh{\omega^{2} uv}
  \|_{Y^{s}} \lesssim \| u \|_{Z^{s}} \| v \|_{Z^{s}}
\end{split}
\end{equation*}
%
where $\omega$ is the pseudo-differential operator defined by
%
%
\begin{equation*}
\begin{split}
  \wh{\omega}(\xi) = \frac{\xi}{ \sqrt{1 + \xi^{2}}}.
\end{split}
\end{equation*}
%
%
These estimates also hold with $uv$ replaced by $u \bar v$ or $\bar u \bar v$.
%
\label{lem:bilinear-estimates}
\end{lemma}
%
%
We conclude from the lemmas above that the Schrodinger type equation is
qualitatively well-posed. Hence, the Tao-Bejenaru machinery applies.

\bibliographystyle{amsalpha-custom}
\bibliography{/Users/davidkarapetyan/math/bib-files/references}

\end{document}

