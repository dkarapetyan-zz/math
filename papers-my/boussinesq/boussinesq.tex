\documentclass[12pt,reqno]{amsart}
\usepackage{amsmath}
\usepackage{fix-cm} %fix font errors
\usepackage{amssymb}
%\usepackage[notcite]{showkeys}
\usepackage{appendix}
\usepackage{cancel}  %for cancelling terms explicity on pdf
\usepackage{yhmath}   %makes fourier transform look nicer, among other things
\usepackage{framed}  %for framing remarks, theorems, etc.
\usepackage{enumerate} %to change enumerate symbols
\usepackage[margin=2.5cm]{geometry}  %page layout
\setcounter{tocdepth}{2} %must come before secnumdepth--else, pain
\setcounter{secnumdepth}{2} %number only sections, not subsections
%\usepackage[pdftex]{graphicx} %for importing pictures into latex--pdf compilation
\numberwithin{equation}{section}  %eliminate need for keeping track of counters
%\numberwithin{figure}{section}
\setlength{\parindent}{0in} %no indentation of paragraphs after section title
\renewcommand{\baselinestretch}{1.1} %increases vert spacing of text
%
\usepackage{hyperref}
\hypersetup{colorlinks=true,
linkcolor=blue,
citecolor=blue,
urlcolor=blue,
}
\usepackage[alphabetic, initials, msc-links]{amsrefs} %for the bibliography; uses cite pkg. Must be loaded after hyperref, otherwise doesn't work properly (conflicts with cref in particular)
\usepackage{cleveref} %must be last loaded package to work properly
\renewcommand{\cref}{\Cref}
%\Crefname{enumi}{}{} %don't write item iv, just iv
%\Crefname{equation}{}{} %don't write Equation (2.1), just (2.1)
\crefformat{equation}{(#2#1#3)} %don't write Equation (2.1), just (2.1)
\Crefformat{equation}{(#2#1#3)} %don't write Equation (2.1), just (2.1)
%
%
\newcommand{\ds}{\displaystyle}
\newcommand{\ts}{\textstyle}
\newcommand{\nin}{\noindent}
\newcommand{\rr}{\mathbb{R}}
\newcommand{\nn}{\mathbb{N}}
\newcommand{\zz}{\mathbb{Z}}
\newcommand{\cc}{\mathbb{C}}
\newcommand{\ci}{\mathbb{T}}
\newcommand{\zzdot}{\dot{\zz}}
\newcommand{\wh}{\widehat}
\newcommand{\p}{\partial}
\newcommand{\ee}{\varepsilon}
\newcommand{\vp}{\varphi}
\newcommand{\wt}{\widetilde}
%
%
%
%
\newtheorem{theorem}{Theorem}[section]
\newtheorem{lemma}[theorem]{Lemma}
\newtheorem{corollary}[theorem]{Corollary}
\newtheorem{claim}[theorem]{Claim}
\newtheorem{prop}[theorem]{Proposition}
\newtheorem{proposition}[theorem]{Proposition}
\newtheorem{no}[theorem]{Notation}
\newtheorem{definition}[theorem]{Definition}
\newtheorem{remark}[theorem]{Remark}
\newtheorem{examp}{Example}[section]
\newtheorem{exercise}[theorem]{Exercise}
%
\makeatletter \renewenvironment{proof}[1][\proofname]
{\par\pushQED{\qed}\normalfont\topsep6\p@\@plus6\p@\relax\trivlist\item[\hskip\labelsep\bfseries#1\@addpunct{.}]\ignorespaces}{\popQED\endtrivlist\@endpefalse}
\makeatother%
%makes proof environment bold instead of italic
%\makeatletter
%\renewcommand\subsubsection{\@startsection{subsubsection}{3}{\z@}%
%{-3.25ex\@plus -1ex \@minus -.2ex}%
%{1.5ex \@plus .2ex}%
%{\normalfont\normalsize \bfseries}}
%\makeatother
%makes subsubsubsection bold instead of italic
%
\newcommand{\uol}{u^\omega_\lambda}
\newcommand{\lbar}{\bar{l}}
\renewcommand{\l}{\lambda}
%\renewcommand{\qed}{\qquad \qedsymbol}
\newcommand{\R}{\mathbb{R}}
\newcommand{\RR}{\mathcal{R}}
\newcommand{\al}{\alpha}
\newcommand{\ve}{q}
\newcommand{\tg}{{tan}}
\newcommand{\m}{q}
\newcommand{\N}{N}
\newcommand{\ta}{{\tilde{a}}}
\newcommand{\tb}{{\tilde{b}}}
\newcommand{\tc}{{\tilde{c}}}
\newcommand{\tS}{{\tilde{S}}}
\newcommand{\tP}{{\tilde{P}}}
\newcommand{\tu}{{\tilde{u}}}
\newcommand{\tw}{{\tilde{w}}}
\newcommand{\tA}{{\tilde{A}}}
\newcommand{\tX}{{\tilde{X}}}
\newcommand{\tphi}{{\tilde{\phi}}}
\synctex=1
\frenchspacing %no extra space after periods
\begin{document}
\title{A Modified Boussinesq equation}
\author{Dan-Andrei Geba, Alexandrou Himonas, and David Karapetyan}
\address{Department of Mathematics, University of Rochester, Rochester, NY 14627}
\address{Department of Mathematics, University of Notre Dame, Notre Dame, IN 46556}
\address{Department of Mathematics, University of Notre Dame, Notre Dame, IN 46556}
\date{04/05/2011}
%
%
\subjclass[2000]{35B30, 35Q55, 35Q72}
\keywords{local well-posedness; ill-posedness.}
\maketitle
\tableofcontents
%
%
\section{Introduction}
%
We consider the initial value problem (ivp) for the fourth order modified Boussinesq
($B_4$) equation 
\begin{gather}
  u_{tt} + u_{xxxx} + (u^2)_{xx} = 0, \quad x \in \rr \ \text{or} \
  \ci, \ t \in \rr
  \label{eqn:mb-2}
  \\
  u(x,0) = u_{0}(x), \quad \p_t u(x, 0) = u_1(x), 
  \label{eqn:mb-init-data-2}
  \\
  \notag
  (u_0, u_1) \in
  H^{s}\times
  H^{s-2}
\end{gather}
%
%
We shall work in the periodic case first, and later generalize our results to
the non-periodic case. We first rewrite the $B_4$ ivp
\cref{eqn:mb-2}-\cref{eqn:mb-init-data-2} in integral form. 
Consider
the linear $B_4$ ivp
\begin{gather}
  u_{tt} + u_{xxxx} = 0,
  \label{lin-mb}
  \\
  u(x, 0)=u_{0}(x), \quad u_{t}(x,0) = u_{1}(x).
  \label{lin-mb-init-data-1}
\end{gather}
Taking the spatial Fourier transform of the linear $B_{4}$ ivp
\cref{lin-mb}-\cref{lin-mb-init-data-1} yields
%
%
\begin{gather}
  \wh{u_{tt}} + n^{4} \wh{u} = 0
  \label{four-trans-lin-mb}
  \\
  \wh{u}(n, 0) = \wh{u_{0}}(n), \quad \wh{u_{t}}(n, 0) = \wh{u_{1}}(n)
  \label{four-trans-lin-mb-data}
\end{gather}
For $n=0$, this admits the unique solution
%
%
\begin{equation}
  \label{nzerosoln}
\begin{split}
  \wh{u}(0,t) = \wh{u_{0}}(0) + \wh{u_{1}}(0)t
\end{split}
\end{equation}
%
%
For the case $n \neq 0$, we substitute the ansatz $e^{\lambda t}$ into
\cref{four-trans-lin-mb} to obtain the characteristic equation
%
%
\begin{equation*}
\begin{split}
  \lambda^{2} + n^{4} = 0
\end{split}
\end{equation*}
%
%
which gives 
%
%
\begin{equation*}
\begin{split}
  \lambda = \pm in^{2}.
\end{split}
\end{equation*}
%
Since
\cref{lin-mb} is second order, and $e^{in^{2}t}$ and $e^{-in^{2}t}$ are
linearly independent solutions to \cref{lin-mb}, it follows that $\left\{
e^{in^{2}t}, e^{-in^{2}t}
\right\}$ is a basis for all solutions of \cref{lin-mb}. Therefore, the general
solution of \cref{lin-mb} takes the form
%
%
\begin{equation}
  \label{explicit-homog-soln}
\begin{split}
  \wh{u}(n,t) = c_{1}e^{in^{2}t} + c_{2}e^{-in^{2}t}
\end{split}
\end{equation}
%
%
which, in conjunction with initial data 
\cref{four-trans-lin-mb-data}, implies
%
%
\begin{gather*}
   c_{1} + c_{2} = \wh{u_{0}}(n)
  \\
   in^{2}c_{1} - in^{2}c_{2} = \wh{u_{1}}(n).
\end{gather*}
%
%
Solving for $c_{1}$ and $c_{2}$, we obtain
%
%
\begin{gather*}
  c_{1} = \frac{1}{2} \wh{u_{0}}(n) + \frac{1}{2in^{2}}\wh{u_{1}}(n),
  \\
  c_{2} = \frac{1}{2} \wh{u_{0}}(n) - \frac{1}{2in^{2}}\wh{u_{1}}(n).
\end{gather*}
%
%
Substituting into \cref{explicit-homog-soln}, we obtain the unique solution for
$n \neq 0$ to
ivp \cref{four-trans-lin-mb}-\cref{four-trans-lin-mb-data}
%
%
\begin{equation}
  \label{lin-mb-four-soln}
\begin{split}
  \wh{u}(n, t) = \wh{u_{0}}(n) \frac{e^{in^{2}t} + e^{-in^{2}t}}{2} +
  \wh{u_{1}}(n)\frac{e^{in^{2}t} - e^{-in^{2}t}}{2 i n^{2}}.
\end{split}
\end{equation}
%
%
Notice that 
\begin{equation}
  \label{remov-sing}
\begin{split}
  \frac{e^{in^{2}t} - e^{-in^{2}t}}{2 i n^{2}} \Big |_{n =0} = t
\end{split}
\end{equation}
%
%
since $n=0$ is a removable singularity. Furthermore, 
%
%
\begin{equation*}
\begin{split}
  \frac{e^{in^{2}t} + e^{-in^{2}t}}{2} \Big |_{n = 0} =1.
\end{split}
\end{equation*}
%
%
Recalling \cref{nzerosoln}, it follows that \cref{lin-mb-four-soln}
is the unique solution to ivp
\cref{four-trans-lin-mb}-\cref{four-trans-lin-mb-data} for all $n \in \zz$.  
Hence, taking the inverse spatial Fourier transform of \cref{lin-mb-four-soln},
we obtain the unique solution to ivp \cref{lin-mb}-\cref{lin-mb-init-data-1}
%
%
\begin{equation*}
\begin{split}
  u(x,t)
  & = \frac{1}{2 \pi}
  \sum_{n \in \zz} e^{inx}\wh{u_{0}}(n) \frac{e^{in^{2}t} + e^{-in^{2}t}}{2}
  \\
  & + \frac{1}{2 \pi} \sum_{n \in \zz} e^{inx}
  \wh{u_{1}}(n) \frac{e^{in^{2}t} - e^{-in^{2}t}}{2i n^{2}}.
\end{split}\end{equation*}
%
%
%
%
%
Turning our attention now to the $B_{4}$ ivp
\cref{eqn:mb-2}-\cref{eqn:mb-init-data-2} and taking the spatial Fourier
transform yields 
%
%
\begin{gather}
  \wh{u_{tt}} + n^{4} \wh{u} = -\wh{(u^{2})_{xx}}
  \label{four-trans-mb}
  \\
  \wh{u}(n, 0) = \wh{u_{0}}(n), \quad \wh{u_{t}}(n, 0) = \wh{u_{1}}(n)
  \label{four-trans-mb-data}
\end{gather}
which for fixed $t$ is a second order ODE in $n$. 
We now need the following.
%
%
%%%%%%%%%%%%%%%%%%%%%%%%%%%%%%%%%%%%%%%%%%%%%%%%%%%%%
%
%
%                General Solution NonHomog 2nd Order Eqn
%
%
%%%%%%%%%%%%%%%%%%%%%%%%%%%%%%%%%%%%%%%%%%%%%%%%%%%%%
%
%
\begin{lemma}[Variation of Parameters]
\label{lem:nonhomog-ode-soln}
The general solution of the $2$nd order nonhomogeneous ODE 
%
%
\begin{equation}
  \label{2nd-order-ode}
\begin{split}
y'' + p(t)y' + q(t)y = g(t)
\end{split}
\end{equation}
%
%
can be written in the form
%
%
\begin{equation*}
\begin{split}
  y = c_{1}y_{1}(t) + c_{2}y_{2}(t) + y_{p}(t),
\end{split}
\end{equation*}
%
%
where $y_{1}$ and $y_{2}$ are linearly independent solutions to the
corresponding homogeneous equation (i.e. $g(t) = 0$), $c_{1}$ and $c_{2}$ are
arbitrary constants, and $y_{p}$ is some specific solution of the nonhomogeneous
equation. Furthermore, one such $y_{p}$ is given by
%
%
\begin{equation}
  \label{2nd-order-ansatz}
\begin{split}
  y_{p} = y_{1}v_{1} + y_{2} v_{2}
\end{split}
\end{equation}
%
%
where $v_{1}$ and $v_{2}$ are solutions to the system
\begin{gather}
  \label{cancel-rel-1}
  y_{1} v_{1}' + y_{2} v_{2}' = 0
  \\
  \label{cancel-rel-2}
  y_{1}' v_{1}' + y_{2}' v_{2}' = g(t).
\end{gather}
\end{lemma}
%
\begin{proof}
Substituting the ansatz \cref{2nd-order-ansatz} into
the left hand side of \cref{2nd-order-ode}, we obtain the expression
%
%
%
%
\begin{equation*}
\begin{split}
  (y_{1}v_{1} + y_{2}v_{2})'' + p(t)(y_{1}v_{1} + y_{2}v_{2})' +
  q(t)(y_{1}v_{1} + y_{2}v_{2}) 
\end{split}
\end{equation*}
%
%
or
%
%
\begin{equation*}
  \begin{split}
    & y_{1}'' v_{1} + 2y_{1}'v_{1}' + y_{1}v_{1}'' + y_{2}''v_{2} + 2y_{2}' v_{2}'
  + y_{2} v_{2}'' + p(t)y_{1}'v_{1} + p(t)y_{1}v_{1}'
  \\
  & + p(t)y_{2}'v_{2} + p(t)y_{2}v_{2}' + q(t)y_{1}v_{1} + q(t)y_{2}v_{2} =g
\end{split}
\end{equation*}
%
Collecting terms, this can be rewritten as
%
%
%
%
\begin{equation*}
\begin{split}
  & (y_{1}v_{1}'' + y_{1}' v_{1}') + (y_{2}v_{2}'' + y_{2}' v_{2}') + y_{1}''
  v_{1} + y_{2}'' v_{2} + (y_{1}'v_{1}' + y_{2}' v_{2}')
  \\
  & + p(t)\left(
  y_{1}'v_{1} + y_{2}' v_{2} + y_{1}v_{1}' + y_{2}v_{2}'
  \right) + q(t)\left( y_{1}v_{1} + y_{2}v_{2} \right)
  \end{split}
\end{equation*}
%
or
%
%
%
%
\begin{equation*}
\begin{split}
  & \cancel{(y_{1}v_{1}' + y_{2}v_{2}')'} + y_{1}''
  v_{1} + y_{2}'' v_{2} + \overbrace{(y_{1}'v_{1}' + y_{2}' v_{2}')}^{g}
  \\
  & + p(t)\left(
  y_{1}'v_{1} + y_{2}' v_{2} + y_{1}v_{1}' + y_{2}v_{2}'
  \right) + q(t)\left( y_{1}v_{1} + y_{2}v_{2} \right)
  \end{split}
\end{equation*}
%
%
or
%
%
\begin{equation*}
\begin{split}
  g + \cancel{v_{1}\left[ y_{1}'' + p(t)y_{1}' + q(t)y_{1} \right]} +
  \cancel{v_{2}\left[ y_{2}'' + p(t)y_{2}' + q(t)y_{2}
  \right]}
\end{split}
\end{equation*}
%
%
where the last cancellation is due to the fact that $y_{1}$ and $y_{2}$ are
solutions to the corresponding homogeneous equation of \cref{2nd-order-ode}
(i.e. $g(t) = 0$). This concludes the proof. 
\end{proof}
%
%
Rewriting the system \cref{cancel-rel-1}-\cref{cancel-rel-2}
as
  \begin{equation*}
  \begin{bmatrix}
    y_{1} & y_{2} \\
    y_{1}' & y_{2}'
  \end{bmatrix}
  \begin{bmatrix}
    v_{1}'
    \\
    v_{2}'
  \end{bmatrix}=
  \begin{bmatrix}
  0 \\
  g
  \end{bmatrix}
\end{equation*}
  we obtain 
  \begin{equation*}
\begin{bmatrix}
  v_{1}'
  \\
  v_{2}'
\end{bmatrix}=
\begin{bmatrix}
  -\frac{y_{2}g}{y_{1}y_{2}' - y_{1}' y_{2}} \\
  \frac{y_{1}g}{y_{1}y_{2}' - y_{1}' y_{2}}.
\end{bmatrix}
\end{equation*}
Integrating from $0$ to $t$, and setting $v_{1}(0) = v_{2}(0) = 0$, we see that
one particular solution is
%
%
\begin{equation*}
\begin{split}
\begin{bmatrix}
  v_{1}
  \\
  v_{2}
\end{bmatrix}=
\begin{bmatrix}
 -\int_{0}^{t} \frac{y_{2}g}{y_{1}y_{2}' - y_{1}' y_{2}} dt' \\
  \int_{0}^{t}\frac{y_{1}g}{y_{1}y_{2}' - y_{1}' y_{2}}dt'.
\end{bmatrix}
\end{split}
\end{equation*}
Therefore,
%
%
\begin{equation*}
\begin{split}
  y_{p} =  \int_{0}^{t}
  \frac{y_{2}(t)y_{1}(t') - y_{1}(t)y_{2}(t')}{y_{1}(t')y_{2}'(t') -
  y_{1}'(t') y_{2}(t')}g \ dt'.
\end{split}
\end{equation*}
%
%
%
%
%
Applying \cref{lem:nonhomog-ode-soln}, it follows that 
the unique solution to ivp
\cref{four-trans-mb}-\cref{four-trans-mb-data} is given by
%
%
\begin{equation*}
\begin{split}
\wh{u}(n, t)
& = \wh{u_{0}}(n) \frac{e^{in^{2}t} + e^{-in^{2}t}}{2} +
  \wh{u_{1}}(n)\frac{e^{in^{2}t} - e^{-in^{2}t}}{2 i n^{2}}
  \\
  & -
  \int_{0}^{t}\frac{e^{in^{2}(t-t')}-e^{-in^{2}(t-t')}}{2in^{2}}
  \wh{(u^{2})_{xx}}(n, t') dt'.
\end{split}
\end{equation*}
%
Hence, taking the inverse Fourier transform we obtain
%
\begin{equation}
  \begin{split}
    u(x,t)
    & = \frac{1}{2\pi}\sum_{n \in \zz} e^{inx} \wh{u_{0}}(n) \frac{e^{in^{2}t} + e^{-in^{2}t}}{2} 
    \\
    & + \frac{1}{2 \pi}\sum_{n \in \zz} e^{inx}
    \wh{u_{1}}(n)\frac{e^{in^{2}t} - e^{-in^{2}t}}{2 i n^{2}} 
    \\
    & - \frac{1}{2 \pi}\sum_{n \in \zz} e^{inx}
    \int_{0}^{t}\frac{e^{in^{2}(t-t')}-e^{-in^{2}(t-t')}}{2 i n^{2}}
    \wh{(u^{2})_{xx}}(n, t') dt'.
  \end{split}
  \label{eqn:pre-integral-form}
\end{equation}
%
Note that \cref{remov-sing} implies
%
%
\begin{equation*}
\begin{split}
  \frac{e^{in^{2}(t - t')} - e^{-in^{2}(t-t')}}{2 i n^{2}} \Big |_{n=0} = t-t'.
\end{split}
\end{equation*}
%
%
But 
%
%
\begin{equation*}
\begin{split}
  \wh{(u^{2})_{xx}}(0, t') = (-n^{2} \wh{u^{2}})(0, t') = 0.
\end{split}
\end{equation*}
%
%
Hence, \cref{eqn:pre-integral-form} simplifies to
%
\begin{equation}
  \begin{split}
    u(x,t)
    & = \frac{1}{2\pi}\sum_{n \in \zz} e^{inx} \wh{u_{0}}(n) \frac{e^{in^{2}t} + e^{-in^{2}t}}{2} 
    \\
    & + \frac{1}{2 \pi}\sum_{n \in \zz} e^{inx}
    \wh{u_{1}}(n)\frac{e^{in^{2}t} - e^{-in^{2}t}}{2 i n^{2}} 
    \\
    & + \frac{1}{4 i \pi}\sum_{n \in \zz} e^{inx}
    \int_{0}^{t}[e^{in^{2}(t-t')}-e^{-in^{2}(t-t')}]
    \wh{u^{2}}(n, t') dt'.
  \end{split}
  \label{eqn:integral-form}
\end{equation}
%
\begin{framed}
  \begin{remark}
For an alternative technique for formulating the $B_{4}$ ivp in integral form,
see the appendix.
\end{remark}
\end{framed}
Hence, we have rewritten the $B_{4}$ ivp
\cref{eqn:mb-2}-\cref{eqn:mb-init-data-2} in integral form, which we will now
localize in time. 
Let $\psi(t)$ be a cutoff function symmetric about the 
origin such that $\psi(t) = 1$ for $|t| \le 1$ and $\text{supp} \, \psi 
= [-2, 2 ]$.
Define $\psi_{\delta}(t) = \psi(t/\delta)$.  Multiplying both sides of expression
$\cref{eqn:integral-form}$ by $\psi_{\delta}(t)$, we obtain
%
%
\begin{align}
  & \psi_{\delta}u(x,t) 
  \label{1hh}
    \\
    & = \frac{1}{2 \pi}\psi_{\delta}(t)
    \sum_{n \in \zz} e^{inx} \wh{u_{0}}(n) \frac{e^{in^{2}t} + e^{-in^{2}t}}{2} 
    \\
    \label{2hh}
    & + \frac{1}{2 \pi}\psi_{\delta}(t) \sum_{n \in \zz} e^{inx}
    \wh{u_{1}}(n)\frac{e^{in^{2}t} - e^{-in^{2}t}}{2 i n^{2}} 
    \\
    \label{3hh}
    & + \frac{1}{4 i \pi}\psi_{\delta}(t) \sum_{n \in \zz} e^{inx}
    \int_{0}^{t}e^{in^{2}(t-t')}
    \wh{w}(n, t') dt'
    \\
    \label{4hh}
    & - \frac{1}{4 i \pi}\psi_{\delta}(t) \sum_{n \in \zz} e^{inx}
    \int_{0}^{t}e^{-in^{2}(t-t')}
    \wh{w}(n, t') dt'
  \end{align}
where $$w(x,t) = \frac{1}{2\pi} \sum_{n \in \zz}
e^{inx} \wh{u^{2}}(n,t).$$  Next, note that for $0 < \delta \le 1$, any $u$
solving
\begin{align}
  & u(x,t) 
    \\
    & = \frac{1}{2 \pi}\psi(t)
    \sum_{n \in \zz} e^{inx} \wh{u_{0}}(n) \frac{e^{in^{2}t} + e^{-in^{2}t}}{2} 
    \\
    & + \frac{1}{2 \pi}\psi(t) \sum_{n \in \zz} e^{inx}
    \wh{u_{1}}(n)\frac{e^{in^{2}t} - e^{-in^{2}t}}{2 i n^{2}} 
    \\
    \label{term-3}
    & + \frac{1}{4 i \pi} \psi_{\delta}(t) \sum_{n \in \zz} e^{inx}
    \int_{0}^{t}e^{in^{2}(t-t')}
    \wh{w}(n, t') dt'
    \\
    \label{term-4}
    & - \frac{1}{4 i \pi} \psi_{\delta} (t) \sum_{n \in \zz} e^{inx}
    \int_{0}^{t}e^{-in^{2}(t-t')}
    \wh{w}(n, t') dt'
  \end{align}
  solves \cref{1hh}-\cref{4hh} for $| t | \le \delta$. Furthermore, replacing
  $w$ with $w_{\delta} \doteq \psi_{2 \delta} w$ yields expressions equivalent
  to \cref{term-3} and \cref{term-4}, respectively. 
  Since \cref{term-3} is a \emph{global} relation in
  $t$, using Fourier inversion we can rewrite it as
%
%
\begin{equation*}
\begin{split}
  & \frac{1}{8 i \pi^{2}}  \psi_{\delta}(t) \sum_{n \in \zz} e^{inx} e^{in^{2}t}
  \int_{0}^{t} \int_{\rr} e^{it'(\tau - n^{2})} \wh{w_{\delta}}(n, \tau) d \tau dt'
\end{split}
\end{equation*}
%
%
which by Fubini and integration is equal to
%
%
\begin{equation*}
\begin{split}
  -\frac{1}{8 \pi^{2}}  \psi_{\delta}(t) \sum_{n \in \zz} \int_{\rr} e^{ixn}
  e^{in^{2}t} \frac{e^{it(\tau - n^{2})} -1}{\tau - n^{2}}\wh{w_{\delta}}(n, \tau) d \tau.
\end{split}
\end{equation*}
%
Next, we localize near the singular curve $\tau =  n^2$.  Multiplying the
integrand by $1 + \psi(\tau -
n^m) - \psi(\tau -
n^m) $ and
rearranging terms, we get
%
%
\begin{equation*}
	\begin{split}
	& - \frac{1}{8 \pi^2}  \psi_{\delta}(t) \sum_{n \in \zz} \int_\rr e^{ixn}  
		e^{it \tau} \frac{1 - \psi(\tau - n^{2}) 
}{\tau - n^{2}} \wh{w_{\delta}}(n, \tau) \ d \tau
		\\
		& + \frac{1}{8 \pi^2}  \psi_{\delta}(t) \sum_{n \in \zz} \int _\rr e^{i(xn + 
		t n^{2})}
		 \frac{1- \psi(\tau - n^{2})}{\tau - n^{2}} \wh{w_{\delta}}(n, \tau) \ d \tau
		\\
		& - \frac{1}{8 \pi^2}  \psi_{\delta}(t) \sum_{n \in \zz} \int_\rr
		e^{i(xn + t n^{2})}
		\frac{\psi(\tau - n^{2})\left[ e^{it(\tau - n^{2})}-1 
		\right]}{\tau - n^{2}} \wh{w_{\delta}}(n, \tau) \ d \tau
	\end{split}
\end{equation*}
%
%
which by a power series expansion of $[e^{it(\tau - n^{2})}-1]$ simplifies  
to
%
%
\begin{align}
		\label{main-int-expression'-2}
		& -\frac{1}{8 \pi^2}  \psi_{\delta}(t) \sum_{n\in \zz} \int_\rr e^{ixn}  
		e^{it \tau} \frac{1 - \psi(\tau -  n^{2}) 
}{\tau -  n^{2}} \wh{w_{\delta}}(n, \tau) \ d \tau
		\\
		\label{main-int-expression'-3}
		& + \frac{1}{8 \pi^2}  \psi_{\delta}(t) \sum_{n\in \zz} \int_\rr e^{i(xn + 
		t n^{2})}
		 \frac{1- \psi(\tau -  n^{2})}{\tau -  n^{2}} \wh{w_{\delta}}(n, \tau) \ d \tau
		\\
		\label{main-int-expression'-4}
		& - \frac{1}{8 \pi^2}  \psi_{\delta}(t) \sum_{k \ge 1} \frac{i^k t^k}{k!}
		\sum_{n \in \zz} \int_\rr e^{i(xn + t n^{2} )}
		\psi(\tau -  n^{2}) (\tau -  n^{2})^{k-1} \wh{w_{\delta}}(n, \tau).
\end{align}
%
Similarly, \cref{term-4} can be rewritten as
%
\begin{align}
		\label{main-int-expression''-2}
		& \frac{1}{8 \pi^2}  \psi_{\delta}(t) \sum_{n\in \zz} \int_\rr e^{ixn}  
		e^{it \tau} \frac{1 - \psi(\tau +  n^m) 
}{\tau +  n^m} \wh{w_{\delta}}(n, \tau) \ d \tau
		\\
		\label{main-int-expression''-3}
		&  - \frac{1}{8 \pi^2}  \psi_{\delta}(t) \sum_{n\in \zz} \int_\rr e^{i(xn - 
		t n^2)}
		 \frac{1- \psi(\tau +  n^m)}{\tau +  n^m} \wh{w_{\delta}}(n, \tau) \ d \tau
		\\
		\label{main-int-expression''-4}
		& + \frac{1}{8 \pi^2}  \psi_{\delta}(t) \sum_{k \ge 1} \frac{i^k t^k}{k!}
		\sum_{n \in \zz} \int_\rr e^{i(xn - t n^2 )}
		\psi(\tau +  n^2) (\tau +  n^m)^{k-1} \wh{w_{\delta}}(n, \tau).
\end{align}
%
%
Therefore, neglecting $\pi$ related constants, we have
%
%
%
\begin{align}
  & u(x,t)
  \label{main1-rel-term-0}
  \\
  \label{main1-rel-term-1}
  & = \psi(t) \sum_{n \in \zz} e^{inx} \wh{u_{0}}(n) \frac{e^{in^{2}t} + e^{-in^{2}t}}{2} 
  \\
  \label{main1-rel-term-2}
  & + \psi(t) \sum_{n \in \zz} e^{inx}
  \wh{u_{1}}(n)\frac{e^{in^{2}t} - e^{-in^{2}t}}{2 i n^{2}} 
  \\
  \label{main1-rel-term-3}
  & +  \psi_{\delta}(t)\sum_{a = \pm 1} \sum_{n\in \zz} \int_\rr e^{ixn}  
  e^{it \tau} \frac{1 - \psi(\tau -  an^{2}) 
}{\tau -  an^{2}} \wh{w_{\delta}}(n, \tau) \ d \tau
  \\
  \label{main1-rel-term-4}
  & + \psi_{\delta}(t) \sum_{a = \pm 1} \sum_{n\in \zz} \int_\rr e^{i(xn + 
  t an^{2})}
  \frac{1- \psi(\tau -  an^{2})}{\tau -  an^{2}} \wh{w_{\delta}}(n, \tau) \ d \tau
  \\
  \label{main1-rel-term-4.5}
  & +  \psi_{\delta}(t) \sum_{a = \pm 1}  \sum_{k \ge 1} \frac{i^k t^k}{k!}
  \sum_{n \in \zz} \int_\rr e^{i(xn + t an^{2} )}
  \psi(\tau -  an^{2}) (\tau -  an^{2})^{k-1} \wh{w_{\delta}}(n, \tau)
  \\
  \label{main1-rel-term-5}
  & \doteq Tu, \quad T=T_{u_0, u_1, \psi, \delta}.
\end{align}
%
%
%
%
%
%
%
%
%
%
%
%
We now introduce the following spaces. 
%
%
\begin{definition}
  Let $\mathcal{Y}$ be the space of functions $F(\cdot)$ such that
  \begin{enumerate}[(I)]
   \item{$F: \ci \times \rr \to \cc$}.
   \item{$F(x, \cdot) \in \mathcal{S}(\rr)$ for each $x \in \ci$}.
   \item{$F(\cdot, t) \in C^{\infty}(\ci)$for each $t \in \rr$}.
  \end{enumerate}
  For $s, b \in \rr$, $X_{s,b}$ denotes the completion of $\mathcal{Y}$ with
  respect to the norm
  %
  %
  \begin{equation}
  \begin{split}
    \|F\|_{X_{s,b}} = \left( \sum_{n \in \zz} (1 + |n|)^{2s} \int_{\rr}
    (1 + | | \tau | - n^{2} |)^{2b} |\wh{F}(n, \tau)|^{2} d \tau\right)^{1/2}.
  \end{split}
  \label{eqn:bous-norm}
  \end{equation}
  %
  \begin{framed}
    %
    %
    \begin{remark}
    Note that the norm here is different than the one used to for the KdV. In
    the KdV case, $b$ has to be equal to $1/2$. Furthermore, to get the embedding
    of the lemma below, one has to add an extra term, thus producing the
    $Y^{s}$ norm of CKSTT\@.
    \label{rem:alternate-space}
    \end{remark}
    %
  \end{framed}
    %
  %
  %
\end{definition}
%
The $X_{s,b}$ spaces have the following important embedding, whose proof is
provided in the appendix.
%
%
%%%%%%%%%%%%%%%%%%%%%%%%%%%%%%%%%%%%%%%%%%%%%%%%%%%%%
%
%
%               Embedding 
%
%
%%%%%%%%%%%%%%%%%%%%%%%%%%%%%%%%%%%%%%%%%%%%%%%%%%%%%
%
%
\begin{lemma}[Lemma 2.3 in~\cite{Farah:2009uq}]
  Let $b > 1/2$. Then $X_{s, b} \subset C(\rr, H^s)$ continuously. That is,
  there exists $c>0$ depending only on $b$ such that
%
%
\begin{equation*}
\begin{split}
  \| u \|_{C(\rr, H^s)} \doteq \sup_{t \in \rr} \| u(t) \|_{H^s} 
  \le c \| u \|_{X_{s,b}}.
\end{split}
\end{equation*}
%
\label{lem:embedding}
\end{lemma}
%
\begin{definition}
  Equip $H^{s} \times H^{s-2}$ with the 
  topology defined by the norm $\|(f_0, f_1)\|_{H^{s} \times H^{s-2}}
  = \|f_0\|_{H^{s}} + \|f_1\|_{H^{s-2}}$.
   We say that the $B_{4}$ ivp
  \cref{eqn:mb-2}-\cref{eqn:mb-init-data-2} is
	\emph{locally well posed} in
  $H^s \times H^{s-2}$ if 
	\begin{enumerate}
    \item Given $(u_{0}, u_{1}) \in H^{s} \times H^{s-2}$
      there exists $\delta>0$ depending on $(u_{0}, u_{1})$
      such that the Cauchy problem
      $\psi_{\delta} u = Tu$ has a solution $u \in C([-\delta,
      \delta], H^s) \cap X_{s,b}$ for $ |t| \le \delta$.
    \item The solution is unique in $C([-\delta, \delta], H^{s}) \cap
      X_{s,b}$.
    \item
      The data to solution map $(u_0, u_{1}) \mapsto u(x,t)$ is continuous. That
      is, given a sequence $\{(u_{0,n}, u_{1,n} ) \} \in H^{s} \times H^{s-2}$
      such that $$\|(u_{0}, u_{1})
      - (u_{0,n}, v_{1,n}) \|_{H^{s} \times
      H^{s-2}} \to 0,$$ with corresponding solutions $u_{n} \in
      C([-\delta_{n},
      \delta_{n}])$ and $u \in C([-\delta_{\infty}, \delta_{\infty}])$
      then there exists $0 < \delta \le \inf\left\{
      \delta_{n}, \delta_{\infty} \right\}$ such that $\psi_{\delta}u_{n} =
      Tu_{n}, \psi_{\delta}u = Tu$ and 
      $$\sup_{t \in [-\delta, \delta]}
      \|u(\cdot, t) - u_{n}(\cdot, t) \|_{H^s} \to 0.$$
  \end{enumerate}
	Otherwise, we say that the $B_{4}$ ivp is \emph{ill-posed}.
\end{definition}
%

%
We are now prepared to state the main result of this paper.
%
%
%
%
%%%%%%%%%%%%%%%%%%%%%%%%%%%%%%%%%%%%%%%%%%%%%%%%%%%%%
%
%
%	Main Result				
%
%
%%%%%%%%%%%%%%%%%%%%%%%%%%%%%%%%%%%%%%%%%%%%%%%%%%%%%
%
%
\begin{theorem}
\label{thm:main}
The initial value problem 
\cref{eqn:mb-2}-\cref{eqn:mb-init-data-2} is locally well-posed in $H^s$ for
$s >
-1/4$ and ill-posed for $s < -2$ in both the periodic and non-periodic cases.
%
%
\end{theorem} 
%
%
%
%
%%%%%%%%%%%%%%%%%%%%%%%%%%%%%%%%%%%%%%%%%%%%%%%%%%%%%
%
%
%                Proof of Thm
%
%
%%%%%%%%%%%%%%%%%%%%%%%%%%%%%%%%%%%%%%%%%%%%%%%%%%%%%
%
%
\section{The Periodic Case} 
\label{sec:periodic}
%
For an arbitrary Banach space $X$, we denote $B_{X}(r) \doteq \left\{f: \| f
\|_{X}< r \right\}$.
To prove well-posedness for the $B_4$ ivp we we will show that for initial data
$\vp \in H^{s}(\ci) \times H^{s-2}(\ci)$, $T = T_{\vp, \delta}$ is a contraction on
$B_{X_{s,b}}(M)$ for suitably small $\delta$, where $\delta >0$ is a constant
depending on $M$ and
$M > 0$ is a constant depending on $\|\vp\|_{H^{s} \times H^{s-2}}$. First, we
will estimate the $X_{s,b}$ norm of
\cref{main1-rel-term-1}-\cref{main1-rel-term-5}. 
The Picard fixed point theorem
will then yield a unique $u \in X_{s,b}$ satisfying $u = Tu$.  An application of
\cref{lem:embedding} will then imply the existence of a unique $u \in
C([-\delta, \delta], H^s(\ci)) \cap X_{s,b}$ solving $\psi_{\delta} u = Tu$ for
$| t | \le \delta$. Local Lipschitz continuity of the flow map will follow from
estimates used to establish the contraction mapping. That is, we will show that
for given $(u_{0}, u_{1}), (v_{0}, v_{1}) \in H^{s} \times H^{s-2}$ there exists
a $\delta > 0$ depending on $\|(u_{0}, u_{1})\|_{H^{s} \times H^{s} \times
H^{s-2}}, \|(v_{0}, v_{1}) \|_{H^{s} \times H^{s-2}}$ such that 
$\psi_{\delta} u = Tu, \psi_{\delta} v = Tv$ and  $$\sup_{t \in [-\delta,
\delta]}\|u(\cdot, t) - v(\cdot, t) \|_{H^{s}} \le c \left( \|u_{0} - v_0
\|_{H^{s}} + \|u_{1} - v_1 \|_{H^{s-2}} \right)$$ for some $c > 0$. Note that
this will imply continuity of the data to solution map.
%
%
%
%
%
%
%
%
\begin{framed}
\begin{remark}
\label{1rem:no-delta-lin}
For well-posedness for large data, it
is crucial that the cutoff for the linear term not depend on $\delta$. To
see this, we estimate \cref{main1-rel-term-1}. We have
%
%
\begin{equation*}
  \begin{split}
    \wh{\cref{main1-rel-term-1}}
    = \frac{\wh{\psi_{\delta}}(\tau -
    n^{2})\wh{u_{0}}(n)}{2} + \frac{\wh{\psi_{\delta}}(\tau +
    n^{2})\wh{u_{0}}(n)}{2}.
  \end{split}
\end{equation*}
%
%
Hence, substituting and applying the inequality 
%
%
\begin{equation}
  \label{1square-ineq}
\begin{split}
(a + b)^{2} \le 4(a^{2} +
b^{2}),\ a, b \in \rr,
\end{split}
\end{equation}
%
%
gives
%
%
\begin{align}
  & \| \cref{main1-rel-term-1} \|_{X_{s,b}}^{2} 
    \notag
    \\
    & = \sum_{n \in \zz}(1 + |n|)^{2s} \int_{\rr}\left( 1 + | | \tau
    |-n^{2} | \right)^{2b} | \frac{\wh{\psi_{\delta}}(\tau - n^{2})\wh{u_{0}(n)}}{2} +
    \frac{\wh{\psi_{\delta}}(\tau + n^{2})\wh{u_{0}}(n)}{2} |^{2} d \tau
    \notag
    \\
    & \le \sum_{n \in \zz} \left( 1 + |n| \right)^{2s} | \wh{u_{0}}(n)
    |^{2} \int_{\rr} | \wh{\psi_{\delta}}(\tau - n^{2}) |^{2}\left( 1 + | | \tau | -
    n^{2} | \right)^{2b} d \tau
    \label{1u-0-norm-comp-1}
    \\
    & + \sum_{n \in \zz} \left( 1 + |n| \right)^{2s} | \wh{u_{0}}(n)
    |^{2} \int_{\rr} | \wh{\psi_{\delta}}(\tau + n^{2}) |^{2}\left( 1 + | | \tau | -
    n^{2} | \right)^{2b} d \tau.
    \label{1u-0-norm-comp-3}
\end{align}
%
%
Since (see appendix)
\begin{equation}
  \begin{split}
    | | \tau | - n^{2} | \le \min\left\{| \tau - n^{2} |, | \tau + n^{2} |
    \right\},
  \end{split}
  \label{1eqn:norm-key-ineq}
\end{equation}
%
we bound \cref{1u-0-norm-comp-1} by
%
%
\begin{equation}
  \label{1uuy}
  \begin{split}
    & \sum_{n \in \zz} \left( 1 + |n| \right)^{2s} | \wh{u_{0}}(n)
    |^{2} \int_{\rr} | \wh{\psi_{\delta}}(\tau - n^{2}) |^{2}\left( 1 +  | \tau  -
    n^{2} | \right)^{2b} d \tau
    \\
    & = \sum_{n \in \zz} \left( 1 + |n| \right)^{2s} | \wh{u_{0}}(n)
    |^{2} \int_{\rr} | \wh{\psi_{\delta}}(\tau') |^{2}\left( 1 +  | \tau'|
    \right)^{2b} d \tau.
    %\\
    %& \le c_{\psi} \sum_{n \in \zz} \left( 1 + |n| \right)^{2s} | \wh{u_{0}}(n)
    %|^{2}
    %\\
    %& = c_{\psi} \| u_{0} \|_{H^{s}}^{2}
  \end{split}
\end{equation}
%
But 
%
%
\begin{equation}
  \label{cutoff-scaling}
\begin{split}
  \wh{\psi_{\delta}}(\tau)
  = \int_{\rr} e^{-it \tau} \psi_{\delta}(t) dt
  & = \int_{\rr} e^{-it \tau} \psi(t/\delta) dt
  \\
  & = \frac{\delta}{2} \int_{\rr} e^{-is \tau \delta/2} \psi(s) ds
  \\
  & = \frac{\delta}{2} \wh{\psi}(\tau \delta/2)
\end{split}
\end{equation}
%
%
which gives
%
%
\begin{equation}
  \label{cutoff-scaling-p}
\begin{split}
  \int_{\rr} (1 + | \tau |)^{2b} | \wh{\psi_{\delta}}(\tau) |^{2} d
  \tau
  & = \left (\frac{\delta}{2} \right )^{2} \int_{\rr} (1 + | \tau |)^{2b} | \wh{\psi}(\delta \tau
  /2) |^{2} d
  \tau
  \\
  & =  \frac{\delta}{2}  \int_{\rr} (1 + | 2 \lambda /\delta |)^{2b} | \wh{\psi}(\lambda) | d \lambda
  \\
  & \le \frac{\delta}{2} \int_{\rr} (1/\delta + | 2 \lambda / \delta |)^{2b} | \wh{\psi}(\lambda) | d
  \lambda, \quad 0 < \delta \le 1
  \\
  & = \delta^{1-2b} c_{\psi,b}.
\end{split}
\end{equation}
where $c_{\psi, b}$ is a constant depending only on $\psi$ and $b$.
Therefore, \cref{1uuy} is bounded by 
\begin{equation*}
\begin{split}
  \delta^{1-2b} c_{\psi,b} \| u_{0} \|^{2}.
\end{split}
\end{equation*}
%
%
%
The
term \cref{1u-0-norm-comp-3} is bounded in similar fashion. Therefore, 
$\|\cref{main1-rel-term-1}\|_{X_{s,b}}^{2} \le c_{\psi,b} \delta^{1-2b}
\|u_{0}\|_{H^s}^2$. Taking square roots of both sides gives
%
%
\begin{equation}
  \begin{split}
    \|\cref{main1-rel-term-1}\|_{X_{s,b}} \le c_{\psi,b} \delta^{1/2-b}
    \|u_{0}\|_{H^s}.
  \end{split}
  \label{1eqn:u-0-fin-est}
\end{equation}
This is a problematic relation for $b > 1/2$, since we have blowup of our bound
as $\delta$ approaches $0$.
\end{remark}
\end{framed}
%
\subsubsection{Estimate for \cref{main1-rel-term-1}}
\label{ssec:est-init-term-1}
We have
%
%
\begin{equation*}
  \begin{split}
    \wh{\cref{main1-rel-term-1}}
    = \frac{\wh{\psi}(\tau -
    n^{2})\wh{u_{0}}(n)}{2} + \frac{\wh{\psi}(\tau +
    n^{2})\wh{u_{0}}(n)}{2}.
  \end{split}
\end{equation*}
%
%
Hence, substituting and applying the inequality 
%
%
\begin{equation}
  \label{square-ineq}
\begin{split}
(a + b)^{2} \le 4(a^{2} +
b^{2}),\ a, b \in \rr,
\end{split}
\end{equation}
%
%
gives
%
%
\begin{align}
  & \| \cref{main1-rel-term-1} \|_{X_{s,b}}^{2} 
    \notag
    \\
    & = \sum_{n \in \zz}(1 + |n|)^{2s} \int_{\rr}\left( 1 + | | \tau
    |-n^{2} | \right)^{2b} | \frac{\wh{\psi}(\tau - n^{2})\wh{u_{0}(n)}}{2} +
    \frac{\wh{\psi}(\tau + n^{2})\wh{u_{0}}(n)}{2} |^{2} d \tau
    \notag
    \\
    & \le \sum_{n \in \zz} \left( 1 + |n| \right)^{2s} | \wh{u_{0}}(n)
    |^{2} \int_{\rr} | \wh{\psi}(\tau - n^{2}) |^{2}\left( 1 + | | \tau | -
    n^{2} | \right)^{2b} d \tau
    \label{u-0-norm-comp-1}
    \\
    & + \sum_{n \in \zz} \left( 1 + |n| \right)^{2s} | \wh{u_{0}}(n)
    |^{2} \int_{\rr} | \wh{\psi}(\tau + n^{2}) |^{2}\left( 1 + | | \tau | -
    n^{2} | \right)^{2b} d \tau.
    \label{u-0-norm-comp-3}
\end{align}
%
%
Since (see appendix)
\begin{equation}
  \begin{split}
    | | \tau | - n^{2} | \le \min\left\{| \tau - n^{2} |, | \tau + n^{2} |
    \right\},
  \end{split}
  \label{eqn:norm-key-ineq}
\end{equation}
%
we bound \cref{u-0-norm-comp-1} by
%
%
\begin{equation}
  \label{uuy}
  \begin{split}
    & \sum_{n \in \zz} \left( 1 + |n| \right)^{2s} | \wh{u_{0}}(n)
    |^{2} \int_{\rr} | \wh{\psi}(\tau - n^{2}) |^{2}\left( 1 +  | \tau  -
    n^{2} | \right)^{2b} d \tau
    \\
    & = \sum_{n \in \zz} \left( 1 + |n| \right)^{2s} | \wh{u_{0}}(n)
    |^{2} \int_{\rr} | \wh{\psi}(\tau') |^{2}\left( 1 +  | \tau'|
    \right)^{2b} d \tau
    \\
    & =c_{\psi,b} \| u_{0} \|^{2}.
\end{split}
\end{equation}
%
%
%
%
\begin{framed}
%
%
\begin{remark}
Note that, for $b \ge 0$,
%
%
\begin{equation*}
\begin{split}
  c_{\psi, b}
  & = \|\wh{\psi}(\tau') \left( 1 +  | \tau'|
  \right)^{b} \|_{L^{2}}
  \\
  & \ge \| \wh{\psi}(\tau') \|_{L^{2}}
  \\
  & = \sqrt{2 \pi} \|
  \psi(t) \|_{L^{2}}
  \\
  & > 1.
\end{split}
\end{equation*}
%
%
Hence, the value of $c_{\psi, b}$ will be no help for handling large data, since there is
no way we may choose a cutoff symmetric around the origin and satisfying
$\psi(t) =1$ for $| t | \le 1/2$ such that $c_{\psi, b}$ is arbitrarily small. 
\label{rem:value-schwartz-const}
\end{remark}
%
%
\end{framed}
%
%
%
%
The
term \cref{u-0-norm-comp-3} is bounded in similar fashion. Therefore, 
$\|\cref{main1-rel-term-1}\|_{X_{s,b}}^{2} \le c_{\psi,b} 
\|u_{0}\|_{H^s}^2$. Taking square roots of both sides gives
%
%
\begin{equation}
  \begin{split}
    \|\cref{main1-rel-term-1}\|_{X_{s,b}} \le c_{\psi,b} 
    \|u_{0}\|_{H^s}.
  \end{split}
  \label{eqn:u-0-fin-est}
\end{equation}
%
%
%
\subsubsection{Estimate for \cref{main1-rel-term-2}}
\label{ssec:est-init-term-2}
We have
%
%
\begin{equation*}
  \begin{split}
    \wh{\psi(t)S_{t}u_{1}}^{x}(n, t)
    & = \psi(t) \wh{u_{1}}(n) \frac{e^{in^2 t} - e^{-in^{2}t}}{2i n^{2}}
    \\
    & = \frac{\psi(t) \wh{u_{1}}(n)e^{in^{2}t}}{2i n^{2}} - \frac{\psi(t)
    \wh{u_{1}}(n)e^{-in^{2}t}}{2i n^{2}}  
  \end{split}
\end{equation*}
%
%
and
%
%
\begin{equation*}
  \begin{split}
    \wh{\psi(t) S_{t}u_{1}}(n, \tau) = \frac{\wh{\psi}(\tau -
    n^{2})\wh{u_{1}}(n)}{2i n^{2}} - \frac{\wh{\psi}(\tau + n^{2})\wh{u_{1}}(n)}{2i
    n^{2}}.
  \end{split}
\end{equation*}
%
Note that 
%
\begin{equation*}
  \begin{split}
    \wh{\psi(t)S_{t}u_{1}}^{x}(0, t)
    & = \psi(t) \wh{u_{1}}(0) t
      \end{split}
\end{equation*}
and so 
%
%
\begin{equation*}
  \begin{split}
    \wh{\psi(t) S_{t}u_{1}}(0, \tau) = i \frac{d}{d \tau} \wh{\psi}(\tau)
    \wh{u_{1}}(0).
  \end{split}
\end{equation*}
%
Hence, substituting and applying \cref{square-ineq}, we have
%
%
\begin{equation}
  \begin{split}
    \| \cref{main1-rel-term-2} \|_{X_{s,b}}^{2} 
    & = \sum_{n \in \zzdot}(1 + |n|)^{2s} \int_{\rr}\left( 1 + | | \tau
    |-n^{2} | \right)^{2b} | \frac{\wh{\psi}(\tau - n^{2})\wh{u_{1}(n)}}{2i
    n^{2}} -
    \frac{\wh{\psi}(\tau + n^{2})\wh{u_{1}}(n)}{2i n^{2}} |^{2} d \tau
    \\
    & + |\wh{u_{1}}(0)|^{2} \int_{\rr} (1 + | \tau |) | i \frac{d}{d \tau}
    \wh{\psi}(\tau)|^{2} d \tau
    \\
    & \le \sum_{n \in \dot{\zz}} n^{-4} \left( 1 + |n| \right)^{2s} | \wh{u_{1}}(n)
    |^{2} \int_{\rr} | \wh{\psi}(\tau - n^{2}) |^{2}\left( 1 + | | \tau | -
    n^{2} | \right)^{2b} d \tau
    \\
    & + \sum_{n \in \dot{\zz}} n^{-4} \left( 1 + |n| \right)^{2s} | \wh{u_{1}}(n)
    |^{2} \int_{\rr} | \wh{\psi}(\tau + n^{2}) |^{2}\left( 1 + | | \tau | -
    n^{2} | \right)^{2b} d \tau
    \\
    & + |\wh{u_{1}}(0)|^{2} \int_{\rr} (1 + | \tau |)^{2b} |\frac{d}{d \tau}
    \wh{\psi}(\tau)|^2 d \tau.
\end{split}
\label{u-1-norm-comp-pre}
\end{equation}
%
%
Applying the inequality
%
%
\begin{equation*}
\begin{split}
  \frac{(1 + |n|)^{2s}}{n^{4}} \le \frac{(1 + |n|)^{2s}}{\frac{1}{16}(1 +
  |n|)^{4}} = 16 (1 + | n |)^{2(s-2)},  \quad s \in \rr, \quad n \ge 1
\end{split}
\end{equation*}
%
to \cref{u-1-norm-comp-pre} gives
%
\begin{equation}
  \begin{split}
    \|  \cref{main1-rel-term-2}\|_{X_{s,b}}^{2} 
    & \lesssim \sum_{n \in \dot{\zz}} \left( 1 + |n| \right)^{2(s-2)} | \wh{u_{1}}(n)
    |^{2} \int_{\rr} | \wh{\psi}(\tau - n^{2}) |^{2}\left( 1 + | | \tau | -
    n^{2} | \right)^{2b} d \tau
    \\
    & + \sum_{n \in \dot{\zz}} \left( 1 + |n| \right)^{2(s-2)} | \wh{u_{1}}(n)
    |^{2} \int_{\rr} | \wh{\psi}(\tau + n^{2}) |^{2}\left( 1 + | | \tau | -
    n^{2} | \right)^{2b} d \tau
    \\
    & + |\wh{u_{1}}(0)|^{2} \int_{\rr} (1 + | \tau |)^{2b} |\frac{d}{d \tau}
    \wh{\psi}(\tau)|^2 d \tau.
\end{split}
\label{u-1-norm-comp}
\end{equation}
%
%
Applying \cref{eqn:norm-key-ineq},
we bound the first term of
\cref{u-1-norm-comp} by
%
%
%
\begin{equation*}
  \begin{split}
    & \sum_{n \in \dot{\zz}} \left( 1 + |n| \right)^{2(s-2)} | \wh{u_{1}}(n)
    |^{2} \int_{\rr} | \wh{\psi}(\tau - n^{2}) |^{2}\left( 1 +  | \tau  -
    n^{2} | \right)^{2b} d \tau
    \\
    & = \sum_{n \in \dot{\zz}} \left( 1 + |n| \right)^{2(s-2)} | \wh{u_{1}}(n)
    |^{2} \int_{\rr} | \wh{\psi}(\tau') |^{2}\left( 1 +  | \tau'|
    \right)^{2b} d \tau
    \\
    & \le c_{\psi,b}  \| u_{1} \|_{H^{s-2}}^{2}. 
  \end{split}
\end{equation*}
%
%
where the last step follows from 
\cref{cutoff-scaling}-\cref{cutoff-scaling-p}. Applying
\cref{eqn:norm-key-ineq} again, the
second term of \cref{u-1-norm-comp} is bounded by
\begin{equation*}
  \begin{split}
    & \sum_{n \in \dot{\zz}} \left( 1 + |n| \right)^{2(s-2)} | \wh{u_{1}}(n)
    |^{2} \int_{\rr} | \wh{\psi}(\tau + n^{2}) |^{2}\left( 1 +  | \tau  +
    n^{2} | \right)^{2b} d \tau
    \\
    & = \sum_{n \in \dot{\zz}} \left( 1 + |n| \right)^{2(s-2)} | \wh{u_{1}}(n)
    |^{2} \int_{\rr} | \wh{\psi}(\tau') |^{2}\left( 1 +  | \tau'|
    \right)^{2b} d \tau
    \\
    & \le c_{\psi,b} \| u_{1} \|_{H^{s-2}}^{2}
  \end{split}
\end{equation*}
while the third term is bounded by  
%
%
\begin{equation*}
\begin{split}
  c_{\psi,b}  \| u_{1} \|_{H^{s-2}}^{2}.
\end{split}
\end{equation*}
%
%
Therefore, 
$\|\cref{main1-rel-term-2}|_{X_{s,b}}^{2} \le c_{\psi,b} 
\|u_{1}\|_{H^{s-2}}^2$ and
taking square roots of both sides gives
%
%
\begin{equation}
  \label{u-1-est}
  \begin{split}
    \|\cref{main1-rel-term-2}|_{X_{s,b}} \le c_{\psi,b} 
    \|u_{1}\|_{H^{s-2}}.
  \end{split}
\end{equation}
%
%
%
%
\subsubsection{Estimate for \cref{main1-rel-term-3}.}
%
%
%%%%%%%%%%%%%%%%%%%%%%%%%%%%%%%%%%%%%%%%%%%%%%%%%%%%%
%
%
%			Schwartz Multiplier	
%
%
%%%%%%%%%%%%%%%%%%%%%%%%%%%%%%%%%%%%%%%%%%%%%%%%%%%%%
%
%
%
We now need the following.
%
\begin{lemma}
\label{lem:schwartz-mult}
For $s, b \in \rr$, we have
%
%
\begin{equation}
	\label{schwartz-mult}
	\begin{split}
    \|\psi_{\delta} f \|_{X_{s,b}} \le c_{\psi, b} \delta^{1/2-b} \|f \|_{X_{s,b}}.
	\end{split}
\end{equation}
%
%
\end{lemma}
%
Hence
%
%
\begin{equation}
  \label{yu}
	\begin{split}
		\|\cref{main1-rel-term-3}\|_{X_{s,b}} 
    & \le c_{\psi, b} \delta^{1/2 - b}
    \sum_{a = \pm 1} \| \sum_{n \in \zz}  e^{ixn} \int_\rr 
		e^{it \tau} \frac{1 - a\psi (\tau - an^{m} ) 
}{\tau - an^{m}} \wh{w_{\delta}}(n, \tau) \ 
		d \tau\|_{X_{s,b}}.
			\end{split}
\end{equation}
%
But
%
%
\begin{equation}
\label{main-int2-est-X-s-part}
\begin{split}
  & \| \sum_{n \in \zz} e^{ixn} \int_\rr 
		e^{it \tau} \frac{1 - a\psi (\tau - an^{m} ) 
  }{\tau - an^{m}} \wh{w_{\delta}}(n, \tau) \ 
		d \tau\|_{X_{s,b}}
		\\
    & = \sum_{a = \pm 1}\left( \sum_{n \in \zz} \left (1 + |n| \right )^{2s} \int_\rr
    (1 + |  |\tau| - n^{m}|)^{2b} \left | \frac{1 - a\psi(\tau - an^{2 
})}{\tau - an^{m}} 
     \wh{w_{\delta}}(n, \tau) \right |^2 \ d 
		\tau \right)^{1/2}
		\\
    & \le \sum_{a = \pm 1}
    \left( \sum_{n \in \zz} \left (1 + |n| \right )^{2s} \int_{| \tau - an^{m}| \ge 1}
    (1 + | |\tau| - n^{m}|)^{2b} \frac{|  \wh{w_{\delta}}(n, \tau)|^2}{|\tau - an^{m}|^2} 
		\ d 
		\tau \right)^{1/2}
  \end{split}
\end{equation}
Applying the inequality
\begin{equation}
	\label{one-plus-ineq}
	\begin{split}
		& \frac{1}{|j|} \le \frac{2}{1 + |j|} , 
		\qquad |j| \ge 1
   	\end{split}
\end{equation}
and \cref{eqn:norm-key-ineq} we bound this by
\begin{equation}
  \label{pre-bi-est}
  \begin{split}
		& 4 \left( \sum_{n \in 
		\zz} \left (1 + |n| \right )^{2s} \int_\rr
    \frac{| \wh{w_{\delta}}(n, \tau) |^2}{(1+ |  |\tau| - 
    n^{m}|)^{2(1-b)}} 
		 \ d \tau 
		\right)^{1/2}
\end{split}
\end{equation}
%
%
%
where $w = \psi_{2 \delta} u^{2}$.
We now need the following, whose proof can be found in the appendix.
%
%
%%%%%%%%%%%%%%%%%%%%%%%%%%%%%%%%%%%%%%%%%%%%%%%%%%%%%
%
%
%               localisation decay est 
%
%
%%%%%%%%%%%%%%%%%%%%%%%%%%%%%%%%%%%%%%%%%%%%%%%%%%%%%
%
%
\begin{lemma}
  Let $\psi(t) \in \mathcal{S}(\rr)$. If $-1/2 < b' \le b < 1/2$, then for any
  $0 < \delta \le 1$
  %
  %
  \begin{equation*}
  \begin{split}
    \| \psi(t/\delta) u \|_{X_{s,b'}} \lesssim_{\psi, b, b'} \delta^{b - b'} \| u
    \|_{X_{s,b}}.
  \end{split}
  \end{equation*}
  %
  %
\label{lem:time-local-est}
\end{lemma}
%
%
Let $b' = -(1-b)$ and apply \cref{lem:time-local-est} to
\cref{pre-bi-est} to obtain the bound
%
\begin{equation}
  \label{gff}
  \begin{split}
    & c_{\psi, a, b} \delta^{-a + (1-b)} \left( \sum_{n \in 
		\zz} \left (1 + |n| \right )^{2s} \int_\rr
		\frac{|\wh{w}(n, \tau) |^2}{(1+ |  |\tau| - 
    n^{m}|)^{2a}} 
		 \ d \tau 
		\right)^{1/2}
\end{split}
\end{equation}
where $ -1/2 < -(1-b) \le -a < 1/2$. We now need the following bilinear
estimate, whose proof we leave for later.
%
%
%%%%%%%%%%%%%%%%%%%%%%%%%%%%%%%%%%%%%%%%%%%%%%%%%%%%%
%
%
%				Proposition
%
%
%%%%%%%%%%%%%%%%%%%%%%%%%%%%%%%%%%%%%%%%%%%%%%%%%%%%%
%
%
\begin{proposition}[Theorem 1.1 in~\cite{Farah:2009uq}]
\label{prop:bilinear-est}
	%
	%
	If $a > 1/4$, $b > 1/2$, and $s \ge -a/2$, 
  then 
	\begin{equation}
    \| w_{fg} \|_{X_{s,-a}}
		    \le c_{a} \|f\|_{X_{s,b}} \|g\|_{X_{s,b}}
	\end{equation}
  where $w_{fg}(x,t)$ = $fg (x,t)$.
%
%
%
%
\end{proposition}
%
Applying \cref{prop:bilinear-est} to \cref{gff}, it follows that
if $1/4 < a \le 1-b < 1/2$ and $s \ge -a/2$ then \cref{gff} is bounded by
%
%
%
%
\begin{equation*}
\begin{split}
c_{\psi, b} 
    \delta^{1-b -a}\|u\|^{2}_{X_{s,b}}
\end{split}
\end{equation*}
%
%
which we apply to \cref{yu} to obtain the estimate
%
\begin{equation}
	\begin{split}
    \|\cref{main1-rel-term-3}\|_{X_{s,b}} \le c_{\psi, b} 
    \delta^{3/2-2b -a}\|u\|^{2}_{X_{s,b}}.
	\end{split}
\end{equation}
%
%
%
Fixing $0 < \ee < 1/12$, and setting $a = 1/2 - 3 \ee$, $b = 1/2 + \ee$, and $s
\ge -1/4 + 2 \ee$, we obtain 
%
%
\begin{equation*}
	\begin{split}
    \|\cref{main1-rel-term-3}\|_{X_{s,b}} \le c_{\psi, b} 
    \delta^{\ee}\|u\|^{2}_{X_{s,b}},
    \quad 0 < \ee < 1/12, \ b = 1/2 + \ee, \ s \ge -1/4 + 2 \ee 
	\end{split}
\end{equation*}
%
%
or
%
%
\begin{equation}
	\label{main-int2-est}
	\begin{split}
    \|\cref{main1-rel-term-3}\|_{X_{s,b}} \le c_{\psi, b} 
    \delta^{b-1/2}\|u\|^{2}_{X_{s,b}}, \quad b > 1/2,  \ s \ge -1/4 + 2(b -1/2) 
	\end{split}
\end{equation}

%
%
\subsubsection{Estimate for \cref{main1-rel-term-4}.}
Letting $$f(x,t) = \sum_{a = \pm 1} \psi(t) \sum_{n \in \zz} e^{i\left( xn +
atn^{m} \right)} 
\int_\rr \frac{1 - \psi\left( \lambda - an^{m} \right)}{\lambda - an^{m}} 
\wh{w_{\delta}} \left( n, \lambda \right) \ d \lambda,$$ we have
%
%
\begin{equation*}
	\begin{split}
		& \wh{f^x}(n, t) = \psi(t) e^{aitn^{m}} \int_\rr
		\frac{1 - \psi\left( \lambda - an^{m} \right)}{\lambda - an^{m}} 
		\wh{w_{\delta}}(n, \lambda) \ d \lambda
	\end{split}
\end{equation*}
and
\begin{equation*}
	\begin{split}
		 \wh{f}\left( n, \tau \right)
		 & = \int_\rr e^{-it\left( \tau - an^{m} 
		\right)} \psi(t) \int_\rr \frac{1 - a\psi\left( 
		\lambda - an^{m} 
		\right)}{\lambda - an^{m}} \wh{w_{\delta}}(n, \lambda) \ d \lambda d \tau
		\\
    & = \wh{\psi}\left( \tau - an^{m} \right) \int_\rr 
		\frac{1 - a\psi\left( 
		\lambda - an^{m} 
		\right)}{\lambda - an^{m}} \wh{w_{\delta}}(n, \lambda) \ d \lambda.
	\end{split}
\end{equation*}
Therefore,
%
%
\begin{equation}
  \label{iu}
	\begin{split}
		& \| \cref{main1-rel-term-4} \|_{X_{s,b}} 
		\\
    & \le c_{\psi, b} \delta^{1/2 - b}
    \sum_{a = \pm 1} \left( \sum_{n \in \zz} \left (1 + |n| \right)^{2s}
    \int_\rr \left( 1 + | |\tau| - n^{m} \right )^{2b} | | \wh{\psi}\left(
    \tau - an^{m} \right) |^2 \ d \tau \right.
		\\
		& \times \left . |
		\int_\rr \frac{1 - \psi\left( \lambda - an^{m} \right)}{\lambda -
		an^{m}} \wh{w_{\delta}}(n, \lambda) \ d \lambda |^2  \right)^{1/2}.
  \end{split}
\end{equation}
Applying \cref{eqn:norm-key-ineq}, we have the bound
%
%
\begin{equation*}
\begin{split}
& \sum_{a = \pm 1} 
\int_\rr \left( 1 + | |\tau| - n^{m} \right )^{2b} | | \wh{\psi}\left(
    \tau - an^{m} \right) |^2 \ d \tau 
    \\
    & \le  \sum_{a = \pm 1} 
    \int_\rr \left( 1 + |\tau - an^{m} | \right )^{2b} | | \wh{\psi}\left(
    \tau - an^{m} \right) |^2 \ d \tau 
    \\
    & = c_{\psi, b}.
\end{split}
\end{equation*}
%
Hence, the right hand side of \cref{iu} is bounded by
%
\begin{equation}
  \begin{split}
    & c_{\psi, b} \delta^{1/2 -b}  \sum_{a = \pm 1}\left( \sum_{n \in \zz} \left (1 + |n| \right )^{2s} | \int_\rr
		\frac{1 - \psi\left( \lambda - an^{m} \right)}{\lambda - an^{m}}
		\wh{w_{\delta}}(n, \lambda) \ d\lambda |^2 \right)^{1/2}
		\\
    & \simeq \delta^{1/2 -b} \sum_{a = \pm 1} \left( \sum_{n \in \zz} \left (1 + |n| \right )^{2s}  \left ( \int_\rr
		\frac{1 - \psi\left( \lambda - an^{m} \right)}{|\lambda - an^{m}|}
		|\wh{w_{\delta}}(n, \lambda) | \ d\lambda \right )^2 \right)^{1/2}
		\\
    & \le \delta^{1/2 - b}\sum_{a = \pm 1} \left( \sum_{n \in \zz} \left (1 + |n| \right )^{2s}  \left ( \int_{| \lambda - 
		an^{m} | \ge 1}
		\frac{|\wh{w_{\delta}}(n, \lambda) |}{|\lambda - an^{m}|}
		\ d\lambda \right )^2 \right)^{1/2}.
  \end{split}
\end{equation}
Applying \cref{one-plus-ineq} then \cref{eqn:norm-key-ineq} we bound this by
\begin{equation}
  \label{fh}
\begin{split}
  4 \delta^{1/2 - b} \left( \sum_{n \in \zz} \left (1 + |n| \right )^{2s}  \left ( \int_\rr
    \frac{|\wh{w_{\delta}}(n, \lambda)|}{1 + | |\lambda| - n^{m}|}
    \ d\lambda \right )^2 \right)^{1/2}.
	\end{split}
\end{equation}
%
By duality, the term in parenthesis is equal to
%
%%
\begin{equation*}
	\begin{split}
    \sup_{\|a_{n}\|_{\ell^{2}}=1}| \sum_{n \in \zz} \left (1 + |n| \right )^{s}
		a_n \int_{\rr} \frac{|\wh{w_{\delta}}(n, \lambda)|}{1 
		+ | |\lambda| - n^{m} |} \ d \lambda |
	\end{split}
\end{equation*}
%
%%
which by Cauchy-Schwartz is bounded by
%
%%
\begin{equation}
	\label{n1m}
	\begin{split}
		&  \sup_{\|a_{n}\|_{\ell^{2}}=1}| \sum_{n \in \zz} \left (1 + |n| \right )^{s} a_n
		\int_{\rr}\frac{| \wh{w_{\delta}}(n, \lambda) |}{1 + | |\lambda| - n^{m} |} \ d \lambda |
		\\
		& \le \sup_{\|a_{n}\|_{\ell^{2}}=1}\sum_{n \in \zz} \int_{\rr} \frac{| a_n |}{\left( 1 + 
		| |\lambda| - n^{m} |
		\right)^{1/2 + \eta}} \cdot \frac{\left( 1 + | n| \right)^s  |
		\wh{w_{\delta}}(n, \lambda) |}{\left( 
		1 + | |\lambda| - n^{m} | \right)^{1/2 - \eta}} \ d \lambda
		\\
		& \le \sup_{\|a_{n}\|_{\ell^{2}}=1} \left( \sum_{n \in \zz} | a_{n} |^2\int_{\rr} \frac{1}{\left( 1 + | |\lambda| - n^{m} | \right)^{1 + 2 \eta}} \ d \lambda  
		\right)^{1/2} 
		\left ( \sum_{n \in \zz}\int_{\rr} \frac{\left (1 + |n| \right )^{2s} | \wh{w_{\delta}}(n, \lambda) 
		|^2}{\left( 1 + | |\lambda| - n^{m} | \right)^{1 - 2 \eta}}\ d \lambda 
		\right)^{1/2}
	\end{split}
\end{equation}
%
%%
Restrict $\eta \in (0, 1/8)$. Splitting the integral into regions $\lambda > 0$
and $\lambda \le 0$ and applying the change of variables $\lambda - n^{m}
= \lambda'$ and $\lambda + n^{m} = \lambda'$ respectively, we obtain  
%%

\begin{equation*}
	\begin{split}
		&  \sup_{\|a_{n}\|_{\ell^{2}}=1}\left( \sum_{n \in \zz} | a_{n} |^2\int_{\rr} \frac{1}{\left( 1 + | |\lambda| -
		n^{m} | \right)^{1 + 2 \eta}} \ d \lambda  
		\right)^{1/2} 
		\\
		& = 2 \sup_{\|a_{n}\|_{\ell^{2}}=1}\left ( \sum_{n \in \zz}
		| a_n |^2 
		\int_{\rr} \frac{1}{\left( 1 + | \lambda' | \right)^{1 + 2 \eta}} \ d 
		\lambda \right)^{1/2}
		\\
		& \simeq 1.
		\end{split}
\end{equation*}
%
Therefore,
%
%
\begin{equation*}
\begin{split}
  \cref{fh}
  & \lesssim \delta^{1/2 - b}\|  w_{\delta} \|_{X_{s, -(1 - 2 \eta)}}.
\end{split}
\end{equation*}
%
%
Letting $b' = -(1- 2 \eta)$ and applying \cref{lem:time-local-est}, we obtain the
bound
\begin{equation*}
  \begin{split}
    \| w_{\delta} \|_{X_{s, -(1 - 2 \eta)}}
     \le  c_{\psi, a} \delta^{-a + (1-2 \eta)} 
    \| w \|_{X_{s, -a}}
  \end{split}
\end{equation*}
%
%%
where $-1/2 < -(1 - 2 \eta) \le -a < 1/2$. Hence, 
%
%
\begin{equation}
  \label{hjh}
\begin{split}
\cref{fh}
  & \le c_{\psi, a} \delta^{1/2 - b -a + (1-2 \eta)} 
    \| w \|_{X_{s, -a}}
\end{split}
\end{equation}
%
Fix $0 < \eta < 1/12$, and set $a = 1/2 - 3 \eta$, $b = 1/2 + \eta$, and $s
\ge -1/4 + 2 \eta$. Using this, and applying 
\cref{prop:bilinear-est} to \cref{hjh} gives the bound
%
%
\begin{equation}
	\begin{split}
    c_{\psi, b} \delta^{1/2}\|u\|^{2}_{X_{s,b}}.
	\end{split}
\end{equation}
%
%
%
We conclude that
%
%
\begin{equation}
  \label{main-int3-est}
	\begin{split}
    \|\cref{main1-rel-term-3}\|_{X_{s,b}} \le c_{\psi, b} 
    \delta^{1/2} \|u\|^{2}_{X_{s,b}}, \quad  b > 1/2,  \ s > -1/4.
	\end{split}
\end{equation}

%
%
%
\subsubsection{Estimate for \cref{main1-rel-term-5}.}
Note that
%
%
\begin{equation}
	\label{1n}
	\begin{split}
    \cref{main1-rel-term-5} \simeq \sum_{a = \pm 1}\sum_{k \ge 1}
		\frac{i^k}{k!}g_k(x,t)
	\end{split}
\end{equation}
%
%
where 
%
%
\begin{equation*}
	\begin{split}
		& g_k(x,t) = t^k \psi(t) \sum_{n \in \zz} e^{i\left( xn + ta n^{m}
		\right)} h_k(n),
		\\
		& h_k(n) = \int_\rr \psi \left( \tau - an^{m} \right) \cdot \left(
		\tau - an^{m} \right)^{k -1} \wh{w_{\delta}}(n, \tau) \ d \tau.
	\end{split}
\end{equation*}
%
%
Hence
%
%
\begin{equation*}
	\begin{split}
		\wh{g_k^x}(n, t) = t^{k} \psi(t) e^{i t an^{m}} h_k(n)
	\end{split}
\end{equation*}
%
%
which gives
%
%
\begin{equation*}
	\begin{split}
		\wh{g_k}(n, \tau)
		& = h_k(n) \int_\rr e^{-it\left( \tau - an^{m} \right)}
		t^{k}\psi(t) \ dt
		\\
		& = h_k(n) \wh{t^{k}\psi(t)} \left( \tau - an^{m} \right).
	\end{split}
\end{equation*}
%
%
Applying this to \cref{1n} and using Minkowski's inequality, we obtain
%
%
\begin{equation}
	\label{2n}
	\begin{split}
		& \|\cref{main1-rel-term-5}\|_{X_{s,b}} 
    \\
    & \lesssim \sum_{a = \pm 1} \left( \sum_{n \in \zz} \left (1 + |n| \right )^{2s}
    \int_\rr \left( 1 + | |\tau| - an^{m} | \right)^{2b}
    | \wh{\sum_{k \ge 1} \frac{i^k}{k!}g_k(x,t)} |^2 \ d \tau
		\right)^{1/2}
		\\
		& \le \sum_{a = \pm 1}\sum_{k \ge 1} \frac{1}{k!}\left( \sum_{n \in \zz} \left (1 + |n| \right )^{2s}
    \int_\rr \left( 1 + | |\tau| - an^{m} | \right)^{2b} | \wh{g_k}(n, \tau) |^2 \
		d \tau \right)^{1/2}
		\\
		& = \sum_{a = \pm 1}\sum_{k \ge 1} \frac{1}{k!} \left( \sum_{n \in \zz} \left (1 + |n| \right )^{2s}
    \int_\rr \left( 1 + | |\tau| - an^{m} | \right)^{2b} | h_k(n) \wh{t^k
		\psi(t)} \left( \tau - an^{m} \right) |^2 \ d \tau \right)^{1/2}
		\\
		& = \sum_{a = \pm 1}\sum_{k \ge 1} \frac{1}{k!} \left( \sum_{n \in \zz} \left (1 + |n| \right )^{2s} |
    h_k(n) |^2 \int_\rr \left( 1 + | |\tau| - an^{m} | \right)^{2b} | \wh{t^k
		\psi(t)} \left( \tau - an^{m} \right) |^2 \ d \tau \right)^{1/2}.
	\end{split}
\end{equation}
%
%
Applying \cref{eqn:norm-key-ineq} and the change
of variable $\tau - an^{m} = \tau'$
gives
%
%
\begin{equation}
	\label{3n}
	\begin{split}
		& \int_\rr \left( 1 + | |\tau| - an^{m} | \right)^{2b} | \wh{t^{k}
		\psi_{\delta}(t)}\left( \tau - an^{m} \right) |^2 \ d \tau
    \\
    & = 2 
    \int_\rr \left( 1 + |\tau'| \right)^{2b} | \wh{t^k \psi_{\delta}(t)}(\tau') |^2 \
		d \tau'
		\\
    & \lesssim
    \int_\rr \left( 1 + |\tau'| \right)^{2b} | \wh{t^k \psi_{\delta}(t)}(\tau')
		|^2 \ d \tau'
		\\
    & \le \| t^{k} \psi \|_{H^{[b] +
  1}}^{2}
\end{split}
\end{equation}
%
%
where $[b]$ denotes the least integer of $b$. Note that
%
%
\begin{equation}
	\label{4n}
	\begin{split}
    & \|t^k \psi \|_{H^{[b] +1}(\rr)}
		\\
    & = \|t^k \psi\|_{L^2(\rr)} + \|\p_t (t^k \psi )
    \|_{L^2(\rr)} 
    \\
    & + \| \p_{t}^{2} (t^{k} \psi) \|_{L^{2}(\rr)} + \cdots + \|
    \p_{t}^{[b] + 1} (t^{k} \psi)\|_{L^{2}}
    \\
    & \le c_{\psi} + k c'_{\psi} + k (k -1) c_{\psi}'' + \cdots +
    k(k-1) \cdots (k - [b]) c_{\psi, b}^{[b] + 1}
    \\
    & \lesssim c_{\psi, b} k(k-1) \cdots (k - [b]).
	\end{split}
\end{equation}
%
%
Hence, applying \cref{3n} and \cref{4n} to \cref{2n}, we obtain
%
%%
\begin{equation}
	\label{5n}
	\begin{split}
		\|\cref{main1-rel-term-5} \|_{X_{s,b}}
		& \lesssim c_{\psi, b} \sum_{a = \pm 1}
    \sum_{k \ge [b] +1} \frac{1}{(k-[b] - 1)!} \left( \sum_{n \in \zz} \left (1 + |n| \right )^{2s} | h_k(n) |^2 
		\right)^{1/2}
		\\
    & \le c_{\psi, b} \sum_{a = \pm 1} \sum_{k \ge [b] +1} \frac{1}{(k-[b] - 1)!}
    \times \sup_{k \ge [b] + 1} \left( \sum_{n \in \zz} \left (1 + |n| \right )^{2s} | 
		h_k(n) |^2 \right)^{1/2}
		\\
    & = c_{\psi, b}  \sum_{a = \pm 1}\sum_{k \ge [b] +1} \frac{1}{(k-[b] - 1)!}
    \\
    & \times \sup_{k \ge [b] + 1} 
		\left( \sum_{n \in \zz} \left (1 + |n| \right )^{2s} |\int_\rr 
		\psi\left( \tau - an^{m} \right) \cdot \left( \tau - an^{m} 
    \right)^{k -1} \wh{w_{\delta}}(n, \tau) \ d \tau|^{2} \right)^{1/2}.
    \end{split}
\end{equation}
%
%%
Recall that $0 \le \psi \le 1, \text{supp} \, \psi \subset [-2,2 ]$. 
This implies $$| \psi\left( \tau - an^{m} \right) \cdot \left( \tau - an^{m}
\right)^{k -1} | \le 2^{k-1} \chi_{| \tau - an^{m} | \le 1}, \qquad k \ge 1.$$ Hence,
we bound the right hand side of \cref{5n} by
%
%%
\begin{equation*}
	\begin{split}
    & c_{\psi, b} \sum_{a = \pm 1}
    \sum_{k \ge [b] +1} \frac{2^{k-1}}{(k-[b] - 1)!}
    \times \left( \sum_{n \in \zz} (1 + | n |)^{2s}| 
		\int_{| \tau - an^{m}  |\le 1}  \wh{w_{\delta}}(n, \tau) \ d \tau |^2 
		\right)^{1/2}
    \\
    & = \frac{e^{2}}{2^{[b]}} c_{\psi, b} \sum_{a = \pm 1} \left( \sum_{n \in \zz} (1 + | n |)^{2s}| 
		\int_{| \tau - an^{m}  |\le 1}  \wh{w_{\delta}}(n, \tau) \ d \tau |^2 
		\right)^{1/2}
    \\
    & \lesssim \sum_{a = \pm 1}
\left[ \sum_{n \in \zz} (1 + | n |)^{2s}\left (  
		\int_{| \tau - an^{m}  |\le 1} | \wh{w_{\delta}}(n, \tau) | \ d \tau \right ) ^2 
		\right]^{1/2}
	\end{split}
\end{equation*}
%
%%
which by the inequality
%
%%
\begin{equation*}
	\begin{split}
		1 \le 
		\frac{2}{1 + | j |}, \qquad | j | \le 1
	\end{split}
\end{equation*}
%
%%
and \cref{eqn:norm-key-ineq}
is bounded by 
%
%%
\begin{equation}
\label{main-int4-est-X-s-part}
	\begin{split}
    & \left[ \sum_{n \in \zz} (1 + | n |)^{2s}\left( \int_\rr
		\frac{|\wh{w_{\delta}}(n, \tau)|}{1 + | |\tau| - n^{m} |} \ d \tau \right ) ^2 
		\right]^{1/2}.
  \end{split}
\end{equation}
%
%%
Recalling the computations from \cref{fh} to \cref{main-int3-est}, we conclude
that 
%
%
\begin{equation}
\label{main-int4-est}
	\begin{split}
    \|\cref{main1-rel-term-5}\|_{X_{s,b}} \le c_{\psi, b} \delta^{1/2}
    \|u\|_{X_{s,b}}^{3}, \quad b > 1/2, \ s > -1/4.
	\end{split}
\end{equation}
%
%
Collecting estimates \cref{main-int2-est}, 
\cref{main-int3-est}, and \cref{main-int4-est}, and recalling 
\cref{main1-rel-term-2}-\cref{main1-rel-term-5}, we obtain
the following.
%
%
\begin{proposition}
\label{prop:contraction}
%
For $1/2 < b \le 1$, $s \ge -1/4 + 2(b -1/2)$, $0 < \delta \le 1$, we have
%
%%
\begin{equation*}
	\begin{split}
    \|Tu\|_{X_{s,b}} \le c \left( \|u_0 \|_{H^s(\ci)} + \|u_1 \|_{H^{s-2}(\ci)}
    + \delta^{b-1/2} \|u\|_{X_{s,b}}^2 
		\right)
	\end{split}
\end{equation*}
%
where $c = c_{\psi, b} > 1$.  
%%
\end{proposition}
%
%
%
\subsection{Proof of Existence and Uniqueness}
\label{sec:proof-b4-per-case}
%
%
%%%%%%%%%%%%%%%%%%%%%%%%%%%%%%%%%%%%%%%%%%%%%%%%%%%%%
%
%% Contraction Proposition
%				 
%%%%%%%%%%%%%%%%%%%%%%%%%%%%%%%%%%%%%%%%%%%%%%%%%%%%%%
%%
%%
%
We will now use \cref{prop:contraction} and the trilinear estimates
to prove local well-posedness for the 
$B_4$ ivp. Suppose
%
%%
\begin{equation*}
	\begin{split}
    \|(u_0, u_{1})\|_{H^s(\ci) \times H^{s-2}(\ci)} \le r.
  \end{split}
\end{equation*}
%
%%
Then $$\|u\|_{X_{s,b}} \le 2rc$$ implies
%
%%
\begin{equation*}
	\begin{split}
		\|Tu \|_{X_{s,b}} 
    & \le c \left[ r + \delta^{b - 1/2} \left( 
		2rc \right)^2 \right].
	\end{split}
\end{equation*}
%
Choosing 
%
%
\begin{equation}
  \label{delta-suf-small}
\begin{split}
  \delta \le \left (\frac{1}{8rc^{2}} \right )^{\frac{1}{b - 1/2}}
\end{split}
\end{equation}
%
%
we obtain 
%%
%
%
%
\begin{equation*}
\begin{split}
\|Tu \|_{X_{s,b}} 
    & \le \frac{3}{2}rc.
  \end{split}
\end{equation*}
%
%
Hence, $T=T_{u_0, u_1, \psi, \delta}$ maps the ball $B_{X_{s,b}}(2rc)$ into
itself. Next, note that 
%
%%
\begin{equation*}
	\begin{split}
    Tu - Tv = \cref{main1-rel-term-3} + \cref{main1-rel-term-4} +
    \cref{main1-rel-term-5} 
  \end{split}
  \label{eqn:integral-form-dif}
\end{equation*}
%
%%
where now $w(x,t) =u^{2} - v^{2}$. Rewriting
%
%%
\begin{equation*}
	\begin{split}
	u^2 - v^2
		& = (u-v)(u+v)
		\end{split}
\end{equation*}
%
%%
and repeating earlier arguments, we obtain
%
%%
%%
\begin{equation}
	\label{20a}
	\begin{split}
		\|Tu - Tv \|_{X_{s,b}}  
    & \le c \delta^{b - 1/2}\|u -v\|_{X_{s,b}} \|u + v \|_{X_{s,b}}
		\\
    & \le c \delta^{b -1/2} \|u -v\|_{X_{s,b}} (\|u\|_{X_{s,b}}+ \|v \|_{X_{s,b}}).
	\end{split}
\end{equation}
%
%%
If $$ u, v \in B_{X_{s,b}} \left (2rc \right )$$ then choosing $\delta$ as in
\cref{delta-suf-small}, we obtain
%
%%
\begin{equation}
	\label{21a}
	\begin{split}
		\|Tu - Tv \|_{X_{s,b}}
    & \le c \delta^{b-1/2} \|u -v \|_{X_{s,b}} \left( 2rc + 
		2rc \right)
		\\
		& = \frac{1}{2} \|u -v \|_{X_{s,b}}. 
	\end{split}
\end{equation}
%
%%
We conclude that $T$ is a contraction on the ball $B_{X_{s,b}}(2rc)$.
A Picard iteration then yields a unique 
$u \in X_{s,b}$ satisfying $u = Tu$. Applying
\cref{lem:embedding}, it follows that $u(x,t) \subset C( [-\delta, \delta], H^s)
\cap X_{s,b}$ is a unique
solution of the $B_{4}$ ivp \cref{eqn:mb-2}-\cref{eqn:mb-init-data-2} for $t
\in [-\delta, \delta]$.
%
%
%
%
%
%
\subsection{Proof of Lipschitz Continuity} 
\label{sec:lip-continuity}
%
%
%
%
Let $$(u_0, u_1), (v_0, v_1)  \subset
B_{H^{s} \times H^{s-2}} \left (r \right ).$$ Then for $\delta$ sufficiently
small (i.e.\ satisfying \cref{delta-suf-small}), there exist $u, v \in
X_{s,b}$ such that $u =
T_{u_0, u_1}u$, $v = T_{v_0, v_1} v$, and so
%
%
\begin{align}
  \notag
    & T_{u_0, u_1}(u) - T_{v_0, v_1}(v)
		\\
    & = \psi(t) \sum_{n \in \zz} e^{inx} \wh{u_{0} - v_{0}}(n) \frac{e^{in^{2}t} + e^{-in^{2}t}}{2} 
\label{main1-rel-term-1g}
  \\
  & + \psi(t) \sum_{n \in \zz} e^{inx}
  \wh{u_{1} - v_{1}}(n)\frac{e^{in^{2}t} - e^{-in^{2}t}}{2 i n^{2}} 
\label{main1-rel-term-2g}
  \\
  & + \psi(t) \sum_{a = \pm 1} \sum_{n\in \zz} \int_\rr e^{ixn}  
  e^{it \tau} \frac{1 - \psi(\tau -  an^{2}) 
}{\tau -  an^{2}} \wh{w}(n, \tau) \ d \tau
\label{main1-rel-term-3g}
  \\
  & + \psi(t) \sum_{a = \pm 1} \sum_{n\in \zz} \int_\rr e^{i(xn + 
  t an^{2})}
  \frac{1- \psi(\tau -  an^{2})}{\tau -  an^{2}} \wh{w}(n, \tau) \ d \tau
\label{main1-rel-term-4g}
  \\
  & + \psi(t) \sum_{a = \pm 1}  \sum_{k \ge 1} \frac{i^k t^k}{k!}
  \sum_{n \in \zz} \int_\rr e^{i(xn + t an^{2} )}
  \psi(\tau -  an^{2}) (\tau -  an^{2})^{k-1} \wh{w}(n, \tau)
  \label{main1-rel-term-5g}
\end{align}
%
where now $w = u^{2} - v^{2}$. Using arguments similar to those in 
\cref{ssec:est-init-term-1}-\cref{ssec:est-init-term-2}
we obtain
%
%
\begin{equation}
	\label{gen-2a}
	\begin{split}
    & \| \cref{main1-rel-term-1g}\|_{X_{s,b}}
		\le c \|u_0 -v_0\|_{H^s},
    \\
    & \| \cref{main1-rel-term-2g}\|_{X_{s,b}}
    \le c \|u_1 -v_1\|_{H^{s-2}}.
	\end{split}
\end{equation}
%
%
%
%
Therefore, from \cref{21a} and \cref{gen-2a}, we obtain
%
%
\begin{equation*}
	\begin{split}
    \|u -v \|_{X_{s,b}}
    & = \|T_{u_0, v_0}(u) - T_{u_1, v_1}(v) \|_{X_{s,b}}
    \\
    & \le
    c \left( \|u_0 -v_0 \|_{H^s\left( \ci \right)} +\|u_1 -v_1
        \|_{H^{s-2}\left( \ci \right)} \right )
        + \frac{1}{2} \|u -v \|_{X_{s,b}}
  \end{split}
\end{equation*}
%
%
which implies
%
%
\begin{equation*}
	\begin{split}
		\frac{1}{2} \|u-v\|_{X_{s,b}} \le
    c \left( \|u_0 -v_0 \|_{H^s\left( \ci \right)} +\|u_1 -v_1
        \|_{H^{s-2}\left( \ci \right)} \right )
      \end{split}
\end{equation*}
%
%
or
%
%
\begin{equation}
	\begin{split}
		\|u -v \|_{X_{s,b}} \le 2 c \left( \|u_0 -v_0 \|_{H^s\left( \ci \right)} +\|u_1 -v_1
        \|_{H^{s-2}\left( \ci \right)} \right ).
	\end{split}
  \label{pre-lem-estimate}
\end{equation}
%
%
Applying  \cref{lem:embedding} to \cref{pre-lem-estimate}, it follows that
for $(u_0, u_1), (v_0, v_1)  \subset
B_{H^{s} \times H^{s-2}} \left (r \right )$, the
associated solutions $u, v \in C([-\delta, \delta], H^{s}(\ci))$ satisfy the estimate%
%
%
	 %
	 %
	 \begin{equation*}
		 \begin{split}
       \sup_{t \in [-\delta, \delta]} \|u(\cdot, t) -v(\cdot, t) \|_{H^s(\ci)} \le
      2 c \left( \|u_0 -v_0 \|_{H^s\left( \ci \right)} +\|u_1 -v_1
        \|_{H^{s-2}\left( \ci \right)} \right ).
		 \end{split}
	 \end{equation*}
	 %
	 %
Hence, the flow map is Lipschitz continuous from $B_{H^{s}
\times H^{s-2}} \left (r \right )$ to $C([-\delta, \delta],
H^{s}(\ci))$, where $\delta = \delta(r)$. This
concludes the proof of well-posedness for the $B_4$ ivp
\cref{eqn:mb-2}-\cref{eqn:mb-init-data-2}. \qquad \qedsymbol
%
%%%%%%%%%%%%%%%%%%%%%%%%%%%%%%%%%%%%%%%%%%%%%%%%%%%%%
%
%
%                Proof of Bilinear Estimate B4 Per
%
%
%%%%%%%%%%%%%%%%%%%%%%%%%%%%%%%%%%%%%%%%%%%%%%%%%%%%%
%
%
\subsection{Proof of \hyperref[prop:bilinear-est]{Periodic Bilinear Estimate}} 
\label{sec:proof-bilin-est}
Since $\|f\|_{X_{s,-a'}} \le \|f\|_{X_{s, -a}}$ for $0 < a < a'$, we may assume
$1/4 < a \le b$ and $b > 1/2$ without loss of generality.
By duality, it suffices to show that for $s \ge -a/2$, 
%
%%
\begin{equation}
	\label{duality-est}
	\begin{split}
	|	\sum_{n}  (1 + |n|)^{s}
		\int_{\rr} \phi(n, \tau) \wh{uv}(n, \tau)(1 
    + | |\tau| - n^{2} |^{-a}) d \tau | \lesssim \|u\|_{X_{s,b}}
    \|v\|_{X_{s,b}}
    \|\phi \|_{L^{2}_{n, \tau}}.
	\end{split}
\end{equation}
Note first that $|\wh{uv}(n, \tau) |  = | \wh{u} *  \wh{v} 
(n, \tau)|$. From this it follows that
%
%
\begin{equation}
	\label{non-lin-rep}
	\begin{split}
		| \wh{uv}(n, \tau)|
    & = | \sum_{n_{1}}  \int_{\tau_{1}}
    \wh{u}\left( n_1,  \tau_1 \right) \wh{v}\left( n - n_1 , \tau - \tau_1   
\right) d \tau_1 |
\\
& \le  \sum_{n_{1}}  \int_{\tau_{1}}
    |\wh{u}\left( n_1,  \tau_1 \right)| |\wh{v}\left( n - n_1 , \tau - \tau_1   
\right)| d \tau_1 
\\
& = \sum_{n_{1}} \int_{\tau_{1}} \frac{c_u\left( n_1, \tau_1 
\right)}{\langle n_1 \rangle ^s \langle |\tau_1| - n_1^{2} | \rangle ^{b}}
\\
& \times \frac{c_{v}\left( n - n_1, \tau - \tau_1 \right)}{\langle n -
n_1 \rangle ^s\ \langle |\tau - \tau_1 | -  (n - n_1)^{2} \rangle^{b}}
  \ d \tau_1 
\end{split}
\end{equation}
%
%
where for clarity of notation we have introduced 
%
%
%
\begin{equation*}
\begin{split}
\langle k \rangle \doteq 1 + |k|
\end{split}
\end{equation*}
%
%
and
%
\begin{equation*}
	\begin{split}
		c_h(n, \tau) =
			\langle n \rangle ^s \langle |\tau| - n^{2} \rangle ^{b} | \wh{h}\left( n, \tau \right) |.
	\end{split}
\end{equation*}
%
%
From our work above, it follows that 
%
%
\begin{equation}
	\label{convo-est-starting-pnt}
	\begin{split}
		 & \langle n \rangle^s \langle \tau - n^{2} \rangle^{-a} | \wh{uv}\left( 
		n, \tau \right) |
		\\
		& \le \langle |\tau| - n^{2} \rangle^{-a}
		\sum_{n_{1}} \int_{\tau_{1}} \frac{\langle n \rangle^{s}}{\langle n_1 \rangle^s
    \langle n - n_1 \rangle^s} 
		\times \frac{c_f(n_1, \tau_1)}{\langle |\tau_1| - n_1^{2} \rangle ^{b}}
		\\
		& \times
		\frac{c_g(n - n_1, \tau - \tau_1 )}{\langle |\tau - \tau_1| - (n - n_1)^{2}
    \rangle^{b}}\ d \tau_1.
	\end{split}
\end{equation}
%
%
Hence, 
%
%
\begin{equation}
  \label{pre-fubini-int-form}
	\begin{split}
    |\text{lhs of} \ \cref{duality-est}|
	& \lesssim \sum_{n} \int_{\tau} \phi(n, \tau) \langle n \rangle^s 
  \sum_{n_{1}}
  \int_{\tau_{1}} c_f(n_1, \tau_1)
		c_g(n - n_1, \tau - \tau_1 )
		\\
    & \times \frac{\langle n \rangle ^{s}}{\langle n_{1} \rangle ^{s} \langle
    n-n_{1} \rangle ^{s}} \times \frac{1}{\langle |\tau| - n^{2} \rangle
    ^{a}\langle |\tau_{1}|-n_{1}^{2} \rangle ^{b}\langle | \tau -
    \tau_{1}|-(n - n_{1})^{2}
    \rangle ^{b}} d \tau_1 d \tau.
	\end{split}
\end{equation}
%
Let $A \subset \rr^{2} \times \zz^{2}$, and $\chi_{A}(\tau, \tau_{1}, n, n_{1})$
be its
characteristic function. Then by Cauchy-Schwartz in
$\tau_{1}, \xi_{1}$
\begin{equation*}
	\begin{split}
    & \sum_{n} \int_{\tau}   \sum_{n_{1}}
    \int_{\tau_{1}} \chi_{A}
    \phi(n, \tau) \langle n \rangle^s \langle \tau - n^{2} \rangle^{-a}
  c_f(n_1, \tau_1)
		c_g(n - n_1, \tau - \tau_1 )
		\\
    & \times \frac{\langle n \rangle ^{s}}{\langle n_{1} \rangle ^{s} \langle
    n-n_{1} \rangle ^{s}} \times \frac{1}{\langle |\tau| - n^{2} \rangle
    ^{a}\langle |\tau_{1}|-n_{1}^{2} \rangle ^{b}\langle | \tau -
    \tau_{1}|-(n - n_{1})^{2}
    \rangle ^{b}} d \tau_1 d \tau.
	\end{split}
\end{equation*}
%
is bounded by 
%
%
\begin{equation}
	\label{10g}
	\begin{split}
    & \sum_{n} \int_{\tau} \phi(n, \tau) \langle | \tau | - n^{2} \rangle
    ^{-a} \langle n \rangle ^{s}
    \\
    & \times \left( \sum_{n_{1}} \int_{\tau_{1}}
    \frac{\chi_{A}}{\langle n_{1} \rangle ^{2s} \langle n-n_{1} \rangle ^{2s} \langle |
    \tau_{1} | - n_{1}^{2}\rangle^{2b}  \langle | \tau - \tau_{1} | -
    (n - n_{1})^{2} \rangle^{2b}} d \tau_{1} \right)^{1/2}
    \\
    & \times \left( \sum_{n_{1}} \int_{\tau_{1}} c_{u}^{2}(n, \tau_{1})
    c_{v}^{2}(n - n_{1}, \tau - \tau_{1}) d \tau_{1} \right)^{1/2} d \tau
  \end{split}
\end{equation}
%
%
Applying Cauchy-Schwartz again, \cref{10g} is bounded by
%
%
\begin{equation*}
  \begin{split}
    & \|\left( \sum_{n_{1}}\int_{\tau_{1}} c_{u}^{2}(n_1, \tau_1)
  c_{v}^{2} (n - n_1, \tau - \tau_{1} ) d \tau_1  \right)^{1/2} \|_{L^{2}(\zz \times
		\rr)}
		\\
    & \times  \|\phi(n, \tau) \langle | \tau | - n^{2} \rangle ^{-a} \langle n
    \rangle ^{s}
		\\
    & \times \left( \sum_{n_{1}} \int_{\tau_{1}} \frac{\chi_{A}}{\langle n_{1}
    \rangle ^{2s} \langle n-n_{1} \rangle ^{2s} \langle | \tau_{1}|-n_{1}^{2}
    \rangle^{2b} \langle  |\tau -
    \tau_{1} | -(n - n_{1})^{2}
    \rangle^{2b}} d \tau_1 \right)^{1/2} \|_{L^2(\zz \times \rr)}
		\\
    & = \|u\|_{X_{s,b}} \|v\|_{X_{s,b}} \label{holder-term}
     \|\phi(n, \tau) \times \left( \langle | \tau | - n^{2} \rangle ^{-2a} \langle n
    \rangle ^{2s} \right .
    \\
    & \times \left . \sum_{n_{1}} \int_{\tau_{1}} \frac{\chi_{A}}{\langle n_{1} \rangle ^{2s} \langle
n-n_{1} \rangle ^{2s}  \langle | \tau_{1}|-n_{1}^{2} \rangle^{2b} \langle  |\tau -
    \tau_{1} | -(n - n_{1})^{2}
    \rangle^{2b}} d \tau_1 \right)^{1/2} \|_{L^2(\zz \times \rr)}.
  \end{split}
\end{equation*}
%
Applying H{\"o}lder, we bound this by 
%
%
\begin{equation}
  \label{integral-bound-1st-form-per}
	\begin{split}
    & \|u\|_{X_{s,b}} \|v\|_{X_{s,b}} \| \phi \|_{L^{2}_{n, \tau}}
    \|\left( \langle | \tau | - n^{2} \rangle ^{-2a} \langle n
    \rangle ^{2s} \right. 
    \\
    & \times \left. 
    \sum_{n_{1}} \int_{\tau_{1}} \frac{\chi_{A}}{\langle n_{1} \rangle ^{2s} \langle
n-n_{1} \rangle ^{2s} \langle | \tau_{1}|-n_{1}^{2} \rangle^{2b} \langle  |\tau -
    \tau_{1} | -(n - n_{1})^{2}
    \rangle^{2b}} d \tau_1 \right)^{1/2} \|_{L^\infty_{n, \tau}}
	\end{split}
\end{equation}
%
%
Let us now return to the right hand side of \cref{pre-fubini-int-form}.
Then by the change of variable $\lambda =
\tau - \tau_{1}$, we obtain
\begin{equation*}
	\begin{split}
    & \sum_{n} \int_{\tau}   \sum_{n_{1}}
    \int_{\tau_{1}} \chi_{A}
    \phi(n, \tau) \langle n \rangle^s \langle \tau - n^{2} \rangle^{-a}
  c_f(n_1, \tau_1)
		c_g(n - n_1, \lambda )
		\\
    & \times \frac{\langle n \rangle ^{s}}{\langle n_{1} \rangle ^{s} \langle
    n-n_{1} \rangle ^{s}} \times \frac{1}{\langle |\tau| - n^{2} \rangle
    ^{a}\langle |\tau_{1}|-n_{1}^{2} \rangle ^{b}\langle | \tau -
    \tau_{1}|-(n - n_{1})^{2}
    \rangle ^{b}} d \tau_1 d \tau
    \\
    & = \sum_{n} \int_{\tau}   \sum_{n_{1}}
    \int_{\lambda} \chi^{*}_{A}
    \phi(n, \tau) \langle n \rangle^s \langle \tau - n^{2} \rangle^{-a}
  c_f(n_1, \tau - \lambda)
		c_g(n - n_1, \lambda )
		\\
    & \times \frac{\langle n \rangle ^{s}}{\langle n_{1} \rangle ^{s} \langle
    n-n_{1} \rangle ^{s}} \times \frac{1}{\langle |\tau| - n^{2} \rangle
    ^{a}\langle |\tau - \lambda|-n_{1}^{2} \rangle ^{b}\langle |
    \lambda|-(n - n_{1})^{2}
    \rangle ^{b}} d \lambda  d \tau
	\end{split}
\end{equation*}
where 
%
%
\begin{equation}
  \label{change-of-var}
\begin{split}
  \chi^{*}_{A}(\tau, \lambda, n, n_{1}) =
  \chi_{A}(\tau, \tau - \lambda, n, n_{1}).
\end{split}
\end{equation}
%
%
Cauchy-Schwartz in
$\lambda, n$ then gives the bound
%
%
%
\begin{equation}
	\label{10gk}
	\begin{split}
    & \sum_{n} \int_{\tau} \phi(n, \tau) \langle | \tau | - n^{2} \rangle
    ^{-a} \langle n \rangle ^{s}
    \\
    & \times \left( \sum_{n_{1}} \int_{\lambda}
    \frac{\chi^{*}_{A}}{\langle n_{1} \rangle ^{2s} \langle n-n_{1} \rangle ^{2s} \langle |
    \tau - \lambda | - n_{1}^{2}\rangle  \langle | \lambda | -
    (n - n_{1})^{2} \rangle} d \lambda \right)^{1/2}
    \\
    & \times \left( \sum_{n_{1}} \int_{\lambda} c_{u}^{2}(n_{1}, \tau - \lambda)
    c_{v}^{2}(n - n_{1}, \lambda) d \lambda \right)^{1/2} d \tau
  \end{split}
\end{equation}
%
%
Applying Cauchy-Schwartz again, \cref{10gk} is bounded by
%
%
\begin{equation*}
  \begin{split}
    & \|\left( \sum_{n_{1}}\int_{\tau_{1}} c_{u}^{2}(n_1, \tau - \lambda)
  c_{v}^{2} (n - n_1, \lambda ) d \tau_1  \right)^{1/2} \|_{L^{2}(\zz \times
		\rr)}
		\\
    & \times  \|\phi(n, \tau) \langle | \tau | - n^{2} \rangle ^{-a} \langle n
    \rangle ^{s}
		\\
    & \times \left( \sum_{n_{1}} \int_{\tau_{1}} \frac{\chi^{*}_{A}}{\langle n_{1}
    \rangle ^{2s} \langle n-n_{1} \rangle ^{2s} \langle | \tau - \lambda|-n_{1}^{2}
    \rangle \langle  |\lambda | -(n - n_{1})^{2}
    \rangle} d \tau_1 \right)^{1/2} \|_{L^2(\zz \times \rr)}
		\\
    & = \|u\|_{X_{s,b}} \|v\|_{X_{s,b}} \label{holder-term-0}
    \left \{\sum_{n} \int_{\tau} |\phi(n, \tau)|^{2} \right .
    \\
    & \left. \times \sum_{n_{1}} \int_{\lambda} \frac{\chi^{*}_{A}
    \langle n \rangle ^{2s}
  }{\langle n_{1} \rangle^{2s} \langle | \tau | - n^{2}
    \rangle ^{2a}  \langle
n-n_{1} \rangle ^{2s}  \langle | \tau - \lambda|-n_{1}^{2}
\rangle \langle  | \lambda | -(n - n_{1})^{2}
    \rangle} d \lambda d \tau \right \}^{1/2}.
  \end{split}
\end{equation*}
%
%
Applying Fubini and H{\"o}lder to the last term gives the bound
%
%
\begin{equation}
  \label{integral-bound-1st-form-per-0}
	\begin{split}
    & \|u\|_{X_{s,b}} \|v\|_{X_{s,b}} \| \phi \|_{L^{2}_{n, \tau}}
    \\
    & \times \left( \sum_{n_{1}} \int_{\tau} \frac{\chi^{*}_{A}
    \langle n \rangle ^{2s}
  }{\langle n_{1} \rangle^{2s} \langle | \tau | - n^{2}
    \rangle ^{2a}  \langle
n-n_{1} \rangle ^{2s}  \langle | \tau - \lambda|-n_{1}^{2}
\rangle \langle  | \lambda | -(n - n_{1})^{2}
    \rangle} d \tau  \right)^{1/2} \|_{L^\infty_{n, \lambda}}.
	\end{split}
\end{equation}
Again, we return to the right hand side of \cref{pre-fubini-int-form}.
We seek to bound
\begin{equation*}
\begin{split}
  & \sum_{n} \int_{\tau}  \sum_{n_{1}}
  \int_{\tau_{1}} \chi_{A} \phi(n, \tau)
    c_f(n_1, \tau_1)
		c_g(n - n_1, \tau - \tau_1 )
		\\
    & \times \frac{\langle n \rangle ^{s}}{\langle n_{1} \rangle ^{s} \langle
    n-n_{1} \rangle ^{s}} \times \frac{1}{\langle \tau - n^{2} \rangle^{a}
\langle |\tau| - n^{2} \rangle
    ^{b}\langle |\tau_{1}|-n_{1}^{2} \rangle ^{b}\langle | \tau|-n_{2}^{2}
    \rangle ^{b}} d \tau_1 d \tau 
   \end{split}
\end{equation*}
in a different manner than before. First, we apply 
Fubini, then Cauchy-Schwartz in $n_{1}, \tau_{1}$ to obtain the bound
%
%
\begin{equation*}
\begin{split}
  & \left[ \sum_{n_{1}} \int_{\tau_{1}} c_{f}^{2}(n_{1}, \tau_{1}) d \tau_{1}
  \right]^{1/2}
  \\
  & \times \left \{\sum_{n_{1}} \int_{\tau_{1}}   
 \left[
 \sum_{n} \int_{\tau}
   \frac{\langle n \rangle ^{s}}{\langle n_{1} \rangle ^{s} \langle
   n - n_{1}\rangle ^{s}} \times \frac{\chi_{A} |\phi(n, \tau)| c_{g}(n -
   n_{1}, \tau - \tau_{1})
}{\langle | \tau | - n^{2} \rangle
  ^{a} \langle | \tau_{1} | - n_{1}^{2} \rangle ^{b} \langle | \tau -
  \tau_{1} | - (n - n_{1}^{2}) \rangle ^{b}} d \tau 
  \right]^{2} d \tau_{1} \right \}^{1/2}
  \\
  & = \| f \|_{X_{s,b}}
  \\
  & \times \left \{\sum_{n_{1}} \int_{\tau_{1}}   
 \left[
 \sum_{n} \int_{\tau}
   \frac{\langle n \rangle ^{s}}{\langle n_{1} \rangle ^{s} \langle
   n - n_{1}\rangle ^{s}} \times \frac{\chi_{A}|\phi(n, \tau)| c_{g}(n -
   n_{1}, \tau - \tau_{1})
}{\langle | \tau | - n^{2} \rangle
  ^{a} \langle | \tau_{1} | - n_{1}^{2} \rangle ^{b} \langle | \tau -
  \tau_{1} | - (n - n_{1}^{2}) \rangle ^{b}} d \tau 
  \right]^{2} d \tau_{1}  \right \}^{1/2}
\end{split}
\end{equation*}
%
Applying Cauchy-Schwartz in $\tau, n$, we bound the last line by 
%
%
\begin{equation*}
\begin{split}
  & \left \{\sum_{n_{1}} \int_{\tau_{1}}   
  \left [ \sum_{n} \int_{\tau}
  | \phi(n, \tau)|^{2} c_{g}^{2}(n - n_{1}, \tau - \tau_{1}) d \tau  
    \right ] \right . 
   \\
   & \left. \times \left [ \sum_{n} \int_{\tau} \frac{\langle n \rangle
   ^{2s}}{\langle n_{1} \rangle ^{2s} \langle n - n_{1}\rangle ^{2s}}
   \times \frac{\chi_{A}}{\langle | \tau | - n^{2} \rangle ^{2a} \langle | \tau_{1} |
   - n_{1}^{2} \rangle^{2b}  \langle | \tau - \tau_{1} | - (n - n_{1}^{2})
   \rangle^{2b}} d \tau  \right ] \right \}^{1/2}d \tau_{1} 
\end{split}
\end{equation*}
%
%
which by Holder is bounded by 
%
%
%
\begin{equation}
  \label{integral-bound-2nd-form-per}
\begin{split}
  & \| \sum_{n} \int_{\rr} \frac{\langle n \rangle ^{2s}}{\langle n_{1} \rangle ^{2s} \langle
  n - n_{1}\rangle ^{2s}}  \times \frac{\chi_{A}}{\langle | \tau | - n^{2} \rangle
  ^{2a} \langle | \tau_{1} | - n_{1}^{2} \rangle^{2b}  \langle | \tau -
  \tau_{1} | - (n - n_{1}^{2}) \rangle^{2b}} d \tau 
  \|_{L^{\infty}_{n_{1}, \tau_{1}}}^{1/2}
  \\
  & \times \|\phi\|_{L^{2}} \| g \|_{X_{s,b}}.
\end{split}
\end{equation}
%
%
Now consider the family $\{A_{j}\}_{1}^{k}, A_{j} \subset \rr^{2} \times
\zz^{2}$ with
$$\bigcup_{1}^{k} A_{j}= \rr^{2} \times
\zz^{2}.$$ From \cref{integral-bound-1st-form-per},
\cref{integral-bound-1st-form-per-0},
\cref{integral-bound-2nd-form-per}, and our preceding argumentation,
we see that the proof of \cref{prop:bilin-est-real} reduces to showing that
either 
%
%
%
%
\begin{equation}
  \label{key-sup-estimate-per-1}
  \begin{split}
     \| \langle | \tau | - n^{2} \rangle ^{-2a} \langle n
    \rangle ^{2s}
    \sum_{n_{1}}
    \int_{\tau_{1}} \frac{\chi_{A_{j}}}{\langle n_{1} \rangle ^{2s} \langle
    n-n_{1} \rangle ^{2s} \langle | \tau_{1}|-n_{1}^{2} \rangle^{2b}  \langle  |\tau -
    \tau_{1} | -(n - n_{1})^{2}
    \rangle^{2b}} d \tau_1  \|_{L^\infty_{n, \tau}} < \infty.
  \end{split}
\end{equation}
%
or
%%
\begin{equation}
  \label{key-sup-estimate-per-2}
\begin{split}
  & \| \frac{1}{\langle n_{1} \rangle ^{2s}
  \langle | \tau_{1} | - n_{1}^{2} \rangle
  ^{2a}} \sum_{n} \int_{\tau} \frac{\langle n \rangle ^{2s}}{\langle
  n - n_{1}\rangle ^{2s}}  \times \frac{\chi_{A_{j}}}{\langle | \tau | - n^{2}
  \rangle^{2b}  \langle | \tau -
  \tau_{1} | - (n - n_{1}^{2}) \rangle^{2b}} d \tau 
  \|_{L^{\infty}_{n_{1}, \tau_{1}}}
\end{split}
\end{equation}
or
%
%
\begin{equation}
  \label{key-sup-estimate-per-3}
\begin{split}
  \| \sum_{n_{1}} \int_{\tau} \frac{\chi^{*}_{A_{j}}
    \langle n \rangle ^{2s}
  }{\langle n_{1} \rangle^{2s} \langle | \tau | - n^{2}
    \rangle ^{2a}  \langle
n-n_{1} \rangle ^{2s}  \langle | \tau - \lambda|-n_{1}^{2}
\rangle^{2b} \langle  | \lambda | -(n - n_{1})^{2}
\rangle^{2b}} d \tau  \|_{L^{\infty}_{n, \lambda}}
\end{split}
\end{equation}
%
%
for each $j \in \left\{1,\dots,k \right\}$. 
By the triangle inequality and the fact that 
%
%
\begin{equation*}
\begin{split}
& | \tau | =
\begin{cases}
  - \tau, \quad & \tau < 0, 
\\
\tau, \quad & \tau > 0
\end{cases}
\end{split}
\end{equation*}
%
%
it follows that the proof of \cref{prop:bilinear-est} reduces to showing that
for any $j$, either 
%
%
\begin{equation}
  \label{sup-est-gen-per-1}
  \begin{split}
    \| \frac{\langle n
    \rangle ^{2s}}{\langle \sigma \rangle ^{2a}}
    \sum_{n_{1}} \int_{\tau_{1}} \frac{\chi_{A_{j}}}{\langle n_{1} \rangle ^{2s} \langle n-n_{1} \rangle ^{2s} 
    \langle \sigma_{1} \rangle^{2b} \langle  \sigma_{2} \rangle^{2b}}
    d \tau_1  \|_{L^{\infty}_{n, \tau}} < \infty
  \end{split}
\end{equation}
%
%
or 
\begin{equation}
  \label{sup-est-gen-per-2}
\begin{split}
  & \| \frac{1}{\langle n_{1} \rangle ^{2s}
  \langle \sigma_{1} \rangle
  ^{2a}} \sum_{n} \int_{\tau} \frac{\langle n \rangle ^{2s}}{\langle
  n - n_{1}\rangle ^{2s}}  \times \frac{\chi_{A_{j}}}{\langle
  \sigma \rangle^{2b}  \langle \sigma_{2} \rangle^{2b}} d \tau 
  \|_{L^{\infty}_{n_{1}, \tau_{1}}} < \infty
\end{split}
\end{equation}
%
or
\begin{equation}
  \label{sup-est-gen-per-3}
\begin{split}
  \| \sum_{n_{1}} \int_{\tau} \frac{\chi^{*}_{A_{j}}
    \langle n \rangle ^{2s}
  }{\langle n_{1} \rangle^{2s} \langle
    n-n_{1} \rangle ^{2s} \langle \sigma^{*}  
    \rangle ^{2a}
    \langle \sigma_{1}^{*} \rangle^{2b}
    \langle  \sigma_{2}^{*} \rangle^{2b}} d \tau  \|_{L^{\infty}_{n, \lambda}}
\end{split}
\end{equation}
%
%
where we consider cases
\begin{enumerate}[(I)]
    \item $ \sigma=\tau+n^2,\quad \sigma_1=\tau_1+n_1^2,\quad \sigma_2=\tau -
      \tau_1+(n - n_1)^2$,
\label{it-1}
    \item $ \sigma=\tau-n^2,\quad \sigma_1=\tau_1-n_1^2,\quad \sigma_2=\tau - \tau_1+(n - n_1)^2$,
\label{it-2}
    \item  $\sigma=\tau+n^2,\quad \sigma_1=\tau_1-n_1^2,\quad \sigma_2=\tau - \tau_1+(n - n_1)^2$,
      \label{it-3}
    \item $\sigma=\tau-n^2,\quad \sigma_1=\tau_1+n_1^2,\quad \sigma_2=\tau - \tau_1-(n - n_1)^2$,
\label{it-4}
    \item $\sigma=\tau+n^2,\quad \sigma_1=\tau_1+n_1^2,\quad \sigma_2=\tau - \tau_1-(n - n_1)^2$,
\label{it-5}
    \item $\sigma=\tau-n^2,\quad \sigma_1=\tau_1-n_1^2,\quad \sigma_2=\tau - \tau_1-(n - n_1)^2$.
\label{it-6}
\end{enumerate}
%
for \cref{sup-est-gen-per-1} and \cref{sup-est-gen-per-2}, and cases
%
\begin{enumerate}[(I)]
\item $ \sigma^{*}=\tau+n^2,\quad \sigma^{*}_1=\tau - \lambda+n_1^2,\quad
  \sigma^{*}_2=\lambda+(n - n_1)^2$, \label{it-1-star} \item $
  \sigma^{*}=\tau-n^2,\quad \sigma^{*}_1=\tau - \lambda-n_1^2,\quad
  \sigma^{*}_2=\lambda+(n - n_1)^2$, \label{it-2-star} \item
  $\sigma^{*}=\tau+n^2,\quad \sigma^{*}_1=\tau - \lambda-n_1^2,\quad
  \sigma^{*}_2=\lambda+(n - n_1)^2$, \label{it-3-star} \item
  $\sigma^{*}=\tau-n^2,\quad \sigma^{*}_1=\tau - \lambda+n_1^2,\quad
  \sigma^{*}_2=\lambda-(n - n_1)^2$, \label{it-4-star} \item
  $\sigma^{*}=\tau+n^2,\quad \sigma^{*}_1=\tau - \lambda+n_1^2,\quad
  \sigma^{*}_2=\lambda-(n - n_1)^2$, \label{it-5-star} \item
  $\sigma^{*}=\tau-n^2,\quad \sigma^{*}_1=\tau - \lambda-n_1^2,\quad
  \sigma^{*}_2= \lambda-(n - n_1)^2$.  \label{it-6-star}
  \end{enumerate}
for \cref{sup-est-gen-per-3}.
%
%
\begin{framed}
\begin{remark}
Note that the cases $\sigma=\tau+n^2,\quad \sigma_1=\tau_1-n_1^2,\quad
\sigma_2=\tau - \tau_1-(n - n_1)^2$ and $\sigma=\tau-n^2,\quad
\sigma_1=\tau_1+n_1^2,\quad \sigma_2=\tau - \tau_1+(n - n_1)^2$ cannot occur, since
$\tau_1< 0, \tau-\tau_1< 0$ implies $\tau<0$ and $\tau_1\geq 0, \tau-\tau_1\geq
0$ implies $\tau\geq 0$. An analogous argument holds for $\sigma^{*},
\sigma_{1}^{*}$ and $\sigma_{2}^{*}$.
\end{remark}
\end{framed}
%
Observe that the transformation $(n, \tau, n_{1}, \tau_{1}) \mapsto -(n, \tau,
n_{1}, \tau_{1})$ reduces \cref{it-3} to \cref{it-4}, \cref{it-2} to
\cref{it-5}, and \cref{it-1} to \cref{it-6}. Furthermore, the change of
variables $\tau_{2} = \tau - \tau_{1}, n_{2} = n - n_{1}$, and the
transformation $(n, \tau, n_{2}, \tau_{2}) \mapsto - (n, \tau, n_{2},
\tau_{2})$ reduces \cref{it-5} to \cref{it-4}. Since $L^{2}$ is invariant
under change of variables and reflections, we may without loss of generality
restrict our attention to cases \cref{it-4} and \cref{it-6}.
 \subsubsection{Case \cref{it-6}} 
\label{ssec:case-it-6}
Let 
%
%
\begin{align*}
A_1&=\{(n, n_1, \tau, \tau_1)\in A: n=0\},\\
A_2&=\{(n, n_1, \tau, \tau_1)\in A: n_1 = n \},\\
A_3&=\{(n, n_1, \tau, \tau_1)\in A: n_1=0 \},\\
A_4&=\{(n, n_1, \tau, \tau_1)\in A: n \neq 0, n_1 \neq 0 \ \text{and} \ n_1 \neq n \}.
\end{align*} 
%
%
%
We seek to bound
\begin{equation*}
\begin{split}
  & \frac{1}{\langle n_{1} \rangle ^{2s}
  \langle \tau_{1} - n_{1}^{2} \rangle
  ^{2a}} \sum_{n} \int_{\tau} \frac{\langle n \rangle ^{2s}}{\langle
  n - n_{1}\rangle ^{2s}}  \times \frac{\chi_{A_{1}}}{\langle
  \tau - n^{2}  \rangle^{2b}  \langle \tau - \tau_{1} - (n - n_{1})^{2}
  \rangle^{2b}} d \tau 
\end{split}
\end{equation*}
which is equal to 
%
\begin{equation}
  \label{case-1-term-1-reduc}
\begin{split}
  & \frac{1}{\langle n_{1} \rangle ^{4s}
  \langle \tau_{1} - n_{1}^{2} \rangle
  ^{2a}} \int_{\tau} \frac{1}{\langle
  \tau  \rangle^{2b}  \langle \tau - \tau_{1} - n_{1}^{2}
  \rangle^{2b}} d \tau.
\end{split}
\end{equation}
%
Following Ginibre, Tsutsumi, Velo~\cite{Ginibre:1997fk}, Kenig, Ponce, Vega~\cite{Kenig:1996aa}, and others,
we now need the following Calculus lemma, whose proof is provided in the
appendix.
%
%%%%%%%%%%%%%%%%%%%%%%%%%%%%%%%%%%%%%%%%%%%%%%%%%%%%%
%
%
%				 Calculus Lemma
%
%
%%%%%%%%%%%%%%%%%%%%%%%%%%%%%%%%%%%%%%%%%%%%%%%%%%%%%
%
%
\begin{lemma}[Lemma 3.1 in Farah periodic]
	\label{lem:calc}
 %
 Fix $p, q > 0$ such that $p +q >1$, and let $r =\min\left\{p, q, p+q-1
 \right\}$. Then 
 %
 \begin{enumerate}[(I)]
   \item
For $\alpha=\beta \ \text{or} p \neq 1 \ \text{or} \ q \neq 1$
 \begin{equation*}
\begin{split}
  & \int_{\rr} \frac{1}{\langle x - \alpha \rangle ^{p} \langle x -
  \beta \rangle
  ^{q}} d x
  \le \frac{c_{p,q}}{\langle \alpha - \beta \rangle ^{r}}, 
  \end{split}
\end{equation*}
  \item
    \begin{equation*}
  \int_{\rr} \frac{1}{\langle x - \alpha \rangle  \langle x - \beta
  \rangle} d x
  \le  \frac{4 \log \langle \alpha - \beta \rangle}{\langle \alpha - \beta
  \rangle}, \quad \alpha \neq \beta.
\end{equation*}
\end{enumerate}
\end{lemma}
\begin{framed}
  %
  %
  \begin{remark}
  \label{rem:farah-wrong}
  This is an improvement on the corresponding lemma Farah has in his paper. His
  lemma is not correct---it fails for $p=q=1$.
  \end{remark}
\end{framed}
  %
  %
%
By \cref{lem:calc}, \cref{case-1-term-1-reduc} is bounded by
%
%
\begin{equation*}
\begin{split}
  & \frac{c}{\langle n_{1} \rangle ^{4s} \langle \tau_{1} - n_{1}^{2} \rangle
  ^{2a}
  \langle \tau_{1} + n_{1}^{2} \rangle^{2b}
} \lesssim
  \frac{1}{\langle n_{1} \rangle ^{4s} \langle \tau_{1} - n_{1}^{2} \rangle^{2a}
\langle \tau_{1} + n_{1}^{2} \rangle^{2a}
}
\end{split}
\end{equation*}
%
%
%
%
which is bounded by
%
%
%
\begin{equation*}
\begin{split}
  \frac{c}{\langle n_{1} \rangle^{4(s + a)}} < \infty, \qquad s \ge -a
\end{split}
\end{equation*}
due to the following lemma, whose proof is provided in the appendix.
%
%
%
%%%%%%%%%%%%%%%%%%%%%%%%%%%%%%%%%%%%%%%%%%%%%%%%%%%%%
%
%
%               important calc lower bound estimate
%
%
%%%%%%%%%%%%%%%%%%%%%%%%%%%%%%%%%%%%%%%%%%%%%%%%%%%%%
%
%
\begin{lemma}
\label{lem:calc-lower-bound}
For $ a, x \in \rr$
\begin{equation}
  \label{simp-est-lower-bound}
\begin{split}
\langle x-a \rangle \langle x+a \rangle  \ge \langle a \rangle.
\end{split}
\end{equation}
\end{lemma}
%
%
%
%
%
\begin{framed}
\begin{remark}
This result is optimal for region $A_{1}$. To illustrate this, we now estimate
%
%
\begin{equation*}
  \begin{split}
     \frac{\langle n
    \rangle ^{2s}}{\langle \tau - n^{2} \rangle ^{2a}}
    \sum_{n_{1}} \int_{\tau_{1}} \frac{\chi_{A_{1}}}{\langle n_{1} \rangle ^{2s} \langle n-n_{1} \rangle ^{2s} 
    \langle \tau_{1} - n_{1}^{2} \rangle^{2b} \langle  \tau - \tau_{1} -
    (n - n_{1})^{2} \rangle^{2b}}
    d \tau_1 
  \end{split}
\end{equation*}
which reduces to 
\begin{equation}
  \label{pathological-equality-case-1}
  \begin{split}
    \frac{1}{\langle \tau \rangle ^{2a}}
    \sum_{n_{1}} \int_{\tau_{1}} \frac{\chi_{A_{1}}}{\langle n_{1} \rangle ^{4s}
    \langle \tau_{1} - n_{1}^{2} \rangle^{2b} \langle  \tau - \tau_{1} -
    n_{1}^{2} \rangle^{2b}}
    d \tau_1.
  \end{split}
\end{equation}
%
%
%
%
Setting $\tau = 0$, we obtain
%
%
%
\begin{equation*}
\begin{split}
  \sum_{n_{1}} \langle & n_{1}\rangle ^{-4s} \int_{\tau_{1}} \frac{1}{\langle
   \tau_{1} - n_{1}^{2} \rangle ^{4b}}d \tau_{1}
   \\
   & = \sum_{n_{1}} \langle
  n_{1}\rangle ^{-4s} \int_{\tau'} \frac{1}{\langle
   \tau' \rangle ^{4b}}d \tau'
   \\
   & \simeq \sum_{n_{1}} \langle n_{1} \rangle ^{-4s}, \quad b > 1/4
   \\
   & < \infty, \quad s > 1/4.
\end{split}
\end{equation*}
%
Hence, we cannot hope to bound \cref{pathological-equality-case-1} for $s \le
1/4$ using this splitting. Instead, consider now
%
%
\begin{equation*}
\begin{split}
  \sum_{n_{1}} \int_{\tau} \frac{\chi^{*}_{A_{1}} \langle n \rangle ^{2s}}{\langle n_{1} \rangle
  ^{2s} \langle n - n_{1} \rangle ^{2s} \langle \tau - n^{2} \rangle ^{2a} \langle
  \tau - \lambda +  n_{1}^{2} \rangle^{2b}
   \langle \lambda - (n - n_{1})^{2} \rangle^{2b}} d \tau
\end{split}
\end{equation*}
%
%
which reduces to 
\begin{equation*}
\begin{split}
  \sum_{n_{1}} \int_{\tau} \frac{\chi^{*}_{A_{1}}}{\langle n_{1} \rangle
  ^{4s} \langle \tau \rangle ^{2a} \langle \tau - \lambda + n_{1}^{2} \rangle^{2b}
   \langle \lambda  - n_{1}^{2} \rangle^{2b}} d \tau
\end{split}
\end{equation*}
Applying \cref{lem:calc} and \cref{lem:calc-lower-bound}, this is bounded by
%
\begin{equation*}
\begin{split}
  & \sum_{n_{1}} \frac{\chi^{*}_{A_{1}}}{\langle n_{1} \rangle ^{4s} \langle
  \lambda- n_{1}^{2} \rangle^{2b}  \langle \lambda + n_{1}^{2} \rangle
  ^{2a}}
  \\
  & \le \sum_{n_{1}}  \frac{\chi_{A_{1}}^{*}}{\langle n_{1} \rangle^{4(s + a )}}
  \\
  & < \infty, \qquad s > 1/4 - a.
\end{split}
\end{equation*}
%
%
%
\end{remark}
\end{framed}
%
Applying \cref{lem:calc}, we now bound 
\begin{equation*}
  \begin{split}
    & \frac{\langle n
    \rangle ^{2s}}{\langle \tau - n^{2} \rangle ^{2a}}
    \sum_{n_{1}} \int_{\tau_{1}} \frac{\chi_{A_{2}}}{\langle n_{1} \rangle ^{2s} \langle n-n_{1} \rangle ^{2s} 
    \langle \tau_{1} - n_{1}^{2} \rangle^{2b} \langle  \tau - \tau_{1} -
    (n - n_{1})^{2} \rangle^{2b}}
    d \tau_1 
    \\
    & = 
    \langle \tau -n^{2} \rangle ^{-2a}\int_{\tau_{1}} \frac{1}{\langle \tau_{1} -
  n^{2} \rangle^{2b} \langle
  \tau - \tau_{1}\rangle^{2b}}d \tau_{1}
  \\
  & \lesssim 
  \langle \tau - n^{2} \rangle ^{-2a-2b} 
  \\
  & < \infty.
\end{split}
\end{equation*}
%
%
Similarly, we bound
%
%
\begin{equation}
\begin{split}
  & \frac{\langle n
    \rangle ^{2s}}{\langle \tau - n^{2} \rangle ^{2a}}
    \sum_{n_{1}} \int_{\tau_{1}} \frac{\chi_{A_{3}}}{\langle n_{1} \rangle ^{2s} \langle n-n_{1} \rangle ^{2s} 
    \langle \tau_{1} - n_{1}^{2} \rangle^{2b} \langle  \tau - \tau_{1} -
    (n - n_{1})^{2} \rangle^{2b}}
    d \tau_1 
    \\
  & = \langle \tau - n^{2} \rangle ^{-2a}
  \int_{\tau_{1}} \frac{1}{\langle \tau_{1} \rangle^{2b}  \langle \tau -
  \tau_{1} - n^{2} \rangle^{2b}}
d \tau_1 
\\
  & \lesssim   \langle \tau - n^{2} \rangle ^{-2a-2b} 
  \\
  & < \infty.
	\end{split}
\end{equation}
%
%
We now 
partition $ A_{4}$ into two parts
\begin{align*}
A_{4,1}&=\{(n, n_1, \tau, \tau_1)\in A_3: |\tau_1-n_1^2|\leq|\tau-n^2|\},\\
A_{4,2}&=\{(n, n_1, \tau, \tau_1)\in A_3: |\tau-n^2|\leq|\tau_1-n_1^2| \}.
\end{align*} 
Furthermore, by the symmetry of the convolution, we may assume without loss of
generality that
$$|(\tau-\tau_1)-(n-n_1)^2|\leq|\tau_1-n_1^2|\}.$$
Then in region $A_{4,1}$
\begin{equation}
\begin{split}
  | \tau - n^{2} |
  & \ge \frac{1}{3}\left[ | \tau_{1} - n_{1}^{2} | + | \tau -
  \tau_{1} - (n - n_{1})^{2}
  | + | \tau - n^{2} | \right]
  \\
  & \ge \frac{1}{3} | - n_{1}^{2} - (n - n_{1})^{2} + n^{2} |
  \\
  & = \frac{2}{3} | n_{1} | | n - n_{1} |
  \\
  & \gtrsim | n_{1} |. 
\end{split}
\label{smoothing-per-4-1}
\end{equation}
%
%
Hence, applying apply \cref{lem:calc} and \cref{smoothing-per-4-1}
we obtain
%
%
%
%
\begin{equation}
  \label{region-a41}
\begin{split}
& \langle \tau - n^{2}  \rangle ^{-2a} \langle n
    \rangle ^{2s}
    \sum_{n_{1}} \int_{\tau_{1}} \frac{\chi_{A_{4,1}}}{\langle n_{1} \rangle ^{2s} \langle n-n_{1} \rangle ^{2s} 
\langle \tau_{1} - n_{1}^{2}  \rangle^{2b} \langle  \tau - \tau_{1} - (n -
n_{1})^{2}  \rangle^{2b}}
d \tau_1 
\\
& \lesssim \langle \tau - n^{2} \rangle ^{-2a} \langle n \rangle ^{2s}
\sum_{n_{1} \in
\zz}  \frac{\chi_{A_{4,1}}}{\langle n_{1} \rangle ^{2s} \langle n - n_{1} \rangle
^{2s} \langle \tau - n^{2} - 2n_{1}^{2} + 2nn_{1}  \rangle^{2b}}
\\
& \lesssim 
\sum_{n_{1} \in
\zz}  \frac{\langle n_1 \rangle ^{-2s} \langle n - n_{1} \rangle ^{-2s}}{\langle
n \rangle ^{-2s} \langle n_{1} \rangle
^{2a}} \times \frac{\chi_{A_{4,1}}}{\langle \tau - n^{2} - 2n_{1}^{2} + 2nn_{1}
\rangle^{2b}}.
\end{split}
\end{equation}
%
%
We now need the following. 
%
%
%%%%%%%%%%%%%%%%%%%%%%%%%%%%%%%%%%%%%%%%%%%%%%%%%%%%%
%
%
%                Integer Bound
%
%
%%%%%%%%%%%%%%%%%%%%%%%%%%%%%%%%%%%%%%%%%%%%%%%%%%%%%
%
%
\begin{lemma}
  Let $n, n_1 \in \zz$ such that $n_{1} \neq 0$ and $n_{1} \neq n$.
  Then
  %
  %
  \begin{equation*}
  \begin{split}
    | n | \le | n - n_{1} | | n_{1} |.
  \end{split}
  \end{equation*}
  %
  %
\label{lem:integer-bound}
\end{lemma}
%
Hence,
%
\begin{equation}
  \label{growth-term-per}
\begin{split}
  \frac{\langle n \rangle ^{2s} \chi_{A_{4}}}{\langle n_{1} \rangle ^{2s} \langle n -
  n_{1} \rangle ^{2s}} \le \langle n_{1} \rangle ^{\gamma(s)},
  \quad 
  \gamma(s) = 
  \begin{cases} 0, \quad & s \ge 0
    \\
    4|s|, \quad & s < 0.
  \end{cases}
\end{split}
\end{equation}
%
%
%
%
Since $a \ge 0$, it follows from \cref{growth-term-per} that 
%
\begin{equation}
  \label{growth-term-control-per}
  \frac{\langle n_1 \rangle ^{-2s} \langle n - n_{1} \rangle
  ^{-2s}\chi_{A_{4}}}{\langle
n \rangle ^{-2s} \langle n_{1} \rangle
^{2a}} \le 1, \quad s \ge -a/2
\end{equation}
%
%
which we use to bound the right hand side of \cref{region-a41} by
%
%
\begin{equation*}
\begin{split}
\sum_{n_{1} \in
\zz} 
\frac{1}{\langle \tau - n^{2} - 2n_{1}^{2} + 2nn_{1}  \rangle^{2b}}
\end{split}
\end{equation*}
%
%
%
which is finite for $b > 1/4$, due to the following lemma, whose proof can be found in
Kenig, Ponce, and Vega
\cite{Kenig-Ponce-Vega-1996-Quadratic-forms-for-the-1-D-semilinear} and in the
appendix.
\begin{lemma}
  \label{lem:sum-estimate}
If $\gamma>1/2$, then
\begin{equation}\label{CI2}
\sup_{(n,\tau)\in \zz \times \rr}\sum_{n_1\in \zz}\frac{1}{(1+|\tau\pm
n_1(n-n_1)|)^{\gamma}}<\infty. 
\end{equation}
\end{lemma}
%
Working now in region $A_{4,2}$, we seek to bound 
\begin{equation}
  \label{region-4-2}
\begin{split}
  &  \frac{1}{\langle n_{1} \rangle ^{2s}
  \langle \tau_{1} - n_{1}^{2} \rangle
  ^{2a}} \sum_{n} \int_{\tau} \frac{\langle n \rangle ^{2s}}{\langle
  n - n_{1}\rangle ^{2s}}  \times \frac{\chi_{A_{4,2}}}{\langle
  \tau - n^{2} \rangle^{2b}  \langle \tau - \tau_{1} - (n - n_{1})^{2} \rangle^{2b}} d \tau 
\end{split}
\end{equation}
%
%
Note that in region $A_{4,2}$
\begin{equation}
  \label{smoothing-per-4-2}
\begin{split}
  | \tau_{1} - n_{1}^{2} |
  & \ge \frac{1}{3}\left[ | \tau_{1} - n_{1}^{2} | + | \tau -
  \tau_{1} - (n - n_{1})^{2}
  | + | \tau - n^{2} | \right]
  \\
  & \ge \frac{1}{3} | - n_{1}^{2} - (n - n_{1})^{2} + n^{2} |
  \\
  & = \frac{2}{3} | n_{1} | | n - n_{1} |
  \\
  & \gtrsim | n_{1} |.
\end{split}
\end{equation}
Hence, applying
\cref{lem:calc}, \cref{growth-term-control-per}, and
\cref{smoothing-per-4-2}, we bound \cref{region-4-2} by
%
%
\begin{equation*}
\begin{split}
&  \frac{c}{\langle n_{1} \rangle ^{2s}
  \langle \tau_{1} - n_{1}^{2} \rangle
  ^{2a}} \sum_{n} \frac{\langle n \rangle ^{2s}}{\langle
  n - n_{1}\rangle ^{2s}}  \times \frac{\chi_{A_{4,2}}}{\langle
  \tau_{1} - 2nn_{1} + n_{1}^{2} \rangle^{2b}} 
  \\
  & \lesssim 
  \sum_{n} \frac{\chi_{A_{4,2}}}{\langle
  \tau_{1} - 2nn_{1} + n_{1}^{2} \rangle^{2b}},
  \quad  s \ge -a/2
  \end{split}
\end{equation*}
%
%
Since the right hand side is bounded for $b > 1/4$ by \cref{lem:sum-estimate}, this
completes the proof for case \cref{it-6}.
\subsubsection{Case \cref{it-4}} 
\label{ssec:case-it-4}
Let 
%
%
\begin{align*}
B_1&=\{(n, n_1, \tau, \tau_1)\in B: n=0\},\\
B_2&=\{(n, n_1, \tau, \tau_1)\in B: n_1 = 0 \},\\
B_3&=\{(n, n_1, \tau, \tau_1)\in B: n \neq 0, n_1 \neq 0 \}.
\end{align*} 
%
%
We seek to bound
\begin{equation*}
\begin{split}
  \sum_{n_{1}} \int_{\tau} \frac{\chi^{*}_{B_{3}}
    \langle n \rangle ^{2s}
  }{\langle n_{1} \rangle^{2s} \langle
    n-n_{1} \rangle ^{2s} \langle \tau - n^{2}    \rangle ^{2a}
    \langle \tau - \lambda + n_{1}^{2} \rangle^{2b}
    \langle  \lambda + n_{1}^{2} \rangle^{2b}} d \tau  
\end{split}
\end{equation*}
which is equal to
\begin{equation*}
\begin{split}
  \sum_{n_{1}} \int_{\tau} \frac{1}
    {\langle n_{1} \rangle^{4s} \langle \tau    \rangle ^{2a}
    \langle \tau - \lambda + n_{1}^{2} \rangle^{2b}
    \langle  \lambda + n_{1}^{2} \rangle^{2b}} d \tau  
\end{split}
\end{equation*}
which by \cref{growth-term-per} and \cref{lem:calc} is bounded by
%
%
\begin{equation*}
\begin{split}
  \sum_{n_{1}} \frac{c}
  {\langle n_{1} \rangle^{4s} \langle n_{1}^{2} - \lambda   \rangle ^{2a}
  \langle n_{1}^{2} + \lambda \rangle^{2b}
   } d \tau.
\end{split}
\end{equation*}
%
%
Applying estimate \cref{lem:calc-lower-bound}, this is bounded by
\begin{equation*}
\begin{split}
  & \sum_{n_{1}} \frac{c}
  {\langle n_{1} \rangle ^{4(s+a)}} 
  \\
  & < \infty, \qquad s > 1/4 -a.
\end{split}
\end{equation*}
\begin{framed}
  \begin{remark}
This result cannot be improved. For pedagogical purposes, we now estimate 
%
\begin{equation*}
  \begin{split}
     \frac{\langle n
     \rangle ^{2s}}{\langle \tau - n^{2} \rangle ^{2a}}
    \sum_{n_{1}} \int_{\tau_{1}} \frac{\chi_{A_{j}}}{\langle n_{1} \rangle ^{2s} \langle n-n_{1} \rangle ^{2s} 
    \langle \tau_{1} + n_{1}^{2} \rangle^{2b} \langle  \tau - \tau_{1} -
    (n - n_{1})^{2} \rangle^{2b}} 
    d \tau_1  
  \end{split}
\end{equation*}
which reduces to
%
\begin{equation}
  \label{pathological-equality}
\begin{split}
   \langle \tau \rangle ^{-2a} \sum_{n_{1}} \langle
   n_{1}\rangle ^{-4s} \int_{\tau_{1}} \frac{\chi_{B_{1}}}{\langle \tau_{1} + n_{1}^{2} \rangle^{2b} \langle
  \tau - \tau_{1} - n_{1}^{2}\rangle^{2b}}d \tau_{1}.
\end{split}
\end{equation}
Applying \cref{lem:calc}, we obtain the bound
%
%
\begin{equation*}
\begin{split}
  c  \langle \tau \rangle
  ^{-2a-2b} \sum_{n_{1}} \langle n_{1} \rangle
  ^{-4s} 
  & \lesssim \sum_{n_{1}} \langle n_{1} \rangle ^{-4s}
  \\
  & < \infty, \quad s > \frac{1}{4}.
\end{split}
\end{equation*}
%
%
Note that we cannot improve our lower bound for $s$ using this splitting.
To see this, we set $\tau = 0$
in the right hand side of \cref{pathological-equality} and obtain
%
%
%
\begin{equation*}
\begin{split}
   \sum_{n_{1}} \langle
   & n_{1}\rangle ^{-4s} \int_{\tau_{1}} \frac{1}{\langle \tau_{1} + n_{1}^{2} \rangle^{2b} \langle
   \tau_{1} + n_{1}^{2}\rangle^{2b}}d \tau_{1} 
   \\
   & = \sum_{n_{1}} \langle
  n_{1}\rangle ^{-4s} \int_{\tau'} \frac{1}{\langle
   \tau' \rangle ^{4b}}d \tau'
   \\
   & \simeq \sum_{n_{1}} \langle n_{1} \rangle ^{-4s}, \quad b > 1/4
   \\
   & < \infty, \quad s > 1/4.
\end{split}
\end{equation*}
%
%
%
Hence, we now try to bound
\begin{equation}
\begin{split}
  & \frac{1}{\langle n_{1} \rangle ^{2s}
  \langle \tau_{1} + n_{1}^{2} \rangle
  ^{2a}} \sum_{n} \int_{\tau} \frac{\langle n \rangle ^{2s}}{\langle
  n - n_{1}\rangle ^{2s}}  \times \frac{\chi_{B_{1}}}{\langle
  \tau - n^{2}  \rangle^{2b}  \langle \tau - \tau_{1} - (n - n_{1})^{2}
  \rangle^{2b}} d \tau 
\end{split}
\end{equation}
instead, and look to obtain a better result. This is equal to 
%
\begin{equation*}
\begin{split}
  & \frac{1}{\langle n_{1} \rangle ^{4s}
  \langle \tau_{1} + n_{1}^{2} \rangle
  ^{2a}} \int_{\tau} \frac{1}{\langle
  \tau  \rangle^{2b}  \langle \tau - \tau_{1} - n_{1}^{2}
  \rangle^{2b}} d \tau
\end{split}
\end{equation*}
%
%
which by \cref{lem:calc} is bounded by
%
%
\begin{equation*}
\begin{split}
  & \frac{c}{\langle n_{1} \rangle ^{4s}
  \langle \tau_{1} + n_{1}^{2} \rangle
  ^{2a + 2b}}
\\
& < \infty, \qquad s \ge 0. 
\end{split}
\end{equation*}
%
\end{remark}
\end{framed}
%
%
\begin{framed}
\begin{remark}
The case $n=0$ is easy to estimate for the Boussinesq, since the
$A_{1}$ and $B_{1}$ terms vanish for the Boussinesq. Heuristically this is
because of the presence of the laplacian term for the Boussinesq. More
specifically, the integral form of the Boussinesq ivp
\begin{equation*}
  \begin{split}
    u(x,t)
    & = \frac{1}{2\pi}\sum_{n \in \zz} e^{inx} \wh{u_{0}}(n) \frac{e^{i(n^{2} + n^{4})^{1/2}t} + e^{-i(n^{2} + n^{4})^{1/2}t}}{2} 
    \\
    & + \frac{1}{2 \pi}\sum_{n \in \zz} e^{inx}
    \wh{u_{1}}(n)\frac{e^{i(n^{2} + n^{4})^{1/2}t} - e^{-i(n^{2} + n^{4})^{1/2}t}}{2 i (n^{2} +
    n^{4})^{1/2}} 
    \\
    & - \frac{1}{2 \pi}\sum_{n \in \zz} e^{inx}
    \int_{0}^{t}\frac{e^{i(n^{2} + n^{4})^{1/2}(t-t')}-e^{-i(n^{2} + n^{4})^{1/2}(t-t')}}{2 i (n^{2} +
    n^{4})^{1/2}}
    \wh{(u^{2})_{xx}}(n, t') dt'.
  \end{split}
\end{equation*}
has a nonlinearity which vanishes for $n =0$, since 
$$\frac{n^{2}}{(n^{2} + n^{4})^{1/2}} |_{n=0} = 0.$$ 
\label{rem:bous-easier}
\end{remark}
\end{framed}
%
%
Applying \cref{lem:calc}, we now bound 
\begin{equation*}
  \begin{split}
    & \frac{\langle n
    \rangle ^{2s}}{\langle \tau - n^{2} \rangle ^{2a}}
    \sum_{n_{1}} \int_{\tau_{1}} \frac{\chi_{B_{2}}}{\langle n_{1} \rangle ^{2s} \langle n-n_{1} \rangle ^{2s} 
    \langle \tau_{1} + n_{1}^{2} \rangle^{2b} \langle  \tau - \tau_{1} -
    (n - n_{1})^{2} \rangle^{2b}}
    d \tau_1 
    \\
    & = 
   \langle \tau -n^{2} \rangle ^{-2a}\int_{\tau_{1}} \frac{1}{\langle \tau_{1} +
  n^{2} \rangle^{2b} \langle
  \tau - \tau_{1}\rangle^{2b}}d \tau_{1}
  \\
  & \lesssim 
  \langle \tau - n^{2} \rangle ^{-2a-2b} 
  \\
  & < \infty.
\end{split}
\end{equation*}
%
%
We now 
partition $ B_{3}$ into three parts
\begin{align*}
B_{3,1}&=\{(n, n_1, \tau, \tau_1)\in B_3:
|\tau-\tau_1-(n-n_1)^2|, |\tau_1+n_1^2| \le |\tau-n^2|\},\\
B_{3,2}&=\{(n, n_1, \tau, \tau_1)\in B_3:
|\tau-\tau_1-(n-n_1)^2|, |\tau-n^2| \le |\tau_1+n_1^2|\},\\
B_{3,3}&=\{(n, n_1, \tau, \tau_1)\in B_3: |\tau_{1}+n_{1}^2|, | \tau - n^{2} | \le |  \tau - \tau_{1} -
(n - n_{1})^{2} |\}.
\end{align*} 
Then in region $B_{3,1}$
\begin{equation}
\begin{split}
  | \tau - n^{2} |
  & \ge \frac{1}{3}\left[ | \tau_{1} + n_{1}^{2} | + | \tau -
  \tau_{1} - (n - n_{1})^{2}
  | + | \tau - n^{2} | \right]
  \\
  & \ge \frac{1}{3} |  n_{1}^{2} - (n - n_{1})^{2} + n^{2} |
  \\
  & = \frac{2}{3} | n_{1} | | n |
  \\
  & \gtrsim | n_{1} |.
\end{split}
\label{smoothing-per-3-1-case-6}
\end{equation}
%
%
Estimating first in region
$B_{3,1}$, we apply \cref{lem:calc} and \cref{smoothing-per-3-1-case-6}
to obtain
%
%
%
%
\begin{equation}
  \label{region-a31-case-6}
\begin{split}
& \langle \tau - n^{2}  \rangle ^{-2a} \langle n
    \rangle ^{2s}
    \sum_{n_{1}} \int_{\tau_{1}} \frac{\chi_{B_{3,1}}}{\langle n_{1} \rangle ^{2s} \langle n-n_{1} \rangle ^{2s} 
\langle \tau_{1} - n_{1}^{2}  \rangle^{2b} \langle  \tau - \tau_{1} - (n -
n_{1})^{2}  \rangle^{2b}}
d \tau_1 
\\
& \lesssim \langle \tau - n^{2} \rangle ^{-2a} \langle n \rangle ^{2s}
\sum_{n_{1} \in
\zz}  \frac{\chi_{B_{3,1}}}{\langle n_{1} \rangle ^{2s} \langle n - n_{1} \rangle
^{2s} \langle \tau - n^{2} - 2n_{1}^{2} + 2nn_{1}  \rangle^{2b}}
\\
& \lesssim 
\sum_{n_{1}}  \frac{\langle n_1 \rangle ^{-2s} \langle n - n_{1} \rangle ^{-2s}}{\langle
n \rangle ^{-2s} \langle n_{1} \rangle
^{2a}} \times \frac{\chi_{B_{3,1}}}{\langle \tau - n^{2} - 2n_{1}^{2} + 2nn_{1}
\rangle^{2b}}.
\end{split}
\end{equation}
%
%
Since $a \ge 0$, it follows from \cref{growth-term-control-per} that the right
hand side is bounded by
%
%
%
%
\begin{equation*}
\begin{split}
  \sum_{n_{1}}   \frac{\chi_{B_{3,1}}}{\langle \tau - n^{2} - 2n_{1}^{2} + 2nn_{1}
\rangle^{2b}}
\end{split}
\end{equation*}
%
%
%
%
%
%
which is finite for $b > 1/4$, due to \cref{lem:sum-estimate}. 
Working now in region $B_{3,2}$, we seek to estimate 
\begin{equation}
  \label{region-B-3-split-3}
\begin{split}
  &  \frac{1}{\langle n_{1} \rangle ^{2s}
  \langle \tau_{1} + n_{1}^{2} \rangle
  ^{2a}} \sum_{n} \int_{\tau} \frac{\langle n \rangle ^{2s}}{\langle
  n - n_{1}\rangle ^{2s}}  \times \frac{\chi_{B_{3,2}}}{\langle
  \tau - n^{2} \rangle^{2b}  \langle \tau - \tau_{1} - (n - n_{1})^{2} \rangle^{2b}
} d \tau.
\end{split}
\end{equation}
%
%
\begin{framed}
\begin{remark}
Farah estimates differently here. His
way is convoluted, and unnecessary; see page $960$ of Farah periodic for details.
\label{rem:fara-dif}
\end{remark}
\end{framed}
%
%
%
Note that in region $B_{3,2}$
\begin{equation}
  \label{smoothing-per-3-2-case-6}
\begin{split}
  | \tau_{1} + n_{1}^{2} |
  & \ge \frac{1}{3}\left[ | \tau_{1} + n_{1}^{2} | + | \tau -
  \tau_{1} - (n - n_{1})^{2}
  | + | \tau - n^{2} | \right]
  \\
  & \ge \frac{1}{3} |  n_{1}^{2} - (n - n_{1})^{2} + n^{2} |
  \\
  & = \frac{2}{3} | n_{1} | | n |
  \\
  & \gtrsim | n_{1} |.
\end{split}
\end{equation}
%
Hence, applying
\cref{lem:calc}, \cref{growth-term-control-per}, and
\cref{smoothing-per-3-2-case-6} to \cref{region-B-3-split-3}, we obtain the bound
%
%
\begin{equation*}
  \begin{split}
    &  \frac{c}{\langle n_{1} \rangle ^{2s}
    \langle \tau_{1} + n_{1}^{2} \rangle
    ^{2a}} \sum_{n} \frac{\langle n \rangle ^{2s}}{\langle
    n - n_{1}\rangle ^{2s}}  \times \frac{\chi_{B_{3,2}}}{\langle
    \tau_{1} - 2nn_{1} + n_{1}^{2} \rangle^{2b}} 
    \\
    & \lesssim 
    \sum_{n} \frac{\chi_{B_{3,2}}}{\langle
    \tau_{1} - 2nn_{1} + n_{1}^{2} \rangle^{2b}},
    \quad s \ge -a/2.
  \end{split}
\end{equation*}
%
%
%
It remains to show that 
%
%
%
\begin{equation}
  \label{sum-bound}
\begin{split}
\sum_{n_{1}} \frac{\chi_{B_{3,2}}}{\langle \tau - n^{2} + 2nn_{1}
\rangle^{2b}} < c, \qquad b > 1/2
\end{split}
\end{equation}
%
%
where $c$ does not depend on the choice of $\tau$ or $n$. 
%
%
To see this, note that if $\tau - n^{2} = 0$, the above follows easily, since
$n \neq 0$ in $B_{3,2}$.
Hence, assuming $\tau - n^{2} \neq 0$, we use the fact that $| \tau - n^{2} |
\gtrsim | n_{1} |$ and $n \neq 0$ in $B_{3,2}$ to obtain 
%
%
\begin{equation*}
\begin{split}
\sum_{n_{1}} \frac{\chi_{B_{3,2}}}{\langle \tau - n^{2} + 2nn_{1}
\rangle^{2b}}
& = \sum_{n_{1}} \frac{\chi_{B_{3,2}}}{(1 + | \tau - n^{2} +
2nn_{1} |)^{2b}}
\\
& \le \sum_{n_{1}} \frac{\chi_{B_{3,2}}}{(1 + | \tau - n^{2}
| | 1 + \frac{2nn_{1}}{\tau - n^{2}} |)^{2b}}
\\
& \lesssim \sum_{n_{1}} \frac{\chi_{B_{3,2}}}{(1 + |n_{1}|
| 1 + \frac{2nn_{1}}{\tau - n^{2}} |)^{2b}}
\\
& \le \sup_{a \in \tau} \sum_{n_{1}} \frac{\chi_{B_{3,2}}}{(1 + |n_{1}|
| 1 + an_{1}|)^{2b}}
\\
& = \sum_{n_{1}} \frac{\chi_{B_{3,2}}}{(1 + |n_{1}|)^{2b}}
\\
& < c, \qquad b > 1/2.
\end{split}
\end{equation*}
%
%
It remains to handle region $B_{3,3}$. It will be enough to bound
%
%
\begin{equation}
  \label{region-B-3-star-split}
\begin{split}
   \sum_{n_{1}} \int_{\tau} \frac{\chi^{*}_{B_{3,3}}
    \langle n \rangle ^{2s}
  }{\langle n_{1} \rangle^{2s} \langle  \tau  - n^{2}
    \rangle ^{2a}  \langle
n-n_{1} \rangle ^{2s}  \langle  \tau - \lambda+n_{1}^{2}
\rangle^{2b} \langle   \lambda  -(n - n_{1})^{2}
\rangle^{2b}} d \tau.
\end{split}
\end{equation}
%
Due to the presence of $\chi^{*}_{B_{3,3}}$ factor, we have the restriction
%
%
\begin{equation*}
\begin{split}
& |\tau - \lambda +n_{1}^2|, | \tau - n^{2} | \le |  \lambda -
(n - n_{1})^{2} | \ \text{and} \ n \neq 0, n_1 \neq 0.
\end{split}
\end{equation*}
%
It follows that
\begin{equation}
  \label{smoothing-per-3-3-case-6}
\begin{split}
  | \lambda - (n - n_{1})^{2} |
  & \ge \frac{1}{3}\left[ | \tau - \lambda + n_{1}^{2} | + | \lambda - (n - n_{1})^{2}
  | + | \tau - n^{2} | \right]
  \\
  & \ge \frac{1}{3} |  n_{1}^{2} - (n - n_{1})^{2} + n^{2} |
  \\
  & = \frac{2}{3} | n_{1} | | n |
  \\
  & \gtrsim | n_{1} |.
\end{split}
\end{equation}
Hence, applying
\cref{growth-term-control-per}, \cref{lem:calc}, and
\cref{smoothing-per-3-3-case-6}, we bound \cref{region-B-3-star-split} by
%
%
\begin{equation*}
\begin{split}
   & \sum_{n_{1}} \int_{\tau} \frac{\chi^{*}_{B_{3,3}}
  }{\langle  \tau  - n^{2}
    \rangle ^{2a}   \langle  \tau - \lambda+n_{1}^{2}
\rangle^{2b}} d \tau, \quad s \ge -a/2.
\\
& \lesssim  \sum_{n_{1}} \frac{\chi_{B_{3,3}^{*}}}{\langle n_{1}^{2} +
n^{2} - \lambda \rangle^{2a}}
\end{split}
\end{equation*}
which is bounded for $a > 1/4$ by
\cref{lem:sum-estimate}. This completes the proof of
\cref{prop:bilinear-est}. \qquad \qedsymbol
%

%%%%%%%%%%%%%%%%%%%%%%%%%%%%%%%%%%%%%%%%%%%%%%%%%%%%%
%
%
%                The non-periodic Case
%
%
%%%%%%%%%%%%%%%%%%%%%%%%%%%%%%%%%%%%%%%%%%%%%%%%%%%%%
%
\section{The Non-Periodic Case} 
\label{sec:non-periodic-case}
%
%
\begin{definition}
  Let $S(\rr^{2})$ denote the space of Schwartz functions on
  $\rr^{2}$.  For $s, b \in \rr$, $\mathcal{X}_{s,b}$
  denotes the completion of $S(\rr^{2})$ with
  respect to the norm
  %
  %
  \begin{equation}
  \begin{split}
    \|F\|_{\mathcal{X}_{s,b}} = \left( \int_{\rr} \int_{\rr} (1 + \xi^{2})^{s}
    (1 + | | \tau | - \xi^{2} |) \wh{F}(n, \tau) d \tau d \xi \right)^{1/2}.
  \end{split}
  \label{eqn:bous-norm-real}
  \end{equation}
  %
  %
  %
  %
\end{definition}
%
%
We need only establish the following bilinear estimate. All other arguments are
analogous to those in the periodic case. \qquad \qedsymbol
%
\begin{proposition}[Theorem 1.1 in Farah nonperiodic]
\label{prop:bilin-est-real}
If $b > 1/2$, $a > 1/4$, and $s \ge -a/2$, 
  then there exists $c > 0$ depending only on $a$, $b$, and $s$ such that
  %
  %
  \begin{equation*}
  \begin{split}
    \| uv \|_{\mathcal{X}_{s,-a}} \le c \| u \|_{\mathcal{X}_{s,b}} \| v \|_{\mathcal{X}_{s,b}}.
  \end{split}
  \end{equation*}
  %
  %
\end{proposition}
%
%
%
\subsection{Proof of \hyperref[prop:bilin-est-real]{Non-Periodic Bilinear Estimate}} 
%
\label{sec:bilin-est-real}
Since $\| f \|_{\mathcal{X}_{s,-a}} \le \| f \|_{\mathcal{X}_{s, -a'}}$ for $a \ge a' \ge 0$, we assume
without loss of generality that $1/4 > a \le b$ and $b > 1/2$ throughout. 
By duality, it suffices to show that 
%
%%
\begin{equation}
	\label{duality-est-real}
	\begin{split}
    |	\int_{\rr} \int_{\rr} (1 + |\xi|)^{s}
		\phi(\xi, \tau) \wh{uv}(\xi, \tau)(1 
    + | |\tau| - \xi^{2} |^{-a}) d \tau d \xi | \lesssim \|u\|_{\mathcal{X}_{s,b}}
    \|v\|_{\mathcal{X}_{s,b}}
    \|\phi \|_{L^{2}_{\xi, \tau}}.
	\end{split}
\end{equation}
Note first that $|\wh{uv}(\xi, \tau) |  = | \wh{u} *  \wh{v} 
(\xi, \tau)|$. From this it follows that
%
%
\begin{equation}
	\label{non-lin-rep-real}
	\begin{split}
		| \wh{uv}(\xi, \tau)|
    & = | \int_{\xi_{1}}  \int_{\tau_{1}}
    \wh{u}\left( \xi_1,  \tau_1 \right) \wh{v}\left( \xi - \xi_1 , \tau - \tau_1   
\right) d \tau_1 |
\\
& \le  \int_{\xi_{1}}  \int_{\tau_{1}}
    |\wh{u}\left( \xi_1,  \tau_1 \right)| |\wh{v}\left( \xi - \xi_1 , \tau - \tau_1   
    \right)| d \tau_1 d \xi_{1}
\\
& = \int_{\xi_1} \int_{\tau_{1}} \frac{c_u\left( \xi_1, \tau_1 
\right)}{\langle \xi_1 \rangle ^s \langle |\tau_1| - \xi_1^{2} | \rangle ^{b}}
\\
& \times \frac{c_{v}\left( \xi - \xi_1, \tau - \tau_1 \right)}{\langle \xi -
\xi_1 \rangle ^s\ \langle |\tau - \tau_1 | -  (\xi - \xi_1)^{2} \rangle^{b}}
\ d \tau_1 d \xi_{1}
\end{split}
\end{equation}
%
%
where 
\begin{equation*}
	\begin{split}
		c_h(\xi, \tau) =
			\langle \xi \rangle ^s \langle |\tau| - \xi^{2} \rangle ^{b} | \wh{h}\left( \xi, \tau \right) |.
	\end{split}
\end{equation*}
%
%
From our work above, it follows that 
%
%
\begin{equation}
	\label{convo-est-starting-pnt-real}
	\begin{split}
		 & \langle \xi \rangle^s \langle \tau - \xi^{2} \rangle^{-a} | \wh{uv}\left( 
		\xi, \tau \right) |
		\\
		& \le \langle |\tau| - \xi^{2} \rangle^{-a}
    \int_{\xi_1} \int_{\tau_{1}} \frac{\langle \xi \rangle^{s}}{\langle \xi_1 \rangle^s
    \langle \xi - \xi_1 \rangle^s} 
		\times \frac{c_f(\xi_1, \tau_1)}{\langle |\tau_1| - \xi_1^{2} \rangle ^{b}}
		\\
		& \times
		\frac{c_g(\xi - \xi_1, \tau - \tau_1 )}{\langle |\tau - \tau_1| - (\xi - \xi_1)^{2}
    \rangle^{b}}\ d \tau_1 d \xi_{1}.
	\end{split}
\end{equation}
%
%
Hence, 
%
%
\begin{equation}
  \label{pre-fubini-int-form-real}
	\begin{split}
    |\text{lhs of} \ \cref{duality-est-real}|
    & \lesssim \int_{\xi} \int_{\tau}   
    \int_{\xi_{1}}  \int_{\tau_{1}} \phi(\xi, \tau)
    c_f(\xi_1, \tau_1)
		c_g(\xi - \xi_1, \tau - \tau_1 )
		\\
    & \times \frac{\langle \xi \rangle ^{s}}{\langle \xi_{1} \rangle ^{s} \langle
    \xi-\xi_{1} \rangle ^{s}} \times \frac{1}{\langle \tau - \xi^{2} \rangle^{a}
\langle |\tau| - \xi^{2} \rangle
    ^{b}\langle |\tau_{1}|-\xi_{1}^{2} \rangle ^{-b}\langle | \tau|-\xi_{2}^{2}
    \rangle ^{b}} d \tau_1 d \xi_{1} d \tau d \xi.
	\end{split}
\end{equation}
%
%
%
Now consider the family $\{A_{j}\}_{1}^{k}, A_{j} \subset \rr^{4}$ with
$$\bigcup_{1}^{k} A_{j}= \rr^{4}.$$ 
As in the periodic case, a combination of Cauchy-Schwartz, Fubini, and
H{\"o}lder to estimate the right hand side of \cref{pre-fubini-int-form-real}
reduces the proof of \cref{prop:bilin-est-real} to showing that
%From \cref{integral-bound-1st-form},
%\cref{integral-bound-2nd-form}, and our preceding argumentation,
%we see that the proof of \cref{prop:bilin-est-real} reduces to showing that
%either 
%
%
%
%
\begin{equation}
  \label{key-sup-estimate-real}
  \begin{split}
     \|   \frac{\langle \xi
     \rangle ^{2s}}{| \tau | - \xi^{2} \rangle ^{2a}}\int_{\xi_{1}}
     \int_{\tau_{1}} \frac{\chi_{A_{j}}}{\langle \xi_{1} \rangle ^{2s} \langle
\xi-\xi_{1} \rangle ^{2s}
\langle | \tau_{1}|-\xi_{1}^{2} \rangle^{2b}  \langle  |\tau -
    \tau_{1} | -(\xi - \xi_{1})^{2}
    \rangle^{2b}} d \tau_1 d \xi_{1} \|_{L^\infty_{\xi, \tau}} < \infty.
  \end{split}
\end{equation}
%
or
%%
\begin{equation}
\begin{split}
  & \| \frac{1}{\langle \xi_{1} \rangle ^{2s}
  \langle | \tau_{1} | - \xi_{1}^{2} \rangle
  ^{2a}} \int_{\xi} \int_{\tau} \frac{\langle \xi \rangle ^{2s}}{\langle
  \xi - \xi_{1}\rangle ^{2s}}  \times \frac{\chi_{A_{j}}}{\langle | \tau | - \xi^{2} \rangle^{2b}  \langle | \tau -
  \tau_{1} | - (\xi - \xi_{1}^{2}) \rangle^{2b}} d \tau d \xi
  \|_{L^{\infty}_{\xi_{1}, \tau_{1}}} < \infty
\end{split}
\end{equation}
or
\begin{equation}
\begin{split}
  \| \int_{\xi_{1}} \int_{\tau} \frac{\chi^{*}_{A}
    \langle \xi \rangle ^{2s}
  }{\langle \xi_{1} \rangle^{2s} \langle | \tau | - \xi^{2}
    \rangle ^{2a}  \langle
\xi-\xi_{1} \rangle ^{2s}  \langle | \tau - \lambda|-\xi_{1}^{2}
\rangle^{2b} \langle  | \lambda | -(\xi - \xi_{1})^{2}
\rangle^{2b}} d \tau d \xi_{1} \|_{L^{\infty}_{\xi, \lambda}} < \infty
\end{split}
\end{equation}
%
%
for each $j \in \left\{0,1,\dots,k \right\}$. 
By the triangle inequality and the fact that 
%
%
\begin{equation*}
\begin{split}
& | \tau | =
\begin{cases}
  - \tau, \quad & \tau < 0, 
\\
\tau, \quad & \tau > 0
\end{cases}
\end{split}
\end{equation*}
%
%
it follows that the proof of \cref{prop:bilinear-est} reduces to showing that
for any $j$, either 
%
%
\begin{equation}
  \label{sup-est-gen-1}
  \begin{split}
    \| \frac{\langle \xi
    \rangle ^{2s}}{\langle \sigma \rangle ^{2a}}
    \int_{\xi_{1}} \int_{\tau_{1}} \frac{\chi_{A_{j}}}{\langle \xi_{1} \rangle ^{2s} \langle \xi-\xi_{1} \rangle ^{2s} 
    \langle \sigma_{1} \rangle^{2b} \langle  \sigma_{2} \rangle^{2b}}
    d \tau_1 d \xi_{1} \|_{L^{\infty}_{\xi, \tau}} < \infty
  \end{split}
\end{equation}
%
%
or 
\begin{equation}
  \label{sup-est-gen-2}
\begin{split}
  & \| \frac{1}{\langle \xi_{1} \rangle ^{2s}
  \langle \sigma_{1} \rangle
  ^{2a}} \int_{\xi} \int_{\tau} \frac{\langle \xi \rangle ^{2s}}{\langle
  \xi - \xi_{1}\rangle ^{2s}}  \times \frac{\chi_{A_{j}}}{\langle
  \sigma \rangle^{2b}  \langle \sigma_{2} \rangle^{2b}} d \tau d \xi
  \|_{L^{\infty}_{\xi_{1}, \tau_{1}}} < \infty
\end{split}
\end{equation}
%
or
\begin{equation}
  \label{sup-est-gen-3}
\begin{split}
  \| \int_{\xi_{1}}  \int_{\tau} \frac{\chi^{*}_{A_{j}}
    \langle \xi \rangle ^{2s}
  }{\langle \xi_{1} \rangle^{2s} \langle
    \xi-\xi_{1} \rangle ^{2s} \langle \sigma^{*}  
    \rangle ^{2a}
    \langle \sigma_{1}^{*} \rangle^{2b}
    \langle  \sigma_{2}^{*} \rangle^{2b}} d \tau d \xi_{1}  \|_{L^{\infty}_{\xi, \lambda}}
    < \infty
\end{split}
\end{equation}
%
for the following cases.
\begin{enumerate}[(I)]
    \item $ \sigma=\tau+\xi^2,\quad \sigma_1=\tau_1+\xi_1^2,\quad \sigma_2=\tau -
      \tau_1+(\xi - \xi_1)^2$,
\label{it-real-1}
    \item $ \sigma=\tau-\xi^2,\quad \sigma_1=\tau_1-\xi_1^2,\quad \sigma_2=\tau - \tau_1+(\xi - \xi_1)^2$,
\label{it-real-2}
    \item  $\sigma=\tau+\xi^2,\quad \sigma_1=\tau_1-\xi_1^2,\quad \sigma_2=\tau - \tau_1+(\xi - \xi_1)^2$,
      \label{it-real-3}
    \item $\sigma=\tau-\xi^2,\quad \sigma_1=\tau_1+\xi_1^2,\quad \sigma_2=\tau - \tau_1-(\xi - \xi_1)^2$,
\label{it-real-4}
    \item $\sigma=\tau+\xi^2,\quad \sigma_1=\tau_1+\xi_1^2,\quad \sigma_2=\tau - \tau_1-(\xi - \xi_1)^2$,
\label{it-real-5}
    \item $\sigma=\tau-\xi^2,\quad \sigma_1=\tau_1-\xi_1^2,\quad \sigma_2=\tau - \tau_1-(\xi - \xi_1)^2$.
\label{it-real-6}
\end{enumerate}
%
for \cref{sup-est-gen-1} and \cref{sup-est-gen-2}, and the analogous cases
%
\begin{enumerate}[(I)]
\item $ \sigma^{*}=\tau+\xi^2,\quad \sigma^{*}_1=\tau - \lambda+\xi_1^2,\quad
  \sigma^{*}_2=\lambda+(\xi - \xi_1)^2$, \label{it-1-star-real} \item $
  \sigma^{*}=\tau-\xi^2,\quad \sigma^{*}_1=\tau - \lambda-\xi_{1}^2,\quad
  \sigma^{*}_2=\lambda+(\xi - \xi_1)^2$, \label{it-2-star-real} \item
  $\sigma^{*}=\tau+\xi^2,\quad \sigma^{*}_1=\tau - \lambda-\xi_1^2,\quad
  \sigma^{*}_2=\lambda+(\xi - \xi_1)^2$, \label{it-3-star-real} \item
  $\sigma^{*}=\tau-\xi^2,\quad \sigma^{*}_1=\tau - \lambda+\xi_1^2,\quad
  \sigma^{*}_2=\lambda-(\xi - \xi_1)^2$, \label{it-4-star-real} \item
  $\sigma^{*}=\tau+\xi^2,\quad \sigma^{*}_1=\tau - \lambda+\xi_1^2,\quad
  \sigma^{*}_2=\lambda-(\xi - \xi_1)^2$, \label{it-5-star-real} \item
  $\sigma^{*}=\tau-\xi^2,\quad \sigma^{*}_1=\tau - \lambda-\xi_1^2,\quad
  \sigma^{*}_2= \lambda-(\xi - \xi_1)^2$.  \label{it-6-star-real}
  \end{enumerate}
for \cref{sup-est-gen-3}.
%
%
\begin{framed}
\begin{remark}
Note that the cases $\sigma=\tau+\xi^2,\quad \sigma_1=\tau_1-\xi_1^2,\quad
\sigma_2=\tau - \tau_1-(\xi - \xi_1)^2$ and $\sigma=\tau-\xi^2,\quad
\sigma_1=\tau_1+\xi_1^2,\quad \sigma_2=\tau - \tau_1+(\xi - \xi_1)^2$ cannot occur, since
$\tau_1< 0, \tau-\tau_1< 0$ implies $\tau<0$ and $\tau_1\geq 0, \tau-\tau_1\geq
0$ implies $\tau\geq 0$. An analogous argument holds for $\sigma^{*},
\sigma_{1}^{*}$ and $\sigma_{2}^{*}$.
\end{remark}
\end{framed}
%
%
Observe that the transformation $(\xi, \tau, \xi_{1}, \tau_{1}) \mapsto -(\xi, \tau,
\xi_{1}, \tau_{1})$ reduces \cref{it-real-3} to \cref{it-real-4}, \cref{it-real-2} to
\cref{it-real-5}, and \cref{it-real-1} to \cref{it-real-6}. Furthermore, the change of
variables $\tau_{2} = \tau - \tau_{1}, \xi_{2} = \xi - \xi_{1}$, and the
transformation $(\xi, \tau, \xi_{2}, \tau_{2}) \mapsto - (\xi, \tau, \xi_{2},
\tau_{2})$ reduces \cref{it-real-5} to \cref{it-real-4}. Since $L^{2}$ is invariant
under change of variables and reflections, we may without loss of generality
restrict our attention to cases \cref{it-real-4} and \cref{it-real-6}.
 \subsubsection{Case \cref{it-real-6}} 
\label{ssec:case-it-real-6}
We partition $\rr^{4}$ into the following three sets 
%
%
\begin{equation*}
\begin{split}
  & A_{1} = \left\{(\xi, \tau, \xi_{1}, \tau_{1}) \subset \rr^{4}: |
  \xi_{1} \le 1 | \ \text{or} \ | \xi - \xi_{1} | \le 1 \right\},
  \\
  & A_{2} = 
  \begin{Bmatrix}
    | \xi_{1} \ge 1 \ \text{and} \ | \xi - \xi_{1} | \ge 1,
    \\
    \langle \tau_{1} - \xi_{1}^{2} \rangle  \le \langle \tau -
  \xi^{2} \rangle
\end{Bmatrix}
  \\
  & A_{3} = 
  \begin{Bmatrix}
    | \xi_{1} | \ge 1 \ \text{and} \ | \xi - \xi_{1} | \ge 1,
    \\
    \langle \tau - \xi^{2} \rangle  \le \langle \tau_{1} - \xi_{1}^{2} \rangle 
  \end{Bmatrix}
\end{split}
\end{equation*}
%
%
%
%
%
and first seek to bound
%
%
\begin{equation}
  \label{case-1-region-1}
  \begin{split}
    \frac{\langle \xi
    \rangle ^{2s}}{\langle \tau - \xi^{2} \rangle ^{2a}}
    \int_{\xi_{1}} \int_{\tau_{1}} \frac{\chi_{A_{1}}}{\langle \xi_{1} \rangle ^{2s} \langle \xi-\xi_{1} \rangle ^{2s} 
    \langle \tau_{1} - \xi_{1}^{2} \rangle^{2b} \langle  \tau - \tau_{1} -
    (\xi - \xi_{1})^{2} \rangle^{2b}}
    d \tau_1 d \xi_{1}.
  \end{split}
\end{equation}
%
%
\begin{framed}
\begin{remark}
Farah splits a bit differently. He considers a series of disjoint smaller sets
whose union is $A_{1}$, and estimates in each region separately.  This is
unnecessary, and complicates matters. The splitting we use here is motivated by
work in Kenig, Ponce, and Vega~\cite{Kenig:1996aa}
\label{rem:farah-dif-splitting-organization}.
\end{remark}
\end{framed}
%
In region $A_{1}$, we note that if $| \xi_{1} | \le 1$
%
%
%
%
\begin{equation*}
\begin{split}
  (1 + | \xi_{1} |)(1 + | \xi - \xi_{1} |)
  & \le (1 + | \xi_{1} |)(1 + | \xi | + \xi_{1})
  \\
  & \le 2 (2 + | \xi |)
  \\
  & \le 4 (1 + | \xi |).
\end{split}
\end{equation*}
%
%
If $| \xi - \xi_{1} |\le 1$, then
%
%
\begin{equation*}
\begin{split}
  (1 + | \xi_{1} |)(1 + | \xi - \xi_{1} |)
  & \le 2 (1 + | \xi_{1} |)
  \\
  & \le 2 (1 + | \xi - \xi_{1} | + | \xi |)
  \\
  & \le 2(2 + | \xi |)
  \\
  & \le 4 (1 + | \xi |).
\end{split}
\end{equation*}
%
%
Hence, in region $A_{1}$, we have the estimate
%
%
\begin{equation}
\begin{split}
  \langle \xi_{1} \rangle \langle \xi - \xi_{1} \rangle  \le 4 \langle \xi \rangle 
\end{split}
\label{splitting-estimate}
\end{equation}
%
%
which we use to bound \cref{case-1-region-1} by
%
%
%
%
\begin{equation*}
\begin{split}
    \frac{1}{\langle \tau - \xi^{2} \rangle ^{2a}}
    \int_{\xi_{1}} \int_{\tau_{1}} \frac{\chi_{A_{1}}}{\langle \tau_{1} -
    \xi_{1}^{2} \rangle^{2b} \langle  \tau - \tau_{1} - (\xi - \xi_{1})^{2}
    \rangle^{2b}} d \tau_1 d \xi_{1}.
\end{split}
\end{equation*}
%
%
Applying \cref{lem:calc}, this in turn is bounded by
%
%
\begin{equation}
  \label{uniform-bound-region-1}
\begin{split}
  & \frac{c_{1}}{\langle \tau - \xi^{2} \rangle^{2a}} \int_{\xi_{1}}
  \frac{\chi_{A_{1}}}{\langle \tau - \xi^{2} + 2 \xi \xi_{1} - 2
  \xi_{1}^{2} \rangle^{2b}} d \xi_{1}
  \\
  & \le c_{1} \int_{\xi_{1}}
  \frac{\chi_{A_{1}}}{\langle \tau - \xi^{2} + 2 \xi \xi_{1} - 2
  \xi_{1}^{2} \rangle^{2b}} d \xi_{1}, \quad a > 0
  \\
& < c, \quad b > 1/4
\end{split}
\end{equation}
%
%
where the last line follows from a corollary to \cref{lem:calc}.
%
%
%%%%%%%%%%%%%%%%%%%%%%%%%%%%%%%%%%%%%%%%%%%%%%%%%%%%%
%
%
%                Corollary to calculus lemma
%
%
%%%%%%%%%%%%%%%%%%%%%%%%%%%%%%%%%%%%%%%%%%%%%%%%%%%%%
%
%
\begin{corollary}[Lemma 3.1 in~\cite{Farah:2009uq}]
  For $a_{0}, a_{1}, a_{2} \in \rr$ with $a_{2} \neq 0$ and $q > 1/2$
  %
  %
  \begin{equation*}
  \begin{split}
    \int_{\rr} \frac{1}{\langle a_{0} + a_{1}x + a_{2}x^{2} \rangle ^{q}} dx \le c
  \end{split}
  \end{equation*}
  %
  where $c$ is a constant independent of the choice of $a_{0}, a_{1}$, and $a_{2}$.
  %
\label{cor:integral-bound}
\end{corollary}
Note that for \cref{uniform-bound-region-1}, the
value of $c$ does not depend on the choice of $\tau$ or $\xi$. 
Next we seek to bound
\begin{equation}
  \label{case-1-region-2}
  \begin{split}
    \frac{\langle \xi
    \rangle ^{2s}}{\langle \tau - \xi^{2} \rangle ^{2a}}
    \int_{\xi_{1}} \int_{\tau_{1}} \frac{\chi_{A_{2}}}{\langle \xi_{1} \rangle ^{2s} \langle \xi-\xi_{1} \rangle ^{2s} 
    \langle \tau_{1} - \xi_{1}^{2} \rangle^{2b} \langle  \tau - \tau_{1} -
    (\xi - \xi_{1})^{2} \rangle^{2b}}
    d \tau_1 d \xi_{1}.
  \end{split}
\end{equation}
Due to the symmetry of the convolution, we assume without loss of generality that
%
%
\begin{equation*}
\begin{split}
  \langle \tau - \tau_{1} - (\xi - \xi_{1})^{2} \rangle \le \langle
  \tau_{1} - \xi_{1}^{2}\rangle.
\end{split}
\end{equation*}
%
%
Then in region $A_{2}$, we have the estimate
%
%
\begin{equation}
\begin{split}
  | \tau - \xi^{2} |
  & \ge \frac{1}{3}\left[ | \tau_{1} - \xi_{1}^{2} | + | \tau -
  \tau_{1} - (\xi - \xi_{1})^{2}
  | + | \tau - \xi^{2} | \right]
  \\
  & \ge \frac{1}{3} | - \xi_{1}^{2} - (\xi - \xi_{1})^{2} + \xi^{2} |
  \\
  & = \frac{2}{3} | \xi_{1} | | \xi - \xi_{1} |
  \\
  & \gtrsim | \xi_{1} | \qquad (| \xi - \xi_{1} |\chi_{A_{2}} \ge 1).
\end{split}
\label{region-2-smoothing}
\end{equation}
%
%
Furthermore, from the inequality
%
%
\begin{equation*}
\begin{split}
  \langle \xi \rangle  \le \langle \xi_{1} \rangle \langle \xi - \xi_{1} \rangle 
\end{split}
\end{equation*}
%
we obtain
%
%
\begin{equation}
  \label{growth-term}
\begin{split}
  \frac{\langle \xi \rangle ^{2s}}{\langle \xi_{1} \rangle ^{2s} \langle \xi -
  \xi_{1} \rangle ^{2s}} \le \langle \xi_{1} \rangle ^{\gamma(s)},
  \quad 
  \gamma(s) = 
  \begin{cases} 0, \quad & s \ge 0
    \\
    4|s|, \quad & s < 0.
  \end{cases}
\end{split}
\end{equation}
Applying \cref{region-2-smoothing} and \cref{growth-term}, we obtain 
%
%
\begin{equation*}
\begin{split}
  \cref{case-1-region-2}
  & \lesssim
  \int_{\xi_{1}} \int_{\tau_{1}}  \frac{\chi_{A_{2}} \langle \xi_{1} \rangle
  ^{\gamma(s) -2a}}
  {\langle \tau_{1} - \xi_{1}^{2} \rangle^{2b} \langle  \tau - \tau_{1} -
    (\xi - \xi_{1})^{2} \rangle^{2b}}
    d \tau_1 d \xi_{1}
    \\
    & \le \int_{\xi_{1}} \int_{\tau_{1}}  \frac{\chi_{A_{2}}}
    {\langle \tau_{1} - \xi_{1}^{2} \rangle^{2b} \langle  \tau - \tau_{1} -
    (\xi - \xi_{1})^{2} \rangle^{2b}}
    d \tau_1 d \xi_{1}, \quad s \ge -a/2.
  \end{split}
\end{equation*}
%
%
By \cref{lem:calc} and \cref{cor:integral-bound}, we bound the right hand
side by
%
%
\begin{equation*}
\begin{split}
  & c_{1} \int_{\xi_{1}}  \frac{1}{\langle \tau - \xi^{2} + 2 \xi \xi_{1} - 2
  \xi_{1}^{2} \rangle^{2b}}d \xi_{1}
  \\
  & \le  c, \quad b > 1/2
\end{split}
\end{equation*}
%
%
where $c$ is independent of the value of $\tau$ and $\xi$. 
%
Lastly, we seek to bound
\begin{equation*}
\begin{split}
  &  \frac{1}{\langle \xi_{1} \rangle ^{2s}
  \langle \tau_{1} - \xi_{1}^{2}  \rangle
  ^{2a}} \int_{\xi} \int_{\tau} \frac{\langle \xi \rangle ^{2s}}{\langle
  \xi - \xi_{1}\rangle ^{2s}}  \times \frac{\chi_{A_{3}}}{\langle
  \tau - \xi^{2} \rangle ^{2a} \langle \tau - \tau_{1} - (\xi -
  \xi_{1})^{2} \rangle^{2b}} d \tau d \xi.
\end{split}
\end{equation*}
Applying \cref{lem:calc}, we bound this by
%
%
\begin{equation}
  \label{pre-A-3-bound}
\begin{split}
  &  \frac{1}{\langle \xi_{1} \rangle ^{2s}
  \langle \tau_{1} - \xi_{1}^{2}  \rangle
  ^{2a}} \int_{\xi} \frac{\langle \xi \rangle ^{2s}}{\langle
  \xi - \xi_{1}\rangle ^{2s}}  \times \frac{\chi_{A_{3}}}{\langle
  \tau_{1} + 2 \xi \xi_{1} - \xi_{1}^{2} \rangle^{2b}} d \xi.
\end{split}
\end{equation}
%
Next, noting that in region $A_{3}$,
%
%
\begin{equation*}
\begin{split}
  | \tau_{1} -\xi_{1}^{2} |
  & \ge \frac{1}{3} \left[ | \tau_{1} - \xi_{1}^{2} | + | \tau - \tau_{1} -
  (\xi - \xi_{1})^{2} | + | \tau - \xi^{2} | \right]
  \\
  & \ge \frac{1}{3} | - \xi_{1}^{2} - (\xi - \xi_{1})^{2} + \xi^{2} |
  \\
  & = \frac{2}{3} | \xi_{1} | | \xi - \xi_{1} |
  \\
  & \gtrsim | \xi_{1} | \quad (| \xi - \xi_{1} |\chi_{A_{2}} \ge 1)
\end{split}
\end{equation*}
%
%
and recalling \cref{growth-term}, we see that \cref{pre-A-3-bound} is bounded by
\begin{equation*}
\begin{split}
  &  \int_{\xi} \frac{\chi_{A_{2,2}} \langle \xi_{1} \rangle ^{\gamma(s) -2a}}{\langle
  \tau_{1} + 2 \xi \xi_{1} - \xi_{1}^{2} \rangle^{2b}} d \xi
\end{split}
\end{equation*}
%
which by the change of variable
%
%
\begin{equation*}
\begin{split}
  & \eta = \tau_{1} - \xi_{1}^{2} + 2 \xi \xi_{1},
  \\
  & d \eta = 2 \xi_{1} d \xi
\end{split}
\end{equation*}
%
%
is equal to
%
%
\begin{equation*}
\begin{split}
  & \frac{1}{2} \langle \xi_{1} \rangle ^{\gamma(s)-2a}  \int_{\eta} 
  \frac{1}{| \xi_{1} |\langle \eta \rangle ^{2a}}d \eta
  \\
  & = \langle \xi_{1} \rangle ^{\gamma(s)-2a-1} \int_{0}^{\infty} \frac{1}{(1 + \eta
  )^{2a}}d \eta
  \\
  & \lesssim 1, \qquad s \ge -\frac{2a+1}{4}.
\end{split}
\end{equation*}
%
%
\subsubsection{Case \cref{it-real-4}} 
\label{ssec:case-it-real-4}
We partition $\rr^{4}$ into three sets 
%
%
\begin{equation*}
\begin{split}
  & A_{1} = \left\{(\xi, \tau, \xi_{1}, \tau_{1}) \subset \rr^{4}: |
  \xi_{1} \le 1 | \right \} \\
  & A_{2} = \left\{(\xi, \tau, \xi_{1}, \tau_{1}) \subset \rr^{4}:|
  \xi_{1} | \ge 1 \ \text{and} \ | \xi| \le 1 \right \}
  \\
  & A_{3} = \left\{(\xi, \tau, \xi_{1}, \tau_{1}) \subset \rr^{4}:|
  \xi_{1} | \ge 1 \ \text{and} \ | \xi| \ge 1 \right \}
  \end{split}
\end{equation*}
%
%
and first seek to bound
%
%
\begin{equation}
  \label{case-2-region-1}
  \begin{split}
    \frac{\langle \xi
    \rangle ^{2s}}{\langle \tau - \xi^{2} \rangle ^{2a}}
    \int_{\xi_{1}} \int_{\tau_{1}} \frac{\chi_{A_{1}}}{\langle \xi_{1} \rangle ^{2s} \langle \xi-\xi_{1} \rangle ^{2s} 
    \langle \tau_{1} + \xi_{1}^{2} \rangle^{2b} \langle  \tau - \tau_{1} -
    (\xi - \xi_{1})^{2} \rangle^{2b}}
    d \tau_1 d \xi_{1}.
  \end{split}
\end{equation}
By inequality \cref{splitting-estimate}, this is bounded by
%
%
\begin{equation*}
\begin{split}
  \frac{1}{\langle \tau - \xi^{2} \rangle ^{2a}} \int_{\xi_{1}}
  \int_{\tau_{1}}
  \frac{\chi_{A_{1}}}{\langle \tau_{1} + \xi_{1}^{2} \rangle^{2b}  \langle \tau
  - \tau_{1} - ( \xi - \xi_{1})^{2}\rangle^{2b}} d \tau_{1} d \xi_{1}.
\end{split}
\end{equation*}
%
%
Applying \cref{lem:calc} then gives the bound
%
%
\begin{equation}
\begin{split}
  \frac{1}{\langle \tau - \xi^{2} \rangle ^{2a}} \int_{\xi_{1}}
  \frac{\chi_{A_{1}}}{\langle \tau - \xi^{2} + 2 \xi \xi_{1} \rangle^{2b}} d
  \xi_{1}.
\end{split}
\label{region-1-case-2-pre-est}
\end{equation}
%
%
Since $| \xi_{1} | \le 2$ in $A_{1}$, this is is bounded by
%
%
\begin{equation*}
\begin{split}
  \int_{| \xi_{1} | \le 2} d \xi_{1} \simeq 1.
\end{split}
\end{equation*}
%
%
Next, we seek to bound
\begin{equation*}
\begin{split}
  &  \frac{1}{\langle \xi_{1} \rangle ^{2s}
  \langle \tau_{1} + \xi_{1}^{2}  \rangle
  ^{2a}} \int_{\xi} \int_{\tau} \frac{\langle \xi \rangle ^{2s}}{\langle
  \xi - \xi_{1}\rangle ^{2s}}  \times \frac{\chi_{A_{2}}}{\langle
  \tau - \xi^{2} \rangle ^{2a} \langle \tau - \tau_{1} - (\xi -
  \xi_{1})^{2} \rangle^{2b}} d \tau d \xi.
\end{split}
\end{equation*}
Applying \cref{growth-term}, this is bounded by
\begin{equation*}
\begin{split}
  &  \langle \xi_{1} \rangle ^{\gamma(s)}
   \int_{\xi} \int_{\tau} \frac{\chi_{A_{2}}}{\langle
  \tau - \xi^{2} \rangle ^{2a} \langle \tau - \tau_{1} - (\xi -
  \xi_{1})^{2} \rangle^{2b}} d \tau d \xi.
\end{split}
\end{equation*}
Applying \cref{lem:calc} then gives the bound
\begin{equation*}
\begin{split}
  & \langle \xi_{1} \rangle ^{\gamma(s)} \int_{\xi} \frac{\chi_{A_{2}}}{\langle
  \tau_{1} + 2 \xi \xi_{1} - \xi_{1}^{2} \rangle^{2b}} d \xi
\end{split}
\end{equation*}
which by the change of variable
%
%
\begin{equation*}
\begin{split}
  & \eta = \tau_{1} - \xi_{1}^{2} + 2 \xi \xi_{1},
  \\
  & d \eta = 2 \xi_{1} d \xi
\end{split}
\end{equation*}
%
%
is equal to
%
%
\begin{equation*}
\begin{split}
  & \frac{\langle \xi_{1} \rangle^{\gamma(s)}}{2}  \int_{\eta} 
  \frac{1}{| \xi_{1} |\langle \eta \rangle ^{2a}}d \eta
  \\
  & = \langle \xi_{1} \rangle ^{\gamma(s) -1} \int_{0}^{\infty} \frac{1}{(1 + \eta
  )^{2a}}d \eta
  \\
  & \lesssim 1, \qquad s \ge -1/4.
\end{split}
\end{equation*}
%
%
We now split $A_{3}$ into the sets
%
%
%
\begin{align*}
A_{3,1}&=\{(\xi, \xi_1, \tau, \tau_1)\in A_2:
|\tau-\tau_1-(\xi-\xi_1)^2|, |\tau_1+\xi_1^2| \le |\tau-\xi^2|\},\\
A_{3,2}&=\{(\xi, \xi_1, \tau, \tau_1)\in A_2:
|\tau-\tau_1-(\xi-\xi_1)^2|, |\tau-\xi^2| \le |\tau_1+\xi_1^2|\},\\
A_{3,3}&=\{(\xi, \xi_1, \tau, \tau_1)\in A_2: |\tau_{1}+\xi_{1}^2|, | \tau - \xi^{2} | \le |  \tau - \tau_{1} -
(\xi - \xi_{1})^{2} |\}
\end{align*} 
and first seek to estimate
%
%
\begin{equation}
  \label{case-2-region-2}
  \begin{split}
    \frac{\langle \xi
    \rangle ^{2s}}{\langle \tau - \xi^{2} \rangle ^{2a}}
    \int_{\xi_{1}} \int_{\tau_{1}} \frac{\chi_{A_{3,1}}}{\langle \xi_{1} \rangle ^{2s} \langle \xi-\xi_{1} \rangle ^{2s} 
    \langle \tau_{1} + \xi_{1}^{2} \rangle^{2b} \langle  \tau - \tau_{1} -
    (\xi - \xi_{1})^{2} \rangle^{2b}}
    d \tau_1 d \xi_{1}.
  \end{split}
\end{equation}
Notice that, unlike case \cref{it-6}, we may \emph{not} assume without loss of generality
that $|\tau - \tau_{1} - (\xi - \xi_1)^{2} | \le | \tau_{1} + \xi^{2} | $.
Applying \cref{lem:calc} to \cref{case-2-region-2} gives the bound
%
%
\begin{equation*}
  \begin{split}
    \frac{\langle \xi
    \rangle ^{2s}}{\langle \tau - \xi^{2} \rangle ^{2a}}
    \int_{\xi_{1}} \frac{\chi_{A_{3,1}}}{\langle \xi_{1} \rangle ^{2s} \langle \xi-\xi_{1} \rangle ^{2s} 
    \langle \tau - \xi^{2} +2 \xi \xi_{1} \rangle^{2b}} d \xi_{1}.
  \end{split}
\end{equation*}
%
%
%
But in $A_{3,1}$
%
%
\begin{equation}
\begin{split}
  | \tau - \xi^{2} |
  & \ge \frac{1}{3} \left[ | \tau_{1} + \xi_{1}^{2} | + | \tau - \tau_{1} -
  (\xi - \xi_{1})^{2} | + | \tau - \xi^{2} | \right]
  \\
  & \ge \frac{1}{3} | \xi_{1}^{2} - (\xi - \xi_{1})^{2} + \xi^{2} |
  \\
  & = \frac{2}{3}| \xi | | \xi_{1} |
  \\
  & \gtrsim  | \xi_{1} |.
\end{split}
\label{case-2-region-A2-key-est}
\end{equation}
%
%
Hence, applying \cref{growth-term} and \cref{case-2-region-A2-key-est}, we obtain
%
%
%
%
\begin{equation*}
\begin{split}
  \cref{case-2-region-2}
  & \lesssim 
  \int_{\xi_{1}} \int_{\tau_{1}} \frac{\chi_{A_{3,1}} \langle \xi_{1}
  \rangle^{\gamma(s) -2a}}
  {\langle \tau_{1} + \xi_{1}^{2} \rangle^{2b} \langle  \tau - \tau_{1} -
    (\xi - \xi_{1})^{2} \rangle^{2b}}
    d \tau_1 d \xi_{1}
    \\
    & \le \int_{\xi_{1}} \int_{\tau_{1}} \frac{\chi_{A_{3,1}}}
    {\langle \tau_{1} + \xi_{1}^{2} \rangle^{2b} \langle  \tau - \tau_{1} -
    (\xi - \xi_{1})^{2} \rangle^{2b}}
    d \tau_1 d \xi_{1}, \quad s \ge -a/2.
\end{split}
\end{equation*}
By \cref{lem:calc} and \cref{cor:integral-bound}, we bound the right hand
side by
%
%
\begin{equation*}
\begin{split}
  & c_{1} \int_{\xi_{1}}  \frac{1}{\langle \tau - \xi^{2} + 2 \xi \xi_{1} 
  \rangle^{2b}}d \xi_{1}
  \\
  & \le c_{1} \int_{\xi_{1}}  \frac{1}{\langle (\tau - \xi^{2} + 2 \xi
  \xi_{1})^{2} \rangle^{b}}d \xi_{1}
  \\
  & \le  c, \qquad b > 1/2
\end{split}
\end{equation*}
where $c$ is independent of the value of $\tau$ and $\xi$. 
Next, we seek to estimate
\begin{equation*}
\begin{split}
  &  \frac{1}{\langle \xi_{1} \rangle ^{2s}
  \langle \tau_{1} + \xi_{1}^{2}  \rangle
  ^{2a}} \int_{\xi} \int_{\tau} \frac{\langle \xi \rangle ^{2s}}{\langle
  \xi - \xi_{1}\rangle ^{2s}}  \times \frac{\chi_{A_{3,2}}}{\langle
  \tau - \xi^{2} \rangle ^{2a} \langle \tau - \tau_{1} - (\xi -
  \xi_{1})^{2} \rangle^{2b}} d \tau d \xi.
\end{split}
\end{equation*}
Applying \cref{lem:calc}, we bound this by
%
%
\begin{equation}
  \label{pre-A-2-2}
\begin{split}
  &  \frac{c}{\langle \xi_{1} \rangle ^{2s}
  \langle \tau_{1} + \xi_{1}^{2}  \rangle
  ^{2a}} \int_{\xi} \frac{\langle \xi \rangle ^{2s}}{\langle
  \xi - \xi_{1}\rangle ^{2s}}  \times \frac{\chi_{A_{3,2}}}{\langle
  \tau_{1} + 2 \xi \xi_{1} - \xi_{1}^{2} \rangle^{2b}} d \xi.
\end{split}
\end{equation}
%
Next, noting that in region $A_{3,2}$,
%
%
\begin{equation}
\begin{split}
  | \tau_{1} + \xi_{1}^{2} |
  & \ge \frac{1}{3} \left[ | \tau_{1} + \xi_{1}^{2} | + | \tau - \tau_{1} -
  (\xi - \xi_{1})^{2} | + | \tau - \xi^{2} | \right]
  \\
  & \ge \frac{1}{3} | \xi_{1}^{2} - (\xi - \xi_{1})^{2} + \xi^{2} |
  \\
  & = \frac{2}{3}| \xi | | \xi_{1} |
  \\
  & \gtrsim  | \xi_{1} |
\end{split}
\label{case-2-region-A-2-2-key-est}
\end{equation}
%
%
and recalling \cref{growth-term}, we see that \cref{pre-A-2-2} is bounded by
\begin{equation*}
\begin{split}
  &  c
  \int_{\xi} \frac{\chi_{A_{3,2}} \langle \xi_{1} \rangle ^{\gamma(s) -2a}}{\langle
  \tau_{1} + 2 \xi \xi_{1} - \xi_{1}^{2} \rangle^{2b}} d \xi
\end{split}
\end{equation*}
which by the change of variable
%
%
\begin{equation*}
\begin{split}
  & \eta = \tau_{1} - \xi_{1}^{2} + 2 \xi \xi_{1},
  \\
  & d \eta = 2 \xi_{1} d \xi
\end{split}
\end{equation*}
%
%
is equal to
%
%
\begin{equation*}
\begin{split}
  & \frac{c \xi_{1}^{\gamma(s) -2a}}{2}  \int_{\eta} 
  \frac{1}{| \xi_{1} |\langle \eta \rangle ^{2a}}d \eta
  \\
  & = c\langle \xi_{1} \rangle ^{\gamma(s) -2a -1} \int_{0}^{\infty} \frac{1}{(1 + \eta
  )^{2a}}d \eta
  \\
  & \lesssim 1, \qquad s \ge -\frac{2a+1}{4}.
\end{split}
\end{equation*}
It remains to handle region $A_{3,3}$. It will be enough to bound
%
%
\begin{equation}
  \label{region-A-2-3-star-split}
\begin{split}
   \int_{\xi_{1}} \int_{\tau} \frac{\chi^{*}_{A_{3,3}}
    \langle \xi \rangle ^{2s}
  }{\langle \xi_{1} \rangle^{2s} \langle  \tau  - \xi^{2}
    \rangle ^{2a}  \langle
\xi-\xi_{1} \rangle ^{2s}  \langle  \tau - \lambda+\xi_{1}^{2}
\rangle^{2b} \langle   \lambda  -(\xi - \xi_{1})^{2}
\rangle^{2b}} d \tau d \xi_{1}.
\end{split}
\end{equation}
%
Due to the presence of $\chi^{*}_{A_{3,3}}$ in the integrand, we have the restriction
%
%
\begin{equation*}
\begin{split}
& |\tau - \lambda +\xi_{1}^2|, | \tau - \xi^{2} | \le |  \lambda -
(\xi - \xi_{1})^{2} | \ \text{and} \  |\xi| \ge 1, |\xi_1| \ge 1.
\end{split}
\end{equation*}
%
It follows that
\begin{equation}
  \label{smoothing-2-3-case-6}
\begin{split}
  | \lambda - (\xi - \xi_{1})^{2} |
  & \ge \frac{1}{3}\left[ | \tau - \lambda + \xi_{1}^{2} | + | \lambda - (\xi - \xi_{1})^{2}
  | + | \tau - \xi^{2} | \right]
  \\
  & \ge \frac{1}{3} |  \xi_{1}^{2} - (\xi - \xi_{1})^{2} + \xi^{2} |
  \\
  & = \frac{2}{3} | \xi_{1} | | \xi |
  \\
  & \gtrsim | \xi_{1} |.
\end{split}
\end{equation}
Hence, applying
\cref{growth-term}, \cref{lem:calc}, and
\cref{smoothing-2-3-case-6}, we bound \cref{region-A-2-3-star-split} by
%
%
\begin{equation*}
\begin{split}
   & \int_{\xi_{1}} \int_{\tau} \frac{\chi^{*}_{B_{3,3}} \langle
   \xi_{1} \rangle ^{\gamma(s) -2a}
  }{\langle  \tau  - \xi^{2}
    \rangle ^{2a}   \langle  \tau - \lambda+\xi_{1}^{2}
    \rangle^{2b}} d \tau d \xi_{1}
\\
& \lesssim  \int_{\xi_{1}} \frac{\chi_{B_{3,3}^{*}}}{\langle \xi_{1}^{2} +
\xi^{2} - \lambda \rangle^{2 a}} d \xi_{1}, \quad s \ge -a/2
\end{split}
\end{equation*}
which is bounded for $a > 1/4$ by \cref{cor:integral-bound}. This concludes the proof of \cref{prop:bilin-est-real}. \qquad \qedsymbol
%
%
\section{Ill-Posedness}
Our motivation will be the work of Bejanaru and Tao
\cite{Bejenaru-Tao-2006-Sharp-well-posedness-and-ill-posedness}. 
%
\begin{definition}
  For $f =f(x)= (f_{1}(x), f_{2}(x)), u = u(x,t), v = v(x,t)$ formally define 
%
%
\begin{equation*}
\begin{split}
  L(f)
  \doteq \frac{1}{2 \pi} \psi(t) \sum_{n \in \zz} e^{inx}
  \wh{f_{1}}(n) \frac{e^{in^{2}t} + e^{-in^{2}t}}{2} 
  & + \psi(t) \sum_{n \in \zz} e^{inx}
  \wh{f_{2}}(n)\frac{e^{in^{2}t} - e^{-in^{2}t}}{2 i n^{2}} 
\end{split}
\end{equation*}
%
%
and
%
%
\begin{equation*}
\begin{split}
N(u, v)
& \doteq \frac{1}{4 \pi i} \psi_{\delta}(t) \sum_{n \in \zz} e^{inx}
    \int_{0}^{t}[e^{in^{2}(t-t')}-e^{-in^{2}(t-t')}]
    \wh{uv}(n, t') dt'.
\end{split}
\end{equation*}
%
%
Then we let $A_{n}: H^{s} \times H^{s-2} \to X_{s,b}, \ n = 1, 2, \dots$ be the
recursively defined maps
%
%
\begin{equation*}
\begin{split}
  & A_{1}(f) \doteq L(f),
  \\
  & A_{n}(f) \doteq \sum_{j, k \in \mathbb{N}: j + k = n} N\left[
  A_{j}(f), A_{k}(f) \right], \quad n > 1.
\end{split}
\end{equation*}

\end{definition}
%
The key tool in our proof of ill-posedness is the following. 
%%
\begin{proposition}
  Let $f \mapsto u[f]$ be the solution map to the localized
  $B_{4}$ which maps a ball $B_{H^{s} \times H^{s-2}}$ in $H^{s} \times
  H^{s-2}$ to a ball $B_{X_{s,b}}$ in $X_{s,b}$. Suppose that the solution map
  is continuous from $B_{H^{s'} \times
  H^{s'-2}}$ to $(B_{X_{s,b}}, \| \|_{X_{s', b}})$ (i.e.\ the
  ball $B_{X_{s,b}}$ equipped with the $X_{s',b}$ topology),
  where $s' < s$. Then for  $n=1,2,\dots$, the operator $A_{n} :
  B_{H^{s'}} \times B_{H^{s'-2}} \to X_{s',b}$ is continuous
  from $B_{H^{s'} \times {H^{s'-2}}}$
  to $(B_{X_{s,b}}, \|  \|_{X_{s',b}})$.
  %
  %
  %
  %
\label{prop:ill-posedness}
\end{proposition}
%
%
%
%
\begin{framed}
%
%
\begin{remark}
  We defer the proof for later.  A sketch of the proof of a more general
  proposition can be found
  in~\cite{Bejenaru-Tao-2006-Sharp-well-posedness-and-ill-posedness}. The key
  requirement, heuristically, is that if the equation in question
  has a $k-linear$ non-linearity, then for the proposition to apply
  it must possess a $k-linear$ estimate on the non-linearity, in addition
  to a suitable bound on the linear term. Bejenaru and Tao refer to this as
  \emph{qualitative well-posedness}. For example, the $B_{4}$ ivp is qualitatively
  well-posed. That is, for given initial data $f \in H^{s} \times H^{s-2}$ and
  functions $u,v \in X_{s,b}$, one has the estimates
\begin{enumerate}
  \item{$\|L(f)\|_{X_{s,b}} \le C \| f \|_{H^s \times H^{s-2}}$}
  \item{$\| N(u, v) \|_{X_{s,b}} \le C \delta^{b - 1/2} \| u \|_{X_{s,b}} \| v \|_{X_{s,b}}$} 
\end{enumerate}
for some constant $C > 0$.
%%
%
%
\label{rem:qual-wp}
\end{remark}
%
%
\end{framed}
%
%
%
\subsection{Ill-Posedness for $s < -2$ on the Circle (Wrong)} 
\label{ssec:circle-ill-pos}
%
The proof will be via contradiction. Let \\ $f_{N} = (f_{N,1},
f_{N,2})$. We have the estimate
%
%
\begin{equation*}
\begin{split}
  & \| N(A_{1}(f_{N}), A_{1}(f_{N})) \|_{X_{s,b}}
  \\
  & \ge
  \| N[A_{1}(f_{N}), A_{1}(f_{N})] \|_{C([0, \delta],
  H^{s})} 
  \\
  & = \sup_{0 \le t \le \delta} \| \psi_{\delta}(t) (1 + | n |)^{s}
  \frac{1}{4 i \pi}
  \int_{0}^{t} \left( e^{in^{2}(t-t')} - e^{-in^{2}(t-t')} \right)
  \wh{[L(f_{N})]^{2}}(n, t') dt' \|_{\ell^{2}_{n}}
  \\
  & = \sup_{0 \le t \le \delta} \| (1 + | n |)^{s} \frac{1}{4 i \pi} 
  \int_{0}^{t} \sum_{n_{1}}
  \\
  & \left( e^{in^{2}(t-t')} - e^{-in^{2}(t-t')} \right)
  \wh{L(f_{N})}(n - n_{1}, t')\wh{L(f_{N})}(n_{1}, t') dt'
  \|_{\ell^{2}_{n}}
  \\
  & = \sup_{0 \le t \le \delta} \| (1 + | n |)^{s} \frac{1}{4 i \pi} 
  \int_{0}^{t} \sum_{n_{1}} \left( e^{in^{2}(t-t')} - e^{-in^{2}(t-t')} \right)
  [\psi(t')]^{2}
  \\
  & \times \frac{1}{2 \pi} \left[ \frac{\wh{f_{N,1}}(n - n_{1})\left( e^{i(n - n_{1})^{2}t'} +
  e^{-i(n - n_{1})^{2}t'} \right)}{2} + \frac{\wh{f_{N,2}}(n - n_{1})\left(
  e^{i(n - n_{1})^{2}t'} - e^{-i(n - n_{1})^{2}t'}
  \right)}{2i(n - n_{1})^{2}} \right]
  \\
  & \times \frac{1}{2 \pi} \left[ \frac{\wh{f_{N,1}}(n_{1})\left( e^{in_{1}^{2}t'} +
  e^{-in_{1}^{2}t'} \right)}{2} + \frac{\wh{f_{N,2}}(n_{1})\left(
  e^{in_{1}^{2}t'} - e^{-in_{1}^{2}t'}
  \right)}{2 i n_{1}^{2}} \right]
  dt' \|_{\ell^{2}_{n}}
\end{split}
\end{equation*}
%
Take $f_{N,2}(x) \equiv 0$. Then the above reduces to
%
%
\begin{equation*}
\begin{split}
  & \sup_{0 \le t \le \delta} \frac{1}{16 i \pi^{2}}\| (1 + | n |)^{s}
  \int_{0}^{t} \sum_{n_{1}} \left( e^{in^{2}(t-t')} - e^{-in^{2}(t-t')} \right)
  \\
  & \times \left[ \frac{\wh{f_{N,1}}(n - n_{1})\left( e^{i(n - n_{1})^{2}t'} +
  e^{-i(n - n_{1})^{2}t'} \right)}{2} \right ]
  \\
  & \times \left[ \frac{\wh{f_{N,1}}(n_{1})\left( e^{in_{1}^{2}t'} +
  e^{-in_{1}^{2}t'} \right)}{2}  \right]
  dt' \|_{\ell^{2}_{n}}
\end{split}
\end{equation*}
%
which we expand to obtain
\begin{equation*}
\begin{split}
& \sup_{0 \le t \le \delta} \frac{1}{8 \pi^{2}}\| (1 + | n |)^{s}
\int_{0}^{t} \sum_{n_{1}} \sin[n^{2}(t-t')] \cos[(n - n_{1})^{2}t']
\cos(n_{1}^{2}t') \wh{f_{N,1}}(n - n_{1})\wh{f_{N,1}}(n_{1}) dt'
\|_{\ell_{2}}.
\end{split}
\end{equation*}
Assume $N$ to be a positive integer. Then 
using the inequality $\| a_{n} \|_{\ell^{2}} \ge
|a_{2N}|$ and the fact that
%
%
\begin{equation*}
\begin{split}
  (1 + 2N)^{s} \ge 2^{s}(1 + N)^{s} \ge
  4^{s} N^{s}, \qquad s \le 0
\end{split}
\end{equation*}
%
%
we bound this below by 
%
%
\begin{equation}
  \label{yut}
\begin{split}
  & \frac{4^{s}}{8\pi^{2}} N^{s} \sup_{0 \le t \le \delta} 
  | \int_{0}^{t} \sum_{n_{1}} \sin[4N^{2}(t-t')] \cos[(2N -n_{1})^{2}t']
\cos(n_{1}^{2}t') \wh{f_{N,1}}(2N - n_{1})\wh{f_{N,1}}(n_{1}) dt' |.
\end{split}
\end{equation}
%
%
%
%
Let
%
%
\begin{equation}
  \label{ill-pos-ce}
\begin{split}
  \wh{f_{N,1}}(n) \doteq
  \begin{cases}
     rN^{-s-\ee},  \quad  & n = N
    \\
     0, \quad  & \text{otherwise}
  \end{cases}
\end{split}
\end{equation}
%
%
Then
%
%
\begin{equation*}
\begin{split}
  \| f_{N,1} \|_{H^{s}}
  & = rN^{-s -\ee} \left( \sum_{n} (1 + | n |)^{2s} |
  \chi_{\left\{0 \right\}} (n - N) |^{2} \right)^{1/2}
  \\
  & \le 2rN^{-\ee}.
\end{split}
\end{equation*}
%
%
Hence, $f_{N,1} \in B_{H^{s}}(2r)$ and $f_{N,1} \to 0$ in $H^{s}$. Therefore,
$(f_{N,1}, 0) \in B_{H^{s} \times H^{s-2}}(2r)$ and $(f_{N,1}, 0) \to 0$ in
$H^{s} \times H^{s-2}$. From the well-posedness theory for the $B_{4}$ equation,
the associated solutions $u_{N}$ share a common lifespan in the interval
$(0, 1)$. We now fix $\delta$ to be this common lifespan. Substituting
\cref{ill-pos-ce} into \cref{yut}, we obtain
%
%
\begin{equation}
  \label{rxx}
\begin{split}
  r^{2}N^{-s -2 \ee} \sup_{0 \le t \le \delta} 
  | \int_{0}^{t} \sin[4N^{2}(t-t')] [\cos(N^{2}t')]^{2} dt' |.
\end{split}
\end{equation}
%
Let $t = N^{-2}$, where we restrict $N \ge
\delta^{-1/2}$. Then \cref{rxx} is bounded below by
%
%
%
%
\begin{equation*}
\begin{split}
\frac{1}{4} r^{2}N^{-s -2 \ee}  
\int_{0}^{N^{-2}} \sin[4N^{2}(N^{-2}-t')] dt'
\end{split}
\end{equation*}
%
%
which evaluates to
%
%
\begin{equation*}
\begin{split}
\frac{1}{16} r^{2} N^{-s -2 -2 \ee}  
[1 - \cos 4]  \to \infty, \qquad s < -2 -2\ee
\end{split}
\end{equation*}
%
%
completing the proof.  \qed
%
%
%
%%%%%%%%%%%%%%%%%%%%%%%%%%%%%%%%%%%%%%%%%%%%%%%%%%%%%
%
%
%                Ill Posedness improvement
%
%
%%%%%%%%%%%%%%%%%%%%%%%%%%%%%%%%%%%%%%%%%%%%%%%%%%%%%
%
%
%
%%%%%%%%%%%%%%%%%%%%%%%%%%%%%%%%%%%%%%%%%%%%%%%%%%%%%
%
%
%                Ill-pos Line
%
%
%%%%%%%%%%%%%%%%%%%%%%%%%%%%%%%%%%%%%%%%%%%%%%%%%%%%%
%
%
\subsection{Ill-Posedness for $s < -2$ on the Line} 
\label{realsec:ill-pos-line}
The proof will be via contradiction. Assume $0 < \delta \le 1$. We have the estimate
%
%
\begin{equation*}
\begin{split}
  & \| N[A_{1}(f_{N}), A_{1}(f_{N})] \|_{X_{s,b}}
  \\
 & \ge
  \| N[A_{1}(f_{N}), A_{1}(f_{N})] \|_{C([0, \delta],
  H^{s})} 
  \\
  & = \sup_{0 \le t \le \delta} \| \psi_{\delta}(t) (1 + | \xi |)^{s}
  \frac{1}{4 i \pi} \int_{0}^{t} \left( e^{i\xi^{2}(t-t')} - e^{-i\xi^{2}(t-t')} \right)
  \wh{L(f_{N})^{2}}(\xi, t') d \xi_{1} dt' \|_{L^{2}_{\xi}}
  \\
  & = \sup_{0 \le t \le \delta} \| (1 + | \xi |)^{s} \frac{1}{4 i \pi} 
  \int_{0}^{t} \int_{\xi_{1}}
  \\
  & \left( e^{i\xi^{2}(t-t')} - e^{-i\xi^{2}(t-t')} \right)
  \wh{L(f_{N})}(\xi - \xi_{1}, t')\wh{L(f_{N})}(\xi_{1}, t') d
  \xi_{1} dt'
  \|_{L^{2}_{\xi}}
  \\
  & = \sup_{0 \le t \le \delta} \| (1 + | \xi |)^{s} \frac{1}{4 i \pi} 
  \int_{0}^{t} \int_{\xi_{1}} \left( e^{i\xi^{2}(t-t')} - e^{-i\xi^{2}(t-t')} \right)
  [\psi(t')]^{2}
  \\
  & \times \frac{1}{2 \pi} \left[ \frac{\wh{f_{N,1}}(\xi - \xi_{1})\left( e^{i(\xi - \xi_{1})^{2}t'} +
  e^{-i(\xi - \xi_{1})^{2}t'} \right)}{2} + \frac{\wh{f_{N,2}}(\xi - \xi_{1})\left(
  e^{i(\xi - \xi_{1})^{2}t'} - e^{-i(\xi - \xi_{1})^{2}t'}
  \right)}{2i(\xi - \xi_{1})^{2}} \right]
  \\
  & \times \frac{1}{2 \pi} \left[ \frac{\wh{f_{N,1}}(\xi_{1})\left( e^{i\xi_{1}^{2}t'} +
  e^{-i\xi_{1}^{2}t'} \right)}{2} + \frac{\wh{f_{N,2}}(\xi_{1})\left(
  e^{i\xi_{1}^{2}t'} - e^{-i\xi_{1}^{2}t'}
  \right)}{2 i \xi_{1}^{2}} \right] d \xi_{1} dt' \|_{L^{2}_{\xi}}
\end{split}
\end{equation*}
%
Take $f_{N,2}(x) \equiv 0$. Then the above reduces to
%
%
\begin{equation*}
\begin{split}
  & \sup_{0 \le t \le \delta} \frac{1}{16 i \pi^{2}}\| (1 + | \xi |)^{s}
  \int_{0}^{t} \int_{\xi_{1}} \left( e^{i\xi^{2}(t-t')} - e^{-i\xi^{2}(t-t')} \right)
  \\
  & \times \left[ \frac{\wh{f_{N,1}}(\xi - \xi_{1})\left( e^{i(\xi - \xi_{1})^{2}t'} +
  e^{-i(\xi - \xi_{1})^{2}t'} \right)}{2} \right ]
  \\
  & \times \left[ \frac{\wh{f_{N,1}}(\xi_{1})\left( e^{i\xi_{1}^{2}t'} +
  e^{-i\xi_{1}^{2}t'} \right)}{2}  \right]
  dt' \|_{L^{2}_{\xi}}
\end{split}
\end{equation*}
%
%
which we expand to obtain 
%
%
\begin{equation}
  \label{realyut}
\begin{split}
& \sup_{0 \le t \le \delta} \frac{1}{8 \pi^{2}}\| (1 + | \xi |)^{s}
\int_{0}^{t} \int_{\xi_{1}} \sin[\xi^{2}(t-t')] \cos[(\xi - \xi_{1})^{2}t']
\cos(\xi_{1}^{2}t') \wh{f_{N,1}}(\xi - \xi_{1})\wh{f_{N,1}}(\xi_{1}) d
\xi_{1}  dt'
\|_{L^{2}_{\xi}}
%\\
%& \gtrsim \sup_{0 \le t \le \delta} \| \chi_{| \xi | \le 1}
 %\int_{0}^{t} \int_{\xi_{1}} \sin[(t-t')] \cos[(1 -\xi_{1})^{2}t']
 %\cos(\xi_{1}^{2}t') \wh{f_{N,1}}(1 - \xi_{1})\wh{f_{N,1}}(\xi_{1}) d \xi_{1} dt'
 %\|_{L^{2}_{\xi}}.
\end{split}
\end{equation}
%
%
%
%
%
%
%
%
%
Let
%
%
\begin{equation}
  \label{realill-pos-ce}
\begin{split}
  \wh{f_{N,1}}(\xi) \doteq r N^{-s - \ee} \chi_{|\xi - N|\le 1}, \quad \ee > 0
\end{split}
\end{equation}
%
and fix $s < -1/2$. Then
%
%
\begin{equation*}
\begin{split}
  \| f_{N,1} \|_{H^{s}}
  & = rN^{-s-\ee} \left( \int_{N-1}^{N+1} (1 + \xi)^{2s} d \xi
  \right)^{1/2}
  \\
  & = rN^{-s-\ee} \left| \frac{1}{2s+1}( (N+2)^{2s+1} - N^{2s+1} ) \right|^{1/2},
  \\
  & = r N^{-\ee} \left| {\frac{1}{2s+1}}\left [ (N+2) \left( \frac{N+2}{N}
  \right)^{2s} -N \right ] \right|^{1/2}  \\
  & \le 2 r N^{-\ee}.
\end{split}
\end{equation*}
%
Hence, for $s \le -1/2$, $f_{N,1} \in B_{H^{s}}(2r)$
and $f_{N,1} \to 0$ in $H^{s}$. Therefore,
$(f_{N,1}, 0) \in B_{H^{s} \times H^{s-2}}(2r)$ and $(f_{N,1}, 0) \to 0$ in
$H^{s} \times H^{s-2}$. Since our initial data all lie in the same ball, from
the well-posedness theory for the $B_{4}$ equation we see that
the associated solutions $u_{N}$ share a common lifespan in the interval
$(0, 1)$. We now fix $\delta$ to be this common lifespan. Note that the
conditions
%
%
\begin{equation*}
\begin{split}
  & | \xi - \xi_{1} - N | \le 1,
  \\
  & | \xi_{1} - N | \le 1
\end{split}
\end{equation*}
%
%
imply 
\begin{equation}
  \label{abs-val-cond}
  \xi - N - 1 \le \xi_{1} \le \xi - N +1, \ N-1 \le \xi_{1} \le N+1.
\end{equation}
If we restrict $2N \le \xi  \le 2N +1$ in \cref{abs-val-cond}, then we obtain
$N \le \xi_{1} \le N+1$. Therefore, substituting
\cref{realill-pos-ce} into \cref{realyut} and localizing around the set $S
\doteq \left\{ \xi: 2N \le \xi \le 2N+1  \right\}$, we obtain the lower bound
%
%
%
\begin{equation*}
\begin{split}
  & \frac{r^{2}}{8 \pi^{2}}
  N^{-2s -2 \ee} \sup_{0 \le t \le \delta} \| \chi_{S}
  (1 + | \xi |)^{s} \int_{0}^{t} \int_{N}^{N+1} 
  \sin[\xi^{2}(t-t')] \cos[(\xi -\xi_{1})^{2}t']
 \cos(\xi_{1}^{2}t') d \xi_{1} dt'
 \|_{L^{2}_{\xi}}.
 \end{split}
\end{equation*}
%
%
Let $t = N^{-2}$, where we restrict $N \ge
\delta^{-1/2}$. Noting that $\xi - \xi_{1} \sim N$, $\xi_{1} \sim N$, and $\xi
\sim N$ thanks
to the restrictions on the domains of integration for $\xi$ and $\xi_{1}$, we bound below by
%
%
%
%
\begin{equation*}
\begin{split}
& \frac{r^{2}}{8 \pi^{2}}
  N^{-2s -2 \ee} \| \chi_{S}
  (1 + | \xi |)^{s} \int_{0}^{N^{-2}} \int_{N}^{N+1} 
  \sin[\xi^{2}(t-t')] \cos[(\xi -\xi_{1})^{2}t']
 \cos(\xi_{1}^{2}t') d \xi_{1} dt'
 \|_{L^{2}_{\xi}}
\\
  & \ge \frac{r^{2}}{32 \pi^{2}} N^{-2s - 2\ee}  \|\chi_{S} (1 + | \xi
  |)^{s}
  \int_{0}^{N^{-2}} \int_{N}^{N+1} \sin[\xi^{2}(N^{-2}-t')] d
  \xi_{1} dt' \|_{L^{2}_{\xi}}
  \\
  & \simeq
  N^{-2s - 2\ee}  \|\chi_{S}  (1 + | \xi
  |)^{s}
  \int_{0}^{N^{-2}} \sin[\xi^{2}(N^{-2}-t')] dt' \|_{L^{2}_{\xi}}
  \\
  & = N^{-2s- 2\ee}  \|\chi_{S}
  [1 - \cos(\xi^{2}N^{-2})](1 + | \xi |)^{s} \xi^{-2}
  \|_{L^{2}_{\xi}}
  \\
  & \gtrsim  N^{-2s- 2\ee}  \|\chi_{S}
  [1 - \cos(\xi^{2}N^{-2})](1 + | \xi |)^{s-2} 
  \|_{L^{2}_{\xi}}
  \\
  & \gtrsim N^{-s- 2 -2\ee}  \|\chi_{S}
  \|_{L^{2}_{\xi}}
  \\
  & = N^{-s -2 -2\ee} \to \infty, \qquad s < -2 -2 \ee
  \end{split}
\end{equation*}
%
%
which completes the proof. \qed
%
%
%
\subsection{Attempt to Improve the Ill-Posedness Index} 
\label{sec:ill-pos-improv}
On the circle, we have
%
%
\begin{equation*}
\begin{split}
  & \| N[A_{1}(f_{N}, g_{N}), A_{2}(f_{N}, {g_{N}})] \|_{C([0, \delta]),
  H^{s}}
  \\
  & = \sup_{0 \le t \le \delta} \| \psi_{\delta}(t) (1 + | n |)^{s}
  \frac{1}{4 i \pi} \int_{0}^{t} \left( e^{in^{2}(t-t')} - e^{-in^{2}(t-t')} \right)
  \\
  & \times \wh{\{N[A_{1}(f_{N}, g_{N}), A_{2}(f_{N}, g_{N})]} \}^{2}(n, t') dt'
  \|_{\ell^{2}_{n}}
  \\
  & = \sup_{0 \le t \le \delta} \| (1 + | n |)^{s} \frac{1}{4 i \pi} 
  \int_{0}^{t} \sum_{n_{1}} \left( e^{in^{2}(t-t')} - e^{-in^{2}(t-t')} \right)
  \\
  & \times \wh{L(f_{N},g_{N})}(n - n_{1}, t')\wh{A_{2}(f_{N},g_{N})}(n_{1}, t') dt'
  \|_{\ell^{2}_{n}}
  \\
  & = \sup_{0 \le t \le \delta} \| (1 + | n |)^{s} \frac{1}{4 i \pi} 
  \int_{0}^{t} \sum_{n_{1}} \left( e^{in^{2}(t-t')} - e^{-in^{2}(t-t')} \right)
  \psi(t')
  \\
  & \times \frac{1}{2 \pi} \left[ \frac{\wh{f_{N}}(n - n_{1})\left( e^{i(n - n_{1})^{2}t'} +
  e^{-i(n - n_{1})^{2}t'} \right)}{2} + \frac{\wh{g_{N}}(n - n_{1})\left(
  e^{i(n - n_{1})^{2}t'} - e^{-i(n - n_{1})^{2}t'}
  \right)}{2i(n - n_{1})^{2}} \right]
  \\
  & \times \wh{N[L(f_{N}, g_{N}), L(f_{N}, g_{N})](n_{1}, t')} dt' \|_{\ell^{2}_{n}}
\end{split}
\end{equation*}
which is equal to
\begin{equation*}
  \begin{split}
  & \sup_{0 \le t \le \delta} \| (1 + | n |)^{s} \frac{1}{4 i \pi} 
  \int_{0}^{t} \sum_{n_{1}} \left( e^{in^{2}(t-t')} - e^{-in^{2}(t-t')} \right)
  \\
  & \times \frac{1}{2 \pi} \left[ \frac{\wh{f_{N}}(n - n_{1})\left( e^{i(n - n_{1})^{2}t'} +
  e^{-i(n - n_{1})^{2}t'} \right)}{2} + \frac{\wh{g_{N}}(n - n_{1})\left(
  e^{i(n - n_{1})^{2}t'} - e^{-i(n - n_{1})^{2}t'}
  \right)}{2i(n - n_{1})^{2}} \right]
  \\
  & \times \frac{1}{4 \pi i} \psi_{\delta}(t') \int_{0}^{t'} \left[ e^{in_{1}^{2}(t' -s)} -
  e^{-in_{1}^{2}(t' -s)} \right] \wh{L(f, g) L(f,g)}(n_{1},s) ds dt'  \|_{\ell^{2}_{n}}
\end{split}
\end{equation*}
or 
%
%
\begin{equation*}
\begin{split}
& \sup_{0 \le t \le \delta} \| (1 + | n |)^{s} \frac{1}{4 i \pi} 
  \int_{0}^{t} \sum_{n_{1}} \left( e^{in^{2}(t-t')} - e^{-in^{2}(t-t')} \right)
  \\
  & \times \frac{1}{2 \pi} \left[ \frac{\wh{f_{N}}(n - n_{1})\left( e^{i(n - n_{1})^{2}t'} +
  e^{-i(n - n_{1})^{2}t'} \right)}{2} + \frac{\wh{g_{N}}(n - n_{1})\left(
  e^{i(n - n_{1})^{2}t'} - e^{-i(n - n_{1})^{2}t'}
  \right)}{2i(n - n_{1})^{2}} \right]
  \\
  & \times \frac{1}{4 \pi i} \int_{0}^{t'} \left[ e^{in_{1}^{2}(t' -s)} -
  e^{-in_{1}^{2}(t' -s)} \right] \sum_{n_{2}} \wh{L(f, g)}(n_{1} - n_{2},s)
  \wh{L(f,g)}(n_{2},s) ds dt'  \|_{\ell^{2}_{n}}
  \\
 & = \sup_{0 \le t \le \delta} \| (1 + | n |)^{s} \frac{1}{4 i \pi} 
  \int_{0}^{t} \sum_{n_{1}} \left( e^{in^{2}(t-t')} - e^{-in^{2}(t-t')} \right)
  \\
  & \times \frac{1}{2 \pi} \left[ \frac{\wh{f_{N}}(n - n_{1})\left( e^{i(n - n_{1})^{2}t'} +
  e^{-i(n - n_{1})^{2}t'} \right)}{2} + \frac{\wh{g_{N}}(n - n_{1})\left(
  e^{i(n - n_{1})^{2}t'} - e^{-i(n - n_{1})^{2}t'}
  \right)}{2i(n - n_{1})^{2}} \right]
  \\
  & \times \frac{1}{4 \pi i} \int_{0}^{t'} \left[ e^{in_{1}^{2}(t' -s)} -
  e^{-in_{1}^{2}(t' -s)} \right]
  \\
  & \times \sum_{n_{2}} 
  \left[ \frac{1}{2 \pi} \psi(s) 
  \wh{f}(n_{1} - n_{2}) \frac{e^{i(n_{1} - n_{2})^{2}s} + e^{-i(n_{1} -
  n_{2})^{2}s}}{2} 
  + \psi(s) \wh{g}(n_{1} - n_{2})\frac{e^{i(n_{1} - n_{2})^{2}s} -
  e^{-i(n_{1} - n_{2})^{2}s}}{2 i (n_{1} - n_{2})^{2}} \right]
  \\
  & \times 
  \left[ \frac{1}{2 \pi} \psi(s) 
  \wh{f}(n_{2}) \frac{e^{in_{2}^{2}s} + e^{-in_{2}^{2}s}}{2} 
  + \psi(s) 
  \wh{g}(n_{2})\frac{e^{in_{2}^{2}s} - e^{-in_{2}^{2}s}}{2 i n_{2}^{2}} \right]
  ds dt'
  \|_{\ell^{2}_{n}}
\end{split}
\end{equation*}
%
%
or
%
%
\begin{equation*}
\begin{split}
& \sup_{0 \le t \le \delta} \| (1 + | n |)^{s} \frac{1}{4 i \pi} 
  \int_{0}^{t} \sum_{n_{1}} \left( e^{in^{2}(t-t')} - e^{-in^{2}(t-t')} \right)
  \\
  & \times \frac{1}{2 \pi} \left[ \frac{\wh{f_{N}}(n - n_{1})\left( e^{i(n - n_{1})^{2}t'} +
  e^{-i(n - n_{1})^{2}t'} \right)}{2} + \frac{\wh{g_{N}}(n - n_{1})\left(
  e^{i(n - n_{1})^{2}t'} - e^{-i(n - n_{1})^{2}t'}
  \right)}{2i(n - n_{1})^{2}} \right]
  \\
  & \times \frac{1}{4 \pi i} \int_{0}^{t'} \left[ e^{in_{1}^{2}(t' -s)} -
  e^{-in_{1}^{2}(t' -s)} \right]
  \\
  & \times \sum_{n_{2}} 
  \left[ \frac{1}{2 \pi} 
  \wh{f}(n_{1} - n_{2}) \frac{e^{i(n_{1} - n_{2})^{2}s} + e^{-i(n_{1} -
  n_{2})^{2}s}}{2} 
  + \wh{g}(n_{1} - n_{2})\frac{e^{i(n_{1} - n_{2})^{2}s} -
  e^{-i(n_{1} - n_{2})^{2}s}}{2 i (n_{1} - n_{2})^{2}} \right]
  \\
  & \times 
  \left[ \frac{1}{2 \pi} 
  \wh{f}(n_{2}) \frac{e^{in_{2}^{2}s} + e^{-in_{2}^{2}s}}{2} 
  + 
  \wh{g}(n_{2})\frac{e^{in_{2}^{2}s} - e^{-in_{2}^{2}s}}{2 i n_{2}^{2}} \right]
  ds dt'
  \|_{\ell^{2}_{n}}.
\end{split}
\end{equation*}
%
%
Take $g(x) \equiv 0$. Then the above reduces to
%
%
\begin{equation*}
\begin{split}
& \sup_{0 \le t \le \delta} \| (1 + | n |)^{s} \frac{1}{4 i \pi} 
  \int_{0}^{t} \sum_{n_{1}} \left( e^{in^{2}(t-t')} - e^{-in^{2}(t-t')} \right)
  \\
  & \times \frac{1}{2 \pi} \left[ \frac{\wh{f_{N}}(n - n_{1})\left( e^{i(n - n_{1})^{2}t'} +
  e^{-i(n - n_{1})^{2}t'} \right)}{2} \right]
  \\
  & \times \frac{1}{4 \pi i} \int_{0}^{t'} \left[ e^{in_{1}^{2}(t' -s)} -
  e^{-in_{1}^{2}(t' -s)} \right]
  \sum_{n_{2}} 
  \left[ \frac{1}{2 \pi} 
  \wh{f}(n_{1} - n_{2}) \frac{e^{i(n_{1} - n_{2})^{2}s} + e^{-i(n_{1} -
  n_{2})^{2}s}}{2} 
  \right]
  \\
  & \times 
  \left[ \frac{1}{2 \pi} 
  \wh{f}(n_{2}) \frac{e^{in_{2}^{2}s} + e^{-in_{2}^{2}s}}{2} 
  \right]
  ds dt'
  \|_{\ell^{2}_{n}}
\end{split}
\end{equation*}
%
or
\begin{equation*}
\begin{split}
  & \frac{1}{32 \pi^{5}} \sup_{0 \le t \le \delta} \| (1 + | n |)^{s}  
  \int_{0}^{t} \sum_{n_{1}} \wh{f_{N}}(n - n_{1}) \sin[n^{2}(t - t')]
  \cos[(n - n_{1})^{2}t']
  \\
  & \times \int_{0}^{t'} \sin[n_{1}^{2}(t' -s)]
  \sum_{n_{2}} 
  \wh{f}(n_{1} - n_{2}) \wh{f}(n_{2}) \cos[(n_{1} - n_{2})^{2}s ]
  \cos(n_{2}^{2}s) ds dt'
  \|_{\ell^{2}_{n}}
  \\
  & \gtrsim \sup_{0 \le t \le \delta} 
  | \int_{0}^{t} \sum_{n_{1}} \wh{f_{N}}(1 - n_{1}) \sin[(t - t')]
  \cos[(1 - n_{1})^{2}t']
  \\
  & \times \int_{0}^{t'} \sin[n_{1}^{2}(t' -s)]
  \sum_{n_{2}} 
  \wh{f}(n_{1} - n_{2}) \wh{f}(n_{2}) \cos[(n_{1} - n_{2})^{2}s ]
  \cos(n_{2}^{2}s) ds dt' |
\end{split}
\end{equation*}
%
where the last step follows from the inequality $\| a_{n} \|_{\ell^{2}} \ge
|a_{1}|$. Choose $f_N$ as in \cref{ill-pos-ce}. Then
$(f_{N}, g_{N}) \in B_{H^{s} \times H^{s-2}}(2r)$ and $(f_{N}, g_{N}) \to 0$ in
$H^{s} \times H^{s-2}$, as before. Substituting, we bound below by
%
%
%
\begin{equation*}
\begin{split}
  & r^{3} N^{-3s - 3 \ee} \sup_{0 \le t \le \delta} 
  | \int_{0}^{t} \sin[(t - t')] \cos(N^{2}t')
  \\
  & \times \int_{0}^{t'} \sin[(N-1)^{2}(t' -s)]
  \cos^{2}(N^{2}s )
  ds dt' |
\end{split}
\end{equation*}
%
Let $t \doteq N^{-\alpha}$, where we restrict $N \le
\delta^{1/\alpha}$. Then, for $\alpha \ge 2$ we bound below by
%
%
\begin{equation*}
\begin{split}
  & \frac{1}{64} r^{3} N^{-3s - 3 \ee} 
  | \int_{0}^{N^{-\alpha}} \sin(N^{-\alpha} - t') 
  \\
  & \times \int_{0}^{t'} \sin[(N-1)^{2}(t' -s)]
  ds dt' |
\end{split}
\end{equation*}
%
%
%
%
%
Integrating in $s$ gives
%
%
\begin{equation}
  \label{gfd}
\begin{split}
  & \frac{1}{64} r^{3} N^{-3s - 3 \ee}(N-1)^{-2} 
| \int_{0}^{N^{-\alpha}} \sin(N^{-\alpha} - t')
\left\{1 -\cos[ (N-1)^{2}t']  \right \} dt' |.
\end{split}
\end{equation}
%
%
Then 
%
%
\begin{equation}
  \label{pre-int}
\begin{split}
& \int_{0}^{N^{-\alpha}} \sin(N^{-\alpha} - t')
\left\{1 -\cos[(N-1)^{2}t']  \right\} dt' 
\\
& = 1 - \cos(N^{-\alpha}) - \int_{0}^{N^{-\alpha}} \sin(N^{-\alpha} -t')
\cos[(N-1)^{2}t'] dt'.
\end{split}
\end{equation}
%
%
But
%
%
\begin{equation*}
\begin{split}
   & \int_{0}^{N^{-\alpha}} \sin(N^{-\alpha -t'}) \cos[(N-1)^{2}t'] 
   \\
   & = \int_{0}^{N^{-\alpha}} \frac{e^{i(N^{-\alpha} -t')} -
  e^{-i(N^{-\alpha}-t)}}{2i} \times \frac{e^{i[(N-1)^{2}t']} + e^{-i[(N-1)^{2}t']}}{2}
  \\
  & = \frac{1}{4 i} \int_{0}^{N^{-\alpha}} \left( e^{i[N^{-\alpha} + (N-1)^{2}t'
  - t']}) + e^{i[N^{-\alpha} - (N-1)^{2}t' -t']} \right.
  \\
  & \left. - e^{-i[N^{-\alpha} - t' -
  (N-1)^{2}t']} - e^{-i[N^{-\alpha} - t' + (N-1)^{2}t']}  \right ) dt'
  \\
  & = \frac{1}{4i} \left \{\left[ \frac{e^{i(N-1)^{2}N^{-\alpha}}}{i[(N-1)^{2}-1]} -
  \frac{e^{-i(N-1)^{2}N^{-\alpha}}}{i[(N-1)^{2} +1]} - \frac{e^{i(N-1)^{2}N^{-
  \alpha}}}{i[(N-1)^{2} +1]} - \frac{e^{-i(N-1)^{2}N^{-\alpha}}}{i[(N-1)^{2} +1]}\right] \right.
  \\
  & - \left. \left[
  \frac{e^{iN^{-\alpha}}}{i[(N-1)^{2} -1]} - \frac{e^{iN^{-\alpha}}}{i[(N-1)^{2} +1]} -
  \frac{e^{-iN^{-\alpha}}}{i[(N-1)^{2} +1]} +
  \frac{e^{-iN^{-\alpha}}}{i[(N-1)^{2} -1]}  \right]  \right \}
  \\
  & = \frac{1}{4} \left\{2 \cos[(N-1)^{2}N^{-\alpha}]\left[
  \frac{1}{(N-1)^{2} +1} - \frac{1}{(N-1)^{2} -1} \right] \right.
  \\
  & + \left. 2 \cos N^{-\alpha}
  \left[ \frac{1}{(N-1)^{2} -1} - \frac{1}{(N-1)^{2} +1} \right] \right\}
\end{split}
\end{equation*}
%
which reduces to
%
%
\begin{equation*}
\begin{split}
 \frac{\cos(N^{-\alpha}) - \cos[(N-1)^{2}N^{-\alpha}]}{(N-1)^{4} -1}.
\end{split}
\end{equation*}
%
%
Substituting into \cref{pre-int}, we obtain
%
%
%
\begin{equation*}
\begin{split}
  1 - \cos (N^{-\alpha}) -
  \frac{\cos(N^{-\alpha}) - \cos[(N-1)^{2}N^{-\alpha}]}{(N-1)^{4} -1}.
\end{split}
\end{equation*}
%
%
Hence, \cref{gfd} reduces to
%
%
%
\begin{equation*}
\begin{split}
  & \frac{1}{64} r^{3} N^{-3s - 3 \ee}(N-1)^{-2} \left | 1 - \cos (N^{-\alpha}) -
  \frac{\cos(N^{-\alpha}) - \cos[(N-1)^{2}N^{-\alpha}]}{(N-1)^{4} -1}\right |
  \\
  & \gtrsim 
  N^{-3s - 3 \ee -2} \left | 1 - \cos (N^{-\alpha}) -
  \frac{\cos(N^{-\alpha}) - \cos[(N-1)^{2}N^{-\alpha}]}{(N-1)^{4} -1}\right |
  \\
  & \ge \frac{1 - \cos (N^{-\alpha})}{N^{3s + 3 \ee +2}}  -N^{-3s - 3 \ee -2} 
  \left [\frac{\cos(N^{-\alpha}) - \cos[(N-1)^{2}N^{-\alpha}]}{(N-1)^{4} -1}
  \right ].
\end{split}
\end{equation*}
%
%
Assume $s < -2/3 - \ee$. Then by l'H\^opital's rule
%
%
\begin{equation*}
\begin{split}
   \lim_{N \to \infty} \frac{1 - \cos (N^{-\alpha})}{N^{3s + 3 \ee +2}}
   & \simeq  \lim_{N \to \infty} \frac{\sin(N^{-\alpha})N^{-\alpha -1}}{N^{3s + 3 \ee +1
}}
  \\
  & = \lim_{N \to 0} \frac{\sin(N^{\alpha})}{N^{-3s - 3 \ee -2
  -\alpha}}
  \\
  & = 1, \qquad \alpha = -3s -3 \ee -2 - \alpha
\end{split}
\end{equation*}
%
%
or $2 \alpha = -3s - 3 \ee -2$. Since $\alpha \ge 2$, we obtain 
$-3s -3 \ee -2 \ge 4$, which gives $s \le -2 -\ee$. Hence, considering the 
higher iterate $A_{2}$ has given us no better an ill-posedness result than
considering the $A_{1}$ iterate. A similar phenomenon should occur for the real
line. 
%
%
%%%%%%%%%%%%%%%%%%%%%%%%%%%%%%%%%%%%%%%%%%%%%%%%%%%%%
%
%
%                Proof of ill-pos props
%
%
%%%%%%%%%%%%%%%%%%%%%%%%%%%%%%%%%%%%%%%%%%%%%%%%%%%%%
%
%
\section{Proof of \hyperref[prop:ill-posedness]{Key Ill-Posedness
Proposition}}
\label{sec:pf-ill-pos-prop}
We will need the following lemma, whose proof is provided at the end of the
section.
%
%
%%%%%%%%%%%%%%%%%%%%%%%%%%%%%%%%%%%%%%%%%%%%%%%%%%%%%
%
%
%               Quantitative wp implies analytic wp
%
%
%%%%%%%%%%%%%%%%%%%%%%%%%%%%%%%%%%%%%%%%%%%%%%%%%%%%%
%
%
\begin{lemma}[Quantitative WP implies Analytic WP]
  \label{lem:qwp-awp}
  For $f \in B_{H^{s} \times H^{s-2}}(r)$, $s > -1/4$ and $\delta=\delta(r)$
sufficiently small: 
\begin{enumerate}[(I)]
  \item
    \label{qwp-it-1}
    There exists a unique solution
    $u[f] \in X_{s,b} \cap C([-\delta, \delta], H^{s})$
to the localized integral $B_{4}$ equation
\cref{main1-rel-term-0}-\cref{main1-rel-term-5}.
\item
  \label{homogen}
  We have the homogeneity property
  %
  %
  \begin{equation}
    \label{eqn:homog}
  \begin{split}
    A_{n}(\lambda f) = \lambda^{n} A_{n}(f) \ \text{for all} \ \lambda \in \rr
  \end{split}
  \end{equation}
  %
  \item
    \label{pre-lip-qwp}
    We have the bound
\begin{equation}
  \label{induct-ineq}
  \begin{split}
    \| A_{n}(f) \|_{X_{s,b}} \le  c_{\psi}^{2n-1} 30^{(n-1)}
    \delta^{(n-1)(b-1/2)} \| f
    \|_{H^{s} \times H^{s-2}}^{n}
  \end{split}
  \end{equation}
  %
  %
  %
  %
  %
  \item
    \label{lip-qwp}
    We have the Lipschitz bound
    %
    %
    \begin{equation}
      \label{lip-qwp-eq}
    \begin{split}
      \| A_{n}(f) - A_{n}(g) \|_{X_{s,b}}
      & \le c_{\psi}^{2n-1} 30^{n-1} 
      \delta^{(n-1)(b-1/2)} \| f-g \|_{H^{s} \times
      H^{s-2}}
      \\
      & \times \left( \| f \|_{H^{s} \times H^{s-2}}
      + \| g \|_{H^{s} \times H^{s-2}}  \right)^{n-1}
    \end{split}
    \end{equation}
    %
    %
\item
  \label{qwp-it-5}
  %
%
For all $f \in B_{H^{s} \times H^{s-2}}(r)$ we have the absolutely convergent
(in $X_{s,b}$) power series expansion
%
%
\begin{equation}
  \label{power-series-soln}
\begin{split}
  u[f] = \sum_{n=1}^{\infty} A_{n}(f)
\end{split}
\end{equation}
\end{enumerate}
%
%
\label{lem:analytic-wp}
\end{lemma}
%
To prove \cref{prop:ill-posedness}, we first assume that the solution
map $f \mapsto u[f]$ is continuous from $B_{H^{s} \times H^{s-2}}(r_{1})$ to
$B_{X_{s,b}}(r_{2})$. Let $\left\{ f_{m} \right\} \subset B_{H^{s} \times
H^{s-2}}(r_{1})$ be a sequence converging to $f \in B_{H^{s} \times
H^{s-2}}(r_{1})$ in the
$H^{s'} \times H^{s'-2}$ topology, $s' < s$. That is
%
%
\begin{equation*}
\begin{split}
  \| f - f_{m} \|_{H^{s} \times H^{s-2}} \to 0.
\end{split}
\end{equation*}
%
%
We will show that
%
%
\begin{equation*}
\begin{split}
  \| A_{n}(f) - A_{n}(f_{m}) \|_{X_{s',b}} \to 0
\end{split}
\end{equation*}
%
%
by induction on $n$. For the base case $n=1$, we note that
%
%
\begin{equation*}
\begin{split}
  \| A_{1}(f) - A_{1}(f_{m}) \|_{X_{s',b}}
  & =  \|L(f) - L(f_{m})  \|_{X_{s',b}}
  \\
  & = \|L(f -f_{m}) \|_{X_{s',b}}
  \\
  & \lesssim \| f - f_{m} \|_{H^{s'} \times H^{s'-2}} \to 0.
\end{split}
\end{equation*}
%
For the inductive step, assume that for al $1 < n' < n$, the operator
$A_{n'}: H^{s'} \times H^{s'-2} \mapsto X_{s',b}$ is continuous from
$(H^{s} \times H^{s-2}, \|  \|_{H^{s'} \times H^{s'-2}})$ to $(X_{s,b}, \|
\|_{X_{s',b}})$. Then
%
%
\begin{equation*}
\begin{split}
  \lim_{m \to \infty} \| u[\lambda f] - u[\lambda f_{m}] \|_{X_{s',b}} = 0.
\end{split}
\end{equation*}
%
%
Applying \cref{qwp-it-5}, this implies
%
%
\begin{equation*}
\begin{split}
  \lim_{m \to \infty} \| \sum_{n'=1}^{\infty} A_{n'}(\lambda f) -
  A_{n'}(\lambda f_{m}) \|_{X_{s',b}} =0
\end{split}
\end{equation*}
%
%
which by \cref{homogen} gives
%
%
\begin{equation}
\begin{split}
  \lim_{m \to \infty} \| \sum_{n'=1}^{\infty} \lambda^{n'}\left[
  A_{n'}(f) - A_{n'}(f_{m})
  \right] \|_{X_{s',b}} =0.
\end{split}
\label{1bh}
\end{equation}
%
%
By the reverse triangle inequality
%
%
\begin{equation}
\begin{split}
  & \| \sum_{n'=1}^{\infty} \lambda^{n'}\left[ A_{n'}(f) - A_{n'}(f_{m}) \right]
  \|_{X_{s',b}}
  \\
  & = \| \sum_{n' < n}^{\infty} \lambda^{n'}\left[ A_{n'}(f) -
  A_{n'}(f_{m}) \right] - \sum_{n' > n}^{\infty} \lambda^{n'}\left[
  A_{n'}(f_{m}) - A_{n'}(f)
  \right] \|_{X_{s',b}}
  \\
  & \ge |\| \sum_{n' < n}^{\infty} \lambda^{n'}\left[ A_{n'}(f) -
  A_{n'}(f_{m}) \right] \|_{X_{s',b}} -
  \| \sum_{n' \ge n}^{\infty} \lambda^{n'}\left[ A_{n'}(f_{m}) -
  A_{n'}(f) \right] \|_{X_{s',b}} \|_{X_{s',b}} |.
\end{split}
\label{2bh}
\end{equation}
%
%
But by our inductive hypothesis
%
%
\begin{equation*}
\begin{split}
  \lim_{m \to \infty} \| \sum_{n'<n}^{\infty} \lambda^{n'} \left[
  A_{n'}(f') - A_{n'}(f_{m})
  \right] \|_{X_{s',b}} = 0.
\end{split}
\end{equation*}
%
%
Using this in conjunction with \cref{1bh}, we see that \cref{2bh} implies
%
%
\begin{equation*}
\begin{split}
  \lim_{m \to \infty} \| \sum_{n' \ge n}^{\infty} \lambda^{n'} \left[
  A_{n'}(f_{m}) - A_{n'}(f)
  \right] \|_{X_{s',b}} = 0.
\end{split}
\end{equation*}
%
%
By the reverse triangle inequality, we obtain
%
%
%
\begin{equation*}
\begin{split}
0 & = \lim_{m \to \infty} \| \sum_{n' \ge n}^{\infty} \lambda^{n'} \left[
  A_{n'}(f_{m}) - A_{n'}(f)
  \right] \|_{X_{s',b}}
  \\
  & = \lim_{m \to \infty}  \|\lambda^{n}\left[ A_{n}(f_{m}) - A_{n}(f) \right] -
  \sum_{n' > n} \lambda^{n'} \left[ A_{n'}(f) - A_{n}(f_{m})
  \right]\|_{X_{s',b}}
  \\
  & = \lim_{m \to \infty}
  \left\{ \lambda^{n} \| A_{n}(f_{m}) - A_{n}(f) \|_{X_{s',b}}
   - \| \sum_{n' > n} \lambda^{n'} \left[ A_{n'}(f) - A_{n}(f_{m})
      \right]\|_{X_{s',b}}\right\}
\end{split}
\end{equation*}
%
%
and so
%
%
\begin{equation*}
\begin{split}
  \lim_{m \to \infty} \lambda^{n} \| A_{n}(f_{m}) - A_{n}(f) \|_{X_{s',b}}
  & = \lim_{m \to \infty} \| \sum_{n' > n} \lambda^{n'} \left[ A_{n'}(f) - A_{n}(f_{m})
      \right]\|_{X_{s',b}}
      \\
      & \le  \sum_{n' > n} \lambda^{n'} \sup_{m} \|  A_{n'}(f) - A_{n}(f_{m})
      \|_{X_{s',b}}
\end{split}
\end{equation*}
%
%
Dividing both sides by $\lambda^{n}$ gives and applying \cref{lip-qwp-eq} gives
%
%
\begin{equation}
  \label{pre-lambda-small-app}
\begin{split}
& \lim_{m \to \infty}  \| A_{n}(f_{m}) - A_{n}(f) \|_{X_{s',b}}
  \\
  & \le \sum_{n' > n} \lambda^{n'-n} \sup_{m} \|  A_{n'}(f) - A_{n'}(f_{m})
      \|_{X_{s',b}}
      \\
      & \le  \sum_{n' > n} c_{\psi}^{2n'-1} 30^{n'-1} 
      \delta^{(n'-1)(b-1/2)}\lambda^{n'-n} \sup_{m}
      \| f-f_{m} \|_{H^{s} \times
      H^{s-2}} \left( \| f \|_{H^{s} \times H^{s-2}}
      + \| f_{m} \|_{H^{s} \times H^{s-2}}  \right)^{n'-1}
      \\
      & \le  \sum_{n' > n} c_{\psi}^{2n'-1} 30^{n'-1} 
      \delta^{(n'-1)(b-1/2)}  \lambda^{n'-n} (2r)^{n'}
\end{split}
\end{equation}
%
where the last step follows from the triangle inequality and the boundedness of
$f_{m}$ and $f$.
%
%
\begin{framed}
\begin{remark}
Notice that here we could have produced a similar estimate by applying the
triangle inequality and \cref{induct-ineq}. Indeed, the proof goes through using
this alternative estimate as well. That is, \cref{lip-qwp} is not needed to
prove \cref{qwp-it-5}. However, for the sake of completeness, we have
included the proof of \cref{lip-qwp} nonetheless.
\label{rem:dont-need-lip}
\end{remark}
\end{framed}
%
%
Note that
%
%
%
\begin{equation*}
\begin{split}
  \sum_{n' > n} c_{\psi}^{2n'-1} 30^{n'-1} 
  \delta^{(n'-1)(b-1/2)}  \lambda^{n'-n} (2r)^{n'}
  & = 2c_{\psi} r \sum_{n' > n} c_{\psi}^{2(n'-1)} 30^{n'-1} 
  \delta^{(n'-1)(b-1/2)}  \lambda^{n'-n} (2r)^{n'-1}
  \\
  & = 2 c_{\psi} r \sum_{n' > n} \lambda^{n'-n} [60 rc_{\psi}^{2}
  \delta^{(b-1/2)}]^{n'-1}
  \\
  & = 2 c_{\psi} r \sum_{k =1}^{\infty} \lambda^{k} [60 rc_{\psi}^{2}
  \delta^{(b-1/2)}]^{k +n -1}
  \\
  & = 2 c_{\psi} r \lambda [60 rc_{\psi}^{2}
  \delta^{(b-1/2)}]^{n}
  \sum_{k =0}^{\infty}  [60 \lambda rc_{\psi}^{2}
  \delta^{(b-1/2)}]^{k},
  \\
  & = \frac{2 c_{\psi} r \lambda [60 rc_{\psi}^{2}
  \delta^{(b-1/2)}]^{n}}{1-60 \lambda r c_{\psi}^{2}
  \delta^{(b-1/2)}}, \qquad \lambda < \frac{1}{60rc_{\psi}^{2} \delta^{(b-1/2)}}.
\end{split}
\end{equation*}
%
Recalling \cref{pre-lambda-small-app}, we see that for any $\lambda < 1/(60 r
c_{\psi}^{2} \delta^{b-1/2})$, we have
%
%
\begin{equation*}
\begin{split}
\lim_{m \to \infty}  \| A_{n}(f_{m}) - A_{n}(f) \|_{X_{s',b}}
  \le \frac{2 c_{\psi} r \lambda [60 rc_{\psi}^{2}
  \delta^{(b-1/2)}]^{n}}{1-60 \lambda r c_{\psi}^{2}
  \delta^{(b-1/2)}}.
\end{split}
\end{equation*}
%
%
Taking $\lambda \to 0$ completes the proof of \cref{prop:ill-posedness}. \qed
%
\subsection{Proof of \hyperref[lem:qwp-awp]{Power Series Decomposition Lemma}} We split the proof into five parts.
\label{ssec:pf-qwp-it-1}
%
\begin{proof}[Proof of \cref{qwp-it-1}]
  It is a consequence of the linear and nonlinear estimates
  %
  %
  \begin{align}
    \label{ill-pos-lin}
    & \| L(f) \|_{X_{s,b}} \le c_{\psi} \| f \|_{H^{s} \times H^{s-2}},
    \\
    & \| N(u,v) \|_{X_{s,b}} \le c_{\psi} \delta^{b-1/2} \| u \|_{X_{s,b}} \| v
    \|_{X_{s,b}}.
    \label{ill-pos-bil}
  \end{align}
  %
  For the proofs of these estimates, and how to obtain \cref{qwp-it-1} from
  them, we refer the reader to \cref{sec:periodic} and
  \cref{sec:non-periodic-case}.
\end{proof}
%
%
%
\begin{proof}[Proof of \cref{homogen}]
  %
  The proof will be via induction on $n$. For the base case $n=1$, we have
  %
  %
  \begin{equation*}
  \begin{split}
    A_{1}(\lambda f) = L(\lambda f) = \lambda L(f).
  \end{split}
  \end{equation*}
  %
  %
  due to the linearity of $L$. For the inductive step we assume
  \cref{eqn:homog} for $n' < n$. Since $N$ is bilinear, we have
  \begin{equation*}
  \begin{split}
    A_{n}(\lambda f)
    & = \sum_{\substack{j, k \in \mathbb{N}\\ j + k = n}} N\left[
    A_{j}(\lambda f), A_{k}(\lambda f) \right]
    \\
    & = \sum_{\substack{j, k \in \mathbb{N}\\ j + k = n}} N\left[
    \lambda^{j} A_{j}(f), \lambda^{k} A_{k}(\lambda f) \right]
    \\
    & = \sum_{\substack{j, k \in \mathbb{N}\\ j + k = n}} \lambda^{j+k}
    N\left[ A_{j}(f),  A_{k}(\lambda f) \right]
    \\
    & = \lambda^{n} A_{n}(f)
  \end{split}
  \end{equation*}
  %
  %
  concluding the proof.
\end{proof}
%
%
%
%
\begin{proof}[Proof of \cref{pre-lip-qwp}]
  We will prove a stronger estimate
\begin{equation}
  \label{op-strong-bound}
  \begin{split}
    \| A_{n}(f) \|_{X_{s,b}} \le  c_{\psi}^{2n-1} 30^{n-1}
    \delta^{(n-1)(b-1/2)} n^{-3}\| f
    \|_{H^{s} \times H^{s-2}}^{n}
  \end{split}
  \end{equation}
via an induction on $n$.
For the base case $n=1$, we have
  %
  \begin{equation*}
  \begin{split}
    \| A_{1}(f) \|_{X_{s,b}}
    & = \| L(f) \|_{X_{s,b}}
    \\
    & \le  c_{\psi} \| f \|_{H^{s} \times H^{s-2}}.
  \end{split}
  \end{equation*}
  %
  For the inductive step, we assume \cref{induct-ineq} for $n' < n$. 
  Then applying \cref{ill-pos-bil} gives
  %
  %
  %
  %
  %
  \begin{equation*}
  \begin{split}
    \| A_{n}(f) \|_{X_{s,b}}
    & = \| \sum_{\substack{j, k \in \mathbb{N}\\ j + k = n}} N\left[
    A_{j}(f), A_{k}(f) \right] \|_{X_{s,b}}
    \\
    & \le \sum_{\substack{j, k \in \mathbb{N}\\ j + k = n}} \| N\left[
    A_{j}(f), A_{k}(f) \right] \|_{X_{s,b}}
    \\
    & \le c_{\psi} \sum_{\substack{j, k \in \mathbb{N}\\ j + k = n}} 
    \delta^{b-1/2} \| A_{j}(f) \|_{X_{s,b}} \| A_{k}(f)\|_{X_{s,b}}
  \end{split}
  \end{equation*}
  %
  %
  which by  \cref{induct-ineq} is bounded by
  %
  %
  \begin{equation}
    \label{pre-sub-sum}
  \begin{split}
    & c_{\psi} \sum_{\substack{j, k \in \mathbb{N}\\ j + k = n}} 
    \delta^{b-1/2}\left[ c_{\psi}^{2j-1}
    30^{j-1}j^{-3} \delta^{(j-1)(b-1/2)}\| f \|_{H^{s} \times
    H^{s-2}}^{j} \right] \left [c_{\psi}^{2k-1}
    30^{k-1} k^{-3} \delta^{(k-1)(b-1/2)}\| f \|_{H^{s} \times
    H^{s-2}}^{k} \right]
    \\
    & = \sum_{\substack{j, k \in \mathbb{N}\\ j + k = n}} c_{\psi}^{2j+2k -1}
    30^{j+k-2} (jk)^{-3} \delta^{(j + k -1)(b-1/2)} \| f \|_{H^{s} \times
    H^{s-2}}^{j+k}
    \\
    &  \le  30^{n-2} \delta^{(n-1)(b-1/2)}
    c_{\psi}^{2n-1} \| f \|^{n+1}_{H^{s} \times H^{s-2}} \times 
    \sum_{\substack{j, k \in \mathbb{N}\\ j + k = n}}  (jk)^{-3}
    \\
    & = 30^{n-1} \delta^{(n-1)(b-1/2)}
    c_{\psi}^{2n-1} \| f \|^{n+1}_{H^{s} \times H^{s-2}} \times \frac{1}{30}
    \sum_{j=1}^{n-1}
    \frac{1}{j^{3}(n-j)^{3}}.
    \end{split}
  \end{equation}
  %
Hence, to complete the proof, it will be enough to show that
  %
  %
  %
  %
  \begin{equation*}
  \begin{split}
  \sum_{j=1}^{n-1}
  \frac{1}{j^{3}(n-j)^{3}} \le \frac{30}{n^{3}}.
  \end{split}
  \end{equation*}
  %
  %
  Using the partial decomposition
  %
  %
  \begin{equation*}
  \begin{split}
    \frac{1}{j^{3}(n-j)^{3}} = \frac{1}{n^{3}} \times \frac{1}{j^{3}} +
    \frac{1}{(n-j)^{2}} \times \frac{3n-1}{n^{2}(n-1)^{3}} +
    \frac{1}{(n-j)^{3}} \times \frac{1}{n^{3}}
  \end{split}
  \end{equation*}
  %
  %
  we obtain the estimate
  %
  %
  \begin{equation}
    \label{pre-sum-est}
  \begin{split}
    \sum_{j=1}^{n-1} \frac{1}{j^{3}(n-j)^{3}}
    & = \frac{1}{n^{3}} \left(
    \sum_{j=1}^{n-1} \frac{1}{j^{3}} + \sum_{j=1}^{n-1}
    \frac{1}{(n-j)^{3}}
    \right) + \frac{3n-1}{n^{2}(n-1)^{3}} \sum_{j=1}^{n-1}
    \frac{1}{(n-j)^{2}}
    \\
    & = \frac{2}{n^{3}} \sum_{j=1}^{n-1} \frac{1}{j^{3}} +
    \frac{3n-1}{n^{2}(n-1)^{3}} \sum_{j=1}^{n-1} \frac{1}{j^{2}}
    \\
    & \le \frac{2}{n^{3}} \sum_{j=1}^{\infty} \frac{1}{j^{3}} +
    \frac{3}{n(n-1)^{3}} \sum_{j=1}^{\infty} \frac{1}{j^{2}}
    \\
    & \le \frac{\pi^{2}}{6} \left[ \frac{2}{n^{3}} + \frac{3}{n(n-1)^{3}} \right]
    \\
    & \le \frac{\pi^{2}}{2} \left[ \frac{1}{n^{3}} + \frac{1}{n(n-1)^{3}}
    \right].
  \end{split}
  \end{equation}
  %
  %
  But %
  %
  \begin{equation*}
  \begin{split}
    \frac{1}{n(n-1)^{3}} \le \frac{5}{n^{3}}.
  \end{split}
  \end{equation*}
  %
  %
  Hence, substituting into \cref{pre-sum-est}, we obtain the bound
  %
  %
  \begin{equation}
    \label{series-bound}
  \begin{split}
    \sum_{j=1}^{n-1} \frac{1}{j^{3}(n-j)^{3}}
    \le \frac{3 \pi^{2}}{n^{3}}
    \le \frac{30}{n^{3}}
  \end{split}
  \end{equation}
  %
  completing the proof.
  %
  %
\end{proof}
%
%
%
\begin{proof}[Proof of \cref{lip-qwp}]
  We will prove the stronger estimate
\begin{equation*}
    \begin{split}
      \| A_{n}(f) - A_{n}(g) \|_{X_{s,b}}
      & \le c_{\psi}^{2n-1}  30^{n-1} 
      \delta^{(n-1)(b-1/2)} n^{-3} \| f-g \|_{H^{s} \times
      H^{s-2}}
      \\
      & \times \left( \| f \|_{H^{s} \times H^{s-2}}
      + \| g \|_{H^{s} \times H^{s-2}}  \right)^{n-1}
    \end{split}
    \end{equation*}
via an induction on $n$. For the base case $n=1$, we have
%
%
\begin{equation*}
\begin{split}
  \| A_{n}(f) - A_{n}(g) \|_{X_{s,b}}
  & = \| L_{n}(f) - L_{n}(g) \|_{X_{s,b}}
  \\
  & = \| L_{n}(f -g) \|_{X_{s,b}}
  \\
  & \le c_{\psi} \| f-g \|_{H^{s} \times H^{s-2}}
\end{split}
\end{equation*}
%
%
where the last step followed from \cref{ill-pos-lin}. For the inductive step,
we assume \cref{lip-qwp-eq} for $n' < n$. Note that
%
%
\begin{equation*}
\begin{split}
  \|A_{n}(f) - A_{n}(g)  \|_{X_{s,b}} = \| \sum_{\substack{j,
  k \in \mathbb{N} \\ j + k =n}}\left\{ N\left[
  A_{j}(f), A_{k}(f) \right] - N\left[
  A_{j}(g), A_{k}(g) \right]  \right\} \|_{X_{s,b}}.
\end{split}
\end{equation*}
%
%
Using the relation
%
%
\begin{equation*}
\begin{split}
  u_{1}u_{2} - v_{1}v_{2} = (u_{1} - v_{1})(u_{2} + v_{2}) - u_{1}v_{2} +
  u_{2} v_{1}
\end{split}
\end{equation*}
%
%
we obtain
%
%
\begin{equation*}
\begin{split}
\|A_{n}(f) - A_{n}(g)  \|_{X_{s,b}}
& = \| \sum_{\substack{j,
k \in \mathbb{N} \\ j + k =n}} \left \{N[A_{j}(f) -
A_{j}(g),  A_{k}(f) + A_{k}(g) ] \right .
\\
& \left . - N\left[ A_{j}(f), A_{k}(g) \right] + N\left[
A_{k}(f), A_{j}(g) \right] \right \}  \|_{X_{s,b}}
\\
& = \| \sum_{\substack{j,
k \in \mathbb{N} \\ j + k =n}}N[A_{j}(f) -
A_{j}(g), A_{k}(f) + A_{k}(g)]
 \|_{X_{s,b}}
\end{split}
\end{equation*}
%
%
where the last step follows from symmetry.
%
%
%
%
Applying the triangle inequality and \cref{ill-pos-bil}, we bound this by
%
%
\begin{equation*}
\begin{split}
  & c_{\psi} \delta^{b-1/2}\sum_{\substack{j,
  k \in \mathbb{N} \\ j + k =n}} \| A_{j}(f) -
  A_{j}(g) \|_{X_{s,b}} \| A_{k}(f) +
  A_{k}(g) \|_{X_{s,b}}
  \\
  & \le c_{\psi} \delta^{b-1/2}\sum_{\substack{j,
  k \in \mathbb{N} \\ j + k =n}} \| A_{j}(f) -
  A_{j}(g) \|_{X_{s,b}} ( \| A_{k}(f) \|_{X_{s,b}} +
  \| A_{k}(g) \|_{X_{s,b}} )
\end{split}
\end{equation*}
%
%
Applying our inductive assumption \cref{lip-qwp-eq} for $n' < n$ and
\cref{op-strong-bound}, this is bounded by
%
%
\begin{equation*}
\begin{split}
  & c_{\psi} \delta^{b-1/2} \sum_{\substack{j,
  k \in \mathbb{N} \\ j + k =n}} c_{\psi}^{2j-1}
  30^{j-1}  \delta^{(j-1)(b-1/2)} j^{-3}\| f -g \|_{H^{s} \times H^{s-2}}
  \\
  & \times \left(
  \| f \|_{H^{s} \times H^{s-2}} + \| g \|_{H^{s} \times H^{s-2}}
  \right)^{j-1} \times c_{\psi}^{2k-1}30^{k-1} 
  \delta^{(k-1)(b -1/2)} k^{-3} \left[ \| f \|_{H^{s} \times H^{s-2}}^{k}
  + \| g \|_{H^{s} \times H^{s-2}}^{k} \right]
  \\
  & \le \sum_{\substack{j,
  k \in \mathbb{N} \\ j + k =n}} c_{\psi}^{2j+2k-1}
  30^{j+k-2} \delta^{(j + k-1)(b-1/2)} \| f -g \|_{H^{s} \times H^{s-2}}
\\
& \times \left( \| f \|_{H^{s} \times H^{s-2}}
+ \| g \|_{H^{s} \times H^{s-2}} \right)^{j + k -1} (jk)^{-3}
\\
& = 
  c_{\psi}^{2n-1}
  30^{n-2} \delta^{(n-1)(b-1/2)} \| f -g \|_{H^{s} \times H^{s-2}}
\left( \| f \|_{H^{s} \times H^{s-2}}
+ \| g \|_{H^{s} \times H^{s-2}} \right)^{n-1} \sum_{j=1}^{n-1}
\frac{1}{j^{3}(n-j)^{3}}.
  \end{split}
\end{equation*}
%
Applying \cref{series-bound} completes the proof. 
%

\end{proof}
%
%
%
\begin{proof}[Proof of \cref{qwp-it-5}]
We first prove that $u_{k} \doteq
\sum_{n=1}^{k}A_{n}(f)$ is absolutely convergent in $X_{s,b}$ (and hence
converges in $X_{s,b}$ to an element of $X_{s,b}$). Applying \cref{induct-ineq} and
  \cref{delta-ill-pos} to obtain 
  %
  %
  \begin{equation*}
  \begin{split}
    \sum_{n=1}^{\infty} \| A_{n}(f) \|_{X_{s,b}}
    & \le \sum_{n=1}^{\infty}
       c_{\psi}^{2n-1} 30^{(n-1)}
    \delta^{(n-1)(b-1/2)} \| f
    \|_{H^{s} \times H^{s-2}}^{n}
    \\
    & \le  
    c_{\psi} \| f \|_{H^{s} \times H^{s-2}}
    \sum_{n=0}^{\infty}
    ( \delta^{b-1/2} c_{\psi}^{2} \| f \|_{H^{s} \times H^{s-2}})^{n}
    \\
    & \le c_{\psi} \| f \|_{H^{s} \times H^{s-2}}
    \sum_{n=0}^{\infty}
    \left (\frac{1}{2} \right )^{n}
    \\
    & < \infty.
  \end{split}
  \end{equation*}
  Next, recalling \cref{qwp-it-1}, we see that for $f \in H^{s} \times
  H^{s-2}$ and $\delta = \delta(r)$ sufficiently small, there exists a unique
  solution $u[f] \in X_{s,b} \cap C([-\delta, \delta], H^{s})$ to the integral
  equation $Tu = u$; that is $L(f) + N(u,u) -u =0$. Since $u_{k}$ is convergent
  in $X_{s,b}$, It remains to show that
  $u_{k}$ converges to this $u$ in $X_{s,b}$. 
Using this as motivation, we will now estimate the error of $u_{k} \doteq
  \sum_{n=1}^{k}A_{n}(f)$. First, note that
  %
  %
  \begin{equation*}
  \begin{split}
    L(f) + N(u_{k}, u_{k})
    & = L(f) + N\left[ \sum_{n=1}^{k} A_{n}(f), \sum_{n=1}^{k} A_{n}(f) \right]
    \\
    & = L(f) + \sum_{1 \le i \le k, 1 \le j \le k} N\left[ A_{i}(f),
    A_{j}(f) \right]
    \\
    & = u_{k} + \sum_{k < i+j \le 2k} N\left[ A_{i}(f), A_{j}(f) \right].
  \end{split}
  \end{equation*}
  %
  %
  From this, we obtain
  %
  %
  \begin{equation*}
  \begin{split}
    & \|L(f) + N(u_{k}, u_{k}) - u_{k}\|_{X_{s,b}}
    \\
    &= \|
    \sum_{k < i+j \le 2k} N\left[ A_{i}(f), A_{j}(f) \right]\|_{X_{s,b}}
    \\
    & \le \sum_{k < i+j \le 2k} \| N\left[ A_{i}(f), A_{j}(f)
    \right]\|_{X_{s,b}}.
  \end{split}
\end{equation*}
Hence, applying \cref{ill-pos-bil} and \cref{induct-ineq}, we bound
this above by
\begin{equation}
  \label{pre-delta-small}
  \begin{split}
    & c_{\psi} \delta^{b-1/2}
    \sum_{k < i+j \le 2k} \| A_{i}(f) \|_{X_{s,b}} \| A_{j}(f)
    \|_{X_{s,b}}
    \\
    & \le c_{\psi} \delta^{b-1/2}
    \sum_{k < i+j \le 2k} \left[ c_{\psi}^{2i-1}
    30^{i-1}\delta^{(i-1)(b-1/2)}\| f \|_{H^{s} \times
    H^{s-2}}^{i} \right] \left [c_{\psi}^{2j-1}
    30^{j-1}\delta^{(j-1)(b-1/2)}\| f \|_{H^{s} \times
    H^{s-2}}^{j} \right]
    \\
    & = \sum_{k < i+j \le 2k} c_{\psi}^{2i+2j -1}
    30^{i+j-2} \delta^{(i + j -1)(b-1/2)} \| f \|_{H^{s} \times
    H^{s-2}}^{i+j}
    \\
    & \le 2  \sum_{k < \ell \le 2k} c_{\psi}^{2 \ell - 1} 30^{\ell-2}
    \delta^{(\ell -1)(b - 1/2)} \| f \|_{H^{s} \times H^{s-2}}^{\ell}
    \\
    & = 2 c_{\psi} \| f \|_{H^{s} \times H^{s-2}} \sum_{k < \ell \le 2k}
    c_{\psi}^{2(\ell - 1)} 30^{\ell-1}
    \delta^{(\ell -1)(b - 1/2)} \| f \|_{H^{s} \times H^{s-2}}^{\ell-1}.
  \end{split}
  \end{equation}
  %
  Fixing $\delta$ sufficiently small such that
  %
  %
  \begin{equation}
    \label{delta-ill-pos}
  \begin{split}
    \| f \|_{H^{s} \times H^{s-2}} \le \frac{1}{60 c_{\psi}^{2}\delta^{b -1/2}}
  \end{split}
  \end{equation}
  %
  %
  we bound the right hand side of \cref{pre-delta-small} by 
  %
  %
  %
  \begin{equation*}
  \begin{split}
  2 c_{\psi} \| f \|_{H^{s} \times H^{s-2}} \sum_{k < \ell \le 2k}
  \left (\frac{1}{2} \right )^{\ell-1}.
\end{split}
  \end{equation*}
  %
  Since this quantity converges to $0$ as $k \to \infty$, we conclude that
  $u_{k} \to u$ in $X_{s,b}$.
    %
  %
which concludes the proof.
  %
\end{proof}
%
%
%
%
\newpage
\begin{appendices}
\section{An Alternative Way to Derive the $B_{4}$ Integral Equation}
\label{ssec:integral-form-deriv}
We will describe an alternative to the method of variation of parameters used in
the paper. 
%
%
\subsection{Reducing to a First Order ODE} 
\label{ssec:first-order-ode}
Taking the spatial Fourier transform of \cref{lin-mb} yields
the ivp
%
%
\begin{gather*}
  \wh{u_{tt}} + n^{4} \wh{u} = \wh{-u^{2}_{xx}},
  \\
  \wh{u}(n, 0) = \wh{u_{0}}(n), \quad \wh{u_{t}}(n, 0) = \wh{u_{1}}(n)
\end{gather*}
%
%
which we rewrite as 
%
%
\begin{gather}
  \label{eqn:lin-mb-ode}
  y_{tt} + n^{4}y = -f,
  \\
  y(n, 0) = y_{0}(n), \quad y_{t}(n, 0) = y_{1}(n)
\label{eqn:lin-mb-ode-init-data}
\end{gather}
%
%
where
%
%
\begin{gather*}
  \label{not-1}
  y = y(n, t) \doteq \wh{u}(n, t), \quad f = f(n, t) \doteq
  \wh{u^{2}_{xx}}(n,t),
  \\
  \label{not-2}
  y_{0}(n) \doteq \wh{u_{0}}(n), \quad y_{1}(n) = \wh{u_{1}}(n).
\end{gather*}
%
%
Viewing $n$ as fixed, we see that \cref{eqn:lin-mb-ode} is a second order
linear nonhomogeneous ODE\@. To solve it, we set 
%
%
\begin{equation*}
  \label{not-3}
\begin{split}
   & v_{1} = y, 
   \\
   & v_{2} = y_{t}
\end{split}
\end{equation*}
%
%
giving
%
%
\begin{equation*}
\begin{split}
  & v_{1}' = v_{2},
  \\
  & v_{2}' = -n^{4}v_{1} - f.
\end{split}
\end{equation*}
%
%
Therefore 
%
%
\begin{equation}
\begin{split}
\frac{d \vec v}{dt} = 
\begin{bmatrix}
0 & 1 \\
-n^{4} & 0
\end{bmatrix}
\begin{bmatrix}
  v_{1}\\
  v_{2}
\end{bmatrix}
-
\begin{bmatrix}
0\\
f
\end{bmatrix}
\doteq A \vec v - \vec f.
\end{split}
\label{eqn:first-order-ode-reduction}
\end{equation}
%
%
Hence, we have reduced solving the $2$nd order ODE \cref{eqn:lin-mb-ode} to
solving the first order ODE \cref{eqn:first-order-ode-reduction}. Multiplying
by the integrating factor $e^{-At}$ on both sides of
\cref{eqn:first-order-ode-reduction}, we obtain
%
%
\begin{equation*}
\begin{split}
  \frac{d}{dt}(e^{-At} \vec v) = -e^{-At} \vec f.
\end{split}
\end{equation*}
%
%
Integrating in time then gives
%
%
\begin{equation*}
\begin{split}
  e^{-At} \vec v(n, t) = \vec v(n, 0) - \int_{0}^{t}e^{-At'} \vec f dt'
\end{split}
\end{equation*}
%
%
or
%
%
\begin{equation}
  \label{ode-vec-soln}
\begin{split}
  \vec v(n, t) = e^{At} \vec v(n, 0) - \int_{0}^{t}e^{A(t - t')} \vec f dt'.
\end{split}
\end{equation}
%
%
We wish to compute $e^{At}$. It is easy to check that $A$ has eigenvalues
$\lambda = \pm in^{2}$, with corresponding eigenvectors 
%
%
\begin{equation*}
\begin{split}
\pm \begin{bmatrix}
1 \\
in^{2}
\end{bmatrix}.
\end{split}
\end{equation*}
%
%
Since $A$ is a square matrix and has no repeated
eigenvalues, it is diagonizable. More precisely,
%
%
%
%
\begin{equation*}
\begin{split}
  A = Q D Q^{-1} = 
  \begin{bmatrix}
  1 & 1
  \\
  in^{2} & -in^{2}
  \end{bmatrix}
  \begin{bmatrix}
    in^{2} & 0 
    \\
    0 & -in^{2}
  \end{bmatrix}
  \begin{bmatrix}
    \frac{1}{2} & \frac{1}{2i n^{2}} \\
    \frac{1}{2} & -\frac{1}{2i n^{2}}
  \end{bmatrix}
\end{split}
\end{equation*}
%
%
where the column vectors of $Q$ are comprised of the eigenvectors of $A$.
Recalling that 
%
%
\begin{equation*}
\begin{split}
  e^{At} \doteq \sum_{n=0}^{\infty} \frac{A^{n}}{n}t^{n}
\end{split}
\end{equation*}
%
%
it is easy to check that 
%
%
\begin{equation*}
\begin{split}
  e^{Q D Q^{-1}} = 
\begin{bmatrix}
  1 & 1
  \\
  in^{2} & -in^{2}
  \end{bmatrix}
  \begin{bmatrix}
    e^{in^{2}} & 0 
    \\
    0 & e^{-in^{2}}
  \end{bmatrix}
  \begin{bmatrix}
    \frac{1}{2} & \frac{1}{2i n^{2}} \\
    \frac{1}{2} & -\frac{1}{2i n^{2}}
  \end{bmatrix}
\end{split}
\end{equation*}
%
%
and 
\begin{equation}
  \label{matrix-expo}
\begin{split}
  e^{Q D Q^{-1}t}
  & = 
\begin{bmatrix}
  1 & 1
  \\
  in^{2} & -in^{2}
  \end{bmatrix}
  \begin{bmatrix}
    e^{in^{2}t} & 0 
    \\
    0 & e^{-in^{2}t}
  \end{bmatrix}
\begin{bmatrix}
    \frac{1}{2} & \frac{1}{2i n^{2}} \\
    \frac{1}{2} & -\frac{1}{2i n^{2}}
  \end{bmatrix}
  \\
  & =
  \begin{bmatrix}
    \frac{1}{2}(e^{in^{2}t} + e^{-in^{2}t}) & \frac{1}{2 i n^{2}} (e^{in^{2}t} -
    e^{-in^{2}t})    \\
    \frac{in^{2}}{2}(e^{in^{2}t} - e^{-in^{2}t}) & \frac{1}{2}(e^{in^{2}t} +
    e^{-in^{2}t})
  \end{bmatrix}.
\end{split}
\end{equation}
%
%
\begin{framed}
\begin{remark}
If a matrix has repeated eigenvalues, it may no longer be diagonizable. However,
any matrix can be written in Jordan canonical form. The above computations
become slightly more complicated in this case. The important observation is that
writing a matrix in Jordan canonical (or, ideally, diagonal) form allows us to
easily compute its exponential. 
\label{rem:jordan-form}
\end{remark}
\end{framed}
%
%
%
%
\begin{framed}
\begin{remark}
\label{rem:simpler-comp}
  Since $A$ is a particularly simple matrix, i.e. $A^{2} = -n^{4} I$, one can
  compute its exponential easily without the use of diagonalization (this is
  not true in general). From the above equality, we obtain $A^{n} =
  (-1)^{n/2} n^{2n} I$ for even $n$, and so
  %
  %
  \begin{equation*}
  \begin{split}
    e^{At}
    & = \sum_{n=0}^{\infty} \frac{(-1)^{n}n^{4n}t^{2n}}{(2n)!}I + A
    \sum_{n=0}^{\infty} \frac{(-1)^{n} n^{4n} t^{2n + 1}}{(2n + 1)!} I 
    \\
    & = \cos n^{2}t \, I - \frac{\sin n^{2}t}{n^{2}}A
    \\
    & = 
    \begin{bmatrix}
      \cos n^{2}t &  -\frac{\sin n^{2}t}{n^{2}}
      \\
      - n^{2} \sin n^{2}t & \cos n^{2}t
    \end{bmatrix}
    \\
    & = \text{rhs of} \ \cref{matrix-expo}
  \end{split}
  \end{equation*}
\end{remark}
\end{framed}
%
%
Substituting \cref{matrix-expo} into \cref{ode-vec-soln} and recalling our notation, we obtain
%
%
\begin{equation*}
\begin{split}
  v_{1}(n, t) = (e^{in^{2}t} + e^{-in^{2}t})v_{1}(n, 0) + \frac{e^{in^{2}} -
  e^{-in^{2}t}}{2 i n^{2}} v_{1}(n, 0) - \int_{0}^{t} \frac{e^{in^{2}(t - t')} -
  e^{-in^{2}(t-t')}}{2 i n^{2}} f dt'
\end{split}
\end{equation*}
%
%
or
\begin{equation*}
\begin{split}
  y(n, t) = (e^{in^{2}t} + e^{-in^{2}t})y_{0} + \frac{e^{in^{2}} -
  e^{-in^{2}t}}{2 i n^{2}} y_{1} - \int_{0}^{t} \frac{e^{in^{2}(t - t')} -
  e^{-in^{2}(t-t')}}{2 i n^{2}} f dt'
\end{split}
\end{equation*}
or
\begin{equation*}
\begin{split}
  \wh{u}(n, t) = (e^{in^{2}t} + e^{-in^{2}t})\wh{u_0} + \frac{e^{in^{2}} -
  e^{-in^{2}t}}{2 i n^{2}} \wh{u_1} - \int_{0}^{t} \frac{e^{in^{2}(t - t')} -
  e^{-in^{2}(t-t')}}{2 i n^{2}} \wh{(u^{2})_{xx}} dt'.
\end{split}
\end{equation*}
%
Taking the inverse Fourier transform then yields \cref{eqn:integral-form}, as
desired.
\section{Farah's Approach to Estimating the Integral $B_{4}$ Equation} 
\label{ssec:farah-approach-reduction}
Recall that the unique solution to ivp
\cref{four-trans-mb}-\cref{four-trans-mb-data} is given by
%
%
\begin{equation*}
\begin{split}
\wh{u}(n, t) = \wh{u_{0}}(n) \frac{e^{in^{2}t} + e^{-in^{2}t}}{2} +
  \wh{u_{1}}(n)\frac{e^{in^{2}t} - e^{-in^{2}t}}{2 i n^{2}} +
  \int_{0}^{t}\frac{e^{in^{2}(t-t')}-e^{-in^{2}(t-t')}}{2in^{2}}
  \wh{(u^{2})_{xx}} dt'.
\end{split}
\end{equation*}
%
%
Taking the inverse Fourier transform then gives
%
\begin{equation}
  \begin{split}
    u(x,t) = R_{t}u_{0} + S_{t}u_{1} + \int_{0}^{t} S_{t-t'}
    (u^{2})_{xx} dt'.
  \end{split}
  \label{eqn:integral-formb}
\end{equation}
%
%
Hence, we have rewritten the $B_{4}$ ivp
\cref{eqn:mb-2}-\cref{eqn:mb-init-data-2} in integral form, which we will now
localize in time. 
Let $\psi(t)$ be a cutoff function symmetric about the 
origin such that $\psi(t) = 1$ for $|t| \le 1$ and $\text{supp} \, \psi 
= [-2, 2 ]$. Define $\psi_{\delta}(t) = \psi(t/\delta)$, $ 0 < \delta \le 1$.
Motivated by $\cref{eqn:integral-formb}$, we consider the equation
%
%
\begin{equation}
  \begin{split}
    u(x,t)
    & = \psi(t) R_{t} u_{0} + \psi(t) S_{t}u_{1} +
    \psi_{\delta}(t) \int_{0}^{t} S_{t-t'}
    (u^{2})_{xx} dt'
    \\
    & \doteq Tu
  \end{split}
  \label{localized-int-eqn}
\end{equation}
%
%
%
%
%
%
%
The computations Farah presents for estimating
$\psi(t) R_{t}u_{0}$ and $\psi(t) S_{t}u_{1}$ are identical
to those in the paper. Therefore, we focus only on Farah's approach for
estimating the nonlinear
term.
\subsection{Estimate for $\psi_{\delta}(t) \int_{0}^{t} S_{t-t'} (u^{2})_{xx} dt'$.}
\label{ssec:non-lin-term}
We define the spatial Fourier transform by 
%
%
\begin{equation*}
\begin{split}
  \tilde{f}(n, t) = \int_{\ci} e^{-inx}f(x,t) dx
\end{split}
\end{equation*}
%
%
and the spacetime Fourier transform by
\begin{equation*}
\begin{split}
  \wh{f}(n, \tau) = \int_{\rr} \int_{\ci} e^{-inx-it\tau}f(x,t) dx dt.
\end{split}
\end{equation*}
%
%
Let $f(x,t) \doteq \psi_{\delta}(t) \int_{0}^{t} S_{t-t'} (u^{2})_{xx} dt'$. 
Then
%
%
\begin{equation}
  \begin{split}
    \wt{f}(n, t)
    & = \psi_{\delta} \int_{0}^{t}-n^{2}\wt{u^{2}}(n, t') \left[
    \frac{e^{in^{2}(t-t')} - e^{-in^{2}(t-t')}}{2 i n^{2}}
    \right] dt'
    \\
    & = - \frac{1}{2i} e^{in^{2}t} \int_{0}^{t} \psi_{\delta}(t) \wt{u^{2}}(n, t')
    e^{-in^{2}t'} dt' + 
    \frac{1}{2i} e^{-in^{2}t} \int_{0}^{t} \psi_{\delta}(t) \wt{u^{2}}(n, t')
    e^{in^{2}t'} dt' \\
    & \doteq - e^{in^{2}t} \wt{w_1}(n, t) + e^{-in^{2}t} \wt{w_2}(n, t)
  \end{split}
  \label{space-four-trans}
\end{equation}
%
where
%
%
\begin{gather*}
  w_{1}(x,t) = \frac{1}{4 i \pi} \sum_{n \in \zz} e^{inx} \left[ \int_{0}^{t}
  \psi_{\delta}(t) \wt{u^{2}}(n, t') e^{-in^{2}t'}
  dt'\right],
  \\
  w_{2}(x,t) = \frac{1}{4 i\pi} \sum_{n \in \zz} e^{inx} \left[ \int_{0}^{t} \psi_{\delta}(t) \wt{u^{2}}(n, t') e^{in^{2}t'} dt'
 \right].
\end{gather*}
%
%
%
Notice that \cref{space-four-trans} is a \emph{global} relation in $t$.
Hence, taking its time Fourier transform gives
%
%
\begin{equation}
  \label{full-fourier-trans-exp}
\begin{split}
  \wh{f}(n, \tau) = -\wh{w_{1}}(n, \tau - n^{2}) + \wh{w_{2}}(n, \tau +
  n^{2}).
\end{split}
\end{equation}
%
%
Therefore, using the definition of the $X_{s,b}$ spaces, and applying
\cref{square-ineq} gives 
%
%
\begin{equation*}
\begin{split}
  \| f \|_{X_{s,b}}^{2}
  & = \sum_{n \in \zz} (1 + |n|)^{2s} \int_{\rr} (1 + |
  | \tau | - n^{2} |)^{2b} | -\wh{w_{1}}(n, \tau - n^{2}) + \wh{w_{2}}(n, \tau +
  n^{2}) |^{2} d \tau
  \\
  & \le 4 \sum_{n \in \zz} (1 + |n|)^{2s} \int_{\rr} (1 + |
  | \tau | - n^{2} |)^{2b} | \wh{w_{1}}(n, \tau - n^{2}) d \tau
  \\
  & + 4 \sum_{n \in \zz} (1 + |n|)^{2s} \int_{\rr} (1 + |
  | \tau | - n^{2} |)^{2b} | \wh{w_{2}}(n, \tau + n^{2}) d \tau.
\end{split}
\end{equation*}
%
%
Applying a change of variable implies
%
%
%
%
\begin{equation}
\begin{split}
  \| f \|_{X_{s,b}}^{2}
  & \le 4 \sum_{n \in \zz} (1 + |n|)^{2s} \int_{\rr} (1 + |
  | \tau + n^{2} | - n^{2} |)^{2b} | \wh{w_{1}}(n, \tau) |^2 d \tau
  \\
  & + 4 \sum_{n \in \zz} (1 + |n|)^{2s} \int_{\rr} (1 + |
  | \tau - n^{2} | - n^{2} |)^{2b} | \wh{w_{2}}(n, \tau )|^2 d \tau.
\end{split}
\label{comp-pre-lemma}
\end{equation}
%
%
%
Applying  to \cref{comp-pre-lemma} yields
%
%
\begin{equation}
  \label{pre-smoothing-lem}
\begin{split}
\| f \|_{X_{s,b}}^{2}
  & \le 4 \sum_{j=1}^{2}  \sum_{n \in \zz} (1 + |n|)^{2s} \int_{\rr} (1 + |
  \tau|)^{2b} | \wh{w_{j}}(n, \tau)|^2 d \tau
  \\
  & = 4 \sum_{j=1}^{2} \sum_{n \in \zz} (1 + |n|)^{2s} \|\wt{w_{j}}(n, t)
  \|^{2}_{H_{t}^{b}}
  \\
  & = \sum_{n \in \zz} \| \psi_{\delta}(t) \int_{0}^{t} \wt{u^2}(n, t')
  e^{in^{2}t'}dt'  \|_{H_{t}^{b}}
  + 
  \sum_{n \in \zz} \| \psi_{\delta}(t) \int_{0}^{t} \wt{u^2}(n, t')
  e^{-in^{2}t'}dt'  \|_{H_{t}^{b}}.
\end{split}
\end{equation}
%
We now need the following, whose proof is provided in~\cite{Ginibre:1996fk} and
the appendix.
%
%
%%%%%%%%%%%%%%%%%%%%%%%%%%%%%%%%%%%%%%%%%%%%%%%%%%%%%
%
%
%                Lemma to Reduce to Bilinear Est Form
%
%
%%%%%%%%%%%%%%%%%%%%%%%%%%%%%%%%%%%%%%%%%%%%%%%%%%%%%
%
%
\begin{lemma}[Lemma 2.2 in~\cite{Farah:2009uq}]
Let $-1/2 < b' \le 0 \le b \le b' +1$ and $\delta \le 1$. Then
%
%
\begin{equation*}
\begin{split}
  \| \psi_{\delta}(t) \int_{0}^{t} g(t') dt' \|_{H^{b}_{t}} \le c_{\psi, b'}\delta^{1-(b - b')} \| g
  \|_{H_{t}^{b'}}
\end{split}
\end{equation*}
%
%
where $c_{\psi, b'}$ is a constant depending only on $\psi$ and $b'$.
\label{lem:pre-bilin-est}
\end{lemma}
%
%
%
%
%
%
Applying the lemma with $b' = -a$ we bound the right hand side
of \cref{pre-smoothing-lem} by
%
%
\begin{equation*}
\begin{split}
  & c_{\psi, a} \delta^{1- (a + b)}\left[ \sum_{n \in \zz} (1 + |n|)^{s} \| \wt{u^{2}}(n, t')
  e^{in^{2}t'} \|_{H_{t}^{-a}}  +
  \sum_{n \in \zz} (1 + |n|)^{s} \| \wt{u^{2}}(n, t')
  e^{-in^{2}t'} \|_{H_{t}^{-a}} \right]
  \\
  & = c_{\psi, a} \delta^{1-(a + b)}
  \left [ \sum_{n \in \zz} (1 + |n|)^{s} \int_{\rr} (1 + | \tau
  |)^{-a} |\wh{u^{2}}(n, \tau - n^{2})|^{2} d \tau  \right .
  \\
  & + \left .
  \sum_{n \in \zz} (1 + |n|)^{s} \int_{\rr} (1 + | \tau
  |)^{-a} |\wh{u^{2}}(n, \tau + n^{2})|^{2} d \tau  \right ]
  \\
  & = c_{\psi, a} \delta^{1- (a + b)}
  \left [ \sum_{n \in \zz} (1 + |n|)^{s} \int_{\rr} (1 + | \tau
  + n^{2}
  |)^{-a} |\wh{u^{2}}(n, \tau )|^{2} d \tau  \right .
  \\
  & + \left . 
  \sum_{n \in \zz} (1 + |n|)^{s} \int_{\rr} (1 + | \tau
  - n^{2} |)^{-a} |\wh{u^{2}}(n, \tau )|^{2} d \tau  \right ]
\end{split}
\end{equation*}
%
%
which by \cref{eqn:norm-key-ineq} is bounded by 
%
%
%
\begin{equation*}
\begin{split}
  & 
  2 c_{\psi, a} \delta^{1- (a + b)}
\sum_{n \in \zz} (1 + |n|)^{s} \int_{\rr} (1 + | |\tau|
  - n^{2} |)^{-a} 
|\wh{u^{2}}(n, \tau)|^{2} d \tau 
  \\
  & = 2 c_{\psi, a} \delta^{1- (a + b)}
\|u^{2} \|_{X_{s,-a}}^{2}.
\end{split}
\end{equation*}
%
%
Substituting back in for $f$ and taking square roots gives
%
%
\begin{equation}
\begin{split}
  \|\psi_{\delta}(t) \int_{0}^{t} S_{t-t'} (u^{2})_{xx} dt'\|_{X_{s,b}} \le
  c_{\psi, a} \delta^{[1- (a + b)]/2}\| u^{2}
  \|_{X_{s,-a}}, \qquad b \le 1-a, \quad 0 \le a < 1/2.
\end{split}
\label{eqn:non-lin-bound}
\end{equation}
%
%
To bound the right hand side, we now require a crucial bilinear
estimate.
%
%
%
%
%
%
Applying \cref{prop:bilinear-est} to \cref{eqn:non-lin-bound}, we conclude that
for $-1/4 < s < 0$, $0 < \ee < 1/2 + 2s$, $b= 1 + 2s - 2 \ee$, and $s \ge 0$,
$\ee = 1/4$, $1/2 < b \le 1$, we have 
%
%
\begin{equation}
\begin{split}
  \|\psi_{\delta}(t) \int_{0}^{t} S_{t-t'} (u^{2})_{xx} dt'\|_{X_{s,b}} \le
  c \delta^{\ee} \| u \|^2_{X_{s,b}}, 
\end{split}
\label{eqn:nonlinear-term-bound}
\end{equation}
%
%
where $c = c_{\psi, b}$ (for \cref{eqn:nonlinear-term-bound} to hold, the value
of $a$ depends on the value of $b$; see below).
%
%
\begin{framed}
\begin{remark}
We obtain the restrictions on $s$, $b$, and $c$ as follows. If $-1/4 < s \le 0$, let
$0 < \epsilon < 1/2 + 2s$ and set
$a = -2s$, $b = 1 + 2s - 2 \epsilon$. Then $0 \le a < 1/2$ and $b \le 1-a$,
so \cref{eqn:non-lin-bound} holds. Furthermore, \cref{prop:bilinear-est} holds,
and substituting for $a,b$ we get $\delta^{[1-(a + b)]/2} = \delta^{\ee}$.
\\
If $ s \ge 0$, choose $1/4 < a < 1/2$, and let $0 < \ee < 1/2 -a$ and $b = 1-a
-\ee$. Then
\cref{eqn:non-lin-bound} holds. Furthermore, 
\cref{prop:bilinear-est} holds, and substituting for $a, b$ we get $\delta^{\left[
1- (a + b) \right]/2} = \delta^{\ee} < \delta^{1/4}$. 
\end{remark}
\end{framed}
%
%
%
%%%%%%%%%%%%%%%%%%%%%%%%%%%%%%%%%%%%%%%%%%%%%%%%%%%%%
%
%
%                Proof of miscellaneous lemmas
%
%
%%%%%%%%%%%%%%%%%%%%%%%%%%%%%%%%%%%%%%%%%%%%%%%%%%%%%
%
%
\section{Proofs of Lemmas and Estimates} 
\label{sec:pfs-lems-est}
%
\begin{proof}[Proof of \cref{lem:embedding}] 
%
%
\begin{equation*}
\begin{split}
  \| u(t) \|_{H^{s}}
  & = \left( \sum_{n} (1 + n^{2})^{s} |
  \wh{u}(n, t) |^{2} \right)^{1/2}
  \\
  & = \left( \sum_{n} (1 + n^{2})^{s} | \frac{1}{2\pi}
  \int_{\rr}e^{i \tau t} \wh{u}(n, \tau) d \tau |^{2} \right)^{1/2}
  \\
  & \lesssim \left[ \sum_{n} (1 + n^{2})^{s} \left( \int_{\rr} |
  \wh{u}(n, \tau)
  | d \tau \right)^{2} \right]^{1/2}
  \\
  & = \left[ \sum_{n} (1 + n^{2})^{s} \left( \int_{\rr} |
  \wh{u}(n, \tau)
  | (1 + | | \tau | - n^{2} |)^{b} (1 + | | \tau | - n^{2} |)^{-b} d \tau
  \right)^{2} \right]^{1/2}.
\end{split}
\end{equation*}
%
%
Cauchy-Schwartz in the $\tau$ variable then gives the bound
%
%
\begin{equation*}
\begin{split}
  & \left \{\sum_{n} (1 + n^{2})^{s} \left[ \left( \int_{\rr} (1 + | |
  \tau | - n^{2}
  |) | \wh{u}(n, \tau) |^{2} d \tau  \right)^{1/2} 
 \left( \int_{\rr} (1 + | |
  \tau | - n^{2}
  |)^{-1}  d \tau  \right)^{1/2}
  \right]^{2}\right \}^{1/2}
  \\
  & \le c_{b} \| u \|_{X_{s,b}}
\end{split}
\end{equation*}
%
where the last step follows from the estimate
%
%
%
%
\begin{equation*}
\begin{split}
\int_{\rr} (1 + | |
  \tau | - n^{2}
  |)^{-1}  d \tau 
  & = \int_{\tau \le 0} (1 + 
  |\tau  + n^{2}
  |)^{-1}  d \tau  + \int_{\tau \ge 0} (1 + | 
  \tau - n^{2}
  |)^{-1}  d \tau
  \\
  & = c_{b}.
\end{split}
\end{equation*}
%
%
completing the proof. 
\end{proof}
%
\begin{proof}[Proof of \cref{eqn:norm-key-ineq}]
By the reverse triangle inequality, we have
%
%
\begin{equation*}
\begin{split}
  | \tau | = | \tau + n^{2} - n^{2} | \ge | | \tau + n^{2} | - n^{2} |.
\end{split}
\end{equation*}
%
%
Furthermore, if $\tau - n^{2} < 0$, then
%
%
\begin{equation*}
\begin{split}
  | | \tau - n^{2} | - n^{2} | = | n^{2} - \tau - n^{2} | = | \tau |
\end{split}
\end{equation*}
%
%
while if $\tau - n^{2} > 0$, then
%
%
\begin{equation*}
\begin{split}
  | | \tau - n^{2} | - n^{2} | \le n^{2} \le \tau = |\tau|
\end{split}
\end{equation*}
%
%
completing the proof. 
\end{proof}
%
%
\begin{proof}[Proof of \cref{lem:schwartz-mult}]
Note that
%
%
\begin{equation*}
	\begin{split}
		\wh{\psi f}\left( n, \tau \right)
		& = \wh{\psi}(\cdot) * \wh{f}(n,
		\cdot)(\tau)
		= \int_\rr \wh{\psi}(\tau_1) \wh{f} \left( n, \tau - \tau_1 \right) 
		d\tau_1
	\end{split}
\end{equation*}
%
%
and hence
%
%
\begin{equation}
	\label{n1b}
	\begin{split}
		\|\psi f\|_{X^s} 
		& = \left( \sum_{n \in \zz} \left (1 + |n| \right )^{2s} \int_\rr \left( 1 + | \tau -
    n^{m} | \right)^{2b} | \int_\rr \wh{\psi}(\tau_1) \wh{f}\left( n, \tau -
		\tau_1
		\right)  d \tau_1 d \tau |^2 \right)^{1/2}
		\\
		& \le \left( \sum_{n \in \zz} \left (1 + |n| \right )^{2s} \int_\rr \left( 1 + | \tau -
		n^{m}
		|
    \right)^{2b} \left( \int_\rr |\wh{\psi}\left( \tau_1 \right) | |\wh{f}\left( n,
		\tau - \tau_1
		\right) |  d \tau_1 d \tau \right)^2 \right)^{1/2}.
	\end{split}
\end{equation}
%
%
Using the relation
%
%
\begin{equation*}
	\begin{split}
		1 + | \tau - n^{m} |
    & = 1 + | \tau - \tau_1 + \tau_{1} - n^{m} |
		\\
		& \le 1 + | \tau_1 | + | \tau - \tau_1 - n^{m} |
		\\
		& \le \left( 1 + | \tau_1 | \right)\left( 1 + | \tau - \tau_1 -
		n^{m} | \right)
	\end{split}
\end{equation*}
%
%
we obtain
%
%
\begin{equation*}
	\begin{split}
		\cref{n1b}
		& \le \left( \sum_{n \in \zz} \left (1 + |n| \right )^{2s} \right.
		\\
		& \times \left . \int_\rr \left(
		\int_\rr \left( 1 + | \tau_1 | \right)^{b} | \wh{\psi}(\tau_1) |
		\left( 1 + | \tau - \tau_1 - n^{m} | \right)^{b} \wh{f}\left( n, \tau
		- \tau_1
		\right)d \tau_1
		\right)^2 d \tau \right)^{1/2}
	\end{split}
\end{equation*}
%
%
which by Minkowski's inequality is bounded by
%
%
\begin{equation}
	\label{n2b}
	\begin{split}
		& \left( \sum_{n \in \zz} \left (1 + |n| \right )^{2s}  \right.
		\\
		& \times \left. \left( \int_\rr \left[ \int_\rr
    \left( 1 + | \tau_{1} | \right)^{2b} | \wh{\psi}(\tau_1) |^2 \left( 1 + |
		\tau - \tau_1 - n^{m} |
    \right)^{2b} | \wh{f}\left( n, \tau - \tau_1 \right) |^2 d \tau 
    \right]^{1/2} d \tau_{1} \right)^2 \right)^{1/2}.
	\end{split}
\end{equation}
%
%
Using the change of variable $\tau - \tau_1 = \lambda$ gives
%
%
\begin{equation}
  \label{uh}
	\begin{split}
		\cref{n2b}
		& = \left( \sum_{n \in \zz} \left (1 + |n| \right )^{2s}\right.
		\\
		& \times \left.  \left( \int_\rr \left[
    \int_\rr \left( 1 + | \tau_1 | \right)^{2b} | \wh{\psi}\left( \tau_1
    \right) |^2 \left( 1 + | \lambda - n^{m} | \right)^{2b} | \wh{f} \left( n,
		\lambda
    \right)|^2 d \lambda \right]^{1/2} d \tau_{1} \right)^2 \right)^{1/2}
		\\
		& =  \left( \sum_{n \in \zz} \left (1 + |n| \right )^{2s} \right.
		\\
		& \times \left. \left( \int_\rr \left( 1 + |
		\tau_1 |
		\right)^{b} | \wh{\psi}(\tau_1) | d \tau_1 \left[ \int_\rr \left( 1 + |
		\lambda - n^{m} |
		\right) | \wh{f}\left( n, \lambda \right) |^2 d \lambda \right]^{1/2}
		\right)^2 \right)^{1/2}.
  \end{split}
\end{equation}
%
%
Applying a computation similar to
\cref{cutoff-scaling}-\cref{cutoff-scaling-p}, the right hand side of \cref{uh} simplifies to
\begin{equation*}
  \begin{split}
    & \left( \sum_{n \in \zz} \left (1 + |n| \right )^{2s} \left( c_{\psi}
    \delta^{-b} \left[ \int_\rr
    \left( 1 + | \lambda - n^{m} | \right)^{2b} | \wh{f}\left( n, \lambda
    \right) |^2 d \lambda
    \right]^{1/2} \right)^{2} \right)^{1/2}
    \\
    & = c_{\psi} \delta^{-b} \|f\|_{X_{s,b}}
	\end{split}
\end{equation*}
%
concluding the proof. 
\end{proof}
%
\begin{proof}[Proof of \cref{lem:calc}]
%
By the change of variable $x \mapsto x + (\alpha + \beta)/2$, we have
%
%
\begin{equation*}
	\begin{split}
    \int_{\rr} \frac{1}{\langle x - \alpha \rangle^{p} \langle  x -
    \beta
    \rangle^{q}}d x.
    & = \int_{\rr} \frac{1}{\langle x - (\alpha - \beta)/2  \rangle^{p}
    \langle  x + (\alpha - \beta)/2 \rangle^{q}} d x.
	\end{split}
\end{equation*}
%
%
Hence, we seek to bound
%
%
%
\begin{equation*}
\begin{split}
  \int_{\rr} \frac{1}{\langle a - x \rangle ^{p} \langle a + x \rangle
  ^{q}} d x
\end{split}
\end{equation*}
%
which for $a =0$ reduces to 
%
%
\begin{equation*}
\begin{split}
  \int_{\rr} \frac{1}{\langle x \rangle ^{p+q}} d x 
  & = 2 \int_{0}^{\infty} \frac{1}{(1 + x)^{p+q}} d x
  \\
  & = \frac{2}{p+q -1}
  \\
  & = \frac{2}{(p+q -1)\langle a \rangle}.
\end{split}
\end{equation*}
%
%
By symmetry, we now assume without loss of generality that $a > 0$ and split 
%
%
\begin{equation*}
\begin{split}
\int_{\rr} \frac{1}{\langle a + x \rangle ^{p} \langle a - x \rangle
  ^{q}} d x
  & = \int_{-2a}^{2a}
  \frac{1}{\langle a + x \rangle ^{p} \langle a - x \rangle
  ^{q}} d x
  \\
  & + \int_{| x | \ge 2a} 
\frac{1}{\langle a + x \rangle ^{p} \langle a - x \rangle
  ^{q}} d x
  \\
  & = I + II.
\end{split}
\end{equation*}
%
%
If $p=1$ and $q=1$, then 
%
%
\begin{equation*}
\begin{split}
  I
  & \le \sup_{-2a \le x \le 2a} \frac{1}{\langle a - x \rangle
} \int_{-2a}^{2a} \frac{1}{\langle a + x \rangle} d x
  \\
  & = \frac{1}{\langle a \rangle} \int_{-2a}^{2a} \frac{1}{(1 + | a -
  x
  |)} d x
  \\
  & = \frac{4}{\langle a \rangle} \int_{0}^{a} \frac{1}{(1 + a -
  x)} d x.
\end{split}
\end{equation*}
%
%
Integrating, we obtain
%
%
\begin{equation*}
 I
 \le 
 \frac{4 \log \langle a \rangle}{\langle a \rangle}, \qquad p =1, \ q =1.
\end{equation*}
Otherwise, assume that $q \neq 1$. Then
\begin{equation*}
\begin{split}
  I
  & \le \sup_{-2a \le x \le 2a} \frac{1}{\langle a + x \rangle
  ^{p}} \int_{-2a}^{2a} \frac{1}{\langle a - x \rangle ^{q}} d x
  \\
  & = \frac{1}{\langle a \rangle ^{p}} \int_{-2a}^{2a} \frac{1}{(1 + | a -
  x
  |)^{q}} d x
  \\
  & = \frac{4}{\langle a \rangle ^{p}} \int_{0}^{a} \frac{1}{(1 + a -
  x)^{q}} d x.
\end{split}
\end{equation*}
Evaluating the integral, we obtain
\begin{equation*}
  I \le \frac{4}{|q-1| \langle a \rangle ^{p +q -1}}, \qquad q \neq 1.
\end{equation*}
%
%
A similar computation yields
\begin{equation*}
  I \le \frac{4}{|q-1| \langle a \rangle ^{p +q -1}}, \qquad p \neq 1.
\end{equation*}
%
%
Also
%
%
\begin{equation*}
\begin{split}
  II 
  & \simeq \int_{x \ge 2a} \frac{1}{\langle a - x \rangle ^{p} \langle a
  + x \rangle ^{q} d x}
  \\ 
  & = \int_{x \ge 2a} \frac{1}{(1 + x - a)^{p} (1 + x +
  a)^{q}} d x
  \\
  & \le \int_{x \ge 2a} \frac{1}{(1 + x -a)^{p+q}} d x
  \\
  & = \frac{1}{[(p + q)-1] \langle a \rangle ^{p+q -1}}, \qquad p + q > 1.
\end{split}
\end{equation*}
%
%
Collecting our estimates for $I$ and $II$ we see that for 
$p, q > 0$ such that $p +q >1$, and $r =\min\left\{p, q, p+q-1
 \right\}$, we have 
%
\begin{align*}
  & \int_{\rr} \frac{1}{\langle a - x \rangle ^{p} \langle a + x \rangle
  ^{q}} d x
  \le \frac{c_{p,q}}{\langle a \rangle ^{r}}, \qquad a = 0 \ \text{or} \
  p \neq 1 \ \text{or} \ q \neq 1
  \\
  & \int_{\rr} \frac{1}{\langle a - x \rangle  \langle a + x \rangle
} d x
  \le  \frac{4 \log \langle a \rangle}{\langle a \rangle}, \qquad a > 0.
  \label{est-2}
\end{align*}
By symmetry, the second inequality also holds for $a < 0$. Substituting $(\alpha -
\beta)$ for $a$ completes the proof.
\end{proof}
%
%
\begin{proof}[Proof of \cref{lem:calc-lower-bound}]
By symmetry, we may assume $ a \ge 0$, $x \ge 0$ without loss of generality.
Then
%
%
\begin{equation*}
\begin{split}
f(x) = 
\begin{cases}
  f_{1}(x) \doteq (1 + x-a)(1 + x + a), \quad & x \ge a \\
  f_{2}(x) \doteq (1 + a -x)(1 + x + a), \quad & 0 \le x \le a.
\end{cases}
\end{split}
\end{equation*}
%
%
Since $f_{1}'(x) = 2 + 2x$, $f_{1}(x)$ has a critical point at $x=-1$. Since
$f'(x) < 0$ for $x <-1$, and $f'(x) > 0$ for $x > -1$, $x=-1$ is the absolute
minimum of $f_{1}(x)$ for all $x \in \rr$. Since $f$ is increasing for $x > -1$,
it follows that $x=a$ is the absolute minimum of $f_{1}$ in the region $x \ge
a$. Furthermore, we have
%
%
\begin{equation*}
\begin{split}
  f_{1}(a) = 1 + 2 a.
\end{split}
\end{equation*}
%
%
Next, note that $f_{2}'(x) = -2x$, which gives the critical point $x = 0$. Since
$f_{2}' < 0$ for $x>0$ and $f_{2}' > 0$ for $x < 0$, we see that $x=0$ is
the absolute maximum of $f_{2}'$ for  $x \in \rr$. Since $f_{2}$ is decreasing
for $x > 0$, it follows that $x = a$
is the absolute minimum of $f_{2}$ in the region $0 \le x \le a$. Furthermore
%
%
\begin{equation*}
\begin{split}
  f_{2}(a) = 1 + 2 a
\end{split}
\end{equation*}
%
%
Hence, 
%
%
\begin{equation*}
\begin{split}
f(x) \ge 1 + 2a \ge 1 + a
\end{split}
\end{equation*}
%
%
completing the proof. 
\end{proof}
%
\begin{proof}[Proof of \cref{lem:sum-estimate}]
Let $\alpha = \alpha(n, \mu), \beta = \beta(n, \mu)$ be roots of the equation
%
%
\begin{equation*}
\begin{split}
  \mu - n_{1}(n - n_{1}) = 0.
\end{split}
\end{equation*}
%
%
Then 
%
%
\begin{equation*}
\begin{split}
\sup_{(n,\tau)\in \zz \times \rr}\sum_{n_1\in \zz}\frac{1}{(1+|\tau-
n_1(n-n_1)|)^{\gamma}}
& = \sup_{(\alpha, \beta )\in \mathbb{C}^{2}}\sum_{n_1\in \zz}\frac{1}{(1 +
|(n_{1} - \alpha)(n_{1} - \beta)|)^{\gamma}}
\end{split}
\end{equation*}
%
%
Since $| n_{1} - \alpha | \ge | n_{1} - \mathfrak{Re}(\alpha) |$, we may assume
without loss of generality that $(\alpha, \beta) \in \rr^{2}$. Next, note that
for fixed $(\alpha, \beta) \in \rr^{2}$, there are at most $10$ integers
$n_{1}$ such that
%
%
\begin{equation*}
\begin{split}
  | n_{1} - \alpha | \le 2 \quad \text{or} \quad | n_{1} - \beta | \le 2.
\end{split}
\end{equation*}
%
%
%
%
\begin{framed}
\begin{remark}
  We consider the equivalent problem of finding integers $n_{1}$ satisfying
  \begin{equation*}
\begin{split}
  | n_{1} - k +c | \le 2 \quad \text{or} \quad | n_{1} - m + d | \le 2, \quad k,
  m \in \zz, \ 0 \le c < 1, \ 0 \le d < 1.
\end{split}
\end{equation*}
They are
%
%
\begin{equation*}
\begin{split}
  &  k, \ k+1, \ k-1,
  \ k-2,  \ k+2 \ (\text{if and only if} \ c=0)
  \\
  &  m,  \ m+1, \ m-1,
  \ m-2,  \ m+2 \ (\text{if and only if} \ d=0).
\end{split}
\end{equation*}
%
%
%
%
This is a list of at most $10$ integers.
\label{rem:pf-counting}
\end{remark}
\end{framed}
%
Therefore
%
%
%
%
\begin{equation*}
\begin{split}
\sup_{(\alpha, \beta )\in \mathbb{R}^{2}}\sum_{n_1\in \zz}\frac{1}{(1 +
|(n_{1} - \alpha)(n_{1} - \beta)|)^{\gamma}} \le 10 +
\sup_{(\alpha, \beta )\in \mathbb{R}^{2}}\sum_{\substack{|n_1 - \alpha| \ge 2 \\
| n_{1} - \beta  | \ge 2}}\frac{1}{(1 +
|(n_{1} - \alpha)(n_{1} - \beta)|)^{\gamma}} 
\end{split}
\end{equation*}
But by Cauchy-Schwartz,
%
%
\begin{equation*}
\begin{split}
\sup_{(\alpha, \beta )\in \mathbb{R}^{2}}\sum_{\substack{|n_1 - \alpha| \ge 2 \\
| n_{1} - \beta  | \ge 2}}\frac{1}{(1 +
|(n_{1} - \alpha)(n_{1} - \beta)|)^{\gamma}}
& \le 
\sup_{(\alpha, \beta )\in \mathbb{R}^{2}}\sum_{\substack{|n_1 - \alpha| \ge 2 \\
| n_{1} - \beta  | \ge 2}}\frac{1}{|(n_{1} - \alpha)(n_{1} - \beta)|^{\gamma}} 
\\
& \le \sup_{(\alpha, \beta )\in \mathbb{R}^{2}}\sum_{|n_1 - \alpha|
\ge 2}\frac{1}{|n_{1} - \alpha|^{2\gamma}} \times \sum_{|n_1 - \beta|
\ge 2}\frac{1}{|n_{1} - \beta|^{2\gamma}} 
\\
& \le 2 \sum_{|n_1 |
\ge 2}\frac{1}{|n_{1} |^{2\gamma}}
\\
& < \infty, \qquad \gamma > 1/2.
\end{split}
\end{equation*}
%
Therefore
%
\begin{equation*}
\begin{split}
\sup_{(n,\tau)\in \zz \times \rr}\sum_{n_1\in \zz}\frac{1}{(1+|\tau-
n_1(n-n_1)|)^{\gamma}} < \infty.
\end{split}
\end{equation*}
An identical argument also gives
\begin{equation*}
\begin{split}
\sup_{(n,\tau)\in \zz \times \rr}\sum_{n_1\in \zz}\frac{1}{(1+|\tau+
n_1(n-n_1)|)^{\gamma}} < \infty
\end{split}
\end{equation*}
which completes the proof.  
\end{proof}
%
\begin{proof}[Proof of \cref{cor:integral-bound}]
%
Let $\alpha = \alpha(a_{0}, a_{1}, a_{2})$, $\beta = \beta(a_{0}, a_{1}, a_{2})$
be the roots of $a_{0} + a_{1}x + a_{2}x^{2}$. Then
%
%
\begin{equation*}
\begin{split}
  \int_{\rr} \frac{1}{\langle a_{0} + a_{1}x + a_{2}x^{2} \rangle ^{q}} dx =
  \int_{\rr} \frac{1}{(1 + | x -\alpha | | x- \beta |)^{q}} dx.
\end{split}
\end{equation*}
%
%
Since $| x - \gamma | \ge | x - \mathfrak{Re}(\gamma) |$, we assume without loss
of generality that $ \alpha, \beta \in \rr$. Then
%
%
\begin{equation*}
\begin{split}
  \int_{\rr} \frac{1}{(1 + | x -\alpha | | x- \beta |)^{q}} dx
  & \le \int_{\rr} \frac{1}{1 + | x- \alpha |^{q} | x - \beta |^{q}} dx
  \\
  & \le \frac{c_{q}}{\langle \alpha - \beta \rangle}, \qquad q > 1/2
\end{split}
\end{equation*}
%
%
where the last line follows from \cref{lem:calc}. This concludes the proof.
\end{proof}
%
\begin{proof}[Proof of \cref{lem:pre-bilin-est}]
We have
%
%
\begin{equation*}
\begin{split}
  \psi_{\delta}(t) \int_{0}^{t} g(t') dt'
  & = \frac{\psi_{\delta}(t)}{2 \pi} \int_{0}^{t} \int_{\rr} e^{it' \tau}
  \wh{g}(\tau) d \tau dt'
  \\
  & \simeq \psi_{\delta}(t) \int_{\rr} \int_{0}^{t} e^{it' \tau} \wh{g}(\tau) dt' d\tau
  \\
  & = \psi_{\delta}(t)  \int_{\rr} \frac{e^{it \tau} -1}{i \tau}
  \wh{g}(\tau) d \tau
  \\
  & = \psi_{\delta}(t) \int_{| \tau |\delta \le 1} \frac{e^{it\tau}
  -1}{i\tau}\wh{g}(\tau) d \tau + \psi_{\delta}(t) \int_{| \tau |\delta \ge 1} \frac{e^{it\tau}
  -1}{i\tau}\wh{g}(\tau) d \tau
  \\
  & = \psi_{\delta}(t) \sum_{k \ge 1}\frac{t^{k}}{k!} \int_{| \tau |\delta \le 1}
  (i\tau)^{k-1} \wh{g}(\tau) d \tau
  - \psi_{\delta}(t) \int_{| \tau |\delta \ge 1}\frac{\wh{g}(\tau)}{i \tau} d \tau
  \\
  & + \psi_{\delta}(t) \int_{| \tau |\delta \ge 1}
  \frac{e^{it \tau}}{i \tau}\wh{g}(\tau) d \tau
  \\
  & = I + II + III.
\end{split}
\end{equation*}
%
%
Then
%
%
\begin{equation}
  \label{h1-norm}
\begin{split}
  & \|I \|_{H^{b}} = \sum_{k \ge 1} \frac{1}{k!} \| t^{k} \psi_{\delta}(t) \|_{H^{b}}
  | \int_{| \tau |\delta \le 1} (i \tau)^{k-1} \wh{g}(\tau) d \tau |
  \\
  & \le \sum_{k \ge 1} \frac{1}{k!} \|t^{k} \psi_{\delta}(t)\|_{H^{b}} \delta^{1-k}\int_{| \tau |\delta \le
  1} | \wh{g}(\tau) | d \tau.
\end{split}
\end{equation}
%
%
%
%
First we estimate
%
%
\begin{equation*}
  \begin{split}
  & \int_{| \tau |\delta \le 1}| \wh{g}(\tau) | d \tau
  \\
  & \le  \| g \|_{H^{b'}} \left( \int_{| \tau | \delta \le 1}  (1 + | \tau
  |)^{-2b'} d \tau \right)^{1/2} 
  \\
  & = \sqrt{2} \| g \|_{H^{b'}} \left( \int_{0 \le \tau \le
  \frac{1}{\delta}}(1 + \tau)^{-2b'} d \tau \right)^{1/2}
  \\
  & = \sqrt{\frac{2}{-2b'+1}} \left[ (1 + \tau)^{-2b'+1} \Big |_{0}^{1/\delta}
  \right]^{1/2} \| g \|_{H^{b'}}, \qquad b'>-1/2
  \\
  & \simeq_{b'} \left[ (1 + 1/\delta)^{-2b'+1} - 1 \right]^{1/2} \| g
  \|_{H^{b'}}
  \\
  & \le \left[ (2/\delta)^{-2b'+1} -1 \right]^{1/2} \| g \|_{H^{b'}}, \qquad \delta \le 1
\end{split}
\end{equation*}
and so 
%
%
\begin{equation}
\label{microl-est}
\begin{split}
\int_{| \tau |\delta \le 1}| \wh{g}(\tau) | d \tau
\lesssim_{b'} \delta^{-1/2 + b'} \| g \|_{H^{b'}}, \qquad -1/2 < b' \le 0 < b \le b' +1.
\end{split}
\end{equation}
%
%
Next, for $k \ge 1$ we consider
%
%
\begin{equation}
  \label{1h}
\begin{split}
  \| t^{k} \psi_{\delta} \|^{2}_{H^{b}} 
  & = \int_{\rr} (1 + | \tau |)^{2b} | \wh{t^{k}\psi_{\delta}(t)}(\tau) |^{2} d \tau
  \\
  & = \int_{\rr} (1 + | \tau |)^{2b} | \int_{\rr} e^{-it \tau} t^{k}
  \psi_{\delta}(t) dt |^{2} d \tau
  \\
  & = \int_{\rr} (1 + | \tau |)^{2b} | \int_{\rr} e^{-it \tau} t^{k}
  \psi(t/\delta) dt |^{2} d \tau
    \end{split}
\end{equation}
%
%
which by the change of variable $s = t/\delta$ is equal to
%
%
%
%
\begin{equation}
  \label{2h}
\begin{split}
  \int_{\rr} (1 + | \xi |)^{2b} | \int_{\rr} e^{-is\delta \xi} (s\delta)^{k} \psi(s) \delta ds
  |^{2} d \xi.
\end{split}
\end{equation}
%
%
Now applying the change of variable $\xi = \lambda/\delta$, this is equal to
%
%
%
\begin{equation}
  \label{3h}
\begin{split}
  & \int_{\rr} \left( 1 + | \lambda/\delta | \right)^{2b} | \int_{\rr} e^{-is\lambda}
  (s\delta)^{k} \psi(s) \delta ds |^{2} \delta^{-1} d \lambda
  \\
  & = \delta^{2k+1}
  \int_{\rr} \left( 1 + | \lambda/\delta | \right)^{2b} | \wh{s^{k}
  \psi(s)}(\lambda )|^{2}  d \lambda
  \\
  & \le \delta^{2k+1}
  \int_{\rr} \left( 1/\delta + | \lambda/\delta | \right)^{2b} | \wh{s^{k}
  \psi(s)}(\lambda )|^{2}  d \lambda, \quad 0 < \delta \le 1
  \\
  & = \delta^{2k +1 -2b} \| s^{k} \psi(s) \|_{H^{b}}.
\end{split}
\end{equation}
%
%
But
%
%
\begin{equation}
	\label{4ng}
	\begin{split}
		\|t^k \psi(t) \|_{H^b(\rr)}^2
    & \le \|t^k \psi(t) \|_{H^1(\rr)}^2, \quad b \le 1
		\\
    & = \left( \|t^k \psi(t)\|_{L^2(\rr)} + \|\p_t \left( t^k \psi(t)
		\right)\|_{L^2(\rr)} \right)^2
		\\
		& \lesssim \|t^{k}\psi(t) \|_{L^2(\rr)}^2 + \|\p_t \left (t^{k}
		\psi(t) \right )\|_{L^2(\rr)}^2
		\\
		& \le \|t^k \psi(t) \|_{L^2(\rr)}^2 + \|t^k \p_t \psi(t)
		\|_{L^2(\rr)}^2 + \|k t^{k -1} \psi(t) \|_{L^2(\rr)}^2
		\\
		& = c_{\psi} + c_{\psi}' + c_{\psi}''k^2 
		\\
    & \lesssim 
    \begin{cases}
      c_{\psi} k^2 \qquad  & k \ge 1
      \\
      c_{\psi}, \qquad & k = 0.
    \end{cases}
	\end{split}
\end{equation}
%
%
%
%
Therefore,
%
%
\begin{equation}
\begin{split}
  \| t^{k}\psi_{\delta}(t) \|_{H^b} \lesssim_{\psi}  k \delta^{k+1/2 -b}.
\end{split}
\label{hb-norm}
\end{equation}
%
%
Substituting estimates \cref{microl-est} and \cref{hb-norm} into the right
hand side of \cref{h1-norm}, we see that
%
%
\begin{equation*}
\begin{split}
  \| I \|_{H^{b}}
  & \lesssim_{\psi, b'} \sum_{k \ge 1}\frac{k}{k!} \delta^{k + 1/2 -b}
  \delta^{1-k} \delta^{-1/2 + b'}
  \\
  & = \sum_{k \ge 1}\frac{1}{(k-1)!}\delta^{1 - (b - b')}
  \\
  & = e \delta^{1 - (b - b')}
\end{split}
\end{equation*}
%
%
and so
%
%
\begin{equation}
\begin{split}
  \| I \|_{H^{b}} \lesssim_{\psi, b'} \delta^{1 - (b - b')}.
\end{split}
\label{rom-1}
\end{equation}
%
%
Next, we estimate
%
%
\begin{equation}
  \label{est-II}
\begin{split}
  \| II \|_{H^{b}}
  & = \| \psi_{\delta}(t) \int_{| \tau |\delta \ge 1} \frac{\wh{g}(\tau)}{i
  \tau} d \tau \|_{H^{b}}
  \\
  & = | \int_{| \tau | \delta \ge 1} \frac{\wh{g}(\tau)}{i \tau} d \tau | \times \| \psi_{\delta}
  \|_{H^{b}}
  \\
  & \le \int_{| \tau |\delta \ge 1} \frac{| \wh{g}(\tau)|}{|\tau|} d \tau  \times \| \psi_{\delta}
  \|_{H^{b}}.
\end{split}
\end{equation}
%
%
Note that 
%
%
\begin{equation*}
\begin{split}
\int_{| \tau |\delta \ge 1} \frac{| \wh{g}(\tau)|}{|\tau|} d \tau 
& = \int_{| \tau |\delta \ge 1} \frac{| \wh{g}(\tau)| (1 + | \tau |)^{b'} (1 + | \tau
|)^{-b'}}{|\tau|} d \tau 
\\
& \le \|g\|_{H^{b'}} \left[ \int_{| \tau | \delta \ge 1} (1 + | \tau |)^{-2b'} | \tau
|^{-2} d \tau \right]^{1/2}.
\end{split}
\end{equation*}
%
%
Applying the inequality 
%
%
\begin{equation}
  \label{sob-term-bound}
\begin{split}
(1 + | \tau |) \le 2 | \tau |, \qquad | \tau |\delta \ge 1, \ 0 < \delta \le 1
\end{split}
\end{equation}
we bound the integral term by
%
%
%
%
\begin{equation*}
\begin{split}
\left[ \int_{| \tau | \delta \ge 1} 4 (1 + | \tau |)^{-2b'-2} | \tau
|^{-2} d \tau \right]^{1/2}
& = \left[ \int_{1/\delta \le  \tau < \infty} 8 (1 +  \tau )^{-2b'-2} 
d \tau \right]^{1/2}
\\
& = c_{b'} \left [ \left( 1 + \frac{1}{\delta} \right)^{-2b' -1} \right ] ^{1/2}, \quad b'> -1/2
\\
& \le c_{b'} \left [ \left (\frac{2}{\delta} \right )^{-2b' -1} \right ]^{1/2}, \quad \delta \le 1
\\
& \simeq_{b'} \delta^{1/2 + b'}.
\end{split}
\end{equation*}
%
It follows that
%
%
\begin{equation}
  \label{est-II-1}
\begin{split}
\int_{| \tau |\delta \ge 1} \frac{| \wh{g}(\tau)|}{|\tau|} d \tau 
\lesssim_{b'} \delta^{1/2 + b'}.
\end{split}
\end{equation}
%
%
%
Furthermore, from computations \cref{1h}-\cref{3h} and \cref{4ng} (both with
$k=0$), we obtain 
%
\begin{equation}
  \label{est-II-2}
\begin{split}
  \| \psi_{\delta} \|_{H^{b}}
  & = \delta^{1/2 - b} \| \psi(t) \|_{H^{b}}
  \\
  & \simeq_{\psi} \delta^{1/2 -b}.
\end{split}
\end{equation}
%
%
Substituting \cref{est-II-1} and \cref{est-II-2} into the right hand side of
\cref{est-II}, we obtain
%
%
\begin{equation}
  \label{rom-2}
\begin{split}
  \| II \|_{H^{b}} \lesssim_{\psi, b'} \delta^{1 - (b - b')}.
\end{split}
\end{equation}
%
%
Lastly, we estimate $III$. Setting 
%
%
\begin{equation*}
\begin{split}
J(t) = \int_{| \tau |\delta \ge 1}
  \frac{e^{it \tau}}{i \tau}\wh{g}(\tau) d \tau
\end{split}
\end{equation*}
%
%
we have
%
%
\begin{equation*}
\begin{split}
  \| III \|_{H^{b}_{t}}
  & = \| (1 + | \cdot |)^{b} \wh{\psi_{\delta}}(\cdot) *
  \wh{J}(\cdot) (\lambda) \|_{L^{2}_{\lambda}}
  \\
  & = \left[ \int_{\rr} (1 + | \lambda |)^{2b} | \int_{\rr}
  \wh{\psi_{\delta}}(\lambda - \tau) \wh{J}(\tau) d \tau |^{2} \right]^{1/2}
  \\
  & = \left[ \int_{\rr}  | \int_{\rr}
 (1 + | \lambda |)^{b} \wh{\psi_{\delta}}(\lambda - \tau) \wh{J}(\tau) d \tau |^{2} \right]^{1/2}
  \\
  & \le \left[ \int_{\rr}  | \int_{\rr}
  \wh{\psi_{\delta}}(\lambda - \tau) (1 + | \lambda - \tau |)^{b} \wh{J}(\tau)
  (1 + | \tau |)^{b}  d \tau |^{2} \right]^{1/2}
  \\
  & = \left[ \int_{\rr} | \wh{\psi_{\delta}}(1 + | \cdot |)^{b} * \wh{J}(1 + | \cdot
  |)^{b} (\tau) d \tau |^{2}  \right]^{1/2}
\end{split}
\end{equation*}
%
%
which by Young's inequality is bounded by
%
%
\begin{equation*}
\begin{split}
  \| \wh{\psi_{\delta}}(\tau) (1 + | \tau |) \|_{L^{1}} \| \wh{J}(\tau) (1 + | \tau
  |)^{b} \|_{L^{2}}
\end{split}
\end{equation*}
%
%
Rewriting
%
%
\begin{equation*}
\begin{split}
  J(t) 
  & = \int_{| \tau |\delta \ge 1}
  \frac{e^{it \tau}}{i \tau}\wh{g}(\tau) d \tau
  \\
  & = \int_{\rr}
  \frac{e^{it \tau}}{i \tau}\wh{g}(\tau) \chi_{| \tau |\delta \ge 1} d \tau
\end{split}
\end{equation*}
%
%
we see that
%
%
\begin{equation*}
\begin{split}
  \wh{J}(\tau) = \wh{g}(\tau) (i \tau)^{-1} \chi_{| \tau |\delta \ge 1}
\end{split}
\end{equation*}
%
%
and so 
%
%
\begin{equation*}
\begin{split}
\| \wh{J}(\tau) (1 + | \tau
|)^{b} \|_{L^{2}}^{2}
& = \int_{| \tau |\delta \ge 1} (1 + | \tau |)^{2b} | \wh{g}(\tau) |^{2}
\tau^{-2} d \tau
\\
& = \int_{| \tau |\delta \ge 1} (1 + | \tau |)^{2b'} | \wh{g}(\tau) |^{2} (1 + | \tau
|)^{2(b - b')} \tau^{-2} d \tau
\\
& \le \sup_{| \tau |\delta \ge 1} (1 + | \tau |)^{2(b - b')} \tau^{-2}
\|g\|_{H^{b'}}^{2}
\end{split}
\end{equation*}
%
%
Applying \cref{sob-term-bound} we bound this by
%
%
\begin{equation*}
\begin{split}
  & \sup_{| \tau |\delta \ge 1} 2^{2(b - b')} (| \tau |)^{2(b - b' -1)} 
\|g\|_{H^{b'}}^{2}
\\
& \le \sup_{| \tau |\delta \ge 1} 4 (| \tau |)^{2(b - b' -1)} 
\|g\|_{H^{b'}}^{2}, \qquad -1/2 < b' \le 0 < b 
\\
& \simeq \delta^{2[1 - (b - b') ]} \| g \|_{H^{b'}}^{2}.
\end{split}
\end{equation*}
%
%
Therefore,
%
%
\begin{equation}
  \label{rom-3}
\begin{split}
  \| III \| \lesssim \delta^{1 - (b - b')} \| g \|_{H^{b'}}.
\end{split}
\end{equation}
%
%
Combining \cref{rom-1}, \cref{rom-2}, and \cref{rom-3} concludes the proof.
\end{proof}
%
%
%
%%%%%%%%%%%%%%%%%%%%%%%%%%%%%%%%%%%%%%%%%%%%%%%%%%%%%
%
%
%                Scaling
%
%
%%%%%%%%%%%%%%%%%%%%%%%%%%%%%%%%%%%%%%%%%%%%%%%%%%%%%
%
%
\section{Scaling} 
\label{sec:scaling}
Our object of investigation is the initial value problem for the
periodic/non-periodic $1$-D ``good'' Boussinesq equation, i.e.,
\begin{equation}
  \aligned
  & u_{tt}-u_{xx}+u_{xxxx}+ (u^2)_{xx}\,=\,0, \quad x\in \mathbb{T}\ \text{or} \ \mathbb{R}, \quad t>0,\\
&u(0,x)\,=\,u_0(x),\qquad u_t(0,x)\,=\,u_1(x).\endaligned
\label{main}
\end{equation}
Due to the fact that the leading terms in the linear operator above are $u_{tt}$ and $u_{xxxx}$, morally speaking, one derivative in time is like two derivatives in space. This is why the Sobolev regularity scale for the initial data should be as follows:
\[
u_0\in H^s(\mathbb{T}\ \text{or} \ \mathbb{R}), \qquad u_1\in H^{s-2}(\mathbb{T}\ \text{or} \ \mathbb{R})
\]
Results for these problems are usually formulated with $u_0=\phi \in H^s$ and $u_1=\psi_x$, $\psi\in H^{s-1}$.
Current state of the art in terms of local well-posedness/ill-posedness for the two problems is:
\begin{itemize}
  \item LWP for both problems when $s>-\frac 14$ (Farah '09, Farah-Scialom '10), with iteration done in
    the norm
    \[
    \|F\|_{X^{s,b}}\,=\,\|<|\tau|-\sqrt{\xi^2+\xi^4}>^b\,<\xi>^s \tilde{F}\|_{L^2_{\tau,\xi}};
    \]
  \item main IP result is for the non-periodic problem when $s<-2$, as the solution map 
    \[
    S: H^s\times H^{s-1} \to C([0,\delta]; H^s), \quad
    S(\phi,\psi)\,=\,u
    \]
    is not $C^2$ at zero (Farah '09);
  \item also, for the non-periodic problem, one can not find a space in which to run a contraction argument based on treating the nonlinearity as bilinear for $s<-2$ (see Theorem 1.4 in Farah '09);
  \item finally, and really puzzling\footnote{for other dispersive equations (e.g., KdV, Schrodinger), there is usually a gap of $\frac 14$ between regularities for the two problems}, the crucial bilinear estimate (equation (5) in both papers) fails basically at the same threshold for both problems: $s\leq -\frac 14$ (non-periodic), $s<-\frac{1}{4}$ (periodic).
\end{itemize}
The equation does not have an associated scaling, however one can do a formal scaling analysis by ignoring one of the two linear terms containing spatial derivatives:
\begin{itemize}
  \item for 
    \[
    u_{tt}+u_{xxxx}+(u^2)_{xx}\,=\,0,
    \]
    one has 
    \[
    u_{\lambda}(t,x)\,=\,\frac{1}{\lambda^2}u\left(\frac{t}{\lambda^2}, \frac{x}{\lambda}\right),
    \]
    which leads to $s_c=-\frac 32$.
\end{itemize}
\begin{proof}
Let $u(x, t)$ be a solution to the $B_4$ equation, that is
%
$$
B_4(u)=
 \partial_t^2u + \partial^4_x u + \partial_x^2(u^2)  = 0
$$
%
We would like to find the constants
$a, b, c$ such that
\[
u_\lambda (x, t) = \lambda^a u(\lambda^b x, \lambda^c t)
\]
is also a solution to $B_4$.  Since 
$$
B_4(u_\lambda)=
\lambda^{a+2c} \partial_t^2u 
+
 \lambda^{a+4b} \partial^4_x u 
 +
  \lambda^{2a+2b}
  \partial_x^2(u^2),  
$$
we see that $u_\lambda$ is a $B_4$ solution only if
$$
a+2c=a+4b=2a+2b,
$$
or
$
c= 2b =a.
$
  Thus
\[
u_\lambda (x, t) = \lambda^{2b} u(\lambda^{b}x,  \lambda^{2b} t).
\]
%
%
Therefore, replacing  $\lambda^b$ with  $ \lambda$ gives the following scaling:
%
\begin{equation}
\label{DP-scal}
\boxed{u(x, t) \ \text{solution to} \  B_4
 \Longrightarrow 
u_\lambda (x, t) = \lambda^2 u(\lambda x, \lambda^2 t)  \ \text{is also a
solution to} \  B_4.}
\end{equation}
\label{rem:scaling}
To find the critical Sobolev index, we compute
%
%
\begin{equation}
\begin{split}
  \| u_{\lambda} \|_{\dot{H}^s(\ci)} 
  & = \lambda^{2} \| u(\lambda x, \lambda^2 t) \|_{\dot{H}^{s}(\ci)}
  \\
  & = \lambda^{2} \left( \int_{\rr} | \xi |^{2s} | \wh{u (\lambda x,
  \lambda^{2} t)}^x (\xi, t)| \right)^{1/2}.
\end{split}
\label{crit-ind-comp}
\end{equation}
%
But
%
%
\begin{equation*}
\begin{split}
  \wh{u(\lambda x, \lambda^{2}t)^x}(\xi, t)
  & = \int_{\rr}e^{-i\xi x}u(\lambda x, \lambda^2 t) dx
  \\
  & = \frac{1}{\lambda} \int_{\rr}e^{-i \frac{n}{\lambda} x'}u(x',
  \lambda^{2} t) dx'
  \\
  & = \frac{1}{\lambda} \wh{u(\cdot, \lambda^{2}t)}(\frac{\xi}{\lambda})
\end{split}
\end{equation*}
%
%
Substituting back into \cref{crit-ind-comp}, we obtain
%
%
\begin{equation*}
\begin{split}
  \| u_{\lambda} \|_{\dot{H}^s(\rr)} 
  & = \lambda^{2} \left( \int_{\rr} | \xi |^{2s} |
  \frac{1}{\lambda}\wh{u(\cdot, \lambda^{2}t)}(\frac{\xi}{\lambda}) |^2 d \xi
  \right)^{1/2}
  \\
  & = \lambda \left( \int_{\rr}| \xi |^{2s} | \wh{u(\cdot,
  \lambda^{2}t)}(\frac{\xi}{\lambda}) |^2 d \xi  \right)^{1/2}
  \\
  & = \lambda \left( \int_{\rr} | \lambda \xi' |^{2s} 
  \wh{u(\cdot, \lambda^{2}t)}(\xi') |^2 \lambda d \xi
  \right)^{1/2}
  \\
  & = \lambda^{s + 3/2} \|u(\cdot, \lambda^{2}t) \|_{\dot{H}^s (\ci)}.
\end{split}
\end{equation*}
%
%
Therefore, $\| u_{\lambda}(0) \|_{\dot{H}^s(\rr)} = \lambda^{s + 3/2} \|
u_{0} \|_{\dot{H}^{s}(\rr)}$. Hence, $s=-3/2$ is the critical Sobolev index.
\end{proof}
%
%
\begin{framed}
\begin{remark}
Since the scaling conserves data in $\dot{H}^{-3/2}$, it
seems that this equation is ``like KdV''.
So one may expect KdV type theorems.
That is, $s_c=-3/4$ on the line and $s_c=-1/2$ on the circle,
if one uses bilinear estimates.
But, Kappeler and collaborators went all the way to $-1$ for KdV.
However KdV is integrable. Is this equation integrable?
Also, people conjecture that the critical index for KdV well-posedness 
in some appropriate sense should be the scaling index which is  $-3/2$.
\label{rem:kdv-like}
\end{remark}
\end{framed}
%
%
\begin{itemize}
  \item for 
    \[
    u_{tt}-u_{xx}+(u^2)_{xx}\,=\,0,
    \]
    one has 
    \[
    u_{\lambda}(t,x)\,=\,u\left(\frac{t}{\lambda}, \frac{x}{\lambda}\right),
    \]
    which leads to $s_c=\frac 12$.
\end{itemize}
\begin{proof}
Let $u(x, t)$ be a solution to the $B_2$ equation, that is
%
$$
B_2(u)=
 \partial_t^2u - \partial^2_x u + \partial_x^2(u^2)  = 0
$$
%
We would like to find the constants
$a, b, c$ such that
\[
u_\lambda (x, t) = \lambda^a u(\lambda^b x, \lambda^c t)
\]
is also a solution to $B_2$.  Since 
$$
B_2(u_\lambda)=
\lambda^{a+2c} \partial_t^2u 
-
 \lambda^{a+2b} \partial^2_x u 
 +
  \lambda^{2a+2b}
  \partial_x^2(u^2),  
$$
we see that $u_\lambda$ is a $B_2$ solution only if
$$
a+2c=a+2b=2a+2b,
$$
or
$
a=0, b=c.
$
  Thus
\[
u_\lambda (x, t) = u(\lambda^{b}x,  \lambda^{b} t).
\]
%
%
Therefore, replacing  $ \lambda^b$ with  $ \lambda$ gives the following scaling:
%
\begin{equation}
\label{B2-scal}
\boxed{u(x, t) \ \text{solution to} \  B_2
 \Longrightarrow 
u_\lambda (x, t) = u(\lambda x, \lambda t) \ \text{is also a
solution to} \  B_2. 
}
\end{equation}
\label{rem:scaling-B2}
To find the critical Sobolev index, we compute
%
%
\begin{equation}
\begin{split}
  \| u_{\lambda} \|_{\dot{H}^s(\ci)} 
  & =  \| u(\lambda x, \lambda t) \|_{\dot{H}^{s}(\ci)}
  \\
  & = \left( \int_{\rr} | \xi |^{2s} | \wh{u(\lambda x,
  \lambda t)}^x (\xi, t)| \right)^{1/2}.
\end{split}
\label{crit-ind-comp-B2}
\end{equation}
%
But
%
%
\begin{equation*}
\begin{split}
  \wh{u(\lambda x, \lambda t)^x}(\xi, t)
  & = \int_{\rr}e^{-i\xi x}u(\lambda x, \lambda t) dx
  \\
  & = \frac{1}{\lambda} \int_{\rr}e^{-i \frac{n}{\lambda} x'}u(x',
  \lambda t) dx'
  \\
  & = \frac{1}{\lambda} \wh{u(\cdot, \lambda t)}(\frac{\xi}{\lambda})
\end{split}
\end{equation*}
%
%
Substituting back into \cref{crit-ind-comp-B2}, we obtain
%
%
\begin{equation*}
\begin{split}
  \| u_{\lambda} \|_{\dot{H}^s(\rr)} 
  & = \left( \int_{\rr} | \xi |^{2s} |
  \frac{1}{\lambda}\wh{u(\cdot, \lambda t)}(\frac{\xi}{\lambda}) |^2 d \xi
  \right)^{1/2}
  \\
  & = \frac{1}{\lambda} \left( \int_{\rr}| \xi |^{2s} | \wh{u(\cdot,
  \lambda t)}(\frac{\xi}{\lambda}) |^2 d \xi  \right)^{1/2}
  \\
  & = \frac{1}{\lambda} \left( \int_{\rr} | \lambda \xi' |^{2s} 
  \wh{u(\cdot, \lambda)}(\xi') |^2 \lambda d \xi
  \right)^{1/2}
  \\
  & = \lambda^{s - 1/2} \|u(\cdot, t) \|_{\dot{H}^s (\rr)}.
\end{split}
\end{equation*}
%
%
Therefore, $\| u_{\lambda(0)} \|_{\dot{H}^s(\rr)} = \lambda^{s - 1/2} \|
u_{0} \|_{\dot{H}^{s}(\rr)}$, and so $s=1/2$ is the critical Sobolev index.
\end{proof}
This might suggest that the current results are not optimal.
\end{appendices}
%
% \bib, bibdiv, biblist are defined by the amsrefs package.
\begin{bibdiv}
\begin{biblist}

\bib{Bejenaru-Tao-2006-Sharp-well-posedness-and-ill-posedness}{article}{
      author={Bejenaru, Ioan},
      author={Tao, Terence},
       title={Sharp well-posedness and ill-posedness results for a quadratic
  non-linear schr{\"o}dinger equation},
        date={2006},
     journal={J. Funct. Anal.},
      volume={233},
      number={1},
       pages={228\ndash 259},
  url={http://www.ams.org.proxy.library.nd.edu/mathscinet-getitem?mr=2204680},
}

\bib{Farah:2009uq}{article}{
      author={Farah, Luiz~Gustavo},
       title={Local solutions in sobolev spaces with negative indices for the
  ``good'' boussinesq equation},
        date={2009},
     journal={Comm. Partial Differential Equations},
      volume={34},
      number={1-3},
       pages={52\ndash 73},
  url={http://www.ams.org.proxy.library.nd.edu/mathscinet-getitem?mr=2512853},
}

\bib{Ginibre:1996fk}{article}{
      author={Ginibre, Jean},
       title={Le probl{\`e}me de cauchy pour des edp semi-lin{\'e}aires
  p{\'e}riodiques en variables d'espace (d'apr{\`e}s bourgain)},
        date={1996},
     journal={Ast{\'e}risque},
      number={237},
       pages={Exp.\ No.\ 796, 4, 163\ndash 187},
  url={http://www.ams.org.proxy.library.nd.edu/mathscinet-getitem?mr=1423623},
        note={S{{\'e}}minaire Bourbaki, Vol. 1994/95},
}

\bib{Ginibre:1997fk}{article}{
      author={Ginibre, J},
      author={Tsutsumi, Y},
      author={Velo, G},
       title={On the cauchy problem for the zakharov system},
        date={1997},
     journal={J. Funct. Anal.},
      volume={151},
      number={2},
       pages={384\ndash 436},
  url={http://www.ams.org.proxy.library.nd.edu/mathscinet-getitem?mr=1491547},
}

\bib{Kenig:1996aa}{article}{
      author={Kenig, Carlos~E},
      author={Ponce, Gustavo},
      author={Vega, Luis},
       title={A bilinear estimate with applications to the kdv equation},
        date={1996},
     journal={J. Amer. Math. Soc.},
      volume={9},
      number={2},
       pages={573\ndash 603},
  url={http://www.ams.org.proxy.library.nd.edu/mathscinet-getitem?mr=1329387},
}

\bib{Kenig-Ponce-Vega-1996-Quadratic-forms-for-the-1-D-semilinear}{article}{
      author={Kenig, Carlos~E},
      author={Ponce, Gustavo},
      author={Vega, Luis},
       title={Quadratic forms for the {\$}1{\$}-d semilinear schr{\"o}dinger
  equation},
        date={1996},
     journal={Trans. Amer. Math. Soc.},
      volume={348},
      number={8},
       pages={3323\ndash 3353},
  url={http://www.ams.org.proxy.library.nd.edu/mathscinet-getitem?mr=1357398},
}

\end{biblist}
\end{bibdiv}
%
% \bib, bibdiv, biblist are defined by the amsrefs package.
%\bibliography{/Users/davidkarapetyan/math/bib-files/references.bib}
%
%\nocite{*}
\end{document}
