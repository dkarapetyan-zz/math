%
\documentclass[12pt,reqno]{amsart}
\usepackage{amssymb}
\usepackage{amsmath} 
\usepackage{cancel}  %for cancelling terms explicity on pdf
\usepackage{yhmath}   %makes fourier transform look nicer, among other things
\usepackage{framed}  %for framing remarks, theorems, etc.
\usepackage[shortalphabetic, initials, msc-links]{amsrefs} %for the bibliography; uses cite pkg
\usepackage{enumerate} %to change enumerate symbols
%\usepackage{showkeys}  %shows source equation labels on the pdf
\usepackage[margin=3cm]{geometry}  %page layout
%\usepackage[pdftex]{graphicx} %for importing pictures into latex--pdf compilation
%\setcounter{secnumdepth}{1} %number only sections, not subsections
\hypersetup{colorlinks=true,
linkcolor=blue,
citecolor=blue,
urlcolor=blue,
}
\synctex=1
\numberwithin{equation}{section}  %eliminate need for keeping track of counters
\numberwithin{figure}{section}
\setlength{\parindent}{0in} %no indentation of paragraphs after section title
\renewcommand{\baselinestretch}{1.1} %increases vert spacing of text
%
%
\newcommand{\ds}{\displaystyle}
\newcommand{\ts}{\textstyle}
\newcommand{\nin}{\noindent}
\newcommand{\rr}{\mathbb{R}}
\newcommand{\nn}{\mathbb{N}}
\newcommand{\zz}{\mathbb{Z}}
\newcommand{\cc}{\mathbb{C}}
\newcommand{\ci}{\mathbb{T}}
\newcommand{\zzdot}{\dot{\zz}}
\newcommand{\wh}{\widehat}
\newcommand{\p}{\partial}
\newcommand{\ee}{\varepsilon}
\newcommand{\vp}{\varphi}
\newcommand{\wt}{\widetilde}
%
%
\theoremstyle{plain}  
\newtheorem{theorem}{Theorem}
\newtheorem{proposition}{Proposition}
\newtheorem{lemma}{Lemma}
\newtheorem{corollary}{Corollary}
\newtheorem{claim}{Claim}
\newtheorem{conjecture}[subsection]{conjecture}
%
\theoremstyle{definition}
\newtheorem{definition}{Definition}
%
\theoremstyle{remark}
\newtheorem{remark}{Remark}
%
%
%\newtheorem{theorem}{Theorem}[section]
%\newtheorem{lemma}[theorem]{Lemma}
%\newtheorem{corollary}[theorem]{Corollary}
%\newtheorem{claim}[theorem]{Claim}
%\newtheorem{prop}[theorem]{Proposition}
%\newtheorem{proposition}[theorem]{Proposition}
%\newtheorem{no}[theorem]{Notation}
%\newtheorem{definition}[theorem]{Definition}
%\newtheorem{remark}[theorem]{Remark}
%\newtheorem{examp}{Example}[section]
%\newtheorem {exercise}[theorem] {Exercise}
%
\def\makeautorefname#1#2{\expandafter\def\csname#1autorefname\endcsname{#2}}
\makeautorefname{equation}{Equation}
\makeautorefname{footnote}{footnote}
\makeautorefname{item}{item}
\makeautorefname{figure}{Figure}
\makeautorefname{table}{Table}
\makeautorefname{part}{Part}
\makeautorefname{appendix}{Appendix}
\makeautorefname{chapter}{Chapter}
\makeautorefname{section}{Section}
\makeautorefname{subsection}{Section}
\makeautorefname{subsubsection}{Section}
\makeautorefname{paragraph}{Paragraph}
\makeautorefname{subparagraph}{Paragraph}
\makeautorefname{theorem}{Theorem}
\makeautorefname{theo}{Theorem}
\makeautorefname{thm}{Theorem}
\makeautorefname{addendum}{Addendum}
\makeautorefname{add}{Addendum}
\makeautorefname{maintheorem}{Main theorem}
\makeautorefname{corollary}{Corollary}
\makeautorefname{lemma}{Lemma}
\makeautorefname{sublemma}{Sublemma}
\makeautorefname{proposition}{Proposition}
\makeautorefname{property}{Property}
\makeautorefname{scholium}{Scholium}
\makeautorefname{step}{Step}
\makeautorefname{conjecture}{Conjecture}
\makeautorefname{question}{Question}
\makeautorefname{definition}{Definition}
\makeautorefname{notation}{Notation}
\makeautorefname{remark}{Remark}
\makeautorefname{remarks}{Remarks}
\makeautorefname{example}{Example}
\makeautorefname{algorithm}{Algorithm}
\makeautorefname{axiom}{Axiom}
\makeautorefname{case}{Case}
\makeautorefname{claim}{Claim}
\makeautorefname{assumption}{Assumption}
\makeautorefname{conclusion}{Conclusion}
\makeautorefname{condition}{Condition}
\makeautorefname{construction}{Construction}
\makeautorefname{criterion}{Criterion}
\makeautorefname{exercise}{Exercise}
\makeautorefname{problem}{Problem}
\makeautorefname{solution}{Solution}
\makeautorefname{summary}{Summary}
\makeautorefname{operation}{Operation}
\makeautorefname{observation}{Observation}
\makeautorefname{convention}{Convention}
\makeautorefname{warning}{Warning}
\makeautorefname{note}{Note}
\makeautorefname{fact}{Fact}
%
%


\newcommand{\uol}{u^\omega_\lambda}
\newcommand{\lbar}{\bar{l}}
\renewcommand{\l}{\lambda}
\newcommand{\R}{\mathbb R}
\newcommand{\RR}{\mathcal R}
\newcommand{\al}{\alpha}
\newcommand{\ve}{q}
\newcommand{\tg}{{tan}}
\newcommand{\m}{q}
\newcommand{\N}{N}
\newcommand{\ta}{{\tilde{a}}}
\newcommand{\tb}{{\tilde{b}}}
\newcommand{\tc}{{\tilde{c}}}
\newcommand{\tS}{{\tilde S}}
\newcommand{\tP}{{\tilde P}}
\newcommand{\tu}{{\tilde{u}}}
\newcommand{\tw}{{\tilde{w}}}
\newcommand{\tA}{{\tilde{A}}}
\newcommand{\tX}{{\tilde{X}}}
\newcommand{\tphi}{{\tilde{\phi}}}


\begin{document}
\title{$1$-D ``good" Boussinesq equation}

\author{Dan-Andrei Geba, Alexandrou Himonas, and David Karapetyan}

\address{Department of Mathematics, University of Rochester, Rochester, NY 14627}
\address{Department of Mathematics, University of Notre Dame, Notre Dame, IN 46556}
\address{Department of Mathematics, University of Notre Dame, Notre Dame, IN 46556}
\date{}

%\begin{abstract}
%\end{abstract}

\subjclass[2000]{35B30, 35Q55, 35Q72}
\keywords{local well-posedness; ill-posedness.}

\maketitle
%
\section{Introduction}
Our object of investigation is the initial value problem for the periodic/non-periodic $1$-D ``good" Boussinesq equation, i.e.,
\begin{equation}
  \aligned
  &u_{tt}-u_{xx}+u_{xxxx}+(u^2)_{xx}\,=\,0, \quad x\in \mathbb{T}\ \text{or} \ \mathbb{R}, \quad t>0,\\
&u(0,x)\,=\,u_0(x),\qquad u_t(0,x)\,=\,u_1(x).\endaligned
\label{main}
\end{equation}

Due to the fact that the leading terms in the linear operator above are $u_{tt}$ and $u_{xxxx}$, morally speaking, one derivative in time is like two derivatives in space. This is why the Sobolev regularity scale for the initial data should be as follows:
\[
u_0\in H^s(\mathbb{T}\ \text{or} \ \mathbb{R}), \qquad u_1\in H^{s-2}(\mathbb{T}\ \text{or} \ \mathbb{R})
\]
Results for these problems are usually formulated with $u_0=\phi \in H^s$ and $u_1=\psi_x$, $\psi\in H^{s-1}$.

Current state of the art in terms of local well-posedness/ill-posedness for the two problems is:
\begin{itemize}
  \item LWP for both problems when $s>-\frac 14$(Farah '09, Farah-Scialom '10), with iteration done in
    the norm
    \[
    \|F\|_{X^{s,b}}\,=\,\|<|\tau|-\sqrt{\xi^2+\xi^4}>^b\,<\xi>^s \tilde{F}\|_{L^2_{\tau,\xi}};
    \]

  \item main IP result is for the non-periodic problem when $s<-2$, as the solution map 
    \[
    S: H^s\times H^{s-1} \to C([0,T]; H^s), \quad
    S(\phi,\psi)\,=\,u
    \]
    is not $C^2$ at zero (Farah '09);

  \item also, for the non-periodic problem, one can not find a space in which to run a contraction argument based on treating the nonlinearity as bilinear for $s<-2$ (see Theorem 1.4 in Farah '09);

  \item finally, and really puzzling\footnote{for other dispersive equations (e.g., KdV, Schrodinger), there is usually a gap of $\frac 14$ between regularities for the two problems}, the crucial bilinear estimate (equation (5) in both papers) fails basically at the same threshold for both problems: $s\leq -\frac 14$ (non-periodic), $s<-\frac{1}{4}$ (periodic).
\end{itemize}

The equation does not have an associated scaling, however one can do a formal scaling analysis by ignoring one of the two linear terms containing spatial derivatives:
\begin{itemize}
  \item for 
    \[
    u_{tt}+u_{xxxx}+(u^2)_{xx}\,=\,0,
    \]
    one has 
    \[
    u_{\lambda}(t,x)\,=\,\frac{1}{\lambda^2}u\left(\frac{t}{\lambda^2}, \frac{x}{\lambda}\right),
    \]
    which leads to $s_c=-\frac 32$;
\begin{framed}
\begin{remark}
We now prove this.
Let $u(x, t)$ be a solution to the $B_4$ equation, that is
%
$$
B_4(u)=
 \partial_t^2u + \partial^4_x u + \partial_x^2(u^2)  = 0
$$
%
We would like to find the constants
$a, b, c$ such that
\[
u_\lambda (x, t) = \lambda^a u(\lambda^b x, \lambda^c t)
\]
is also a solution to $B_4$.  Since 
$$
B_4(u_\lambda)=
\lambda^{a+2c} \partial_t^2u 
+
 \lambda^{a+4b} \partial^4_x u 
 +
  \lambda^{2a+2b}
  \partial_x^2(u^2),  
$$
we see that $u_\lambda$ is a $B_4$ solution only if
$$
a+2c=a+4b=2a+2b,
$$
or
$
c= 2b =a.
$
  Thus
\[
u_\lambda (x, t) = \lambda^{2b} u(\lambda^{b}x,  \lambda^{2b} t).
\]
%
%
Therefore, replacing  $ \lambda^b$ with  $ \lambda$ gives the following scaling:
%
\begin{equation}
\label{DP-scal}
\boxed{
u(x, t) \text{ solution to }  B_4
 \Longrightarrow 
u_\lambda (x, t) = \lambda^2 u(\lambda x, \lambda^2 t)  \text { is also a
solution to }  B_4. 
}
\end{equation}
  \label{rem:scaling}
To find the critical Sobolev index, we compute
%
%
\begin{equation}
\begin{split}
  \| u_{\lambda} \|_{\dot{H}^s(\ci)} 
  & = \lambda^{2} \| u(\lambda x, \lambda^2 t) \|_{\dot{H}^{s}(\ci)}
  \\
  & = \lambda^{2} \left( \int_{\rr} | \xi |^{2s} | \wh{u(\lambda x,
  \lambda^{2} t)}^x (\xi, t)| \right)^{1/2}.
\end{split}
\label{crit-ind-comp}
\end{equation}
%
But
%
%
\begin{equation*}
\begin{split}
  \wh{u(\lambda x, \lambda^{2}t)^x}(\xi, t)
  & = \int_{\rr}e^{-i\xi x}u(\lambda x, \lambda^2 t) dx
  \\
  & = \frac{1}{\lambda} \int_{\rr}e^{-i \frac{n}{\lambda} x'}u(x',
  \lambda^{2} t) dx'
  \\
  & = \frac{1}{\lambda} \wh{u(\cdot, \lambda^{2}t)}(\frac{\xi}{\lambda})
\end{split}
\end{equation*}
%
%
Substituting back into \eqref{crit-ind-comp}, we obtain
%
%
\begin{equation*}
\begin{split}
  \| u_{\lambda} \|_{\dot{H}^s(\rr)} 
  & = \lambda^{2} \left( \int_{\rr} | \xi |^{2s} |
  \frac{1}{\lambda}\wh{u(\cdot, \lambda^{2}t)}(\frac{\xi}{\lambda}) |^2 d \xi
  \right)^{1/2}
  \\
  & = \lambda \left( \int_{\rr}| \xi |^{2s} | \wh{u(\cdot,
  \lambda^{2}t)}(\frac{\xi}{\lambda}) |^2 d \xi  \right)^{1/2}
  \\
  & = \lambda \left( \int_{\rr} | \lambda \xi' |^{2s} 
  \wh{u(\cdot, \lambda^{2}t)}(\xi') |^2 \lambda d \xi
  \right)^{1/2}
  \\
  & = \lambda^{s + 3/2} \|u(\cdot, \lambda^{2}t) \|_{\dot{H}^s (\ci)}.
\end{split}
\end{equation*}
%
%
Therefore, $\| u_{\lambda(0)} \|_{\dot{H}^s(\rr)} = \lambda^{s + 3/2} \|
u_{0} \|_{\dot{H}^{s}(\rr)}$, and so $s=-3/2$ is the critical Sobolev index.
\end{remark}
\end{framed}
\nin Since the scaling conserves data in $\dot{H}^{-3/2}$......
It seems that this equation is ``like KdV".
So one may expect KdV type theorems...
That is, $s_c=-3/4$ on the line and $s_c=-1/2$ on the circle,
if one uses bilinear estimates.
But, Kappeler and collaborators went all the way to $-1$ for KdV.
However KdV is integrable. Is this equation integrable?
Also, people conjecture that the critical index for KdV well-posedness 
in some appropriate sense should be the scaling index which is  $-3/2$.


  \item for 
    \[
    u_{tt}-u_{xx}+(u^2)_{xx}\,=\,0,
    \]
    one has 
    \[
    u_{\lambda}(t,x)\,=\,u\left(\frac{t}{\lambda}, \frac{x}{\lambda}\right),
    \]
    which leads to $s_c=\frac 12$.
\begin{framed}
\begin{remark}
We now prove this. 
Let $u(x, t)$ be a solution to the $B_2$ equation, that is
%
$$
B_2(u)=
 \partial_t^2u - \partial^2_x u + \partial_x^2(u^2)  = 0
$$
%
We would like to find the constants
$a, b, c$ such that
\[
u_\lambda (x, t) = \lambda^a u(\lambda^b x, \lambda^c t)
\]
is also a solution to $B_2$.  Since 
$$
B_2(u_\lambda)=
\lambda^{a+2c} \partial_t^2u 
-
 \lambda^{a+2b} \partial^2_x u 
 +
  \lambda^{2a+2b}
  \partial_x^2(u^2),  
$$
we see that $u_\lambda$ is a $B_2$ solution only if
$$
a+2c=a+2b=2a+2b,
$$
or
$
a=0, b=c.
$
  Thus
\[
u_\lambda (x, t) = u(\lambda^{b}x,  \lambda^{b} t).
\]
%
%
Therefore, replacing  $ \lambda^b$ with  $ \lambda$ gives the following scaling:
%
\begin{equation}
\label{B2-scal}
\boxed{
u(x, t) \text{ solution to }  B_2
 \Longrightarrow 
u_\lambda (x, t) = u(\lambda x, \lambda t)  \text { is also a
solution to }  B_2. 
}
\end{equation}
  \label{rem:scaling-B2}
To find the critical Sobolev index, we compute
\\
%
%
\begin{equation}
\begin{split}
  \| u_{\lambda} \|_{\dot{H}^s(\ci)} 
  & =  \| u(\lambda x, \lambda t) \|_{\dot{H}^{s}(\ci)}
  \\
  & = \left( \int_{\rr} | \xi |^{2s} | \wh{u(\lambda x,
  \lambda t)}^x (\xi, t)| \right)^{1/2}.
\end{split}
\label{crit-ind-comp-B2}
\end{equation}
%
But
%
\\
%
\begin{equation*}
\begin{split}
  \wh{u(\lambda x, \lambda t)^x}(\xi, t)
  & = \int_{\rr}e^{-i\xi x}u(\lambda x, \lambda t) dx
  \\
  & = \frac{1}{\lambda} \int_{\rr}e^{-i \frac{n}{\lambda} x'}u(x',
  \lambda t) dx'
  \\
  & = \frac{1}{\lambda} \wh{u(\cdot, \lambda t)}(\frac{\xi}{\lambda})
\end{split}
\end{equation*}
%
%
Substituting back into \eqref{crit-ind-comp-B2}, we obtain
%
%
\begin{equation*}
\begin{split}
  \| u_{\lambda} \|_{\dot{H}^s(\rr)} 
  & = \left( \int_{\rr} | \xi |^{2s} |
  \frac{1}{\lambda}\wh{u(\cdot, \lambda t)}(\frac{\xi}{\lambda}) |^2 d \xi
  \right)^{1/2}
  \\
  & = \frac{1}{\lambda} \left( \int_{\rr}| \xi |^{2s} | \wh{u(\cdot,
  \lambda t)}(\frac{\xi}{\lambda}) |^2 d \xi  \right)^{1/2}
  \\
  & = \frac{1}{\lambda} \left( \int_{\rr} | \lambda \xi' |^{2s} 
  \wh{u(\cdot, \lambda)}(\xi') |^2 \lambda d \xi
  \right)^{1/2}
  \\
  & = \lambda^{s - 1/2} \|u(\cdot, t) \|_{\dot{H}^s (\rr)}.
\end{split}
\end{equation*}
%
%
Therefore, $\| u_{\lambda(0)} \|_{\dot{H}^s(\rr)} = \lambda^{s - 1/2} \|
u_{0} \|_{\dot{H}^{s}(\rr)}$, and so $s=1/2$ is the critical Sobolev index.
\end{remark}
\end{framed}


\end{itemize}

This might suggest that the current results are not optimal.


%

%
%%%%%%%%%%%%%%%%%%%%%%%%%%%%%%%%%%%%%%%%%%%%%%%%%%%%%
%
%
%             Fourth order Modified Boussinesq  equation
%
%
%%%%%%%%%%%%%%%%%%%%%%%%%%%%%%%%%%%%%%%%%%%%%%%%%%%%%


%
\section{Fourth order Modified Boussinesq  equation}
%
We consider the initial value problem (ivp) for the fourth order modified Boussinesq
($B_4$) equation 
\begin{gather}
  u_{tt}   + u_{xxxx} + (u^2)_{xx} = 0,
  \label{eqn:mb-2}
  \\
  u(x,0) = u_{0}(x), \quad \p_t u(x, 0) = u_1(x), 
  \label{eqn:mb-init-data-2}
  \\
  \notag
  (u_0, u_1) \in
  H^{s}(\ci) \times
  H^{s-1}(\ci)  
\end{gather}
and prove the following.
%
%
%%%%%%%%%%%%%%%%%%%%%%%%%%%%%%%%%%%%%%%%%%%%%%%%%%%%%
%
%
%                Main Theorem
%
%
%%%%%%%%%%%%%%%%%%%%%%%%%%%%%%%%%%%%%%%%%%%%%%%%%%%%%
%
%
\begin{theorem}
  The $B_{4}$ ivp is well-posed
  \begin{enumerate}[(i)]
    \item In $H^s(\rr)$ if $s > -1/4$
    \item In $H^{s}(\ci)$ if $s > 1/4$,
  \end{enumerate}
  and the data-to-solution map is locally Lipschitz. Furthermore, these results
  are optimal in the sense that uniform continuity of the flow map fails for $s
  \le 1/4$ in the periodic case and $s < -1/4$ in the real case. 
  \label{thm:wp-2}
\end{theorem}
%

%
%%%%%%%%%%%%%%%%%%%%%%%%%%%%
%
%
%           Scaling for B4
%
%
%%%%%%%%%%%%%%%%%%%%%%%%%%%
%
%
%



%
%
%%%%%%%%%%%%%%%%%%%%%%%%%%%%%%%%%%%%%%%%%%%%%%%%%%%%%
%
%
%                The Periodic Case
%
%
%%%%%%%%%%%%%%%%%%%%%%%%%%%%%%%%%%%%%%%%%%%%%%%%%%%%%
%
%
\subsection{The Periodic Case} 
\label{ssec:periodic-case}
We will first rewrite the $B_4$ ivp
\eqref{eqn:mb-2}-\eqref{eqn:mb-init-data-2} in integral form. Consider
the linear $B_4$ ivp
\begin{gather}
  u_{tt} + u_{xxxx} = 0,
  \label{lin-mb}
  \\
  u(x, 0)=u_{0}(x), \quad u_{t}(x,0) = u_{1}(x).
  \label{lin-mb-init-data-1}
\end{gather}
Taking the spatial Fourier transform yields the ivp
\begin{gather*}
  \wh{u_{tt}^{x}} + n^{4} \wh{u^{x}} = 0,
  \\
  \wh{u^{x}}(\cdot, 0) = \wh{u_{0}}(n), \quad
  \wh{u_{t}^{x}}(\cdot, 0) = \wh{u_{1}}(n)
\end{gather*}
which admits the unique solution (see appendix)
%
%
\begin{equation*}
  \begin{split}
    \wh{u^{x}}(n, t) = \wh{u_{0}}(n) \frac{e^{in^{2}t} + e^{in^{2}t}}{2} + 
    \wh{u_{1}}(n) \frac{e^{in^{2}t} - e^{-in^{2}t}}{2i n^{2}}.
  \end{split}
\end{equation*}
%
%
%
%
\begin{framed}
\begin{remark}
  Note that $$g(n) \doteq \frac{e^{in^{2}t} - e^{-in^{2}t}}{2i n^{2}}$$ has a removable
  singularity at $n=0$. Since $$\lim_{n \to 0} g(n) = t$$ we may analytically
  extend $g(n)$ to the entire complex plane. 
\label{rem:analytic-extension}
\end{remark}
\end{framed}
%
%
%
Therefore,
%
%
\begin{equation*}
  \begin{split}
    u(x,t) = R_t u_{0} + S_{t}u_{1}
  \end{split}
\end{equation*}
%
is the unique solution to the ivp
\eqref{lin-mb}-\eqref{lin-mb-init-data-1}, where $R_{t}$ and $S_{t}$ are linear operators defined via the relation
%
%
\begin{gather}
  \label{sin-cos-op}
  \wh{R_{t}\vp} = \wh{\vp}(n) \frac{e^{in^{2}t} + e^{-in^{2}t}}{2} , \quad 
  \wh{S_{t}\vp} = \wh{\vp}(n) \frac{e^{in^{2}t} - e^{-in^{2}t}}{2i n^{2}}.
\end{gather}
%
%
By Duhamel's principle, it
follows that the $B_4$ ivp \eqref{eqn:mb-2}-\eqref{eqn:mb-init-data-2} can
be written in the integral form (see appendix for all details)
%
%
\begin{equation}
  \begin{split}
    u(x,t) = R_{t}u_{0} + S_{t}u_{1} - \int_{0}^{t} S_{t-t'}
    (u^{2})_{xx} dt'
  \end{split}
  \label{eqn:integral-form}
\end{equation}
%
%
which we will now localize in time.
Let $\psi(t)$ be a cutoff function symmetric about the 
origin such that $\psi(t) = 1$ for $|t| \le T$ and $\text{supp} \, \psi 
= [-2T, 2T ]$. Multiplying the right hand side of expression
$\eqref{eqn:integral-form}$ by $\psi(t)$, we obtain
%
%
\begin{equation}
  \begin{split}
    \psi(t) u(x,t)
    & = \psi(t) R_{t} u_{0} + \psi(t) S_{t}u_{1} -
    \psi(t) \int_{0}^{t} S_{t-t'}
    (u^{2})_{xx} dt'
    \\
    & \doteq Tu
  \end{split}
  \label{localized-int-eqn}
\end{equation}
where $T=T_{u_0, u_1}$.We now introduce the following spaces. 
%
%
\begin{definition}
  Let $\mathcal{Y}$ be the space of functions $F(\cdot)$ such that
  \begin{enumerate}[(i)]
   \item{$F: \ci \times \rr \to \cc$ }.
   \item{ $F(x, \cdot) \in S(\rr)$ for each $x \in \ci$}.
   \item{ $F(\cdot, t) \in C^{\infty}(\ci)$for each $t \in \rr$}.
  \end{enumerate}
  For $s, b \in \rr$, $X_{s,b}$ denotes the completion of $\mathcal{Y}$ with
  respect to the norm
  %
  %
  \begin{equation}
  \begin{split}
    \|F\|_{X_{s,b}} = \left( \sum_{n \in \zz} (1 + |n|)^{2s} \int_{\rr}
    (1 + | | \tau | - n^{2} |)^{2b} \wh{F}(n, \tau) d \tau\right)^{1/2}.
  \end{split}
  \label{eqn:bous-norm}
  \end{equation}
  %
  %
  %
  %
\end{definition}
%
The $X_{s,b}$ spaces have the following important embedding, whose proof is
provided in the appendix.
%
%
%%%%%%%%%%%%%%%%%%%%%%%%%%%%%%%%%%%%%%%%%%%%%%%%%%%%%
%
%
%               Embedding 
%
%
%%%%%%%%%%%%%%%%%%%%%%%%%%%%%%%%%%%%%%%%%%%%%%%%%%%%%
%
%
\begin{lemma}
  Let $b > 1/2$. Then $X_{s, b} \subset C(\rr, H^s)$ continuously. That is, there exists $c>0$ depending only on $b$ such that
%
%
\begin{equation*}
\begin{split}
  \| u \|_{C(\rr, H^s) } \le c \| u \|_{X_{s,b}}.
\end{split}
\end{equation*}
%
\label{lem:embedding}
\end{lemma}
%
%
We will 
show that for initial data $\vp \in {H}^s(\ci)$, $T$ is a contraction on $B_M 
\subset {X}_{s,b}$, where $B_M$ is the ball centered at the origin of radius $M = 
M_{\vp}> 0$, by estimating the $X_{s,b}$
norm of \eqref{localized-int-eqn}. The Picard fixed point theorem will
then yield a unique solution to
\eqref{localized-int-eqn}. An application of \autoref{lem:embedding}
will then imply the existence of a unique, local
solution $u \in C([-T, T], H^s(\ci))$ to the $B_4$ ivp. Local Lipschitz continuity of the flow map will follow from estimates used to establish the contraction mapping. %
%
%
%
%
%
%
%
%
%
\subsubsection{Estimate for $\psi(t) R_{t}u_{0}$.} 
\label{sssec:est-init-term-1}
We have
%
%
\begin{equation*}
  \begin{split}
    \wh{\psi(t)R_{t}u_{0}}^{x}(n, t)
    & = \psi(t) \wh{u_{0}}(n) \frac{e^{in^2 t} + e^{-in^{2}t}}{2}
    \\
    & = \frac{\psi(t) \wh{u_{0}}(n)e^{in^{2}t}}{2} + \frac{\psi(t)
    \wh{u_{0}}(n)e^{-in^{2}t}}{2}  
  \end{split}
\end{equation*}
%
%
and
%
%
\begin{equation*}
  \begin{split}
    \wh{\psi(t) R_{t}u_{0}}(n, \tau) = \frac{\wh{\psi}(\tau -
    n^{2})\wh{u_{0}}(n)}{2} + \frac{\wh{\psi}(\tau - n^{2})\wh{u_{0}}(n)}{2}.
  \end{split}
\end{equation*}
%
%
Hence, substituting and applying the inequality $(a + b)^{2} \le 4(a^{2} +
b^{2}),\ a, b \in \rr$, we have

%
%
\begin{align}
    & \| \psi(t) R_{t}u_{0} \|_{X_{s,b}}^{2} 
    \notag
    \\
    & = \sum_{n \in \zz}(1 + |n|)^{2s} \int_{\rr}\left( 1 + | | \tau
    |-n^{2} | \right)^{2b} | \frac{\wh{\psi}(\tau - n^{2})\wh{u_{0}(n)}}{2} +
    \frac{\wh{\psi}(\tau + n^{2})\wh{u_{0}}(n)}{2} |^{2} d \tau
    \notag
    \\
    & \le \sum_{n \in \zz} \left( 1 + |n| \right)^{2s} | \wh{u_{0}}(n)
    |^{2} \int_{\rr} | \wh{\psi}(\tau - n^{2}) |^{2}\left( 1 + | | \tau | -
    n^{2} | \right)^{2b} d \tau
    \label{u-0-norm-comp-1}
    \\
    & + \sum_{n \in \zz} \left( 1 + |n| \right)^{2s} | \wh{u_{0}}(n)
    |^{2} \int_{\rr} | \wh{\psi}(\tau + n^{2}) |^{2}\left( 1 + | | \tau | -
    n^{2} | \right)^{2b} d \tau.
    \label{u-0-norm-comp-3}
\end{align}
%
Noting that
\begin{equation}
  \begin{split}
    | | \tau | - n^{2} | \le \min\left\{ | \tau - n^{2} |, | \tau + n^{2} | \right\}
  \end{split}
  \label{eqn:norm-key-ineq}
\end{equation}
%
%
we bound \eqref{u-0-norm-comp-1} by
%
%
\begin{equation*}
  \begin{split}
    & \sum_{n \in \zz} \left( 1 + |n| \right)^{2s} | \wh{u_{0}}(n)
    |^{2} \int_{\rr} | \wh{\psi}(\tau - n^{2}) |^{2}\left( 1 +  | \tau  -
    n^{2} | \right)^{2b} d \tau
    \\
    & = \sum_{n \in \zz} \left( 1 + |n| \right)^{2s} | \wh{u_{0}}(n)
    |^{2} \int_{\rr} | \wh{\psi}(\tau') |^{2}\left( 1 +  | \tau'| \right)^{2b} d \tau
    \\
    & = c_{\psi, b} \sum_{n \in \zz} \left( 1 + |n| \right)^{2s} | \wh{u_{0}}(n)
    |^{2} 
    \\
    & = c_{\psi, b} \| u_{0} \|_{H^{s}}^{2}
  \end{split}
\end{equation*}
%
%
where $c_{\psi, b}$ is a constant depending only on $\psi$ and $b$. The
term \eqref{u-0-norm-comp-3} is bounded in similar fashion. Therefore, 
$\|\psi(t) R_{t} u_{0}\|_{X_{s,b}}^{2} \le c_{\psi, b}
\|u_{0}\|_{H^s}^2$ and
taking square roots of both sides gives
%
%
\begin{equation}
  \begin{split}
    \|\psi(t) R_{t} u_{0}\|_{X_{s,b}} \le c_{\psi, b}
    \|u_{0}\|_{H^s}.
  \end{split}
  \label{eqn:u-0-fin-est}
\end{equation}
%
%

\subsubsection{Estimate for $\psi(t) S_{t}u_{1}$.}
\label{sssec:estimate-init-term-2}
We have
%
%
\begin{equation*}
  \begin{split}
    \wh{\psi(t)S_{t}u_{1}}^{x}(n, t)
    & = \psi(t) \wh{u_{1}}(n) \frac{e^{in^2 t} - e^{-in^{2}t}}{2i n^{2}}
    \\
    & = \frac{\psi(t) \wh{u_{1}}(n)e^{in^{2}t}}{2i n^{2}} - \frac{\psi(t)
    \wh{u_{1}}(n)e^{-in^{2}t}}{2i n^{2}}  
  \end{split}
\end{equation*}
%
%
and
%
%
\begin{equation*}
  \begin{split}
    \wh{\psi(t) S_{t}u_{1}}(n, \tau) = \frac{\wh{\psi}(\tau -
    n^{2})\wh{u_{1}}(n)}{2i n^{2}} + \frac{\wh{\psi}(\tau - n^{2})\wh{u_{1}}(n)}{2i
    n^{2}}.
  \end{split}
\end{equation*}
%
Note that 
%
\begin{equation*}
  \begin{split}
    \wh{\psi(t)S_{t}u_{1}}^{x}(0, t)
    & = \psi(t) \wh{u_{1}}(0) t
      \end{split}
\end{equation*}
and so 
%
%
\begin{equation*}
  \begin{split}
    \wh{\psi(t) S_{t}u_{1}}(0, \tau) = \frac{d}{d \tau} \wh{\psi}(\tau)
    \wh{u_{1}}(0).
  \end{split}
\end{equation*}
%
Hence, substituting and applying the inequality $(a + b)^{2} \le 4(a^{2} +
b^{2}),\ a, b \in \rr$, we have
%
%
\begin{equation}
  \begin{split}
    \| \psi(t) S_{t}u_{1} \|_{X_{s,b}}^{2} 
    & = \sum_{n \in \zzdot}(1 + |n|)^{2s} \int_{\rr}\left( 1 + | | \tau
    |-n^{2} | \right)^{2b} | \frac{\wh{\psi}(\tau - n^{2})\wh{u_{1}(n)}}{2i
    n^{2}} -
    \frac{\wh{\psi}(\tau + n^{2})\wh{u_{1}}(n)}{2i n^{2}} |^{2} d \tau
    \\
    & + |\wh{u_{1}}(0)|^{2} \int_{\rr} (1 + | \tau |)^{2b} \frac{d }{d \tau}
    \wh{\psi}(\tau) d \tau
    \\
    & \le \sum_{n \in \dot{\zz}} n^{-2s} \left( 1 + |n| \right)^{2s} | \wh{u_{1}}(n)
    |^{2} \int_{\rr} | \wh{\psi}(\tau - n^{2}) |^{2}\left( 1 + | | \tau | -
    n^{2} | \right)^{2b} d \tau
    \\
    & + \sum_{n \in \dot{\zz}} n^{-2s} \left( 1 + |n| \right)^{2s} | \wh{u_{1}}(n)
    |^{2} \int_{\rr} | \wh{\psi}(\tau + n^{2}) |^{2}\left( 1 + | | \tau | -
    n^{2} | \right)^{2b} d \tau
    \\
    & + |\wh{u_{1}}(0)|^{2} \int_{\rr} (1 + | \tau |)^{2b} |\frac{d }{d \tau}
    \wh{\psi}(\tau)|^2 d \tau.
\end{split}
\label{u-1-norm-comp-pre}
\end{equation}
%
%
Applying the inequality
%
%
\begin{equation*}
\begin{split}
  \frac{(1 + |n|)^{2s}}{ n^{2}} \le \frac{(1 + |n|)^{2s}}{\frac{1}{4}(1 +
  |n|)^{2}} = 4 (1 + | n |)^{2(s-1)},  \quad s \in \rr, \quad n \ge 1
\end{split}
\end{equation*}
%
to \eqref{u-1-norm-comp-pre} gives
%
\begin{equation}
  \begin{split}
    \| \psi(t) S_{t}u_{1} \|_{X_{s,b}}^{2} 
    & \lesssim \sum_{n \in \dot{\zz}} \left( 1 + |n| \right)^{2(s-1)} | \wh{u_{1}}(n)
    |^{2} \int_{\rr} | \wh{\psi}(\tau - n^{2}) |^{2}\left( 1 + | | \tau | -
    n^{2} | \right)^{2b} d \tau
    \\
    & + \sum_{n \in \dot{\zz}} \left( 1 + |n| \right)^{2(s-1)} | \wh{u_{1}}(n)
    |^{2} \int_{\rr} | \wh{\psi}(\tau + n^{2}) |^{2}\left( 1 + | | \tau | -
    n^{2} | \right)^{2b} d \tau
    \\
    & + |\wh{u_{1}}(0)|^{2} \int_{\rr} (1 + | \tau |)^{2b} |\frac{d }{d \tau}
    \wh{\psi}(\tau)|^2 d \tau.
\end{split}
\label{u-1-norm-comp}
\end{equation}
%
%
Applying \eqref{eqn:norm-key-ineq},
we bound the first term of
\eqref{u-1-norm-comp} by
%
%
%
\begin{equation*}
  \begin{split}
    & \sum_{n \in \dot{\zz}} \left( 1 + |n| \right)^{2(s-1)} | \wh{u_{1}}(n)
    |^{2} \int_{\rr} | \wh{\psi}(\tau - n^{2}) |^{2}\left( 1 +  | \tau  -
    n^{2} | \right)^{2b} d \tau
    \\
    & = \sum_{n \in \dot{\zz}} \left( 1 + |n| \right)^{2(s-1)} | \wh{u_{1}}(n)
    |^{2} \int_{\rr} | \wh{\psi}(\tau') |^{2}\left( 1 +  | \tau'| \right)^{2b} d \tau
    \\
    & = c_{\psi, b} \sum_{n \in \dot{\zz}} \left( 1 + |n| \right)^{2(s-1)} | \wh{u_{1}}(n)
    |^{2} 
    \\
    & \le c_{\psi, b} \| u_{1} \|_{H^{s-1}}^{2}
  \end{split}
\end{equation*}
%
%
where $c_{\psi, b}$ is a constant depending only on $\psi$ and $b$. Applying
\eqref{eqn:norm-key-ineq} again, the
second term of \eqref{u-1-norm-comp} is bounded by
\begin{equation*}
  \begin{split}
    & \sum_{n \in \dot{\zz}} \left( 1 + |n| \right)^{2(s-1)} | \wh{u_{1}}(n)
    |^{2} \int_{\rr} | \wh{\psi}(\tau + n^{2}) |^{2}\left( 1 +  | \tau  -
    n^{2} | \right)^{2b} d \tau
    \\
    & = \sum_{n \in \dot{\zz}} \left( 1 + |n| \right)^{2(s-1)} | \wh{u_{1}}(n)
    |^{2} \int_{\rr} | \wh{\psi}(\tau') |^{2}\left( 1 +  | \tau'| \right)^{2b} d \tau
    \\
    & = c_{\psi, b} \sum_{n \in \dot{\zz}} \left( 1 + |n| \right)^{2(s-1)} | \wh{u_{1}}(n)
    |^{2} 
    \\
    & \le c_{\psi, b} \| u_{1} \|_{H^{s-1}}^{2}
  \end{split}
\end{equation*}
while the third term is bounded by  
%
%
\begin{equation*}
\begin{split}
  c_{\psi, b} \| u_{1} \|_{H^{s-1}}^{2}.
\end{split}
\end{equation*}
%
%
Therefore, 
$\|\psi(t) S_{t} u_{1}\|_{X_{s,b}}^{2} \le c_{\psi, b}
\|u_{1}\|_{H^{s-1}}^2$ and
taking square roots of both sides gives
%
%
\begin{equation}
  \begin{split}
    \|\psi(t) S_{t} u_{1}\|_{X_{s,b}} \le c_{\psi, b}
    \|u_{1}\|_{H^{s-1}}.
  \end{split}
  \label{eqn:u-1-fin-est}
\end{equation}

\subsubsection{Estimate for $\psi(t) \int_{0}^{t} S_{t-t'} (u^{2})_{xx} dt'$.}
\label{sssec:non-lin-term}
We define the spatial Fourier transform by 
%
%
\begin{equation*}
\begin{split}
  \tilde{f}(n, t) = \int_{\ci} e^{-inx}f(x,t) dx
\end{split}
\end{equation*}
%
%
and the spacetime Fourier transform by
\begin{equation*}
\begin{split}
  \wh{f}(n, \tau) = \int_{\rr} \int_{\ci} e^{-inx-it\tau}f(x,t) dx dt
\end{split}
\end{equation*}
%
%
Let $f(x,t) \doteq \psi(t) \int_{0}^{t} S_{t-t'} (u^{2})_{xx} dt'$. 
Then
%
%
\begin{equation}
  \begin{split}
    \wt{f}(n, t)
    & = \frac{\psi(t)}{2i} \int_{0}^{t}\wt{u^{2}}(n, t') \left[
    e^{in^{2}(t-t')} - e^{-in^{2}(t-t')}
    \right] dt'
    \\
    & = \frac{1}{2i} e^{in^{2}t} \int_{0}^{t} \psi(t) \wt{u^{2}}(n, t') e^{-in^{2}t'}
    dt' - 
    \frac{1}{2i} e^{-in^{2}t} \int_{0}^{t} \psi(t) \wt{u^{2}}(n, t') e^{in^{2}t'} dt'
    \\
    & \doteq e^{in^{2}t} \wt{w_1}(n, t) - e^{-in^{2}t} \wt{w_2}(n, t)
  \end{split}
  \label{space-four-trans}
\end{equation}
%
where
%
%
\begin{gather*}
  w_{1}(x,t) = \frac{1}{4 i \pi} \sum_{n \in \zz} e^{inx} \left[ \int_{0}^{t} \psi(t) \wt{u^{2}}(n, t') e^{in^{2}t'}
  dt'\right],
  \\
  w_{2}(x,t) = \frac{1}{4 i\pi} \sum_{n \in \zz} e^{inx} \left[ \int_{0}^{t} \psi(t) \wt{u^{2}}(n, t') e^{-in^{2}t'} dt'
 \right].
\end{gather*}
%
%
%
Notice that \eqref{space-four-trans} is a \emph{global} relation in $t$.
Hence, taking its time Fourier transform gives
%
%
\begin{equation}
  \label{full-fourier-trans-exp}
\begin{split}
  \wh{f}(n, \tau) = \wh{w_{1}}(n, \tau - n^{2}) - \wh{w_{2}}(n, \tau +
  n^{2}).
\end{split}
\end{equation}
%
%
Therefore, using the definition of the $X_{s,b}$ spaces, and applying the
inequality 
%
%
\begin{equation*}
\begin{split}
  (a + b)^{2} \le 4(a^{2} + b^{2})
\end{split}
\end{equation*}
%
%
gives 
%
%
\begin{equation*}
\begin{split}
  \| f \|_{X_{s,b}}^{2}
  & = \sum_{n \in \zz} (1 + |n|)^{2s} \int_{\rr} (1 + |
  | \tau | - n^{2} |)^{2b} | \wh{w_{1}}(n, \tau - n^{2}) - \wh{w_{2}}(n, \tau +
  n^{2}) |^{2} d \tau
  \\
  & \le 4 \sum_{n \in \zz} (1 + |n|)^{2s} \int_{\rr} (1 + |
  | \tau | - n^{2} |)^{2b} | \wh{w_{1}}(n, \tau - n^{2}) d \tau
  \\
  & + 4 \sum_{n \in \zz} (1 + |n|)^{2s} \int_{\rr} (1 + |
  | \tau | - n^{2} |)^{2b} | \wh{w_{1}}(n, \tau + n^{2}) d \tau.
\end{split}
\end{equation*}
%
%
Applying a change of variable implies
%
%
%
%
\begin{equation}
\begin{split}
  \| f \|_{X_{s,b}}^{2}
  & \le 4 \sum_{n \in \zz} (1 + |n|)^{2s} \int_{\rr} (1 + |
  | \tau + n^{2} | - n^{2} |)^{2b} | \wh{w_{1}}(n, \tau) |^2 d \tau
  \\
  & + 4 \sum_{n \in \zz} (1 + |n|)^{2s} \int_{\rr} (1 + |
  | \tau - n^{2} | - n^{2} |)^{2b} | \wh{w_{2}}(n, \tau )|^2 d \tau.
\end{split}
\label{comp-pre-lemma}
\end{equation}
%
%
We now need the following lemma, whose proof is provided in the appendix.
%
%
%%%%%%%%%%%%%%%%%%%%%%%%%%%%%%%%%%%%%%%%%%%%%%%%%%%%%
%
%
%                Bound for modified principal symbol
%
%
%%%%%%%%%%%%%%%%%%%%%%%%%%%%%%%%%%%%%%%%%%%%%%%%%%%%%
%
%
\begin{lemma}
For any $n, \tau \in \rr$, we have
\label{lem:mod-princ-symb-bound}
%
%
\begin{equation*}
\begin{split}
  \max\left\{ | | \tau + n^{2} | - n^{2} |, | | \tau - n^{2} | - n^{2} |
  \right\} \le | \tau |.
\end{split}
\end{equation*}
%
%
\end{lemma}
%
%
Applying \autoref{lem:mod-princ-symb-bound} to \eqref{comp-pre-lemma} yields
%
%
\begin{equation*}
\begin{split}
\| f \|_{X_{s,b}}^{2}
  & \le 4 \sum_{j=1}^{2}  \sum_{n \in \zz} (1 + |n|)^{2s} \int_{\rr} (1 + |
  \tau|)^{2b} | \wh{w_{j}}(n, \tau)|^2 d \tau
  \\
  & = 4 \sum_{j=1}^{2} \sum_{n \in \zz} (1 + |n|)^{2s} \|\wt{w_{j}}(n, t)
  \|^{2}_{H_{t}^{b}}
  \\
  & = \sum_{n \in \zz} \| \psi(t) \int_{0}^{t} \wt{u^2}(n, t')
  e^{in^{2}t'}dt'  \|_{H_{t}^{b}}
  + 
  \sum_{n \in \zz} \| \psi(t) \int_{0}^{t} \wt{u^2}(n, t')
  e^{-in^{2}t'}dt'  \|_{H_{t}^{b}}.
\end{split}
\end{equation*}
%
We now need the following, whose proof is provided in \cite{Ginibre:1996fk} and the appendix.
%
%
%%%%%%%%%%%%%%%%%%%%%%%%%%%%%%%%%%%%%%%%%%%%%%%%%%%%%
%
%
%                Lemma to Reduce to Bilinear Est Form
%
%
%%%%%%%%%%%%%%%%%%%%%%%%%%%%%%%%%%%%%%%%%%%%%%%%%%%%%
%
%
\begin{lemma}
Let $-1/2 < b' \le 0 < b \le b' + 1$. Then
%
%
\begin{equation*}
\begin{split}
  \| \psi(t) \int_{0}^{t} g(t') dt' \|_{H^{b}_{t}} \le T^{1-(b-b')} \| g
  \|_{H_{t}^{b'}}.
\end{split}
\end{equation*}
%
%
\label{lem:pre-bilin-est}
\end{lemma}
%
%
Applying the lemma, we bound the right hand side of \eqref{comp-pre-lemma} by
%
%
\begin{equation*}
\begin{split}
  & T^{1-(b - b')} \left( \sum_{n \in \zz} (1 + |n|)^{s} \| \wt{u^{2}}(n, t')
  e^{in^{2}t'} \|_{H_{t}^{b'}}  +
  \sum_{n \in \zz} (1 + |n|)^{s} \| \wt{u^{2}}(n, t')
  e^{-in^{2}t'} \|_{H_{t}^{b'}} \right)
  \\
  & = T^{1-(b -b')}\left( \sum_{n \in \zz} (1 + |n|)^{s} \int_{\rr} (1 + | \tau
  |)^{2b'} \wh{u^{2}}(n, \tau - n^{2}) d \tau \right. 
  \\
  & +
  \left . \sum_{n \in \zz} (1 + |n|)^{s} \int_{\rr} (1 + | \tau
  |)^{2b'} \wh{u^{2}}(n, \tau + n^{2}) d \tau  \right)
  \\
  & = T^{1-(b -b')}\left( \sum_{n \in \zz} (1 + |n|)^{s} \int_{\rr} (1 + | \tau
  + n^{2}
  |)^{2b'} \wh{u^{2}}(n, \tau ) d \tau \right .
  \\
  & +
  \left. \sum_{n \in \zz} (1 + |n|)^{s} \int_{\rr} (1 + | \tau
  - n^{2} |)^{2b'} \wh{u^{2}}(n, \tau) d \tau  \right).
\end{split}
\end{equation*}
%
%
Hence, applying \autoref{lem:mod-princ-symb-bound} we obtain
%
%
%
\begin{equation*}
\begin{split}
  \| f \|_{X_{s,b}}^{2}
  & \le T^{1-(b - b')} 
  \sum_{n \in \zz} (1 + |n|)^{s} \int_{\rr} (1 + | |\tau|
  - n^{2} |)^{2b'} \wh{u^{2}}(n, \tau) d \tau  
  \\
  & = T^{(1-b - b')} \|u^{2} \|_{X_{s,b'}}^{2}.
\end{split}
\end{equation*}
%
%
Taking square roots and substituting back in for $f$ gives
%
%
\begin{equation}
\begin{split}
  \|\psi(t) \int_{0}^{t} S_{t-t'} (u^{2})_{xx} dt'\|_{X_{s,b}} \le c_{\psi, b,
  b'} \| u^{2} \|_{X_{s,b'}}.
\end{split}
\label{eqn:non-lin-bound}
\end{equation}
%
%
To bound the right hand side, we now require a crucial bilinear
estimate.
%
%
%%%%%%%%%%%%%%%%%%%%%%%%%%%%%%%%%%%%%%%%%%%%%%%%%%%%%
%
%
%                Bilinear Estimate
%
%
%%%%%%%%%%%%%%%%%%%%%%%%%%%%%%%%%%%%%%%%%%%%%%%%%%%%%
%
%
\begin{proposition}
\label{prop:bilin-est}
  Let $s > 1/4$ and $u,v \in X_{s, -a}$. If $b > 1/2$, and $1/4 < a< 1/2$ 
  then there exists $c > 0$ depending only on $a$, $b$, and $s$ such that
  %
  %
  \begin{equation*}
  \begin{split}
    \| uv \|_{X_{s,-a}} \le c \| u \|_{X_{s,b}} \| v \|_{X_{s,b}}.
  \end{split}
  \end{equation*}
  %
  %
\end{proposition}
%
%
Applying the bilinear estimate to \eqref{eqn:non-lin-bound}, we conclude that
for given $s > -1/4$, we can choose $b$ and  $b'$ such that
%
%
\begin{equation}
\begin{split}
\|\psi(t) \int_{0}^{t} S_{t-t'} (u^{2})_{xx} dt'\|_{X_{s,b}} \le c_{\psi, b
  } \| u \|^2_{X_{s,b}}.
\end{split}
\label{eqn:nonlinear-term-bound}
\end{equation}
%
%
%
\begin{framed}
\begin{remark}
  Note that, in order to apply \autoref{prop:bilin-est}, our choice of $b'$
  depends on our choice of $b$. This is why the constant on the right hand side
  of \eqref{eqn:nonlinear-term-bound} depends only on $\psi$ and $b$. 
\label{rem:b'-fun-b}
\end{remark}
\end{framed}
%
%
%
\subsubsection{Proof of Existence and Uniqueness in the Periodic Case}
\label{sssec:proof-b4-per-case}
%
%
Collecting estimates \eqref{eqn:u-0-fin-est}, \eqref{eqn:u-1-fin-est}, and
\eqref{eqn:nonlinear-term-bound}, we 
we obtain the following.
%%
%%%%%%%%%%%%%%%%%%%%%%%%%%%%%%%%%%%%%%%%%%%%%%%%%%%%%
%
%% Contraction Proposition
%				 
%%%%%%%%%%%%%%%%%%%%%%%%%%%%%%%%%%%%%%%%%%%%%%%%%%%%%%
%%
%%
%
\begin{proposition}
\label{prop:contraction}
Let $s > \frac{1}{4}$. Then
%
%%
\begin{equation*}
	\begin{split}
    \|Tu\|_{X_{s,b}} \le c_{\psi, b} \left( \|u_0 \|_{H^s(\ci)} + \|u_1 \|_{H^{s-1}(\ci)}
    + \|u\|_{X_{s,b}}^2 
		\right).
	\end{split}
\end{equation*}
%
%%
\end{proposition}
We will now use \autoref{prop:contraction} to prove local well-posedness for the 
$B_4$ ivp. Let $c = c_{\psi, b}$. For given $u_0, u_1$, we may choose $\psi$ such
that 
%
%%
\begin{equation*}
	\begin{split}
    \|u_0\|_{H^s(\ci)} \le \frac{3}{32c^2}, \quad \|u_1\|_{H^{s-1}(\ci)} \le \frac{3}{32c^2}.
	\end{split}
\end{equation*}
%
%%
Then $$\|u\|_{X_{s,b}} \le \frac{1}{4c}$$ implies
%
%%
\begin{equation*}
	\begin{split}
		\|T u \|_{X_{s,b}} 
		& \le c \left[ \frac{3}{32c^2} + \frac{3}{32c^2}+ \left( 
		\frac{1}{4c} \right)^2 \right]
		=  \frac{1}{4c}.
	\end{split}
\end{equation*}
%
%%
Hence, $T=T_{u_0, u_1}$ maps the ball $B\left( 0, \frac{1}{4c} \right) \subset
X_{s,b}$ into itself. Next, note that 
%
%%
\begin{equation*}
	\begin{split}
		Tu - Tv = 
    \int_{0}^{t} S_{t-t'}
    (u^{2} - v^{2})_{xx} dt'.
  \end{split}
  \label{eqn:integral-form-dif}
\end{equation*}
%
%%
Rewriting
%
%%
\begin{equation*}
	\begin{split}
	\p_x^2 (u^2 - v^2)	
		& = \p_x^2[(u-v)(u+v)]
		\end{split}
\end{equation*}
%
%%
and repeating the arguments used in \autoref{sssec:non-lin-term},
we obtain
%
%%
%%
\begin{equation}
	\label{20a}
	\begin{split}
		\|Tu - Tv \|_{X_{s,b}}  
		& \le c_{\psi, b} \|u -v\|_{X_{s,b}} \|u + v \|_{X_{s,b}}
		\\
		& \le c_{\psi, b} \|u -v\|_{X_{s,b}} (\|u\|_{X_{s,b}}+ \|v \|_{X_{s,b}}).
	\end{split}
\end{equation}
%
%%
If $$ u, v \in B(0, \frac{1}{4c}) \subset X_{s,b},$$ then
%
%%
\begin{equation}
	\label{21a}
	\begin{split}
		\|Tu - Tv \|_{X_{s,b}}
		& \le c \|u -v \|_{X_{s,b}} \left( \frac{1}{4c} + 
		\frac{1}{4c} \right)
		\\
		& = \frac{1}{2} \|u -v \|_{X_{s,b}}. 
	\end{split}
\end{equation}
%
%%
We conclude that $T = T_{\vp}$ is a contraction on the ball $B(0, 
\frac{1}{4c}) \subset X_{s,b}$. A Picard iteration then yields a unique solution
$u \in X_{s,b}$ to \eqref{localized-int-eqn}. Applying
\autoref{lem:embedding}, it follows that $u(x,t) \subset C( [-T, T], H^s$ is a unique
solution of the B4 ivp \eqref{eqn:mb-2}-\eqref{eqn:mb-init-data-2} for $t
\in [-T, T]$.
%
%
\subsubsection{Proof of Lipschitz Continuity in the Periodic Case} 
\label{sssec:lip-continuity}
%
%
We first define our notion of continuity.
%
%
\begin{definition}
  Let $X, Y$ be Banach spaces, and equip $X \times Y$ with the product
  topology (i.e. if $\|(f_0, f_1)\|_{X \times Y} = \|f_0\|_{X} + \|f_1\|_{Y}$).
  We say that the data to solution
  map $(f_0, f_1) \mapsto f(t)$ of the ivp $T_{f_0, f_1} f =
  0$ is \emph{locally Lipschitz} in $X \times Y$ if for
  $(u_0, u_1), (v_0, v_1) \in B_R \doteq \{(f_0,f_1) \in X \times Y: \|f_0\|_{X} +
  \|f_1\|_{Y}< R\},$ there exist $C, T>0$ depending on $R$ and local solutions
  $u(x,t), v(x,t)$
  for $t \in [-T, T]$ of $T_{u_0, u_1}u=0, T_{v_0, v_1}v=0$ satisfying
	$$\|u(\cdot, t) - v(\cdot, t)
  \|_X \le C \left( \|u_{0} - v_0 \|_{X} + \|u_{1} - v_1 \|_{Y}
  \right), \quad t \in [-T, T].$$ We
	say the flow map is \emph{locally uniformly
	continuous} in $X$ if for
	$u_0, v_0 \in B_R$ there exists $T >0$ depending on $R$ and local solutions
  $u(x,t), v(x,t)$
  for $t \in [-T, T]$ of $T_{u_0, u_1}u=0, T_{v_0, v_1}v=0$ such that 
	$$ \|u(\cdot, t) - v(\cdot, t) \|_{X} \to
  0 \ \ \text{if}  \ \ \|u_0 - v_0 \|_{X}, \|u_1 - v_1 \|_{Y} \to 0, \quad
  t \in
  [-T, T]. $$ 
\end{definition}
%
%
Notice that any locally Lipschitz flow map is locally uniformly continuous. 
Next, we shall establish local Lipschitz continuity in $X_{s,b}$ of the flow
map. Let $(u_0, u_1), (v_0, v_1) \subset \in H^{s}(\ci) \times H^{s-1}(\ci)  $
be given. Choose $\psi$ such that $$(u_0, u_1), (v_0, v_1)  \subset B(0,
\frac{15}{64c^{3}}).$$ Then there exist $u, v \in X_{s,b}$ such that $u =
T_{u_0, u_1}$, $v = T_{v_0, v_1}$, and so
%
%
\begin{equation}
	\label{gen-1a}
	\begin{split}
		& T_{u_0, u_1}(u) - T_{v_0, v_1}(v)
		\\
    & = \psi(t ) R_{t}(u_{0} - v_0) + \psi(t) S_{t}(u_{1} - v_1)
    + \psi(t) \int_{0}^{t} S_{t-t'}
    (u^{2} - v^{2} )_{xx} dt'.
		\end{split}
\end{equation}
%
%
Using arguments similar to those in 
\autoref{sssec:est-init-term-1}-\autoref{sssec:estimate-init-term-2}
we obtain
%
%
\begin{equation}
	\label{gen-2a}
	\begin{split}
		& \| \psi(t ) R_t (u_0 - v_0)\|_{X_{s,b}}
		\le c_{\psi, b} \|u_0 -v_0\|_{H^s},
    \\
    & \| \psi(t) S_t (u_1 - v_1)\|_{X_{s,b}}
    \le c_{\psi, b} \|u_1 -v_1\|_{H^{s-1}}.
	\end{split}
\end{equation}
%
%
Therefore, from \eqref{21a}-\eqref{gen-2a}, we obtain
%
%
\begin{equation*}
	\begin{split}
    \|u -v \|_{X_{s,b}}
    & = \|T_{u_0, v_0}(u) - T_{u_1, v_1}(v) \|_{X_{s,b}}
    \\
    & \le
    c_{\psi, b} \left( \|u_0 -v_0 \|_{H^s\left( \ci \right)} +\|u_1 -v_1
        \|_{H^{s-1}\left( \ci \right)} + \frac{1}{2} \|u -v \|_{X_{s,b}}\right)
  \end{split}
\end{equation*}
%
%
which implies
%
%
\begin{equation*}
	\begin{split}
		\frac{1}{2} \|u-v\|_{X_{s,b}} \le
    c_{\psi, b} \left( \|u_0 -v_0 \|_{H^s\left( \ci \right)} +\|u_1 -v_1
        \|_{H^{s-1}\left( \ci \right)} \right )
      \end{split}
\end{equation*}
%
%
or
%
%
\begin{equation}
	\begin{split}
		\|u -v \|_{X_{s,b}} \le 2 c_{\psi, b} \left( \|u_0 -v_0 \|_{H^s\left( \ci \right)} +\|u_1 -v_1
        \|_{H^{s-1}\left( \ci \right)} \right ).
	\end{split}
  \label{pre-lem-estimate}
\end{equation}
%
%
Applying  
\autoref{lem:embedding} to \eqref{pre-lem-estimate}, we obtain 

%
%
%
	 %
	 %
	 \begin{equation*}
		 \begin{split}
			\|u(\cdot, t) -v(\cdot, t) \|_{H^s(\ci)} \le
      2 c_{\psi, b} \left( \|u_0 -v_0 \|_{H^s\left( \ci \right)} +\|u_1 -v_1
        \|_{H^{s-1}\left( \ci \right)} \right ).
		 \end{split}
	 \end{equation*}
	 %
	 %
Since $u,v$ are the unique solutions to the ivp
\eqref{localized-int-eqn}, it follows that $u(x,t), v(x,t), t \in [-T, T]$ are unique
local solutions to \eqref{eqn:mb-2} with
initial data $(u_0, u_1), (v_0, v_1)$, respectively.
Hence, the flow map of the $B_4$ ivp is locally Lipschitz continuous in
$H^s(\ci)$. This
concludes the proof of well-posedness for the $B_4$ ivp
\eqref{eqn:mb-2}-\eqref{eqn:mb-init-data-2}. \qquad \qedsymbol

%
%%%%%%%%%%%%%%%%%%%%%%%%%%%%%%%%%%%%%%%%%%%%%%%%%%%%%
%
%
%                The non-periodic Case
%
%
%%%%%%%%%%%%%%%%%%%%%%%%%%%%%%%%%%%%%%%%%%%%%%%%%%%%%
%
%
%
%%%%%%%%%%%%%%%%%%%%%%%%%%%%%%%%%%%%%%%%%%%%%%%%%%%%%
%
%
%                Proof of Bilinear Estimate B4 Per
%
%
%%%%%%%%%%%%%%%%%%%%%%%%%%%%%%%%%%%%%%%%%%%%%%%%%%%%%
%
%
\subsubsection{Proof of \autoref{prop:bilin-est}} 
\label{sssec:proof-bilin-est}
By
duality, it suffices to show that when $b>1$, $1/4 < a <1/2$ are satisfied, we have
%
%%
\begin{equation}
	\label{duality-est}
	\begin{split}
	|	\sum_{n \in \zzdot}  (1 + |n|)^{s}
		\int_{\rr} \phi(n, \tau) \wh{uv}(n, \tau)(1 
    + | |\tau| - n^{2} |^{-a}) d \tau | \lesssim \|u\|_{X_{s,b}}
    \|v\|_{X_{s,b}}
    \|\phi \|_{L^{2}_{n, \tau}}.
	\end{split}
\end{equation}
Note first that $|\wh{uv}(n, \tau) |  = | \wh{u} *  \wh{v} 
(n, \tau)|$. From this it follows that
%
%
\begin{equation}
	\label{non-lin-rep}
	\begin{split}
		| \wh{uv}(n, \tau)|
    & = | \sum_{n_{1} \in \zz }  \int
    \wh{u}\left( n_1,  \tau_1 \right) \wh{v}\left( n - n_1 , \tau - \tau_1   
\right) d \tau_1 |
\\
& \le  \sum_{n_{1} \in \zz }  \int
    |\wh{u}\left( n_1,  \tau_1 \right)| |\wh{v}\left( n - n_1 , \tau - \tau_1   
\right)| d \tau_1 
\\
& = \sum_{n_1 \in \zz } \int \frac{c_u\left( n_1, \tau_1 
\right)}{\langle n_1 \rangle ^s \langle |\tau_1| - n_1^{2} | \rangle ^{b}}
\\
& \times \frac{c_{v}\left( n - n_1, \tau - \tau_1 \right)}{\langle n -
n_1 \rangle ^s\ \langle |\tau - \tau_1 | -  (n - n_1)^{2} \rangle^{b}}
  \ d \tau_1 
\end{split}
\end{equation}
%
%
where for clarity of notation we have introduced 
%
%
%
\begin{equation*}
\begin{split}
\langle k \rangle \doteq 1 + |k|
\end{split}
\end{equation*}
%
%
and
%
\begin{equation*}
	\begin{split}
		c_h(n, \tau) =
			\langle n \rangle ^s \langle |\tau| - n^{2} \rangle ^{b} | \wh{h}\left( n, \tau \right) |.
	\end{split}
\end{equation*}
%
%
From our work above, it follows that 
%
%
\begin{equation}
	\label{convo-est-starting-pnt}
	\begin{split}
		 & \langle n \rangle^s \langle \tau - n^{2} \rangle^{-a} | \wh{uv}\left( 
		n, \tau \right) |
		\\
		& \le \langle |\tau| - n^{2} \rangle^{-a}
		\sum_{n_1 \in \zz} \int \frac{\langle n \rangle^{s}}{\langle n_1 \rangle^s
    \langle n - n_1 \rangle^s} 
		\times \frac{c_f(n_1, \tau_1)}{\langle |\tau_1| - n_1^{2} \rangle ^{b}}
		\\
		& \times
		\frac{c_g(n - n_1, \tau - \tau_1 )}{\langle |\tau - \tau_1| - (n - n_1)^{2}
    \rangle^{b}}\ d \tau_1.
	\end{split}
\end{equation}
%
%
Hence, 
%
%
\begin{equation}
  \label{pre-fubini-int-form}
	\begin{split}
    |\text{lhs of} \ \eqref{duality-est}|
	& \lesssim \sum_{n \in \zz} \int_{\rr} \phi(n, \tau) \langle n \rangle^s \langle \tau - n^{2} \rangle^{-a}
  \sum_{n_1 \in \zz}
  \int_{\rr} c_f(n_1, \tau_1)
		c_g(n - n_1, \tau - \tau_1 )
		\\
    & \times \frac{\langle n \rangle ^{s}}{\langle n_{1} \rangle ^{s} \langle
    n-n_{1} \rangle ^{s}} \times \frac{1}{\langle |\tau| - n^{2} \rangle
    ^{b}\langle |\tau_{1}|-n_{1}^{2} \rangle ^{-b}\langle | \tau|-n_{2}^{2}
    \rangle ^{b}} d \tau_1 d \tau.
	\end{split}
\end{equation}
%
%
which by Cauchy-Schwartz is bounded by
%
%
\begin{equation}
	\label{10g}
	\begin{split}
    & \sum_{n \in \zz} \int_{\rr} \phi(n, \tau) \langle | \tau | - n^{2} \rangle
    ^{-a} \langle n \rangle ^{s}
    \\
    & \times \left( \sum_{n_{1} \in \zz} \int_{\rr}
    \frac{1}{\langle n_{1} \rangle ^{2s} \langle n-n_{1} \rangle ^{2s} \langle |
    \tau_{1} | - n_{1}^{2}\rangle ^{2b} \langle | \tau - \tau_{1} | -
    (n - n_{1})^{2} \rangle ^{2b}} d \tau_{1} \right)^{1/2}
    \\
    & \times \left( \sum_{n_{1} \in \zz} \int_{\rr} c_{u}^{2}(n, \tau_{1})
    c_{v}^{2}(n - n_{1}, \tau - \tau_{1}) d \tau_{1} \right)^{1/2} d \tau
  \end{split}
\end{equation}
%
%
Applying Cauchy-Schwartz again, \eqref{10g} is bounded by
%
%
\begin{equation*}
  \begin{split}
  & \|\left( \sum_{n_{1} \in \zz }\int_{\rr } c_{u}^{2}(n_1, \tau_1)
  c_{v}^{2} (n - n_1, \tau - \tau_{1} ) d \tau_1  \right)^{1/2} \|_{L^{2}(\zz \times
		\rr)}
		\\
    & \times  \|\phi(n, \tau) \langle | \tau | - n^{2} \rangle ^{-a} \langle n
    \rangle ^{s}
		\\
    & \times \left( \sum_{n_{1} \in \zz} \int_{\rr} \frac{1}{ \langle n_{1}
    \rangle ^{2s} \langle n-n_{1} \rangle ^{2s} \langle | \tau_{1}|-n_{1}^{2}
    \rangle^{2b} \langle  |\tau -
    \tau_{1} | -(n - n_{1}^{2}
    \rangle^{2b} } d \tau_1 \right)^{1/2} \|_{L^2(\zz \times \rr)}
		\\
    & = \|u\|_{X_{s,b}} \|v\|_{X_{s,b}} \label{holder-term}
     \|\phi(n, \tau)     \\
    & \times \left( \langle | \tau | - n^{2} \rangle ^{-2a} \langle n
    \rangle ^{2s}
\sum_{n_{1} \in \zz} \int_{\rr} \frac{1}{ \langle n_{1} \rangle ^{2s} \langle
n-n_{1} \rangle ^{2s}  \langle | \tau_{1}|-n_{1}^{2} \rangle^{2b} \langle  |\tau -
    \tau_{1} | -(n - n_{1}^{2}
    \rangle^{2b} } d \tau_1 \right)^{1/2} \|_{L^2(\zz \times \rr)}.
  \end{split}
\end{equation*}
%
Applying H{\"o}lder, we bound this by 
%
%
\begin{equation*}
	\begin{split}
    & \|u\|_{X_{s,b}} \|v\|_{X_{s,b}} \| \phi \|_{L^{2}_{n, \tau}}
    \\
    & \times \|\left( \langle | \tau | - n^{2} \rangle ^{-2a} \langle n
    \rangle ^{2s}
\sum_{n_{1} \in \zz} \int_{\rr} \frac{1}{ \langle n_{1} \rangle ^{2s} \langle
n-n_{1} \rangle ^{2s} \langle | \tau_{1}|-n_{1}^{2} \rangle^{2b} \langle  |\tau -
    \tau_{1} | -(n - n_{1}^{2}
    \rangle ^{2b} } d \tau_1 \right)^{1/2} \|_{L^\infty_{n, \tau}}
	\end{split}
\end{equation*}
%
%
Hence, to complete the proof, it will be enough
to show that 
%
%
%
%
\begin{equation}
  \label{key-sup-estimate}
	\begin{split}
		 \| \langle | \tau | - n^{2} \rangle ^{-2a} \langle n
    \rangle ^{2s}
\sum_{n_{1} \in \zz} \int_{\rr} \frac{1}{  \langle n_{1} \rangle ^{2s} \langle
n-n_{1} \rangle ^{2s} \langle | \tau_{1}|-n_{1}^{2} \rangle^{2b}  \langle  |\tau -
    \tau_{1} | -(n - n_{1}^{2}
    \rangle ^{2b} } d \tau_1 \|_{L^\infty_{n, \tau}} < \infty.
	\end{split}
\end{equation}
%
%
By the triangle inequality and the fact that 
%
%
\begin{equation*}
\begin{split}
& | \tau | =
\begin{cases}
  - \tau, \quad & \tau < 0, 
\\
\tau, \quad & \tau > 0
\end{cases}
\end{split}
\end{equation*}
%
%
\eqref{key-sup-estimate} will be proved if we can bound the
$L^{\infty}_{\tau, n}$ norm of the quantity
%
%
\begin{equation}
  \label{sup-est-gen}
\begin{split}
		  \langle \sigma \rangle ^{-2a} \langle n
    \rangle ^{2s}
\sum_{n_{1} \in \zz} \int_{\rr} \frac{1}{ \langle n_{1} \rangle ^{2s} \langle n-n_{1} \rangle ^{2s} 
\langle \sigma_{1} \rangle^{2b} \langle  \sigma_{2} \rangle^{2b} }
d \tau_1 
	\end{split}
\end{equation}
%
for the following cases.
\begin{enumerate}[(i)]
    \item $ \sigma=\tau+n^2,\quad \sigma_1=\tau_1+n_1^2,\quad \sigma_2=\tau -
      \tau_1+(n - n_1)^2$,
\label{it-1}
    \item $ \sigma=\tau-n^2,\quad \sigma_1=\tau_1-n_1^2,\quad \sigma_2=\tau - \tau_1+(n - n_1)^2$,
\label{it-2}
    \item  $\sigma=\tau+n^2,\quad \sigma_1=\tau_1-n_1^2,\quad \sigma_2=\tau - \tau_1+(n - n_1)^2$,
      \label{it-3}
    \item $\sigma=\tau-n^2,\quad \sigma_1=\tau_1+n_1^2,\quad \sigma_2=\tau - \tau_1-(n - n_1)^2$,
\label{it-4}
    \item $\sigma=\tau+n^2,\quad \sigma_1=\tau_1+n_1^2,\quad \sigma_2=\tau - \tau_1-(n - n_1)^2$,
\label{it-5}
    \item $\sigma=\tau-n^2,\quad \sigma_1=\tau_1-n_1^2,\quad \sigma_2=\tau - \tau_1-(n - n_1)^2$.
\label{it-6}
\end{enumerate}
%
%
\begin{framed}
\begin{remark}
Note that the cases $\sigma=\tau+n^2,\quad \sigma_1=\tau_1-n_1^2,\quad
\sigma_2=\tau - \tau_1-(n - n_1)^2$ and $\sigma=\tau-n^2,\quad
\sigma_1=\tau_1+n_1^2,\quad \sigma_2=\tau - \tau_1+(n - n_1)^2$ cannot occur, since
$\tau_1< 0, \tau-\tau_1< 0$ implies $\tau<0$ and $\tau_1\geq 0, \tau-\tau_1\geq
0$ implies $\tau\geq 0$.
\end{remark}
\end{framed}
%
Observe that the transformation $(n, \tau, n_{1}, \tau_{1}) \mapsto -(n, \tau,
n_{1}, \tau_{1})$ reduces \eqref{it-3} to \eqref{it-4}, \eqref{it-2} to
\eqref{it-5}, and \eqref{it-1} to \eqref{it-6}. Furthermore, the change of
variables $\tau_{2} = \tau - \tau_{1}, n_{2} = n - n_{1}$, and the
transformation $(n, \tau, n_{2}, \tau_{2}) \mapsto - (n, \tau, n_{2},
\tau_{2})$ reduces \eqref{it-5} to \eqref{it-4}. Since $L^{2}$ is invariant
under change of variables and reflections, we may without loss of generality
restrict our attention to cases \eqref{it-4} and \eqref{it-6}.
 \subsubsection{Case \eqref{it-6}} 
\label{sssec:case-it-6}
Let 
%
%
\begin{align*}
A_1&=\{(n, n_1, \tau, \tau_1)\in A: n=0\},\\
A_2&=\{(n, n_1, \tau, \tau_1)\in A: n_1 = n \},\\
A_3&=\{(n, n_1, \tau, \tau_1)\in A: n_1=0 \},\\
A_4&=\{(n, n_1, \tau, \tau_1)\in A: n \neq 0, n_1 \neq 0 \text{ and } n_1 \neq n \}.
\end{align*} 
%
%
Then 
%
%
\begin{equation}
  \label{n=0-real}
\begin{split}
  |  \eqref{sup-est-gen} \chi_{A_{1}}| = | \langle \tau \rangle ^{-2a} \sum_{n_{1}} \langle
  n_{1}\rangle ^{-4s} \int_{\rr} \frac{1}{\langle \tau_{1} - n_{1}^{2} \rangle ^{2b}\langle
  \tau - \tau_{1} - n_{1}^{2}\rangle ^{2b}}d \tau_{1} |
\end{split}
\end{equation}
%
%
Following Ginibre, Tsutsumi, Velo --- and Kenig \cite{Kenig:1996aa}, and others,
we now need the following Calculus lemma.
%
%
%%%%%%%%%%%%%%%%%%%%%%%%%%%%%%%%%%%%%%%%%%%%%%%%%%%%%
%
%
%				 Calculus Lemma
%
%
%%%%%%%%%%%%%%%%%%%%%%%%%%%%%%%%%%%%%%%%%%%%%%%%%%%%%
%
%
\begin{lemma}
	\label{lem:calc}
 %
 %
 For $r > 1/2$
\begin{equation*}
  \int_{\rr} \frac{1} {\langle  \theta \rangle^{r} \langle  a - \theta
  \rangle^{r}}\leq\frac{\log 2} {\langle a \rangle^{r}}.
\end{equation*}
 %
 %
 \end{lemma}
%
%%%%%%%%%%%%%%%%%%%%%%%%%%%%%%%%%%%%%%%%%%%%%%%%%%%%%
%
%
%             Secod order Modified Boussinesq  equation
%
%
%%%%%%%%%%%%%%%%%%%%%%%%%%%%%%%%%%%%%%%%%%%%%%%%%%%%%

Applying the calculus lemma, it follows from \eqref{n=0-real} that
%
\begin{equation*}
\begin{split}
  |\eqref{sup-est-gen} \chi_{A_{1}} |  & \lesssim \| \langle \tau \rangle ^{-2a} \sum_{n_{1} \in \zz} \frac{\langle n_{1} \rangle
  ^{-4s}}{ \langle \tau - 2n_{1}^{2}  \rangle ^{2b}} \|_{L^{\infty}_{\tau}}
  \\
  & \simeq \sum_{n_{1}} \langle n_{1} \rangle ^{-4s - 4b}
  \\
  & < \infty, \quad s > \frac{1-4b}{4}.
\end{split}
\end{equation*}
%
%
%
%
%\begin{framed}
%\begin{remark}
%Later, we will see that we will need to assume $b>1/4$, which gives $s>0$. The
%well-posedness result for the Boussinesq is $s>-1/4$. To see why, we note that
%for the Boussinesq, the analogue of the left hand side of \eqref{sup-est-gen} is

%\begin{equation}
  %\label{sup-est-gen-Boussinesq}
%\begin{split}
		 %\| \langle \sigma \rangle ^{-2a} \langle n
     %\rangle ^{2s} \frac{ n^{4}}{n^{2} + n^{4}}
%\sum_{n_{1} \in \zz} \int_{\rr} \frac{1}{ \langle n_{1} \rangle ^{2s} \langle n-n_{1} \rangle ^{2s} 
%\langle \sigma_{1} \rangle \langle  \sigma_{2} \rangle
%d \tau_1} \|_{L^\infty_{n, \tau}} < \infty
	%\end{split}
%\end{equation}

%\label{rem:difference-b4-boussinesq}
%%
%%
%which for $n=0$ is trivially bounded, since $$\frac{n^{4}}{n^{2} +
%n^{4}}=0$$ at $n=0$
%($n=0$ is a removable singularity. See \autoref{rem:analytic-extension}). This is \emph{NOT} the case for the
%$B_{4}$ ivp, since the Boussinesq
%quantity $$\frac{n^{4}}{n^{2} + n^{4}}$$ is replaced by
%the $B_{4}$ quantity $$\frac{n^{4}}{n^{4}}$$ which is equal to $1$ at
%$n=0$. In short, we miss the extra smoothing coming from the Laplacian that is
%available for the Boussinesq. This same pathological behavior occurs for the
%NLS---lack of smoothing near the $0th$ Fourier mode is what makes us unable to get a well-posedness result better than
%$s > 0$ in the periodic case--it CHEATS us out of the extra $1/4$ smoothing we
%should be getting, and which is found in the case on the line.
%%
%%
%\end{remark}
%\end{framed}
%
%
Similarly, substitution and \autoref{lem:calc} give
%
%
\begin{equation}
\begin{split}
  |  \eqref{sup-est-gen} \chi_{A_{2}}|
  & = | \langle \tau -n^{2} \rangle ^{-2a}\int_{\rr} \frac{1}{\langle \tau_{1} -
  n^{2} \rangle ^{2b}\langle
  \tau - \tau_{1}\rangle ^{2b}}d \tau_{1} |
  \\
  & \lesssim |  \langle \tau - n^{2} \rangle ^{-2a-2b} |
  \\
  & < \infty, \quad b \ge -a,
\end{split}
\end{equation}
%
%
%
%
\begin{equation}
\begin{split}
  |  \eqref{sup-est-gen} \chi_{A_{3}}|
  & = \langle \tau - n^{2} \rangle ^{-2a}
  \int_{\rr} \frac{1}{ \langle \tau_{1} \rangle^{2b}  \langle \tau -
  \tau_{1} - n^{2} \rangle^{2b}}
d \tau_1 
\\
  & \lesssim |  \langle \tau - n^{2} \rangle ^{-2a-2b} |
  \\
  & < \infty, \quad b \ge -a.
	\end{split}
\end{equation}
%
%
and
%
%
\begin{equation*}
\begin{split}
  | \eqref{sup-est-gen} \chi_{A_{4}} |
  & = 
  \langle \tau - n^{2}  \rangle ^{-2a} \langle n
    \rangle ^{2s}
    \sum_{n_{1} \in \zz} \int_{\rr} \frac{\chi_{A_{4}}}{ \langle n_{1} \rangle ^{2s} \langle n-n_{1} \rangle ^{2s} 
\langle \tau_{1} - n_{1}^{2}  \rangle \langle  \tau - \tau_{1} - (n -
n_{1})^{2}  \rangle}
d \tau_1 
\\
& = \langle \tau - n^{2}  \rangle ^{-2a} \langle n
    \rangle ^{2s}
    \sum_{j=1}^{2} \sum_{n_{1} \in \zz} \int_{\rr} \frac{\chi_{A_{4,j}}}{ \langle n_{1} \rangle ^{2s} \langle n-n_{1} \rangle ^{2s} 
\langle \tau_{1} - n_{1}^{2}  \rangle \langle  \tau - \tau_{1} - (n -
n_{1})^{2}  \rangle}
d \tau_1 
\end{split}
\end{equation*}
%
%
where we have partioned $ A_{4}$ into two parts
\begin{align*}
A_{4,1}&=\{(n, n_1, \tau, \tau_1)\in A_3: |\tau_1-n_1^2|\leq|\tau-n^2|\},\\
A_{4,2}&=\{(n, n_1, \tau, \tau_1)\in A_3: |\tau-n^2|\leq|\tau_1-n_1^2| \}.
\end{align*} 
Furthermore, by the symmetry of the convolution, we may assume without loss of
generality that
$$|(\tau-\tau_1)-(n-n_1)^2|\leq|\tau_1-n_1^2|\}.$$
Noting that 
%
%
\begin{equation*}
\begin{split}
  \tau - n^{2} - \left[ (\tau - n_{1}^{2}) + (\tau - \tau_{1}) - (n -
  n_{1})^{2} \right] = 2n_{1}(n - n_{1}).
\end{split}
\end{equation*}
%
%
we have in region $A_{4}$ 
%
%
\begin{enumerate}[(i)]
  \item{$\langle \tau - n^{2} \rangle  \gtrsim \langle n_{1}(n - n_{1}) \rangle$, or}
    \\
  \item{  $\langle \tau_{1} - n_{1}^{2} \rangle  \gtrsim \langle n_{1}(n -
    n_{1}) \rangle$, or}
    \\
  \item{ $\langle \tau - \tau_{1} - (n - n_{1})^{2} \rangle  \gtrsim \langle n_{1}(n -
    n_{1}) \rangle$}.
\end{enumerate}
In region $A_{4,1}$, these reduce to the single case $ \langle \tau - n^{2} \rangle
\gtrsim \langle n_{1}(n - n_{1}) \rangle$, and in region $A_{4,2}$ to the single
case $ \langle \tau_{1} - n_{1}^{2} \rangle
\gtrsim \langle n_{1}(n - n_{1}) \rangle$. Estimating first in region
$A_{4,1}$, we apply \autoref{lem:calc} and obtain
%
%
%
%
\begin{equation}
  \label{region-a41}
\begin{split}
& \langle \tau - n^{2}  \rangle ^{-2a} \langle n
    \rangle ^{2s}
    \sum_{n_{1} \in \zz} \int_{\rr} \frac{\chi_{A_{4,1}}}{ \langle n_{1} \rangle ^{2s} \langle n-n_{1} \rangle ^{2s} 
\langle \tau_{1} - n_{1}^{2}  \rangle \langle  \tau - \tau_{1} - (n -
n_{1})^{2}  \rangle}
d \tau_1 
\\
& \lesssim \langle \tau - n^{2} \rangle ^{-2a} \langle n \rangle ^{2s}
\sum_{n_{1} \in
\zz}  \frac{\chi_{A_{4,1}}}{\langle n_{1} \rangle ^{2s} \langle n - n_{1} \rangle
^{2s} \langle \tau - n^{2} - 2n_{1}^{2} + 2nn_{1}  \rangle ^{2b}}
\\
& \lesssim 
\sum_{n_{1} \in
\zz}  \frac{\langle n_1 \rangle ^{-2s} \langle n - n_{1} \rangle ^{-2s}}{\langle
n \rangle ^{-2s} \langle n_{1}(n - n_{1}) \rangle
^{2a}} \times \frac{\chi_{A_{4,1}}}{\langle \tau - n^{2} - 2n_{1}^{2} + 2nn_{1}
\rangle ^{2b}}.
\end{split}
\end{equation}
%
%
But
%
%
\begin{equation}
  \label{prelim-int-est}
\begin{split}
& \frac{\langle n_1 \rangle ^{-2s} \langle n - n_{1} \rangle ^{-2s}}{\langle
n \rangle ^{-2s} \langle n_{1}(n - n_{1}) \rangle
^{2a}}
\\
& \lesssim  \frac{| n_1 | ^{-2s} | n - n_{1} | ^{-2s}}{|
n | ^{-2s} | n_{1}(n - n_{1}) |
^{2a}}, \quad n \neq 0, n_1 \neq 0, n \neq n_1 
\\
& = | n - n_{1} |^{-2s-2a}|n_{1}|^{-2s-2a}| n |^{2s}
\\
& \lesssim \begin{cases}
  | n |^{\cancel{-2s}-2a}| n_{1} |^{-2s-2a} | n_{1} |^{-2s-2a}\cancel{| n
  |^{2s}}, \quad & s < 0, \quad a \ge -s
\\
| n - n_{1} |^{\cancel{-2s}-2a}|n_{1}|^{\cancel{-2s}-2a}{\cancel{| n
-n_{1}|^{2s}}\cancel{| n_{1}
|^{2s}}}, \quad & s \ge 0. 
\end{cases}
\end{split}
\end{equation}
%
%
where the last step follows from the following.
%
%
%
%%%%%%%%%%%%%%%%%%%%%%%%%%%%%%%%%%%%%%%%%%%%%%%%%%%%%
%
%
%                Integer Bound
%
%
%%%%%%%%%%%%%%%%%%%%%%%%%%%%%%%%%%%%%%%%%%%%%%%%%%%%%
%
%
\begin{lemma}
  Let $n, n_1 \in \zz$ such that $n_{1} \neq 0$ and $n_{1} \neq n$.
  Then
  %
  %
  \begin{equation*}
  \begin{split}
    | n | \le | n - n_{1} | | n_{1} |.
  \end{split}
  \end{equation*}
  %
  %
\label{lem:integer-bound}
\end{lemma}
%
Since $a \ge 0$, it follows from \eqref{prelim-int-est} that 
%
\begin{equation*}
\frac{\langle n_1 \rangle ^{-2s} \langle n - n_{1} \rangle ^{-2s}}{\langle
n \rangle ^{-2s} \langle n_{1}(n - n_{1}) \rangle
^{2a}} \lesssim 1, \quad s \ge 0 \text{ or } s < 0, a \ge s
\end{equation*}
%
%
which we use to bound the right hand side of \eqref{region-a41} by
%
%
\begin{equation*}
\begin{split}
\sum_{n_{1} \in
\zz} 
\frac{1}{\langle \tau - n^{2} - 2n_{1}^{2} + 2nn_{1}  \rangle ^{2b}}
\end{split}
\end{equation*}
%
%
%
which is finite for $b > 1/4$, due to the following lemma, which can be found in. 
\begin{lemma}
  \label{lem:sum-estimate}
If $\gamma>1/2$, then
\begin{equation}\label{CI2}
\sup_{(n,\tau)\in \zz \times \rr}\sum_{n_1\in \zz}\frac{1}{(1+|\tau\pm n_1(n-n_1)|)^{\gamma}}<\infty. 
\end{equation}
\end{lemma}
%
Recalling that 
$$ \langle \tau_{1} - n_{1}^{2} \rangle
\gtrsim \langle n_{1}(n - n_{1}) \rangle$$
%
in region $A_{4,2}$, we have
%
%
\begin{equation*}
\begin{split}
& \langle \tau - n^{2}  \rangle ^{-2a} \langle n
    \rangle ^{2s}
    \sum_{n_{1} \in \zz} \int_{\rr} \frac{\chi_{A_{4,1}}}{ \langle n_{1} \rangle ^{2s} \langle n-n_{1} \rangle ^{2s} 
\langle \tau_{1} - n_{1}^{2}  \rangle \langle  \tau - \tau_{1} - (n -
n_{1})^{2}  \rangle}
d \tau_1 
\\
& \lesssim \langle \tau - n^{2}  \rangle ^{-2a}     \sum_{n_{1} \in \zz} \int_{\rr} \frac{\chi_{A_{4,1}} \langle n
    \rangle ^{2s}
}{ \langle n_{1} \rangle ^{2s} \langle n-n_{1} \rangle ^{2s} 
\langle n_{1}(n - n_{1}) \rangle ^{2b} \langle  \tau - \tau_{1} - (n -
n_{1})^{2}  \rangle}
d \tau_1 
\end{split}
\end{equation*}
%
%
which by computation similar to \eqref{prelim-int-est}
is bounded by
%
%
\begin{equation}
  \label{prelim-int-est-2}
\begin{split}
\langle \tau - n^{2}  \rangle ^{-2a} \sum_{n_{1} \in \zz} \int_{\rr} \frac{\chi_{A_{4,1}} }{ \langle  \tau - \tau_{1} - (n -
n_{1})^{2}  \rangle^{2b}}
d \tau_1 
\end{split}
\end{equation}
%
%
for $s \ge 0$ or $s \le 0, b \ge -2s$. If $b > 1/2$, then integrating in
$\tau_{1}$ and discarding the  $\langle \tau - n^{2}  \rangle ^{-2a}$ term, we
then bound \eqref{prelim-int-est-2} by
%
%
\begin{equation*}
\begin{split}
  C \sum_{n_{1} \in \zz} \frac{\chi_{A_{4,1}}}{\langle \tau - (n -
  n_{1})^{2} \rangle ^{2b -1}}
  & \simeq
  \sum_{n_{1} \in \zz} \frac{\chi_{A_{4,1}}}{\langle \tau - n^{2} +
  2nn_{1} - n_{1}^{2}
  \rangle ^{2b -1}}
  \\
  & < \infty, \quad b > 1/2
\end{split}
\end{equation*}
%
%
where the last step follows from  \autoref{lem:sum-estimate}.
\subsubsection{Case \eqref{it-4}} 
\label{sssec:case-it-4}
Let 
%
%
\begin{align*}
B_1&=\{(n, n_1, \tau, \tau_1)\in B: n=0\},\\
B_2&=\{(n, n_1, \tau, \tau_1)\in B: n_1 = 0 \},\\
B_3&=\{(n, n_1, \tau, \tau_1)\in B: n \neq 0, n_1 \neq 0 \}.
\end{align*} 
%
%
Then 
%
%
\begin{equation}
  \label{pathological-equality}
\begin{split}
  |  \eqref{sup-est-gen} \chi_{B_{1}}| = | \langle \tau \rangle ^{-2a} \sum_{n_{1}} \langle
  n_{1}\rangle ^{-4s} \int_{\rr} \frac{1}{\langle \tau_{1} + n_{1}^{2} \rangle ^{2b}\langle
  \tau - \tau_{1} - n_{1}^{2}\rangle ^{2b}}d \tau_{1} |
\end{split}
\end{equation}
Applying \autoref{lem:calc}, it follows that
%
\begin{equation*}
\begin{split}
  |\eqref{sup-est-gen} \chi_{B_{1}} |  & \lesssim \| \langle \tau \rangle
  ^{-2a-2b} \sum_{n_{1} \in \zz} \langle n_{1} \rangle
  ^{-4s}  \|_{L^{\infty}_{\tau}}
  \\
  & \simeq \sum_{n_{1} \in \zz} \langle n_{1} \rangle ^{-4s}
  \\
  & < \infty, \quad s > \frac{1}{4}.
\end{split}
\end{equation*}
%
%
Note that we cannot improve our lower bound for $s$. To see this, we set $\tau = 0$
in the right hand side of \eqref{pathological-equality} and obtain
%
%
%
\begin{equation*}
\begin{split}
   \sum_{n_{1}} \langle
  & n_{1}\rangle ^{-4s} \int_{\rr} \frac{1}{\langle \tau_{1} + n_{1}^{2} \rangle ^{2b}\langle
   \tau_{1} + n_{1}^{2}\rangle ^{2b}}d \tau_{1} 
   \\
   & = \sum_{n_{1}} \langle
  n_{1}\rangle ^{-4s} \int_{\rr} \frac{1}{\langle
   \tau' \rangle ^{4b}}d \tau'
   \\
   & \simeq \sum_{n_{1}} \langle n_{1} \rangle ^{-4s}, \quad b > 1/4
   \\
   & < \infty, \quad s > 1/4.
\end{split}
\end{equation*}
%
%
%
\begin{framed}
\begin{remark}
This is a pathological case which does not exist for the Boussinesq. This is
because  $$\frac{n^{2}}{n^{2} + n^{4}} |_{n=0} = 0$$ while
$$\frac{n^{2}}{n^{2}} |_{n=0} = 1.$$ Since the term
\eqref{sup-est-gen}$\chi_{B_{1}}$ does not vanish for the $B_4$ equation, it
lowers our well-posedness result a full half-derivative, i.e.
from $s > -1/4$ (Boussinesq periodic) to $s > 1/4$ ($B_{4}$ periodic). 
%
\end{remark}
\end{framed}
Similarly, substitution and \autoref{lem:calc} give
%
%
\begin{equation}
\begin{split}
  |  \eqref{sup-est-gen} \chi_{B_{2}}|
  & = | \langle \tau +n^{2} \rangle ^{-2a}\int_{\rr} \frac{1}{\langle \tau_{1}\rangle ^{2b}\langle
  \tau - \tau_{1} - n^{2} \rangle ^{2b}}d \tau_{1} |
  \\
  & \lesssim |  \langle \tau - n^{2} \rangle ^{-2a-2b} |
  \\
  & < \infty, \quad b \ge -a.
\end{split}
\end{equation}
and
%
%
\begin{equation*}
\begin{split}
  | \eqref{sup-est-gen} \chi_{B_{3}} |
  & = 
  \langle \tau - n^{2}  \rangle ^{-2a} \langle n
    \rangle ^{2s}
    \sum_{n_{1} \in \zz} \int_{\rr} \frac{\chi_{B_{3}}}{ \langle n_{1} \rangle ^{2s} \langle n-n_{1} \rangle ^{2s} 
\langle \tau_{1} - n_{1}^{2}  \rangle \langle  \tau - \tau_{1} - (n -
n_{1})^{2}  \rangle}
d \tau_1 
\\
& = \langle \tau - n^{2}  \rangle ^{-2a} \langle n
    \rangle ^{2s}
    \sum_{j=1}^{2} \sum_{n_{1} \in \zz} \int_{\rr} \frac{\chi_{B_{3,j}}}{ \langle n_{1} \rangle ^{2s} \langle n-n_{1} \rangle ^{2s} 
\langle \tau_{1} - n_{1}^{2}  \rangle \langle  \tau - \tau_{1} - (n -
n_{1})^{2}  \rangle}
d \tau_1 
\end{split}
\end{equation*}
%
%
where we have partioned $ B_{3}$ into two parts
\begin{align*}
B_{3,1}&=\{(n, n_1, \tau, \tau_1)\in B_3: |\tau_1-n_1^2|\leq|\tau-n^2|\},\\
B_{3,2}&=\{(n, n_1, \tau, \tau_1)\in B_3: |\tau-n^2|\leq|\tau_1-n_1^2| \}.
\end{align*} 
Furthermore, by the symmetry of the convolution, we may assume without loss of
generality that
$$|(\tau-\tau_1)-(n-n_1)^2|\leq|\tau_1-n_1^2|\}.$$
Noting that 
%
%
\begin{equation*}
\begin{split}
  \tau - n^{2} - \left[ (\tau + n_{1}^{2}) + (\tau - \tau_{1}) - (n -
  n_{1})^{2} \right] = 2n_{1}n
\end{split}
\end{equation*}
%
%
we have in region $B_{3}$ 
%
%
\begin{enumerate}[(i)]
  \item{$\langle \tau - n^{2} \rangle  \gtrsim \langle n_{1}n \rangle$, or}
    \\
  \item{  $\langle \tau_{1} - n_{1}^{2} \rangle  \gtrsim \langle n_{1}n \rangle$, or}
    \\
  \item{ $\langle \tau - \tau_{1} - (n - n_{1})^{2} \rangle  \gtrsim \langle n_{1}n
    \rangle$}.
\end{enumerate}
In region $B_{3,1}$, these reduce to the single case $ \langle \tau - n^{2} \rangle
\gtrsim \langle n_{1}n \rangle$, and in region $B_{3,2}$ to the single
case $ \langle \tau_{1} - n_{1}^{2} \rangle
\gtrsim \langle n_{1}n \rangle$. Estimating first in region
$B_{3,1}$, we apply \autoref{lem:calc} and obtain
%
%
%
%
\begin{equation}
  \label{region-b31}
\begin{split}
& \langle \tau - n^{2}  \rangle ^{-2a} \langle n
    \rangle ^{2s}
    \sum_{n_{1} \in \zz} \int_{\rr} \frac{\chi_{B_{3,1}}}{ \langle n_{1} \rangle ^{2s} \langle n-n_{1} \rangle ^{2s} 
\langle \tau_{1} + n_{1}^{2}  \rangle \langle  \tau - \tau_{1} - (n -
n_{1})^{2}  \rangle}
d \tau_1 
\\
& \lesssim \langle \tau - n^{2} \rangle ^{-2a} \langle n \rangle ^{2s}
\sum_{n_{1} \in
\zz}  \frac{\chi_{B_{3,1}}}{\langle n_{1} \rangle ^{2s} \langle n - n_{1} \rangle
^{2s} \langle \tau - n^{2} + 2nn_{1}  \rangle ^{2b}}
\\
& \lesssim 
\sum_{n_{1} \in
\zz}  \frac{\langle n_1 \rangle ^{-2s} \langle n - n_{1} \rangle ^{-2s}}{\langle
n \rangle ^{-2s} \langle n_{1}n \rangle
^{2a}} \times \frac{\chi_{B_{3,1}}}{\langle \tau - n^{2} + 2nn_{1}
\rangle ^{2b}}.
\end{split}
\end{equation}
%
%
But for $n = n_{1}$, 
%
%
\begin{equation}
  \label{prelim-int-est-b-1}
\begin{split}
\frac{\langle n_1 \rangle ^{-2s} \langle n - n_{1} \rangle ^{-2s}}{\langle
n \rangle ^{-2s} \langle n_{1}n \rangle
^{2a}} \le 1
\end{split}
\end{equation}
while for $n \neq 0, n_{1} \neq 0, n \neq n_{1}$, 
\begin{equation}
  \begin{split}
  \label{prelim-int-est-b-2}
& \frac{\langle n_1 \rangle ^{-2s} \langle n - n_{1} \rangle ^{-2s}}{\langle
n \rangle ^{-2s} \langle n_{1}n \rangle
^{2a}} 
\\
& \le \begin{cases}
  | n_{1} |^{-4s-2a} , \quad & s < 0
\\
1, \quad  & s \ge 0. 
\end{cases}
\end{split}
\end{equation}
%
%
where the last step follows from \autoref{lem:integer-bound}. Since $a \ge 0$,
it follows from \eqref{prelim-int-est-b-1}-\eqref{prelim-int-est-b-2} that 
%
\begin{equation*}
\frac{\langle n_1 \rangle ^{-2s} \langle n - n_{1} \rangle ^{-2s}}{\langle
n \rangle ^{-2s} \langle n_{1}(n - n_{1}) \rangle
^{2a}} \le 1, \quad s \ge 0 \text{ or } s<0, s \ge -a/2 
\end{equation*}
%
%
which we use to bound the right hand side of \eqref{region-b31} by
%
%
\begin{equation*}
\begin{split}
\sum_{n_{1} \in
\zz} 
\frac{1}{\langle \tau - n^{2} + 2nn_{1}  \rangle ^{2b}}
\end{split}
\end{equation*}
%
%
%
which is finite for $b > 1/4$, due to \autoref{lem:sum-estimate}.
Recalling that 
$$ \langle \tau_{1} - n_{1}^{2} \rangle
\gtrsim \langle n_{1}n \rangle$$
%
in region $B_{3,2}$, we have
%
%
\begin{equation*}
\begin{split}
& \langle \tau - n^{2}  \rangle ^{-2a} \langle n
    \rangle ^{2s}
    \sum_{n_{1} \in \zz} \int_{\rr} \frac{\chi_{B_{3,1}}}{ \langle n_{1} \rangle ^{2s} \langle n-n_{1} \rangle ^{2s} 
\langle \tau_{1} + n_{1}^{2}  \rangle \langle  \tau - \tau_{1} - (n -
n_{1})^{2}  \rangle}
d \tau_1 
\\
& \lesssim \langle \tau - n^{2}  \rangle ^{-2a}     \sum_{n_{1} \in \zz} \int_{\rr} \frac{\chi_{B_{3,1}} \langle n
    \rangle ^{2s}
}{ \langle n_{1} \rangle ^{2s} \langle n-n_{1} \rangle ^{2s} 
\langle n_{1}n \rangle ^{2b} \langle  \tau - \tau_{1} - (n -
n_{1})^{2}  \rangle}
d \tau_1 
\end{split}
\end{equation*}
%
%
which by computation similar to \eqref{prelim-int-est-b-2}
is bounded by
%
%
\begin{equation}
  \label{prelim-int-est-2-b}
\begin{split}
\langle \tau - n^{2}  \rangle ^{-2a} \sum_{n_{1} \in \zz} \int_{\rr} \frac{\chi_{B_{3,1}} }{ \langle  \tau - \tau_{1} - (n -
n_{1})^{2}  \rangle^{2b}}
d \tau_1 
\end{split}
\end{equation}
%
%
for $s \ge 0$ or $s \le 0, b \ge -2s$. If $b > 1/2$, then integrating in
$\tau_{1}$ and discarding the  $\langle \tau - n^{2}  \rangle ^{-2a}$ term, we
then bound \eqref{prelim-int-est-2-b} by
%
%
\begin{equation*}
\begin{split}
  C \sum_{n_{1} \in \zz} \frac{\chi_{B_{3,1}}}{\langle \tau - (n -
  n_{1})^{2} \rangle ^{2b -1}}
  & \simeq
  \sum_{n_{1} \in \zz} \frac{\chi_{B_{3,1}}}{\langle \tau - n^{2} +
  2nn_{1} - n_{1}^{2}
  \rangle ^{2b -1}}
  \\
  & < \infty, \quad b > 1/2
\end{split}
\end{equation*}
%
%
where the last step follows from  \autoref{lem:sum-estimate}. This completes the
proof of \autoref{prop:bilin-est}. \qquad \qedsymbol
\subsection{The Non-Periodic Case} 
\label{ssec:non-periodic-case}
We now introduce the following spaces. 
%
%
\begin{definition}
  Let $S(\rr^{2})$ denote the space of Schwartz functions on
  $\rr^{2}$.  For $s, b \in \rr$, $\mathcal{X}_{s,b}$
  denotes the completion of $S(\rr^{2})$ with
  respect to the norm
  %
  %
  \begin{equation}
  \begin{split}
    \|F\|_{\mathcal{X}_{s,b}} = \left( \sum_{n \in \zz} (1 + \xi^{2})^{s} \int_{\rr}
    (1 + | | \tau | - \xi^{2} |)^{2b} \wh{F}(n, \tau) d \tau\right)^{1/2}.
  \end{split}
  \label{eqn:bous-norm-real}
  \end{equation}
  %
  %
  %
  %
\end{definition}
%
%
We need only establish the following bilinear estimate. All other arguments are
analagous to those in the periodic case.
%
\begin{proposition}
\label{prop:bilin-est-real}
  Let $s > -1/4$ and $u,v \in \mathcal{X}_{s, -a}$. If either
  \begin{enumerate}[(i)]
   \item{$s \ge 0$, $b > 1$, and $1/4 < a< 1/2$ }
     \label{first-it-real}
   \item{ $-1/4 < s< 0$, $b > 1/2$, and $1/4 < a < 1/2$ such that $| s | <
     a/2$,}
     \label{sec-item-real}
  \end{enumerate}
 then there exists $c > 0$ depending only on $a$, $b$, and $s$ such that
  %
  %
  \begin{equation*}
  \begin{split}
    \| uv \|_{\mathcal{X}_{s,-a}} \le c \| u \|_{\mathcal{X}_{s,b}} \| v \|_{\mathcal{X}_{s,b}}.
  \end{split}
  \end{equation*}
  %
  %
\end{proposition}


\subsubsection{Proof of \autoref{prop:bilin-est-real}.} 
\label{sssec:bilin-est-real}
By
duality, it suffices to show that when $b>1/2$, $1/4 < a <1/2$ are satisfied, we have
%
%%
\begin{equation}
	\label{duality-est-real}
	\begin{split}
    |	\int_{\rr} \int_{\rr} (1 + |\xi|)^{s}
		\phi(\xi, \tau) \wh{uv}(\xi, \tau)(1 
    + | |\tau| - \xi^{2} |^{-a}) d \tau d \xi | \lesssim \|u\|_{\mathcal{X}_{s,b}}
    \|v\|_{\mathcal{X}_{s,b}}
    \|\phi \|_{L^{2}_{\xi, \tau}}.
	\end{split}
\end{equation}
Note first that $|\wh{uv}(\xi, \tau) |  = | \wh{u} *  \wh{v} 
(\xi, \tau)|$. From this it follows that
%
%
\begin{equation}
	\label{non-lin-rep-real}
	\begin{split}
		| \wh{uv}(\xi, \tau)|
    & = | \sum_{\xi_{1} \in \zz }  \int
    \wh{u}\left( \xi_1,  \tau_1 \right) \wh{v}\left( \xi - \xi_1 , \tau - \tau_1   
\right) d \tau_1 |
\\
& \le  \sum_{\xi_{1} \in \zz }  \int
    |\wh{u}\left( \xi_1,  \tau_1 \right)| |\wh{v}\left( \xi - \xi_1 , \tau - \tau_1   
\right)| d \tau_1 
\\
& = \sum_{\xi_1 \in \zz } \int \frac{c_u\left( \xi_1, \tau_1 
\right)}{\langle \xi_1 \rangle ^s \langle |\tau_1| - \xi_1^{2} | \rangle ^{b}}
\\
& \times \frac{c_{v}\left( \xi - \xi_1, \tau - \tau_1 \right)}{\langle \xi -
\xi_1 \rangle ^s\ \langle |\tau - \tau_1 | -  (\xi - \xi_1)^{2} \rangle^{b}}
  \ d \tau_1 
\end{split}
\end{equation}
%
%
where for clarity of notation we have introduced 
%
%
%
\begin{equation*}
\begin{split}
\langle k \rangle \doteq 1 + |k|
\end{split}
\end{equation*}
%
%
and
%
\begin{equation*}
	\begin{split}
		c_h(\xi, \tau) =
			\langle \xi \rangle ^s \langle |\tau| - \xi^{2} \rangle ^{b} | \wh{h}\left( \xi, \tau \right) |.
	\end{split}
\end{equation*}
%
%
From our work above, it follows that 
%
%
\begin{equation}
	\label{convo-est-starting-pnt-real}
	\begin{split}
		 & \langle \xi \rangle^s \langle \tau - \xi^{2} \rangle^{-a} | \wh{uv}\left( 
		\xi, \tau \right) |
		\\
		& \le \langle |\tau| - \xi^{2} \rangle^{-a}
		\sum_{\xi_1 \in \zz} \int \frac{\langle \xi \rangle^{s}}{\langle \xi_1 \rangle^s
    \langle \xi - \xi_1 \rangle^s} 
		\times \frac{c_f(\xi_1, \tau_1)}{\langle |\tau_1| - \xi_1^{2} \rangle ^{b}}
		\\
		& \times
		\frac{c_g(\xi - \xi_1, \tau - \tau_1 )}{\langle |\tau - \tau_1| - (\xi - \xi_1)^{2}
    \rangle^{b}}\ d \tau_1.
	\end{split}
\end{equation}
%
%
Hence, 
%
%
\begin{equation}
  \label{pre-fubini-int-form-real}
	\begin{split}
    |\text{lhs of} \ \eqref{duality-est-real}|
    & \lesssim \int_{\rr} \int_{\rr}     \int_{\rr}  \int_{\rr} \phi(\xi, \tau)
    c_f(\xi_1, \tau_1)
		c_g(\xi - \xi_1, \tau - \tau_1 )
		\\
    & \times \frac{\langle \xi \rangle ^{s}}{\langle \xi_{1} \rangle ^{s} \langle
    \xi-\xi_{1} \rangle ^{s}} \times \frac{1}{ \langle \tau - \xi^{2} \rangle^{a}
\langle |\tau| - \xi^{2} \rangle
    ^{b}\langle |\tau_{1}|-\xi_{1}^{2} \rangle ^{-b}\langle | \tau|-\xi_{2}^{2}
    \rangle ^{b}} d \tau_1 d \xi_{1} d \tau d \xi.
	\end{split}
\end{equation}
%
Let $A \subset \rr^{4}$, and $\chi_{A}(\xi, \tau, \xi_{1}, \tau_{1})$ be its
characteristic function. Then by Cauchy-Schwartz in
$\tau_{1}, \xi_{1}$, we bound
%
%
%
\begin{equation*}
\begin{split}
  & \int_{\rr} \int_{\rr}     \int_{\rr}  \int_{\rr} \chi_{A} \phi(\xi, \tau)
    c_f(\xi_1, \tau_1)
		c_g(\xi - \xi_1, \tau - \tau_1 )
		\\
    & \times \frac{\langle \xi \rangle ^{s}}{\langle \xi_{1} \rangle ^{s} \langle
    \xi-\xi_{1} \rangle ^{s}} \times \frac{1}{ \langle \tau - \xi^{2} \rangle^{a}
\langle |\tau| - \xi^{2} \rangle
    ^{b}\langle |\tau_{1}|-\xi_{1}^{2} \rangle ^{-b}\langle | \tau|-\xi_{2}^{2}
    \rangle ^{b}} d \tau_1 d \xi_{1} d \tau d \xi
\end{split}
\end{equation*}
%
%
by
%
%
\begin{equation}
	\label{10g-real}
	\begin{split}
    & \int_{\rr} \int_{\rr} \phi(\xi, \tau) \langle | \tau | - \xi^{2} \rangle
    ^{-a} \langle \xi \rangle ^{s}
    \\
    & \times \left( \int_{\rr} \int_{\rr}
    \frac{\chi_{A}}{\langle \xi_{1} \rangle ^{2s} \langle \xi-\xi_{1} \rangle ^{2s} \langle |
    \tau_{1} | - \xi_{1}^{2}\rangle ^{2b} \langle | \tau - \tau_{1} | -
    (\xi - \xi_{1})^{2} \rangle ^{2b}} d \tau_{1} d \xi_{1} \right)^{1/2}
    \\
    & \times \left( \int_{\rr} \int_{\rr} c_{u}^{2}(\xi, \tau_{1})
    c_{v}^{2}(\xi - \xi_{1}, \tau - \tau_{1}) d \tau_{1} d \xi_{1}
    \right)^{1/2} d \tau d
    \xi.
  \end{split}
\end{equation}
%
%
Applying Cauchy-Schwartz in $\tau, \xi$, \eqref{10g-real} is bounded by
%
%
\begin{equation*}
  \begin{split}
    & \|\left( \int_{\rr} \int_{\rr } c_{u}^{2}(\xi_1, \tau_1)
  c_{v}^{2} (\xi - \xi_1, \tau - \tau_{1} ) d \tau_1 d \xi_{1}  \right)^{1/2} \|_{L^{2}(\zz \times
		\rr)}
		\\
    & \times  \|\phi(\xi, \tau) \langle | \tau | - \xi^{2} \rangle ^{-a} \langle \xi
    \rangle ^{s}
		\\
    & \times \left( \int_{\rr} \int_{\rr} \frac{\chi_{A}}{ \langle \xi_{1}
    \rangle ^{2s} \langle \xi-\xi_{1} \rangle ^{2s} \langle | \tau_{1}|-\xi_{1}^{2}
    \rangle^{2b} \langle  |\tau -
    \tau_{1} | -(\xi - \xi_{1}^{2}
    \rangle^{2b} } d \tau_1 d \xi_{1} \right)^{1/2} \|_{L^2(\zz \times \rr)}
		\\
    & = \|u\|_{\mathcal{X}_{s,b}} \|v\|_{\mathcal{X}_{s,b}} \label{holder-term-real}
     \|\phi(\xi, \tau)     \\
    & \times \left( \langle | \tau | - \xi^{2} \rangle ^{-2a} \langle \xi
    \rangle ^{2s}
    \int_{\rr} \int_{\rr} \frac{\chi_{A}}{ \langle \xi_{1} \rangle ^{2s} \langle
\xi-\xi_{1} \rangle ^{2s}  \langle | \tau_{1}|-\xi_{1}^{2} \rangle^{2b} \langle  |\tau -
    \tau_{1} | -(\xi - \xi_{1}^{2}
    \rangle^{2b} } d \tau_1 d \xi_{1} \right)^{1/2} \|_{L^2(\zz \times \rr)}.
  \end{split}
\end{equation*}
%
Applying H{\"o}lder, we bound this by 
%
%
\begin{equation}
  \label{integral-bound-1st-form}
	\begin{split}
    & \|u\|_{\mathcal{X}_{s,b}} \|v\|_{\mathcal{X}_{s,b}} \| \phi \|_{L^{2}_{\xi, \tau}}
    \\
    & \times \|\left( \langle | \tau | - \xi^{2} \rangle ^{-2a} \langle \xi
    \rangle ^{2s}
    \sum_{n_{1} \in \zz} \int_{\rr} \frac{\chi_{A}}{ \langle \xi_{1} \rangle ^{2s} \langle
\xi-\xi_{1} \rangle ^{2s} \langle | \tau_{1}|-\xi_{1}^{2} \rangle^{2b} \langle  |\tau -
    \tau_{1} | -(\xi - \xi_{1}^{2}
    \rangle ^{2b} } d \tau_1 \right)^{1/2} \|_{L^\infty_{\xi, \tau}}.
	\end{split}
\end{equation}
%
%
Hence, to complete the proof, it will be enough
to show that 
%
%
%
%
\begin{equation}
  \label{key-sup-estimate-real-1}
	\begin{split}
		 \| \langle | \tau | - \xi^{2} \rangle ^{-2a} \langle \xi
    \rangle ^{2s}
\sum_{\xi_{1} \in \zz} \int_{\rr} \frac{1}{  \langle \xi_{1} \rangle ^{2s} \langle
\xi-\xi_{1} \rangle ^{2s} \langle | \tau_{1}|-\xi_{1}^{2} \rangle^{2b}  \langle  |\tau -
    \tau_{1} | -(\xi - \xi_{1}^{2}
    \rangle ^{2b} } d \tau_1 \|_{L^\infty_{\xi, \tau}} < \infty.
	\end{split}
\end{equation}
%
%
By the triangle inequality and the fact that 
%
%
\begin{equation*}
\begin{split}
& | \tau | =
\begin{cases}
  - \tau, \quad & \tau < 0, 
\\
\tau, \quad & \tau > 0
\end{cases}
\end{split}
\end{equation*}
%
%
\eqref{key-sup-estimate-real} will be proved if we can bound the
$L^{\infty}_{\tau, n}$ norm of the quantity
%
%
\begin{equation}
  \label{sup-est-gen-real}
\begin{split}
		  \langle \sigma \rangle ^{-2a} \langle n
    \rangle ^{2s}
\sum_{n_{1} \in \zz} \int_{\rr} \frac{1}{ \langle n_{1} \rangle ^{2s} \langle n-n_{1} \rangle ^{2s} 
\langle \sigma_{1} \rangle^{2b} \langle  \sigma_{2} \rangle^{2b} }
d \tau_1 
	\end{split}
\end{equation}
Let us now return to the right hand side of \eqref{pre-fubini-int-form-real}.
Let $A \subset \rr^{4}$, and $\chi_{A}(\xi, \tau, \xi_{1}, \tau_{1})$ be its
characteristic function, as before.  We seek to bound
\begin{equation*}
\begin{split}
  & \int_{\rr} \int_{\rr}     \int_{\rr}  \int_{\rr} \chi_{A} \phi(\xi, \tau)
    c_f(\xi_1, \tau_1)
		c_g(\xi - \xi_1, \tau - \tau_1 )
		\\
    & \times \frac{\langle \xi \rangle ^{s}}{\langle \xi_{1} \rangle ^{s} \langle
    \xi-\xi_{1} \rangle ^{s}} \times \frac{1}{ \langle \tau - \xi^{2} \rangle^{a}
\langle |\tau| - \xi^{2} \rangle
    ^{b}\langle |\tau_{1}|-\xi_{1}^{2} \rangle ^{-b}\langle | \tau|-\xi_{2}^{2}
    \rangle ^{b}} d \tau_1 d \xi_{1} d \tau d \xi
\end{split}
\end{equation*}
in a slightly different manner than before. First, we apply 
Fubini, then Cauchy-Schwartz in $\xi_{1}, \tau_{1}$ to obtain the bound
%
%
\begin{equation*}
\begin{split}
  & \left[ \int_{\rr} \int_{\rr} c_{f}^{2}(\xi_{1}, \tau_{1}) d \tau_{1} d
  \xi_{1} \right]^{1/2}
  \\
  & \times \left \{ \int_{\rr} \int_{\rr}   
 \left[
  \int_{\rr} \int_{\rr}
   \frac{\langle \xi \rangle ^{s}}{\langle \xi_{1} \rangle ^{s} \langle
   \xi - \xi_{1}\rangle ^{s}} \times \frac{|\phi(\xi, \tau)| c_{g}(\xi -
   \xi_{1}, \tau - \tau_{1})
}{\langle | \tau | - \xi^{2} \rangle
  ^{a} \langle | \tau_{1} | - \xi_{1}^{2} \rangle ^{b} \langle | \tau -
  \tau_{1} | - (\xi - \xi_{1}^{2}) \rangle ^{b}} d \tau d \xi 
  \right]^{2} \right \}^{1/2}
  \\
  & = \| f \|_{X_{s,b}}
  \\
  & \times \left \{ \int_{\rr} \int_{\rr}   
 \left[
  \int_{\rr} \int_{\rr}
   \frac{\langle \xi \rangle ^{s}}{\langle \xi_{1} \rangle ^{s} \langle
   \xi - \xi_{1}\rangle ^{s}} \times \frac{|\phi(\xi, \tau)| c_{g}(\xi -
   \xi_{1}, \tau - \tau_{1})
}{\langle | \tau | - \xi^{2} \rangle
  ^{a} \langle | \tau_{1} | - \xi_{1}^{2} \rangle ^{b} \langle | \tau -
  \tau_{1} | - (\xi - \xi_{1}^{2}) \rangle ^{b}} d \tau d \xi 
  \right]^{2} d \tau_{1} d \xi_{1} \right \}^{1/2}
\end{split}
\end{equation*}
%
Applying Cauchy-Schwartz in $\tau, \xi$, we bound the last line by 
%
%
\begin{equation*}
\begin{split}
& \left \{ \int_{\rr} \int_{\rr}   
  \left [ \int_{\rr} \int_{\rr}
  | \phi(\xi, \tau)|^{2} c_{g}^{2}(\xi - \xi_{1}, \tau - \tau_{1}) d \tau d \xi 
    \right ] \right . 
   \\
   & \left. \times \left [ \int_{\rr} \int_{\rr} \frac{\langle \xi \rangle
   ^{2s}}{\langle \xi_{1} \rangle ^{2s} \langle \xi - \xi_{1}\rangle ^{2s}}
   \times \frac{\chi_{A}}{\langle | \tau | - \xi^{2} \rangle ^{2a} \langle | \tau_{1} |
   - \xi_{1}^{2} \rangle ^{2b} \langle | \tau - \tau_{1} | - (\xi - \xi_{1}^{2})
   \rangle ^{2b}} d \tau d \xi \right ] \right \}^{1/2}d \tau_{1} d \xi_{1}
\end{split}
\end{equation*}
%
%
which by Holder is bounded by 
%
%
%
\begin{equation}
  \label{integral-bound-2nd-form}
\begin{split}
  & \| \int_{\rr} \int_{\rr} \frac{\langle \xi \rangle ^{2s}}{\langle \xi_{1} \rangle ^{2s} \langle
  \xi - \xi_{1}\rangle ^{2s}}  \times \frac{\chi_{A}}{\langle | \tau | - \xi^{2} \rangle
  ^{2a} \langle | \tau_{1} | - \xi_{1}^{2} \rangle ^{2b} \langle | \tau -
  \tau_{1} | - (\xi - \xi_{1}^{2}) \rangle ^{2b}} d \tau d \xi
  \|_{L^{\infty}_{\xi_{1}, \tau_{1}}}^{1/2}
  \\
  & \times \|\phi\|_{L^{2}} \| g \|_{X_{s,b}}
\end{split}
\end{equation}
%
%
Now consider the family $\{A_{j}\}_{1}^{n}, A_{j} \subset \rr^{4}$ with
$$\bigcup_{1}^{n} A_{j}= \rr^{4}.$$ From \eqref{integral-bound-1st-form},
\eqref{integral-bound-2nd-form}, and our preceding argumentation,
we see that the proof of \autoref{prop:bilin-est-real} reduces to showing that
either 
%
%
%
%
\begin{equation}
  \label{key-sup-estimate-real}
  \begin{split}
     \| \langle | \tau | - \xi^{2} \rangle ^{-2a} \langle \xi
    \rangle ^{2s}
    \int_{\rr} \int_{\rr} \frac{\chi_{A_{j}}}{ \langle \xi_{1} \rangle ^{2s} \langle
\xi-\xi_{1} \rangle ^{2s} \langle | \tau_{1}|-\xi_{1}^{2} \rangle^{2b}  \langle  |\tau -
    \tau_{1} | -(\xi - \xi_{1}^{2}
    \rangle ^{2b} } d \tau_1 d \xi_{1} \|_{L^\infty_{\xi, \tau}} < \infty.
  \end{split}
\end{equation}
%
or
%%
\begin{equation}
\begin{split}
  & \| \int_{\rr} \int_{\rr} \frac{\langle \xi \rangle ^{2s}}{\langle \xi_{1} \rangle ^{2s} \langle
  \xi - \xi_{1}\rangle ^{2s}}  \times \frac{\chi_{A_{j}}}{\langle | \tau | - \xi^{2} \rangle
  ^{2a} \langle | \tau_{1} | - \xi_{1}^{2} \rangle ^{2b} \langle | \tau -
  \tau_{1} | - (\xi - \xi_{1}^{2}) \rangle ^{2b}} d \tau d \xi
  \|_{L^{\infty}_{\xi_{1}, \tau_{1}}}
\end{split}
\end{equation}
for each $j \in \left\{ 0,1,\dots,n \right\}$. 
By the triangle inequality and the fact that 
%
%
\begin{equation*}
\begin{split}
& | \tau | =
\begin{cases}
  - \tau, \quad & \tau < 0, 
\\
\tau, \quad & \tau > 0
\end{cases}
\end{split}
\end{equation*}
%
%
\eqref{key-sup-estimate-real} will be proved if we can bound the
$L^{\infty}_{\tau, \xi}$ norm of the quantity
%
%
\begin{equation}
  \label{sup-est-gen-real-1}
\begin{split}
      \langle \sigma \rangle ^{-2a} \langle \xi
    \rangle ^{2s}
    \int_{\rr} \int_{\rr} \frac{1}{ \langle \xi_{1} \rangle ^{2s} \langle \xi-\xi_{1} \rangle ^{2s} 
\langle \sigma_{1} \rangle^{2b} \langle  \sigma_{2} \rangle^{2b} }
d \tau_1 d \xi_{1} 
  \end{split}
\end{equation}
%
for the following cases.
\begin{enumerate}[(i)]
    \item $ \sigma=\tau+\xi^2,\quad \sigma_1=\tau_1+\xi_1^2,\quad \sigma_2=\tau -
      \tau_1+(\xi - \xi_1)^2$,
\label{it-real-1}
    \item $ \sigma=\tau-\xi^2,\quad \sigma_1=\tau_1-\xi_1^2,\quad \sigma_2=\tau - \tau_1+(\xi - \xi_1)^2$,
\label{it-real-2}
    \item  $\sigma=\tau+\xi^2,\quad \sigma_1=\tau_1-\xi_1^2,\quad \sigma_2=\tau - \tau_1+(\xi - \xi_1)^2$,
      \label{it-real-3}
    \item $\sigma=\tau-\xi^2,\quad \sigma_1=\tau_1+\xi_1^2,\quad \sigma_2=\tau - \tau_1-(\xi - \xi_1)^2$,
\label{it-real-4}
    \item $\sigma=\tau+\xi^2,\quad \sigma_1=\tau_1+\xi_1^2,\quad \sigma_2=\tau - \tau_1-(\xi - \xi_1)^2$,
\label{it-real-5}
    \item $\sigma=\tau-\xi^2,\quad \sigma_1=\tau_1-\xi_1^2,\quad \sigma_2=\tau - \tau_1-(\xi - \xi_1)^2$.
\label{it-real-6}
\end{enumerate}
%
%
\begin{framed}
\begin{remark}
Note that the cases $\sigma=\tau+\xi^2,\quad \sigma_1=\tau_1-\xi_1^2,\quad
\sigma_2=\tau - \tau_1-(\xi - \xi_1)^2$ and $\sigma=\tau-\xi^2,\quad
\sigma_1=\tau_1+\xi_1^2,\quad \sigma_2=\tau - \tau_1+(\xi - \xi_1)^2$ cannot occur, since
$\tau_1< 0, \tau-\tau_1< 0$ implies $\tau<0$ and $\tau_1\geq 0, \tau-\tau_1\geq
0$ implies $\tau\geq 0$.
\end{remark}
\end{framed}
%
Observe that the transformation $(\xi, \tau, \xi_{1}, \tau_{1}) \mapsto -(\xi, \tau,
\xi_{1}, \tau_{1})$ reduces \eqref{it-real-3} to \eqref{it-real-4}, \eqref{it-real-2} to
\eqref{it-real-5}, and \eqref{it-real-1} to \eqref{it-real-6}. Furthermore, the change of
variables $\tau_{2} = \tau - \tau_{1}, \xi_{2} = \xi - \xi_{1}$, and the
transformation $(\xi, \tau, \xi_{2}, \tau_{2}) \mapsto - (\xi, \tau, \xi_{2},
\tau_{2})$ reduces \eqref{it-real-5} to \eqref{it-real-4}. Since $L^{2}$ is invariant
under change of variables and reflections, we may without loss of generality
restrict our attention to cases \eqref{it-real-4} and \eqref{it-real-6}.
 \subsubsection{Case \eqref{it-real-6}} 
\label{sssec:case-it-real-6}
Noting that 
%
%
\begin{equation*}
\begin{split}
  \tau - \xi^{2} - \left[ (\tau + \xi_{1}^{2}) + (\tau - \tau_{1}) - (\xi -
  \xi_{1})^{2} \right] = 2\xi_{1}\xi
\end{split}
\end{equation*}
%
%
we have 
%
%
\begin{enumerate}[(i)]
  \item{$\langle \tau - \xi^{2} \rangle  \gtrsim \langle \xi_{1}\xi \rangle$, or}
    \\
  \item{  $\langle \tau_{1} - \xi_{1}^{2} \rangle  \gtrsim \langle \xi_{1}\xi \rangle$, or}
    \\
  \item{ $\langle \tau - \tau_{1} - (\xi - \xi_{1})^{2} \rangle  \gtrsim \langle \xi_{1}\xi
    \rangle$}.
\end{enumerate}
%By the symmetry of the convolution, we assume without loss of generality
%that
%%
%%
%\begin{equation*}
%\begin{split}
  %\langle \tau_{1} - \xi_{1} \rangle \le  \langle \tau - \tau_{1} - (\xi - \xi_{1})^{2} \rangle 
%\end{split}
%\end{equation*}
%
%
We now partition $\rr^{4}$ into three sets 
%
%
%\begin{align*}
%B_1&=\{(\xi, \xi_1, \tau, \tau_1)\in \rr^4: |\xi_1| \le 1\},\\
%B_2&=\{(\xi, \xi_1, \tau, \tau_1)\in \rr^4: |\xi_1| \ge 1 \text{ and } | \xi
%| \le 1 \},\\
%B_3&=\{(\xi, \xi_1, \tau, \tau_1)\in \rr^4: |\xi_1| \ge 1, | \xi | \ge 1, \text{
%and } | \xi | \ge | \xi_{1}|/2 \},\\
%B_4&=\{(\xi, \xi_1, \tau, \tau_1)\in \rr^4: | \xi_{1} | \ge 1, | \xi| \ge 1 ,
%\text{ and } | \xi | \le | \xi_{1} |/2 \}.
%\end{align*} 
%
\begin{align*}
B_1&=\{(\xi, \xi_1, \tau, \tau_1)\in \rr^4: \langle \tau - \xi^{2} \rangle
\gtrsim \langle \xi_{1}\xi \rangle\},\\
B_2&=\{(\xi, \xi_1, \tau, \tau_1)\in \rr^4: \langle \tau_{1} - \xi_{1}^{2} \rangle  \gtrsim \langle \xi_{1}\xi \rangle \},\\
B_3&=\{(\xi, \xi_1, \tau, \tau_1)\in \rr^4: \langle \tau - \tau_{1} - (\xi -
\xi_{1})^{2} \rangle  \gtrsim \langle \xi_{1}\xi \rangle \}.
\end{align*} 

%
Then from the inequality
%
%
\begin{equation*}
\begin{split}
  \langle \xi \rangle  \le \langle \xi_{1} \rangle \langle \xi - \xi_{1} \rangle 
\end{split}
\end{equation*}
%
we obtain
%
%
\begin{equation*}
\begin{split}
  \langle \xi \rangle ^{s} \le \langle \xi_{1} \rangle ^{\gamma(s)},
  \quad 
  \gamma(s) = 
  \begin{cases} 0, \quad & s \ge 0
    \\
    4|s|, \quad & s < 0.
  \end{cases}
\end{split}
\end{equation*}
%
%
%
%
%
and so
%
%
\begin{equation}
\begin{split}
  |  \eqref{sup-est-gen-real} \chi_{B_{1}}|
  & =  \langle \tau - \xi^{2} \rangle
  ^{-2a} \langle \xi \rangle ^{2s} \int _{\rr} \int_{\rr} \frac{\chi_{B_{1}}}{
  \langle \xi_{1} \rangle ^{2s} \langle \xi-\xi_{1}\rangle ^{2s} \langle \tau_{1} - \xi_{1}^{2} \rangle ^{2b}\langle
  \tau - \tau_{1} - (\xi -\xi_{1})^{2}\rangle ^{2b}}d \tau_{1} d \xi_{1} 
  \\
  & \le  \int_{\rr} \int_{\rr} \frac{\chi_{B_{1}}
  \langle
  \xi_{1}\rangle ^{\gamma(s) -2a} 
}{ \langle \tau_{1} - \xi_{1}^{2} \rangle ^{2b}\langle
  \tau - \tau_{1} - (\xi - \xi_{1})^{2}\rangle ^{2b}}
  d \tau_{1} d \xi_{1} 
  \\
  & \le \int_{\rr} \int_{\rr} \frac{\chi_{B_{1}}
  }{ \langle \tau_{1} - \xi_{1}^{2} \rangle ^{2b}\langle
  \tau - \tau_{1} - (\xi - \xi_{1})^{2}\rangle ^{2b}}
  d \tau_{1} d \xi_{1}, \quad s \ge 0 \text{ or } s \ge -a/2
  \\
  & \le \int _{\rr} 
  \frac{\chi_{B_{1}}}{\langle \tau - \xi^{2} + 2 \xi
  \xi_{1} - 2 \xi_{1}^{2} \rangle ^{2b}}d \xi_{1}, \quad (\autoref{lem:calc}) 
  \\
  & < \infty
\end{split}
\end{equation}
%
where the final step follows from a corollary to \autoref{lem:calc}.
%
%
%%%%%%%%%%%%%%%%%%%%%%%%%%%%%%%%%%%%%%%%%%%%%%%%%%%%%
%
%
%                Corollary to calculus lemma
%
%
%%%%%%%%%%%%%%%%%%%%%%%%%%%%%%%%%%%%%%%%%%%%%%%%%%%%%
%
%
\begin{corollary}
  For $a_{0}, a_{1}, a_{2} \in \rr$ with $a_{2} \neq 0$ and $q > 1/2$
  %
  %
  \begin{equation*}
  \begin{split}
    \int_{\rr} \frac{1}{\langle a_{0} + a_{1}x + a_{2}x^{2} \rangle ^{q}} dx \le c
  \end{split}
  \end{equation*}
  %
  %
\label{cor:integral-bound}
\end{corollary}
%
%Similarly, 
%%
%%
%\begin{equation}
%\begin{split}
  %|  \eqref{sup-est-gen-real} \chi_{B_{2}}|
  %& =  \langle \tau - \xi^{2} \rangle
  %^{-2a} \langle \xi \rangle ^{2s} \int _{\rr} \int_{\rr} \frac{\chi_{B_{1}}}{
  %\langle \xi_{1} \rangle ^{2s} \langle \xi-\xi_{1}\rangle ^{2s} \langle \tau_{1} - \xi_{1}^{2} \rangle ^{2b}\langle
  %\tau - \tau_{1} - (\xi -\xi_{1})^{2}\rangle ^{2b}}d \tau_{1} d \xi_{1} 
  %\\
  %& \le  \int_{\rr} \int_{\rr} \frac{\chi_{B_{1}}
  %\langle
  %\xi_{1}\rangle ^{\gamma(s) -2b} 
%}{ \langle \tau_{1} - \xi_{1}^{2} \rangle ^{2b}\langle
  %\tau - \tau_{1} - (\xi - \xi_{1})^{2}\rangle ^{2b}}
  %d \tau_{1} d \xi_{1} 
  %\\
  %& \le \int_{\rr} \int_{\rr} \frac{\chi_{B_{1}}
  %}{ \langle \tau_{1} - \xi_{1}^{2} \rangle ^{2b}\langle
  %\tau - \tau_{1} - (\xi - \xi_{1})^{2}\rangle ^{2b}}
  %d \tau_{1} d \xi_{1}, \quad s \ge 0 \text{ or } s \ge -a/2
  %\\
  %& \le \int _{\rr} 
  %\frac{\chi_{B_{1}}}{\langle \tau - \xi^{2} + 2 \xi
  %\xi_{1} - 2 \xi_{1}^{2} \rangle ^{2b}}d \xi_{1}, \quad (\autoref{lem:calc}) 
  %\\
  %& < \infty
%\end{split}
%\end{equation}

\section{Second order Modified Boussinesq  equation}
\label{sec:intro}
We consider the initial value problem (ivp) for a modified Boussinesq
equation ($B_4$) equation 
\begin{gather}
  u_{tt} - u_{xx} + (u^2)_{xx} = 0,
  \label{eqn:mb}
  \\
  u(x,0) = u_{0}(x), \quad u_{0} \in H^{s}
  \label{eqn:mb-init-data}
\end{gather}
and conjecture the following.
%
%
%%%%%%%%%%%%%%%%%%%%%%%%%%%%%%%%%%%%%%%%%%%%%%%%%%%%%
%
%
%                Main Theorem
%
%
%%%%%%%%%%%%%%%%%%%%%%%%%%%%%%%%%%%%%%%%%%%%%%%%%%%%%
%
%
\begin{theorem}
  If $s>s_c$ then then the  i.v.p, for the fourth order modified
  Boussinesq  equation is well-posed
  \begin{itemize}
    \item In $H^s(\rr)$ if $s > s_c$
    \item In $H^{s}(\ci)$ if $s > s_c + 1/4$,
  \end{itemize}
  and the data-to-solution map is  ?? (continuous?, Lip?, smooth, analytic?). 
  \label{thm:wp}
\end{theorem}
%
%
%
%
%
Since the scaling conserves data in $\dot{H}^{-3/2}$......
It seems that this equation is ``like KdV".
So one may expect KdV type theorems...
That is, $s_c=-3/4$ on the line and $s_c=-1/2$ on the circle,
if one uses bilinear estimates.
But, Kappeler and collaborators went all the way to $-1$ for KdV.
However KdV is integrable. Is this equation integrable?
Also, people conjecture that the critical index for KdV well-posedness 
in some appropriate sense should be the scaling index which is  $-3/2$.

\newpage
\appendix
\section{}
%\subsection{Proof of \autoref{lem:embedding}}
%%
%%
%\begin{equation*}
%\begin{split}
  %\| u(t) - u(t') \|_{H^s}^{2}
  %& = \sum_{n \in \zz} (1 + |n|)^{2s} [\wt{u}(n, t) - \wt{u}(n, t')]
  %\\
  %& = \sum_{n \in \zz} (1 + |n|)^{2s} \int_{\rr} (e^{it\tau} - e^{it'
  %\tau})\wh{u}(n, \tau) d \tau
  %\\
  %& \le 2 \sum_{n \in \zz} (1 + |n|)^{2s} \int_{\rr} \wh{u}(n, \tau) d \tau
  %\\
  %& \simeq \sum_{n \in \zz} (1 + |n|)^{2s} \int_{\rr} (1 + | | \tau | -
  %n^{2} |)^{b}(1 + | | \tau | - n^{2} |)^{-b} | \wh{u}(n, \tau) | d \tau.
%\end{split}
%\end{equation*}
%%
%%
%Applying Cauchy-Schwartz in $\tau$, we bound this by
%%
%%
%\begin{equation*}
%\begin{split}
  %& \sum_{n \in \zz} (1 + |n|)^{2s} \left[ \int_{\rr} (1 + | | \tau | -
  %n^{2} |)^{2b} | \wh{u}(n, \tau) |^{2} d \tau \right]^{1/2} \left[ \int_{\rr}
  %(1 + | | \tau | - n^{2} |)^{-2b} d \tau \right]^{1/2}
  %\\
  %& = \sum_{n \in \zz} (1 + |n|)^{2s} \left[ \int_{\rr} (1 + | | \tau | -
  %n^{2} |)^{2b} | \wh{u}(n, \tau) |^{2} d \tau \right]^{1/2} \left[ \int_{\rr}
  %(1 +  | \tau' | )^{-2b} d \tau' \right]^{1/2}
  %\\
  %& = c \sum_{n \in \zz} (1 + |n|)^{2s} \left[ \int_{\rr} (1 + | | \tau | -
  %n^{2} |)^{2b} | \wh{u}(n, \tau) |^{2} d \tau \right]^{1/2} \qquad (b > 1/2). 
%\end{split}
%\end{equation*}
%%
%%
%%
%%
%Applying Cauchy-Schwartz in $n$ then gives the bound
%%
%%
%\begin{equation*}
%\begin{split}
  %\sum_{n \in \zz} (1 + |n|)^{2s} \int_{\rr} (1 + | | \tau | - n^{2}
  %|)^{2b} \wh{u}(n, \tau) d \tau = \| u \|_{X_{s,b}}^{2}.
%\end{split}
%\end{equation*}
%%
%%
%An application of dominated convergence completes the proof. \qquad \qedsymbol
\subsection{Deriving the Integral form of the $B_{4}$ ivp
\eqref{lin-mb}-\eqref{lin-mb-init-data-1}.} 
\label{ssec:integral-form-deriv}
We will describe two different methods.
\subsubsection{Variation of Parameters} 
\label{sssec:var-param}
Taking the spatial Fourier transform of the linear $B_{4}$ ivp
\eqref{lin-mb}-\eqref{lin-mb-init-data-1} yields
%
%
\begin{gather}
  \wh{u_{tt}} + n^{4} \wh{u} = 0
  \label{four-trans-lin-mb}
  \\
  \wh{u}(n, 0) = \wh{u_{0}}(n), \quad \wh{u_{t}}(n, 0) = \wh{u_{1}}(n)
  \label{four-trans-lin-mb-data}
\end{gather}
Substituting the ansatz $e^{\lambda t}$ into \eqref{four-trans-lin-mb}, we
obtain the characteristic equation
%
%
\begin{equation*}
\begin{split}
  \lambda^{2} + n^{4} = 0
\end{split}
\end{equation*}
%
%
which gives 
%
%
\begin{equation*}
\begin{split}
  \lambda = \pm in^{2}.
\end{split}
\end{equation*}
%
Since
\eqref{lin-mb} is second order, and $e^{in^{2}t}$ and $e^{-in^{2}t}$ are
linearly independent solutions to \eqref{lin-mb}, it follows that $\left\{
e^{in^{2}t}, e^{-in^{2}t}
\right\}$ is a basis for all solutions of \eqref{lin-mb}. Therefore, the general
solution of \eqref{lin-mb} takes the form
%
%
\begin{equation}
  \label{explicit-homog-soln}
\begin{split}
  \wh{u}(n,t) = c_{1}e^{in^{2}t} + c_{2}e^{-in^{2}t}.
\end{split}
\end{equation}
%
%
which, in conjunction with initial data 
\eqref{four-trans-lin-mb-data}, implies
%
%
\begin{gather*}
   c_{1} + c_{2} = \wh{u_{0}}(n)
  \\
   in^{2}c_{1} - in^{2}c_{2} = \wh{u_{1}}(n).
\end{gather*}
%
%
Solving for $c_{1}$ and $c_{2}$, we obtain
%
%
\begin{gather*}
  c_{1} = \frac{1}{2} \wh{u_{0}}(n) + \frac{1}{2in^{2}}\wh{u_{1}}(n),
  \\
  c_{2} = \frac{1}{2} \wh{u_{0}}(n) - \frac{1}{2in^{2}}\wh{u_{1}}(n).
\end{gather*}
%
%
Substituting into \eqref{explicit-homog-soln}, we obtain the unique solution to
ivp \eqref{four-trans-lin-mb}-\eqref{four-trans-lin-mb-data}
%
%
\begin{equation*}
\begin{split}
  \wh{u}(n, t) = \wh{u_{0}}(n) \frac{e^{in^{2}t} + e^{-in^{2}t}}{2} +
  \wh{u_{1}}(n)\frac{e^{in^{2}t} - e^{-in^{2}}t}{2 i n^{2}}
\end{split}
\end{equation*}
%
or
%
%
\begin{equation*}
\begin{split}
  u(x,t) = R_{t}u_{0} + S_{t}u_{1}
\end{split}
\end{equation*}
%
%
where $R_{t}$ and $S_{t}$ are defined as in \eqref{sin-cos-op}.
Turning our attention now to the $B_{4}$ ivp
\eqref{eqn:mb-2}-\eqref{eqn:mb-init-data-2} and taking the spatial Fourier
transform yields 
%
%
\begin{gather}
  \wh{u_{tt}} + n^{4} \wh{u} = -\wh{(u^{2})_{xx}}
  \label{four-trans-mb}
  \\
  \wh{u}(n, 0) = \wh{u_{0}}(n), \quad \wh{u_{t}}(n, 0) = \wh{u_{1}}(n)
  \label{four-trans-mb-data}
\end{gather}
which for fixed $t$ is a second order ODE in $n$. 
We now need the following.
%
%
%%%%%%%%%%%%%%%%%%%%%%%%%%%%%%%%%%%%%%%%%%%%%%%%%%%%%
%
%
%                General Solution NonHomog 2nd Order Eqn
%
%
%%%%%%%%%%%%%%%%%%%%%%%%%%%%%%%%%%%%%%%%%%%%%%%%%%%%%
%
%
\begin{lemma}
\label{lem:nonhomog-ode-soln}
The general solution of the $2$nd order nonhomogeneous ODE 
%
%
\begin{equation}
  \label{2nd-order-ode}
\begin{split}
y'' + p(t)y' + q(t)y = g(t)
\end{split}
\end{equation}
%
%
can be written in the form
%
%
\begin{equation*}
\begin{split}
  y = c_{1}y_{1}(t) + c_{2}y_{2}(t) + y_{p}(t),
\end{split}
\end{equation*}
%
%
where $y_{1}$ and $y_{2}$ are linearly independent solutions to the
corresponding homogeneous equation (i.e. $g(t) = 0$), $c_{1}$ and $c_{2}$ are
arbitrary constants, and $y_{p}$ is some specific solution of the nonhomogeneous
equation. Furthermore, one such $y_{p}$ is given by
%
%
\begin{equation}
  \label{2nd-order-ansatz}
\begin{split}
  y_{p} = y_{1}v_{1} + y_{2} v_{2}
\end{split}
\end{equation}
%
%
where $v_{1}$ and $v_{2}$ are solutions to the system
\begin{gather}
  \label{cancel-rel-1}
  y_{1} v_{1}' + y_{2} v_{2}' = 0
  \\
  \label{cancel-rel-2}
  y_{1}' v_{1}' + y_{2}' v_{2}' = g(t).
\end{gather}
\end{lemma}
%
{\bf Proof.} Substituting the ansatz \eqref{2nd-order-ansatz} into
the left hand side of \eqref{2nd-order-ode}, we obtain the expression
%
%
%
%
\begin{equation*}
\begin{split}
  (y_{1}v_{1} + y_{2}v_{2})'' + p(t)(y_{1}v_{1} + y_{2}v_{2})' +
  q(t)(y_{1}v_{1} + y_{2}v_{2}) 
\end{split}
\end{equation*}
%
%
or
%
%
\begin{equation*}
  \begin{split}
    & y_{1}'' v_{1} + 2y_{1}'v_{1}' + y_{1}v_{1}'' + y_{2}''v_{2} + 2y_{2}' v_{2}'
  + y_{2} v_{2}'' + p(t)y_{1}'v_{1} + p(t)y_{1}v_{1}'
  \\
  & + p(t)y_{2}'v_{2} + p(t)y_{2}v_{2}' + q(t)y_{1}v_{1} + q(t)y_{2}v_{2} =g
\end{split}
\end{equation*}
%
Collecting terms, this can be rewritten as
%
%
%
%
\begin{equation*}
\begin{split}
  & (y_{1}v_{1}'' + y_{1}' v_{1}') + (y_{2}v_{2}'' + y_{2}' v_{2}') + y_{1}''
  v_{1} + y_{2}'' v_{2} + (y_{1}'v_{1}' + y_{2}' v_{2}')
  \\
  & + p(t)\left(
  y_{1}'v_{1} + y_{2}' v_{2} + y_{1}v_{1}' + y_{2}v_{2}'
  \right) + q(t)\left( y_{1}v_{1} + y_{2}v_{2} \right)
  \end{split}
\end{equation*}
%
or
%
%
%
%
\begin{equation*}
\begin{split}
  & \cancel{(y_{1}v_{1}' + y_{2}v_{2}')'} + y_{1}''
  v_{1} + y_{2}'' v_{2} + \overbrace{(y_{1}'v_{1}' + y_{2}' v_{2}')}^{g}
  \\
  & + p(t)\left(
  y_{1}'v_{1} + y_{2}' v_{2} + y_{1}v_{1}' + y_{2}v_{2}'
  \right) + q(t)\left( y_{1}v_{1} + y_{2}v_{2} \right)
  \end{split}
\end{equation*}
%
%
or
%
%
\begin{equation*}
\begin{split}
  g + \cancel{v_{1}\left[ y_{1}'' + p(t)y_{1}' + q(t)y_{1} \right]} +
  \cancel{v_{2}\left[ y_{2}'' + p(t)y_{2}' + q(t)y_{2}
  \right]}
\end{split}
\end{equation*}
%
%
where the last cancellation is due to the fact that $y_{1}$ and $y_{2}$ are
solutions to the corresponding homogeneous equation of \eqref{2nd-order-ode}
(i.e. $g(t) = 0$). This concludes the proof. \qquad \qedsymbol
%
%
\\
Applying \autoref{lem:nonhomog-ode-soln}, we see that to find the analog of
$y_p$ for the ODE \eqref{four-trans-mb}, we must solve the system
%
%
\begin{gather*}
  e^{in^{2}t}v_{1}' + e^{-in^{2}t}v_{2}'=0
  \\
  in^{2}e^{in^{2}t}v_{1}' - in^{2}e^{-in^{2}t}v_{2}' = -\wh{(u^{2})_{xx}}
\end{gather*}
%
%
or
\begin{gather*}
    \begin{bmatrix}
      e^{in^{2}t} & e^{-in^{2}t}\\
      in^{2}e^{in^{2}t} & -in^{2}e^{-in^{2}t}
    \end{bmatrix}
    \begin{bmatrix}
      v_{1}' \\
      v_{2}'
    \end{bmatrix}
    =
    \begin{bmatrix}
    0\\
    -\wh{(u^{2})_{xx}}
    \end{bmatrix}.
\end{gather*}
This admits the solution
\begin{gather*}
    \begin{bmatrix}
      v_{1}' \\
      v_{2}'
    \end{bmatrix}
    =
    \frac{1}{2in^{2}}
    \begin{bmatrix}
      \wh{(u^{2})_{xx}}e^{-in^{2}t}
      \\
      -\wh{(u^{2})_{xx}}e^{in^{2}t}
    \end{bmatrix}, \quad
\begin{bmatrix}
      v_{1}(0) \\
      v_{2}(0)
    \end{bmatrix}
    =
    \begin{bmatrix}
    0 \\
     0
    \end{bmatrix}
\end{gather*}
which we integrate from $0$ to $t$ to obtain
\begin{gather*}
    \begin{bmatrix}
      v_{1} \\
      v_{2}
    \end{bmatrix}
    =
    \frac{1}{2in^{2}}
    \begin{bmatrix}
      \int_{0}^{t} \wh{(u^{2})_{xx}}e^{-in^{2}t'} dt'
      \\
      \int_{0}^{t} -\wh{(u^{2})_{xx}}e^{in^{2}t'} dt'
    \end{bmatrix}
\end{gather*}
Using relation \eqref{2nd-order-ansatz}, we now see that
%
%
\begin{equation*}
\begin{split}
  y_{p}= e^{in^{2}t}\int_{0}^{t}
  \frac{\wh{(u^{2})_{xx}}e^{-in^{2}t'}}{2in^{2}} dt'
  - e^{-in^{2}t}\int_{0}^{t} \frac{\wh{(u^{2})_{xx}}e^{in^{2}t'}}{2in^{2}} dt'
\end{split}
\end{equation*}
%
%
which when rewritten gives
%
%
\begin{equation*}
\begin{split}
  y_{p}= \int_{0}^{t}\frac{e^{in^{2}(t-t')}-e^{-in^{2}(t-t')}}{2in^{2}}
  \wh{(u^{2})_{xx}} dt'.
\end{split}
\end{equation*}
%
%
Therefore, the unique solution to ivp
\eqref{four-trans-mb}-\eqref{four-trans-mb-data} is given by
%
%
\begin{equation*}
\begin{split}
\wh{u}(n, t) = \wh{u_{0}}(n) \frac{e^{in^{2}t} + e^{-in^{2}t}}{2} +
  \wh{u_{1}}(n)\frac{e^{in^{2}t} - e^{-in^{2}}t}{2 i n^{2}} +
  \int_{0}^{t}\frac{e^{in^{2}(t-t')}-e^{-in^{2}(t-t')}}{2in^{2}}
  \wh{(u^{2})_{xx}} dt'.
\end{split}
\end{equation*}
%
%
Taking the inverse Fourier transform then gives
%
\begin{equation*}
\begin{split}
  u(x,t) = R_{t}u_{0} + S_{t}u_{1} + \int_{0}^{t} S_{t-t'}
  \wh{(u^{2})_{xx}}dt'.
\end{split}
\end{equation*}
%
Hence, we have rewritten the $B_{4}$ ivp
\eqref{eqn:mb-2}-\eqref{eqn:mb-init-data-2} in integral form, as desired. 
%
%
\subsubsection{Reducing to a First Order ODE} 
\label{sssec:first-order-ode}
Taking the spatial Fourier transform of \eqref{lin-mb} yields
the ivp
%
%
\begin{gather*}
  \wh{u_{tt}} + n^{4} \wh{u} = \wh{-u^{2}_{xx}},
  \\
  \wh{u}(n, 0) = \wh{u_{0}}(n), \quad \wh{u_{t}}(n, 0) = \wh{u_{1}}(n)
\end{gather*}
%
%
which we rewrite as 
%
%
\begin{gather}
  \label{eqn:lin-mb-ode}
  y_{tt} + n^{4}y = -f,
  \\
  y(n, 0) = y_{0}(n), \quad y_{t}(n, 0) = y_{1}(n)
\label{eqn:lin-mb-ode-init-data}
\end{gather}
%
%
where
%
%
\begin{gather*}
  \label{not-1}
  y = y(n, t) \doteq \wh{u}(n, t), \quad f = f(n, t) \doteq
  \wh{u^{2}_{xx}}(n,t),
  \\
  \label{not-2}
  y_{0}(n) \doteq \wh{u_{0}}(n), \quad y_{1}(n) = \wh{u_{1}}(n).
\end{gather*}
%
%
Viewing $n$ as fixed, we see that \eqref{eqn:lin-mb-ode} is a second order
linear nonhomogeneous ODE. To solve it, we set 
%
%
\begin{equation*}
  \label{not-3}
\begin{split}
   & v_{1} = y, 
   \\
   & v_{2} = y_{t}
\end{split}
\end{equation*}
%
%
giving
%
%
\begin{equation*}
\begin{split}
  & v_{1}' = v_{2},
  \\
  & v_{2}' = -n^{4}v_{1} - f.
\end{split}
\end{equation*}
%
%
Using the notation
%
%
\begin{equation*}
  \vec v =
  \begin{bmatrix}
     v_{1}  \\
     v_{2} 
  \end{bmatrix},
\end{equation*}
%
%
we then obtain
%
%
\begin{equation}
\begin{split}
\frac{d \vec v}{dt} = 
\begin{bmatrix}
0 & 1 \\
-n^{4} & 0
\end{bmatrix}
\begin{bmatrix}
  v_{1}\\
  v_{2}
\end{bmatrix}
-
\begin{bmatrix}
0\\
f
\end{bmatrix}
\doteq A \vec v - \vec f
\end{split}
\label{eqn:first-order-ode-reduction}
\end{equation}
%
%
Hence, we have reduced solving the $2$nd order ODE \eqref{eqn:lin-mb-ode} to
solving the first order ODE \eqref{eqn:first-order-ode-reduction}. Multiplying
by the integrating factor $e^{-At}$ on both sides of
\eqref{eqn:first-order-ode-reduction}, we obtain
%
%
\begin{equation*}
\begin{split}
  \frac{d}{dt}(e^{-At} \vec v) = -e^{-At} \vec f.
\end{split}
\end{equation*}
%
%
Integrating in time then gives
%
%
\begin{equation*}
\begin{split}
  e^{-At} \vec v(n, t) = \vec v(n, 0) - \int_{0}^{t}e^{-At'} \vec f dt'
\end{split}
\end{equation*}
%
%
or
%
%
\begin{equation}
  \label{ode-vec-soln}
\begin{split}
  \vec v(n, t) = e^{At} \vec v(n, 0) - \int_{0}^{t}e^{A(t - t')} \vec f dt'.
\end{split}
\end{equation}
%
%
We wish to compute $e^{At}$. It is easy to check that $A$ has eigenvalues
$\lambda = \pm in^{2}$, with corresponding eigenvectors 
%
%
\begin{equation*}
\begin{split}
\pm \begin{bmatrix}
1 \\
in^{2}
\end{bmatrix}.
\end{split}
\end{equation*}
%
%
Since $A$ is a square matrix and has no repeated
eigenvalues, it is diagonizable. More precisely,
%
%
%
%
\begin{equation*}
\begin{split}
  A = Q D Q^{-1} = 
  \begin{bmatrix}
  1 & 1
  \\
  in^{2} & -in^{2}
  \end{bmatrix}
  \begin{bmatrix}
    in^{2} & 0 
    \\
    0 & -in^{2}
  \end{bmatrix}
  \begin{bmatrix}
    \frac{1}{2} & \frac{1}{2i n^{2}} \\
    \frac{1}{2} & -\frac{1}{2i n^{2} }
  \end{bmatrix}
\end{split}
\end{equation*}
%
%
where the column vectors of $Q$ are comprised of the eigenvectors of $A$.
Recalling that 
%
%
\begin{equation*}
\begin{split}
  e^{At} \doteq \sum_{n=0}^{\infty} \frac{A^{n}}{n}t^{n}
\end{split}
\end{equation*}
%
%
it is easy to check that 
%
%
\begin{equation*}
\begin{split}
  e^{Q D Q^{-1}} = 
\begin{bmatrix}
  1 & 1
  \\
  in^{2} & -in^{2}
  \end{bmatrix}
  \begin{bmatrix}
    e^{in^{2}} & 0 
    \\
    0 & e^{-in^{2}}
  \end{bmatrix}
  \begin{bmatrix}
    \frac{1}{2} & \frac{1}{2i n^{2}} \\
    \frac{1}{2} & -\frac{1}{2i n^{2} }
  \end{bmatrix}
\end{split}
\end{equation*}
%
%
and 
\begin{equation}
  \label{matrix-exp}
\begin{split}
  e^{Q D Q^{-1}t}
  & = 
\begin{bmatrix}
  1 & 1
  \\
  in^{2} & -in^{2}
  \end{bmatrix}
  \begin{bmatrix}
    e^{in^{2}t} & 0 
    \\
    0 & e^{-in^{2}t}
  \end{bmatrix}
\begin{bmatrix}
    \frac{1}{2} & \frac{1}{2i n^{2}} \\
    \frac{1}{2} & -\frac{1}{2i n^{2} }
  \end{bmatrix}
  \\
  & =
  \begin{bmatrix}
    \frac{1}{2}(e^{in^{2}t} + e^{-in^{2}t}) & \frac{1}{2 i n^{2}} (e^{in^{2}t} -
    e^{-in^{2}t})    \\
    \frac{in^{2}}{2}(e^{in^{2}t} - e^{-in^{2}t}) & \frac{1}{2}(e^{in^{2}t} +
    e^{-in^{2}t})
  \end{bmatrix}.
\end{split}
\end{equation}
%
%
\begin{framed}
\begin{remark}
If a matrix has repeated eigenvalues, it may no longer be diagonizable. However,
any matrix can be written in Jordan canonical form. The above computations
become slighly more complicated in this case. The important observation is that
writing a matrix in Jordan canonical (or, ideally, diagonal) form allows us to
easily compute its exponential. 
\label{rem:jordan-form}
\end{remark}
\end{framed}
%
%
%
%
\begin{framed}
\begin{remark}
  Since $A$ is a particularly simple matrix, i.e. $A^{2} = -n^{4} I$, one can
  compute its exponential easily without the use of diagonalization (this is
  not true in general). From the above equality, we obtain $A^{n} =
  (-1)^{n/2} n^{2n} I$ for even $n$, and so
  %
  %
  \begin{equation*}
  \begin{split}
    e^{At}
    & = \sum_{n=0}^{\infty} \frac{(-1)^{n}n^{4n}t^{2n}}{(2n)!}I + A
    \sum_{n=0}^{\infty} \frac{(-1)^{n} n^{4n} t^{2n + 1}}{(2n + 1)!} I 
    \\
    & = \cos n^{2}t \, I - \frac{\sin n^{2}t}{n^{2}}A
    \\
    & = 
    \begin{bmatrix}
      \cos n^{2}t &  -\frac{\sin n^{2}t}{n^{2}}
      \\
      - n^{2} \sin n^{2}t & \cos n^{2}t
    \end{bmatrix}
    \\
    & = \text{rhs of }\eqref{matrix-exp}
  \end{split}
  \end{equation*}
  %
  %
\label{rem:simpler-comp}
\end{remark}
\end{framed}
%
%
Substituting \eqref{matrix-exp} into \eqref{ode-vec-soln} and recalling our notation, we obtain
%
%
\begin{equation*}
\begin{split}
  v_{1}(n, t) = (e^{in^{2}t} + e^{-in^{2}t})v_{1}(n, 0) + \frac{e^{in^{2}} -
  e^{-in^{2}t}}{2 i n^{2}} v_{1}(n, 0) + \int_{0}^{t} \frac{e^{in^{2}(t - t')} -
  e^{-in^{2}(t-t')}}{2 i n^{2}} f dt'
\end{split}
\end{equation*}
%
%
or
\begin{equation*}
\begin{split}
  y(n, t) = (e^{in^{2}t} + e^{-in^{2}t})y_{0} + \frac{e^{in^{2}} -
  e^{-in^{2}t}}{2 i n^{2}} y_{1} + \int_{0}^{t} \frac{e^{in^{2}(t - t')} -
  e^{-in^{2}(t-t')}}{2 i n^{2}} f dt'
\end{split}
\end{equation*}
or
\begin{equation*}
\begin{split}
  \wh{u}(n, t) = (e^{in^{2}t} + e^{-in^{2}t})\wh{u_0} + \frac{e^{in^{2}} -
  e^{-in^{2}t}}{2 i n^{2}} \wh{u_1} + \int_{0}^{t} \frac{e^{in^{2}(t - t')} -
  e^{-in^{2}(t-t')}}{2 i n^{2}} \wh{(u^{2})_{xx}} dt'.
\end{split}
\end{equation*}
%
Taking the inverse Fourier transform then yields \eqref{eqn:integral-form}, as
desired.
%
%
\subsection{Proof of \autoref{lem:mod-princ-symb-bound}} 
\label{ssec:pf-mod-princ}
By the reverse triangle inequality, we have
%
%
\begin{equation*}
\begin{split}
  | \tau | = | \tau + n^{2} - n^{2} | \ge | | \tau + n^{2} | - n^{2} |.
\end{split}
\end{equation*}
%
%
Furthermore, if $\tau - n^{2} < 0$, then
%
%
\begin{equation*}
\begin{split}
  | | \tau - n^{2} | - n^{2} | = | n^{2} - \tau - n^{2} | = | \tau |
\end{split}
\end{equation*}
%
%
while if $\tau - n^{2} > 0$, then
%
%
\begin{equation*}
\begin{split}
  | | \tau - n^{2} | - n^{2} | \le n^{2} \le \tau = |\tau|
\end{split}
\end{equation*}
%
%
completing the proof. \qquad \qedsymbol
%
%
\subsection{Proof of \autoref{lem:calc}}
%
%
%
By the change of variable $\theta \mapsto a/2 + x$, we have
%
%
\begin{equation*}
	\begin{split}
		\int_{\rr} \frac{1}{(1 + | \theta |)(1 + | a - \theta |)}d \theta
	= \int_{\rr} \frac{1}{(1 + |  a/2 + x |)(1 + | a/2 - x |)}d x.
	\end{split}
\end{equation*}
%
%
Hence, it suffices to show that
%
%
\begin{equation*}
	\begin{split}
		\int_{\rr} \frac{1}{(1 + | a - \theta |)(1 + | a + \theta |)}d \theta
		\lesssim \frac{\log(2 + | a |)}{1 + | a |}.
	\end{split}
\end{equation*}
%
%
Let us leave the case $a = 0$ for last. By symmetry, the cases $a<0$ and $a >0$
are equivalent. Hence, to cover the case $a \neq0$, we may assume
without loss of generality that $a >0$.
%
%
Then
\begin{equation}
	\label{a1}
	\begin{split}
		& \int_{\rr} \frac{1}{(1 + | a - \theta |)(1 + | a + \theta |)}d \theta
		\\
		& = \int_{| \theta| \le a+1 } \frac{1}{(1 + | a - \theta |)(1 + | a + \theta
		|)}d \theta + \int_{| \theta| \ge a+1 } \frac{1}{(1 + | a - \theta |)(1 + |
		a + \theta |)}d \theta.
	\end{split}
\end{equation}
Estimating the second integral of \eqref{a1}, we have
\begin{equation*}
	\begin{split}
		& \int_{| \theta| \ge a+1 } \frac{1}{(1 + | a - \theta |)(1 + | a + \theta
		|)}d \theta 
		\\
		& = \int_{\theta \ge a + 1} \frac{1}{(1 + \theta-a)(1 + \theta+a)} d \theta
		+ \int_{\theta \le -a -1} \frac{1}{(1 + \theta - a) (1 + \theta + a)}d \theta
		\\
		& = \frac{1}{2a} \int_{\theta \ge a + 1} \left[ \frac{1}{1 + \theta -a} -
		\frac{1}{1 + \theta+a} \right] d \theta
		+ \frac{1}{2a} \int_{\theta \le -a-1} \left[ \frac{1}{1 + \theta+a}
		-\frac{1}{1 + \theta -a} \right] d \theta
		\\
		& = \frac{1}{a} \log(1+a)
		\\
		& \lesssim \frac{\log(2 + |a|)}{1 + | a |}.
	\end{split}
\end{equation*}
To evaluate the first integral of \eqref{a1}, we split into the cases $a \le \theta \le
a+1$, $-a \le \theta \le 0$, $0 \le \theta \le a$, and $a \le \theta \le a+1$.
However, note that 
%
%
\begin{equation*}
	\begin{split}
		& \int_{a}^{a+1} \frac{1}{(1 + | a - \theta |)(1 + | a + \theta |)}d \theta =
		\int_{-a-1}^{-a} \frac{1}{(1 + | a - \theta |)(1 + | a + \theta |)}d \theta,
		\\
		& \int_{0}^{a} \frac{1}{(1 + | a - \theta |)(1 + | a + \theta |)}d \theta =
		\int_{-a}^{0} \frac{1}{(1 + | a - \theta |)(1 + | a + \theta |)}d \theta.
	\end{split}
\end{equation*}
%
%
Therefore, we need only consider the cases $a \le \theta \le a+1$ and $0 \le
\theta \le a$.
%
%
\subsection{Case $a \le \theta \le a+1$}
We have
%
%
\begin{equation*}
	\begin{split}
		\int_{a}^{a+1} \frac{1}{(1 + | a-\theta |)(1 + | a + \theta |)}d \theta
		& = \int_{a}^{a+1} \frac{1}{(1 + \theta -a)(1 + a + \theta)}d \theta
		\\
		& = \frac{1}{2a} \int_{a}^{a+1} \left[ \frac{1}{1 + \theta -a} -
		\frac{1}{1 + \theta + a}  \right]d \theta
		\\
		& =\frac{1}{2a} \log\left( \frac{1 + \theta -a}{1 + \theta + a} \right) \Big
		|_a^{a+1}
		\\
		& = \frac{1}{2a} \log\left( \frac{2a+1}{a+1} \right)
		\\
		& \lesssim\frac{\log 2}{2a}
		\\
		& \lesssim \frac{\log(2 + | a |)}{1 + | a |}.
	\end{split}
\end{equation*}
%
%
\subsection{Case $0 \le \theta \le a$}
We have
%
%
\begin{equation*}
	\begin{split}
		\int_{0}^{a} \frac{1}{(1 + | a - \theta |)(1 + | a + \theta |)}d \theta
		& = \int_{0}^{a} \frac{1}{(1 +  a - \theta )(1 +  a + \theta )}d \theta
		\\
		& = \frac{1}{2(1 + a)} \int_{0}^{a} \left[ \frac{1}{1 + a - \theta} +
		\frac{1}{1 + a + \theta} \right]d \theta
		\\
		& = \frac{1}{2(1 + a)} \log \left( \frac{1 + a + \theta}{1 + a - \theta}
		\right) \Big |_{0}^{a}
		\\
		& = \frac{\log\left( 1 + 2a \right)}{2\left( 1 + a \right)}
		\\
		& \lesssim \frac{\log(2 + | a |)}{1 + | a |}.
	\end{split}
\end{equation*}
%
%
This completes the proof for the case $a \neq 0$. Lastly, for the case
$a =0$, we use dominated convergence and our preceding work to
conclude that
%
%
\begin{equation*}
	\begin{split}
		\int_{\rr} \frac{1}{(1 + | \theta|)^2} d \theta
		& = \lim_{a \to 0}
		\int_{\rr} \frac{1}{(1 + | a - \theta |)(1 + | a + \theta |)}d \theta
		\\
		& \lesssim \lim_{a \to 0} \frac{\log(2 + | a |)}{1 + | a |}
		\\
		& =  \log 2
		\\
		& = \frac{\log(2 + | 0 |)}{1 + | 0 |}. \qquad \qed
	\end{split}
\end{equation*}


%\nocite{*}
\bibliography{/Users/davidkarapetyan/math/bib-files/references.bib}
%
%
\end{document}
