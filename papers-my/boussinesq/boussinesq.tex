%
\documentclass[12pt,reqno]{amsart}
\usepackage{amssymb}
\usepackage{amsmath} 
\usepackage{cancel}  %for cancelling terms explicity on pdf
\usepackage{yhmath}   %makes fourier transform look nicer, among other things
\usepackage{framed}  %for framing remarks, theorems, etc.
\usepackage[shortalphabetic, initials, msc-links]{amsrefs} %for the bibliography; uses cite pkg
\usepackage{enumerate} %to change enumerate symbols
%\usepackage{showkeys}  %shows source equation labels on the pdf
\usepackage[margin=3cm]{geometry}  %page layout
%\usepackage[pdftex]{graphicx} %for importing pictures into latex--pdf compilation
%\setcounter{secnumdepth}{1} %number only sections, not subsections
\hypersetup{colorlinks=true,
linkcolor=blue,
citecolor=blue,
urlcolor=blue,
}
\synctex=1
\numberwithin{equation}{section}  %eliminate need for keeping track of counters
\numberwithin{figure}{section}
\setlength{\parindent}{0in} %no indentation of paragraphs after section title
\renewcommand{\baselinestretch}{1.1} %increases vert spacing of text
%
%
\newcommand{\ds}{\displaystyle}
\newcommand{\ts}{\textstyle}
\newcommand{\nin}{\noindent}
\newcommand{\rr}{\mathbb{R}}
\newcommand{\nn}{\mathbb{N}}
\newcommand{\zz}{\mathbb{Z}}
\newcommand{\cc}{\mathbb{C}}
\newcommand{\ci}{\mathbb{T}}
\newcommand{\zzdot}{\dot{\zz}}
\newcommand{\wh}{\widehat}
\newcommand{\p}{\partial}
\newcommand{\ee}{\varepsilon}
\newcommand{\vp}{\varphi}
\newcommand{\wt}{\widetilde}
%
%
\theoremstyle{plain}  
\newtheorem{theorem}{Theorem}
\newtheorem{proposition}{Proposition}
\newtheorem{lemma}{Lemma}
\newtheorem{corollary}{Corollary}
\newtheorem{claim}{Claim}
\newtheorem{conjecture}[subsection]{conjecture}
%
\theoremstyle{definition}
\newtheorem{definition}{Definition}
%
\theoremstyle{remark}
\newtheorem{remark}{Remark}
%
%
%\newtheorem{theorem}{Theorem}[section]
%\newtheorem{lemma}[theorem]{Lemma}
%\newtheorem{corollary}[theorem]{Corollary}
%\newtheorem{claim}[theorem]{Claim}
%\newtheorem{prop}[theorem]{Proposition}
%\newtheorem{proposition}[theorem]{Proposition}
%\newtheorem{no}[theorem]{Notation}
%\newtheorem{definition}[theorem]{Definition}
%\newtheorem{remark}[theorem]{Remark}
%\newtheorem{examp}{Example}[section]
%\newtheorem {exercise}[theorem] {Exercise}
%
\def\makeautorefname#1#2{\expandafter\def\csname#1autorefname\endcsname{#2}}
\makeautorefname{equation}{Equation}
\makeautorefname{footnote}{footnote}
\makeautorefname{item}{item}
\makeautorefname{figure}{Figure}
\makeautorefname{table}{Table}
\makeautorefname{part}{Part}
\makeautorefname{appendix}{Appendix}
\makeautorefname{chapter}{Chapter}
\makeautorefname{section}{Section}
\makeautorefname{subsection}{Section}
\makeautorefname{subsubsection}{Section}
\makeautorefname{paragraph}{Paragraph}
\makeautorefname{subparagraph}{Paragraph}
\makeautorefname{theorem}{Theorem}
\makeautorefname{theo}{Theorem}
\makeautorefname{thm}{Theorem}
\makeautorefname{addendum}{Addendum}
\makeautorefname{add}{Addendum}
\makeautorefname{maintheorem}{Main theorem}
\makeautorefname{corollary}{Corollary}
\makeautorefname{lemma}{Lemma}
\makeautorefname{sublemma}{Sublemma}
\makeautorefname{proposition}{Proposition}
\makeautorefname{property}{Property}
\makeautorefname{scholium}{Scholium}
\makeautorefname{step}{Step}
\makeautorefname{conjecture}{Conjecture}
\makeautorefname{question}{Question}
\makeautorefname{definition}{Definition}
\makeautorefname{notation}{Notation}
\makeautorefname{remark}{Remark}
\makeautorefname{remarks}{Remarks}
\makeautorefname{example}{Example}
\makeautorefname{algorithm}{Algorithm}
\makeautorefname{axiom}{Axiom}
\makeautorefname{case}{Case}
\makeautorefname{claim}{Claim}
\makeautorefname{assumption}{Assumption}
\makeautorefname{conclusion}{Conclusion}
\makeautorefname{condition}{Condition}
\makeautorefname{construction}{Construction}
\makeautorefname{criterion}{Criterion}
\makeautorefname{exercise}{Exercise}
\makeautorefname{problem}{Problem}
\makeautorefname{solution}{Solution}
\makeautorefname{summary}{Summary}
\makeautorefname{operation}{Operation}
\makeautorefname{observation}{Observation}
\makeautorefname{convention}{Convention}
\makeautorefname{warning}{Warning}
\makeautorefname{note}{Note}
\makeautorefname{fact}{Fact}
%
%


\newcommand{\uol}{u^\omega_\lambda}
\newcommand{\lbar}{\bar{l}}
\renewcommand{\l}{\lambda}
\newcommand{\R}{\mathbb R}
\newcommand{\RR}{\mathcal R}
\newcommand{\al}{\alpha}
\newcommand{\ve}{q}
\newcommand{\tg}{{tan}}
\newcommand{\m}{q}
\newcommand{\N}{N}
\newcommand{\ta}{{\tilde{a}}}
\newcommand{\tb}{{\tilde{b}}}
\newcommand{\tc}{{\tilde{c}}}
\newcommand{\tS}{{\tilde S}}
\newcommand{\tP}{{\tilde P}}
\newcommand{\tu}{{\tilde{u}}}
\newcommand{\tw}{{\tilde{w}}}
\newcommand{\tA}{{\tilde{A}}}
\newcommand{\tX}{{\tilde{X}}}
\newcommand{\tphi}{{\tilde{\phi}}}


\begin{document}
\title{$1$-D ``good'' Boussinesq equation}

\author{Dan-Andrei Geba, Alexandrou Himonas, and David Karapetyan}

\address{Department of Mathematics, University of Rochester, Rochester, NY 14627}
\address{Department of Mathematics, University of Notre Dame, Notre Dame, IN 46556}
\address{Department of Mathematics, University of Notre Dame, Notre Dame, IN 46556}
\date{}

%\begin{abstract}
%\end{abstract}

\subjclass[2000]{35B30, 35Q55, 35Q72}
\keywords{local well-posedness; ill-posedness.}

\maketitle
%
\section{Introduction}
Our object of investigation is the initial value problem for the
periodic/non-periodic $1$-D ``good'' Boussinesq equation, i.e.,
\begin{equation}
  \aligned
  & u_{tt}-u_{xx}+u_{xxxx}+ (u^2)_{xx}\,=\,0, \quad x\in \mathbb{T}\ \text{or} \ \mathbb{R}, \quad t>0,\\
&u(0,x)\,=\,u_0(x),\qquad u_t(0,x)\,=\,u_1(x).\endaligned
\label{main}
\end{equation}

Due to the fact that the leading terms in the linear operator above are $u_{tt}$ and $u_{xxxx}$, morally speaking, one derivative in time is like two derivatives in space. This is why the Sobolev regularity scale for the initial data should be as follows:
\[
u_0\in H^s(\mathbb{T}\ \text{or} \ \mathbb{R}), \qquad u_1\in H^{s-2}(\mathbb{T}\ \text{or} \ \mathbb{R})
\]
Results for these problems are usually formulated with $u_0=\phi \in H^s$ and $u_1=\psi_x$, $\psi\in H^{s-1}$.

Current state of the art in terms of local well-posedness/ill-posedness for the two problems is:
\begin{itemize}
  \item LWP for both problems when $s>-\frac 14$ (Farah '09, Farah-Scialom '10), with iteration done in
    the norm
    \[
    \|F\|_{X^{s,b}}\,=\,\|<|\tau|-\sqrt{\xi^2+\xi^4}>^b\,<\xi>^s \tilde{F}\|_{L^2_{\tau,\xi}};
    \]

  \item main IP result is for the non-periodic problem when $s<-2$, as the solution map 
    \[
    S: H^s\times H^{s-1} \to C([0,T]; H^s), \quad
    S(\phi,\psi)\,=\,u
    \]
    is not $C^2$ at zero (Farah '09);

  \item also, for the non-periodic problem, one can not find a space in which to run a contraction argument based on treating the nonlinearity as bilinear for $s<-2$ (see Theorem 1.4 in Farah '09);

  \item finally, and really puzzling\footnote{for other dispersive equations (e.g., KdV, Schrodinger), there is usually a gap of $\frac 14$ between regularities for the two problems}, the crucial bilinear estimate (equation (5) in both papers) fails basically at the same threshold for both problems: $s\leq -\frac 14$ (non-periodic), $s<-\frac{1}{4}$ (periodic).
\end{itemize}

The equation does not have an associated scaling, however one can do a formal scaling analysis by ignoring one of the two linear terms containing spatial derivatives:
\begin{itemize}
  \item for 
    \[
    u_{tt}+u_{xxxx}+(u^2)_{xx}\,=\,0,
    \]
    one has 
    \[
    u_{\lambda}(t,x)\,=\,\frac{1}{\lambda^2}u\left(\frac{t}{\lambda^2}, \frac{x}{\lambda}\right),
    \]
    which leads to $s_c=-\frac 32$;
\begin{framed}
\begin{remark}
We now prove this.
Let $u(x, t)$ be a solution to the $B_4$ equation, that is
%
$$
B_4(u)=
 \partial_t^2u + \partial^4_x u + \partial_x^2(u^2)  = 0
$$
%
We would like to find the constants
$a, b, c$ such that
\[
u_\lambda (x, t) = \lambda^a u(\lambda^b x, \lambda^c t)
\]
is also a solution to $B_4$.  Since 
$$
B_4(u_\lambda)=
\lambda^{a+2c} \partial_t^2u 
+
 \lambda^{a+4b} \partial^4_x u 
 +
  \lambda^{2a+2b}
  \partial_x^2(u^2),  
$$
we see that $u_\lambda$ is a $B_4$ solution only if
$$
a+2c=a+4b=2a+2b,
$$
or
$
c= 2b =a.
$
  Thus
\[
u_\lambda (x, t) = \lambda^{2b} u(\lambda^{b}x,  \lambda^{2b} t).
\]
%
%
Therefore, replacing  $ \lambda^b$ with  $ \lambda$ gives the following scaling:
%
\begin{equation}
\label{DP-scal}
\boxed{
u(x, t) \text{ solution to }  B_4
 \Longrightarrow 
u_\lambda (x, t) = \lambda^2 u(\lambda x, \lambda^2 t)  \text { is also a
solution to }  B_4. 
}
\end{equation}
\label{rem:scaling}
To find the critical Sobolev index, we compute
%
%
\begin{equation}
\begin{split}
  \| u_{\lambda} \|_{\dot{H}^s(\ci)} 
  & = \lambda^{2} \| u(\lambda x, \lambda^2 t) \|_{\dot{H}^{s}(\ci)}
  \\
  & = \lambda^{2} \left( \int_{\rr} | \xi |^{2s} | \wh{u (\lambda x,
  \lambda^{2} t)}^x (\xi, t)| \right)^{1/2}.
\end{split}
\label{crit-ind-comp}
\end{equation}
%
But
%
%
\begin{equation*}
\begin{split}
  \wh{u(\lambda x, \lambda^{2}t)^x}(\xi, t)
  & = \int_{\rr}e^{-i\xi x}u(\lambda x, \lambda^2 t) dx
  \\
  & = \frac{1}{\lambda} \int_{\rr}e^{-i \frac{n}{\lambda} x'}u(x',
  \lambda^{2} t) dx'
  \\
  & = \frac{1}{\lambda} \wh{u(\cdot, \lambda^{2}t)}(\frac{\xi}{\lambda})
\end{split}
\end{equation*}
%
%
Substituting back into \eqref{crit-ind-comp}, we obtain
%
%
\begin{equation*}
\begin{split}
  \| u_{\lambda} \|_{\dot{H}^s(\rr)} 
  & = \lambda^{2} \left( \int_{\rr} | \xi |^{2s} |
  \frac{1}{\lambda}\wh{u(\cdot, \lambda^{2}t)}(\frac{\xi}{\lambda}) |^2 d \xi
  \right)^{1/2}
  \\
  & = \lambda \left( \int_{\rr}| \xi |^{2s} | \wh{u(\cdot,
  \lambda^{2}t)}(\frac{\xi}{\lambda}) |^2 d \xi  \right)^{1/2}
  \\
  & = \lambda \left( \int_{\rr} | \lambda \xi' |^{2s} 
  \wh{u(\cdot, \lambda^{2}t)}(\xi') |^2 \lambda d \xi
  \right)^{1/2}
  \\
  & = \lambda^{s + 3/2} \|u(\cdot, \lambda^{2}t) \|_{\dot{H}^s (\ci)}.
\end{split}
\end{equation*}
%
%
Therefore, $\| u_{\lambda(0)} \|_{\dot{H}^s(\rr)} = \lambda^{s + 3/2} \|
u_{0} \|_{\dot{H}^{s}(\rr)}$, and so $s=-3/2$ is the critical Sobolev index.
\end{remark}
\end{framed}
\nin
Since the scaling conserves data in $\dot{H}^{-3/2}$\ldots
It seems that this equation is ``like KdV''.
So one may expect KdV type theorems\ldots
That is, $s_c=-3/4$ on the line and $s_c=-1/2$ on the circle,
if one uses bilinear estimates.
But, Kappeler and collaborators went all the way to $-1$ for KdV.
However KdV is integrable. Is this equation integrable?
Also, people conjecture that the critical index for KdV well-posedness 
in some appropriate sense should be the scaling index which is  $-3/2$.
  \item for 
    \[
    u_{tt}-u_{xx}+(u^2)_{xx}\,=\,0,
    \]
    one has 
    \[
    u_{\lambda}(t,x)\,=\,u\left(\frac{t}{\lambda}, \frac{x}{\lambda}\right),
    \]
    which leads to $s_c=\frac 12$.
\begin{framed}
\begin{remark}
We now prove this. 
Let $u(x, t)$ be a solution to the $B_2$ equation, that is
%
$$
B_2(u)=
 \partial_t^2u - \partial^2_x u + \partial_x^2(u^2)  = 0
$$
%
We would like to find the constants
$a, b, c$ such that
\[
u_\lambda (x, t) = \lambda^a u(\lambda^b x, \lambda^c t)
\]
is also a solution to $B_2$.  Since 
$$
B_2(u_\lambda)=
\lambda^{a+2c} \partial_t^2u 
-
 \lambda^{a+2b} \partial^2_x u 
 +
  \lambda^{2a+2b}
  \partial_x^2(u^2),  
$$
we see that $u_\lambda$ is a $B_2$ solution only if
$$
a+2c=a+2b=2a+2b,
$$
or
$
a=0, b=c.
$
  Thus
\[
u_\lambda (x, t) = u(\lambda^{b}x,  \lambda^{b} t).
\]
%
%
Therefore, replacing  $ \lambda^b$ with  $ \lambda$ gives the following scaling:
%
\begin{equation}
\label{B2-scal}
\boxed{
u(x, t) \text{ solution to }  B_2
 \Longrightarrow 
u_\lambda (x, t) = u(\lambda x, \lambda t)  \text { is also a
solution to }  B_2. 
}
\end{equation}
\label{rem:scaling-B2}
To find the critical Sobolev index, we compute
\\
%
%
\begin{equation}
\begin{split}
  \| u_{\lambda} \|_{\dot{H}^s(\ci)} 
  & =  \| u(\lambda x, \lambda t) \|_{\dot{H}^{s}(\ci)}
  \\
  & = \left( \int_{\rr} | \xi |^{2s} | \wh{u(\lambda x,
  \lambda t)}^x (\xi, t)| \right)^{1/2}.
\end{split}
\label{crit-ind-comp-B2}
\end{equation}
%
But
%
\\
%
\begin{equation*}
\begin{split}
  \wh{u(\lambda x, \lambda t)^x}(\xi, t)
  & = \int_{\rr}e^{-i\xi x}u(\lambda x, \lambda t) dx
  \\
  & = \frac{1}{\lambda} \int_{\rr}e^{-i \frac{n}{\lambda} x'}u(x',
  \lambda t) dx'
  \\
  & = \frac{1}{\lambda} \wh{u(\cdot, \lambda t)}(\frac{\xi}{\lambda})
\end{split}
\end{equation*}
%
%
Substituting back into \eqref{crit-ind-comp-B2}, we obtain
%
%
\begin{equation*}
\begin{split}
  \| u_{\lambda} \|_{\dot{H}^s(\rr)} 
  & = \left( \int_{\rr} | \xi |^{2s} |
  \frac{1}{\lambda}\wh{u(\cdot, \lambda t)}(\frac{\xi}{\lambda}) |^2 d \xi
  \right)^{1/2}
  \\
  & = \frac{1}{\lambda} \left( \int_{\rr}| \xi |^{2s} | \wh{u(\cdot,
  \lambda t)}(\frac{\xi}{\lambda}) |^2 d \xi  \right)^{1/2}
  \\
  & = \frac{1}{\lambda} \left( \int_{\rr} | \lambda \xi' |^{2s} 
  \wh{u(\cdot, \lambda)}(\xi') |^2 \lambda d \xi
  \right)^{1/2}
  \\
  & = \lambda^{s - 1/2} \|u(\cdot, t) \|_{\dot{H}^s (\rr)}.
\end{split}
\end{equation*}
%
%
Therefore, $\| u_{\lambda(0)} \|_{\dot{H}^s(\rr)} = \lambda^{s - 1/2} \|
u_{0} \|_{\dot{H}^{s}(\rr)}$, and so $s=1/2$ is the critical Sobolev index.
\end{remark}
\end{framed}


\end{itemize}

This might suggest that the current results are not optimal.


%

%
%%%%%%%%%%%%%%%%%%%%%%%%%%%%%%%%%%%%%%%%%%%%%%%%%%%%%
%
%
%             Fourth order Modified Boussinesq  equation
%
%
%%%%%%%%%%%%%%%%%%%%%%%%%%%%%%%%%%%%%%%%%%%%%%%%%%%%%


%
\section{Fourth order Modified Boussinesq  equation}
%
We consider the initial value problem (ivp) for the fourth order modified Boussinesq
($B_4$) equation 
\begin{gather}
  u_{tt}   + u_{xxxx} + (u^2)_{xx} = 0,
  \label{eqn:mb-2}
  \\
  u(x,0) = u_{0}(x), \quad \p_t u(x, 0) = u_1(x), 
  \label{eqn:mb-init-data-2}
  \\
  \notag
  (u_0, u_1) \in
  H^{s}(\ci) \times
  H^{s-1}(\ci)  
\end{gather}
and prove the following.
%
%
%%%%%%%%%%%%%%%%%%%%%%%%%%%%%%%%%%%%%%%%%%%%%%%%%%%%%
%
%
%                Main Theorem
%
%
%%%%%%%%%%%%%%%%%%%%%%%%%%%%%%%%%%%%%%%%%%%%%%%%%%%%%
%
%
\begin{theorem}
  The $B_{4}$ ivp is well-posed
  \begin{enumerate}[(i)]
    \item In $H^s(\rr)$ if $s > -1/4$
    \item In $H^{s}(\ci)$ if $s > -1/4$,
  \end{enumerate}
  and the data-to-solution map is locally Lipschitz. Furthermore, these results
  are optimal in the sense that uniform continuity of the flow map fails for $s
  \le 1/4$ in the periodic and non-periodic cases.
\label{thm:wp-2}
\end{theorem}
%

%
%%%%%%%%%%%%%%%%%%%%%%%%%%%%
%
%
%           Scaling for B4
%
%
%%%%%%%%%%%%%%%%%%%%%%%%%%%
%
%
%



%
%
%%%%%%%%%%%%%%%%%%%%%%%%%%%%%%%%%%%%%%%%%%%%%%%%%%%%%
%
%
%                The Periodic Case
%
%
%%%%%%%%%%%%%%%%%%%%%%%%%%%%%%%%%%%%%%%%%%%%%%%%%%%%%
%
%
\subsection{The Periodic Case} 
\label{ssec:periodic-case}
We will first rewrite the $B_4$ ivp
\eqref{eqn:mb-2}-\eqref{eqn:mb-init-data-2} in integral form. 
Consider
the linear $B_4$ ivp
\begin{gather}
  u_{tt} + u_{xxxx} = 0,
  \label{lin-mb}
  \\
  u(x, 0)=u_{0}(x), \quad u_{t}(x,0) = u_{1}(x).
  \label{lin-mb-init-data-1}
\end{gather}
Taking the spatial Fourier transform of the linear $B_{4}$ ivp
\eqref{lin-mb}-\eqref{lin-mb-init-data-1} yields
%
%
\begin{gather}
  \wh{u_{tt}} + n^{4} \wh{u} = 0
  \label{four-trans-lin-mb}
  \\
  \wh{u}(n, 0) = \wh{u_{0}}(n), \quad \wh{u_{t}}(n, 0) = \wh{u_{1}}(n)
  \label{four-trans-lin-mb-data}
\end{gather}
Substituting the ansatz $e^{\lambda t}$ into \eqref{four-trans-lin-mb}, we
obtain the characteristic equation
%
%
\begin{equation*}
\begin{split}
  \lambda^{2} + n^{4} = 0
\end{split}
\end{equation*}
%
%
which gives 
%
%
\begin{equation*}
\begin{split}
  \lambda = \pm in^{2}.
\end{split}
\end{equation*}
%
Since
\eqref{lin-mb} is second order, and $e^{in^{2}t}$ and $e^{-in^{2}t}$ are
linearly independent solutions to \eqref{lin-mb}, it follows that $\left\{
e^{in^{2}t}, e^{-in^{2}t}
\right\}$ is a basis for all solutions of \eqref{lin-mb}. Therefore, the general
solution of \eqref{lin-mb} takes the form
%
%
\begin{equation}
  \label{explicit-homog-soln}
\begin{split}
  \wh{u}(n,t) = c_{1}e^{in^{2}t} + c_{2}e^{-in^{2}t}.
\end{split}
\end{equation}
%
%
which, in conjunction with initial data 
\eqref{four-trans-lin-mb-data}, implies
%
%
\begin{gather*}
   c_{1} + c_{2} = \wh{u_{0}}(n)
  \\
   in^{2}c_{1} - in^{2}c_{2} = \wh{u_{1}}(n).
\end{gather*}
%
%
Solving for $c_{1}$ and $c_{2}$, we obtain
%
%
\begin{gather*}
  c_{1} = \frac{1}{2} \wh{u_{0}}(n) + \frac{1}{2in^{2}}\wh{u_{1}}(n),
  \\
  c_{2} = \frac{1}{2} \wh{u_{0}}(n) - \frac{1}{2in^{2}}\wh{u_{1}}(n).
\end{gather*}
%
%
Substituting into \eqref{explicit-homog-soln}, we obtain the unique solution to
ivp \eqref{four-trans-lin-mb}-\eqref{four-trans-lin-mb-data}
%
%
\begin{equation*}
\begin{split}
  \wh{u}(n, t) = \wh{u_{0}}(n) \frac{e^{in^{2}t} + e^{-in^{2}t}}{2} +
  \wh{u_{1}}(n)\frac{e^{in^{2}t} - e^{-in^{2}}t}{2 i n^{2}}
\end{split}
\end{equation*}
%
or
%
%
\begin{equation*}
\begin{split}
  u(x,t) = R_{t}u_{0} + S_{t}u_{1}
\end{split}
\end{equation*}
%
where $R_{t}$ and $S_{t}$ are linear operators defined via the relation
%
%
\begin{gather}
  \label{sin-cos-op}
  \wh{R_{t}\vp} = \wh{\vp}(n) \frac{e^{in^{2}t} + e^{-in^{2}t}}{2} , \quad 
  \wh{S_{t}\vp} = \wh{\vp}(n) \frac{e^{in^{2}t} - e^{-in^{2}t}}{2i n^{2}}.
\end{gather}
%
Turning our attention now to the $B_{4}$ ivp
\eqref{eqn:mb-2}-\eqref{eqn:mb-init-data-2} and taking the spatial Fourier
transform yields 
%
%
\begin{gather}
  \wh{u_{tt}} + n^{4} \wh{u} = -\wh{(u^{2})_{xx}}
  \label{four-trans-mb}
  \\
  \wh{u}(n, 0) = \wh{u_{0}}(n), \quad \wh{u_{t}}(n, 0) = \wh{u_{1}}(n)
  \label{four-trans-mb-data}
\end{gather}
which for fixed $t$ is a second order ODE in $n$. 
We now need the following.
%
%
%%%%%%%%%%%%%%%%%%%%%%%%%%%%%%%%%%%%%%%%%%%%%%%%%%%%%
%
%
%                General Solution NonHomog 2nd Order Eqn
%
%
%%%%%%%%%%%%%%%%%%%%%%%%%%%%%%%%%%%%%%%%%%%%%%%%%%%%%
%
%
\begin{lemma}[Variation of Parameters]
\label{lem:nonhomog-ode-soln}
The general solution of the $2$nd order nonhomogeneous ODE 
%
%
\begin{equation}
  \label{2nd-order-ode}
\begin{split}
y'' + p(t)y' + q(t)y = g(t)
\end{split}
\end{equation}
%
%
can be written in the form
%
%
\begin{equation*}
\begin{split}
  y = c_{1}y_{1}(t) + c_{2}y_{2}(t) + y_{p}(t),
\end{split}
\end{equation*}
%
%
where $y_{1}$ and $y_{2}$ are linearly independent solutions to the
corresponding homogeneous equation (i.e. $g(t) = 0$), $c_{1}$ and $c_{2}$ are
arbitrary constants, and $y_{p}$ is some specific solution of the nonhomogeneous
equation. Furthermore, one such $y_{p}$ is given by
%
%
\begin{equation}
  \label{2nd-order-ansatz}
\begin{split}
  y_{p} = y_{1}v_{1} + y_{2} v_{2}
\end{split}
\end{equation}
%
%
where $v_{1}$ and $v_{2}$ are solutions to the system
\begin{gather}
  \label{cancel-rel-1}
  y_{1} v_{1}' + y_{2} v_{2}' = 0
  \\
  \label{cancel-rel-2}
  y_{1}' v_{1}' + y_{2}' v_{2}' = g(t).
\end{gather}
\end{lemma}
%
{\bf Proof.} Substituting the ansatz \eqref{2nd-order-ansatz} into
the left hand side of \eqref{2nd-order-ode}, we obtain the expression
%
%
%
%
\begin{equation*}
\begin{split}
  (y_{1}v_{1} + y_{2}v_{2})'' + p(t)(y_{1}v_{1} + y_{2}v_{2})' +
  q(t)(y_{1}v_{1} + y_{2}v_{2}) 
\end{split}
\end{equation*}
%
%
or
%
%
\begin{equation*}
  \begin{split}
    & y_{1}'' v_{1} + 2y_{1}'v_{1}' + y_{1}v_{1}'' + y_{2}''v_{2} + 2y_{2}' v_{2}'
  + y_{2} v_{2}'' + p(t)y_{1}'v_{1} + p(t)y_{1}v_{1}'
  \\
  & + p(t)y_{2}'v_{2} + p(t)y_{2}v_{2}' + q(t)y_{1}v_{1} + q(t)y_{2}v_{2} =g
\end{split}
\end{equation*}
%
Collecting terms, this can be rewritten as
%
%
%
%
\begin{equation*}
\begin{split}
  & (y_{1}v_{1}'' + y_{1}' v_{1}') + (y_{2}v_{2}'' + y_{2}' v_{2}') + y_{1}''
  v_{1} + y_{2}'' v_{2} + (y_{1}'v_{1}' + y_{2}' v_{2}')
  \\
  & + p(t)\left(
  y_{1}'v_{1} + y_{2}' v_{2} + y_{1}v_{1}' + y_{2}v_{2}'
  \right) + q(t)\left( y_{1}v_{1} + y_{2}v_{2} \right)
  \end{split}
\end{equation*}
%
or
%
%
%
%
\begin{equation*}
\begin{split}
  & \cancel{(y_{1}v_{1}' + y_{2}v_{2}')'} + y_{1}''
  v_{1} + y_{2}'' v_{2} + \overbrace{(y_{1}'v_{1}' + y_{2}' v_{2}')}^{g}
  \\
  & + p(t)\left(
  y_{1}'v_{1} + y_{2}' v_{2} + y_{1}v_{1}' + y_{2}v_{2}'
  \right) + q(t)\left( y_{1}v_{1} + y_{2}v_{2} \right)
  \end{split}
\end{equation*}
%
%
or
%
%
\begin{equation*}
\begin{split}
  g + \cancel{v_{1}\left[ y_{1}'' + p(t)y_{1}' + q(t)y_{1} \right]} +
  \cancel{v_{2}\left[ y_{2}'' + p(t)y_{2}' + q(t)y_{2}
  \right]}
\end{split}
\end{equation*}
%
%
where the last cancellation is due to the fact that $y_{1}$ and $y_{2}$ are
solutions to the corresponding homogeneous equation of \eqref{2nd-order-ode}
(i.e. $g(t) = 0$). This concludes the proof. \qquad \qedsymbol
%
%
Rewriting the system \eqref{cancel-rel-1}-\eqref{cancel-rel-2}
as
  \begin{equation*}
  \begin{bmatrix}
    y_{1} & y_{2} \\
    y_{1}' & y_{2}'
  \end{bmatrix}
  \begin{bmatrix}
    v_{1}'
    \\
    v_{2}'
  \end{bmatrix}=
  \begin{bmatrix}
  0 \\
  g
  \end{bmatrix}
\end{equation*}
  we obtain 
  \begin{equation*}
\begin{bmatrix}
  v_{1}'
  \\
  v_{2}'
\end{bmatrix}=
\begin{bmatrix}
  -\frac{y_{2}g}{y_{1}y_{2}' - y_{1}' y_{2}} \\
  \frac{y_{1}g}{y_{1}y_{2}' - y_{1}' y_{2}}.
\end{bmatrix}
\end{equation*}
Integrating from $0$ to $t$, and setting $v_{1}(0) = v_{2}(0) = 0$, we see that
one particular solution is
%
%
\begin{equation*}
\begin{split}
\begin{bmatrix}
  v_{1}
  \\
  v_{2}
\end{bmatrix}=
\begin{bmatrix}
 -\int_{0}^{t} \frac{y_{2}g}{y_{1}y_{2}' - y_{1}' y_{2}} dt' \\
  \int_{0}^{t}\frac{y_{1}g}{y_{1}y_{2}' - y_{1}' y_{2}}dt'.
\end{bmatrix}
\end{split}
\end{equation*}
Therefore,
%
%
\begin{equation*}
\begin{split}
  y_{p} =  \int_{0}^{t}
  \frac{ y_{2}(t)y_{1}(t') - y_{1}(t)y_{2}(t')}{y_{1}(t')y_{2}'(t') -
  y_{1}'(t') y_{2}(t')}g \ dt'.
\end{split}
\end{equation*}
%
%
%
%
%
Applying \autoref{lem:nonhomog-ode-soln}, it follows that 
the unique solution to ivp
\eqref{four-trans-mb}-\eqref{four-trans-mb-data} is given by
%
%
\begin{equation*}
\begin{split}
\wh{u}(n, t) = \wh{u_{0}}(n) \frac{e^{in^{2}t} + e^{-in^{2}t}}{2} +
  \wh{u_{1}}(n)\frac{e^{in^{2}t} - e^{-in^{2}}t}{2 i n^{2}} +
  \int_{0}^{t}\frac{e^{in^{2}(t-t')}-e^{-in^{2}(t-t')}}{2in^{2}}
  \wh{(u^{2})_{xx}} dt'.
\end{split}
\end{equation*}
%
%
Taking the inverse Fourier transform then gives
%
\begin{equation}
  \begin{split}
    u(x,t) = R_{t}u_{0} + S_{t}u_{1} + \int_{0}^{t} S_{t-t'}
    (u^{2})_{xx} dt'.
  \end{split}
  \label{eqn:integral-form}
\end{equation}
%
%
\begin{framed}
  \begin{remark}
For an alternative technique for formulating the $B_{4}$ ivp in integral form,
see the appendix.
\end{remark}
\end{framed}
Hence, we have rewritten the $B_{4}$ ivp
\eqref{eqn:mb-2}-\eqref{eqn:mb-init-data-2} in integral form, which we will now
localize in time. 
Let $\psi(t)$ be a cutoff function symmetric about the 
origin such that $\psi(t) = 1$ for $|t| \le T$ and $\text{supp} \, \psi 
= [-2T, 2T ]$. Multiplying the right hand side of expression
$\eqref{eqn:integral-form}$ by $\psi(t)$, we obtain
%
%
\begin{equation}
  \begin{split}
    \psi(t) u(x,t)
    & = \psi(t) R_{t} u_{0} + \psi(t) S_{t}u_{1} +
    \psi(t) \int_{0}^{t} S_{t-t'}
    (u^{2})_{xx} dt'
    \\
    & \doteq Tu
  \end{split}
  \label{localized-int-eqn}
\end{equation}
where $T=T_{u_0, u_1}$.We now introduce the following spaces. 
%
%
\begin{definition}
  Let $\mathcal{Y}$ be the space of functions $F(\cdot)$ such that
  \begin{enumerate}[(i)]
   \item{$F: \ci \times \rr \to \cc$ }.
   \item{ $F(x, \cdot) \in S(\rr)$ for each $x \in \ci$}.
   \item{ $F(\cdot, t) \in C^{\infty}(\ci)$for each $t \in \rr$}.
  \end{enumerate}
  For $s, b \in \rr$, $X_{s,b}$ denotes the completion of $\mathcal{Y}$ with
  respect to the norm
  %
  %
  \begin{equation}
  \begin{split}
    \|F\|_{X_{s,b}} = \left( \sum_{n \in \zz} (1 + |n|)^{2s} \int_{\rr}
    (1 + | | \tau | - n^{2} |)^{2b} \wh{F}(n, \tau) d \tau\right)^{1/2}.
  \end{split}
  \label{eqn:bous-norm}
  \end{equation}
  %
  %
  %
  %
\end{definition}
%
The $X_{s,b}$ spaces have the following important embedding, whose proof is
provided in the appendix.
%
%
%%%%%%%%%%%%%%%%%%%%%%%%%%%%%%%%%%%%%%%%%%%%%%%%%%%%%
%
%
%               Embedding 
%
%
%%%%%%%%%%%%%%%%%%%%%%%%%%%%%%%%%%%%%%%%%%%%%%%%%%%%%
%
%
\begin{lemma}[Lemma 2.3 in \cite{Farah:2009uq}]
  Let $b > 1/2$. Then $X_{s, b} \subset C(\rr, H^s)$ continuously. That is, there exists $c>0$ depending only on $b$ such that
%
%
\begin{equation*}
\begin{split}
  \| u \|_{C(\rr, H^s) } \le c \| u \|_{X_{s,b}}.
\end{split}
\end{equation*}
%
\label{lem:embedding}
\end{lemma}
%
%
We will 
show that for initial data $\vp \in {H}^s(\ci)$, $T$ is a contraction on $B_M 
\subset {X}_{s,b}$, where $B_M$ is the ball centered at the origin of radius $M = 
M_{\vp}> 0$, by estimating the $X_{s,b}$
norm of \eqref{localized-int-eqn}. The Picard fixed point theorem will
then yield a unique solution to
\eqref{localized-int-eqn}. An application of \autoref{lem:embedding}
will then imply the existence of a unique, local
solution $u \in C([-T, T], H^s(\ci))$ to the $B_4$ ivp. Local Lipschitz continuity of the flow map will follow from estimates used to establish the contraction mapping. %
%
%
%
%
%
%
%
%
%
\subsubsection{Estimate for $\psi(t) R_{t}u_{0}$.} 
\label{sssec:est-init-term-1}
We have
%
%
\begin{equation*}
  \begin{split}
    \wh{\psi(t)R_{t}u_{0}}^{x}(n, t)
    & = \psi(t) \wh{u_{0}}(n) \frac{e^{in^2 t} + e^{-in^{2}t}}{2}
    \\
    & = \frac{\psi(t) \wh{u_{0}}(n)e^{in^{2}t}}{2} + \frac{\psi(t)
    \wh{u_{0}}(n)e^{-in^{2}t}}{2}  
  \end{split}
\end{equation*}
%
%
and
%
%
\begin{equation*}
  \begin{split}
    \wh{\psi(t) R_{t}u_{0}}(n, \tau) = \frac{\wh{\psi}(\tau -
    n^{2})\wh{u_{0}}(n)}{2} + \frac{\wh{\psi}(\tau + n^{2})\wh{u_{0}}(n)}{2}.
  \end{split}
\end{equation*}
%
%
Hence, substituting and applying the inequality 
%
%
\begin{equation}
  \label{square-ineq}
\begin{split}
(a + b)^{2} \le 4(a^{2} +
b^{2}),\ a, b \in \rr,
\end{split}
\end{equation}
%
%
we have

%
%
\begin{align}
    & \| \psi(t) R_{t}u_{0} \|_{X_{s,b}}^{2} 
    \notag
    \\
    & = \sum_{n \in \zz}(1 + |n|)^{2s} \int_{\rr}\left( 1 + | | \tau
    |-n^{2} | \right)^{2b} | \frac{\wh{\psi}(\tau - n^{2})\wh{u_{0}(n)}}{2} +
    \frac{\wh{\psi}(\tau + n^{2})\wh{u_{0}}(n)}{2} |^{2} d \tau
    \notag
    \\
    & \le \sum_{n \in \zz} \left( 1 + |n| \right)^{2s} | \wh{u_{0}}(n)
    |^{2} \int_{\rr} | \wh{\psi}(\tau - n^{2}) |^{2}\left( 1 + | | \tau | -
    n^{2} | \right)^{2b} d \tau
    \label{u-0-norm-comp-1}
    \\
    & + \sum_{n \in \zz} \left( 1 + |n| \right)^{2s} | \wh{u_{0}}(n)
    |^{2} \int_{\rr} | \wh{\psi}(\tau + n^{2}) |^{2}\left( 1 + | | \tau | -
    n^{2} | \right)^{2b} d \tau.
    \label{u-0-norm-comp-3}
\end{align}
%
Noting that
\begin{equation}
  \begin{split}
    | | \tau | - n^{2} | \le \min\left\{ | \tau - n^{2} |, | \tau + n^{2} | \right\}
  \end{split}
  \label{eqn:norm-key-ineq}
\end{equation}
%
%
we bound \eqref{u-0-norm-comp-1} by
%
%
\begin{equation*}
  \begin{split}
    & \sum_{n \in \zz} \left( 1 + |n| \right)^{2s} | \wh{u_{0}}(n)
    |^{2} \int_{\rr} | \wh{\psi}(\tau - n^{2}) |^{2}\left( 1 +  | \tau  -
    n^{2} | \right)^{2b} d \tau
    \\
    & = \sum_{n \in \zz} \left( 1 + |n| \right)^{2s} | \wh{u_{0}}(n)
    |^{2} \int_{\rr} | \wh{\psi}(\tau') |^{2}\left( 1 +  | \tau'| \right)^{2b} d \tau
    \\
    & = c_{\psi, b} \sum_{n \in \zz} \left( 1 + |n| \right)^{2s} | \wh{u_{0}}(n)
    |^{2} 
    \\
    & = c_{\psi, b} \| u_{0} \|_{H^{s}}^{2}
  \end{split}
\end{equation*}
%
%
where $c_{\psi, b}$ is a constant depending only on $\psi$ and $b$. The
term \eqref{u-0-norm-comp-3} is bounded in similar fashion. Therefore, 
$\|\psi(t) R_{t} u_{0}\|_{X_{s,b}}^{2} \le c_{\psi, b}
\|u_{0}\|_{H^s}^2$ and
taking square roots of both sides gives
%
%
\begin{equation}
  \begin{split}
    \|\psi(t) R_{t} u_{0}\|_{X_{s,b}} \le c_{\psi, b}
    \|u_{0}\|_{H^s}.
  \end{split}
  \label{eqn:u-0-fin-est}
\end{equation}
%
%

\subsubsection{Estimate for $\psi(t) S_{t}u_{1}$.}
\label{sssec:estimate-init-term-2}
We have
%
%
\begin{equation*}
  \begin{split}
    \wh{\psi(t)S_{t}u_{1}}^{x}(n, t)
    & = \psi(t) \wh{u_{1}}(n) \frac{e^{in^2 t} - e^{-in^{2}t}}{2i n^{2}}
    \\
    & = \frac{\psi(t) \wh{u_{1}}(n)e^{in^{2}t}}{2i n^{2}} - \frac{\psi(t)
    \wh{u_{1}}(n)e^{-in^{2}t}}{2i n^{2}}  
  \end{split}
\end{equation*}
%
%
and
%
%
\begin{equation*}
  \begin{split}
    \wh{\psi(t) S_{t}u_{1}}(n, \tau) = \frac{\wh{\psi}(\tau -
    n^{2})\wh{u_{1}}(n)}{2i n^{2}} + \frac{\wh{\psi}(\tau + n^{2})\wh{u_{1}}(n)}{2i
    n^{2}}.
  \end{split}
\end{equation*}
%
Note that 
%
\begin{equation*}
  \begin{split}
    \wh{\psi(t)S_{t}u_{1}}^{x}(0, t)
    & = \psi(t) \wh{u_{1}}(0) t
      \end{split}
\end{equation*}
and so 
%
%
\begin{equation*}
  \begin{split}
    \wh{\psi(t) S_{t}u_{1}}(0, \tau) = -\frac{d}{d \tau} \wh{\psi}(\tau)
    \wh{u_{1}}(0).
  \end{split}
\end{equation*}
%
Hence, substituting and applying \eqref{square-ineq}, we have
%
%
\begin{equation}
  \begin{split}
    \| \psi(t) S_{t}u_{1} \|_{X_{s,b}}^{2} 
    & = \sum_{n \in \zzdot}(1 + |n|)^{2s} \int_{\rr}\left( 1 + | | \tau
    |-n^{2} | \right)^{2b} | \frac{\wh{\psi}(\tau - n^{2})\wh{u_{1}(n)}}{2i
    n^{2}} -
    \frac{\wh{\psi}(\tau + n^{2})\wh{u_{1}}(n)}{2i n^{2}} |^{2} d \tau
    \\
    & + |\wh{u_{1}}(0)|^{2} \int_{\rr} (1 + | \tau |)^{2b} | - \frac{d }{d \tau}
    \wh{\psi}(\tau)|^{2} d \tau
    \\
    & \le \sum_{n \in \dot{\zz}} n^{-2s} \left( 1 + |n| \right)^{2s} | \wh{u_{1}}(n)
    |^{2} \int_{\rr} | \wh{\psi}(\tau - n^{2}) |^{2}\left( 1 + | | \tau | -
    n^{2} | \right)^{2b} d \tau
    \\
    & + \sum_{n \in \dot{\zz}} n^{-2s} \left( 1 + |n| \right)^{2s} | \wh{u_{1}}(n)
    |^{2} \int_{\rr} | \wh{\psi}(\tau + n^{2}) |^{2}\left( 1 + | | \tau | -
    n^{2} | \right)^{2b} d \tau
    \\
    & + |\wh{u_{1}}(0)|^{2} \int_{\rr} (1 + | \tau |)^{2b} |\frac{d }{d \tau}
    \wh{\psi}(\tau)|^2 d \tau.
\end{split}
\label{u-1-norm-comp-pre}
\end{equation}
%
%
Applying the inequality
%
%
\begin{equation*}
\begin{split}
  \frac{(1 + |n|)^{2s}}{ n^{2}} \le \frac{(1 + |n|)^{2s}}{\frac{1}{4}(1 +
  |n|)^{2}} = 4 (1 + | n |)^{2(s-1)},  \quad s \in \rr, \quad n \ge 1
\end{split}
\end{equation*}
%
to \eqref{u-1-norm-comp-pre} gives
%
\begin{equation}
  \begin{split}
    \| \psi(t) S_{t}u_{1} \|_{X_{s,b}}^{2} 
    & \lesssim \sum_{n \in \dot{\zz}} \left( 1 + |n| \right)^{2(s-1)} | \wh{u_{1}}(n)
    |^{2} \int_{\rr} | \wh{\psi}(\tau - n^{2}) |^{2}\left( 1 + | | \tau | -
    n^{2} | \right)^{2b} d \tau
    \\
    & + \sum_{n \in \dot{\zz}} \left( 1 + |n| \right)^{2(s-1)} | \wh{u_{1}}(n)
    |^{2} \int_{\rr} | \wh{\psi}(\tau + n^{2}) |^{2}\left( 1 + | | \tau | -
    n^{2} | \right)^{2b} d \tau
    \\
    & + |\wh{u_{1}}(0)|^{2} \int_{\rr} (1 + | \tau |)^{2b} |\frac{d }{d \tau}
    \wh{\psi}(\tau)|^2 d \tau.
\end{split}
\label{u-1-norm-comp}
\end{equation}
%
%
Applying \eqref{eqn:norm-key-ineq},
we bound the first term of
\eqref{u-1-norm-comp} by
%
%
%
\begin{equation*}
  \begin{split}
    & \sum_{n \in \dot{\zz}} \left( 1 + |n| \right)^{2(s-1)} | \wh{u_{1}}(n)
    |^{2} \int_{\rr} | \wh{\psi}(\tau - n^{2}) |^{2}\left( 1 +  | \tau  -
    n^{2} | \right)^{2b} d \tau
    \\
    & = \sum_{n \in \dot{\zz}} \left( 1 + |n| \right)^{2(s-1)} | \wh{u_{1}}(n)
    |^{2} \int_{\rr} | \wh{\psi}(\tau') |^{2}\left( 1 +  | \tau'| \right)^{2b} d \tau
    \\
    & = c_{\psi, b} \| u_{1} \|_{H^{s-1}}^{2}
  \end{split}
\end{equation*}
%
%
where $c_{\psi, b}$ is a constant depending only on $\psi$ and $b$. Applying
\eqref{eqn:norm-key-ineq} again, the
second term of \eqref{u-1-norm-comp} is bounded by
\begin{equation*}
  \begin{split}
    & \sum_{n \in \dot{\zz}} \left( 1 + |n| \right)^{2(s-1)} | \wh{u_{1}}(n)
    |^{2} \int_{\rr} | \wh{\psi}(\tau + n^{2}) |^{2}\left( 1 +  | \tau  -
    n^{2} | \right)^{2b} d \tau
    \\
    & = \sum_{n \in \dot{\zz}} \left( 1 + |n| \right)^{2(s-1)} | \wh{u_{1}}(n)
    |^{2} \int_{\rr} | \wh{\psi}(\tau') |^{2}\left( 1 +  | \tau'| \right)^{2b} d \tau
    \\
    & = c_{\psi, b} \| u_{1} \|_{H^{s-1}}^{2}
  \end{split}
\end{equation*}
while the third term is bounded by  
%
%
\begin{equation*}
\begin{split}
  c_{\psi, b} \| u_{1} \|_{H^{s-1}}^{2}.
\end{split}
\end{equation*}
%
%
Therefore, 
$\|\psi(t) S_{t} u_{1}\|_{X_{s,b}}^{2} \le c_{\psi, b}
\|u_{1}\|_{H^{s-1}}^2$ and
taking square roots of both sides gives
%
%
\begin{equation}
  \begin{split}
    \|\psi(t) S_{t} u_{1}\|_{X_{s,b}} \le c_{\psi, b}
    \|u_{1}\|_{H^{s-1}}.
  \end{split}
  \label{eqn:u-1-fin-est}
\end{equation}

\subsubsection{Estimate for $\psi(t) \int_{0}^{t} S_{t-t'} (u^{2})_{xx} dt'$.}
\label{sssec:non-lin-term}
We define the spatial Fourier transform by 
%
%
\begin{equation*}
\begin{split}
  \tilde{f}(n, t) = \int_{\ci} e^{-inx}f(x,t) dx
\end{split}
\end{equation*}
%
%
and the spacetime Fourier transform by
\begin{equation*}
\begin{split}
  \wh{f}(n, \tau) = \int_{\rr} \int_{\ci} e^{-inx-it\tau}f(x,t) dx dt.
\end{split}
\end{equation*}
%
%
Let $f(x,t) \doteq \psi(t) \int_{0}^{t} S_{t-t'} (u^{2})_{xx} dt'$. 
Then
%
%
\begin{equation}
  \begin{split}
    \wt{f}(n, t)
    & = \frac{\psi(t)}{2i} \int_{0}^{t}\wt{u^{2}}(n, t') \left[
    e^{in^{2}(t-t')} - e^{-in^{2}(t-t')}
    \right] dt'
    \\
    & = \frac{1}{2i} e^{in^{2}t} \int_{0}^{t} \psi(t) \wt{u^{2}}(n, t') e^{-in^{2}t'}
    dt' - 
    \frac{1}{2i} e^{-in^{2}t} \int_{0}^{t} \psi(t) \wt{u^{2}}(n, t') e^{in^{2}t'} dt'
    \\
    & \doteq e^{in^{2}t} \wt{w_1}(n, t) - e^{-in^{2}t} \wt{w_2}(n, t)
  \end{split}
  \label{space-four-trans}
\end{equation}
%
where
%
%
\begin{gather*}
  w_{1}(x,t) = \frac{1}{4 i \pi} \sum_{n \in \zz} e^{inx} \left[ \int_{0}^{t} \psi(t) \wt{u^{2}}(n, t') e^{in^{2}t'}
  dt'\right],
  \\
  w_{2}(x,t) = \frac{1}{4 i\pi} \sum_{n \in \zz} e^{inx} \left[ \int_{0}^{t} \psi(t) \wt{u^{2}}(n, t') e^{-in^{2}t'} dt'
 \right].
\end{gather*}
%
%
%
Notice that \eqref{space-four-trans} is a \emph{global} relation in $t$.
Hence, taking its time Fourier transform gives
%
%
\begin{equation}
  \label{full-fourier-trans-exp}
\begin{split}
  \wh{f}(n, \tau) = \wh{w_{1}}(n, \tau - n^{2}) - \wh{w_{2}}(n, \tau +
  n^{2}).
\end{split}
\end{equation}
%
%
Therefore, using the definition of the $X_{s,b}$ spaces, and applying
\eqref{square-ineq} gives 
%
%
\begin{equation*}
\begin{split}
  \| f \|_{X_{s,b}}^{2}
  & = \sum_{n \in \zz} (1 + |n|)^{2s} \int_{\rr} (1 + |
  | \tau | - n^{2} |)^{2b} | \wh{w_{1}}(n, \tau - n^{2}) - \wh{w_{2}}(n, \tau +
  n^{2}) |^{2} d \tau
  \\
  & \le 4 \sum_{n \in \zz} (1 + |n|)^{2s} \int_{\rr} (1 + |
  | \tau | - n^{2} |)^{2b} | \wh{w_{1}}(n, \tau - n^{2}) d \tau
  \\
  & + 4 \sum_{n \in \zz} (1 + |n|)^{2s} \int_{\rr} (1 + |
  | \tau | - n^{2} |)^{2b} | \wh{w_{1}}(n, \tau + n^{2}) d \tau.
\end{split}
\end{equation*}
%
%
Applying a change of variable implies
%
%
%
%
\begin{equation}
\begin{split}
  \| f \|_{X_{s,b}}^{2}
  & \le 4 \sum_{n \in \zz} (1 + |n|)^{2s} \int_{\rr} (1 + |
  | \tau + n^{2} | - n^{2} |)^{2b} | \wh{w_{1}}(n, \tau) |^2 d \tau
  \\
  & + 4 \sum_{n \in \zz} (1 + |n|)^{2s} \int_{\rr} (1 + |
  | \tau - n^{2} | - n^{2} |)^{2b} | \wh{w_{2}}(n, \tau )|^2 d \tau.
\end{split}
\label{comp-pre-lemma}
\end{equation}
%
%
We now need the following lemma, whose proof is provided in the appendix.
%
%
%%%%%%%%%%%%%%%%%%%%%%%%%%%%%%%%%%%%%%%%%%%%%%%%%%%%%
%
%
%                Bound for modified principal symbol
%
%
%%%%%%%%%%%%%%%%%%%%%%%%%%%%%%%%%%%%%%%%%%%%%%%%%%%%%
%
%
\begin{lemma}
For any $n, \tau \in \rr$, we have
\label{lem:mod-princ-symb-bound}
%
%
\begin{equation*}
\begin{split}
  \max\left\{ | | \tau + n^{2} | - n^{2} |, | | \tau - n^{2} | - n^{2} |
  \right\} \le | \tau |.
\end{split}
\end{equation*}
%
%
\end{lemma}
%
%
Applying \autoref{lem:mod-princ-symb-bound} to \eqref{comp-pre-lemma} yields
%
%
\begin{equation*}
\begin{split}
\| f \|_{X_{s,b}}^{2}
  & \le 4 \sum_{j=1}^{2}  \sum_{n \in \zz} (1 + |n|)^{2s} \int_{\rr} (1 + |
  \tau|)^{2b} | \wh{w_{j}}(n, \tau)|^2 d \tau
  \\
  & = 4 \sum_{j=1}^{2} \sum_{n \in \zz} (1 + |n|)^{2s} \|\wt{w_{j}}(n, t)
  \|^{2}_{H_{t}^{b}}
  \\
  & = \sum_{n \in \zz} \| \psi(t) \int_{0}^{t} \wt{u^2}(n, t')
  e^{in^{2}t'}dt'  \|_{H_{t}^{b}}
  + 
  \sum_{n \in \zz} \| \psi(t) \int_{0}^{t} \wt{u^2}(n, t')
  e^{-in^{2}t'}dt'  \|_{H_{t}^{b}}.
\end{split}
\end{equation*}
%
We now need the following, whose proof is provided in~\cite{Ginibre:1996fk} and
the appendix.
%
%
%%%%%%%%%%%%%%%%%%%%%%%%%%%%%%%%%%%%%%%%%%%%%%%%%%%%%
%
%
%                Lemma to Reduce to Bilinear Est Form
%
%
%%%%%%%%%%%%%%%%%%%%%%%%%%%%%%%%%%%%%%%%%%%%%%%%%%%%%
%
%
\begin{lemma}[Lemma 2.2 in \cite{Farah:2009uq}]
Let $-1/2 < -a \le 0 < b \le 1-a$. Then
%
%
\begin{equation*}
\begin{split}
  \| \psi(t) \int_{0}^{t} g(t') dt' \|_{H^{b}_{t}} \le T^{1-(a+b)} \| g
  \|_{H_{t}^{-a}}.
\end{split}
\end{equation*}
%
%
\label{lem:pre-bilin-est}
\end{lemma}
%
%
Applying the lemma with $b = 1 - a$ (ATTENTION: Farah makes no mention of this--but it is
ncessary to take $b = 1-a$ in order to obtain as much smoothing as possible,
i.e. $H_t^{-a}$, want as large an $a$ as possible), we bound the right hand side of \eqref{comp-pre-lemma} by
%
%
\begin{equation*}
\begin{split}
  & \sum_{n \in \zz} (1 + |n|)^{s} \| \wt{u^{2}}(n, t')
  e^{in^{2}t'} \|_{H_{t}^{-a}}  +
  \sum_{n \in \zz} (1 + |n|)^{s} \| \wt{u^{2}}(n, t')
  e^{-in^{2}t'} \|_{H_{t}^{-a}} 
  \\
  & = \sum_{n \in \zz} (1 + |n|)^{s} \int_{\rr} (1 + | \tau
  |)^{2-a} \wh{u^{2}}(n, \tau - n^{2}) d \tau 
  \\
  & +
  \sum_{n \in \zz} (1 + |n|)^{s} \int_{\rr} (1 + | \tau
  |)^{2-a} \wh{u^{2}}(n, \tau + n^{2}) d \tau 
  \\
  & = \sum_{n \in \zz} (1 + |n|)^{s} \int_{\rr} (1 + | \tau
  + n^{2}
  |)^{2-a} \wh{u^{2}}(n, \tau ) d \tau 
  \\
  & +
  \sum_{n \in \zz} (1 + |n|)^{s} \int_{\rr} (1 + | \tau
  - n^{2} |)^{2-a} \wh{u^{2}}(n, \tau) d \tau  
\end{split}
\end{equation*}
%
%
which by \autoref{lem:mod-princ-symb-bound} is bounded by 
%
%
%
\begin{equation*}
\begin{split}
  & 
  \sum_{n \in \zz} (1 + |n|)^{s} \int_{\rr} (1 + | |\tau|
  - n^{2} |)^{2-a} \wh{u^{2}}(n, \tau) d \tau  
  \\
  & = \|u^{2} \|_{X_{s,-a}}^{2}.
\end{split}
\end{equation*}
%
%
Substituting back in for $f$ and taking square roots gives
%
%
\begin{equation}
\begin{split}
  \|\psi(t) \int_{0}^{t} S_{t-t'} (u^{2})_{xx} dt'\|_{X_{s,b}} \le \| u^{2}
  \|_{X_{s,-a}}, \qquad b = 1-a, \quad a < 1/2
\end{split}
\label{eqn:non-lin-bound}
\end{equation}
%
%
To bound the right hand side, we now require a crucial bilinear
estimate.
%
%
%%%%%%%%%%%%%%%%%%%%%%%%%%%%%%%%%%%%%%%%%%%%%%%%%%%%%
%
%
%                Bilinear Estimate
%
%
%%%%%%%%%%%%%%%%%%%%%%%%%%%%%%%%%%%%%%%%%%%%%%%%%%%%%
%
%
\begin{proposition}[Theorem 1.1 in \cite{Farah:2009uq}]
\label{prop:bilin-est}
  If $a > 1/4$, $b > 1/2$, and $s \ge -a/2$, 
  then 
  %
  %
  \begin{equation*}
  \begin{split}
    \| uv \|_{X_{s,-a}} \le  \| u \|_{X_{s,b}} \| v \|_{X_{s,b}}.
  \end{split}
  \end{equation*}
  %
  %
\end{proposition}
%
%
Applying the bilinear estimate to \eqref{eqn:non-lin-bound}, we conclude that
for $s > -1/4$, $b = 1 + 2s$, we have 
%
%
\begin{equation}
\begin{split}
\|\psi(t) \int_{0}^{t} S_{t-t'} (u^{2})_{xx} dt'\|_{X_{s,b}} \le 
  \| u \|^2_{X_{s,b}}. 
\end{split}
\label{eqn:nonlinear-term-bound}
\end{equation}
%
%
%
%
%
%
\subsubsection{Proof of Existence and Uniqueness in the Periodic Case}
\label{sssec:proof-b4-per-case}
%
%
Recalling \eqref{localized-int-eqn}, and collecting estimates \eqref{eqn:u-0-fin-est}, \eqref{eqn:u-1-fin-est}, and
\eqref{eqn:nonlinear-term-bound}, we obtain the following.
%%
%%%%%%%%%%%%%%%%%%%%%%%%%%%%%%%%%%%%%%%%%%%%%%%%%%%%%
%
%% Contraction Proposition
%				 
%%%%%%%%%%%%%%%%%%%%%%%%%%%%%%%%%%%%%%%%%%%%%%%%%%%%%%
%%
%%
%
\begin{proposition}
\label{prop:contraction}
Let  $s > -1/4$. Then for $b = 1 + 2s$, we have
%
%%
\begin{equation*}
	\begin{split}
    \|Tu\|_{X_{s,b}} \le c_{\psi, b} \left( \|u_0 \|_{H^s(\ci)} + \|u_1 \|_{H^{s-1}(\ci)}
    + \|u\|_{X_{s,b}}^2 
		\right).
	\end{split}
\end{equation*}
%
%%
\end{proposition}
We will now use \autoref{prop:contraction} to prove local well-posedness for the 
$B_4$ ivp. Let $c = c_{\psi, b}$. For given $u_0, u_1$, we may choose $\psi$ such
that 
%
%%
\begin{equation*}
	\begin{split}
    \|u_0\|_{H^s(\ci)} \le \frac{3}{32c^2}, \quad \|u_1\|_{H^{s-1}(\ci)} \le \frac{3}{32c^2}.
	\end{split}
\end{equation*}
%
%%
Then $$\|u\|_{X_{s,b}} \le \frac{1}{4c}$$ implies
%
%%
\begin{equation*}
	\begin{split}
		\|T u \|_{X_{s,b}} 
		& \le c \left[ \frac{3}{32c^2} + \frac{3}{32c^2}+ \left( 
		\frac{1}{4c} \right)^2 \right]
		=  \frac{1}{4c}.
	\end{split}
\end{equation*}
%
%%
Hence, $T=T_{u_0, u_1}$ maps the ball $B\left( 0, \frac{1}{4c} \right) \subset
X_{s,b}$ into itself. Next, note that 
%
%%
\begin{equation*}
	\begin{split}
		Tu - Tv = 
    \int_{0}^{t} S_{t-t'}
    (u^{2} - v^{2})_{xx} dt'.
  \end{split}
  \label{eqn:integral-form-dif}
\end{equation*}
%
%%
Rewriting
%
%%
\begin{equation*}
	\begin{split}
	\p_x^2 (u^2 - v^2)	
		& = \p_x^2[(u-v)(u+v)]
		\end{split}
\end{equation*}
%
%%
and repeating the arguments used in \autoref{sssec:non-lin-term},
we obtain
%
%%
%%
\begin{equation}
	\label{20a}
	\begin{split}
		\|Tu - Tv \|_{X_{s,b}}  
		& \le c_{\psi, b} \|u -v\|_{X_{s,b}} \|u + v \|_{X_{s,b}}
		\\
		& \le c_{\psi, b} \|u -v\|_{X_{s,b}} (\|u\|_{X_{s,b}}+ \|v \|_{X_{s,b}}).
	\end{split}
\end{equation}
%
%%
If $$ u, v \in B(0, \frac{1}{4c}) \subset X_{s,b},$$ then
%
%%
\begin{equation}
	\label{21a}
	\begin{split}
		\|Tu - Tv \|_{X_{s,b}}
		& \le c \|u -v \|_{X_{s,b}} \left( \frac{1}{4c} + 
		\frac{1}{4c} \right)
		\\
		& = \frac{1}{2} \|u -v \|_{X_{s,b}}. 
	\end{split}
\end{equation}
%
%%
We conclude that $T = T_{\vp}$ is a contraction on the ball $B(0, 
\frac{1}{4c}) \subset X_{s,b}$. A Picard iteration then yields a unique solution
$u \in X_{s,b}$ to \eqref{localized-int-eqn}. Applying
\autoref{lem:embedding}, it follows that $u(x,t) \subset C( [-T, T], H^s)$ is a unique
solution of the B4 ivp \eqref{eqn:mb-2}-\eqref{eqn:mb-init-data-2} for $t
\in [-T, T]$.
%
%
\subsubsection{Proof of Lipschitz Continuity in the Periodic Case} 
\label{sssec:lip-continuity}
%
%
We first define our notion of continuity.
%
%
\begin{definition}
  Let $X, Y$ be Banach spaces, and equip $X \times Y$ with the 
  topology defined by the norm $\|(f_0, f_1)\|_{X \times Y}
  = \|f_0\|_{X} + \|f_1\|_{Y}$.
  We say that the data to solution
  map $(f_0, f_1) \mapsto f(t)$ of the ivp $T_{f_0, f_1} f =
  0$ is \emph{locally Lipschitz} in $X \times Y$ if for
  $(u_0, u_1), (v_0, v_1) \in B_R \doteq \{(f_0,f_1) \in X \times Y: \|f_0\|_{X} +
  \|f_1\|_{Y}< R\}$, there exist $C, T>0$ depending on $R$ and local solutions
  $u(x,t), v(x,t)$
  for $t \in [-T, T]$ of $T_{u_0, u_1}u=0, T_{v_0, v_1}v=0$ satisfying
	$$\|u(\cdot, t) - v(\cdot, t)
  \|_X \le C \left( \|u_{0} - v_0 \|_{X} + \|u_{1} - v_1 \|_{Y}
  \right), \quad t \in [-T, T].$$ We
	say the flow map is \emph{locally uniformly
	continuous} in $X$ if for
	$u_0, v_0 \in B_R$ there exists $T >0$ depending on $R$ and local solutions
  $u(x,t), v(x,t)$
  for $t \in [-T, T]$ of $T_{u_0, u_1}u=0, T_{v_0, v_1}v=0$ such that 
	$$ \|u(\cdot, t) - v(\cdot, t) \|_{X} \to
  0 \ \ \text{if}  \ \ \|u_0 - v_0 \|_{X}, \|u_1 - v_1 \|_{Y} \to 0, \quad
  t \in
  [-T, T]. $$ 
\end{definition}
%
%
Notice that any locally Lipschitz flow map is locally uniformly continuous. 
Next, we shall establish local Lipschitz continuity in $X_{s,b}$ of the flow
map. Let $(u_0, u_1), (v_0, v_1) \subset \in H^{s}(\ci) \times H^{s-1}(\ci)  $
be given. Choose $\psi$ such that $$(u_0, u_1), (v_0, v_1)  \subset B(0,
\frac{15}{64c^{3}}).$$ Then there exist $u, v \in X_{s,b}$ such that $u =
T_{u_0, u_1}$, $v = T_{v_0, v_1}$, and so
%
%
\begin{equation}
	\label{gen-1a}
	\begin{split}
		& T_{u_0, u_1}(u) - T_{v_0, v_1}(v)
		\\
    & = \psi(t ) R_{t}(u_{0} - v_0) + \psi(t) S_{t}(u_{1} - v_1)
    + \psi(t) \int_{0}^{t} S_{t-t'}
    (u^{2} - v^{2} )_{xx} dt'.
		\end{split}
\end{equation}
%
%
Using arguments similar to those in 
\autoref{sssec:est-init-term-1}-\autoref{sssec:estimate-init-term-2}
we obtain
%
%
\begin{equation}
	\label{gen-2a}
	\begin{split}
		& \| \psi(t ) R_t (u_0 - v_0)\|_{X_{s,b}}
		\le c_{\psi, b} \|u_0 -v_0\|_{H^s},
    \\
    & \| \psi(t) S_t (u_1 - v_1)\|_{X_{s,b}}
    \le c_{\psi, b} \|u_1 -v_1\|_{H^{s-1}}.
	\end{split}
\end{equation}
%
%
Therefore, from \eqref{21a}-\eqref{gen-2a}, we obtain
%
%
\begin{equation*}
	\begin{split}
    \|u -v \|_{X_{s,b}}
    & = \|T_{u_0, v_0}(u) - T_{u_1, v_1}(v) \|_{X_{s,b}}
    \\
    & \le
    c_{\psi, b} \left( \|u_0 -v_0 \|_{H^s\left( \ci \right)} +\|u_1 -v_1
        \|_{H^{s-1}\left( \ci \right)} + \frac{1}{2} \|u -v \|_{X_{s,b}}\right)
  \end{split}
\end{equation*}
%
%
which implies
%
%
\begin{equation*}
	\begin{split}
		\frac{1}{2} \|u-v\|_{X_{s,b}} \le
    c_{\psi, b} \left( \|u_0 -v_0 \|_{H^s\left( \ci \right)} +\|u_1 -v_1
        \|_{H^{s-1}\left( \ci \right)} \right )
      \end{split}
\end{equation*}
%
%
or
%
%
\begin{equation}
	\begin{split}
		\|u -v \|_{X_{s,b}} \le 2 c_{\psi, b} \left( \|u_0 -v_0 \|_{H^s\left( \ci \right)} +\|u_1 -v_1
        \|_{H^{s-1}\left( \ci \right)} \right ).
	\end{split}
  \label{pre-lem-estimate}
\end{equation}
%
%
Applying  
\autoref{lem:embedding} to \eqref{pre-lem-estimate}, we obtain 

%
%
%
	 %
	 %
	 \begin{equation*}
		 \begin{split}
			\|u(\cdot, t) -v(\cdot, t) \|_{H^s(\ci)} \le
      2 c_{\psi, b} \left( \|u_0 -v_0 \|_{H^s\left( \ci \right)} +\|u_1 -v_1
        \|_{H^{s-1}\left( \ci \right)} \right ).
		 \end{split}
	 \end{equation*}
	 %
	 %
Since $u,v$ are the unique solutions to the ivp
\eqref{localized-int-eqn}, it follows that $u(x,t), v(x,t), t \in [-T, T]$ are unique
local solutions to \eqref{eqn:mb-2} with
initial data $(u_0, u_1), (v_0, v_1)$, respectively.
Hence, the flow map of the $B_4$ ivp is locally Lipschitz continuous in
$H^s(\ci)$. This
concludes the proof of well-posedness for the $B_4$ ivp
\eqref{eqn:mb-2}-\eqref{eqn:mb-init-data-2}. \qquad \qedsymbol

%
%%%%%%%%%%%%%%%%%%%%%%%%%%%%%%%%%%%%%%%%%%%%%%%%%%%%%
%
%
%                The non-periodic Case
%
%
%%%%%%%%%%%%%%%%%%%%%%%%%%%%%%%%%%%%%%%%%%%%%%%%%%%%%
%
%
%
%%%%%%%%%%%%%%%%%%%%%%%%%%%%%%%%%%%%%%%%%%%%%%%%%%%%%
%
%
%                Proof of Bilinear Estimate B4 Per
%
%
%%%%%%%%%%%%%%%%%%%%%%%%%%%%%%%%%%%%%%%%%%%%%%%%%%%%%
%
%
\subsubsection{Proof of \autoref{prop:bilin-est}} 
\label{sssec:proof-bilin-est}
Since $\|f\|_{X_{s,-a'}} \le \|f\|_{X_{s, -a}}$ for $0 < a < a'$, we may assume
$1/4 < a \le b$ and $b > 1/2$ without loss of generality.
By duality, it suffices to show that for $s \ge -a/2$, 
%
%%
\begin{equation}
	\label{duality-est}
	\begin{split}
	|	\sum_{n \in \zzdot}  (1 + |n|)^{s}
		\int_{\rr} \phi(n, \tau) \wh{uv}(n, \tau)(1 
    + | |\tau| - n^{2} |^{-a}) d \tau | \lesssim \|u\|_{X_{s,b}}
    \|v\|_{X_{s,b}}
    \|\phi \|_{L^{2}_{n, \tau}}.
	\end{split}
\end{equation}
Note first that $|\wh{uv}(n, \tau) |  = | \wh{u} *  \wh{v} 
(n, \tau)|$. From this it follows that
%
%
\begin{equation}
	\label{non-lin-rep}
	\begin{split}
		| \wh{uv}(n, \tau)|
    & = | \sum_{n_{1} \in \zz }  \int
    \wh{u}\left( n_1,  \tau_1 \right) \wh{v}\left( n - n_1 , \tau - \tau_1   
\right) d \tau_1 |
\\
& \le  \sum_{n_{1} \in \zz }  \int
    |\wh{u}\left( n_1,  \tau_1 \right)| |\wh{v}\left( n - n_1 , \tau - \tau_1   
\right)| d \tau_1 
\\
& = \sum_{n_1 \in \zz } \int \frac{c_u\left( n_1, \tau_1 
\right)}{\langle n_1 \rangle ^s \langle |\tau_1| - n_1^{2} | \rangle ^{b}}
\\
& \times \frac{c_{v}\left( n - n_1, \tau - \tau_1 \right)}{\langle n -
n_1 \rangle ^s\ \langle |\tau - \tau_1 | -  (n - n_1)^{2} \rangle^{b}}
  \ d \tau_1 
\end{split}
\end{equation}
%
%
where for clarity of notation we have introduced 
%
%
%
\begin{equation*}
\begin{split}
\langle k \rangle \doteq 1 + |k|
\end{split}
\end{equation*}
%
%
and
%
\begin{equation*}
	\begin{split}
		c_h(n, \tau) =
			\langle n \rangle ^s \langle |\tau| - n^{2} \rangle ^{b} | \wh{h}\left( n, \tau \right) |.
	\end{split}
\end{equation*}
%
%
From our work above, it follows that 
%
%
\begin{equation}
	\label{convo-est-starting-pnt}
	\begin{split}
		 & \langle n \rangle^s \langle \tau - n^{2} \rangle^{-a} | \wh{uv}\left( 
		n, \tau \right) |
		\\
		& \le \langle |\tau| - n^{2} \rangle^{-a}
		\sum_{n_1 \in \zz} \int \frac{\langle n \rangle^{s}}{\langle n_1 \rangle^s
    \langle n - n_1 \rangle^s} 
		\times \frac{c_f(n_1, \tau_1)}{\langle |\tau_1| - n_1^{2} \rangle ^{b}}
		\\
		& \times
		\frac{c_g(n - n_1, \tau - \tau_1 )}{\langle |\tau - \tau_1| - (n - n_1)^{2}
    \rangle^{b}}\ d \tau_1.
	\end{split}
\end{equation}
%
%
Hence, 
%
%
\begin{equation}
  \label{pre-fubini-int-form}
	\begin{split}
    |\text{lhs of} \ \eqref{duality-est}|
	& \lesssim \sum_{n \in \zz} \int_{\rr} \phi(n, \tau) \langle n \rangle^s 
  \sum_{n_1 \in \zz}
  \int_{\rr} c_f(n_1, \tau_1)
		c_g(n - n_1, \tau - \tau_1 )
		\\
    & \times \frac{\langle n \rangle ^{s}}{\langle n_{1} \rangle ^{s} \langle
    n-n_{1} \rangle ^{s}} \times \frac{1}{\langle |\tau| - n^{2} \rangle
    ^{a}\langle |\tau_{1}|-n_{1}^{2} \rangle ^{b}\langle | \tau -
    \tau_{1}|-(n - n_{1})^{2}
    \rangle ^{b}} d \tau_1 d \tau.
	\end{split}
\end{equation}
%
Let $A \subset \rr^{2} \times \zz^{2}$, and $\chi_{A}(\tau, \tau_{1}, n, n_{1})$
be its
characteristic function. Then by Cauchy-Schwartz in
$\tau_{1}, \xi_{1}$
\begin{equation*}
	\begin{split}
    & \sum_{n \in \zz} \int_{\rr}   \sum_{n_1 \in \zz}
    \int_{\rr} \chi_{A}
    \phi(n, \tau) \langle n \rangle^s \langle \tau - n^{2} \rangle^{-a}
  c_f(n_1, \tau_1)
		c_g(n - n_1, \tau - \tau_1 )
		\\
    & \times \frac{\langle n \rangle ^{s}}{\langle n_{1} \rangle ^{s} \langle
    n-n_{1} \rangle ^{s}} \times \frac{1}{\langle |\tau| - n^{2} \rangle
    ^{a}\langle |\tau_{1}|-n_{1}^{2} \rangle ^{b}\langle | \tau -
    \tau_{1}|-(n - n_{1})^{2}
    \rangle ^{b}} d \tau_1 d \tau.
	\end{split}
\end{equation*}
%
is bounded by 
%
%
\begin{equation}
	\label{10g}
	\begin{split}
    & \sum_{n \in \zz} \int_{\rr} \phi(n, \tau) \langle | \tau | - n^{2} \rangle
    ^{-a} \langle n \rangle ^{s}
    \\
    & \times \left( \sum_{n_{1} \in \zz} \int_{\rr}
    \frac{\chi_{A}}{\langle n_{1} \rangle ^{2s} \langle n-n_{1} \rangle ^{2s} \langle |
    \tau_{1} | - n_{1}^{2}\rangle ^{2b} \langle | \tau - \tau_{1} | -
    (n - n_{1})^{2} \rangle ^{2b}} d \tau_{1} \right)^{1/2}
    \\
    & \times \left( \sum_{n_{1} \in \zz} \int_{\rr} c_{u}^{2}(n, \tau_{1})
    c_{v}^{2}(n - n_{1}, \tau - \tau_{1}) d \tau_{1} \right)^{1/2} d \tau
  \end{split}
\end{equation}
%
%
Applying Cauchy-Schwartz again, \eqref{10g} is bounded by
%
%
\begin{equation*}
  \begin{split}
  & \|\left( \sum_{n_{1} \in \zz }\int_{\rr } c_{u}^{2}(n_1, \tau_1)
  c_{v}^{2} (n - n_1, \tau - \tau_{1} ) d \tau_1  \right)^{1/2} \|_{L^{2}(\zz \times
		\rr)}
		\\
    & \times  \|\phi(n, \tau) \langle | \tau | - n^{2} \rangle ^{-a} \langle n
    \rangle ^{s}
		\\
    & \times \left( \sum_{n_{1} \in \zz} \int_{\rr} \frac{\chi_{A}}{ \langle n_{1}
    \rangle ^{2s} \langle n-n_{1} \rangle ^{2s} \langle | \tau_{1}|-n_{1}^{2}
    \rangle^{2b} \langle  |\tau -
    \tau_{1} | -(n - n_{1})^{2}
    \rangle^{2b} } d \tau_1 \right)^{1/2} \|_{L^2(\zz \times \rr)}
		\\
    & = \|u\|_{X_{s,b}} \|v\|_{X_{s,b}} \label{holder-term}
     \|\phi(n, \tau)     \\
    & \times \left( \langle | \tau | - n^{2} \rangle ^{-2a} \langle n
    \rangle ^{2s}
    \sum_{n_{1} \in \zz} \int_{\rr} \frac{\chi_{A}}{ \langle n_{1} \rangle ^{2s} \langle
n-n_{1} \rangle ^{2s}  \langle | \tau_{1}|-n_{1}^{2} \rangle^{2b} \langle  |\tau -
    \tau_{1} | -(n - n_{1})^{2}
    \rangle^{2b} } d \tau_1 \right)^{1/2} \|_{L^2(\zz \times \rr)}.
  \end{split}
\end{equation*}
%
Applying H{\"o}lder, we bound this by 
%
%
\begin{equation}
  \label{integral-bound-1st-form-per}
	\begin{split}
    & \|u\|_{X_{s,b}} \|v\|_{X_{s,b}} \| \phi \|_{L^{2}_{n, \tau}}
    \\
    & \times \|\left( \langle | \tau | - n^{2} \rangle ^{-2a} \langle n
    \rangle ^{2s}
    \sum_{n_{1} \in \zz} \int_{\rr} \frac{\chi_{A}}{ \langle n_{1} \rangle ^{2s} \langle
n-n_{1} \rangle ^{2s} \langle | \tau_{1}|-n_{1}^{2} \rangle^{2b} \langle  |\tau -
    \tau_{1} | -(n - n_{1})^{2}
    \rangle ^{2b} } d \tau_1 \right)^{1/2} \|_{L^\infty_{n, \tau}}
	\end{split}
\end{equation}
%
%
Let us now return to the right hand side of \eqref{pre-fubini-int-form}.
Then by the change of variable $\lambda =
\tau - \tau_{1}$, we obtain
\begin{equation*}
	\begin{split}
    & \sum_{n \in \zz} \int_{\rr}   \sum_{n_1 \in \zz}
    \int_{\rr} \chi_{A}
    \phi(n, \tau) \langle n \rangle^s \langle \tau - n^{2} \rangle^{-a}
  c_f(n_1, \tau_1)
		c_g(n - n_1, \lambda )
		\\
    & \times \frac{\langle n \rangle ^{s}}{\langle n_{1} \rangle ^{s} \langle
    n-n_{1} \rangle ^{s}} \times \frac{1}{\langle |\tau| - n^{2} \rangle
    ^{a}\langle |\tau_{1}|-n_{1}^{2} \rangle ^{b}\langle | \tau -
    \tau_{1}|-(n - n_{1})^{2}
    \rangle ^{b}} d \tau_1 d \tau
    \\
    & = \sum_{n \in \zz} \int_{\rr}   \sum_{n_1 \in \zz}
    \int_{\rr} \chi^{*}_{A}
    \phi(n, \tau) \langle n \rangle^s \langle \tau - n^{2} \rangle^{-a}
  c_f(n_1, \tau - \lambda)
		c_g(n - n_1, \lambda )
		\\
    & \times \frac{\langle n \rangle ^{s}}{\langle n_{1} \rangle ^{s} \langle
    n-n_{1} \rangle ^{s}} \times \frac{1}{\langle |\tau| - n^{2} \rangle
    ^{a}\langle |\tau - \lambda|-n_{1}^{2} \rangle ^{b}\langle |
    \lambda|-(n - n_{1})^{2}
    \rangle ^{b}} d \lambda  d \tau
	\end{split}
\end{equation*}
where 
%
%
\begin{equation}
  \label{change-of-var}
\begin{split}
  \chi^{*}_{A}(\tau, \tau - \lambda, n, n_{1}) =
  \chi_{A}(\tau, \tau_{1}, n, n_{1}).
\end{split}
\end{equation}
%
%
Cauchy-Schwartz in
$\lambda, \xi_{1}$ then gives the bound
%
%
%
\begin{equation}
	\label{10g*}
	\begin{split}
    & \sum_{n \in \zz} \int_{\rr} \phi(n, \tau) \langle | \tau | - n^{2} \rangle
    ^{-a} \langle n \rangle ^{s}
    \\
    & \times \left( \sum_{n_{1} \in \zz} \int_{\rr}
    \frac{\chi^{*}_{A}}{\langle n_{1} \rangle ^{2s} \langle n-n_{1} \rangle ^{2s} \langle |
    \tau - \lambda | - n_{1}^{2}\rangle ^{2b} \langle | \lambda | -
    (n - n_{1})^{2} \rangle ^{2b}} d \lambda \right)^{1/2}
    \\
    & \times \left( \sum_{n_{1} \in \zz} \int_{\rr} c_{u}^{2}(n_{1}, \tau - \lambda)
    c_{v}^{2}(n - n_{1}, \lambda) d \lambda \right)^{1/2} d \tau
  \end{split}
\end{equation}
%
%
Applying Cauchy-Schwartz again, \eqref{10g*} is bounded by
%
%
\begin{equation*}
  \begin{split}
  & \|\left( \sum_{n_{1} \in \zz }\int_{\rr } c_{u}^{2}(n_1, \tau - \lambda)
  c_{v}^{2} (n - n_1, \lambda ) d \tau_1  \right)^{1/2} \|_{L^{2}(\zz \times
		\rr)}
		\\
    & \times  \|\phi(n, \tau) \langle | \tau | - n^{2} \rangle ^{-a} \langle n
    \rangle ^{s}
		\\
    & \times \left( \sum_{n_{1} \in \zz} \int_{\rr} \frac{\chi^{*}_{A}}{ \langle n_{1}
    \rangle ^{2s} \langle n-n_{1} \rangle ^{2s} \langle | \tau - \lambda|-n_{1}^{2}
    \rangle^{2b} \langle  |\lambda | -(n - n_{1})^{2}
    \rangle^{2b} } d \tau_1 \right)^{1/2} \|_{L^2(\zz \times \rr)}
		\\
    & = \|u\|_{X_{s,b}} \|v\|_{X_{s,b}} \label{holder-term*}
    \left \{ \sum_{n \in \zz} \int_{\rr} |\phi(n, \tau)|^{2} \right .
    \\
    & \left. \times \sum_{n_{1} \in \zz} \int_{\rr} \frac{\chi^{*}_{A}
    \langle n \rangle ^{2s}
    }{ \langle n_{1} \rangle^{2s} \langle | \tau | - n^{2}
    \rangle ^{2a}  \langle
n-n_{1} \rangle ^{2s}  \langle | \tau - \lambda|-n_{1}^{2}
\rangle^{2b} \langle  | \lambda | -(n - n_{1})^{2}
    \rangle^{2b} } d \lambda d \tau \right \}^{1/2}.
  \end{split}
\end{equation*}
%
%
Applying Fubini and H{\"o}lder to the last term gives the bound
%
%
\begin{equation}
  \label{integral-bound-1st-form-per*}
	\begin{split}
    & \|u\|_{X_{s,b}} \|v\|_{X_{s,b}} \| \phi \|_{L^{2}_{n, \tau}}
    \\
    & \times \left( \sum_{n_{1} \in \zz} \int_{\rr} \frac{\chi^{*}_{A}
    \langle n \rangle ^{2s}
    }{ \langle n_{1} \rangle^{2s} \langle | \tau | - n^{2}
    \rangle ^{2a}  \langle
n-n_{1} \rangle ^{2s}  \langle | \tau - \lambda|-n_{1}^{2}
\rangle^{2b} \langle  | \lambda | -(n - n_{1})^{2}
    \rangle^{2b} } d \tau  \right)^{1/2} \|_{L^\infty_{n, \lambda}}.
	\end{split}
\end{equation}
Again, we return to the right hand side of \eqref{pre-fubini-int-form}.
We seek to bound
\begin{equation*}
\begin{split}
  & \sum_{n \in \zz} \int_{\rr}  \sum_{n_{1} \in \zz }
  \int_{\rr} \chi_{A} \phi(n, \tau)
    c_f(n_1, \tau_1)
		c_g(n - n_1, \tau - \tau_1 )
		\\
    & \times \frac{\langle n \rangle ^{s}}{\langle n_{1} \rangle ^{s} \langle
    n-n_{1} \rangle ^{s}} \times \frac{1}{ \langle \tau - n^{2} \rangle^{a}
\langle |\tau| - n^{2} \rangle
    ^{b}\langle |\tau_{1}|-n_{1}^{2} \rangle ^{b}\langle | \tau|-n_{2}^{2}
    \rangle ^{b}} d \tau_1 d \tau 
   \end{split}
\end{equation*}
in a different manner than before. First, we apply 
Fubini, then Cauchy-Schwartz in $n_{1}, \tau_{1}$ to obtain the bound
%
%
\begin{equation*}
\begin{split}
  & \left[ \sum_{n_{1} \in \zz} \int_{\rr} c_{f}^{2}(n_{1}, \tau_{1}) d \tau_{1}
  \right]^{1/2}
  \\
  & \times \left \{ \sum_{n_{1} \in \zz} \int_{\rr}   
 \left[
 \sum_{n \in \zz} \int_{\rr}
   \frac{\langle n \rangle ^{s}}{\langle n_{1} \rangle ^{s} \langle
   n - n_{1}\rangle ^{s}} \times \frac{\chi_{A} |\phi(n, \tau)| c_{g}(n -
   n_{1}, \tau - \tau_{1})
}{\langle | \tau | - n^{2} \rangle
  ^{a} \langle | \tau_{1} | - n_{1}^{2} \rangle ^{b} \langle | \tau -
  \tau_{1} | - (n - n_{1}^{2}) \rangle ^{b}} d \tau 
  \right]^{2} d \tau_{1} \right \}^{1/2}
  \\
  & = \| f \|_{X_{s,b}}
  \\
  & \times \left \{ \sum_{n_{1} \in \zz} \int_{\rr}   
 \left[
 \sum_{n \in \zz} \int_{\rr}
   \frac{\langle n \rangle ^{s}}{\langle n_{1} \rangle ^{s} \langle
   n - n_{1}\rangle ^{s}} \times \frac{\chi_{A}|\phi(n, \tau)| c_{g}(n -
   n_{1}, \tau - \tau_{1})
}{\langle | \tau | - n^{2} \rangle
  ^{a} \langle | \tau_{1} | - n_{1}^{2} \rangle ^{b} \langle | \tau -
  \tau_{1} | - (n - n_{1}^{2}) \rangle ^{b}} d \tau 
  \right]^{2} d \tau_{1}  \right \}^{1/2}
\end{split}
\end{equation*}
%
Applying Cauchy-Schwartz in $\tau, n$, we bound the last line by 
%
%
\begin{equation*}
\begin{split}
  & \left \{ \sum_{n_{1} \in \zz} \int_{\rr}   
  \left [ \sum_{n \in \zz} \int_{\rr}
  | \phi(n, \tau)|^{2} c_{g}^{2}(n - n_{1}, \tau - \tau_{1}) d \tau  
    \right ] \right . 
   \\
   & \left. \times \left [ \sum_{n \in \zz} \int_{\rr} \frac{\langle n \rangle
   ^{2s}}{\langle n_{1} \rangle ^{2s} \langle n - n_{1}\rangle ^{2s}}
   \times \frac{\chi_{A}}{\langle | \tau | - n^{2} \rangle ^{2a} \langle | \tau_{1} |
   - n_{1}^{2} \rangle ^{2b} \langle | \tau - \tau_{1} | - (n - n_{1}^{2})
   \rangle ^{2b}} d \tau  \right ] \right \}^{1/2}d \tau_{1} 
\end{split}
\end{equation*}
%
%
which by Holder is bounded by 
%
%
%
\begin{equation}
  \label{integral-bound-2nd-form-per}
\begin{split}
  & \| \sum_{n \in \zz} \int_{\rr} \frac{\langle n \rangle ^{2s}}{\langle n_{1} \rangle ^{2s} \langle
  n - n_{1}\rangle ^{2s}}  \times \frac{\chi_{A}}{\langle | \tau | - n^{2} \rangle
  ^{2a} \langle | \tau_{1} | - n_{1}^{2} \rangle ^{2b} \langle | \tau -
  \tau_{1} | - (n - n_{1}^{2}) \rangle ^{2b}} d \tau 
  \|_{L^{\infty}_{n_{1}, \tau_{1}}}^{1/2}
  \\
  & \times \|\phi\|_{L^{2}} \| g \|_{X_{s,b}}.
\end{split}
\end{equation}
%
%
Now consider the family $\{A_{j}\}_{1}^{k}, A_{j} \subset \rr^{2} \times
\zz^{2}$ with
$$\bigcup_{1}^{k} A_{j}= \rr^{2} \times
\zz^{2}.$$ From \eqref{integral-bound-1st-form-per},
\eqref{integral-bound-1st-form-per*},
\eqref{integral-bound-2nd-form-per}, and our preceding argumentation,
we see that the proof of \autoref{prop:bilin-est-real} reduces to showing that
either 
%
%
%
%
\begin{equation}
  \label{key-sup-estimate-per-1}
  \begin{split}
     \| \langle | \tau | - n^{2} \rangle ^{-2a} \langle n
    \rangle ^{2s}
    \sum_{n_{1} \in \zz}
    \int_{\rr} \frac{\chi_{A_{j}}}{ \langle n_{1} \rangle ^{2s} \langle
n-n_{1} \rangle ^{2s} \langle | \tau_{1}|-n_{1}^{2} \rangle^{2b}  \langle  |\tau -
    \tau_{1} | -(n - n_{1})^{2}
    \rangle ^{2b} } d \tau_1  \|_{L^\infty_{n, \tau}} < \infty.
  \end{split}
\end{equation}
%
or
%%
\begin{equation}
  \label{key-sup-estimate-per-2}
\begin{split}
  & \| \frac{1}{\langle n_{1} \rangle ^{2s}
  \langle | \tau_{1} | - n_{1}^{2} \rangle
  ^{2a}} \sum_{n \in \zz} \int_{\rr} \frac{\langle n \rangle ^{2s}}{\langle
  n - n_{1}\rangle ^{2s}}  \times \frac{\chi_{A_{j}}}{\langle | \tau | - n^{2} \rangle ^{2b} \langle | \tau -
  \tau_{1} | - (n - n_{1}^{2}) \rangle ^{2b}} d \tau 
  \|_{L^{\infty}_{n_{1}, \tau_{1}}}
\end{split}
\end{equation}
or
%
%
\begin{equation}
  \label{key-sup-estimate-per-3}
\begin{split}
  \| \sum_{n_{1} \in \zz} \int_{\rr} \frac{\chi_{A_{j}}
    \langle n \rangle ^{2s}
    }{ \langle n_{1} \rangle^{2s} \langle | \tau | - n^{2}
    \rangle ^{2a}  \langle
n-n_{1} \rangle ^{2s}  \langle | \tau - \lambda|-n_{1}^{2}
\rangle^{2b} \langle  | \lambda | -(n - n_{1})^{2}
\rangle^{2b} } d \tau  \|_{L^{\infty}_{n, \lambda}}
\end{split}
\end{equation}
%
%
for each $j \in \left\{ 1,\dots,k \right\}$. 
By the triangle inequality and the fact that 
%
%
\begin{equation*}
\begin{split}
& | \tau | =
\begin{cases}
  - \tau, \quad & \tau < 0, 
\\
\tau, \quad & \tau > 0
\end{cases}
\end{split}
\end{equation*}
%
%
it follows that the proof of \autoref{prop:bilin-est} reduces to showing that
for any $j$, either 
%
%
\begin{equation}
  \label{sup-est-gen-per-1}
  \begin{split}
    \| \frac{ \langle n
    \rangle ^{2s}}{\langle \sigma \rangle ^{2a}}
    \sum_{n_{1} \in \zz} \int_{\rr} \frac{\chi_{A_{j}}}{ \langle n_{1} \rangle ^{2s} \langle n-n_{1} \rangle ^{2s} 
    \langle \sigma_{1} \rangle^{2b} \langle  \sigma_{2} \rangle^{2b} }
    d \tau_1  \|_{L^{\infty}_{n, \tau}} < \infty
  \end{split}
\end{equation}
%
%
or 
\begin{equation}
  \label{sup-est-gen-per-2}
\begin{split}
  & \| \frac{1}{\langle n_{1} \rangle ^{2s}
  \langle \sigma_{1} \rangle
  ^{2a}} \sum_{n \in \zz} \int_{\rr} \frac{\langle n \rangle ^{2s}}{\langle
  n - n_{1}\rangle ^{2s}}  \times \frac{\chi_{A_{j}}}{\langle
  \sigma \rangle ^{2b} \langle \sigma_{2} \rangle ^{2b}} d \tau 
  \|_{L^{\infty}_{n_{1}, \tau_{1}}} < \infty
\end{split}
\end{equation}
%
or
\begin{equation}
  \label{sup-est-gen-per-3}
\begin{split}
  \| \sum_{n_{1} \in \zz} \int_{\rr} \frac{\chi^{*}_{A_{j}}
    \langle n \rangle ^{2s}
    }{ \langle n_{1} \rangle^{2s} \langle
    n-n_{1} \rangle ^{2s} \langle \sigma^{*}  
    \rangle ^{2a}
    \langle \sigma_{1}^{*} \rangle^{2b}
    \langle  \sigma_{2}^{*} \rangle^{2b}  } d \tau  \|_{L^{\infty}_{n, \lambda}}
\end{split}
\end{equation}
%
%
where we consider cases
\begin{enumerate}[(i)]
    \item $ \sigma=\tau+n^2,\quad \sigma_1=\tau_1+n_1^2,\quad \sigma_2=\tau -
      \tau_1+(n - n_1)^2$,
\label{it-1}
    \item $ \sigma=\tau-n^2,\quad \sigma_1=\tau_1-n_1^2,\quad \sigma_2=\tau - \tau_1+(n - n_1)^2$,
\label{it-2}
    \item  $\sigma=\tau+n^2,\quad \sigma_1=\tau_1-n_1^2,\quad \sigma_2=\tau - \tau_1+(n - n_1)^2$,
      \label{it-3}
    \item $\sigma=\tau-n^2,\quad \sigma_1=\tau_1+n_1^2,\quad \sigma_2=\tau - \tau_1-(n - n_1)^2$,
\label{it-4}
    \item $\sigma=\tau+n^2,\quad \sigma_1=\tau_1+n_1^2,\quad \sigma_2=\tau - \tau_1-(n - n_1)^2$,
\label{it-5}
    \item $\sigma=\tau-n^2,\quad \sigma_1=\tau_1-n_1^2,\quad \sigma_2=\tau - \tau_1-(n - n_1)^2$.
\label{it-6}
\end{enumerate}
%
for \eqref{sup-est-gen-per-1} and \eqref{sup-est-gen-per-2}, and cases
%
\begin{enumerate}[(i)]
\item $ \sigma^{*}=\tau+n^2,\quad \sigma^{*}_1=\tau - \lambda+n_1^2,\quad
  \sigma^{*}_2=\lambda+(n - n_1)^2$, \label{it-1-star} \item $
  \sigma^{*}=\tau-n^2,\quad \sigma^{*}_1=\tau - \lambda-n_1^2,\quad
  \sigma^{*}_2=\lambda+(n - n_1)^2$, \label{it-2-star} \item
  $\sigma^{*}=\tau+n^2,\quad \sigma^{*}_1=\tau - \lambda-n_1^2,\quad
  \sigma^{*}_2=\lambda+(n - n_1)^2$, \label{it-3-star} \item
  $\sigma^{*}=\tau-n^2,\quad \sigma^{*}_1=\tau - \lambda+n_1^2,\quad
  \sigma^{*}_2=\lambda-(n - n_1)^2$, \label{it-4-star} \item
  $\sigma^{*}=\tau+n^2,\quad \sigma^{*}_1=\tau - \lambda+n_1^2,\quad
  \sigma^{*}_2=\lambda-(n - n_1)^2$, \label{it-5-star} \item
  $\sigma^{*}=\tau-n^2,\quad \sigma^{*}_1=\tau - \lambda-n_1^2,\quad
  \sigma^{*}_2= \lambda-(n - n_1)^2$.  \label{it-6-star}
  \end{enumerate}
for \eqref{sup-est-gen-per-3}.
%
%
\begin{framed}
\begin{remark}
Note that the cases $\sigma=\tau+n^2,\quad \sigma_1=\tau_1-n_1^2,\quad
\sigma_2=\tau - \tau_1-(n - n_1)^2$ and $\sigma=\tau-n^2,\quad
\sigma_1=\tau_1+n_1^2,\quad \sigma_2=\tau - \tau_1+(n - n_1)^2$ cannot occur, since
$\tau_1< 0, \tau-\tau_1< 0$ implies $\tau<0$ and $\tau_1\geq 0, \tau-\tau_1\geq
0$ implies $\tau\geq 0$. An analogous argument holds for $\sigma^{*},
\sigma_{1}^{*}$ and $\sigma_{2}^{*}$.
\end{remark}
\end{framed}
%
Observe that the transformation $(n, \tau, n_{1}, \tau_{1}) \mapsto -(n, \tau,
n_{1}, \tau_{1})$ reduces \eqref{it-3} to \eqref{it-4}, \eqref{it-2} to
\eqref{it-5}, and \eqref{it-1} to \eqref{it-6}. Furthermore, the change of
variables $\tau_{2} = \tau - \tau_{1}, n_{2} = n - n_{1}$, and the
transformation $(n, \tau, n_{2}, \tau_{2}) \mapsto - (n, \tau, n_{2},
\tau_{2})$ reduces \eqref{it-5} to \eqref{it-4}. Since $L^{2}$ is invariant
under change of variables and reflections, we may without loss of generality
restrict our attention to cases \eqref{it-4} and \eqref{it-6}.
 \subsubsection{Case \eqref{it-6}} 
\label{sssec:case-it-6}
Let 
%
%
\begin{align*}
A_1&=\{(n, n_1, \tau, \tau_1)\in A: n=0\},\\
A_2&=\{(n, n_1, \tau, \tau_1)\in A: n_1 = n \},\\
A_3&=\{(n, n_1, \tau, \tau_1)\in A: n_1=0 \},\\
A_4&=\{(n, n_1, \tau, \tau_1)\in A: n \neq 0, n_1 \neq 0 \text{ and } n_1 \neq n \}.
\end{align*} 
%
%
%
We seek to bound
\begin{equation*}
\begin{split}
  & \frac{1}{\langle n_{1} \rangle ^{2s}
  \langle \tau_{1} - n_{1}^{2} \rangle
  ^{2a}} \sum_{n \in \zz } \int_{\rr} \frac{\langle n \rangle ^{2s}}{\langle
  n - n_{1}\rangle ^{2s}}  \times \frac{\chi_{A_{1}}}{\langle
  \tau - n^{2}  \rangle ^{2b} \langle \tau - \tau_{1} - (n - n_{1})^{2}
  \rangle ^{2b}} d \tau 
\end{split}
\end{equation*}
which is equal to 
%
\begin{equation}
  \label{case-1-term-1-reduc}
\begin{split}
  & \frac{1}{\langle n_{1} \rangle ^{4s}
  \langle \tau_{1} - n_{1}^{2} \rangle
  ^{2a}} \int_{\rr} \frac{1}{\langle
  \tau  \rangle ^{2b} \langle \tau - \tau_{1} - n_{1}^{2}
  \rangle ^{2b}} d \tau.
\end{split}
\end{equation}
%
Following Ginibre, Tsutsumi, Velo~\cite{Ginibre:1997fk}, Kenig, Ponce, Vega~\cite{Kenig:1996aa}, and others,
we now need the following Calculus lemma.
%
%%%%%%%%%%%%%%%%%%%%%%%%%%%%%%%%%%%%%%%%%%%%%%%%%%%%%
%
%
%				 Calculus Lemma
%
%
%%%%%%%%%%%%%%%%%%%%%%%%%%%%%%%%%%%%%%%%%%%%%%%%%%%%%
%
%
\begin{lemma}
	\label{lem:calc}
 %
 %
 For $r > 1/2$
\begin{equation*}
  \int_{\rr} \frac{1} {\langle  \theta \rangle^{r} \langle  a - \theta
  \rangle^{r}} d \theta \leq\frac{\log 2} {\langle a \rangle^{r}}.
\end{equation*}
 %
 %
 \end{lemma}

%
By \autoref{lem:calc}, \eqref{case-1-term-1-reduc} is bounded by
%
%
\begin{equation*}
\begin{split}
  & \frac{c}{\langle n_{1} \rangle ^{4s} \langle \tau_{1} - n_{1}^{2} \rangle
  ^{2a}
  \langle \tau_{1} + n_{1}^{2} \rangle
  ^{2b}} .
\end{split}
\end{equation*}
%
Since the minimum of the function $f(x) = (x-a)(x+a)$ occurs at $x = 0$ and
$\langle n_{1} \rangle ^{2} \le 4\langle n_{1}^{2} \rangle  $, we
bound the above by 
%
%
\begin{equation*}
\begin{split}
  \frac{C}{\langle n_{1} \rangle ^{2(2s + a + b)}} < \infty, \qquad s \ge
  -(a + b)/2.
\end{split}
\end{equation*}
%
\begin{framed}
\begin{remark}
This result is optimal for region $A_{1}$. To illustrate this, we now estimate
%
%
\begin{equation*}
  \begin{split}
     \frac{ \langle n
    \rangle ^{2s}}{\langle \tau - n^{2} \rangle ^{2a}}
    \sum_{n_{1} \in \zz} \int_{\rr} \frac{\chi_{A_{1}}}{ \langle n_{1} \rangle ^{2s} \langle n-n_{1} \rangle ^{2s} 
    \langle \tau_{1} - n_{1}^{2} \rangle^{2b} \langle  \tau - \tau_{1} -
    (n - n_{1})^{2} \rangle^{2b} }
    d \tau_1 
  \end{split}
\end{equation*}
which reduces to 
\begin{equation}
  \label{pathological-equality-case-1}
  \begin{split}
    \frac{ 1}{\langle \tau \rangle ^{2a}}
    \sum_{n_{1} \in \zz} \int_{\rr} \frac{\chi_{A_{1}}}{ \langle n_{1} \rangle ^{4s}
    \langle \tau_{1} - n_{1}^{2} \rangle^{2b} \langle  \tau - \tau_{1} -
    n_{1}^{2} \rangle^{2b} }
    d \tau_1.
  \end{split}
\end{equation}
%
%
%
%
Setting $\tau = 0$, we obtain
%
%
%
\begin{equation*}
\begin{split}
   \sum_{n_{1}} \langle & n_{1}\rangle ^{-4s} \int_{\rr} \frac{1}{\langle
   \tau_{1} - n_{1}^{2} \rangle ^{4b}}d \tau_{1}
   \\
   & = \sum_{n_{1}} \langle
  n_{1}\rangle ^{-4s} \int_{\rr} \frac{1}{\langle
   \tau' \rangle ^{4b}}d \tau'
   \\
   & \simeq \sum_{n_{1}} \langle n_{1} \rangle ^{-4s}, \quad b > 1/4
   \\
   & < \infty, \quad s > 1/4.
\end{split}
\end{equation*}
%
Hence, we cannot hope to bound \eqref{pathological-equality-case-1} for $s \le
1/4$ using this splitting. Instead, consider now
%
%
\begin{equation*}
\begin{split}
  \sum_{n_{1} \in \zz} \int_{\rr} \frac{\chi^{*}_{A_{1}}}{\langle n_{1} \rangle
  ^{4s} \langle \tau - n^{2} \rangle ^{2a} \langle \tau_{1} - n_{1}^{2} \rangle
  ^{2b} \langle \tau - \tau_{1} - (n - n_{1})^{2} \rangle ^{2b}}
\end{split}
\end{equation*}
%
%
which reduces to 
\begin{equation*}
\begin{split}
  \sum_{n_{1} \in \zz} \int_{\rr} \frac{\chi^{*}_{A_{1}}}{\langle n_{1} \rangle
  ^{4s} \langle \tau \rangle ^{2a} \langle \tau_{1} - n_{1}^{2} \rangle
  ^{2b} \langle \tau - \tau_{1} - n_{1}^{2} \rangle ^{2b}}
\end{split}
\end{equation*}
Applying \autoref{lem:calc} and the fact that the function $f(x) =
(x-a)(x+a)$ achieves its minimum at $x=0$, we obtain the bounds
%
%
\begin{equation*}
\begin{split}
  & \sum_{n_{1} \in \zz} \frac{\chi^{*}_{A_{1}}}{\langle n_{1} \rangle ^{4s} \langle
  \tau_{1} - n_{1}^{2} \rangle ^{2b} \langle \tau_{1} + n_{1}^{2} \rangle
  ^{2a}}
  \\
  & \le \sum_{n_{1} \in \zz}  \frac{\chi_{A_{1}}^{*}}{\langle n_{1} \rangle
  ^{2(2s + a + b)}}
  \\
  & < \infty, \qquad s > 1/4 - (a + b)/2.
\end{split}
\end{equation*}
%
%
\end{remark}
\end{framed}
%
Applying \autoref{lem:calc}, we now bound 
\begin{equation*}
  \begin{split}
    & \frac{ \langle n
    \rangle ^{2s}}{\langle \tau - n^{2} \rangle ^{2a}}
    \sum_{n_{1} \in \zz} \int_{\rr} \frac{\chi_{A_{2}}}{ \langle n_{1} \rangle ^{2s} \langle n-n_{1} \rangle ^{2s} 
    \langle \tau_{1} - n_{1}^{2} \rangle^{2b} \langle  \tau - \tau_{1} -
    (n - n_{1})^{2} \rangle^{2b} }
    d \tau_1 
    \\
    & = 
   \langle \tau -n^{2} \rangle ^{-2a}\int_{\rr} \frac{1}{\langle \tau_{1} -
  n^{2} \rangle ^{2b}\langle
  \tau - \tau_{1}\rangle ^{2b}}d \tau_{1}
  \\
  & \lesssim 
  \langle \tau - n^{2} \rangle ^{-2a-2b} 
  \\
  & < \infty.
\end{split}
\end{equation*}
%
%
Similarly, we bound
%
%
\begin{equation}
\begin{split}
  & \frac{ \langle n
    \rangle ^{2s}}{\langle \tau - n^{2} \rangle ^{2a}}
    \sum_{n_{1} \in \zz} \int_{\rr} \frac{\chi_{A_{3}}}{ \langle n_{1} \rangle ^{2s} \langle n-n_{1} \rangle ^{2s} 
    \langle \tau_{1} - n_{1}^{2} \rangle^{2b} \langle  \tau - \tau_{1} -
    (n - n_{1})^{2} \rangle^{2b} }
    d \tau_1 
    \\
  & = \langle \tau - n^{2} \rangle ^{-2a}
  \int_{\rr} \frac{1}{ \langle \tau_{1} \rangle^{2b}  \langle \tau -
  \tau_{1} - n^{2} \rangle^{2b}}
d \tau_1 
\\
  & \lesssim   \langle \tau - n^{2} \rangle ^{-2a-2b} 
  \\
  & < \infty.
	\end{split}
\end{equation}
%
%
We now 
partition $ A_{4}$ into two parts
\begin{align*}
A_{4,1}&=\{(n, n_1, \tau, \tau_1)\in A_3: |\tau_1-n_1^2|\leq|\tau-n^2|\},\\
A_{4,2}&=\{(n, n_1, \tau, \tau_1)\in A_3: |\tau-n^2|\leq|\tau_1-n_1^2| \}.
\end{align*} 
Furthermore, by the symmetry of the convolution, we may assume without loss of
generality that
$$|(\tau-\tau_1)-(n-n_1)^2|\leq|\tau_1-n_1^2|\}.$$
Then in region $A_{4,1}$
\begin{equation}
\begin{split}
  | \tau - n^{2} |
  & \ge \frac{1}{3}\left[ | \tau_{1} - n_{1}^{2} | + | \tau -
  \tau_{1} - (n - n_{1})^{2}
  | + | \tau - n^{2} | \right]
  \\
  & \ge \frac{1}{3} | - n_{1}^{2} - (n - n_{1})^{2} + n^{2} |
  \\
  & = \frac{2}{3} | n_{1} | | n - n_{1} |
  \\
  & \gtrsim | n_{1} |. 
\end{split}
\label{smoothing-per-4-1}
\end{equation}
%
%
Hence, applying apply \autoref{lem:calc} and \eqref{smoothing-per-4-1}
we obtain
%
%
%
%
\begin{equation}
  \label{region-a41}
\begin{split}
& \langle \tau - n^{2}  \rangle ^{-2a} \langle n
    \rangle ^{2s}
    \sum_{n_{1} \in \zz} \int_{\rr} \frac{\chi_{A_{4,1}}}{ \langle n_{1} \rangle ^{2s} \langle n-n_{1} \rangle ^{2s} 
\langle \tau_{1} - n_{1}^{2}  \rangle \langle  \tau - \tau_{1} - (n -
n_{1})^{2}  \rangle}
d \tau_1 
\\
& \lesssim \langle \tau - n^{2} \rangle ^{-2a} \langle n \rangle ^{2s}
\sum_{n_{1} \in
\zz}  \frac{\chi_{A_{4,1}}}{\langle n_{1} \rangle ^{2s} \langle n - n_{1} \rangle
^{2s} \langle \tau - n^{2} - 2n_{1}^{2} + 2nn_{1}  \rangle ^{2b}}
\\
& \lesssim 
\sum_{n_{1} \in
\zz}  \frac{\langle n_1 \rangle ^{-2s} \langle n - n_{1} \rangle ^{-2s}}{\langle
n \rangle ^{-2s} \langle n_{1} \rangle
^{2a}} \times \frac{\chi_{A_{4,1}}}{\langle \tau - n^{2} - 2n_{1}^{2} + 2nn_{1}
\rangle ^{2b}}.
\end{split}
\end{equation}
%
%
We now need the following. 
%
%
%%%%%%%%%%%%%%%%%%%%%%%%%%%%%%%%%%%%%%%%%%%%%%%%%%%%%
%
%
%                Integer Bound
%
%
%%%%%%%%%%%%%%%%%%%%%%%%%%%%%%%%%%%%%%%%%%%%%%%%%%%%%
%
%
\begin{lemma}
  Let $n, n_1 \in \zz$ such that $n_{1} \neq 0$ and $n_{1} \neq n$.
  Then
  %
  %
  \begin{equation*}
  \begin{split}
    | n | \le | n - n_{1} | | n_{1} |.
  \end{split}
  \end{equation*}
  %
  %
\label{lem:integer-bound}
\end{lemma}
%
Hence,
%
\begin{equation}
  \label{growth-term-per}
\begin{split}
  \frac{\langle n \rangle ^{2s} \chi_{A_{4}}}{\langle n_{1} \rangle ^{2s} \langle n -
  n_{1} \rangle ^{2s}} \le \langle n_{1} \rangle ^{\gamma(s)},
  \quad 
  \gamma(s) = 
  \begin{cases} 0, \quad & s \ge 0
    \\
    4|s|, \quad & s < 0.
  \end{cases}
\end{split}
\end{equation}
%
%
%
%
Since $a \ge 0$, it follows from \eqref{growth-term-per} that 
%
\begin{equation}
  \label{growth-term-control-per}
  \frac{\langle n_1 \rangle ^{-2s} \langle n - n_{1} \rangle
  ^{-2s}\chi_{A_{4}}}{\langle
n \rangle ^{-2s} \langle n_{1} \rangle
^{2a}} \le 1, \quad s \ge -a/2
\end{equation}
%
%
which we use to bound the right hand side of \eqref{region-a41} by
%
%
\begin{equation*}
\begin{split}
\sum_{n_{1} \in
\zz} 
\frac{1}{\langle \tau - n^{2} - 2n_{1}^{2} + 2nn_{1}  \rangle ^{2b}}
\end{split}
\end{equation*}
%
%
%
which is finite for $b > 1/4$, due to the following lemma, which can be found in
Kenig, Ponce, and Vega
\cite{Kenig-Ponce-Vega-1996-Quadratic-forms-for-the-1-D-semilinear}.
\begin{lemma}
  \label{lem:sum-estimate}
If $\gamma>1/2$, then
\begin{equation}\label{CI2}
\sup_{(n,\tau)\in \zz \times \rr}\sum_{n_1\in \zz}\frac{1}{(1+|\tau\pm
n_1(n-n_1)|)^{\gamma}}<\infty. 
\end{equation}
\end{lemma}
%
Working now in region $A_{4,2}$, we seek to bound 
\begin{equation}
  \label{region-4-2}
\begin{split}
  &  \frac{1}{\langle n_{1} \rangle ^{2s}
  \langle \tau_{1} - n_{1}^{2} \rangle
  ^{2a}} \sum_{n \in \zz} \int_{\rr} \frac{\langle n \rangle ^{2s}}{\langle
  n - n_{1}\rangle ^{2s}}  \times \frac{\chi_{A_{4,2}}}{\langle
  \tau - n^{2} \rangle ^{2b} \langle \tau - \tau_{1} - (n - n_{1})^{2} \rangle ^{2b}} d \tau 
\end{split}
\end{equation}
%
%
Note that in region $A_{4,2}$
\begin{equation}
  \label{smoothing-per-4-2}
\begin{split}
  | \tau_{1} - n_{1}^{2} |
  & \ge \frac{1}{3}\left[ | \tau_{1} - n_{1}^{2} | + | \tau -
  \tau_{1} - (n - n_{1})^{2}
  | + | \tau - n^{2} | \right]
  \\
  & \ge \frac{1}{3} | - n_{1}^{2} - (n - n_{1})^{2} + n^{2} |
  \\
  & = \frac{2}{3} | n_{1} | | n - n_{1} |
  \\
  & \gtrsim | n_{1} |.
\end{split}
\end{equation}
Hence, applying
\autoref{lem:calc}, \eqref{growth-term-control-per}, and
\eqref{smoothing-per-4-2}, we bound \eqref{region-4-2} by
%
%
\begin{equation*}
\begin{split}
&  \frac{c}{\langle n_{1} \rangle ^{2s}
  \langle \tau_{1} - n_{1}^{2} \rangle
  ^{2a}} \sum_{n \in \zz} \frac{\langle n \rangle ^{2s}}{\langle
  n - n_{1}\rangle ^{2s}}  \times \frac{\chi_{A_{4,2}}}{\langle
  \tau_{1} - 2nn_{1} + n_{1}^{2} \rangle ^{2b}} 
  \\
  & \lesssim 
  \sum_{n \in \zz} \frac{\chi_{A_{4,2}}}{\langle
  \tau_{1} - 2nn_{1} + n_{1}^{2} \rangle ^{2b}},
  \quad  s \ge -a/2
  \end{split}
\end{equation*}
%
%
Since the right hand side is bounded for $b > 1/4$ by \autoref{lem:sum-estimate}, this
completes the proof for case \eqref{it-6}.
\subsubsection{Case \eqref{it-4}} 
\label{sssec:case-it-4}
Let 
%
%
\begin{align*}
B_1&=\{(n, n_1, \tau, \tau_1)\in B: n=0\},\\
B_2&=\{(n, n_1, \tau, \tau_1)\in B: n_1 = 0 \},\\
B_3&=\{(n, n_1, \tau, \tau_1)\in B: n \neq 0, n_1 \neq 0 \}.
\end{align*} 
%
%
We seek to bound

\begin{equation*}
\begin{split}
  \sum_{n_{1} \in \zz} \int_{\rr} \frac{\chi^{*}_{B_{3}}
    \langle n \rangle ^{2s}
    }{ \langle n_{1} \rangle^{2s} \langle
    n-n_{1} \rangle ^{2s} \langle \tau - n^{2}    \rangle ^{2a}
    \langle \tau - \lambda + n_{1}^{2} \rangle^{2b}
    \langle  \lambda + n_{1}^{2} \rangle^{2b}  } d \tau  
\end{split}
\end{equation*}
which is equal to
\begin{equation*}
\begin{split}
  \sum_{n_{1} \in \zz} \int_{\rr} \frac{1}
    { \langle n_{1} \rangle^{4s} \langle \tau    \rangle ^{2a}
    \langle \tau - \lambda + n_{1}^{2} \rangle^{2b}
    \langle  \lambda + n_{1}^{2} \rangle^{2b}  } d \tau  
\end{split}
\end{equation*}
which by \eqref{growth-term-per} and \autoref{lem:calc} is bounded by
%
%
\begin{equation*}
\begin{split}
  \sum_{n_{1} \in \zz} \frac{1}
  { \langle n_{1} \rangle^{4s} \langle n_{1}^{2} - \lambda   \rangle ^{2a}
  \langle n_{1}^{2} + \lambda \rangle^{2b}
     } d \tau  .
\end{split}
\end{equation*}
%
%
Since the minimum of the function $f(x) = (x-a)(x+a)$ occurs at $a=0$, the above
is bounded by
\begin{equation*}
\begin{split}
  & \sum_{n_{1} \in \zz} \frac{1}
  { \langle n_{1} \rangle^{4(s+a+b)}} 
  \\
  & < \infty, \qquad s > 1/4 -a -b
\end{split}
\end{equation*}

\begin{framed}
  \begin{remark}
    This result cannot be improved. For pedagogical purposes, we now estimate 
%
%
\begin{equation}
  \label{pathological-equality}
\begin{split}
   \langle \tau \rangle ^{-2a} \sum_{n_{1}} \langle
  n_{1}\rangle ^{-4s} \int_{\rr} \frac{\chi_{B_{1}}}{\langle \tau_{1} + n_{1}^{2} \rangle ^{2b}\langle
  \tau - \tau_{1} - n_{1}^{2}\rangle ^{2b}}d \tau_{1}.
\end{split}
\end{equation}
Applying \autoref{lem:calc}, we obtain the bound
%
%
\begin{equation*}
\begin{split}
  c  \langle \tau \rangle
  ^{-2a-2b} \sum_{n_{1} \in \zz} \langle n_{1} \rangle
  ^{-4s} 
  & \lesssim \sum_{n_{1} \in \zz} \langle n_{1} \rangle ^{-4s}
  \\
  & < \infty, \quad s > \frac{1}{4}.
\end{split}
\end{equation*}
%
%
Note that we cannot improve our lower bound for $s$ using this splitting.
To see this, we set $\tau = 0$
in the right hand side of \eqref{pathological-equality} and obtain
%
%
\newpage
%
\begin{equation*}
\begin{split}
   \sum_{n_{1}} \langle
  & n_{1}\rangle ^{-4s} \int_{\rr} \frac{1}{\langle \tau_{1} + n_{1}^{2} \rangle ^{2b}\langle
   \tau_{1} + n_{1}^{2}\rangle ^{2b}}d \tau_{1} 
   \\
   & = \sum_{n_{1}} \langle
  n_{1}\rangle ^{-4s} \int_{\rr} \frac{1}{\langle
   \tau' \rangle ^{4b}}d \tau'
   \\
   & \simeq \sum_{n_{1}} \langle n_{1} \rangle ^{-4s}, \quad b > 1/4
   \\
   & < \infty, \quad s > 1/4.
\end{split}
\end{equation*}
%
%
%
Hence, we now try to bound
\begin{equation}
\begin{split}
  & \frac{1}{\langle n_{1} \rangle ^{2s}
  \langle \tau_{1} + n_{1}^{2} \rangle
  ^{2a}} \sum_{n \in \zz } \int_{\rr} \frac{\langle n \rangle ^{2s}}{\langle
  n - n_{1}\rangle ^{2s}}  \times \frac{\chi_{B_{1}}}{\langle
  \tau - n^{2}  \rangle ^{2b} \langle \tau - \tau_{1} - (n - n_{1})^{2}
  \rangle ^{2b}} d \tau 
\end{split}
\end{equation}
instead, and look to obtain a better result. This is equal to 
%
\begin{equation*}
\begin{split}
  & \frac{1}{\langle n_{1} \rangle ^{4s}
  \langle \tau_{1} + n_{1}^{2} \rangle
  ^{2a}} \int_{\rr} \frac{1}{\langle
  \tau  \rangle ^{2b} \langle \tau - \tau_{1} - n_{1}^{2}
  \rangle ^{2b}} d \tau
\end{split}
\end{equation*}
%
%
which by \autoref{lem:calc} is bounded by
%
%
\begin{equation*}
\begin{split}
  & \frac{c}{\langle n_{1} \rangle ^{4s}
  \langle \tau_{1} + n_{1}^{2} \rangle
  ^{2a + 2b}}
\\
& < \infty, \qquad s \ge 0. 
\end{split}
\end{equation*}
%
%
We remark that the case $n=0$ is trivial to estimate for the Boussinesq. This is
because  $$\frac{n^{2}}{n^{2} + n^{4}} |_{n=0} = 0$$ while
$$\frac{n^{2}}{n^{2}} |_{n=0} = 1.$$ 
Hence, the existence of the extra $n = 0$ term 
for the $B_{4}$ equation requires some extra work to be done, compared to the
Boussinesq equation. 
\end{remark}
\end{framed}
Applying \autoref{lem:calc}, we now bound 
\begin{equation*}
  \begin{split}
    & \frac{ \langle n
    \rangle ^{2s}}{\langle \tau - n^{2} \rangle ^{2a}}
    \sum_{n_{1} \in \zz} \int_{\rr} \frac{\chi_{B_{2}}}{ \langle n_{1} \rangle ^{2s} \langle n-n_{1} \rangle ^{2s} 
    \langle \tau_{1} + n_{1}^{2} \rangle^{2b} \langle  \tau - \tau_{1} -
    (n - n_{1})^{2} \rangle^{2b} }
    d \tau_1 
    \\
    & = 
   \langle \tau -n^{2} \rangle ^{-2a}\int_{\rr} \frac{1}{\langle \tau_{1} +
  n^{2} \rangle ^{2b}\langle
  \tau - \tau_{1}\rangle ^{2b}}d \tau_{1}
  \\
  & \lesssim 
  \langle \tau - n^{2} \rangle ^{-2a-2b} 
  \\
  & < \infty.
\end{split}
\end{equation*}
%
%
We now 
partition $ B_{3}$ into three parts
\begin{align*}
B_{3,1}&=\{(n, n_1, \tau, \tau_1)\in B_3:
|\tau-\tau_1-(n-n_1)^2|, |\tau_1+n_1^2| \le |\tau-n^2|\},\\
B_{3,2}&=\{(n, n_1, \tau, \tau_1)\in B_3:
|\tau-\tau_1-(n-n_1)^2|, |\tau-n^2| \le |\tau_1+n_1^2|\},\\
B_{3,3}&=\{(n, n_1, \tau, \tau_1)\in B_3: |\tau_{1}+n_{1}^2|, | \tau - n^{2} | \le |  \tau - \tau_{1} -
(n - n_{1})^{2} |\}.
\end{align*} 
Then in region $B_{3,1}$
\begin{equation}
\begin{split}
  | \tau - n^{2} |
  & \ge \frac{1}{3}\left[ | \tau_{1} + n_{1}^{2} | + | \tau -
  \tau_{1} - (n - n_{1})^{2}
  | + | \tau - n^{2} | \right]
  \\
  & \ge \frac{1}{3} |  n_{1}^{2} - (n - n_{1})^{2} + n^{2} |
  \\
  & = \frac{2}{3} | n_{1} | | n |
  \\
  & \gtrsim | n_{1} |.
\end{split}
\label{smoothing-per-3-1-case-6}
\end{equation}
%
%
Estimating first in region
$B_{3,1}$, we apply \autoref{lem:calc} and \eqref{smoothing-per-3-1-case-6}
to obtain
%
%
%
%
\begin{equation}
  \label{region-a31-case-6}
\begin{split}
& \langle \tau - n^{2}  \rangle ^{-2a} \langle n
    \rangle ^{2s}
    \sum_{n_{1} \in \zz} \int_{\rr} \frac{\chi_{B_{3,1}}}{ \langle n_{1} \rangle ^{2s} \langle n-n_{1} \rangle ^{2s} 
\langle \tau_{1} - n_{1}^{2}  \rangle \langle  \tau - \tau_{1} - (n -
n_{1})^{2}  \rangle}
d \tau_1 
\\
& \lesssim \langle \tau - n^{2} \rangle ^{-2a} \langle n \rangle ^{2s}
\sum_{n_{1} \in
\zz}  \frac{\chi_{B_{3,1}}}{\langle n_{1} \rangle ^{2s} \langle n - n_{1} \rangle
^{2s} \langle \tau - n^{2} - 2n_{1}^{2} + 2nn_{1}  \rangle ^{2b}}
\\
& \lesssim 
\sum_{n_{1} \in
\zz}  \frac{\langle n_1 \rangle ^{-2s} \langle n - n_{1} \rangle ^{-2s}}{\langle
n \rangle ^{-2s} \langle n_{1} \rangle
^{2a}} \times \frac{\chi_{B_{3,1}}}{\langle \tau - n^{2} - 2n_{1}^{2} + 2nn_{1}
\rangle ^{2b}}.
\end{split}
\end{equation}
%
%
Since $a \ge 0$, it follows from \eqref{growth-term-control-per} that the right
hand side is bounded by
%
%
%
%
\begin{equation*}
\begin{split}
\sum_{n_{1} \in
\zz} \frac{\chi_{B_{3,1}}}{\langle \tau - n^{2} - 2n_{1}^{2} + 2nn_{1}
\rangle ^{2b}}
\end{split}
\end{equation*}
%
%
%
%
%
%
which is finite for $b > 1/4$, due to \autoref{lem:sum-estimate}. 
Working now in region $B_{3,2}$ (ATTENTION: Farah estimates differently here. His
way is convoluted, and unnecessary; see page 960 of Farah periodic), we seek to estimate 
\begin{equation}
  \label{region-B-3-split-3}
\begin{split}
  &  \frac{1}{\langle n_{1} \rangle ^{2s}
  \langle \tau_{1} + n_{1}^{2} \rangle
  ^{2a}} \sum_{n \in \zz} \int_{\rr} \frac{\langle n \rangle ^{2s}}{\langle
  n - n_{1}\rangle ^{2s}}  \times \frac{\chi_{B_{3,2}}}{\langle
  \tau - n^{2} \rangle ^{2b} \langle \tau - \tau_{1} - (n - n_{1})^{2} \rangle
  ^{2b}} d \tau.
\end{split}
\end{equation}
%
Note that in region $B_{3,2}$
\begin{equation}
  \label{smoothing-per-3-2-case-6}
\begin{split}
  | \tau_{1} + n_{1}^{2} |
  & \ge \frac{1}{3}\left[ | \tau_{1} + n_{1}^{2} | + | \tau -
  \tau_{1} - (n - n_{1})^{2}
  | + | \tau - n^{2} | \right]
  \\
  & \ge \frac{1}{3} |  n_{1}^{2} - (n - n_{1})^{2} + n^{2} |
  \\
  & = \frac{2}{3} | n_{1} | | n |
  \\
  & \gtrsim | n_{1} |.
\end{split}
\end{equation}
%
Hence, applying
\autoref{lem:calc}, \eqref{growth-term-control-per}, and
\eqref{smoothing-per-3-2-case-6} to \eqref{region-B-3-split-3}, we obtain the bound
%
%
\begin{equation*}
  \begin{split}
    &  \frac{c}{\langle n_{1} \rangle ^{2s}
    \langle \tau_{1} + n_{1}^{2} \rangle
    ^{2a}} \sum_{n \in \zz} \frac{\langle n \rangle ^{2s}}{\langle
    n - n_{1}\rangle ^{2s}}  \times \frac{\chi_{B_{3,2}}}{\langle
    \tau_{1} - 2nn_{1} + n_{1}^{2} \rangle ^{2b}} 
    \\
    & \lesssim 
    \sum_{n \in \zz} \frac{\chi_{B_{3,2}}}{\langle
    \tau_{1} - 2nn_{1} + n_{1}^{2} \rangle ^{2b}},
    \quad s \ge -a/2.
  \end{split}
\end{equation*}
%
%
%
It remains to show that 
%
%
%
\begin{equation}
  \label{sum-bound}
\begin{split}
\sum_{n_{1} \in
\zz} \frac{\chi_{B_{3,2}}}{\langle \tau - n^{2} + 2nn_{1}
\rangle ^{2b}} < c, \qquad b > 1/2
\end{split}
\end{equation}
%
%
where $c$ does not depend on the choice of $\tau$ or $n$. 
%
%
To see this, note that if $\tau - n^{2} = 0$, the above follows easily, since
$n \neq 0$ in $B_{3,2}$.
Otherwise, we use the fact that $| \tau - n^{2} | \gtrsim | n_{1} |$ and
$n \neq 0$ in $B_{3,2}$ to obtain 
%
%
\begin{equation*}
\begin{split}
\sum_{n_{1} \in
\zz} \frac{\chi_{B_{3,2}}}{\langle \tau - n^{2} + 2nn_{1}
\rangle ^{2b}}
& = \sum_{n_{1} \in \zz} \frac{\chi_{B_{3,2}}}{(1 + | \tau - n^{2} +
2nn_{1} |)^{2b}}
\\
& \le \sum_{n_{1} \in \zz} \frac{\chi_{B_{3,2}}}{1 + | \tau - n^{2} +
2nn_{1} |^{2b}}, \quad b \ge 1/2
\\
& \le \sum_{n_{1} \in \zz} \frac{\chi_{B_{3,2}}}{1 + | \tau - n^{2}
|^{2b} | 1 + \frac{2nn_{1}}{\tau - n^{2}} |^{2b}}
\\
& \lesssim \sum_{n_{1} \in \zz} \frac{\chi_{B_{3,2}}}{1 + |n_{1}|^{2b}
| 1 + \frac{2nn_{1}}{\tau - n^{2}} |^{2b}}
\\
& \le \sup_{a \in \rr} \sum_{n_{1} \in \zz} \frac{\chi_{B_{3,2}}}{1 + |n_{1}|^{2b}
| 1 + an_{1}|^{2b} }
\\
& = \sum_{n_{1} \in \zz} \frac{\chi_{B_{3,2}}}{1 + |n_{1}|^{2b}}
\\
& < c, \qquad b > 1/2.
\end{split}
\end{equation*}
%
%
It remains to handle region $B_{3,3}$. It will be enough to bound
%
%
\begin{equation}
  \label{region-B-3-star-split}
\begin{split}
   \sum_{n_{1} \in \zz} \int_{\rr} \frac{\chi^{*}_{B_{3,3}}
    \langle n \rangle ^{2s}
    }{ \langle n_{1} \rangle^{2s} \langle  \tau  - n^{2}
    \rangle ^{2a}  \langle
n-n_{1} \rangle ^{2s}  \langle  \tau - \lambda+n_{1}^{2}
\rangle^{2b} \langle   \lambda  -(n - n_{1})^{2}
\rangle^{2b} } d \tau.
\end{split}
\end{equation}
%
Due to the presence of $\chi^{*}_{B_{3,3}}$ factor, we have the restriction
%
%
\begin{equation*}
\begin{split}
& |\tau - \lambda +n_{1}^2|, | \tau - n^{2} | \le |  \lambda -
(n - n_{1})^{2} | \text{ and }  n \neq 0, n_1 \neq 0.
\end{split}
\end{equation*}
%
It follows that
\begin{equation}
  \label{smoothing-per-3-3-case-6}
\begin{split}
  | \lambda - (n - n_{1})^{2} |
  & \ge \frac{1}{3}\left[ | \tau - \lambda + n_{1}^{2} | + | \lambda - (n - n_{1})^{2}
  | + | \tau - n^{2} | \right]
  \\
  & \ge \frac{1}{3} |  n_{1}^{2} - (n - n_{1})^{2} + n^{2} |
  \\
  & = \frac{2}{3} | n_{1} | | n |
  \\
  & \gtrsim | n_{1} |.
\end{split}
\end{equation}
Hence, applying
\eqref{growth-term-control-per}, \autoref{lem:calc}, and
\eqref{smoothing-per-3-3-case-6}, we bound \eqref{region-B-3-star-split} by
%
%
\begin{equation*}
\begin{split}
   & \sum_{n_{1} \in \zz} \int_{\rr} \frac{\chi^{*}_{B_{3,3}}
    }{ \langle  \tau  - n^{2}
    \rangle ^{2a}   \langle  \tau - \lambda+n_{1}^{2}
\rangle^{2b} } d \tau, \quad s \ge -a/2.
\\
& \lesssim  \sum_{n_{1} \in \zz} \frac{\chi_{B_{3,3}^{*}}}{\langle n_{1}^{2} +
n^{2} - \lambda \rangle^{2a }}
\end{split}
\end{equation*}
which is bounded for $a > 1/4$ by
\autoref{lem:sum-estimate}. This completes the proof of
\autoref{prop:bilin-est}. \qquad \qedsymbol
%
%
\subsection{The Non-Periodic Case} 
\label{ssec:non-periodic-case}
We now introduce the following spaces. 
%
%
\begin{definition}
  Let $S(\rr^{2})$ denote the space of Schwartz functions on
  $\rr^{2}$.  For $s, b \in \rr$, $\mathcal{X}_{s,b}$
  denotes the completion of $S(\rr^{2})$ with
  respect to the norm
  %
  %
  \begin{equation}
  \begin{split}
    \|F\|_{\mathcal{X}_{s,b}} = \left( \sum_{n \in \zz} (1 + \xi^{2})^{s} \int_{\rr}
    (1 + | | \tau | - \xi^{2} |)^{2b} \wh{F}(n, \tau) d \tau\right)^{1/2}.
  \end{split}
  \label{eqn:bous-norm-real}
  \end{equation}
  %
  %
  %
  %
\end{definition}
%
%
We need only establish the following bilinear estimate. All other arguments are
analogous to those in the periodic case.
%
\begin{proposition}[Theorem 1.1 in Farah periodic]
\label{prop:bilin-est-real}
If $b > 1/2$, $a > 1/4$, and $s \ge -a/2$, 
  then there exists $c > 0$ depending only on $a$, $b$, and $s$ such that
  %
  %
  \begin{equation*}
  \begin{split}
    \| uv \|_{\mathcal{X}_{s,-a}} \le c \| u \|_{\mathcal{X}_{s,b}} \| v \|_{\mathcal{X}_{s,b}}.
  \end{split}
  \end{equation*}
  %
  %
\end{proposition}


\subsubsection{Proof of \autoref{prop:bilin-est-real}.} 
\label{sssec:bilin-est-real}
Since $\| f \|_{\mathcal{X}_{s,-a}} \le \| f \|_{\mathcal{X}_{s, -a'}}$ for $a \ge a' \ge 0$, we assume
without loss of generality that $1/4 > a \le b$ and $b > 1/2$ throughout. 
By duality, it suffices to show that 
%
%%
\begin{equation}
	\label{duality-est-real}
	\begin{split}
    |	\int_{\rr} \int_{\rr} (1 + |\xi|)^{s}
		\phi(\xi, \tau) \wh{uv}(\xi, \tau)(1 
    + | |\tau| - \xi^{2} |^{-a}) d \tau d \xi | \lesssim \|u\|_{\mathcal{X}_{s,b}}
    \|v\|_{\mathcal{X}_{s,b}}
    \|\phi \|_{L^{2}_{\xi, \tau}}.
	\end{split}
\end{equation}
Note first that $|\wh{uv}(\xi, \tau) |  = | \wh{u} *  \wh{v} 
(\xi, \tau)|$. From this it follows that
%
%
\begin{equation}
	\label{non-lin-rep-real}
	\begin{split}
		| \wh{uv}(\xi, \tau)|
    & = | \sum_{\xi_{1} \in \zz }  \int
    \wh{u}\left( \xi_1,  \tau_1 \right) \wh{v}\left( \xi - \xi_1 , \tau - \tau_1   
\right) d \tau_1 |
\\
& \le  \sum_{\xi_{1} \in \zz }  \int
    |\wh{u}\left( \xi_1,  \tau_1 \right)| |\wh{v}\left( \xi - \xi_1 , \tau - \tau_1   
\right)| d \tau_1 
\\
& = \sum_{\xi_1 \in \zz } \int \frac{c_u\left( \xi_1, \tau_1 
\right)}{\langle \xi_1 \rangle ^s \langle |\tau_1| - \xi_1^{2} | \rangle ^{b}}
\\
& \times \frac{c_{v}\left( \xi - \xi_1, \tau - \tau_1 \right)}{\langle \xi -
\xi_1 \rangle ^s\ \langle |\tau - \tau_1 | -  (\xi - \xi_1)^{2} \rangle^{b}}
  \ d \tau_1 
\end{split}
\end{equation}
%
%
where for clarity of notation we have introduced 
%
%
%
\begin{equation*}
\begin{split}
\langle k \rangle \doteq 1 + |k|
\end{split}
\end{equation*}
%
%
and
%
\begin{equation*}
	\begin{split}
		c_h(\xi, \tau) =
			\langle \xi \rangle ^s \langle |\tau| - \xi^{2} \rangle ^{b} | \wh{h}\left( \xi, \tau \right) |.
	\end{split}
\end{equation*}
%
%
From our work above, it follows that 
%
%
\begin{equation}
	\label{convo-est-starting-pnt-real}
	\begin{split}
		 & \langle \xi \rangle^s \langle \tau - \xi^{2} \rangle^{-a} | \wh{uv}\left( 
		\xi, \tau \right) |
		\\
		& \le \langle |\tau| - \xi^{2} \rangle^{-a}
		\sum_{\xi_1 \in \zz} \int \frac{\langle \xi \rangle^{s}}{\langle \xi_1 \rangle^s
    \langle \xi - \xi_1 \rangle^s} 
		\times \frac{c_f(\xi_1, \tau_1)}{\langle |\tau_1| - \xi_1^{2} \rangle ^{b}}
		\\
		& \times
		\frac{c_g(\xi - \xi_1, \tau - \tau_1 )}{\langle |\tau - \tau_1| - (\xi - \xi_1)^{2}
    \rangle^{b}}\ d \tau_1.
	\end{split}
\end{equation}
%
%
Hence, 
%
%
\begin{equation}
  \label{pre-fubini-int-form-real}
	\begin{split}
    |\text{lhs of} \ \eqref{duality-est-real}|
    & \lesssim \int_{\rr} \int_{\rr}     \int_{\rr}  \int_{\rr} \phi(\xi, \tau)
    c_f(\xi_1, \tau_1)
		c_g(\xi - \xi_1, \tau - \tau_1 )
		\\
    & \times \frac{\langle \xi \rangle ^{s}}{\langle \xi_{1} \rangle ^{s} \langle
    \xi-\xi_{1} \rangle ^{s}} \times \frac{1}{ \langle \tau - \xi^{2} \rangle^{a}
\langle |\tau| - \xi^{2} \rangle
    ^{b}\langle |\tau_{1}|-\xi_{1}^{2} \rangle ^{-b}\langle | \tau|-\xi_{2}^{2}
    \rangle ^{b}} d \tau_1 d \xi_{1} d \tau d \xi.
	\end{split}
\end{equation}
%
%Let $A \subset \rr^{4}$, and $\chi_{A}(\xi, \tau, \xi_{1}, \tau_{1})$ be its
%characteristic function. Then by Cauchy-Schwartz in
%$\tau_{1}, \xi_{1}$, we bound
%%
%%
%%
%\begin{equation*}
%\begin{split}
  %& \int_{\rr} \int_{\rr}     \int_{\rr}  \int_{\rr} \chi_{A} \phi(\xi, \tau)
    %c_f(\xi_1, \tau_1)
		%c_g(\xi - \xi_1, \tau - \tau_1 )
		%\\
    %& \times \frac{\langle \xi \rangle ^{s}}{\langle \xi_{1} \rangle ^{s} \langle
    %\xi-\xi_{1} \rangle ^{s}} \times \frac{1}{ \langle \tau - \xi^{2} \rangle^{a}
%\langle |\tau| - \xi^{2} \rangle
    %^{b}\langle |\tau_{1}|-\xi_{1}^{2} \rangle ^{-b}\langle | \tau|-\xi_{2}^{2}
    %\rangle ^{b}} d \tau_1 d \xi_{1} d \tau d \xi
%\end{split}
%\end{equation*}
%%
%%
%by
%%
%%
%\begin{equation}
	%\label{10g-real}
	%\begin{split}
    %& \int_{\rr} \int_{\rr} \phi(\xi, \tau) \langle | \tau | - \xi^{2} \rangle
    %^{-a} \langle \xi \rangle ^{s}
    %\\
    %& \times \left( \int_{\rr} \int_{\rr}
    %\frac{\chi_{A}}{\langle \xi_{1} \rangle ^{2s} \langle \xi-\xi_{1} \rangle ^{2s} \langle |
    %\tau_{1} | - \xi_{1}^{2}\rangle ^{2b} \langle | \tau - \tau_{1} | -
    %(\xi - \xi_{1})^{2} \rangle ^{2b}} d \tau_{1} d \xi_{1} \right)^{1/2}
    %\\
    %& \times \left( \int_{\rr} \int_{\rr} c_{u}^{2}(\xi, \tau_{1})
    %c_{v}^{2}(\xi - \xi_{1}, \tau - \tau_{1}) d \tau_{1} d \xi_{1}
    %\right)^{1/2} d \tau d
    %\xi.
  %\end{split}
%\end{equation}
%%
%%
%Applying Cauchy-Schwartz in $\tau, \xi$, \eqref{10g-real} is bounded by
%%
%%
%\begin{equation*}
  %\begin{split}
    %& \|\left( \int_{\rr} \int_{\rr } c_{u}^{2}(\xi_1, \tau_1)
  %c_{v}^{2} (\xi - \xi_1, \tau - \tau_{1} ) d \tau_1 d \xi_{1}  \right)^{1/2} \|_{L^{2}(\zz \times
		%\rr)}
		%\\
    %& \times  \|\phi(\xi, \tau) \langle | \tau | - \xi^{2} \rangle ^{-a} \langle \xi
    %\rangle ^{s}
		%\\
    %& \times \left( \int_{\rr} \int_{\rr} \frac{\chi_{A}}{ \langle \xi_{1}
    %\rangle ^{2s} \langle \xi-\xi_{1} \rangle ^{2s} \langle | \tau_{1}|-\xi_{1}^{2}
    %\rangle^{2b} \langle  |\tau -
    %\tau_{1} | -(\xi - \xi_{1})^{2}
    %\rangle^{2b} } d \tau_1 d \xi_{1} \right)^{1/2} \|_{L^2(\zz \times \rr)}
		%\\
    %& = \|u\|_{\mathcal{X}_{s,b}} \|v\|_{\mathcal{X}_{s,b}} \label{holder-term-real}
     %\|\phi(\xi, \tau)     \\
    %& \times \left( \langle | \tau | - \xi^{2} \rangle ^{-2a} \langle \xi
    %\rangle ^{2s}
    %\int_{\rr} \int_{\rr} \frac{\chi_{A}}{ \langle \xi_{1} \rangle ^{2s} \langle
%\xi-\xi_{1} \rangle ^{2s}  \langle | \tau_{1}|-\xi_{1}^{2} \rangle^{2b} \langle  |\tau -
    %\tau_{1} | -(\xi - \xi_{1})^{2}
    %\rangle^{2b} } d \tau_1 d \xi_{1} \right)^{1/2} \|_{L^2(\zz \times \rr)}.
  %\end{split}
%\end{equation*}
%%
%Applying H{\"o}lder, we bound this by 
%%
%%
%\begin{equation}
  %\label{integral-bound-1st-form}
	%\begin{split}
    %& \|u\|_{\mathcal{X}_{s,b}} \|v\|_{\mathcal{X}_{s,b}} \| \phi \|_{L^{2}_{\xi, \tau}}
    %\\
    %& \times \|\left( \langle | \tau | - \xi^{2} \rangle ^{-2a} \langle \xi
    %\rangle ^{2s}
    %\sum_{n_{1} \in \zz} \int_{\rr} \frac{\chi_{A}}{ \langle \xi_{1} \rangle ^{2s} \langle
%\xi-\xi_{1} \rangle ^{2s} \langle | \tau_{1}|-\xi_{1}^{2} \rangle^{2b} \langle  |\tau -
    %\tau_{1} | -(\xi - \xi_{1})^{2}
    %\rangle ^{2b} } d \tau_1 \right)^{1/2} \|_{L^\infty_{\xi, \tau}}.
	%\end{split}
%\end{equation}
%%
%%
%%Hence, to complete the proof, it will be enough
%%to show that 
%%%
%%%
%%%
%%%
%%\begin{equation}
  %%\label{key-sup-estimate-real-1}
	%%\begin{split}
		 %%\| \langle | \tau | - \xi^{2} \rangle ^{-2a} \langle \xi
    %%\rangle ^{2s}
%%\sum_{\xi_{1} \in \zz} \int_{\rr} \frac{1}{  \langle \xi_{1} \rangle ^{2s} \langle
%%\xi-\xi_{1} \rangle ^{2s} \langle | \tau_{1}|-\xi_{1}^{2} \rangle^{2b}  \langle  |\tau -
    %%\tau_{1} | -(\xi - \xi_{1}^{2}
    %%\rangle ^{2b} } d \tau_1 \|_{L^\infty_{\xi, \tau}} < \infty.
	%%\end{split}
%%\end{equation}
%%
%%
%%By the triangle inequality and the fact that 
%%%
%%%
%%\begin{equation*}
%%\begin{split}
%%& | \tau | =
%%\begin{cases}
  %%- \tau, \quad & \tau < 0, 
%%\\
%%\tau, \quad & \tau > 0
%%\end{cases}
%%\end{split}
%%\end{equation*}
%%%
%%%
%%\eqref{key-sup-estimate-real} will be proved if we can bound the
%%$L^{\infty}_{\tau, n}$ norm of the quantity
%%%
%%%
%%\begin{equation}
  %%\label{sup-est-gen-real}
%%\begin{split}
			%%\langle \sigma \rangle ^{-2a} \langle n
    %%\rangle ^{2s}
%%\sum_{n_{1} \in \zz} \int_{\rr} \frac{1}{ \langle n_{1} \rangle ^{2s} \langle n-n_{1} \rangle ^{2s} 
%%\langle \sigma_{1} \rangle^{2b} \langle  \sigma_{2} \rangle^{2b} }
%%d \tau_1 
	%%\end{split}
%%\end{equation}
%Let us now return to the right hand side of \eqref{pre-fubini-int-form-real}.
%Let $A \subset \rr^{4}$, and $\chi_{A}(\xi, \tau, \xi_{1}, \tau_{1})$ be its
%characteristic function, as before.  We seek to bound
%\begin{equation*}
%\begin{split}
  %& \int_{\rr} \int_{\rr}     \int_{\rr}  \int_{\rr} \chi_{A} \phi(\xi, \tau)
    %c_f(\xi_1, \tau_1)
		%c_g(\xi - \xi_1, \tau - \tau_1 )
		%\\
    %& \times \frac{\langle \xi \rangle ^{s}}{\langle \xi_{1} \rangle ^{s} \langle
    %\xi-\xi_{1} \rangle ^{s}} \times \frac{1}{ \langle \tau - \xi^{2} \rangle^{a}
%\langle |\tau| - \xi^{2} \rangle
    %^{b}\langle |\tau_{1}|-\xi_{1}^{2} \rangle ^{-b}\langle | \tau|-\xi_{2}^{2}
    %\rangle ^{b}} d \tau_1 d \xi_{1} d \tau d \xi
%\end{split}
%\end{equation*}
%in a different manner than before. First, we apply 
%Fubini, then Cauchy-Schwartz in $\xi_{1}, \tau_{1}$ to obtain the bound
%%
%%
%\begin{equation*}
%\begin{split}
  %& \left[ \int_{\rr} \int_{\rr} c_{f}^{2}(\xi_{1}, \tau_{1}) d \tau_{1} d
  %\xi_{1} \right]^{1/2}
  %\\
  %& \times \left \{ \int_{\rr} \int_{\rr}   
 %\left[
  %\int_{\rr} \int_{\rr}
   %\frac{\langle \xi \rangle ^{s}}{\langle \xi_{1} \rangle ^{s} \langle
   %\xi - \xi_{1}\rangle ^{s}} \times \frac{|\phi(\xi, \tau)| c_{g}(\xi -
   %\xi_{1}, \tau - \tau_{1})
%}{\langle | \tau | - \xi^{2} \rangle
  %^{a} \langle | \tau_{1} | - \xi_{1}^{2} \rangle ^{b} \langle | \tau -
  %\tau_{1} | - (\xi - \xi_{1}^{2}) \rangle ^{b}} d \tau d \xi 
  %\right]^{2} \right \}^{1/2}
  %\\
  %& = \| f \|_{X_{s,b}}
  %\\
  %& \times \left \{ \int_{\rr} \int_{\rr}   
 %\left[
  %\int_{\rr} \int_{\rr}
   %\frac{\langle \xi \rangle ^{s}}{\langle \xi_{1} \rangle ^{s} \langle
   %\xi - \xi_{1}\rangle ^{s}} \times \frac{|\phi(\xi, \tau)| c_{g}(\xi -
   %\xi_{1}, \tau - \tau_{1})
%}{\langle | \tau | - \xi^{2} \rangle
  %^{a} \langle | \tau_{1} | - \xi_{1}^{2} \rangle ^{b} \langle | \tau -
  %\tau_{1} | - (\xi - \xi_{1}^{2}) \rangle ^{b}} d \tau d \xi 
  %\right]^{2} d \tau_{1} d \xi_{1} \right \}^{1/2}
%\end{split}
%\end{equation*}
%%
%Applying Cauchy-Schwartz in $\tau, \xi$, we bound the last line by 
%%
%%
%\begin{equation*}
%\begin{split}
%& \left \{ \int_{\rr} \int_{\rr}   
  %\left [ \int_{\rr} \int_{\rr}
  %| \phi(\xi, \tau)|^{2} c_{g}^{2}(\xi - \xi_{1}, \tau - \tau_{1}) d \tau d \xi 
    %\right ] \right . 
   %\\
   %& \left. \times \left [ \int_{\rr} \int_{\rr} \frac{\langle \xi \rangle
   %^{2s}}{\langle \xi_{1} \rangle ^{2s} \langle \xi - \xi_{1}\rangle ^{2s}}
   %\times \frac{\chi_{A}}{\langle | \tau | - \xi^{2} \rangle ^{2a} \langle | \tau_{1} |
   %- \xi_{1}^{2} \rangle ^{2b} \langle | \tau - \tau_{1} | - (\xi - \xi_{1}^{2})
   %\rangle ^{2b}} d \tau d \xi \right ] \right \}^{1/2}d \tau_{1} d \xi_{1}
%\end{split}
%\end{equation*}
%%
%%
%which by Holder is bounded by 
%%
%%
%%
%\begin{equation}
  %\label{integral-bound-2nd-form}
%\begin{split}
  %& \| \int_{\rr} \int_{\rr} \frac{\langle \xi \rangle ^{2s}}{\langle \xi_{1} \rangle ^{2s} \langle
  %\xi - \xi_{1}\rangle ^{2s}}  \times \frac{\chi_{A}}{\langle | \tau | - \xi^{2} \rangle
  %^{2a} \langle | \tau_{1} | - \xi_{1}^{2} \rangle ^{2b} \langle | \tau -
  %\tau_{1} | - (\xi - \xi_{1}^{2}) \rangle ^{2b}} d \tau d \xi
  %\|_{L^{\infty}_{\xi_{1}, \tau_{1}}}^{1/2}
  %\\
  %& \times \|\phi\|_{L^{2}} \| g \|_{X_{s,b}}
%\end{split}
%\end{equation}
%
%
Now consider the family $\{A_{j}\}_{1}^{k}, A_{j} \subset \rr^{4}$ with
$$\bigcup_{1}^{k} A_{j}= \rr^{4}.$$ 
As in the periodic case, a combination of Cauchy-Schwartz, Fubini, and
H{\"o}lder to estimate the right hand side of \eqref{pre-fubini-int-form-real}
reduces the proof of \autoref{prop:bilin-est-real} to showing that

%From \eqref{integral-bound-1st-form},
%\eqref{integral-bound-2nd-form}, and our preceding argumentation,
%we see that the proof of \autoref{prop:bilin-est-real} reduces to showing that
%either 
%
%
%
%
\begin{equation}
  \label{key-sup-estimate-real}
  \begin{split}
     \| \langle  \frac{\langle \xi
     \rangle ^{2s}}{ | \tau | - \xi^{2} \rangle ^{2a}}\int_{\rr} \int_{\rr} \frac{\chi_{A_{j}}}{ \langle \xi_{1} \rangle ^{2s} \langle
\xi-\xi_{1} \rangle ^{2s}
\langle | \tau_{1}|-\xi_{1}^{2} \rangle^{2b}  \langle  |\tau -
    \tau_{1} | -(\xi - \xi_{1})^{2}
    \rangle ^{2b} } d \tau_1 d \xi_{1} \|_{L^\infty_{\xi, \tau}} < \infty.
  \end{split}
\end{equation}
%
or
%%
\begin{equation}
\begin{split}
  & \| \frac{1}{\langle \xi_{1} \rangle ^{2s}
  \langle | \tau_{1} | - \xi_{1}^{2} \rangle
  ^{2a}} \int_{\rr} \int_{\rr} \frac{\langle \xi \rangle ^{2s}}{\langle
  \xi - \xi_{1}\rangle ^{2s}}  \times \frac{\chi_{A_{j}}}{\langle | \tau | - \xi^{2} \rangle ^{2b} \langle | \tau -
  \tau_{1} | - (\xi - \xi_{1}^{2}) \rangle ^{2b}} d \tau d \xi
  \|_{L^{\infty}_{\xi_{1}, \tau_{1}}} < \infty
\end{split}
\end{equation}
or
\begin{equation}
\begin{split}
  \| \int_{\rr} \int_{\rr} \frac{\chi_{A}
    \langle \xi \rangle ^{2s}
    }{ \langle \xi_{1} \rangle^{2s} \langle | \tau | - \xi^{2}
    \rangle ^{2a}  \langle
\xi-\xi_{1} \rangle ^{2s}  \langle | \tau - \lambda|-\xi_{1}^{2}
\rangle^{2b} \langle  | \lambda | -(\xi - \xi_{1})^{2}
\rangle^{2b} } d \tau d \xi_{1} \|_{L^{\infty}_{\xi, \lambda}} < \infty
\end{split}
\end{equation}

for each $j \in \left\{ 0,1,\dots,k \right\}$. 
By the triangle inequality and the fact that 
%
%
\begin{equation*}
\begin{split}
& | \tau | =
\begin{cases}
  - \tau, \quad & \tau < 0, 
\\
\tau, \quad & \tau > 0
\end{cases}
\end{split}
\end{equation*}
%
%
it follows that the proof of \autoref{prop:bilin-est} reduces to showing that
for any $j$, either 
%
%
\begin{equation}
  \label{sup-est-gen-1}
  \begin{split}
    \| \frac{ \langle \xi
    \rangle ^{2s}}{\langle \sigma \rangle ^{2a}}
    \int_{\rr} \int_{\rr} \frac{\chi_{A_{j}}}{ \langle \xi_{1} \rangle ^{2s} \langle \xi-\xi_{1} \rangle ^{2s} 
    \langle \sigma_{1} \rangle^{2b} \langle  \sigma_{2} \rangle^{2b} }
    d \tau_1 d \xi_{1} \|_{L^{\infty}_{\xi, \tau}} < \infty
  \end{split}
\end{equation}
%
%
or 
\begin{equation}
  \label{sup-est-gen-2}
\begin{split}
  & \| \frac{1}{\langle \xi_{1} \rangle ^{2s}
  \langle \sigma_{1} \rangle
  ^{2a}} \int_{\rr} \int_{\rr} \frac{\langle \xi \rangle ^{2s}}{\langle
  \xi - \xi_{1}\rangle ^{2s}}  \times \frac{\chi_{A_{j}}}{\langle
  \sigma \rangle ^{2b} \langle \sigma_{2} \rangle ^{2b}} d \tau d \xi
  \|_{L^{\infty}_{\xi_{1}, \tau_{1}}} < \infty
\end{split}
\end{equation}
%
or
\begin{equation}
  \label{sup-est-gen-3}
\begin{split}
  \| \int_{\rr}  \int_{\rr} \frac{\chi^{*}_{A_{j}}
    \langle \xi \rangle ^{2s}
    }{ \langle \xi_{1} \rangle^{2s} \langle
    \xi-\xi_{1} \rangle ^{2s} \langle \sigma^{*}  
    \rangle ^{2a}
    \langle \sigma_{1}^{*} \rangle^{2b}
    \langle  \sigma_{2}^{*} \rangle^{2b}  } d \tau d \xi_{1}  \|_{L^{\infty}_{\xi, \lambda}}
    < \infty
\end{split}
\end{equation}
%
for the following cases.
\begin{enumerate}[(i)]
    \item $ \sigma=\tau+\xi^2,\quad \sigma_1=\tau_1+\xi_1^2,\quad \sigma_2=\tau -
      \tau_1+(\xi - \xi_1)^2$,
\label{it-real-1}
    \item $ \sigma=\tau-\xi^2,\quad \sigma_1=\tau_1-\xi_1^2,\quad \sigma_2=\tau - \tau_1+(\xi - \xi_1)^2$,
\label{it-real-2}
    \item  $\sigma=\tau+\xi^2,\quad \sigma_1=\tau_1-\xi_1^2,\quad \sigma_2=\tau - \tau_1+(\xi - \xi_1)^2$,
      \label{it-real-3}
    \item $\sigma=\tau-\xi^2,\quad \sigma_1=\tau_1+\xi_1^2,\quad \sigma_2=\tau - \tau_1-(\xi - \xi_1)^2$,
\label{it-real-4}
    \item $\sigma=\tau+\xi^2,\quad \sigma_1=\tau_1+\xi_1^2,\quad \sigma_2=\tau - \tau_1-(\xi - \xi_1)^2$,
\label{it-real-5}
    \item $\sigma=\tau-\xi^2,\quad \sigma_1=\tau_1-\xi_1^2,\quad \sigma_2=\tau - \tau_1-(\xi - \xi_1)^2$.
\label{it-real-6}
\end{enumerate}
%
for \eqref{sup-est-gen-1} and \eqref{sup-est-gen-2}, and the analogous cases
%
\begin{enumerate}[(i)]
\item $ \sigma^{*}=\tau+\xi^2,\quad \sigma^{*}_1=\tau - \lambda+\xi_1^2,\quad
  \sigma^{*}_2=\lambda+(\xi - \xi_1)^2$, \label{it-1-star-real} \item $
  \sigma^{*}=\tau-\xi^2,\quad \sigma^{*}_1=\tau - \lambda-\xi_{1}^2,\quad
  \sigma^{*}_2=\lambda+(\xi - \xi_1)^2$, \label{it-2-star-real} \item
  $\sigma^{*}=\tau+\xi^2,\quad \sigma^{*}_1=\tau - \lambda-\xi_1^2,\quad
  \sigma^{*}_2=\lambda+(\xi - \xi_1)^2$, \label{it-3-star-real} \item
  $\sigma^{*}=\tau-\xi^2,\quad \sigma^{*}_1=\tau - \lambda+\xi_1^2,\quad
  \sigma^{*}_2=\lambda-(\xi - \xi_1)^2$, \label{it-4-star-real} \item
  $\sigma^{*}=\tau+\xi^2,\quad \sigma^{*}_1=\tau - \lambda+\xi_1^2,\quad
  \sigma^{*}_2=\lambda-(\xi - \xi_1)^2$, \label{it-5-star-real} \item
  $\sigma^{*}=\tau-\xi^2,\quad \sigma^{*}_1=\tau - \lambda-\xi_1^2,\quad
  \sigma^{*}_2= \lambda-(\xi - \xi_1)^2$.  \label{it-6-star-real}
  \end{enumerate}
for \eqref{sup-est-gen-3}.
%
%
\begin{framed}
\begin{remark}
Note that the cases $\sigma=\tau+\xi^2,\quad \sigma_1=\tau_1-\xi_1^2,\quad
\sigma_2=\tau - \tau_1-(\xi - \xi_1)^2$ and $\sigma=\tau-\xi^2,\quad
\sigma_1=\tau_1+\xi_1^2,\quad \sigma_2=\tau - \tau_1+(\xi - \xi_1)^2$ cannot occur, since
$\tau_1< 0, \tau-\tau_1< 0$ implies $\tau<0$ and $\tau_1\geq 0, \tau-\tau_1\geq
0$ implies $\tau\geq 0$. An analogous argument holds for $\sigma^{*},
\sigma_{1}^{*}$ and $\sigma_{2}^{*}$.
\end{remark}
\end{framed}
%
%
Observe that the transformation $(\xi, \tau, \xi_{1}, \tau_{1}) \mapsto -(\xi, \tau,
\xi_{1}, \tau_{1})$ reduces \eqref{it-real-3} to \eqref{it-real-4}, \eqref{it-real-2} to
\eqref{it-real-5}, and \eqref{it-real-1} to \eqref{it-real-6}. Furthermore, the change of
variables $\tau_{2} = \tau - \tau_{1}, \xi_{2} = \xi - \xi_{1}$, and the
transformation $(\xi, \tau, \xi_{2}, \tau_{2}) \mapsto - (\xi, \tau, \xi_{2},
\tau_{2})$ reduces \eqref{it-real-5} to \eqref{it-real-4}. Since $L^{2}$ is invariant
under change of variables and reflections, we may without loss of generality
restrict our attention to cases \eqref{it-real-4} and \eqref{it-real-6}.
 \subsubsection{Case \eqref{it-real-6}} 
\label{sssec:case-it-real-6}
We partition $\rr^{4}$ into the following three sets 
%
%
\begin{equation*}
\begin{split}
  & A_{1} = \left\{ (\xi, \tau, \xi_{1}, \tau_{1}) \subset \rr^{4}: |
  \xi_{1} \le 1 | \text{ or } | \xi - \xi_{1} | \le 1 \right\},
  \\
  & A_{2} = 
  \begin{Bmatrix}
    | \xi_{1} \ge 1 \text{ and } | \xi - \xi_{1} | \ge 1,
    \\
    \langle \tau_{1} - \xi_{1}^{2} \rangle  \le \langle \tau -
  \xi^{2} \rangle
\end{Bmatrix}
  \\
  & A_{3} = 
  \begin{Bmatrix}
    | \xi_{1} | \ge 1 \text{ and } | \xi - \xi_{1} | \ge 1,
    \\
    \langle \tau - \xi^{2} \rangle  \le \langle \tau_{1} - \xi_{1}^{2} \rangle 
  \end{Bmatrix}
\end{split}
\end{equation*}
%
[ATTENTION: Farah splits a bit differently. He in particular does not have a set
like $A_1$-it is split into smaller pieces. This is redundant. Instead,  we follow KPV--makes things cleaner and easier.]
and first seek to bound
%
%
\begin{equation}
  \label{case-1-region-1}
  \begin{split}
    \frac{ \langle \xi
    \rangle ^{2s}}{\langle \tau - \xi^{2} \rangle ^{2a}}
    \int_{\rr} \int_{\rr} \frac{\chi_{A_{1}}}{ \langle \xi_{1} \rangle ^{2s} \langle \xi-\xi_{1} \rangle ^{2s} 
    \langle \tau_{1} - \xi_{1}^{2} \rangle^{2b} \langle  \tau - \tau_{1} -
    (\xi - \xi_{1})^{2} \rangle^{2b} }
    d \tau_1 d \xi_{1}.
  \end{split}
\end{equation}

In region $A_{1}$, we note that if $| \xi_{1} | \le 1$
%
%
%
%
\begin{equation*}
\begin{split}
  (1 + | \xi_{1} |)(1 + | \xi - \xi_{1} |)
  & \le (1 + | \xi_{1} |)(1 + | \xi | + \xi_{1})
  \\
  & \le 2 (2 + | \xi |)
  \\
  & \le 4 (1 + | \xi |).
\end{split}
\end{equation*}
%
%
If $| \xi - \xi_{1} |\le 1$, then
%
%
\begin{equation*}
\begin{split}
  (1 + | \xi_{1} |)(1 + | \xi - \xi_{1} |)
  & \le 2 (1 + | \xi_{1} |)
  \\
  & \le 2 (1 + | \xi - \xi_{1} | + | \xi |)
  \\
  & \le 2(2 + | \xi |)
  \\
  & \le 4 (1 + | \xi |).
\end{split}
\end{equation*}
%
%
Hence, in region $A_{1}$, we have the estimate
%
%
\begin{equation}
\begin{split}
  \langle \xi_{1} \rangle \langle \xi - \xi_{1} \rangle  \le 4 \langle \xi \rangle 
\end{split}
\label{splitting-estimate}
\end{equation}
%
%
which we use to bound \eqref{case-1-region-1} by
%
%
%
%
\begin{equation*}
\begin{split}
    \frac{ 1}{\langle \tau - \xi^{2} \rangle ^{2a}}
    \int_{\rr} \int_{\rr} \frac{\chi_{A_{1}}}{  
    \langle \tau_{1} - \xi_{1}^{2} \rangle^{2b} \langle  \tau - \tau_{1} -
    (\xi - \xi_{1})^{2} \rangle^{2b} }
    d \tau_1 d \xi_{1}.
\end{split}
\end{equation*}
%
%
Applying \autoref{lem:calc}, this in turn is bounded by
%
%
\begin{equation}
  \label{uniform-bound-region-1}
\begin{split}
  & \frac{c_{1}}{\langle \tau - \xi^{2} \rangle^{2a}} \int_{\rr}
  \frac{\chi_{A_{1}}}{\langle \tau - \xi^{2} + 2 \xi \xi_{1} - 2
  \xi_{1}^{2} \rangle ^{2b}} d \xi_{1}
  \\
  & \le c_{1} \int_{\rr}
  \frac{\chi_{A_{1}}}{\langle \tau - \xi^{2} + 2 \xi \xi_{1} - 2
  \xi_{1}^{2} \rangle ^{2b}} d \xi_{1}, \quad a > 0
  \\
& < c, \quad b > 1/4
\end{split}
\end{equation}
%
%
where the last line follows from a corollary to \autoref{lem:calc}.
%
%
%%%%%%%%%%%%%%%%%%%%%%%%%%%%%%%%%%%%%%%%%%%%%%%%%%%%%
%
%
%                Corollary to calculus lemma
%
%
%%%%%%%%%%%%%%%%%%%%%%%%%%%%%%%%%%%%%%%%%%%%%%%%%%%%%
%
%
\begin{corollary}[Lemma 3.1 in \cite{Farah:2009uq}]
  For $a_{0}, a_{1}, a_{2} \in \rr$ with $a_{2} \neq 0$ and $q > 1/2$
  %
  %
  \begin{equation*}
  \begin{split}
    \int_{\rr} \frac{1}{\langle a_{0} + a_{1}x + a_{2}x^{2} \rangle ^{q}} dx \le c
  \end{split}
  \end{equation*}
  %
  where $c$ is a constant independent of the choice of $a_{0}, a_{1}$, and $a_{2}$.
  %
\label{cor:integral-bound}
\end{corollary}
Note that for \eqref{uniform-bound-region-1}, the
value of $c$ does not depend on the choice of $\tau$ or $\xi$. 
Next we seek to bound
\begin{equation}
  \label{case-1-region-2}
  \begin{split}
    \frac{ \langle \xi
    \rangle ^{2s}}{\langle \tau - \xi^{2} \rangle ^{2a}}
    \int_{\rr} \int_{\rr} \frac{\chi_{A_{2}}}{ \langle \xi_{1} \rangle ^{2s} \langle \xi-\xi_{1} \rangle ^{2s} 
    \langle \tau_{1} - \xi_{1}^{2} \rangle^{2b} \langle  \tau - \tau_{1} -
    (\xi - \xi_{1})^{2} \rangle^{2b} }
    d \tau_1 d \xi_{1}.
  \end{split}
\end{equation}
Due to the symmetry of the convolution, we assume without loss of generality that
%
%
\begin{equation*}
\begin{split}
  \langle \tau - \tau_{1} - (\xi - \xi_{1})^{2} \rangle \le \langle
  \tau_{1} - \xi_{1}^{2}\rangle 
\end{split}
\end{equation*}
%
%
Then in region $A_{2}$, we have the estimate
%
%
\begin{equation}
\begin{split}
  | \tau - \xi^{2} |
  & \ge \frac{1}{3}\left[ | \tau_{1} - \xi_{1}^{2} | + | \tau -
  \tau_{1} - (\xi - \xi_{1})^{2}
  | + | \tau - \xi^{2} | \right]
  \\
  & \ge \frac{1}{3} | - \xi_{1}^{2} - (\xi - \xi_{1})^{2} + \xi^{2} |
  \\
  & = \frac{2}{3} | \xi_{1} | | \xi - \xi_{1} |
  \\
  & \gtrsim | \xi_{1} | \quad (| \xi - \xi_{1} |\chi_{A_{2}} \ge 1).
\end{split}
\label{region-2-smoothing}
\end{equation}
%
%
Furthermore, from the inequality
%
%
\begin{equation*}
\begin{split}
  \langle \xi \rangle  \le \langle \xi_{1} \rangle \langle \xi - \xi_{1} \rangle 
\end{split}
\end{equation*}
%
we obtain
%
%
\begin{equation}
  \label{growth-term}
\begin{split}
  \frac{\langle \xi \rangle ^{2s}}{\langle \xi_{1} \rangle ^{2s} \langle \xi -
  \xi_{1} \rangle ^{2s}} \le \langle \xi_{1} \rangle ^{\gamma(s)},
  \quad 
  \gamma(s) = 
  \begin{cases} 0, \quad & s \ge 0
    \\
    4|s|, \quad & s < 0.
  \end{cases}
\end{split}
\end{equation}
Applying \eqref{region-2-smoothing} and \eqref{growth-term}, we obtain 
%
%
\begin{equation*}
\begin{split}
  \eqref{case-1-region-2}
  & \lesssim
  \int_{\rr} \int_{\rr}  \frac{\chi_{A_{2}} \langle \xi_{1} \rangle
  ^{\gamma(s) -2a}}{ 
  \langle \tau_{1} - \xi_{1}^{2} \rangle^{2b} \langle  \tau - \tau_{1} -
    (\xi - \xi_{1})^{2} \rangle^{2b} }
    d \tau_1 d \xi_{1}
    \\
    & \le \int_{\rr} \int_{\rr}  \frac{\chi_{A_{2}}}{ 
    \langle \tau_{1} - \xi_{1}^{2} \rangle^{2b} \langle  \tau - \tau_{1} -
    (\xi - \xi_{1})^{2} \rangle^{2b} }
    d \tau_1 d \xi_{1}, \quad s \ge -a/2.
  \end{split}
\end{equation*}
%
%
By \autoref{lem:calc} and \autoref{cor:integral-bound}, we bound the right hand
side by
%
%
\begin{equation*}
\begin{split}
  & c_{1} \int_{\rr}  \frac{1}{\langle \tau - \xi^{2} + 2 \xi \xi_{1} - 2
  \xi_{1}^{2} \rangle ^{2b}}d \xi_{1}
  \\
  & \le  c, \quad b > 1/2
\end{split}
\end{equation*}
%
%
where $c$ is independent of the value of $\tau$ and $\xi$. 
%
Lastly, we seek to bound
\begin{equation*}
\begin{split}
  &  \frac{1}{\langle \xi_{1} \rangle ^{2s}
  \langle \tau_{1} - \xi_{1}^{2}  \rangle
  ^{2a}} \int_{\rr} \int_{\rr} \frac{\langle \xi \rangle ^{2s}}{\langle
  \xi - \xi_{1}\rangle ^{2s}}  \times \frac{\chi_{A_{3}}}{\langle
  \tau - \xi^{2} \rangle ^{2a} \langle \tau - \tau_{1} - (\xi -
  \xi_{1})^{2} \rangle ^{2b}} d \tau d \xi.
\end{split}
\end{equation*}
Applying \autoref{lem:calc}, we bound this by
%
%
\begin{equation}
  \label{pre-A-3-bound}
\begin{split}
  &  \frac{1}{\langle \xi_{1} \rangle ^{2s}
  \langle \tau_{1} - \xi_{1}^{2}  \rangle
  ^{2a}} \int_{\rr} \frac{\langle \xi \rangle ^{2s}}{\langle
  \xi - \xi_{1}\rangle ^{2s}}  \times \frac{\chi_{A_{3}}}{\langle
  \tau_{1} + 2 \xi \xi_{1} - \xi_{1}^{2} \rangle ^{2b}} d \xi.
\end{split}
\end{equation}
%
Next, noting that in region $A_{3}$,
%
%
\begin{equation*}
\begin{split}
  | \tau_{1} -\xi_{1}^{2} |
  & \ge \frac{1}{3} \left[ | \tau_{1} - \xi_{1}^{2} | + | \tau - \tau_{1} -
  (\xi - \xi_{1})^{2} | + | \tau - \xi^{2} | \right]
  \\
  & \ge \frac{1}{3} | - \xi_{1}^{2} - (\xi - \xi_{1})^{2} + \xi^{2} |
  \\
  & = \frac{2}{3} | \xi_{1} | | \xi - \xi_{1} |
  \\
  & \gtrsim | \xi_{1} | \quad (| \xi - \xi_{1} |\chi_{A_{2}} \ge 1)
\end{split}
\end{equation*}
%
%
and recalling \eqref{growth-term}, we see that \eqref{pre-A-3-bound} is bounded by
\begin{equation*}
\begin{split}
  &  \int_{\rr} \frac{\chi_{A_{2,2}} \langle \xi_{1} \rangle ^{\gamma(s) -2a}}{\langle
  \tau_{1} + 2 \xi \xi_{1} - \xi_{1}^{2} \rangle ^{2b}} d \xi
\end{split}
\end{equation*}
%
which by the change of variable
%
%
\begin{equation*}
\begin{split}
  & \eta = \tau_{1} - \xi_{1}^{2} + 2 \xi \xi_{1},
  \\
  & d \eta = 2 \xi_{1} d \xi
\end{split}
\end{equation*}
%
%
is equal to
%
%
\begin{equation*}
\begin{split}
  & \frac{1}{2} \langle \xi_{1} \rangle ^{\gamma(s)-2a}  \int_{\rr} 
  \frac{1}{| \xi_{1} |\langle \eta \rangle ^{2a} }d \eta
  \\
  & = \langle \xi_{1} \rangle ^{\gamma(s)-2a-1} \int_{0}^{\infty} \frac{1}{(1 + \eta
  )^{2a}}d \eta
  \\
  & \lesssim 1, \qquad s \ge -\frac{2a+1}{4}.
\end{split}
\end{equation*}
%
%
\subsubsection{Case \eqref{it-real-4}} 
\label{sssec:case-it-real-4}
We partition $\rr^{4}$ into three sets 
%
%
\begin{equation*}
\begin{split}
  & A_{1} = \left\{ (\xi, \tau, \xi_{1}, \tau_{1}) \subset \rr^{4}: |
  \xi_{1} \le 1 | \right \} \\
  & A_{2} = \left\{ (\xi, \tau, \xi_{1}, \tau_{1}) \subset \rr^{4}:|
  \xi_{1} | \ge 1 \text{ and } | \xi| \le 1 \right \}
  \\
  & A_{3} = \left\{ (\xi, \tau, \xi_{1}, \tau_{1}) \subset \rr^{4}:|
  \xi_{1} | \ge 1 \text{ and } | \xi| \ge 1 \right \}
  \end{split}
\end{equation*}
%
%
and first seek to bound
%
%
\begin{equation}
  \label{case-2-region-1}
  \begin{split}
    \frac{ \langle \xi
    \rangle ^{2s}}{\langle \tau - \xi^{2} \rangle ^{2a}}
    \int_{\rr} \int_{\rr} \frac{\chi_{A_{1}}}{ \langle \xi_{1} \rangle ^{2s} \langle \xi-\xi_{1} \rangle ^{2s} 
    \langle \tau_{1} + \xi_{1}^{2} \rangle^{2b} \langle  \tau - \tau_{1} -
    (\xi - \xi_{1})^{2} \rangle^{2b} }
    d \tau_1 d \xi_{1}.
  \end{split}
\end{equation}
By inequality \eqref{splitting-estimate}, this is bounded by
%
%
\begin{equation*}
\begin{split}
  \frac{1}{\langle \tau - \xi^{2} \rangle ^{2a}} \int_{\rr} \int_{\rr}
  \frac{\chi_{A_{1}}}{\langle \tau_{1} + \xi_{1}^{2} \rangle ^{2b} \langle \tau
  - \tau_{1} - ( \xi - \xi_{1})^{2}\rangle ^{2b}} d \tau_{1} d \xi_{1}.
\end{split}
\end{equation*}
%
%
Applying \autoref{lem:calc} then gives the bound
%
%
\begin{equation}
\begin{split}
  \frac{1}{\langle \tau - \xi^{2} \rangle ^{2a}} \int_{\rr}
  \frac{\chi_{A_{1}}}{\langle \tau - \xi^{2} + 2 \xi \xi_{1} \rangle ^{2b}} d
  \xi_{1}.
\end{split}
\label{region-1-case-2-pre-est}
\end{equation}
%
%
Since $| \xi_{1} | \le 2$ in $A_{1}$, this is is bounded by
%
%
\begin{equation*}
\begin{split}
  \int_{| \xi_{1} | \le 2} d \xi_{1} \simeq 1.
\end{split}
\end{equation*}
%
%
Next, we seek to bound
\begin{equation*}
\begin{split}
  &  \frac{1}{\langle \xi_{1} \rangle ^{2s}
  \langle \tau_{1} + \xi_{1}^{2}  \rangle
  ^{2a}} \int_{\rr} \int_{\rr} \frac{\langle \xi \rangle ^{2s}}{\langle
  \xi - \xi_{1}\rangle ^{2s}}  \times \frac{\chi_{A_{2}}}{\langle
  \tau - \xi^{2} \rangle ^{2a} \langle \tau - \tau_{1} - (\xi -
  \xi_{1})^{2} \rangle ^{2b}} d \tau d \xi.
\end{split}
\end{equation*}
Applying \eqref{growth-term}, this is bounded by
\begin{equation*}
\begin{split}
  &  \langle \xi_{1} \rangle ^{\gamma(s)}
   \int_{\rr} \int_{\rr} \frac{\chi_{A_{2}}}{\langle
  \tau - \xi^{2} \rangle ^{2a} \langle \tau - \tau_{1} - (\xi -
  \xi_{1})^{2} \rangle ^{2b}} d \tau d \xi.
\end{split}
\end{equation*}
Applying \autoref{lem:calc} then gives the bound
\begin{equation*}
\begin{split}
  & \langle \xi_{1} \rangle ^{\gamma(s)} \int_{\rr} \frac{\chi_{A_{2,2}} }{\langle
  \tau_{1} + 2 \xi \xi_{1} - \xi_{1}^{2} \rangle ^{2b}} d \xi
\end{split}
\end{equation*}
which by the change of variable
%
%
\begin{equation*}
\begin{split}
  & \eta = \tau_{1} - \xi_{1}^{2} + 2 \xi \xi_{1},
  \\
  & d \eta = 2 \xi_{1} d \xi
\end{split}
\end{equation*}
%
%
is equal to
%
%
\begin{equation*}
\begin{split}
  & \frac{\langle \xi_{1} \rangle^{\gamma(s)}}{2}  \int_{\rr} 
  \frac{1}{| \xi_{1} |\langle \eta \rangle ^{2a} }d \eta
  \\
  & = \langle \xi_{1} \rangle ^{\gamma(s) -1} \int_{0}^{\infty} \frac{1}{(1 + \eta
  )^{2a}}d \eta
  \\
  & \lesssim 1, \qquad s \ge -1/4.
\end{split}
\end{equation*}

%If $| \xi | \ge 1$, then using the change of variable
%%
%%
%\begin{equation*}
%\begin{split}
  %& \eta = \tau - \xi^{2} + 2 \xi \xi_{1}
  %\\
  %& d \eta = 2 \xi d \xi_{1}
%\end{split}
%\end{equation*}
%%
%%
%to \eqref{region-1-case-2-pre-est} gives
%\begin{equation*}
  %\begin{split}
 %\frac{1}{2 \langle \tau - \xi^{2} \rangle^{2a} } \int_{\rr} \frac{1}{| \xi
%|\langle \eta \rangle ^{2b}} d \eta
%& \le  \int_{\rr} \frac{1}{\langle \eta \rangle
%^{2b}} d \eta
%\\
%& \simeq 1, \qquad b > 1/2.
%\end{split}
%\end{equation*}
%
%
We now split $A_{3}$ into the sets
%
%
%
\begin{align*}
A_{3,1}&=\{(\xi, \xi_1, \tau, \tau_1)\in A_2:
|\tau-\tau_1-(\xi-\xi_1)^2|, |\tau_1+\xi_1^2| \le |\tau-\xi^2|\},\\
A_{3,2}&=\{(\xi, \xi_1, \tau, \tau_1)\in A_2:
|\tau-\tau_1-(\xi-\xi_1)^2|, |\tau-\xi^2| \le |\tau_1+\xi_1^2|\},\\
A_{3,3}&=\{(\xi, \xi_1, \tau, \tau_1)\in A_2: |\tau_{1}+\xi_{1}^2|, | \tau - \xi^{2} | \le |  \tau - \tau_{1} -
(\xi - \xi_{1})^{2} |\}
\end{align*} 
and first seek to estimate
%
%
\begin{equation}
  \label{case-2-region-2}
  \begin{split}
    \frac{ \langle \xi
    \rangle ^{2s}}{\langle \tau - \xi^{2} \rangle ^{2a}}
    \int_{\rr} \int_{\rr} \frac{\chi_{A_{3,1}}}{ \langle \xi_{1} \rangle ^{2s} \langle \xi-\xi_{1} \rangle ^{2s} 
    \langle \tau_{1} + \xi_{1}^{2} \rangle^{2b} \langle  \tau - \tau_{1} -
    (\xi - \xi_{1})^{2} \rangle^{2b} }
    d \tau_1 d \xi_{1}.
  \end{split}
\end{equation}
Notice that, unlike case \eqref{it-6}, we may \emph{not} assume without loss of generality
that $|\tau - \tau_{1} - (\xi - \xi_1)^{2} | \le | \tau_{1} + \xi^{2} | $.
Applying \autoref{lem:calc} to \eqref{case-2-region-2} gives the bound
%
%
\begin{equation*}
  \begin{split}
    \frac{ \langle \xi
    \rangle ^{2s}}{\langle \tau - \xi^{2} \rangle ^{2a}}
    \int_{\rr} \frac{\chi_{A_{3,1}}}{ \langle \xi_{1} \rangle ^{2s} \langle \xi-\xi_{1} \rangle ^{2s} 
    \langle \tau - \xi^{2} +2 \xi \xi_{1} \rangle^{2b} } d \xi_{1}.
  \end{split}
\end{equation*}
%
%
%
But in $A_{3,1}$
%
%
\begin{equation}
\begin{split}
  | \tau - \xi^{2} |
  & \ge \frac{1}{3} \left[ | \tau_{1} + \xi_{1}^{2} | + | \tau - \tau_{1} -
  (\xi - \xi_{1})^{2} | + | \tau - \xi^{2} | \right]
  \\
  & \ge \frac{1}{3} | \xi_{1}^{2} - (\xi - \xi_{1})^{2} + \xi^{2} |
  \\
  & = \frac{2}{3}| \xi | | \xi_{1} |
  \\
  & \gtrsim  | \xi_{1} |.
\end{split}
\label{case-2-region-A2-key-est}
\end{equation}
%
%
Hence, applying \eqref{growth-term} and \eqref{case-2-region-A2-key-est}, we obtain
%
%
%
%
\begin{equation*}
\begin{split}
  \eqref{case-2-region-2}
  & \lesssim 
  \int_{\rr} \int_{\rr} \frac{\chi_{A_{3,1}} \langle \xi_{1}
  \rangle^{\gamma(s) -2a}}{  
    \langle \tau_{1} + \xi_{1}^{2} \rangle^{2b} \langle  \tau - \tau_{1} -
    (\xi - \xi_{1})^{2} \rangle^{2b} }
    d \tau_1 d \xi_{1}
    \\
    & \le \int_{\rr} \int_{\rr} \frac{\chi_{A_{3,1}}}{  
    \langle \tau_{1} + \xi_{1}^{2} \rangle^{2b} \langle  \tau - \tau_{1} -
    (\xi - \xi_{1})^{2} \rangle^{2b} }
    d \tau_1 d \xi_{1}, \quad s \ge -a/2.
\end{split}
\end{equation*}
By \autoref{lem:calc} and \autoref{cor:integral-bound}, we bound the right hand
side by
%
%
\begin{equation*}
\begin{split}
  & c_{1} \int_{\rr}  \frac{1}{\langle \tau - \xi^{2} + 2 \xi \xi_{1} 
  \rangle ^{2b}}d \xi_{1}
  \\
  & = c_{1} \int_{\rr}  \frac{1}{\left [ \langle \tau - \xi^{2} + 2 \xi \xi_{1} 
  \rangle ^{2} \right ]^{b}}d \xi_{1}
  \\
  & \le  c, \quad b > 1/2
\end{split}
\end{equation*}
where $c$ is independent of the value of $\tau$ and $\xi$. 
Next, we seek to estimate
\begin{equation*}
\begin{split}
  &  \frac{1}{\langle \xi_{1} \rangle ^{2s}
  \langle \tau_{1} + \xi_{1}^{2}  \rangle
  ^{2a}} \int_{\rr} \int_{\rr} \frac{\langle \xi \rangle ^{2s}}{\langle
  \xi - \xi_{1}\rangle ^{2s}}  \times \frac{\chi_{A_{3,2}}}{\langle
  \tau - \xi^{2} \rangle ^{2a} \langle \tau - \tau_{1} - (\xi -
  \xi_{1})^{2} \rangle ^{2b}} d \tau d \xi.
\end{split}
\end{equation*}
Applying \autoref{lem:calc}, we bound this by
%
%
\begin{equation}
  \label{pre-A-2-2}
\begin{split}
  &  \frac{1}{\langle \xi_{1} \rangle ^{2s}
  \langle \tau_{1} + \xi_{1}^{2}  \rangle
  ^{2a}} \int_{\rr} \frac{\langle \xi \rangle ^{2s}}{\langle
  \xi - \xi_{1}\rangle ^{2s}}  \times \frac{\chi_{A_{3,2}}}{\langle
  \tau_{1} + 2 \xi \xi_{1} - \xi_{1}^{2} \rangle ^{2b}} d \xi.
\end{split}
\end{equation}
%
Next, noting that in region $A_{3,2}$,
%
%
\begin{equation}
\begin{split}
  | \tau_{1} + \xi_{1}^{2} |
  & \ge \frac{1}{3} \left[ | \tau_{1} + \xi_{1}^{2} | + | \tau - \tau_{1} -
  (\xi - \xi_{1})^{2} | + | \tau - \xi^{2} | \right]
  \\
  & \ge \frac{1}{3} | \xi_{1}^{2} - (\xi - \xi_{1})^{2} + \xi^{2} |
  \\
  & = \frac{2}{3}| \xi | | \xi_{1} |
  \\
  & \gtrsim  | \xi_{1} |
\end{split}
\label{case-2-region-A-2-2-key-est}
\end{equation}
%
%
and recalling \eqref{growth-term}, we see that \eqref{pre-A-2-2} is bounded by
\begin{equation*}
\begin{split}
  &  \int_{\rr} \frac{\chi_{A_{3,2}} \langle \xi_{1} \rangle ^{\gamma(s) -2a}}{\langle
  \tau_{1} + 2 \xi \xi_{1} - \xi_{1}^{2} \rangle ^{2b}} d \xi
\end{split}
\end{equation*}

which by the change of variable
%
%
\begin{equation*}
\begin{split}
  & \eta = \tau_{1} - \xi_{1}^{2} + 2 \xi \xi_{1},
  \\
  & d \eta = 2 \xi_{1} d \xi
\end{split}
\end{equation*}
%
%
is equal to
%
%
\begin{equation*}
\begin{split}
  & \frac{\xi_{1}^{\gamma(s) -2a}}{2}  \int_{\rr} 
  \frac{1}{| \xi_{1} |\langle \eta \rangle ^{2a} }d \eta
  \\
  & = \langle \xi_{1} \rangle ^{\gamma(s) -2a -1} \int_{0}^{\infty} \frac{1}{(1 + \eta
  )^{2a}}d \eta
  \\
  & \lesssim 1, \qquad s \ge -\frac{2a+1}{4}.
\end{split}
\end{equation*}
It remains to handle region $A_{3,3}$. It will be enough to bound
%
%
\begin{equation}
  \label{region-A-2-3-star-split}
\begin{split}
   \sum_{\xi_{1} \in \zz} \int_{\rr} \frac{\chi^{*}_{A_{3,3}}
    \langle \xi \rangle ^{2s}
    }{ \langle \xi_{1} \rangle^{2s} \langle  \tau  - \xi^{2}
    \rangle ^{2a}  \langle
\xi-\xi_{1} \rangle ^{2s}  \langle  \tau - \lambda+\xi_{1}^{2}
\rangle^{2b} \langle   \lambda  -(\xi - \xi_{1})^{2}
\rangle^{2b} } d \tau.
\end{split}
\end{equation}
%
Due to the presence of $\chi^{*}_{A_{3,3}}$ factor, we have the restriction
%
%
\begin{equation*}
\begin{split}
& |\tau - \lambda +\xi_{1}^2|, | \tau - \xi^{2} | \le |  \lambda -
(\xi - \xi_{1})^{2} | \text{ and }  |\xi| \ge 1, |\xi_1| \ge 1.
\end{split}
\end{equation*}
%
It follows that
\begin{equation}
  \label{smoothing-2-3-case-6}
\begin{split}
  | \lambda - (\xi - \xi_{1})^{2} |
  & \ge \frac{1}{3}\left[ | \tau - \lambda + \xi_{1}^{2} | + | \lambda - (\xi - \xi_{1})^{2}
  | + | \tau - \xi^{2} | \right]
  \\
  & \ge \frac{1}{3} |  \xi_{1}^{2} - (\xi - \xi_{1})^{2} + \xi^{2} |
  \\
  & = \frac{2}{3} | \xi_{1} | | \xi |
  \\
  & \gtrsim | \xi_{1} |.
\end{split}
\end{equation}
Hence, applying
\eqref{growth-term}, \autoref{lem:calc}, and
\eqref{smoothing-2-3-case-6}, we bound \eqref{region-A-2-3-star-split} by
%
%
\begin{equation*}
\begin{split}
   & \sum_{\xi_{1} \in \zz} \int_{\rr} \frac{\chi^{*}_{B_{3,3}} \langle
   \xi_{1} \rangle ^{\gamma(s) -2a}
    }{ \langle  \tau  - \xi^{2}
    \rangle ^{2a}   \langle  \tau - \lambda+\xi_{1}^{2}
\rangle^{2b} } d \tau
\\
& \lesssim  \sum_{\xi_{1} \in \zz} \frac{\chi_{B_{3,3}^{*}}}{\langle \xi_{1}^{2} +
\xi^{2} - \lambda \rangle^{2 a} }, \quad s \ge -a/2
\end{split}
\end{equation*}
which is bounded for $a > 1/4$ by
\autoref{lem:sum-estimate}. This concludes the proof of \autoref{prop:bilin-est-real}. \qquad \qedsymbol
%
%
%%
%%
%\begin{equation}
%\begin{split}
  %|  \eqref{sup-est-gen-real} \chi_{B_{2}}|
  %& =  \langle \tau - \xi^{2} \rangle
  %^{-2a} \langle \xi \rangle ^{2s} \int _{\rr} \int_{\rr} \frac{\chi_{B_{1}}}{
  %\langle \xi_{1} \rangle ^{2s} \langle \xi-\xi_{1}\rangle ^{2s} \langle \tau_{1} - \xi_{1}^{2} \rangle ^{2b}\langle
  %\tau - \tau_{1} - (\xi -\xi_{1})^{2}\rangle ^{2b}}d \tau_{1} d \xi_{1} 
  %\\
  %& \le  \int_{\rr} \int_{\rr} \frac{\chi_{B_{1}}
  %\langle
  %\xi_{1}\rangle ^{\gamma(s) -2b} 
%}{ \langle \tau_{1} - \xi_{1}^{2} \rangle ^{2b}\langle
  %\tau - \tau_{1} - (\xi - \xi_{1})^{2}\rangle ^{2b}}
  %d \tau_{1} d \xi_{1} 
  %\\
  %& \le \int_{\rr} \int_{\rr} \frac{\chi_{B_{1}}
  %}{ \langle \tau_{1} - \xi_{1}^{2} \rangle ^{2b}\langle
  %\tau - \tau_{1} - (\xi - \xi_{1})^{2}\rangle ^{2b}}
  %d \tau_{1} d \xi_{1}, \quad s \ge 0 \text{ or } s \ge -a/2
  %\\
  %& \le \int _{\rr} 
  %\frac{\chi_{B_{1}}}{\langle \tau - \xi^{2} + 2 \xi
  %\xi_{1} - 2 \xi_{1}^{2} \rangle ^{2b}}d \xi_{1}, \quad (\autoref{lem:calc}) 
  %\\
  %& < \infty
%\end{split}
%\end{equation}

\section{Second order Modified Boussinesq  equation}
\label{sec:intro}
We consider the initial value problem (ivp) for a modified Boussinesq
equation ($B_4$) equation 
\begin{gather}
  u_{tt} - u_{xx} + (u^2)_{xx} = 0,
  \label{eqn:mb}
  \\
  u(x,0) = u_{0}(x), \quad u_{0} \in H^{s}
  \label{eqn:mb-init-data}
\end{gather}
and conjecture the following.
%
%
%%%%%%%%%%%%%%%%%%%%%%%%%%%%%%%%%%%%%%%%%%%%%%%%%%%%%
%
%
%                Main Theorem
%
%
%%%%%%%%%%%%%%%%%%%%%%%%%%%%%%%%%%%%%%%%%%%%%%%%%%%%%
%
%
\begin{theorem}
  If $s>s_c$ then then the  i.v.p, for the fourth order modified
  Boussinesq  equation is well-posed
  \begin{itemize}
    \item In $H^s(\rr)$ if $s > s_c$
    \item In $H^{s}(\ci)$ if $s > s_c + 1/4$,
  \end{itemize}
  and the data-to-solution map is ?? (continuous?, Lip?, smooth, analytic?). 
  \label{thm:wp}
\end{theorem}
%
%
%
%
%
Since the scaling conserves data in $\dot{H}^{-3/2}$\ldots
It seems that this equation is ``like KdV''.
So one may expect KdV type theorems\ldots
That is, $s_c=-3/4$ on the line and $s_c=-1/2$ on the circle,
if one uses bilinear estimates.
But, Kappeler and collaborators went all the way to $-1$ for KdV.
However KdV is integrable. Is this equation integrable?
Also, people conjecture that the critical index for KdV well-posedness 
in some appropriate sense should be the scaling index which is  $-3/2$.

\newpage
\appendix
\section{}
%\subsection{Proof of \autoref{lem:embedding}}
%%
%%
%\begin{equation*}
%\begin{split}
  %\| u(t) - u(t') \|_{H^s}^{2}
  %& = \sum_{n \in \zz} (1 + |n|)^{2s} [\wt{u}(n, t) - \wt{u}(n, t')]
  %\\
  %& = \sum_{n \in \zz} (1 + |n|)^{2s} \int_{\rr} (e^{it\tau} - e^{it'
  %\tau})\wh{u}(n, \tau) d \tau
  %\\
  %& \le 2 \sum_{n \in \zz} (1 + |n|)^{2s} \int_{\rr} \wh{u}(n, \tau) d \tau
  %\\
  %& \simeq \sum_{n \in \zz} (1 + |n|)^{2s} \int_{\rr} (1 + | | \tau | -
  %n^{2} |)^{b}(1 + | | \tau | - n^{2} |)^{-b} | \wh{u}(n, \tau) | d \tau.
%\end{split}
%\end{equation*}
%%
%%
%Applying Cauchy-Schwartz in $\tau$, we bound this by
%%
%%
%\begin{equation*}
%\begin{split}
  %& \sum_{n \in \zz} (1 + |n|)^{2s} \left[ \int_{\rr} (1 + | | \tau | -
  %n^{2} |)^{2b} | \wh{u}(n, \tau) |^{2} d \tau \right]^{1/2} \left[ \int_{\rr}
  %(1 + | | \tau | - n^{2} |)^{-2b} d \tau \right]^{1/2}
  %\\
  %& = \sum_{n \in \zz} (1 + |n|)^{2s} \left[ \int_{\rr} (1 + | | \tau | -
  %n^{2} |)^{2b} | \wh{u}(n, \tau) |^{2} d \tau \right]^{1/2} \left[ \int_{\rr}
  %(1 +  | \tau' | )^{-2b} d \tau' \right]^{1/2}
  %\\
  %& = c \sum_{n \in \zz} (1 + |n|)^{2s} \left[ \int_{\rr} (1 + | | \tau | -
  %n^{2} |)^{2b} | \wh{u}(n, \tau) |^{2} d \tau \right]^{1/2} \qquad (b > 1/2). 
%\end{split}
%\end{equation*}
%%
%%
%%
%%
%Applying Cauchy-Schwartz in $n$ then gives the bound
%%
%%
%\begin{equation*}
%\begin{split}
  %\sum_{n \in \zz} (1 + |n|)^{2s} \int_{\rr} (1 + | | \tau | - n^{2}
  %|)^{2b} \wh{u}(n, \tau) d \tau = \| u \|_{X_{s,b}}^{2}.
%\end{split}
%\end{equation*}
%%
%%
%An application of dominated convergence completes the proof. \qquad \qedsymbol
\subsection{Deriving the Integral form of the $B_{4}$ ivp
\eqref{lin-mb}-\eqref{lin-mb-init-data-1}.} 
\label{ssec:integral-form-deriv}
We will describe an alternative to the method of variation of parameters used in
the paper. 
%
%
\subsubsection{Reducing to a First Order ODE} 
\label{sssec:first-order-ode}
Taking the spatial Fourier transform of \eqref{lin-mb} yields
the ivp
%
%
\begin{gather*}
  \wh{u_{tt}} + n^{4} \wh{u} = \wh{-u^{2}_{xx}},
  \\
  \wh{u}(n, 0) = \wh{u_{0}}(n), \quad \wh{u_{t}}(n, 0) = \wh{u_{1}}(n)
\end{gather*}
%
%
which we rewrite as 
%
%
\begin{gather}
  \label{eqn:lin-mb-ode}
  y_{tt} + n^{4}y = -f,
  \\
  y(n, 0) = y_{0}(n), \quad y_{t}(n, 0) = y_{1}(n)
\label{eqn:lin-mb-ode-init-data}
\end{gather}
%
%
where
%
%
\begin{gather*}
  \label{not-1}
  y = y(n, t) \doteq \wh{u}(n, t), \quad f = f(n, t) \doteq
  \wh{u^{2}_{xx}}(n,t),
  \\
  \label{not-2}
  y_{0}(n) \doteq \wh{u_{0}}(n), \quad y_{1}(n) = \wh{u_{1}}(n).
\end{gather*}
%
%
Viewing $n$ as fixed, we see that \eqref{eqn:lin-mb-ode} is a second order
linear nonhomogeneous ODE. To solve it, we set 
%
%
\begin{equation*}
  \label{not-3}
\begin{split}
   & v_{1} = y, 
   \\
   & v_{2} = y_{t}
\end{split}
\end{equation*}
%
%
giving
%
%
\begin{equation*}
\begin{split}
  & v_{1}' = v_{2},
  \\
  & v_{2}' = -n^{4}v_{1} - f.
\end{split}
\end{equation*}
%
%
Therefore 
%
%
\begin{equation}
\begin{split}
\frac{d \vec v}{dt} = 
\begin{bmatrix}
0 & 1 \\
-n^{4} & 0
\end{bmatrix}
\begin{bmatrix}
  v_{1}\\
  v_{2}
\end{bmatrix}
-
\begin{bmatrix}
0\\
f
\end{bmatrix}
\doteq A \vec v - \vec f.
\end{split}
\label{eqn:first-order-ode-reduction}
\end{equation}
%
%
Hence, we have reduced solving the $2$nd order ODE \eqref{eqn:lin-mb-ode} to
solving the first order ODE \eqref{eqn:first-order-ode-reduction}. Multiplying
by the integrating factor $e^{-At}$ on both sides of
\eqref{eqn:first-order-ode-reduction}, we obtain
%
%
\begin{equation*}
\begin{split}
  \frac{d}{dt}(e^{-At} \vec v) = -e^{-At} \vec f.
\end{split}
\end{equation*}
%
%
Integrating in time then gives
%
%
\begin{equation*}
\begin{split}
  e^{-At} \vec v(n, t) = \vec v(n, 0) - \int_{0}^{t}e^{-At'} \vec f dt'
\end{split}
\end{equation*}
%
%
or
%
%
\begin{equation}
  \label{ode-vec-soln}
\begin{split}
  \vec v(n, t) = e^{At} \vec v(n, 0) - \int_{0}^{t}e^{A(t - t')} \vec f dt'.
\end{split}
\end{equation}
%
%
We wish to compute $e^{At}$. It is easy to check that $A$ has eigenvalues
$\lambda = \pm in^{2}$, with corresponding eigenvectors 
%
%
\begin{equation*}
\begin{split}
\pm \begin{bmatrix}
1 \\
in^{2}
\end{bmatrix}.
\end{split}
\end{equation*}
%
%
Since $A$ is a square matrix and has no repeated
eigenvalues, it is diagonizable. More precisely,
%
%
%
%
\begin{equation*}
\begin{split}
  A = Q D Q^{-1} = 
  \begin{bmatrix}
  1 & 1
  \\
  in^{2} & -in^{2}
  \end{bmatrix}
  \begin{bmatrix}
    in^{2} & 0 
    \\
    0 & -in^{2}
  \end{bmatrix}
  \begin{bmatrix}
    \frac{1}{2} & \frac{1}{2i n^{2}} \\
    \frac{1}{2} & -\frac{1}{2i n^{2} }
  \end{bmatrix}
\end{split}
\end{equation*}
%
%
where the column vectors of $Q$ are comprised of the eigenvectors of $A$.
Recalling that 
%
%
\begin{equation*}
\begin{split}
  e^{At} \doteq \sum_{n=0}^{\infty} \frac{A^{n}}{n}t^{n}
\end{split}
\end{equation*}
%
%
it is easy to check that 
%
%
\begin{equation*}
\begin{split}
  e^{Q D Q^{-1}} = 
\begin{bmatrix}
  1 & 1
  \\
  in^{2} & -in^{2}
  \end{bmatrix}
  \begin{bmatrix}
    e^{in^{2}} & 0 
    \\
    0 & e^{-in^{2}}
  \end{bmatrix}
  \begin{bmatrix}
    \frac{1}{2} & \frac{1}{2i n^{2}} \\
    \frac{1}{2} & -\frac{1}{2i n^{2} }
  \end{bmatrix}
\end{split}
\end{equation*}
%
%
and 
\begin{equation}
  \label{matrix-expo}
\begin{split}
  e^{Q D Q^{-1}t}
  & = 
\begin{bmatrix}
  1 & 1
  \\
  in^{2} & -in^{2}
  \end{bmatrix}
  \begin{bmatrix}
    e^{in^{2}t} & 0 
    \\
    0 & e^{-in^{2}t}
  \end{bmatrix}
\begin{bmatrix}
    \frac{1}{2} & \frac{1}{2i n^{2}} \\
    \frac{1}{2} & -\frac{1}{2i n^{2} }
  \end{bmatrix}
  \\
  & =
  \begin{bmatrix}
    \frac{1}{2}(e^{in^{2}t} + e^{-in^{2}t}) & \frac{1}{2 i n^{2}} (e^{in^{2}t} -
    e^{-in^{2}t})    \\
    \frac{in^{2}}{2}(e^{in^{2}t} - e^{-in^{2}t}) & \frac{1}{2}(e^{in^{2}t} +
    e^{-in^{2}t})
  \end{bmatrix}.
\end{split}
\end{equation}
%
%
\begin{framed}
\begin{remark}
If a matrix has repeated eigenvalues, it may no longer be diagonizable. However,
any matrix can be written in Jordan canonical form. The above computations
become slightly more complicated in this case. The important observation is that
writing a matrix in Jordan canonical (or, ideally, diagonal) form allows us to
easily compute its exponential. 
\label{rem:jordan-form}
\end{remark}
\end{framed}
%
%
%
\pagebreak
%
\begin{framed}
\begin{remark}
\label{rem:simpler-comp}
  Since $A$ is a particularly simple matrix, i.e. $A^{2} = -n^{4} I$, one can
  compute its exponential easily without the use of diagonalization (this is
  not true in general). From the above equality, we obtain $A^{n} =
  (-1)^{n/2} n^{2n} I$ for even $n$, and so
  %
  %
  \begin{equation*}
  \begin{split}
    e^{At}
    & = \sum_{n=0}^{\infty} \frac{(-1)^{n}n^{4n}t^{2n}}{(2n)!}I + A
    \sum_{n=0}^{\infty} \frac{(-1)^{n} n^{4n} t^{2n + 1}}{(2n + 1)!} I 
    \\
    & = \cos n^{2}t \, I - \frac{\sin n^{2}t}{n^{2}}A
    \\
    & = 
    \begin{bmatrix}
      \cos n^{2}t &  -\frac{\sin n^{2}t}{n^{2}}
      \\
      - n^{2} \sin n^{2}t & \cos n^{2}t
    \end{bmatrix}
    \\
    & = \text{rhs of }\eqref{matrix-expo}
  \end{split}
  \end{equation*}
\end{remark}
\end{framed}
%
%
Substituting \eqref{matrix-expo} into \eqref{ode-vec-soln} and recalling our notation, we obtain
%
%
\begin{equation*}
\begin{split}
  v_{1}(n, t) = (e^{in^{2}t} + e^{-in^{2}t})v_{1}(n, 0) + \frac{e^{in^{2}} -
  e^{-in^{2}t}}{2 i n^{2}} v_{1}(n, 0) + \int_{0}^{t} \frac{e^{in^{2}(t - t')} -
  e^{-in^{2}(t-t')}}{2 i n^{2}} f dt'
\end{split}
\end{equation*}
%
%
or
\begin{equation*}
\begin{split}
  y(n, t) = (e^{in^{2}t} + e^{-in^{2}t})y_{0} + \frac{e^{in^{2}} -
  e^{-in^{2}t}}{2 i n^{2}} y_{1} + \int_{0}^{t} \frac{e^{in^{2}(t - t')} -
  e^{-in^{2}(t-t')}}{2 i n^{2}} f dt'
\end{split}
\end{equation*}
or
\begin{equation*}
\begin{split}
  \wh{u}(n, t) = (e^{in^{2}t} + e^{-in^{2}t})\wh{u_0} + \frac{e^{in^{2}} -
  e^{-in^{2}t}}{2 i n^{2}} \wh{u_1} + \int_{0}^{t} \frac{e^{in^{2}(t - t')} -
  e^{-in^{2}(t-t')}}{2 i n^{2}} \wh{(u^{2})_{xx}} dt'.
\end{split}
\end{equation*}
%
Taking the inverse Fourier transform then yields \eqref{eqn:integral-form}, as
desired.
%
%
\subsection{Proof of \autoref{lem:mod-princ-symb-bound}} 
\label{ssec:pf-mod-princ}
By the reverse triangle inequality, we have
%
%
\begin{equation*}
\begin{split}
  | \tau | = | \tau + n^{2} - n^{2} | \ge | | \tau + n^{2} | - n^{2} |.
\end{split}
\end{equation*}
%
%
Furthermore, if $\tau - n^{2} < 0$, then
%
%
\begin{equation*}
\begin{split}
  | | \tau - n^{2} | - n^{2} | = | n^{2} - \tau - n^{2} | = | \tau |
\end{split}
\end{equation*}
%
%
while if $\tau - n^{2} > 0$, then
%
%
\begin{equation*}
\begin{split}
  | | \tau - n^{2} | - n^{2} | \le n^{2} \le \tau = |\tau|
\end{split}
\end{equation*}
%
%
completing the proof. \qquad \qedsymbol
%
%
\subsection{Proof of \autoref{lem:calc}}
%
%
%
By the change of variable $\theta \mapsto a/2 + x$, we have
%
%
\begin{equation*}
	\begin{split}
		\int_{\rr} \frac{1}{(1 + | \theta |)(1 + | a - \theta |)}d \theta
	= \int_{\rr} \frac{1}{(1 + |  a/2 + x |)(1 + | a/2 - x |)}d x.
	\end{split}
\end{equation*}
%
%
Hence, it suffices to show that
%
%
\begin{equation*}
	\begin{split}
		\int_{\rr} \frac{1}{(1 + | a - \theta |)(1 + | a + \theta |)}d \theta
		\lesssim \frac{\log(2 + | a |)}{1 + | a |}.
	\end{split}
\end{equation*}
%
%
Let us leave the case $a = 0$ for last. By symmetry, the cases $a<0$ and $a >0$
are equivalent. Hence, to cover the case $a \neq0$, we may assume
without loss of generality that $a >0$.
%
%
Then
\begin{equation}
	\label{a1}
	\begin{split}
		& \int_{\rr} \frac{1}{(1 + | a - \theta |)(1 + | a + \theta |)}d \theta
		\\
		& = \int_{| \theta| \le a+1 } \frac{1}{(1 + | a - \theta |)(1 + | a + \theta
		|)}d \theta + \int_{| \theta| \ge a+1 } \frac{1}{(1 + | a - \theta |)(1 + |
		a + \theta |)}d \theta.
	\end{split}
\end{equation}
Estimating the second integral of \eqref{a1}, we have
\begin{equation*}
	\begin{split}
		& \int_{| \theta| \ge a+1 } \frac{1}{(1 + | a - \theta |)(1 + | a + \theta
		|)}d \theta 
		\\
		& = \int_{\theta \ge a + 1} \frac{1}{(1 + \theta-a)(1 + \theta+a)} d \theta
		+ \int_{\theta \le -a -1} \frac{1}{(1 + \theta - a) (1 + \theta + a)}d \theta
		\\
		& = \frac{1}{2a} \int_{\theta \ge a + 1} \left[ \frac{1}{1 + \theta -a} -
		\frac{1}{1 + \theta+a} \right] d \theta
		+ \frac{1}{2a} \int_{\theta \le -a-1} \left[ \frac{1}{1 + \theta+a}
		-\frac{1}{1 + \theta -a} \right] d \theta
		\\
		& = \frac{1}{a} \log(1+a)
		\\
		& \lesssim \frac{\log(2 + |a|)}{1 + | a |}.
	\end{split}
\end{equation*}
To evaluate the first integral of \eqref{a1}, we split into the cases $a \le \theta \le
a+1$, $-a \le \theta \le 0$, $0 \le \theta \le a$, and $a \le \theta \le a+1$.
However, note that 
%
%
\begin{equation*}
	\begin{split}
		& \int_{a}^{a+1} \frac{1}{(1 + | a - \theta |)(1 + | a + \theta |)}d \theta =
		\int_{-a-1}^{-a} \frac{1}{(1 + | a - \theta |)(1 + | a + \theta |)}d \theta,
		\\
		& \int_{0}^{a} \frac{1}{(1 + | a - \theta |)(1 + | a + \theta |)}d \theta =
		\int_{-a}^{0} \frac{1}{(1 + | a - \theta |)(1 + | a + \theta |)}d \theta.
	\end{split}
\end{equation*}
%
%
Therefore, we need only consider the cases $a \le \theta \le a+1$ and $0 \le
\theta \le a$.
%
%
\subsection{Case $a \le \theta \le a+1$}
We have
%
%
\begin{equation*}
	\begin{split}
		\int_{a}^{a+1} \frac{1}{(1 + | a-\theta |)(1 + | a + \theta |)}d \theta
		& = \int_{a}^{a+1} \frac{1}{(1 + \theta -a)(1 + a + \theta)}d \theta
		\\
		& = \frac{1}{2a} \int_{a}^{a+1} \left[ \frac{1}{1 + \theta -a} -
		\frac{1}{1 + \theta + a}  \right]d \theta
		\\
		& =\frac{1}{2a} \log\left( \frac{1 + \theta -a}{1 + \theta + a} \right) \Big
		|_a^{a+1}
		\\
		& = \frac{1}{2a} \log\left( \frac{2a+1}{a+1} \right)
		\\
		& \lesssim\frac{\log 2}{2a}
		\\
		& \lesssim \frac{\log(2 + | a |)}{1 + | a |}.
	\end{split}
\end{equation*}
%
%
\subsection{Case $0 \le \theta \le a$}
We have
%
%
\begin{equation*}
	\begin{split}
		\int_{0}^{a} \frac{1}{(1 + | a - \theta |)(1 + | a + \theta |)}d \theta
		& = \int_{0}^{a} \frac{1}{(1 +  a - \theta )(1 +  a + \theta )}d \theta
		\\
		& = \frac{1}{2(1 + a)} \int_{0}^{a} \left[ \frac{1}{1 + a - \theta} +
		\frac{1}{1 + a + \theta} \right]d \theta
		\\
		& = \frac{1}{2(1 + a)} \log \left( \frac{1 + a + \theta}{1 + a - \theta}
		\right) \Big |_{0}^{a}
		\\
		& = \frac{\log\left( 1 + 2a \right)}{2\left( 1 + a \right)}
		\\
		& \lesssim \frac{\log(2 + | a |)}{1 + | a |}.
	\end{split}
\end{equation*}
%
%
This completes the proof for the case $a \neq 0$. Lastly, for the case
$a =0$, we use dominated convergence and our preceding work to
conclude that
%
%
\begin{equation*}
	\begin{split}
		\int_{\rr} \frac{1}{(1 + | \theta|)^2} d \theta
		& = \lim_{a \to 0}
		\int_{\rr} \frac{1}{(1 + | a - \theta |)(1 + | a + \theta |)}d \theta
		\\
		& \lesssim \lim_{a \to 0} \frac{\log(2 + | a |)}{1 + | a |}
		\\
		& =  \log 2
		\\
		& = \frac{\log(2 + | 0 |)}{1 + | 0 |}. \qquad \qed
	\end{split}
\end{equation*}


%\nocite{*}
%\bibliography{/Users/davidkarapetyan/math/bib-files/references.bib}
%
% \bib, bibdiv, biblist are defined by the amsrefs package.
\begin{bibdiv}
\begin{biblist}

\bib{Farah:2009uq}{article}{
      author={Farah, Luiz~Gustavo},
       title={Local solutions in {S}obolev spaces with negative indices for the
  ``good'' {B}oussinesq equation},
        date={2009},
        ISSN={0360-5302},
     journal={Comm. Partial Differential Equations},
      volume={34},
      number={1-3},
       pages={52\ndash 73},
         url={http://dx.doi.org/10.1080/03605300802682283},
      review={\MR{2512853 (2010k:35404)}},
}


\bib{Ginibre:1996fk}{article}{
      author={Ginibre, Jean},
       title={Le probl{\`e}me de {C}auchy pour des {EDP} semi-lin{\'e}aires
  p{\'e}riodiques en variables d'espace (d'apr{\`e}s {B}ourgain)},
        date={1996},
        ISSN={0303-1179},
     journal={Ast{\'e}risque},
      number={237},
       pages={Exp.\ No.\ 796, 4, 163\ndash 187},
        note={S{{\'e}}minaire Bourbaki, Vol. 1994/95},
      review={\MR{1423623 (98e:35154)}},
}

\bib{Ginibre:1997fk}{article}{
      author={Ginibre, J.},
      author={Tsutsumi, Y.},
      author={Velo, G.},
       title={On the {C}auchy problem for the {Z}akharov system},
        date={1997},
        ISSN={0022-1236},
     journal={J. Funct. Anal.},
      volume={151},
      number={2},
       pages={384\ndash 436},
         url={http://dx.doi.org/10.1006/jfan.1997.3148},
      review={\MR{1491547 (2000c:35220)}},
}

\bib{Kenig:1996aa}{article}{
      author={Kenig, Carlos~E.},
      author={Ponce, Gustavo},
      author={Vega, Luis},
       title={A bilinear estimate with applications to the {K}d{V} equation},
        date={1996},
        ISSN={0894-0347},
     journal={J. Amer. Math. Soc.},
      volume={9},
      number={2},
       pages={573\ndash 603},
         url={http://dx.doi.org/10.1090/S0894-0347-96-00200-7},
      review={\MR{1329387 (96k:35159)}},
}

\bib{Kenig-Ponce-Vega-1996-Quadratic-forms-for-the-1-D-semilinear}{article}{
      author={Kenig, Carlos~E.},
      author={Ponce, Gustavo},
      author={Vega, Luis},
       title={Quadratic forms for the {$1$}-{D} semilinear {S}chr{\"o}dinger
  equation},
        date={1996},
        ISSN={0002-9947},
     journal={Trans. Amer. Math. Soc.},
      volume={348},
      number={8},
       pages={3323\ndash 3353},
         url={http://dx.doi.org/10.1090/S0002-9947-96-01645-5},
      review={\MR{1357398 (96j:35233)}},
}

\end{biblist}
\end{bibdiv}
% \bib, bibdiv, biblist are defined by the amsrefs package.
\end{document}
