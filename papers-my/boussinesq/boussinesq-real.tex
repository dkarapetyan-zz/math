%
%
\documentclass[12pt,reqno]{amsart}
\usepackage{amsmath}
\usepackage{amssymb}
\usepackage{cancel}  %for cancelling terms explicity on pdf
\usepackage{yhmath}   %makes fourier transform look nicer, among other things
\usepackage{framed}  %for framing remarks, theorems, etc.
\usepackage{enumerate} %to change enumerate symbols
\usepackage[margin=2.5cm]{geometry}  %page layout
\numberwithin{equation}{section}  %eliminate need for keeping track of counters
\setlength{\parindent}{0in} %no indentation of paragraphs after section title
\renewcommand{\baselinestretch}{1.1} %increases vert spacing of text
%
\usepackage{hyperref}
\hypersetup{colorlinks=true,
linkcolor=blue,
citecolor=blue,
urlcolor=blue,
}
\synctex=1
%\usepackage[alphabetic, initials, msc-links]{amsrefs} %for the bibliography;
%%uses cite pkg. Must be loaded after hyperref, otherwise doesn't work properly
%%(conflicts with cref in particular)
%
%
\newcommand{\ds}{\displaystyle}
\newcommand{\ts}{\textstyle}
\newcommand{\nin}{\noindent}
\newcommand{\rr}{\mathbb{R}}
\newcommand{\nn}{\mathbb{N}}
\newcommand{\zz}{\mathbb{Z}}
\newcommand{\cc}{\mathbb{C}}
\newcommand{\ci}{\mathbb{T}}
\newcommand{\zzdot}{\dot{\zz}}
\newcommand{\wh}{\widehat}
\newcommand{\p}{\partial}
\newcommand{\ee}{\varepsilon}
\newcommand{\vp}{\varphi}
\newcommand{\wt}{\widetilde}
%
%
%
%
\newtheorem{theorem}{Theorem}[section]
\newtheorem{lemma}[theorem]{Lemma}
\newtheorem{corollary}[theorem]{Corollary}
\newtheorem{claim}[theorem]{Claim}
\newtheorem{prop}[theorem]{Proposition}
\newtheorem{proposition}[theorem]{Proposition}
\newtheorem{no}[theorem]{Notation}
\newtheorem{definition}[theorem]{Definition}
\newtheorem{remark}[theorem]{Remark}
\newtheorem{examp}{Example}[section]
\newtheorem{exercise}[theorem]{Exercise}
%
\makeatletter \renewenvironment{proof}[1][\proofname]
{\par\pushQED{\qed}\normalfont\topsep6\p@\@plus6\p@\relax\trivlist\item[\hskip\labelsep\bfseries#1\@addpunct{.}]\ignorespaces}{\popQED\endtrivlist\@endpefalse}
\makeatother%
%makes proof environment bold instead of italic
%\makeatletter
%\renewcommand\subsubsection{\@startsection{subsubsection}{3}{\z@}%
%{-3.25ex\@plus -1ex \@minus -.2ex}%
%{1.5ex \@plus .2ex}%
%{\normalfont\normalsize \bfseries}}
%\makeatother
%makes subsubsubsection bold instead of italic
%
\def\sgn{\operatorname{sgn}}
\newcommand{\uol}{u^\omega_\lambda}
\newcommand{\lbar}{\bar{l}}
\renewcommand{\l}{\lambda}
\newcommand{\R}{\mathbb{R}}
\newcommand{\RR}{\mathcal{R}}
\newcommand{\al}{\alpha}
\newcommand{\ve}{q}
\newcommand{\tg}{{tan}}
\newcommand{\m}{q}
\newcommand{\N}{N}
\newcommand{\ta}{{\tilde{a}}}
\newcommand{\tb}{{\tilde{b}}}
\newcommand{\tc}{{\tilde{c}}}
\newcommand{\tS}{{\tilde{S}}}
\newcommand{\tP}{{\tilde{P}}}
\newcommand{\tu}{{\tilde{u}}}
\newcommand{\tw}{{\tilde{w}}}
\newcommand{\tA}{{\tilde{A}}}
\newcommand{\tX}{{\tilde{X}}}
\newcommand{\tphi}{{\tilde{\phi}}}
\begin{document}
\title{Boussinesq Non-Periodic Ill-Posedness}
\author{Dan-Andrei Geba, Alexandrou Himonas, and David Karapetyan}
\address{Department of Mathematics, University of Rochester, Rochester, NY 14627}
\address{Department of Mathematics, University of Notre Dame, Notre Dame, IN 46556}
\address{Department of Mathematics, University of Notre Dame, Notre Dame, IN 46556}
\date{\today}
%
%
\subjclass[2000]{35B30, 35Q55, 35Q72}
\keywords{local well-posedness; ill-posedness.}
\maketitle
%
%

%
%
%
%
%
%
\section{Ill-Posedness for $s < -1/2$}
Our motivation will be the work of Bejenaru and Tao
\cite{Bejenaru-Tao-2006-Sharp-well-posedness-and-ill-posedness}. 
%
\begin{definition}
  For $f =f(x)= (f_{1}(x), 0), u = u(x,t), v = v(x,t)$ formally define 
%
%
\begin{equation*}
\begin{split}
  L(f)
  \doteq \frac{1}{2 \pi} \psi(t) \int_{\rr} e^{i\xi x}
  \wh{f_{1}}(\xi) \cos \sqrt{\xi^{2} + \xi^{4}}t d \xi
  \end{split}
\end{equation*}
%
%
and
%
%
\begin{equation*}
\begin{split}
N(u, v)
& \doteq \frac{1}{2 \pi } \psi_{\delta}(t) \int_{\rr} e^{i\xi x}
\int_{0}^{t}\sin \sqrt{\xi^{2} + \xi^{4}}(t-t')
\frac{\xi^{2}}{\sqrt{\xi^{2} + \xi^{4}}}\wh{uv}(n, t') dt' d \xi
\end{split}
\end{equation*}
%
%
where $\psi(t)$ is a cutoff function
symmetric about the origin, equal to unity in $[-1,1]$, and supported in
$[-2, 2]$, with $\psi_{\delta}(t) = \psi(t/\delta)$.
Let $$A_{n}: H^{s} \times H^{s-2} \to C([0, \delta], H^{s}), \ n = 1, 2, \dots$$ be the
recursively defined maps
%
%
\begin{equation*}
\begin{split}
  & A_{1}(f) \doteq L(f),
  \\
  & A_{n}(f) \doteq \sum_{j, k \in \mathbb{N}: j + k = n} N\left[
  A_{j}(f), A_{k}(f) \right], \quad n > 1.
\end{split}
\end{equation*}
\end{definition}
%
%
%
For example, 
%
%
\begin{equation*}
\begin{split}
  & A_{2}(f_{N}) = N(A_{1}(f_{N}), A_{1}(f_{N}))
  \\
  & A_{3}(f_{N}) = N(A_{1}(f_{N}), A_{2}(f_{N})) + N(A_{2}(f_{N}), A_{2}(f_{N}))
  \\
  & A_{4}(f_{N})= N(A_{1}(f_{N}), A_{3}(f_{N})) + N(A_{2}(f_{N}), A_{2}(f_{N}))
  + N(A_{3}(f_{N}), A_{1}(f_{N}))
\end{split}
\end{equation*}
%
%
and so on. Due to the well-posedness of 
Boussinesq on the line for $s>-1/2$ established by
Kishimoto and Tsugawa \cite{Kishimoto:2010ly} and the work of
Tao and Bejenaru \cite{Bejenaru-Tao-2006-Sharp-well-posedness-and-ill-posedness},
we obtain the
following. 
%%
\begin{lemma}
  \label{lem:qwp-awp}
  For $f \in B_{H^{s} \times H^{s-2}}(r)$, $s > -1/2$,  and $\delta=\delta(r)$
sufficiently small, we have the absolutely convergent
(in $C([0, \delta], H^{s})$) power series expansion
%
%
\begin{equation}
  \label{power-series-soln}
\begin{split}
  u[f] = \sum_{n=1}^{\infty} A_{n}(f).
\end{split}
\end{equation}
%
%
\label{lem:analytic-wp}
\end{lemma}
%
%
From this, it follows that we have failure of continuity at $\vec{0} =
(0, 0)$, due to the
following.
%
%
%
%
%
%%%%%%%%%%%%%%%%%%%%%%%%%%%%%%%%%%%%%%%%%%%%%%%%%%%%%
%
%
%                Failure of Continuity at 0
%
%
%%%%%%%%%%%%%%%%%%%%%%%%%%%%%%%%%%%%%%%%%%%%%%%%%%%%%
%
%
\begin{theorem}
  Let $N$ be a positive integer, and $\ee > 0, \alpha > 0, r > 0$ real. Consider
  initial data $f_{N}(x) = (f_{N_{1}}(x), 0)$, where
  %
  %
    \begin{equation*}
  \begin{split}
    \wh{f_{N,1}}(\xi) = N^{-\alpha
    \ee}\chi_{[\frac{1}{2}N^{\alpha},N^{\alpha}]}(\xi).
  \end{split}
  \end{equation*}
  %
  %
  Then  
  %
  %
    \begin{enumerate}[(I)]
      \item{ For $N$ sufficiently large, 
        $ f_{N} \in B_{H^{-1/2 + \ee} \times H^{-5/2 + \ee}}(r)$}.
        \label{1uu}
        \\
  \item{ For $s<-1/2$, there exists $\alpha$ sufficiently small such that
    $$\frac{\|A_{2}(f_{N}) \|_{H^{s}}}{\| f_{N} \|_{H^{s}}} \to \infty.$$}
    \label{2uu}
\end{enumerate}
  %
  %
\label{thm:ill-pos}
\end{theorem}
%
%
%
%
%
%
%
%
%
%
%
\begin{proof}[Proof of \ref{1uu}]
    We have
  %
  %
  %
  %
  %
%
\begin{equation}
  \label{ill-pos-ce}
  \begin{split}
    \| f_{N,1} \|_{H^{s}}
    & = 
    N^{-\alpha \ee}\left( \int_{\rr} (1 + | \xi |)^{2s} \chi_{[0,N]}^{2}(\xi
    ) d \xi \right)^{1/2}
    \\
    & = N^{- \alpha \ee}
    \left[ \frac{1}{2s+1} (1 +  \xi )^{2s+1} \big
    |_{\frac{1}{2}N^{\alpha}}^{N^{\alpha}} \right]^{1/2},
    \quad s > -1/2.
\end{split}
\end{equation}
%
%
If $s = -1/2 + \ee$, this simplifies to
%
%
\begin{equation*}
\begin{split}
  \frac{N^{-\alpha \ee}}{\sqrt{\ee}}\left[ (1 + N^{\alpha})^{\ee} - (1 +
  \frac{1}{2}N^{\alpha})^{\ee} \right]^{1/2} \sim \frac{1}{\sqrt{\ee}}N^{-\alpha \ee/2}
  \to 0.
  \end{split}
\end{equation*}
%
%
Therefore, for given $r > 0$, there exists $N'$ such that for all $N > N'$, we
have
%
%
\begin{equation*}
  \begin{split}
    & (f_{N,1}, 0) \in B_{H^{-1/2 + \ee} \times H^{-5/2 + \ee}}(r).
  \end{split}
\end{equation*}
\end{proof}
%
%
%
\begin{proof}[Proof of \ref{2uu}]
Note that since the Boussinesq is well-posed on the line for $s
> -1/2$, and since $f_{N} \in B_{H^{-1/2 + \ee} \times H^{-5/2 + \ee}}(r)$,
the associated
solutions $u[f_{N}]$ have common lifespan $\delta$ in $H^{-1/2 + \ee} \times 
H^{-5/2 + \ee}$ (and hence, weaker topologies) \emph{which does not depend
on $N$}. Furthermore, since $r$
can be chosen to be arbitrarily small, we can assume without loss of generality that
$\delta =1$. Observe that
%
%
\begin{equation}
  \label{pre-loc}
\begin{split}
  & \| A_{2}(f_{N}) \|_{C([0, 1], H^{s})}^{2} 
  \\
  & =  \| N[A_{1}(f_{N}), A_{1}(f_{N})] \|_{C([0, 1],
  H^{s})}^{2} 
  \\
& \simeq \sup_{0 \le t \le 1}\| \frac{\xi^{2}}{\sqrt{\xi^{2} + \xi^{4}}}(1 + | \xi |)^{s}
\\
& \int_{0}^{t} \int_{\xi_{1}} \sin[\sqrt{\xi^{2} + \xi^{4}}(t-t')]
\cos[\sqrt{(\xi - \xi_{1})^{2} + (\xi - \xi_{1})^{4}}t']
\cos(\sqrt{\xi_{1}^{2} + \xi_{1}^{4}}t') \wh{f_{N}}(\xi - \xi_{1})\wh{f_{N}}(\xi_{1}) dt'
\|_{L_{2}}^{2}
\end{split}
\end{equation}
%
which is equal to
%
%
\begin{equation}
  \label{yu}
\begin{split}
  & \sup_{0 \le t \le 1} N^{-4 \alpha \ee}\int_{N^{\alpha} \le \xi \le 2 N^{\alpha}}
  \Big[ (1 + \xi)^{s}
  \frac{\xi^{2}}{\sqrt{\xi^{2} + \xi^{4}}} \int_{0}^{t}
  \int_{\frac{1}{2}
  N^{\alpha} \le \xi_{1} \le N^{\alpha}}
  \\
  & \sin[\sqrt{\xi^{2} + \xi^{4}}(t-t')]
  \cos[\sqrt{(\xi - \xi_{1})^{2} + (\xi - \xi_{1})^{4}}t']
\cos(\sqrt{\xi_{1}^{2} + \xi_{1}^{4}}t') d \xi_{1} dt' \Big ]^{2} d \xi.
\end{split}
\end{equation}
%
%
Set $t = N^{-2\alpha}$. Then for $N>>1$, we bound \eqref{yu} below by
%
%
%
%
\begin{equation*}
\begin{split}
  & c N^{-4 \alpha \ee}\int_{N^{\alpha} \le \xi \le 2 N^{\alpha}}
  \Big[ (1 + \xi)^{s}
  \frac{\xi^{2}}{\sqrt{\xi^{2} + \xi^{4}}} \int_{0}^{N^{-2 \alpha}}
  \int_{\frac{1}{2}
  N^{\alpha} \le \xi_{1} \le N^{\alpha}}
  \sin[\sqrt{\xi^{2} + \xi^{4}}(N^{-2 \alpha}-t')] d \xi_{1} dt' \Big ]^{2} d
  \xi.
\end{split}
\end{equation*}
%
%
Integrating in $\xi_{1}$ and $dt'$ and simplifying, this is equal to
%
%
%
%
\begin{equation*}
\begin{split}
  & c N^{2 \alpha - 4 \alpha \ee} \int_{N^{\alpha} \le \xi \le 2N^{\alpha}}
  (1 + \xi)^{2s} \frac{1}{(1 + \xi^{2})^{2}}
  \left ( 1 - \cos \frac{\sqrt{\xi^{2} + \xi^{4}}}{N^{2 \alpha}} \right )^{2} d \xi
  \\
  & \gtrsim N^{2 \alpha - 4 \alpha \ee} \int_{N^{\alpha} \le \xi \le 2N^{\alpha}}
  (1 + \xi)^{2s -4} 
\left ( 1 - \cos \frac{\sqrt{\xi^{2} + \xi^{4}}}{N^{2 \alpha}} \right )^{2} d \xi
    \\
    & \gtrsim N^{2\alpha - 4 \alpha \ee + \alpha(2s -4) } \int_{N^{\alpha} \le
    \xi \le 2 N^{\alpha}} d \xi
  \\
  & = N^{2s \alpha - 4 \alpha \ee - \alpha}.
\end{split}
\end{equation*}
%
%
Therefore,
%
%
%
\begin{equation}
  \label{a2-b}
\begin{split}
  \| A_{2}(f_{N}) \|_{C([0, 1], H^{s})} \gtrsim N^{s \alpha - 2 \alpha \ee -
  \alpha/2}.  
\end{split}
\end{equation}
%
%
%
Now, following the computations in \eqref{ill-pos-ce}, we see that for $s <
-1/2$
%
%
\begin{equation}
  \label{init-b}
\begin{split}
  \| f_{N,1} \|_{H^{s}} & \simeq N^{-\alpha \ee}\left[ (1 +
  \frac{1}{2}N)^{2 s +1} - (1 + N)^{2 s +1} \right]^{1/2}
  \\
  & = N^{-\alpha \ee}(1 + N)^{s + 1/2} \left[ \left( \frac{1 + N/2}{1 +
  N} \right)^{2 s +1} -1 \right]^{1/2}
  \\
  & \lesssim N^{s + 1/2 -\alpha \ee}, \quad N > > 1. 
\end{split}
\end{equation}
%
%
Therefore, from \eqref{a2-b} and \eqref{init-b}, we see that
%
%
\begin{equation*}
\begin{split}
  \frac{\| A_{2}(f_{N}) \|_{H^{s}}}{\| f_{N} \|_{H^{s}}}
  & \gtrsim \frac{N^{s \alpha - 2 \alpha \ee - \alpha/2}}{N^{s + 1/2 - \alpha \ee}}
  \\
  & = N^{s( \alpha -1) -1/2 - \alpha(\ee +1/2)} 
  \\
  & \to \infty, \quad s <
  \frac{1}{1 - \alpha}\left[ -\frac{1}{2} - \alpha(\ee + \frac{1}{2}) \right].
\end{split}
\end{equation*}
%
%
Since $\alpha$ and $\ee$ can be chosen to be arbitrarily small, the proof is
complete. 
%
%
%
%
%
\end{proof}
%
%
%
%
%
\begin{thebibliography}{KT10}

\bibitem[BT06]{Bejenaru-Tao-2006-Sharp-well-posedness-and-ill-posedness}
I.~Bejenaru and T.~Tao, \emph{Sharp well-posedness and ill-posedness results
  for a quadratic non-linear schr{\"o}dinger equation}, J. Funct. Anal.
  \textbf{233} (2006), no.~1, 228--259.

\bibitem[KT10]{Kishimoto:2010ly}
N.~Kishimoto and K.~Tsugawa, \emph{Local well-posedness for quadratic nonlinear
  {S}chr{\"o}dinger equations and the ``good'' {B}oussinesq equation},
  Differential Integral Equations \textbf{23} (2010), no.~5-6, 463--493.

\end{thebibliography}
%
%
%
%
%
%\bibliography{/Users/davidkarapetyan/math/bib-files/references.bib}
%\bibliographystyle{amsalpha-custom}
\end{document}
