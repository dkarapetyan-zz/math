%
%
\documentclass[12pt,reqno]{amsart}
\usepackage{amsmath}
\usepackage{amssymb}
\usepackage{cancel}  %for cancelling terms explicity on pdf
\usepackage{yhmath}   %makes fourier transform look nicer, among other things
\usepackage{framed}  %for framing remarks, theorems, etc.
\usepackage{enumerate} %to change enumerate symbols
\usepackage[margin=2.5cm]{geometry}  %page layout
\numberwithin{equation}{section}  %eliminate need for keeping track of counters
\setlength{\parindent}{0in} %no indentation of paragraphs after section title
\renewcommand{\baselinestretch}{1.1} %increases vert spacing of text
%
\usepackage{hyperref}
\hypersetup{colorlinks=true,
linkcolor=blue,
citecolor=blue,
urlcolor=blue,
}
%\usepackage[alphabetic, initials, msc-links]{amsrefs} %for the bibliography;
%%uses cite pkg. Must be loaded after hyperref, otherwise doesn't work properly
%%(conflicts with cref in particular)
%
%
\newcommand{\ds}{\displaystyle}
\newcommand{\ts}{\textstyle}
\newcommand{\nin}{\noindent}
\newcommand{\rr}{\mathbb{R}}
\newcommand{\nn}{\mathbb{N}}
\newcommand{\zz}{\mathbb{Z}}
\newcommand{\cc}{\mathbb{C}}
\newcommand{\ci}{\mathbb{T}}
\newcommand{\zzdot}{\dot{\zz}}
\newcommand{\wh}{\widehat}
\newcommand{\p}{\partial}
\newcommand{\ee}{\varepsilon}
\newcommand{\vp}{\varphi}
\newcommand{\wt}{\widetilde}
%
%
%
%
\newtheorem{theorem}{Theorem}[section]
\newtheorem{lemma}[theorem]{Lemma}
\newtheorem{corollary}[theorem]{Corollary}
\newtheorem{claim}[theorem]{Claim}
\newtheorem{prop}[theorem]{Proposition}
\newtheorem{proposition}[theorem]{Proposition}
\newtheorem{no}[theorem]{Notation}
\newtheorem{definition}[theorem]{Definition}
\newtheorem{remark}[theorem]{Remark}
\newtheorem{examp}{Example}[section]
\newtheorem{exercise}[theorem]{Exercise}
%
\makeatletter \renewenvironment{proof}[1][\proofname]
{\par\pushQED{\qed}\normalfont\topsep6\p@\@plus6\p@\relax\trivlist\item[\hskip\labelsep\bfseries#1\@addpunct{.}]\ignorespaces}{\popQED\endtrivlist\@endpefalse}
\makeatother%
%makes proof environment bold instead of italic
%\makeatletter
%\renewcommand\subsubsection{\@startsection{subsubsection}{3}{\z@}%
%{-3.25ex\@plus -1ex \@minus -.2ex}%
%{1.5ex \@plus .2ex}%
%{\normalfont\normalsize \bfseries}}
%\makeatother
%makes subsubsubsection bold instead of italic
%
\def\sgn{\operatorname{sgn}}
\def\sech{\operatorname{sech}}
\newcommand{\uol}{u^\omega_\lambda}
\newcommand{\lbar}{\bar{l}}
\renewcommand{\l}{\lambda}
\newcommand{\R}{\mathbb{R}}
\newcommand{\RR}{\mathcal{R}}
\newcommand{\al}{\alpha}
\newcommand{\ve}{q}
\newcommand{\tg}{{tan}}
\newcommand{\m}{q}
\newcommand{\N}{N}
\newcommand{\ta}{{\tilde{a}}}
\newcommand{\tb}{{\tilde{b}}}
\newcommand{\tc}{{\tilde{c}}}
\newcommand{\tS}{{\tilde{S}}}
\newcommand{\tP}{{\tilde{P}}}
\newcommand{\tu}{{\tilde{u}}}
\newcommand{\tw}{{\tilde{w}}}
\newcommand{\tA}{{\tilde{A}}}
\newcommand{\tX}{{\tilde{X}}}
\newcommand{\tphi}{{\tilde{\phi}}}
\begin{document}
\title{A Modified Boussinesq equation}
\author{Dan-Andrei Geba, Alexandrou Himonas, and David Karapetyan}
\address{Department of Mathematics, University of Rochester, Rochester, NY 14627}
\address{Department of Mathematics, University of Notre Dame, Notre Dame, IN 46556}
\address{Department of Mathematics, University of Notre Dame, Notre Dame, IN 46556}
\date{04/26/2011}
%
%
\subjclass[2000]{35B30, 35Q55, 35Q72}
\keywords{local well-posedness; ill-posedness.}
\maketitle
%
%
%%%%%%%%%%%%%%%%%%%%%%%%%%%%%%%%%%%%%%%%%%%%%%%%%%%%%
%
%
%                Traveling waves for B4
%
%
%%%%%%%%%%%%%%%%%%%%%%%%%%%%%%%%%%%%%%%%%%%%%%%%%%%%%
%
%
\section{Traveling Waves for the Boussinesq Equation} 
\label{sec:trav-wave}
We are looking for functions $f(x)$ such that
%
%
\begin{equation}
  \label{ansatz}
\begin{split}
u(x,t) = f(x-ct)
\end{split}
\end{equation}
%
%
satisfies the Boussinesq equation
%
%
\begin{equation}
  \label{bous-eqn}
\begin{split}
  u_{tt} - u_{xx} + u_{xxxx} + (u^{2})_{xx} = 0
\end{split}
\end{equation}
%
%
Substituting \eqref{ansatz} into \eqref{bous-eqn} gives
%
%
\begin{equation}
\begin{split}
  (c^{2}-1)f'' + f'''' + (f^{2})'' = 0
\end{split}
\label{post-sub}
\end{equation}
%
%
where $$f^{(k)} = f^{(k)}(x-ct) = \frac{d^{k}f}{dx^{k}} \Big |_{x-ct}.$$ 
Integrating \eqref{post-sub} twice with repspect to $x$, we obtain
%
%
\begin{equation}
  \label{pre-const-elim}
\begin{split}
  (c^{2}-1)f + f'' + f^{2} + ax +b = 0, \quad a,b \in \rr.
\end{split}
\end{equation}
%
%

%
%
In order to construct traveling wave solutions from $f$, we need to arrange that
$\lim_{x \to \infty} f(x) = D \in \rr$. But if $a \neq 0$ in
\eqref{pre-const-elim}, then it will be impossible to construct a traveling wave
solution using \eqref{pre-const-elim}. To see this, we proceed by contradiction
and assume that $\lim_{x \to
\infty} f(x) = D$. Then
%
%
\begin{equation*}
\begin{split}
  0 = \lim_{x \to \infty} f'' = \lim_{x \to \infty} [-f^{2} -f(c^{2}-1) - ax -b]
\end{split}
\end{equation*}
%
%
which would imply
%
%
\begin{equation*}
\begin{split}
  \lim_{x \to \infty} | f(x) |  = \infty
\end{split}
\end{equation*}
%
%
which is a contradiction. Hence, taking $a = 0$ in \eqref{pre-const-elim} we
obtain
%
\begin{equation}
  \label{2nd-order-ode-pre}
\begin{split}
  f'' = (1-c^{2})f - f^{2} -b 
\end{split}
\end{equation}
%
%
Multiplying both sides of \eqref{2nd-order-ode-pre} by $f'$, integrating, and rearranging terms we obtain 
%
\begin{equation}
  \label{first-order-ode}
\begin{split}
  (f')^{2} = -\frac{2}{3} \left [ f^{3} + \frac{3}{2} (c^{2}-1)f^{2} -3bf 
  \right ]
\end{split}
\end{equation}
%
%
where we have taken the integration constant to be zero. 
We now restrict our attention to solutions of \eqref{first-order-ode} where
we attain a local maximum $M$ and local minimum $m$. Then $b = b(m, M)$ is
completely determined by the expression
%
%
\begin{equation*}
\begin{split}
& 0 = -\frac{2}{3} \left [ M^{3} + \frac{3}{2} (c^{2} -1)M^{2} -3bM  
  \right ]
\end{split}
\end{equation*}
%
giving
%
%
\begin{equation}
  \label{b-val}
\begin{split}
  b = \frac{M^{2}}{3} + \frac{M}{2}(c^{2}-1).
\end{split}
\end{equation}
%
%
%
Furthermore, recall that for a cubic polynomial $g(x) = x^{3} +bx^{2} +cx +d$
with roots $r_{1}, r_{2}, r_{3}$, we have $r_{1} + r_{2} + r_{3} = -b$.
%
%
\begin{framed}
%
%
\begin{remark}
\label{rem:fact-pf}
For a proof, expand 
$g(x) = (x-a)(x-b)(x-c)$.
\end{remark}
%
%
\end{framed}
%
%
Combining this information with the fundamental theorem of algebra, 
we see that \eqref{first-order-ode} has the factorization
%
%
%
%
\begin{equation}
\begin{split}
  \left ( f'\right )^{2}
  = \frac{2}{3} \left (f-m\right )
  \left( M-f\right )
  \left [ f + \frac{3}{2}(c^{2} -1) +m +M  \right ]
\end{split}
\label{ode-fact}
\end{equation}
%
For fixed $c$, we choose $m, M$ such that 
\begin{equation}
  \label{max-min-bound}
  m + \frac{3}{2}(c^{2} -1) + m + M > 0 
\end{equation}
%
ensuring that the third term in the product on the right hand side of
\eqref{ode-fact} is always positive.
%
%\begin{framed}
%%
%%
%\begin{remark}
%Note that this is always possible, for any $c$. To se this, note that relation
%\eqref{max-min-bound} is equivalent to
%%
%%
%\begin{equation*}
%\begin{split}
  %2m+M < \frac{3}{2}(c^{2} -1) < m+2M
%\end{split}
%\end{equation*}
%%
%%
%Hence, it will be enough to show that for fixed $C$, there exist $m, M$ such
%that
%%
%%
%\begin{equation*}
%\begin{split}
  %2m+M < C < m+2M
%\end{split}
%\end{equation*}
%%
%%
%For fixed $C > 0$, simply take $m > 0$, $M = 2m$. Then we reduce to showing that
%there exists $m$ such that
%\begin{equation*}
%\begin{split}
  %4m < C < 5m
%\end{split}
%\end{equation*}
%But this is clearly true (take $m = C/5 + \ee$, where $\ee > 0$ is as small
%as we like). For fixed $C < 0$, take $m < 0$, $m = 2M$. Then we reduce to showing that
%there exists $M$ such that
%\begin{equation*}
%\begin{split}
  %5M < C < 4M
%\end{split}
%\end{equation*}
%But this is also true (take $M = C/5 - \ee$, where $\ee > 0$ is as small
%as we like). 
%\label{rem:pos}
%\end{remark}
%%
%%
%\end{framed}
%
%

Then our solution to \eqref{ode-fact} has a
local maximum $M$ and a local minimum $m$. Consider the Cauchy problem
%
\begin{gather}
  \label{ode-cauchy}
   \left ( f'\right )^{2}
  = \frac{2}{3} \left (f-m\right )
  \left(M-f \right )
  \left [ f + \frac{3}{2}(c^{2} -1) +m +M \right ],
  \\
\label{ode-cauchy-data}
   f(0) = m.
\end{gather}
%
%
We have the following result.
%
%
%%%%%%%%%%%%%%%%%%%%%%%%%%%%%%%%%%%%%%%%%%%%%%%%%%%%%
%
%
%                
%
%
%%%%%%%%%%%%%%%%%%%%%%%%%%%%%%%%%%%%%%%%%%%%%%%%%%%%%
%
%
\begin{lemma}
  For fixed $c \ge 1$ and $M > m > 0$,  there exists a positive
  number $\ell = \ell(m, M)$ and an even $2\ell$-periodic analytic function $f =
  f(x)$ which solves the initial value problem
\begin{align}
  \label{2nd-order-ode}
& f'' = (1-c^{2})f - f^{2} - \left [\frac{M^{2}}{3} + \frac{M}{2}(c^{2}-1)
\right]
\\
\label{2nd-order-ode-data}
& f(0)=m, \quad f'(0) = 0.
\end{align}
Furthermore, the function
  $f(x)$ satisfies
  %
  %
  \begin{equation}
    \label{f-bound}
  \begin{split}
  m \le f(x) \le M
  \end{split}
  \end{equation}
  %
  %
  and the function $u(x,t) = f(x-ct)$ is a traveling wave solution of the
  Boussinesq equation. 
\label{lem:ode-solution}
\end{lemma}
%
%
%
\begin{proof}
By the fundamental ode theorem, the Cauchy problem
\begin{gather}
  \label{ode-cauchy-left}
   f'
  = -\left \{ \frac{2}{3} \left (f-m\right )
  \left( M -f \right )
  \left [ f + \frac{3}{2}(c^{2} -1) +m +M \right ] \right
  \}^{1/2},
  \\
\label{ode-cauchy-data-left}
   f(0) = m
\end{gather}
admits a unique solution in the domain $-\ell \le x \le 0$, where $\ell =
\ell(m, M)$ and $f(-\ell) = M$. Reflecting this solution across the $y$-axis
gives the unique solution to the Cauchy problem 
\begin{gather}
  \label{ode-cauchy-right}
   f'
  = \left \{ \frac{2}{3} \left (f-m\right )
  \left( M -f \right )
  \left [ f + \frac{3}{2}(c^{2} -1) +m +M  \right ] \right
  \}^{1/2},
  \\
\label{ode-cauchy-data-right}
   f(0) = m
\end{gather}
in the domain $0 \le x \le \ell$, where $f(\ell) = M$. 
Changing the
initial data from $f(0) = m$ to $f(k\ell)=m$ where $k \in \left\{ 2, 4, 6,\dots
\right\}$, repeating the process above, and then joining the resulting curves
together at the points $n \ell$, where $n \in \left\{ 1,3,5,\dots \right\}$, we
obtain a $2\ell$-periodic
solution to the Cauchy problem \eqref{ode-cauchy}-\eqref{ode-cauchy-data}.
Hence, it is a solution to the Cauchy problem
\eqref{2nd-order-ode}-\eqref{2nd-order-ode-data}. Analyticity follows from the
Cauchy-Kovalevskaya theorem. The
estimate \eqref{f-bound} follows by construction.
%
%
\end{proof}
%
%
%
%
%
%%%%%%%%%%%%%%%%%%%%%%%%%%%%%%%%%%%%%%%%%%%%%%%%%%%%%
%
%
%                Period Estimate
%
%
%%%%%%%%%%%%%%%%%%%%%%%%%%%%%%%%%%%%%%%%%%%%%%%%%%%%%
%
%
\begin{lemma}
  The period $2 \ell$ of the solution to the Cauchy problem
  \eqref{2nd-order-ode}-\eqref{2nd-order-ode-data} described in Lemma
  \ref{lem:ode-solution} satisfies
  %
  %
  \begin{equation}
  \begin{split}
    \ell \sim \left [m + M + \frac{3}{2}(c^{2}-1) \right ]^{-1/2}.
  \end{split}
  \label{period-est}
  \end{equation}
  %
  %
\label{lem:period-est}
\end{lemma}
%
%
%
%
\begin{proof}
For $0 < x < \ell$, we have $f'(x) > 0$. Hence, the function $f = f(x)$ has an
inverse $x = x(f)$, and so by the fundamental theorem of calculus and
\eqref{ode-cauchy-right}
%
%
\begin{equation*}
\begin{split}
\ell & = x(m + M) - x(m)
\\
& = \int_{m}^{M} \frac{dx}{df} df
\\
& = \int_{m}^{M} \frac{1}{ \left\{ \frac{2}{3}(f-m)(M-f)
\left[ f +
\frac{3}{2}(c^{2}-1) + m + M \right] \right\}^{1/2}} df.
\end{split}
\end{equation*}
%
%
Bounding from above, we have
%
%
\begin{equation*}
\begin{split}
\ell
& \le \frac{\sqrt{\frac{3}{2}}}{\left[ m + \frac{3}{2}(c^{2}-1) + m + M
\right]^{1/2}} \int_{m}^{M} \frac{1}{(f-m)^{1/2}(M-f)^{1/2}}df
\\
& = \frac{\sqrt{\frac{3}{2}}}
{\left[ m + \frac{3}{2}(c^{2}-1) + m + M
\right]^{1/2}} 
\\
& \times \left[ \int_{m}^{\frac{M+m}{2}} \frac{1}{(f-m)^{1/2}(M-f)^{1/2}}df 
+ \int_{\frac{M+m}{2}}^{M} \frac{1}{(f-m)^{1/2}(M-f)^{1/2}}df
\right]
\\
& = \frac{\sqrt{\frac{3}{2}}}
{\left[ m + \frac{3}{2}(c^{2}-1) + m + M
\right]^{1/2}} 
\\
& \times \frac{1}{\left( \frac{M-m}{2} \right)^{1/2}} \left[ \int_{m}^{\frac{M+m}{2}} \frac{1}{(f-m)^{1/2}}df 
+ \int_{\frac{M+m}{2}}^{M} \frac{1}{(M-f)^{1/2}}df
\right]
\\
& = \frac{4 \sqrt{\frac{3}{2}}}
{\left[ m + \frac{3}{2}(c^{2}-1) + m + M
\right]^{1/2}} 
\\
& \le 2 \sqrt{6} \left [m+M + \frac{3}{2} (c^{2} -1) \right ]^{-1/2}.
\end{split}
\end{equation*}
%
%
Bounding from below, we have
%
%
\begin{equation*}
\begin{split}
\ell
& \ge  \frac{1}{ \left\{ \frac{2}{3}(M-m)(M-m)
\left[ m + M +
\frac{3}{2}(c^{2}-1) + m + M \right] \right\}^{1/2}} \int_{m}^{M} df
\\
& = \frac{1}{\left\{ \frac{2}{3}
\left[ m + M +
\frac{3}{2}(c^{2}-1) + m + M \right] \right\}^{1/2}} 
\\
& = \frac{\sqrt{\frac{3}{2}}}{\left[ 2m + 2M 
+ \frac{3}{2}(c^{2}-1) \right]^{1/2}} 
\\
& = \frac{\sqrt{\frac{3}{4}}}{\left[ m + M 
+ \frac{3}{4}(c^{2}-1) \right]^{1/2}} 
\\
& \ge \frac{\sqrt{3}}{2} \left[ m + M 
+ \frac{3}{2}(c^{2}-1) \right]^{-1/2}
\end{split}
\end{equation*}
%
%
which completes the proof.
\end{proof}
%
%
We have the estimate
%
%
\begin{equation*}
\begin{split}
  \| f' \|^{2}_{L^{2}(-\ell, \ell)}
  & \simeq \int_{0}^{\ell} (f')^{2} dx
  \\
  & \simeq \int_{0}^{\ell} (f-m)(M-f)\left[ f + \frac{3}{2}(c^{2}-1)+m + M
  \right]dx
  \\
  & \le (M-m)^{2} \left[ M + \frac{3}{2}(c^{2}-1) +m + M \right] \ell
  \\
  & \lesssim (M-m)^{2} \left[ m + 2M + \frac{3}{2}(c^{2}-1)  \right]
  \left[ m + M + \frac{3}{2}(c^{2}-1)\right]^{-1/2}
  \\
  & \le 2 (M-m)^{2}\left[ M + \frac{3}{2}(c^{2}-1) +m + M \right]^{1/2}
\end{split}
\end{equation*}
%
%
Furthermore,
\begin{equation*}
\begin{split}
  \| f'' \|^{2}_{L^{2}(-\ell, \ell)}
  & \simeq \int_{0}^{\ell} (f'')^{2} dx
  \\
  & \simeq \int_{0}^{\ell}
  \left\{ (1-c^{2})f - f^{2} - \left [\frac{M^{2}}{3} + \frac{M}{2}(c^{2}-1)
  \right] \right\}dx
  \\
  & \le (M-m)^{2} \left[ M + \frac{3}{2}(c^{2}-1) +m + M \right] \ell
  \\
  & \lesssim (M-m)^{2} \left[ m + 2M + \frac{3}{2}(c^{2}-1)  \right]
  \left[ m + M + \frac{3}{2}(c^{2}-1)\right]^{-1/2}
  \\
  & \le 2 (M-m)^{2}\left[ M + \frac{3}{2}(c^{2}-1) +m + M \right]^{1/2}
\end{split}
\end{equation*}


%
%%%%%%%%%%%%%%%%%%%%%%%%%%%%%%%%%%%%%%%%%%%%%%%%%%%%%
%
%
%                Solitons
%
%
%%%%%%%%%%%%%%%%%%%%%%%%%%%%%%%%%%%%%%%%%%%%%%%%%%%%%
%
%
\section{Solitons for the Boussinesq} 
\label{sec:soliton}
We rewrite \eqref{first-order-ode} as
%
%
\begin{equation}
  \label{first-order-rewritten}
\begin{split}
  \frac{(f')^{2}}{f} = -\frac{2}{3}\left[ f^{2} + \frac{3}{2}(c^{2}-1)f - 3b -
  \frac{3d}{2f}\right].
\end{split}
\end{equation}
%
%
Assume $\lim_{|x| \to \infty} f'' =0$. Then by L'H\^opital's rule
%
%
\begin{equation*}
\begin{split}
  \lim_{|x| \to \infty} \frac{(f')^{2}}{f} = \lim_{|x| \to \infty} \frac{2f'
  f''}{f'} = \lim_{|x| \to \infty} 2f'' = 0.
\end{split}
\end{equation*}
%
%
Coupling this with our earlier assumption of $\lim_{| x | \to \infty} f' = \lim_{| x
| \to \infty} f = 0$, we see that we must have $b =d=0$. Hence, from 
\eqref{first-order-ode} we obtain 
%
%
\begin{equation}
  \label{bous-ode-simp}
\begin{split}
  (f')^{2} = \frac{2}{3} f^{2} \left [ \frac{3}{2} (1-c^{2}) -f \right ]. 
\end{split}
\end{equation}
%
%
Assume $$0 \le |c| < 1 \ \text{and} \ 0 \le f \le \frac{3}{2}(1-c^{2}).$$ 
%
%
Then
%
%
\begin{equation*}
\begin{split}
  f' = \sqrt{\frac{2}{3}}f \left[ \frac{3}{2}(1 - c^{2}) -f \right]^{1/2}
\end{split}
\end{equation*}
%
%
Letting $y = f(x)$, this gives
%
%
\begin{equation*}
\begin{split}
\frac{dy}{ f \left[ \frac{3}{2}(1 - c^{2}) -f \right]^{1/2}} = \sqrt{\frac{2}{3}}
dx
\end{split}
\end{equation*}
%
%
Integrating both sides, and using the formula
%
%
\begin{equation*}
\begin{split}
  \int \frac{dy}{y \sqrt{a + by}} = -\frac{2}{ \sqrt{a}} \tanh^{-1}
  \frac{\sqrt{a+ by}}{ \sqrt{a}}
\end{split}
\end{equation*}
%
%
we obtain
%
%
\begin{equation*}
\begin{split}
  -\frac{2}{\sqrt{\frac{3}{2}(1-c^{2})}} \tanh^{-1}
  \frac{\sqrt{\frac{3}{2}(1-c^{2}) - y}}{\sqrt{\frac{3}{2}(1-c^{2})}} =
  \sqrt{\frac{2}{3}}x + A, \quad A \in \rr
\end{split}
\end{equation*}
%
%
or
\begin{equation*}
\begin{split}
  \tanh^{-1}
  \frac{\sqrt{\frac{3}{2}(1-c^{2}) - y}}{\sqrt{\frac{3}{2}(1-c^{2})}} =
  - \frac{\sqrt{(1-c^{2})}}{2}x + B, \quad B \in \rr
\end{split}
\end{equation*}
%
Assume that
%
%
\begin{equation*}
\begin{split}
  y(x_{0}) = \frac{3}{2}(1-c^{2})
\end{split}
\end{equation*}
%
%
for fixed $x_{0}$. Then
%
%
\begin{equation*}
\begin{split}
  0 = \tanh^{-1}(0) = - \frac{\sqrt{(1-c^{2})}}{2}x_{0} + B
\end{split}
\end{equation*}
%
%
or
%
%
\begin{equation*}
\begin{split}
B = \frac{\sqrt{(1-c^{2})}}{2}x_{0}.
\end{split}
\end{equation*}
%
%
Therefore,
%
%
\begin{equation*}
\begin{split}
  \frac{\sqrt{\frac{3}{2}(1-c^{2}) - y}}{\sqrt{\frac{3}{2}(1-c^{2})}} =
  \tanh \left [- \frac{\sqrt{(1-c^{2})}}{2}(x -x_{0}) \right ] 
\end{split}
\end{equation*}
%
%
or
%
%
%
\begin{equation*}
\begin{split}
  y = \frac{3}{2}(1-c^{2}) \left \{ 1 - \tanh^{2} \left [- \frac{\sqrt{(1-c^{2})}}{2}(x -x_{0})
  \right ] \right \}
  \end{split}
\end{equation*}
%
%
Using the identities
%
%
\begin{equation*}
\begin{split}
  & 1 - \tanh^{2}x = \sech^{2}x,
  \\
  & \sech^{2}(-x) = \sech^{2}x
\end{split}
\end{equation*}
%
we conclude that
%
%
\begin{equation*}
\begin{split}
  y(x) = \frac{3}{2}(1-c^{2}) \sech^{2} \left [\frac{\sqrt{(1-c^{2})}}{2}(x -x_{0})
  \right ]. 
  \end{split}
\end{equation*}
%
%
%
%%%%%%%%%%%%%%%%%%%%%%%%%%%%%%%%%%%%%%%%%%%%%%%%%%%%%
%
%
%                B4 no Solitons
%
%
%%%%%%%%%%%%%%%%%%%%%%%%%%%%%%%%%%%%%%%%%%%%%%%%%%%%%
%
%
\section{Why the $B_{4}$ Equation Doesn't Seem to Admit Solitons} 
\label{sec:B4-soliton-fail}
%
%
The analog of \eqref{bous-ode-simp} for the $B_{4}$ ivp
\begin{equation}
  \label{b4}
\begin{split}
  u_{tt} + u_{xxxx} + (u^{2})_{xx} = 0
\end{split}
\end{equation}
is
\begin{equation}
  \label{b4-ode-simp}
\begin{split}
  (f')^{2} = \frac{2}{3} f^{2} \left [ -\frac{3}{2}c^{2} -f \right ]. 
\end{split}
\end{equation}
Assume $f \le -\frac{3}{2}c^{2}$. Then 
%
\begin{equation*}
\begin{split}
  f' = \sqrt{\frac{2}{3}}f \left[ -\frac{3}{2}c^{2} -f \right]^{1/2}
\end{split}
\end{equation*}
%
%
Letting $y = f(x)$, this gives
%
%
\begin{equation*}
\begin{split}
\frac{dy}{ f \left[ -\frac{3}{2}c^{2} -f \right]^{1/2}} = \sqrt{\frac{2}{3}}
dx
\end{split}
\end{equation*}
%
Integrating both sides, we obtain
%
%
%
%
\begin{equation*}
\begin{split}
\int \frac{dy}{ f \left[ -\frac{3}{2}c^{2} -f \right]^{1/2}} = \sqrt{\frac{2}{3}}
x + A, \quad A \in \rr
\end{split}
\end{equation*}
%
%
Using the substitution $y = -\frac{3}{2}c^{2}\sec^{2} \theta$, we see that
%
%
\begin{equation*}
\begin{split}
\int \frac{dy}{ f \left[ -\frac{3}{2}c^{2} -f \right]^{1/2}} 
& = \int \frac{-3 c^{2} \sec^{2}\theta \tan \theta d
\theta}{-\frac{3}{2}c^{2} \sec^{2} \theta \left[ -\frac{3}{2}c^{2} +
\frac{3}{2}c^{2} \sec^{2} \theta \right]^{1/2}}
\\
& = \int \frac{4}{3c^{2}} d \theta
\\
& = \frac{4}{3c^{2}} \theta
\\
& = \frac{4}{3c^{2}} \sec^{-2}\left (-\frac{2}{3c^{2}}y \right )
\end{split}
\end{equation*}
%
%
and so
%
%
\begin{equation*}
\begin{split}
 \frac{4}{3c^{2}} \sec^{-2}\left (-\frac{2}{3c^{2}}y \right )
= \sqrt{\frac{2}{3}}
x + A
\end{split}
\end{equation*}
%
%
implying that $y$ is periodic. But we assumed a priori that $y(x) \to 0$ as $x
\to \infty$. Since $y \not \equiv 0$, we have a contradiction.
\end{document}
