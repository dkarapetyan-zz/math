%
\documentclass[12pt,reqno]{amsart}
\usepackage{amssymb}
\usepackage{appendix}
\usepackage[showonlyrefs=true]{mathtools} %amsmath extension package
\usepackage{cancel}  %for cancelling terms explicity on pdf
\usepackage{yhmath}   %makes fourier transform look nicer, among other things
\usepackage{framed}  %for framing remarks, theorems, etc.
\usepackage{enumerate} %to change enumerate symbols
\usepackage[margin=2.5cm]{geometry}  %page layout
\setcounter{tocdepth}{1} %must come before secnumdepth--else, pain
\setcounter{secnumdepth}{1} %number only sections, not subsections
%\usepackage[pdftex]{graphicx} %for importing pictures into latex--pdf compilation
\numberwithin{equation}{section}  %eliminate need for keeping track of counters
%\numberwithin{figure}{section}
\setlength{\parindent}{0in} %no indentation of paragraphs after section title
\renewcommand{\baselinestretch}{1.1} %increases vert spacing of text
%
\usepackage{hyperref}
\hypersetup{colorlinks=true,
linkcolor=blue,
citecolor=blue,
urlcolor=blue,
}
\usepackage[alphabetic, initials, msc-links]{amsrefs} %for the bibliography; uses cite pkg. Must be loaded after hyperref, otherwise doesn't work properly (conflicts with cref in particular)
\usepackage{cleveref} %must be last loaded package to work properly
%
%
\newcommand{\ds}{\displaystyle}
\newcommand{\ts}{\textstyle}
\newcommand{\nin}{\noindent}
\newcommand{\rr}{\mathbb{R}}
\newcommand{\nn}{\mathbb{N}}
\newcommand{\zz}{\mathbb{Z}}
\newcommand{\cc}{\mathbb{C}}
\newcommand{\ci}{\mathbb{T}}
\newcommand{\zzdot}{\dot{\zz}}
\newcommand{\wh}{\widehat}
\newcommand{\p}{\partial}
\newcommand{\ee}{\varepsilon}
\newcommand{\vp}{\varphi}
\newcommand{\wt}{\widetilde}
%
%
%
%
\newtheorem{theorem}{Theorem}[section]
\newtheorem{lemma}[theorem]{Lemma}
\newtheorem{corollary}[theorem]{Corollary}
\newtheorem{claim}[theorem]{Claim}
\newtheorem{prop}[theorem]{Proposition}
\newtheorem{proposition}[theorem]{Proposition}
\newtheorem{no}[theorem]{Notation}
\newtheorem{definition}[theorem]{Definition}
\newtheorem{remark}[theorem]{Remark}
\newtheorem{examp}{Example}[section]
\newtheorem {exercise}[theorem] {Exercise}
%
\makeatletter \renewenvironment{proof}[1][\proofname] {\par\pushQED{\qed}\normalfont\topsep6\p@\@plus6\p@\relax\trivlist\item[\hskip\labelsep\bfseries#1\@addpunct{.}]\ignorespaces}{\popQED\endtrivlist\@endpefalse} \makeatother%
%makes proof environment bold instead of italic
\newcommand{\uol}{u^\omega_\lambda}
\newcommand{\lbar}{\bar{l}}
\renewcommand{\l}{\lambda}
\newcommand{\R}{\mathbb R}
\newcommand{\RR}{\mathcal R}
\newcommand{\al}{\alpha}
\newcommand{\ve}{q}
\newcommand{\tg}{{tan}}
\newcommand{\m}{q}
\newcommand{\N}{N}
\newcommand{\ta}{{\tilde{a}}}
\newcommand{\tb}{{\tilde{b}}}
\newcommand{\tc}{{\tilde{c}}}
\newcommand{\tS}{{\tilde S}}
\newcommand{\tP}{{\tilde P}}
\newcommand{\tu}{{\tilde{u}}}
\newcommand{\tw}{{\tilde{w}}}
\newcommand{\tA}{{\tilde{A}}}
\newcommand{\tX}{{\tilde{X}}}
\newcommand{\tphi}{{\tilde{\phi}}}
\synctex=1
\begin{document}
\title{A Modified Boussinesq equation}
\author{Dan-Andrei Geba, Alexandrou Himonas, and David Karapetyan}
\address{Department of Mathematics, University of Rochester, Rochester, NY 14627}
\address{Department of Mathematics, University of Notre Dame, Notre Dame, IN 46556}
\address{Department of Mathematics, University of Notre Dame, Notre Dame, IN 46556}
\date{}
%
%
\subjclass[2000]{35B30, 35Q55, 35Q72}
\keywords{local well-posedness; ill-posedness.}
\maketitle
%
%
\section{Introduction}
%
We consider the initial value problem (ivp) for the fourth order modified Boussinesq
($B_4$) equation 
\begin{gather}
  u_{tt}   + u_{xxxx} + (u^2)_{xx} = 0, \quad x \in \rr \text{ or }
  \ci,\text{ } t \in \rr
  \label{eqn:mb-2}
  \\
  u(x,0) = u_{0}(x), \quad \p_t u(x, 0) = u_1(x), 
  \label{eqn:mb-init-data-2}
  \\
  \notag
  (u_0, u_1) \in
  H^{s}\times
  H^{s-2}
\end{gather}
%
%
We shall work in the periodic case first, and later generalize our results to
the non-periodic case. Let $\psi(t)$ is a cutoff function symmetric about the 
origin such that $\psi(t) = 1$ for $|t| \le 1/2$ and $\text{supp} \, \psi 
= [-1, 1 ]$. Define $\psi_{\delta}(t) = \psi(2t/\delta)$.
Neglecting $\pi$ related constants, we consider the time
localized integral form of the $B_{4}$ equation
%
\begin{align}
  & \psi_{\delta} u(x,t)
  \notag
  \\
  \label{main-rel-term-1}
  & = \psi_{\delta}(t) \sum_{n \in \zz} e^{inx} \wh{u_{0}}(n) \frac{e^{in^{2}t} + e^{-in^{2}t}}{2} 
  \\
  \label{main-rel-term-2}
  & + \psi_{\delta}(t) \sum_{n \in \zz} e^{inx}
  \wh{u_{1}}(n)\frac{e^{in^{2}t} - e^{-in^{2}}t}{2 i n^{2}} 
  \\
  \label{main-rel-term-3}
  & + \psi_{\delta}(t) \sum_{a = \pm 1} \sum_{n\in \zz} \int_\rr e^{ixn}  
  e^{it \tau} \frac{ 1 - \psi(\tau -  an^{2}) 
  }{\tau -  an^{2}} \wh{w}(n, \tau) \ d \tau
  \\
  \label{main-rel-term-4}
  & + \psi_{\delta}(t) \sum_{a = \pm 1} \sum_{n\in \zz} \int_\rr e^{i(xn + 
  t an^{2})}
  \frac{1- \psi(\tau -  an^{2})}{\tau -  an^{2}} \wh{w}(n, \tau) \ d \tau
  \\
  & + \psi_{\delta}(t) \sum_{a = \pm 1}  \sum_{k \ge 1} \frac{i^k t^k}{k!}
  \sum_{n \in \zz} \int_\rr e^{i(xn + t an^{2} )}
  \psi(\tau -  an^{2}) (\tau -  an^{2})^{k-1} \wh{w}(n, \tau)
  \\
  \label{main-rel-term-5}
  & \doteq Tu
\end{align}
%
%
where $T=T_{u_0, u_1, \psi, \delta}$.  We now introduce the following spaces. 
%
%
\begin{definition}
  Let $\mathcal{Y}$ be the space of functions $F(\cdot)$ such that
  \begin{enumerate}[(i)]
   \item{$F: \ci \times \rr \to \cc$ }.
   \item{ $F(x, \cdot) \in \mathcal{S}(\rr)$ for each $x \in \ci$}.
   \item{ $F(\cdot, t) \in C^{\infty}(\ci)$for each $t \in \rr$}.
  \end{enumerate}
  For $s, b \in \rr$, $X_{s,b}$ denotes the completion of $\mathcal{Y}$ with
  respect to the norm
  %
  %
  \begin{equation}
  \begin{split}
    \|F\|_{X_{s,b}} = \left( \sum_{n \in \zz} (1 + |n|)^{2s} \int_{\rr}
    (1 + | | \tau | - n^{2} |)^{2b} |\wh{F}(n, \tau)|^{2} d \tau\right)^{1/2}.
  \end{split}
  \label{eqn:bous-norm}
  \end{equation}
  %
  \begin{framed}
    %
    %
    \begin{remark}
    Note that the norm here is different than the one used to for the KdV. In
    the KdV case, $b$ has to be equal to $1/2$. Furthermore, to get the embedding
    of the lemma below, one has to add an extra term, thus producing the
    $Y^{s}$ norm of CKSTT.
    \label{rem:alternate-space}
    \end{remark}
    %
  \end{framed}
    %
  %
  %
\end{definition}
%
The $X_{s,b}$ spaces have the following important embedding.
%
%%%%%%%%%%%%%%%%%%%%%%%%%%%%%%%%%%%%%%%%%%%%%%%%%%%%%
%
%
%               Embedding 
%
%
%%%%%%%%%%%%%%%%%%%%%%%%%%%%%%%%%%%%%%%%%%%%%%%%%%%%%
%
%
\begin{lemma}
  Let $b > 1/2$. Then $X_{s, b} \subset C(\rr, H^s)$ continuously. That is,
  there exists $c>0$ depending only on $b$ such that
%
%
\begin{equation*}
\begin{split}
  \| u \|_{C(\rr, H^s) } \doteq \sup_{t \in \rr} \| u(t) \|_{H^s } 
  \le c \| u \|_{X_{s,b}}.
\end{split}
\end{equation*}
%
\label{lem:embedding}
\end{lemma}
%
%
%
%
\section{The Periodic Case} 
\label{sec:periodic}
%
%
To prove well-posedness for the $B_4$ ivp we we will 
show that for initial data $\vp \in B_{H^{s}(\ci) \times H^{s-2}(\ci)}(R)$, $T$ is a contraction on
$B_{X^{s}}(M_{R})$, where $M_{R}$ is a constant depending on $R$, 
by estimating the $X_{s,b}$
norm of \eqref{main-rel-term-1}-\eqref{main-rel-term-5}. The 
Picard fixed point theorem will
then yield a unique $u \in X^{s}$ satisfying $u = Tu$.
An application of
\cref{lem:embedding} will then imply the existence of a unique
$u \in C([-\delta, \delta], H^s(\ci))$ solving $\psi_{\delta} u = Tu$ for $| t | \le \delta$.
Lipschitz continuity of the flow map (and hence, uniform
continuity) will follow from estimates used to establish the contraction
mapping. 
%
%
%
%
%
%
%
%
%
\subsection{Estimate for \eqref{main-rel-term-1}}
\label{ssec:est-init-term-1}
We have
%
%
\begin{equation*}
  \begin{split}
    \wh{\eqref{main-rel-term-1}}
    = \frac{\wh{\psi_{\delta}}(\tau -
    n^{2})\wh{u_{0}}(n)}{2} + \frac{\wh{\psi_{\delta}}(\tau +
    n^{2})\wh{u_{0}}(n)}{2}.
  \end{split}
\end{equation*}
%
%
Hence, substituting and applying the inequality 
%
%
\begin{equation}
  \label{square-ineq}
\begin{split}
(a + b)^{2} \le 4(a^{2} +
b^{2}),\ a, b \in \rr,
\end{split}
\end{equation}
%
%
we have
%
%
\begin{align}
  & \| \eqref{main-rel-term-1} \|_{X^{s}}^{2} 
    \notag
    \\
    & = \sum_{n \in \zz}(1 + |n|)^{2s} \int_{\rr}\left( 1 + | | \tau
    |-n^{2} | \right) | \frac{\wh{\psi_{\delta}}(\tau - n^{2})\wh{u_{0}(n)}}{2 } +
    \frac{\wh{\psi_{\delta}}(\tau + n^{2})\wh{u_{0}}(n)}{2 } |^{2} d \tau
    \notag
    \\
    & \le \sum_{n \in \zz} \left( 1 + |n| \right)^{2s} | \wh{u_{0}}(n)
    |^{2} \int_{\rr} | \wh{\psi_{\delta}}(\tau - n^{2}) |^{2}\left( 1 + | | \tau | -
    n^{2} | \right) d \tau
    \label{u-0-norm-comp-1}
    \\
    & + \sum_{n \in \zz} \left( 1 + |n| \right)^{2s} | \wh{u_{0}}(n)
    |^{2} \int_{\rr} | \wh{\psi_{\delta}}(\tau + n^{2}) |^{2}\left( 1 + | | \tau | -
    n^{2} | \right) d \tau.
    \label{u-0-norm-comp-3}
\end{align}
%
%
Noting that
\begin{equation}
  \begin{split}
    | | \tau | - n^{2} | \le \min\left\{ | \tau - n^{2} |, | \tau + n^{2} |
    \right\},
  \end{split}
  \label{eqn:norm-key-ineq}
\end{equation}
%
we bound \eqref{u-0-norm-comp-1} by
%
%
\begin{equation*}
  \begin{split}
    & \sum_{n \in \zz} \left( 1 + |n| \right)^{2s} | \wh{u_{0}}(n)
    |^{2} \int_{\rr} | \wh{\psi_{\delta}}(\tau - n^{2}) |^{2}\left( 1 +  | \tau  -
    n^{2} | \right) d \tau
    \\
    & = \sum_{n \in \zz} \left( 1 + |n| \right)^{2s} | \wh{u_{0}}(n)
    |^{2} \int_{\rr} | \wh{\psi_{\delta}}(\tau') |^{2}\left( 1 +  | \tau'| \right) d \tau
    \\
    & \le c_{\psi, \delta} \sum_{n \in \zz} \left( 1 + |n| \right)^{2s} | \wh{u_{0}}(n)
    |^{2}
    \\
    & = c_{\psi, \delta} \| u_{0} \|_{H^{s}}^{2}
  \end{split}
\end{equation*}
%
%
where $c_{\psi, \delta}$ is a constant depending only on $\psi$ and $\delta$. The
term \eqref{u-0-norm-comp-3} is bounded in similar fashion. Therefore, 
$\|\eqref{main-rel-term-1}\|_{X^{s}}^{2} \le c_{\psi, \delta}
\|u_{0}\|_{H^s}^2$. Taking square roots of both sides gives
%
%
\begin{equation}
  \begin{split}
    \|\eqref{main-rel-term-1}\|_{X^{s}} \le c_{\psi, \delta}
    \|u_{0}\|_{H^s}.
  \end{split}
  \label{eqn:u-0-fin-est}
\end{equation}
%
%
%
%
\subsection{Estimate for \eqref{main-rel-term-2}}
\label{ssec:est-init-term-2}
We have
%
%
\begin{equation*}
  \begin{split}
    \wh{\psi_{\delta}(t)S_{t}u_{1}}^{x}(n, t)
    & = \psi_{\delta}(t) \wh{u_{1}}(n) \frac{e^{in^2 t} - e^{-in^{2}t}}{2i n^{2}}
    \\
    & = \frac{\psi_{\delta}(t) \wh{u_{1}}(n)e^{in^{2}t}}{2i n^{2}} - \frac{\psi_{\delta}(t)
    \wh{u_{1}}(n)e^{-in^{2}t}}{2i n^{2}}  
  \end{split}
\end{equation*}
%
%
and
%
%
\begin{equation*}
  \begin{split}
    \wh{\psi_{\delta}(t) S_{t}u_{1}}(n, \tau) = \frac{\wh{\psi_{\delta}}(\tau -
    n^{2})\wh{u_{1}}(n)}{2i n^{2}} - \frac{\wh{\psi_{\delta}}(\tau + n^{2})\wh{u_{1}}(n)}{2i
    n^{2}}.
  \end{split}
\end{equation*}
%
Note that 
%
\begin{equation*}
  \begin{split}
    \wh{\psi_{\delta}(t)S_{t}u_{1}}^{x}(0, t)
    & = \psi_{\delta}(t) \wh{u_{1}}(0) t
      \end{split}
\end{equation*}
and so 
%
%
\begin{equation*}
  \begin{split}
    \wh{\psi_{\delta}(t) S_{t}u_{1}}(0, \tau) = i \frac{d}{d \tau} \wh{\psi_{\delta}}(\tau)
    \wh{u_{1}}(0).
  \end{split}
\end{equation*}
%
Hence, substituting and applying \eqref{square-ineq}, we have
%
%
\begin{equation}
  \begin{split}
    \| \eqref{main-rel-term-2} \|_{X^{s}}^{2} 
    & = \sum_{n \in \zzdot}(1 + |n|)^{2s} \int_{\rr}\left( 1 + | | \tau
    |-n^{2} | \right) | \frac{\wh{\psi_{\delta}}(\tau - n^{2})\wh{u_{1}(n)}}{2i
    n^{2}} -
    \frac{\wh{\psi_{\delta}}(\tau + n^{2})\wh{u_{1}}(n)}{2i n^{2}} |^{2} d \tau
    \\
    & + |\wh{u_{1}}(0)|^{2} \int_{\rr} (1 + | \tau |) | i \frac{d }{d \tau}
    \wh{\psi_{\delta}}(\tau)|^{2} d \tau
    \\
    & \le \sum_{n \in \dot{\zz}} n^{-4} \left( 1 + |n| \right)^{2s} | \wh{u_{1}}(n)
    |^{2} \int_{\rr} | \wh{\psi_{\delta}}(\tau - n^{2}) |^{2}\left( 1 + | | \tau | -
    n^{2} | \right) d \tau
    \\
    & + \sum_{n \in \dot{\zz}} n^{-4} \left( 1 + |n| \right)^{2s} | \wh{u_{1}}(n)
    |^{2} \int_{\rr} | \wh{\psi_{\delta}}(\tau + n^{2}) |^{2}\left( 1 + | | \tau | -
    n^{2} | \right) d \tau
    \\
    & + |\wh{u_{1}}(0)|^{2} \int_{\rr} (1 + | \tau |) |\frac{d }{d \tau}
    \wh{\psi_{\delta}}(\tau)|^2 d \tau.
\end{split}
\label{u-1-norm-comp-pre}
\end{equation}
%
%
Applying the inequality
%
%
\begin{equation*}
\begin{split}
  \frac{(1 + |n|)^{2s}}{ n^{4}} \le \frac{(1 + |n|)^{2s}}{\frac{1}{16}(1 +
  |n|)^{4}} = 16 (1 + | n |)^{2(s-2)},  \quad s \in \rr, \quad n \ge 1
\end{split}
\end{equation*}
%
to \eqref{u-1-norm-comp-pre} gives
%
\begin{equation}
  \begin{split}
    \|  \eqref{main-rel-term-2}\|_{X^{s}}^{2} 
    & \lesssim \sum_{n \in \dot{\zz}} \left( 1 + |n| \right)^{2(s-2)} | \wh{u_{1}}(n)
    |^{2} \int_{\rr} | \wh{\psi_{\delta}}(\tau - n^{2}) |^{2}\left( 1 + | | \tau | -
    n^{2} | \right) d \tau
    \\
    & + \sum_{n \in \dot{\zz}} \left( 1 + |n| \right)^{2(s-2)} | \wh{u_{1}}(n)
    |^{2} \int_{\rr} | \wh{\psi_{\delta}}(\tau + n^{2}) |^{2}\left( 1 + | | \tau | -
    n^{2} | \right) d \tau
    \\
    & + |\wh{u_{1}}(0)|^{2} \int_{\rr} (1 + | \tau |) |\frac{d }{d \tau}
    \wh{\psi_{\delta}}(\tau)|^2 d \tau.
\end{split}
\label{u-1-norm-comp}
\end{equation}
%
%
Applying \eqref{eqn:norm-key-ineq},
we bound the first term of
\eqref{u-1-norm-comp} by
%
%
%
\begin{equation*}
  \begin{split}
    & \sum_{n \in \dot{\zz}} \left( 1 + |n| \right)^{2(s-2)} | \wh{u_{1}}(n)
    |^{2} \int_{\rr} | \wh{\psi_{\delta}}(\tau - n^{2}) |^{2}\left( 1 +  | \tau  -
    n^{2} | \right) d \tau
    \\
    & = \sum_{n \in \dot{\zz}} \left( 1 + |n| \right)^{2(s-2)} | \wh{u_{1}}(n)
    |^{2} \int_{\rr} | \wh{\psi_{\delta}}(\tau') |^{2}\left( 1 +  | \tau'| \right) d \tau
    \\
    & \le c_{\psi, \delta} \| u_{1} \|_{H^{s-2}}^{2}. 
  \end{split}
\end{equation*}
%
%
Applying
\eqref{eqn:norm-key-ineq} again, the
second term of \eqref{u-1-norm-comp} is bounded by
\begin{equation*}
  \begin{split}
    & \sum_{n \in \dot{\zz}} \left( 1 + |n| \right)^{2(s-2)} | \wh{u_{1}}(n)
    |^{2} \int_{\rr} | \wh{\psi_{\delta}}(\tau + n^{2}) |^{2}\left( 1 +  | \tau  -
    n^{2} | \right) d \tau
    \\
    & = \sum_{n \in \dot{\zz}} \left( 1 + |n| \right)^{2(s-2)} | \wh{u_{1}}(n)
    |^{2} \int_{\rr} | \wh{\psi_{\delta}}(\tau') |^{2}\left( 1 +  | \tau'| \right) d \tau
    \\
    & \le c_{\psi, \delta} \| u_{1} \|_{H^{s-2}}^{2}
  \end{split}
\end{equation*}
while the third term is bounded by  
%
%
\begin{equation*}
\begin{split}
  c_{\psi, \delta} \| u_{1} \|_{H^{s-2}}^{2}.
\end{split}
\end{equation*}
%
%
Therefore, 
$\|\eqref{main-rel-term-2}|_{X^{s}}^{2} \le c_{\psi, \delta}
\|u_{1}\|_{H^{s-2}}^2$ and
taking square roots of both sides gives
%
%
\begin{equation*}
  \begin{split}
    \|\eqref{main-rel-term-2}|_{X^{s}} \le c_{\psi, \delta}
    \|u_{1}\|_{H^{s-2}}.
  \end{split}
\end{equation*}
%
%
%
%
%
%
% \bib, bibdiv, biblist are defined by the amsrefs package.
\begin{bibdiv}
\begin{biblist}
\bib{Farah:2009uq}{article}{
      author={Farah, Luiz~Gustavo},
       title={Local solutions in {S}obolev spaces with negative indices for the
  ``good'' {B}oussinesq equation},
        date={2009},
        ISSN={0360-5302},
     journal={Comm. Partial Differential Equations},
      volume={34},
      number={1-3},
       pages={52\ndash 73},
         url={http://dx.doi.org/10.1080/03605300802682283},
      review={\MR{2512853 (2010k:35404)}},
}
\bib{Ginibre:1996fk}{article}{
      author={Ginibre, Jean},
       title={Le probl{\`e}me de {C}auchy pour des {EDP} semi-lin{\'e}aires
  p{\'e}riodiques en variables d'espace (d'apr{\`e}s {B}ourgain)},
        date={1996},
        ISSN={0303-1179},
     journal={Ast{\'e}risque},
      number={237},
       pages={Exp.\ No.\ 796, 4, 163\ndash 187},
        note={S{{\'e}}minaire Bourbaki, Vol. 1994/95},
      review={\MR{1423623 (98e:35154)}},
}
\bib{Ginibre:1997fk}{article}{
      author={Ginibre, J.},
      author={Tsutsumi, Y.},
      author={Velo, G.},
       title={On the {C}auchy problem for the {Z}akharov system},
        date={1997},
        ISSN={0022-1236},
     journal={J. Funct. Anal.},
      volume={151},
      number={2},
       pages={384\ndash 436},
         url={http://dx.doi.org/10.1006/jfan.1997.3148},
      review={\MR{1491547 (2000c:35220)}},
}
\bib{Kenig:1996aa}{article}{
      author={Kenig, Carlos~E.},
      author={Ponce, Gustavo},
      author={Vega, Luis},
       title={A bilinear estimate with applications to the {K}d{V} equation},
        date={1996},
        ISSN={0894-0347},
     journal={J. Amer. Math. Soc.},
      volume={9},
      number={2},
       pages={573\ndash 603},
         url={http://dx.doi.org/10.1090/S0894-0347-96-00200-7},
      review={\MR{1329387 (96k:35159)}},
}
\bib{Kenig-Ponce-Vega-1996-Quadratic-forms-for-the-1-D-semilinear}{article}{
      author={Kenig, Carlos~E.},
      author={Ponce, Gustavo},
      author={Vega, Luis},
       title={Quadratic forms for the {$1$}-{D} semilinear {S}chr{\"o}dinger
  equation},
        date={1996},
        ISSN={0002-9947},
     journal={Trans. Amer. Math. Soc.},
      volume={348},
      number={8},
       pages={3323\ndash 3353},
         url={http://dx.doi.org/10.1090/S0002-9947-96-01645-5},
      review={\MR{1357398 (96j:35233)}},
}
\end{biblist}
\end{bibdiv}
%\bibliography{/Users/davidkarapetyan/math/bib-files/references.bib}
%
%\nocite{*}
\end{document}
