%
\documentclass[12pt,reqno]{amsart}
\usepackage{amssymb}
\usepackage{appendix}
\usepackage[showonlyrefs=true]{mathtools} %amsmath extension package
\usepackage{cancel}  %for cancelling terms explicity on pdf
\usepackage{yhmath}   %makes fourier transform look nicer, among other things
\usepackage{framed}  %for framing remarks, theorems, etc.
\usepackage{enumerate} %to change enumerate symbols
\usepackage[margin=2.5cm]{geometry}  %page layout
\setcounter{tocdepth}{1} %must come before secnumdepth--else, pain
\setcounter{secnumdepth}{1} %number only sections, not subsections
%\usepackage[pdftex]{graphicx} %for importing pictures into latex--pdf compilation
\numberwithin{equation}{section}  %eliminate need for keeping track of counters
%\numberwithin{figure}{section}
\setlength{\parindent}{0in} %no indentation of paragraphs after section title
\renewcommand{\baselinestretch}{1.1} %increases vert spacing of text
%
\usepackage{hyperref}
\hypersetup{colorlinks=true,
linkcolor=blue,
citecolor=blue,
urlcolor=blue,
}
\usepackage[alphabetic, initials, msc-links]{amsrefs} %for the bibliography; uses cite pkg. Must be loaded after hyperref, otherwise doesn't work properly (conflicts with cref in particular)
\usepackage{cleveref} %must be last loaded package to work properly
\renewcommand{\cref}{\Cref}
\Crefname{enumi}{}{} %don't write item iv, just iv
%\Crefname{equation}{}{} %don't write Equation (2.1), just (2.1)
\crefformat{equation}{(#2#1#3)} %don't write Equation (2.1), just (2.1)
\Crefformat{equation}{(#2#1#3)} %don't write Equation (2.1), just (2.1)

%
%
\newcommand{\ds}{\displaystyle}
\newcommand{\ts}{\textstyle}
\newcommand{\nin}{\noindent}
\newcommand{\rr}{\mathbb{R}}
\newcommand{\nn}{\mathbb{N}}
\newcommand{\zz}{\mathbb{Z}}
\newcommand{\cc}{\mathbb{C}}
\newcommand{\ci}{\mathbb{T}}
\newcommand{\zzdot}{\dot{\zz}}
\newcommand{\wh}{\widehat}
\newcommand{\p}{\partial}
\newcommand{\ee}{\varepsilon}
\newcommand{\vp}{\varphi}
\newcommand{\wt}{\widetilde}
%
%
%
%
\newtheorem{theorem}{Theorem}[section]
\newtheorem{lemma}[theorem]{Lemma}
\newtheorem{corollary}[theorem]{Corollary}
\newtheorem{claim}[theorem]{Claim}
\newtheorem{prop}[theorem]{Proposition}
\newtheorem{proposition}[theorem]{Proposition}
\newtheorem{no}[theorem]{Notation}
\newtheorem{definition}[theorem]{Definition}
\newtheorem{remark}[theorem]{Remark}
\newtheorem{examp}{Example}[section]
\newtheorem {exercise}[theorem] {Exercise}
%
\makeatletter \renewenvironment{proof}[1][\proofname] {\par\pushQED{\qed}\normalfont\topsep6\p@\@plus6\p@\relax\trivlist\item[\hskip\labelsep\bfseries#1\@addpunct{.}]\ignorespaces}{\popQED\endtrivlist\@endpefalse} \makeatother%
%makes proof environment bold instead of italic
\newcommand{\uol}{u^\omega_\lambda}
\newcommand{\lbar}{\bar{l}}
\renewcommand{\l}{\lambda}
\newcommand{\R}{\mathbb R}
\newcommand{\RR}{\mathcal R}
\newcommand{\al}{\alpha}
\newcommand{\ve}{q}
\newcommand{\tg}{{tan}}
\newcommand{\m}{q}
\newcommand{\N}{N}
\newcommand{\ta}{{\tilde{a}}}
\newcommand{\tb}{{\tilde{b}}}
\newcommand{\tc}{{\tilde{c}}}
\newcommand{\tS}{{\tilde S}}
\newcommand{\tP}{{\tilde P}}
\newcommand{\tu}{{\tilde{u}}}
\newcommand{\tw}{{\tilde{w}}}
\newcommand{\tA}{{\tilde{A}}}
\newcommand{\tX}{{\tilde{X}}}
\newcommand{\tphi}{{\tilde{\phi}}}
\synctex=1
\begin{document}
\title{A Modified Boussinesq equation}
\author{Dan-Andrei Geba, Alexandrou Himonas, and David Karapetyan}
\address{Department of Mathematics, University of Rochester, Rochester, NY 14627}
\address{Department of Mathematics, University of Notre Dame, Notre Dame, IN 46556}
\address{Department of Mathematics, University of Notre Dame, Notre Dame, IN 46556}
\date{}
%
%
\subjclass[2000]{35B30, 35Q55, 35Q72}
\keywords{local well-posedness; ill-posedness.}
\maketitle
%
%
\section{Introduction}
%
We consider the initial value problem (ivp) for the fourth order modified Boussinesq
($B_4$) equation 
\begin{gather}
  u_{tt}   + u_{xxxx} + (u^2)_{xx} = 0, \quad x \in \rr \text{ or }
  \ci,\text{ } t \in \rr
  \label{eqn:mb-2}
  \\
  u(x,0) = u_{0}(x), \quad \p_t u(x, 0) = u_1(x), 
  \label{eqn:mb-init-data-2}
  \\
  \notag
  (u_0, u_1) \in
  H^{s}\times
  H^{s-2}
\end{gather}
%
%
We shall work in the periodic case first, and later generalize our results to
the non-periodic case. Let $\psi(t)$ is a cutoff function symmetric about the 
origin such that $\psi(t) = 1$ for $|t| \le 1/2$ and $\text{supp} \, \psi 
= [-1, 1 ]$. Neglecting $\pi$ related constants, we consider the time
localized integral form of the $B_{4}$ equation
%
\begin{align}
  & \psi u(x,t)
  \notag
  \\
  \label{main1-rel-term-1}
  & = \psi(t) \sum_{n \in \zz} e^{inx} \wh{u_{0}}(n) \frac{e^{in^{2}t} + e^{-in^{2}t}}{2} 
  \\
  \label{main1-rel-term-2}
  & + \psi(t) \sum_{n \in \zz} e^{inx}
  \wh{u_{1}}(n)\frac{e^{in^{2}t} - e^{-in^{2}}t}{2 i n^{2}} 
  \\
  \label{main1-rel-term-3}
  & + \psi(t) \sum_{a = \pm 1} \sum_{n\in \zz} \int_\rr e^{ixn}  
  e^{it \tau} \frac{ 1 - \psi(\tau -  an^{2}) 
  }{\tau -  an^{2}} \wh{w}(n, \tau) \ d \tau
  \\
  \label{main1-rel-term-4}
  & + \psi(t) \sum_{a = \pm 1} \sum_{n\in \zz} \int_\rr e^{i(xn + 
  t an^{2})}
  \frac{1- \psi(\tau -  an^{2})}{\tau -  an^{2}} \wh{w}(n, \tau) \ d \tau
  \\
  & + \psi(t) \sum_{a = \pm 1}  \sum_{k \ge 1} \frac{i^k t^k}{k!}
  \sum_{n \in \zz} \int_\rr e^{i(xn + t an^{2} )}
  \psi(\tau -  an^{2}) (\tau -  an^{2})^{k-1} \wh{w}(n, \tau)
  \\
  \label{main1-rel-term-5}
  & \doteq Tu
\end{align}
%
%
where $T=T_{u_0, u_1, \psi}$.  We now introduce the following spaces. 
%
%
\begin{definition}
  Let $\mathcal{Y}$ be the space of functions $F(\cdot)$ such that
  \begin{enumerate}[(i)]
   \item{$F: \ci \times \rr \to \cc$ }.
   \item{ $F(x, \cdot) \in \mathcal{S}(\rr)$ for each $x \in \ci$}.
   \item{ $F(\cdot, t) \in C^{\infty}(\ci)$for each $t \in \rr$}.
  \end{enumerate}
  For $s, b \in \rr$, $X_{s,b}$ denotes the completion of $\mathcal{Y}$ with
  respect to the norm
  %
  %
  \begin{equation}
  \begin{split}
    \|F\|_{X_{s,b}} = \left( \sum_{n \in \zz} (1 + |n|)^{2s} \int_{\rr}
    (1 + | | \tau | - n^{2} |)^{2b} |\wh{F}(n, \tau)|^{2} d \tau\right)^{1/2}.
  \end{split}
  \label{eqn:bous-norm}
  \end{equation}
  %
  \begin{framed}
    %
    %
    \begin{remark}
    Note that the norm here is different than the one used to for the KdV. In
    the KdV case, $b$ has to be equal to $1/2$. Furthermore, to get the embedding
    of the lemma below, one has to add an extra term, thus producing the
    $Y^{s}$ norm of CKSTT.
    \label{rem:alternate-space}
    \end{remark}
    %
  \end{framed}
    %
  %
  %
\end{definition}
%
The $X_{s,b}$ spaces have the following important embedding.
%
%%%%%%%%%%%%%%%%%%%%%%%%%%%%%%%%%%%%%%%%%%%%%%%%%%%%%
%
%
%               Embedding 
%
%
%%%%%%%%%%%%%%%%%%%%%%%%%%%%%%%%%%%%%%%%%%%%%%%%%%%%%
%
%
\begin{lemma}
  Let $b > 1/2$. Then $X_{s, b} \subset C(\rr, H^s)$ continuously. That is,
  there exists $c>0$ depending only on $b$ such that
%
%
\begin{equation*}
\begin{split}
  \| u \|_{C(\rr, H^s) } \doteq \sup_{t \in \rr} \| u(t) \|_{H^s } 
  \le c \| u \|_{X_{s,b}}.
\end{split}
\end{equation*}
%
\label{lem:embedding}
\end{lemma}
%
%
%
%
\section{The Periodic Case} 
\label{sec:periodic}
%
%
To prove well-posedness for the $B_4$ ivp we we will 
show that for initial data $\vp \in B_{H^{s}(\ci) \times H^{s-2}(\ci)}(R)$, $T$ is a contraction on
$B_{X_{s,b}}(M_{R})$, where $M_{R}$ is a constant depending on $R$, 
by estimating the $X_{s,b}$
norm of \eqref{main1-rel-term-1}-\eqref{main1-rel-term-5}. The 
Picard fixed point theorem will
then yield a unique $u \in X_{s,b}$ satisfying $u = Tu$.
An application of
\cref{lem:embedding} will then imply the existence of a unique
$u \in C([-1/2, 1/2], H^s(\ci))$ solving $\psi u = Tu$ for $| t | \le 1/2$.
Lipschitz continuity of the flow map (and hence, uniform
continuity) will follow from estimates used to establish the contraction
mapping. 
%
%
%
%
%
%
%
%
%
\subsection{Estimate for \eqref{main1-rel-term-1}}
\label{ssec:est-init-term-1}
We have
%
%
\begin{equation*}
  \begin{split}
    \wh{\eqref{main1-rel-term-1}}
    = \frac{\wh{\psi}(\tau -
    n^{2})\wh{u_{0}}(n)}{2} + \frac{\wh{\psi}(\tau +
    n^{2})\wh{u_{0}}(n)}{2}.
  \end{split}
\end{equation*}
%
%
Hence, substituting and applying the inequality 
%
%
\begin{equation}
  \label{square-ineq}
\begin{split}
(a + b)^{2} \le 4(a^{2} +
b^{2}),\ a, b \in \rr,
\end{split}
\end{equation}
%
%
we have
%
%
\begin{align}
  & \| \eqref{main1-rel-term-1} \|_{X_{s,b}}^{2} 
    \notag
    \\
    & = \sum_{n \in \zz}(1 + |n|)^{2s} \int_{\rr}\left( 1 + | | \tau
    |-n^{2} | \right)^{2b} | \frac{\wh{\psi}(\tau - n^{2})\wh{u_{0}(n)}}{2 } +
    \frac{\wh{\psi}(\tau + n^{2})\wh{u_{0}}(n)}{2 } |^{2} d \tau
    \notag
    \\
    & \le \sum_{n \in \zz} \left( 1 + |n| \right)^{2s} | \wh{u_{0}}(n)
    |^{2} \int_{\rr} | \wh{\psi}(\tau - n^{2}) |^{2}\left( 1 + | | \tau | -
    n^{2} | \right)^{2b} d \tau
    \label{u-0-norm-comp-1}
    \\
    & + \sum_{n \in \zz} \left( 1 + |n| \right)^{2s} | \wh{u_{0}}(n)
    |^{2} \int_{\rr} | \wh{\psi}(\tau + n^{2}) |^{2}\left( 1 + | | \tau | -
    n^{2} | \right)^{2b} d \tau.
    \label{u-0-norm-comp-3}
\end{align}
%
%
Noting that
\begin{equation}
  \begin{split}
    | | \tau | - n^{2} | \le \min\left\{ | \tau - n^{2} |, | \tau + n^{2} |
    \right\},
  \end{split}
  \label{eqn:norm-key-ineq}
\end{equation}
%
we bound \eqref{u-0-norm-comp-1} by
%
%
\begin{equation*}
  \begin{split}
    & \sum_{n \in \zz} \left( 1 + |n| \right)^{2s} | \wh{u_{0}}(n)
    |^{2} \int_{\rr} | \wh{\psi}(\tau - n^{2}) |^{2}\left( 1 +  | \tau  -
    n^{2} | \right)^{2b} d \tau
    \\
    & = \sum_{n \in \zz} \left( 1 + |n| \right)^{2s} | \wh{u_{0}}(n)
    |^{2} \int_{\rr} | \wh{\psi}(\tau') |^{2}\left( 1 +  | \tau'|
    \right)^{2b} d \tau
    \\
    & \le c_{\psi, b} \sum_{n \in \zz} \left( 1 + |n| \right)^{2s} | \wh{u_{0}}(n)
    |^{2}
    \\
    & = c_{\psi, b} \| u_{0} \|_{H^{s}}^{2}.
  \end{split}
\end{equation*}
%
%
The
term \eqref{u-0-norm-comp-3} is bounded in similar fashion. Therefore, 
$\|\eqref{main1-rel-term-1}\|_{X_{s,b}}^{2} \le c_{\psi, b}
\|u_{0}\|_{H^s}^2$. Taking square roots of both sides gives
%
%
\begin{equation}
  \begin{split}
    \|\eqref{main1-rel-term-1}\|_{X_{s,b}} \le c_{\psi, b}
    \|u_{0}\|_{H^s}.
  \end{split}
  \label{eqn:u-0-fin-est}
\end{equation}
%
%
%
%
\subsection{Estimate for \eqref{main1-rel-term-2}}
\label{ssec:est-init-term-2}
We have
%
%
\begin{equation*}
  \begin{split}
    \wh{\psi(t)S_{t}u_{1}}^{x}(n, t)
    & = \psi(t) \wh{u_{1}}(n) \frac{e^{in^2 t} - e^{-in^{2}t}}{2i n^{2}}
    \\
    & = \frac{\psi(t) \wh{u_{1}}(n)e^{in^{2}t}}{2i n^{2}} - \frac{\psi(t)
    \wh{u_{1}}(n)e^{-in^{2}t}}{2i n^{2}}  
  \end{split}
\end{equation*}
%
%
and
%
%
\begin{equation*}
  \begin{split}
    \wh{\psi(t) S_{t}u_{1}}(n, \tau) = \frac{\wh{\psi}(\tau -
    n^{2})\wh{u_{1}}(n)}{2i n^{2}} - \frac{\wh{\psi}(\tau + n^{2})\wh{u_{1}}(n)}{2i
    n^{2}}.
  \end{split}
\end{equation*}
%
Note that 
%
\begin{equation*}
  \begin{split}
    \wh{\psi(t)S_{t}u_{1}}^{x}(0, t)
    & = \psi(t) \wh{u_{1}}(0) t
      \end{split}
\end{equation*}
and so 
%
%
\begin{equation*}
  \begin{split}
    \wh{\psi(t) S_{t}u_{1}}(0, \tau) = i \frac{d}{d \tau} \wh{\psi}(\tau)
    \wh{u_{1}}(0).
  \end{split}
\end{equation*}
%
Hence, substituting and applying \eqref{square-ineq}, we have
%
%
\begin{equation}
  \begin{split}
    \| \eqref{main1-rel-term-2} \|_{X_{s,b}}^{2} 
    & = \sum_{n \in \zzdot}(1 + |n|)^{2s} \int_{\rr}\left( 1 + | | \tau
    |-n^{2} | \right)^{2b} | \frac{\wh{\psi}(\tau - n^{2})\wh{u_{1}(n)}}{2i
    n^{2}} -
    \frac{\wh{\psi}(\tau + n^{2})\wh{u_{1}}(n)}{2i n^{2}} |^{2} d \tau
    \\
    & + |\wh{u_{1}}(0)|^{2} \int_{\rr} (1 + | \tau |) | i \frac{d }{d \tau}
    \wh{\psi}(\tau)|^{2} d \tau
    \\
    & \le \sum_{n \in \dot{\zz}} n^{-4} \left( 1 + |n| \right)^{2s} | \wh{u_{1}}(n)
    |^{2} \int_{\rr} | \wh{\psi}(\tau - n^{2}) |^{2}\left( 1 + | | \tau | -
    n^{2} | \right)^{2b} d \tau
    \\
    & + \sum_{n \in \dot{\zz}} n^{-4} \left( 1 + |n| \right)^{2s} | \wh{u_{1}}(n)
    |^{2} \int_{\rr} | \wh{\psi}(\tau + n^{2}) |^{2}\left( 1 + | | \tau | -
    n^{2} | \right)^{2b} d \tau
    \\
    & + |\wh{u_{1}}(0)|^{2} \int_{\rr} (1 + | \tau |) |\frac{d }{d \tau}
    \wh{\psi}(\tau)|^2 d \tau.
\end{split}
\label{u-1-norm-comp-pre}
\end{equation}
%
%
Applying the inequality
%
%
\begin{equation*}
\begin{split}
  \frac{(1 + |n|)^{2s}}{ n^{4}} \le \frac{(1 + |n|)^{2s}}{\frac{1}{16}(1 +
  |n|)^{4}} = 16 (1 + | n |)^{2(s-2)},  \quad s \in \rr, \quad n \ge 1
\end{split}
\end{equation*}
%
to \eqref{u-1-norm-comp-pre} gives
%
\begin{equation}
  \begin{split}
    \|  \eqref{main1-rel-term-2}\|_{X_{s,b}}^{2} 
    & \lesssim \sum_{n \in \dot{\zz}} \left( 1 + |n| \right)^{2(s-2)} | \wh{u_{1}}(n)
    |^{2} \int_{\rr} | \wh{\psi}(\tau - n^{2}) |^{2}\left( 1 + | | \tau | -
    n^{2} | \right)^{2b} d \tau
    \\
    & + \sum_{n \in \dot{\zz}} \left( 1 + |n| \right)^{2(s-2)} | \wh{u_{1}}(n)
    |^{2} \int_{\rr} | \wh{\psi}(\tau + n^{2}) |^{2}\left( 1 + | | \tau | -
    n^{2} | \right)^{2b} d \tau
    \\
    & + |\wh{u_{1}}(0)|^{2} \int_{\rr} (1 + | \tau |) |\frac{d }{d \tau}
    \wh{\psi}(\tau)|^2 d \tau.
\end{split}
\label{u-1-norm-comp}
\end{equation}
%
%
Applying \eqref{eqn:norm-key-ineq},
we bound the first term of
\eqref{u-1-norm-comp} by
%
%
%
\begin{equation*}
  \begin{split}
    & \sum_{n \in \dot{\zz}} \left( 1 + |n| \right)^{2(s-2)} | \wh{u_{1}}(n)
    |^{2} \int_{\rr} | \wh{\psi}(\tau - n^{2}) |^{2}\left( 1 +  | \tau  -
    n^{2} | \right)^{2b} d \tau
    \\
    & = \sum_{n \in \dot{\zz}} \left( 1 + |n| \right)^{2(s-2)} | \wh{u_{1}}(n)
    |^{2} \int_{\rr} | \wh{\psi}(\tau') |^{2}\left( 1 +  | \tau'| \right) d \tau
    \\
    & \le c_{\psi, b} \| u_{1} \|_{H^{s-2}}^{2}. 
  \end{split}
\end{equation*}
%
%
Applying
\eqref{eqn:norm-key-ineq} again, the
second term of \eqref{u-1-norm-comp} is bounded by
\begin{equation*}
  \begin{split}
    & \sum_{n \in \dot{\zz}} \left( 1 + |n| \right)^{2(s-2)} | \wh{u_{1}}(n)
    |^{2} \int_{\rr} | \wh{\psi}(\tau + n^{2}) |^{2}\left( 1 +  | \tau  -
    n^{2} | \right)^{2b} d \tau
    \\
    & = \sum_{n \in \dot{\zz}} \left( 1 + |n| \right)^{2(s-2)} | \wh{u_{1}}(n)
    |^{2} \int_{\rr} | \wh{\psi}(\tau') |^{2}\left( 1 +  | \tau'| \right) d \tau
    \\
    & \le c_{\psi, b} \| u_{1} \|_{H^{s-2}}^{2}
  \end{split}
\end{equation*}
while the third term is bounded by  
%
%
\begin{equation*}
\begin{split}
  c_{\psi, b} \| u_{1} \|_{H^{s-2}}^{2}.
\end{split}
\end{equation*}
%
%
Therefore, 
$\|\eqref{main1-rel-term-2}|_{X_{s,b}}^{2} \le c_{\psi, b}
\|u_{1}\|_{H^{s-2}}^2$ and
taking square roots of both sides gives
%
%
\begin{equation*}
  \begin{split}
    \|\eqref{main1-rel-term-2}|_{X_{s,b}} \le c_{\psi, b}
    \|u_{1}\|_{H^{s-2}}.
  \end{split}
\end{equation*}
%
%
\subsection{Estimate for \cref{main1-rel-term-3}.}
%
%
%%%%%%%%%%%%%%%%%%%%%%%%%%%%%%%%%%%%%%%%%%%%%%%%%%%%%
%
%
%			Schwartz Multiplier	
%
%
%%%%%%%%%%%%%%%%%%%%%%%%%%%%%%%%%%%%%%%%%%%%%%%%%%%%%
%
%
%
%
We have
%
%
\begin{equation}
  \label{yu}
	\begin{split}
		\|\cref{main1-rel-term-3}\|_{X_{s,b}} 
    & \le 
    \sum_{a = \pm 1} \| \sum_{n \in \zz} (t) e^{ixn} \int_\rr 
		e^{it \tau} \frac{ 1 - a\psi (\tau - an^{m} ) 
		}{\tau - an^{m}} \wh{w}(n, \tau) \ 
		d \tau\|_{X_{s,b}}.
			\end{split}
\end{equation}
%
But
%
%
\begin{equation}
\label{main-int2-est-X-s-part}
\begin{split}
  & \| \sum_{n \in \zz} e^{ixn} \int_\rr 
		e^{it \tau} \frac{ 1 - a\psi (\tau - an^{m} ) 
    }{\tau - an^{m}} (t) \wh{w}(n, \tau) \ 
		d \tau\|_{X_{s,b}}
		\\
    & = \sum_{a = \pm 1}\left( \sum_{n \in \zz} \left (1 + |n| \right )^{2s} \int_\rr
    (1 + |  |\tau| - n^{m}|)^{2b} \left | \frac{1 - a\psi(\tau - an^{2 
		})}{\tau - an^{m}} 
     \wh{w}(n, \tau) \right |^2 \ d 
		\tau \right)^{1/2}
		\\
    & \le \sum_{a = \pm 1}
    \left( \sum_{n \in \zz} \left (1 + |n| \right )^{2s} \int_{| \tau - an^{m }| \ge 1}
    (1 + | |\tau| - n^{m}|)^{2b} \frac{|  \wh{w}(n, \tau)|^2 }{|\tau - an^{m }|^2} 
		\ d 
		\tau \right)^{1/2}
  \end{split}
\end{equation}
Applying the inequality
\begin{equation}
	\label{one-plus-ineq}
	\begin{split}
		& \frac{1}{|j|} \le \frac{2}{1 + |j| } , 
		\qquad |j| \ge 1
   	\end{split}
\end{equation}
and \cref{eqn:norm-key-ineq} we bound this by
\begin{equation}
  \label{pre-bi-est}
  \begin{split}
		& 4 \left( \sum_{n \in 
		\zz} \left (1 + |n| \right )^{2s} \int_\rr
    \frac{| \wh{w}(n, \tau) |^2}{(1+ |  |\tau| - 
    n^{m}|)^{2(1-b)}} 
		 \ d \tau 
		\right)^{1/2}
\end{split}
\end{equation}
%
%
%
We now need the following bilinear
estimate, whose proof we leave for later.
%
%
%%%%%%%%%%%%%%%%%%%%%%%%%%%%%%%%%%%%%%%%%%%%%%%%%%%%%
%
%
%				Proposition
%
%
%%%%%%%%%%%%%%%%%%%%%%%%%%%%%%%%%%%%%%%%%%%%%%%%%%%%%
%
%
\begin{proposition}[Theorem 1.1 in~\cite{Farah:2009uq}]
\label{prop:bilinear-est}
	%
	%
	If $a > 1/4$, $b > 1/2$, and $s \ge -a/2$, 
  then 
	\begin{equation}
		\left( \sum_{n \in \zz} \left (1 + |n| \right )^{2s} \int_\rr
		\frac{|\wh{w_{fg}}(n, \tau) |^2}{\left (1+ |  |\tau| - 
    n^{m}| \right )^{2a}} 
		 \ d \tau 
		\right)^{1/2}
    \le c_{a} \|f\|_{X_{s,b}} \|g\|_{X_{s,b}}
	\end{equation}
  where $w_{fg}(x,t)$ = $fg (x,t)$.
%
%
%
%
\end{proposition}
%
Applying \cref{prop:bilinear-est} to \cref{pre-bi-est}, it follows that
if $1/4 < 1-b < 1/2$ and $s \ge -(1-b)/2$ then
%
%
%
%
%
\begin{equation}
	\label{main-int2-est}
	\begin{split}
    \|\cref{main1-rel-term-3}\|_{X_{s,b}} \le c_{\psi, b} 
    \|f\|_{X_{s,b}}
    \|g\|_{X_{s,b}}
	\end{split}
\end{equation}

%
%
\subsection{Estimate for \cref{main1-rel-term-4}.}
Letting $$f(x,t) = \sum_{a = \pm 1} \psi(t) \sum_{n \in \zz} e^{i\left( xn +
atn^{m} \right)} 
\int_\rr \frac{1 - \psi\left( \lambda - an^{m} \right)}{\lambda - an^{m}} 
\wh{w} \left( n, \lambda \right) \ d \lambda,$$ we have
%
%
\begin{equation*}
	\begin{split}
		& \wh{f^x}(n, t) = \psi(t) e^{aitn^{m}} \int_\rr
		\frac{1 - \psi\left( \lambda - an^{m} \right)}{\lambda - an^{m}} 
		\wh{w}(n, \lambda) \ d \lambda
	\end{split}
\end{equation*}
and
\begin{equation*}
	\begin{split}
		 \wh{f}\left( n, \tau \right)
		 & = \int_\rr e^{-it\left( \tau - an^{m} 
		\right)} \psi(t) \int_\rr \frac{1 - a\psi\left( 
		\lambda - an^{m} 
		\right)}{\lambda - an^{m}} \wh{w}(n, \lambda) \ d \lambda d \tau
		\\
    & = \wh{\psi}\left( \tau - an^{m} \right) \int_\rr 
		\frac{1 - a\psi\left( 
		\lambda - an^{m} 
		\right)}{\lambda - an^{m}} \wh{w}(n, \lambda) \ d \lambda.
	\end{split}
\end{equation*}
Therefore,
%
%
\begin{equation}
  \label{iu}
	\begin{split}
		& \| \cref{main1-rel-term-4} \|_{X_{s,b}} 
		\\
    & \le c_{\psi, b}  \sum_{a = \pm 1} \left( \sum_{n \in \zz} \left (1 + |n| \right)^{2s}
    \int_\rr \left( 1 + | |\tau| - n^{m } \right )^{2b} | | \wh{\psi}\left(
    \tau - an^{m } \right) |^2 \ d \tau \right.
		\\
		& \times \left . |
		\int_\rr \frac{1 - \psi\left( \lambda - an^{m } \right)}{\lambda -
		an^{m }} \wh{w}(n, \lambda) \ d \lambda |^2  \right)^{1/2}.
  \end{split}
\end{equation}
Applying \cref{eqn:norm-key-ineq}, we have the bound
%
%
\begin{equation*}
\begin{split}
& \sum_{a = \pm 1} 
\int_\rr \left( 1 + | |\tau| - n^{m } \right )^{2b} | | \wh{\psi}\left(
    \tau - an^{m } \right) |^2 \ d \tau 
    \\
    & \le  \sum_{a = \pm 1} 
    \int_\rr \left( 1 + |\tau - an^{m } | \right )^{2b} | | \wh{\psi}\left(
    \tau - an^{m } \right) |^2 \ d \tau 
    \\
    & = c_{\psi, b}.
\end{split}
\end{equation*}
%
Hence, \cref{iu} is bounded by
%
\begin{equation}
  \begin{split}
    & c_{\psi, b}  \sum_{a = \pm 1}\left( \sum_{n \in \zz} \left (1 + |n| \right )^{2s} | \int_\rr
		\frac{1 - \psi\left( \lambda - an^{m } \right)}{\lambda - an^{m }}
		\wh{w}(n, \lambda) \ d\lambda |^2 \right)^{1/2}
		\\
		& \simeq \sum_{a = \pm 1} \left( \sum_{n \in \zz} \left (1 + |n| \right )^{2s}  \left ( \int_\rr
		\frac{1 - \psi\left( \lambda - an^{m } \right)}{|\lambda - an^{m }|}
		|\wh{w}(n, \lambda) | \ d\lambda \right )^2 \right)^{1/2}
		\\
		& \le \sum_{a = \pm 1} \left( \sum_{n \in \zz} \left (1 + |n| \right )^{2s}  \left ( \int_{| \lambda - 
		an^{m } | \ge 1}
		\frac{|\wh{w}(n, \lambda) | }{|\lambda - an^{m }|}
		\ d\lambda \right )^2 \right)^{1/2}.
  \end{split}
\end{equation}
Applying \cref{one-plus-ineq} then \cref{eqn:norm-key-ineq} we bound this by
\begin{equation}
  \label{fh}
\begin{split}
    4 \left( \sum_{n \in \zz} \left (1 + |n| \right )^{2s}  \left ( \int_\rr
    \frac{|\wh{w}(n, \lambda)| }{1 + | |\lambda| - n^{m }|}
    \ d\lambda \right )^2 \right)^{1/2}.
	\end{split}
\end{equation}
%
%
%
%%
We now need the following corollary to
\cref{prop:bilinear-est}.
%
%
%%%%%%%%%%%%%%%%%%%%%%%%%%%%%%%%%%%%%%%%%%%%%%%%%%%%%
%
%
%				Second bilinear Estimate 
%
%
%%%%%%%%%%%%%%%%%%%%%%%%%%%%%%%%%%%%%%%%%%%%%%%%%%%%%
%
%
\begin{corollary}
\label{cor:bilinear-estimate2}
	If $a > 1/2$, $b > 1/2$, and $s \ge -a/2$, 
  then 
%
%
\begin{equation}
	\label{bilinear-estimate2}
	\begin{split}
		\left( \sum_{n \in \zz} \left (1 + |n| \right )^{2s}  \left ( \int_\rr 
    \frac{|\wh{w_{fg}}(n, \tau) |}{(1 + | |\tau| - n^{m } |)^{2a}}
		 \ d\tau \right)^2  \right)^{1/2} \lesssim \|f\|_{X_{s,b}} \|g\|_{X_{s,b}}.
	\end{split}
\end{equation}
\end{corollary}
Applying \cref{cor:bilinear-estimate2} to \cref{fh}, we
conclude that
%
%
\begin{equation}
	\label{main-int3-est}
	\begin{split}
		\|\cref{main1-rel-term-4}\|_{X_{s,b}} 
    \le c_{\psi, b}  \|f\|_{X_{s,b}} \|g\|_{X_{s,b}}. 
	\end{split}
\end{equation}
%
%
%
\subsection{Estimate for \cref{main1-rel-term-5}.}
Note that
%
%
\begin{equation}
	\label{1n}
	\begin{split}
    \cref{main1-rel-term-5} \simeq \sum_{a = \pm 1}\sum_{k \ge 1}
		\frac{i^k}{k!}g_k(x,t)
	\end{split}
\end{equation}
%
%
where 
%
%
\begin{equation*}
	\begin{split}
		& g_k(x,t) = t^k \psi(t) \sum_{n \in \zz} e^{i\left( xn + ta n^{m}
		\right)} h_k(n),
		\\
		& h_k(n) = \int_\rr \psi \left( \tau - an^{m } \right) \cdot \left(
		\tau - an^{m } \right)^{k -1} \wh{w}(n, \tau) \ d \tau.
	\end{split}
\end{equation*}
%
%
Hence
%
%
\begin{equation*}
	\begin{split}
		\wh{g_k^x}(n, t) = t^{k} \psi(t) e^{i t an^{m }} h_k(n)
	\end{split}
\end{equation*}
%
%
which gives
%
%
\begin{equation*}
	\begin{split}
		\wh{g_k}(n, \tau)
		& = h_k(n) \int_\rr e^{-it\left( \tau - an^{m } \right)}
		t^{k}\psi(t) \ dt
		\\
		& = h_k(n) \wh{t^{k}\psi(t)} \left( \tau - an^{m } \right).
	\end{split}
\end{equation*}
%
%
Applying this to \cref{1n} and using Minkowski's inequality, we obtain
%
%
\begin{equation}
	\label{2n}
	\begin{split}
		& \|\cref{main1-rel-term-5}\|_{X_{s,b}} 
    \\
    & \lesssim \sum_{a = \pm 1} \left( \sum_{n \in \zz} \left (1 + |n| \right )^{2s}
    \int_\rr \left( 1 + | |\tau| - an^{m } | \right)^{2b}
    | \wh{\sum_{k \ge 1} \frac{i^k}{k!}g_k(x,t)} |^2 \ d \tau
		\right)^{1/2}
		\\
		& \le \sum_{a = \pm 1}\sum_{k \ge 1} \frac{1}{k!}\left( \sum_{n \in \zz} \left (1 + |n| \right )^{2s}
    \int_\rr \left( 1 + | |\tau| - an^{m } | \right)^{2b} | \wh{g_k}(n, \tau) |^2 \
		d \tau \right)^{1/2}
		\\
		& = \sum_{a = \pm 1}\sum_{k \ge 1} \frac{1}{k!} \left( \sum_{n \in \zz} \left (1 + |n| \right )^{2s}
    \int_\rr \left( 1 + | |\tau| - an^{m } | \right)^{2b} | h_k(n) \wh{t^k
		\psi(t)} \left( \tau - an^{m } \right) |^2 \ d \tau \right)^{1/2}
		\\
		& = \sum_{a = \pm 1}\sum_{k \ge 1} \frac{1}{k!} \left( \sum_{n \in \zz} \left (1 + |n| \right )^{2s} |
    h_k(n) |^2 \int_\rr \left( 1 + | |\tau| - an^{m } | \right)^{2b} | \wh{t^k
		\psi(t)} \left( \tau - an^{m } \right) |^2 \ d \tau \right)^{1/2}.
	\end{split}
\end{equation}
%
%
Applying \cref{eqn:norm-key-ineq} and the change
of variable $\tau - an^{m } = \tau'$
gives
%
%
\begin{equation}
	\label{3n}
	\begin{split}
		& \int_\rr \left( 1 + | |\tau| - an^{m } | \right)^{2b} | \wh{t^{k}
		\psi(t)}\left( \tau - an^{m } \right) |^2 \ d \tau
    \\
    & \le \int_\rr \left( 1 + | \tau - an^{m } | \right)^{2b} | \wh{t^{k}
		\psi(t)}\left( \tau - an^{m } \right) |^2 \ d \tau
		\\
    & = \int_\rr \left( 1 + |\tau'| \right)^{2b} | \wh{t^k \psi(t)}(\tau') |^2 \
		d \tau'
		\\
    & \le \int_\rr \left( 1 + |\tau'| \right)^{2b} | \wh{t^k \psi(t)}(\tau')
		|^2 \ d \tau'
		\\
    & \lesssim \int_\rr \left( 1 + | \tau' | \right)^{2b}| \wh{t^{k}
		\psi(t)}(\tau') |^2 \ d \tau'
    \\
    & \le \| s^{k} \psi \|_{H^{[b] + 1}}^{2}
\end{split}
\end{equation}
%
%
where $[b]$ denotes the least integer of $b$. Note that
%
%
\begin{equation}
	\label{4n}
	\begin{split}
    & \|t^k \psi \|_{H^{[b] +1}(\rr)}
		\\
    & = \|t^k \psi\|_{L^2(\rr)} + \|\p_t (t^k \psi )
    \|_{L^2(\rr)} 
    \\
    & + \| \p_{t}^{2} (t^{k} \psi) \|_{L^{2}(\rr)} + \ldots + \|
    \p_{t}^{[b] + 1} (t^{k} \psi)\|_{L^{2}}
    \\
    & \le c_{\psi} + k c'_{\psi} + k (k -1) c_{\psi}'' + \ldots +
    k(k-1) \ldots (k - [b]) c_{\psi, b}^{[b] + 1}
    \\
    & \lesssim c_{\psi, b} k(k-1) \cdots (k - [b]).
	\end{split}
\end{equation}
%
%
Hence, applying \cref{3n} and \cref{4n} to \cref{2n}, we obtain
%
%%
\begin{equation}
	\label{5n}
	\begin{split}
		\|\cref{main1-rel-term-5} \|_{X_{s,b}}
		& \lesssim c_{\psi, b} \sum_{a = \pm 1}
    \sum_{k \ge [b] +1} \frac{1}{(k-[b] - 1)!} \left( \sum_{n \in \zz} \left (1 + |n| \right )^{2s} | h_k(n) |^2 
		\right)^{1/2}
		\\
    & \le c_{\psi, b}  \sum_{a = \pm 1} \sum_{k \ge [b] +1} \frac{1}{(k-[b] - 1)!}
    \times \sup_{k \ge [b] + 1} \left( \sum_{n \in \zz} \left (1 + |n| \right )^{2s} | 
		h_k(n) |^2 \right)^{1/2}
		\\
    & = c_{\psi, b}  \sum_{a = \pm 1}\sum_{k \ge [b] +1} \frac{1}{(k-[b] - 1)!}
    \\
    & \times \sup_{k \ge [b] + 1} 
		\left( \sum_{n \in \zz} \left (1 + |n| \right )^{2s} |\int_\rr 
		\psi\left( \tau - an^{m } \right) \cdot \left( \tau - an^{m } 
    \right)^{k -1} \wh{w}(n, \tau) \ d \tau|^{2} \right)^{1/2}.
    \end{split}
\end{equation}
%
%%
Recall that $0 \le \psi \le 1, \text{ supp} \, \psi \subset [-1,1 ]$. 
This implies $$| \psi\left( \tau - an^{m } \right) \cdot \left( \tau - an^{m }
\right)^{k -1} | \le \chi_{| \tau - an^{m } | \le 1}, \qquad k \ge 1.$$ Hence,
we bound the double sum on the right hand side of \cref{5n} by
%
%%
\begin{equation*}
	\begin{split}
    & \sum_{a = \pm 1}
    \sum_{k \ge [b] +1} \frac{1}{(k-[b] - 1)!}
    \times \left( \sum_{n \in \zz} (1 + | n |)^{2s}| 
		\int_{| \tau - an^{m}  |\le 1}  \wh{w}(n, \tau) \ d \tau |^2 
		\right)^{1/2}
    \\
    & = e \sum_{a = \pm 1} \left( \sum_{n \in \zz} (1 + | n |)^{2s}| 
		\int_{| \tau - an^{m}  |\le 1}  \wh{w}(n, \tau) \ d \tau |^2 
		\right)^{1/2}
    \\
    & \le e \sum_{a = \pm 1}
\left[ \sum_{n \in \zz} (1 + | n |)^{2s}\left (  
		\int_{| \tau - an^{m}  |\le 1} | \wh{w}(n, \tau) | \ d \tau \right ) ^2 
		\right]^{1/2}
	\end{split}
\end{equation*}
%
%%
which by the inequality
%
%%
\begin{equation*}
	\begin{split}
		1 \le 
		\frac{2}{1 + | j |}, \qquad | j | \le 1
	\end{split}
\end{equation*}
%
%%
and \cref{eqn:norm-key-ineq}
is bounded by 
%
%%
\begin{equation}
\label{main-int4-est-X-s-part}
	\begin{split}
		& 4e \left[ \sum_{n \in \zz} (1 + | n |)^{2s}\left ( \int_\rr
		\frac{|\wh{w}(n, \tau)|}{1 + | |\tau| - n^{m } |} \ d \tau \right ) ^2 
		\right]^{1/2}.
  \end{split}
\end{equation}
%
%%
Applying \cref{cor:bilinear-estimate2} to \cref{main-int4-est-X-s-part}, we
conclude that
%
%
\begin{equation}
\label{main-int4-est}
	\begin{split}
    \|\cref{main1-rel-term-5}\|_{X_{s,b}} \le c_{\psi}     \|u\|_{X_{s,b}}^{3}.
	\end{split}
\end{equation}
%
%
Collecting estimates \cref{main-int2-est}, 
\cref{main-int3-est}, and \cref{main-int4-est}, and recalling 
\cref{main1-rel-term-2}-\cref{main1-rel-term-5}, we obtain the following. 
%
%
\begin{proposition}
\label{prop:contraction}

For $1/4 < 1-b < 1/2$ and $s \ge -(1-b)/2$, we have
%%
\begin{equation*}
	\begin{split}
    \|Tu\|_{X_{s,b}} \le c \left( \|u_0 \|_{H^s(\ci)} + \|u_1 \|_{H^{s-2}(\ci)}
    +  \|u\|_{X_{s,b}}^2 
		\right)
	\end{split}
\end{equation*}
%
where $c = c_{\psi, b} > 1$.  
%%
\end{proposition}
%
\subsection{Proof of Existence and Uniqueness}
\label{sec:proof-b4-per-case}
%
%
%%%%%%%%%%%%%%%%%%%%%%%%%%%%%%%%%%%%%%%%%%%%%%%%%%%%%
%
%% Contraction Proposition
%				 
%%%%%%%%%%%%%%%%%%%%%%%%%%%%%%%%%%%%%%%%%%%%%%%%%%%%%%
%%
%%
%
We will now use \cref{prop:contraction} to prove local well-posedness for the 
$B_4$ ivp. Suppose
%
%%
\begin{equation*}
	\begin{split}
    \|(u_0, u_{1})\|_{H^s(\ci) \times H^{s-2}(\ci)} \le r.
  \end{split}
\end{equation*}
%
%%
Then $$\|u\|_{X_{s,b}} \le 2rc$$ implies
%
%%
\begin{equation*}
	\begin{split}
		\|Tu \|_{X_{s,b}} 
    & \le c \left[ r + \left( 
		2rc \right)^2 \right].
	\end{split}
\end{equation*}
%
If 
%
%
\begin{equation}
  \label{delta-suf-small}
\begin{split}
  r \le  \frac{1}{8c^{2}} 
\end{split}
\end{equation}
%
%
we obtain 
%%
%
%
%
\begin{equation*}
\begin{split}
\|Tu \|_{X_{s,b}} 
    & \le \frac{3}{2}rc.
  \end{split}
\end{equation*}
%
%
Hence, $T=T_{u_0, u_1, \psi}$ maps the ball $B_{X_{s,b}}(2rc)$ into
itself. Next, note that 
%
%%
\begin{equation*}
	\begin{split}
    Tu - Tv = \cref{main1-rel-term-3} + \cref{main1-rel-term-4} +
    \cref{main1-rel-term-5} 
  \end{split}
  \label{eqn:integral-form-dif}
\end{equation*}
%
%%
where now $w(x,t) =u^{2} - v^{2}$. Rewriting
%
%%
\begin{equation*}
	\begin{split}
	u^2 - v^2
		& = (u-v)(u+v)
		\end{split}
\end{equation*}
%
%%
and repeating earlier arguments, we obtain
%
%%
%%
\begin{equation}
	\label{20a}
	\begin{split}
		\|Tu - Tv \|_{X_{s,b}}  
    & \le c \|u -v\|_{X_{s,b}} \|u + v \|_{X_{s,b}}
		\\
    & \le c \|u -v\|_{X_{s,b}} (\|u\|_{X_{s,b}}+ \|v \|_{X_{s,b}}).
	\end{split}
\end{equation}
%
%%
If $$ u, v \in B_{X_{s,b}} \left (2rc \right )$$ then 
%%
\begin{equation}
	\label{21a}
	\begin{split}
		\|Tu - Tv \|_{X_{s,b}}
    & \le c  \|u -v \|_{X_{s,b}} \left( 2rc + 
		2rc \right)
		\\
		& = \frac{1}{2} \|u -v \|_{X_{s,b}}. 
	\end{split}
\end{equation}
%
%%
We conclude that $T$ is a contraction on the ball $B_{X_{s,b}}(2rc)$.
A Picard iteration then yields a unique 
$u \in X_{s,b}$ satisfying $u = Tu$. Applying
\cref{lem:embedding}, it follows that $u(x,t) \subset C( [-1/2, 1/2], H^s)$ is a unique
solution of the B4 ivp \cref{eqn:mb-2}-\cref{eqn:mb-init-data-2} for $t
\in [-1/2, 1/2]$.
%
%
\subsection{Proof of Lipschitz Continuity} 
\label{sec:lip-continuity}
%
%
%
%
Let $$(u_0, u_1), (v_0, v_1)  \subset
B_{H^{s} \times H^{s-2}} \left (r \right ).$$ Then for initial data sufficiently
small (i.e. with size satisfying \cref{delta-suf-small}), there exist $u, v \in
X_{s,b}$ such that $u =
T_{u_0, u_1}u$, $v = T_{v_0, v_1} v$, and so
%
%
\begin{align}
  \notag
    & T_{u_0, u_1}(u) - T_{v_0, v_1}(v)
		\\
    & = \psi(t) \sum_{n \in \zz} e^{inx} \wh{u_{0} - v_{0} }(n) \frac{e^{in^{2}t} + e^{-in^{2}t}}{2} 
\label{main1-rel-term-1g}
  \\
  & + \psi(t) \sum_{n \in \zz} e^{inx}
  \wh{u_{1} - v_{1} }(n)\frac{e^{in^{2}t} - e^{-in^{2}t}}{2 i n^{2}} 
\label{main1-rel-term-2g}
  \\
  & + \psi(t) \sum_{a = \pm 1} \sum_{n\in \zz} \int_\rr e^{ixn}  
  e^{it \tau} \frac{ 1 - \psi(\tau -  an^{2}) 
  }{\tau -  an^{2}} \wh{w}(n, \tau) \ d \tau
\label{main1-rel-term-3g}
  \\
  & + \psi(t) \sum_{a = \pm 1} \sum_{n\in \zz} \int_\rr e^{i(xn + 
  t an^{2})}
  \frac{1- \psi(\tau -  an^{2})}{\tau -  an^{2}} \wh{w}(n, \tau) \ d \tau
\label{main1-rel-term-4g}
  \\
  & + \psi(t) \sum_{a = \pm 1}  \sum_{k \ge 1} \frac{i^k t^k}{k!}
  \sum_{n \in \zz} \int_\rr e^{i(xn + t an^{2} )}
  \psi(\tau -  an^{2}) (\tau -  an^{2})^{k-1} \wh{w}(n, \tau)
  \label{main1-rel-term-5g}
\end{align}
%
where now $w = u^{2} - v^{2}$. Using arguments similar to those in 
\cref{ssec:est-init-term-1}-\cref{ssec:est-init-term-2}
we obtain
%
%
\begin{equation}
	\label{gen-2a}
	\begin{split}
    & \| \cref{main1-rel-term-1g}\|_{X_{s,b}}
		\le c \|u_0 -v_0\|_{H^s},
    \\
    & \| \cref{main1-rel-term-2g}\|_{X_{s,b}}
    \le c \|u_1 -v_1\|_{H^{s-2}}.
	\end{split}
\end{equation}
%
%
%
%
Therefore, from \cref{21a} and \cref{gen-2a}, we obtain
%
%
\begin{equation*}
	\begin{split}
    \|u -v \|_{X_{s,b}}
    & = \|T_{u_0, v_0}(u) - T_{u_1, v_1}(v) \|_{X_{s,b}}
    \\
    & \le
    c \left( \|u_0 -v_0 \|_{H^s\left( \ci \right)} +\|u_1 -v_1
        \|_{H^{s-2}\left( \ci \right)} \right )
        + \frac{1}{2} \|u -v \|_{X_{s,b}}
  \end{split}
\end{equation*}
%
%
which implies
%
%
\begin{equation*}
	\begin{split}
		\frac{1}{2} \|u-v\|_{X_{s,b}} \le
    c \left( \|u_0 -v_0 \|_{H^s\left( \ci \right)} +\|u_1 -v_1
        \|_{H^{s-2}\left( \ci \right)} \right )
      \end{split}
\end{equation*}
%
%
or
%
%
\begin{equation}
	\begin{split}
		\|u -v \|_{X_{s,b}} \le 2 c \left( \|u_0 -v_0 \|_{H^s\left( \ci \right)} +\|u_1 -v_1
        \|_{H^{s-2}\left( \ci \right)} \right ).
	\end{split}
  \label{pre-lem-estimate}
\end{equation}
%
%
Applying  \cref{lem:embedding} to \cref{pre-lem-estimate}, it follows that
for $(u_0, u_1), (v_0, v_1)  \subset
B_{H^{s} \times H^{s-2}} \left (r \right )$, the
associated solutions $u, v \in C([-1/2, 1/2], H^{s}(\ci))$ satisfy the estimate%
%
%
	 %
	 %
	 \begin{equation*}
		 \begin{split}
       \sup_{t \in [-1/2, 1/2]} \|u(\cdot, t) -v(\cdot, t) \|_{H^s(\ci)} \le
      2 c \left( \|u_0 -v_0 \|_{H^s\left( \ci \right)} +\|u_1 -v_1
        \|_{H^{s-2}\left( \ci \right)} \right ).
		 \end{split}
	 \end{equation*}
	 %
	 %
Hence, the flow map is Lipschitz continuous from $B_{H^{s}
\times H^{s-2}} \left (r \right )$ to $C([-1/2, 1/2],
H^{s}(\ci))$. This
concludes the proof of well-posedness for the $B_4$ ivp
\cref{eqn:mb-2}-\cref{eqn:mb-init-data-2}. \qquad \qedsymbol

%
%
%
%
% \bib, bibdiv, biblist are defined by the amsrefs package.
\begin{bibdiv}
\begin{biblist}
\bib{Farah:2009uq}{article}{
      author={Farah, Luiz~Gustavo},
       title={Local solutions in {S}obolev spaces with negative indices for the
  ``good'' {B}oussinesq equation},
        date={2009},
        ISSN={0360-5302},
     journal={Comm. Partial Differential Equations},
      volume={34},
      number={1-3},
       pages={52\ndash 73},
         url={http://dx.doi.org/10.1080/03605300802682283},
      review={\MR{2512853 (2010k:35404)}},
}
\bib{Ginibre:1996fk}{article}{
      author={Ginibre, Jean},
       title={Le probl{\`e}me de {C}auchy pour des {EDP} semi-lin{\'e}aires
  p{\'e}riodiques en variables d'espace (d'apr{\`e}s {B}ourgain)},
        date={1996},
        ISSN={0303-1179},
     journal={Ast{\'e}risque},
      number={237},
       pages={Exp.\ No.\ 796, 4, 163\ndash 187},
        note={S{{\'e}}minaire Bourbaki, Vol. 1994/95},
      review={\MR{1423623 (98e:35154)}},
}
\bib{Ginibre:1997fk}{article}{
      author={Ginibre, J.},
      author={Tsutsumi, Y.},
      author={Velo, G.},
       title={On the {C}auchy problem for the {Z}akharov system},
        date={1997},
        ISSN={0022-1236},
     journal={J. Funct. Anal.},
      volume={151},
      number={2},
       pages={384\ndash 436},
         url={http://dx.doi.org/10.1006/jfan.1997.3148},
      review={\MR{1491547 (2000c:35220)}},
}
\bib{Kenig:1996aa}{article}{
      author={Kenig, Carlos~E.},
      author={Ponce, Gustavo},
      author={Vega, Luis},
       title={A bilinear estimate with applications to the {K}d{V} equation},
        date={1996},
        ISSN={0894-0347},
     journal={J. Amer. Math. Soc.},
      volume={9},
      number={2},
       pages={573\ndash 603},
         url={http://dx.doi.org/10.1090/S0894-0347-96-00200-7},
      review={\MR{1329387 (96k:35159)}},
}
\bib{Kenig-Ponce-Vega-1996-Quadratic-forms-for-the-1-D-semilinear}{article}{
      author={Kenig, Carlos~E.},
      author={Ponce, Gustavo},
      author={Vega, Luis},
       title={Quadratic forms for the {$1$}-{D} semilinear {S}chr{\"o}dinger
  equation},
        date={1996},
        ISSN={0002-9947},
     journal={Trans. Amer. Math. Soc.},
      volume={348},
      number={8},
       pages={3323\ndash 3353},
         url={http://dx.doi.org/10.1090/S0002-9947-96-01645-5},
      review={\MR{1357398 (96j:35233)}},
}
\end{biblist}
\end{bibdiv}
%\bibliography{/Users/davidkarapetyan/math/bib-files/references.bib}
%
%\nocite{*}
\end{document}
