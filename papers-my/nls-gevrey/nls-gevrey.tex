%
\documentclass[12pt,reqno]{amsart}
\usepackage{amssymb}
\usepackage{cancel}  %for cancelling terms explicity on pdf
\usepackage{yhmath}   %makes fourier transform look nicer, among other things
\usepackage{framed}  %for framing remarks, theorems, etc.
\usepackage{enumerate}
%\usepackage{showkeys}  %shows source equation labels on the pdf
\usepackage[margin=3cm]{geometry}  %page layout
%\usepackage[pdftex]{graphicx} %for importing pictures into latex--pdf compilation
\setcounter{secnumdepth}{1} %number only sections, not subsections
\usepackage[shortalphabetic, initials, msc-links]{amsrefs} %for the bibliography; uses cite pkg
\hypersetup{colorlinks=true, linkcolor=blue, citecolor=blue, urlcolor=blue}
\synctex=1
\numberwithin{equation}{section}  %eliminate need for keeping track of counters
\numberwithin{figure}{section}
\setlength{\parindent}{0in} %no indentation of paragraphs after section title
\renewcommand{\baselinestretch}{1.1} %increases vert spacing of text
%
%
\newcommand{\ds}{\displaystyle}
\newcommand{\ts}{\textstyle}
\newcommand{\nin}{\noindent}
\newcommand{\rr}{\mathbb{R}}
\newcommand{\nn}{\mathbb{N}}
\newcommand{\zz}{\mathbb{Z}}
\newcommand{\cc}{\mathbb{C}}
\newcommand{\ci}{\mathbb{T}}
\newcommand{\zzdot}{\dot{\zz}}
\newcommand{\wh}{\widehat}
\newcommand{\p}{\partial}
\newcommand{\ee}{\varepsilon}
\newcommand{\vp}{\varphi}
%
%
\theoremstyle{plain}  
\newtheorem{theorem}{Theorem}
\newtheorem{proposition}{Proposition}
\newtheorem{lemma}{Lemma}
\newtheorem{corollary}{Corollary}
\newtheorem{claim}{Claim}
\newtheorem{conjecture}[subsection]{conjecture}
%
\theoremstyle{definition}
\newtheorem{definition}{Definition}
%
\theoremstyle{remark}
\newtheorem{remark}{Remark}
%
% Goes with hyperref package and \autoref command--now when you type
% \autoref{some-environment}, it typsets the reference with the appropriate
% environment name. For example, \autoref{some-theorem} gives Theorem 2,
% \autoref{some lemma} gives Lemma 2, and so on
\def\makeautorefname#1#2{\expandafter\def\csname#1autorefname\endcsname{#2}}
\makeautorefname{equation}{Equation}
\makeautorefname{footnote}{footnote}
\makeautorefname{item}{item}
\makeautorefname{figure}{Figure}
\makeautorefname{table}{Table}
\makeautorefname{part}{Part}
\makeautorefname{appendix}{Appendix}
\makeautorefname{chapter}{Chapter}
\makeautorefname{section}{Section}
\makeautorefname{subsection}{Section}
\makeautorefname{subsubsection}{Section}
\makeautorefname{paragraph}{Paragraph}
\makeautorefname{subparagraph}{Paragraph}
\makeautorefname{theorem}{Theorem}
\makeautorefname{theo}{Theorem}
\makeautorefname{thm}{Theorem}
\makeautorefname{addendum}{Addendum}
\makeautorefname{add}{Addendum}
\makeautorefname{maintheorem}{Main theorem}
\makeautorefname{corollary}{Corollary}
\makeautorefname{lemma}{Lemma}
\makeautorefname{sublemma}{Sublemma}
\makeautorefname{proposition}{Proposition}
\makeautorefname{property}{Property}
\makeautorefname{scholium}{Scholium}
\makeautorefname{step}{Step}
\makeautorefname{conjecture}{Conjecture}
\makeautorefname{question}{Question}
\makeautorefname{definition}{Definition}
\makeautorefname{notation}{Notation}
\makeautorefname{remark}{Remark}
\makeautorefname{remarks}{Remarks}
\makeautorefname{example}{Example}
\makeautorefname{algorithm}{Algorithm}
\makeautorefname{axiom}{Axiom}
\makeautorefname{case}{Case}
\makeautorefname{claim}{Claim}
\makeautorefname{assumption}{Assumption}
\makeautorefname{conclusion}{Conclusion}
\makeautorefname{condition}{Condition}
\makeautorefname{construction}{Construction}
\makeautorefname{criterion}{Criterion}
\makeautorefname{exercise}{Exercise}
\makeautorefname{problem}{Problem}
\makeautorefname{solution}{Solution}
\makeautorefname{summary}{Summary}
\makeautorefname{operation}{Operation}
\makeautorefname{observation}{Observation}
\makeautorefname{convention}{Convention}
\makeautorefname{warning}{Warning}
\makeautorefname{note}{Note}
\makeautorefname{fact}{Fact}
%
%
\begin{document}
\title{Well-posedness of NLS in Analytic Spaces on the Circle}
\author{A. Himonas, D. Karapetyan, and G. Petronilho}
\date{November 10, 2010}
%
\maketitle
%
%
%
%
%
\section{Introduction}
\label{sec:introduction}
We consider the Cauchy problem for the NLS equation
%
\begin{gather}
  \label{eqn:nls}
  i\partial_tu+\partial_x^2u+\lambda |u|^2u=0,
  \\
  \label{eqn:nls-data}
  u(x,0)=\varphi(x) \in \mathcal{C}^\omega(\mathbb{T}),
\end{gather}
%
where $\lambda =\pm 1$, and prove the following theorem.
%
\begin{theorem}
  \label{thm:nls-analyt}
  The solution $u(x,t)$  to the Cauchy problem for NLS \eqref{eqn:nls}--
  \eqref{eqn:nls-data}  with   initial data $\varphi(x)$  in  the Gevrey space
  $G^1(\mathbb{T})=\mathcal{C}^\omega(\mathbb{T})$ belongs to the anisotropic
  Gevrey space $G^{1, \sigma}( \mathbb{T}\times [-T, T])$,  $\sigma \ge 2$, where
  $T>0$ is the lifespan of the solution. Furthermore, $\sigma$ is sharp in the
  sense that $u(x,t)$ as a function of  time  may fail to be in $G^r$, $1\le
  r<2$, near $0$.
\end{theorem}
%
%
%
%
\begin{framed}
\begin{remark}
The cubic NLS has a number of invariances. 
\begin{enumerate}[(i)]
  \item{ $u_{\gamma} = \gamma u(\gamma x, \gamma^{2} t)$ with critical Sobolev index
    $s_{c} = -1/2$}.
    \label{invar-it-1}
  \item{ $u_{a,b}(x,t) = u(x -a, t - b), a \in \rr, b \in \rr$}.
    \label{invar-it-2}
  \item{ $u_{\theta}(x,t) = e^{i \theta} u(x, t), \theta \in \rr$}.
    \label{invar-it-3}
  \item{ $u_{c}(x,t) = e^{icx - i| c^{2} |t}u(x -2tc, t), c \in \rr$}.
    \label{invar-it-4}
\end{enumerate}
{\bf Proof of \eqref{invar-it-1}.}
Let $u(x, t)$ be a solution to the $NLS$ equation, that is
%
$$
NLS(u)=
i \p_t u + \p_x^{2} u  + | u |^{2} u  = 0
$$
%
We would like to find the constants
$a, b, c$ such that
\[
u_\gamma (x, t) = \gamma^a u(\gamma^b x, \gamma^c t)
\]
is also a solution to $NLS$.  Since 
$$
NLS(u_\gamma)=
i \gamma^{a + c} \p_t u + \gamma^{a + 2b} \p_x^{2} u  + \gamma^{3a} | u |^{2} u 
$$
we see that $u_\gamma$ is a $NLS$ solution only if
$$
a=b, c = 2a.
$$
Setting $a=1$ completes the proof. 
To find the critical Sobolev index, we compute
%
%
\begin{equation}
\begin{split}
  \| u_{\gamma} \|_{\dot{H}^s(\ci)} 
  & = \gamma \| u(\gamma x, \gamma^2 t) \|_{\dot{H}^{s}(\ci)}
  \\
  & = \gamma \left( \int_{\rr} | \xi |^{2s} | \wh{u(\gamma x,
  \gamma^{2} t)}^x (\xi, t)| \right)^{1/2}.
\end{split}
\label{crit-ind-comp}
\end{equation}
%
But
%
%
\begin{equation*}
\begin{split}
  \wh{u(\gamma x, \gamma^{2}t)^x}(\xi, t)
  & = \int_{\rr}e^{-i\xi x}u(\gamma x, \gamma^2 t) dx
  \\
  & = \frac{1}{\gamma} \int_{\rr}e^{-i \frac{n}{\gamma} x'}u(x',
  \gamma^{2} t) dx'
  \\
  & = \frac{1}{\gamma} \wh{u(\cdot, \gamma^{2}t)}(\frac{\xi}{\gamma})
\end{split}
\end{equation*}
%
%
Substituting back into \eqref{crit-ind-comp}, we obtain
%
%
\begin{equation*}
\begin{split}
  \| u_{\gamma} \|_{\dot{H}^s(\rr)} 
  & = \gamma\left( \int_{\rr} | \xi |^{2s} |
  \frac{1}{\gamma}\wh{u(\cdot, \gamma^{2}t)}(\frac{\xi}{\gamma}) |^2 d \xi
  \right)^{1/2}
  \\
  & = \left( \int_{\rr}| \xi |^{2s} | \wh{u(\cdot,
  \gamma^{2}t)}(\frac{\xi}{\gamma}) |^2 d \xi  \right)^{1/2}
  \\
  & = \left( \int_{\rr} | \gamma \xi' |^{2s} 
  \wh{u(\cdot, \gamma^{2}t)}(\xi') |^2 \gamma d \xi
  \right)^{1/2}
  \\
  & = \gamma^{s + 1/2} \|u(\cdot, \gamma^{2}t) \|_{\dot{H}^s (\ci)}.
\end{split}
\end{equation*}
%
%
Therefore, $\| u_{\gamma(0)} \|_{\dot{H}^s(\rr)} = \gamma^{s + 1/2} \|
u_{0} \|_{\dot{H}^{s}(\rr)}$, and so $s=-1/2$ is the critical Sobolev index.
\label{rem:invariances}
\\
{\bf Proof of \eqref{invar-it-2}.} Let $u(x, t)$ be a solution to the $NLS$ equation.
We would like to find constants
$a, b \in \rr$ such that $$u_{a,b}(x,t) = u(x -a, t - b)$$ is a solution the NLS
equation. Since
%
%
\begin{equation*}
\begin{split}
  NLS(u_{a,b}) = i \p_t u + \p_x^{2} u  + | u |^{2} u 
\end{split}
\end{equation*}
%
%
we see that $u_{a,b}$ is a solution to the NLS equation for all $a, b \in \rr$.
This completes the proof.
\\
{\bf Proof of \eqref{invar-it-3}.}
Let $u(x, t)$ be a solution to the $NLS$ equation.
We would like to find a constant
$c \in \cc$ such that
\[
u_c (x, t) = c u(x, t)
\]
is also a solution to $NLS$.  Since 
$NLS(u_{c}) = c i \p_{t} u  + c \p_{x}^{2} u + |c|^2c | u |^{2}u$, we obtain
$|c|^{2} = 1$, or $c = e^{i \theta}$, for any $\theta \in \rr$.
\\
{\bf Proof of \eqref{invar-it-4}.} Pending. Want to find a clever (i.e. natural)
way of coming up with the galilean invariance, rather than just verifying the relation.
\end{remark}
\end{framed}
%
%

%
%
%%%%%%%%%%%%%%%%%%%%%%%%%%%%%%%%%%%%%%%%%%%%%%%%%%%%%
%
%
%                Analyticity in Space
%
%
%%%%%%%%%%%%%%%%%%%%%%%%%%%%%%%%%%%%%%%%%%%%%%%%%%%%%
%
%
\section{Introduction to the Proof of Spatial Analyticity} 
\label{sec:space-anal}
We follow \cite{Gorsky:2005fk}, and differentiate
\eqref{eqn:nls}-\eqref{eqn:nls-data} $k$-times to obtain the system
%
%
\begin{gather}
  i \p_t u_k + \p^2_x u_k + B_{k}= 0, 
  \label{eqn:nls-system}
  \\
  u_{k}(x,0) = \vp_k(x), \quad k=0,1,2,\ldots
  \label{eqn:nls-system-init-data}
\end{gather}
%
%
where
\begin{gather*}
  u_k \doteq \p_x^k u, \quad \vp_k \doteq \p_x^k \vp, \quad \text{and}
  \quad B_k = B_{k}(u,u) =
  \lambda \sum_{j=0}^{k} \sum_{i=0}^{j} {k \choose j}{j \choose i} u_{k-j} u_{j-i}
  \bar{u_i}
  \label{eqn:system-notat}
\end{gather*}
Rewrite the system in integral form
%
%
\begin{equation}
\begin{split}
  u_k(x,t) = W(t) \vp_{k}(x) - \int_{0}^{t} W(t- \tau)B_{k}(x, \tau) d \tau
  \label{eqn:sys-integral-form}
\end{split}
\end{equation}
%
%
where $W(t) = e^{it \Delta}$. To localize in time, we introduce a cutoff
function $\psi(t) \in C^{\infty}_{0}(-1,1)$ with $0 \le \psi \le 1$ and
$\psi(t) \equiv 1$ for $| t | < 1/2$, and multiply \eqref{eqn:sys-integral-form}
by $\psi(t)$ to obtain
%
%
\begin{align}
\psi(t) u_k(x,t)
& = \psi(t) W(t) \vp_{k}(x) - \psi(t) \int_{0}^{t} W(t-
\tau)B_{k}(x, \tau) d \tau \notag
\\
\label{main-int-expression-1}
    & = \frac{1}{2 \pi} \psi(t) \sum_{n \in \zz} e^{i(xn + tn^{m 
    })} \widehat{\vp_k}(n) 
    \\
    & + \frac{1}{4 \pi^2} \psi(t) \sum_{n\in \zz} \int_\rr e^{ixn}  
    e^{it \tau} \frac{ 1 - \psi(\tau -  n^m) 
    }{\tau -  n^m} \wh{B_k}(n, \tau) \ d \tau
    \label{main-int-expression-2}
    \\
    \label{main-int-expression-3}
    & - \frac{1}{4 \pi^2} \psi(t) \sum_{n\in \zz} \int_\rr e^{i(xn + 
    t n^m)}
     \frac{1- \psi(\tau -  n^m)}{\tau -  n^m} \wh{B_k}(n, \tau) \ d \tau
    \\
    \label{main-int-expression-4}
    & + \frac{1}{4 \pi^2} \psi(t) \sum_{k \ge 1} \frac{i^k t^k}{k!}
    \sum_{n \in \zz} \int_\rr e^{i(xn + t n^m )}
    \psi(\tau -  n^m) (\tau -  n^m)^{k-1} \wh{B_k}(n, \tau)  
    \\
    & \doteq T_k(u_0, u_1,\cdots,u_k) \notag
\end{align}
%
where $T_k = T_{k,\vp_k}$.
%
The strategy for proving spatial analyticity will be to show that there exists a
map constructed from the $\{T_k\}$ which is a
contraction on an appropriate subspace of the following spaces.
\begin{definition}
	Denote $Y^s$ to be the space of all
	functions $u$ on $\ci \times \rr$ with
	bounded norm
\begin{equation}
	\label{Y-s-norm}
	\begin{split}
    \|u\|_{Y^s} = \|u\|_{X^s} + \|u\|_E
  \end{split}
\end{equation}
%
%
%
%
where
%
\begin{equation}
	\label{X^s-norm}
	\begin{split}
		& \|u\|_{X^s}
		= \left ( \sum_{n\in \zz} \left (1 + |n| \right )^{2s} \int_\rr \left ( 1 + | 
		\tau - n^{2} \right ) | \wh{u} ( n, \tau ) |^2
		\right )^{1/2},
    \\
    & \|u\|_E = \left( \sum_{n \in \zz} | n |^{2s} \left(
    \int_{\rr} | \wh{u}(n, \tau) |d \tau
    \right)^{2} \right)^{1/2}.
	\end{split}
\end{equation}
%
%
%
\end{definition}
The $Y^s$ spaces were first introduced by Colliander, Keel, Staffilani, Takaoka,
and Tao \cite{Colliander:2003kx}, and are a natural extension of the Bourgain spaces
\cite{Bourgain-Fourier-transfo-1}, \cite{Bourgain-Fourier-transfo}. Our main
motivation for introducing the $Y^s$ spaces is that, unlike the
Bourgain spaces, they possess the
following important property.
%
%
\begin{lemma}
	\label{lem:cutoff-loc-soln}
	Let $\psi(t)$ be a smooth cutoff function with $\psi(t) =1$ for $t \in [-T, T]$. If
	$\psi(t)u(x,t) \in Y^s$, then $u \in C([-T, T], H^s(\ci))$.
\end{lemma}
%
%
%
%
%%%%%%%%%%%%%%%%%%%%%%%%%%%%%%%%%%%%%%%%%%%%%%%%%%%%%
%
%
%                Spaces for Picard Iteration
%
%
%%%%%%%%%%%%%%%%%%%%%%%%%%%%%%%%%%%%%%%%%%%%%%%%%%%%%
%
%
\section{Introducing Appropriate Spaces for the Picard Iteration}
\label{sec:picard-spaces}
Note that $\vp \in
\mathcal{C}^{\omega}(\ci)$ if and only if
%
%
\begin{equation}
\begin{split}
  \|\vp \|_{H^s(\ci)} \le M_{0} \left( \frac{1}{2 C_0} \right)^{k} k!, \quad k
  \in \mathbb{N}_0
  \label{eqn:analy-condition}
\end{split}
\end{equation}
%
%
for some positive constants $M_0$ and $C_0$. Therefore, if we let
%
%
\begin{equation*}
\begin{split}
  \left\{ \vp_k \right\} \doteq \left( \vp_{0}, \vp_{1}, \vp_{2}, \cdots
  \right), \quad \vp_k = \p_x^k \vp,
\end{split}
\end{equation*}
%
%
and define the norm
%
%
\begin{equation*}
\begin{split}
  \|\left\{ \vp_k \right\} \|_{s} \doteq \sum_{k=0}^\infty \frac{C_0^k}{k!}
  \|\vp_k \|_{H^s(\ci)},
\end{split}
\end{equation*}
%
%
then by \eqref{eqn:analy-condition} we have
%
%
\begin{equation*}
\begin{split}
  \|\left\{ \vp_k \right\} \|_s < \infty.
\end{split}
\end{equation*}
%
%
Therefore, if we denote 
%
%
\begin{equation*}
\begin{split}
  | | |\left\{ u_k \right\} | | | \doteq \sum_{k=0}^\infty \frac{C_0^k}{k!}
  \|u_k \|_{H^s(\ci)}
\end{split}
\end{equation*}
%
%
then a natural space for expressing spatial analyticity is
%
%
\begin{equation*}
\begin{split}
  \mathcal{A}(Y^s) \doteq \left\{ (u_0, u_1, u_2, \cdots ) \doteq \left\{ u_k
  \right\}: u_j \in Y^s, j \in \mathbb{N}_0, \ \text{and}\ | | |\left\{ u_k
  \right\} | | | < \infty \right\}.
\end{split}
\end{equation*}
%
%
In the non-periodic case, such norms have been introduced by Kato and Ogawa
\cite{Kato:2000vn}. Our strategy is to show the existence of solutions $\left\{
u_k \right\}$ to \eqref{eqn:nls-system}-\eqref{eqn:nls-system-init-data}
satisfying
%
%
\begin{equation*}
\begin{split}
  | | |\left\{ u_k \right\} | | | < \infty.
\end{split}
\end{equation*}
%
%
This will imply that there exists $M_1 >0$ such that
%
%
\begin{equation*}
\begin{split}
  \frac{C_0^k}{k!} \|u_k\|_{Y^s} < M_1, \quad k \in \mathbb{N}_0
\end{split}
\end{equation*}
%
%
or
%
%
\begin{equation*}
\begin{split}
  \|u_k\|_{Y^s} \le M_1 \left( \frac{1}{C_0}^k  \right)^k, \quad k \in
  \mathbb{N}_0.
\end{split}
\end{equation*}
%
%
Then by \autoref{lem:cutoff-loc-soln} we will have that
%
%
\begin{equation*}
\begin{split}
  \|\p_x^k u(\cdot, t) \|_{H^s(\ci)} = \|u_k(\cdot, t) \|_{H^s(\ci)} \le \|u_k
  \|_{Y^s} \le c M_1 \left( \frac{1}{C_0} \right)^k k!, \quad k \in
  \mathbb{N}_0,
\end{split}
\end{equation*}
%
%
implying $u(\cdot, t) \in \mathcal{C}^\omega (\ci)$. 
%
%
%
%%%%%%%%%%%%%%%%%%%%%%%%%%%%%%%%%%%%%%%%%%%%%%%%%%%%%
%
%
%                Proof of Spatial Analyticity
%
%
%%%%%%%%%%%%%%%%%%%%%%%%%%%%%%%%%%%%%%%%%%%%%%%%%%%%%
%
%
\section{Proof of Spatial Analyticity} 
\label{sec:proof-spat-anal}
We proceed by estimating the $Y^s$ norms of
\eqref{main-int-expression-1}-\eqref{main-int-expression-4}. This will allow us
to show that $T_k$ is a contraction on $\mathcal{A}(Y^s)$. A Picard iteration
will then complete the proof. 
%
%
\subsection{Estimate for \eqref{main-int-expression-1}.}
Letting $f_k(x,t) = \psi(t) \sum_{n \in \zz} e^{i(xn + tn^{2})} 
\wh{\vp_k}(n)$, we have $\wh{f_k}(n,t) = \psi(t) \wh{\vp_k}(n) e^{itn^{2}}$,
from which we obtain
%
%
\begin{equation*}
  \begin{split}
    \wh{f_k}(n, \tau)
    & = \wh{\vp}(n) \int_\rr e^{-it( \tau - n^{2})} 
    \psi(t) \ d t
    = \wh{\psi}(\tau - n^{2}) \wh{\vp_k}(n).
  \end{split}
\end{equation*}
%
%
Since $\wh{\psi}(\xi)$ is Schwartz for $|\xi| \ge T$, we see that 
%
%
\begin{equation}
  \begin{split}
  \label{main-int1-est}
    \|\eqref{main-int-expression-1}\|_{Y^s}
    & = \left (  \sum_{n\in \zz} \left (1 + |n| \right )^s \int_\rr \left( 1 + | \tau - n^{2} 
    | \right )
    | \wh{\psi}(\tau - n^{2}) \wh{\vp_k}(n) |^2 d \tau \right)^{1/2} 
    \\
    & + \left[ \sum_{n \in \zz }\left( 1 + | n | \right)^{2s} \left( \int_{\rr} |
    \wh{\psi}(\tau - n^{2})\wh{\vp_k}(n) | d \tau
    \right)^{2} \right]^{1/2}
    \\
    & \le c_{\psi}
    \|\vp_k\|_{H^s(\ci)}.
  \end{split}
\end{equation}
%
%
%
%
\subsection{Estimate for \eqref{main-int-expression-2}.}
We now need the following lemma, whose proof is provided in the appendix.
%
%
%%%%%%%%%%%%%%%%%%%%%%%%%%%%%%%%%%%%%%%%%%%%%%%%%%%%%
%
%
%			Schwartz Multiplier	
%
%
%%%%%%%%%%%%%%%%%%%%%%%%%%%%%%%%%%%%%%%%%%%%%%%%%%%%%
%
%
\begin{lemma}
\label{lem:schwartz-mult}
  For $\psi \in S(\rr)$,
%
%
\begin{gather}
  \label{schwartz-mult-piece-1}
    \|\psi f \|_{X^s} \le c_{\psi} \|f \|_{X^s},
    \\
    \label{schwartz-mult-piece-2}
    \|\psi f \|_{E} \le c_{\psi} \|f \|_{E},
  \end{gather}
  and hence
  \begin{gather}
    \label{schwartz-mult}
\|\psi f \|_{Y^s} \le c_{\psi} \|f \|_{Y^s}.
\end{gather}
%
%
\end{lemma}
%
%
Hence,
%
%
\begin{equation}
  \label{main-int2-est-X-s-part}
  \begin{split}
    \|\eqref{main-int-expression-2}\|_{X^s} 
    & \lesssim 
    \left( \| \sum_{n \in \zz} e^{ixn} \int_\rr 
    e^{it \tau} \frac{ 1 - \psi (\tau - n^{2} ) 
    }{\tau - n^{2}} \wh{B_k}(n, \tau) \ 
    d \tau\|_{X^s} \right)^{1/2}
    \\
    & =  \left( \sum_{n \in \zz} \left (1 + |n| \right )^{2s} \int_\rr
    (1 + |\tau - n^{2}|) \left | \frac{1 - \psi(\tau - n^{2 
    })}{\tau - n^{2}} 
    \wh{B_k}(n, \tau) \right |^2 \ d 
    \tau \right)^{1/2}
    \\
    & \le \left( \sum_{n \in \zz} \left (1 + |n| \right )^{2s} \int_{| \tau - n^{2}| \ge 1}
    (1 + |\tau - n^{2}|) \frac{|\wh{B_k}(n, \tau)|^2 }{|\tau - n^{2}|^2} 
    \ d 
    \tau \right)^{1/2}
    \\
    & \lesssim  \left( \sum_{n \in 
    \zz} \left (1 + |n| \right )^{2s} \int_\rr
    \frac{|\wh{B_k}(n, \tau) |^2}{1+ |\tau - 
    n^{2}|} 
     \ d \tau 
    \right)^{1/2}
    \\
    & \lesssim 
    \sum_{j=0}^{k} \sum_{i=0}^{j} {k \choose j}{j \choose i}
    \|u_{k-j}\|_{X^s} \| u_{j-i}\|_{X^s}
    \| u_i \|_{X^s}
  \end{split}
\end{equation}
%
%
where the last two steps follow from the inequality 
%
\begin{equation}
  \label{one-plus-ineq}
  \begin{split}
    \frac{1}{|\tau - n^{2}| } \le \frac{2}{1 + |\tau - n^{2}| }, 
    \qquad |\tau - n^{2}| \ge 1
  \end{split}
\end{equation}
%
%
and the following trilinear estimate, whose proof we leave for later.
%
%
%%%%%%%%%%%%%%%%%%%%%%%%%%%%%%%%%%%%%%%%%%%%%%%%%%%%%
%
%
%				Proposition
%
%
%%%%%%%%%%%%%%%%%%%%%%%%%%%%%%%%%%%%%%%%%%%%%%%%%%%%%
%
%
\begin{proposition}
\label{prop:trilinear-est}
  %
  %
  For any $s \ge 0$ and $b \ge 3/4$, we have
  \begin{equation}
    \begin{split}
    & \left( \sum_{n \in \zz} \left (1 + |n| \right )^{2s} \int_\rr
    \frac{|\wh{{B_k}(f,g,h)}(n, \tau) |^2}{\left (1+ |\tau - 
    n^{2}| \right ) ^b} 
     \ d \tau 
    \right)^{1/2}
    \\
    & \lesssim \sum_{j=0}^{k} \sum_{i=0}^{j} {k \choose j}{j \choose i}
    \|f_{k-j}\|_{X^s} \| g_{j-i}\|_{X^s}
    \| h_i \|_{X^s}
  \end{split}
  \end{equation}
  where $$B_k(f,g,h)(x,t) = \sum_{j=0}^{k} \sum_{i=0}^{j} {k \choose j}{j \choose
  i} f_{k-j} g_{j-i} \bar{h_i}.$$
%
%
%
%
\end{proposition}
%
%
Furthermore,
%
%
%
%
\begin{equation}
  \label{main-int-expression-2-Y-s-part}
  \begin{split}
    \|\eqref{main-int-expression-2} \|_{E}
    & \lesssim \left( \| \sum_{n \in \zz} e^{ixn} \int_\rr 
    e^{it \tau} \frac{ 1 - \psi (\tau - n^{2} ) 
    }{\tau - n^{2}} \wh{{B_k}}(n, \tau) \ 
    d \tau\|_{E} \right)^{1/2}
    \\
    & = \left[ \sum_{n \in \zz}(1 + | n |)^{2s} \left(
    \int_{\rr}\frac{1 - \psi(\tau - n^{2})}{\tau - n^{2}} \wh{{B_k}}(n, \tau) d
    \tau \right)^{2} \right]^{1/2}
    \\
    & \lesssim \sum_{j=0}^{k} \sum_{i=0}^{j} {k \choose j}{j \choose i}
    \|u_{k-j}\|_{X^s} \| u_{j-i}\|_{X^s}
    \| u_i \|_{X^s}
  \end{split}
\end{equation}
%
%
where the last step follows from the following corollary to the preceding trilinear
estimate.
%
%
%%%%%%%%%%%%%%%%%%%%%%%%%%%%%%%%%%%%%%%%%%%%%%%%%%%%%
%
%
%				Second trilinear Estimate 
%
%
%%%%%%%%%%%%%%%%%%%%%%%%%%%%%%%%%%%%%%%%%%%%%%%%%%%%%
%
%
\begin{corollary}
\label{cor:trilinear-estimate2}
  For $s \ge 0$ we have
%
%
\begin{equation}
  \label{trilinear-estimate2}
  \begin{split}
    & \left( \sum_{n \in \zz} \left (1 + |n| \right )^{2s}  \left ( \int_\rr 
    \frac{|\wh{{B_k}(f,g,h)}(n, \tau) |}{1 + | \tau - n^{2} |}
     \ d\tau \right)^2  \right)^{1/2} 
     \\
     & \lesssim \sum_{j=0}^{k} \sum_{i=0}^{j} {k \choose j}{j \choose i}
    \|f_{k-j}\|_{X^s} \| g_{j-i}\|_{X^s}
    \| h_i \|_{X^s}.
  \end{split}
\end{equation}
\end{corollary}
%
%
Combining \eqref{main-int2-est-X-s-part} and
\eqref{main-int-expression-2-Y-s-part}, we conclude that
%
%
%
%
\begin{equation}
  \label{main-int2-est}
  \begin{split}
    \|\eqref{main-int-expression-2}\|_{Y^s} \le c_{\psi}
    \sum_{j=0}^{k} \sum_{i=0}^{j} {k \choose j}{j \choose i}
    \|u_{k-j}\|_{X^s} \| u_{j-i}\|_{X^s}
    \| u_i \|_{X^s}.
  \end{split}
\end{equation}
%
%
\subsection{Estimate for \eqref{main-int-expression-3}.}
Letting $$f_k(x,t) = \psi(t) \sum_{n \in \zz} e^{i\left( xn + tn^{2} \right)} 
\int_\rr \frac{1 - \psi\left( \lambda - n^{2} \right)}{\lambda - n^{2}} 
\wh{{B_k}} \left( n, \lambda \right) \ d \lambda,$$ we have
%
%
\begin{equation*}
  \begin{split}
    & \wh{f_k^x}(n, t) = \psi(t) e^{itn^{2}} \int_\rr
    \frac{1 - \psi\left( \lambda - n^{2} \right)}{\lambda - n^{2}} 
    \wh{{B_k}}(n, \lambda) \ d \lambda
  \end{split}
\end{equation*}
and
\begin{equation*}
  \begin{split}
     \wh{f_k}\left( n, \tau \right)
     & = \int_\rr e^{-it\left( \tau - n^{2} 
    \right)} \psi(t) \int_\rr \frac{1 - \psi\left( 
    \lambda - n^{2} 
    \right)}{\lambda - n^{2}} \wh{{B_k}}(n, \lambda) \ d \lambda d \tau
    \\
    & = \wh{\psi}\left( \tau - n^{2} \right) \int_\rr 
    \frac{1 - \psi\left( 
    \lambda - n^{2} 
    \right)}{\lambda - n^{2}} \wh{{B_k}}(n, \lambda) \ d \lambda.
  \end{split}
\end{equation*}
Therefore,
%
%
\begin{equation*}
  \begin{split}
    & \| \eqref{main-int-expression-3} \|_{X^s} 
    \\
    & = \left( \sum_{n \in \zz} \left (1 + |n| \right )^{2s} \int_\rr \left( 1 + | \tau - n^{m
    } \right ) | | \wh{\psi}\left( \tau - n^{2} \right) |^2 \ d \tau
    \right.
    \\
    & \times \left . |
    \int_\rr \frac{1 - \psi\left( \lambda - n^{2} \right)}{\lambda -
    n^{2}} \wh{{B_k}}(n, \lambda) \ d \lambda |^2  \right)^{1/2}
    \\
    & \lesssim \left( \sum_{n \in \zz} \left (1 + |n| \right )^{2s} | \int_\rr
    \frac{1 - \psi\left( \lambda - n^{2} \right)}{\lambda - n^{2}}
    \wh{{B_k}}(n, \lambda) \ d\lambda |^2 \right)^{1/2}
    \\
    & \le \left( \sum_{n \in \zz} \left (1 + |n| \right )^{2s}  \left ( \int_\rr
    \frac{1 - \psi\left( \lambda - n^{2} \right)}{|\lambda - n^{2}|}
    |\wh{{B_k}}(n, \lambda) | \ d\lambda \right )^2 \right)^{1/2}
    \\
    & \le \left( \sum_{n \in \zz} \left (1 + |n| \right )^{2s}  \left ( \int_{| \lambda - 
    n^{2} | \ge 1}
    \frac{|\wh{{B_k}}(n, \lambda) | }{|\lambda - n^{2}|}
    \ d\lambda \right )^2 \right)^{1/2}.
  \end{split}
\end{equation*}
%
%
Applying estimate \eqref{one-plus-ineq} then gives
%
%%
\begin{equation}
  \label{main-int3-est-X-s-part}
  \begin{split}
    \| \eqref{main-int-expression-3} \|_{X^s}
    & \lesssim \left( \sum_{n \in \zz} \left (1 + |n| \right )^{2s}  \left ( \int_\rr
    \frac{|\wh{{B_k}}(n, \lambda)| }{1 + |\lambda - n^{2}|}
     \ d\lambda \right )^2 \right)^{1/2}
     \\
    & \lesssim \sum_{j=0}^{k} \sum_{i=0}^{j} {k \choose j}{j \choose i}
    \|u_{k-j}\|_{X^s} \| u_{j-i}\|_{X^s}
    \| u_i \|_{X^s}
  \end{split}
\end{equation}
%
%%
where the last step follows from \autoref{cor:trilinear-estimate2}.
Furthermore, 
%
%
\begin{equation}
  \label{main-int-estimate-3-Y-s-part}
  \begin{split}
    \|\eqref{main-int-expression-3}\|_{E}
    & = \left[ \sum_{n \in \zz} (1 + | n |)^{2s} \int_{\rr} |
    \wh{\psi}(\tau - n^{2}) |^{2} \left( \int_{\rr}\frac{1 - \psi(\lambda -
    n^{2})}{\lambda - n^{2}} \wh{{B_k}}(n, \lambda) d \lambda \right)^{2} d \tau
    \right]^{1/2}
    \\
    & \le c_{\psi} \left[ \sum_{n \in \zz} (1 + | n |)^{2s} \left(
    \int_{\rr} \frac{1 - \psi(\lambda - n^{2})}{\lambda - n^{2}}
    \wh{{B_k}}(n, \lambda) d \lambda
    \right)^{2}\right]^{1/2}
    \\
    & \le 2 c_{\psi} \left[ \sum_{n \in \zz} (1 + | n |)^{2s} \left(
    \int_{\rr} \frac{\wh{{B_k}}(n, \lambda) }{1 + |\lambda - n^{2}|}
    d \lambda
    \right)^{2}\right]^{1/2}
    \\
    & \lesssim 
    \sum_{j=0}^{k} \sum_{i=0}^{j} {k \choose j}{j \choose i}
    \|u_{k-j}\|_{X^s} \| u_{j-i}\|_{X^s}
    \| u_i \|_{X^s}
  \end{split}
\end{equation}
%
%
where the last two steps follow from \eqref{one-plus-ineq} and
\autoref{cor:trilinear-estimate2}, respectively. Combining
\eqref{main-int3-est-X-s-part} and \eqref{main-int-estimate-3-Y-s-part}, we
conclude that
%
%
\begin{equation}
  \label{main-int3-est}
  \begin{split}
    \|\eqref{main-int-expression-3}\|_{Y^s} 
    \lesssim \sum_{j=0}^{k} \sum_{i=0}^{j} {k \choose j}{j \choose i}
    \|u_{k-j}\|_{X^s} \| u_{j-i}\|_{X^s}
    \| u_i \|_{X^s}.
  \end{split}
\end{equation}
%
%
%
\subsection{Estimate for \eqref{main-int-expression-4}.}
Note that
%
%
\begin{equation}
  \label{1n}
  \begin{split}
    \eqref{main-int-expression-4} \simeq \sum_{k \ge 1}
    \frac{i^k}{k!}g_k(x,t)
  \end{split}
\end{equation}
%
%
where 
%
%
\begin{equation*}
  \begin{split}
    & g_k(x,t) = t^k \psi(t) \sum_{n \in \zz} e^{i\left( xn + tn^{2}
    \right)} h_k(n),
    \\
    & h_k(n) = \int_\rr \psi \left( \tau - n^{2} \right) \cdot \left(
    \tau - n^{2} \right)^{k -1} \wh{{B_k}}(n, \tau) \ d \tau.
  \end{split}
\end{equation*}
%
%
Hence
%
%
\begin{equation*}
  \begin{split}
    \wh{g_k^x}(n, t) = t^{k} \psi(t) e^{i t n^{2}} h_k(n)
  \end{split}
\end{equation*}
%
%
which gives
%
%
\begin{equation*}
  \begin{split}
    \wh{g_k}(n, \tau)
    & = h_k(n) \int_\rr e^{-it\left( \tau - n^{2} \right)}
    t^{k}\psi(t) \ dt
    \\
    & = h_k(n) \wh{t^{k}\psi(t)} \left( \tau - n^{2} \right).
  \end{split}
\end{equation*}
%
%
Applying this to \eqref{1n}, we obtain
%
%
\begin{equation}
  \label{2n}
  \begin{split}
    \|\eqref{main-int-expression-4}\|_{X^s} 
    & \simeq \left( \sum_{n \in \zz} \left (1 + |n| \right )^{2s} \int_\rr \left( 1 + | \tau -
    n^{2}
    |
    \right) | \wh{\sum_{k \ge 1} \frac{i^k}{k!}g_k(x,t)} |^2 \ d \tau
    \right)^{1/2}
    \\
    & \le \sum_{k \ge 1} \frac{1}{k!}\left( \sum_{n \in \zz} \left (1 + |n| \right )^{2s}
    \int_\rr \left( 1 + | \tau - n^{2} | \right) | \wh{g_k}(n, \tau) |^2 \
    d \tau \right)^{1/2}
    \\
    & = \sum_{k \ge 1} \frac{1}{k!} \left( \sum_{n \in \zz} \left (1 + |n| \right )^{2s}
    \int_\rr \left( 1 + | \tau - n^{2} | \right) | h_k(n) \wh{t^k
    \psi(t)} \left( \tau - n^{2} \right) |^2 \ d \tau \right)^{1/2}
    \\
    & = \sum_{k \ge 1} \frac{1}{k!} \left( \sum_{n \in \zz} \left (1 + |n| \right )^{2s} |
    h_k(n) |^2 \int_\rr \left( 1 + | \tau - n^{2} | \right) | \wh{t^k
    \psi(t)} \left( \tau - n^{2} \right) |^2 \ d \tau \right)^{1/2}.
  \end{split}
\end{equation}
%
%
Notice that for fixed $n$, the change of variable $\tau - n^{2} \to \tau'$
gives
%
%
\begin{equation}
  \label{3n}
  \begin{split}
    \int_\rr \left( 1 + | \tau - n^{2} | \right) | \wh{t^{k}
    \psi(t)}\left( \tau - n^{2} \right) |^2 \ d \tau
    & = \int_\rr \left( 1 + |\tau'| \right) | \wh{t^k \psi(t)}(\tau') |^2 \
    d \tau'
    \\
    & \le \int_\rr \left( 1 + |\tau'| \right)^2 | \wh{t^k \psi(t)}(\tau')
    |^2 \ d \tau'
    \\
    & \lesssim \int_\rr \left( 1 + | \tau' |^2 \right) | \wh{t^{k}
    \psi(t)}(\tau') |^2 \ d \tau'
    \\
    & = \|t^k \psi(t) \|_{H^1(\rr)}^2.
  \end{split}
\end{equation}
%
%
But
%
%
\begin{equation}
  \label{4n}
  \begin{split}
    \|t^k \psi(t) \|_{H^1(\rr)}^2
    & = \left( \|t^k \psi(t)\|_{L^2(\rr)} + \|\p_t \left( t^k \psi(t)
    \right)\|_{L^2(\rr)} \right)^2
    \\
    & \lesssim \|t^{k}\psi(t) \|_{L^2(\rr)}^2 + \|\p_t \left (t^{k}
    \psi(t) \right )\|_{L^2(\rr)}^2
    \\
    & \le \|t^k \psi(t) \|_{L^2(\rr)}^2 + \|t^k \p_t \psi(t)
    \|_{L^2(\rr)}^2 + \|k t^{k -1} \psi(t) \|_{L^2(\rr)}^2
    \\
    & = c_{\psi} + c_{\psi}' + k^2 c_{\psi}''
    \\
    & \lesssim k^2.
  \end{split}
\end{equation}
%
%
Hence, applying \eqref{3n} and \eqref{4n} to \eqref{2n}, we obtain
%
%%
\begin{equation}
  \label{5n}
  \begin{split}
    \|\eqref{main-int-expression-4} \|_{X^s}
    & \lesssim
    \sum_{k \ge 1} \frac{k}{k!} \left( \sum_{n \in \zz} \left (1 + |n| \right )^{2s} | h_k(n) |^2 
    \right)^{1/2}
    \\
    & \le \sum_{k \ge 1} \frac{k}{k!}
    \cdot \sup_{k \ge 1} \left( \sum_{n \in \zz} \left (1 + |n| \right )^{2s} | 
    h_k(n) |^2 \right)^{1/2}
    \\
    & = \sum_{k \ge 1} \frac{k}{k!} \cdot \sup_{k \ge 1} 
    \left( \sum_{n \in \zz} \left (1 + |n| \right )^{2s} \int_\rr 
    \psi\left( \tau - n^{2} \right) \cdot \left( \tau - n^{2} 
    \right)^{k -1} \wh{{B_k}}(n, \tau) \ d \tau \right)^{1/2}.
  \end{split}
\end{equation}
%
%%
Recall that $\text{supp} \, |\psi| \subset [0, T ]$. Pick $T \le 1$. 
Then $| \psi\left( \tau - n^{2} \right) \cdot \left( \tau - n^{2} \right)^{k 
-1} | \le \chi_{| \tau - n^{2} | \le 1}$ for all $k \ge 1$. Hence, \eqref{5n} gives
%
%%
\begin{equation*}
  \begin{split}
    \|\eqref{main-int-expression-4} \|_{X^s} 
    & \lesssim \sum_{k \ge 1} \frac{k}{k!} \cdot \left( \sum_{n \in \zz} | 
    \int_{| \tau - n^{2}  |\le 1} | \wh{{B_k}}(n, \tau) \ d \tau |^2 
    \right)^{1/2}
  \end{split}
\end{equation*}
%
%%
which by the inequality
%
%%
\begin{equation*}
  \begin{split}
    \frac{1 + | \tau - n^{2} |}{1 + | \tau  - n^{2} |} \le 
    \frac{2}{1 + | \tau - n^{2} |}, \qquad | \tau - n^{2}  | \le 1
  \end{split}
\end{equation*}
%
%%
implies
%
%%
\begin{equation}
\label{main-int4-est-X-s-part}
  \begin{split}
    \|\eqref{main-int-expression-4}\|_{X^s}
    & \lesssim \left( \sum_{n \in \zz} | \int_{| \tau - n^{2}| \le 1 }
    \frac{\wh{{B_k}}(n, \tau)}{1 + | \tau - n^{2} |} \ d \tau |^2 
    \right)^{1/2}
    \\
    & \le \left( \sum_{n \in \zz} | \int_\rr
    \frac{\wh{{B_k}}(n, \tau)}{1 + | \tau - n^{2} |} \ d \tau |^2 
    \right)^{1/2} \\
    & \le \left( \sum_{n \in \zz} \left( \int_\rr 
    \frac{|\wh{{B_k}}(n, \tau)|}{1 + | \tau - n^{2} |}  \ d \tau  \right)^2
    \right)^{1/2} \\
    & \lesssim 
    \sum_{j=0}^{k} \sum_{i=0}^{j} {k \choose j}{j \choose i}
    \|u_{k-j}\|_{X^s} \| u_{j-i}\|_{X^s}
    \| u_i \|_{X^s}.
  \end{split}
\end{equation}
%
%%
where the last step follows from \autoref{cor:trilinear-estimate2}. Similarly,
we have
%
%
\begin{equation}
\label{main-int4-est-Y-s-part}
  \begin{split}
    \|\eqref{main-int-expression-4}\|_{E}
    & \simeq \left[ \sum_{n \in
    \zz}(1 + | n |)^{2s} \left( \int_{\rr} | \sum_{k \ge 1}
    \wh{\frac{i^{k}}{k!}g_{k}(x,t)(n, \tau)} |d \tau \right)^{2} \right]^{1/2}
    \\
    & \le \sum_{k \ge 1} \frac{1}{k!} \left[ \sum_{n \in \zz} (1 + | n
    |)^{2s} \left( \int_{\rr} | \wh{g}(n, \tau) | d \tau \right)^{2}
    \right]^{1/2}
    \\
    & = \sum_{k \ge 1} \frac{1}{k!} \left[ \sum_{n \in \zz} (1 + | n
    |)^{2s} | h_{k}(n) |^2 \left( \int_{\rr} | \wh{t^{k} \psi(t)}(\tau -
    n^{2}) |d \tau \right)^{2} \right]^{1/2}
    \\
    & = c_{\psi} \sum_{k \ge 1} \frac{1}{k!} \left[ \sum_{n \in \zz} (1 + | n
    |)^{2s} | h_{k}(n) |^2 \right]^{1/2}
    \\
    & \lesssim 
    \sum_{j=0}^{k} \sum_{i=0}^{j} {k \choose j}{j \choose i}
    \|u_{k-j}\|_{X^s} \| u_{j-i}\|_{X^s}
    \| u_i \|_{X^s}
  \end{split}
\end{equation}
%
%
where the last step follows from the computations starting from \eqref{5n}
through \eqref{main-int4-est-X-s-part}.
Combining \eqref{main-int4-est-X-s-part} and \eqref{main-int4-est-Y-s-part}, we
have
%
%
\begin{equation}
\label{main-int4-est}
  \begin{split}
    \|\eqref{main-int-expression-4}\|_{Y^s} \lesssim 
    \sum_{j=0}^{k} \sum_{i=0}^{j} {k \choose j}{j \choose i}
    \|u_{k-j}\|_{X^s} \| u_{j-i}\|_{X^s}
    \| u_i \|_{X^s}.
  \end{split}
\end{equation}
%
%
Collecting estimates \eqref{main-int1-est}, \eqref{main-int2-est}, 
\eqref{main-int3-est}, and \eqref{main-int4-est}, and recalling 
\eqref{main-int-expression-1}-\eqref{main-int-expression-4}, we see that
$$\|T_k(u_0, u_1, \cdots, u_k)\|_{Y^s} \le c_\psi \left( \|\vp_k \|_{H^s(\ci)} + 
\sum_{j=0}^{k} \sum_{i=0}^{j} {k \choose j}{j \choose i} \|u_{k-j}\|_{X^s} \|
u_{j-i}\|_{X^s} \| u_i \|_{X^s}  \right )$$ 
which by the inequality $\|u\|_{X^s} \le \|u\|_{Y^s}$ yields the following.
%%
%%%%%%%%%%%%%%%%%%%%%%%%%%%%%%%%%%%%%%%%%%%%%%%%%%%%%
%
%% Contraction Proposition
%				 
%%%%%%%%%%%%%%%%%%%%%%%%%%%%%%%%%%%%%%%%%%%%%%%%%%%%%%
%%
%%
%
\begin{proposition}
\label{prop:contraction}
  Let $s \ge0$. Then
%
%%
\begin{equation*}
  \begin{split}
    \|T_k(u_0, u_1, \cdots, u_k)\|_{Y^s} \le c_\psi  \left( \|\vp_k
    \|_{H^s(\ci)} +
    \sum_{j=0}^{k} \sum_{i=0}^{j} {k \choose j}{j \choose i}
    \|u_{k-j}\|_{Y^s} \| u_{j-i}\|_{Y^s}
    \| u_i \|_{Y^s}\right).
  \end{split}
\end{equation*}
%
%%
\end{proposition}
\subsection{Setting Up the Picard Iteration}
We will now use \autoref{prop:contraction} to prove local well-posedness for
analytic initial data for the NLS ivp. We first establish that the map 
%
%
\begin{equation*}
\begin{split}
  \left\{ u_k \right\}_{k=0}^{\infty} \mapsto T\left( \left\{ u_k
  \right\}_{k=0}^{\infty} \right) = \left( T_0(u_0), T_1(u_0, u_1), \cdots \right)
\end{split}
\end{equation*}
%
%
goes from $\mathcal{A}(Y^s)$ to $\mathcal{A}(Y^s)$, and that it is a contraction
in an appropriate ball.
%
Applying \autoref{prop:contraction}, we have the estimate
%
%
\begin{equation}
\begin{split}
  | | |T(\left\{ u_{k} \right\}) | | |
  & = \sum_{k=0}^{\infty} \frac{C_0^k}{k!} \|T_{k}(u_0, u_1, \cdots, u_k)
  \|_{Y^s}
  \\
  & \le c_{\psi} \sum_{k=0}^{\infty} \frac{C_0^k}{k!}\left[ \| \vp_{k}
  \|_{H^s(\ci)} + \sum_{j=0}^{k} \sum_{j=0}^{i} {k \choose j } {j \choose i }
  \|u_{k-j}\|_{Y^s} \| u_{j-1}\|_{Y^s} \|u_{i} \|_{Y^s} \| \right]
  \\
  & = c_{\psi}\left( | | | \left\{ \vp_{k} \right\} | | |_{s} +
  \sum_{k=0}^{\infty} \sum_{j=0}^{k}
  \sum_{j=0}^{i} {k \choose j } {j \choose i }
  \|u_{k-j}\|_{Y^s} \| u_{j-1}\|_{Y^s} \| u_{i} \|_{Y^s}\right).
\end{split}
\label{eqn:map-onto-ball}
\end{equation}
%
%
We now rewrite the second term
%
%
\begin{equation*}
\begin{split}
  & \sum_{k=0}^{\infty} \sum_{j=0}^{k}  \sum_{j=0}^{i} {k \choose j } {j \choose i } \|u_{k-j}\|_{Y^s}
\| u_{j-1}\|_{Y^s} \| u_{i} \|_{Y^s} 
  \\
  & = \sum_{k=0}^{\infty} \sum_{j=0}^{k}  \sum_{j=0}^{i} \frac{k!}{\cancel{j!}(k-j)!}
  \cdot \frac{\cancel{j!}}{i!(j-i)!} \| u_{k-j} \|_{Y^s} \| u_{j-i} \|_{Y^s} \|
  u_{i} \|_{Y^s}
  \\
  & = \sum_{k=0}^{\infty} \sum_{j=0}^{k}  \sum_{j=0}^{i}  k! \frac{\|
  u_{k-j} \|_{Y^s}}{(k-j)!} \frac{\| u_{j-i} \|_{Y^s}}{(j-1)!} \frac{\|
  u_{i} \|_{Y^s}}{i!}
\end{split}
\end{equation*}
%
%
which implies 
%
%
\begin{equation}
\begin{split}
  & \sum_{k=0}^{\infty} \frac{C_0^k}{k!} \sum_{j=0}^{k}  \sum_{j=0}^{i} {k \choose j } {j \choose i } \|u_{k-j}\|_{Y^s}
\| u_{j-1}\|_{Y^s} \| u_{i} \|_{Y^s} 
\\
& = \sum_{k=0}^{\infty} \sum_{j=0}^{k}  \sum_{j=0}^{i} C_{0}^{k} \frac{\|
u_{k-j} \|_{Y^s}}{(k-j)!} \frac{\| u_{j-i} \|_{Y^s}}{(j-i)} \frac{\|
u_{i} \|_{Y^s}}{i!}
\\
& = \sum_{k=0}^{\infty} \sum_{j=0}^{k}  \sum_{j=0}^{i}
\frac{C_{0}^{k-j}}{(k-j)!} \| u_{k-j} \|_{Y^s} \frac{C_{0}^{j-i}}{(j-i)!}\|
u_{j-i} \|_{Y^s} \frac{C_{0}^{i}}{i!} \| u_{i} \|_{Y^s}
\\
& \le \left( \sum_{k=0} \frac{C_0^k}{k!} \| u_{k} \|_{Y^s} \right)^{3}
= | | | \left\{ u_{k} \right\} | | |^3_{s}.
\label{eqn:low-dim-to-high-comp}
\end{split}
\end{equation}
%
%
Substituting into \eqref{eqn:map-onto-ball}, we obtain the relation
%
%
\begin{equation}
\begin{split}
  | | | T(\left\{ u_{k} \right\}) | | |
  \le c_{\psi} \left( | | | \left\{ \vp_{k} \right\} | | |_{s} +
  | | | \left\{ u_{k} \right\} | | |^3_{s} \right ).
\end{split}
\label{eqn:onto-relation}
\end{equation}
%
%
Let $c = c_{\psi}^{1/2}$. For given $\vp$, we may choose $\psi$ such
that 
%
%%
\begin{equation*}
  \begin{split}
    \|\{\vp\}\|_{s} \le \frac{15}{64c^3}.
  \end{split}
\end{equation*}
%
%%
Then if $| | |\{u_k\} | | | \le \frac{1}{4c}$, \eqref{eqn:onto-relation} gives
%
%%
\begin{equation*}
  \begin{split}
    |  | |T (\{u_k \}) | | |
    & \le c^2 \left[ \frac{15}{64c^3} + \left( 
    \frac{1}{4c} \right)^3 \right]
    =  \frac{1}{4c}.
  \end{split}
\end{equation*}
%
%%
Hence, $T=T_{\vp}$ maps the ball $B\left( 0, \frac{1}{4c} \right) \subset
\mathcal{A}(Y^s)$ into 
itself. Next, note that
%
%%
\begin{equation*}
  \begin{split}
    T_k(u_0, u_1,\cdots,u_k) - T_k(v_0, v_1, \cdots, v_k)
    = \eqref{main-int-expression-2} + \eqref{main-int-expression-3} 
    + \eqref{main-int-expression-4}
  \end{split}
\end{equation*}
%
%%
where now we replace $B_k(u,u,u)$ with $B_k(u,u,u) - B_k(v,v,v) = \p_x^k(u | u
|^2 - v | v |^{2})$. Rewriting
%
%%
\begin{equation*}
  \begin{split}
    u | u |^{2} - v | v |^{2}
    & = | u |^2 \left( u -v \right) + v\left( | u 
    |^2 - | v |^2
    \right)
    \\
    & = u \bar u \left( u -v \right) + v u \bar u - v v \bar v
    \\
    & = u \bar u \left( u - v \right) + v \bar u\left( u - v \right) + v 
    \bar u v - v v \bar v
    \\
    & = u \bar u \left( u -v \right) + v \bar u\left( u - v \right) + v v 
    \left( \overline{u -v} \right)
  \end{split}
\end{equation*}
%
%%
the triangle inequality and linearity of $\p_x^k$ and the Fourier transform then give
%
%%
\begin{equation*}
  \begin{split}
    | \mathcal{F}[{B_k}(u,u,u) - B_k(v,v,v)](n, \tau) |
    & \le | \mathcal{F}[\p_x^k[u \overline{u} \left (u -v \right )]] | +
    | \mathcal{F}[\p_x^k[v \overline{u} (u -v)]] | + |\mathcal{F}[\p_x^k[v v 
    (\overline{u-v})]|
    \\
    & \doteq | \wh{{B_k}(u-v, u, u} | + | \wh{{B_k}(u-v, u, v} | + |
    \wh{{B_k}(u-v,v,v)} |.
  \end{split}
\end{equation*}
%
%%
%
%%
Hence, $T_k(u_0, u_1,\cdots,u_k) - T_k(v_0, v_1, \cdots, v_k)
 = L_k(u, u-v,u) + L_k(v,u-v,u) + L_k(v, v, u-v)$, where
\begin{align}
  \label{main-int-exp-mod1}
  & \frac{1}{4 \pi^2} \psi(t) \sum_{n\in \zz} \int_\rr e^{ixn}  
    e^{it \tau} \frac{ 1 - \psi(\tau - n^{2}) 
    }{\tau - n^{2}} \wh{{B_k}(f,g,h)}(n, \tau) \ d \tau
    \\
    \label{main-int-exp-mod2}
    - & \frac{1}{4 \pi^2} \psi(t) \sum_{n\in \zz} \int_\rr e^{i(xn + 
    tn^{2})}
    \frac{1- \psi(\tau - n^{2})}{\tau - n^{2}} \wh{{B_k}(f,g,h)}(n, \tau) \ d \tau
    \\
    \label{main-int-exp-mod3}
    + & \frac{1}{4 \pi^2} \psi(t) \sum_{k \ge 1} \frac{i^k t^k}{k!}
    \sum_{n \in \zz} \int_\rr e^{i(xn + tn^{2} )}
    \psi(\tau - n^{2}) (\tau - n^{2})^{k-1} \wh{{B_k}(f,g,h)}(n, \tau)  
    \\
    \doteq & L_k(f,g,h). \notag
\end{align}
Repeating the arguments used to estimate 
\eqref{main-int-expression-2}-\eqref{main-int-expression-4}, we obtain
%
%%
\begin{equation*}
  \begin{split}
    & \|L_k(f, g, h)\|_{Y^s} \le c_\psi \sum_{j=0}^{k} \sum_{i=0}^{j} {k \choose
    j}{j \choose i} \|f_{k-j}\|_{Y^s} \| g_{j-i}\|_{Y^s} \| h_i \|_{Y^s}.
\end{split}
\end{equation*}
%
which implies
%
%%
\begin{equation}
  \label{20a}
  \begin{split}
    & | | |T_k(\left\{ u_k \right\}) - T_k(\left\{ v_k \right\}) | | |
    \\
    & = \sum_{k=0}^{\infty} \frac{C_0^k}{k!}\|T_k(u_0, u_1, \cdots, u_k) -
    T_k(v_0, v_1, \cdots, v_k) \|_{Y^s}
    \\
    & = \sum_{k=0}^{\infty} \frac{C_0^k}{k!}\left( \| L_k(u, u-v,u) +
    L_k(v,u-v,u) + L_k(v, v, u-v) \|_{Y^s} \right)
    \\
    & \le \sum_{k=0}^{\infty} \frac{C_0^k}{k!}\left( \| L_k(u, u-v,u) \|_{Y^s}
    + \|L_k(v,u-v,u)\|_{Y^s} +
    \|L_k(v, v, u-v) \|_{Y^s}\right)
    \\
    & \lesssim_{\psi} \sum_{k=0}^{\infty} \sum_{j=0}^{k}
    \sum_{i=0}^{j}\frac{C_{0}^{k}}{k!} {k \choose j } {j \choose i}
    \|u_{k-j} -v_{k-j} \|_{Y^s}
    \\
    & \times \left( \|u_{j-i}\|_{Y^s} \|u_{i}\|_{Y^s} +
    \|u_{j-i}\|_{Y^s} \| v_{i} \|_{Y^s} + \| v_{j-i} \|_{Y^s} \| v_i \|_{Y^s}  \right)
    %\\
    %& = c^2 \|u -v\|_{Y^s} \left( \|u\|_{Y^s} + \|v\|_{Y^s} \right)^2.
  \end{split}
\end{equation}
%
%%
Using a computation similar to \eqref{eqn:low-dim-to-high-comp}, we bound this
by
%
%
\begin{equation*}
\begin{split}
  & | | |\left\{ u_{k} \right\} - \left\{ v_{k} \right\}| | |_{s}\left( | | |\left\{
  u_{k} \right\} | | |_{s}^{2} + | | |\left\{ u_{k} \right\} | | |_{s} | |
  | \left\{ v_k \right\} | | |_{s} + | | |\{v_{k}\} | | |_{s}^{2}
  \right)
  \\
  & \le | | |\left\{ u_{k} \right\} - \left\{ v_{k} \right\}| | |_{s}
  \left( | | |\left\{
  u_{k} \right\} | | |_{s}+ | | |\{v_{k}\} | | |_{s} \right)^2.
\end{split}
\end{equation*}
%
%
Hence,  
%
%
\begin{equation*}
\begin{split}
  | | |T_k(\left\{ u_k \right\}) - T_k(\left\{ v_k \right\}) | | |
  \le c_{\psi} | | |\left\{ u_{k} \right\} - \left\{ v_{k} \right\}| | |_{s}
  \left( | | |\left\{
  u_{k} \right\} | | |_{s}+ | | |\{v_{k}\} | | |_{s} \right)^2.
\end{split}
\end{equation*}
%
%
Therefore, if $u, v \in B(0, \frac{1}{4c}) \subset \mathcal{A}(Y^s)$, it follows that
%
%%
\begin{equation}
  \label{21a}
  \begin{split}
    |  | |T\{u_k\} - T\{v_k\} |  | |
    & \le c^2 | | |u -v |  | |_{s} \left( \frac{1}{4c} + 
    \frac{1}{4c} \right)^2
    \\
    & = \frac{1}{4} \|u -v \|_{Y^s}. 
  \end{split}
\end{equation}
%
%%
We conclude that $T = T_{\vp}$ is a contraction on the ball $B(0, 
\frac{1}{4c}) \subset \mathcal{A}(Y^s)$. A Picard iteration, coupled with
\autoref{lem:cutoff-loc-soln} then yields a unique, local
solution in $\mathcal{A}(Y^s)$ to the NLS ivp
\eqref{eqn:nls}-\eqref{eqn:nls-data}. This completes the proof. \qquad
\qedsymbol
%
%
%
%
\section{Proof of \autoref{prop:trilinear-est}.}
%
%
By the triangle inequality, it is enough to show that for any $s \ge 0$ and $b
\ge 3/4$, we have
  \begin{equation}
    \label{trilin-est-simp}
    \begin{split}
    & \left( \sum_{n \in \zz} \left (1 + |n| \right )^{2s} \int_\rr
    \frac{|\wh{{w}_{fgh}}(n, \tau) |^2}{\left (1+ |\tau - 
    n^{2}| \right ) ^b} 
     \ d \tau 
    \right)^{1/2}
    \lesssim \|f\|_{X^s} \| g\|_{X^s}
    \| h \|_{X^s}
  \end{split}
  \end{equation}
  where $w_{fgh}(x,t) = f g \bar{h}$.
%
%
%
%
Note first that $|\wh{{w}_{fgh}}(n, \tau) |  = | \wh{f} * ( \wh{g} 
* \wh{\bar h})(n, \tau)|$ and $| \wh{\bar{h}}(n, \tau) | = |\overline{ \wh{\overline{h}} 
}(n, \tau)| = | \wh{h}(-n, -\tau) |$. It follows that
%
%
\begin{equation}
  \label{non-lin-rep}
  \begin{split}
    | \wh{{w}_{fgh}}(n, \tau)|
    & = | \sum_{n_1 + n_2 + n_3 = n}  \int_{\tau_1 + \tau_2 + \tau_3 = \tau} \wh{f}\left( n_1,  \tau_1 
\right) \wh{g}\left( n_2, \tau_2  
\right) \wh{\bar h}\left( n_3, \tau_3 \right) d \tau_1 d \tau_2 d \tau_3 |
\\
& \le \sum_{n_1 + n_2 + n_3 = n}  \int_{\tau_1 + \tau_2 + \tau_3 = \tau} | \wh{f}\left( n_1, \tau_1 
\right) | \times  | \wh{g}\left( n_2, \tau_2 
\right) | \times | \wh{\bar h}\left( n_3, \tau_3 \right) | d \tau_1 d \tau_2 d 
\tau_3
\\
& \le \sum_{n_1 + n_2 + n_3 = n}  \int_{\tau_1 + \tau_2 + \tau_3 = \tau} | \wh{f}\left( n_1, \tau_1 
\right) | \times | \wh{g}\left( n_2, \tau_2 
\right) | \times | \wh{h}\left( -n_3, - \tau_3 \right) | d \tau_1 d \tau_2 d 
\tau_3
\\
& = \sum_{n_1 + n_2 + n_3 = n} \int_{\tau_1 + \tau_2 + \tau_3 = \tau} \frac{c_f\left( n_1, \tau_1 
\right)}{\left (1 + |n_1| \right )^s \left( 1 + | \tau_1 - n_1^2 | \right)^{b/2}}
\\
& \times \frac{c_{g}\left( n_2, \tau_2 \right)}{\left (1 + |n_2| \right ) 
^s\left( 1 + | \tau_2 -  n_2^2| 
\right)^{b/2}}
 \times \frac{c_{h}\left( -n_3, -\tau_3 \right)}{\left (1 + |n_3| \right ) ^s\left( 1 + | 
\tau_3 + n_3^2 | \right)^{b/2}} \ d \tau_1 d \tau_2 d \tau_3
\end{split}
\end{equation}
%
%
where 
%
%
\begin{equation*}
  \begin{split}
    c_\sigma(n, \tau) = \left (1 + |n| \right ) ^s \left( 1 + | \tau - n^{2} |  
    \right)^{b/2} | \wh{\sigma}\left( n, \tau \right) | .
  \end{split}
\end{equation*}
%
%
Hence
%
%
\begin{equation}
  \label{convo-est-starting-pnt}
  \begin{split}
     & \left (1 + |n| \right )^s \left( 1 + | \tau - n^{2} | \right)^{-b/2} | \wh{{w}_{fgh}}\left( 
    n, \tau \right) |
    \\
    & \le \left( 1 + | \tau - n^{2} | \right)^{-b/2}
    \sum_{n_1 + n_2 + n_3 = n} \int_{\tau_1 + \tau_2 + \tau_3 = \tau} \frac{\left (1 + |n| \right )^s}{\left (1 +
    |n_1| \right )^s \left (1 + | n_2| \right )^s \left (1 + |n_3| \right )^s} 
    \\
    & \times \frac{c_f(n_1, \tau_1)}{\left( 1 + | \tau_1 - n_1^2 | 
    \right)^{b/2}}
    \times
    \frac{c_g(n_2, \tau_2)}{\left( 1 + | \tau_2 - n_2^2 | 
    \right)^{b/2}} \times
    \frac{c_h(-n_3, -\tau_3)}{\left( 1 + | \tau_3 + n_3^2 | 
    \right)^{b/2}}\ d \tau_1 d \tau_2 d \tau_3.
  \end{split}
\end{equation}
%
%
For $s \ge 0$, observe that
%
%
\begin{equation}
  \label{deriv-bound-easy-s}
  \begin{split}
    \frac{\left (1 + |n| \right ) ^s}{\left (1 + |n_1| \right ) ^s \left (1 + |n_2| \right ) ^s \left (1 + |n_3| \right ) ^s} 
    \le 3^{s}
  \end{split}
\end{equation}
%
%
by the following lemma, whose proof is provided in the appendix.
%
%
\begin{lemma}
\label{lem:splitting}
  For $v \ge 0$ and $a, b, c \in \zz$, we have
%
%
\begin{equation}
  \label{splitting}
  \begin{split}
    \left ( 1 + |a +b + c| \right)^v \le 3^v \left(1 + | a | \right)^v \left(
    1 + | b | \right)^v \left( 1 + | c | \right)^v.
  \end{split}
\end{equation}
%
%
\end{lemma}
%
%
Hence, from \eqref{convo-est-starting-pnt} and \eqref{deriv-bound-easy-s}, we 
obtain
%
\begin{equation*}
  \begin{split}
    \left (1 + |n| \right )^s \left( 1 +  | \tau - n^{2}  | \right)^{-b/2} | 
    \wh{{w}_{fgh}}\left( n, \tau \right) | 
    & \lesssim \sum_{n_1 + n_2 + n_3 = n} \int_{\tau_1 + \tau_2 + \tau_3 = \tau} \frac{1}{\left( 1 +
    | \tau - n^{2}| 
    \right)^{b/2}}  
    \\
    & \times
    \sum_{n_1 + n_2 + n_3 = n} \int_{\tau_1 + \tau_2 + \tau_3 = \tau} \frac{c_f\left( n_1, \tau_1 
    \right)}{\left (1 + |n_1| \right )^s \left( 1 + | \tau_1 - n_1^2 |
    \right)^{b/2}}
    \\
    & \times \frac{c_{g}\left( n_2, \tau_2 \right)}{\left (1 + |n_2| \right ) 
    ^s\left( 1 + | \tau_2 -  n_2^2| 
    \right)^{b/2}}
    \\
    & \times \frac{c_{h}\left( -n_3, -\tau_3 \right)}{\left (1 + |n_3| \right ) ^s\left( 1 + | 
    \tau_3 + n_3^2 | \right)^{b/2}} \ d \tau_1 d \tau_2 d \tau_3
    \\
    & = \left( 1 + | \tau - n^{2} | \right)^{-b/2}
    \wh{C_f C_{g} C^+_{h}} \left( n, \tau \right)
  \end{split}
\end{equation*}
%
%
where
%
%
\begin{equation*}
  \begin{split}
    C_\sigma(x, t) = \left[ \frac{c_\sigma\left( n, \tau \right)}{\left( 
    1 + | \tau - n^{2} | \right)^{b/2}} \right]^\vee,
    \ \ C^+_\sigma(x, t) = \left[ \frac{c_\sigma\left( -n, -\tau \right)}{\left( 
    1 + | \tau + n^{2} | \right)^{b/2}} \right]^\vee.
  \end{split}
\end{equation*}
%
%
Therefore
%
%
\begin{equation}
  \label{gen-holder-pre-estimate}
  \begin{split}
    & \| \left( 1 + |n | \right)^s
    \left( 1 + | \tau - n^{2} | \right)^{-b/2} \wh{{w}_{fgh}}(n, 
    \tau)		
    \|_{L^2(\zz \times \rr)}
    \\
    & \lesssim \| \left( 1 + | \tau - n^{2} | \right)^{-b/2}
    \wh{C_f C_{g} C^+_{h}} \|_{L^2(\zz \times \rr)}.
  \end{split}
\end{equation}
%
We now require the following multiplier estimate, whose proof can be found in 
\cite{Himonas-Misiolek-2001-A-priori-estimates-for-Schrodinger}.
%
%
%%%%%%%%%%%%%%%%%%%%%%%%%%%%%%%%%%%%%%%%%%%%%%%%%%%%%
%
%
%			Four Mult Est	
%
%
%%%%%%%%%%%%%%%%%%%%%%%%%%%%%%%%%%%%%%%%%%%%%%%%%%%%%
%
%
%
%
%
%
%
%
\begin{lemma}
  \label{lem:four-mult-est-L4}
  Let $(x, t) \in \ci \times \rr $ and $(n, \tau) \in \zz \times \rr$ be 
  the dual variables. Let $v$ be a positive even integer. Then there is a 
  constant $c_v > 0$ such that
%
%
\begin{equation}
  \label{four-mult-est-L4*}
  \begin{split}
    \| \left( 1 + | \tau - n^v | 
    \right)^{-\frac{v+1}{4v}}
    \wh{f}\|_{L^2(\zz \times \rr)} \le c_v \|f \|_{L^{4/3}( \ci \times \rr)}.
  \end{split}
\end{equation}
%
%
\end{lemma}
%
%
Applying \autoref{cor:four-mult-est-L4} and generalized H\"{o}lder to the 
right-hand-side of \eqref{gen-holder-pre-estimate} gives
%
%
\begin{equation}
  \label{gen-holder-piece-1}
  \begin{split}
    \|\left( 1 + | \tau - n^{2} | \right)^{-b/2} \wh{C_f C_{ 
    g } C^+_{h}}\|_{L^2(\zz \times \rr)}
    & \lesssim  \|C_f C_{g} C^+_{h} \|_{L^{4/3}(\ci \times \rr)}
    \\
    & \le \|C_f \|_{L^4(\ci \times \rr)} \|C_{g}\|_{L^4(\ci \times \rr)} 
    \|C^+_{h}\|_{L^4(\ci \times \rr)}.
  \end{split}
\end{equation}
%
%
Note that a change of variable gives
%
%
\begin{equation*}
  \begin{split}
    C_\sigma^+(x, t)
    & = \sum_{n \in \zz} \int_\rr e^{i(nx +  \tau t)} \frac{c_\sigma\left( -n, -\tau \right)}{\left( 
    1 + | \tau + n^{2} | \right)^{b/2}} \ d \tau
    \\
    & = - \sum_{n \in \zz} \int_\rr e^{-i(nx +   \tau t )}
    \frac{c_\sigma\left( n, \tau \right)}{\left( 
    1 + | \tau - n^{2} | \right)^{b/2}} \ d \tau
  \end{split}
\end{equation*}
%
%
and so
%
%
\begin{equation*}
  \begin{split}
    C_\sigma^+(-x, -t) = -C_\sigma(x, t).
  \end{split}
\end{equation*}
%
%
We will now the need the following dual estimate of
\autoref{lem:four-mult-est-L4}.
%
\begin{corollary}
  \label{cor:four-mult-est-L4}
  Let $(x, t) \in \ci \times \rr $ and $(n, \tau) \in \zz \times \rr$ be 
  the dual variables. Let $v$ be a positive even integer. Then there is a 
  constant $c_v > 0$ such that
%
%
\begin{equation}
  \label{four-mult-est-L4}
  \begin{split}
    \|f\|_{L^4(\ci \times \rr)} \le c_v \|\left( 1 + | \tau - n^v | 
    \right)^\frac{v+1}{4v} \wh{f} \|_{L^2( \zz \times \rr)}
  \end{split}
\end{equation}
for every test function $f(x, t)$. 
%
%
%
%
\end{corollary}
%
%
Recalling that $L^4(\ci \times \rr)$ is invariant under the transformation $(x, 
t) \mapsto (-x,-t)$ and applying 
\autoref{cor:four-mult-est-L4}, we obtain
%
%
\begin{equation}
  \label{C-sig-estimate}
  \begin{split}
    \| C^+_\sigma \|_{L^4(\ci \times \rr)} = \|C_\sigma \|_{L^4(\ci \times \rr)} 
    & \lesssim \|\left( 1 + | \tau - n^{2} | 
    \right)^{(m +1)/4m} \left( 1 + | \tau - n^{2} | 
    \right)^{-b/2} c_\sigma \|_{L^2(\zz \times \rr)}
    \\
    & = \|\left( 1 + | \tau - n^{2} | 
    \right)^{[m(1 - 2b) + 1]/4m } c_\sigma \|_{L^2(\zz \times \rr)}
    \\
    & \le \|c_\sigma \|_{L^2(\zz \times \rr)}  \qquad (\text{since  } [m(1 - 2b) + 
    1]/4m \le 0 )
    \\
    & = \|\sigma\|_{X^s}.
  \end{split}
\end{equation}
%
%
We conclude from \eqref{gen-holder-pre-estimate}, \eqref{gen-holder-piece-1}, 
and \eqref{C-sig-estimate} that
%
%
%
%
\begin{equation*}
  \begin{split}
    \| \left( 1 + |n | \right)^s \left( 1 + | \tau - n^{2} | \right)^{-b/2} \wh{{w}_{fgh}} 
    (n, \tau) \|_{L^2(\zz \times \rr)} \lesssim 
    \|f\|_{X^s}\|g\|_{X^s}\|h\|_{X^s}. \qquad \qed
  \end{split}
\end{equation*}
%
%
%
%
%
%
%
\section{Proof of \autoref{cor:trilinear-estimate2}.}
By the triangle inequality, it is enough to show that
for $s \ge 0$ we have
%
%
\begin{equation}
  \begin{split}
    & \left( \sum_{n \in \zz} \left (1 + |n| \right )^{2s}  \left ( \int_\rr 
    \frac{|\wh{{w}_{fgh}}(n, \tau) |}{1 + | \tau - n^{2} |}
     \ d\tau \right)^2  \right)^{1/2} 
     \lesssim \|f\|_{X^s} \| g\|_{X^s}
    \| h \|_{X^s}.
  \end{split}
\end{equation}
%
Hence, by duality, it suffices to show that 
%
%%
\begin{equation*}
  \begin{split}
    \sum_{n \in \zz} \left (1 + |n| \right )^{s}
    a_n \int_{\rr} \frac{|\wh{{w}_{fgh}}(n, \tau)|}{1 
    + | \tau - n^{2} |} \ d \tau \lesssim \|f\|_{X^s} \|g\|_{X^s} \|h\|_{X^s}
    \|a_n \|_{\ell^2}
  \end{split}
\end{equation*}
%
%%
for $\{a_n\} \in \ell^2$. By the triangle inequality 
and Cauchy-Schwartz,
%
%%
\begin{equation}
  \label{1m}
  \begin{split}
    & | \sum_{n \in \zz} \left (1 + |n| \right )^{s} a_n
    \int_{\rr}\frac{| \wh{{w}_{fgh}}(n, \tau) |}{1 + | \tau - n^{2} |} \ d \tau |
    \\
    & \le \sum_{n \in \zz} \int_{\rr} \frac{| a_n |}{\left( 1 + 
    | \tau - n^{2} |
    \right)^{1/2 + \eta}} \cdot \frac{\left( 1 + | n| \right)^s  |
    \wh{{w}_{fgh}}(n, \tau) |}{\left( 
    1 + | \tau - n^{2} | \right)^{1/2 - \eta}} \ d \tau
    \\
    & \le \left( \sum_{n \in \zz} | a_{n} |^2\int_{\rr} \frac{1}{\left( 1 + | \tau - n^{2} | \right)^{1 + 2 \eta}} \ d \tau  
    \right)^{1/2} 
    \left ( \sum_{n \in \zz}\int_{\rr} \frac{\left (1 + |n| \right )^{2s} | \wh{{w}_{fgh}}(n, \tau) 
    |^2}{\left( 1 + | \tau - n^{2} | \right)^{1 - 2 \eta}}\ d \tau 
    \right)^{1/2}
  \end{split}
\end{equation}
%
%%
Restrict $\eta \in (0, 1/8)$. Applying the change of variable $\tau - n^{2}
\mapsto \tau'$ we obtain  %
%%

\begin{equation*}
  \begin{split}
    & \left( \sum_{n \in \zz} | a_{n} |^2\int_{\rr} \frac{1}{\left( 1 + | \tau -
    n^{2} | \right)^{1 + 2 \eta}} \ d \tau  
    \right)^{1/2} 
    \\
    & = \left ( \sum_{n \in \zz}
    | a_n |^2 
    \int_{\rr} \frac{1}{\left( 1 + | \tau' | \right)^{1 + 2 \eta}} \ d 
    \tau \right)^{1/2}
    \\
    & \simeq \|a_n\|_{\ell^2}
    \end{split}
\end{equation*}
while \eqref{trilin-est-simp} gives the bound
\begin{equation*}
  \begin{split}
    \left ( \sum_{n \in \zz}\int_{\rr} \frac{\left (1 + |n| \right )^{2s} | \wh{{w}_{fgh}}(n, \tau) 
    |^2}{\left( 1 + | \tau - n^{2} | \right)^{1 - 2 \eta}}\ d \tau 
    \right)^{1/2} \lesssim \|f\|_{X^s} \|g\|_{X^s} \|h\|_{X^s}
  \end{split}
\end{equation*}
%
%%
completing the proof.
\qquad \qedsymbol
%





%%%%%%%%%%%%%%%%%%%%%%%%%%%%%%
%
%
%
%
%                     $G^2$  regularity in time
%
%
%
%
%
%%%%%%%%%%%%%%%%%%%%%%%%%%%%%



%
\section{$G^2$  regularity in time} 
\label{g2-rreg-3}
%
%
Here we shall  follow the work of  Hannah, Himonas and Petronilho
\cite{Hannah:2006uq}
to move the analyticity in space to $G^2$ regularity in time.
We have 
%
%
\begin{gather}
  \label{eqn:nls-t}
  \partial_tu=i\partial_x^2u+i\lambda |u|^2u,
  \\
  \label{eqn:nls-t-data}
  u(x,0)=\varphi(x) \in \mathcal{C}^\omega(\mathbb{T}),
\end{gather}
%
where $\lambda = \pm 1$. As we
shall see, the value of $\lambda$ does not influence the proof. Therefore, we
set $\lambda =1$ without loss of generality. 








%%%%%%%%%%%%%%%%%%%%%%%%%%%%%%
%
%
%
%
%                     $G^2$  regularity in time
%
%
%
%
%
%%%%%%%%%%%%%%%%%%%%%%%%%%%%%



%
\section{Construction of non-analytic in time solutions} 
\label{g2-rreg}
%
%
Here we would like to find an analog to the construction
 of non-analytic in time solutions for KdV (or gKdV)
 (see  Gorsky-Himonas \cite{Gorsky:2005fk} and Byers-Himonas
 \cite{Byers-2006-Existence-time-for-the-Camassa-Holm})
 for NLS. 
 
 








%%%%%%%%%%%%%%%%%%%%%%%%%%%%%%
%
%
%
%
%                     Appendix
%
%
%
%
%
%%%%%%%%%%%%%%%%%%%%%%%%%%%%%





%
\appendix
\section{}
\subsection{Proof of \autoref{lem:cutoff-loc-soln}}
%
%
\begin{equation*}
  \begin{split}
    \lim_{t_{n} \to t} \|u(\cdot, t) - u(\cdot, t_{n})\|_{H^s(\ci)} 
    & = \lim_{t_{n} \to t} \|\psi(t) u(\cdot, t) - \psi(t_n) u(\cdot, t_{n})\|_{H^s(\ci)} 
    \\
    & = \lim_{t_n \to t} \left[ \sum_{n \in \zz}\left( 1 + | n |
    \right)^{2s} | \psi(t)  \wh{u}(n, t) - \psi(t_n) \wh{ u}(n, t_n) |^2 \right]^{1/2}
    \\
    & = \lim_{t_n \to t} \left[ \sum_{n \in \zz} \left( 1 + | n |
    \right)^{2s} | \int_{\rr} (e^{it \tau} - e^{it_{n} \tau}) \wh{\psi u}(n,
    \tau) d \tau |^2 \right]^{1/2}.
  \end{split}
\end{equation*}
    It is clear that
    %
    %
    \begin{equation*}
      \begin{split}
        \left( 1 + | n |
        \right)^{2s} | \int_{\rr} (e^{it \tau} - e^{it_{n}\tau}) \wh{\psi u}(n, \tau) d \tau |^2 
    & \le 4  \left( 1 + | n |
    \right)^{2s} \left ( \int_{\rr} |\wh{\psi u}(n, \tau)| d \tau
    \right )^2 
  \end{split}
\end{equation*}
and 
%
%
\begin{equation*}
  \begin{split}
 \sum_{n \in \zz} \left( 1 + | n |
    \right)^{2s} \left ( \int_{\rr} |\wh{\psi u}(n, \tau)| d \tau
    \right ) ^2 
    \le \|\psi u \|_{Y^s}^2 
  \end{split}
\end{equation*}
which is bounded by assumption.
Applying dominated convergence completes the proof. \qquad \qedsymbol
%
%
\subsection{Proof of \autoref{lem:schwartz-mult}}
Note that
%
%
\begin{equation*}
	\begin{split}
		\wh{\psi f}\left( n, \tau \right)
		& = \wh{\psi}(\cdot) * \wh{f}(n,
		\cdot)(\tau)
		= \int_\rr \wh{\psi}(\tau_1) \wh{f} \left( n, \tau - \tau_1 \right) 
		d\tau_1
	\end{split}
\end{equation*}
%
%
and hence
%
%
\begin{equation}
	\label{1b}
	\begin{split}
		\|\psi f\|_{X^s} 
		& = \left( \sum_{n \in \zz} \left (1 + |n| \right )^{2s} \int_\rr \left( 1 + | \tau -
		n^{m} | \right) | \int_\rr \wh{\psi}(\tau_1) \wh{f}\left( n, \tau -
		\tau_1
		\right)  d \tau_1 d \tau |^2 \right)^{1/2}
		\\
		& \le \left( \sum_{n \in \zz} \left (1 + |n| \right )^{2s} \int_\rr \left( 1 + | \tau -
		n^{m }
		|
		\right) \left( \int_\rr \wh{\psi}\left( \tau_1 \right) \wh{f}\left( n,
		\tau - \tau_1
		\right)  d \tau_1 d \tau \right)^2 \right)^{1/2}.
	\end{split}
\end{equation}
%
%
Using the relation
%
%
\begin{equation*}
	\begin{split}
		1 + | \tau - n^{m } |
		& = 1 + | \tau + \tau_1 - n^{m} |
		\\
		& \le 1 + | \tau_1 | + | \tau - \tau_1 - n^{m} |
		\\
		& \le \left( 1 + | \tau_1 | \right)\left( 1 + | \tau - \tau_1 -
		n^{m} | \right),
	\end{split}
\end{equation*}
%
%
we obtain
%
%
\begin{equation*}
	\begin{split}
		\eqref{1b}
		& \le \left( \sum_{n \in \zz} \left (1 + |n| \right )^{2s} \right.
		\\
		& \times \left . \int_\rr \left(
		\int_\rr \left( 1 + | \tau_1 | \right)^{1/2} | \wh{\psi}(\tau_1) |
		\left( 1 + | \tau - \tau_1 - n^{m} | \right)^{1/2} \wh{f}\left( n, \tau
		- \tau_1
		\right)d \tau_1
		\right)^2 d \tau \right)^{1/2}
	\end{split}
\end{equation*}
%
%
which by Minkowski's inequality is bounded by
%
%
\begin{equation}
	\label{2b}
	\begin{split}
		& \left( \sum_{n \in \zz} \left (1 + |n| \right )^{2s}  \right.
		\\
		& \times \left. \left( \int_\rr \left[ \int_\rr
		\left( 1 + | \tau_{1} | \right) | \wh{\psi}(\tau_1) |^2 \left( 1 + |
		\tau - \tau_1 - n^{m} |
		\right) | \wh{f}\left( n, \tau - \tau_1 \right) |^2 d \tau_1 
		\right]^{1/2} d \tau \right)^2 \right)^{1/2}.
	\end{split}
\end{equation}
%
%
Using the change of variable $\tau - \tau_1 \to \lambda$ gives
%
%
\begin{equation*}
	\begin{split}
		\eqref{2b}
		& = \left( \sum_{n \in \zz} \left (1 + |n| \right )^{2s}\right.
		\\
		& \times \left.  \left( \int_\rr \left[
		\int_\rr \left( 1 + | \tau_1 | \right) | \wh{\psi}\left( \tau_1
		\right) |^2 \left( 1 + | \lambda - n^{m} | \right) | \wh{f} \left( n,
		\lambda
		\right)|^2 d \tau_1 \right]^{1/2} d \lambda \right)^2 \right)^{1/2}
		\\
		& =  \left( \sum_{n \in \zz} \left (1 + |n| \right )^{2s} \right.
		\\
		& \times \left. \left( \int_\rr \left( 1 + |
		\tau_1 |
		\right)^{1/2} | \wh{\psi}(\tau_1) | d \tau_1 \left[ \int_\rr \left( 1 + |
		\lambda - n^{m} |
		\right) | \wh{f}\left( n, \lambda \right) |^2 d \lambda \right]^{1/2}
		\right)^2 \right)^{1/2}
		\\
		& = c_{\psi} \left( \sum_{n \in \zz} \left (1 + |n| \right )^{2s} \left( \left[ \int_\rr
		\left( 1 + | \lambda - n^{m} | \right) | \wh{f}\left( n, \lambda
		\right) |^2 d \lambda
		\right]^{\cancel{1/2}} \right)^{\cancel{2}} \right)^{1/2}
		\\
		& = c_{\psi} \|f\|_{X^s},
	\end{split}
\end{equation*}
%
%
which proves \eqref{schwartz-mult-piece-1}.
Estimating 
%
%
\begin{equation*}
\begin{split}
  \| \psi f \|_{E}^{2}
  & = \sum_{n \in \zz} | n |^{2s} \left( \int_{\rr} |
  \wh{\psi f}(n, \tau)
  | d \tau \right)^{2}
  \\
  & = \sum_{n \in \zz} | n |^{2s} \left( \int_{\rr} \wh{\psi} * \wh{f}
  (n, \tau) d \tau \right)^{2}
  \\
  & \le \| \wh{\psi} \|_{L^1}^{2} \sum_{n \in \zz} | n |^{2s} \left(
  \int_{\rr} \wh{f}(n, \tau) d \tau
  \right)^{2} \quad \text{(Young's Inequality)}
  \\
  & = c_{\psi} \| f \|_{E}^2.
\end{split}
\end{equation*}
%
%
and taking square roots of both sides gives \eqref{schwartz-mult-piece-2}. Combining
\eqref{schwartz-mult-piece-1} and \eqref{schwartz-mult-piece-2} we obtain
\eqref{schwartz-mult}, completing the proof.  \qquad \qedsymbol
%
%
\subsection{Proof of \autoref{lem:splitting}.} We have
%
%
\begin{equation}
	\label{6a}
	\begin{split}
		1 + | a + b + c| 
		& \le 1 + | a | + | b | + | c |
		\\
		& \le 1 + | a | + 1 + | b | + 1 + | c |
		\\
		& \le 3\left( \max\{1+| a |, 1+| b |, 1+ | c | \}\right)
		\\
		& \le 3 \left( 1 + | a | \right)\left( 1 + | b | \right) \left( 1 + |
		c |
		\right), \qquad a, b, c \in \zz.
	\end{split}
\end{equation}
%
%
Raising both sides of expression $\eqref{6a}$ to the $v$ power completes 
the proof. \qquad \qedsymbol 

  




%\nocite{*}
\bibliography{/Users/davidkarapetyan/math/bib-files/references.bib}
% \bib, bibdiv, biblist are defined by the amsrefs package.
\end{document}
