%
\documentclass[12pt,reqno]{amsart}
\usepackage{amssymb}
\usepackage{appendix}
\usepackage[notref, notcite]{showkeys}
\usepackage{tikz}
\usepackage[showonlyrefs=true]{mathtools} %amsmath extension package
\usepackage{cancel}  %for cancelling terms explicity on pdf
\usepackage[normalem]{ulem} %for striking out non-math text
\usepackage{yhmath}   %makes fourier transform look nicer, among other things
\usepackage{framed}  %for framing remarks, theorems, etc.
\usepackage{enumerate} %to change enumerate symbols
\usepackage[margin=2.5cm]{geometry}  %page layout
%\setcounter{tocdepth}{1} %must come before secnumdepth--else, pain
%\setcounter{secnumdepth}{1} %number only sections, not subsections
%\usepackage[pdftex]{graphicx} %for importing pictures into latex--pdf compilation
\numberwithin{equation}{section}  %eliminate need for keeping track of counters
\numberwithin{figure}{section}
\setlength{\parindent}{0in} %no indentation of paragraphs after section title
\renewcommand{\baselinestretch}{1.1} %increases vert spacing of text
%
\usepackage{hyperref}
\hypersetup{colorlinks=true,
linkcolor=blue,
citecolor=blue,
urlcolor=blue,
}
%
\DeclareMathOperator{\sgn}{sgn}
%
\newcommand{\ds}{\displaystyle}
\newcommand{\ts}{\textstyle}
\newcommand{\nin}{\noindent}
\newcommand{\rr}{\mathbb{R}}
\newcommand{\nn}{\mathbb{N}}
\newcommand{\zz}{\mathbb{Z}}
\newcommand{\cc}{\mathbb{C}}
\newcommand{\ci}{\mathbb{T}}
\newcommand{\zzdot}{\dot{\zz}}
\newcommand{\wh}{\widehat}
\newcommand{\p}{\partial}
\newcommand{\ee}{\varepsilon}
\newcommand{\vp}{\varphi}
\newcommand{\wt}{\widetilde}
%
%
%
%
\newtheorem{theorem}{Theorem}[section]
\newtheorem{lemma}[theorem]{Lemma}
\newtheorem{corollary}[theorem]{Corollary}
\newtheorem{claim}[theorem]{Claim}
\newtheorem{prop}[theorem]{Proposition}
\newtheorem{proposition}[theorem]{Proposition}
\newtheorem{no}[theorem]{Notation}
\newtheorem{definition}[theorem]{Definition}
\newtheorem{remark}[theorem]{Remark}
\newtheorem{examp}{Example}[section]
\newtheorem {exercise}[theorem]{Exercise}
%
%\makeatletter \renewenvironment{proof}[1][\proofname] {\par\pushQED{\qed}\normalfont\topsep6\p@\@plus6\p@\relax\trivlist\item[\hskip\labelsep\bfseries#1\@addpunct{.}]\ignorespaces}{\popQED\endtrivlist\@endpefalse} \makeatother%
%makes proof environment bold instead of italic
\newcommand{\uol}{u^\omega_\lambda}
\newcommand{\lbar}{\bar{l}}
\renewcommand{\l}{\lambda}
\newcommand{\R}{\mathbb R}
\newcommand{\RR}{\mathcal R}
\newcommand{\al}{\alpha}
\newcommand{\ve}{q}
\newcommand{\tg}{{tan}}
\newcommand{\m}{q}
\newcommand{\N}{N}
\newcommand{\ta}{{\tilde{a}}}
\newcommand{\tb}{{\tilde{b}}}
\newcommand{\tc}{{\tilde{c}}}
\newcommand{\tS}{{\tilde S}}
\newcommand{\tP}{{\tilde P}}
\newcommand{\tu}{{\tilde{u}}}
\newcommand{\tw}{{\tilde{w}}}
\newcommand{\tA}{{\tilde{A}}}
\newcommand{\tX}{{\tilde{X}}}
\newcommand{\tphi}{{\tilde{\phi}}}
\synctex=1
\begin{document}
\title[H\"older Continuity of the Data to Solution Map for HR]{H\"older Continuity of the Data to Solution Map for HR in the
Weak Topology}
\author{David Karapetyan}
\address{Department of Mathematics  \\
    University  of Notre Dame\\
        Notre Dame, IN 46556 }
        \date{\today}
        %
        \maketitle
        %
        %
        %
        %
        %
        %        
        %
        %
%%%%%%%%%%%%%%%%%%%%%%%%%%%%%%%%%%%%%%%%%%%%%%%%%%%%%
%
%
%				Optimality
%
%
%%%%%%%%%%%%%%%%%%%%%%%%%%%%%%%%%%%%%%%%%%%%%%%%%%%%%
%
%
\section{Optimality} 
\label{sec:optimality}
\subsection{Burgers} 
\label{ssec:burgers-opt}

We first consider the Burgers initial value problem
%
%
\begin{gather}
    \label{burgers}
u_{t} + uu_{x} = 0,
\\
\label{burgers-init}
u(x,0) = u_{0}(x)
\end{gather}
%
%
To solve this, we apply the method of characteristics. That is, we seek a curve
$s \mapsto (x(s), t(s))$ such that
%
%
\begin{equation*}
\begin{split}
\frac{d}{ds} u(x(s), t(s)) = 0.
\end{split}
\end{equation*}
%
Formally differentiating the left hand side, we obtain
%
%
\begin{equation*}
\begin{split}
\frac{du}{dx} \frac{dx}{ds} + \frac{du}{dt} \frac{dt}{ds}.
\end{split}
\end{equation*}
%
%
Setting
%
%
\begin{gather}
    \label{char-ode-space}
    \frac{dx}{ds} = u,
    \\
    \label{char-ode-time}
    \frac{dt}{ds}=1
\end{gather}
and recalling the Burgers equation \eqref{burgers}, we see that
%
%
\begin{equation*}
\begin{split}
\frac{d}{ds} u(x(s), t(s)) = 0.
\end{split}
\end{equation*}
%
%
Hence, $u$ is constant along the characteristic curve $(x(s), t(s))$ given by
the characteristic ode's \eqref{char-ode-space}-\eqref{char-ode-time}. Solving
them, we obtain
\begin{gather}
    \label{0j}
    x(s) = c_{1} + \int_{0}^{s} u(x(s'), t(s'))ds'
    \\
    t(s) = s + c_{2}.
\end{gather}
and since $u$ is constant along the characteristic curve $(x(s), t(s))$, this reduces to
%
%
\begin{gather}
    \label{1j}
    x(s) = c_{1} + su
    \\
    \label{2j}
    t(s) = s + c_{2}.
\end{gather}
%
Again, using the fact that $u$ is constant along $(x(s), t(s))$, we see that
%
%
%
%
\begin{equation*}
\begin{split}
u(x(s),t(s)) = u(x(-c_{2}), t(-c_{2})) = u(c_{1} - c_{2}u, 0 ) = u_{0}(c_{1} - c_{2} u)
\end{split}
\end{equation*}
%
which implies
%
%
\begin{equation*}
\begin{split}
  u(x(0), t(0)) = u(c_{1}, c_{2}) = u_{0}(c_{1} - c_{2}u). 
\end{split}
\end{equation*}
%
%
%
Since $c_{1}, c_{2}$ are arbitrary constants, a relabeling allows us to conclude that
%
%
%
%
\begin{equation*}
\begin{split}
u(x,t) = u_{0}(x - tu).
\end{split}
\end{equation*}
%
%
We now need the following.
%
%
%%%%%%%%%%%%%%%%%%%%%%%%%%%%%%%%%%%%%%%%%%%%%%%%%%%%%
%
%
%			solution burgers 
%
%
%%%%%%%%%%%%%%%%%%%%%%%%%%%%%%%%%%%%%%%%%%%%%%%%%%%%%
%
%
\begin{lemma}
%
Let 
\begin{equation}
  \label{key-init-data}
  u_{0}^{\lambda}(x) = (\lambda +
x_{+}^{\alpha + 1}) \vp(x)
\end{equation}
be a family of initial data indexed by $\lambda \in
[-1, 1]$, where $\alpha > 0$, $x_{+} \doteq \max\{0, x\}$ and $\vp$ is a smooth cutoff
function equal to the identity in $[-2, 2]$ and with support in $[-4,4]$. Then the associated solutions $u^{\lambda}(x,t)$ take the form
\begin{equation}
    \label{u-lam-explicit-form}
    \begin{split}
        u^{\lambda}(x,t) = \lambda + (x - \lambda
        t)^{\alpha + 1}_{+} p(t(x- \lambda
        t)_{+}), \quad | x | \le 1, \quad
        | \lambda | \le 1
    \end{split}
\end{equation}
%
%
for sufficiently small $| t |$, where $p(z)$ is a power series in $z$ with $p(0) =1$ and a positive radius of convergence.
\label{lem:sol-burg}
\end{lemma}
%
%
\begin{framed}
%
%
\begin{remark}
Kato's statement of the theorem has $p(t(x - \lambda t)^{\alpha}_{+})$. This is a typo.
\label{rem:kato-typo}
\end{remark}
%
%
\end{framed}
%
%

%
%
%
%
%
Applying Lemma~\ref{lem:sol-burg}, we have
%
%
\begin{equation*}
\begin{split}
u^{\lambda} - u^{0} = \lambda  + \sum_{k=0}^{\infty}c_{k}t^{k}\left[ (x - \lambda
  t)^{\alpha + 1 + k}_{+} - x_{+}^{\alpha + 1 + k}
  \right].
\end{split}
\end{equation*}
%
%
Furthermore, for integer $s \ge 2$ (we need to have well-posedness, and don't want to deal with fractional differentiation) and real $\alpha > (s-2)$
%
%
\begin{equation*}
\begin{split}
\frac{d^{s}}{dx^{s}}(u^{\lambda} - u^{0}) & = (\alpha + 1)(\alpha) \cdots (\alpha
+ 2 -s)\{\sum_{k=1}^{\infty}c_{k}t^{k}\left[ (x - \lambda t)^{\alpha + 1 + k -s}_{+} -
  x_{+}^{\alpha + 1 + k -s} \right] \}.
\end{split}
\end{equation*}
%
%
Hence
%
%
\begin{equation*}
\begin{split}
\| u^{\lambda} - u^{0} \|_{H^{s}}^{2} 
& \ge \| \frac{d^{s}}{dx^{s}}(u^{\lambda} - u^{0}) \|_{L^{2}}^{2}
\\
& \ge \int_{0}^{\lambda t} | \frac{d^{s}}{dx^{s}}(u^{\lambda} - u^{0}) |^{2} dx
\\
& \simeq \int_{0}^{\lambda t}
\{ \sum_{k=0}^{\infty}c_{k}t^{k}\left[ (x - \lambda t)^{\alpha + 1 + k -s}_{+} -
  x_{+}^{\alpha + 1 + k -s} \right] \}^{2}dx
\\
& = \int_{0}^{\lambda t}
( \sum_{k=0}^{\infty}c_{k}t^{k} x_{+}^{\alpha + 1 + k -s} )^{2}dx
\\
& =\int_{0}^{\lambda t}
x^{2(\alpha + 1 -s)} [ \sum_{k=0}^{\infty}c_{k}(tx_{+})^{k} ]^{2}dx.
\end{split}
\end{equation*}
%
But since $\sum_{k=0}^{\infty}c_{k}(tx_{+})^{k}$ is continuous in
$x$ and $c_{0} =1$, we can find $T > 0$ sufficiently small such that for $0 \le x \le T$ 
%
%
\begin{equation*}
\begin{split}
  [\sum_{k=0}^{\infty}c_{k}(x_{+})^{k}]^{2} \ge 1/2. 
\end{split}
\end{equation*}
%
%
Hence, restricting $t \le \min \{1, T/\lambda \}$, we obtain the bound
%
%
%
\begin{equation*}
\begin{split}
\| u^{\lambda} - u^{0} \|_{H^{s}}^{2} 
& \gtrsim \int_{0}^{\lambda t} x^{2(\alpha + 1 -s)} dx 
\\
& \simeq  | \lambda t |^{2 \alpha + 3 -2s}
\end{split}
\end{equation*}
%
%
and so
\begin{equation}
    \begin{split}
    \label{holder-lb}
\| u^{\lambda} - u^{0} \|_{H^{s}} & > c_{\alpha, s} | \lambda t |^{\alpha -s +
    3/2}, \quad t \le \min \{1, T/\lambda \}.
\end{split}
\end{equation}
%
%
Suppose the map $u_{0} \mapsto u(t)$ is locally
H\"older continuous in $H^{r}$, $r<s$,
with index $\gamma > 0$. Then for a sequence of initial data $f_{0}^{\lambda} \in B_{H^{s}}(R)$, there exists $t_{0} > 0$ depending on $R$ such that for $0 < t <
t_{0}$
%
%
\begin{equation*}
\begin{split}
  \| f^{\lambda}(t) - f^{0}(t) \|_{H^{r}}
  \le c_{r, \gamma} \| f^{\lambda}_{0} - f^{0}_{0} \|_{H^{r}}^{\gamma}.
  \\
  &
\end{split}
\end{equation*}
%
%
To disprove H\"older continuity, we proceed by contradiction and choose a uniformly bounded sequence motivated by \eqref{key-init-data}, that is
%
%
\begin{equation*}
\begin{split}
  u_{0}^{\lambda}(x) = \left\{ (\lambda + x^{\alpha + 1}_{+})\vp \right\}_{0 \le
      \lambda < 1}.
\end{split}
\end{equation*}
%
%
Assume H\"older continuity holds. Then there exists $t_{0} > 0$ independent of $\lambda$ such that for $0 < t \le t_{0}$
%
%
\begin{equation}
  \label{rg}
\begin{split}
  \| u^{\lambda}(t) - u^{0}(t) \| 
  & \le c_{\gamma} \| u_{0}^{\lambda} - u^{0}_{0} \|^{\gamma}_{H^{r}}
  \\
  & = c_{\gamma, \vp} \lambda^{\gamma}.
\end{split}
\end{equation}
%
%
Next, fix $\lambda$, and restrict $t_{0} < \{1, T/\lambda \}$.
Then for $t = t_{0}$, \eqref{holder-lb} gives
%
%
\begin{equation*}
\begin{split}
  \| u^{\lambda}(t_{0}) - u^{0}(t_{0}) \|_{H^{r}} > c_{\alpha,s, t_{0}} \lambda^{\alpha - s + 3/2}.
\end{split}
\end{equation*}
%
Recall that our only restriction on $\alpha$ was that $\alpha > s-2$. 
Let $\ee <<1$, and choose $\alpha = s - 3/2 + \gamma - \ee$. 
Then
\begin{equation*}
\begin{split}
  \| u^{\lambda}(t_{0}) - u^{0}(t_{0}) \| 
  &  > c_{\alpha,s, t_{0}} \lambda^{\gamma - \ee}
\end{split}
\end{equation*}
%
which for 
%
%
\begin{equation*}
\begin{split}
  c_{\alpha,s, t_{0}}\lambda^{-\ee} > c_{\gamma, \vp}
\end{split}
\end{equation*}
%
%
or 
%
%
\begin{equation}
  \label{lam-bound}
\begin{split}
    \left (\frac{c_{\gamma, \vp}}{c_{\alpha, s, t_{0}}} \right)^{\ee} < \lambda < 1
\end{split}
\end{equation}
%
%
gives
%
%
\begin{equation*}
\begin{split}
  \| u^{\lambda}(t_{0}) - u^{0}(t_{0}) \| 
  &  > c_{\gamma, \vp} \lambda^{\gamma}
\end{split}
\end{equation*}
%
contradicting \eqref{rg}.
%
\begin{framed}
  \begin{remark}
  We can always choose $\ee > 0$ sufficiently small such that
\begin{equation*}
\begin{split}
    \left (\frac{c_{\alpha, s, t_{0}}}{c_{\gamma, \vp}}
\right )^{\ee} < 1
\end{split}
\end{equation*}
and so relation \eqref{lam-bound} makes sense.
\end{remark}
\end{framed}
Hence, since $\gamma$ was arbitrary, we see that Burgers
initial value problem on the line with initial data $u_{0} \in H^{s}$, $s \ge 2$
is not H\"older continuous for any exponent. This completes the proof.
%
%
\begin{proof}[Proof of Lemma~\ref{lem:sol-burg}]
Since $u_{0}^{\lambda}$ is uniformly bounded in $\lambda$ in $H^{s}$, the
$u^{\lambda}$ have a common lifespan $T$ and are uniformly bounded (via an
energy estimate--see \cite{Karapetyan:2010fk}). Therefore, choosing $T$
sufficiently small, we have
%
%
\begin{equation}
    \label{burg-sol-cont}
\begin{split}
    | t u^{\lambda}(x,t) | \le 1, \quad x \in \rr, \ | t | \le T, \ | \lambda | \le 1.
\end{split}
\end{equation}
%
%
%
%
Set
%
%
\begin{equation}
    \label{burg-sol-not}
\begin{split}
y(x,t) = x - t u^{\lambda}(x,t).
\end{split}
\end{equation}
%
%
Then by \eqref{burg-sol-cont}, we have
%
%
\begin{equation*}
\begin{split}
    | y(x,t) | & \le | x | + | tu^{\lambda} |
    \\
    & \le 2, \quad | x | \le 1, \ | t | \le T, | \lambda | \le 1.
\end{split}
\end{equation*}
%
%
Hence, $\vp(y) =1$ and so
%
%
\begin{equation}
    \label{burg-sol-redux}
\begin{split}
u^{\lambda}(x,t) 
& = u_{0}^{\lambda}(x - tu)
\\
& = [\lambda + (x - tu^{\lambda})_{+}^{\alpha + 1}] \vp(x - tu^{\lambda})
\\
& = [\lambda + (x - tu^{\lambda})_{+}^{\alpha + 1}].
\end{split}
\end{equation}
%
%
Multiplying both sides by $t$, we get
%
%
\begin{equation*}
\begin{split}
t u^{\lambda}(x,t) = t \lambda + t (x - tu^{\lambda})_{+}^{\alpha + 1}
\end{split}
\end{equation*}
%
%
which implies
%
%
\begin{equation*}
\begin{split}
x - tu^{\lambda} = x - t \lambda - t(x - tu^{\lambda})_{+}^{\alpha + 1}.
\end{split}
\end{equation*}
%
%
Set 
\begin{equation}
    \label{y-for-u}
    y(x,t) = x - tu^{\lambda}. 
\end{equation}
%
%
%
%
Then
%
%
\begin{equation*}
\begin{split}
y = x - t \lambda - t y_{+}^{\alpha + 1}
\end{split}
\end{equation*}
%
%
or
%
\begin{equation*}
\begin{split}
y + ty_{+}^{\alpha + 1} = x - t \lambda.
\end{split}
\end{equation*}
%
%
Without loss of generality, assume $t \ge 0$, $\lambda \ge 0$. 
Then this is equivalent to the system
%
%
\begin{gather}
    \label{y-simp}
    y= x - t \lambda, \quad x \le t \lambda
    \\
    \label{y-rest}
    y + ty^{\alpha + 1} = x - t \lambda, \quad x \ge t \lambda
\end{gather}
for $x \ge t \lambda$. We shall first solve \eqref{y-rest} without any
restrictions on $x$, that is, we are looking for a
solution to 
%
%
\begin{equation}
    \label{f-eq}
\begin{split}
f + tf^{\alpha + 1} = x - t \lambda
\end{split}
\end{equation}
%
%
which we will then use to 
construct a solution to the system
\eqref{y-simp}-\eqref{y-rest}. Let
%
%
\begin{equation*}
\begin{split}
    g(x,t) = f(x + \lambda t, t).
\end{split}
\end{equation*}
%
%
Then
%
%
\begin{equation}
    \label{g-eq}
\begin{split}
g + t g^{\alpha + 1} = x
\end{split}
\end{equation}
%
%
which implies
%
%
\begin{equation*}
\begin{split}
g(0, t) = 0.
\end{split}
\end{equation*}
%
%
For the sake of clarity, we handle the case $\alpha =2$ first.
Differentiating both sides of \eqref{g-eq} with respect to $x$ once, twice, and
three times, respectively, we obtain
%
%
\begin{gather*}
    g_{x}\left \{ 2t g + 1 \right \} = 1,
    \\
    g_{xx}(3tg^{2} + 1) + 6tgg_{x}^{2} = 0,
    \\
    g_{xxx}(3tg^{2} + 1) + 18tgg_{x}g_{xx} + 6tg_{x}^{3} = 0
\end{gather*}
%
%
which implies
%
%
\begin{gather*}
g_{x}(0, t) =1,
\\
g_{xx}(0, t) = 0,
\\
g_{xxx}(0, t) = -6t.
\end{gather*}
%
Continuing in this fashion, we obtain
%
%
\begin{equation*}
\begin{split}
\frac{d^{k}}{dx^{k}} g |_{(0, t)} = c_{k}t^{k-2}
\end{split}
\end{equation*}
%
%
and hence
%
%
\begin{equation*}
\begin{split}
g(x,t) & = x - tx^{3} + ct^{2}x^{4} + \ldots
\\
& = x - t[x^{3}(1 + c_{1}tx + c_{2}t^{2}x^{2} + \ldots)] 
\end{split}
\end{equation*}
%
For general integer valued $\alpha \ge 0$, a similar argument gives
%
\begin{equation}
  \label{eq-gen}
\begin{split}
g(x,t) & = x - tx^{\alpha + 1} + ct^{2}x^{\alpha + 2} + \ldots
\\
& = x - t[x^{\alpha + 1}(1 + c_{1}tx + c_{2}t^{2}x^{2} + \ldots)].
\end{split}
\end{equation}

Therefore
%
%
\begin{equation*}
\begin{split}
f(x,t) 
& = x - \lambda t - t \left \{ (x - \lambda t)^{\alpha + 1}[1 + c_{1}t(x - \lambda t) + c_{2}t^{2}(x - \lambda t)^{2} + \ldots] \right \}
\\
& = x - t \left \{ \lambda  + (x - \lambda t)^{\alpha + 1} [1 + c_{1}t(x - \lambda t) + c_{2}t^{2}(x - \lambda t)^{2} + \ldots] \right \}
\end{split}
\end{equation*}
%
solves \eqref{f-eq} for $x \ge \lambda t$. By observation, we see that
%
%
\begin{equation*}
\begin{split}
y(x,t) 
& = x - t \left \{ \lambda  + (x - \lambda
    t)_{+}^{\alpha + 1} [1 + c_{1}t(x - \lambda t)_{+} +
    c_{2}t^{2}(x - \lambda t)_{+}^{2} + \ldots] \right \}
\end{split}
\end{equation*}
%
%
solves the system \eqref{y-simp}-\eqref{y-rest}.
%
%
%
Recalling that
%
%
\begin{equation*}
\begin{split}
y(x,t) = x - tu^{\lambda}
\end{split}
\end{equation*}
%
%
we conclude that
%
%
%
%
\begin{equation}
    \label{ode-sol}
\begin{split}
u^{\lambda}(x,t) = \lambda + (x - \lambda
t)_{+}^{\alpha + 1}[1 + c_{1}t(x - \lambda t)_{+} +
c_{2}t^{2}(x - \lambda t)_{+}^{2} + \ldots]
\end{split}
\end{equation}
where $c_{k} = c_{k}(\alpha)$. 
%
To extend this result for arbitrary $\alpha \ge 0$, it will be enough to solve \eqref{g-eq} for arbitrary $\alpha \ge 0$.
%
%
%
%
However, notice that solutions to this equation will \emph{not} be analytic at $x = 0$. (For example, $(g^{1/2})' \to \infty$ as $x \to 0$, since $g(0, t) = 0$). However, for integer $k \ge 0$, we have already shown that there exists $g = g(x,t)$ analytic in $x$ and $t$ in a neighborhood of the origin (that is, $|x| \le 1$ and $|t| \le T$) and satisfying
%
%
\begin{equation}
  \label{yit}
\begin{split}
  g + tg^{k} = x
\end{split}
\end{equation}
%
which has form \eqref{eq-gen}. Substituting this relation, we obtain
\begin{equation}
  \label{yit2}
\begin{split}
  g^{k} = x^{k}p(xt)
\end{split}
\end{equation}
where $p(xt)$ is analytic for $xt$ in a neighborhood of $0$, and $p(0) = 1$. From \eqref{yit} and \eqref{yit2}, we see that 
%
%
%
%
%
\begin{equation}
  \label{uit}
\begin{split}
  (g-x)x^{r-k}= -t x^{r}p(xt).
\end{split}
\end{equation}
%
%
Let 
%
%
\begin{equation*}
\begin{split}
  h \doteq x + c(g -x)x^{r-k}
\end{split}
\end{equation*}
%
where $c = c(xt)$ is to be determined later.
Then \eqref{yit2} implies
%
%
\begin{equation}
  \label{yit3}
\begin{split}
  h -x = -\frac{t x^{r} p(xt)}{c}.
\end{split}
\end{equation}
%
%
But 
%
%
\begin{equation*}
\begin{split}
  h = x + c(g -x)x^{r-k} = x[1 + c(g-x)x^{r-k-1}]
\end{split}
\end{equation*}
%
and so
%
%
%
\begin{equation*}
\begin{split}
  h^{r} = x^{r}\left[ 1 + c(g-x)x^{r-k-1} \right]^{r}.
\end{split}
\end{equation*}
%
%
Assume for the time being that we can choose $c = c(xt)$ such that
%
%
\begin{equation}
  \label{yit4}
\begin{split}
\frac{p(xt)}{c} = \left[ 1 + c(g-x)x^{r-k-1} \right]^{r}.
\end{split}
\end{equation}
%
%
Then
%
%
\begin{equation*}
\begin{split}
  h^{r} = \frac{x^{r}p(xt)}{c}
\end{split}
\end{equation*}
%
%
which we combine with \eqref{yit3} to obtain
%
%
%
\begin{equation*}
\begin{split}
  h + th^{r} = x.
\end{split}
\end{equation*}
%
%
Next, we show that there exists a unique $c = c(x,t) $ such that \eqref{yit4} holds. Proceeding, we rewrite \eqref{yit4} via \eqref{uit} to obtain
%
%
\begin{equation}
  \label{pii}
\begin{split}
  0 = p(xt) - c \left[ 1 - ctx^{r-1}p(xt) \right]^{r} \doteq P(x,t,c).
\end{split}
\end{equation}
%
%
which is continuous in $(x, t)$ and analytic in $(x,t)$, where $0 < |x| \le 1$, $0 < |t| < T$. Furthermore, $D_{(x,t)}P$ exists and is invertible for $(x,t)$ where $0 < |x| \le 1$, $0 < |t| < T$. Hence, by the implicit function theorem,
there exists a unique $c = c(xt)$ analytic for $|xt| > 0$ sufficiently small and satisfying \eqref{pii} in this region. 
\\
\\
We now extend $c(xt)$ to an analytic function at $0$ as follows. Define
%
%
\begin{equation*}
\begin{split}
  c^{*}(xt) = 
  \begin{cases} 
    c(xt), \quad & x > 0, \ t > 0 
    \\
    \lim_{y \to 0} c(y,t), \quad & x = 0, \ t > 0
    \\
    \lim_{s \to 0} c(x,s), \quad & x > 0, \ t = 0.
  \end{cases}
\end{split}
\end{equation*}
%
%
Since $0 = P(0, t, c) = P(x, 0, c) = 1 - c$, it follows that
%
%
\begin{equation*}
\begin{split}
  c^{*}(xt) = 
  \begin{cases} 
    c(xt), \quad & x > 0, \ t > 0 
    \\
    1, \quad & x = 0, t > 0
    \\
    1, \quad & x > 0, t = 0
  \end{cases}
\end{split}
\end{equation*}
%
which is analytic for sufficiently small $|xt|\ge 0$. Hence, the same holds true for
%
%
%
\begin{equation*}
\begin{split}
  p^{*}(xt) \doteq \frac{p(xt)}{c^{*}(xt)}.
\end{split}
\end{equation*}
%
%
In particular, $p^{*}(0) =1$. Putting all our work together, we see that we have found a solution to \eqref{g-eq} for arbitrary $\alpha > 0$ the This completes the proof.
\end{proof}

%%%%%%%%%%%%%%%%%%%%%%%%%%%%%%%%%%%%%%%%%%%%%%%%%%%%%
%
%
%				Optimality HR
%
%
%%%%%%%%%%%%%%%%%%%%%%%%%%%%%%%%%%%%%%%%%%%%%%%%%%%%%
%
%
\section{Optimality for HR} 
\label{sec:op-hr}
We recall the hyperelastic-rod (HR) ivp
\begin{gather}
    \label{hr}
    u_{t} + \frac{\gamma}{2}(u^{2})_{x} + P_{x} = 0
    \\
    \label{hr-data}
    u(x,0) = u_{0}(x)
\end{gather}
where
%
%
\begin{equation*}
\begin{split}
P(x,t) \doteq \frac{1}{2}e^{-| x |} * \left [\frac{3 - \gamma}{2}
    u^{2}(x,t) + \frac{\gamma}{2} u_{2}^{2}(x,t) \right ].
\end{split}
\end{equation*}
%
\begin{framed}
We remark that $e^{-| x |}$ is a weak solution to the ode $(1 + \p_{x}^{2})u =
2\delta$. Taking Fourier transforms of both sides, it follows that
$\widehat{e^{-| \cdot |}}(\xi) = 2/(1 + \xi^{2})$. Hence,
%
%
\begin{equation*}
\begin{split}
\frac{1}{2} e^{-| x |} * f = (1 - \p_{x}^2)f.
\end{split}
\end{equation*}
%
%
Therefore, the form of HR which we all know and love is equivalent to
\eqref{hr}.
\end{framed}
%
Following Bressan \cite{Bressan_2007_Global-conserva}, we let $\xi \in \rr$ be an energy variable, and for fixed $t \in \rr$ define the map $t \mapsto y(\xi, t)$ implicitly by 
%
%
\begin{equation}
\label{potent-def}
\begin{split}
    \int_{0}^{y(\xi, t)} [1 + u^{2}(x,t)]dx = \xi.
\end{split}
\end{equation}
%
%
Consider also the initial value problem
%
%
\begin{gather}
    \label{en-eq}
\frac{dy}{dt}(\xi, t) = \gamma u(y(\xi, t), t),
\\
\label{en-data}
y(\xi, 0) = \xi
\end{gather}
%
%
and define
\begin{gather}
    \label{var-1}
    w(\xi, t) \doteq u(y(\xi, t), t)
    \\
    \label{var-2}
    v(\xi, t) \doteq u_{x}(y(\xi, t), t)
    \\
    \label{var-3}
    q(\xi, t) \doteq y_{\xi}(\xi, t)
\end{gather}
where we adopt the notation $$u_{x}(y(\xi, t),t) \doteq \frac{d}{dz}u(z,t) \big
|_{z = y(\xi, t)}$$ for the remainder of the paper. Using ivp
\eqref{en-eq}-\eqref{en-data}, we will rewrite the HR ivp
\eqref{hr}-\eqref{hr-data} as an ode system for the variables \eqref{var-1},
\eqref{var-2}, and \eqref{var-3}. Proceeding, we note that by the chain rule
%
%
\begin{equation}
\label{w-deriv}
\begin{split}
\frac{d}{dt}w(\xi, t)
& = \gamma u_{x} y_{t}(y(\xi, t), t) + u_{t}(y(\xi, t), )
\\
& = \gamma u u_{x}(y(\xi, t ), t) + u_{t}(y(\xi, t))
\\
& = -P_{x}(y(\xi, t), t).
\end{split}
\end{equation}
%
%
Now, we want to get $P_{x}(y(\xi, t), t)$ in ``nice'' form as we will be estimating later. Notice that
%
%
\begin{equation*}
\begin{split}
P(x,t) = \frac{1}{2} \int_{\rr}e^{-| x - x_{1} |} \left [\frac{3 - \gamma}{2}
u^{2}(x_{1}, t) + \frac{\gamma}{2} u_{x}^{2}(x_{1}, t) \right ]
\end{split}
\end{equation*}
%
%
and so 
%
%
\begin{equation}
\label{yuu}
\begin{split}
P(y(\xi, t), t) = \frac{1}{2} \int_{\rr} e^{-| y(\xi, t) - x_{1} |} \left [ \frac{3 - \gamma}{2} u^{2} + \frac{\gamma}{2}u_{x}^{2} \right ] (x_{1}, t) d x_{1}.
\end{split}
\end{equation}
%
%
Furthermore
%
%
\begin{equation}
\begin{split}
\label{p}
P_{x}(x,t)
& = \frac{1}{2}\p_{x} e^{-| x |}* \left [ \frac{3 - \gamma}{2} u^{2} + \frac{\gamma}{2} u_{x}^{2} \right ] 
\\
& = -\frac{1}{2} \int_{\rr} \sgn(x - x_{1}) e^{-| x - x_{1} |} \left [ \frac{3 - \gamma}{2} u^{2}(x_{1}, t) + \frac{\gamma}{2} u_{x}^{2}(x_{1}, t) \right ] 
\\
& =-\frac{1}{2} \left ( \int_{x}^{\infty} - \int_{-\infty}^{x} \right )
e^{-| x - x_{1} |} \left [ \frac{3 - \gamma}{2} u^{2}(x_{1}, t) +
\frac{\gamma}{2} u_{x}^{2}(x_{1}, t) \right ] dx_{1}
\end{split}
\end{equation}
%
%
and so%
%
\begin{equation}
\label{p-deriv}
\begin{split}
P_{x}(y(\xi, t), t)
& = -\frac{1}{2} \left ( \int_{y(\xi, t)}^{\infty} - \int_{-\infty}^{y(\xi, t)} \right ) e^{-| y(\xi, t) - x_{1} |} \left [ \frac{3 - \gamma}{2} u^{2}(x_{1}, t) +
\frac{\gamma}{2} u_{x}^{2}(x_{1}, t) \right ] dx_{1}.
\end{split}
\end{equation}
%
%
Adopt the notation 
%
%
\begin{equation*}
\begin{split}
q(\xi, t) \doteq y_{\xi}(\xi, t).
\end{split}
\end{equation*}
%
%
Then applying the change of variable $x_{1} = y(\xi_{1}, t)$, we obtain
%
%
\begin{equation*}
\begin{split}
\eqref{yuu} & = -\frac{1}{2} \int_{\rr} e^{-| y(\xi, t) - y(\xi_{1}, t) |} \left [
\frac{3 -\gamma}{2} w^{2}q + \frac{\gamma}{2} v^{2} q  \right ](\xi_{1}, t) d \xi_{1}
\\
& = -\frac{1}{2} \int_{\rr} e^{-| \int_{\xi_{1}}^{\xi} q(\lambda, t) d \lambda |} \left [
\frac{3 -\gamma}{2} w^{2}q + \frac{\gamma}{2} v^{2} q  \right ](\xi_{1}, t) d \xi_{1}
\\
& \doteq Q = Q(v, w, q)(\xi, t).
\end{split}
\end{equation*}
%
%
Similarly
%
%
\begin{equation}
\label{R-def}
\begin{split}
\eqref{p-deriv} & = \frac{1}{2} \left ( \int_{\xi}^{\infty} -
\int_{-\infty}^{\xi} \right ) e^{-| \int_{\xi_{1}}^{\xi} q(\lambda, t) d \lambda 
|} \left [ \frac{3 - \gamma}{2} w^{2} q + \frac{\gamma}{2} v^{2} q \right ] (\xi_{1}, t) d \xi_{1}
\\
& \doteq R = R(v, w, q)(\xi, t). 
\end{split}
\end{equation}
%
Substituting this into \eqref{w-deriv}, we see that
%
%
\begin{equation}
\label{w-ode}
\begin{split}
\frac{d}{dt}w(\xi, t) = -R(w, v, q)(\xi, t).
\end{split}
\end{equation}
%
%
Now, we find analogues of \eqref{w-ode} for $v$ and $q$. Observe that by the
chain rule 
%
\begin{equation}
\label{lkk}
\begin{split}
\frac{d}{dt}v(\xi, t)
&  = u_{xx}(y(\xi, t), t) y_{t}(\xi, t) + u_{xt}(y(\xi, t), t)
\\
& =  (\gamma u_{xx}u + u_{xt})(y(\xi, t), t).
\end{split}
\end{equation}
%
Differentiating \eqref{hr} formally with respect to $x$,
and using the fact that
%
%
\begin{equation*}
\begin{split}
P_{xx}
&  = \frac{1}{2} \p_{x}^{2}e^{-| x |}* \left [ \frac{3 - \gamma}{2}u^{2} + \frac{\gamma}{2} u_{x}^{2} \right ] 
\\
& = \frac{1}{2} \left [ e^{-| x |} - 2 \delta \right ] * \left [ \frac{3 - \gamma}{2}u^{2} + \frac{\gamma}{2} u_{x}^{2} \right ] 
\\
& = \frac{1}{2} e^{-| x |} * \left [ \frac{3 - \gamma}{2}u^{2} +
\frac{\gamma}{2} u_{x}^{2} \right ] - \left [ \frac{3 -
\gamma}{2}u^{2} + \frac{\gamma}{2} u_{x}^{2} \right ] 
\\
& = P - \left [ \frac{3 -
\gamma}{2}u^{2} + \frac{\gamma}{2} u_{x}^{2} \right ] 
\end{split}
\end{equation*}
%
%
we obtain
%
%
\begin{equation*}
\begin{split}
u_{xt} + \gamma u u_{xx} + \gamma u_{x}^{2} + P - \left [ \frac{3 -
\gamma}{2}u^{2} + \frac{\gamma}{2} u_{x}^{2} \right ]  = 0.
\end{split}
\end{equation*}
%
%
Substituting this into \eqref{lkk}, we get
%
%
\begin{equation*}
\begin{split}
\frac{d}{dt}v(\xi, t)
& = \left \{ \gamma u_{xx} u + \big[- \gamma uu_{xx} - u_{x}^{2} - P + \left ( \frac{3 - \gamma}{2} u^{2} + \frac{\gamma}{2}u_{x}^{2} \right ) ] \right \}(y(\xi, t), t)
\\
& = \left \{- P  + \frac{3 - \gamma}{2} u^{2} + \frac{\gamma-2}{2} u_{x}^{2} \right \}(y(\xi, t), t)
\\
& = \left \{-Q + \frac{3- \gamma}{2 \gamma^{2}}w^{2} + \frac{\gamma-2}{2}
v^{2} \right \}(\xi, t).
\end{split}
\end{equation*}
%
Lastly,
%
%
%
%
\begin{equation*}
\begin{split}
\frac{dq}{dt}(\xi, t)
& = \frac{d}{d\xi}\frac{d}{dt}y(\xi, t)
\\
& = \gamma \frac{d}{d \xi}u(y(\xi, t), t)
\\
& = \gamma \frac{du}{dy}(y(\xi,t),t)
\\
& = \gamma u_{x}(y(\xi, t), t)q(\xi, t)
\\
& = \gamma vq(\xi, t).
\end{split}
\end{equation*}
%
%
Next, note that $y(\xi, 0) = 0$, and so
%
%
\begin{equation*}
\begin{split}
w(\xi, 0)
& =  u(y(\xi, 0), 0)
\\
& =  u(\xi, 0)
\\
& =  u_{0}(\xi).
\end{split}
\end{equation*}
%
%


Also, 
%
%
\begin{equation*}
\begin{split}
v(\xi, 0)
& = u_{x}(y(\xi, 0), 0)
\\
& = u_{x}(\xi, 0)
\\
& = u_{0}'(\xi)
\end{split}
\end{equation*}
%
%
and
%
%
\begin{equation*}
\begin{split}
q(\xi,0)
& = y_{\xi}(\xi, 0)
\\
& = 1.
\end{split}
\end{equation*}
%
%
Hence, we are interested in solving the ode Cauchy-problem
%
%
%
%
\begin{gather}
\label{ode-system}
\frac{d}{dt}(w, v, q) = \left ( -R, -Q + \frac{3 - \gamma}{2 \gamma^{2}}w^{2} + \frac{\gamma -2}{2}v^{2}, \frac{vq}{\gamma} \right ) 
\\
\label{ode-system-init}
(w, v, q)(0) = (u_{0}(\xi), u_{0}'(\xi), 1).
\end{gather}
%
Denoting
%
%
\begin{equation*}
\begin{split}
  Y = H^{1} \times L^{\infty} \cap L^{2} \times L^{\infty},
\end{split}
\end{equation*}
%
%
we recall the following.
%
\begin{proposition}[Metric Space ODE Theorem]
	\label{prop:ode-thm}
  Let $X$ be a topological vector space over $\rr$
  with topology induced by a metric $d$, $E \subset X$ open, and $(-a, a)$ an
	open interval in $\rr$. Suppose $f: (-a, a) \times E \to X$ satisfies the
	inequality
	%
	%
	\begin{equation}
		\label{stronger-ode}
		\begin{split}
      d[f(t, x), f(t, y)] \le c d(x, y), \qquad \forall t \in (-a, a),
			\qquad \forall x, y \in E
		\end{split}
	\end{equation}
	%
  Then for given $\vp \in E$, there exists sufficiently small $h > 0$ and a
  unique differentiable map $u: (-h, h) \to E$ which for all $t \in (-h, h)$
  satisfies 
	%
	\begin{gather}
    \label{ode-thm-eq}
			u'(t) = f(t, u(t)),
			\\
      \label{ode-thm-init-data}
			u(0) = \vp.
	\end{gather}
\end{proposition}
We will obtain the following result as a corollary. 
%
%
%%%%%%%%%%%%%%%%%%%%%%%%%%%%%%%%%%%%%%%%%%%%%%%%%%%%%
%
%
%				Existence of Solution to ODE system
%
%
%%%%%%%%%%%%%%%%%%%%%%%%%%%%%%%%%%%%%%%%%%%%%%%%%%%%%
%
%
\begin{theorem}
  Assume $u_{0} \in H^{1}$. Then for sufficiently small $T > 0$
the Cauchy problem \eqref{ode-system}-\eqref{ode-system-init} has a unique
solution $(w, v, q) \in C^{1}([0, T], Y)$.
\label{thm:ode-sys-sol}
\end{theorem}
%
%
\begin{proof}
  Due to \eqref{prop:ode-thm}, it will
  be will be enough to show that the map $$(w, v, q) \mapsto
\left ( -R, -Q + \frac{3 - \gamma}{2 \gamma^{2}}w^{2} + \frac{\gamma
-2}{2}v^{2}, \frac{vw}{\gamma} \right ) \doteq F(w, v, q)$$ is Lipschitz on
an open subset $U$ of $Y$. 
%
Using the notation $Q_{1} \doteq Q(w_{1}, v_{1}, q_{1})$ and $Q_{2} \doteq
Q(w_{2}, v_{2}, q_{2})$, we have
%
%
\begin{equation}
  \label{lip-diff}
\begin{split}
  & (w_{2}, v_{2}, q_{2}) - (w_{1}, v_{1}, q_{1})
  \\
  & = \left( -(R_{2} - R_{1}),
  -(Q_{2} - Q_{1}) + \frac{3 - \gamma}{2 \gamma^{2}}(w_{2}^{2} -
  w_{1}^{2}) + \frac{\gamma -2}{2} (v_{2}^{2} - v_{1}^{2}),
  \frac{1}{\gamma}(v_{2} w_{2} - v_{1} w_{1}) \right).
\end{split}
\end{equation}
%
Let $r>0$, $w_{1}, w_{2} \in B_{H^{1}}(r)$, $v_{1}, v_{2} \in B_{L^{\infty} \cap L^{2}}(r)$, and
$q_{1}, q_{2} \in B_{L^{\infty}}(r)$, where $0 < c < q_{1} < r$ and $0 < c <
q_{2} <r$.
Note that $L^{\infty} \cap L^{2}$ is an algebra, since
%
%
\begin{equation*}
\begin{split}
  \| fg \|_{L^{\infty} \cap L^{2}} & = \| fg \|_{L^{\infty}} + \| fg \|_{L^{2}}
  \\
  & \le \| f \|_{L^{\infty}} \| g \|_{L^{\infty}} + \| f \|_{L^{2}}\| g \|_{L^{\infty}}
  \\
  & \le \| f \|_{L^{\infty} \cap L^{2}} \| g \|_{L^{\infty} \cap L^{2}}.
\end{split}
\end{equation*}
%
%
Hence
%
%
%
\begin{equation}
  \label{1aa}
\begin{split}
  \| v_{2}w_{2} - v_{1}w_{1} \|_{L^{\infty} \cap L^{2}}
  & = \| v_{2}w_{2} \pm v_{2}w_{1} - v_{1}w_{1} \|_{L^{\infty} \cap L^{2}}
  \\
  & \le \| v_{2}(w_{2} - w_{1}) \|_{L^{\infty} \cap L^{2}}  
  + \| w_{1}(v_{2} - v_{1}) \|_{L^{\infty} \cap L^{2}}  
  \\
  & \lesssim_{r} \| w_{2} - w_{1} \|_{L^{\infty} \cap L^{2}} +
  \| v_{2} - v_{1} \|_{L^{\infty} \cap L^{2}}
\end{split}
\end{equation}
%
%
and
%
\begin{equation}
  \label{2aa}
\begin{split}
  \| w^{2}_{2} - w_{1}^{2} \|_{L^{\infty} \cap L^{2}} 
  & = \| (w_{2} - w_{1})(w_{2} + w_{1}) \|_{L^{\infty} \cap L^{2}}
    \\
    & \lesssim_{r} \| w_{2} - w_{1} \|_{L^{2} \cap L^{\infty}}.
\end{split}
\end{equation}
%
Similarly
%
%
\begin{equation}
  \label{3aa}
\begin{split}
\| v^{2}_{2} - v_{1}^{2} \|_{L^{\infty} \cap L^{2}} 
& \lesssim_{r}\| v_{2} - v_{1} \|_{L^{\infty} \cap L^{2}}.
\end{split}
\end{equation}
%
%
It remains to estimate $(R_{2} - R_{1})$ and $(Q_{2} - Q_{1})$ in $H^{1}$
and $L^{\infty} \cap L^{2}$, respectively. First, note that
%
%
\begin{equation}
\begin{split}
  \| Q_{2} - Q_{1} \|_{L^{\infty}}
  & \le \| \frac{3 - \gamma}{2} (w_{2}^{2} q_{2} - w_{1}^{2} q_{1}) +
  \frac{\gamma}{2}(v_{2}^{2}q_{2} - v_{1}^{2}q_{1}) \|_{L^{1}}.
\end{split}
\end{equation}
%
%
Now
%
%
\begin{equation*}
\begin{split}
  w_{2}^{2} q_{2} - w_{1}^{2} q_{1}
  & = w_{2}^{2} q_{2} - w_{1}^{2} q_{2} + w_{1}^{2} q_{2} - w_{1}^{2} q_{1}
  \\
  & = (w_{2} - w_{1})(w_{2} + w_{1})q_{2} + (q_{2} - q_{1})w_{1}^{2}.
\end{split}
\end{equation*}
%
%
Similarly
%
%
\begin{equation*}
\begin{split}
  v_{2}^{2}q_{2} - v_{1}^{2} q_{1} = (v_{2} - v_{1})(v_{2}  + v_{1})q_{2} + (q_{2} - q_{1})v_{1}^{2}.
\end{split}
\end{equation*}
%
%
Hence, applying H\'older, we have 
\begin{equation}
  \label{q-init-bound}
  \begin{split}
  & \| \frac{3 - \gamma}{2} (w_{2}^{2} q_{2} - w_{1}^{2} q_{1}) +
  \frac{\gamma}{2}(v_{2}^{2}q_{2} - v_{1}^{2}q_{1}) \|_{L^{1}}
  \\
  & \lesssim \| q_{2} \|_{L^{\infty}} \| (w_{2} - w_{1})(w_{2} + w_{1})
  + (v_{2} - v_{1})(v_{2} + v_{1}) \|_{L^{1}}
  + \| q_{2} - q_{1} \|_{L^{\infty}} \| w_{1}^{2} + v_{1}^{2}\|_{L^{1}} 
  \\
  & \le \| q_{2} \|_{L^{\infty}} \| (w_{2} - w_{1})(w_{2} + w_{1})
  + (v_{2} - v_{1})(v_{2} + v_{1}) \|_{L^{1}}
  + \| q_{2} - q_{1} \|_{L^{\infty}} (\| w_{1}\|_{L^{2}}^{2} + \|
  v_{1}\|_{L^{2}}^{2}) 
  \\
  & \lesssim_{r} \| w_{2} - w_{1} \|_{L^{2}} + \| v_{2} - v_{1} \|_{L^{2}} + \| q_{2} - q_{1} \|_{L^{\infty}}
\end{split}
\end{equation}
where the last step follows from Cauchy-Schwartz and the a priori bounds
on $w_{i}, v_{i}, q_{i}$, $i \in \{1,2\}$. Therefore, 
%
%
\begin{equation}
  \label{pi}
\begin{split}
  \| Q_{2} - Q_{1} \|_{L^{\infty}} \lesssim_{r}\| w_{2} -w_{1} \|_{L^{2}} + \| v_{2} - v_{1} \|_{L^{2}} + \| q_{2} - q_{1} \|_{L^{\infty}}.
\end{split}
\end{equation}
%
%
Also, note that
%
%
%
%
\begin{equation}
  \label{I-II-split}
\begin{split}
  & Q_{2} - Q_{1}
  \\
  & = \frac{1}{2} \int_{\rr} \left\{ e^{-| \int_{\xi_{1}}^{\xi}
  q_{2}(\lambda) d \lambda|}\left[ \frac{3 - \gamma}{2} w_{2}^{2} q_{2} +
    \frac{\gamma}{2} v_{2}^{2} q_{2} \right] - e^{-| \int_{\xi_{1}}^{\xi}
    q_{1}(\lambda) d \lambda |} \left[ \frac{3 - \gamma}{2} w_{1}^{2}
      q_{1} + \frac{\gamma}{2} v_{1}^{2} q_{1} \right] \right\} d \xi_{1}
      \\
      & = I + II
\end{split}
\end{equation}
%
%
where
\begin{gather*}
  I = \int_{\rr} e^{-| \int_{\xi_{1}}^{\xi} q_{1}(\lambda) |} \left[ \frac{3-
  \gamma}{2}(w_{2}^{2} q_{2} - w_{1}^{2} q_{1}) +
  \frac{\gamma}{2}(v_{2}^{2} q_{2} - v_{1}^{2} q_{1}) \right]
  \\
  II = \int_{\rr} f\left[ \frac{3-\gamma}{2}w_{2}^{2}q_{2} + \frac{\gamma}{2}
    v_{2}^{2} q_{2} \right]d \xi_{1}, \quad
    f = e^{-| \int_{\xi_{1}}^{\xi} q_{2}(\lambda) d \lambda |} -
      e^{- | \int_{\xi_{1}}^{\xi} q_{1}(\lambda) d \lambda |}.
\end{gather*}
Observe that
%
\begin{equation}
  \label{q-l2-pre}
\begin{split}
  \| I \|_{L^{2}}^{2} & = \frac{1}{4} \int_{\rr} | \int_{\rr} e^{-| \int_{\xi_{1}}^{\xi}q_{1}(\lambda) d \lambda |}  \left[ \frac{3 - \gamma}{2}(w_{2}^{2} q_{2} -
  w_{1}^{2}q_{1}) + \frac{\gamma}{2}(v_{2}^{2} q_{2} - v_{1}^{2} q_{1}) \right]  d
  \xi_{1} |^{2} d \xi.
\end{split}
\end{equation}
%
%We shall first bound the exponential by analyzing it's argument. More precisley, recall that
%%
%%
%\begin{equation*}
%\begin{split}
  %\int_{0}^{y(\xi, t)} \left[ 1 + u(x, t) \right]^{2} dx = \xi.
%\end{split}
%\end{equation*}
%%
%%
%Differentiating both sides with respect to $\xi$ gives
%%
%%
%\begin{equation*}
%\begin{split}
  %y_{\xi} = \frac{1}{1 + y^{2}}.
%\end{split}
%\end{equation*}
%%%
%%%
%Fix $t$ and assume without loss of generality that $\xi \ge \xi_{1}$. If $y(s) \le 1$ for all $s \in [\xi_{1}, \xi]$, then
%%
%%
%\begin{equation*}
%\begin{split}
  %| y(\xi) - y(\xi_{1}) | 
  %& = | \int_{\xi_{1}}^{\xi} y_{s} ds|
  %\\
  %& = | \int_{\xi_{1}}^{\xi} \frac{1}{1 + y^{2}} ds |
  %\\
  %& \ge \frac{1}{2}| \xi - \xi_{1} |.
%\end{split}
%\end{equation*}
%%%
%%%
%Otherwise, since $y(s)$ is continuous and increasing, there exists $\xi_{1} \le \xi^{*} \le \xi$ such that $y(\xi^{*}) \ge 1$. Then
%%%
%%%
%\begin{equation*}
%\begin{split}
  %| y(\xi) - y(\xi_{1}) | & = | \int_{\left\{ s \in [\xi_{1}, \xi^{*}]: y \le 1
  %\right\}} \frac{1}{1 + y^{2}} ds + 
  %\int_{\left\{ s \in [\xi_{1}, \xi^{*}]: y \ge 1    
  %\right\}}  \frac{1}{1 + y^{2}} ds | 
  %\\
  %& \ge | \int_{\left \{ s \in [\xi_{1}, \xi^{*}]: y \le 1 \right\}}
  %\frac{1}{1 + y^{2}} ds + 
  %\int_{\left\{ s \in [\xi_{1}, \xi^{*}]: 1 \le y \le 2 \right\}} \frac{1}{1 +
  %y^{2}} ds \\
  %& \ge \frac{1}{2} | \xi^{*} - \xi_{1} | + \frac{1}{5} | \xi - \xi^{*} |
  %\\
  %& \ge \frac{2}{5} | \xi - \xi_{1} |.
%\end{split}
%\end{equation*}
%%
%%
%%
Since $0 < c_{1} < q_{1}$, we bound \eqref{q-l2-pre} by
%
%
\begin{equation*}
\begin{split}
  & \frac{1}{4} \int_{\rr} \left( \int_{\rr} | e^{-c_{1} | \xi - \xi_{1} |} \left[ \frac{3 - \gamma}{2}(w_{2}^{2} q_{2} - w_{1}^{2}q_{1}) + \frac{\gamma}{2}(v_{2}^{2} q_{2} - v_{1}^{2} q_{1}) \right] | d \xi_{1} \right)^{2} d \xi
  \\
  & = \frac{1}{4} \| e^{-c_{1} | \cdot |} * \left[ \frac{3 - \gamma}{2}(w_{2}^{2} q_{2} - w_{1}^{2}q_{1}) + \frac{\gamma}{2}(v_{2}^{2} q_{2} - v_{1}^{2} q_{1}) \right]  \|_{L^{2}}^{2}.
\end{split}
\end{equation*}
%
Applying Young's inequality, we obtain the bound
%
%
\begin{equation*}
\begin{split}
  \| e^{-c_{1} | x |    } \|_{L^{2}}^{2} \|  \frac{3 -
  \gamma}{2}(w_{2}^{2} q_{2} - w_{1}^{2}q_{1}) + \frac{\gamma}{2}(v_{2}^{2} q_{2} -
  v_{1}^{2} q_{1})\|_{L^{1}}^{2}
  & \lesssim \|  \frac{3 -
  \gamma}{2}(w_{2}^{2} q_{2} - w_{1}^{2}q_{1}) + \frac{\gamma}{2}(v_{2}^{2} q_{2} -
  v_{1}^{2} q_{1})\|_{L^{1}}^{2}
  \\
  & \lesssim_{r}
  (\| w_{2} - w_{1} \|_{L^{2}} + \| v_{2} - v_{1} \|_{L^{2}} + \| q_{2} - q_{1} \|_{L^{\infty}})^{2}
\end{split}
\end{equation*}
%
%
where the last step follows from \eqref{q-init-bound}. Therefore
%
%
\begin{equation}
  \label{I-est}
  \| I \|_{L^{2}} \lesssim_{r} \| w_{2} - w_{1} \|_{L^{2}}+ 
  \| v_{2} - v_{1} \|_{L^{2}} + \| q_{2} - q_{1} \|_{L^{\infty}}
\end{equation}
To bound $II$, we first note that
%
%
\begin{equation*}
\begin{split}
  f & = e^{- | \int_{\xi_{1}}^{\xi} q_{1}(\lambda) d \lambda|} \left[
    e^{| \int_{\xi_{1}}^{\xi} q_{1}(\lambda) |}e^{-|
      \int_{\xi_{1}}^{\xi}q_{2}(\lambda) d \lambda |} - 1
    \right]
    \\
    & = e^{- | \int_{\xi_{1}}^{\xi} q_{1}(\lambda) d \lambda |} \left(
    \sum_{n = 1}^{\infty} \frac{z^{n}}{n!}
    \right)
\end{split}
\end{equation*}
%
%
where
%
%
\begin{equation*}
\begin{split}
  z = | \int_{\xi_{1}}^{\xi} q_{1}(\lambda)d \lambda | - | \int_{\xi_{1}}^{\xi} q_{2}(\lambda) d \lambda |.
\end{split}
\end{equation*}
%
%
By the reverse triangle inequality
%
%
\begin{equation*}
\begin{split}
| z | & \le | \int_{\xi_{1}}^{\xi} q_{1}(\lambda)d \lambda  -  \int_{\xi_{1}}^{\xi} q_{2}(\lambda) d \lambda |
\\
& = | \int_{\xi_{1}}^{\xi} [q_{1} -  q_{2}](\lambda) d \lambda |
\\
& \le \| q_{1} - q_{2} \|_{L^{\infty}} | \xi - \xi_{1} |.
\end{split}
\end{equation*}
%
%
Due to the fact
that $0 < c_{1} <q_{1}$, we also have
%
%
\begin{equation*}
\begin{split}
  | \int_{\xi_{1}}^{\xi} q_{1}(\lambda) d \lambda | \ge c_{1} | \xi - \xi_{1} |.
\end{split}
\end{equation*}
%
%
Hence
%
%
%
\begin{equation*}
\begin{split}
  | f | &  \le e^{-c_{1} | \xi - \xi_{1} |} \sum_{n = 1}^{\infty} \frac{\left[ \| q_{2} - q_{1} \|_{L^{\infty}}| \xi - \xi_{1} | \right]^{n}}{n!}
  \\
  & \le e^{-c_{1} | \xi - \xi_{1} |} \| q_{2} - q_{1} \|_{L^{\infty}} \sum_{n =
  1}^{\infty} \frac{(\| q_{2} - q_{1} \|_{L^{\infty}})^{n-1}| \xi - \xi_{1} |^{n}
    }{n!}
\\
&  \le e^{-c_{1} | \xi - \xi_{1} |} \| q_{2} - q_{1} \|_{L^{\infty}} \sum_{n =
1}^{\infty} \frac{(2c_{2} )^{n-1}| \xi - \xi_{1} |^{n}
    }{n!}
    \\
    & \le e^{-c_{1}| \xi - \xi_{1} |} \| q_{2} - q_{1} \|_{L^{\infty}} e^{2c_{2}| \xi - \xi_{1} |}
    \\
    & = e^{-2c_{1} + c_{2}| \xi - \xi_{1} |} \| q_{2} - q_{1} \|_{L^{\infty}}.
\end{split}
\end{equation*}
%
%
Choose $c_{1}, c_{2}$ such that $-2c_{1} + c_{2} = -1/4$. Then we obtain
%
%
\begin{equation*}
\begin{split}
  | f | \le e^{-\frac{1}{4} | \xi - \xi_{1} |} \| q_{2} - q_{1} \|_{L^{\infty}}
\end{split}
\end{equation*}
%
%
and so
%
%
%
%
\begin{equation*}
\begin{split}
  \| II \|_{L^{2}}^{2}
  & = \| q_{2} - q_{1} \|_{L^{\infty}}^{2} \int_{\rr} |\int_{\rr} f\left[ \frac{3-\gamma}{2}w_{2}^{2}q_{2} + \frac{\gamma}{2}
    v_{2}^{2} q_{2} \right]d \xi_{1} | ^{2} d \xi  
    \\
    & \le \| q_{2} - q_{1} \|_{L^{\infty}}^{2} \int_{\rr} \left (\int_{\rr} e^{-\frac{1}{4}| \xi - \xi_{1} |} |\left[ \frac{3-\gamma}{2}w_{2}^{2}q_{2} + \frac{\gamma}{2}
    v_{2}^{2} q_{2} \right] | d \xi_{1} \right ) ^{2} d \xi 
    \\
    & = \| q_{2} - q_{1} \|_{L^{\infty}}^{2} \| e^{-\frac{1}{4} | \cdot |} * \left[ \frac{3 - \gamma}{2}w_{2}^{2} q_{2}
    + \frac{\gamma}{2}v_{2}^{2} q_{2}  \right]  \|_{L^{2}}^{2}.
  \end{split}
\end{equation*}
%
%
Applying Young's inequality, we bound this by
%
%
%
%
\begin{equation*}
\begin{split}
\| e^{-\frac{1}{4} | \cdot |} * \left[ \frac{3 - \gamma}{2}w_{2}^{2} q_{2}
    + \frac{\gamma}{2}v_{2}^{2} q_{2}  \right]  \|_{L^{2}}^{2}
    & \le \| e^{-\frac{1}{4}| x |} \|_{L^{2}}^{2} \| \frac{3 - \gamma}{2}w_{2}^{2} q_{2}
    + \frac{\gamma}{2}v_{2}^{2} q_{2}\|_{L^{1}}^{2}
    \\
    & \lesssim \| q_{2} \|_{L^{2}}^{2} \| \frac{3 - \gamma}{2} w_{2}^{2} + \frac{\gamma}{2} v_{2}^{2} \|_{L^{1}}
    \\
    & \lesssim_{r} 1 
\end{split}
\end{equation*}
%
since $w_{2}, v_{2}, q_{2} \in B_{L^{2}}(r)$.
%
Therefore
%
%
\begin{equation}
  \label{II-est}
\begin{split}
  \| II \|_{L^{2}} \lesssim_{r} \| q_{2} - q_{1} \|_{L^{\infty}}.
\end{split}
\end{equation}
%
%
Combining \eqref{I-est}, \eqref{II-est} and recalling \eqref{I-II-split}, we obtain
%
%
%
%
\begin{equation}
  \label{yi}
\begin{split}
  \| Q_{2} - Q_{1} \|_{L^{2}} \lesssim_{r} \| w_{2} - w_{1} \|_{L^{2}} + \| v_{2} - v_{1} \|_{L^{2}} + \| q_{2} - q_{1} \|_{L^{\infty}}
\end{split}
\end{equation}
%
%
Combining \eqref{pi} and \eqref{yi}, we see that
%
%
\begin{equation}
  \label{Q-diff-fin-est}
\begin{split}
  \| Q_{2} - Q_{1} \|_{L^{2} \cap L^{\infty}} \lesssim_{r} \| w_{2} - w_{1} \|_{L^{2}} + \| v_{2} - v_{1}
  \|_{L^{2}} + \| q_{2} - q_{1} \|_{L^{\infty}}.
\end{split}
\end{equation}
%
%
Next, we turn our attention to the term 
$(R_{2} - R_{1})$. Recall that if
%
%
\begin{equation*}
\begin{split}
F(\xi) = \int_{a(\xi)}^{b(\xi)} f(\xi, \tau) d \tau
\end{split}
\end{equation*}
%
%
with $a(\xi), b(\xi)$ continuous over some region $[\xi_{0}, \xi_{1}]$ and $f, f_{\xi}$ continuous over $[\xi_{0}, \xi_{1}] \times [\tau_{0}, \tau_{1}]$, then by the Leibniz integral rule, we have
%
%
\begin{equation*}
\begin{split}
F'(\xi) = b'(\xi) f(b(\xi)) - a'(\xi) f(a(\xi)) + \int_{a(\xi)}^{b(\xi)} f_{\xi}(\xi, \tau) d \tau.
\end{split}
\end{equation*}
%
%
Recall the definition of $R$ in \eqref{R-def}, and let $f(\xi, \xi_{1}, t)$ denote
its integrand. Then
%
%
\begin{equation*}
\begin{split}
\frac{d}{d \xi} \int_{\xi}^{\infty}f(\xi, \xi_{1}, t) d \xi_{1}
& = -f(\xi, \xi, t) + \int_{\xi}^{\infty} f_{\xi}(\xi, \xi_{1}, t) d \xi_{1}.
\end{split}
\end{equation*}
%
%
Similarly
\begin{equation*}
\begin{split}
\frac{d}{d \xi} \int_{-\infty}^{\xi}f(\xi, \xi_{1}, t) d \xi_{1}
& = f(\xi, \xi, t) + \int_{-\infty}^{\xi} f_{\xi}(\xi, \xi_{1}, t) d \xi_{1}.
\end{split}
\end{equation*}
Note that
\begin{gather*}
    f(\xi, \xi, t) = \left [ \frac{3 - \gamma}{2}w^{2}q + \frac{\gamma}{2} v^{2} q \right ](\xi, t)
\end{gather*}
and
%
%
\begin{equation}
\label{oii}
\begin{split}
f_{\xi}(\xi, \xi_{1}, t) = \sgn \left [ \int_{\xi_{1}}^{\xi} q(\lambda, t) d \lambda \right ]q(\xi, t) \left [ \frac{3- \gamma}{2} w^{2} q + \frac{\gamma}{2} v^{2}q \right ] (\xi_{1}, t).
\end{split}
\end{equation}
%
%
Now, from \eqref{potent-def}, it follows that $y(\xi, t)$ is an increasing function of $\xi$. Therefore, 
%
%
\begin{equation*}
\begin{split}
\sgn \left [ \int_{\xi_1}^{\xi} q(\lambda, t) d \lambda \right ]  = \sgn(\xi - \xi_1)
\end{split}
\end{equation*}
%
%
which in conjunction with \eqref{oii} gives
%
%
\begin{equation*}
\begin{split}
\int_{\xi}^{\infty} f_{\xi}(\xi, \xi_{1}, t) d \xi_{1} - \int_{-\infty}^{\xi} f_{\xi}(\xi, \xi_{1}, t) d \xi_{1}
& = q(\xi, t) \int_{-\infty}^{\infty} \left [ \frac{3- \gamma}{2} w^{2} q + \frac{\gamma}{2} v^{2}q \right ] (\xi_{1}, t).
\\
& = q Q(\xi, t).
\end{split}
\end{equation*}
%
%
Therefore,
%
%
\begin{equation*}
\begin{split}
R_{\xi}(\xi, t) = \frac{1}{2} \left [ \frac{3- \gamma}{2} w^{2} + \frac{\gamma}{2} v^{2} + Q \right ]q(\xi, t)
\end{split}
\end{equation*}
%
and so
%
%
\begin{equation*}
\begin{split}
  (R_{2})_{\xi} - (R_{1})_{\xi} 
  & \simeq \left[ \frac{3 - \gamma}{2} w_{2}^{2} +
  \frac{\gamma}{2}v_{2}^{2} + Q_{2}  \right]q_{2} - \left[ \frac{3 - \gamma}{2}
  w_{1}^{2} + \frac{\gamma}{2}v_{1}^{2} + Q_{1}  \right]q_{1}  
  \\
  & = \left[ \frac{3 - \gamma}{2} (w_{2}^{2} - w_{1}^{2}) +
  \frac{\gamma}{2}(v_{2}^{2} - v_{1}^{2}) + Q_{2} - Q_{1}  \right]q_{1} + \left[
  \frac{3 + \gamma}{2} w_{2}^{2} + \frac{\gamma}{2}v_{2}^{2} + Q_{2}
  \right](q_{2} - q_{1}).  
\end{split}
\end{equation*}
%
By H\"older and the algebra property of $L^{\infty}$, we have the estimates
%
%
%
\begin{equation*}
\begin{split}
  \| (w_{2}^{2} - w_{1}^{2})q_{1} \|_{L^{2}}
  & = \| (w_{2} - w_{1})(w_{2} + w_{1})q \|_{L^{2}} \le \| w_{2} - w_{1} \|_{L^{2}} 
  \| w_{2} + w_{1} \|_{L^{\infty}} \| q_{2} \|_{L^{\infty}}
  \\
  & \lesssim_{r} \| w_{2} - w_{1} \|_{L^{2}}
\end{split}
\end{equation*}
%
%
and 
%
\begin{equation*}
\begin{split}
  \| (v_{2}^{2} - v_{1}^{2})q_{1} \|_{L^{2}}
  & = \| (v_{2} - v_{1})(v_{2} + v_{1})q \|_{L^{2}}
  \\
  & \le \| v_{2} - v_{1} \|_{L^{2}} 
  \| v_{2} + v_{1} \|_{L^{\infty}} \| q_{2} \|_{L^{\infty}}
  \\
  & \lesssim_{r} \| v_{2} - v_{1} \|_{L^{2}}.
\end{split}
\end{equation*}
Applying H\"older and \eqref{yi}, we also have
%
%
%
\begin{equation*}
\begin{split}
  \| (Q_{2} - Q_{1})q_{1} \|_{L^{2}} 
  & \le \| Q_{2} - Q_{1} \|_{L^{2}} \| q_{1} \|_{L^{\infty}}
  \\
  & \le (\| v_{2} - v_{1} \|_{L^{2}} + \| q_{2} - q_{1} \|_{L^{\infty}})\| q_{1} \|_{L^{\infty}}
  \\
  & \lesssim_{r} \| v_{2} - v_{1} \|_{L^{2}} + \| q_{2} - q_{1} \|_{L^{\infty}}.
\end{split}
\end{equation*}
%
%
Lastly,
%
\begin{equation*}
\begin{split}
 \| \left[
  \frac{3 + \gamma}{2} w_{2}^{2} + \frac{\gamma}{2}v_{2}^{2} + Q_{2}
  \right](q_{2} - q_{1}) \|_{L^{\infty}}
  & \lesssim_{r} \| q_{2} - q_{1} \|_{L^{\infty}} 
\end{split}
\end{equation*}
%
%
Hence
%
%
\begin{equation*}
\begin{split}
  \| (R_{2})_{\xi} - (R_{1})_{\xi} \|_{L^{\infty}} \lesssim_{r} \| w_{2} - w_{1} \|_{L^{\infty}} + \| v_{2} - v_{1} \|_{L^{\infty}}
+ \| q_{2} - q_{1} \|_{L^{\infty}}.
\end{split}
\end{equation*}
%
%
which implies
\begin{equation}
  \label{R-diff-fin-est}
\begin{split}
  \| R_{2} - R_{1} \|_{H^{1}} \lesssim_{r} \| w_{2} - w_{1} \|_{L^{\infty}} + \| v_{2} - v_{1} \|_{L^{\infty}}
+ \| q_{2} - q_{1} \|_{L^{\infty}}.
\end{split}
\end{equation}
%
%
Recalling \eqref{lip-diff}, combining estimates \eqref{1aa}-\eqref{3aa},
\eqref{Q-diff-fin-est}, and \eqref{R-diff-fin-est}, and applying the estimates
$\| f \|_{L^{2}} \le \| f \|_{H^{1}}$, $\| f \|_{L^{\infty}} \lesssim \| f \|_{H^{1}}$ and $\| f \|_{L^{\infty}} \le \| f
\|_{L^{2} \cap L^{\infty}}$, we conclude that
%
%
%
\begin{equation*}
\begin{split}
    \|F(w_{2}, v_{2}, q_{2}) - F(w_{1}, v_{1}, q_{1})\|_{Y}
    & \lesssim_{r} \| w_{2} - w_{1} \|_{H^{1}} + \| v_{2} - v_{1} \|_{L^{2} \cap L^{\infty}}  + \| q_{2} - q_{1} \|_{L^{\infty}}
    \\
    & = \| (w_{2}, v_{2}, q_{2}) - (w_{1}, v_{1}, q_{1}) \|_{Y}
\end{split}
\end{equation*}
%
%
ending the proof.
%
\end{proof} 
%
%
%
%%%%%%%%%%%%%%%%%%%%%%%%%%%%%%%%%%%%%%%%%%%%%%%%%%%%%
%
%
%				norm equivalence
%
%
%%%%%%%%%%%%%%%%%%%%%%%%%%%%%%%%%%%%%%%%%%%%%%%%%%%%%
%
%
\begin{lemma}
We have
%
%
\begin{equation*}
\begin{split}
  \| u(y(\xi)) \|_{H^{s}_{\xi}} \sim \| u(\xi) \|_{H^{s}_{\xi}}.
\end{split}
\end{equation*}
%
%
%
%
%
%
\label{lem:norm-equiv}
\end{lemma}
%
%
%
%
\begin{proof}
For non-integer $s \ge 0$, write
%
%
\begin{equation*}
\begin{split}
  s = k + \sigma, \ k \in \mathbb{N}, \ 0 < \sigma < 1. 
\end{split}
\end{equation*}
%
%
Then 
%
%
\begin{equation*}
\begin{split}
  \| u(y(\xi)) \|_{H^{s}_{\xi}} = \sum_{j = 0}^{k} \| \p_{\xi}^{j} u (y(\xi))
  \|_{L^{2}_{\xi}} + \int_{\rr} \int_{\rr} \frac{| u(y(\xi)) - u(y(\eta))
    |^{2}}{| \xi - \eta |^{1 + 2 \sigma}} d \xi d \eta.
\end{split}
\end{equation*}
%
%
For the second term, we use the change of variable $z_{1} = y(\xi)$ and $z_{2} = y(\eta)$. Since the map $\xi \mapsto y(\xi)$ is a $C^{1}$ diffeomorphism on $\rr$, we have
%
%
\begin{gather*}
   \xi= y^{-1}(z_{1}), \quad \eta = y^{-1}(z_{2})
  \\
   d \xi(z_{1}) = \frac{dz_{1}}{y'(y^{-1}(z_{1}))}, \quad d \xi(z_{2}) = \frac{dz_{2}}{y'(y^{-1}(z_{2}))}
\end{gather*}
%
%
and so
%
%
\begin{equation}
  \label{hg}
\begin{split}
\int_{\rr} \int_{\rr} \frac{| u(y(\xi)) - u(y(\eta))
    |^{2}}{| \xi - \eta |^{1 + 2 \sigma}} d \xi d \eta & = \int_{\rr} \int_{\rr}
    \frac{| u(z_{1}) - u(z_{2}) |^{2}}{| y^{-1}(z_{1}) - y^{-1}(z_{2}) |^{1 + 2
    \sigma}} \frac{1}{y'(y^{-1}(z_{1})) y'(y^{-1}(z_{2}))} d z_{1} d z_{2}
    \\
    & \sim \int_{\rr} \int_{\rr} \frac{| u(z_{1}) - u(z_{2}) |^{2}}{|
      y^{-1}(z_{1}) - y^{-1}(z_{2}) |^{1 + 2 \sigma}} d z_{1} dz_{2}
\end{split}
\end{equation}
%
%
where the last step follows from the fact that since $y$ is $C^{1}$
diffeomorphic  and non-negative, we can find $c_{1}, c_{2} > 0$ such that
%
%
\begin{equation*}
\begin{split}
  c_{1} \le \inf y' \le \sup y' \le c_{2}.
\end{split}
\end{equation*}
%
%
From this fact, it also follows that
%
%
\begin{equation*}
\begin{split}
  | y^{-1}(z_{1}) - y^{-1}(z_{2}) | = | \int_{z_{2}}^{z_{1}} (y^{-1})'(z) dz | \sim | z_{1} - z_{2} |.
\end{split}
\end{equation*}
%
%
Applying this relation to \eqref{hg}, we see that
%
%
\begin{equation*}
\begin{split}
\int_{\rr} \int_{\rr} \frac{| u(y(\xi)) - u(y(\eta))
    |^{2}}{| \xi - \eta |^{1 + 2 \sigma}} d \xi d \eta
    \sim 
\int_{\rr} \int_{\rr} \frac{| u(z_{1}) - u(z_{2}) |^{2}}{| z_{1} - z_{2} |^{1 + 2 \sigma}} d z_{1} dz_{2}.
\end{split}
\end{equation*}
%
Now
\begin{gather*}
  [u(y(\xi))]' = u'(y(\xi))y'
  \\
  [u(y(\xi))]'' = u'(y(\xi))y'' + u''(y(\xi))(y')^{2}
  \\
[u(y(\xi))]''' = u'(y(\xi))y''' + 3 y' y'' u''(y(\xi)) + u'''(y(\xi))y'y'y'.
\end{gather*}
In general,
%
%
\begin{equation*}
\begin{split}
  [u(y(\xi))]^{m}
  & = u^{(1)}(y(\xi))y^{(m)} + c_{1} u^{(2)}(y(\xi)) y^{(m-1)}y^{(1)} + c_{2} u^{(3)}(y(\xi)) y^{(m-2)}y^{(2)} 
  \\
  & +\cdots+ c_{m} u^{(m)}(y(\xi)) \underbrace{y^{(1)}\ldots y^{(1)}}_{\text{m copies}}.
\end{split}
\end{equation*}
%
%
%
\end{proof}
%
%
%
%
%
%
\providecommand{\bysame}{\leavevmode\hbox to3em{\hrulefill}\thinspace}
\providecommand{\MR}{\relax\ifhmode\unskip\space\fi MR }
% \MRhref is called by the amsart/book/proc definition of \MR.
\providecommand{\MRhref}[2]{%
  \href{http://www.ams.org/mathscinet-getitem?mr=#1}{#2}
}
\providecommand{\href}[2]{#2}
\begin{thebibliography}{HKM09}

\bibitem[BBM72]{Benjamin_1972_Model-equations}
T.~B. Benjamin, J.~L. Bona, and J.~J. Mahony, \emph{Model equations for long
  waves in nonlinear dispersive systems}, Philos. Trans. Roy. Soc. London Ser.
  A \textbf{272} (1972), no.~1220, 47--78.

\bibitem[BC07]{Bressan_2007_Global-conserva}
A.~Bressan and A.~Constantin, \emph{Global conservative solutions of the
  camassa-holm equation}, Arch. Ration. Mech. Anal. \textbf{183} (2007), no.~2,
  215--239.

\bibitem[BCK06]{Bendahmane_2006_Hsp-1-perturbat}
M.~Bendahmane, G.~M. Coclite, and K.~H. Karlsen, \emph{$h^1$-perturbations of
  smooth solutions for a weakly dissipative hyperelastic-rod wave equation},
  Mediterr. J. Math. \textbf{3} (2006), no.~3-4, 419--432.

\bibitem[Ben72]{Benjamin_1972_The-stability-o}
T.~B. Benjamin, \emph{The stability of solitary waves}, Proc. Roy. Soc.
  (London) Ser. A \textbf{328} (1972), 153--183.

\bibitem[BL76]{bergh1976interpolation}
J.~Bergh and J.~L{\"o}fstr{\"o}m, \emph{Interpolation spaces: an introduction},
  vol. 223, Springer-Verlag, 1976.

\bibitem[BS88]{Bennett:1988ys}
C.~Bennett and R.~Sharpley, \emph{Interpolation of operators}, vol. 129,
  Academic Pr, 1988.

\bibitem[Che11]{Chen:2011fk}
R.~M. Chen, \emph{The h{\"o}lder continuity of the solution map to the
  $b$-family equation in weak topology}, 2011.

\bibitem[CHK05]{Coclite_2005_Global-weak-sol}
G.~M. Coclite, H.~Holden, and K.~H. Karlsen, \emph{Global weak solutions to a
  generalized hyperelastic-rod wave equation}, SIAM J. Math. Anal. \textbf{37}
  (2005), no.~4, 1044--1069 (electronic).

\bibitem[CS00]{Constantin_2000_Stability-of-a-}
A.~Constantin and W.~A. Strauss, \emph{Stability of a class of solitary waves
  in compressible elastic rods}, Phys. Lett. A \textbf{270} (2000), no.~3-4,
  140--148.

\bibitem[Dai98]{Dai_1998_Model-equations}
H.-H. Dai, \emph{Model equations for nonlinear dispersive waves in a
  compressible mooney-rivlin rod}, Acta Mech. \textbf{127} (1998), no.~1-4,
  193--207.

\bibitem[DDH00]{Dai_2000_Head-on-collisi}
H.-H. Dai, S.~Dai, and Y.~Huo, \emph{Head-on collision between two solitary
  waves in a compressible mooney-rivlin elastic rod}, Wave Motion \textbf{32}
  (2000), no.~2, 93--111.

\bibitem[DH00]{Dai_2000_Solitary-shock-}
H.-H. Dai and Y.~Huo, \emph{Solitary shock waves and other travelling waves in
  a general compressible hyperelastic rod}, R. Soc. Lond. Proc. Ser. A Math.
  Phys. Eng. Sci. \textbf{456} (2000), no.~1994, 331--363.

\bibitem[HKM09]{Himonas_2009_Non-uniform-dep-per}
A.~Himonas, C.~E. Kenig, and G.~Misiolek, \emph{Non-uniform dependence for the
  periodic ch equation.}, To appear in Communications in Partial Differential
  Equations (2009).

\bibitem[HR07]{Holden_2007_Global-conserva}
H.~Holden and X.~Raynaud, \emph{Global conservative solutions of the
  generalized hyperelastic-rod wave equation}, J. Differential Equations
  \textbf{233} (2007), no.~2, 448--484.

\bibitem[Kar10]{Karapetyan:2010fk}
D.~Karapetyan, \emph{Non-uniform dependence and well-posedness for the
  hyperelastic rod equation}, J. Differential Equations \textbf{249} (2010),
  no.~4, 796--826. \MR{2652154 (2011g:35352)}

\bibitem[Kat75]{Kato_1975_Quasi-linear-eq}
T.~Kato, \emph{Quasi-linear equations of evolution, with applications to
  partial differential equations}, 1975, pp.~25--70. Lecture Notes in Math.,
  Vol. 448.

\bibitem[Len06]{Lenells_2006_Traveling-waves}
J.~Lenells, \emph{Traveling waves in compressible elastic rods}, Discrete
  Contin. Dyn. Syst. Ser. B \textbf{6} (2006), no.~1, 151--167 (electronic).

\bibitem[Mus07]{Mustafa_2007_Global-conserva}
O.~G. Mustafa, \emph{Global conservative solutions of the hyperelastic rod
  equation}, Int. Math. Res. Not. IMRN (2007), no.~13, Art. ID rnm040, 26.

\bibitem[RB01]{Rodriguez-Blanco_2001_On-the-Cauchy-p}
G.~Rodr\'iguez-Blanco, \emph{On the cauchy problem for the camassa-holm
  equation}, Nonlinear Anal. \textbf{46} (2001), no.~3, Ser. A: Theory Methods,
  309--327.

\bibitem[Tay91]{Taylor_1991_Pseudodifferent}
M.~E. Taylor, \emph{Pseudodifferential operators and nonlinear pde}, vol. 100,
  1991.

\bibitem[Tay95]{Taylor:1995kx}
M.~E. Taylor, \emph{Partial differential equations i, basic theory}, Applied
  Mathematical Sciences, vol. 115, Springer-Verlag, New York, 1995.

\bibitem[Yin03]{Yin_2003_On-the-Cauchy-p}
Z.~Yin, \emph{On the cauchy problem for a nonlinearly dispersive wave
  equation}, J. Nonlinear Math. Phys. \textbf{10} (2003), no.~1, 10--15.

\bibitem[Zho05]{Zhou_2005_Local-well-pose}
Y.~Zhou, \emph{Local well-posedness and blow-up criteria of solutions for a rod
  equation}, Math. Nachr. \textbf{278} (2005), no.~14, 1726--1739.

\end{thebibliography}
%
%
%
%
%\bibliographystyle{amsalpha-custom}
%\bibliography{/Users/davidkarapetyan/math/bib-files/references}

\end{document}
