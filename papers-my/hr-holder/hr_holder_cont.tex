%
\documentclass[12pt,reqno]{amsart}
\usepackage{amssymb}
\usepackage{appendix}
%\usepackage[notref, notcite]{showkeys}
\usepackage{tikz}
\usepackage[showonlyrefs=true]{mathtools} %amsmath extension package
\usepackage{cancel}  %for cancelling terms explicity on pdf
\usepackage[normalem]{ulem} %for striking out non-math text
\usepackage{yhmath}   %makes fourier transform look nicer, among other things
\usepackage{framed}  %for framing remarks, theorems, etc.
\usepackage{enumerate} %to change enumerate symbols
\usepackage[margin=2.5cm]{geometry}  %page layout
\setcounter{tocdepth}{1} %must come before secnumdepth--else, pain
\setcounter{secnumdepth}{1} %number only sections, not subsections
%\usepackage[pdftex]{graphicx} %for importing pictures into latex--pdf compilation
\numberwithin{equation}{section}  %eliminate need for keeping track of counters
\numberwithin{figure}{section}
\setlength{\parindent}{0in} %no indentation of paragraphs after section title
\renewcommand{\baselinestretch}{1.1} %increases vert spacing of text
%
\usepackage{hyperref}
\hypersetup{colorlinks=true,
linkcolor=blue,
citecolor=blue,
urlcolor=blue,
}
%
\DeclareMathOperator{\sgn}{sgn}
%
\newcommand{\ds}{\displaystyle}
\newcommand{\ts}{\textstyle}
\newcommand{\nin}{\noindent}
\newcommand{\rr}{\mathbb{R}}
\newcommand{\nn}{\mathbb{N}}
\newcommand{\zz}{\mathbb{Z}}
\newcommand{\cc}{\mathbb{C}}
\newcommand{\ci}{\mathbb{T}}
\newcommand{\zzdot}{\dot{\zz}}
\newcommand{\wh}{\widehat}
\newcommand{\p}{\partial}
\newcommand{\ee}{\varepsilon}
\newcommand{\vp}{\varphi}
\newcommand{\wt}{\widetilde}
%
%
%
%
\newtheorem{theorem}{Theorem}
\newtheorem{lemma}{Lemma}
\newtheorem{corollary}[theorem]{Corollary}
\newtheorem{claim}[theorem]{Claim}
\newtheorem{prop}[theorem]{Proposition}
\newtheorem{proposition}[theorem]{Proposition}
\newtheorem{no}[theorem]{Notation}
\newtheorem{definition}[theorem]{Definition}
\newtheorem{remark}[theorem]{Remark}
\newtheorem{examp}{Example}[section]
\newtheorem {exercise}[theorem]{Exercise}
%
%\makeatletter \renewenvironment{proof}[1][\proofname] {\par\pushQED{\qed}\normalfont\topsep6\p@\@plus6\p@\relax\trivlist\item[\hskip\labelsep\bfseries#1\@addpunct{.}]\ignorespaces}{\popQED\endtrivlist\@endpefalse} \makeatother%
%makes proof environment bold instead of italic
\newcommand{\uol}{u^\omega_\lambda}
\newcommand{\lbar}{\bar{l}}
\renewcommand{\l}{\lambda}
\newcommand{\R}{\mathbb R}
\newcommand{\RR}{\mathcal R}
\newcommand{\al}{\alpha}
\newcommand{\ve}{q}
\newcommand{\tg}{{tan}}
\newcommand{\m}{q}
\newcommand{\N}{N}
\newcommand{\ta}{{\tilde{a}}}
\newcommand{\tb}{{\tilde{b}}}
\newcommand{\tc}{{\tilde{c}}}
\newcommand{\tS}{{\tilde S}}
\newcommand{\tP}{{\tilde P}}
\newcommand{\tu}{{\tilde{u}}}
\newcommand{\tw}{{\tilde{w}}}
\newcommand{\tA}{{\tilde{A}}}
\newcommand{\tX}{{\tilde{X}}}
\newcommand{\tphi}{{\tilde{\phi}}}
\synctex=1
\begin{document}
\title[H\"older Continuity of the Data to Solution Map for HR]{H\"older Continuity of the Data to Solution Map for HR in the
Weak Topology}
\author{David Karapetyan}
\address{Department of Mathematics  \\
    University  of Notre Dame\\
        Notre Dame, IN 46556 }
        \date{\today}
        %
        \maketitle
        %
        %
        %
        %
        %
        %        
        %
        \section{Introduction}
%
%
%
We consider the hyperelastic rod (HR) Cauchy problem
\begin{gather}
 \p_t u =  -\gamma u \p_x u -
 \p_{x} (1 - \p_{x}^{2})^{-1} \left[ \frac{3-\gamma}{2}u^2 +
\frac{\gamma}{2} \left( \p_x u \right)^2
\right],
\label{hyperelastic-rod-equation}
\\
 u(x,0) = u_0(x), \; \; x \in \rr, \; \; t \in \rr
\label{init-cond}
\end{gather}
%
%
where  $\gamma$  is a  nonzero constant. The HR equation was first
derived by Dai in \cite{Dai_1998_Model-equations} as a one-dimensional 
model for finite-length and
small-amplitude axial deformation waves in thin cylindrical
rods composed of a compressible Mooney-Rivlin
material. The derivation relied upon a reductive perturbation technique, 
and took into account the nonlinear dispersion of pulses propagating 
along a rod. It was assumed that each cross-section of the rod is 
subject to a stretching and rotation in space. The solution $u(x,t)$ to the 
HR equation represents the radial stretch relative
to a pre-stressed state, while $\gamma$ is a fixed constant depending upon 
the pre-stress and the material used in
the rod, with values ranging from $- 29.4760$ to $3.4174$.
%
\\
\\
The well-posedness of the HR equation has been studied by several authors. 
In Yin \cite{Yin_2003_On-the-Cauchy-p} and Zhou 
\cite{Zhou_2005_Local-well-pose}, a proof of local well-posedness in Sobolev 
spaces $H^s$,  $s > 3/2$, is described  on the line and the circle, respectively. 
Their approach is to rewrite the HR equation   
in its non-local form, and then to verify the conditions needed to apply 
Kato's semi-group theory \cite{Kato_1975_Quasi-linear-eq}. Using an alternative,
Galerkin type method outlined in Taylor \cite{Taylor_1991_Pseudodifferent}, local
well-posedness in Sobolev spaces $H^s$,  $s > 3/2$ on the line and the circle is
established in \cite{Karapetyan:2010fk}. 
For details on how this is done for CH ($\gamma =1$) on the line, see Rodriguez-Blanco 
\cite{Rodriguez-Blanco_2001_On-the-Cauchy-p}. Blow-up criteria 
is also investigated in \cite{Yin_2003_On-the-Cauchy-p} and 
\cite{Zhou_2005_Local-well-pose}, as well as by Constantin and Strauss 
\cite{Constantin_2000_Stability-of-a-}. 
\\
\\
Setting $\gamma = 0$ gives the celebrated 
BBM equation, which was proposed by 
Benjamin, Bona, and Mahony 
\cite{Benjamin_1972_Model-equations} as a model for 
the unidirectional evolution of long waves.
Solitary-wave solutions to this 
equation are global and orbitally stable (see Benjamin 
\cite{Benjamin_1972_The-stability-o}, 
\cite{Benjamin_1972_Model-equations}, and 
\cite{Constantin_2000_Stability-of-a-}).
For more general $\gamma$, the existence of global 
solutions to HR on the line with constant $H^1$ energy
was proved recently by Mustafa \cite{Mustafa_2007_Global-conserva}
using the approach developed by Bressan and 
Constantin in \cite{Bressan_2007_Global-conserva}. Using a vanishing 
viscosity argument, Coclite, 
Holden, and Karlsen \cite{Coclite_2005_Global-weak-sol}
established existence of a strongly continuous semigroup of global 
weak solutions of HR on the line for initial data in $H^1$.
Bendahmane, Coclite, and Karlsen 
\cite{Bendahmane_2006_Hsp-1-perturbat} extended this result to traveling 
wave solutions that are supersonic solitary shockwaves.
For more information on the existence of global solutions to the HR
equation, see Holden and Raynaud \cite{Holden_2007_Global-conserva}
and \cite{Yin_2003_On-the-Cauchy-p}. 
\\
\\
There is a variety of traveling wave solutions to the HR equation that can be 
obtained using various combinations of peaks, cusps, compactons, 
fractal-like waves, and plateaus (see Lenells 
\cite{Lenells_2006_Traveling-waves}). Orbital stability of solitary wave 
solutions was proved in \cite{Constantin_2000_Stability-of-a-}.
Solitary shock wave formation was 
analyzed in Dai and Huo \cite{Dai_2000_Solitary-shock-} using traveling 
wave solutions of the HR equation to derive a system of ordinary differential 
equations, with a vertical singular line in the phase plane corresponding with the 
formation of shock waves. Head-on collisions between two solitary 
waves was investigated in the work of Hui-Hui Dai, 
Shiqiang Dai, and Huo \cite{Dai_2000_Head-on-collisi} using a reductive 
perturbation method coupled with the technique of strained coordinates. 
\\
\\
In this work we study the continuity properties of the data-to-solution map for
the HR 
equation, expanding upon the work in \cite{Karapetyan:2010fk}, in which it was
shown that the data-to-solution map is not better than continuous in Sobolev
spaces $H^{s}$ on the line
and the circle. More precisely, following \cite{Chen:2011fk} we show the
following result:
%
%
\begin{theorem}
For $\gamma \neq 0$, the
data to solution map for HR is H\"older continuous from $B_{H^{s}}(R)$ (in
the topology of $H^{r}$) to $C([0, T], H^{r})$, where $T = T(R)$, for $s >
3/2$, $-1 \le r < s$. More
precisely, consider the sets
  %
  %
  \begin{equation*}
  \begin{split}
      & \Omega_{1} = \left\{ (s, \ r) \in \rr^{2}:
     \ s>3/2, \ -1 \le r \le s-1, \ s + r \ge 2  \right\}
    \\
    & \Omega_{2} = \left\{  (s, \ r) \in \rr^{2}:
     \ s>3/2, \ -1 \le r < 2-s \right\}
    \\
    & \Omega_{3} = \left\{ (s, \ r) \in \rr^{2}:
    \  s>3/2, \  s-1 < r < s  \right\}.
    \end{split}
\end{equation*}
  %
  %
\label{thm:main-thm}
\end{theorem}
%
\begin{center}
\begin{tikzpicture}[scale=1.5]
% Draw thin grid lines with color 40% gray + 60% white

% Draw x and y axis lines
\draw [->] (0,0) -- (3,0) node [below] {$s$};
\draw [->] (0,-1) -- (0,3) node [left] {$r$};
\draw [->, dashed] (0,0) -- (3,3);
\draw [->, dashed] (0,-1) -- (3,2);
\draw [->, dashed] (0,2) -- (3,-1);
\draw [->, dashed] (0,-1) -- (3,-1);
\draw [->, dashed] (3/2,-1) -- (3/2, 3);
\fill[color=green, fill opacity=0.3] (1.5, 0.5) -- (3,2) -- (3,0) -- (3,-1);
\fill[color=red, fill opacity=0.3] (1.5, 0.5) -- (1.5,1.5) -- (3,3) -- (3,2);
\fill[color=blue, fill opacity=0.3] (1.5, 0.5) -- (1.5, -1) -- (3, -1);


\foreach \x/\xtext in {1, 2}
    \draw[shift={(\x,0)}]  node[below] {$\xtext$};
\foreach \y/\ytext in {-1, 1, 2}
    \draw[shift={(0,\y)}]  node[left] {$\ytext$};
    \draw (2,1.5) node {$\Omega_{3}$};
    \draw (2,0.5) node {$\Omega_{1}$};
    \draw (2,-0.5) node {$\Omega_{2}$};
\end{tikzpicture}
\end{center}
%
%
Then for two initial data $u_{0}, v_{0} \in B_{H^{s}}(R)$, there exist unique
corresponding solutions $u(x,t), v(x,t)$ for $0 \le t \le T= T(R)$ to the
HR equation \eqref{hyperelastic-rod-equation} which satisfy 
%
%
\begin{equation*}
\begin{split}
  \| u(t) - v(t) \|_{H^{r}} \le C \| u_{0} - v_{0} \|_{H^{r}}^{\alpha(s, r)},
  \quad 0
  \le t \le T
\end{split}
\end{equation*}
%
%
where 
%
%
\begin{equation*}
\begin{split}
\alpha = 
\begin{cases}
   1, \quad & (s,r) \in \Omega_{1} 
  \\
   2(s-1)/(s-r),  \quad & (s, r) \in \Omega_{2}
  \\
   s-r, \quad & (s, r) \in \Omega_{3}.
\end{cases}
\end{split}
\end{equation*}
%

%
%%%%%%%%%%%%%%%%%%%%%%%%%%%%%%%%%%%%%%%%%%%%%%%%%%%%%
%
%
%                Main theorem
%
%
%%%%%%%%%%%%%%%%%%%%%%%%%%%%%%%%%%%%%%%%%%%%%%%%%%%%%
%
%
We remark that this result improves upon the H\"older continuity result in
\cite{Chen:2011fk} (there it is for $0 \le r < s$) by a full degree in $r$. 
%
%
%
%
\section{Proof of Theorem~\ref{thm:main-thm}}
%
%
The proof in the periodic case is analogous to that in the non-periodic case.
Hence, we restrict our attention to the non-periodic case. 
%
%
%
\subsection{Region $\Omega_{1}$} 
\label{ssec:reg-m-imp}
%
%
Let $u_{0}(x), v_{0}(x)
\in B_{H^{s}}(R)$, $s > 3/2$ be two initial datum. Then from
the well-posedness theory for HR (see \cite{Karapetyan:2010fk}), we
know that there exist unique corresponding solutions $u, v \in C(I,
B_{H^{s}}(2R))$ to \eqref{hyperelastic-rod-equation}.
Set $v=u-w$. Then $v$ solves the Cauchy-problem
%
%
\begin{align}
	\label{uniqueness-exp}
& \p_t v
=  -\frac{\gamma}{2} \p_x [v(u + w)] 
\\
\notag
& \phantom{\p_t v = }-\p_x (1 - \p_{x}^{2})^{-1} \left\{
\frac{3-\gamma}{2}[v(u+w)] + \frac{\gamma}{2}[\p_x v \cdot \p_x (u+w)]
\right\},
\\
& v(x,0) = u_{0}(x) - w_{0}(x).
\label{uniqueness-init-data}
\end{align}
%
%
%
Let

\begin{equation*}
	D^{m} = (1 - \p_x^2)^{m/2}, \quad m \in \rr.
\end{equation*}
%
Applying $D^r$ to both sides of \eqref{uniqueness-exp}, then 
multiplying both sides by $D^r v$ and integrating, we obtain
%
%
\begin{equation}
\begin{split}
 \frac{1}{2} \frac{d}{dt} \|v\|_{H^r}^2
 = & -\frac{\gamma}{2} \int_{\rr} D^r \p_x [v(u+w)] \cdot
D^r v \ dx
\\
& - \frac{3-\gamma}{2} \int_{\rr}  D^{r -2}
\p_x[v(u+w)] \cdot
D^r v \ dx  
\\
& - \frac{\gamma}{2} \int_{\rr} D^{r 
-2} \p_x [ \p_x v
\cdot \p_x (u+w)]\cdot D^r v \ dx.
\label{2v}
\end{split}
\end{equation}
%
%
We now estimate \eqref{2v} in parts.

\subsubsection{Estimate of Integral 1} Note that
%
%
\begin{equation}
\begin{split}
& \left |  -\frac{\gamma}{2} \int_{\rr} D^r \p_x [v(u+w)] \cdot
D^r v \ dx \right |
\\
& =
\left |
-\frac{\gamma}{2} \int_{\rr} \left[ D^r \p_x, \ u+w \right]v \cdot
D^r v \ dx - \frac{\gamma}{2} \int_{\rr} (u+w) D^r
\p_x v \cdot D^r v\ dx
\right | \\
& \lesssim \left |
\int_{\rr} \left[ D^r \p_x, \ u+w \right]v \cdot
D^r v \ dx \right |
+ \left | \int_{\rr} (u+w) D^r \p_x v
\cdot D^r v\
dx \right |.
\label{4v}
\end{split}
\end{equation}
%
%
Observe that integrating by parts gives
%
%
\begin{equation}
\begin{split}
\left | \int_{\rr} (u+w) D^r \p_x v \cdot
D^r v \ dx \right |
\le \|\p_x (u+w)\|_{L^\infty}
\|v\|_{H^r}^2.
\label{4'v}
\end{split}
\end{equation}
%
%
%
%
To estimate the remaining integral of \eqref{4v}, we shall need the following
following result taken from \cite{Himonas_2009_Non-uniform-dep-per}:
%
\begin{lemma}
\label{cor1}
If $s > 3/2$ and $-1 \le r  \le s -1$, then
%
%
\begin{equation}
\begin{split}
\|[D^r \p_x ,f]g\|_{L^2} \le C \|f\|_{H^s} \|g\|_{H^r}.
\label{15}
\end{split}
\end{equation}
%
%
\end{lemma}
%
%
Set $s > 3/2$ and $-1 \le r \le s -1$. An application of 
Cauchy-Schwartz and Lemma~\ref{cor1} then yields 
%
%
\begin{equation}
\begin{split}
 \left | \int_{\rr} [D^r \p_x, \ u+w] v
\cdot D^r v \ dx \right |
& \lesssim \|u+w\|_{H^s} 
\|v\|_{H^r}^2.
\label{7v}
\end{split}
\end{equation}
%
%
Combining \eqref{4'v} and \eqref{7v} and applying the Sobolev Imbedding 
Theorem, we obtain the estimate
%
%
\begin{equation}
\begin{split}
\left |  -\frac{\gamma}{2} \int_{\rr} D^r \p_x [v(u+w)] \cdot
D^r v \ dx \right |
 \lesssim \|u+w\|_{H^s} \|v\|_{H^r}^2, \quad s > 3/2, \ -1 \le r \le s-1.
\label{8v}
\end{split}
\end{equation}
%
%

\subsubsection{Estimate of Integral 2} We shall need the following.
%
%
%%%%%%%%%%%%%%%%%%%%%%%%%%%%%%%%%%%%%%%%%%%%%%%%%%%%%
%
%
%                frac deriv est
%
%
%%%%%%%%%%%%%%%%%%%%%%%%%%%%%%%%%%%%%%%%%%%%%%%%%%%%%
%
%
\begin{lemma}
For $s > 3/2$, $r \le s$, $s + r \ge 2$, we have
%
%
\begin{equation*}
\begin{split}
  \| fg \|_{H^{r-1}} \lesssim \| f \|_{H^{r-1}} \| g \|_{H^{s-1}}.
\end{split}
\end{equation*}
%
%
\label{lem:frac-deriv}
\end{lemma}
%
%
Applying Cauchy-Schwartz and Lemma~\ref{lem:frac-deriv}, we obtain
%
%
%
\begin{equation}
\begin{split}
\left | - \frac{3-\gamma}{2} \int_{\rr}  D^{r -2}
\p_x[v(u+w)] \cdot
D^r v \ dx  \right |
 & \lesssim \|u+w\|_{H^{s}} \|v\|_{H^r}^2
 \label{3v}
\end{split}
\end{equation}
%
for $s > 3/2, \ r \le s, \ \text{and} \ s + r \ge 2$.
%
%
\subsubsection{Estimate of Integral 3} Applying
Cauchy-Schwartz, Lemma~\ref{lem:frac-deriv} and the inequality $\| f_{x}
\|_{H^{m-1}} \le \| f \|_{H^{m}}$, we obtain
%
%
%
\begin{equation}
\begin{split}
\left | - \frac{\gamma}{2} \int_{\rr} D^{r 
-2} \p_x [ \p_x v
\cdot \p_x (u+w)]\cdot D^r v \ dx \right | 
 \lesssim \|u+w \|_{H^{s}}
\|v\|_{H^r}^2
\label{3'v}
\end{split}
\end{equation}
%
%
for $s > 3/2, \ r \le s, \ \text{and} \ s + r \ge 2$.
%
%
%
%
Grouping \eqref{8v}, \eqref{3v}, and \eqref{3'v}, we see that
%
%
\begin{equation}
\begin{split}
\frac{1}{2} \frac{d}{dt}
\|v\|_{H^r}^2
& \lesssim \|u+w\|_{H^s}
\|v\|_{H^r}^2, \quad | t | < T
\\
& \le 4R \| v \|_{H^{r}}^{2}
\label{9v}
\end{split}
\end{equation}
%
%
%
%
%
which by Gronwall gives
%
%
\begin{equation}
  \label{lip-ineq}
\begin{split}
  & \| u(t) - w(t) \|_{H^{r}} \le C \| u_{0} - w_{0} \|_{H^{r}}, 
  \\
  & \text{for} \ | t | < T,
  \ s > 3/2, \ -1 \le r \le s-1, \ s + r \ge 2.
\end{split}
\end{equation}
%
%
%
%
%
%
%
%
%
%
%
\subsection{Region $\Omega_{2}$} 
\label{ssec:case-4}
%
We have the estimate
\begin{equation}
  \label{fgh}
\begin{split}
  \| v(t) \|_{H^{r}}
  & < \|v(t) \|_{H^{2-s}}.
    \end{split}
\end{equation}
%
We see that \eqref{lip-ineq} is valid for $r = 2-s$, $3/2 < s \le 3$.
Hence, applying \eqref{lip-ineq} to \eqref{fgh}, we obtain 
%
%
%
%
\begin{equation*}
\begin{split}
\| v(t) \|_{H^{r}}
 \lesssim \|v(0) \|_{H^{2-s}}.
\end{split}
\end{equation*}
%
%
%
%
%
We need the following interpolation
result. 
%
%
%%%%%%%%%%%%%%%%%%%%%%%%%%%%%%%%%%%%%%%%%%%%%%%%%%%%%
%
%
%                interp
%
%
%%%%%%%%%%%%%%%%%%%%%%%%%%%%%%%%%%%%%%%%%%%%%%%%%%%%%
%
%
\begin{lemma}
  For $m_{1} < m < m_{2}$,
  %
  %
  \begin{equation*}
  \begin{split}
    \| f \|_{H^{m}} \le \| f \|_{H^{m_{1}}}^{(m_{2}-m)/(m_{2} - m_{1})} \| f
    \|_{H^{m_{2}}}^{(m -m_{1})/(m_{2} - m_{1})}.
  \end{split}
  \end{equation*}
  %
  %
  %
  %
   %
  %
\label{lem:interp}
\end{lemma}
%
Applying the lemma with $m_{1} =r$, $m = 2-s$, and $m_{2} = s$ (notice
$m_{2} > m$ for $s > 1$), we bound 
%
%
\begin{equation*}
\begin{split}
    \| v(0) \|_{H^{2-s}} 
    & \le \| v(0) \|_{H^{r}}^{\frac{2(s-1)}{s-r}} \|v(0)
  \|_{H^{s}}^{\frac{2-s-r}{s-r}}
  \\
  & = \| u_{0} - w_{0} \|_{H^{r}}^{\frac{2(s-1)}{s-r}} \|u_{0} - w_{0}
  \|_{H^{s}}^{\frac{2-s-r}{s-r}}
  \\
  & \lesssim \| u_{0} - w_{0} \|_{H^{r}}^{\frac{2(s-1)}{s-r}}.
\end{split}
\end{equation*}
%
We conclude that
%
%
\begin{equation*}
\begin{split}
  \| u(t) - w(t) \|_{H^{r}} \lesssim \|u_{0} - w_{0} \|_{H^{r}}^{\frac{2(s-1)}{s-r}}.
\end{split}
\end{equation*}
%
%
%
%
\subsection{Region $\Omega_{3}$} 
\label{ssec:case-2}
%
%
Applying Lemma~\ref{lem:interp} with $m_{1} = s-1$, $m =r$ and $m_{2} = s$, 
we obtain
%
%
\begin{equation}
  \label{pre-lip-ap}
\begin{split}
  \| v(t) \|_{H^{r}} & \lesssim \| v(t) \|_{H^{s-1}}^{s-r} \|v(t) \|_{H^{s}}^{1-s+r}
  \\
  & \simeq \| v(t) \|_{H^{s-1}}^{s-r}.
\end{split}
\end{equation}
%
%
We see that \eqref{lip-ineq} is valid for  $r = s-1$, $s \ge 3/2$. Hence,
applying \eqref{lip-ineq} to \eqref{pre-lip-ap} we obtain 
%
%
\begin{equation*}
\begin{split}
  \| v(t) \|_{H^{r}} & \lesssim \|v(0) \|_{H^{s-1}}^{s-r}
   \\
   & \le \|v(0) \|_{H^{r}}^{s-r},
\end{split}
\end{equation*}
%
that is,
\begin{equation*}
  \| u(t) - w(t) \|_{H^{r}} \lesssim  \|u_{0} - w_{0}\|_{H^{r}}^{s-r}.
\end{equation*}
%
This completes the proof of Theorem~\ref{thm:main-thm}. \qed
%
%
\providecommand{\bysame}{\leavevmode\hbox to3em{\hrulefill}\thinspace}
\providecommand{\MR}{\relax\ifhmode\unskip\space\fi MR }
% \MRhref is called by the amsart/book/proc definition of \MR.
\providecommand{\MRhref}[2]{%
  \href{http://www.ams.org/mathscinet-getitem?mr=#1}{#2}
}
\providecommand{\href}[2]{#2}
\begin{thebibliography}{HKM09}

\bibitem[BBM72]{Benjamin_1972_Model-equations}
T.~B. Benjamin, J.~L. Bona, and J.~J. Mahony, \emph{Model equations for long
  waves in nonlinear dispersive systems}, Philos. Trans. Roy. Soc. London Ser.
  A \textbf{272} (1972), no.~1220, 47--78.

\bibitem[BC07]{Bressan_2007_Global-conserva}
A.~Bressan and A.~Constantin, \emph{Global conservative solutions of the
  camassa-holm equation}, Arch. Ration. Mech. Anal. \textbf{183} (2007), no.~2,
  215--239.

\bibitem[BCK06]{Bendahmane_2006_Hsp-1-perturbat}
M.~Bendahmane, G.~M. Coclite, and K.~H. Karlsen,
  \emph{$h^1$-perturbations of smooth solutions for a weakly dissipative
  hyperelastic-rod wave equation}, Mediterr. J. Math. \textbf{3} (2006),
  no.~3-4, 419--432.

\bibitem[Ben72]{Benjamin_1972_The-stability-o}
T.~B. Benjamin, \emph{The stability of solitary waves}, Proc. Roy. Soc.
  (London) Ser. A \textbf{328} (1972), 153--183.

\bibitem[CHK05]{Coclite_2005_Global-weak-sol}
G.~M. Coclite, H.~Holden, and K.~H. Karlsen, \emph{Global weak solutions to a
  generalized hyperelastic-rod wave equation}, SIAM J. Math. Anal. \textbf{37}
  (2005), no.~4, 1044--1069 (electronic).

\bibitem[CLZ11]{Chen:2011fk}
R.~M. Chen, Y.~Liu, and P.~Zhang, \emph{The h{\"o}lder continuity of the
  solution map to the {\$}b{\$}-family equation in weak topology}, submitted 2011.

\bibitem[CS00]{Constantin_2000_Stability-of-a-}
A.~Constantin and W.~A. Strauss, \emph{Stability of a class of solitary waves
  in compressible elastic rods}, Phys. Lett. A \textbf{270} (2000), no.~3-4,
  140--148.

\bibitem[Dai98]{Dai_1998_Model-equations}
H.-H. Dai, \emph{Model equations for nonlinear dispersive waves in a
  compressible mooney-rivlin rod}, Acta Mech. \textbf{127} (1998), no.~1-4,
  193--207.

\bibitem[DDH00]{Dai_2000_Head-on-collisi}
H.-H. Dai, S.~Dai, and Y.~Huo, \emph{Head-on collision between two solitary
  waves in a compressible mooney-rivlin elastic rod}, Wave Motion \textbf{32}
  (2000), no.~2, 93--111.

\bibitem[DH00]{Dai_2000_Solitary-shock-}
H.-H. Dai and Y.~Huo, \emph{Solitary shock waves and other travelling waves in
  a general compressible hyperelastic rod}, R. Soc. Lond. Proc. Ser. A Math.
  Phys. Eng. Sci. \textbf{456} (2000), no.~1994, 331--363.

\bibitem[HKM09]{Himonas_2009_Non-uniform-dep-per}
A.~Himonas, C.~E. Kenig, and G.~Misiolek, \emph{Non-uniform dependence for the
  periodic ch equation.}, To appear in Communications in Partial Differential
  Equations (2009).

\bibitem[HR07]{Holden_2007_Global-conserva}
H.~Holden and X.~Raynaud, \emph{Global conservative solutions of the
  generalized hyperelastic-rod wave equation}, J. Differential Equations
  \textbf{233} (2007), no.~2, 448--484.

\bibitem[Kar10]{Karapetyan:2010fk}
D.~Karapetyan, \emph{Non-uniform dependence and well-posedness for the
  hyperelastic rod equation}, J. Differential Equations \textbf{249} (2010),
  no.~4, 796--826.

\bibitem[Kat75]{Kato_1975_Quasi-linear-eq}
T.~Kato, \emph{Quasi-linear equations of evolution, with applications to
  partial differential equations}, 1975, pp.~25--70. Lecture Notes in Math.,
  Vol. 448.

\bibitem[Len06]{Lenells_2006_Traveling-waves}
J.~Lenells, \emph{Traveling waves in compressible elastic rods}, Discrete
  Contin. Dyn. Syst. Ser. B \textbf{6} (2006), no.~1, 151--167 (electronic).

\bibitem[Mus07]{Mustafa_2007_Global-conserva}
O.~G. Mustafa, \emph{Global conservative solutions of the hyperelastic rod
  equation}, Int. Math. Res. Not. IMRN (2007), no.~13, Art. ID rnm040, 26.

\bibitem[RB01]{Rodriguez-Blanco_2001_On-the-Cauchy-p}
G.~Rodr\'iguez-Blanco, \emph{On the cauchy problem for the camassa-holm
  equation}, Nonlinear Anal. \textbf{46} (2001), no.~3, Ser. A: Theory Methods,
  309--327.

\bibitem[Tay91]{Taylor_1991_Pseudodifferent}
M.~E. Taylor, \emph{Pseudodifferential operators and nonlinear pde}, vol. 100,
  1991.

\bibitem[Yin03]{Yin_2003_On-the-Cauchy-p}
Z.~Yin, \emph{On the cauchy problem for a nonlinearly dispersive wave
  equation}, J. Nonlinear Math. Phys. \textbf{10} (2003), no.~1, 10--15.

\bibitem[Zho05]{Zhou_2005_Local-well-pose}
Y.~Zhou, \emph{Local well-posedness and blow-up criteria of solutions for a rod
  equation}, Math. Nachr. \textbf{278} (2005), no.~14, 1726--1739.

\end{thebibliography}
%
%
%
%
%
%
%
%\bibliographystyle{amsalpha-custom}
%\bibliography{/Users/davidkarapetyan/math/bib-files/references}

\end{document}
