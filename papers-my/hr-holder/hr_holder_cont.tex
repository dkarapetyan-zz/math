%
\documentclass[12pt,reqno]{amsart}
\usepackage{amssymb}
\usepackage{appendix}
\usepackage{showkeys}
\usepackage{tikz}
\usepackage[showonlyrefs=true]{mathtools} %amsmath extension package
\usepackage{cancel}  %for cancelling terms explicity on pdf
\usepackage[normalem]{ulem} %for striking out non-math text
\usepackage{yhmath}   %makes fourier transform look nicer, among other things
\usepackage{framed}  %for framing remarks, theorems, etc.
\usepackage{enumerate} %to change enumerate symbols
\usepackage[margin=2.5cm]{geometry}  %page layout
%\setcounter{tocdepth}{1} %must come before secnumdepth--else, pain
%\setcounter{secnumdepth}{1} %number only sections, not subsections
%\usepackage[pdftex]{graphicx} %for importing pictures into latex--pdf compilation
\numberwithin{equation}{section}  %eliminate need for keeping track of counters
\numberwithin{figure}{section}
\setlength{\parindent}{0in} %no indentation of paragraphs after section title
\renewcommand{\baselinestretch}{1.1} %increases vert spacing of text
%
\usepackage{hyperref}
\hypersetup{colorlinks=true,
linkcolor=blue,
citecolor=blue,
urlcolor=blue,
}
%
\DeclareMathOperator{\sgn}{sgn}
%
\newcommand{\ds}{\displaystyle}
\newcommand{\ts}{\textstyle}
\newcommand{\nin}{\noindent}
\newcommand{\rr}{\mathbb{R}}
\newcommand{\nn}{\mathbb{N}}
\newcommand{\zz}{\mathbb{Z}}
\newcommand{\cc}{\mathbb{C}}
\newcommand{\ci}{\mathbb{T}}
\newcommand{\zzdot}{\dot{\zz}}
\newcommand{\wh}{\widehat}
\newcommand{\p}{\partial}
\newcommand{\ee}{\varepsilon}
\newcommand{\vp}{\varphi}
\newcommand{\wt}{\widetilde}
%
%
%
%
\newtheorem{theorem}{Theorem}[section]
\newtheorem{lemma}[theorem]{Lemma}
\newtheorem{corollary}[theorem]{Corollary}
\newtheorem{claim}[theorem]{Claim}
\newtheorem{prop}[theorem]{Proposition}
\newtheorem{proposition}[theorem]{Proposition}
\newtheorem{no}[theorem]{Notation}
\newtheorem{definition}[theorem]{Definition}
\newtheorem{remark}[theorem]{Remark}
\newtheorem{examp}{Example}[section]
\newtheorem {exercise}[theorem]{Exercise}
%
%\makeatletter \renewenvironment{proof}[1][\proofname] {\par\pushQED{\qed}\normalfont\topsep6\p@\@plus6\p@\relax\trivlist\item[\hskip\labelsep\bfseries#1\@addpunct{.}]\ignorespaces}{\popQED\endtrivlist\@endpefalse} \makeatother%
%makes proof environment bold instead of italic
\newcommand{\uol}{u^\omega_\lambda}
\newcommand{\lbar}{\bar{l}}
\renewcommand{\l}{\lambda}
\newcommand{\R}{\mathbb R}
\newcommand{\RR}{\mathcal R}
\newcommand{\al}{\alpha}
\newcommand{\ve}{q}
\newcommand{\tg}{{tan}}
\newcommand{\m}{q}
\newcommand{\N}{N}
\newcommand{\ta}{{\tilde{a}}}
\newcommand{\tb}{{\tilde{b}}}
\newcommand{\tc}{{\tilde{c}}}
\newcommand{\tS}{{\tilde S}}
\newcommand{\tP}{{\tilde P}}
\newcommand{\tu}{{\tilde{u}}}
\newcommand{\tw}{{\tilde{w}}}
\newcommand{\tA}{{\tilde{A}}}
\newcommand{\tX}{{\tilde{X}}}
\newcommand{\tphi}{{\tilde{\phi}}}
\synctex=1
\begin{document}
\title[H\"older Continuity of the Data to Solution Map for HR]{H\"older Continuity of the Data to Solution Map for HR in the
Weak Topology}
\author{David Karapetyan}
\address{Department of Mathematics  \\
    University  of Notre Dame\\
        Notre Dame, IN 46556 }
        \date{\today}
        %
        \maketitle
        %
        %
        %
        %
        %
        %        
        %
        \section{Introduction}
%
%
%
We consider the hyperelastic rod (HR) Cauchy problem
\begin{gather}
 \p_t u =  -\gamma u \p_x u -
 \p_{x} D^{-2} \left[ \frac{3-\gamma}{2}u^2 +
\frac{\gamma}{2} \left( \p_x u \right)^2
\right],
\label{hyperelastic-rod-equation}
\\
 u(x,0) = u_0(x), \; \; x \in \rr, \; \; t \in \rr
\label{init-cond}
\end{gather}
%
%
where 
\begin{equation*}
	D^{m} = (1 - \p_x^2)^{m/2}, \quad m \in \rr
\end{equation*}
and  $\gamma$  is a  nonzero constant. The HR equation was first
derived by Dai in \cite{Dai_1998_Model-equations} as a one-dimensional 
model for finite-length and
small-amplitude axial deformation waves in thin cylindrical
rods composed of a compressible Mooney-Rivlin
material. The derivation relied upon a reductive perturbation technique, 
and took into account the nonlinear dispersion of pulses propagating 
along a rod. It was assumed that each cross-section of the rod is 
subject to a stretching and rotation in space. The solution $u(x,t)$ to the 
HR equation represents the radial stretch relative
to a pre-stressed state, while $\gamma$ is a fixed constant depending upon 
the pre-stress and the material used in
the rod, with values ranging from $- 29.4760$ to $3.4174$.
%
\\
\\
The well-posedness of the HR equation has been studied by several authors. 
In Yin \cite{Yin_2003_On-the-Cauchy-p} and Zhou 
\cite{Zhou_2005_Local-well-pose}, a proof of local well-posedness in Sobolev 
spaces $H^s$,  $s > 3/2$, is described  on the line and the circle, respectively. 
Their approach is to rewrite the HR equation   
in its non-local form, and then to verify the conditions needed to apply 
Kato's semi-group theory \cite{Kato_1975_Quasi-linear-eq}. Using an alternative,
Galerkin type method outlined in Taylor \cite{Taylor_1991_Pseudodifferent}, local
well-posedness in Sobolev spaces $H^s$,  $s > 3/2$ on the line and the circle is
established in \cite{Karapetyan:2010fk}. 
For details on how this is done for CH ($\gamma =1$) on the line, see Rodriguez-Blanco 
\cite{Rodriguez-Blanco_2001_On-the-Cauchy-p}. Blow-up criteria 
is also investigated in \cite{Yin_2003_On-the-Cauchy-p} and 
\cite{Zhou_2005_Local-well-pose}, as well as by Constantin and Strauss 
\cite{Constantin_2000_Stability-of-a-}. 
\\
\\
Setting $\gamma = 0$ gives the celebrated 
BBM equation, which was proposed by 
Benjamin, Bona, and Mahony 
\cite{Benjamin_1972_Model-equations} as a model for 
the unidirectional evolution of long waves.
Solitary-wave solutions to this 
equation are global and orbitally stable (see Benjamin 
\cite{Benjamin_1972_The-stability-o}, 
\cite{Benjamin_1972_Model-equations}, and 
\cite{Constantin_2000_Stability-of-a-}).
For more general $\gamma$, the existence of global 
solutions to HR on the line with constant $H^1$ energy
was proved recently by Mustafa \cite{Mustafa_2007_Global-conserva}
using the approach developed by Bressan and 
Constantin in \cite{Bressan_2007_Global-conserva}. Using a vanishing 
viscosity argument, Coclite, 
Holden, and Karlsen \cite{Coclite_2005_Global-weak-sol}
established existence of a strongly continuous semigroup of global 
weak solutions of HR on the line for initial data in $H^1$.
Bendahmane, Coclite, and Karlsen 
\cite{Bendahmane_2006_Hsp-1-perturbat} extended this result to traveling 
wave solutions that are supersonic solitary shockwaves.
For more information on the existence of global solutions to the HR
equation, see Holden and Raynaud \cite{Holden_2007_Global-conserva}
and \cite{Yin_2003_On-the-Cauchy-p}. 
\\
\\
There is a variety of traveling wave solutions to the HR equation that can be 
obtained using various combinations of peaks, cusps, compactons, 
fractal-like waves, and plateaus (see Lenells 
\cite{Lenells_2006_Traveling-waves}). Orbital stability of solitary wave 
solutions was proved in \cite{Constantin_2000_Stability-of-a-}.
Solitary shock wave formation was 
analyzed in Dai and Huo \cite{Dai_2000_Solitary-shock-} using traveling 
wave solutions of the HR equation to derive a system of ordinary differential 
equations, with a vertical singular line in the phase plane corresponding with the 
formation of shock waves. Head-on collisions between two solitary 
waves was investigated in the work of Hui-Hui Dai, 
Shiqiang Dai, and Huo \cite{Dai_2000_Head-on-collisi} using a reductive 
perturbation method coupled with the technique of strained coordinates. 
\\
\\
In this work we study the continuity properties of the data-to-solution map for
the HR 
equation, expanding upon the work in \cite{Karapetyan:2010fk}, in which it was
shown that the data-to-solution map is not better than continuous in Sobolev
spaces $H^{s}$ on the line
and the circle. More precisely, following \cite{Chen:2011fk} we show the
following result:
%
%
\begin{theorem}
For $\gamma \neq 0$, the
data to solution map for HR is H\"older continuous from $B_{H^{s}}(R)$ (in
the topology of $H^{r}$) to $C([0, T], H^{r})$, where $T = T(R)$, for $s >
3/2$, $-1 \le r < s$. More
precisely, decompose the set $\Omega = \left\{ (s, r) \in \rr^{2}  \right\}$
into the pieces
  %
  %
  \begin{equation*}
  \begin{split}
    & \Omega_{1} = \left\{ (s, \ r):  \ s < 3/2 \right\}
    \\
    & \Omega_{2} = \left\{ (s, \ r):
     \ s>3/2, \ r < -1  \right\}
    \\
    & \Omega_{3} = \left\{ (s, \ r):
     \ s>3/2, \ r > s  \right\}
    \\
    & \Omega_{4} = \left\{ (s, \ r):
     \ s>3/2, \ -1 \le r \le s-1, \ s + r \ge 2  \right\}
    \\
    & \Omega_{5} = \left\{ (s, \ r):
     \ s>3/2, \ -1 \le r < 2-s \right\}
    \\
    & \Omega_{6} = \left\{ (s, \ r):
    \  s>3/2, \  s-1 < r < s  \right\}.
    \end{split}
\end{equation*}
  %
  %
\label{thm:main-thm}
\end{theorem}
%
\begin{center}
\begin{tikzpicture}[scale=2]
% Draw thin grid lines with color 40% gray + 60% white

% Draw x and y axis lines
\draw [->] (0,0) -- (3,0) node [below] {$s$};
\draw [->] (0,-2) -- (0,3) node [left] {$r$};
\draw [->, dashed] (0,0) -- (3,3);
\draw [->, dashed] (0,-1) -- (3,2);
\draw [->, dashed] (0,2) -- (3,-1);
\draw [->, dashed] (0,-1) -- (3,-1);
\draw [->, dashed] (3/2,-2) -- (3/2, 3);
\fill[color=green, fill opacity=0.3] (1.5, 0.5) -- (3,2) -- (3,0) -- (3,-1);
\fill[color=red, fill opacity=0.3] (1.5, 0.5) -- (1.5,1.5) -- (3,3) -- (3,2);
\fill[color=yellow, fill opacity=0.3] (0, -2) -- (1.5, -2) -- (1.5, 3) -- (0, 3);
\fill[color=blue, fill opacity=0.3] (1.5, 0.5) -- (1.5, -1) -- (3, -1);
\fill[color=brown, fill opacity=0.3] (1.5, 1.5) -- (3, 3) -- (1.5, 3);
\fill[color=cyan, fill opacity=0.3] (1.5, -1) -- (3, -1) -- (3, -2) -- (1.5, -2);


\foreach \x/\xtext in {1, 2}
    \draw[shift={(\x,0)}]  node[below] {$\xtext$};
\foreach \y/\ytext in {-1, 1, 2}
    \draw[shift={(0,\y)}]  node[left] {$\ytext$};
    \draw (1,0.5) node {$\Omega_{1}$};
    \draw (2,2.5) node {$\Omega_{3}$};
    \draw (2,1.5) node {$\Omega_{6}$};
    \draw (2,-1.5) node {$\Omega_{2}$};
    \draw (2,0.5) node {$\Omega_{4}$};
    \draw (2,-0.5) node {$\Omega_{5}$};
\end{tikzpicture}
\end{center}
%
%
Then for two initial data $u_{0}, v_{0} \in B_{H^{s}}(R)$, there exist unique
corresponding solutions $u(x,t), v(x,t)$ for $0 \le t \le T= T(R)$ to the
HR equation \eqref{hyperelastic-rod-equation} which satisfy 
%
%
\begin{equation*}
\begin{split}
  \| u(t) - v(t) \|_{H^{r}} \le C \| u_{0} - v_{0} \|^{\alpha(s, r, \gamma)},
  \quad 0
  \le t \le T
\end{split}
\end{equation*}
%
%
where 
%
%
\begin{equation*}
\begin{split}
\alpha = 
\begin{cases}
   1, \quad & (s,r, \gamma) \in \Omega_{4} 
  \\
   2(s-1)/(2-r),  \quad & (s, r, \gamma) \in \Omega_{5}
  \\
   s-r, \quad & (s, r, \gamma) \in \Omega_{6}.
\end{cases}
\end{split}
\end{equation*}
%

%
%%%%%%%%%%%%%%%%%%%%%%%%%%%%%%%%%%%%%%%%%%%%%%%%%%%%%
%
%
%                Main theorem
%
%
%%%%%%%%%%%%%%%%%%%%%%%%%%%%%%%%%%%%%%%%%%%%%%%%%%%%%
%
%
We remark that this result improves upon the H\"older continuity result in
\cite{Chen:2011fk} (there it is for $0 \le r < s$) by a full degree in $r$. 
%
%
%
%
\section{Proof of Theorem~\ref{thm:main-thm}}
%
%
The proof in the periodic case is analogous to that in the non-periodic case.
Hence, we restrict our attention to the non-periodic case. 

%
%
\subsection{Region $\Omega_{1}$} 
\label{ssec:reg-2}
This is open. In particular, well-posedness of HR must be proved or disproved
below $3/2$ before we can even consider discussing H\"older continuity in weak
topologies.
\subsection{Region $\Omega_{2}$} 
\label{ssec:reg-6}
Open. The lower bound on $r$ comes from Lemma~\ref{cor1}. Perhaps the lemma can
be strengthened.
\subsection{Region $\Omega_{3}$} 
\label{ssec:reg-7}
It makes little sense to talk about this region, 
i.e.\ we assume a priori that our initial
data is in $H^{s}$ (i.e.\ it may not even be in $H^{r}$ if $r > s$). If it is in
$H^{r}$, then we are back where we started.
\subsection{Region $\Omega_{4}$} 
\label{ssec:reg-m-imp}
%
%
Let $u_{0}(x), v_{0}(x)
\in B_{H^{s}}(R)$, $s > 3/2$ be two initial datum. Then from
the well-posedness theory for HR \cite{Karapetyan:2010fk}, we
know that there exists unique corresponding solutions $u, v \in C(I,
B_{H^{s}}(2R))$ to \eqref{hyperelastic-rod-equation}.
Set $v=u-w$. Then $v$ solves the Cauchy-problem
%
%
\begin{align}
	\label{uniqueness-exp}
& \p_t v
=  -\frac{\gamma}{2} \p_x [v(u + w)] 
\\
\notag
& \phantom{\p_t v = }- D^{-2} \p_x \left\{
\frac{3-\gamma}{2}[v(u+w)] + \frac{\gamma}{2}[\p_x v \cdot \p_x (u+w)]
\right\},
\\
& v(x,0) = u_{0}(x) - v_{0}(x).
\label{uniqueness-init-data}
\end{align}
%
%
%
%
Applying $D^r$ to both sides of \eqref{uniqueness-exp}, then 
multiplying both sides by $D^r v$ and integrating, we obtain
%
%
\begin{equation}
\begin{split}
 \frac{1}{2} \frac{d}{dt} \|v\|_{H^r}^2
 = & -\frac{\gamma}{2} \int_{\rr} D^r \p_x [v(u+w)] \cdot
D^r v \ dx
\\
& - \frac{3-\gamma}{2} \int_{\rr}  D^{r -2}
\p_x[v(u+w)] \cdot
D^r v \ dx  
\\
& - \frac{\gamma}{2} \int_{\rr} D^{r 
-2} \p_x [ \p_x v
\cdot \p_x (u+w)]\cdot D^r v \ dx.
\label{2v}
\end{split}
\end{equation}
%
%
We now estimate \eqref{2v} in parts.

\subsection{Estimate of Integral 1} Note that
%
%
\begin{equation}
\begin{split}
& \left |  -\frac{\gamma}{2} \int_{\rr} D^r \p_x [v(u+w)] \cdot
D^r v \ dx \right |
\\
& =
\left |
-\frac{\gamma}{2} \int_{\rr} \left[ D^r \p_x, \ u+w \right]v \cdot
D^r v \ dx - \frac{\gamma}{2} \int_{\rr} (u+w) D^r
\p_x v \cdot D^r v\ dx
\right | \\
& \le \left |
-\frac{\gamma}{2} \int_{\rr} \left[ D^r \p_x, \ u+w \right]v \cdot
D^r v \ dx \right |
+ \left | \frac{\gamma}{2} \int_{\rr} (u+w) D^r \p_x v
\cdot D^r v\
dx \right |.
\label{4v}
\end{split}
\end{equation}
%
%
Observe that integrating by parts gives
%
%
\begin{equation}
\begin{split}
\left | \frac{\gamma}{2}\int_{\rr} (u+w) D^r \p_x v \cdot
D^r v \ dx \right |
\lesssim \|\p_x (u+w)\|_{L^\infty}
\|v\|_{H^r}^2.
\label{4'v}
\end{split}
\end{equation}
%
%
%
%
To estimate the remaining integral of \eqref{4v}, we shall need the following
following result taken from \cite{Himonas_2009_Non-uniform-dep-per}:
%
\begin{lemma}
\label{cor1}
If $s > 3/2$ and $-1 \le r  \le s -1$, then
%
%
\begin{equation}
\begin{split}
\|[D^r \p_x ,f]g\|_{L^2} \le C \|f\|_{H^s} \|g\|_{H^r}.
\label{15}
\end{split}
\end{equation}
%
%
\end{lemma}
%
%
Set $s > 3/2$ and $-1 \le r \le s -1$. An application of 
Cauchy-Schwartz and Lemma~\ref{cor1} then yields 
%
%
\begin{equation}
\begin{split}
 \left | -\frac{\gamma}{2} \int_{\rr} [D^r \p_x, \ u+w] v
\cdot D^r v \ dx \right |
& \lesssim \|u+w\|_{H^s} 
\|v\|_{H^r}^2.
\label{7v}
\end{split}
\end{equation}
%
%
Combining \eqref{4'v} and \eqref{7v} and applying the Sobolev Imbedding 
Theorem, we obtain the estimate
%
%
\begin{equation}
\begin{split}
\left |  -\frac{\gamma}{2} \int_{\rr} D^r \p_x [v(u+w)] \cdot
D^r v \ dx \right |
 \lesssim \|u+w\|_{H^s} \|v\|_{H^r}^2, \quad s > 3/2, \ -1 \le r \le s-1.
\label{8v}
\end{split}
\end{equation}
%
%

\subsection{Estimate of Integral 2} We shall need the following.
%
%
%%%%%%%%%%%%%%%%%%%%%%%%%%%%%%%%%%%%%%%%%%%%%%%%%%%%%
%
%
%                frac deriv est
%
%
%%%%%%%%%%%%%%%%%%%%%%%%%%%%%%%%%%%%%%%%%%%%%%%%%%%%%
%
%
\begin{lemma}
For $s > 3/2$, $r \le s$, $s + r \ge 2$, we have
%
%
\begin{equation*}
\begin{split}
  \| fg \|_{H^{r-1}} \le \| f \|_{H^{r-1}} \| g \|_{H^{s}}.
\end{split}
\end{equation*}
%
%
\label{lem:frac-deriv}
\end{lemma}
%
%
%
%
%
%
Applying Cauchy-Schwartz and Lemma~\ref{lem:frac-deriv}, we obtain
%
%
%
\begin{equation*}
\begin{split}
\left | - \frac{3-\gamma}{2} \int_{\rr}  D^{r -2}
\p_x[v(u+w)] \cdot
D^r v \ dx  \right |
 & \lesssim \|u+w\|_{H^{r -1}} \|v\|_{H^r}^2
\end{split}
\end{equation*}
%
%
which implies
\begin{equation}
\begin{split}
\left | - \frac{3-\gamma}{2} \int_{\rr}  D^{r -2}
\p_x[v(u+w)] \cdot
D^r v \ dx  \right |
 & \lesssim \|u+w\|_{H^{s}} \|v\|_{H^r}^2
 \label{3v}
\end{split}
\end{equation}
%
for $s > 3/2, \ r \le s, \ \text{and} \ s + r \ge 2$.
%
%
\subsection{Estimate of Integral 3} We first apply
Cauchy-Schwartz to obtain
%
%
\begin{equation*}
\begin{split}
\left | - \frac{\gamma}{2} \int_{\rr} D^{r 
-2} \p_x [ \p_x v
\cdot \p_x (u+w)]\cdot D^r v \ dx \right | 
 \lesssim 
\|[\p_x v \cdot \p_x (u+w)] \|_{H^{r -1}}
\|v\|_{H^r}.
\end{split}
\end{equation*}
%
We now need the following result.
%
%
%
\begin{lemma}
\label{impo}
If  $s > 3/2$, $r \le s$, and $s + r \ge 2$,  then
%
%
\begin{equation}
\begin{split}
  \|f_{x}g_{x}\|_{H^{r - 1}} \le C \|f\|_{H^{r}}
\|g\|_{H^{s}}.
\label{11}
\end{split}
\end{equation}
%
%
\end{lemma}
%
Applying the lemma, we conclude that
%
\begin{equation}
\begin{split}
\left | - \frac{3-\gamma}{2} \int_{\rr}  D^{r -2}
\p_x[v(u+w)] \cdot
D^r v \ dx  \right |
 \lesssim \|u+w \|_{H^{s}}
\|v\|_{H^r}^2
\label{3'v}
\end{split}
\end{equation}
%
%
for $s > 3/2, \ r \le s, \ \text{and} \ s + r \ge 2$.
%
%
%
%
Grouping \eqref{8v}, \eqref{3v}, and \eqref{3'v}, and 
applying
the Sobolev Imbedding Theorem, we obtain
%
%
\begin{equation}
\begin{split}
\frac{1}{2} \frac{d}{dt}
\|v\|_{H^r}^2
& \lesssim \|u+w\|_{H^s}
\|v\|_{H^r}^2, \quad | t | < T
\\
& \le 2R \| v \|_{H^{r}}^{2}.
\label{9v}
\end{split}
\end{equation}
%
%
%
%
%
Letting $y(t) = \| v \|^{2}_{H^{r}}$, we obtain
%
%
%
\begin{equation*}
\begin{split}
  \frac{dy}{dt} \le cy
\end{split}
\end{equation*}
%
where $c = c(s, r, R)$. 
This admits the solution
%
%
\begin{equation*}
\begin{split}
  y(t) \le y(0) e^{ct}, \quad | t | < T
\end{split}
\end{equation*}
%
%
which implies
%
%
\begin{equation*}
\begin{split}
  y(t) \le y(0) e^{cT}.
\end{split}
\end{equation*}
%
%
Substituting back in for $y$, we see that
%
%
\begin{equation*}
\begin{split}
  \| v \|_{H^{r}}^{2} \le \| v(0) \|^{2}_{H^{r}} e^{cT}
\end{split}
\end{equation*}
%
%
or
%
%
\begin{equation}
  \label{lip-ineq}
\begin{split}
  & \| u(t) - w(t) \|_{H^{r}} \le C \| u_{0} - w_{0} \|_{H^{r}}, 
  \\
  & \text{for} \ | t | < T,
  \ s > 3/2, \ -1 \le r \le s-1, \ s + r \ge 2.
\end{split}
\end{equation}
%
Hence, in region $\Omega_{4}$, the data to solution map is locally Lipschitz from
$B_{H^{s}(R)}$ (in the $H^{r}$
topology) to $C([-T, T], H^{r})$, with Lipschitz constant $C = C(s, r, R)$.
%
%
%
%
%
%
%
%
%
%
\subsection{Region $\Omega_{5}$} 
\label{ssec:case-4}
%
%
Note that   $-1 \le 2-s \le s-1$ for $s>3/2$ and $s + (2-s) = 2$.
Hence, applying \eqref{lip-ineq}, we bound 
%
%
%
%
\begin{equation}
  \label{fgh}
\begin{split}
  \| v(t) \|_{H^{r}}
  & \le \|v(t) \|_{H^{2-s}}
  \\
  & \lesssim \|v(0) \|_{H^{2-s}}.
  \end{split}
\end{equation}
%
We need the following interpolation
result. 
%
%
%%%%%%%%%%%%%%%%%%%%%%%%%%%%%%%%%%%%%%%%%%%%%%%%%%%%%
%
%
%                interp
%
%
%%%%%%%%%%%%%%%%%%%%%%%%%%%%%%%%%%%%%%%%%%%%%%%%%%%%%
%
%
\begin{lemma}
  For $m_{1} < m < m_{2}$,
  %
  %
  \begin{equation*}
  \begin{split}
    \| f \|_{H^{m}} \le \| f \|_{H^{m_{1}}}^{(m_{2}-m)/(m_{2} - m_{1})} \| f
    \|_{H^{m_{2}}}^{(m -m_{1})/(m_{2} - m_{1})}.
  \end{split}
  \end{equation*}
  %
  %
  %
  %
   %
  %
\label{lem:interp}
\end{lemma}
%
Applying the lemma with $m_{1} =r$, $m = 2-s$, and $m_{2} = s$, we bound \eqref{fgh} by
%
%
\begin{equation*}
\begin{split}
  \| v(0) \|_{H^{r}}^{\frac{2(s-1)}{2-r}} \|v(0)
  \|_{H^{s}}^{\frac{2-s-r}{s-r}}
  & = \| u_{0} - w_{0} \|_{H^{r}}^{\frac{2(s-1)}{2-r}} \|u_{0} - w_{0}
  \|_{H^{s}}^{\frac{2-s-r}{s-r}}
  \\
  & \lesssim \| u_{0} - w_{0} \|_{H^{r}}^{\frac{2(s-1)}{2-r}}.
\end{split}
\end{equation*}
%
We conclude that
%
%
\begin{equation*}
\begin{split}
  \| u(t) - w(t) \|_{H^{r}} \lesssim \|u_{0} - w_{0} \|_{H^{r}}^{\frac{2(s-1)}{2-r}}.
\end{split}
\end{equation*}
%
%
%
%
\subsection{Region $\Omega_{6}$} 
\label{ssec:case-2}
%
%
Applying Lemma~\ref{lem:interp} and the fact that 
%
%
\begin{equation*}
\begin{split}
  \|v\|_{H^{s}} = \|u - w \|_{H^{s}} \le 4R
\end{split}
\end{equation*}
%
%
we obtain
%
%
\begin{equation}
  \label{pre-lip-ap}
\begin{split}
  \| v(t) \|_{r} & \lesssim \| v(t) \|_{H^{s-1}}^{s-r} \|v(t) \|_{H^{s}}^{1-s+r}
  \\
  & \simeq \| v(t) \|_{H^{s-1}}^{s-r}.
\end{split}
\end{equation}
%
%
Notice that $-1 \le s-1 \le s-1$, and $s + (s-1) = 2s-1 \ge 2$ for $s >3/2$.
Hence, we may apply \eqref{lip-ineq} to bound \eqref{pre-lip-ap} by
%
%
\begin{equation*}
\begin{split}
  C \|v(0) \|_{H^{s-1}}^{s-r}
  & \lesssim \|v(0) \|_{H^{r}}^{s-r} 
  = \|u_{0} - w_{0}\|_{H^{r}}^{s-r}.
\end{split}
\end{equation*}
%
%
This completes the proof of Theorem~\ref{thm:main-thm}. \qed
%
%
%%%%%%%%%%%%%%%%%%%%%%%%%%%%%%%%%%%%%%%%%%%%%%%%%%%%%
%
%
%				Optimality
%
%
%%%%%%%%%%%%%%%%%%%%%%%%%%%%%%%%%%%%%%%%%%%%%%%%%%%%%
%
%
\section{Optimality} 
\label{sec:optimality}
\subsection{Burgers} 
\label{ssec:burgers-opt}

We first consider the Burgers initial value problem
%
%
\begin{gather}
    \label{burgers}
u_{t} + uu_{x} = 0,
\\
\label{burgers-init}
u(x,0) = u_{0}(x)
\end{gather}
%
%
To solve this, we apply the method of characteristics. That is, we seek a curve
$s \mapsto (x(s), t(s))$ such that
%
%
\begin{equation*}
\begin{split}
\frac{d}{ds} u(x(s), t(s)) = 0.
\end{split}
\end{equation*}
%
Formally differentiating the left hand side, we obtain
%
%
\begin{equation*}
\begin{split}
\frac{du}{dx} \frac{dx}{ds} + \frac{du}{dt} \frac{dt}{ds}.
\end{split}
\end{equation*}
%
%
Setting
%
%
\begin{gather}
    \label{char-ode-space}
    \frac{dx}{ds} = u,
    \\
    \label{char-ode-time}
    \frac{dt}{ds}=1
\end{gather}
and recalling the Burgers equation \eqref{burgers}, we see that
%
%
\begin{equation*}
\begin{split}
\frac{d}{ds} u(x(s), t(s)) = 0.
\end{split}
\end{equation*}
%
%
Hence, $u$ is constant along the characteristic curve $(x(s), t(s))$ given by
the characteristic ode's \eqref{char-ode-space}-\eqref{char-ode-time}. Solving
them, we obtain
\begin{gather}
    \label{0j}
    x(s) = c_{1} + \int_{0}^{s} u(x(s'), t(s'))ds'
    \\
    t(s) = s + c_{2}.
\end{gather}
We remark that by the inverse function theorem, \eqref{0j} admits a solution
$u(x(s), t(s))$ for $s$ in some bounded interval in $\rr$.  By construction, this $u$ will be constant along the characteristic curve $(x(s), t(s))$. Hence, we obtain
%
%
\begin{gather}
    \label{1j}
    x(s) = c_{1} + su
    \\
    \label{2j}
    t(s) = s + c_{2}.
\end{gather}
%
%
Setting $s = -c_{2}$, it follows that 
%
%
%
\begin{equation*}
\begin{split}
u(x(-c_{2}), t(-c_{2})) = u(c_{1} - c_{2}u, 0 ) = u_{0}(c_{1} - c_{2} u).
\end{split}
\end{equation*}
%
%
and so
%
%
%
%
\begin{equation*}
\begin{split}
u(x,t) = u_{0}(x - tu).
\end{split}
\end{equation*}
%
%
We shall prove the following.
%
%
%%%%%%%%%%%%%%%%%%%%%%%%%%%%%%%%%%%%%%%%%%%%%%%%%%%%%
%
%
%			solution burgers 
%
%
%%%%%%%%%%%%%%%%%%%%%%%%%%%%%%%%%%%%%%%%%%%%%%%%%%%%%
%
%
\begin{lemma}
%
Let $u_{0}^{\lambda}(x) = (\lambda +
x_{+}^{\alpha + 1}) \vp(x)$ be a family of initial data indexed by $\lambda \in
[-1, 1]$, where $\alpha > 0$, $x_{+} \doteq \max\{0, x\}$ and $\vp$ is a smooth cutoff
function equal to the identity in $[-2, 2]$ and with support in $[-4,4]$. Then the associated solutions $u^{\lambda}(x,t)$ take the form
\begin{equation}
    \label{u-lam-explicit-form}
    \begin{split}
        u^{\lambda}(x,t) = \lambda + (x - \lambda t)^{\alpha + 1}_{+} p(t(x- \lambda t)^{\alpha}_{+}), \quad | x | \le 1, \quad | \lambda | \le 1
    \end{split}
\end{equation}
%
%
for sufficiently small $| t |$, where $p(z)$ is a power series in $z$ with $p(0) =1$ and a positive radius of convergence.
\label{lem:sol-burg}
\end{lemma}
%
%
%
%
\begin{proof}(Kato)
Since $u_{0}^{\lambda}$ is uniformly bounded in $\lambda$ in $H^{s}$, the
$u^{\lambda}$ have a common lifespan $T$ and are uniformly bounded (via an
energy estimate--see HR). Therefore, choosing $T$ sufficiently small, we have
%
%
\begin{equation}
    \label{burg-sol-cont}
\begin{split}
    | t u^{\lambda}(x,t) | \le 1, \quad x \in \rr, \ | t | \le T, \ | \lambda | \le 1.
\end{split}
\end{equation}
%
%
%
%
Set
%
%
\begin{equation}
    \label{burg-sol-not}
\begin{split}
y(x,t) = x - t u^{\lambda}(x,t).
\end{split}
\end{equation}
%
%
Then by \eqref{burg-sol-cont}, we have
%
%
\begin{equation*}
\begin{split}
    | y(x,t) | & \le | x | + | tu^{\lambda} |
    \\
    & \le 2, \quad | x | \le 1, \ | t | \le T, | \lambda | \le 1.
\end{split}
\end{equation*}
%
%
Hence, $\vp(y) =1$ and so
%
%
\begin{equation}
    \label{burg-sol-redux}
\begin{split}
u^{\lambda}(x,t) 
& = u_{0}^{\lambda}(x - tu)
\\
& = [\lambda + (x - tu^{\lambda})_{+}^{\alpha + 1}] \vp(x - tu^{\lambda})
\\
& = [\lambda + (x - tu^{\lambda})_{+}^{\alpha + 1}].
\end{split}
\end{equation}
%
%
Multiplying both sides by $t$, we get
%
%
\begin{equation*}
\begin{split}
t u^{\lambda}(x,t) = t \lambda + t (x - tu^{\lambda})_{+}^{\alpha + 1}
\end{split}
\end{equation*}
%
%
which implies
%
%
\begin{equation*}
\begin{split}
x - tu^{\lambda} = x - t \lambda - t(x - tu^{\lambda})_{+}^{\alpha + 1}.
\end{split}
\end{equation*}
%
%
Set 
\begin{equation}
    \label{y-for-u}
    y(x,t) = x - tu^{\lambda}. 
\end{equation}
%
%
%
%
Then
%
%
\begin{equation*}
\begin{split}
y = x - t \lambda - t y_{+}^{\alpha + 1}
\end{split}
\end{equation*}
%
%
or
%
\begin{equation*}
\begin{split}
y + ty_{+}^{\alpha + 1} = x - t \lambda
\end{split}
\end{equation*}
%
%
which implies
%
%
\begin{equation}
    \label{y-equation}
\begin{split}
[1 + ty_{+}^{\alpha + 1}]y_{+} = (x - t \lambda)_{+}.
\end{split}
\end{equation}
%
%
Solving this equation for $y$ and then substituting back into \eqref{y-for-u} completes the proof.
\end{proof}
\begin{framed}
    \textbf{I spent $7+$ hours, but couldn't figure out how to solve
\eqref{y-equation} in order to get \eqref{u-lam-explicit-form}. I go about
proving Lemma~\ref{lem:sol-burg} in slightly different fashion; see the next
section}.
\end{framed}
%
%
\begin{proof}(David)
Notice from \eqref{burg-sol-redux} that
%
%
\begin{equation*}
\begin{split}
u^{\lambda}(x, 0) = \lambda + x_{+}^{\alpha +1}.
\end{split}
\end{equation*}
%
%
Furthermore, \eqref{burg-sol-redux} implies that
%
%
%
%
\begin{equation*}
\begin{split}
u^{\lambda} - \lambda
& = (x - tu^{\lambda})^{\alpha + 1}
\\
& = [x - t(u^{\lambda} \pm \lambda)]^{\alpha + 1}_{+}
\\
& = [x - t(u^{\lambda - \lambda}) - t \lambda]_{+}^{\alpha + 1}.
\end{split}
\end{equation*}
%
%
Set $y = u^{\lambda} - \lambda$. Then
%
%
\begin{equation*}
\begin{split}
y = [x - t \lambda - ty]_{+}^{\alpha + 1}
\end{split}
\end{equation*}
%
%
or
%
%
\begin{equation*}
\begin{split}
y^{1/(\alpha + 1)} = [x - t \lambda - ty]_{+}.
\end{split}
\end{equation*}
%
%
Therefore, $y(t \lambda, t) = 0$, and so $$u^{\lambda}(t \lambda, t) = \lambda.$$
%
%
Hence, it is natural to guess that $u^{\lambda}(x,t) = \lambda + (x - \lambda
t)^{\alpha + 1}_{+} p(z)$ for some power series $p(z)$, where $z = z(x,t)$ and
$p(z(x,0)) =1$. We now show that this is indeed the case. Substituting our ansatz
into \eqref{burg-sol-redux}, we see that
%
%
\begin{equation*}
\begin{split}
(x - \lambda t)^{\alpha + 1}_{+} p(z) = \{x - t[\lambda + (x - \lambda
    t)_{+}^{\alpha +1} p(z)]\} 
\end{split}
\end{equation*}
%
%
which implies
%
%
\begin{equation*}
\begin{split}
x - \lambda t - (t + 1)(x - \lambda t)_{+}^{\alpha + 1} p(z) = 0.
\end{split}
\end{equation*}
%
%
Assume $x - \lambda t \neq 0$. Dividing through by $x - \lambda t$ and simplifying, we obtain
%
%
\begin{equation*}
\begin{split}
p(z) =\frac{1}{(t+1)(x - \lambda t)^{\alpha}}.
\end{split}
\end{equation*}
%
%
%
\begin{framed}
    \textbf{This function is NOT analytic at $x = \lambda t$. 
    Hence, I am very confused by Kato's claim that we can arrange that $p(z)$ be a
    power series with $p(0)=1$ and a positive radius of convergence. I feel that
    I have just shown that this is impossible. I will accept it for the time
    being.}
\end{framed}
%
%
%
\end{proof}
%
%
Applying Lemma~\ref{lem:sol-burg}, we have
%
%
\begin{equation*}
\begin{split}
u^{\lambda} - u^{0} = \lambda + (x - \lambda t)^{\alpha + 1}_{+}p(t(x- \lambda t)^{\alpha}_{+}) - (x)_{+}^{\alpha + 1}p(tx_{+}^{\alpha}) + \ \text{positive higher order terms}.
\end{split}
\end{equation*}
%
%
Furthermore, for integer $s \ge 2$ (we need to have well-posedness) and real $\alpha > (s-1)$
%
%
\begin{equation*}
\begin{split}
\frac{d^{s}}{dx^{s}}(u^{\lambda} - u^{0}) = (\alpha + 1)(\alpha) \cdots (\alpha + 1 -s)[(x - \lambda t)^{\alpha + 1 -s}_{+} - (x)_{+}^{\alpha + 1 -s}] + \ \text{positive higher order terms}.
\end{split}
\end{equation*}
%
%
Hence
%
%
\begin{equation*}
\begin{split}
\| u^{\lambda} - u^{0} \|_{H^{s}}^{2} 
& \ge \| \frac{d^{s}}{dx^{s}}(u^{\lambda} - u^{0}) \|_{L^{2}}^{2}
\\
& \ge \int_{-1}^{1} | \frac{d^{s}}{dx^{s}}(u^{\lambda} - u^{0}) |^{2} dx
\\
& \ge \int_{-1}^{1} \{(\alpha + 1)(\alpha) \cdots (\alpha + 1 -s   )[(x - \lambda t)_{+}^{\alpha + 1 -s} - x_{+}^{\alpha + 1 -s}]\}^{2} dx
\\
& \simeq \int_{0}^{\lambda t} (x^{\alpha + 1 -s})^{2} dx + \int_{\lambda t}^{1} [(x - \lambda t)^{\alpha +1 -s} -x^{\alpha + 1 -s}]^{2} dx
\\
& \ge \int_{0}^{\lambda t} x^{2(\alpha + 1 -s)} dx 
\\
& \simeq  | \lambda t |^{2 \alpha + 3 -2s}
\end{split}
\end{equation*}
%
%
and so
\begin{equation}
    \begin{split}
    \label{holder-lb}
\| u^{\lambda} - u^{0} \|_{H^{s}} & \ge c_{\alpha, s} | \lambda t |^{\alpha -s + 3/2}
\\
&  = c_{\alpha, s} | \lambda t |^{3/4}, \quad \alpha = s - 3/4.
\end{split}
\end{equation}
But
%
%
\begin{equation}
    \label{init-lb}
\begin{split}
    \| u^{\lambda}(0) - u^{0}(0) \|_{H^{r}} 
    & \simeq |\lambda |.
\end{split}
\end{equation}
%
Hence, from \eqref{holder-lb} and \eqref{init-lb},
we see that for any fixed constant $c > 0$ there exists sufficiently small $T = T(c) >
0$ such that for all $|t| \le T$
%
\begin{equation*}
\| u^{\lambda} - u^{0} \|_{H^{s}} \ge c \| u^{\lambda}(0) - u^{0}(0) \|_{H^{r}}.
\end{equation*}
Therefore, the Burgers initial value problem on the line with initial data
$u_{0} \in H^{s}, s \ge 2$ is not H\"older continuous for any exponent.
%
%
%%%%%%%%%%%%%%%%%%%%%%%%%%%%%%%%%%%%%%%%%%%%%%%%%%%%%
%
%
%				Optimality HR
%
%
%%%%%%%%%%%%%%%%%%%%%%%%%%%%%%%%%%%%%%%%%%%%%%%%%%%%%
%
%
\section{Optimality for HR} 
\label{sec:op-hr}
We recall the hyperelastic-rod (HR) ivp
\begin{gather}
    \label{hr}
    u_{t} + \frac{\gamma}{2}(u^{2})_{x} + P_{x} = 0
    \\
    \label{hr-data}
    u(x,0) = u_{0}(x)
\end{gather}
where
%
%
\begin{equation*}
\begin{split}
P(x,t) \doteq \frac{1}{2}e^{-| x |} * \left [\frac{3 - \gamma}{2}
    u^{2}(x,t) + \frac{\gamma}{2} u_{2}^{2}(x,t) \right ].
\end{split}
\end{equation*}
%
\begin{framed}
We remark that $e^{-| x |}$ is a weak solution to the ode $(1 + \p_{x}^{2})u =
2\delta$. Taking Fourier transforms of both sides, it follows that
$\widehat{e^{-| \cdot |}}(\xi) = 2/(1 + \xi^{2})$. Hence,
%
%
\begin{equation*}
\begin{split}
\frac{1}{2} e^{-| x |} * f = (1 - \p_{x}^2)f
\end{split}
\end{equation*}
%
%
Therefore, the form of HR which we all know and love is equivalent to
\eqref{hyperelastic-rod-equation}.
\end{framed}
%
Following Bressan \cite{Bressan_2007_Global-conserva}, we let $\xi \in \rr$ be an energy variable, and for fixed $t \in \rr$ define the map $t \mapsto y(\xi, t)$ implicitly by 
%
%
\begin{equation}
\label{potent-def}
\begin{split}
    \int_{0}^{y(\xi, t)} [1 + u^{2}(x,t)]dx = \xi.
\end{split}
\end{equation}
%
%
Consider also the initial value problem
%
%
\begin{gather}
    \label{en-eq}
\frac{dy}{dt}(\xi, t) = u(y(\xi, t), t),
\\
\label{en-data}
y(\xi, 0) = \xi
\end{gather}
%
%
and define
\begin{gather}
    \label{var-1}
    w(\xi, t) \doteq \gamma u(y(\xi, t), t)
    \\
    \label{var-2}
    v(\xi, t) \doteq u_{x}(y(\xi, t), t)
    \\
    \label{var-3}
    q(\xi, t) \doteq y_{\xi}(\xi, t)
\end{gather}
where we adopt the notation $$u_{x}(y(\xi, t),t) \doteq \frac{d}{dz}u(z,t) \big
|_{z = y(\xi, t)}$$ for the remainder of the paper. Using ivp
\eqref{en-eq}-\eqref{en-data}, we will rewrite the HR ivp
\eqref{hr}-\eqref{hr-data} as an ode system for the variables \eqref{var-1},
\eqref{var-2}, and \eqref{var-3}. Proceeding, we note that by the chain rule
%
%
\begin{equation}
\label{w-deriv}
\begin{split}
\frac{d}{dt}w(\xi, t)
& = \gamma u_{y} y_{t}(y(\xi, t), t) + u_{t}(y(\xi, t), )
\\
& = \gamma u u_{x}(y(\xi, t ), t) + u_{t}(y(\xi, t))
\\
& = -P_{x}(y(\xi, t), t).
\end{split}
\end{equation}
%
%
Now, we want to get $P_{x}(y(\xi, t), t)$ in "nice" form as we will be estimating later. Notice that
%
%
\begin{equation*}
\begin{split}
P(x,t) = \frac{1}{2} \int_{\rr}e^{-| x - x_{1} |} \left [\frac{3 - \gamma}{2}
u^{2}(x_{1}, t) + \frac{\gamma}{2} u_{x}^{2}(x_{1}, t) \right ]
\end{split}
\end{equation*}
%
%
and so 
%
%
\begin{equation}
\label{yuu}
\begin{split}
P(y(\xi, t), t) = \frac{1}{2} \int_{\rr} e^{-| y(\xi, t) - x_{1} |} \left [ \frac{3 - \gamma}{2} u^{2} + \frac{\gamma}{2}u_{x}^{2} \right ] (x_{1}, t) d x_{1}.
\end{split}
\end{equation}
%
%
Furthermore
%
%
\begin{equation}
\begin{split}
\label{p}
P_{x}(x,t)
& = \frac{1}{2}\p_{x} e^{-| x |}* \left [ \frac{3 - \gamma}{2} u^{2} + \frac{\gamma}{2} u_{x}^{2} \right ] 
\\
& = \frac{1}{2} \int_{\rr} \sgn(x - x_{1}) e^{-| x - x_{1} |} \left [ \frac{3 - \gamma}{2} u^{2}(x_{1}, t) + \frac{\gamma}{2} u_{x}^{2}(x_{1}, t) \right ] 
\\
& =\frac{1}{2} \left ( \int_{x}^{\infty} - \int_{-\infty}^{x} \right )
e^{-| x - x_{1} |} \left [ \frac{3 - \gamma}{2} u^{2}(x_{1}, t) +
\frac{\gamma}{2} u_{x}^{2}(x_{1}, t) \right ] dx_{1}
\end{split}
\end{equation}
%
%
and so%
%
\begin{equation}
\label{p-deriv}
\begin{split}
P_{x}(y(\xi, t), t)
& = \frac{1}{2} \left ( \int_{y(\xi, t)}^{\infty} - \int_{-\infty}^{y(\xi, t)} \right ) e^{-| y(\xi, t) - x_{1} |} \left [ \frac{3 - \gamma}{2} u^{2}(x_{1}, t) +
\frac{\gamma}{2} u_{x}^{2}(x_{1}, t) \right ] dx_{1}.
\end{split}
\end{equation}
%
%
Adopt the notation 
%
%
\begin{equation*}
\begin{split}
q(\xi, t) \doteq y_{\xi}(\xi, t).
\end{split}
\end{equation*}
%
%
Then applying the change of variable $x_{1} = y(\xi_{1}, t)$, we obtain
%
%
\begin{equation*}
\begin{split}
\eqref{yuu} & = \frac{1}{2} \int_{\rr} e^{-| y(\xi, t) - y(\xi_{1}, t) |} \left [
\frac{3 -\gamma}{2} w^{2}q + \frac{\gamma}{2} v^{2} q  \right ](\xi_{1}, t) d \xi_{1}
\\
& = \frac{1}{2} \int_{\rr} e^{-| \int_{\xi_{1}}^{\xi} q(\lambda, t) d \lambda |} \left [
\frac{3 -\gamma}{2} w^{2}q + \frac{\gamma}{2} v^{2} q  \right ](\xi_{1}, t) d \xi_{1}
\\
& \doteq Q = Q(v, w, q)(\xi, t).
\end{split}
\end{equation*}
%
%
Similarly
%
%
\begin{equation}
\label{R-def}
\begin{split}
\eqref{p-deriv} & = \frac{1}{2} \left ( \int_{\xi}^{\infty} -
\int_{-\infty}^{\xi} \right ) e^{-| \int_{\xi_{1}}^{\xi} q(\lambda, t) d \lambda 
|} \left [ \frac{3 - \gamma}{2} w^{2} q + \frac{\gamma}{2} v^{2} q \right ] (\xi_{1}, t) d \xi_{1}
\\
& \doteq R = R(v, w, q)(\xi, t). 
\end{split}
\end{equation}
%
Substituting this into \eqref{w-deriv}, we see that
%
%
\begin{equation}
\label{w-ode}
\begin{split}
\frac{d}{dt}w(\xi, t) = -R(w, v, q)(\xi, t).
\end{split}
\end{equation}
%
%
Now, we find analogues of \eqref{w-ode} for $v$ and $q$. Observe that by the
chain rule 
%
\begin{equation}
\label{lkk}
\begin{split}
\frac{d}{dt}v(\xi, t)
&  = \frac{du_{x}}{dy}(y(\xi, t), t) \frac{dy}{dt}(\xi, t) + u_{xt}(y(\xi, t), t)
\\
& =  (u_{xx}u + u_{xt})(y(\xi, t), t).
\end{split}
\end{equation}
%
Differentiating \eqref{hyperelastic-rod-equation} formally with respsect to $x$,
and using the fact that
%
%
\begin{equation*}
\begin{split}
P_{xx}
&  = \frac{1}{2} \p_{x}^{2}e^{-| x |}* \left [ \frac{3 - \gamma}{2}u^{2} + \frac{\gamma}{2} u_{x}^{2} \right ] 
\\
& = \frac{1}{2} \left [ e^{-| x |} - 2 \delta \right ] * \left [ \frac{3 - \gamma}{2}u^{2} + \frac{\gamma}{2} u_{x}^{2} \right ] 
\\
& = \frac{1}{2} e^{-| x |} * \left [ \frac{3 - \gamma}{2}u^{2} +
\frac{\gamma}{2} u_{x}^{2} \right ] - \left [ \frac{3 -
\gamma}{2}u^{2} + \frac{\gamma}{2} u_{x}^{2} \right ] 
\\
& = P - \left [ \frac{3 -
\gamma}{2}u^{2} + \frac{\gamma}{2} u_{x}^{2} \right ] 
\end{split}
\end{equation*}
%
%
we obtain
%
%
\begin{equation*}
\begin{split}
u_{xt} + u u_{xx} + u_{x}^{2} + P - \left [ \frac{3 -
\gamma}{2}u^{2} + \frac{\gamma}{2} u_{x}^{2} \right ]  = 0.
\end{split}
\end{equation*}
%
%
Substituting this into \eqref{lkk}, we get
%
%
\begin{equation*}
\begin{split}
\frac{d}{dt}v(\xi, t)
& = \left \{ u_{xx} u + \big[-uu_{xx} - u_{x}^{2} - P + \left ( \frac{3 - \gamma}{2} u^{2} + \frac{\gamma}{2}u_{x}^{2} \right ) ] \right \}(y(\xi, t), t)
\\
& = \left \{- P  + \frac{3 - \gamma}{2} u^{2} + \frac{\gamma-2}{2} u_{x}^{2} \right \}(y(\xi, t), t)
\\
& = \left \{-Q + \frac{3- \gamma}{2 \gamma^{2}}w^{2} + \frac{\gamma-2}{2}
v^{2} \right \}(\xi, t).
\end{split}
\end{equation*}
%
Lastly,
%
%
%
%
\begin{equation*}
\begin{split}
\frac{dq}{dt}(\xi, t)
& = \frac{d}{d\xi}\frac{d}{dt}y(\xi, t)
\\
& = \frac{d}{d \xi}u(y(\xi, t), t)
\\
& = \frac{du}{dy}(y(\xi,t),t)
\\
& = u_{x}(y(\xi, t), t)q(\xi, t)
\\
& = \frac{vq}{\gamma}(\xi,t).
\end{split}
\end{equation*}
%
%
Next, note that $y(\xi, 0) = 0$, and so
%
%
\begin{equation*}
\begin{split}
w(\xi, 0)
& = \gamma u(y(\xi, 0), 0)
\\
& = \gamma u(\xi, 0)
\\
& = \gamma u_{0}(\xi).
\end{split}
\end{equation*}
%
%


Also, 
%
%
\begin{equation*}
\begin{split}
v(\xi, 0)
& = u_{x}(y(\xi, 0), 0)
\\
& = u_{x}(\xi, 0)
\\
& = u_{0}'(\xi)
\end{split}
\end{equation*}
%
%
and
%
%
\begin{equation*}
\begin{split}
q(\xi,0)
& = y_{\xi}(\xi, 0)
\\
& = 1.
\end{split}
\end{equation*}
%
%
Hence, we are interested in solving the ode Cauchy-problem
%
%
%
%
\begin{gather}
\label{ode-system}
\frac{d}{dt}(w, v, q) = \left ( -R, -Q + \frac{3 - \gamma}{2 \gamma^{2}}w^{2} + \frac{\gamma -2}{2}v^{2}, \frac{vq}{\gamma} \right ) 
\\
\label{ode-system-init}
(w, v, q)(0) = (\gamma u_{0}(\xi), u_{0}'(\xi), 1).
\end{gather}
%
Denoting
%
%
\begin{equation*}
\begin{split}
  Y = H^{1} \times L^{\infty} \cap L^{2} \times L^{\infty},
\end{split}
\end{equation*}
%
%
we recall the following.
%
\begin{proposition}[Metric Space ODE Theorem]
	\label{prop:ode-thm}
  Let $X$ be a topological vector space over $\rr$
  with topology induced by a metric $d$, $E \subset X$ open, and $(-a, a)$ an
	open interval in $\rr$. Suppose $f: (-a, a) \times E \to X$ satisfies the
	inequality
	%
	%
	\begin{equation}
		\label{stronger-ode}
		\begin{split}
      d[f(t, x), f(t, y)] \le c d(x, y), \qquad \forall t \in (-a, a),
			\qquad \forall x, y \in E
		\end{split}
	\end{equation}
	%
  Then for given $\vp \in E$, there exists sufficiently small $h > 0$ and a
  unique differentiable map $u: (-h, h) \to E$ which for all $t \in (-h, h)$
  satisfies 
	%
	\begin{gather}
    \label{ode-thm-eq}
			u'(t) = f(t, u(t)),
			\\
      \label{ode-thm-init-data}
			u(0) = \vp.
	\end{gather}
\end{proposition}
We will obtain the following result as a corollary. 
%
%
%%%%%%%%%%%%%%%%%%%%%%%%%%%%%%%%%%%%%%%%%%%%%%%%%%%%%
%
%
%				Existence of Solution to ODE system
%
%
%%%%%%%%%%%%%%%%%%%%%%%%%%%%%%%%%%%%%%%%%%%%%%%%%%%%%
%
%
\begin{theorem}
  Assume $u_{0} \in H^{1} \cap C^{1}$. Then for sufficiently small $T > 0$
the Cauchy problem \eqref{ode-system}-\eqref{ode-system-init} has a unique
solution $(w, v, q) \in C^{1}([0, T], Y)$.
\label{thm:ode-sys-sol}
\end{theorem}
%
%
\begin{proof}
  Due to \eqref{prop:ode-thm}, it will
  be will be enough to show that the map $$(w, v, q) \mapsto
\left ( -R, -Q + \frac{3 - \gamma}{2 \gamma^{2}}w^{2} + \frac{\gamma
-2}{2}v^{2}, \frac{vw}{\gamma} \right ) \doteq F(w, v, q)$$ is Lipschitz on
an open subset $U$ of $Y$. 
%
Using the notation $Q_{1} \doteq Q(w_{1}, v_{1}, q_{1})$ and $Q_{2} \doteq
Q(w_{2}, v_{2}, q_{2})$, we have
%
%
\begin{equation}
  \label{lip-diff}
\begin{split}
  & (w_{2}, v_{2}, q_{2}) - (w_{1}, v_{1}, q_{1})
  \\
  & = \left( -(R_{2} - R_{1}),
  -(Q_{2} - Q_{1}) + \frac{3 - \gamma}{2 \gamma^{2}}(w_{2}^{2} -
  w_{1}^{2}) + \frac{\gamma -2}{2} (v_{2}^{2} - v_{1}^{2}),
  \frac{1}{\gamma}(v_{2} w_{2} - v_{1} w_{1}) \right).
\end{split}
\end{equation}
%
Let $r>0$, $w_{1}, w_{2} \in B_{C^{1} \cap L^{2}}(r)$, $v_{1}, v_{2} \in B_{L^{\infty} \cap L^{2}}(r)$, and
$q_{1}, q_{2} \in L^{\infty}(r)$, where $0 < c < q_{1} < r$ and $0 < c < q_{2} <r$.
Note that $L^{\infty} \cap L^{2}$ is an algebra, since
%
%
\begin{equation*}
\begin{split}
  \| fg \|_{L^{\infty} \cap L^{2}} & = \| fg \|_{L^{\infty}} + \| fg \|_{L^{2}}
  \\
  & \le \| f \|_{L^{\infty}} \| g \|_{L^{\infty}} + \| f \|_{L^{2}}\| g \|_{L^{\infty}}
  \\
  & \le \| f \|_{L^{\infty} \cap L^{2}} \| g \|_{L^{\infty} \cap L^{2}}.
\end{split}
\end{equation*}
%
%
Hence
%
%
%
\begin{equation}
  \label{1aa}
\begin{split}
  \| v_{2}w_{2} - v_{1}w_{1} \|_{L^{\infty} \cap L^{2}}
  & = \| v_{2}w_{2} \pm v_{2}w_{1} - v_{1}w_{1} \|_{L^{\infty} \cap L^{2}}
  \\
  & \le \| v_{2}(w_{2} - w_{1}) \|_{L^{\infty} \cap L^{2}}  
  + \| w_{1}(v_{2} - v_{1}) \|_{L^{\infty} \cap L^{2}}  
  \\
  & \lesssim_{r} \| w_{2} - w_{1} \|_{L^{\infty} \cap L^{2}} +
  \| v_{2} - v_{1} \|_{L^{\infty} \cap L^{2}}
\end{split}
\end{equation}
%
%
and
%
\begin{equation}
  \label{2aa}
\begin{split}
  \| w^{2}_{2} - w_{1}^{2} \|_{L^{\infty} \cap L^{2}} 
  & = \| (w_{2} - w_{1})(w_{2} + w_{1}) \|_{L^{\infty} \cap L^{2}}
    \\
    & \lesssim_{r} \| w_{2} - w_{1} \|_{L^{2} \cap L^{\infty} \cap L^{2}}.
\end{split}
\end{equation}
%
Similarly
%
%
\begin{equation}
  \label{3aa}
\begin{split}
\| v^{2}_{2} - v_{1}^{2} \|_{L^{\infty} \cap L^{2}} 
& \lesssim_{r}\| v_{2} - v_{1} \|_{L^{\infty} \cap L^{2}}.
\end{split}
\end{equation}
%
%
It remains to estimate $(R_{2} - R_{1})$ and $(Q_{2} - Q_{1})$ in $H^{1}$
and $L^{\infty} \cap L^{2}$, respectively. First, note that
%
%
\begin{equation}
\begin{split}
  \| Q_{2} - Q_{1} \|_{L^{\infty}}
  & \le \| \frac{3 - \gamma}{2} (w_{2}^{2} q_{2} - w_{1}^{2} q_{1}) +
  \frac{\gamma}{2}(v_{2}^{2}q_{2} - v_{1}^{2}q_{1}) \|_{L^{1}}
\end{split}
\end{equation}
%
%
Now
%
%
\begin{equation*}
\begin{split}
  w_{2}^{2} q_{2} - w_{1}^{2} q_{1}
  & = w_{2}^{2} q_{2} - w_{1}^{2} q_{2} + w_{1}^{2} q_{2} - w_{1}^{2} q_{1}
  \\
  & = (w_{2} - w_{1})(w_{2} + w_{1})q_{2} + (q_{2} - q_{1})w_{1}^{2}.
\end{split}
\end{equation*}
%
%
Similarly
%
%
\begin{equation*}
\begin{split}
  v_{2}^{2}q_{2} - v_{1}^{2} q_{1} = (v_{2} - v_{1})(v_{2}  + v_{1})q_{2} + (q_{2} - q_{1})v_{1}^{2}.
\end{split}
\end{equation*}
%
%
Hence, applying H\'older, we have 
\begin{equation}
  \label{q-init-bound}
  \begin{split}
  & \| \frac{3 - \gamma}{2} (w_{2}^{2} q_{2} - w_{1}^{2} q_{1}) +
  \frac{\gamma}{2}(v_{2}^{2}q_{2} - v_{1}^{2}q_{1}) \|_{L^{1}}
  \\
  & \lesssim \| q_{2} \|_{L^{\infty}} \| (w_{2} - w_{1})(w_{2} + w_{1})
  + (v_{2} - v_{1})(v_{2} + v_{1}) \|_{L^{1}}
  + \| q_{2} - q_{1} \|_{L^{\infty}} \| w_{1}^{2} + v_{1}^{2}\|_{L^{1}} 
  \\
  & \le \| q_{2} \|_{L^{\infty}} \| (w_{2} - w_{1})(w_{2} + w_{1})
  + (v_{2} - v_{1})(v_{2} + v_{1}) \|_{L^{1}}
  + \| q_{2} - q_{1} \|_{L^{\infty}} (\| w_{1}\|_{L^{2}}^{2} + \|
  v_{1}\|_{L^{2}}^{2}) 
  \\
  & \lesssim_{r} \| v_{2} - v_{1} \|_{L^{2}} + \| q_{2} - q_{1} \|_{L^{\infty}}
\end{split}
\end{equation}
where the last step follows from Cauchy-Schwartz and the a priori bounds
on $w_{i}, v_{i}, q_{i}$, $i \in \{1,2\}$. Therefore, 
%
%
\begin{equation}
  \label{pi}
\begin{split}
  \| Q_{2} - Q_{1} \|_{L^{\infty}} \lesssim_{r} \| v_{2} - v_{1} \|_{L^{2}} + \| q_{2} - q_{1} \|_{L^{\infty}}.
\end{split}
\end{equation}
%
%
Also, note that
%
%
%
%
\begin{equation}
  \label{I-II-split}
\begin{split}
  & Q_{2} - Q_{1}
  \\
  & = \frac{1}{2} \int_{\rr} \left\{ e^{-| \int_{\xi_{1}}^{\xi}
  q_{2}(\lambda) d \lambda|}\left[ \frac{3 - \gamma}{2} w_{2}^{2} q_{2} +
    \frac{\gamma}{2} v_{2}^{2} q_{2} \right] - e^{-| \int_{\xi_{1}}^{\xi}
    q_{1}(\lambda) d \lambda |} \left[ \frac{3 - \gamma}{2} w_{1}^{2}
      q_{1} + \frac{\gamma}{2} v_{1}^{2} q_{1} \right] \right\} d \xi_{1}
      \\
      & = I + II
\end{split}
\end{equation}
%
%
where
\begin{gather*}
  I = \int_{\rr} e^{-| \int_{\xi_{1}}^{\xi} q_{1}(\lambda) |} \left[ \frac{3-
  \gamma}{2}(w_{2}^{2} q_{2} - w_{1}^{2} q_{1}) +
  \frac{\gamma}{2}(v_{2}^{2} q_{2} - v_{1}^{2} q_{1}) \right]
  \\
  II = \int_{\rr} f\left[ \frac{3-\gamma}{2}w_{2}^{2}q_{2} + \frac{\gamma}{2}
    v_{2}^{2} q_{2} \right]d \xi_{1}, \quad
    f = e^{-| \int_{\xi_{1}}^{\xi} q_{2}(\lambda) d \lambda |} -
      e^{- | \int_{\xi_{1}}^{\xi} q_{1}(\lambda) d \lambda |}.
\end{gather*}
Observe that
%
\begin{equation}
  \label{q-l2-pre}
\begin{split}
  \| I \|_{L^{2}}^{2} & = \frac{1}{4} \int_{\rr} | \int_{\rr} e^{-| \int_{\xi_{1}}^{\xi}q_{1}(\lambda) d \lambda |}  \left[ \frac{3 - \gamma}{2}(w_{2}^{2} q_{2} -
  w_{1}^{2}q_{1}) + \frac{\gamma}{2}(v_{2}^{2} q_{2} - v_{1}^{2} q_{1}) \right]  d
  \xi_{1} |^{2} d \xi.
\end{split}
\end{equation}
%
%We shall first bound the exponential by analyzing it's argument. More precisley, recall that
%%
%%
%\begin{equation*}
%\begin{split}
  %\int_{0}^{y(\xi, t)} \left[ 1 + u(x, t) \right]^{2} dx = \xi.
%\end{split}
%\end{equation*}
%%
%%
%Differentiating both sides with respect to $\xi$ gives
%%
%%
%\begin{equation*}
%\begin{split}
  %y_{\xi} = \frac{1}{1 + y^{2}}.
%\end{split}
%\end{equation*}
%%%
%%%
%Fix $t$ and assume without loss of generality that $\xi \ge \xi_{1}$. If $y(s) \le 1$ for all $s \in [\xi_{1}, \xi]$, then
%%
%%
%\begin{equation*}
%\begin{split}
  %| y(\xi) - y(\xi_{1}) | 
  %& = | \int_{\xi_{1}}^{\xi} y_{s} ds|
  %\\
  %& = | \int_{\xi_{1}}^{\xi} \frac{1}{1 + y^{2}} ds |
  %\\
  %& \ge \frac{1}{2}| \xi - \xi_{1} |.
%\end{split}
%\end{equation*}
%%%
%%%
%Otherwise, since $y(s)$ is continuous and increasing, there exists $\xi_{1} \le \xi^{*} \le \xi$ such that $y(\xi^{*}) \ge 1$. Then
%%%
%%%
%\begin{equation*}
%\begin{split}
  %| y(\xi) - y(\xi_{1}) | & = | \int_{\left\{ s \in [\xi_{1}, \xi^{*}]: y \le 1
  %\right\}} \frac{1}{1 + y^{2}} ds + 
  %\int_{\left\{ s \in [\xi_{1}, \xi^{*}]: y \ge 1    
  %\right\}}  \frac{1}{1 + y^{2}} ds | 
  %\\
  %& \ge | \int_{\left \{ s \in [\xi_{1}, \xi^{*}]: y \le 1 \right\}}
  %\frac{1}{1 + y^{2}} ds + 
  %\int_{\left\{ s \in [\xi_{1}, \xi^{*}]: 1 \le y \le 2 \right\}} \frac{1}{1 +
  %y^{2}} ds \\
  %& \ge \frac{1}{2} | \xi^{*} - \xi_{1} | + \frac{1}{5} | \xi - \xi^{*} |
  %\\
  %& \ge \frac{2}{5} | \xi - \xi_{1} |.
%\end{split}
%\end{equation*}
%%
%%
%%
Since $0 < c_{1} < q_{1}$, we bound \eqref{q-l2-pre} by
%
%
\begin{equation*}
\begin{split}
  & \frac{1}{4} \int_{\rr} \left( \int_{\rr} | e^{-c_{1} | \xi - \xi_{1} |} \left[ \frac{3 - \gamma}{2}(w_{2}^{2} q_{2} - w_{1}^{2}q_{1}) + \frac{\gamma}{2}(v_{2}^{2} q_{2} - v_{1}^{2} q_{1}) \right] | d \xi_{1} \right)^{2} d \xi
  \\
  & = \frac{1}{4} \| e^{-c_{1} | \cdot |} * \left[ \frac{3 - \gamma}{2}(w_{2}^{2} q_{2} - w_{1}^{2}q_{1}) + \frac{\gamma}{2}(v_{2}^{2} q_{2} - v_{1}^{2} q_{1}) \right]  \|_{L^{2}}^{2}.
\end{split}
\end{equation*}
%
Applying Young's inequality, we obtain the bound
%
%
\begin{equation*}
\begin{split}
  \| e^{-c_{1} | x |    } \|_{L^{2}}^{2} \|  \frac{3 -
  \gamma}{2}(w_{2}^{2} q_{2} - w_{1}^{2}q_{1}) + \frac{\gamma}{2}(v_{2}^{2} q_{2} -
  v_{1}^{2} q_{1})\|_{L^{1}}^{2}
  & \lesssim \|  \frac{3 -
  \gamma}{2}(w_{2}^{2} q_{2} - w_{1}^{2}q_{1}) + \frac{\gamma}{2}(v_{2}^{2} q_{2} -
  v_{1}^{2} q_{1})\|_{L^{1}}^{2}
  \\
  & \lesssim_{r}
  (\| v_{2} - v_{1} \|_{L^{2}} + \| q_{2} - q_{1} \|_{L^{\infty}})^{2}
\end{split}
\end{equation*}
%
%
where the last step follows from \eqref{q-init-bound}. Therefore
%
%
\begin{equation}
  \label{I-est}
  \| I \|_{L^{2}} \lesssim_{r}
  \| v_{2} - v_{1} \|_{L^{2}} + \| q_{2} - q_{1} \|_{L^{\infty}}
\end{equation}
To bound $II$, we first note that
%
%
\begin{equation*}
\begin{split}
  f & = e^{- | \int_{\xi_{1}}^{\xi} q_{1}(\lambda) d \lambda|} \left[
    e^{| \int_{\xi_{1}}^{\xi} q_{1}(\lambda) |}e^{-|
      \int_{\xi_{1}}^{\xi}q_{2}(\lambda) d \lambda |} - 1
    \right]
    \\
    & = e^{- | \int_{\xi_{1}}^{\xi} q_{1}(\lambda) d \lambda |} \left(
    \sum_{n = 1}^{\infty} \frac{z^{n}}{n!}
    \right)
\end{split}
\end{equation*}
%
%
where
%
%
\begin{equation*}
\begin{split}
  z = | \int_{\xi_{1}}^{\xi} q_{1}(\lambda)d \lambda | - | \int_{\xi_{1}}^{\xi} q_{2}(\lambda) d \lambda |.
\end{split}
\end{equation*}
%
%
By the reverse triangle inequality
%
%
\begin{equation*}
\begin{split}
| z | & \le | \int_{\xi_{1}}^{\xi} q_{1}(\lambda)d \lambda  -  \int_{\xi_{1}}^{\xi} q_{2}(\lambda) d \lambda |
\\
& = | \int_{\xi_{1}}^{\xi} [q_{1} -  q_{2}](\lambda) d \lambda |
\\
& \le \| q_{1} - q_{2} \|_{L^{\infty}} | \xi - \xi_{1} |.
\end{split}
\end{equation*}
%
%
Due to the fact
that $0 < c_{1} <q_{1}$, we also have
%
%
\begin{equation*}
\begin{split}
  | \int_{\xi_{1}}^{\xi} q_{1}(\lambda) d \lambda | \ge c_{1} | \xi - \xi_{1} |.
\end{split}
\end{equation*}
%
%
Hence
%
%
%
\begin{equation*}
\begin{split}
  | f | &  \le e^{-c_{1} | \xi - \xi_{1} |} \sum_{n = 1}^{\infty} \frac{\left[ \| q_{2} - q_{1} \|_{L^{\infty}}| \xi - \xi_{1} | \right]^{n}}{n!}
  \\
  & \le e^{-c_{1} | \xi - \xi_{1} |} \| q_{2} - q_{1} \|_{L^{\infty}} \sum_{n =
  1}^{\infty} \frac{(\| q_{2} - q_{1} \|_{L^{\infty}})^{n-1}| \xi - \xi_{1} |^{n}
    }{n!}
\\
&  \le e^{-c_{1} | \xi - \xi_{1} |} \| q_{2} - q_{1} \|_{L^{\infty}} \sum_{n =
1}^{\infty} \frac{(2c_{2} )^{n-1}| \xi - \xi_{1} |^{n}
    }{n!}
    \\
    & \le e^{-c_{1}| \xi - \xi_{1} |} \| q_{2} - q_{1} \|_{L^{\infty}} e^{2c_{2}| \xi - \xi_{1} |}
    \\
    & = e^{-2c_{1} + c_{2}| \xi - \xi_{1} |} \| q_{2} - q_{1} \|_{L^{\infty}}.
\end{split}
\end{equation*}
%
%
Choose $c_{1}, c_{2}$ such that $-2c_{1} + c_{2} = -1/4$. Then we obtain
%
%
\begin{equation*}
\begin{split}
  | f | \le e^{-\frac{1}{4} | \xi - \xi_{1} |} \| q_{2} - q_{1} \|_{L^{\infty}}
\end{split}
\end{equation*}
%
%
and so
%
%
%
%
\begin{equation*}
\begin{split}
  \| II \|_{L^{2}}^{2}
  & = \| q_{2} - q_{1} \|_{L^{\infty}}^{2} \int_{\rr} |\int_{\rr} f\left[ \frac{3-\gamma}{2}w_{2}^{2}q_{2} + \frac{\gamma}{2}
    v_{2}^{2} q_{2} \right]d \xi_{1} | ^{2} d \xi  
    \\
    & \le \| q_{2} - q_{1} \|_{L^{\infty}}^{2} \int_{\rr} \left (\int_{\rr} e^{-\frac{1}{4}| \xi - \xi_{1} |} |\left[ \frac{3-\gamma}{2}w_{2}^{2}q_{2} + \frac{\gamma}{2}
    v_{2}^{2} q_{2} \right] | d \xi_{1} \right ) ^{2} d \xi 
    \\
    & = \| q_{2} - q_{1} \|_{L^{\infty}}^{2} \| e^{-\frac{1}{4} | \cdot |} * \left[ \frac{3 - \gamma}{2}w_{2}^{2} q_{2}
    + \frac{\gamma}{2}v_{2}^{2} q_{2}  \right]  \|_{L^{2}}^{2}.
  \end{split}
\end{equation*}
%
%
Applying Young's inequality, we bound this by
%
%
%
%
\begin{equation*}
\begin{split}
\| e^{-\frac{1}{4} | \cdot |} * \left[ \frac{3 - \gamma}{2}w_{2}^{2} q_{2}
    + \frac{\gamma}{2}v_{2}^{2} q_{2}  \right]  \|_{L^{2}}^{2}
    & \le \| e^{-\frac{1}{4}| x |} \|_{L^{2}}^{2} \| \frac{3 - \gamma}{2}w_{2}^{2} q_{2}
    + \frac{\gamma}{2}v_{2}^{2} q_{2}\|_{L^{1}}^{2}
    \\
    & \lesssim \| q_{2} \|_{L^{2}}^{2} \| \frac{3 - \gamma}{2} w_{2}^{2} + \frac{\gamma}{2} v_{2}^{2} \|_{L^{1}}
    \\
    & \lesssim_{r} 1 
\end{split}
\end{equation*}
%
since $w_{2}, v_{2}, q_{2} \in B_{L^{2}}(r)$.
%
Therefore
%
%
\begin{equation}
  \label{II-est}
\begin{split}
  \| II \|_{L^{2}} \lesssim_{r} \| q_{2} - q_{1} \|_{L^{\infty}}.
\end{split}
\end{equation}
%
%
Combining \eqref{I-est}, \eqref{II-est} and recalling \eqref{I-II-split}, we obtain
%
%
%
%
\begin{equation}
  \label{yi}
\begin{split}
  \| Q_{2} - Q_{1} \|_{L^{2}} \lesssim_{r} \| v_{2} - v_{1} \|_{L^{2}} + \| q_{2} - q_{1} \|_{L^{\infty}}
\end{split}
\end{equation}
%
%
Combining \eqref{pi} and \eqref{yi}, we see that
%
%
\begin{equation}
  \label{Q-diff-fin-est}
\begin{split}
  \| Q_{2} - Q_{1} \|_{L^{2} \cap L^{\infty}} \lesssim_{r} \| v_{2} - v_{1}
  \|_{L^{2}} + \| q_{2} - q_{1} \|_{L^{\infty}}
\end{split}
\end{equation}
%
%
Next, we turn our attention to the term 
$(R_{2} - R_{1})$. Recall that if
%
%
\begin{equation*}
\begin{split}
F(\xi) = \int_{a(\xi)}^{b(\xi)} f(\xi, \tau) d \tau
\end{split}
\end{equation*}
%
%
with $a(\xi), b(\xi)$ continuous over some region $[\xi_{0}, \xi_{1}]$ and $f, f_{\xi}$ continuous over $[\xi_{0}, \xi_{1}] \times [\tau_{0}, \tau_{1}]$, then by the Leibniz integral rule, we have
%
%
\begin{equation*}
\begin{split}
F'(\xi) = b'(\xi) f(b(\xi)) - a'(\xi) f(a(\xi)) + \int_{a(\xi)}^{b(\xi)} f_{\xi}(\xi, \tau) d \tau.
\end{split}
\end{equation*}
%
%
Recall the definition of $R$ in \eqref{R-def}, and let $f(\xi, \xi_{1}, t)$ denote
its integrand. Then
%
%
\begin{equation*}
\begin{split}
\frac{d}{d \xi} \int_{\xi}^{\infty}f(\xi, \xi_{1}, t) d \xi_{1}
& = -f(\xi, \xi, t) + \int_{\xi}^{\infty} f_{\xi}(\xi, \xi_{1}, t) d \xi_{1}.
\end{split}
\end{equation*}
%
%
Similarly
\begin{equation*}
\begin{split}
\frac{d}{d \xi} \int_{-\infty}^{\xi}f(\xi, \xi_{1}, t) d \xi_{1}
& = f(\xi, \xi, t) + \int_{-\infty}^{\xi} f_{\xi}(\xi, \xi_{1}, t) d \xi_{1}.
\end{split}
\end{equation*}
Note that
\begin{gather*}
    f(\xi, \xi, t) = \left [ \frac{3 - \gamma}{2}w^{2}q + \frac{\gamma}{2} v^{2} q \right ](\xi, t)
\end{gather*}
and
%
%
\begin{equation}
\label{oii}
\begin{split}
f_{\xi}(\xi, \xi_{1}, t) = \sgn \left [ \int_{\xi_{1}}^{\xi} q(\lambda, t) d \lambda \right ]q(\xi, t) \left [ \frac{3- \gamma}{2} w^{2} q + \frac{\gamma}{2} v^{2}q \right ] (\xi_{1}, t).
\end{split}
\end{equation}
%
%
Now, from \eqref{potent-def}, it follows that $y(\xi, t)$ is an increasing function of $\xi$. Therefore, 
%
%
\begin{equation*}
\begin{split}
\sgn \left [ \int_{\xi_1}^{\xi} q(\lambda, t) d \lambda \right ]  = \sgn(\xi - \xi_1)
\end{split}
\end{equation*}
%
%
which in conjunction with \eqref{oii} gives
%
%
\begin{equation*}
\begin{split}
\int_{\xi}^{\infty} f_{\xi}(\xi, \xi_{1}, t) d \xi_{1} - \int_{-\infty}^{\xi} f_{\xi}(\xi, \xi_{1}, t) d \xi_{1}
& = q(\xi, t) \int_{-\infty}^{\infty} \left [ \frac{3- \gamma}{2} w^{2} q + \frac{\gamma}{2} v^{2}q \right ] (\xi_{1}, t).
\\
& = q Q(\xi, t).
\end{split}
\end{equation*}
%
%
Therefore,
%
%
\begin{equation*}
\begin{split}
R_{\xi}(\xi, t) = \frac{1}{2} \left [ \frac{3- \gamma}{2} w^{2} + \frac{\gamma}{2} v^{2} + Q \right ]q(\xi, t)
\end{split}
\end{equation*}
%
and so
%
%
\begin{equation*}
\begin{split}
  (R_{2})_{\xi} - (R_{1})_{\xi} 
  & \simeq \left[ \frac{3 - \gamma}{2} w_{2}^{2} +
  \frac{\gamma}{2}v_{2}^{2} + Q_{2}  \right]q_{2} - \left[ \frac{3 - \gamma}{2}
  w_{1}^{2} + \frac{\gamma}{2}v_{1}^{2} + Q_{1}  \right]q_{1}  
  \\
  & = \left[ \frac{3 - \gamma}{2} (w_{2}^{2} - w_{1}^{2}) +
  \frac{\gamma}{2}(v_{2}^{2} - v_{1}^{2}) + Q_{2} - Q_{1}  \right]q_{1} + \left[
  \frac{3 + \gamma}{2} w_{2}^{2} + \frac{\gamma}{2}v_{2}^{2} + Q_{2}
  \right](q_{2} - q_{1}).  
\end{split}
\end{equation*}
%
By H\"older and the algebra property of $L^{\infty}$, we have the estimates
%
%
%
\begin{equation*}
\begin{split}
  \| (w_{2}^{2} - w_{1}^{2})q_{1} \|_{L^{2}}
  & = \| (w_{2} - w_{1})(w_{2} + w_{1})q \|_{L^{2}} \le \| w_{2} - w_{1} \|_{L^{2}} 
  \| w_{2} + w_{1} \|_{L^{\infty}} \| q_{2} \|_{L^{\infty}}
  \\
  & \lesssim_{r} \| w_{2} - w_{1} \|_{L^{2}}
\end{split}
\end{equation*}
%
%
and 
%
\begin{equation*}
\begin{split}
  \| (v_{2}^{2} - v_{1}^{2})q_{1} \|_{L^{2}}
  & = \| (v_{2} - v_{1})(v_{2} + v_{1})q \|_{L^{2}}
  \\
  & \le \| v_{2} - v_{1} \|_{L^{2}} 
  \| v_{2} + v_{1} \|_{L^{\infty}} \| q_{2} \|_{L^{\infty}}
  \\
  & \lesssim_{r} \| v_{2} - v_{1} \|_{L^{2}}.
\end{split}
\end{equation*}
Applying H\"older and \eqref{yi}, we also have
%
%
%
\begin{equation*}
\begin{split}
  \| (Q_{2} - Q_{1})q_{1} \|_{L^{2}} 
  & \le \| Q_{2} - Q_{1} \|_{L^{2}} \| q_{1} \|_{L^{\infty}}
  \\
  & \le (\| v_{2} - v_{1} \|_{L^{2}} + \| q_{2} - q_{1} \|_{L^{\infty}})\| q_{1} \|_{L^{\infty}}
  \\
  & \lesssim_{r} \| v_{2} - v_{1} \|_{L^{2}} + \| q_{2} - q_{1} \|_{L^{\infty}}.
\end{split}
\end{equation*}
%
%
Lastly,
%
\begin{equation*}
\begin{split}
 \| \left[
  \frac{3 + \gamma}{2} w_{2}^{2} + \frac{\gamma}{2}v_{2}^{2} + Q_{2}
  \right](q_{2} - q_{1}) \|_{L^{\infty}}
  & \lesssim_{r} \| q_{2} - q_{1} \|_{L^{\infty}} 
\end{split}
\end{equation*}
%
%
Hence
%
%
\begin{equation*}
\begin{split}
  \| (R_{2})_{\xi} - (R_{1})_{\xi} \|_{L^{\infty}} \lesssim_{r} \| w_{2} - w_{1} \|_{L^{\infty}} + \| v_{2} - v_{1} \|_{L^{\infty}}
+ \| q_{2} - q_{1} \|_{L^{\infty}}.
\end{split}
\end{equation*}
%
%
which implies
\begin{equation}
  \label{R-diff-fin-est}
\begin{split}
  \| R_{2} - R_{1} \|_{H^{1}} \lesssim_{r} \| w_{2} - w_{1} \|_{L^{\infty}} + \| v_{2} - v_{1} \|_{L^{\infty}}
+ \| q_{2} - q_{1} \|_{L^{\infty}}.
\end{split}
\end{equation}
%
%
Recalling \eqref{lip-diff}, combining estimates \eqref{1aa}-\eqref{3aa},
\eqref{Q-diff-fin-est}, and \eqref{R-diff-fin-est}, and applying the estimates
$\| f \|_{L^{\infty}} \lesssim \| f \|_{H^{1}}$ and $\| f \|_{L^{\infty}} \le \| f
\|_{L^{2} \cap L^{\infty}}$, we conclude that
%
%
%
\begin{equation*}
\begin{split}
  \|(w_{2}, v_{2}, q_{2}) - (w_{2}, v_{2}, q_{2})\|_{Y} \lesssim_{r} \| w_{2} - w_{1} \|_{H^{1}} + \| v_{2} - v_{1} \|_{L^{2} \cap L^{\infty}}  + \| q_{2} - q_{1} \|_{L^{\infty}}
\end{split}
\end{equation*}
%
%
ending the proof.
%
\end{proof} 
%
\section{Proofs of Lemmas} 
\label{sec:pf-lemmas}
%
%
%
\begin{proof}[Proof of Lemma~\ref{lem:frac-deriv}]
For the non-periodic case we have
%
%
\begin{equation*}
\begin{split}
  \| fg\|_{H^{r-1}}^{2}
  & = \int_{\rr} (1 + \xi^{2})^{r-1}| \int_{\rr}
  \wh{f}(\eta) \wh{g}( \xi - \eta) d \eta |^{2} d \xi
  \\
  & \le \int_{\rr} (1 + \xi^{2})^{r-1}\left [ \int_{\rr}
  | \wh{f}(\eta) |  | \wh{g}(\xi - \eta) | 
  d \eta \right ]^{2} d \xi
  \\
  & = \int_{\rr}  (1 + \xi^{2})^{r-1}\left [ \int_{\rr}
  | \wh{f}(\eta) |  | \wh{g}(\xi - \eta) | (1 +
  \eta^{2})^{\frac{1-s}{2}} (1 + \eta^{2})^{\frac{s-1}{2}}
  d \eta \right ]^{2} d \xi
  \\
  & \le \int_{\rr}  (1 + \xi^{2})^{r-1}\left [ \int_{\rr} 
  | \wh{f}(\eta) | | \wh{g}(\xi - \eta) | (1 +
  \eta^{2})^{\frac{1-s}{2}} (1 + \eta^{2})^{\frac{s}{2}}
  d \eta \right ]^{2} d \xi.
\end{split}
\end{equation*}
%
Applying Cauchy Schwartz in $\eta$, we bound this by
%
%
%
\begin{equation}
  \label{np-key-term}
\begin{split}
  \| f \|_{H^{s}}^{2} \int_{\rr}  (1 + \xi^{2})^{r-1}\int_{\rr} \frac{|
  \wh{g}(\xi - \eta) |^{2}}{(1 + \eta^{2})^{s-1}} d \eta d \xi.
  \end{split}
\end{equation}
%
%
Applying a change of variable, we get
%
\begin{equation*}
\begin{split}
  \| f \|_{H^{s}}^{2} \int_{\rr} (1 + \xi^{2})^{r-1} \int_{\rr}
\frac{| \wh{g}(\eta) |^{2}}{[1 + (\xi - \eta)^{2}]^{s-1}} d \eta d \xi
  \end{split}
\end{equation*}
which by Fubini is equal to
%
%
\begin{equation}
  \label{int-pre-calc-lem}
\begin{split}
  \|f \|_{H^{s}}^{2} \int_{\rr} | \wh{g}(\eta) |^{2} \int_{\rr} \frac{1}{\left[
  1 + (\xi - \eta)^{2} \right]^{s-1} (1 + \xi^{2})^{1-r}} d \xi d \eta.
\end{split}
\end{equation}
%
%
We now need the following lemma, whose proof is provided in the appendix.
%
%
\begin{lemma}
	\label{lem:calc}
 %
 Fix $p, q > 0$ such that $p +q >1$, and let $r =\min\left\{p, q, p+q-1
 \right\}$. Adopt the notation
 $\langle x - \alpha \rangle  \doteq 1 + | x - \alpha |$. Then 
 %
 \begin{enumerate}[(I)]
   \item
For $\alpha=\beta \ \text{or} \ p \neq 1 \ \text{or} \ q \neq 1$
 \begin{equation*}
\begin{split}
  & \int_{\rr} \frac{1}{\langle x - \alpha \rangle ^{p} \langle x -
  \beta \rangle
  ^{q}} d x
  \le \frac{c_{p,q}}{\langle \alpha - \beta \rangle ^{r}}, 
  \end{split}
\end{equation*}
  \item
    \begin{equation*}
  \int_{\rr} \frac{1}{\langle x - \alpha \rangle  \langle x - \beta
  \rangle} d x
  \le  \frac{4 \log \langle \alpha - \beta \rangle}{\langle \alpha - \beta
  \rangle}, \quad \alpha \neq \beta.
\end{equation*}
\end{enumerate}
\end{lemma}
To be able to apply the lemma to the integral term in \eqref{int-pre-calc-lem}, 
we must first check
that its conditions are met. Let $ s = 3/2 + \ee$, $r = 1- \delta$, $\ee > 0$, $
\delta \ge 0$ and observe that
%
%
\begin{equation*}
\begin{split}
2(s-1) + 2(1-r)
& = 2(s-r)
\\
& = 2[3/2 + \ee - (1 - \delta)]
\\
& = 2(1/2 + \ee + \delta)
\\
& = 1 + 2 \ee + 2 \delta > 1.
\end{split}
\end{equation*}
%
%
Furthermore, $2(s-1), 2(1-r) > 0$. Hence, Lemma~\ref{lem:calc} is applicable. 
Lastly, observe that 
%
%
\begin{equation*}
\begin{split}
  \min\left\{ 2(s-1), 2(1-r), 2(s-1) + 2(1-r) -1 \right\}
  & = \min\left\{ 1 + 2 \ee, 2 \delta, 2(\ee + \delta) \right\}
  \\
  & = \min\left\{ 1 + 2 \ee, 2 \delta\right\}
  \\
  & = 2 \delta, \quad \delta \le 1/2 + \ee.
\end{split}
\end{equation*}
%
%
Hence, the integral term of \eqref{int-pre-calc-lem} is bounded by
\begin{equation*}
\begin{split}
  C_{s,r} \int_{\rr}  | \wh{g}(\eta) |^{2} \int_{\rr} \frac{1}{\left[ 1
  + 2 |\eta| \right]^{2 \delta}} d \xi d \eta 
  & \lesssim
  \int_{\rr}  | \wh{g}(\eta) |^{2} \int_{\rr} \frac{1}{\left[ 1
  + |\eta| \right]^{2 \delta}} d \xi d \eta  
  \\
  & \le \int_{\rr}  | \wh{g}(\eta) |^{2} \int_{\rr} \frac{1}{\left[ 1
  + \eta^{2} \right]^{\delta}} d \xi d \eta  
  \\
  & \le \| w \|_{H^{-\delta}}^{2}
  \\
  & = \| w \|_{H^{r-1}}^{2}.
\end{split}
\end{equation*}
%
Note that this bound is valid for all $0 \le \delta \le 1/2 + \ee$, $\ee >
0$. Recasting this in terms of $r$ and $s$, we have 
$$1-r \le 1/2 + s - 3/2, \quad r \le 1, \ s > 3/2,$$ which is equivalent to 
$$s + r \ge 2,  \quad  r \le 1, \ s > 3/2.$$ Therefore, 
%
%
%
%
\begin{equation}
  \label{yhh}
\begin{split}
  \| f g \|_{H^{r-1}} \lesssim \| f \|_{H^{s}} \| g \|_{H^{r-1}},
  \quad s + r \ge 2, \ s > 3/2, \ r \le 1.
\end{split}
\end{equation}
We will now establish this bound for $r=s$ where $s > 3/2$, and then interpolate to obtain
bounds for the whole range $1 \le r \le s$, where again $s > 3/2$. We shall need the following. 
%
%
\begin{lemma}[Algebra Property]
  \label{lem:alg-prop}
If  $s>1/2$ then there is $c_s>0$ such that 
%
%
%
\begin{equation} \label{KP-com-est}
  \| fg\|_{H^{s}} \le c_s \| f \|_{H^{s}} \| g \|_{H^{s}}
\end{equation}
%
%
%
\end{lemma}
%
%
%
%
%
Hence,
%
%
\begin{equation}
  \label{pre-interp-1}
\begin{split}
  \| f g \|_{H^{s-1}}
  & \lesssim   \|f  \|_{H^{s-1}} \| g \|_{H^{s-1}}, \quad s >3/2
  \\
  & \le \| f \|_{H^{s}} \| g \|_{H^{s-1}}.
\end{split}
\end{equation}
%
%
%
%
%
%
%
We now wish to interpolate to obtain estimates for the whole range $1 \le r \le s$.
We need the following.
%
%
%%%%%%%%%%%%%%%%%%%%%%%%%%%%%%%%%%%%%%%%%%%%%%%%%%%%%
%
%
%                
%
%
%%%%%%%%%%%%%%%%%%%%%%%%%%%%%%%%%%%%%%%%%%%%%%%%%%%%%
%
%
\begin{proposition}[Sobolev Interpolation]
  For fixed $k \le q, m \le s$ suppose that \\ $T: H^{k} \to H^{m}$ continuously
and $T: H^{q} \to H^{s}$. Then\\ $T: H^{\theta q + (1 - \theta)k} \to H^{\theta
s + (1 - \theta) m}$ continuously for all $\theta \in [0,1)$.
\label{prop:sob-interp}
\end{proposition}
%
To apply Proposition~\ref{prop:sob-interp}, we note that \eqref{yhh}
and \eqref{pre-interp-1} imply
%
%
\begin{equation*}
\begin{split}
  \| f g \|_{H^{r-1}} \lesssim \| g \|_{H^{r-1}}, \quad
  r=1, \  r =s, \ \| f \|_{H^{s}} =1.
\end{split}
\end{equation*}
%
%
That is, for fixed $f \in H^{s}$ with $\| f \|_{H^{s}} =1$, the map $g \mapsto
Tg = fg$ is continuous from $L^{2}$ to $L^{2}$ and from $H^{s-1}$ to
$H^{s-1}$. Therefore, by Proposition~\ref{prop:sob-interp}, it is continuous from
$H^{\theta (s-1) }$ to $H^{\theta (s-1)}$ for all $\theta \in
[0, 1)$. Setting $\theta = (r-1)/(s-1)$, $ 1 \le r \le s$, we obtain that $T$ is
continuous from $H^{r-1}$ to $H^{r-1}$; that is
%
%
\begin{equation*}
\begin{split}
  \| f g \|_{H^{r-1}} \lesssim \| g \|_{H^{r-1}}, \quad 1 \le r \le s, \
  \| f \|_{H^{s}} =1
\end{split}
\end{equation*}
which for general $f \in H^{s}$ implies 
%
\begin{equation}
  \label{hhh}
\begin{split}
  \| f g \|_{H^{r-1}} \lesssim \|f \|_{H^{s}}
  \| g \|_{H^{r-1}}, \quad 1 \le r \le s. 
\end{split}
\end{equation}
%
Combining \eqref{yhh} and \eqref{hhh} completes the proof in the non-periodic
case. For the periodic case, we note that the analogue of the integral term in \eqref{np-key-term} is
%
%
%
\begin{equation*}
\begin{split}
  \sum_{n \in \zz}   (1 + n^{2})^{r-1}\int_{\ci} \frac{| \wh{g}(n - \eta)
  |^{2}}{(1 + \eta^{2})^{s-1}} d \eta. 
\end{split}
\end{equation*}
%
%
The remainder of the proof is analogous to that in the non-periodic case.
\end{proof}
%
\begin{proof}[Proof of Lemma~\ref{impo}]
We have
%
%
\begin{equation*}
\begin{split}
  \| f_{x} g_{x} \|_{H^{r-1}}^{2}
  & = \int_{\rr} (1 + \xi^{2})^{r-1}| \int_{\rr} 
  \eta \wh{f}(\eta) (\xi - \eta) \wh{f}(\xi - \eta) d \eta |^{2} d \xi
  \\
  & \le \int_{\rr} (1 + \xi^{2})^{r-1}\left [ \int_{\rr} |
  \eta| | \wh{f}(\eta) | | \xi - \eta |  | \wh{g}(\xi - \eta) | 
  d \eta \right ]^{2} d \xi
  \\
  & = \int_{\rr}  (1 + \xi^{2})^{r-1}\left [ \int_{\rr} |
  \eta| | \wh{f}(\eta) | | \xi - \eta |  | \wh{g}(\xi - \eta) | (1 +
  \eta^{2})^{\frac{1-s}{2}} (1 + \eta^{2})^{\frac{s-1}{2}}
  d \eta \right ]^{2} d \xi
  \\
  & \le \int_{\rr}  (1 + \xi^{2})^{r-1}\left [ \int_{\rr}
  | \wh{f}(\eta) | |
  \xi -\eta|  | \wh{g}(\xi - \eta) | (1 +
  \eta^{2})^{\frac{1-s}{2}} (1 + \eta^{2})^{\frac{s}{2}}
  d \eta \right ]^{2} d \xi.
\end{split}
\end{equation*}
%
Applying Cauchy Schwartz in $\eta$, we bound this by
%
%
%
\begin{equation}
  \label{np-key-term-d}
\begin{split}
  \| f \|_{H^{s}}^{2} \int_{\rr}  (1 + \xi^{2})^{r-1}\int_{\rr} \frac{(\xi -
  \eta)^{2} | \wh{g}(\xi - \eta) |^{2}}{(1 + \eta^{2})^{s-1}} d \eta d \xi.
  \end{split}
\end{equation}
%
%
Applying a change of variable, we get
%
\begin{equation*}
\begin{split}
  \| f \|_{H^{s}}^{2} \int_{\rr} (1 + \xi^{2})^{r-1} \int_{\rr} \frac{\eta^{2}
  | \wh{g}(\eta) |^{2}}{[1 + (\xi - \eta)^{2}]^{s-1}} d \eta d \xi
  \end{split}
\end{equation*}
which by Fubini is equal to
%
%
\begin{equation*}
\begin{split}
  \int_{\rr} \eta^{2} | \wh{g}(\eta) |^{2} \int_{\rr} \frac{1}{\left[ 1 + (\xi -
  \eta)^{2} \right]^{s-1} (1 + \xi^{2})^{1-r}} d \xi d \eta.
\end{split}
\end{equation*}
%
Using the inequality 
\begin{equation}
  \label{simp-ineq}
  1 + a^{2} \ge (1 + a)^{2}/4
\end{equation}
and a change of
variable, we bound this by
%
%
%
\begin{equation}
  \label{pre-int-lem-calc}
\begin{split}
  C_{s,r} \int_{\rr} \eta^{2} | \wh{g}(\eta) |^{2} \int_{\rr} \frac{1}{\left[ 1
  + |\xi - \eta| \right]^{2(s-1)} (1 + |\xi|)^{2(1-r)}} d \xi d \eta.
\end{split}
\end{equation}
%
%
To be able to apply the lemma to \eqref{pre-int-lem-calc}, we must first check
that its conditions are met. Let $ s = 3/2 + \ee$, $r = 1- \delta$, $\ee > 0$, $
\delta \ge 0$ and observe that
%
%
\begin{equation*}
\begin{split}
2(s-1) + 2(1-r)
& = 2(s-r)
\\
& = 2[3/2 + \ee - (1 - \delta)]
\\
& = 2(1/2 + \ee + \delta)
\\
& = 1 + 2 \ee + 2 \delta > 1.
\end{split}
\end{equation*}
%
%
Furthermore, $2(s-1), 2(1-r) > 0$. Hence, Lemma~\ref{lem:calc} is applicable. 
Lastly, observe that 
%
%
\begin{equation*}
\begin{split}
  \min\left\{ 2(s-1), 2(1-r), 2(s-1) + 2(1-r) -1 \right\}
  & = \min\left\{ 1 + 2 \ee, 2 \delta, 2(\ee + \delta) \right\}
  \\
  & = \min\left\{ 1 + 2 \ee, 2 \delta\right\}
  \\
  & = 2 \delta, \quad \delta \le 1/2 + \ee.
\end{split}
\end{equation*}
%
%
Hence, \eqref{pre-int-lem-calc} is bounded by
\begin{equation*}
\begin{split}
  C_{s,r} \int_{\rr} \eta^{2} | \wh{g}(\eta) |^{2} \int_{\rr} \frac{1}{\left[ 1
  + 2 |\eta| \right]^{2 \delta}} d \xi d \eta 
  & \lesssim
  \int_{\rr} \eta^{2} | \wh{g}(\eta) |^{2} \int_{\rr} \frac{1}{\left[ 1
  + |\eta| \right]^{2 \delta}} d \xi d \eta  
  \\
  & \le \int_{\rr} \eta^{2} | \wh{g}(\eta) |^{2} \int_{\rr} \frac{1}{\left[ 1
  + \eta^{2} \right]^{\delta}} d \xi d \eta  
  \\
  & \le \| w \|_{H^{1- \delta}}^{2}
  \\
  & = \| w \|_{H^{r}}^{2}.
\end{split}
\end{equation*}
%
Note that this bound is valid for all $0 \le \delta \le 1/2 + \ee$, $\ee >
0$. Recasting this in terms of $r$ and $s$, we have 
$$1-r \le 1/2 + s - 3/2, \quad r \le 1, \ s > 3/2,$$ which is equivalent to 
$$s + r \ge 2,  \quad  r \le 1, \ s > 3/2.$$ Therefore, 
%
%
%
%
\begin{equation}
  \label{r-lt-1-range}
\begin{split}
  \| f_{x} g_{x} \|_{H^{r-1}} \lesssim \| f \|_{H^{s}} \| g \|_{H^{r}},
  \quad s + r \ge 2, \ s > 3/2, \ r \le 1.
\end{split}
\end{equation}
%
%
We will now establish this bound for $r=s$ where $s > 3/2$, and then interpolate
to obtain bounds for the whole range $1 \le r \le s$, where again $s > 3/2$.
Applying Lemma~\ref{lem:alg-prop}, we obtain
%
%
%
\begin{equation}
  \label{pre-interp-2}
\begin{split}
  \| f_{x} g_{x} \|_{H^{s-1}}
  & \lesssim   \|f_{x}  \|_{H^{s-1}} \| g \|_{H^{s-1}}, \quad s >3/2
  \\
  & \le \| f \|_{H^{s}} \| g \|_{H^{s}}.
\end{split}
\end{equation}
%
%
%
%
%
%
%
%
To apply Proposition~\ref{prop:sob-interp}, we note that \eqref{r-lt-1-range}
and \eqref{pre-interp-2} imply
%
%
\begin{equation*}
\begin{split}
  \| f_{x} g_{x} \|_{H^{r-1}} \lesssim \| g \|_{H^{r}}, \quad
  r=1, \  r =s, \ \| f \|_{H^{s}} =1.
\end{split}
\end{equation*}
%
%
That is, for fixed $f \in H^{s}$ with $\| f \|_{H^{s}} =1$, the map $g \mapsto
Tg = f_{x} g_{x}$ is continuous from $H^{1}$ to $L^{2}$ and from $H^{s}$ to
$H^{s-1}$. Therefore, by Proposition~\ref{prop:sob-interp}, it is continuous from
$H^{\theta s + 1 - \theta}$ to $H^{\theta(s-1)}$ for all $\theta \in
(0, 1)$. Setting $\theta = (r-1)/(s-1)$, $ 1 \le r \le s$, we obtain that $T$ is continuous from $H^{r}$ to $H^{r-1}$; that is
%
%
\begin{equation*}
\begin{split}
  \| f_{x} g_{x} \|_{H^{r-1}} \lesssim \| g \|_{H^{r}}, \quad 1 \le r \le s, \
  \| f \|_{H^{s}} =1
\end{split}
\end{equation*}
which for general $f \in H^{s}$ implies 
%
\begin{equation}
  \label{r-g-1-range}
\begin{split}
  \| f_{x} g_{x} \|_{H^{r-1}} \lesssim \|f \|_{H^{s}}
  \| g \|_{H^{r}}, \quad 1 \le r \le s. 
\end{split}
\end{equation}
%
Combining \eqref{r-lt-1-range} and \eqref{r-g-1-range} completes the proof in
the non-periodic case. For the periodic case, we note that the analogue of the
integral term in \eqref{np-key-term-d} is
%
%
%
\begin{equation*}
\begin{split}
  \sum_{n \in \zz}   (1 + n^{2})^{r-1}\int_{\ci} \frac{(n - \eta)^{2}| \wh{g}(n
  - \eta) |^{2}}{(1 + \eta^{2})^{s-1}} d \eta. 
\end{split}
\end{equation*}
%
%
The remainder of the proof is analogous to that in the non-periodic case.
\end{proof}
%Following Klainerman \cite{Klainerman:fk}, we introduce the following.  
%\begin{definition}
  %Let $T_{z}$ be a family of linear operators indexed by $z \in D$. We say the
  %$T_{z}$ are an analytic family of operators if
  %\begin{enumerate}
    %\item{$T_{z}$ maps simple functions into measurable functions}
    %\item{ The map $z \mapsto T_{z}$ is analytic in the interior of the strip
      %%
      %%
      %\begin{equation*}
      %\begin{split}
        %0 \le \text{Re}\, z \le 1
      %\end{split}
      %\end{equation*}
      %%
      %%
      %and bounded and continuous on the boundary.}
  %\end{enumerate}
%\end{definition}
%%
%%
%%%%%%%%%%%%%%%%%%%%%%%%%%%%%%%%%%%%%%%%%%%%%%%%%%%%%%
%%
%%
%%                stein-interp
%%
%%
%%%%%%%%%%%%%%%%%%%%%%%%%%%%%%%%%%%%%%%%%%%%%%%%%%%%%%
%%
%%
%\begin{lemma}[Stein Complex Interpolation]
  %Let $T_{z}$ be an analytic family of operators and assume there are positive
  %constants $M_{0}, M_{1}$ such that, for every $b \in \rr$
  %%
  %%
  %\begin{equation*}
  %\begin{split}
    %\| T_{ib} \|_{L^{q_{0}}} \le M_{0} \| f \|_{L^{p_{0}}}, \quad \|
    %T_{1 + ib} f \|_{L^{q_{1}}} \le M \| f \|_{L^{p_{1}}}
  %\end{split}
  %\end{equation*}
  %%
  %%
  %with $1 \le q_{0}, p_{0}, q_{1}, p_{1} \le \infty$. Then, for $z = a + ib \in
  %D$, $T_{z}$ extends to a bounded operator from $L^{p}$ to $L^{q}$ and
  %%
  %%
  %\begin{equation*}
  %\begin{split}
    %\| T_{z} f \|_{L^{q}} \le M_{0}^{1-a} M_{1}^{a} \| f \|_{L^{p}}
  %\end{split}
  %\end{equation*}
  %%
  %%
  %where
  %%
  %%
  %\begin{equation*}
  %\begin{split}
    %\frac{1}{p} = \frac{1-a}{p_{0}} + \frac{a}{p_{1}}, \quad \frac{1}{q} =
    %\frac{1-a}{q_{0}} + \frac{a}{q_{1}}.
  %\end{split}
  %\end{equation*}
  %%
  %%
%\label{lem:stein}
%\end{lemma}
%%
%%
%%
%To apply the lemma, we need some preliminaries. First, recall that for $r \in
%\rr$
%%
%%
%\begin{equation*}
%\begin{split}
  %(1 - \p_x^{2})^{r}h(x) 
  %\overset{\text{it.}}{=} & \int_{\rr} e^{ix
  %\xi}(1 + \xi^{2})^{r} \wh{h}(\xi) d \xi
  %\\
   %\overset{\text{abs. conv.}}{=}  & \lim_{j \to \infty} \frac{1}{2 \pi} \int_{\rr}
  %\int_{\rr} e^{i(x-y)}
  %\chi( \xi/j) (1 + \xi^{2})^{r}dy d \xi
%\end{split}
%\end{equation*}
%%
%%
%where $\chi(\xi)$ is a cutoff function symmetric about the origin. 
%%we fix $g \in H^{s}$ with $\| g\|_{H^{s}} = 1$
%%
%%
%%
%%
%
%
\begin{proof}[Proof of Lemma~\ref{lem:interp}]
 We have
 %
 %
 \begin{equation}
   \label{pre-hold}
 \begin{split}
   \| f \|_{H^{s}}^{2}
   & = \int_{\rr} (1 + \xi^{2})^{s} | \wh{f}(\xi) |^{2} d \xi
   \\
   & = \int_{\rr} | \wh{f}(\xi) | (1 + \xi^{2})^{s_{1}/p} | \wh{f}(\xi) |
   (1 + \xi^{2})^{s_{2}/q} 
 \end{split}
 \end{equation}
 %
 %
 where
 %
 %
 \begin{equation*}
 \begin{split}
   s_{1}/p + s_{2}/q =s, \quad 1/p + 1/q =1.
 \end{split}
 \end{equation*}
 %
 %
This implies 
 %
 %
 \begin{equation*}
 \begin{split}
   1/p = (s_{2} -s)/(s_{2} -s_{1}), \quad 1/q = (s -s_{1})/(s_{2} -s_{1}). 
 \end{split}
 \end{equation*}
 %
 %
 Applying H\"older to the right hand side of \eqref{pre-hold}, we obtain the
 bound
 %
 %
 \begin{equation*}
 \begin{split}
   \| f \|_{H^{s}}^{s}
   & \le  \left[ \int_{\rr} (1 + \xi^{2})^{s_{1}} | \wh{f}(\xi) |^{2} d \xi
   \right]^{(s_{2} - s)/(s_{2} -s_{1})} \left[ \int_{\rr} (1 + \xi^{2})^{s_{1}}
   | \wh{f}(\xi) |^{2} d \xi \right]^{(s - s_{1})/(s_{2} -s_{1})}
   \\
   & = \| f \|_{H^{s_{1}}}^{2 (s_{2} - s)/(s_{2} -s_{1})}
   \| f \|_{H^{s_{2}}}^{2 (s - s_{1})/(s_{2} -s_{1})}.
 \end{split}
 \end{equation*}
 %
 %
 Taking square roots of both sides completes the proof in the non-periodic case.
 The proof in the periodic case is analogous to the non-periodic proof (i.e.\
 integrals are replaced with sums in the proof).
  \end{proof}
\begin{proof}[Proof of Lemma~\ref{lem:calc}]
%
By the change of variable $x \mapsto x + (\alpha + \beta)/2$, we have
%
%
\begin{equation}
  \label{rry}
	\begin{split}
    \int_{\rr} \frac{1}{\langle x - \alpha \rangle^{p} \langle  x -
    \beta
    \rangle^{q}}d x
    & = \int_{\rr} \frac{1}{\langle x - (\alpha - \beta)/2  \rangle^{p}
    \langle  x + (\alpha - \beta)/2 \rangle^{q}} d x
    \\
    & \simeq \int_{\rr} \frac{1}{\langle x - (\alpha - \beta)  \rangle^{p}
    \langle  x + (\alpha - \beta) \rangle^{q}} d x.
  \\
  & = \int_{\rr} \frac{1}{\langle a - x \rangle ^{p} \langle a + x \rangle
  ^{q}} d x, \quad a = \alpha - \beta
\end{split}
\end{equation}
%
which for $a =0$ reduces to 
%
%
\begin{equation*}
\begin{split}
  \int_{\rr} \frac{1}{\langle x \rangle ^{p+q}} d x 
  & = 2 \int_{0}^{\infty} \frac{1}{(1 + x)^{p+q}} d x
  \\
  & = \frac{2}{p+q -1}
  \\
  & = \frac{2}{(p+q -1)\langle a \rangle}.
\end{split}
\end{equation*}
%
%
By symmetry, we now assume without loss of generality that $a > 0$ and split 
%
%
\begin{equation*}
\begin{split}
\int_{\rr} \frac{1}{\langle a + x \rangle ^{p} \langle a - x \rangle
  ^{q}} d x
  & = \int_{-2a}^{2a}
  \frac{1}{\langle a + x \rangle ^{p} \langle a - x \rangle
  ^{q}} d x
  \\
  & + \int_{| x | \ge 2a} 
\frac{1}{\langle a + x \rangle ^{p} \langle a - x \rangle
  ^{q}} d x
  \\
  & = I + II.
\end{split}
\end{equation*}
%
%
If $p=1$ and $q=1$, then 
%
%
\begin{equation*}
\begin{split}
  I
  & \le \sup_{-2a \le x \le 2a} \frac{1}{\langle a - x \rangle
} \int_{-2a}^{2a} \frac{1}{\langle a + x \rangle} d x
  \\
  & = \frac{1}{\langle a \rangle} \int_{-2a}^{2a} \frac{1}{(1 + | a -
  x
  |)} d x
  \\
  & = \frac{4}{\langle a \rangle} \int_{0}^{a} \frac{1}{(1 + a -
  x)} d x.
\end{split}
\end{equation*}
%
%
Integrating, we obtain
%
%
\begin{equation*}
 I
 \le 
 \frac{4 \log \langle a \rangle}{\langle a \rangle}, \qquad p =1, \ q =1.
\end{equation*}
Otherwise, assume that $q \neq 1$. Then
\begin{equation*}
\begin{split}
  I
  & \le \sup_{-2a \le x \le 2a} \frac{1}{\langle a + x \rangle
  ^{p}} \int_{-2a}^{2a} \frac{1}{\langle a - x \rangle ^{q}} d x
  \\
  & = \frac{1}{\langle a \rangle ^{p}} \int_{-2a}^{2a} \frac{1}{(1 + | a -
  x
  |)^{q}} d x
  \\
  & = \frac{4}{\langle a \rangle ^{p}} \int_{0}^{a} \frac{1}{(1 + a -
  x)^{q}} d x.
\end{split}
\end{equation*}
Evaluating the integral, we obtain
\begin{equation*}
  I \le \frac{4}{|q-1| \langle a \rangle ^{p +q -1}}, \qquad q \neq 1.
\end{equation*}
%
%
A similar computation yields
\begin{equation*}
  I \le \frac{4}{|q-1| \langle a \rangle ^{p +q -1}}, \qquad p \neq 1.
\end{equation*}
%
%
Also
%
%
\begin{equation*}
\begin{split}
  II 
  & \simeq \int_{x \ge 2a} \frac{1}{\langle a - x \rangle ^{p} \langle a
  + x \rangle ^{q} d x}
  \\ 
  & = \int_{x \ge 2a} \frac{1}{(1 + x - a)^{p} (1 + x +
  a)^{q}} d x
  \\
  & \le \int_{x \ge 2a} \frac{1}{(1 + x -a)^{p+q}} d x
  \\
  & = \frac{1}{[p + q-1] \langle a \rangle ^{p+q -1}}, \qquad p + q > 1.
\end{split}
\end{equation*}
%
%
Collecting our estimates for $I$ and $II$ we see that for 
$p, q > 0$ such that $p +q >1$, and $r =\min\left\{p, q, p+q-1
 \right\}$, we have 
%
\begin{align*}
  \int_{\rr} \frac{1}{\langle a - x \rangle ^{p} \langle a + x \rangle
  ^{q}} d x
  \le \frac{c_{p,q}}{\langle a \rangle ^{r}}, \qquad & a = 0 \ \text{or} \
  p \neq 1 \ \text{or} \ q \neq 1
  \\
   \int_{\rr} \frac{1}{\langle a - x \rangle  \langle a + x \rangle
} d x
  \le  \frac{4 \log \langle a \rangle}{\langle a \rangle}, \qquad & a \neq 0.
  \label{est-2}
\end{align*}
By symmetry, the second inequality also holds for $a < 0$. Recalling
\eqref{rry}, the proof is complete.
\end{proof}

  \section{Proof of Proposition~\ref{prop:sob-interp}}
\label{ssec:pf-sob-interp}
The proof follows immediately from the following two lemmas and the continuous
embedding $H^{s'} \subset H^{s}$, $s < s'$.
%
%
%%%%%%%%%%%%%%%%%%%%%%%%%%%%%%%%%%%%%%%%%%%%%%%%%%%%%
%
%
%                
%
%
%%%%%%%%%%%%%%%%%%%%%%%%%%%%%%%%%%%%%%%%%%%%%%%%%%%%%
%
%
\begin{lemma}
\label{lem:rest-bound}
Let $X, Y, U, V$ be Banach spaces with $U \subset X, V \subset Y$ continuously,
and suppose $T$ is a bounded linear operator from $X$ to $Y$. If $T$ maps $U$
to $V$, then $T$ is bounded from $U$ to $V$.
\end{lemma}
%
%
%
%%%%%%%%%%%%%%%%%%%%%%%%%%%%%%%%%%%%%%%%%%%%%%%%%%%%%
%
%
%               Interpolation spaces 
%
%
%%%%%%%%%%%%%%%%%%%%%%%%%%%%%%%%%%%%%%%%%%%%%%%%%%%%%
%
%
\begin{lemma}
\label{lem:interp-spaces}
For fixed $k \le q, \ m \le s$ suppose that $T: H^{k} \to H^{m}$
continuously and $T: H^{q} \to H^{s}$. Then
$T: H^{\theta q + (1 - \theta)k} \to H^{\theta s +
(1 - \theta) m}$ for all $\theta \in [0,1]$.
%
%
%
\end{lemma}
%
%
%
Hence, to complete the proof of Proposition~\ref{prop:sob-interp},
it will be enough to prove these lemmas.
%
\begin{proof}[Proof of Lemma~\ref{lem:rest-bound}]
We first recall the following.
\begin{definition}
Let $X$ and $Y$ be Banach spaces, and $T: X \to Y$ be linear. We say that $T$ is
\emph{closed} if 
%
%
%\begin{equation*}
%\begin{split}
  %\Gamma(T) = \left\{ (x, Tx) \in X \times Y \right\}
%\end{split}
%\end{equation*}
%%
%%
%is a closed subspace of $X \times Y$ in the product topology. That is,
%%
%%
%\begin{equation*}
%\begin{split}
  %\| x -x_{n} \|_{X} + \| Tx - Tx_{n} \|_{Y} \to 0.
%\end{split}
%\end{equation*}
%%
%%
%
$x_{n} \to x$ and $Tx_{n} \to y$ imply $y = Tx$. 
\end{definition}

%
%
%%%%%%%%%%%%%%%%%%%%%%%%%%%%%%%%%%%%%%%%%%%%%%%%%%%%%
%
%
%                closed graph theorem
%
%
%%%%%%%%%%%%%%%%%%%%%%%%%%%%%%%%%%%%%%%%%%%%%%%%%%%%%
%
%
\begin{lemma}[Closed Graph Theorem]
\label{lem:closed-graph}
If $X$ and $Y$ are Banach, and $T: X \to Y$ is a closed linear map, then $T$ is
bounded.
\end{lemma}
%
%
Proceeding with the proof of Lemma~\ref{lem:rest-bound},
suppose $x_{n} \to x$ in $U$; that is
%
%
\begin{equation}
  \label{1hh}
\begin{split}
  \| x - x_{n} \|_{U} \to 0.
\end{split}
\end{equation}
%
%
Since $U \subset X$
continuously and
$T: X \to Y$ is bounded, we have
%
%
\begin{equation}
  \label{2hh}
\begin{split}
  \| Tx - Tx_{n} \|_{Y} = \| T(x - x_{n}) \|_{Y} \lesssim \| x -
  x_{n} \|_{X} \le  \| x - x_{n} \|_{U} \to 0.
\end{split}
\end{equation}
%
%
Hence, from \eqref{2hh}, the continuous embedding $V \subset Y$, 
and the uniqueness of the limit,
it follows that if $Tx_{n} \to v$ in $V$, then $v = Tx$. 
Therefore, $T$ is closed from
$U$ to $V$. Applying the closed graph theorem concludes the proof. 
\end{proof}
%
%
%
\begin{proof}[Proof of Lemma~\ref{lem:interp-spaces}]
  We refer the reader to \cite{Taylor:1995kx}, pp.322-324.
\end{proof}
%
%
For additional reading on interpolation spaces, please consult
\cite{Taylor:1995kx}, \cite{bergh1976interpolation}, and \cite{Bennett:1988ys}.
%
%
\providecommand{\bysame}{\leavevmode\hbox to3em{\hrulefill}\thinspace}
\providecommand{\MR}{\relax\ifhmode\unskip\space\fi MR }
% \MRhref is called by the amsart/book/proc definition of \MR.
\providecommand{\MRhref}[2]{%
  \href{http://www.ams.org/mathscinet-getitem?mr=#1}{#2}
}
\providecommand{\href}[2]{#2}
\begin{thebibliography}{HKM09}

\bibitem[BBM72]{Benjamin_1972_Model-equations}
T.~B. Benjamin, J.~L. Bona, and J.~J. Mahony, \emph{Model equations for long
  waves in nonlinear dispersive systems}, Philos. Trans. Roy. Soc. London Ser.
  A \textbf{272} (1972), no.~1220, 47--78.

\bibitem[BC07]{Bressan_2007_Global-conserva}
A.~Bressan and A.~Constantin, \emph{Global conservative solutions of the
  camassa-holm equation}, Arch. Ration. Mech. Anal. \textbf{183} (2007), no.~2,
  215--239.

\bibitem[BCK06]{Bendahmane_2006_Hsp-1-perturbat}
M.~Bendahmane, G.~M. Coclite, and K.~H. Karlsen, \emph{$h^1$-perturbations of
  smooth solutions for a weakly dissipative hyperelastic-rod wave equation},
  Mediterr. J. Math. \textbf{3} (2006), no.~3-4, 419--432.

\bibitem[Ben72]{Benjamin_1972_The-stability-o}
T.~B. Benjamin, \emph{The stability of solitary waves}, Proc. Roy. Soc.
  (London) Ser. A \textbf{328} (1972), 153--183.

\bibitem[BL76]{bergh1976interpolation}
J.~Bergh and J.~L{\"o}fstr{\"o}m, \emph{Interpolation spaces: an introduction},
  vol. 223, Springer-Verlag, 1976.

\bibitem[BS88]{Bennett:1988ys}
C.~Bennett and R.~Sharpley, \emph{Interpolation of operators}, vol. 129,
  Academic Pr, 1988.

\bibitem[Che11]{Chen:2011fk}
R.~M. Chen, \emph{The h{\"o}lder continuity of the solution map to the
  $b$-family equation in weak topology}, 2011.

\bibitem[CHK05]{Coclite_2005_Global-weak-sol}
G.~M. Coclite, H.~Holden, and K.~H. Karlsen, \emph{Global weak solutions to a
  generalized hyperelastic-rod wave equation}, SIAM J. Math. Anal. \textbf{37}
  (2005), no.~4, 1044--1069 (electronic).

\bibitem[CS00]{Constantin_2000_Stability-of-a-}
A.~Constantin and W.~A. Strauss, \emph{Stability of a class of solitary waves
  in compressible elastic rods}, Phys. Lett. A \textbf{270} (2000), no.~3-4,
  140--148.

\bibitem[Dai98]{Dai_1998_Model-equations}
H.-H. Dai, \emph{Model equations for nonlinear dispersive waves in a
  compressible mooney-rivlin rod}, Acta Mech. \textbf{127} (1998), no.~1-4,
  193--207.

\bibitem[DDH00]{Dai_2000_Head-on-collisi}
H.-H. Dai, S.~Dai, and Y.~Huo, \emph{Head-on collision between two solitary
  waves in a compressible mooney-rivlin elastic rod}, Wave Motion \textbf{32}
  (2000), no.~2, 93--111.

\bibitem[DH00]{Dai_2000_Solitary-shock-}
H.-H. Dai and Y.~Huo, \emph{Solitary shock waves and other travelling waves in
  a general compressible hyperelastic rod}, R. Soc. Lond. Proc. Ser. A Math.
  Phys. Eng. Sci. \textbf{456} (2000), no.~1994, 331--363.

\bibitem[HKM09]{Himonas_2009_Non-uniform-dep-per}
A.~Himonas, C.~E. Kenig, and G.~Misiolek, \emph{Non-uniform dependence for the
  periodic ch equation.}, To appear in Communications in Partial Differential
  Equations (2009).

\bibitem[HR07]{Holden_2007_Global-conserva}
H.~Holden and X.~Raynaud, \emph{Global conservative solutions of the
  generalized hyperelastic-rod wave equation}, J. Differential Equations
  \textbf{233} (2007), no.~2, 448--484.

\bibitem[Kar10]{Karapetyan:2010fk}
D.~Karapetyan, \emph{Non-uniform dependence and well-posedness for the
  hyperelastic rod equation}, J. Differential Equations \textbf{249} (2010),
  no.~4, 796--826. \MR{2652154 (2011g:35352)}

\bibitem[Kat75]{Kato_1975_Quasi-linear-eq}
T.~Kato, \emph{Quasi-linear equations of evolution, with applications to
  partial differential equations}, 1975, pp.~25--70. Lecture Notes in Math.,
  Vol. 448.

\bibitem[Len06]{Lenells_2006_Traveling-waves}
J.~Lenells, \emph{Traveling waves in compressible elastic rods}, Discrete
  Contin. Dyn. Syst. Ser. B \textbf{6} (2006), no.~1, 151--167 (electronic).

\bibitem[Mus07]{Mustafa_2007_Global-conserva}
O.~G. Mustafa, \emph{Global conservative solutions of the hyperelastic rod
  equation}, Int. Math. Res. Not. IMRN (2007), no.~13, Art. ID rnm040, 26.

\bibitem[RB01]{Rodriguez-Blanco_2001_On-the-Cauchy-p}
G.~Rodr\'iguez-Blanco, \emph{On the cauchy problem for the camassa-holm
  equation}, Nonlinear Anal. \textbf{46} (2001), no.~3, Ser. A: Theory Methods,
  309--327.

\bibitem[Tay91]{Taylor_1991_Pseudodifferent}
M.~E. Taylor, \emph{Pseudodifferential operators and nonlinear pde}, vol. 100,
  1991.

\bibitem[Tay95]{Taylor:1995kx}
M.~E. Taylor, \emph{Partial differential equations i, basic theory}, Applied
  Mathematical Sciences, vol. 115, Springer-Verlag, New York, 1995.

\bibitem[Yin03]{Yin_2003_On-the-Cauchy-p}
Z.~Yin, \emph{On the cauchy problem for a nonlinearly dispersive wave
  equation}, J. Nonlinear Math. Phys. \textbf{10} (2003), no.~1, 10--15.

\bibitem[Zho05]{Zhou_2005_Local-well-pose}
Y.~Zhou, \emph{Local well-posedness and blow-up criteria of solutions for a rod
  equation}, Math. Nachr. \textbf{278} (2005), no.~14, 1726--1739.

\end{thebibliography}
%
%
%
%
%\bibliographystyle{amsalpha-custom}
%\bibliography{/Users/davidkarapetyan/math/bib-files/references}

\end{document}
