\documentclass[12pt,reqno]{amsart}
\usepackage{amscd}
\usepackage{amsfonts}
\usepackage{amsmath}
\usepackage{amssymb}
\usepackage{amsthm}
\usepackage{appendix}
\usepackage{fancyhdr}
\usepackage{latexsym}
\usepackage{cancel}
\usepackage{amsxtra}
\synctex=1
\usepackage[colorlinks=true, pdfstartview=fitv, linkcolor=blue,
citecolor=blue, urlcolor=blue]{hyperref}
\input epsf
\input texdraw
\input txdtools.tex
\input xy
\xyoption{all}
%%%%%%%%%%%%%%%%%%%%%%
\usepackage{color}
\definecolor{red}{rgb}{1.00, 0.00, 0.00}
\definecolor{darkgreen}{rgb}{0.00, 1.00, 0.00}
\definecolor{blue}{rgb}{0.00, 0.00, 1.00}
\definecolor{cyan}{rgb}{0.00, 1.00, 1.00}
\definecolor{magenta}{rgb}{1.00, 0.00, 1.00}
\definecolor{deepskyblue}{rgb}{0.00, 0.75, 1.00}
\definecolor{darkgreen}{rgb}{0.00, 0.39, 0.00}
\definecolor{springgreen}{rgb}{0.00, 1.00, 0.50}
\definecolor{darkorange}{rgb}{1.00, 0.55, 0.00}
\definecolor{orangered}{rgb}{1.00, 0.27, 0.00}
\definecolor{deeppink}{rgb}{1.00, 0.08, 0.57}
\definecolor{darkviolet}{rgb}{0.58, 0.00, 0.82}
\definecolor{saddlebrown}{rgb}{0.54, 0.27, 0.07}
\definecolor{black}{rgb}{0.00, 0.00, 0.00}
\definecolor{dark-magenta}{rgb}{.5,0,.5}
\definecolor{myblack}{rgb}{0,0,0}
\definecolor{darkgray}{gray}{0.5}
\definecolor{lightgray}{gray}{0.75}
%%%%%%%%%%%%%%%%%%%%%%
%%%%%%%%%%%%%%%%%%%%%%%%%%%%
%  for importing pictures  %
%%%%%%%%%%%%%%%%%%%%%%%%%%%%
\usepackage[pdftex]{graphicx}
\usepackage{epstopdf}
% \usepackage{graphicx}
%% page setup %%
\setlength{\textheight}{20.8truecm}
\setlength{\textwidth}{14.8truecm}
\marginparwidth  0truecm
\oddsidemargin   01truecm
\evensidemargin  01truecm
\marginparsep    0truecm
\renewcommand{\baselinestretch}{1.1}
%% new commands %%
\newcommand{\tf}{\tilde{f}}
\newcommand{\ti}{\tilde}
\newcommand{\bigno}{\bigskip\noindent}
\newcommand{\ds}{\displaystyle}
\newcommand{\medno}{\medskip\noindent}
\newcommand{\smallno}{\smallskip\noindent}
\newcommand{\nin}{\noindent}
\newcommand{\ts}{\textstyle}
\newcommand{\rr}{\mathbb{R}}
\newcommand{\p}{\partial}
\newcommand{\zz}{\mathbb{Z}}
\newcommand{\cc}{\mathbb{C}}
\newcommand{\ci}{\mathbb{T}}
\newcommand{\ee}{\varepsilon}
\newcommand{\vp}{\varphi}
\def\refer #1\par{\noindent\hangindent=\parindent\hangafter=1 #1\par}
%% equation numbers %%
\renewcommand{\theequation}{\thesection.\arabic{equation}}
%% new environments %%
%\swapnumbers
\theoremstyle{plain}  % default
\newtheorem{theorem}{Theorem}
\newtheorem{proposition}{Proposition}
\newtheorem{lemma}{Lemma}
\newtheorem{corollary}{Corollary}
\newtheorem{claim}{Claim}
\newtheorem{remark}{Remark}
\newtheorem{conjecture}[subsection]{conjecture}
\theoremstyle{definition}
\newtheorem{definition}{Definition}
%
\begin{document}
\date{10/02/09}
\title{Non-Uniform Dependence for the
Hyperelastic Rod Equation}
\author{{\it David Karapetyan}}
\begin{abstract}
The solution map for the Hyperelastic Rod equation is not uniformly continuous
	from any bounded set of $H^s$ into $C([-T, T]; H^s)$
	for $s>3/2$ in the periodic case and for $s>1$ in the non-periodic case.
	The proof is based on the method of approximate solutions.
\end{abstract}
\maketitle
\markboth{Non-Uniform Dependence for the Hyperelastic Rod Equation}{David Karapetyan}
\parindent0in
\parskip0.1in
%\end{titlepage}
%%%%%%%%%%%%%%%%%%%%%%%%
%
%      introduction
%
%%%%%%%%%%%%%%%%%%%%%%%%
\section{introduction}
\setcounter{equation}{0}
%
We consider the periodic initial value problem for
the hyperelastic-rod (HR)  equation
\begin{equation}
	\label{hr}
	\p_t u
	-
	\p_t \p_x^2 u
	+
	3u\p_x u
	=
	\gamma \big (
	2\p_x u \p_x^2 u
	+
	u \p_x^3 u
	\big ),
\end{equation}
\begin{equation}
	\label{hr-data} u(x, 0) = u_0 (x),
	\quad x  \in \ci, \text{  or  } \rr \quad t \in \rr,
\end{equation}
where $\gamma$ is a constant. Equation \ref{hr} was first
derived by Dai in \cite{d} as a model  for finite-length and
small-amplitude axial deformation waves in thin cylindrical
rods composed of a compressible isotropic hyperelastic
material. Local well-posedness and blow-up criteria for
solutions were established by  Zhou in \cite{z}. Orbital
stability of a class of solitary waves for this equation was
proved in \cite{cs} by Constantin and Strauss.
\\
\\
Motivated by the work of Himonas and Kenig \cite{hk} and
Himonas and  Misiolek \cite{hm} we prove that 
the solution map for the HR equation is not uniformly
continuous in both the periodic and non-periodic case.
%



\begin{theorem}
	\label{hr-non-unif-dependence}
	For $s>3/2$ in the periodic case and for $s>1$ in
	the non-periodic case, the flow map $u_0 \to u(t)$ of the
	Cauchy-problem \eqref{hr}-\eqref{hr-data} is not uniformly continuous
	from any bounded set of $H^s$ into $C([-T, T]; H^s)$.
	More precisely, in both cases there exist two sequences of solutions $u_n(t)$
	and $v_n(t)$ in $C([-T, T]; H^s)$ such that
	%
	%
	\begin{equation}
		\label{h-s-bdd}
		\| u_n(t)  \|_{H^s}
		+
		\| v_n(t)  \|_{H^s}
		\lesssim
		1,
	\end{equation}
	%
	\begin{equation}
		\label{zero-limit-at-0}
		\lim_{n\to\infty}
		\|
		u_n(0)
		-
		v_n(0)
		\|_{H^s}
		=
		0,
	\end{equation}
	%
	%
	and
	%
	%
	\begin{equation}
		\label{bdd-away-from-0}
		\liminf_{n\to\infty}
		\|
		u_n(t)
		-
		v_n(t)
	\|_{H^s}
		\gtrsim
		\sin ( \gamma t),
		\quad
		|t|\le T.
	\end{equation}
	%
	%
	\end{theorem}
	For $\gamma \neq 3$ this result was proved by Olson in \cite{o} using
	traveling wave solutions. In this paper
	we will eliminate the restriction on $\gamma$ using approximate solutions to the HR
	equation. In Section 1 we introduce the approximate
	solutions we will be using, and derive a functional representation of
	the error in Section 2. In Section 3 we find an upper bound for the
	error. This leads to a crucial lemma in Section 4 that gives a
	decaying bound for the difference of approximate and actual solutions.
	In Section 5 the lemma is used in an interpolation which allows us to 
	conclude the non-uniform dependence on initial
	data of solutions to the HR equation, independent of the choice of 
	$\gamma$.
\vskip0.1in
\section{
The non-periodic case
}
\setcounter{equation}{0}
\vskip0.1in
We consider the Cauchy problem for the Hyperelastic Rod Equation (HR)
\begin{equation}
	\begin{split}
		\p_t u + \gamma u \p_x u + \p_x \left( 1 - \p_x^2
		\right)^{-1}  \left[ \frac{3-\gamma}{2}u^2 +
		\frac{\gamma}{2} \left( \p_x u \right)^2
		\right] = 0,
		\label{apple1'}
	\end{split}
\end{equation}
%
\begin{equation}
	\begin{split}
		u(x,0) = u_0(x), \; \; x \in \rr, \; \; t \in \rr. 
		\label{apple2'}
	\end{split}
\end{equation}
Our approximate solutions $u^{\omega, \lambda} = u^{\omega,
\lambda}(x,t)$ to \eqref{apple1'}-\eqref{apple2'} will
consist of a low frequency and a high frequency part,
i.e.
\begin{equation}
	\label{apple1}
	u^{\omega,\lambda} = u_\ell + u^h.
\end{equation}
The high frequency part is given by 
\begin{equation}
	\begin{split}
		u^h = u^{h,\omega,\lambda}(x,t) =
		\lambda^{-\frac{\delta}{2} -s}
		\phi \left (\frac{x}{\lambda^\delta}\right )
		\cos(\lambda x - \gamma \omega t)
	\end{split}
\end{equation}
where $\phi$ is a $C^\infty$ cutoff function such that
\begin{equation*}
	\phi = 
	\begin{cases}
		1, &\text{if $|x|<1$;} \\
		0, &\text{if $|x| \ge 2$.} 
	\end{cases}
\end{equation*}
By Theorem \ref{thm:HR_existence_continous_dependence} ,
we let the low frequency part $u_\ell = u_{l,
\omega, \lambda}(x,t)$ be the unique solution to the HR equation
\begin{equation}
	\p_t u_\ell + \gamma u_\ell \p_x u_\ell + \p_x (1- \p_x^2)^{-1}  \left[
	\frac{3- \gamma}{2}(u_\ell)^2 + \frac{\gamma}{2}\left( \p_x u_\ell
	\right)^2 \right] = 0
	\label{apple1*}
\end{equation}
with initial data
\begin{equation}
	u_\ell(x,0) = \omega \lambda^{-1} \tilde{\phi} \left(
	\frac{x}{\lambda^{\delta}}
	\right), \quad x \in \rr, \quad t \in \rr
	\label{apple1**}
\end{equation}
where $\tilde{\phi}$ is a $C^{\infty}_0(\rr)$ function such that
\begin{equation}
	\label{apple1***}
	\tilde{\phi}(x) = 1 \; \;  \text{if} \; \;
	x \in \text{supp} \; \phi.
\end{equation}
Let $\Lambda^{-1} = \p_x (1 - \p_x^2)^{-1} $. Substituting the
approximate solution $u^{\omega, \lambda} = u_\ell + u^h$ into the HR
equation, and recalling that $u_\ell$ is a solution to the
HR equation, we obtain
\begin{equation}
	\begin{split}
		E 
		& = \p_t u^h + \gamma u_\ell \p_x u^h + \gamma u^h \p_x u_\ell +
		\gamma u^h \p_x u^h
		\\
		& + \Lambda^{-1} \left\{ \frac{3-\gamma}{2}\left[ \left( u^h
		\right)^2 + 2u_\ell u^h
		\right]+ \frac{\gamma}{2}\left[ \left( \p_x u^h \right)^2 + 2
		\p_x u_\ell \p_x u^h\right] \right\}.
		\label{apple2star}
	\end{split}
\end{equation}
%\begin{equation}
%    \begin{split}
%  	  E 
%  	  &= \p_t u^{\omega,\lambda} + \gamma u^{\omega,\lambda} \p_x
%  	  u^{\omega,\lambda} + \Lambda^{-1} \left[
%  	  \frac{3- \gamma}{2}\left( u^{\omega,\lambda} \right)^2 +
%  	  \frac{\gamma}{2}\left( \p_x u^{\omega, \lambda}
%  	  \right)^2 \right]
%  	  \\
%  	  & = \p_t u_\ell + \p_t u^h + \gamma u_\ell \p_x u_\ell + \gamma u_\ell
%  	  \p_x u^h + \gamma u^h \p_x u_\ell + \gamma u^h \p_x u^h
%  	  \\
%  	  & + \Lambda^{-1}  \left\{ \frac{3-\gamma}{2}\left[ u_\ell^2 +
%  	  \left( u^h \right)^2 + 2u_\ell u^h
%  	  \right] + \frac{\gamma}{2}\left[ \left( \p_x u_\ell \right)^2 +
%  	  \left( \p_x u^h \right)^2 + 2\p_x u_\ell \p_x u^h\right]
%  	  \right\}.
%  	  \label{apple2}
%    \end{split}
%\end{equation}
%Recalling that $u_\ell$ is a solution to the HR equation, we can
%simplify \eqref{apple2} to obtain
Using a straightforward calculation of derivatives, and
noting that $\tilde{\phi} (x) = 1$ for $x \in \text{supp} \;
\phi$,
we deduce
\begin{equation}
	\begin{split}
		\p_t u^h + \gamma u_\ell \p_x u^h 
		& = \gamma \lambda\left[ u_\ell(x,0) - u_\ell(x,t)
		\right]\lambda^{-\frac{\delta}{2}-s} \phi\left(
		\frac{x}{\lambda^\delta}
		\right) \sin(\lambda x - \gamma \omega t)
		\\
		& + \gamma u_\ell(x,t) \cdot \lambda^{-\frac{3\delta}{2}-s}
		\phi'\left( \frac{x}{\lambda^\delta} \right)\cos\left( \lambda
		x - \gamma \omega t
		\right).
						 \label{apple5}
					 \end{split}
				 \end{equation}
				 Therefore, applying \eqref{apple5} to \eqref{apple2star}, we see that the error
				 $E$ of our approximate solution is given by
				 \begin{equation*}
					 E=E_1 + E_2 + \dots + E_8
				 \end{equation*}
				 where
				 \begin{equation}
					 \label{all_errors_together}
					 \begin{split}
						  E_1 & = \gamma \lambda \left[ u_\ell(x,0) - u_\ell(x,t)
						 \right] \lambda^{-\frac{\delta}{2}-s} \phi\left(
						 \frac{x}{\lambda^ \delta}
						 \right)\sin(\lambda x - \gamma \omega t)
						 \\
						 E_2 & = \gamma u_\ell(x,t) \cdot \lambda^{-\frac{3\delta}{2}-s}
						 \phi'\left( \frac{x}{\lambda^\delta} \right)\cos\left( \lambda
						 x - \gamma \omega t
						 \right)
						 \\
						 E_3 & = \gamma u^h \p_x u_\ell, \; \; E_4 = \gamma u^h \p_x u^h
						 \\
						 E_5 & = \Lambda^{-1}\left[ \frac{3-\gamma}{2} \left(
						 u^h \right)^2 
						 \right], \; \; E_6 = \Lambda^{-1}
						 \left[ (3- \gamma)u_\ell u^h \right]
						 \\
						 E_7 & = \Lambda^{-1} \left[ \frac{\gamma}{2} \left(
						 \p_x u^h \right)^2 \right ], \; \;
						 E_8 = \Lambda^{-1} \left[ \gamma \p_x u_\ell \p_x u^h \right]
						 .
						 \end{split}
				 \end{equation}
				 %
				 %
				 %
				 \vskip0.1in
				 {\bf An Upper Bound in $H^1$ For the Error of the Approximate
				 Solutions.}
				 \vskip0.1in
				 For estimating the $H^1(\ci)$ norm of $u^h$, we need the
				 following result, whose proof can be found in \cite{hk}:
				  \begin{lemma}
					 \label{applea}
					 Let $\psi \in S(\rr)$, $\alpha \in \rr$. Then for $s \ge 0$ we have
					 \begin{equation}
						 \begin{split}
							 \lim_{\lambda \to \infty} \lambda^{-\frac{\delta}{2}-s}
							 \|\psi \left( \frac{x}{\lambda^\delta} \right)\cos(\lambda
							 x - \alpha) \|_{H^s(\rr)} = \frac{1}{\sqrt
							 2}\|\psi\|_{L^2(\rr)}.
							 \label{apple6}
						 \end{split}
					 \end{equation}
					 Relation \eqref{apple6} remains true if $\cos$ is
					 replaced by $\sin$.
				 \end{lemma}
				 Next, we provide an upper bound for the $H^1(\ci)$ norm of
				 $u_l$; details can again be found in \cite{hk}:
				 %
				\begin{lemma}
					\label{appleb}
					Let $0<\delta<2$, with $\omega$ belonging to a bounded
					subset of $\rr$. Then the initial value problem
					\eqref{apple1*}-\eqref{apple1**} has a unique solution
					$u_\ell \in C\left( [0,T], H^s(\rr) \right)$ for all $s
					\ge 0$, which 
					satisfies
					\begin{equation}
						\label{apple10'}
						\|u_\ell(t)\|_{H^s(\rr)} \le c_s \lambda^{-1 +
						\frac{\delta}{2}}, \quad |t| \le T.
					\end{equation}
				\end{lemma}
								We will also need the following:
\begin{lemma}
	\label{applec}
	For any $f,g \in L^2(\rr)$,
	\begin{equation*}
		\|fg\|_{H^1(\rr)} \le \sqrt{2} \|f\|_{C^1(\rr)} \|g\|_{H^1(\rr)}.
	\end{equation*}
\end{lemma}
%
We are now prepared to estimate the $H^1$ norms of each $E_i$.
\vskip0.1in
{\bf Estimating the $H^1$ norm of $\hyperref[all_errors_together]{E_1}$.} We have
%\vskip-0.4in
\begin{equation*}
	\begin{split}
		\|E_1\|_{H^1(\rr)}
		& = \| \gamma \lambda \left[ u_\ell(x,0) - u_\ell(x,t) \right]
		\lambda^{-\frac{\delta}{2}-s} \phi\left( \frac{x}{\lambda^\delta}
		\right ) \sin (\lambda x - \gamma \omega t )\|_{H^1(\rr)}
		\\
		& = |\gamma| \lambda^{1 -\frac{\delta}{2} -s } \|\left[ u_\ell(x,0) - u_\ell(x,t)
		\right] \phi\left( \frac{x}{\lambda^\delta} \right )
		\sin\left( \lambda x - \gamma \omega t
		\right) \|_{H^1(\rr)}.
	\end{split}
\end{equation*}
Applying Lemma \ref{applec}, we obtain
\begin{equation}
	\begin{split}
		\|E_1\|_{H^1(\rr)} \le |\gamma| \lambda^{1 - \frac{\delta}{2} -s } \|\phi
		\left( \frac{x}{\lambda^\delta} \right) \sin (\lambda x - \gamma \omega t)
		\|_{C^1(\rr)} \|[u_\ell (x,0) - u_\ell (x,t) ] \|_{H^1(\rr)}.
		\label{apple14}
	\end{split}
\end{equation}
We now estimate the right hand side of \eqref{apple14} in pieces. For the first piece, we
have
\begin{equation*}
	\begin{split}
		& \|\phi \left( \frac{x}{\lambda^\delta} \right) \sin (\lambda x - \gamma \omega t)
		\|_{C^1(\rr)} 
		\\
		&
		\le \|\phi \left( \frac{x}{\lambda^\delta} \right) \|_{L^\infty(\rr)} + \lambda
		\|\phi\left( \frac{x}{\lambda^\delta} \right)\|_{L^\infty(\rr)} +
		\lambda^{-\delta} \|\phi'\left( \frac{x}{\lambda^\delta}
		\right)\|_{L^\infty(\rr)}
	\end{split}
\end{equation*}
which gives
\begin{equation}
	\begin{split}
		\|\phi\left( \frac{x}{\lambda^\delta} \right) \sin(\lambda x - \gamma \omega t)
		\|_{C^1(\rr)}
		\lesssim \lambda.
		\label{apple15}
	\end{split}
\end{equation}
For the next piece of \eqref{apple14}, we observe that the fundamental theorem
of calculus gives
\begin{equation*}
	u_\ell(x,t) - u_\ell(x,0) = \int_0^t \p_\tau u_\ell(x,\tau) \; d \tau.
\end{equation*}
Hence
\begin{equation}
	\begin{split}
		\|u_\ell(x,t) - u_\ell(x,0)\|_{H^1(\rr)}
		& = \left \| \int_0^t \p_\tau
		u_\ell(x,\tau) \; d \tau \right \|_{H^1(\rr)}
		\\
		& \le  \int_0^t \|\p_\tau u_\ell (x,\tau) \|_{H^1(\rr)} \; d \tau.
		\label{apple100}
	\end{split}
\end{equation}
We want to estimate the right hand side of \eqref{apple100}. Recalling
\eqref{apple1'}, we have
\begin{equation}
	\label{apple101}
	\begin{split}
		\|\p_\tau u_\ell(x,\tau) \|_{H^1(\rr)}
		& =  \|-\gamma u_\ell \p_x u_\ell + \Lambda^{-1} \left[
		\frac{3-\gamma}{2}(u_\ell)^2 + \frac{\gamma}{2} \left( \p_x u_\ell \right)^2
		\right] \|_{H^1(\rr)}
		\\
		& \le \|\gamma u_\ell \p_x u_\ell \|_{H^1(\rr)} + \|\Lambda^{-1} \left[
		\frac{3-\gamma}{2} (u_\ell)^2 + \frac{\gamma}{2} \left( \p_x u_\ell \right)^2
		\right] \|_{H^1(\rr)}.
	\end{split}
\end{equation}
Applying the algebra property of Sobolev spaces, we obtain
\begin{equation*}
	\begin{split}
		\|\gamma u_\ell \p_x u_\ell \|_{H^1(\rr)} &
		= \|\gamma \p_x (u_\ell)^2 \|_{H^1(\rr)}
		\\
		& \le |\gamma| \cdot \| (u_\ell)^2 \|_{H^2(\rr)}
		\\
		& \lesssim \|u_\ell\|_{H^2(\rr)}^2
	\end{split}
\end{equation*}
which by \eqref{apple10'} reduces to
\begin{equation}
	\begin{split}
		\|\gamma u_\ell \p_x u_\ell \|_{H^1(\rr)} \lesssim \lambda^{-2 + \delta}.
		\label{apple102}
	\end{split}
\end{equation}
Next, note that, for any $u \in L^2(\rr)$, we
have
\begin{equation}
	\begin{split}
		\|\Lambda^{-1} u \|_{H^1(\rr)} 
		\le \|u\|_{L^2(\rr)}.
		\label{apple27}
	\end{split}
\end{equation}
Hence, applying \eqref{apple27}, the algebra property of Sobolev spaces,
and the Sobolev Imbedding Theorem, we obtain
\begin{equation*}
	\begin{split}
		\|\Lambda^{-1} \left[ \frac{3-\gamma}{2}u^2 +
		\frac{\gamma}{2}\left( \p_x u \right)^2 \right] \|_{H^1(\rr)}
		& \lesssim \|u\|_{H^2(\rr)}^2
	\end{split}
\end{equation*}
which by \eqref{apple10'} reduces to 
\begin{equation}
	\begin{split}
		\|\Lambda^{-1} \left[ \frac{3-\gamma}{2}u^2 +
		\frac{\gamma}{2}\left( \p_x u \right)^2 \right] \|_{H^1(\rr)}
		\lesssim \lambda^{-2 + \delta}.
		\label{apple104}
	\end{split}
\end{equation}
Substituting \eqref{apple102} and \eqref{apple104} into the right hand side of
\eqref{apple101}, and recalling \eqref{apple100}, we obtain
\begin{equation}
	\begin{split}
		\|u_\ell(x,t) - u_\ell(x,0)\|_{H^1(\rr)} \lesssim \lambda^{-2 + \delta}.
		\label{apple105}
	\end{split}
\end{equation}
By estimates \eqref{apple14}, \eqref{apple15}, and \eqref{apple105}, we conclude that 
\begin{equation}
	\begin{split}
		\|E_1\|
		& \lesssim \lambda^{\frac{\delta}{2}-s}.
		\label{apple106}
	\end{split}
\end{equation}
%We now estimate the second piece of \eqref{apple14} by applying the triangle inequality and
%Lemma \ref{appleb} to obtain
%\begin{equation}
%	\begin{split}
%		\|\left[ u_\ell(x,0) - u_\ell(x,t) \right] \|_{H^1(\rr)} \le 2 c_1 \lambda^{-1 +
%		\frac{\delta}{2}}.
%		\label{apple16}
%	\end{split}
%\end{equation}
%Substituting estimates \eqref{apple15} and \eqref{apple16} into \eqref{apple14} gives
%\begin{equation*}
%	\begin{split}
%		\|E_1\|_{H^1(\rr)}
%		& \le |\gamma| \lambda^{1 - \frac{\delta}{2} -s} \cdot \lambda
%		\cdot 2c_1 \lambda^{-1 + \frac{\delta}{2}}
%	\end{split}
%\end{equation*}
%which simplifies to
%\begin{equation}
%	\begin{split}
%		\|E_1\|_{H^1(\rr)}	
%		 \le 2 |\gamma| c_1 \lambda^{-s}.
%		\label{apple17}
%	\end{split}
%\end{equation}
%
\vskip0.1in
{\bf Estimating the $H^1$ norm of $\hyperref[all_errors_together]{E_2}$.} Applying Lemma \ref{applec}, we have
\begin{equation}
	\begin{split}
		\|E_2\|_{H^1(\rr)} 
		& = \gamma \lambda^{-\frac{3 \delta}{2} -s } \|u_\ell(x,t) \cdot
		\phi'\left( \frac{x}{\lambda^\delta} \right) \cos (\lambda x - \gamma \omega t)
		\|_{H^1(\rr)}
		\\
		& \le c_s \gamma \lambda^{\frac{-3 \delta}{2} -s } \|u_\ell(x,t) \|_{H^1(\rr)}
		\|\phi'\left( \frac{x}{\lambda^\delta} \right )
		\cos(\lambda x - \gamma \omega t 
		\|_{C^1(\rr)}.
		\label{apple18}
	\end{split}
\end{equation}
We note that
\begin{equation*}
	\begin{split}
		& \|\phi'\left( \frac{x}{\lambda^\delta} \right) \cos(\lambda x - \gamma \omega t)
		\|_{C^1(\rr)}
		\\
		& \le \|\phi' \left( \frac{x}{\lambda^\delta} \right)\|_{L^\infty(\rr)} +
		\lambda \|\phi'\left( \frac{x}{\lambda^\delta} \right)\|_{L^\infty(\rr)}
		+ \lambda^{-\delta} \|\phi''\left( \frac{x}{\lambda^\delta} \right)
		\|_{L^\infty(\rr)}
	\end{split}
\end{equation*}
which gives
\begin{equation}
	\begin{split}
		\|\phi'\left( \frac{x}{\lambda^\delta} \right) \cos(\lambda x - \gamma \omega t)
		\|_{C^1(\rr)} \lesssim \lambda.
		\label{apple19}
	\end{split}
\end{equation}
Applying estimates \eqref{apple19} and \eqref{apple10'} to \eqref{apple18}, we obtain
\begin{equation*}
	\begin{split}
	\label{apple20}
	\|E_2\|_{H^1(\rr)} \lesssim \lambda^{-\delta -s }.
\end{split}
\end{equation*}
%
%
\vskip0.1in
%
%
{\bf Estimating the $H^1$ norm of $\hyperref[all_errors_together]{E_3}$.} 
By Lemma \ref{applec}, we deduce
\begin{equation}
	\begin{split}
		\|\gamma u^h \p_x u_\ell \| \le \sqrt{2} |\gamma| \cdot \|u^h\|_{C^1(\rr)}
		\|u_\ell\|_{H^1(\rr)}.
		\label{apple21}
	\end{split}
\end{equation}
Now, note that
\begin{equation}
	\begin{split}
		\|u^h\|_{L^\infty(\rr)} 
		& = \lambda^{-\frac{\delta}{2} -s } \|\phi\left( \frac{x}{\lambda^\delta}
		\right) \cos \left( \lambda x - \gamma \omega t \right) \|_{L^\infty(\rr)}
		\\
		& \lesssim \lambda^{-\frac{\delta}{2} -s }.
		\label{apple22}
	\end{split}
\end{equation}
and 
\begin{equation}
	\begin{split}
		& \|\p_x u^h \|_{L^\infty(\rr)}
		\\
		& = \lambda^{-\frac{\delta}{2}-s} \|\phi\left(
		\frac{x}{\lambda^\delta}
		\right) \cdot -\lambda \sin(\lambda x - \gamma \omega t) + \lambda^{-\delta}
		\phi'\left( \frac{x}{\lambda^\delta}\right) \cos(\lambda x - \gamma \omega
		t) \|_{L^\infty(\rr)}
		\\
		& \lesssim \lambda^{1 - \frac{\delta}{2} -s }.
		\label{apple23}
	\end{split}
\end{equation}
Therefore, from \eqref{apple22} and \eqref{apple23} it follows that
\begin{equation}
	\begin{split}
		\|u^h\|_{C^1(\rr)} \lesssim \lambda^{-\frac{\delta}{2} -s } + \lambda^{1
		-\frac{\delta}{2} -s}
		\approx \lambda^{1- \frac{\delta}{2} -s}.
		\label{apple24}
	\end{split}
\end{equation}
Substituting estimates \eqref{apple24} and  \eqref{apple10'} into \eqref{apple21} we obtain
\begin{equation}
	\begin{split}
		\|\gamma u^h \p_x u_\ell \|_{H^1(\rr)} \lesssim \lambda^{-s}.
		\label{apple24'}
	\end{split}
\end{equation}
\vskip0.1in
{\bf Estimating the $H^1$ norm of $\hyperref[all_errors_together]{E_4}$.} Applying Lemma \ref{applec} we have
\begin{equation}
	\begin{split}
		\|\gamma u^h \p_x u^h\|_{H^1(\rr)} \lesssim \|u^h\|_{C^1(\rr)}
		\|u^h\|_{H^1(\rr)}.
		\label{apple25}
	\end{split}
\end{equation}
Substituting in \eqref{apple24}, and recalling that $\|u^h\|_{H^1(\rr)} \approx 1$ by Lemma
\ref{applea}, we obtain
\begin{equation}
	\begin{split}
		\|u^h \p_x u^h \|_{H^1(\rr)} \lesssim \lambda^{1-\frac{\delta}{2}-s}.
		\label{apple26}
	\end{split}
\end{equation}
\vskip0.1in
%
%
{\bf Estimating the $H^1$ norm of $\hyperref[all_errors_together]{E_5}$.}
Applying \eqref{apple27}, we obtain
\begin{equation}
	\begin{split}
		\|E_5\|_{H^1(\rr)}
		& = \|\Lambda^{-1}\left[ \frac{3-\gamma}{2}(u^h)^2
		\right]\|_{H^1(\rr)}
		\\
		& \lesssim \|u^h\|_{L^\infty(\rr)} \|u^h\|_{L^2(\rr)}.
		\label{apple28}
	\end{split}
\end{equation}
Substituting \eqref{apple6} and \eqref{apple22} into \eqref{apple28}, we conclude that
\begin{equation}
	\begin{split}
		\|E_5\|_{H^1(\rr)} \lesssim \lambda^{-\frac{\delta}{2}-s}.
		\label{apple29}
	\end{split}
\end{equation}
%
%
\vskip0.1in
%
%
{\bf Estimating the $H^1$ norm of $\hyperref[all_errors_together]{E_6}$.} Applying \eqref{apple27}, we obtain
\begin{equation}
	\begin{split}
		\|E_6\|_{H^1(\rr)} 
		& = \|\Lambda^{-1} \left[ (3 -\gamma) u_\ell u^h \right]\|_{H^1(\rr)}
		\\
		& \lesssim \|u_\ell\|_{L^2(\rr)} \|u^h\|_{L^\infty(\rr)}.
		\label{apple30}
	\end{split}
\end{equation}
which by Lemma \ref{appleb} and \eqref{apple22} reduces to
\begin{equation}
	\begin{split}
		\|E_6\|_{H^1(\rr)} \lesssim \lambda^{-1-s}.
		\label{apple31}
	\end{split}
\end{equation}
%
%
\vskip0.1in
%
%
{\bf Estimating the $H^1$ norm of $\hyperref[all_errors_together]{E_7}$.} Applying \eqref{apple27}, we obtain
\begin{equation}
	\begin{split}
		\|E_7\|_{H^1(\rr)} 
		& = \|\Lambda^{-1} \left[ \frac{\gamma}{2}\left( \p_x u \right)^2
		\right]\|_{H^1(\rr)}
		\\
		& \lesssim  \|\p_x u^h\|_{L^\infty(\rr)} \|u^h\|_{H^1(\rr)}.
		\label{apple32}
	\end{split}
\end{equation}
which by Lemma \ref{applea} and \eqref{apple23} reduces to
\begin{equation}
	\begin{split}
		\|E_7\|_{H^1(\rr)} \lesssim \lambda^{1-\frac{\delta}{2}-s}.
		\label{apple33'}
	\end{split}
\end{equation}
%
%
\vskip0.1in
%
%
{\bf Estimating the $H^1$ norm of $\hyperref[all_errors_together]{E_8}$.} Applying \eqref{apple27}, we have
\begin{equation}
	\begin{split}
		\|E_8\|_{H^1(\rr)}
		& = \|\Lambda^{-1}\left[ \gamma \p_x u_\ell \p_x u^h \right]\|_{H^1(\rr)}
		\\
		& \lesssim \|u_\ell\|_{H^2(\rr)} \|\p_x u^h\|_{L^\infty(\rr)}.
		\label{apple33}
	\end{split}
\end{equation}
which by Lemma \ref{appleb} and \eqref{apple23} reduces to
\begin{equation}
	\begin{split}
		\|E_8\|_{H^1(\rr)} \lesssim \lambda^{-s}.
		\label{apple34}
	\end{split}
\end{equation}
Collecting all our estimates for the $E_i$ and recalling that we have assumed
$1<\delta<2$, we obtain
\begin{equation*}
	\begin{split}
		\|E\|_{H^1(\rr)}
		 \lesssim \lambda^{\frac{\delta}{2} -s }, \qquad \lambda >>1.
	\end{split}
\end{equation*}
We now summarize our result:
%
%
\begin{proposition}
	Let $1<\delta<2$. Then for $s > 1$, bounded $\omega$, and
	$\lambda >>1$ we are assured the decay of the error $E$ of the
	approximate solutions to the HR equation; specifically
	\begin{equation}
		\begin{split}
			\|E(t)\|_{H^1(\rr)} \lesssim \lambda^{-r_s}
			\label{apple35}
		\end{split}
	\end{equation}
	where
	\begin{equation}
		\begin{split}
			r_s = s - \frac{\delta}{2}.   
			\label{appler_s}
		\end{split}
	\end{equation}
\end{proposition}
%
%
%
%
%
{\bf Estimating the $H^1$ norm of the difference \\
between approximate and actual
	solutions.}
	\vskip0.1in
	We wish now to estimate the difference between approximate and actual solutions to
	the HR i.v.p with common initial data $u_0 \in H^s$. Let
	$u_{\omega,\lambda}(x,t)$ be the unique solution to the HR equation
	with initial data $u^{\omega,\lambda}(x,0)$; that is,
	$u_{\omega,\lambda}$ solves the initial value problem
	\begin{align}
		& \p_t u_{\omega,\lambda} + \gamma u_{\omega,\lambda} \p_x u_{\omega,\lambda} + \Lambda^{-1} \left[
		\frac{3- \gamma}{2}\left( u_{\omega,\lambda} \right)^2 + \frac{\gamma}{2}\left(
		\p_x u_{\omega,\lambda} \right)^2
		\right], \; \; x\in \rr, \; \; t \in \rr,
		\label{apple50}
		\\
		& u_{\omega,\lambda} = u^{\omega,\lambda}(x,0)=\omega \lambda^{-1}
		\tilde{\phi} \left( \frac{x}{\lambda^\delta} \right)
		+ \lambda^{-\frac{\delta}{2} -s}
		\phi\left( \frac{x}{\lambda^\delta} \right) \cos(\lambda x).
		\label{apple41}
	\end{align}

	
	%
%
%
Letting $v = u^{\omega,\lambda} - u_{\omega,\lambda}$, we will now prove the following critical lemma:
\vskip0.2in
\begin{lemma}
	\label{applelem:bound_for_difference-of-approx-and-actual-soln}
	For $s > 1$ and $1<\delta<2$ we have 
			\begin{equation} 
				\|
				v(t)
				\|_{H^1(\rr)}
				\doteq
				\label{applediffer-H1-est} 
				\|
				u^{\omega,\lambda}(t) 
				- 
				u_{\omega,\lambda}(t)
				\|_{H^1(\rr)}
				\lesssim 
				n^{-r_s}, 
				\quad
				|t| \le T
			\end{equation}
			where $r_s$ is defined as in \eqref{appler_s}.
			%
			\end{lemma}
			{\bf{Proof.}} Our plan will be to calculate an energy-estimate for $v$.
			To do so, we must first calculate $\p_t v$. Subtracting $\p_t
			u_{\omega, \lambda}$ from $\p_t u^{\omega,\lambda}$ and
			recalling that $u_{\omega,t}$ is a solution to the HR Cauchy
			problem \eqref{apple50}-\eqref{apple41},
			and $u_{\omega,\lambda}$ is an approximate solution, we obtain
			\begin{equation*}
				\begin{split}
					\p_t v 
					& = E + \gamma(v \p_x v - v \p_x u^{\omega,\lambda} - u^{\omega,\lambda} \p_x v) 
					\\
					& + \p_x \left( 1 - \p_x^2 \right)^{-1}  \left[ \frac{3-
					\gamma}{2}v^2 + \frac{\gamma}{2}\left( \p_x v \right)^2 - \left(
					3 - \gamma \right)u^{\omega,\lambda} v -
					\gamma \p_x u^{\omega,\lambda} \p_x v \right].
				\end{split}
			\end{equation*}
			It follows immediately that
		\begin{equation}
			\label{applev-dtv-pseudo-functional-equality}
			\begin{split}
			v(1-\p_x^2)\p_t v &= v(1- \p_x^2)E + v\gamma(1- \p_x^2)(v\p_x v 
			- v\p_x u^{\omega,\lambda} -
			u^{\omega,\lambda} \p_x v)
			\\
			&+ v\p_x \left[ \frac{3-\gamma}{2}v^2 + \frac{\gamma}{2}(\p_x v)^2 -
			(3-\gamma)u^{\omega,\lambda} v - \gamma \p_x u^{\omega,\lambda} \p_x v \right].
		\end{split}
	\end{equation}
	Applying the equality $v\p_t v = v(1-\p_x^2) \p_t v + v\p_x^2 \p_t v$ to
	\eqref{applev-dtv-pseudo-functional-equality}, we obtain
	\begin{equation*}
		\begin{split}
		v \p_t v &= v(1- \p_x^2)E + v\gamma(1- \p_x^2)(v\p_x v - v\p_x u^{\omega,\lambda} -
			u^{\omega,\lambda} \p_x v)
			\\
			&+ v\p_x \left[ \frac{3-\gamma}{2}v^2 + \frac{\gamma}{2}(\p_x v)^2 -
			(3-\gamma)u^{\omega,\lambda} v - \gamma \p_x u^{\omega,\lambda} \p_x v
			\right] + v\p_x^2 \p_t v.
		\end{split}
	\end{equation*}
	Hence
	\begin{equation}
		\label{appleenergy-est}
		\begin{split}
			&\frac{1}{2} \frac{d}{dt} \|v\|_{H^1(\rr)}^2  
			\\
		& =  \int_{\rr} \left[ v(1-\p_x^2)E \right]dx
		- \gamma \int_{\rr} \left[ v(1-\p_x^2)(v\p_x u^{\omega,\lambda} + u^{\omega,\lambda} \p_x v) \right]dx
		\\
		&- \int_{\rr}\left[ \left( 3-\gamma \right)v \p_x\left( u^{\omega,\lambda}v \right) + \gamma v
		\p_x \left( \p_x u^{\omega,\lambda} \p_x v \right)\right]dx
		\\
		&+  \int_{\rr}
		\left[ \gamma v \left( 1-\p_x^2 \right)\left( v \p_x v \right) + v
		\p_x \left( \frac{3-\gamma}{2} v^2 + \frac{\gamma}{2}\left( \p_x v \right)^2
		\right) \right . +  v \p_x^2 \p_t v + \p_x v \p_t \p_x v\bigg]dx.
	\end{split}
\end{equation}
We compute the last term of \eqref{appleenergy-est} first:
\begin{equation*}
	\begin{split}
	&  \int_{\rr} \bigg[ \gamma v \left( 1-\p_x^2 \right)(v\p_x v) + v \p_x\left(
	\frac{3-\gamma}{2}v^2 + \frac{\gamma}{2}\left( \p_x v \right )^2 \right)
	+ v\p_x^2 \p_t v + \p_x v \p_t \p_x v
	\bigg]dx
	\\
	& =  \int_{\rr} \left[ 3v^2 \p_x v - \gamma v^2 \p_x^3 v - 2 \gamma v \p_x v \p_x^2
	v + v \p_x^2 \p_t v + \p_x v \p_t \p_x v \right]dx
	\\
	&=  \int_{\rr} \left[ \p_x (v^3) - \gamma \p_x (v^2 \p_x^2 v) + \p_x\left( v \p_t
	\p_x v
	\right) \right]dx
	\\
	& = 0.
\end{split}
\end{equation*}
Therefore			
\begin{equation}
	\label{appleenergy-estimate-simplified}
	\begin{split}
		\frac{1}{2}	\frac{d}{dt} \|v(t)\|_{H^1(\rr)}^2
		& =  \int_{\rr}
		 v\left( 1-\p_x^2
	\right)E \; dx
	\\
	&-  \gamma  \int_{\rr}  v\left( 1-\p_x^2 \right)\left( v \p_x u^{\omega,\lambda}
	+ u^{\omega,\lambda} \p_x v
	\right) \; dx
	\\
	& -  \int_{\rr} \left[ \left( 3-\gamma \right)v \p_x \left( u^{\omega,\lambda}v \right) + \gamma v
	\p_x \left( \p_x u^{\omega,\lambda} \p_x v \right)\right]dx.
\end{split}
\end{equation}
We now estimate the right-hand-side of \eqref{appleenergy-estimate-simplified}:
%
\begin{equation}
	\begin{split}
		\label{applefirst_piece}
	\left |\int_{\rr} \left [v (1- \p_x^2)E \right ] dx \right |
	& \lesssim
	\left( \|v\|_{L^2(\rr)}
	\|E\|_{L^2(\rr)} + \|\p_x v \|_{L^2(\rr)}
		\|\p_xE\|_{L^2(\rr)}\right)
	\\
	&
	\lesssim
	\|v\|_{H^1(\rr)} \|E\|_{H^1(\rr)}.
\end{split}
\end{equation}
For the second piece of \eqref{appleenergy-estimate-simplified} we have
\begin{equation}
	\begin{split}
		\label{applesecond-piece}
		& -\gamma \int_{\rr} v\left( 1-\p_x^2 \right)\left( v \p_x u^{\omega,\lambda} +
		u^{\omega,\lambda} \p_x v
		\right) \; dx
		\\
		& = -\gamma \int_{\rr} \left[ v^2 \p_x u^{\omega,\lambda} - v \p_x^2\left( v \p_x u^{\omega,\lambda}
		\right) \right]dx
		\\
		&   -\gamma \int_{\rr}\left[ vu^{\omega,\lambda} \p_x v - v \p_x^2\left( u^{\omega,\lambda} \p_x v \right)
		\right]dx.
	\end{split}
\end{equation}
We estimate the first term of \eqref{applesecond-piece} in parts:
\begin{equation*}
	\begin{split}
		\left | -\gamma \int_{\rr} \left[ v^2 \p_x u^{\omega,\lambda} \right]dx \right |
		& \lesssim \|\p_x u^{\omega,\lambda} \|_{L^\infty(\rr)} \|v\|_{H^1(\rr)}^2
	\end{split}
\end{equation*}
and by the product rule, Cauchy-Schwartz, and the Sobolev Imbedding Theorem,
we also have 
\begin{equation*}
	\begin{split}
		\left |-  \gamma \int_{\rr} \left [-v \p_x^2 \left(v \p_x u^{\omega,\lambda}
		\right) \right ]  dx \right |
		& \lesssim \left ( \|v\|_{L^2(\rr)} \|\p_x v\|_{L^2(\rr)} \|\p_x^2
		u^{\omega,\lambda} \|_{L^\infty(\rr)} \right .
		\\
		&+ \|\p_x v \|_{L^2(\rr)}^2 \|\p_x u^{\omega,\lambda}\|_{L^\infty(\rr)} \left )
		\right .
		\\
		& \lesssim \left( \|\p_x u^{\omega,\lambda} \|_{L^\infty(\rr)}+ \|\p_x^2
		u^{\omega,\lambda} \|_{L^\infty(\rr)} \right )\|v\|_{H^1(\rr)}^2 .
	\end{split}
\end{equation*}
Hence,
\begin{equation}
	\label{applepart1}
	\begin{split}
		& \left | - \gamma \int_{\rr} \left[ v^2 \p_x u^{\omega,\lambda} - v\p_x^2 \left( v \p_x u^{\omega,\lambda} \right)
		\right]dx \right |
		 \\
		 &  \lesssim  \left( \|\p_x u^{\omega,\lambda} \|_{L^\infty (\rr)} + \| \p_x^2 u^{\omega,\lambda}
		\|_{L^\infty(\rr)}
		\right)
		\|v \|_{H^1 (\rr)}^2.
	\end{split}
\end{equation}
Next, we estimate the second term of \eqref{applesecond-piece} in parts. For the first part, we apply Cauchy-Schwartz to obtain:
\begin{equation}
	\label{applefirst-part}
	\begin{split}
		\left |  -\gamma \int_{\rr} \left[ v u^{\omega,\lambda} \p_x v \right] dx \right |
		& \lesssim \|u^{\omega,\lambda}\|_{L^\infty(\rr)} \|v\|_{H^1(\rr)}^2
	\end{split}
\end{equation}
For the second part, we use integration by parts and the product rule to
obtain
\begin{equation}
	\label{appleboo}
	\begin{split}
		 \left | -\gamma \int_{\rr} 
		 -v \p_x^2 \big ( u^{\omega,\lambda} \p_x v \big )
		\; dx \right | 
		& \simeq \left | \int_{\rr} \left[ 
		\p_x\left( u^{\omega,\lambda}\left( \p_x v
		\right)^2 \right) + \p_x u^{\omega,\lambda}\left( \p_x v \right)^2
		\right]dx \right |.
	\end{split}
\end{equation}
Since $u^{\omega,\lambda}$ is compactly supported on $\rr$, \eqref{appleboo}
gives
\begin{equation}
	\label{applesecond-part}
	\begin{split}
		 \left | -\gamma \int_{\rr} \left [-v \p_x^2 \big ( u^{\omega,\lambda} \p_x v \big )\right
		] dx \right | 
		& \lesssim \|\p_x u^{\omega,\lambda} \|_{L^\infty(\rr)} \|v \|_{H^1(\rr)}^2.
	\end{split}
\end{equation}
Grouping \eqref{applefirst-part} and \eqref{applesecond-part} we obtain
\begin{equation}
	\label{applepart2}
	\left |  -\gamma \int_{\rr} \left[ v u^{\omega,\lambda} \p_x v - v \p_x^2\left( u^{\omega,\lambda} \p_x v \right)
	\right]dx \right | \lesssim \left( \|u^{\omega,\lambda}\|_{L^\infty(\rr)} + \|\p_x u^{\omega,\lambda}
	\|_{L^\infty(\rr)}
	\right)\|v\|_{H^1(\rr)}^2.
\end{equation}
which, combined with \eqref{applepart1} yields an estimate for
\eqref{applesecond-piece}:
\begin{equation}
	\begin{split}
		\label{applesecond-piece-final}
		\left | -\gamma \int_{\rr}
		\left[ v\left( 1-\p_x^2 \right)\left( v \p_x u^{\omega,\lambda} + u^{\omega,\lambda} \p_x v
		\right) \right] dx \right |
		&\lesssim \left( \|u^{\omega,\lambda}\|_{L^\infty(\rr)}\| + \|\p_x u^{\omega,\lambda}
		\|_{L^\infty(\rr)} \right . 
		\\
		& + \|\p_x^2 u^{\omega,\lambda} \|_{L^\infty(\rr)}
		\big )\|v\|_{H^1(\rr)}^2.
	\end{split}
\end{equation}
We now estimate the final piece of the right-hand-side of
\eqref{appleenergy-estimate-simplified}, i.e.
\begin{equation}
	\label{applelast_piece}
	-\int_{\rr} \left[ \left( 3 -\gamma \right)v \p_x \left( u^{\omega,\lambda} v \right) + \gamma
	v \p_x \left( \p_x u^{\omega,\lambda} \p_x v \right)\right]dx.
\end{equation}
We will estimate in parts:
\begin{equation}
	\begin{split}
		\label{applelast_piece_part1}
		\left | -\int_{\rr}  \left( 3- \gamma \right)v \p_x\left( u^{\omega,\lambda} v \right)
		 dx \right | 
		& \lesssim \|\p_x v \|_{L^2(\rr)} \|u^{\omega,\lambda} v \|_{L^2(\rr)}
		\\
		& \lesssim \|u^{\omega,\lambda}\|_{L^\infty(\rr)} \|v\|_{H^1(\rr)}^2
	\end{split}
\end{equation}
and
\begin{equation}
	\begin{split}
		\label{applelast_piece_part2}
		\left | -\int_{\rr}  \gamma v \p_x \left( \p_x u^{\omega,\lambda} \p_x v
		\right) dx  \right | 
		& \lesssim \|\p_x v \|_{L^2(\rr)} \| \p_x u^{\omega,\lambda} \p_x v \|_{L^2(\rr)}
		\\
		& \lesssim \|\p_x u^{\omega,\lambda} \|_{L^\infty(\rr)} \|v \|_{H^1(\rr)}^2.
	\end{split}
\end{equation}
Using \eqref{applelast_piece_part1} and \eqref{applelast_piece_part2}, we now have the
following estimate for \eqref{applelast_piece}:
\begin{equation}
	\begin{split}
	\label{applelast_piece_final}
	& \left | -\int_{\rr} \left[ \left( 3-\gamma \right)v
	\p_x \left( u^{\omega,\lambda} v \right) + \gamma
	v \p_x \left( \p_x u^{\omega,\lambda} \p_x v \right)\right]dx \right |
	\\
	& \lesssim \big(
	\|u^{\omega,\lambda}\|_{L^\infty(\rr)}
	 + \|\p_x u^{\omega,\lambda} \|_{L^\infty(\rr)} \big)
	\|v\|_{H^1(\rr)}^2.
\end{split}
\end{equation}
Combining \eqref{applefirst_piece}, \eqref{applesecond-piece-final},
and \eqref{applelast_piece_final}, we can
simplify \eqref{appleenergy-estimate-simplified} to obtain
\begin{equation}
	\begin{split}
		\label{appleenergy-estimate-best}
		\frac{d}{dt} \|v(t)\|_{H^1(\rr)}^2
		& \lesssim \left( \|u^{\omega,\lambda}\|_{L^\infty(\rr)} + \|
		\p_x u^{\omega,\lambda} \|_{L^\infty(\rr)} + \|\p_x^2 u^{\omega,\lambda} \|_{L^\infty (\rr)} \right)
		\|v\|_{H^1(\rr)}^2 
		\\
		&+ \|v\|_{H^1(\rr)} \|E\|_{H^1(\rr)}.
	\end{split}
\end{equation}
Now, observe that
\begin{equation}
	\begin{split}
		\p_x^2 u^h 
		& = \lambda^{-\frac{\delta}{2}-s} \Big[ - \lambda^2 \phi\left(
		\frac{x}{\lambda^\delta} \right ) \cos(\lambda x - \gamma \omega t) \\
		& - 2\lambda^{1 -\delta } \phi'\left( \frac{x}{\lambda^\delta}
		\right )
		\sin(\lambda x - \gamma \omega t ) + \lambda^{-2\delta} \phi''\left(
		\frac{x}{\lambda^\delta} \right) \cos (\lambda x - \gamma \omega t) \Big].
		\label{apple51}
	\end{split}
\end{equation}
Since the $\lambda^2$ term dominates inside the brackets, \eqref{apple51} yields
\begin{equation}
	\begin{split}
		\|\p_x^2 u^h \|_{L^\infty(\rr)} \lesssim
		\lambda^{2-\frac{\delta}{2}-s}.
		\label{apple52}
	\end{split}
\end{equation}
Combining \eqref{apple22}, \eqref{apple23}, and \eqref{apple52}, we obtain
\begin{equation}
	\begin{split}
		\|u^h\|_{L^\infty(\rr)} + \|\p_x u^h\|_{L^\infty(\rr)} + \|\p_x^2
		u^h\|_{L^\infty(\rr)} \lesssim \lambda^{-\left(
		\frac{\delta}{2} + s -2 \right)}.
		\label{apple53}
	\end{split}
\end{equation}
Furthermore, we have
\begin{equation}
	\begin{split}
		\|u_\ell\|_{L^\infty(\rr)} + \|\p_x u_\ell \|_{L^\infty(\rr)} + \|\p_x^2
		u_\ell\|_{L^\infty(\rr)} \le c_s \|u_\ell\|_{H^3(\rr)}.
		\label{apple54}
	\end{split}
\end{equation}
Applying Lemma \ref{appleb} to \eqref{apple54}, we see that
\begin{equation}
	\begin{split}
		\|u_\ell\|_{L^\infty(\rr)} + \|\p_x u_\ell \|_{L^\infty(\rr)} + \|\p_x^2
		u_\ell\|_{L^\infty(\rr)} \lesssim \lambda^{-(1 - \frac{\delta}{2})},
		\quad |t| \le T.
		\label{apple55}
	\end{split}
\end{equation}
Combining \eqref{apple53} and \eqref{apple55}, we obtain
\begin{equation}
	\begin{split}
		\|u^{\omega,\lambda}\|_{L^\infty(\rr)} + \|\p_x u^{\omega,\lambda}\|_{L^\infty(\rr)} + \|\p_x^2
		u^{\omega,\lambda}\|_{L^\infty(\rr)}
		& \lesssim \lambda^{-\rho_s},
		\label{apple56}
	\end{split}
\end{equation}
where
\begin{equation}
	\begin{split}
		\rho_s = \text{min} \left\{ \frac{\delta}{2} + s -2, \; 1-
		\frac{\delta}{2} \right\}.
		\label{apple57}
	\end{split}
\end{equation}
Note that for $s>1$, we can assure $\rho_s > 0$
by choosing a suitable $1<\delta<2$.
Substituting \eqref{apple35} and \eqref{apple56} into \eqref{appleenergy-estimate-best},
we get
\begin{equation}
	\label{apple58}
	\frac{d}{dt} \|v(t)\|_{H^(\rr)}^2 \lesssim \lambda^{-\rho_s}
	\|v\|_{H^1(\rr)}^2 + \lambda^{-r_s}
	\|v \|_{H^1(\rr)}
\end{equation}
where we recall the definition of $r_s$ from \eqref{appler_s}. By Gronwall's Inequality, we conclude that for $s>1$ and
suitably chosen $1<\delta<2$, we are assured the
decay of $\|v(t)\|_{H^1(\rr)}$, i.e. 
\begin{equation}
	\label{appleen-est-fin!}
	\|v(t)\|_{H^1(\rr)} 
	\lesssim
	\lambda^{-r_s}, \quad |t| \le T,\quad \lambda>>1 . \qquad \Box
\end{equation}
%
%
%
\vskip0.1in
{\bf Non-Uniform Dependence for $s > 1$.}
\vskip0.1in
Let $u_{\pm 1,\lambda}$ be solutions to the HR i.v.p. with common initial data $u^{\pm 1,
n}(0)$, respectively.
We wish to show that the $H^s$ norm of the difference of $u_{\pm 1,
n}$ and the associated approximate solution $u^{\pm 1,\lambda}$
decays as $n \to \infty$. In order for \eqref{appleen-est-fin!} to hold,
we assume $s > 1$; recalling Theorem \ref{thm:HR_existence_continous_dependence}
, we
have
\begin{equation}
	\begin{split}
		\|u_{\pm 1,\lambda} (t) \|_{H^{2s-1}(\rr)}
		& \le 2 \|u^{\pm 1,\lambda}(0) \|_{H^{2s-1}(\rr)}, \qquad
		|t| \le T.
		\label{apple60}
	\end{split}
\end{equation}
Furthermore, recalling \eqref{apple1}-\eqref{apple1***}, we have 
\begin{equation}
	\begin{split}
		& \|u^{\pm 1, \lambda}(t)\|_{H^{2s-1}(\rr)}
		\\
		& \le \|u_{\ell, \pm \omega, \lambda}\|_{H^{2s-1}(\rr)} +
		 \| \lambda^{-\frac{\delta}{2} -s} \phi \left(
		\frac{x}{\lambda^\delta} \right) \cos(\lambda x \mp \gamma \omega t)
		\|_{H^{2s-1}(\rr)}
		\\
		& = \|u_{\ell, \pm \omega, \lambda}\|_{H^{2s-1}(\rr)}
		+
		\lambda^{s-1} \cdot
		\lambda^{-\frac{\delta}{2}-(2s-1)} \|\phi \left(
		\frac{x}{\lambda^\delta} \right) \cos(\lambda x \mp \gamma \omega t)
		\|_{H^{2s-1}(\rr)}.
		\label{apple61}
	\end{split}
\end{equation}
Applying Lemma \ref{applea} and Lemma \ref{appleb} to \eqref{apple61}, we obtain
\begin{equation}
	\begin{split}
		\|u^{\pm 1, \lambda}(t) \|_{H^{2s-1}(\rr)}
		& \lesssim \lambda^{s-1}.
		\label{apple62}
	\end{split}
\end{equation}
Hence, using \eqref{apple60}, \eqref{apple62}, and the triangle inequality, we deduce
\begin{equation}
	\begin{split}
		\|u^{\pm 1, \lambda}(t) - u_{\pm 1, \lambda}(t) \|_{H^{2s-1}(\rr)}
		\lesssim \lambda^{s-1}.
		\label{apple63}
	\end{split}
\end{equation}
Furthermore, by Lemma
\ref{applelem:bound_for_difference-of-approx-and-actual-soln}, we have
\begin{equation}
	\begin{split}
		\|u^{\pm 1, \lambda}(t) - u_{\pm 1, \lambda} \|_{H^1(\rr)} \lesssim
		\lambda^{-r_s}.
		\label{apple64}
	\end{split}
\end{equation}
		We now wish to interpolate in order to obtain an estimate of the $H^s (\rr)$
		norm of the difference of the approximate and actual solutions:
		\begin{lemma}
			\label{apple403}
			For all $\psi \in L^2(\rr)$,
			\begin{equation*}
				\|\psi \|_{H^s (\rr)}^2 \leq  \| \psi \|_{H^1 (\rr)} \| \psi
				\|_{H^{2s-1}}. 
			\end{equation*}
		\end{lemma}
			Hence, using Lemma \ref{apple403} to interpolate between estimates
			\eqref{apple63} and \eqref{apple64}, we obtain
			\begin{equation}
				\begin{split}
					\|u^{\pm 1, \lambda}(t) - u_{\pm 1, \lambda}(t)
					\|_{H^s(\rr)}
					\lesssim \lambda^{\frac{\delta -2}{4}}.
					\label{apple65}
				\end{split}
			\end{equation}
			Next, we will use estimate \eqref{apple65} to prove non-uniform
			dependence when $s > 1$.




%%%%%%%%%%%%% Behavior at time  t = 0  %%%%%%%%%%%% 
  

\vskip0.1in
%
{\bf Behavior at time $t=0$.}  We have
%
%
\begin{equation}
	\begin{split}
		\|u_{1,\lambda}(0) - u_{-1,\lambda}(0) \|_{H^s(\rr)} 
		& = \|u^{1,\lambda}(0) - u^{-1,\lambda}(0) \|_{H^s(\rr)}
		\\
		& = 2 \lambda^{-1} \| \tilde{\phi}\left( \frac{x}{\lambda^\delta}
		\right) \|_{H^s(\rr)}.
		\label{apple}
	\end{split}
\end{equation}
Applying the estimate
\begin{equation}
	\begin{split}
		\|\tilde{\phi}\left( \frac{x}{\lambda^\delta}
		\right)\|_{H^{k}(\rr)} \le
		\lambda^{\frac{\delta}{2}}\|\tilde{\phi}\|_{H^{k}(\rr)},
		\qquad k\ge 0,
	\end{split}
\end{equation}
and recalling that $1<\delta<2$, we conclude that
\begin{equation}
	\begin{split}
		\|u_{1,\lambda}(0) - u_{-1,\lambda}(0) \| \le 2
		\lambda^{\frac{\delta}{2}-1} \|\tilde{\phi} \|_{H^s(\rr)} \to 0
		\; \; \text{as} \; \; \lambda \to \infty.
		\label{apple70}
	\end{split}
\end{equation}
%
%
%%%%%%%%%%%%%% Behavior at time  t >0  %%%%%%%%%%%% 
%  
%
\vskip0.1in
{\bf Behavior at time  $t>0$.}  Using the reverse triangle inequality, we have
%
%
%
\begin{equation} 
	\label{appleHR-slns-differ-t-pos}
	\begin{split}
		\|
		u_{1,\lambda}(t)
		-
		u_{- 1,\lambda}(t)
		\|_{H^s(\rr)}
		&
		\ge
		\|
		u^{1,\lambda}(t)
		-
		u^{- 1,\lambda}(t)
		\|_{H^s(\rr)}
		\\
		&
		-
		\|
		u^{1,\lambda}(t)
		-
		u_{1,\lambda}(t)
		\|_{H^s(\rr)}
		\\
		&
		-
		\|
		-u^{-1,\lambda}(t)
		+
		u_{-1,\lambda}(t)
		\|_{H^s(\rr)} .
	\end{split}
\end{equation}
%
%
Using estimate \eqref{apple65} for the last two terms 
in \eqref{appleHR-slns-differ-t-pos} we obtain
%
%
%
\begin{equation} 
	\label{appleHR-slns-differ-t-pos-est}
	\|
	u_{1,\lambda}(t)
	-
	u_{- 1,\lambda}(t)
	\|_{H^s(\rr)}
	\ge
	\|
	u^{1,\lambda}(t)
	-
	u^{- 1,\lambda}(t)
	\|_{H^s(\rr)}
	-
	c \lambda^{\frac{\delta - 2}{4}}
\end{equation}
where c is a positive, non-zero constant. Letting $\lambda$ go to $\infty$ in
\eqref{appleHR-slns-differ-t-pos-est}
yields
%
\begin{equation} 
	\label{appleHR-slns-to-ap-est}
	\liminf_{n\to\infty}
	\|
	u_{1,\lambda}(t)
	-
	u_{- 1,\lambda}(t)
	\|_{H^s(\rr)}
	\ge
	\liminf_{n\to\infty}
	\|
	u^{1,\lambda}(t)
	-
	u^{- 1,\lambda}(t)
	\|_{H^s(\rr)}.
\end{equation}
%
%
Hence, by \eqref{appleHR-slns-to-ap-est}, we see we have reduced the problem of
analyzing the growth of the difference of actual solutions to the more
manageable problem of analyzing the growth of the associated approximate
solutions. Using the identity 
$$
\cos \alpha -\cos \beta
=
-2
\sin(\frac{\alpha + \beta}{2})
\sin(\frac{\alpha - \beta}{2})
$$
gives
\begin{equation}
	\label{apple80}
	\begin{split}
u^{1,\lambda}(t)
-
u^{- 1,\lambda}(t)
=
u_{\ell,1,\lambda}(t) - u_{\ell,-1,\lambda}(t) + 2\lambda^{-\frac{\delta}{2}-s}
\phi\left( \frac{x}{\lambda^\delta} \right)\sin(\lambda x) \sin(\gamma t).
\end{split}
\end{equation}
Now, by Lemma \ref{appleb}, we have
\begin{equation*}
	\begin{split}
	\|u_{\ell,-1,\lambda}(t) - u_{\ell,1,\lambda}(t)\|_{H^s(\rr)} \lesssim
	\lambda^{-1 + \frac{\delta}{2}};
	\end{split}
\end{equation*}
hence applying the reverse triangle inequality to \eqref{apple80}, we obtain
\begin{equation} 
	\label{apple90}
	\begin{split}
	& \|
	u^{1,\lambda}(t)
	-
	u^{- 1,\lambda}(t)
	\|_{H^s(\rr)}
	\\
	& \ge 2 \lambda^{-\frac{\delta}{2}-s} \|\phi\left(
	\frac{x}{\lambda^\delta} \right) \sin(\lambda x) \|_{H^s(\rr)} |\sin \gamma t|
	\\
	& - \|u_{\ell,-1,\lambda}(t) - u_{\ell,1,\lambda}(t)\|_{H^s(\rr)} 
	\\
	& \gtrsim \lambda^{-\frac{\delta}{2}-s} \|\phi\left(
	\frac{x}{\lambda^\delta} \right ) \sin(\lambda x) \|_{H^s(\rr)} |\sin \gamma t| -
	\lambda^{-1 + \frac{\delta}{2}}.
\end{split}
\end{equation}
%
%
Letting $\lambda$ go to $\infty$, Lemma \ref{applea}
with \eqref{apple90}  gives
%
%
\begin{equation} 
	\label{apple91}
	\liminf_{\lambda \to\infty}
	\|
	u^{1,\lambda}(t)
	-
	u^{- 1,\lambda}(t)
	\|_{H^s(\rr)}
	\gtrsim
	|\sin \gamma t|.
\end{equation}
Combining \eqref{appleHR-slns-to-ap-est} with \eqref{apple91}, we see that
\begin{equation}
	\begin{split}
		\liminf_{\lambda \to \infty} \|u_{1,\lambda}(t) -
		u_{-1,\lambda}(t) \|_{H^s(\rr)} \gtrsim |\sin
		\gamma t |, \qquad |t| \le T,
		\label{apple92}
	\end{split}
\end{equation}
proving \eqref{bdd-away-from-0}. Furthermore, a computation analogous
to that in \eqref{apple60}-\eqref{apple62} yields
\begin{equation}
	\label{apple93}
	\begin{split}
		\|u_{\pm 1, \lambda} (t) \|_{H^{s}(\rr)}
		& \lesssim 1.
	\end{split}
\end{equation}
Collecting \eqref{apple70}, \eqref{apple92}, and \eqref{apple93}, we
conclude that we have proven Theorem \ref{hr-non-unif-dependence} for the
non-periodic case.

	%
	%%%%%%%%%%%%%%%%%%%%%%%%%%%%%%%%%%
	%
	%
	%
	%             Proof of Theorem in the Periodic case
	%
	%
	%
	%%%%%%%%%%%%%%%%%%%%%%%%%%%%%%%%%%
	%
	\section{Proof of Theorem \ref{hr-non-unif-dependence}
 in the Periodic case} 
	\setcounter{equation}{0}
	%
	%
	%
	%
	We will consider approximate solutions of form
	\begin{equation}
		\label{approx-solutions-form}
		u^{\omega,n}(x,t) = \omega n^{-1} + n^{-s} \cos \left( nx - \gamma \omega t
		\right) 
	\end{equation}
	with integer valued $n >>1$, bounded $\omega \in \rr$, and a fixed constant $\gamma \in \rr$. We rewrite 
	the HR i.v.p as
	\begin{align}
			 \p_t u + \gamma u \p_x u \ + & \ \p_x (1 - \p_x^2)^{-1} 
			 \left[ \frac{3 - \gamma}{2}u^2 +
			\frac{\gamma}{2}(\p_x u)^2 \right] = 0,
			\label{hyperelastic-rod-equation}
			\\
			& {u(x,0) = u_0(x)},
			\label{init-cond}
			\end{align}
	and aim to substitute our approximate solution into the left hand side 
	in order to obtain a functional representation of its error. Hence, some 
	preliminary calculations are necessary. We omit the superscripts $w,n$ for clarity: 
	\begin{equation}
		\begin{split}
			 \p_t u
			 & = \gamma \omega n^{-s} \sin\left( nx - \gamma \omega t \right),
			\\
			 \p_x u
			 & = -n^{-s+1} \sin(nx - \gamma \omega t),
			\\
			\gamma u \p_x u
			& = - \gamma \omega n^{-s} \sin\left( nx - \gamma
			\omega t \right) - \frac{\gamma}{2}n^{-2s+1}\sin\left( 2\left( nx - \gamma
			\omega t
			\right) \right),
			\\
			u^2 
			& = \omega^2 n^{-2} + 2\omega n^{-s -1} \cos \left( nx - \gamma \omega t
			\right) + n^{-2s} \cos^2\left( nx - \gamma \omega t \right),
			\\
			(\p_x u)^2 
			& = n^{-2s+2} \sin^2\left( nx - \gamma \omega t \right).
			\label{calculation of functional representation of error}
		\end{split}
	\end{equation}
	Using these relations, we obtain
	\begin{equation}
		\begin{split}
			\p_t u + \gamma u \p_x u
			& + \p_x(1- \p_x^2)^{-1} \left[
			\frac{3-\gamma}{2}u^2 + \frac{\gamma}{2}(\p_x u)^2 \right]
			\\
			& = \cancel{\gamma \omega n^{-s} \sin(nx - \gamma \omega t)} -
			\cancel{\gamma \omega n^{-s}\sin\left( nx - \gamma \omega t \right)}
			\\
			& -
			\frac{\gamma}{2}n^{-2s+1}\sin\left( 2\left( nx - \gamma \omega t \right)
			\right)
			\\
			& + \p_x \left( 1-\p_x^2 \right)^{-1}\bigg[ \frac{3-\gamma}{2} \bigg (
			\omega^2 n^{-2} + 2 \omega n^{-s -1}\cos( nx - \gamma \omega t )
			\\
			& + n^{-2s}\cos^2\left( nx - \gamma \omega t \right) \bigg ) + \frac{\gamma}{2}
			n^{-2s+2}\sin^2\left( nx - \gamma \omega t \right)
			\bigg]
			\\
			& \doteq E.
			\label{functional-representation-of-error}
		\end{split}
	\end{equation}
	Since $\frac{(3-\gamma)}{2}w^2 n^{-2}$ is a constant, it's derivative vanishes;
	hence we can rewrite the error $E$ as
	\begin{equation}
		\begin{split}
			E= E_1 + E_2 + E_3 + E_4
			\label{57}
		\end{split}
	\end{equation}
	where
	\begin{align}
		\label{90*}
			& E_1 =
			- \frac{\gamma}{2}n^{-2s+1}\sin\left[ 2\left( nx - \gamma 
			\omega t \right)
			\right],
			\\
			\label{90**}
			& E_2 = \p_x \left( 1-\p_x^2 \right)^{-1}\bigg[ \frac{3-\gamma}{2} \bigg (
			2 \omega n^{-s -1}\cos( nx - \gamma \omega t )
			\bigg )
			\bigg ],
			\\
			\label{90***}
			& E_3 = \p_x \left( 1-\p_x^2 \right)^{-1}\bigg[ 
			\frac{3-\gamma}{2} \bigg (
			 n^{-2s}\cos^2\left( nx - \gamma \omega t \right) \bigg )
			\bigg ],
			\\
			& E_4 = \frac{\gamma}{2}
			n^{-2s+2}\sin^2\left( nx - \gamma \omega t \right).
			\label{90}
	\end{align}
%
%
%
\noindent
\vskip0.1in
{\bf  Estimate for the  Error of the Approximate Solutions.}
%
%
First, we will need the following two lemmas:
%
%
%
	 \begin{proposition}
		 \label{1n}
		 For nonzero $k \in \rr$, we have
		 \begin{equation}
			 \begin{split}
				 \|\sin(k(nx-c))\|_{H^\sigma(\ci)} \simeq n^\sigma.
				 \label{1m}
			 \end{split}
		 \end{equation}
		The same result holds with $\sin$ replaced by $\cos$.
	\end{proposition}
		%
		{\bf Proof.} The Fourier transform of $\psi_n(x) = \sin[k(nx-c)]$
		is
		\begin{equation*}
			\begin{split}
				\widehat{\psi_n}(\xi)
				& = \int_0^{2\pi} e^{-ix \xi} \sin [k(nx-c)]
				\ dx
				\\
				& = \int_0^{2\pi} e^{-ix \xi} \left( \frac{e^{ik(nx-c)} -
				e^{-ik(nx-c)}}{2i} \right) \ dx
				\\
				& = \frac{1}{2i} \int_0^{2\pi} e^{i[x(kn- \xi)-kc]} \ dx
				- \frac{1}{2i}\int_0^{2\pi} e^{-i[x(kn+\xi) - kc]} \ dx.
			\end{split}
		\end{equation*}
		Therefore
		\begin{equation*}
			\begin{split}
				\widehat{\psi_n}(\xi) =
				\begin{cases}
					- i \pi e^{-ikc}, \qquad & \xi = kn\\
					i \pi e^{ikc}, \qquad & \xi = -kn\\
					0,  \qquad & \xi \neq \pm kn.
				\end{cases}
			\end{split}
		\end{equation*}
		Hence
		\begin{equation*}
			\begin{split}
				\|\psi_n\|_{H^\sigma(\ci)}^2
				& = \sum_{\xi \in \zz}
				(1+\xi^2)^{\sigma} \widehat{\psi_n}(\xi) 
				\\
				& = i \pi (1+k^2 n^2)^{\sigma} (e^{ikc} - e^{-ikc})
				\\
				& \simeq n^{2 \sigma}, \qquad n>>1
			\end{split}
		\end{equation*}
		from which we obtain \eqref{1m}. Furthermore, since
		\begin{equation*}
			\begin{split}
				\cos[k(nx-c)]
				&= \sin[k(nx-c)- \pi/2] \\
				& = \sin\{k[nx - (c + \pi/2k)]\},
			\end{split}
		\end{equation*}
		we conclude \eqref{1m} holds when $\sin$ is replaced by $\cos$.
		$\qquad \Box$
		\begin{proposition}
			\label{2n}
			For nonzero $k \in \rr$, $n >>1$,
			\begin{equation}
				\label{2m}
				\begin{split}
					\|\sin^2[k(nx-c)] \|_{H^\sigma(\ci)} \simeq
					\begin{cases}
					1, \qquad & \sigma \le 0
					\\
					n^\sigma,\qquad &\sigma > 0.
				\end{cases}
				\end{split}
			\end{equation}
			The same result holds with $\sin$ replaced by $\cos$.
		\end{proposition}
		{\bf Proof.} The Fourier transform of $\psi_n(x) = \sin^2[k(nx-c)]$
		is
		\begin{equation*}
			\begin{split}
				\widehat{\psi_n}(\xi) 
				& = \int_0^{2\pi} e^{-ix \xi} \sin^2[k(nx-c)] \ dx
				\\
				& = \int_0^{2\pi} e^{-ix \xi} \left( \frac{e^{ik(nx-c)} -
				e^{-ik(nx-c)}}{2i} \right)^2 \ dx
				\\
				& = -\frac{1}{4} \int_0^{2\pi} e^{-ix \xi} (e^{2ik(nx-c)} +
				e^{-2ik(nx-c)} -2) \ dx
				\\
				& = -\frac{1}{4} \int_0^{2\pi} e^{ix(2kn - \xi) - 2ikc} \
				dx - \frac{1}{4} \int_0^{2\pi} e^{-ix(2kn + \xi) + 2ikc} \ dx
				\\
				& + \frac{1}{2} \int_0^{2\pi} e^{-ix \xi} \ dx.
			\end{split}
		\end{equation*}
	Hence, for nonzero $k \in \rr$ and $n >>1$ we have
	\begin{equation*}
		\begin{split}
			\widehat{\psi_n}(\xi) = 
			\begin{cases}
				-\frac{\pi}{2}e^{-2ikc}, \qquad & \xi=2kn
				\\
				-\frac{\pi}{2}e^{2ikc}, \qquad & \xi = -2kn
				\\
				\pi, \qquad & \xi = 0
				\\
				0, \qquad & \xi \neq 0, \ \pm 2kn.
			\end{cases}
		\end{split}
	\end{equation*}
	which gives
	\begin{equation*}
		\begin{split}
			\|\psi_n\|_{H^\sigma(\ci)}^2 
			& = \sum_{\xi \in \zz} (1+ \xi^2)^\sigma \widehat{\psi_n}(\xi)
			\\
			& = -\frac{\pi}{2}(1+4k^2n^2)^\sigma (e^{2ikc} + e^{-2ikc}) +
			\pi
			\\
			& \simeq 
			\begin{cases}
				1,  \qquad & \sigma \le 0 
				\\
				n^{2 \sigma}, &  \sigma > 0
			\end{cases}
		\end{split}
	\end{equation*}
	from which \eqref{2m} follows. Since $\sin^2[k(nx-c)] = 1-
	\cos^2[k(nx-c)]$, \eqref{2m} holds with $\sin$ replaced by
	$\cos$. $\qquad \Box$
	%
	\vskip0.1in
	{\bf An Estimate for $\hyperref[90*]{E_1}$.}
	We apply Proposition \ref{1n} to obtain
	\begin{equation}
		\label{85}
		\begin{split}
			\|E_1\|_{H^\sigma(\ci)}
			& = 
			\left\| - \frac{\gamma}{2}n^{-2s+1}\sin\left( 2\left(
			nx - \gamma \omega t \right)\right )
			\right\|_{H^\sigma(\ci)}
			\\
			& \lesssim
			n^{-2s + \sigma + 1}
		\end{split}
	\end{equation}
	{\bf An Estimate for $\hyperref[90**]{E_2}$.}
	We will need the following:
	%
	\begin{remark}
		\label{lem:operator-norm-lemma}
		For $u \in L^2(\ci)$ and arbitrary $k$, we have
		\begin{equation}
			\begin{split}
				\|\p_x (1 -\p_x^2)^{-1}u \|_{H^{k}(\ci))} \le
				\|u\|_{H^{k-1}(\ci)}.
				\label{operator norm of pseudo-diff operator we use}
			\end{split}
		\end{equation}
	\end{remark}
	{\bf Proof:} Let $u \in L^2(\ci)$. Then
	\begin{equation*}
		\begin{split}
			\|\p_x \left( 1- \p_x^2 \right)^{-1} u \|_{H^{k}(\ci)}
			& = \sum_{\xi \in \zz}  \left[ \xi\left( 1+\xi^2 \right)^{-1} \right]^2
			\cdot \left( 1 + \xi^2 \right)^{k} \cdot |\hat{u}(\xi)|^2 
			\\
			& \le \sum_{\xi \in \zz}  \left( 1+ \xi^2 \right)^{k-1} \cdot 
			|\hat{u}(\xi)|^2 
			\\
			& \le \|u\|_{H^{k-1}(\ci)}.
			\qquad \Box
		\end{split}
	\end{equation*}
	Applying Remark \ref{lem:operator-norm-lemma}, we obtain
		\begin{equation}
		\begin{split}
			 \|E_2\|_{H^\sigma(\ci)} & = \left \|\p_x(1-\p_x^2)^{-1}
			\left[ \frac{3-\gamma}{2}
			\left( 2 \omega n^{-s -1}\cos( nx - \gamma \omega t )
			\right) \right] \right \|_{H^\sigma(\ci)}
			\\
			& \le \left |\frac{3-\gamma}{2}\right |
			\left \|2 \omega n^{-s -1}\cos( nx - \gamma \omega t )
			\right \|_{H^{\sigma -1 }(\ci)}
			\label{non-local_term_first_piece_without_constant}
		\end{split}
	\end{equation}
	which by Proposition \ref{1n} gives
		\begin{equation}
			\label{3.10}
		\begin{split}
			\|E_2\|_{H^\sigma(\ci)}
			& \lesssim n^{-s + \sigma -2}.
		\end{split}
	\end{equation}
	%
{\bf An Estimate for $\hyperref[90]{E_3}$.}
Applying Remark \ref{lem:operator-norm-lemma}, we obtain
		\begin{equation}
		\begin{split}
			\|E_3\|_{H^\sigma(\ci)} & = \left \|\p_x(1-\p_x^2)^{-1}
			\left[ \frac{3-\gamma}{2}\left( n^{-2s}\cos^2\left( nx - \gamma 
			\omega
			t\right)\right) \right] \right \|_{H^\sigma(\ci)}
			\\
			& \le \left |\frac{3-\gamma}{2}\right |
			\left \| n^{-2s}\cos^2\left( nx - \gamma \omega t \right)
			\right \|_{H^{\sigma -1 }(\ci)}
			\end{split}
	\end{equation}
which by Proposition \ref{2n} gives
		\begin{equation}
			\label{yuoo}
		\begin{split}
			\|E_3\|_{H^\sigma(\ci)}
			& \lesssim 
			\begin{cases}
				n^{-2s}, \qquad & \sigma \le 1  \\
				n^{-2s +\sigma -1}, \qquad & \sigma
				> 1.
			\end{cases}
		\end{split}
	\end{equation}
	{\bf An Estimate for $\hyperref[90]{E_4}$.}
	We now apply Remark \ref{lem:operator-norm-lemma} and Proposition 
	\ref{2n} to obtain
	\begin{equation*}
		\begin{split}
			\|E_4\|_{H^\sigma(\ci)}
			& = \Big \|\p_x \left( 1 - \p_x^2 \right)^{-1} 
			\frac{\gamma}{2}n^{-2s+2}\sin^2\left(
			nx - \gamma \omega t \right) \Big \|_{H^\sigma(\ci)}
			\\
			& \leq \left | \frac{\gamma}{2} \right |  \| 
			n^{-2s+2}\sin^2\left( nx - \gamma \omega t
			\right)\|_{H^{\sigma -1}(\ci)}
			\end{split}
	\end{equation*}
	which by Proposition \ref{2n} gives
	\begin{equation}
			\label{estimate_for_second_piece_of_non_local_final}
		\begin{split}
			\|E_4\|_{H^\sigma(\ci)} \lesssim 
			\begin{cases}
				n^{-2s+2}, \qquad & \sigma \le 1
				\\
				n^{-2s+ \sigma + 1}, \qquad & \sigma >1.
			\end{cases}
		\end{split}
	\end{equation}
	Grouping estimates \eqref{85}, \eqref{3.10}, \eqref{yuoo}, and
	\eqref{estimate_for_second_piece_of_non_local_final} we see that for bounded $\omega$ and $n >> 1$
	we have the following upper bound for the $H^\sigma(\ci)$ error
	of our approximate solutions:
	\begin{equation*}
		\begin{split}
			\|E(t)\|_{H^{\sigma}(\ci)} \lesssim 
			  \begin{cases}
				  n^{-s-1} + n^{-2s +2}, \qquad & \sigma \le 1
				  \\
				  n^{-s + \sigma - 2} + n^{-2s + \sigma + 1}, \qquad &
				  \sigma > 1.
			  \end{cases}
		\end{split}
	\end{equation*}
		Noting that $\|E(t)\|_{H^\sigma(\ci)}$ blows up as $\sigma$ 
		increases,
	regardless of the choice of $s$, we restrict our attention to the case $\sigma \le 1$
	and obtain the following:
	  \begin{lemma}
		  \label{lem:error_of_approx_solution}
		  Let $u^{\omega,n}$ be an approximate solution to the HR i.v.p., 
		  with $\sigma \le 1$,  $\omega$ bounded, and $n >> 1$.
		  Then for the error $E$ we have
		  \begin{equation}
			  \begin{split}
				  \|E(t)\|_{H^\sigma(\ci)} \lesssim n^{-r_s}
				  \label{total-error-approx-solution}
			  \end{split}
		  \end{equation}
		  where
		  \begin{equation}
			  \begin{split}
			r_s = 
			\begin{cases}
				2(s-1)   & \text{if} \quad s \le 3,\\  
				s+1  & \text{if} \quad s > 3. \\
			\end{cases}
			\label{r_s-definition}
			  \end{split}
		  \end{equation}
	  \end{lemma}
	 % 
	 
	 
	 
	 %%%%%%%%%%%%%%%%%%%%%%%%%%%%%%%%%
	 %
	 %
	 %
	 %   Proof of  Theorem in periodic case for s between 3/2 and 2
	 %
	 %
	 %
	 %%%%%%%%%%%%%%%%%%%%%%%%%%%%%%%%%%%
	 
	 
	 
	 
	 
	 
	 {\bf A Critical Lemma.}
	We wish now to estimate the difference between approximate and actual solutions to
	the HR i.v.p with common initial data $u_0 \in H^s$. To do so, we must first
	establish the existence and lifespan of solutions $u(x,t)$ with initial data $u_0$.
	We have the following:
\begin{theorem}
	\label{thm:HR_existence_continous_dependence}
For  $s>3/2$  the following  results  hold:
%
\vskip0.05in
\noindent
(i) If $u_0\in H^s(\ci)$  then  there exists a unique solution to
the Cauchy problem  \eqref{hr}--\eqref{hr-data} 
  in $C([-T, T]; H^s(\ci))$, where the life-span  $T$ depends on the size
  of the initial data $u_0$, that is
  $T=T(\|u_0 \|_{H^s(\ci)})$.
  
  \noindent
(ii)
 The flow  map $u_0 \to u(t)$  is continuous from
 bounded sets of $H^s(\ci)$ into $C([-T, T]; H^s(\ci))$.
%
 \noindent
\\
(iii)  The  lifespan $T$ satisfies the lower bound estimate 
%
     \begin{equation}
   \label{Life-span-est}
T
\ge
\frac{1}{2c_s}
\frac{1}{\|
u_0
  \|_{H^s(\ci)}},
   \end{equation}
   %
and the solution $u$ satisfies the estimate
%
     \begin{equation}
   \label{u_x-Linfty-Hs}
\|
u(t)
  \|_ {H^s(\ci))}
  \le
  2
  \|
u_0
  \|_{H^s(\ci)},\,\, |t|\le T.
   \end{equation}
   %
 \end{theorem}
%
A proof of these results is provided in the appendix.
%
%
%
Let $v=u^{\omega,n} -
u_{\omega,n}$, where $u_{\omega,n}$ denotes a solution to
the Cauchy-problem \eqref{hyperelastic-rod-equation}-\eqref{init-cond} with
initial data $u_0(x) = u^{\omega,n}(x,0)$. We are now prepared to prove the following critical lemma:

\begin{lemma}
	\label{lem:bound_for_difference-of-approx-actual-soln}
	If \ $s > 3/2 $ and $\sigma = 1/2 + \ee$ for an appropriately
	chosen $\ee = \ee(s) > 0$, then 
			\begin{equation} 
				\|
				v(t)
				\|_{H^\sigma(\ci)}
				\doteq
				\label{differ-Hsigma-est} 
				\|
				u^{\omega, n}(t) 
				- 
				u_{\omega, n}(t)
				\|_{H^\sigma(\ci)}
				\lesssim 
				n^{-r_s}, 
				\quad
				|t| \le T.
			\end{equation}
			\end{lemma}
{\bf{Proof.}} Recall the HR Cauchy problem
\begin{align}
	\label{1.1}
	&\p_t u  = -\gamma u \p_x u - \p_x\left( 1-\p_x^2
		\right)^{-1}\left[ \frac{3-\gamma}{2}
		u^2 + \frac{\gamma}{2}\left( \p_x
		u
		\right)^2 \right],
		\\
		\label{1.2}
		& u(x,0) = u_0.
\end{align}
and its approximate solutions of form
	\begin{equation}
		\label{approx-solns-form}
		u^{\omega,n}(x,t) = \omega n^{-1} + n^{-s} \cos \left( nx - \gamma \omega t
		\right) 
	\end{equation}
	with integer valued $n >>1$, bounded $\omega \in \rr$, and a fixed 
	constant $\gamma \in \rr$.	Then the approximate solution 
	$u^{\omega,n}$ satisfies the equation
\begin{equation}
	\begin{split}
		\label{1.4}
		\p_t u^{\omega,n} = E - \gamma u^{\omega,n} \p_x u^{\omega,n} 
		- \p_x\left( 1-\p_x^2 \right)^{-1} \left[
		\frac{3-\gamma}{2} \left( u^{\omega,n} \right)^2 +
		\frac{\gamma}{2}\left( \p_x u^{\omega,n} \right)^2 \right].
	\end{split}
\end{equation}
Let $u_{\omega,n}$ denote the solution to the i.v.p
\begin{align}
	\label{1.5}
	&\p_t u_{\omega,n}  = -\gamma u_{\omega,n} \p_x u_{\omega,n} - 
	\p_x\left( 1-\p_x^2
		\right)^{-1}\left[ \frac{3-\gamma}{2}\left(
		u_{\omega,n} \right)^2 + \frac{\gamma}{2}\left( \p_x
		u_{\omega,n}
		\right)^2 \right],
		\\
		& u_{\omega,n}(x,0) = u^{\omega,n}(x,0).
\end{align}
Subtracting \eqref{1.5} from \eqref{1.4}, we see that the
difference $v = u^{\omega,n} - u_{\omega,n}$ satisfies the i.v.p
\begin{align}
		\label{1.7}
		& \p_t v  =  E - \frac{\gamma}{2} \p_x
		\left[ \left( u^{\omega,n} + u_{\omega,n} \right)v \right]
		 \notag
		\\
		& - \p_x(1-\p_x^2)^{-1} \left[
		\frac{3-\gamma}{2} \left( u^{\omega,n} + u_{\omega,n}
		\right) v +
		\frac{\gamma}{2}\left( \p_x u^{\omega,n} +
		\p_x u_{\omega,n}
		\right) \p_x v
		\right], 
		\\
		& v(x,0)=0.
\end{align}
Applying $D^\sigma$ to both sides of \eqref{1.7}, multiplying by
$D^\sigma v$, and integrating, we obtain the
relation
\begin{equation}
	\begin{split}
		\frac{1}{2}\frac{d}{dt}\|v(t)\|_{H^\sigma(\ci)}^2
		& = \int_{\ci} D^\sigma E \cdot D^\sigma
		v \ dx
		\\
		& - \frac{\gamma}{2}\int_{\ci} D^\sigma
		\p_x \left[ \left( u^{\omega,n} + u_{\omega,n} \right)v
		\right]\cdot D^\sigma v \ dx
		\\
		& - \frac{3-\gamma}{2}\int_{\ci} D^{\sigma
		-2} \p_x \left[ \left( u^{\omega,n} + u_{\omega,n}
		\right)v \right] \cdot D^\sigma v \ dx
		\\
		& - \frac{\gamma}{2}\int_{\ci} D^{\sigma
		-2}
		\p_x \left[ \left( \p_x u^{\omega,n} + \p_x u_{\omega,n}
		\right)\cdot \p_x v \right] \cdot
		D^\sigma v \ dx.
		\label{X}
	\end{split}
\end{equation}
We now estimate each term of the right hand side
of \eqref{X}.
\vskip0.1in
{\bf Estimate for Term 1.} Applying Cauchy-Schwartz, we obtain
\begin{equation}
	\begin{split}
	\left |\int_{\ci} D^\sigma E \cdot D^\sigma v \ dx \right |
		& \le \|D^\sigma E \cdot D^\sigma v \|_{L^1(\ci)}
		\\
		& \le \|E\|_{H^\sigma(\ci)} \|v\|_{H^\sigma(\ci)}.
		\label{est_for_1}
	\end{split}
\end{equation}
%
{\bf Estimate for Term 2.} We can rewrite
\begin{equation}
	\begin{split}
		-\frac{\gamma}{2} \int_{\ci} D^\sigma \p_x \left[ \left( u^{\omega,n} + u_{\omega,n}
		\right)v \right] \cdot D^\sigma v \ dx
		 = & -\frac{\gamma}{2}\int_{\ci} \left[ D^\sigma \p_x , u^{\omega,n} + u_{\omega,n}
		\right]v \cdot D^\sigma v \ dx
		\\
		& - \frac{\gamma}{2} \int_{\ci} (u^{\omega,n} + u_{\omega,n})
		D^\sigma \p_x v \cdot
		D^\sigma v \ dx.
		\label{est_for_2}
	\end{split}
\end{equation}
We now estimate \eqref{est_for_2} in parts. For the first term, we have
\begin{equation}
	\begin{split}
		\left | \frac{\gamma}{2} \int_{\ci} (u^{\omega,n} + u_{\omega,n})
		D^\sigma \p_x v \cdot
		D^\sigma v \ dx \right |
		& = \bigg | \frac{\gamma}{4}\int_{\ci} (u^{\omega,n} +
		u_{\omega,n}) \cdot \p_x (D^\sigma v)^2 \ dx \bigg |
		\\
		& = \left | -\frac{\gamma}{4} \int_{\ci} \p_x(u^{\omega,n} + u_{\omega,n}) \cdot
		(D^\sigma v)^2  \ dx \right |
		\\
		& \lesssim \|\p_x(u^{\omega,n} + u_{\omega,n}) \|_{L^\infty(\ci)}
		\|v\|_{H^\sigma(\ci)}^2.
		\label{2'}
	\end{split}
\end{equation}
To deal with the remaining term, we will need the following commutator
estimate:
\begin{theorem}
	\label{thm10}
	Assume $1<p<\infty$, $m \ge 0$. Consider a pseudo-differential operator $P
	\in \Psi^m$; then for $\rho >n/p + 1$, $s \ge 0$, and $s+m \le \rho$ we
	have
	\begin{equation}
		\begin{split}
			\|[P,f]v\|_{H^{s,p}} \le C \|f\|_{H^{\rho,p}}
			\|v\|_{H^{s+m-1,p}}.
			\label{5}
		\end{split}
	\end{equation}
\end{theorem}
%
A proof of \eqref{5} can be found in \cite{t2}. We also have the following
corollary:
\begin{corollary}
	\label{cor1}
If $\rho > 3/2$ and $0 \le \sigma + 1 \le \rho$, then
\begin{equation}
	\begin{split}
		\|[D^\sigma \p_x ,f]v\|_{L^2(\ci)} \le C \|f\|_{H^\rho} \|v\|_{H^\sigma}.
		\label{15}
	\end{split}
\end{equation}
\end{corollary}
{\bf Proof.} It follows from Theorem \ref{thm10} by setting $n=1$, $p=2$,
$s=0$, and noting  that $D^\sigma \p_x \in
\Psi^{\sigma + 1}$. $\qquad \Box$
\vskip0.1in
Let $\sigma = 1/2 + \ee$ and $\rho = 3/2 + \ee$, where $\ee > 0$ is
arbitrarily small. Then
applying Corollary \ref{cor1}, we obtain
\begin{equation}
	\begin{split}
		\|[D^\sigma \p_x, u^{\omega,n} + u_{\omega,n}]v\|_{L^2(\ci)} \le C \|u^{\omega,n} + u_{\omega,n}
		\|_{H^{\rho}(\ci)} \|v\|_{H^\sigma(\ci)}.
		\label{6}
	\end{split}
\end{equation}
Applying Cauchy-Schwartz and estimate \eqref{6} gives
\begin{equation}
	\begin{split}
		\left | -\frac{\gamma}{2} \int_{\ci} [D^\sigma \p_x , u^{\omega,n} + u_{\omega,n}]v
		\cdot D^\sigma v \ dx \right | \lesssim \|u^{\omega,n} +
		u_{\omega,n}\|_{H^{\rho}(\ci)} \|v\|_{H^\sigma(\ci)}^2.
		\label{7}
	\end{split}
\end{equation}
Combining estimates \eqref{2'} and \eqref{7} we conclude that
\begin{equation}
	\begin{split}
		& \left | -\frac{\gamma}{2} \int_{\ci} D^\sigma \p_x \left[ \left( u^{\omega,n} + u_{\omega,n}
		\right)v \right]  \cdot D^\sigma v \ dx \right |
		\\
		& \lesssim (\|u^{\omega,n} + u_{\omega,n}\|_{H^{\rho}(\ci)} 
		 + \|\p_x u^{\omega,n} +
		\p_x u_{\omega,n}\|_{L^\infty(\rr)} ) \cdot \|v\|_{H^\sigma(\ci)}^2.
		\label{8}
	\end{split}
\end{equation}
%
\vskip0.1in
{\bf Estimate for Term 3.} Using Cauchy-Schwartz, and recalling that
$\sigma = 1/2 + \ee$,  we obtain
\begin{equation}
	\begin{split}
		\bigg | -\frac{3-\gamma}{2} \int_{\ci} D^{\sigma -2} \p_x \left[
		(u^{\omega,n} + u_{\omega,n})v \right]
		& \cdot D^\sigma v \ dx \bigg |
		\\
		& \lesssim
		\|D^{\sigma -2 } \p_x [(u^{\omega,n} + u_{\omega,n})v] \cdot D^\sigma v
		\|_{L^1(\ci)}
		\\
		& \lesssim \|D^{\sigma -2 } \p_x [(u^{\omega,n} +
		u_{\omega,n})v\|_{L^2(\ci)} \cdot \|D^\sigma v \|_{L^2(\ci)}
		\\
		& \lesssim \|(u^{\omega,n} + u_{\omega,n})v \|_{H^{\sigma -1 }(\ci)}
		\|v\|_{H^\sigma(\ci)}
		\\
		& \lesssim \|(u^{\omega,n} + u_{\omega,n})v \|_{L^2(\ci)} \|v\|_{H^\sigma(\ci)}
		\\
		& \lesssim \|u^{\omega,n} + u_{\omega,n} \|_{L^\infty(\ci)} \|v\|_{H^\sigma(\ci)}^2.
		\label{9}
	\end{split}
\end{equation}
%
{\bf Estimate for Term 4.}
We will need the following:
\begin{lemma}
	\label{impo}
	For $1/2 < \sigma < 1 $, $f,g \in \mathcal{S'}$,
	\begin{equation}
		\begin{split}
			\|fg\|_{H^{\sigma - 1}} \le C \|f\|_{H^{\sigma}}
			\cdot \|g\|_{H^{\sigma -1}}.
			\label{11}
		\end{split}
	\end{equation}
\end{lemma}
%
	Hence, applying Cauchy-Schwartz and Lemma \ref{impo}, we obtain
	\begin{equation}
		\begin{split}
			& \left | -\frac{\gamma}{2} \int_{\ci} D^{\sigma -2 } \p_x \left[
			\left( \p_x u^{\omega,n} + \p_x u_{\omega,n} \right) \cdot \p_x v
			\right] \cdot D^\sigma v \ dx \right |
			\\
			& \lesssim \|D^{\sigma -2} \p_x [(\p_x u^{\omega,n} + \p_x
			u_{\omega,n}) \cdot \p_x v]\|_{L^2(\ci)} \cdot \|D^\sigma v
			\|_{L^2(\ci)}
			\\
			& \lesssim \|(\p_x u^{\omega,n} + \p_x u_{\omega,n}) \cdot \p_x v
			\|_{H^{\sigma -1}(\ci)} \| \cdot \|v\|_{H^\sigma (\ci)} 
			\\
			& \lesssim \|\p_x u^{\omega,n} + 
			\p_x u_{\omega,n}
			\|_{H^\sigma(\ci)} \|v\|_{H^\sigma(\ci)}^2.
			\label{12}
		\end{split}
	\end{equation}
Collecting estimates \eqref{est_for_1}, \eqref{8}, \eqref{9}, and
\eqref{12}, and applying the Sobolev Imbedding Theorem, we deduce
\begin{equation}
	\begin{split}
		\frac{1}{2}\frac{d}{dt} \|v\|_{H^\sigma(\ci)}^2
		& \lesssim
		(\|u^{\omega,n} + u_{\omega,n}\|_{H^{\rho}(\ci)} +
		\|\p_x(u^{\omega,n} + u_{\omega,n}) \|_{H^\sigma(\ci)})
		\cdot \|v\|_{H^\sigma(\ci)}^2
		\\
		& + \|E\|_{H^\sigma(\ci)}
		\|v\|_{H^\sigma(\ci)}.
		\label{10}
	\end{split}
\end{equation}
We now estimate the right hand side of \eqref{10} in parts. Applying
Proposition \ref{1n} and Theorem \ref{thm:HR_existence_continous_dependence} 
gives
\begin{equation}
	\begin{split}
		 \|u^{\omega,n} + u_{\omega,n} \|_{H^\rho(\ci)}
		 & \le
		\|u^{\omega,n}\|_{H^\rho(\ci)} + \|u_{\omega,n}\|_{H^\rho(\ci)}
		\\
		& \lesssim n^{\rho -s} + \|u^{\omega,n}(0)\|_{H^\rho(\ci)}
		\\
		& \lesssim n^{\rho -s}
		\label{3r}
	\end{split}
\end{equation}
and
\begin{equation}
	\begin{split}
		\|\p_x(u^{\omega,n} + u_{\omega,n}) \|_{H^\sigma(\ci)} 
		& \le \|u^{\omega,n} + u_{\omega,n}\|_{H^{\sigma + 1}(\ci)}
		\\
		& = \|u^{\omega,n} + u_{\omega,n}\|_{H^{\rho}(\ci)}
		\\
		& \le \|u^{\omega,n} \|_{H^{\rho}(\ci)} + \|u_{\omega,n}
		\|_{H^{\rho}(\ci)}
		\\
		& \lesssim n^{\rho -s} + \|u^{\omega,n}(0)\|_{H^{\rho}(\ci)}
		\\
		& \lesssim n^{\rho -s}.
		\label{4r}
	\end{split}
\end{equation}
Applying Lemma \ref{lem:error_of_approx_solution}, and
substituting \eqref{total-error-approx-solution}, \eqref{3r},
and \eqref{4r} into \eqref{10}, we obtain
\begin{equation}
	\begin{split}
		\frac{1}{2}\frac{d}{dt}\|v\|_{H^\sigma(\ci)}^2 \lesssim n^{\rho - s}
		\|v\|_{H^\sigma(\ci)}^2 + n^{-r_s}\|v\|_{H^\sigma(\ci)}.
		\label{200r}
	\end{split}
\end{equation}
Setting $y(t) = \|v(t)\|_{H^\sigma(\ci)}$  in \eqref{200r}, derivating the
left-hand-side, and dividing through by $y(t)$ yields
\begin{equation}
	\label{sub8}
	\frac{d}{dt} y(t) \le cn^{\rho - s}y + cn^{-r_s}, \quad c\in \rr^{+}.
\end{equation}
We multiply both sides of \eqref{sub8} by the integrating factor $e^{-tcn^{\rho - s}}$,
which gives
\begin{equation*}
	e^{-tcn^{\rho - s}}\frac{d}{dt} y(t) \le cn^{\rho - s} e^{-tcn^{\rho - s}}y
	+ cn^{-r_s}e^{-tcn^{\rho - s}}.
\end{equation*}
It follows that
\begin{equation*}
	\frac{d}{dt}\left (e^{-tcn^{\rho - s}} y \right ) \le cn^{-r_s}e^{-tcn^{\rho - s}} .
\end{equation*}
Hence,
\begin{equation*}
	\begin{split}
	\int_0^t  \frac{d}{d \tau} \left[ e^{-\tau cn^{\rho - s}} y(\tau)
	\right] \; d \tau
	& \le \int_0^t  c n^{-r_s} e^{-\tau cn^{\rho - s}}  \; d \tau
	\\
	& \le \int_0^t cn^{-r_s} \; d \tau
\end{split}
\end{equation*}
from which we obtain
\begin{equation}
	\label{almost8}
	e^{-tcn^{\rho - s}} y(t) - y(0) \le ctn^{-r_s}.
\end{equation}
Noting that $y(0)=0$, we can simplify \eqref{almost8} to obtain
\begin{equation}
	\label{gronwall-ineq8}
	y(t) \le ctn^{-r_s} e^{tcn^{\rho - s}}.
\end{equation}
Substituting back in $\|v(t)\|_{H^\sigma(\ci)}$ for $y$, we see that we are assured the
decay of $\|v(t)\|_{H^\sigma(\ci)}$ only when $s \ge \rho$. Recall that
we previously set $\sigma = 1/2 + \ee$, $\rho = 3/2 + \ee$, where $\ee$ was
chosen to be arbitrarily small. Hence, we conclude that for $s>3/2$
\begin{equation}
	\label{en-est-fin!8}
	\|v(t)\|_{H^\sigma(\ci)} 
	\lesssim
	n^{-r_s}, \quad |t| \le T,\quad n>>1 . 
\end{equation}
This concludes the proof of Lemma
\ref{lem:bound_for_difference-of-approx-actual-soln}. $\qquad \Box$

%
%
%
%
{\bf Non-Uniform Dependence for $3/2<s<2$.}
%
%
%
Let $u_{\pm 1, n}$ be solutions to the HR i.v.p. with common initial data $u^{\pm 1,
n}(0)$, respectively.
We wish to show that the $H^s$ norm of the difference of $u_{\pm 1,
n}$ and the associated approximate solution $u^{\pm 1, n}$
decays as $n \to \infty$. Due to Lemma
\ref{lem:bound_for_difference-of-approx-actual-soln} we assume
$s > 3/2 $ and $\sigma = 1/2 + \ee$ for an appropriately
chosen $\ee= \ee(s) > 0$. Then by Proposition \ref{1n} and
Theorem \ref{thm:HR_existence_continous_dependence}
we obtain
\begin{equation}
	\begin{split}
		\|u_{\pm 1, n} (t) \|_{H^{2s - \sigma}(\ci)}
		& \le 2 \|u^{\pm 1, n}(0) \|_{H^{2s - \sigma}(\ci)}
		\\
		& \lesssim n^{s- \sigma}.
			\label{final-est-Hk-norm-sol}
	\end{split}
\end{equation}
Furthermore, by Proposition \ref{1n}, we have
\begin{equation}
	\begin{split}
		\|u^{\pm 1, n} (t) \|_{H^{2s - \sigma} (\ci)}
		& = \|\pm n^{-1} + n^{-s} \cos(nx \mp \gamma \omega t) \|_{H^{2s - \sigma}(\ci)}
		\\
		& \le \| \pm n^{-1} \|_{H^{2s - \sigma}(\ci)} +
		\|n^{-s} \cos(nx \mp \gamma \omega
		t) \|_{H^{2s - \sigma}(\ci)}
		\\
		& \lesssim n^{-1} + n^{s-\sigma}
		\\
		& \lesssim n^{s-\sigma}.
		\label{4}
	\end{split}
\end{equation}
		Therefore, \eqref{final-est-Hk-norm-sol}, \eqref{4}, and the triangle
		inequality yield
		\begin{equation}
			\begin{split}
				\|u^{\pm 1, n} (t) - u_{\pm 1, n}(t)\|_{H^{2s - \sigma}(\ci)}
				\lesssim n^{s-\sigma}.
				\label{5h}
			\end{split}
		\end{equation}
		Recalling
		Lemma \ref{lem:bound_for_difference-of-approx-actual-soln}, we also
		have
		\begin{equation}
			\begin{split}
				\|u^{\pm 1, n}(t) - u_{\pm 1, n} (t) \|_{H^\sigma (\ci)} 
				\lesssim n^{-r_s}
				\label{6h}.
			\end{split}
		\end{equation}
		We now wish to interpolate in order to obtain an estimate of the $H^s (\ci)$
		norm of the difference of the approximate and actual solutions:
		\begin{lemma}
			\label{7h}
			For all $\psi \in L^2(\ci)$,
			\begin{equation*}
				\|\psi \|_{H^s (\ci)} \leq \left( \| \psi \|_{H^k (\ci)} \| \psi
				\|_{H^{2s - k}} \right)^{\frac{1}{2}}.
			\end{equation*}
		\end{lemma}
		{\bf{Proof:}}
			\begin{equation*}
				\begin{split}
					\|\psi\|^2_{H^s (\ci)} & = \sum_{\xi \in \zz} (1 + 
					\xi^2)^s |\hat{\psi}(\xi) |^2
				\\
				& = \sum_{\xi \in \zz} (1 + \xi^2)^{s - 
				\frac{k}{2}}|\hat{\psi}(\xi)| \cdot
				( 1 + \xi^2)^{\frac{k}{2}}|\hat{\psi}(\xi)|
				\\
				& \le \|\psi\|_{H^k(\ci)}
				\|\psi\|_{H^{2s - k}(\ci)}
			\end{split}
			\end{equation*}
			from which we obtain 
		\begin{equation*}
				\|\psi \|_{H^s (\ci)} \leq \left( \| \psi \|_{H^k (\ci)} \| \psi
				\|_{H^{2s - k}} \right)^{\frac{1}{2}}. \qquad \Box
			\end{equation*}
			Applying Lemma \ref{7h}, and estimates \eqref{5h} and \eqref{6h}
			yields
			\begin{equation}
				\label{comp-200}
				\begin{split}
					\|u^{\pm 1,n}(t) - u_{\pm 1, n}(t) \|_{H^s (\ci)}
					& \le ( \| u^{\pm 1,n}(t)
					- u_{\pm 1, n}(t) \|_{H^\sigma (\ci)}
					\\
					& \cdot \| u^{\pm 1,n}(t)
					- u_{\pm 1, n}(t)\|_{H^{2s - \sigma}(\ci)} )^{\frac{1}{2}}
					\\
					& \lesssim (n^{-r_s} \cdot n^{s-\sigma})^{\frac{1}{2}}.
				\end{split}
			\end{equation}
			Recalling \eqref{r_s-definition}, we see that for $s \le 3$,
			\eqref{comp-200} reduces to 
			\begin{equation}
				\begin{split}
					\|u^{\pm 1,n}(t) - u_{\pm 1, n}(t) \|_{H^s (\ci)}
					& \lesssim (n^{-2(s-1)} \cdot n^{s-\sigma})^{\frac{1}{2}}
					\\
					& \lesssim n^{(2-s - \sigma)/2}
					\label{8h}
				\end{split}
			\end{equation}
			and for $s > 3$ reduces to 
			\begin{equation}
				\begin{split}
					\|u^{\pm 1,n}(t) - u_{\pm 1, n}(t) \|_{H^s (\ci)}
					& \lesssim \left( n^{-\left( s+1 \right)} \cdot
					n^{s-\sigma}
					\right)^{\frac{1}{2}}
					\\
					& \lesssim n^{(-1-\sigma)/2}.
					\label{9h}
				\end{split}
			\end{equation}
			Since $s > 3/2 $ by assumption, we now recall \eqref{8h},
			\eqref{9h}, and the
			relation $\sigma = 1/2 + \ee(s)$ to conclude 
			\begin{equation}
				\begin{split}
					\|u^{\pm 1,n}(t) - u_{\pm 1, n}(t) \|_{H^s (\ci)} \lesssim
					n^{-\ee(s)/2}.
					\label{10h}
				\end{split}
			\end{equation}
			
%
%
%%%%%%%%%%%%%% Behavior at time  t = 0  %%%%%%%%%%%% 
%  
%
\vskip0.1in
{\bf Behavior at time $t=0$.}  We have
%
%
\begin{equation} 
	\label{HR-slns-differ-t-0} 
	\|
	u_{1, n}(0)
	-
	u_{-1, n}(0)
	\|_{H^s(\ci)}
	=
	\|
	2   n^{-1}
	\|_{H^s(\ci)}
	\simeq
	n^{-1}
	\longrightarrow 
	0
	\,\,
	\text{as}
	\,\,
	n \to \infty.
\end{equation}
%
%
%%%%%%%%%%%%%% Behavior at time  t >0  %%%%%%%%%%%% 
%  
%
\vskip0.1in
{\bf Behavior at time  $t>0$.}  Using the reverse triangle inequality, we 
have
%
%
%
\begin{equation} 
	\label{HR-slns-differ-t-pos}
	\begin{split}
		\|
		u_{1, n}(t)
		-
		u_{- 1, n}(t)
		\|_{H^s(\ci)}
		&
		\ge
		\|
		u^{1, n}(t)
		-
		u^{- 1, n}(t)
		\|_{H^s(\ci)}
		\\
		&
		-
		\|
		u^{1, n}(t)
		-
		u_{1, n}(t)
		\|_{H^s(\ci)}
		\\
		&
		-
		\|
		-u^{-1, n}(t)
		+
		u_{-1, n}(t)
		\|_{H^s(\ci)} .
	\end{split}
\end{equation}
%
%
Using estimate \eqref{10h} for the last two terms 
in \eqref{HR-slns-differ-t-pos} we obtain
%
%
%
\begin{equation} 
	\label{HR-slns-differ-t-pos-est}
	\|
	u_{1, n}(t)
	-
	u_{- 1, n}(t)
	\|_{H^s(\ci)}
	\ge
	\|
	u^{1, n}(t)
	-
	u^{- 1, n}(t)
	\|_{H^s(\ci)}
	-
	c n^{- \ee(s)/2}
\end{equation}
where $c \in \rr^+$. Letting $n$ go to $\infty$ in
\eqref{HR-slns-differ-t-pos-est}
yields
%
\begin{equation} 
	\label{HR-slns-to-ap-est}
	\liminf_{n\to\infty}
	\|
	u_{1, n}(t)
	-
	u_{- 1, n}(t)
	\|_{H^s(\ci)}
	\ge
	\liminf_{n\to\infty}
	\|
	u^{1, n}(t)
	-
	u^{- 1, n}(t)
	\|_{H^s(\ci)}.
\end{equation}
%
%
Hence, by \eqref{HR-slns-to-ap-est}, we see that in order to find a lower bound for
the difference of the unknown solution sequences it is
sufficient to find a lower bound for the difference of the
associated approximate solutions. We set out to do so; using the identity 
$$
\cos \alpha -\cos \beta
=
-2
\sin(\frac{\alpha + \beta}{2})
\sin(\frac{\alpha - \beta}{2})
$$
we obtain
$$
u^{1, n}(t)
-
u^{- 1, n}(t)
=
2
n^{-1}
+
2
n^{-  s}
\sin (n x) \sin (\gamma t).
$$
Therefore
%
%
\begin{equation} 
	\label{B--ap-below-est-1}
	\|
	u^{1, n}(t)
	-
	u^{- 1, n}(t)
	\|_{H^s(\ci)}
	\ge
	2
	n^{  -  s}
	\|
	\sin (n x)
	\|_{H^s(\ci)}
	|\sin \gamma t|.
\end{equation}
%
%
Letting $n$ go to $\infty$, Proposition \ref{1n}
and \eqref{B--ap-below-est-1} imply
%
%
\begin{equation} 
	\label{HR-ap-below-est}
	\liminf_{n\to\infty}
	\|
	u^{1, n}(t)
	-
	u^{- 1, n}(t)
	\|_{H^s(\ci)}
	\gtrsim
	|\sin \gamma t|.
\end{equation}
%
%
Combining  \eqref{HR-slns-to-ap-est} and  \eqref{HR-ap-below-est}
yields
%
%
\begin{equation} 
	\label{HR-slns-below-est-fin}
	\liminf_{n\to\infty}
	\|
	u_{1, n}(t)
	-
	u_{- 1, n}(t)
	\|_{H^s(\ci)}
	\gtrsim
	|\sin \gamma t|
\end{equation}
%
%
proving \eqref{bdd-away-from-0}. Furthermore, by Proposition \ref{1n} and
Theorem \ref{thm:HR_existence_continous_dependence}  we obtain
\begin{equation}
	\begin{split}
		\label{solutions-are-small}
		\|u_{\pm 1, n} (t) \|_{H^{s}(\ci)}
		& \lesssim \|u^{\pm 1, n}(0) \|_{H^{s}(\ci)}
		\\
		& \lesssim 1
	\end{split}
\end{equation}
Collecting  \eqref{HR-slns-differ-t-0}, \eqref{HR-slns-below-est-fin}, and
\eqref{solutions-are-small}, we conclude that we have proven Theorem
\ref{hr-non-unif-dependence} for the periodic case.
%
%
%
%
%
%	




	 
	 
	 
	 %%%%%%%%%%%%%%%%%%%%%%%%%%%%%%%%%
	 %
	 %
	 %
	 %   Proof of  Theorem in periodic case for s greater or equal to 2
	 %
	 %
	 %
	 %%%%%%%%%%%%%%%%%%%%%%%%%%%%%%%%%%%
	 
	 
	 
	 
	 
	 
	\section{
A Proof of Theorem \ref{hr-non-unif-dependence} In the Periodic Case For
$s\ge 2$ }
	\setcounter{equation}{0}
	
	In this situation we use the following initial value problem
	satisfied  by the difference
	$
	v=
		u^{\omega, n}(t) 
				- 
				u_{\omega, n}(t)
				$:
				%
				%
				%
	\begin{align}
	\label{v-eq}
					\p_t v 
					& = E + \gamma(v \p_x v - v \p_x u^{\omega,n} - u^{\omega,n} \p_x v) 
					\\
					& + \p_x\left( 1 - \p_x^2 \right)^{-1} \left[ \frac{3-
					\gamma}{2}v^2 + \frac{\gamma}{2}\left( \p_x v \right)^2 - \left(
					3 - \gamma \right)u^{\omega,n} v -
					\gamma \p_x u^{\omega,n} \p_x v \right],
					\notag
					\\
					v(0) & =0
					\label{v-data}
			\end{align}
							%
	%
	%
and prove the following key result:
%
\begin{lemma}
	\label{lem:bound_for_difference-of-approx-and-actual-soln}
If $s \ge 2$ then
			\begin{equation} 
				\|
				v(t)
				\|_{H^1(\ci)}
				=
				\label{differ-H1-est} 
				\|
				u^{\omega, n}(t) 
				- 
				u_{\omega, n}(t)
				\|_{H^1(\ci)}
				\lesssim 
				n^{-r_s}, 
				\quad
				|t| \le T.
			\end{equation}
			%
			\end{lemma}
%
			{\bf{Proof.}} 
Straightforward computations give		
\begin{equation}
	\label{energy-estimate-simplified}
	\begin{split}
		\frac{d}{dt} \|v(t)\|_{H^1(\ci)}^2
		& = 2 \int_{\ci}
		 v\left( 1-\p_x^2
	\right)E \; dx
	\\
	&+ 2 \gamma  \int_{\ci}  v\left( 1-\p_x^2 \right)\left( v \p_x u^{\omega,n}
	- u^{\omega,n} \p_x v
	\right) \; dx
	\\
	& - 2 \int_{\ci} \left[ \left( 3-\gamma \right)v \p_x \left( u^{\omega,n}v \right) + \gamma v
	\p_x \left( \p_x u^{\omega,n} \p_x v \right)\right] \; dx.
\end{split}
\end{equation}
Applying Cauchy-Schwartz and the inequality $ab + cd \le (a^2 +
c^2)^{\frac{1}{2}}(b^2 + d^2)^{\frac{1}{2}}$ for $a,b,c,d \in \rr$, we
obtain
\begin{equation}
	\begin{split}
		\label{energy-estimate-best}
		\frac{d}{dt} \|v(t)\|_{H^1(\ci)}^2
		& \lesssim \left( \|u^{\omega,n}\|_{L^\infty(\ci)} + \|
		\p_x u^{\omega,n} \|_{L^\infty(\ci)} + \|\p_x^2 u^{\omega,n} \|_{L^\infty (\ci)} \right)
		\|v\|_{H^1(\ci)}^2 
		\\
		&+ \|v\|_{H^1(\ci)} \|E\|_{H^1(\ci)}.
	\end{split}
\end{equation}
Since we have
\begin{equation}
	\label{L-infty-error}
	\begin{split}
		\|u^{\omega,n} \|_{L^\infty(\ci)} + \|\p_x u^{\omega,n} \|_{L^\infty(\ci)}
		+ \|\p_x^2 u^{\omega,n} \|_{L^\infty(\ci)}
		& \lesssim 
		n^{-s} + n^{-s+1} + n^{-s + 2} \\
		& \lesssim n^{-s + 2}  ,
	\end{split}
\end{equation}
substituting \eqref{L-infty-error} and \eqref{total-error-approx-solution} into
\eqref{energy-estimate-best} gives
\begin{equation}
	\label{en-est-fin!}
	\frac{d}{dt} \|v(t)\|_{H^1(\ci)}^2 \lesssim n^{-s+2} \|v\|_{H^1(\ci)}^2 + n^{-r_s}
	\|v \|_{H^1(\ci)}
\end{equation}
where $r_s$ is defined in \eqref{r_s-definition}.
Applying Gronwall's inequality completes the proof. $\Box$
%
\vskip0.1in
Next, note that Proposition \ref{1n}, Theorem
\ref{thm:HR_existence_continous_dependence}, and the triangle inequality
yield
\begin{equation}
	\begin{split}
		\|u^{\pm 1, n} (t) - u_{\pm 1, n}(t)\|_{H^{2s - 1}(\ci)}
		\lesssim n^{s-1}.
		\label{5hprimus}
	\end{split}
\end{equation}
Hence, interpolating and applying Lemma
\ref{lem:bound_for_difference-of-approx-and-actual-soln} and
\eqref{5hprimus}, we obtain
		%
			\begin{equation*}
				\begin{split}
					\|u^{\pm 1,n}(t) - u_{\pm 1, n}(t) \|_{H^s (\ci)}
					& \le ( \| u^{\pm 1,n}(t)
					- u_{\pm 1, n}(t) \|_{H^1 (\ci)}
					\\
					& \cdot \| u^{\pm 1,n}(t)
					- u_{\pm 1, n}(t)\|_{H^{2s-1}(\ci)} )^{\frac{1}{2}}
					\\
					& \lesssim (n^{-r_s} \cdot n^{s-1})^{\frac{1}{2}}.
				\end{split}
			\end{equation*}
			Recalling \eqref{r_s-definition}, we see that for $s \ge 2$ this reduces to
			\begin{equation}
				\begin{split}
					\|u^{\pm 1,n}(t) - u_{\pm 1, n}(t) \|_{H^s (\ci)} \lesssim
					n^{-\frac{1}{2}}.
					\label{10v}
				\end{split}
			\end{equation}
%
The rest of the proof is the same as in the case $s>3/2$.
%
%
%	
%
%
%
%
%
%
%
%
%
%
%%%%%%%%%%%%%%%%%%%%%%%%%%%%%%%%%%%%%%%%%%%%%%%%%%%%%%%%%%%%%%%%%%%%
\begin{appendices}
\section{Well-Posedness for the Periodic Case}
%
%
%
%
We will now prove well-posedness for the periodic case, after which we will
provide the necessary details to extend the argument to the non-periodic case.
\vskip0.1in
\subsection{Existence.}
\vskip0.1in
\label{existence}
\setcounter{equation}{0}
\vskip0.1in
Here we will prove the existence of a solution to the HR i.v.p. and inequalities
\eqref{Life-span-est} and \eqref{u_x-Linfty-Hs}.  We begin by mollifying the HR equation, so that we may apply the following ODE
theorem: 
%
\vskip0.1in
\begin{theorem}
	\label{ode_theorem}
	Let  $Y$  be a Banach space, $X\subset Y$ be an open subset,
	$I' \subset \rr$, and $f: I' \times X\to Y$ a continuously differentiable
	map.  Then for any $t_{0} \in I'$ and $x_{0} \in X$ there exists an
	open ball $I \subset I'$ and a unique differentiable mapping $u:I
	\to Y$ such that for all $t \in I$,  $u'(t) = f(t, u)$
	and $u(t_{0}) = x_{0}.$
\end{theorem}
%
To see why we cannot apply the Banach Space ODE Theorem to the HR equation as is,
we use a counterexample. Let $u=x^{-1/2} \chi_{[0,1]}$ and $s=0$. Then $u \in H^s$ but
$u\p_x u \notin H^s$. Hence, returning to the general case, we see that the
HR equation as is can not be thought as an ODE on the space $H^s$.  To
deal with this problem we will replace the i.v.p \eqref{hr}--\eqref{hr-data} by  
\begin{equation}
	\label{hr-moli}
	\p_t  u_\ee =
	-\gamma J_\ee u_\ee \partial_x  J_\ee  u_\ee - \p_x (1-\p_x^2)^{-1} 
	\left [\frac{3-\gamma}{2}u^2 + \frac{\gamma}{2}(\p_x u)^2 \right ],
\end{equation} 
%
\begin{equation} 
	\label{hr-moli-data} 
	u_\ee(x, 0) = u_0 (x),
\end{equation}
%
where $J_\ee$ is defined as follows: Pick a non-negative $j(x) \in
\mathcal{S}(\rr)$ and let
\begin{equation*}
	\begin{split}
		j_\ee(x) = \frac{1}{\ee}j\left( \frac{x}{\ee} \right).
	\end{split}
\end{equation*}
	We then define $J_\ee$ to be the ``Friedrichs mollifier''
	\begin{equation}
		\begin{split}
			J_\ee f(x) = j_\ee * f(x), \quad \ee>0.
		\end{split}
	\end{equation}
%
%
Notice that the right hand side of \eqref{hr-moli} is a map from $H^s(\ci)$
to $H^s(\ci)$.  In order to apply the ODE Theorem, we will also need to
show that it is a continuously differentiable map:
%
\vskip0.2in





%
%
\begin{lemma}
	Let $f_\ee:H^s(\ci) \to H^s(\ci)$ be given by 
	\begin{equation}
		\label{f_ep}
		f_{\ee}(u) = -\gamma  J_\varepsilon u \partial_x J_\varepsilon u
		- \p_x (1-\p_x^2)^{-1} \left
		[\frac{3-\gamma}{2}u^2 + \frac{\gamma}{2}(\p_x u)^2 \right ].
	\end{equation}
	Then $f_\ee$  is a continuously differentiable map.
\end{lemma}
%
%
{\bf Proof.} We explicitly calculate the derivative of $f_\ee$ at an
arbitrary $w \in H^s(\ci)$:
\begin{equation*}
	\begin{split}
		[Df_{\ee}(u)](w)
		=
		& -\gamma (J_\varepsilon w \cdot \partial_x J_\varepsilon u +
		J_\varepsilon u \cdot \partial_x J_\varepsilon w)
		\\
		& - (1-\p_x ^2)^{-1}
		\p_x \left [(3-\gamma)w u + \gamma\p_x w \p_x u \right ].
	\end{split}
\end{equation*}
Let $w_n \xrightarrow{H^s(\ci)} w$. Then it is easy to check that
%
\begin{equation}
	\begin{split}
		& -\gamma (J_\varepsilon w_n \cdot \partial_x J_\varepsilon u 
		+ J_\varepsilon u \cdot \partial_x J_\varepsilon w_n)
		+ (1-\p_x ^2)^{-1}
		\p_x \left [(3-\gamma)w_n u + \gamma\p_x w_n \p_x u \right ]
		\\
		& \xrightarrow{H^s(\ci)} 
		 -\gamma (J_\varepsilon w \cdot \partial_x J_\varepsilon u 
		+ J_\varepsilon u \cdot \partial_x J_\varepsilon w) + (1-\p_x ^2)^{-1}
		\p_x \left [(3-\gamma)w u + \gamma\p_x w \p_x u \right ].
	\end{split}
\end{equation}
This concludes the proof. $\quad \square$
\vskip0.1in
Hence, by Theorem \ref{ode_theorem}, for each $\ee > 0$ there exists a
unique solution $u_\ee \in C(I, H^s(\ci))$ satisfying the Cauchy-problem
\eqref{hr-moli}-\eqref{hr-moli-data}. Next, we analyze the size and
lifespan of the family $\{u_\ee\}$ of solutions.


%%%%%%%%%%%%%%%%%%%%%%%%
%
%     Estimates  for Life-span and Sobolev norm of $u_\ee$
%
%%%%%%%%%%%%%%%%%%%%%%%%


%
%
{\bf Estimates  for Life-span and Sobolev norm of $u_\ee$.}
%
We will show that there is a lower bound  $T$
for $T_\ee$, which is  independent of $\ee\in(0, 1]$.
This is based on the following differential
inequality for the solution $u_\ee$:
%
\begin{equation} 
	\label{B-diff-ineq}
	\frac 12
	\frac{d}{dt}
	\|u_\ee(t)\|_{H^{s}(\ci)}^2
	\le
	c_s
	\|u_\ee(t)\|_{H^{s}(\ci)}^3,
	\quad
	|t| \le T_\ee.
\end{equation}
%
%
We will prove this inequality  by
following the approach used for quasilinear symmetric
hyperbolic systems in Taylor  \cite{t1}. In what follows we will suppress the
$t$ parameter for the sake of clarity.
%
For any $s\in \ci$ let   $D^s=(1-\p_x^2)^{s/2}$ be the  operator
defined by 
%
$$ \widehat{D^s f}(\xi) \doteq (1 + \xi^2)^{s/2} \widehat{f}(\xi), $$
%
where $ \widehat{f}$ is the Fourier transform
%
$$ \widehat{f}(\xi) =  \frac{1}{2\pi}\int_{\ci} e^{-i \xi x} f(x) \ dx.  $$
%
Applying the operator $D^s$ to  both sides of  \eqref{hr-moli},
then  multiplying the resulting equation by $D^s J_\ee u_\ee$
and integrating it for $x\in\ci$ gives
%
\begin{equation} 
	\begin{split}
		\label{B-moli-int}
		\frac 12
		\frac{d}{dt} \|u_\ee \|_{H^s}^2
		=
		&-
		\gamma \int_{\ci}  D^s(J_\ee u_\ee \partial_x J_\ee u_\ee) \cdot
		D^s J_\ee u_\ee  \  dx
		\\
		&- \frac{3 -\gamma}{2} \int_{\ci} D^{s-2} \p_x (u_{\ee}^2) 
		\cdot D^s J_\ee u_{\ee} \ dx
		\\
		&- \frac{\gamma}{2} \int_{\ci}  D^{s-2} \p_x (\p_x u_\ee)^2
		\cdot D^s J_\ee u_\ee  \ dx.
	\end{split}
\end{equation}
%
We will estimate the right hand side of \eqref{B-moli-int} in parts. In
what follows next we use the fact that  $D^s$ and $J_\ee$ commute and
that  $J_\ee$ satisfies the properties 
%
\begin{equation} 
	\label{J-e-inner-prod-property}
	(J_\ee f, g)_{L^2(\ci)}=( f, J_\ee g)_{L^2(\ci)}
\end{equation}
%
and
%
\begin{equation} 
	\label{Je-u-Hs}
	\| J_\ee u \|_{H^s(\ci)}
	\le
	\|  u \|_{H^s(\ci)}.
\end{equation}
%


%%%%%%%%%%%% Burgers term energy estimate %%%%%%%%%%%%
%
%
%
\noindent
Letting 
%
\begin{equation} 
	\label{v-Je-ue}
	v=J_\ee u_\ee
\end{equation}
%
%
we have
%
\begin{equation} 
	\begin{split}
		\label{B-moli-int-v}
		& -  \gamma \int_{\ci}   D^s (J_{\ee} u_{\ee} \p_x J_\ee u_\ee)
		 \cdot D^s
		J_{\ee}u_\ee \ dx  
		\\
		& = - \gamma \int_\ci
		 D^s(v \partial_x v) \cdot   D^s v \ dx
		\\
		& = - \gamma \int_\ci
		\left [ 
		D^s(v\p_x v)  -  v D^s (\p_xv)
		\right ] 
		D^s v \ dx - \gamma \int_\ci
		v D^s (\p_xv)
		D^s v \ dx.
	\end{split}
\end{equation}
%
%
%
We now estimate \eqref{B-moli-int-v} in parts. Applying the Cauchy-Schwarz inequality gives
%
\begin{equation} 
	\label{int1-est-calc2}
	\begin{split}
		& \Big|
		- \gamma \int_\ci
		\big[ 
		D^s(v\p_x v)  -  v D^s (\p_xv)
		\big]
		D^s v   \, dx
		\Big|
		\\
		& \le
		|\gamma| \cdot \|
		D^s(v\p_x v)  -  v D^s (\p_xv)
		\|_{L^2(\ci)}
		\|
		D^s v 
		\|_{L^2(\ci)}
		\\
		&\le
		|\gamma| \cdot \|
		D^s(v\p_x v)  -  v D^s (\p_xv)
		\|_{L^2(\ci)}
		\|
		v
		\|_{H^s(\ci)}
		\\
		&\le c_s \| \p_x v \|_{L^\infty(\ci)} 
		\| v \|_{H^s(\ci)}^2,
	\end{split}
\end{equation}
%
where the last step follows from 
%
\begin{equation} 
	\label{int1-est-calc3}
	\| D^s(v\p_x v)  -  v D^s (\p_xv) \|_{L^2(\ci)}
	\le
	2 c_s^{\prime}    \| \p_x v \|_{L^\infty(\ci)} 
	\| v \|_{H^s(\ci)},
\end{equation}
which we prove below by using the following Kato-Ponce commutator 
estimate:  
\begin{lemma} 
	\label{KP-lemma}
	[Kato-Ponce]
	If  $s>0$ then there is $c_s^{\prime}>0$ such that 
	%
	\begin{equation} 
		\label{KP-com-est}
		\| D^{s} \big(fg) -  f D^s g\|_{L^2(\ci)}
		\le
		c_s^{\prime}\big(
		\| D^{s}f \|_{L^2(\ci)}    \| g \|_{L^\infty(\ci)} 
		+
		\| \p_xf \|_{L^\infty(\ci)}    \| D^{s-1}g \|_{L^2(\ci)}   
		\big).
	\end{equation}
	%
	\end{lemma}
	%
	%
	In fact, applying  this estimate with $f=v$ and $g=\p_xv$ gives 
	%
	\begin{equation} 
		\label{int1-est-calc4}
		\begin{split}
			& \| D^s(v\p_x v)  -  v D^s (\p_xv) \|_{L^2(\ci)}
			\\
			& \le
			{c_s}^\prime \big(
			\| D^{s}v \|_{L^2(\ci)}    \| \p_x v \|_{L^\infty(\ci)} 
			+
			\| \p_xv \|_{L^\infty(\ci)}    \| D^{s-1}\p_x v \|_{L^2(\ci)}   
			\big)
			\\
			& \le
			2{c_s}^\prime    \| \p_x v \|_{L^\infty(\ci)} 
			\| v \|_{H^s(\ci)}, 
		\end{split}
	\end{equation}
	%
	which  is the desired estimate  \eqref{int1-est-calc3}.
	Next, we have
	%
	%
	%
	\begin{equation} 
		\label{int1-est-calc5}
		\begin{split}
			\Big|
			-\gamma \int_\ci
			v D^s (\p_x v)
			\cdot  D^s v \ dx
			\Big|
			& =
			\left | \frac{\gamma}{2} \right | \cdot \Big|
			\int_\ci
			v \p_x\left(D^s v\right)^2  dx
			\Big|
			\\
			& =
			\left | \frac{\gamma}{2} \right | \cdot \Big | \int_\ci
			\p_x v \, (D^s v)^2 \ dx
			\Big|
			\\
			& \le
			\left | \frac{\gamma}{2} \right |  \cdot \int_\ci
			\Big | \p_x v \, (D^s v)^2   
			\Big| \ dx
			\\
			& \lesssim
			\| \p_x v \|_{L^\infty(\ci)} 
			\| v \|_{H^s(\ci)}^2.
		\end{split}
	\end{equation}
	%
	%
	%
	Combining inequalities  \eqref{int1-est-calc2} and
	\eqref{int1-est-calc5} and applying the Sobolev Imbedding Theorem, we
	have
	%
	\begin{equation} 
		\label{burgers_est'}
		\begin{split}
			\Big|
			-\gamma \int_\ci
			D^s(v \partial_x v) \cdot   D^s v \, dx  
			\Big|
			&\le
			{c_s}^\prime
			\| \p_x v \|_{L^\infty(\ci)} 
			\|  v \|_{H^s(\ci)}^2
			\\
			& \le {c_s}^\prime \| v \|_{C^1(\ci)} \| v \|_{H^s(\ci)}^2
			\\
			& \le {c_s}^{\prime \prime} \| v \|_{H^s(\ci)}^3
			\\
			& \le {c_s}^{\prime \prime} \| u_\ee \|_{H^s(\ci)}^3.
		\end{split}
	\end{equation}
	%
	Next we estimate
	\begin{equation}
		\begin{split}
			\left | - \frac{3 -\gamma}{2} \int_\ci D^{s-2} \p_x u_\ee^2 \cdot
			D^s J_\ee u_\ee \; dx \right |
			& \le \left | \frac{3- \gamma}{2} \right | \int_\ci \left |
			D^{s-2} \p_x u_\ee^2 \cdot D^s J_\ee u_\ee \; dx \right | 
			\\
			& \le \left | \frac{3- \gamma}{2} \right |
			\|D^{s-2} \p_x u_\ee^2 \|_{L^2(\ci)} 
			\|D^s J_\ee u_\ee \|_{L^2(\ci)}
			\\
			& \le \left | \frac{3- \gamma}{2} \right |
			\|D^{s-1} u_\ee^2 \|_{L^2(\ci)} 
			\|D^s u_\ee \|_{L^2(\ci)}
			\\
			& \lesssim \| u_\ee^2 \|_{H^s(\ci)} \| u_\ee \|_{H^s(\ci)}.
		\end{split}
	\end{equation}
	%
	%
	Applying the algebra property, we obtain
	%
	\begin{equation}
		\label{hl1}
		\begin{split}
			\left | - \frac{3 -\gamma}{2} \int_\ci D^{s-2} \p_x u_\ee^2 \cdot
			D^s J_\ee u_\ee \; dx \right |
			\lesssim \| u_\ee \|_{H^s(\ci)}^3.
		\end{split}
	\end{equation}
	%
	%
	We also have
	\begin{equation}
		\begin{split}
			\left |- \frac{\gamma}{2} \int_\ci D^{s-2} \p_x (\p_x u_\ee)^2 \cdot
			D^s J_\ee u_\ee \; dx \right |
			& \le \left | \frac{\gamma}{2} \right | \int_\ci \left | D^{s-2} \p_x (\p_x u_\ee)^2 \right |
			\cdot \left |D^s J_\ee u_\ee \right | \; dx
			\\
			& \le \left | \frac{\gamma}{2} \right |
			\| D^{s-1} (\p_x u_\ee)^2 \|_{L^2(\ci)}
			\| D^s J_\ee u_\ee \|_{L^2(\ci)}
			\\
			& \lesssim \|(\p_x u_\ee)^2 \|_{H^{s-1}(\ci)}
			\| J_\ee u_\ee \|_{H^{s-1}(\ci)} 
			\\
			& \lesssim \|(\p_x u_\ee)^2 \|_{H^{s-1}(\ci)} \| u_\ee \|_{H^{s-1}(\ci)} 
		\end{split}
	\end{equation}
	and applying the algebra property yields
	\begin{equation}
		\label{hl2}
		\begin{split}
		\left | - \frac{\gamma}{2} \int_\ci D^{s-2} (\p_x u_\ee)^2 \cdot
		D^s J_\ee u_\ee \; dx \right |
		& \lesssim \| \p_x u_\ee \|_{H^{s-1}(\ci)}^2 \| u_\ee \|_{H^s(\ci)} 
		\\
		& \lesssim \|u_\ee\|_{H^s(\ci)}^3.
	\end{split}
	\end{equation}
	%
	Combining \eqref{burgers_est'}, \eqref{hl1}, and \eqref{hl2}, we obtain
	\eqref{B-diff-ineq}.
	%%%%%%%%%%%%%%%%%%%%%%%%%%%%%%%%%%%
	%  
	%           Lifespan for CH  solution    
	% 
	%%%%%%%%%%%%%%%%%%%%%%%%%%%%%%%%%%%
	%
	%
	%   
	%
	\vskip0.1in
	\noindent
	{\bf  Lifespan estimate of $u_\ee$.} To derive an explicit formula for
	$T_\ee$ we proceed as follows.  Letting  $y(t)=
	\|u_\ee(t)\|_{H^s(\ci)}^2$ inequality  \eqref{B-diff-ineq} takes the
	form
	%
	\begin{equation} 
		\label{energy-y-ineq}
		\frac 12
		y^{-3/2}\frac{dy}{dt}
		\le
		c_s,
		\qquad
		y(0)=y_0=  \|u_0\|_{H^s(\ci)}^2.
	\end{equation}
	%
	Suppose $t$ is non-negative. Integrating  \eqref{energy-y-ineq} from  0  to $t$ gives
	%
	\begin{equation} 
		\label{energy-y-ineq-calc1}
		\frac{1}{\sqrt{y_0}}  - \frac{1}{\sqrt{y(t)}} 
		\le
		c_s t.
	\end{equation}
	%
	%
	Replacing $y(t)$ with   $\|u_\ee(t)\|_{H^s(\ci)}^2$  and solving for  $\|u_\ee(t)\|_{H^s(\ci)}$
	we obtain the formula
	%
	\begin{equation} 
		\label{norm-u(t)-formula}
		\|u_\ee(t)\|_{H^s(\ci)}
		\le
		\frac{ \|u_0\|_{H^s(\ci)}}{1-c_s\|u_0\|_{H^s(\ci)} t}, \quad t\ge
		0.
	\end{equation}
	%
	Now, from \eqref{norm-u(t)-formula} we see that  $\|u_\ee(t)\|_{H^s(\ci)}$ is finite  if 
	%
	\begin{equation*} 
		\label{Lifespan-calc1}
		c_s    \|u_0\|_{H^s(\ci)} t<1,
	\end{equation*}
	%
	or
	%
	\begin{equation} 
		t
		<
		\frac{1}{ c_s \|u_0\|_{H^s(\ci)}}.
	\end{equation}
	%
	Similarly, if $t$ is negative, then 
	\begin{equation} 
		\label{norm-u(t)-formula-prime}
		\|u_\ee(t)\|_{H^s(\ci)}
		\le
		\frac{ \|u_0\|_{H^s(\ci)}}{1+c_s\|u_0\|_{H^s(\ci)} t}, \quad t < 0.
	\end{equation}
	from which it follows that $\|u_\ee(t)\|_{H^s(\ci)}$ is finite  if 
	%
	\begin{equation} 
		t
		>
		 \frac{-1}{ c_s \|u_0\|_{H^s(\ci)}}.
	\end{equation}
	Therefore, the  solution  $u_\ee(t)$ to the mollified CH Cauchy
	problem exists for $|t| <T_0$, where
	%
	\begin{equation} 
		\label{CH-Lifespan}
		T_0
		=
		\frac{1}{ c_s \|u_0\|_{H^s(\ci)}}.
	\end{equation}
	%
	%%%%%%%%%%%%%%%%%%%%%%%%%%%%%%%%%%%
	%  
	%            Norm of   
	% 
	%%%%%%%%%%%%%%%%%%%%%%%%%%%%%%%%%%%
	%
	%
	%   
	%
	\noindent
	{\bf  Size of the solution estimate.} If we choose  $T=\frac12 T_0$, that is
	%
	\begin{equation} 
		\label{T-def}
		T
		=
		\frac{1}{2 c_s \|u_0\|_{H^s(\ci)}},
	\end{equation}
	%
	then for $|t| \le T$, estimates \eqref{norm-u(t)-formula} and
	\eqref{norm-u(t)-formula-prime} imply 
	%
	\begin{equation*} 
		\label{u(t)-u(0)-bound}
		\|u_\ee(t)\|_{H^s(\ci)}
		\le
		\frac{ \|u_0\|_{H^s(\ci)}}{1-(c_s\|u_0\|_{H^s(\ci)})/(2 c_s \|u_0\|_{H^s(\ci)})},
	\end{equation*}
	%
	or 
	%
	\begin{equation} 
		\|u_\ee(t)\|_{H^s(\ci)}
		\le
		  2 \|u_0\|_{H^s(\ci)},
		\quad 
		|t| \le T.
	\end{equation}
	%
	\vskip0.1in
	Thus we have obtained a lower bound for $T_\ee$ and an upper bound for
	$\|u_\ee(t)\|_{H^s(\ci)}$ independent of $\ee\in (0, 1]$. The following
	lemma summarizes these results and provides an estimate for the
	$H^{s-1}(\ci)$ norm of $\p_t u_\ee(t)$:
	%
	%
	\begin{lemma}
		\label{hr_wp}
		Let  $u_0(x) \in  H^s(\ci)$, $s >3/2$. Then for any $\ee\in (0, 1]$
		the i.v.p. for the mollified HR equation 
		%
		\begin{equation} 
			\label{hr-moli-2}
			\partial_t  u_\ee 
			=
			-\gamma (J_\ee u_\ee \partial_x  J_\ee  u_\ee) - \p_x (1-\p_x^2)^{-1} \left
			[\frac{3-\gamma}{2}(u_\ee)^2 + \frac{\gamma}{2}(\p_x u_\ee)^2
			\right ], 
		\end{equation} 
		%
		\begin{equation} 
			\label{burgers-moli-data-2} 
			u_\ee(x, 0) = u_0 (x),
		\end{equation}
		%
		has a unique solution $u_\ee( t)\in C([-T, T]; H^s(\ci))$. 
		In particular,
		%
		\begin{equation} 
			\label{life-est}
			T
			=
			\frac{1}{2 c_s \|u_0\|_{H^s(\ci)}},
		\end{equation}
		%
		is independent of $\ee$ and
		is a lower bound for the lifespan of $u_\ee( t)$ and
		%
		\begin{equation}
			\label{u-e-Hs-bound}
			\|u_\ee(t)\|_{H^s(\ci)}
			\le
			2 \|u_0 \|_{H^s(\ci)},
			\quad
			|t| \le T.
		\end{equation}
		%
		Furthermore,  $u_\ee( t)\in C^1([T, T]; H^{s-1}(\ci))$ and 
		satisfies
		\begin{equation}
			\label{dt-u-e-Hs-bound}
			\|\p_t u_\ee(t)\|_{H^{s-1}(\ci)}
			\lesssim
			\|u_0 \|_{H^s(\ci)}^2,
			\quad
			|t| \le T.
		\end{equation}
		% 
		Here  $c_s$ is a constant depending only on $s$.
	\end{lemma}
	%
	%
	{\bf Proof.}  It suffices to prove  \eqref{dt-u-e-Hs-bound}.
	Using equation \eqref{hr-moli-2}, for any $t\in [-T, T]$ we have
	%
	\begin{equation*}
		\begin{split}
			& \| \partial_t u_\varepsilon(t) \|_{H^{s-1}(\ci)}  
			\\
			& = 
			\| -\gamma (J_\ee u_\ee \partial_x  J_\ee  u_\ee) -
			\p_x (1-\p_x^2)^{-1} \left [\frac{3-\gamma}{2} (u_\ee)^2 +
			\frac{\gamma}{2}(\p_x u_\ee)^2 \right ] \|_{H^{s-1}(\ci)}
			\\
			& \lesssim  
			\| J_\ee u_\ee \partial_x  J_\ee  u_\ee \|_{H^{s-1}(\ci)}
			+ \|\p_x (1-\p_x^2)^{-1} (u_\ee)^2 \|_{H^{s-1}(\ci)}
			\\
			& + \| \p_x (1-\p_x^2)^{-1}(\p_x u_\ee)^2\|_{H^{s-1}(\ci)}.
			\end{split}
		\end{equation*}
		We break this into three parts:
		\begin{equation}
			\label{bixi}
			\begin{split}
				\| J_\ee u_\ee \p_x J_\ee u_\ee \|_{H^{s-1}(\ci)}
				& = 
				\frac{1}{2}\|\p_x[(J_\varepsilon u_\varepsilon
				)^2]\|_{H^{s-1}(\ci)}
				\\
				& \lesssim \|(J_\varepsilon u_\varepsilon )^2\|_{H^s(\ci)}.
			\end{split}
		\end{equation}
		Applying the algebra property of Sobolev spaces and estimate
		\eqref{u-e-Hs-bound} to \eqref{bixi} gives 
		%
		\begin{equation}
			\label{deriv1}
			\begin{split}
				\|J_\ee u_\ee \p_x J_\ee u_\ee  
				\|_{H^{s-1}(\ci)}
				& \lesssim
				\|J_\varepsilon u_\varepsilon \|_{H^s(\ci)}^2
				\\
				&\lesssim
				\| u_\varepsilon \|_{H^s(\ci)}^2
				\\
				&\lesssim
				\|u_0\|_{H^s(\ci)}^2.
			\end{split}
		\end{equation}
		We also have
		\begin{equation*}
			\begin{split}
				\|\p_x (1-\p_x^2)^{-1} (u_\ee)^2\|_{H^{s-1}(\ci)}
				& \le \| (u_\ee)^2\|_{H^{s-1}(\ci)}
				\end{split}
		\end{equation*}
		which by the algebra property and estimate \eqref{u-e-Hs-bound}
		gives
		\begin{equation}
			\begin{split}
				\label{deriv2}
				\|\p_x (1-\p_x^2)^{-1} (u_\ee)^2\|_{H^{s-1}(\ci)}
				& \lesssim \|u_\ee\|^2_{H^s(\ci)} 
				\\
				& \lesssim  \|u_0\|^2_{H^s(\ci)}.
			\end{split}
		\end{equation}
		Similarly,
		\begin{equation}
			\begin{split}
				\label{deriv3}
				\|\p_x (1-\p_x^2)^{-1} (\p_x u_\ee)^2\|_{H^{s-1}(\ci)}
				& \lesssim \|\p_x u_\ee\|^2_{H^{s-1}(\ci)} 
				\\
				& \lesssim  \|u_\ee \|^2_{H^s(\ci)}
				\\
				& \lesssim \|u_0\|^2_{H^s(\ci)}.
			\end{split}
		\end{equation}
		Combining \eqref{deriv1}, \eqref{deriv2}, and \eqref{deriv3}, we
		obtain \eqref{dt-u-e-Hs-bound}. $\qquad \Box$

		%%%%%%%%%%%%%%%%%%%%%%%%
		%
		%     Choosing  a convergent subsequence
		%
		%%%%%%%%%%%%%%%%%%%%%%%%

		{\bf Choosing  a convergent subsequence.}
		%
		Next we shall show that  the family $\{ u_\ee\}$ has a convergent subsequence
		whose limit $u$ solves the Hyperelastic i.v.p. 
		Let
		$$
		I= [-T, T].
		$$
		By Lemma \ref{hr_wp} we have 
		%
		\begin{equation}
			\label{C-1-fam}
			\{u_\ee\}\subset C(I, H^s(\ci))\cap C^1(I, H^{s-1}(\ci))
		\end{equation}
		%
		and bounded. Since $I$ is compact, we have  
		%
		\begin{equation}
			\label{Lip-1-fam}
			\{u_\ee\}\subset L^{\infty}(I, H^s(\ci))\cap C^1(I,
			H^{s-1}(\ci)).
		\end{equation}
		%
		Now, by the Riesz Lemma, we can identify $H^s(\rr)$ with
		$(H^s(\rr))^*$, where for $w, \psi \in H^s(\rr)$ the duality is
		defined by 
		\begin{equation*}
			T_w(\psi) = <w, \psi>_{H^s(\rr)}.
		\end{equation*}
		Hence, by the Riesz Representation Theorem it follows that we can
		identify \\ $L^\infty(I, H^s(\ci)) $ with the dual space of $L^1(I,
		H^{s}(\ci)$, where for $v\in L^\infty(I, H^s(\ci)) $ and $ \phi \in
		L^1(I, H^{s}(\ci))$ the duality is defined by  
		%
		\begin{equation}
			T_v(\phi) = \int_I <v (t), \phi (t)>_{H^s(\rr)} dt  = \int_I
			 \int_{\rr}
			 \widehat{v}(\xi, t) \overline{\widehat{\phi}}(\xi, t) \cdot (1
			 + \xi^2)^s \ d \xi dt.
		\end{equation}
		%
		Next, we recall Aloaglu's Theorem:
		\begin{theorem}
			If $X$ is a normed vector space,
			the closed unit ball $B^* = \{f \in X^* : \|f\| \le
			1\}$ in $X^*$ is compact in the $weak^*$ topology.
		\end{theorem}
		Therefore the bounded family $\{u_\ee\}$ is compact 
		in the weak$^*$ topology of \\
		$L^\infty(I, H^s(\ci))$. More precisely,
		there is a sequence  $\{ u_{\ee_n} \}$ converging
		weakly to a $ u\in L^{\infty}(I, H^s(\ci))$;
		that is 
		%
		\begin{equation}
			\label{weak-conv}
			\lim_{n\to \infty} T_{u_{\ee_n}}(\phi)  =  T_u (\phi) 
			\; \;		
			\text{ for all } \;\;  \phi \in L^1(I, H^{s}(\ci)).
		\end{equation}
		%
		In order to show that  $u$ solves the HR i.v.p. we need to 
		obtain a stronger  convergence for  $u_{\ee_n}$ so that 
		we can take the limit in the mollified HR equation.
		In fact we will prove that 
		%
		\begin{equation}
			\label{strong-conv}
			u_{\ee_n}\longrightarrow u
			\quad
			\text{ in } \,\,   C(I, H^{s-\sigma}(\ci)),\ \text{for any} \
			\, 0 < \sigma <
			1.
		\end{equation}
		%
		For this we will need the following interpolation  result:
		%%%%%%%%%%%%%%%%%%%%%%%%%%%
		%
		%
		%                 Interpolation Lemma
		%
		%
		%%%%%%%%%%%%%%%%%%%%%%%%%%%
		\begin{lemma}
			\label{interpolation-lem}
			(Interpolation)     Let  $s > \frac{3}{2}$.
			If $v \in C(I, H^s(\ci)) \cap C^1(I, H^{s-1}(\ci))$
			then $v \in C^\sigma (I, H^{s- \sigma}(\ci))$ for  $0 < \sigma < 1$.
		\end{lemma}
		%
		{\bf Proof.}  We have
		\begin{equation*}
			\begin{split}
				& \frac{\|v(t) - v(t')\|^2_{H^{s - \sigma}}}{|t - t'|^{2\sigma}}
				\\
				& = 
				\sum_{\xi \in \zz} (1 + \xi^2)^{s- \sigma} 
				\frac{|\hat{v}(\xi, t) - \hat{v}(\xi, t')|^2}{|t-t'|^{2\sigma}} d\xi\\
				& = \sum_{\xi \in \zz} (1+\xi^2)^s 
				\bigg(\frac{1}{(1+ \xi^2)|t - t'|^2} \bigg)^\sigma |\hat{v}(\xi, t)- \hat{v}(\xi, t')|^2 d\xi\\
				& \leq \sum_{\xi \in \zz}(1+\xi^2)^s \bigg( 1 + \frac{1}{(1+\xi^2)|t-t'|^2} \bigg)
				|\hat{v}(\xi,t) - \hat{v}(\xi,t')|^2 d\xi \\
				& \leq \sum_{\xi \in \zz} (1+ \xi^2)^s |\hat{v}(\xi, t)- \hat{v}(\xi, t')|^2 d\xi
				+ \sum_{\xi \in \zz} (1+ \xi^2)^{s-1} \frac{|\hat{v}(\xi, t) - \hat{v} (\xi, t')|^2}{|t-t'|^2} \\
				& \leq  \sup_t \|v(t)\|_{H^s(\ci)}^2 + \sup_t
				\| \partial_t v(t) \|_{H^{s-1}(\ci)}^2
				\\
				& < \infty.
				\\
			\end{split}
		\end{equation*}
		%
		%
		\vskip0.1in
		Next, using this lemma we will show that the family $\{u_\ee\}$ is
		equicontinuous in $C(I, H^{s-\sigma}(\ci))$, $0 < \sigma < 1$. We
		will follow this by proving that there exists a sub-family
		$\{u_{\ee_n} \}$ that is precompact in $C(I,
		H^{s-\sigma}(\ci))$. These two facts, in conjunction with Ascoli's
		Theorem, will yield
		\begin{equation}
			\label{strong-conv2}
			u_{\ee_n} \to u \; \; \text{in} \; \; C(I,H^{s-\sigma}(\ci)),
			\quad
			0 < \sigma < 1.
		\end{equation}

		%%%%%%%%%%%%%%%%%%%%%%
		%
		%
		%       Equicontinuity
		%
		%
		%%%%%%%%%%%%%%%%%%%%%%

		%
		\vskip0.1in
		\nin
		{\bf  Equicontinuity of $\{u_\ee\}_\ee$  in
		$C(I,H^{s-\sigma}(\ci))$.} Applying  Lemma \ref{interpolation-lem} gives 
		%
		\begin{equation}
			\label{equic-1}
			\sup_{t \neq t'} \frac { \|u_\ee(t) - u_\ee(t') \|_{H^{s -
			\sigma}(\ci)}}{|t - t'|^\sigma} < c<\infty
		\end{equation}
		%
		or
		%
		\begin{equation}
			\label{equic-2}
			\|u_\ee(t) - u_\ee(t') \|_{H^{s - \sigma}(\ci)}< c|t - t'|^\sigma, 
			\text{ for all }  \,\,  t, t'\in I,
		\end{equation}
		%
		which shows that  the family  $\{u_\ee\}$ is equicontinuous in 
		$C(I, H^{s-\sigma}(\ci))$. $\qquad \Box$
		%
		\vskip0.1in
		\nin
		%
		%%%%%%%%%%%%%%%%%%%%%%
		%
		%
		%      PreCompactness
		%
		%
		%%%%%%%%%%%%%%%%%%%%%%%%%%
		%
		%
		%
		%
		%		
		{\bf Precompactness of $\{u_\ee(t)\}$ in $H^{s-\sigma}(\ci))$.}
		Now recall that
		\begin{equation}
			\label{compact-1}
			\|u_\ee(t)\|_{H^{s}(\ci)}
			\le
			2 \|u_0 \|_{H^s(\ci)}, \,
			\quad
			t\in I.
		\end{equation}
		%
		By Kondrachov's Theorem, the inclusion $H^s(\ci) \subset H^{s-
		\sigma }(\ci)$ is compact. By \eqref{compact-1},
		it follows that $\{u_\ee(t)\}$ is precompact in $H^{s-\sigma}(\ci)$.
		$\quad \Box$
		%
		%
		\vskip0.1in
		%
		%
		We are now in a position to apply Ascoli's Theorem: 
		\begin{theorem}
			\label{Ascoli}
			(Ascoli)  Let $X$ be a Banach space, $I$ be a compact metric space,
			and $C(I,X)$  be the set of continuous functions $f: I\longrightarrow X$.
			Suppose $S \subset C(I,X)$  has the following properties:
			%
			\begin{itemize}
				\item[(1)]   $S$ is  equicontinuous.
				\item[(2)]  For each $x \in M$ that the set $S(x) = \{f(x)\}$  is  precompact in $X$.
			\end{itemize} 
			%
			Then $S$  is  precompact  in  $C(I,X)$.
		\end{theorem}
		Compiling our previous results on equicontinuity and precompactness
		and applying Theorem \ref{Ascoli}, we
		conclude that there exists a subfamily $\left\{ u_{\ee_n} \right\}$
		such that
		\begin{equation}
			\label{strong-conv-of-u_ep}
			u_{\ee_n} \to u \; \; \text{in} \; \; C(I, H^{s-\sigma}(\ci)).
		\end{equation}
		%
		%
		%
		%%%%%%%%%%%%%%%%%%%%%%%%%%%%%%%%%
		%
		%
		%     Verifying that the limit $u$ solves Burgers equation
		%
		%
		%%%%%%%%%%%%%%%%%%%%%%%%%%%%%%%%%

		{\bf Verifying that the limit $u$ solves the HR equation.} 
		The following lemma will play a crucial role in our proof of the
		existence of a solution to the HR i.v.p.
		\begin{lemma}
			\label{lem:cc}
			We have
			\begin{equation}
				\begin{split}
					\label{burgers_and_nonlocal_conv}
				&  J_{\varepsilon_n} u_{\varepsilon_n} 
				\cdot J_{\varepsilon_n} \p_x u_{\varepsilon_n} 
				\to  u \partial_x u \; \; 
				\text{in} \; \;
				C(I, H^{s-\sigma-1}(\ci)). 
			\end{split}
			\end{equation}
		\end{lemma}
		%
		{\bf Proof.} It is implied by the following propositions:
		\begin{proposition}
			\label{prop:1aa}
			\begin{equation}
				\begin{split}
					 J_{\ee_n} u_{\ee_n} \to  u \ \ \text{in} \ \
					C(I, H^{s-\sigma}(\ci)).
					\label{}
				\end{split}
			\end{equation}
		\end{proposition}
			{\bf Proof.} Note that
			\begin{equation}
				\begin{split}
					& \| u -  J_{\ee_n} u_{\ee_n}
					\|_{C(I, H^{s-\sigma}(\ci))}
					\\
					&= \| u -  J_{\ee_n} u_{\ee_n} \pm 
					u_{\ee_n} \|_{C(I, H^{s-\sigma}(\ci))}
					\\
					& = \| u -  u_{\ee_n}
					\|_{C(I,H^{s-\sigma}(\ci))} + \| (I - J_{\ee_n})
					u_{\ee_n} \|_{C(I, H^{s-\sigma}(\ci))}.
					\label{1bb}
				\end{split}
			\end{equation}
			Applying the estimates
			\begin{equation*}
				\begin{split}
					& \|I-J_{\ee_n} \|_{L(H^{s-\sigma}(\ci), H^{s -
					\sigma}(\ci))} = o(1),
					\\
					& \|u_{\ee_n}\|_{H^{s-\sigma}(\ci)} \le 2
					\|u_0\|_{H^{s-\sigma}(\ci)}
				\end{split}
			\end{equation*}
			to \eqref{1bb} gives
			\begin{equation}
				\label{2bb}
				\begin{split}
					\| u -  J_{\ee_n} u_{\ee_n}\|_{H^{s-\sigma}(\ci)}
					\le \left( \| u -  u_{\ee_n}
					\|_{C(I, H^{s-\sigma}(\ci))} + o(1) \cdot \|u_0
					\|_{H^{s-\sigma}(\ci)} \right).
				\end{split}
			\end{equation}
			Letting $\ee_n \to 0$ in \eqref{2bb} and applying
			\eqref{strong-conv-of-u_ep} completes the proof. $\quad \Box$
			%
			%
			\begin{proposition}
				\label{prop:dd}
				\begin{equation}
					\begin{split}
						 J_{\ee_n} \p_x u_{\ee_n} \to  \p_x u \ \
						\text{in} \ \ C(I, H^{s-\sigma-1}(\ci)).
						\label{0dd}
					\end{split}
				\end{equation}
			\end{proposition}
			{\bf Proof.} 
			\begin{equation*}
				\begin{split}
					\|\p_x u - J_\ee \p_x u_{\ee_n} \|_{C(I,
					H^{s-\sigma-1}))}  
					& = \|\p_x u - \p_x J_\ee u_{\ee_n} \|_{C(I,
					H^{s-\sigma-1}(\ci))} 
					\\
					& \le \|u - J_\ee u_{\ee_n} \|_{C(I,
					H^{s-\sigma}(\ci))}.
				\end{split}
			\end{equation*}
			Applying Proposition \ref{prop:1aa} completes the proof. $\quad
			\Box$
			%
			\vskip0.1in
			This completes the proof of Lemma \ref{lem:cc}. $\quad \Box$
		%
		\vskip0.1in
		Note that since $\|\p_x (1-\p_x^2)^{-1}\|_{L(H^s(\ci), H^s(\ci))}
		\le 1$ for all $s \in \rr$, it follows immediately from
		\eqref{strong-conv-of-u_ep} that
		\begin{equation}
			\begin{split}
				& \p_x(1- \p_x^2)^{-1} \left( \frac{3-\gamma}{2}
				(u_{\ee_n})^2
				 + \frac{\gamma}{2} (\p_x u_{\ee_n})^2 \right )
				 \\
				 & \to
				 \p_x(1- \p_x^2)^{-1} \left( \frac{3-\gamma}{2} u^2
				 + \frac{\gamma}{2} (\p_x u)^2 \right ) \ \
				 \text{in} \ \ C(I, H^{s-\sigma-1}(\ci)).
				\label{non-local-convergence}
			\end{split}
		\end{equation}
		Combining \eqref{burgers_and_nonlocal_conv} and
		\eqref{non-local-convergence}, and applying the Sobolev Imbedding
		Theorem, we deduce 
		\begin{equation}
			\begin{split}
				& -\gamma (J_{\ee_n} u_{\ee_n} \cdot J_{\ee_n} \p_x
				u_{\ee_n}) - \p_x(1- \p_x^2)^{-1} \left( \frac{3-\gamma}{2}
				(u_{\ee_n})^2
				 + \frac{\gamma}{2} (\p_x u_{\ee_n})^2 \right )
				 \\
				 \to & -\gamma u \p_x u -
				 \p_x(1- \p_x^2)^{-1} \left( \frac{3-\gamma}{2} u^2
				 + \frac{\gamma}{2} (\p_x u)^2 \right ) \ \
				 \text{in} \ \ C(I, C(\ci)).
				\label{loc-non-loc-tog}
			\end{split}
		\end{equation}
		Furthermore, we note that the convergence  
		%
		\begin{equation}
			\label{weak-conv-2}
			T_{u_{\ee_n}}(\phi)  \longrightarrow  T_u(\phi) \;
			\text{ for all } \;  \phi \in L^1(I, H^{s}(\ci))
		\end{equation}
		%
		can be restated as 
		%
		\begin{equation}
			u_{\ee_n}  \longrightarrow  u
			\quad
			\text{ in }  \,\,
			\mathcal{D}'(I\times \ci).
		\end{equation}
		%
		This implies 
		%
		\begin{equation}
			\label{distib-conv-2}
			\p_tu_{\ee_n}  \longrightarrow  \p_tu
			\quad
			\text{ in }  \,\, \mathcal{D}'(I\times \ci).
		\end{equation}
		%
		Since for all $n$ we have 
		%
		\begin{equation}
			\p_tu_{\ee_n} 
			=
			-\gamma (J_{\varepsilon_n} u_{\varepsilon_n}  \cdot
			J_{\varepsilon_n}\partial_x u_{\varepsilon_n}) - \p_x (1-
			\p_x^2)^{-1} \left
			[\frac{3-\gamma}{2}(u_\ee)^2 + \frac{\gamma}{2}(\p_x u_\ee)^2 \right ] 
		\end{equation}
		%
		by the uniqueness  of the limit in \eqref{loc-non-loc-tog}
		we must have
		%
		\begin{equation}
			\label{1000y}
			\partial_t u =- \gamma u \partial_x u- \p_x (1- \p_x^2)^{-1} \left
			[\frac{3-\gamma}{2}u^2 + \frac{\gamma}{2}(\p_x u)^2 \right ].
		\end{equation}
		%
		Thus we have constructed a solution $u \in L^\infty(I, H^s(\ci))$
		to the HR i.v.p. $\qquad \Box$
		\vskip0.1in
		It remains to prove that $u \in C(I, H^s(\ci)).$
		%%%%%%%%%%%%%%%%%%%%%%%%%%
%
%
%Proof that  $u \in C(I, H^s(\ci)) \bigcap C^1(I, H^{s-1}(\ci))$.
%
%
%
%%%%%%%%%%%%%%%%%%%%%%%%%%
\vskip0.1in
{\bf Proof that $u \in C(I, H^s(\ci))$.} 
We first outline our strategy. Since \\
$u \in L^\infty(I, H^s(\ci))$, it is a
continuous function from $I$ to $H^s(\ci)$ with respect to the weak
topology on $I$; that is, for $\{t_n\} \subset I$ such that $t_n \to t$, we
have
\begin{equation}
	\begin{split}
		<u(t_n), \ v>_{H^s(\ci)} \ \longrightarrow \
		<u(t), \ v>_{H^s(\ci)}, \quad \forall
		v \in H^s(\ci).
		\label{1ff}
	\end{split}
\end{equation}
Next, note that
\begin{equation}
	\begin{split}
		\|u(t) - u(t_n) \|_{H^s(\ci)}^2
		& = <u(t) - u(t_n), \ u(t) -
		u(t_n)>_{H^s(\ci)}
		\\
		& = \|u(t)\|_{H^s(\ci)}^2 + \|u(t_n)\|_{H^s(\ci)}^2
		\\
		& - <u(t_n), \
		u(t) >_{H^s(\ci)} - <u(t), u(t_n)>_{H^s(\ci)}.
		\label{2ff}
	\end{split}
\end{equation}
Applying \eqref{1ff} and \eqref{2ff}, we see that
\begin{equation}
	\begin{split}
		\lim_{n \to \infty} \|u(t) - u(t_n)\|_{H^s(\ci)}^2 = \left[ \lim_{n
		\to \infty} \|u(t_n)\|_{H^s(\ci)}^2
		\right] - \|u(t)\|_{H^s(\ci)}^2.
		\label{3ff}
	\end{split}
\end{equation}
Hence, by \eqref{3ff}, to prove that $u \in C(I, H^s(\ci))$, it will be
enough to show that the map $t \mapsto \|u(t)\|_{H^s(\ci)}$ is a continuous
function of $t$. However, this will follow from the energy
estimate
		\begin{equation}
			\label{en-est-u}
			\frac{1}{2} \frac{d}{dt} \|u(t)\|_{H^s(\ci)}^2
			\le c_s \|u(t)\|_{H^s(\ci)}^3, \quad |t| \le T
		\end{equation}
		which we now derive. Applying $D^s$ to both sides of
		\eqref{1000y}, multiplying the
		resulting equation by $D^s u$, and integrating for $x\in \ci$, we obtain
		\begin{equation}
			\begin{split}
				\label{bound-int}
				\frac 12
				\frac{d}{dt} \|u \|_{H^s}^2
				=
				&-
				\gamma \int_{\ci}   D^s (u \p_x u) \cdot
				D^s u \  dx
				\\
				&- \frac{3 -\gamma}{2} \int_{\ci}  D^{s-2} \p_x (u^2) 
				\cdot D^s u \ dx
				\\
				&- \frac{\gamma}{2} \int_{\ci}   D^{s-2} \p_x (\p_x u)^2
				\cdot D^s u \ dx.
			\end{split}
		\end{equation}
		First we estimate
	\begin{equation}
		\begin{split}
			\left | - \frac{3 -\gamma}{2} \int_\ci D^{s-2} \p_x (u^2) \cdot
			D^s u \; dx \right |
			& \le \left | \frac{3 -\gamma}{2} \right |
			\int_\ci \left |
			D^{s-2} \p_x (u^2) \cdot D^s u \right | dx 
			\\
			& \le \left | \frac{3 -\gamma}{2} \right |
			\|D^{s-2} \p_x (u^2) \|_{L^2(\ci)} 
			\|D^s u \|_{L^2(\ci)}
			\\
			& \le \left | \frac{3 -\gamma}{2} \right |
			\|D^{s-1} (u^2) \|_{L^2(\ci)} 
			\|D^s u \|_{L^2(\ci)}
			\\
			& = \left | \frac{3 -\gamma}{2} \right |
			\| u^2 \|_{H^{s-1}(\ci)} \| u \|_{H^s(\ci)}
			\\
			& \le
			\left | \frac{3 -\gamma}{2} \right | \| u^2 \|_{H^s(\ci)} \| u
			\|_{H^s(\ci)}.
		\end{split}
	\end{equation}
	%
	%
	Applying the algebra property, we obtain
	%
	\begin{equation}
		\label{hl1-prime}
		\begin{split}
			\left | -\frac{3 -\gamma}{2} \int_\ci D^{s-2} \p_x u^2 \cdot
			D^s u \; dx \right |
			\le c_s' \| u \|_{H^s(\ci)}^3.
		\end{split}
	\end{equation}
	%
	%
	We also have
	\begin{equation}
		\begin{split}
			\left | -\frac{\gamma}{2} \int_\ci D^{s-2} \p_x (\p_x u)^2 \cdot
			D^s u \; dx \right |
			& \le \left | \frac{\gamma}{2} \right |
			\int_\ci \left | D^{s-2} \p_x (\p_x u)^2 
			\cdot D^s u \right | \; dx
			\\
			& \le \left | \frac{\gamma}{2} \right |
			\| D^{s-2} \p_x (\p_x u)^2 \|_{L^2(\ci)}
			\| D^s u \|_{L^2(\ci)}
			\\
			&  \le \left | \frac{\gamma}{2} \right | \|(\p_x u)^2
			\|_{H^{s-1}(\ci)} \| u \|_{H^s(\ci)} 
		\end{split}
	\end{equation}
	and applying the algebra property yields
	\begin{equation}
		\label{hl2-prime}
		\left | -\frac{\gamma}{2} \int_\ci D^{s-2} (\p_x u)^2 \cdot
		D^s u \; dx \right |
		\le c_s'' \|u\|_{H^s(\ci)}^3.
	\end{equation}
	It remains to estimate 
	\begin{equation*}
		- \gamma \int_{\ci} \left [  D^s (u \p_x u) \cdot
		D^s u \right ]  dx
	\end{equation*}
	We have
	\begin{equation} 
	\begin{split}
		\label{B-moli-int-v'}
		-  \gamma \int_{\ci} \left [D^s (u \p_x u) \cdot D^s
		u \right ] \ dx
		= &- \gamma  \int_\ci
		\left [ D^s(u \partial_x u) \cdot   D^s u \right ] \ dx
		\\
		=& - \gamma \int_\ci
		\big[ 
		D^s(u\p_x u)  -  u D^s (\p_xu)
		\big] \cdot
		D^s u   \ dx
		\\
		&
		- \gamma \int_\ci
		u D^s (\p_xu) \cdot
		D^su \ dx.
	\end{split}
\end{equation}
%
%
%
We now estimate \eqref{B-moli-int-v'} in parts. Applying the Cauchy-Schwarz inequality gives
%
\begin{equation*} 
	\begin{split}
		& \Big|
		- \gamma \int_\ci
		\big[ 
		D^s(u\p_x u)  -  u D^s (\p_xu)
		\big] \cdot
		D^s u   \, dx
		\Big|
		\\
		& \le
		|\gamma| \cdot \|
		D^s(u\p_x u)  -  u D^s (\p_xu)
		\|_{L^2(\ci)}
		\|
		D^s u 
		\|_{L^2(\ci)}
		\\
		& =
		|\gamma| \cdot \| D^s(u\p_x u)  -  u D^s (\p_xu)
		\|_{L^2(\ci)}
		\|
		u
		\|_{H^s(\ci)}
			\end{split}
\end{equation*}
Applying \eqref{int1-est-calc3}, we obtain
\begin{equation*}
\begin{split}
		\Big|
		- \gamma \int_\ci
		\big[ 
		D^s(u\p_x u)  -  u D^s (\p_xu)
		\big]
		D^s u   \, dx
		\Big|
		&\le
		 c_s'''   \| \p_x u \|_{L^\infty(\ci)} 
		\| u \|_{H^s(\ci)}^2.
	\end{split}
\end{equation*}
%\label{int1-est-calc2'}
%
Next, we apply Cauchy-Schwartz and the Sobolev Imbedding Theorem to deduce 
	%
	%
	%
	\begin{equation} 
		\label{int1-est-calc5'}
		\begin{split}
			\Big|
			\int_\ci
			\left [u D^s (\p_x u)
			\cdot  D^s u \right ] dx
			\Big|
			& =
			\frac{1}{2} \Big|
			   \int_\ci
			\left [u \p_x\left(D^s u\right)^2 \right ] \ dx
			\Big|
			\\
			& \le
			\frac{1}{2} \int_\ci \Big |
			\left [\p_x u \, (D^s u)^2  \right ] 
			\Big| \ dx
			\\
			& \le
			\frac{1}{2}
			\| \p_x u \|_{L^\infty(\ci)} 
			\| u \|_{H^s(\ci)}^2.
			\\
			& \le c_s'''' \|u\|_{H^s(\ci)}^3.
		\end{split}
	\end{equation}
	%
	%
	%
	Combining \eqref{hl1-prime}, \eqref{hl2-prime},
	and \eqref{int1-est-calc5'}, we obtain \eqref{en-est-u}, as desired.
	\vskip0.1in
	{\bf Size of the solution}. 
	Letting  $y(t)=  \|u(t)\|_{H^s(\ci)}^2$ inequality \eqref{en-est-u}
	takes the form
	%
	\begin{equation} 
		\label{energy-y-ineq'}
		\frac 12
		y^{-3/2}\frac{dy}{dt}
		\le
		c_s,
		\qquad
		y(0)=y_0=  \|u_0\|_{H^s(\ci)}^2.
	\end{equation}
	%
	Suppose $t$ is non-negative. Then integrating  \eqref{energy-y-ineq'}
	from  0 to $t$ gives
	%
	\begin{equation*} 
		\frac{1}{\sqrt{y_0}}  - \frac{1}{\sqrt{y(t)}} 
		\le 
		c_s t.
	\end{equation*}
	%
	%
	Replacing $y(t)$ with   $\|u(t)\|_{H^s(\ci)}^2$  and solving for  $\|u(t)\|_{H^s(\ci)}$
	we obtain the formula
	%
	\begin{equation} 
		\label{norm-u(t)-formula'}
		\|u(t)\|_{H^s(\ci)}
		\le
		\frac{ \|u_0\|_{H^s(\ci)}}{1-c_s\|u_0\|_{H^s(\ci)} t}.
	\end{equation}
	%
	Now, note that our solution $u$ inherits the common lifespan $T$ of the family
	$\{u_\ee\}$; that is, $u$ has lifespan
	\begin{equation*}
		T
		=
		\frac{1}{2 c_s \|u_0\|_{H^s(\ci)}}.
	\end{equation*}
	Substituting into \eqref{norm-u(t)-formula'} we obtain	
	%
	\begin{equation*} 
		\label{u(t)-u(0)-bound'}
		\|u(t)\|_{H^s(\ci)}
		\le
		\frac{ \|u_0\|_{H^s(\ci)}}{1-(c_s\|u_0\|_{H^s(\ci)})/(2 c_s \|u_0\|_{H^s(\ci)})},
	\end{equation*}
	%
	which simplifies to 
	%
	\begin{equation*}
		\|u(t)\|_{H^s(\ci)}
		\le
		2 \|u_0\|_{H^s(\ci)},
		\quad 
		0\le t \le T.
	\end{equation*}
	Similarly, for negative $t$, we have
	\begin{equation*}
		\|u(t)\|_{H^s(\ci)}
		\le
		2 \|u_0\|_{H^s(\ci)},
		\quad 
		-T \le t < 0.
	\end{equation*}
	Hence,
	\begin{equation}
		\label{uniform_bound_for_u}
		\|u(t)\|_{H^s(\ci)}
		\le
		2 \|u_0\|_{H^s(\ci)},
		\quad 
		|t| \le T.
	\end{equation}
		%
		\vskip0.1in
		{\bf Space of the solution}.
	Derivating the left hand side of \eqref{en-est-u} and simplifying, we obtain
	\begin{equation}
		\label{en-est-u-simplified}
	\frac{d}{dt} \|u(t)\|_{H^s(\ci)} \le c_s \|u(t)\|_{H^s(\ci)}^2, \quad |t| \le T.
	\end{equation}
	Since $\|u(t)\|_{H^s(\ci)}$
	is uniformly bounded for $|t| \le T$ by
	\eqref{uniform_bound_for_u}, we conclude from
	\eqref{en-est-u-simplified} that the map $t \mapsto
	\|u(t)\|_{H^s(\ci)}$ is Lipschitz continuous in $t$, for $|t| \le T$.
	Therefore, by \eqref{3ff}, $u \in C(I, H^s(\ci))$. 
	%
	%
	\vskip0.2in
	%
	%
	%
	
	\subsection{Uniqueness.}
	%
	%
	Let $u,\omega \in C(I, H^s(\ci)), \ s > 3/2$ be two solutions to the
	Cauchy-problem \eqref{hyperelastic-rod-equation}-\eqref{init-cond} with
	common initial data. Let $v=u-w$; since
	\begin{align*}
		\p_t u 
		& = - \gamma u \p_x u - D^{-2} \p_x \left[ \frac{3-\gamma}{2} u^2 +
		\frac{\gamma}{2}\left( \p_x u \right)^2 \right]
		\\
		\p_t w & = -\gamma w \p_x w - D^{-2} \p_x \left[
		\frac{3-\gamma}{2} w^2 + \frac{\gamma}{2}(\p_x w)^2 
		\right]
	\end{align*}
	we subtract the two equations to obtain 
	\begin{equation*}
		\begin{split}
			\p_t v
			= -\frac{\gamma}{2} \p_x [v(u + w)] - D^{-2} \p_x \left\{
			\frac{3-\gamma}{2}[v(u+w)] + \frac{\gamma}{2}[\p_x v \cdot \p_x (u+w)]
			\right\}
		\end{split}
	\end{equation*}
	and hence
	\begin{equation}
		\begin{split}
			D^\sigma \p_t v = -\frac{\gamma}{2} D^\sigma \p_x [v(u+w)] - D^{\sigma -2} \p_x
			\left\{ \frac{3-\gamma}{2} [v(u+w)] + \frac{\gamma}{2} [\p_x v
			\cdot \p_x
			(u+w)]
			\right\}.
			\label{1v}
		\end{split}
	\end{equation}
	Multiplying both sides of \eqref{1v} by $D^\sigma v$ and integrating, we obtain
	\begin{equation}
		\begin{split}
			\frac{1}{2} \frac{d}{dt} \|v\|_{H^\sigma(\ci)}^2
			& =  \overbrace{-\frac{\gamma}{2} \int_{\ci} D^\sigma \p_x [v(u+w)] \cdot
			D^\sigma v \ dx}^i
			\\
			& \overbrace{- \frac{3-\gamma}{2} \int_{\ci}  D^{\sigma -2}
			\p_x[v(u+w)] \cdot
			D^\sigma v \ dx}^{ii} 
			\\
			& - \overbrace{\frac{\gamma}{2} \int_{\ci} D^{\sigma -2} \p_x [ \p_x v
			\cdot \p_x (u+w)]\cdot D^\sigma v \ dx }^{iii}.
			\label{2v}
		\end{split}
	\end{equation}
	We will estimate (\hyperref[2v]{ii}) first.
	Applying Cauchy-Schwartz, we have 
	\begin{equation*}
		\begin{split}
			|ii|
			& \le \left | \frac{3-\gamma}{2} \right | \|D^{\sigma -2}
			\p_x [v(u+w)] \cdot D^\sigma
			v  \|_{L^1(\ci)}
			\\
			 & \le  \left | \frac{3-\gamma}{2} \right | \|D^{\sigma -2} \p_x [v(u+w)]
			\|_{L^2(\ci)} \|v\|_{H^\sigma(\ci)}
			\\
			& \lesssim \|v(u+w)\|_{H^{\sigma -1}(\ci)} \|v\|_{H^\sigma(\ci)}
		\end{split}
	\end{equation*}
	which by the algebra property and the Sobolev
	Imbedding Theorem gives
\begin{equation}
		\begin{split}
		|ii| \lesssim \|u+w\|_{H^{\sigma -1}(\ci)} \|v\|_{H^\sigma(\ci)}^2.
			\label{3v}
		\end{split}
	\end{equation}
	To estimate (\hyperref[2v]{iii}) we first apply
	Cauchy-Schwartz and the Sobolev Imbedding Theorem:
	\begin{equation*}
		\begin{split}
		|iii| & \le	\left | \frac{\gamma}{2} \right | \|D^{\sigma -2} \p_x
			[\p_x v \cdot \p_x (u+w)] \cdot D^\sigma v  \|_{L^1(\ci)} 
			\\
			& \le  \left | \frac{\gamma}{2} \right | \|D^{\sigma -2} \p_x
			[\p_x v \cdot \p_x (u+w)] \|_{L^2(\ci)}
			\|v\|_{H^\sigma(\ci)}
			\\
			& \le \left |\frac{\gamma}{2} \right|
			\|[\p_x v \cdot \p_x (u+w)] \|_{H^{\sigma -1}(\ci)}
			\|v\|_{H^\sigma(\ci)}.
		\end{split}
	\end{equation*}
	Restrict $1/2 < \sigma < 1$. Then applying Lemma \ref{impo}, we conclude
	that
	\begin{equation}
		\begin{split}
			|iii|
			& \le C \left | \frac{\gamma}{2} \right |
			\|\p_x(u+w) \|_{H^{\sigma}(\ci)}
			\|\p_x v\|_{H^{\sigma -1}(\ci)} \|v\|_{H^\sigma(\ci)}
			\\
			& \lesssim \|u+w \|_{H^{\sigma + 1}(\ci)}
			\|v\|_{H^\sigma(\ci)}^2.
			\label{3'v}
		\end{split}
	\end{equation}
	It remains to estimate (\hyperref[2v]{i}).
	Proceeding, we rewrite
	\begin{equation}
		\begin{split}
			|i| & =
			\left |
			-\frac{\gamma}{2} \int_{\ci} \left[ D^\sigma \p_x, \ u+w \right]v \cdot
			D^\sigma v \ dx - \frac{\gamma}{2} \int_{\ci} (u+w) D^\sigma
			\p_x v \cdot D^\sigma v\ dx
			\right | 
			\\
			& \le \left |
			-\frac{\gamma}{2} \int_{\ci} \left[ D^\sigma \p_x, \ u+w \right]v \cdot
			D^\sigma v \ dx \right |
			+ \left | \frac{\gamma}{2} \int_{\ci} (u+w) D^\sigma \p_x v
			\cdot D^\sigma v\
			dx \right |.
			\label{4v}
		\end{split}
	\end{equation}
	We now estimate \eqref{4v} in pieces. Observe that by integrating by parts
	and applying Cauchy-Schwartz we have
	\begin{equation}
		\begin{split}
			\left | \frac{\gamma}{2}\int_{\ci} (u+w) D^\sigma \p_x v \cdot
			D^\sigma v \ dx \right |
			& = \left | -\frac{\gamma}{2} \int_{\ci} \p_x (u+w) D^\sigma v
			\cdot D^\sigma v \ dx \right |
			\\
			& \lesssim \|\p_x (u+w) D^\sigma v \|_{L^2(\ci)} \|D^\sigma
			v\|_{L^2(\ci)}
			\\
			& \lesssim \|\p_x (u+w)\|_{L^\infty(\ci)}
			\|v\|_{H^\sigma(\ci)}^2.
			\label{4'v}
		\end{split}
	\end{equation}
	To estimate the remaining piece of \eqref{4v}, we recall first that we
	have the restriction $1/2 < \sigma < 1$. However, this will not prevent
	us from applying Corollary \ref{cor1}; in fact, choosing $\ 3/2 < \rho
	< s,  \ 1/2< \sigma <\min\{1, \ \rho -1 \}$, we obtain
	\begin{equation}
		\begin{split}
			\left | -\frac{\gamma}{2} \int_{\ci} [D^\sigma \p_x, \ u+w] v
			\cdot D^\sigma v \ dx \right |
			& \le \left | \frac{\gamma}{2} \right| \int_{\ci} \left |
			[D^\sigma \p_x, \ u+w] v
			\cdot D^\sigma v \right | dx 
			\\
			& \lesssim \|[D^\sigma \p_x, \ u+w]v\|_{L^2(\ci)}
			\|v\|_{H^\sigma(\ci)} \\
			& \lesssim \|u+w\|_{H^\rho(\ci)} \|v\|_{H^\sigma(\ci)}^2.
			\label{7v}
		\end{split}
	\end{equation}
	Combining \eqref{4'v} and \eqref{7v} and applying the Sobolev Imbedding
	Theorem, we obtain the estimate
	\begin{equation}
		\begin{split}
			|i| \lesssim \|u+w\|_{H^\rho(\ci)} \|v\|_{H^\sigma(\ci)}^2.
			\label{8v}
		\end{split}
	\end{equation}
	Recall \eqref{2v}. Grouping \eqref{3v}, \eqref{3'v}, and \eqref{8v}, and applying
	the Sobolev Imbedding Theorem, we see that 
	\begin{equation}
		\begin{split}
			\frac{1}{2} \frac{d}{dt}
			\|v\|_{H^\sigma(\ci)}^2 \lesssim \|u+w\|_{H^\rho(\ci)}
			\|v\|_{H^\sigma(\ci)}^2.
			\label{9v}
		\end{split}
	\end{equation}
	By Gronwall's inequality, \eqref{9v} gives
	\begin{equation}
		\label{10lv}
		\begin{split}
			\|v\|_{H^\sigma(\ci)}
			& \lesssim e^{\int_0^t \|u+w\|_{H^{\rho}}}
			\|v_0\|_{H^\sigma(\ci)}, \qquad |t| \le T.
		\end{split}
	\end{equation}
	First, note that $v_0 = u_0 - w_0 = 0$; secondly, $\|u + w \|_{H^\rho}
	\le \|u + w \|_{H^s(\ci)} < \infty$ for $|t| \le T$ by
	the triangle inequality and \eqref{u_x-Linfty-Hs}. Hence, from
	\eqref{10lv} we obtain
	\begin{equation*}
		\begin{split}
			\|v\|_{H^\sigma(\ci)}
			& \lesssim \|v_0\|_{H^\sigma(\ci)}, \quad |t| \le T	
			\\
			& = 0.
		\end{split}
	\end{equation*}
	We conclude that solutions to the HR i.v.p. with initial data in
	$H^s(\ci)$ are unique for $s > 3/2$.  $\qquad
	\Box$
	%
	%
	%
	%
	\subsection{Continuous Dependence.}
	Let $\left\{ u_{0, n} \right\}_n \subset H^s(\ci)$ be a uniformly bounded
sequence converging to $u_0$ in $H^s(\ci)$.
Consider solutions $u $, $u^\ee$, $u^\ee_n$, and $u_n$ to the Cauchy-problem
\eqref{hyperelastic-rod-equation}-\eqref{init-cond}
with associated initial data $u_0$, $J_\ee u_0$,
$J_\ee u_{0,n}$, and $u_{0,n}$, respectively, where $J_\ee$ is defined as follows: Pick a function $\widehat{j}(\xi) \in \mathcal{S}(\rr)$ such that
	\begin{equation}
		\label{0u}
		\begin{split}
			& 0 \le \widehat{j}(\xi) \le 1,
			\\
			& \widehat{j}(\xi) = 1 \ \ \text{if} \ \ |\xi| \le 1.
		\end{split}
	\end{equation}
	Since $\sum_{-M}^M \widehat{j}(\ee \xi) e^{i \xi x}$ converges uniformly as $M \to
	\infty$, we can let
	\begin{equation}
		\begin{split}
			j_\ee (x) = \frac{1}{2 \pi}\sum_{\xi \in \zz}
			\widehat{j}(\ee \xi) e^{i \xi x}, \quad \ee > 0
			\label{parseval-def}
		\end{split}
	\end{equation}
	which is equivalent to stating that we can find $\left\{ j_\ee
	\right\} \subset \mathcal{S}(\ci)$ such that
	\begin{equation}
		\begin{split}
			\widehat{j_\ee} = \widehat{j }(\ee \xi), \quad \ee > 0.
			\label{widehat-def}
		\end{split}
	\end{equation}
	We then define $J_\ee$ to be the ``Friedrichs mollifier''
	\begin{equation}
		\label{0'u}
		\begin{split}
			J_\ee f(x) = j_\ee * f(x), \quad \ee>0.
		\end{split}
	\end{equation}
We remark that we have constructed the operator $J_\ee$ in this manner in
order that inequality \eqref{widehat-def} is satisfied; this will prove
crucial later on.
%
Applying
the triangle inequality, we have
\begin{equation*}
	\begin{split}
		\|u - u_n\|_{H^s(\ci)}
		& \le \|u - u^\ee\|_{H^s(\ci)}
		+ \|u^\ee - u^{\ee}_n \|_{H^s(\ci) }
		+  \|u^{\ee}_n - u_n \|_{H^s(\ci)}.
	\end{split}
\end{equation*}
Therefore, to prove continuous dependence, it will be enough to show 
\begin{align}
	& \lim_{\substack{n\to \infty \\ \ee \to 0}} \|u - u^\ee\|_{H^s(\ci)}
	=0,
	\label{enough_to_prove1}
	\\
	& \lim_{\substack{n\to \infty \\ \ee \to 0}} \|u^\ee - u^{\ee}_n
	\|_{H^s(\ci)} = 0,
	\label{enough_to_prove2}
	\\
	& \lim_{\substack{n\to \infty \\ \ee \to 0}}
	\|u^{\ee}_n - u_n \|_{H^s(\ci)} =0
	\label{enough_to_prove3}
\end{align}
where we define 
\medskip
	\begin{equation}
		\label{lim-not}
		\begin{split}
			\lim_{\substack{n\to \infty \\ \ee \to 0}} (\cdot) \doteq \lim_{\ee \to
			\infty} [\lim_{n \to \infty} (\cdot )].
		\end{split}
	\end{equation}
\vskip0.1in
{\bf Proof of \eqref{enough_to_prove1}.}
		Consider two solutions $u $ and $u^\ee$ to the Cauchy-problem
	\eqref{hyperelastic-rod-equation}-\eqref{init-cond}
	with associated initial data $u_0$ and
	$J_\ee u_0$, respectively. Set $v= u -u^\ee $. Then $v$ solves the
	Cauchy-problem
	\begin{align}
		\label{4u}
		\p_t v 
		& =  - \gamma (v \p_x v + v \p_x u^\ee + u^\ee \p_x v)  
		\\
		& - D^{-2} \p_x \left\{ \left (\frac{3-\gamma}{2} \right )(v^2 +
		2u^\ee v) + \frac{\gamma}{2}\left[ (\p_x v)^2 + 2 \p_x u^\ee \p_x v \right]
		\right\}, \notag
		\\
		& v(0) = (I- J_\ee)u_0.
		\label{5u}
	\end{align}
	Applying the operator $D^s$ to both sides of \eqref{4u}, multiplying by
	$D^s v$ and integrating, we have
	\begin{equation}
		\begin{split}
			\frac{1}{2}\frac{d}{dt} \|v\|_{H^s(\ci)} = A + B
			\label{6u}
		\end{split}
	\end{equation}
	where
	\begin{equation}
		\begin{split}
			A
			& =  -\gamma \int_{\ci} D^s(v \p_x v) \cdot D^s v \
			dx
			- \frac{3- \gamma}{2} \int_\ci D^{s-2} \p_x (v^2) \cdot D^s v
			\ dx
			\\
			& - \frac{\gamma}{2}\int_\ci D^{s-2} \p_x (\p_x v)^2 \cdot D^s
			v \ dx
			\label{7u}
		\end{split}
	\end{equation}
	and
	\begin{equation}
		\begin{split}
			B 
			= & \ \overbrace{-\gamma \int_\ci D^s (v \p_x u^\ee ) \cdot D^s v \
			 dx}^{(i)} \ \overbrace{-\gamma \int_\ci D^s (u^\ee \p_x v) \cdot D^s v \
			 dx}^{(ii)}
			  \\
			  & \ \overbrace{- \ ( 3- \gamma) \int_\ci D^{s-2} \p_x (u^\ee v) \cdot D^s
			 v \ dx}^{(iii)}
			 \\
			 & \overbrace{-\gamma \int_\ci D^{s-2} \p_x
			(\p_x u^\ee \cdot \p_x v) \cdot D^s v \
			dx}^{(iv)}.
			\label{8u}
		\end{split}
	\end{equation}
	We now provide estimates for $A$ and $B$:
	\vskip0.1in
	{\bf A.} 
	Recalling the proof of \eqref{en-est-u}, with $u$ replaced by
	$v$ gives 
	\begin{equation}
		\begin{split}
			|A| \lesssim \|v\|_{H^s(\ci)}^3, \quad |t| \le T.
			\label{8'u}
		\end{split}
	\end{equation}
%
	{\bf B.} We now estimate in parts:
	%
	%
	%
	\vskip0.1in
		%
	%
\vskip0.1in	
{\bf Estimate of (\hyperref[8u]{i}).} 
We can rewrite
	\begin{equation}
		\begin{split}
			(i)
			= & -\gamma \int_\ci \left[ D^s(v \p_x u^\ee) - v D^s
			\p_x u^\ee \right] \cdot D^s v \ dx
			\\
			& -  \gamma \int_\ci v D^s \p_x u^\ee \cdot D^s v \ dx.
			\label{1wap'}
		\end{split}
	\end{equation}
	Estimating in parts, we have
	\begin{equation}
		\begin{split}
			& |-\gamma \int_\ci \left[ D^s(v \p_x u^\ee) - v D^s
			\p_x u^\ee \right] \cdot D^s v \ dx |
			\\
			& \le |\gamma| \int_\ci |\left[ D^s(v \p_x u^\ee ) - v D^s
			\p_x u^\ee \right] \cdot D^s v| \ dx
			\\
			& \le |\gamma| \cdot \|D^s (v \p_x u^\ee) - v D^s \p_x u^\ee
			\|_{L^2(\ci)} \|v\|_{H^s(\ci)}.
			\label{1wap}
		\end{split}
	\end{equation}
	Applying the Kato-Ponce estimate \eqref{KP-com-est} to \eqref{1wap}, we
	obtain
	\begin{equation*}
		\begin{split}
			& | -\gamma \int_\ci \left[ D^s(v \p_x u^\ee) - v D^s
			\p_x u^\ee \right] \cdot D^s v \ dx |
			\\
			& \le c_s |\gamma| \cdot ( \|D^s v \|_{L^2(\ci)} \|\p_x
			u^\ee\|_{L^\infty(\ci)} + \|\p_x v \|_{L^\infty(\ci)} \|D^{s-1}
			\p_x u^\ee \|_{L^2(\ci)}) \cdot \|v\|_{H^s(\ci)}
		\end{split}
	\end{equation*}
	which by the Sobolev Imbedding Theorem simplifies to
	\begin{equation}
		\begin{split}
			| -\gamma \int_\ci \left[ D^s(v \p_x u^\ee ) - v D^s
			\p_x u^\ee \right] \cdot D^s v \ dx |
			\lesssim \|u^\ee \|_{H^s(\ci)} \|v\|_{H^s(\ci)}^2.
			\label{2wap}
		\end{split}
	\end{equation}
	For the remaining piece of \eqref{1wap'}, we have
	\begin{equation*}
		\begin{split}
			| - \gamma \int_\ci v D^s \p_x u^\ee \cdot D^s v \ dx |
			& \le |\gamma| \int_\ci |v D^s \p_x u^\ee \cdot D^s v | \ dx
			\\
			& \le |\gamma| \cdot \|v\|_{L^\infty(\ci)} \|D^s \p_x u^\ee
			\|_{L^2(\ci)} \|D^s v\|_{L^2(\ci)}
		\end{split}
	\end{equation*}
	which by the Sobolev Imbedding Theorem gives 
	\begin{equation}
		\begin{split}
			| - \gamma \int_\ci u^\ee D^s \p_x v \cdot D^s v \ dx |
			\lesssim \|u^\ee \|_{H^{s+1}(\ci)} \|v\|_{H^{s-1}(\ci)}
			\|v\|_{H^s(\ci)}.
			\label{3wap}
		\end{split}
	\end{equation}
	Combining estimates \eqref{2wap} and \eqref{3wap} we conclude that
	\begin{equation}
		\begin{split}
			|(i)| \lesssim \|u^\ee \|_{H^s(\ci)} \|v\|_{H^s(\ci)}^2 + 
			\|u^\ee \|_{H^{s+1}(\ci)} \|v\|_{H^{s-1}(\ci)}
			\|v\|_{H^s(\ci)}.
			\label{4wap}
		\end{split}
	\end{equation}
%
\vskip0.1in
{\bf Estimate of (\hyperref[8u]{ii}).} We can rewrite
	\begin{equation}
		\begin{split}
			(ii)
			= & -\gamma \int_\ci \left[ D^s(u^\ee \p_x v) - u^\ee D^s
			\p_x v \right] \cdot D^s v \ dx
			\\
			& -  \gamma \int_\ci u^\ee D^s \p_x v \cdot D^s v \ dx.
			\label{1wa'}
		\end{split}
	\end{equation}
	Estimating in parts, we have
	\begin{equation}
		\begin{split}
			& |-\gamma \int_\ci \left[ D^s(u^\ee \p_x v) - u^\ee D^s
			\p_x v \right] \cdot D^s v \ dx |
			\\
			& \le |\gamma| \int_\ci |\left[ D^s(u^\ee \p_x v) - u^\ee D^s
			\p_x v \right] \cdot D^s v| \ dx
			\\
			& \le |\gamma| \cdot \|D^s (u^\ee \p_x v) - u^\ee D^s \p_x v
			\|_{L^2(\ci)} \|v\|_{H^s(\ci)}.
			\label{1wa}
		\end{split}
	\end{equation}
	Applying the Kato-Ponce estimate \eqref{KP-com-est} to \eqref{1wa}, we
	obtain
	\begin{equation*}
		\begin{split}
			& | -\gamma \int_\ci \left[ D^s(u^\ee \p_x v) - u^\ee D^s
			\p_x v \right] \cdot D^s v \ dx |
			\\
			& \le c_s |\gamma| \cdot ( \|D^s u^\ee \|_{L^2(\ci)} \|\p_x
			v\|_{L^\infty(\ci)} + \|\p_x u^\ee \|_{L^\infty(\ci)} \|D^{s-1}
			\p_x v \|_{L^2(\ci)}) \cdot \|v\|_{H^s(\ci)}
		\end{split}
	\end{equation*}
	which by the Sobolev Imbedding Theorem simplifies to
	\begin{equation}
		\begin{split}
			| -\gamma \int_\ci \left[ D^s(u^\ee \p_x v) - u^\ee D^s
			\p_x v \right] \cdot D^s v \ dx |
			\lesssim \|u^\ee \|_{H^s(\ci)} \|v\|_{H^s(\ci)}^2.
			\label{2wa}
		\end{split}
	\end{equation}
	For the remaining piece of \eqref{1wa'}, we have
	\begin{equation*}
		\begin{split}
			| - \gamma \int_\ci u^\ee D^s \p_x v \cdot D^s v \ dx |
			& = \left | -\frac{\gamma}{2} \int_\ci u^\ee \p_x (D^s v)^2 \
			dx \right |
			\\
			& = \left | \frac{\gamma}{2} \int_\ci \p_x u^\ee (D^s v)^2 \ dx
			\right |
			\\
			& \le \left | \frac{\gamma}{2} \right | \int_\ci |\p_x u^\ee
			(D^s v)^2 | dx
			\\
			& \le \left | \frac{\gamma}{2} \right | \|\p_x u^\ee
			\|_{L^\infty(\ci)} \|v\|_{H^s(\ci)}^2
		\end{split}
	\end{equation*}
	and applying the Sobolev Imbedding Theorem gives
	\begin{equation}
		\begin{split}
			| - \gamma \int_\ci u^\ee D^s \p_x v \cdot D^s v \ dx |
			\lesssim \|u^\ee \|_{H^s(\ci)} \|v\|_{H^s(\ci)}^2.
			\label{3wa}
		\end{split}
	\end{equation}
	Combining estimates \eqref{2wa} and \eqref{3wa} we conclude that
	\begin{equation}
		\begin{split}
			|(ii)| \lesssim \|u^\ee \|_{H^s(\ci)} \|v\|_{H^s(\ci)}^2.
			\label{4wa}
		\end{split}
	\end{equation}

\vskip0.1in
{\bf Estimate of (\hyperref[8u]{iii}).} We have
	\begin{equation}
		\begin{split}
			|(iii)|
			& \le |3-\gamma| \int_{\ci} |D^{s-2} \p_x (u^\ee v) \cdot D^s v
			\ | \ dx
			\\
			& \le |3- \gamma| \cdot  \|D^{s-2}\p_x (u^\ee v)
			\|_{L^2(\ci)} \cdot \|v\|_{H^s(\ci)}
			\\
			& \le |3- \gamma| \cdot  \| u^\ee v \|_{H^{s -1}(\ci)} \cdot \|v\|_{H^s(\ci)}
			\label{12u}
		\end{split}
	\end{equation}
	and applying the algebra property and the Sobolev Imbedding Theorem gives
	\begin{equation}
		\begin{split}
			|(iii)| & \lesssim \|u^\ee\|_{H^{s-1}(\ci)} \|v\|_{H^{s-1}(\ci)}
			\|v\|_{H^s(\ci)}
			\\
			& \lesssim \|u^\ee\|_{H^{s}(\ci)} \|v\|_{H^{s}(\ci)}^2.
			\label{13u}
		\end{split}
	\end{equation}
	%
	%
	%
	%
	\vskip0.1in
	{\bf Estimate of (\hyperref[8u]{iv}).} We have
	\begin{equation*}
		\begin{split}
			|(iv)|
			& \le |\gamma| \cdot \|D^{s-2} \p_x (\p_x u^\ee \cdot \p_x v)
			\|_{L^2(\ci)} \|D^s v\|_{L^2(\ci)}
			\\
			& \le |\gamma| \cdot \|\p_x u^\ee \cdot \p_x v \|_{H^{s-1}(\ci)}
			\|v\|_{H^s(\ci)}
		\end{split}
	\end{equation*}
	and applying the algebra property gives
	\begin{equation*}
		\begin{split}
			|(iv)|
			& \le |\gamma| \cdot \|\p_x u^\ee \|_{H^{s-1}(\ci)} \|\p_x v
			\|_{H^{s-1}(\ci)} \|v\|_{H^s(\ci)}
			\\
			& \lesssim \|u^\ee\|_{H^s(\ci)} \|v\|_{H^s(\ci)}^2.
		\end{split}
	\end{equation*}
	Hence, collecting our estimates for (\hyperref[8u]{i}),
	(\hyperref[8u]{ii}), (\hyperref[8u]{iii}), and (\hyperref[8u]{iv})
	yields
		\begin{equation}
		\begin{split}
			|B| 
			& \lesssim
			\|u^\ee\|_{H^s(\ci)}
			\|v\|_{H^s(\ci)}^2 + \|u^\ee\|_{H^{s+1}(\ci)}
			\|v\|_{H^{s-1}(\ci)} \|v\|_{H^s(\ci)}.
			\label{14u}
		\end{split}
	\end{equation}
	Combining estimates \eqref{8'u} and \eqref{14u} and recalling
	\eqref{6u}, we obtain
	\begin{equation}
		\begin{split}
			\frac{1}{2}\frac{d}{dt}\|v\|_{H^{s}(\ci)}^2
			& \le c_s(\|v\|_{H^s(\ci)}^3 + \|u^\ee\|_{H^s(\ci)}
			\|v\|_{H^s(\ci)}^2
			\\
			& + \|u^\ee\|_{H^{s+1}(\ci)}
			\|v\|_{H^{s-1}(\ci)} \|v\|_{H^s(\ci)})
			\label{15u}
		\end{split}
	\end{equation}
	where $c_s$ is a constant depending only on $s$.
	Note that the first two terms in the parentheses on the right hand side
	of \eqref{15u} will offer us little trouble;
	it is the third term that requires special care (due to the
	$\|u^\ee\|_{H^{s+1}(\ci)}$ factor, which becomes increasingly large as
	$\ee$ decreases). More precisely:
	%
	%
	%
	\begin{remark}
	\label{lem5r}
	For $r \ge s > 3/2$ and $0 < \ee <<1$, 
	\begin{equation}
		\begin{split}
			\|u^{\ee} (t) \|_{H^r(\ci)} \le C \, \ee^{s-r}
			\label{700r}
		\end{split}
	\end{equation}
	where $C = C(r, \|u_0\|_{H^s(\ci)})$.
\end{remark}	
{\bf Proof.} By part (iii) of Theorem
\ref{thm:HR_existence_continous_dependence}, proved in Section
\ref{existence}, we have
\begin{equation}
	\begin{split}
		\|u^\ee(t) \|_{H^r(\ci)}^2
		& \le C' \|u^\ee (0)\|_{H^r(\ci)}^2
		\\
		& = C' \|J_\ee u_0\|_{H^r(\ci)}^2
		\\
		& = C' \sum_{\xi \in \zz} |\widehat{j_\ee} (\xi) \widehat{u_0}(\xi)
		|^2 \cdot (1 + \xi^2)^r
		\label{0qr}
	\end{split}
\end{equation}
Recall how we chose the mollifier $J_\ee$; it will now play a fundamental role. Since
\eqref{widehat-def} holds by construction, \eqref{0qr} gives 
\begin{equation}
	\begin{split}
		\|u^\ee(t) \|_{H^r(\ci)}^2
		& = C' \sum_{\xi \in \zz} |\widehat{j }(\ee \xi)|^2 \cdot (1 +
		\xi^2)^{r-s} \cdot |\widehat{u_0}(\xi)|^2 \cdot (1 + \xi^2)^s
		\\
		& = C'|\widehat{u_0}(0)|^2 +
		C' \sum_{\xi \in \zz \setminus {0}} |\widehat{j }(\ee \xi)|^2 \cdot (1 +
		\xi^2)^{r-s} \cdot |\widehat{u_0}(\xi)|^2 \cdot (1 + \xi^2)^s.
		\label{1qr}
	\end{split}
\end{equation}
Assume $r \ge s$. Since $\widehat{j }(\xi) \in \mathcal{S}(\rr)$, 
\begin{equation}
	\label{schwartz}
	\begin{split}
		|\widehat{j }(\ee \xi)| \le c_r |\ee \xi |^{s-r}, \quad \xi \neq 0.
	\end{split}
\end{equation}
Applying \eqref{schwartz} to \eqref{1qr}, we obtain
\begin{equation}
	\label{calc_ue}
	\begin{split}
		\|u^\ee (t)\|_{H^r(\ci)}^2 
		& \le C' |\widehat{u_0}(0) |^2 + c_r \sum_{\xi \in \zz \setminus
		{0}} |\ee \xi |^{2(s-r)} \cdot (1 + \xi^2)^{r-s}
		|\widehat{u_0}(\xi) |^2 \cdot (1 + \xi^2)^s
		\\
		& \le C' |\widehat{u_0}(0) |^2 + 2^{r-s} c_r \ee^{2(s-r)}
		\sum_{\xi \in \zz \setminus {0}} |\widehat{u_0}(\xi)|^2 \cdot (1 +
		\xi^2)^s
		\\
		& \le C' \|u_0\|_{H^s(\ci)}^2 + 2^{r-s} c_r \ee^{2(s-r)}
		\|u_0\|_{H^s(\ci)}^2
		\\
		& = (C' + 2^{r-s} c_r \ee^{2(s-r)}) \cdot \|u_0\|^2_{H^s(\ci)}.
	\end{split}
\end{equation}
Assuming $0 < \ee <<1$, we conclude from \eqref{calc_ue} that 
\begin{equation*}
	\begin{split}
		\|u^\ee(t)\|_{H^s(\ci)} \le C \ee^{s-r}
	\end{split}
\end{equation*}
where $C = C(r, \|u_0\|_{H^s(\ci)})$. $\qquad \Box$
\vskip0.1in
	We remark that, despite the blowup of $\|u^\ee \|_{H^{s+1}(\ci)}$
	as $\ee$ becomes small, our difficulties would have been
	amplified if we had originally taken $v=w-u$ for some arbitrary
	solution $w$ to the HR
	i.v.p with initial data $w_0 \in H^s(\ci)$, for then we would be dealing with
	\eqref{15u}, with $w$ substituted in for $u^\ee$. However, note that 
	$\|w\|_{H^{s+1}(\ci)}$ might not even be bounded, whereas $\|u^\ee
	\|_{H^{s+1}(\ci)}$ is always bounded, for any $\ee > 0$.
	%
	%
	\vskip0.1in
	In light of the blowup of $\|u^\ee \|_{H^{s+1}(\ci)}$,
	our strategy in tackling
	\eqref{15u} will be as follows. First, we will obtain an estimate for
	$\|v\|_{H^\sigma(\ci)}$ for suitably chosen $\sigma < s-1$. Then, we
	will use this estimate to interpolate between $\|v\|_{H^\sigma(\ci)}$
	and $\|v\|_{H^s(\ci)}$,
	yielding an estimate for $\|v\|_{H^{s-1}(\ci)}$ which will allow us to control the growth of
	$\|u^\ee\|_{H^{s+1}(\ci)}$. 
	%
	%
	%
	%
\begin{lemma} 
	\label{lem6r}
	For $\sigma$ such that $1/2 < \sigma < 1$ and $\sigma + 1 \le s$, we have
	\begin{equation}
	\begin{split}
		\|v\|_{H^{\sigma}(\ci)} \le C \cdot o(\ee^{s- \sigma }), \qquad |t| \le T
	\end{split}
\end{equation}
where $C=C(\|u_0\|_{H^s(\ci)})$.
\end{lemma}
%
%
%
{\bf Proof.}
Recall that $v$ solves the Cauchy-problem \eqref{4u}-\eqref{5u}.
Applying $D^\sigma$ to both sides of \eqref{4u}, multiplying by
$D^\sigma v$, and integrating, we obtain the
relation
\begin{equation*}
	\begin{split}
		\frac{1}{2}\frac{d}{dt}\|v(t)\|_{H^\sigma(\ci)}^2
		= & - \frac{\gamma}{2}\int_{\ci} D^\sigma
		\p_x \left[ \left( u + u^\ee \right)v
		\right]\cdot D^\sigma v \ dx
		\\
		& - \frac{3-\gamma}{2}\int_{\ci} D^{\sigma
		-2} \p_x \left[ \left( u + u^\ee
		\right)v \right] \cdot D^\sigma v \ dx
		\\
		& - \frac{\gamma}{2}\int_{\ci} D^{\sigma
		-2}
		\p_x \left[ \left( \p_x u + \p_x u^\ee
		\right)\cdot \p_x v \right] \cdot
		D^\sigma v \ dx.
	\end{split}
\end{equation*}
Repeating calculations \eqref{X}-\eqref{12}, with $E$ set to zero,
$u^{\omega,n}$ replaced by $u$, $u_{\omega,n}$ replaced by $u^\ee$, and
$\sigma$ and $\rho$ chosen such that
%
\begin{equation}
	\label{size_of_sigma}
	\begin{split}
	& 1/2 < \sigma < 1,
	\\
	& \sigma + 1 \le \rho \le s 
	\end{split}
\end{equation}
yields
 \begin{equation*}
	\begin{split}
		\frac{1}{2}\frac{d}{dt} \|v\|_{H^\sigma(\ci)}^2
		& \le
		c_s' (\|u^{\ee} + u\|_{H^{\rho}(\ci)} +
		\|\p_x(u^{\ee} + u) \|_{H^\sigma(\ci)})
		\cdot \|v\|_{H^\sigma(\ci)}^2.
	\end{split}
\end{equation*}
\medskip
By the Sobolev Imbedding Theorem, it follows that 
\begin{equation}
	\begin{split}
		\frac{1}{2}\frac{d}{dt} \|v\|_{H^{\sigma}(\ci)}^2
		& \le
		c_s \cdot \|u^{\ee}
		+ u\|_{H^{s}(\ci)}\cdot \|v\|_{H^{\sigma}(\ci)}^2.
		\label{10x}
	\end{split}
\end{equation}
Hence, applying the triangle inequality and
part (iii) of Theorem \ref{thm:HR_existence_continous_dependence} (proved
in Section \ref{existence}) to \eqref{10x} yields
%
\begin{equation}
	\begin{split}
		\label{11x}
		\frac{1}{2}\frac{d}{dt} \|v\|_{H^{\sigma}(\ci)}^2
		& \le
		c_s (\|u^{\ee}(0)\|_{H^{s}(\ci)}
		+ \|u(0)\|_{H^{s}(\ci)})\cdot \|v\|_{H^{\sigma}(\ci)}^2
		\\
		& = c_s (\|J_\ee u_0\|_{H^{s}(\ci)}
		+ \|u_0\|_{H^{s}(\ci)})\cdot \|v\|_{H^{\sigma}(\ci)}^2.
	\end{split}
\end{equation}
We now need the following:
\begin{proposition}
	\label{lem3r}
	For arbitrary $u \in L^2(\ci)$,
	\begin{equation}
		\begin{split}
			\|J_\ee u\|_{H^s(\ci)} \le \|u\|_{H^s(\ci)}.
			\label{lem100u}
		\end{split}
	\end{equation}
\end{proposition}
%
%
%
%
{\bf Proof.}
\begin{equation*}
	\begin{split}
		\|J_\ee u\|_{H^s(\ci)} 
		& = \left[\sum_{\xi \in \zz} |\widehat{j_\ee * u}(\xi) |^2
		(1+\xi^2)^s \right ]^{1/2}
		\\
		& = \left [ \sum_{\xi \in \zz} |\widehat{j_\ee} (\xi) \widehat{u}(\xi) |^2
		(1+ \xi^2)^s \right ]^{1/2}
		\\
		& = \left [ \sum_{\xi \in \zz} |\widehat{j}(\ee \xi)
		\widehat{u}(\xi)|^2 ( 1+ \xi^2)^s \right ]^{1/2}
	\end{split}
\end{equation*}
and since $|\widehat{j }(\ee \xi) | \le 1$ by \eqref{0u}, the result
follows. $\qquad \Box$
\vskip0.1in
Using estimate \eqref{11x}, and applying Proposition \ref{lem3r}, 
we obtain the critical estimate 
\begin{equation}
	\begin{split}
		\label{12x}
		\frac{1}{2}\frac{d}{dt} \|v\|_{H^{\sigma}(\ci)}^2
		& \le
	 C \|v\|_{H^{\sigma}(\ci)}^2
\end{split}
\end{equation}
where $C = C(\|u_0\|_{H^s(\ci)})$. Differentiating the left hand side of
\eqref{12x} and simplifying, we obtain
\begin{equation}
	\begin{split}
		\frac{d}{dt}\|v\|_{H^{\sigma}(\ci)} \le C \|v\|_{H^{\sigma}(\ci)}.
		\label{100x}
	\end{split}
\end{equation}
Let $y(t) = \|v\|_{H^{\sigma}(\ci)}$. Then \eqref{100x} gives
\begin{equation*}
	\begin{split}
		\frac{1}{y(t)}\frac{dy}{dt} \le C.
	\end{split}
\end{equation*}
Hence,
\begin{equation*}
	\begin{split}
		\int_0^t \frac{1}{y(\tau)} \frac{dy}{d \tau}
		\le \int_0^t C \ d \tau, \qquad |t| \le T
	\end{split}
\end{equation*}
from which we obtain
\begin{equation}
	\begin{split}
		\ln |y(t) | - \ln |y(0)| \le C t.
		\label{101x}
	\end{split}
\end{equation}
Simplifying \eqref{101x}, we have
\begin{equation*}
	\begin{split}
		\ln \left |\frac{y(t)}{y(0)} \right | \le C t
	\end{split}
\end{equation*}
Since $y(t)$ is non-negative for all $t \in \rr$, this yields the estimate
\begin{equation*}
	\begin{split}
		y(t) \le y(0) e^{C t}, \qquad |t| \le T.
	\end{split}
\end{equation*}
Substituting back in $\|v\|_{H^{\sigma}(\ci)}$ for $y$, we get
\begin{equation}
	\label{conc-lemma}
	\begin{split}
		\|v\|_{H^{\sigma}(\ci)}
		& \le e^{C t}\|v(0)\|_{H^{\sigma}(\ci)}
		\\
		& = e^{C t}\|u(0) - u^\ee(0) \|_{H^{\sigma}(\ci)}
		\\
		& = e^{C t}\|u_0 - J_\ee u_0 \|_{H^{\sigma}(\ci)}.
	\end{split}
\end{equation}
We now require a critical operator norm estimate which will play an
important role later on.
\begin{proposition}
	\label{lem4r}
	For $r \le s$ and $\ee>0$
	\begin{equation}
	\label{0r}
		\begin{split}
			\|I - J_\ee\|_{L(H^s(\ci), H^r(\ci))} \le o(\ee^{s-r}).
		\end{split}
	\end{equation}
\end{proposition}
Using Proposition \ref{lem4r}, we conclude from estimate \eqref{conc-lemma} that
\begin{equation*}
	\begin{split}
		\|v\|_{H^{\sigma}(\ci)} \le C \cdot o(\ee^{s - \sigma}), \qquad |t|
		\le T
	\end{split}
\end{equation*}
where $C=C(\|u_0\|_{H^s(\ci)})$, completing the proof of Lemma \ref{lem6r}.
$\qquad \Box$
%
\vskip0.1in
{\bf Proof of Proposition \ref{lem4r}.}
Pick an arbitrary $u \in H^s(\ci)$ such that $\|u\|_{H^s(\ci)} = 1$, and $r, s \in \rr$ such that $r \le s$. Using the fact that
$\widehat{j_\ee}(\xi) = \widehat{j}(\ee \xi)$ by construction, we have 
\begin{equation}
	\begin{split}
		\|u - J_\ee u\|_{H^r(\ci)}^2 
		& = \sum_{\xi \in \zz} |\widehat{u}(\xi) - \widehat{j_\ee * u}(\xi) |^2
		(1+\xi^2)^r
		\\
		& = \sum_{\xi \in \zz} |\widehat{u}(\xi) - \widehat{j_\ee}(\xi)
		\widehat{u}(\xi) |^2 (1+\xi^2)^r
		\\
		& = \sum_{\xi \in \zz} | [1- \widehat{j_\ee}(\xi] \cdot \widehat{u}(\xi) |^2
		(1+\xi^2)^r
		\\
		& = \sum_{\xi \in \zz} | [1- \widehat{j}(\ee \xi)] \cdot \widehat{u}(\xi) |^2
		(1+\xi^2)^r.
		\label{1r}
	\end{split}
\end{equation}
Assume $r \le s$. Then by construction (see \ref{0u}) we have
\begin{equation*}
	\begin{split}
		|1 - \widehat{j } (\xi) | \le |\xi|^{s-r}
	\end{split}
\end{equation*}
for all $\xi \in \rr$; hence
\begin{equation}
	\begin{split}
		|1 - \widehat{ j }(\ee \xi)| \le |\ee \xi |^{s-r}, \quad \forall
		\xi \in \rr, \ \ee > 0.
		\label{2r}
	\end{split}
\end{equation}
Applying \eqref{2r} to \eqref{1r} and recalling that $r \le s$, we obtain
\begin{equation}
	\label{2pr}
	\begin{split}
	\|u - J_\ee u\|_{H^r(\ci)}^2 
	& \le \sum_{\xi \in \zz}  |\ee \xi |^{2(s-r)}
	|\widehat{u}(\xi)|^2 (1 + \xi^2)^r
	\\
	& = \ee^{2(s-r)} \sum_{\xi \in \zz} |\widehat{u}(\xi)|^2  \cdot (\xi^2)^{s-r}
	(1 + \xi^2)^{r-s} (1 + \xi^2)^{s}
	\\
	& \le \ee^{2(s-r)}
	\sum_{\xi \in \zz} |\widehat{u}(\xi)|^2 (1 + \xi^2)^s
	\\
	& =  \ee^{2(s-r)}.
	\end{split}
\end{equation}
Furthermore,
\begin{equation*}
	\begin{split}
		& |[1- \widehat{j_\ee}(\xi)] \cdot \widehat{u}(\xi)|^2 (1 + \xi^2)^r \le
		|\widehat{u}(\xi)|^2 (1 + \xi^2)^r, \quad \ee > 0, \ \text{and}
		\\
		& \sum_{\xi \in \zz} |\widehat{u}(\xi)|^2 (1 + \xi^2)^r < \infty;
	\end{split}
\end{equation*}
therefore, by the dominated convergence theorem for series
\begin{equation}
	\label{o1}
	\begin{split}
		\lim_{\ee \to 0} \|u - J_\ee u \|_{H^r }^2 
		& = \lim_{\ee \to 0} \sum_{\xi \in \zz} |[1-\widehat{j_\ee}(\xi)]
		\widehat{u}(\xi) |^2 (1 + \xi^2)^r
		\\
		& = \lim_{\ee \to 0} \sum_{\xi \in \zz} |[1-\widehat{j}(\ee \xi)]
		\widehat{u}(\xi) |^2 (1 + \xi^2)^r
		\\
		& = \sum_{\xi \in \zz} \lim_{\ee \to 0} |[1-\widehat{j}(\ee \xi)]
		\widehat{u}(\xi) |^2 (1 + \xi^2)^r
		\\
		& = 0.
	\end{split}
\end{equation}
To complete the proof of Proposition \ref{lem4r}, we take note of the following interpolation result:
\begin{remark}
	\label{lem2r}
	For $\sigma < r \le s$ and arbitrary $u \in L^2(\ci)$,
	\begin{equation}
		\begin{split}
			\|u\|_{H^{r}(\ci)} \le
			\|u\|_{H^\sigma(\ci)}^{(r-s)/(\sigma -s)}
			\|u\|_{H^s(\ci)}^{1 - (r-s)/(\sigma -s)}.
			\label{16u}
		\end{split}
	\end{equation}
\end{remark}
%
%
%
%
{\bf Proof.} Assuming $u \in L^2(\ci)$ and $\sigma < r \le s$,
we rewrite and apply Holder's inequality:
\begin{equation*}
	\begin{split}
		&\|u\|_{H^{r}(\ci)}^2
		\\
		& = \sum_{\xi \in \zz} |\widehat{u}(\xi)|^2 (1 + \xi^2)^{r}
		\\
		& = \sum_{\xi \in \zz}
		\left [|\widehat{u}(\xi)|^2 (1 + \xi^2)^\sigma \right ]^{(r-s)/(\sigma -s)}
		\cdot \left [ |\widehat{u}(\xi )
		|^2 (1+ \xi^2)^s \right ] ^{1 - (r-s)/(\sigma -s)} 
		\\
		& \le \|\left[ |\widehat{u}(\xi)|^2 (1 + \xi^2)^\sigma
		\right]^{(r-s)/(\sigma -s)} \|_{l^{(\sigma -s)/(r-s)}(\zz)}
		\\
		& \cdot \|\left[ |\widehat{u}(\xi)|^2 (1 + \xi^2)^\sigma
		\right]^{1- (r-s)/(\sigma -s)} \|_{l^{1/[1 -(\sigma -s)/(r-s)]}(\zz)}
		\\
		& = \|v\|_{H^\sigma(\ci)}^{2(r-s)/(\sigma -s)}
		\|v\|_{H^s(\ci)}^{2[1 - (r-s)/(\sigma -s)]}
	\end{split}
\end{equation*}
from which the result follows. 
\vskip0.1in
Assume without loss of generality that $s > 0$. Applying Remark \ref{lem2r}, and estimates \eqref{2pr} and \eqref{o1}, we
see that for $r>0$ 
\begin{equation*}
	\begin{split}
		\|u - J_\ee u \|_{H^r(\ci)}
		& \le \|u - J_\ee u
		\|_{L^2(\ci)}^{(s-r)/s} \|u - J_\ee u \|_{H^s(\ci)}^{1 -
		(s-r)/s}
		\\
		& = \left( \ee^{s} \right)^{(s-r)/s} \cdot o(1)
		\\
		& = o(\ee^{s-r})
	\end{split}
\end{equation*}
Similarly, for $r < 0$
\begin{equation*}
	\begin{split}
		\|u - J_\ee u \|_{H^r(\ci)}^2
		& \le \|u - J_\ee u
		\|_{H^\sigma(\ci)}^{(r-s)/(\sigma - s)} \|u - J_\ee u \|_{H^s(\ci)}^{1 -
		(r-s)/(\sigma -s)}
		\\
		& = \left( \ee^{s-\sigma} \right)^{(r-s)/(\sigma -s)} \cdot o(1)
		\\
		& = o(\ee^{s-r})
	\end{split}
\end{equation*}
Lastly, for the case $r=0$, we note that \eqref{o1} implies $\|u - J_\ee u
\|_{H^r(\ci)} \le o(1)$ for all $r \le s$. Hence, the proof of Proposition
\ref{lem4r} is complete.  $\quad \Box$
\vskip0.1in
%
%
%
%
%
%
%
%
%
%
%
%
%
%
%
%
%
%
%
%
%
%{\bf Proof.} By \eqref{uniform_bound_for_u}, we have
%\begin{equation}
%	\begin{split}
%		\|u^\ee(t, \cdot \|_{H^r(\ci)}^2
%		& \le C \|u^\ee (0, \cdot)
%		\|_{H^r(\ci)}^2
%		\\
%		& = \|J_\ee u_0 \|_{H^r(\ci)}^2
%		\\
%		& = \sum_{\xi \in \zz} |\widehat{j_\ee}(\xi) \widehat{u_0}(\xi) |^2
%		\cdot (1 + \xi^2)^r
%		\\
%		& \le \sum_{\xi \in \zz} |[1-\widehat{ j_\ee}(\xi)] \widehat{u_0}(\xi) |^2
%		\cdot (1 + \xi^2)^r
%		\\
%		& + \sum_{\xi \in \zz} |\widehat{j_\ee}(\xi) \widehat{u_0}(\xi) |^2
%		\cdot (1 + \xi^2)^r.
%		\label{1q}
%	\end{split}
%\end{equation}
%Since $J_\ee u_0$ is smooth, by \eqref{uniform_bound_for_u} we have
%\begin{equation*}
%	\begin{split}
%		\sum_{\xi \in \zz} |\widehat{j_\ee}(\xi) \widehat{u_0}(\xi) |^2
%		\cdot (1 + \xi^2)^r
%		= \|J_\ee u_0\|_{H^r(\ci)} = C_r
%	\end{split}
%\end{equation*}
%for all $r \ge 3/2$.

%
%
%
%
%
%
%
%
\vskip0.1in
We now return to analyzing
\eqref{15u}. Applying Remark \ref{lem5r}, Remark \ref{lem2r}, and Lemma
\ref{lem6r}, we have
\begin{equation}
	\begin{split}
		\label{200x}
		\|u^\ee \|_{H^{s+1}(\ci)} \|v \|_{H^{s-1}(\ci)} \|v\|_{H^s(\ci)}
		& \le C''' \ee^{-1} \cdot \|v\|_{H^\sigma(\ci)}^{1/(s-\sigma)}
		\|v\|_{H^s(\ci)}^{2 - 1/(s- \sigma)}
		\\
		& \le C'' \ee^{-1} \cdot o(\ee^{s- \sigma})^{1/(s-\sigma)}
		\|v\|_{H^s(\ci)}^{2- 1/(s-\sigma)}
		\\
		& \le C'' \cdot o(1) \cdot \|v\|_{H^s(\ci)}^{2- 1/(s-\sigma)}
	\end{split}
\end{equation}
where we stress that $C'' = C''(\|u_0\|_{H^s(\ci)})$ does not depend on $\ee$.
Hence, $\|v\|_{H^{s-1}(\ci)}$ has proved to be sufficient to control the
growth of $\|u^\ee \|_{H^{s+1}(\ci)}$. For the remaining terms of
\eqref{15u}, we leave $\|v\|_{H^s(\ci)}^3$ as is, and note that by Remark \ref{lem5r}
\begin{equation}
	\begin{split}
		\|u^\ee\|_{H^s(\ci)} \|v\|_{H^s(\ci)}^2 \le C'
		\cdot \|v\|_{H^s(\ci)}^2
		\label{u-ep-bound}
	\end{split}
\end{equation}
where $C' = C'(\|u_0\|_{H^s(\ci)})$. Hence, applying \eqref{u-ep-bound} and \eqref{200x} to \eqref{15u}, we obtain
\begin{equation}
	\begin{split}
		\frac{1}{2} \frac{d}{dt} \|v\|_{H^s(\ci)}^2 \le C (
		\|v\|_{H^s(\ci)}^3 + \|v\|_{H^s(\ci)}^2 + \ee ^{-1} o(\ee) \|v\|_{H^s(\ci)}^{2-
		1/(s- \sigma)}).
		\label{201x}
	\end{split}
\end{equation}
where $C=C(\|u_0\|_{H^s(\ci)}$ does not depend on $\ee$. We also remark 
that $\|v(t)\|_{H^s(\ci)}$ is uniformly bounded for all $\ee > 0$, since by
the triangle inequality, Proposition \ref{lem3r}, and part (iii) of Theorem
\ref{thm:HR_existence_continous_dependence} we have
\begin{equation*}
	\begin{split}
		\|v(t) \|_{H^s(\ci)}
		& = \|u - u^\ee \|_{H^s(\ci)}
		\\
		& \le \|u \|_{H^s(\ci)} + \|u^\ee \|_{H^s(\ci)}
		\\
		& \le 2( \|u_0\|_{H^s(\ci)} + \|J_\ee u_0\|_{H^s(\ci)})
		\\
		& \le 4 \|u_0\|_{H^s(\ci)}.
	\end{split}
\end{equation*}
Hence, \eqref{201x} gives
\begin{equation}
	\begin{split}
		\lim_{\ee \to 0} \frac{1}{2} \frac{d}{dt} \|v\|_{H^s(\ci)}^2 \le
		 \lim_{\ee \to 0} C (
		\|v\|_{H^s(\ci)}^3 + \|v\|_{H^s(\ci)}^2).
		\label{202x}
	\end{split}
\end{equation}
Differentiating the left hand side of
\eqref{202x} and simplifying, it follows that
\begin{equation*}
	\begin{split}
		\lim_{\ee \to 0}\frac{d}{dt} \|v\|_{H^s(\ci)} \le
		\lim_{\ee \to 0} C (\|v\|_{H^s(\ci)}^2 +
		\|v\|_{H^s(\ci)}).
	\end{split}
\end{equation*}
Letting $y = \|v\|_{H^s(\ci)}$ and rearranging, we obtain
\begin{equation*}
	\begin{split}
		\lim_{\ee \to 0} \ \frac{1}{y(y+1)} \frac{dy}{dt} \le C	
	\end{split}
\end{equation*}
which can be rewritten as
\begin{equation*}
	\begin{split}
		\lim_{\ee \to 0}
		\left( \frac{1}{y} - \frac{1}{y+1} \right)\frac{dy}{dt} \le C
	\end{split}
\end{equation*}
implying
\begin{equation}
	\label{est-int'}
	\begin{split}
		\lim_{\ee \to 0} \left [
\int_0^t \frac{1}{y} \frac{dy}{d \tau} \ d \tau
		- \int_0^t \frac{1}{y+1} \frac{dy}{d \tau} \ d \tau \right ]
		\le \int_0^t C \ d \tau, \quad |t| \le T.
	\end{split}
\end{equation}
Hence \eqref{est-int'} gives 
\begin{equation}
	\begin{split}
	\lim_{\ee \to 0} 
	\left [ \ln \left | \frac{y(t)}{y(0)}
	\cdot \frac{y(0) + 1}{y(t) + 1} \right | \right ] \le C t.
		\label{301''qx}
	\end{split}
\end{equation}
Exponentiating both sides of \eqref{301''qx}, and noting that $f(x) = e^x$
is a continuous function on $\rr$, we must have
\begin{equation*}
	\begin{split}
		\lim_{\ee \to 0}  \
		\left | \frac{y(t)}{y(0)} \cdot \frac{y(0) + 1}{y(t) + 1} \right | \le e^{C t}.
	\end{split}
\end{equation*}
Rearranging, and recalling that $y(t) = \|v(t)\|_{H^s(\ci)} \ge 0$, we obtain
\begin{equation*}
	\begin{split}
		\lim_{\ee \to 0} \frac{y(t)}{y(t) + 1}
		\le \lim_{\ee \to 0} \frac{e^{C t} \cdot y(0)}{y(0) + 1} \le
		\lim_{\ee \to 0} e^{C t} \cdot y(0).
	\end{split}
\end{equation*}
Substituting back in $\|v(t)\|_{H^s(\ci)}$ for $y(t)$ gives
\begin{equation}
	\begin{split}
		\lim_{\ee \to 0}	\frac{\|v(t)\|_{H^s(\ci)}}{\|v(t)\|_{H^s(\ci)} + 1}  \le
		\lim_{\ee \to 0} e^{C t} \cdot \|v(0)\|_{H^s(\ci)}.
		\label{303'qx}
	\end{split}
\end{equation}
Since 
\begin{equation*}
	\label{303''qx}
	\begin{split}
		\|v(0)\|_{H^s(\ci)} = \|u_0 - J_\ee u_0 \|_{H^s(\ci)} \le
		\|u_0\|_{H^s(\ci)} \cdot o(1)
	\end{split}
\end{equation*}
by Proposition \ref{lem4r}, we conclude from \eqref{303'qx} that
\begin{equation}
	\label{304qx}
	\begin{split}
		\lim_{\ee \to 0} \|v(t)\|_{H^s(\ci)} = \lim_{\ee \to 0}
		\|u^\ee(t) - u(t)\|_{H^s(\ci)}= 0, \qquad |t| \le T,
	\end{split}
\end{equation}
and since the family $\left\{ u^\ee - u \right\}_\ee$ does not depend on $n$,
the proof of \eqref{enough_to_prove1} is complete. 
\vskip0.1in
%
%
%
%
%
%
%
%
\vskip0.1in
{\bf Proof of \eqref{enough_to_prove2}.} 
Let $v = u^\ee_n - u^\ee$. Then $v$ solves the Cauchy problem
\begin{align}
		\label{4qu}
		\p_t v 
		& =  -\gamma (v \p_x v + v \p_x u^\ee + u^\ee \p_x v)  
		\\
		& - D^{-2} \p_x \left\{ \left (\frac{3-\gamma}{2} \right )(v^2 +
		2u^\ee v) + \frac{\gamma}{2}\left[ (\p_x v)^2 + 2 \p_x u^\ee \p_x v \right]
		\right\}, \notag
		\\
		& v(0) =J_\ee(u_{0,n} - u_0).
		\label{5qu}
	\end{align}
Applying the operator $D^s$ to both sides of \eqref{4qu}, multiplying by
	$D^s$ and integrating, we have
	\begin{equation}
		\begin{split}
			\frac{1}{2}\frac{d}{dt} \|v\|_{H^s(\ci)} = A + B
			\label{6qu}
		\end{split}
	\end{equation}
	where
	\begin{equation}
		\begin{split}
			A
			& =  -\gamma \int_{\ci} D^s(v \p_x v) \cdot D^s v \
			dx
			- \frac{3- \gamma}{2} \int_\ci D^{s-2} \p_x (v^2) \cdot D^s v
			\ dx
			\\
			& - \frac{\gamma}{2}\int_\ci D^{s-2} \p_x (\p_x v)^2 \cdot D^s
			v \ dx
			\label{7qu}
		\end{split}
	\end{equation}
	and
	\begin{equation}
		\begin{split}
			B 
			 = &  \overbrace{-\gamma \int_\ci D^s (u^\ee \p_x v) \cdot D^s v \
			 dx}^{(i)}
			 \ \overbrace{-\gamma \int_\ci D^s (v \p_x u^\ee ) \cdot D^s v \
			 dx}^{(ii)}
			 \\
			  & \overbrace{- \ ( 3- \gamma) \int_\ci D^{s-2} \p_x (u^\ee v) \cdot D^s
			 v \ dx}^{(iii)}
			 \\
			 & \overbrace{-\gamma \int_\ci D^{s-2} \p_x
			(\p_x u^\ee \cdot \p_x v) \cdot D^s v \
			dx}^{(iv)}.
			\label{8qu}
		\end{split}
	\end{equation}
	Estimating as in \eqref{8'u}-\eqref{14u}, we obtain
	\begin{equation}
		\begin{split}
			\frac{1}{2}\frac{d}{dt}\|v\|_{H^{s}(\ci)}^2
			& \le c_s(\|v\|_{H^s(\ci)}^3 + \|u^\ee\|_{H^s(\ci)}
			\|v\|_{H^s(\ci)}^2
			\\
			& + \|u^\ee\|_{H^{s+1}(\ci)}
			\|v\|_{H^{s-1}(\ci)} \|v\|_{H^s(\ci)}).
			\label{15qu}
		\end{split}
	\end{equation}
	We now aim to control the growth of $\|u^\ee\|_{H^{s+1}(\ci)}$ by
	$\|v\|_{H^{s-1}(\ci)}$. To do so, we will need an estimate for
	$\|v\|_{H^{s-1}(\ci)}$, which we will obtain through the following lemma:
%
%
%
%
\begin{lemma}
	\label{lem:left}
	For $\sigma$ such that $1/2 < \sigma < 1$ and $\sigma + 1 \le s$, we have
	\begin{equation}
	\label{lem6rq}
	\begin{split}
		\|v\|_{H^{\sigma}(\ci)} = 
		\|u^\ee_n - u^\ee\|_{H^\sigma(\ci)}
		\le C \cdot o(\ee^{s- \sigma }) + \|u_0 - u_{0,n} \|_{H^s(\ci)}, \qquad |t| \le T
	\end{split}
\end{equation}
where $C=C(\|u_0\|_{H^s(\ci)})$.
\end{lemma}
%
%
%
{\bf Proof.}
Repeating calculations \eqref{X}-\eqref{12}, with $E$ set to zero, $u^{\omega,n}$
replaced by $u^\ee_n$, $u_{\omega,n}$ replaced by $u^\ee$, and $\sigma$ and $\rho$ chosen such that
\begin{equation}
	\label{size_of_sigma'}
	\begin{split}
	& 1/2 < \sigma < 1,
	\\
	& \sigma + 1 \le \rho \le s 
	\end{split}
\end{equation}
yields
 \begin{equation*}
	\begin{split}
		\frac{1}{2}\frac{d}{dt} \|v\|_{H^\sigma(\ci)}^2
		& \le
		C'' (\|u^{\ee}_n + u^\ee \|_{H^{\rho}(\ci)} +
		\|\p_x(u^{\ee}_n + u^\ee) \|_{H^\sigma(\ci)})
		\cdot \|v\|_{H^\sigma(\ci)}^2.
	\end{split}
\end{equation*}
\medskip
It follows that 
\begin{equation}
	\begin{split}
		\frac{1}{2}\frac{d}{dt} \|v\|_{H^{\sigma}(\ci)}^2
		& \le
		C'' \cdot \|u^{\ee}_n
		+ u^\ee\|_{H^{s}(\ci)}\cdot \|v\|_{H^{\sigma}(\ci)}^2.
		\label{10qx}
	\end{split}
\end{equation}
Applying the triangle inequality and
part (iii) of Theorem \ref{thm:HR_existence_continous_dependence} (proved in
Section \ref{existence})
to \eqref{10qx} yields
%
\begin{equation}
	\begin{split}
		\label{11qx}
		\frac{1}{2}\frac{d}{dt} \|v\|_{H^{\sigma}(\ci)}^2
		& \le
		C' (\|u^{\ee}_n(0)\|_{H^{s}(\ci)}
		+ \|u^\ee(0)\|_{H^{s}(\ci)})\cdot \|v\|_{H^{\sigma}(\ci)}^2
		\\
		& = C' (\|J_\ee u_{0,n}\|_{H^{s}(\ci)}
		+ \|J_\ee u_0\|_{H^{s}(\ci)})\cdot \|v\|_{H^{\sigma}(\ci)}^2.
	\end{split}
\end{equation}
Note that the family $\left\{ u_{0,n} \right\}_n$ is uniformly bounded in
$H^s(\ci)$. Hence, applying Lemma \ref{lem3r} to \eqref{11qx} we obtain the critical estimate 
\begin{equation}
	\begin{split}
		\label{12qx}
		\frac{1}{2}\frac{d}{dt} \|v\|_{H^{\sigma}(\ci)}^2
		& \le
	C \|v\|_{H^{\sigma}(\ci)}^2
\end{split}
\end{equation}
with $C = C(\|u_0\|_{H^s(\ci)}, \ R)$, where
\begin{equation}
	\label{r-def}
	R = \inf \left\{ R' \in \rr:\ \{u_{0,n}\} \subset B_{H^s(\ci)}(R',0)
	\right\}.
\end{equation}
Differentiating the left hand side of \eqref{12qx} and simplifying, we
obtain
\begin{equation}
	\begin{split}
		\frac{d}{dt}\|v\|_{H^{\sigma}(\ci)} \le C \|v\|_{H^{\sigma}(\ci)}.
		\label{100qx}
	\end{split}
\end{equation}
Let $y(t) = \|v\|_{H^{\sigma}(\ci)}$. Then \eqref{100qx} gives
\begin{equation*}
	\begin{split}
		\frac{1}{y(t)}\frac{dy}{dt} \le C.
	\end{split}
\end{equation*}
Hence,
\begin{equation*}
	\begin{split}
		\int_0^t \frac{1}{y(\tau)} \frac{dy}{d \tau}
		\le \int_0^t C \ d \tau, \qquad |t| \le T
	\end{split}
\end{equation*}
from which we obtain
\begin{equation}
	\begin{split}
		\ln |y(t) | - \ln |y(0)| \le C t.
		\label{101qx}
	\end{split}
\end{equation}
Simplifying \eqref{101qx}, we have
\begin{equation*}
	\begin{split}
		\ln \left |\frac{y(t)}{y(0)} \right | \le C t
	\end{split}
\end{equation*}
which yields the estimate
\begin{equation*}
	\begin{split}
		y(t) \le y(0) e^{C t}, \qquad |t| \le T.
	\end{split}
\end{equation*}
Substituting back in $\|v\|_{H^{\sigma}(\ci)}$ for $y$, we get
\begin{equation*}
	\begin{split}
		\|v\|_{H^{\sigma}(\ci)}
		& \le e^{C t}\|v(0)\|_{H^{\sigma}(\ci)}
		\\
		& = e^{C t}\|u^\ee(0) - u^\ee_n(0) \|_{H^{\sigma}(\ci)}.
	\end{split}
\end{equation*}
To conclude the proof, we apply the following:
\begin{proposition}
		\label{lem11r}
	For $r \le s$,
	\begin{equation}
		\begin{split}
			\|u^\ee(0) - u_n^\ee (0) \|_{H^r(\ci)} \le C
			\cdot o(\ee^{s-r}) + \|u_0 - u_{0,n} \|_{H^s(\ci)}
			\label{3w}
		\end{split}
	\end{equation}
	where $C=C(\|u_0\|_{H^s(\ci)})$ does not depend on $n$.
\end{proposition}
%
%
Recalling \eqref{r-def}, we deduce by Proposition \ref{lem11r}
\begin{equation*}
	\begin{split}
		\|v\|_{H^{\sigma}(\ci)} \le C \cdot o(\ee^{s - \sigma}) + \|u_0 -
		u_{0,n} \|_{H^s(\ci)} \qquad |t| \le T
	\end{split}
\end{equation*}
where $C=C(\|u_0\|_{H^s(\ci)}), \ R)$ does not depend
on $n$, completing the proof of Lemma \ref{lem:left}. $\qquad \Box$
%
\vskip0.1in
%
{\bf Proof of Proposition \ref{lem11r}.} We write
\medskip
\begin{equation}
	\begin{split}
		\|u^\ee(0) - u_n^\ee (0) \|_{H^r(\ci)} 
		& = \|J_\ee u_0 - J_\ee u_{0,n} \|_{H^r(\ci)}
		\\
		& \le \|J_\ee u_0 - u_0 \|_{H^r(\ci)} + \| u_0 - u_{0,n}
		\|_{H^r(\ci)}
		\\
		& + \|u_{0,n} - J_\ee u_{0,n} \|_{H^r(\ci)}
		\\
		& \le \|I - J_\ee\|_{L(H^s(\ci), H^r(\ci))} \|u_0\|_{H^s(\ci)}
		\\
		& +
		\|u_0 - u_{0,n} \|_{H^r(\ci)} + 
		\|I - J_\ee\|_{L(H^s(\ci), H^r(\ci))} \|u_{0,n}\|_{H^s(\ci)}.
		\label{4w}
	\end{split}
\end{equation}
Applying Proposition \ref{lem4r} to \eqref{4w}, and recalling that the family
$\left\{ u_{0,n} \right\}_n$ belongs to a bounded subset of
$H^s(\ci)$, we have
\medskip
\begin{equation}
	\label{finito}
	\begin{split}
		\|u^\ee(0) - u_n^\ee (0) \|_{H^r(\ci)} 
		& \le
		C' \cdot o(\ee^{s-r}) \cdot \|u_0\|_{H^s(\ci)}
		 \\
		 & + \|u_0 - u_{0,n} \|_{H^r(\ci)} + C' \cdot o (\ee^{s-r}) \cdot
		 \|u_{0,n}\|_{H^s(\ci)}
		 \\
		 & \le
		 C' \cdot o(\ee^{s-r}) \cdot \|u_0\|_{H^s(\ci)}
		 \\
		 & + \|u_0 - u_{0,n} \|_{H^s(\ci)} + C' \cdot o (\ee^{s-r}) \cdot
		 R
	\end{split}
\end{equation}
where $R$ is defined as in \ref{r-def}. The result follows immediately from
\eqref{finito}. $\qquad \Box$
\vskip0.1in
We are now prepared to interpolate. Recall \eqref{15qu}. Applying Remark \ref{lem5r}, Remark \ref{lem2r}, and
Proposition \ref{lem11r} gives
\begin{equation*}
	\begin{split}
		& \|u^\ee \|_{H^{s+1}(\ci)} \|v\|_{H^{s-1}(\ci)} \|v\|_{H^s
		(\ci)}
		\\
		&\le C' \ee^{-1} \cdot \|v\|_{H^\sigma(\ci)}^{1/(s-\sigma)}
		\|v\|_{H^s(\ci)}^{2 - 1/(s- \sigma)}
		\\
		& \le C' \ee^{-1} \cdot \Big [C \cdot o(\ee^{s- \sigma}) + \|u_0 -
		u_{0,n}\|_{H^s(\ci)} \Big ]^{1/(s-\sigma)}
		\cdot \|v\|_{H^s(\ci)}^{2- 1/(s-\sigma)}
	\end{split}
\end{equation*}
from which we obtain
\begin{equation}
	\begin{split}
		\label{200qx}
		\|u^\ee\|_{H^{s+1}(\ci)} \|v\|_{H^{s-1}(\ci)} \|v \|_{H^s(\ci)}
		& \lesssim  o(1) + \ee^{-1}
		\|u_0-u_{0,n}\|_{H^s(\ci)}^{1/(s-\sigma)}\|v\|_{H^s(\ci)}^{2- 1/(s-\sigma)}.
	\end{split}
\end{equation}
We wish to control the growth of the second term of the
right hand side of \eqref{200qx}.
First, note that the triangle inequality, part (iii) of Theorem
\ref{thm:HR_existence_continous_dependence} and Proposition \ref{lem3r} imply
\begin{equation}
	\begin{split}
		\|v\|_{H^s(\ci)} & = \|u^\ee_n - u^\ee \|_{H^s(\ci)} 
		\\
		& \le \|u^\ee_n\|_{H^s(\ci)} + \|u^\ee \|_{H^s(\ci)}  
		\\
		& \le 2\left[  \|J_\ee u_{0,n}\|_{H^s(\ci)} + \|J_\ee u_0 \|_{H^s(\ci)} 
		 \right]
		\\
		& \le 2 \left[ \|u_{0,n} \|_{H^s(\ci)} + \|u_0 \|_{H^s(\ci)} 
		\right], \qquad |t| \le T
		\label{growth_v}
	\end{split}
\end{equation}
and since $\{u_{0,n}\}_n$ belongs to a bounded subset of
$H^s(\ci)$, we see from \eqref{growth_v} that $\|v \|_{H^s(\ci)}$ is
uniformly bounded in $n$ \emph{and} $\ee$.  Secondly, since $\|u_0 -
u_{0,n} \|_{H^s(\ci)} \to 0$ uniformly in $n$, then for any given $\ee$ we
can chose a family $\{N_j\} $ such that
\begin{equation}
	\begin{split}
		\|u_0 - u_{0,n} \|_{H^s(\ci)} \lesssim
		\frac{\ee^{(s-\sigma)}}{2^{j(s -\sigma)}}, \quad n >
		N_j.
		\label{uniform_n}
	\end{split}
\end{equation}
Thirdly, by Remark \ref{lem5r}, we have 
\begin{equation}
	\label{u-ee-bound}
	\|u^\ee \|_{H^s(\ci)} \le C(\|u_0\|_{H^s(\ci)}), \quad \forall \ee > 0.
\end{equation}
Applying \eqref{200qx} to \eqref{15qu} in light of 
\eqref{growth_v}, \eqref{uniform_n}, and \eqref{u-ee-bound}, we obtain
\begin{equation*}
		\begin{split}
			\lim_{n \to \infty }
			\frac{1}{2}\frac{d}{dt}\|v\|_{H^{s}(\ci)}^2
			& \le
			C \lim_{n \to \infty} \Big [\|v\|_{H^s(\ci)}^3 +
			\|v\|_{H^s(\ci)}^2 + o(1)\Big ]
		\end{split}
	\end{equation*}
	for every $\ee > 0$, where $C = C(\|u_0\|_{H^s(\ci)}, \ R)$ with
	$R$ defined as in \eqref{r-def}; hence we have
\begin{equation}
		\begin{split}
			\lim_{\substack{n \to \infty \\ \ee \to 0} }
			\frac{1}{2}\frac{d}{dt}\|v\|_{H^{s}(\ci)}^2
			& \le C
			\lim_{\substack{n \to \infty \\ \ee \to 0}}
			\Big [\|v\|_{H^s(\ci)}^3 + 
			\|v\|_{H^s(\ci)}^2 \Big ].
			\label{15qx}
		\end{split}
	\end{equation}
	We differentiate the left hand side of \eqref{15qx} and obtain
\begin{equation*}
	\begin{split}
		\lim_{\substack{n \to \infty \\ \ee \to 0}}\frac{d}{dt}
		\|v\|_{H^s(\ci)} \le C
		\lim_{\substack{n \to \infty \\ \ee \to 0}} \left [\|v\|_{H^s(\ci)}^2 +
		\|v\|_{H^s(\ci)} \right ].
	\end{split}
\end{equation*}
Letting $y = \|v\|_{H^s(\ci)}$ and rearranging gives
\begin{equation*}
	\begin{split}
		\lim_{\substack{n \to \infty \\ \ee \to 0} } \ \frac{1}{y(y+1)} \frac{dy}{dt}
		\le	C
	\end{split}
\end{equation*}
which can be rewritten as
\begin{equation*}
	\begin{split}
		\lim_{\substack{n \to \infty \\ \ee \to 0} }
		\left( \frac{1}{y} - \frac{1}{y+1} \right)\frac{dy}{dt} \le C 
	\end{split}
\end{equation*}
implying
\begin{equation}
	\label{est-int}
	\begin{split}
		\lim_{\substack{n \to \infty \\ \ee \to 0} } \left [
\int_0^t \frac{1}{y} \frac{dy}{d \tau} \ d \tau
		- \int_0^t \frac{1}{y+1} \frac{dy}{d \tau} \ d \tau \right ]
		\le \int_0^t C \ d \tau, \quad |t| \le T.
	\end{split}
\end{equation}
Recalling that $y(t) = \|v(t)\|_{H^s(\ci)} > 0$, \eqref{est-int} gives 
\begin{equation}
	\begin{split}
	\lim_{\substack{n \to \infty \\ \ee \to 0} }
	\left [ \ln \left ( \frac{y(t)}{y(0)}
	\cdot \frac{y(0) + 1}{y(t) + 1} \right ) \right ] \le C t.
		\label{301'qx}
	\end{split}
\end{equation}
Exponentiating both sides of \eqref{301'qx}, and noting that $f(x) = e^x$
is a continuous function on $\rr$, we must have
\begin{equation*}
	\begin{split}
		\lim_{\substack{n \to \infty \\ \ee \to 0} } \
		\frac{y(t)}{y(0)} \cdot \frac{y(0) + 1}{y(t) + 1} \le e^{C t}.
	\end{split}
\end{equation*}
Rearranging, we obtain
\begin{equation*}
	\begin{split}
		\lim_{\substack{n \to \infty \\ \ee \to 0}} \frac{y(t)}{y(t) + 1}
		\le \lim_{\substack{n \to \infty \\ \ee \to 0}} \frac{e^{C t} \cdot y(0)}{y(0) + 1} \le
		\lim_{\substack{n \to \infty \\ \ee \to 0}} e^{C t} \cdot y(0).
	\end{split}
\end{equation*}
Substituting back in $\|v(t)\|_{H^s(\ci)}$ for $y(t)$ gives
\begin{equation}
	\begin{split}
		\lim_{\substack{n \to \infty \\ \ee \to 0}}	\frac{\|v(t)\|_{H^s(\ci)}}{\|v(t)\|_{H^s(\ci)} + 1}  \le
		\lim_{\substack{n \to \infty \\ \ee \to 0}} e^{C t} \cdot \|v(0)\|_{H^s(\ci)}.
		\label{303qx}
	\end{split}
\end{equation}
Since by Proposition \ref{lem3r} 
\begin{equation*}
	\begin{split}
	\lim_{\substack{n \to \infty \\ \ee \to 0} }
	\|v(0)\|_{H^s(\ci)}
	& = \lim_{\substack{n \to \infty \\ \ee \to 0} }
	\|J_\ee u_{0,n} - J_\ee u_0 \|_{H^s(\ci)} 
	\\
	& \le \lim_{n \to \infty } \|u_{0,n} - u_0 \|_{H^s(\ci)}
	\\
	& = 0
	\end{split}
\end{equation*}
we deduce from \eqref{303qx} that
\begin{equation*}
	\begin{split}
		\lim_{\substack{n \to \infty \\ \ee \to 0}} \|v(t)\|_{H^s(\ci)} = 0, \qquad |t| \le T
	\end{split}
\end{equation*}
completing the proof of \eqref{enough_to_prove2}. $\quad \Box$
%
%
%
\vskip0.1in
{\bf Proof of \eqref{enough_to_prove3}.} 
Let $v = u_n - u^\ee_n$. Then $v$ solves the Cauchy problem
\begin{align}
		\label{a4qu}
		\p_t v 
		& =  -\gamma (v \p_x v + v \p_x u^\ee_n + u^\ee_n \p_x v)  
		\\
		& - D^{-2} \p_x \left\{ \left (\frac{3-\gamma}{2} \right )(v^2 +
		2u^\ee_n v) + \frac{\gamma}{2}\left[ (\p_x v)^2 + 2 \p_x u^\ee_n \p_x v \right]
		\right\}, \notag
		\\
		& v(0) = (I- J_\ee)u_{0,n}.
		\label{a5qu}
	\end{align}
Applying the operator $D^s$ to both sides of \eqref{a4qu}, multiplying by
	$D^s$ and integrating, we have
	\begin{equation}
		\begin{split}
			\frac{1}{2}\frac{d}{dt} \|v\|_{H^s(\ci)} = A + B
			\label{a6qu}
		\end{split}
	\end{equation}
	where
	\begin{equation}
		\begin{split}
			A
			& =  -\gamma \int_{\ci} D^s(v \p_x v) \cdot D^s v \
			dx
			- \frac{3- \gamma}{2} \int_\ci D^{s-2} \p_x (v^2) \cdot D^s v
			\ dx
			\\
			& - \frac{\gamma}{2}\int_\ci D^{s-2} \p_x (\p_x v)^2 \cdot D^s
			v \ dx
			\label{a7qu}
		\end{split}
	\end{equation}
	and
	\begin{equation}
		\begin{split}
			B 
			 = &  \overbrace{-\gamma \int_\ci D^s (u^\ee_n \p_x v) \cdot D^s v \
			 dx}^{(i)}
			 \ \overbrace{-\gamma \int_\ci D^s (v \p_x u^\ee_n ) \cdot D^s v \
			 dx}^{(ii)}
			 \\
			  & \overbrace{- \ ( 3- \gamma) \int_\ci D^{s-2} \p_x (u^\ee_n v) \cdot D^s
			 v \ dx}^{(iii)}
			 \\
			 & \overbrace{-\gamma \int_\ci D^{s-2} \p_x
			(\p_x u^\ee_n \cdot \p_x v) \cdot D^s v \
			dx}^{(iv)}.
			\label{a8qu}
		\end{split}
	\end{equation}
	Estimating as in \eqref{8'u}-\eqref{14u}, we obtain
	\begin{equation}
		\begin{split}
			\frac{1}{2}\frac{d}{dt}\|v\|_{H^{s}(\ci)}^2
			& \le c_s(\|v\|_{H^s(\ci)}^3 + \|u^\ee_n\|_{H^s(\ci)}
			\|v\|_{H^s(\ci)}^2
			\\
			& + \|u^\ee_n\|_{H^{s+1}(\ci)}
			\|v\|_{H^{s-1}(\ci)} \|v\|_{H^s(\ci)}).
			\label{a15qu}
		\end{split}
	\end{equation}
	Note that the first two terms in parentheses on the right hand side
	of \eqref{a15qu} will offer us little trouble;
	it is the third term that requires special care (due to the
	$\|u^\ee_n\|_{H^{s+1}(\ci)}$ factor, which becomes increasingly large as
	$\ee$ decreases). More precisely:
	%
	%
	%
	\begin{remark}
	\label{lem5r'}
	For $r \ge s > 3/2$ and $0 < \ee <<1$ 
	\begin{equation}
		\begin{split}
			\|u^\ee_n (t, \cdot) \|_{H^r(\ci)} \le C \, \ee^{s-r}
			\label{700r'}
		\end{split}
	\end{equation}
	for all $n \in \mathbb{N}$, with $C = C(r, R)$, where $R$ is defined as
	in \eqref{r-def}.
\end{remark}
{\bf Proof.} By part (iii) of Theorem
\ref{thm:HR_existence_continous_dependence}, proved in Section
\ref{existence}, we have
\begin{equation}
	\begin{split}
		\|u^\ee_n \|_{H^r(\ci)}^2
		& \le C' \|u^\ee_n (0)\|_{H^r(\ci)}^2
		\\
		& = C' \|J_\ee u_{0,n}\|_{H^r(\ci)}^2
		\\
		& = C' \sum_{\xi \in \zz} |\widehat{j_\ee} (\xi) \widehat{u_{0,n}}(\xi)
		|^2 \cdot (1 + \xi^2)^r
		\\
		& = C' \sum_{\xi \in \zz} |\widehat{j }(\ee \xi)|^2 \cdot (1 +
		\xi^2)^{r-s} \cdot |\widehat{u_{0,n}}(\xi)|^2 \cdot (1 + \xi^2)^s
		\\
		& = C'|\widehat{u_{0,n}}(0)|^2 +
		C' \sum_{\xi \in \zz \setminus {0}} |\widehat{j }(\ee \xi)|^2 \cdot (1 +
		\xi^2)^{r-s} \cdot |\widehat{u_{0,n}}(\xi)|^2 \cdot (1 + \xi^2)^s.
		\label{1qr'}
	\end{split}
\end{equation}
Assume $r \ge s$. Since $\widehat{j }(\xi) \in \mathcal{S}(\rr)$, 
\begin{equation}
	\label{schwartz'}
	\begin{split}
		|\widehat{j }(\ee \xi)| \le c_r |\ee \xi |^{s-r}, \quad \xi \neq 0.
	\end{split}
\end{equation}
Applying \eqref{schwartz'} to \eqref{1qr'}, we obtain
\begin{equation}
	\label{calc_ue'}
	\begin{split}
		\|u^\ee_n \|_{H^r(\ci)}^2 
		& \le C' |\widehat{u_{0,n}}(0) |^2 + c_r \sum_{\xi \in \zz \setminus
		{0}} |\ee \xi |^{2(s-r)} \cdot (1 + \xi^2)^{r-s}
		|\widehat{u_{0,n}}(\xi) |^2 \cdot (1 + \xi^2)^s
		\\
		& \le C' |\widehat{u_{0,n}}(0) |^2 + 2^{r-s} c_r \ee^{2(s-r)}
		\sum_{\xi \in \zz \setminus {0}} |\widehat{u_{0,n}}(\xi)|^2 \cdot (1 +
		\xi^2)^s
		\\
		& \le C' \|u_{0,n}\|_{H^s(\ci)}^2 + 2^{r-s} c_r \ee^{2(s-r)}
		\|u_{0,n}\|_{H^s(\ci)}^2
		\\
		& = (C' + 2^{r-s} c_r \ee^{2(s-r)}) \cdot \|u_{0,n}\|^2_{H^s(\ci)}.
	\end{split}
\end{equation}
Assuming $0 < \ee <<1$, and noting that
\begin{equation*}
	\begin{split}
		\|u_{0,n}\|_{H^s(\ci)} \le R, \quad \forall n \in \mathbb{N}
	\end{split}
\end{equation*}
where $R$ is defined as in \eqref{r-def},
we conclude from \eqref{calc_ue'} that 
\begin{equation*}
	\begin{split}
		\|u^\ee_n\|_{H^s(\ci)} \le C \ee^{s-r}
	\end{split}
\end{equation*}
where $C = C(r, R)$. $\qquad \Box$
%
\vskip0.1in
In light of Remark \ref{lem5r'}, we now aim to control the growth of
$\|u^\ee_n\|_{H^{s+1}(\ci)}$ by $\|v\|_{H^{s-1}(\ci)}$. As before, we will
first obtain an estimate for $\|v\|_{H^\sigma(\ci)}$ for suitably chosen
$\sigma < s-1$. Then, we will use this estimate to interpolate between
$\|v\|_{H^\sigma(\ci)}$ and $\|v\|_{H^s(\ci)}$, yielding an estimate for
$\|v\|_{H^{s-1}(\ci)}$ which will allow us to control the growth of
$\|u^\ee_n\|_{H^{s+1}(\ci)}$. 
%
%
%
%
\begin{proposition}
	\label{prop:180}
If $\sigma$ is chosen appropriately in the range $1/2 < \sigma < 1$ and
$\sigma + 1 < s$, then for all $n \in \mathbb{N}$ 
	\begin{equation}
	\label{alem6rq}
	\begin{split}
		\|v\|_{H^{\sigma}(\ci)} = 
		\|u_n - u^\ee_n\|_{H^\sigma(\ci)}
		\le C \cdot o(\ee^{s- \sigma }), \qquad |t| \le T
	\end{split}
\end{equation}
with $C = C(R)$, where $R$ is defined as in \eqref{r-def}.
\end{proposition}
%
%
%
{\bf Proof.}
Recall that $v$ solves the Cauchy-problem \eqref{a4qu}-\eqref{a5qu}.
Applying $D^\sigma$ to both sides of \eqref{a4qu}, multiplying by
$D^\sigma v$, and integrating, we obtain the
relation
\begin{equation*}
	\begin{split}
		\frac{1}{2}\frac{d}{dt}\|v(t)\|_{H^\sigma(\ci)}^2
		= & - \frac{\gamma}{2}\int_{\ci} D^\sigma
		\p_x \left[ \left( u_n + u^\ee_n \right)v
		\right]\cdot D^\sigma v \ dx
		\\
		& - \frac{3-\gamma}{2}\int_{\ci} D^{\sigma
		-2} \p_x \left[ \left( u_n + u^\ee_n
		\right)v \right] \cdot D^\sigma v \ dx
		\\
		& - \frac{\gamma}{2}\int_{\ci} D^{\sigma
		-2}
		\p_x \left[ \left( \p_x u_n + \p_x u^\ee_n
		\right)\cdot \p_x v \right] \cdot
		D^\sigma v \ dx.
	\end{split}
\end{equation*}
Repeating calculations \eqref{X}-\eqref{12}, with $E$ set to zero,
$u^{\omega,n}$ replaced by $u$, $u_{\omega,n}$ replaced by $u^\ee$, and
$\sigma$ and $\rho$ chosen such that
%
\begin{equation}
	\begin{split}
	& 1/2 < \sigma < 1,
	\\
	& \sigma + 1 \le \rho \le s 
	\end{split}
\end{equation}
yields
 \begin{equation*}
	\begin{split}
		\frac{1}{2}\frac{d}{dt} \|v\|_{H^\sigma(\ci)}^2
		& \le
		C'' (\|u_n + u^\ee_n \|_{H^{\rho}(\ci)} +
		\|\p_x(u_n + u^\ee_n) \|_{H^\sigma(\ci)})
		\cdot \|v\|_{H^\sigma(\ci)}^2.
	\end{split}
\end{equation*}
\medskip
It follows that 
\begin{equation}
	\begin{split}
		\frac{1}{2}\frac{d}{dt} \|v\|_{H^{\sigma}(\ci)}^2
		& \le
		C'' \cdot \|u_n
		+ u^\ee_n\|_{H^{s}(\ci)}\cdot \|v\|_{H^{\sigma}(\ci)}^2.
		\label{a10qx}
	\end{split}
\end{equation}
Applying the triangle inequality and
part (iii) of Theorem \ref{thm:HR_existence_continous_dependence} (proved in
Section \ref{existence})
to \eqref{a10qx} yields
%
\begin{equation}
	\begin{split}
		\label{a11qx}
		\frac{1}{2}\frac{d}{dt} \|v\|_{H^{\sigma}(\ci)}^2
		& \le
		C' (\|u_n(0)\|_{H^{s}(\ci)}
		+ \|u^\ee_n(0)\|_{H^{s}(\ci)})\cdot \|v\|_{H^{\sigma}(\ci)}^2
		\\
		& = C' (\|u_{0,n}\|_{H^{s}(\ci)}
		+ \|J_\ee u_{0,n}\|_{H^{s}(\ci)})\cdot \|v\|_{H^{\sigma}(\ci)}^2.
	\end{split}
\end{equation}
Note that the family $\left\{ u_{0,n} \right\}_n$ is uniformly bounded in
$H^s(\ci)$. Hence, applying Proposition \ref{lem3r} to \eqref{a11qx} we obtain the critical estimate 
\begin{equation}
	\begin{split}
		\label{a12qx}
		\frac{1}{2}\frac{d}{dt} \|v\|_{H^{\sigma}(\ci)}^2
		& \le
	C \|v\|_{H^{\sigma}(\ci)}^2
\end{split}
\end{equation}
with $C = C(R)$, where $R$ is defined as in \eqref{r-def}. Note that $C$
does not depend on $n$ or $\ee$. Differentiating
the left hand side of \eqref{a12qx} and simplifying, we obtain
\begin{equation}
	\begin{split}
		\frac{d}{dt}\|v\|_{H^{\sigma}(\ci)} \le C \|v\|_{H^{\sigma}(\ci)}.
		\label{a100qx}
	\end{split}
\end{equation}
Let $y(t) = \|v\|_{H^{\sigma}(\ci)}$. Then \eqref{a100qx} gives
\begin{equation*}
	\begin{split}
		\frac{1}{y(t)}\frac{dy}{dt} \le C.
	\end{split}
\end{equation*}
Hence,
\begin{equation*}
	\begin{split}
		\int_0^t \frac{1}{y(\tau)} \frac{dy}{d \tau}
		\le \int_0^t C \ d \tau, \qquad |t| \le T
	\end{split}
\end{equation*}
from which we obtain
\begin{equation}
	\begin{split}
		\ln |y(t) | - \ln |y(0)| \le C t.
		\label{a101qx}
	\end{split}
\end{equation}
Simplifying \eqref{a101qx}, we have
\begin{equation*}
	\begin{split}
		\ln \left |\frac{y(t)}{y(0)} \right | \le C t
	\end{split}
\end{equation*}
which yields the estimate
\begin{equation*}
	\begin{split}
		y(t) \le y(0) e^{C t}, \qquad |t| \le T.
	\end{split}
\end{equation*}
Substituting back in $\|v\|_{H^{\sigma}(\ci)}$ for $y$, we get
\begin{equation}
	\label{vsig-est}
	\begin{split}
		\|v\|_{H^{\sigma}(\ci)}
		& \le e^{C t}\|v(0)\|_{H^{\sigma}(\ci)}
		\\
		& = e^{C t}\|u_n(0) - u^\ee_n(0) \|_{H^{\sigma}(\ci)}
		\\
		& = e^{C t}\|u_{0,n} - J_\ee u_{0,n}\|_{H^{\sigma}(\ci)}.
	\end{split}
\end{equation}
Applying Proposition \ref{lem4r} to \eqref{vsig-est}, we obtain 
\begin{equation}
	\label{almost}
	\begin{split}
		\|v\|_{H^\sigma (\ci)} \le e^{Ct} \|u_{0,n}\|_{H^\sigma(\ci)} \cdot
		o(\ee^{s-\sigma})
	\end{split}
\end{equation}
and since $\|u_{0,n}\|_{H^s(\ci)} \le R$ for all $n \in \mathbb{N}$, where
$R$ is defined as in \eqref{r-def}, we conclude from estimate \eqref{almost} that
\begin{equation*}
	\begin{split}
		\|v\|_{H^\sigma(\ci)} \le C(R) \cdot o(\ee^{s-\sigma})
	\end{split}
\end{equation*}
completing the proof. $\quad \Box$
\vskip0.1in
We are now prepared to interpolate. Recall \eqref{a15qu}. Applying Remark
\ref{lem2r}, Remark \ref{lem5r'}, and
Proposition \ref{prop:180} gives
\begin{equation*}
	\begin{split}
		& \|u^\ee_n \|_{H^{s+1}(\ci)} \|v\|_{H^{s-1}(\ci)} \|v\|_{H^s
		(\ci)}
		\\
		&\le C'' \ee^{-1} \cdot \|v\|_{H^\sigma(\ci)}^{1/(s-\sigma)}
		\|v\|_{H^s(\ci)}^{2 - 1/(s- \sigma)}
		\\
		& \le C'' \ee^{-1} \cdot \Big [C' \cdot o(\ee^{s- \sigma})\Big ]^{1/(s-\sigma)}
		\cdot \|v\|_{H^s(\ci)}^{2- 1/(s-\sigma)}
	\end{split}
\end{equation*}
from which we obtain
\begin{equation}
	\begin{split}
		\label{a200qx}
		\|u^\ee_n\|_{H^{s+1}(\ci)} \|v\|_{H^{s-1}(\ci)} \|v \|_{H^s(\ci)}
		& \le  C \cdot o(1) \cdot \|v\|_{H^s(\ci)}^{2- 1/(s-\sigma)}.
	\end{split}
\end{equation}
where $C=C(R)$ does not depend on $\ee$ or $n$. We wish to control the growth of the right hand side of \eqref{a200qx}.
First, note that the triangle inequality, part (iii) of Theorem
\ref{thm:HR_existence_continous_dependence}, and Proposition \ref{lem3r} imply
\begin{equation}
	\begin{split}
		\|v\|_{H^s(\ci)} & = \|u_n - u^\ee_n \|_{H^s(\ci)} 
		\\
		& \le \|u_n \|_{H^s(\ci)} + \|u^\ee_n\|_{H^s(\ci)}
		\\
		& \le 2\left[ \|u_{0,n} \|_{H^s(\ci)} + \|J_\ee u_{0,n}
		\|_{H^s(\ci)} \right]
		\\
		& \le 4 \|u_{0,n} \|_{H^s(\ci)}, \qquad |t| \le T
		\label{agrowth_v}
	\end{split}
\end{equation}
and since $\{u_{0,n}\}_n$ belongs to a bounded subset of
$H^s(\ci)$, we see from \eqref{agrowth_v} that $\|v \|_{H^s(\ci)}$ is
uniformly bounded in $n$ \emph{and} $\ee$.  Secondly, by Remark \ref{lem5r'}, we have 
\begin{equation}
	\label{au-ee-bound}
	\|u^\ee_n \|_{H^s(\ci)} \le C(R), \ \ \text{for all} \ \ 0 < \ee <<1, \ n \in
	\mathbb{N}.
\end{equation}
Applying \eqref{a200qx}, \eqref{agrowth_v}, and \eqref{au-ee-bound}
to \eqref{a15qu}, it follows that 
\begin{equation*}
	\label{lim-est-in}
		\begin{split}
			\lim_{n \to \infty }
			\frac{1}{2}\frac{d}{dt}\|v\|_{H^{s}(\ci)}^2
			& \le
			C \lim_{n \to \infty} \Big [\|v\|_{H^s(\ci)}^3 +
			\|v\|_{H^s(\ci)}^2 + o(1)\Big ]
		\end{split}
	\end{equation*}
	for every $0 < \ee <<1$, where $C = C(\|u_0\|_{H^s(\ci)}, \ R)$.
	Therefore
	\begin{equation}
		\begin{split}
			\lim_{\substack{n \to \infty \\ \ee \to 0} }
			\frac{1}{2}\frac{d}{dt}\|v\|_{H^{s}(\ci)}^2
			& \le C
			\lim_{\substack{n \to \infty \\ \ee \to 0}}
			\Big [\|v\|_{H^s(\ci)}^3 + 
			\|v\|_{H^s(\ci)}^2 \Big ].
			\label{a15qx}
		\end{split}
	\end{equation}
	We differentiate the left hand side of \eqref{a15qx} and obtain
\begin{equation*}
	\begin{split}
		\lim_{\substack{n \to \infty \\ \ee \to 0}}\frac{d}{dt}
		\|v\|_{H^s(\ci)} \le C
		\lim_{\substack{n \to \infty \\ \ee \to 0}} \left [\|v\|_{H^s(\ci)}^2 +
		\|v\|_{H^s(\ci)} \right ].
	\end{split}
\end{equation*}
Letting $y = \|v\|_{H^s(\ci)}$ and rearranging gives
\begin{equation*}
	\begin{split}
		\lim_{\substack{n \to \infty \\ \ee \to 0} } \ \frac{1}{y(y+1)} \frac{dy}{dt}
		\le	C
	\end{split}
\end{equation*}
which can be rewritten as
\begin{equation*}
	\begin{split}
		\lim_{\substack{n \to \infty \\ \ee \to 0} }
		\left( \frac{1}{y} - \frac{1}{y+1} \right)\frac{dy}{dt} \le C 
	\end{split}
\end{equation*}
implying
\begin{equation}
	\label{aest-int}
	\begin{split}
		\lim_{\substack{n \to \infty \\ \ee \to 0} } \left [
\int_0^t \frac{1}{y} \frac{dy}{d \tau} \ d \tau
		- \int_0^t \frac{1}{y+1} \frac{dy}{d \tau} \ d \tau \right ]
		\le \int_0^t C \ d \tau, \quad |t| \le T.
	\end{split}
\end{equation}
Hence \eqref{aest-int} gives 
\begin{equation}
	\begin{split}
	\lim_{\substack{n \to \infty \\ \ee \to 0} }	\left [ \ln \left | \frac{y(t)}{y(0)}
	\cdot \frac{y(0) + 1}{y(t) + 1} \right | \right ] \le C t.
		\label{20b}
	\end{split}
\end{equation}
Exponentiating both sides of \eqref{20b}, and noting that $f(x) = e^x$
is a continuous function on $\rr$, we must have
\begin{equation*}
	\begin{split}
		\lim_{\substack{n \to \infty \\ \ee \to 0} }	
		\left |
		\frac{y(t)}{y(0)} \cdot \frac{y(0) + 1}{y(t) + 1} \right | \le e^{C t}.
	\end{split}
\end{equation*}
Recalling that $y(t) = \|v(t)\|_{H^s(\ci)} \ge 0$, we obtain
\begin{equation*}
	\begin{split}
		\lim_{\substack{n \to \infty \\ \ee \to 0} }	
		\frac{y(t)}{y(0)} \cdot \frac{y(0) + 1}{y(t) + 1} \le e^{C t}.
	\end{split}
\end{equation*}
Rearranging, it follows that 
\begin{equation*}
	\begin{split}
		\lim_{\substack{n \to \infty \\ \ee \to 0}} \frac{y(t)}{y(t) + 1}
		\le \lim_{\substack{n \to \infty \\ \ee \to 0}} \frac{e^{C t} \cdot y(0)}{y(0) + 1} \le
		\lim_{\substack{n \to \infty \\ \ee \to 0}} e^{C t} \cdot y(0).
	\end{split}
\end{equation*}
Substituting back in $\|v(t)\|_{H^s(\ci)}$ for $y(t)$ gives
\begin{equation}
	\begin{split}
		\lim_{\substack{n \to \infty \\ \ee \to 0}}	\frac{\|v(t)\|_{H^s(\ci)}}{\|v(t)\|_{H^s(\ci)} + 1}  \le
		\lim_{\substack{n \to \infty \\ \ee \to 0}} e^{C t} \cdot \|v(0)\|_{H^s(\ci)}.
		\label{a303qx}
	\end{split}
\end{equation}
Since by Proposition \ref{lem4r} 
\begin{equation*}
	\begin{split}
	\lim_{\substack{n \to \infty \\ \ee \to 0} }
	\|v(0)\|_{H^s(\ci)}
	& = \lim_{\substack{n \to \infty \\ \ee \to 0} }
	\|u_{0,n} - J_\ee u_{0,n} \|_{H^s(\ci)} 
	\\
	& \le  \lim_{\substack{n \to \infty \\ \ee \to 0}}
	\left [ \|u_{0,n}\|_{H^s(\ci)} \cdot o(1) \right ]
	\\
	& = \lim_{\ee \to 0} \left [ \|u_0\|_{H^s(\ci)} \cdot o(1) \right ]
	\\
	& = 0
	\end{split}
\end{equation*}
we deduce from \eqref{a303qx} that
\begin{equation*}
	\begin{split}
		\lim_{\substack{n \to \infty \\ \ee \to 0}} \|v(t)\|_{H^s(\ci)} = 0, \qquad |t| \le T
	\end{split}
\end{equation*}
completing the proof of \eqref{enough_to_prove3}. $\quad \Box$
%
%
%
\vskip0.1in
\section{Extending Well-Posedness to the Non-Periodic Case.}
\label{sec:defs}
\setcounter{equation}{0}
\vskip0.1in
The method will be analogous to that of the periodic case, with two major
modifications. First, we must choose a different mollifier $J_\ee$ in the
proof of continuous dependence. Pick a
function $j(x) \in \mathcal{S}(\rr)$ such that
\begin{equation*}
		\begin{split}
			& 0 \le \widehat{j}(\xi) \le 1,
			\\
			& \widehat{j}(\xi) = 1 \ \ \text{if} \ \ |\xi| \le 1.
		\end{split}
	\end{equation*}
Letting
\begin{equation*}
	\begin{split}
		j_\ee(x) = \frac{1}{\ee} j \left (\frac{x}{\ee} \right )
	\end{split}
\end{equation*}
it can be verified that 
		\begin{equation*}
		\begin{split}
			\widehat{j_\ee}(\xi) = \widehat{j }(\ee \xi), \quad \ee > 0.
		\end{split}
	\end{equation*}
We then define $J_\ee$ to be the ``Friedrichs mollifier''
	\begin{equation*}
		\begin{split}
			J_\ee f(x) = j_\ee * f(x), \quad \ee>0.
		\end{split}
	\end{equation*}
Given this construction, the proofs of Remark \ref{lem5r}, Remark
\ref{lem5r'}, and Proposition \ref{lem4r} for the non-periodic case will be
analogous to those in the periodic case.
\vskip0.1in
Secondly, in the proof of existence, we will have difficulties in arranging
that the solutions $\{u_\ee\}$ to the mollified HR i.v.p. converge in $C(I,
H^{s- \sigma}(\rr))$, $0 < \sigma < 1$ to a candidate solution $u$ of the HR
i.v.p. We will get around this by considering the family $\left\{ \varphi
u_\ee \right\}$ instead.
%
%
\vskip0.1in
%
%
We divide our work into three parts:
\vskip0.1in
\subsection{Existence.}
\vskip0.1in
Mirroring the argument in the periodic case, we see that the bounded
family $\{u_\ee\}$ is compact in the weak* topology of $L^\infty(I,
H^{s}(\rr))$. More precisely, there is a sequence  $\{ u_{\ee_n} \}$
converging weak* to a $ u\in L^{\infty}(I, H^s(\rr))$; that is 
		%
		\begin{equation*}
			\label{hhweak-conv}
			\lim_{n\to \infty} T_{u_{\ee_n}}(\varphi)  =  T_u (\varphi) 
			\; \;		
			\text{ for all } \;\;  \varphi \in L^1(I, H^{s}(\rr))
		\end{equation*}
		where
		\begin{equation}
			T_v(\varphi) = \int_I <v (t), \varphi (t)>_{H^s(\rr)} dt  = \int_I
			 \int_\rr
			 \widehat{v}(\xi, t) \bar{\widehat{\varphi}} (\xi, t) \cdot (1 +
			 \xi^2)^s \ d \xi \; dt.
		\end{equation}

		%
		Similarly, $\left\{ \p_x u_{\ee_n} \right\}$ is compact in the
		weak* topology of $L^\infty(I, H^{s-1}(\rr))$ and converges weak*
		to $\p_x u$. Hence, for any $k \in \mathbb{N}$, we have
		\begin{align}
			\label{base-weak}
				& (u_{\ee_n})^k \xrightarrow{\text{weak*}} u^k \ \
				\text{on} \ \
				L^\infty(I, H^s(\rr)),
				\\
				\label{base-weak-2}
				& (\p_x u_{\ee_n})^k \xrightarrow{\text{weak*}} (\p_x u)^k
				\ \ \text{on} \ \
				L^\infty(I, H^{s-1}(\rr)). 
		\end{align}
		In order to show that $u$ solves the HR i.v.p., it would
		suffice to obtain a stronger convergence for  $u_{\ee_n}$ so that 
		we could take the limit in the mollified HR equation. However,
		this is difficult, and unnecessary. Rather, our approach will be to
		show that for any pseudo-differential operator
		$P \in \Psi^0$ and arbitrary $\vp \in S(\rr)$, $k \in
		\mathbb{N}$, $0< \sigma < 1$, we have
		%
		%
			\begin{align}
			\label{hhstrong-conv}
			& \varphi P [(u_{\ee_n})^k] \longrightarrow \varphi P [u^k]  
			\quad
			\text{ in } \,\,   C(I, H^{s-\sigma}(\rr)), \ \,
			\\
			\label{hhstrong-conv-next}
			& \varphi P [(\p_x u_{\ee_n})^k] \longrightarrow \varphi P
			[(\p_x u)^k]  
			\quad
			\text{ in } \,\,   C(I, H^{s-\sigma -1}(\rr)), \ \ 
		\end{align}
		%
		which will then be applied to a rewritten version of the HR
		i.v.p. Our focus will be on proving \eqref{hhstrong-conv}; since the proof of
		\eqref{hhstrong-conv-next} is similar, we will omit the
		details. First, we will need the following
		interpolation result:
		%%%%%%%%%%%%%%%%%%%%%%%%%%%
		%
		%
		%                 Interpolation Lemma
		%
		%
		%%%%%%%%%%%%%%%%%%%%%%%%%%%
		\begin{lemma}
			\label{hhinterpolation-lem}
			(Interpolation)     Let  $s > \frac{3}{2}$.
			If $v \in C(I, H^s(\rr)) \cap C^1(I, H^{s-1}(\rr))$
			then $v \in C^\sigma (I, H^{s- \sigma}(\rr))$ for  $0 < \sigma < 1$.
		\end{lemma}
		%
		{\bf Proof.} It is analogous to the proof in the periodic case.
		$\quad \Box$
		\vskip0.1in
		Fix $k \in \mathbb{N}$. Using Lemma \ref{hhinterpolation-lem}, we
		will show that the family
		\begin{equation*}
			\begin{split}
			 \{\varphi P[(u_\ee)^k]\}_\ee
		\end{split}
	\end{equation*}
		is equicontinuous in $C(I, H^{s-\sigma}(\rr))$ 
		for $0 < \sigma < 1$ and $\varphi = \varphi(x) \in \mathcal{S}(\rr)$.
		We will follow this by proving that
		there exists a sub-family $\{\varphi P[(u_{\ee_n}(t))^k]\}_n$
		that is precompact in $H^{s-\sigma}(\rr)$ for $\sigma > 0$. 
		These two facts, in conjunction with Ascoli's Theorem, will
		yield
		\begin{equation*}
			\label{hhstrong-conv2}
			\varphi P[(u_\ee)^k] \to \tilde{u}
			\; \; \text{in} \; \; C(I,H^{s-\sigma}(\rr))
		\end{equation*}
		for $0 < \sigma < 1$.
		We will then show that $\tilde{u} = \varphi P[u^k]$, from which it will
		follow that
		\begin{equation*}
			\label{hhphiplus}
			\begin{split}
				\varphi P[(u_\ee)^k] \to \varphi P[u^k]
				\; \; \text{in} \; \; C(I,H^{s-\sigma}(\rr)).
			\end{split}
		\end{equation*}
		

		%%%%%%%%%%%%%%%%%%%%%%
		%
		%
		%       Equicontinuity
		%
		%
		%%%%%%%%%%%%%%%%%%%%%%

		%
		\vskip0.1in
		\nin
		{\bf  Equicontinuity of $\{ \varphi P [(u_\ee)^k]\}_\ee$  in $C(I,
		H^{s-\sigma}(\rr)$}).
		%
		%
		Since $\varphi \in \mathcal{S}(\rr)$, the map $u \mapsto \vp u$
		is a bounded linear function on $H^s(\rr)$, for arbitrary $s \in
		\rr$, where  
		\begin{equation}
			\begin{split}
				\|\varphi u\|_{H^s(\rr)} \le C(s, \varphi)
				\|u\|_{H^s(\rr)}, \quad \forall s\in \rr.
				\label{hhschwartz-estimate}
			\end{split}
		\end{equation}
		Furthermore, $$P: H^s(\rr) \to H^s(\rr)$$ is bounded and linear,
		with 
		\begin{equation}
			\label{operator-normaa}
			\|P\|_{L(H^s(\rr), H^s(\rr))} \le 1.
		\end{equation}
		Hence, the map 
		\begin{equation}
			\label{the-map}
			\begin{split}
			& T: H^s(\rr) \to H^s(\rr),
			\\
			& T(u) = \vp P u 
		\end{split}
	\end{equation}
	is bounded and linear, with 
	\begin{equation}
		\begin{split}
			\|T\|_{L(H^s(\rr), H^s(\rr))} \le C(s, \vp).
			\label{op-norm-product}
		\end{split}
	\end{equation}
	Therefore, applying Lemma
		\ref{hhinterpolation-lem} gives 
		%
		\begin{equation*}
			\begin{split}
			\label{hhequic-1}
			& \sup_{t \neq t'} \frac {\| \varphi P [(u_\ee(t))^k] - \varphi
			P [(u_\ee(t'))^k] \|_{H^{s -
			\sigma  }(\rr)}}{|t - t'|}
			\\
			& \le \sup_{t \neq t'}  \frac { \|\vp P \|_{L(H^{s-\sigma}(\rr),
			H^{s-\sigma}(\rr))} \cdot \|   [u_\ee(t)]^k  - 
			[u_\ee(t')]^k \|_{H^{s -
			\sigma }(\rr)}}{|t - t'|}
			\\
			& \le C(s, \vp) \cdot \sup_{t \neq t'}  \frac { \|   [u_\ee(t)]^k  - 
			[u_\ee(t')]^k \|_{H^{s -
			\sigma }(\rr)}}{|t - t'|}
			\\
			&< c
		\end{split}
		\end{equation*}
		%
		or
		%
		\begin{equation*}
			\label{hhequic-2}
			\|\varphi P [(u_\ee(t))^k] - \varphi
			P [(u_\ee(t'))^k \|_{H^{s - \sigma }(\rr)}< c|t -
			t'|, 
			\text{ for all }  \,\,  t, t'\in I,
		\end{equation*}
		%
		which shows that  the family  $\{\varphi P [(u_\ee)^k]\}_\ee$ is
		equicontinuous in $C(I, H^{s-\sigma }(\rr))$.  $\quad \Box$
		%
		\vskip0.1in
		\nin
		%
		%%%%%%%%%%%%%%%%%%%%%%
		%
		%
		%      PreCompactness
		%
		%
		%%%%%%%%%%%%%%%%%%%%%%%%%%
		%
		%
		%
		%
		%		
		{\bf Precompactness of $\{\varphi P [(u_\ee(t))^k]\}_\ee$ in
		$H^{s-\sigma  }(\rr)$}.
		Applying the algebra property of Sobolev
		Spaces, and recalling \eqref{the-map}-\eqref{op-norm-product}, we have
		\begin{equation}
			\begin{split}
			\label{hhcompact-1}
			 \|\varphi P [(u_\ee(t))^k]\|_{H^{s}(\rr)}
			& \le  C(s, \vp) \cdot \|[u_\ee(t)]^k\|_{H^{s}(\rr)}
			\\
			& \le C(s, \vp) \cdot \|u_\ee(t)\|^k_{H^{s}(\rr)}.
			\end{split}
		\end{equation}
		%
		Letting $|t| \le T$, we now apply Lemma \ref{hr_wp} to
		\eqref{hhcompact-1} to obtain
		\begin{equation*}
			\begin{split}
			\|\varphi P [(u_\ee(t))^k]\|_{H^{s}(\rr)}
			\le 2^k C(s, \vp) \cdot  \|u_0 \|^k_{H^s(\rr)} < \infty.
			\end{split}
		\end{equation*}
		Therefore, by Reillich's Theorem, the family $\left\{
		\varphi P [(u_\ee(t))^k] \right\}_\ee$ is
		precompact in $H^{s- \sigma }(\rr)$ for all $\sigma > 0$ and $|t| \le T$. $\quad
		\Box$ 
		\vskip0.1in
		Hence, compiling our previous results on equicontinuity and precompactness
		and applying Ascoli's Theorem, we
		conclude that we can find $\tilde{u}$ and a subfamily 
		\\ $\left\{
		\varphi P [(u_{\ee_n})^k]
		\right\}_n$ such that
		\begin{equation}
			\label{hhstrong-conv-of-u_ep}
			\varphi P [(u_{\ee_n})^k] \to \tilde{u}
			\; \; \text{in} \; \; C(I, H^{s-\sigma}(\rr)).
		\end{equation}
		%
		%
		
		\vskip0.1in
		We would now like to find out what $\tilde{u}$ is:
		%
		%
		%
		\vskip0.1in
		\begin{lemma}
			\label{hhlem:crit-conv}
			For arbitrary $k \in \mathbb{N}$,
			\begin{equation}
				\begin{split}
					\varphi P [(u_{\ee_n})^k] \xrightarrow{weak^*}
					\varphi P [u^k] \ \ \text{on} \ \ L^\infty(I,
					H^{s-\sigma}(\rr)).
					\label{hhcrit-conv-est}
				\end{split}
			\end{equation}
		\end{lemma}
		{\bf Proof.} 
		Fix $k \in \mathbb{N}$ and recall that the operators 
		\begin{equation*}
			\begin{split}
			 & T_\varphi: H^s(\rr) \to H^s(\rr)\\
			 & T_\varphi u = \varphi u
		\end{split}
	\end{equation*}
and 
\begin{equation*}
	\begin{split}
		P:H^s(\rr) \to H^s(\rr)
	\end{split}
\end{equation*}
	are continuous; therefore 
	\begin{equation*}
		\begin{split}
			T_\vp P: H^s(\rr) \to H^s(\rr)
		\end{split}
	\end{equation*}
	continuously. Hence, its adjoint  $(T_\varphi P)^*$
	exists and
		\begin{equation*}
			(T_\varphi P)^*: H^s(\rr) \to H^s(\rr) 
		\end{equation*}
		continuously. Therefore, applying \eqref{base-weak}, we conclude that
		\begin{equation}
			\label{widpseudo}
			\begin{split}
				& \int_I <\varphi P[u^k] - \varphi
				P [(u_{\ee_n})^k],\  f>_{H^{s-\sigma }(\rr)} dt
				\\
				&= \int_I <u^k - 
				 (u_{\ee_n})^k, \ (T_\vp P)^* f>_{H^{s-\sigma }(\rr)} \to 0
			\end{split}
		\end{equation}
		completing the proof. $\quad \Box$
		\vskip0.1in
		%
		%
		Now, recalling \eqref{hhstrong-conv-of-u_ep} and applying Lemma
		\ref{hhlem:crit-conv}, we obtain
			\begin{equation}
			\begin{split}
				\vp P [(u_{\ee_n})^k] \to \vp P [u^k] \ \ \text{in}  \ \ C(I,
				H^{s-\sigma}(\rr))
				\label{hhvp_u_ep_conv}
			\end{split}
		\end{equation}
		for arbitrary $k \in \mathbb{N}$.  Using precisely the same
		strategy we used to prove \eqref{hhvp_u_ep_conv} (applied now to
		the family $\{ \vp P [(\p_x u_{\ee})^k] \}_\ee$), one can also show
	\begin{equation}
			\begin{split}
			\vp P [ (\p_x u_{\ee_n})^k] \to \vp P [(\p_x u)^k] \ \ \text{in}  \ \ C(I,
				H^{s-\sigma -1 }(\rr)).
			\end{split}
		\end{equation}
		We summarize our result below:
		\begin{theorem}
		\label{hhthm:crit1}
		Let $P \in \Psi^0$ be a pseudo-differential operator. Then for
		arbitrary $k \in \mathbb{N}$, 
			\begin{equation}
			\begin{split}
				& \vp P [(u_{\ee_n})^k] \to \vp P [u^k] \ \ \text{in}  \ \ C(I,
				H^{s-\sigma }(\rr)),
				\\
				& 
				\vp P [(\p_x u_{\ee_n})^k] \to \vp P [(\p_x u)^k] \ \
				\text{in}  \ \ C(I,
				H^{s-\sigma -1}(\rr)).
				\label{hhdx_vp_u_ep_conv}
			\end{split}
		\end{equation}
	\end{theorem}
		\vskip0.1in
		{\bf Verifying that the weak* limit $u$ solves the HR equation.} 
		We recall the mollified HR i.v.p
		\begin{align}
			& \p_t u_{\ee_n}  = -\gamma(J_{\ee_n} u_{\ee_n} \cdot \p_x
			J_{\ee_n} u_{\ee_n})
			\label{hh1gr}
			\\
			& u(x,0) = u_0(x).
			\label{hh2gr}
		\end{align}
		Multiplying both sides of \eqref{hh1gr} by $\varphi$ and rewriting,
		we obtain
		\begin{equation}
			\label{hh3}
			\begin{split}
				\p_t(u_{\ee_n} \varphi) = -\gamma \vp (J_{\ee_n} u_{\ee_n} \cdot
				J_{\ee_n} \p_x u_{\ee_n}).
			\end{split}
		\end{equation}
		The following lemma will play a crucial role in our proof of the
		existence of a solution to the HR i.v.p.
		\begin{lemma}
			\label{hhlem:cc}
			For $\vp \in \mathcal{S}(\rr)$ such that
			$\vp^\frac{1}{2} \in \mathcal{S}(\rr)$, we have
			\begin{equation}
				\begin{split}
					\label{hhburgers_and_nonlocal_conv}
				& \vp (J_{\varepsilon_n} u_{\varepsilon_n} 
				\cdot J_{\varepsilon_n}\partial_x u_{\varepsilon_n}) 
				\to \vp u \partial_x u \; \; 
				\text{in} \; \;
				C(I, H^{s-\sigma-1}(\rr)). 
			\end{split}
			\end{equation}
		\end{lemma}
		%
		{\bf Proof.} We will need a couple of propositions:
		\begin{proposition}
			For arbitrary $\vp \in \mathcal{S}(\rr)$
			\label{hhprop:1aa}
			\begin{equation}
				\begin{split}
					\vp J_{\ee_n} u_{\ee_n} \to \vp u \ \ \text{in} \ \
					C(I, H^{s-\sigma}(\rr)).
					\label{hh}
				\end{split}
			\end{equation}
		\end{proposition}
			{\bf Proof.} Note that
			\begin{equation}
				\begin{split}
					& \|\vp u - \vp J_{\ee_n} u_{\ee_n}
					\|_{C(I, H^{s-\sigma}(\rr))}
					\\
					&= \|\vp u - \vp J_{\ee_n} u_{\ee_n} \pm \vp
					u_{\ee_n} \|_{C(I, H^{s-\sigma}(\rr))}
					\\
					& = \|\vp u - \vp u_{\ee_n}
					\|_{C(I, H^{s-\sigma}(\rr))} + \|\vp (I - J_{\ee_n})
					u_{\ee_n} \|_{C(I, H^{s-\sigma}(\rr))}.
					\label{hh1bb}
				\end{split}
			\end{equation}
			Applying \eqref{hhschwartz-estimate} and the estimates
			\begin{equation*}
				\begin{split}
					& \|I-J_{\ee_n} \|_{L(H^{s-\sigma}(\rr), H^{s -
					\sigma}(\rr))} = o(1),
					\\
					& \|u_{\ee_n}\|_{H^{s-\sigma}(\rr)} \le 2
					\|u_0\|_{H^{s-\sigma}(\rr)}
				\end{split}
			\end{equation*}
			to \eqref{hh1bb} gives
			\begin{equation}
				\label{hh2bb}
				\begin{split}
					\|\vp u - \vp J_{\ee_n} u_{\ee_n}\|_{H^{s-\sigma}(\rr)}
					\le \left( \|\vp u - \vp u_{\ee_n}
					\|_{C(I, H^{s-\sigma}(\rr))} + C(s, \vp) \cdot o(1) \cdot \|u_0
					\|_{H^{s-\sigma}(\rr)} \right).
				\end{split}
			\end{equation}
			Letting $\ee \to 0$ in \eqref{hh2bb} and applying Theorem
			\ref{hhthm:crit1} completes the proof. $\quad \Box$
			%
			%
			\begin{proposition}
				\label{hhprop:dd}
				For arbitrary $ \vp \in \mathcal{S}(\rr)$,
				\begin{equation}
					\begin{split}
						\vp J_{\ee_n} \p_x u_{\ee_n} \to \vp u \ \
						\text{in} \ \ C(I, H^{s-\sigma-1}(\rr)).
						\label{hh0dd}
					\end{split}
				\end{equation}
			\end{proposition}
			{\bf Proof.} The result follows from Theorem \ref{hhthm:crit1}.
			The proof is nearly identical to that of
			Proposition \ref{hhprop:1aa}, with $s-1$ substituted for $s$
			and $\p_x u_{\ee_n}$ substituted for $u_{\ee_n}$. $\quad \Box$
			%
			%
			\vskip0.1in
			We now have enough tools to prove Lemma \ref{hhlem:cc}. Restrict the
			choice of $\vp$ such that $\vp^\frac{1}{2} \in S(\rr)$
			(Such Schwartz functions exist; as an example, take the square
			of the Gaussian). Using this fact, and applying Proposition
			\ref{hhprop:1aa} and Proposition \ref{hhprop:dd}, we conclude that
			\begin{equation*}
				\begin{split}
					\vp J_{\ee_n} u_{\ee_n} \p_x J_{\ee_n} u_{\ee_n} 
					& = \vp^\frac{1}{2} J_{\ee_n} u_{\ee_n} \cdot
					\vp^\frac{1}{2} \p_x J_{\ee_n} u_{\ee_n}
					\\
					& \to \vp^\frac{1}{2} u \cdot \vp^\frac{1}{2} \p_x u = \vp
					u \p_x u
				\end{split}
			\end{equation*}
			completing the proof of Lemma \ref{hhlem:cc}. $\quad \Box$
%
%
\vskip0.1in
%
%
By Theorem \ref{hhthm:crit1} it follows immediately that
		\begin{equation}
			\begin{split}
				& \vp \p_x(1- \p_x^2)^{-1} \left( \frac{3-\gamma}{2}
				(u_{\ee_n})^2
				 + \frac{\gamma}{2} (\p_x u_{\ee_n})^2 \right )
				 \\
				 & \to
				 \vp \p_x(1- \p_x^2)^{-1} \left( \frac{3-\gamma}{2} u^2
				 + \frac{\gamma}{2} (\p_x u)^2 \right ) \ \
				 \text{in} \ \ C(I, H^{s-\sigma-1}(\rr)).
				\label{llnon-local-convergence}
			\end{split}
		\end{equation}
		Combining \eqref{hhburgers_and_nonlocal_conv} and
		\eqref{llnon-local-convergence}, and applying the Sobolev Imbedding
		Theorem, we deduce 
		\begin{equation}
			\begin{split}
				& -\gamma \vp (J_{\ee_n} u_{\ee_n} \cdot J_{\ee_n} \p_x
				u_{\ee_n}) -
				\vp \p_x(1- \p_x^2)^{-1} \left( \frac{3-\gamma}{2}
				(u_{\ee_n})^2
				 + \frac{\gamma}{2} (\p_x u_{\ee_n})^2 \right )
				 \\
				 \to & -\gamma \vp u \p_x u -
				 \vp \p_x(1- \p_x^2)^{-1} \left( \frac{3-\gamma}{2} u^2
				 + \frac{\gamma}{2} (\p_x u)^2 \right ) \ \
				 \text{in} \ \ C(I, C(\rr)).
				\label{llloc-non-loc-tog}
			\end{split}
		\end{equation}
		%
		Next, we note that the convergence  
		%
		\begin{equation}
			\label{hhweak-conv-2}
			T_{\vp u_{\ee_n}}(f)  \longrightarrow  T_{\vp u} (f) \;
			\text{ for all } \;  f \in L^1(I, H^{-s}(\rr))
		\end{equation}
		%
		can be restated as 
		%
		\begin{equation}
			\vp u_{\ee_n}  \longrightarrow  \vp u
			\quad
			\text{ in }  \,\,
			\mathcal{D}'(I\times \rr).
		\end{equation}
		%
		This implies 
		%
		\begin{equation}
			\label{hhdistib-conv-2}
			\p_t(\vp u_{\ee_n})  \longrightarrow  \p_t (\vp u)
			\quad
			\text{ in }  \,\, \mathcal{D}'(I\times \rr).
		\end{equation}
		%
		Since for all $n$ we have 
		%
		\begin{equation}
			\begin{split}
			 \p_t (\vp u_{\ee_n})
			 = & -\gamma \vp
			(J_{\varepsilon_n} u_{\varepsilon_n}  \cdot
			J_{\varepsilon_n}\partial_x u_{\varepsilon_n})
			\\
			& -
			\vp \p_x(1- \p_x^2)^{-1} \left( \frac{3-\gamma}{2} (u_{\ee_n})^2
			 + \frac{\gamma}{2} (\p_x u_{\ee_n})^2 \right )
		 \end{split}
		\end{equation}
		%
		it follows from \eqref{hhdistib-conv-2} and the uniqueness of the
		limit in \eqref{llloc-non-loc-tog} that
		\begin{equation}
			\begin{split}
			 \p_t (\vp u)
			 = & -\gamma \vp
			u \p_x u - \vp \p_x(1- \p_x^2)^{-1} \left( \frac{3-\gamma}{2} u^2
			 + \frac{\gamma}{2} (\p_x u)^2 \right )
			\label{hhadone}
			\end{split}
		\end{equation}
		Further restricting $\vp \in \mathcal{S}(\rr)$ to be nonzero in
		$\rr$, we
		can divide both sides of \eqref{hhadone} by $\vp$ to obtain
		\begin{equation}
			\label{hh2yy}
			\begin{split}
			 \p_t  u
			 = & -\gamma
			u \p_x u - \p_x(1- \p_x^2)^{-1} \left( \frac{3-\gamma}{2} u^2
			 + \frac{\gamma}{2} (\p_x u)^2 \right ).
			\end{split}
		\end{equation}
		Thus we have constructed a solution $u \in L^\infty(I, H^s(\rr))$
		to the HR i.v.p. 
		\vskip0.1in
\subsection{Uniqueness.} The proof is analogous to that in the periodic case.
\vskip0.1in
\subsection{Continuous Dependence.} The proof is analogous to the proof in
the periodic case, with one important caveat. Recall the introduction to Section
\ref{sec:defs} in the appendix; specifically, how we defined the operator
$J_\ee$. By construction, the proofs of Remark \ref{lem5r}, Remark
\ref{lem5r'}, and Proposition \ref{lem4r} for the non-periodic case will be
analogous to those in the periodic case. Hence, how we
construct the mollifier $J_\ee$ plays a critical role in the proofs of
well-posedness for the HR i.v.p. in both the periodic and non-periodic cases. %
%
\end{appendices}








	











\newpage











	\begin{thebibliography}{100000}
			%
			%
			\bibitem[CS]{cs} A. Constantin and W. Strauss, {\em Stability
				of a class of solitary waves in compressible elastic rods},
				Phys. Lett. A  {\bf 270} (2000), no. 3-4, 140--148.
			\bibitem[D]{d} 
				R. Danchin,  
				\textit{A few remarks on the Camassa-Holm equation}, 
				Differential Integral Equations {\bf 14} (2001),
				no. 8, 953--988.
			\bibitem[HK]{hk} A. Himonas and C. Kenig, 
				{\em Non-uniform dependence on initial data for
				the  CH equation on the line}, 
				Differential and Integral Equations,
				{\bf 22}, No. 3-4, (2009), 201-- 224.				
			\bibitem[HM]{hm} A. Himonas and G. Misio\l ek,
				{\em High-frequency smooth solutions and well-posedness of
				the Camassa-Holm equation.}
				Int. Math. Res. Not. {\bf 2005}, no. 51, 3135--3151. 
			\bibitem[HM2]{hm2} 
				A. Himonas,  and  G. Misio\l ek 
				\textit{The Cauchy problem for an integrable shallow water equation},
				Differential Integral Equations {\bf 14} (2001),
				no. 7, 821--831.
			\bibitem[Mi2]{mi2}
				G. Misiolek,
				{\em Classical solutions of the periodic Camassa-Holm equation},
				Geom. Funct. Anal. (GAFA) {\bf 12}, no. 5, (2002), 1080--1104.
			\bibitem[O]{o} E. Olson, {\em The initial value problem for two nonlinear
				evolution equations}, Differential Integral Equations, {\bf 19} (2006),
				no. 10, 1081--1102.  
			\bibitem[T1]{t1} 
				M. E. Taylor,
				{\em Pseudodifferential Operators and
				Nonlinear PDE}, Birkhauser, Boston 1991.
			\bibitem[T2]{t2}
				M. E. Taylor,
				{\em Commutator Estimates}, 
				Proceedings of the American Mathematical Society,
				{\bf 131}, No. 5, (2002), 1501--1507.				
			\bibitem[Z]{z}
				Y. Zhou, {\em Local well-posedness and blow-up criteria of
				solutions for a rod equation}, Math. Nachr. {\bf 278} (2005),
				no. 14, 1726--1739.

		\end{thebibliography}

		\end{document}


