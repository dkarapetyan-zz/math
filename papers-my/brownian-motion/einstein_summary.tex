\documentclass[12pt,reqno]{amsart}
\usepackage{amscd}
\usepackage{amsfonts}
\usepackage{amsmath}
\usepackage{amssymb}
\usepackage{amsthm}
\usepackage{appendix}
\usepackage{fancyhdr}
\usepackage{latexsym}
\usepackage{pdfsync}
\usepackage{cancel}
\usepackage[colorlinks=true, pdfstartview=fitv, linkcolor=blue,
citecolor=blue, urlcolor=blue]{hyperref}
%%%%%%%%%%%%%%%%%%%%%%
\usepackage{color}
\definecolor{red}{rgb}{1.00, 0.00, 0.00}
\definecolor{darkgreen}{rgb}{0.00, 1.00, 0.00}
\definecolor{blue}{rgb}{0.00, 0.00, 1.00}
\definecolor{cyan}{rgb}{0.00, 1.00, 1.00}
\definecolor{magenta}{rgb}{1.00, 0.00, 1.00}
\definecolor{deepskyblue}{rgb}{0.00, 0.75, 1.00}
\definecolor{darkgreen}{rgb}{0.00, 0.39, 0.00}
\definecolor{springgreen}{rgb}{0.00, 1.00, 0.50}
\definecolor{darkorange}{rgb}{1.00, 0.55, 0.00}
\definecolor{orangered}{rgb}{1.00, 0.27, 0.00}
\definecolor{deeppink}{rgb}{1.00, 0.08, 0.57}
\definecolor{darkviolet}{rgb}{0.58, 0.00, 0.82}
\definecolor{saddlebrown}{rgb}{0.54, 0.27, 0.07}
\definecolor{black}{rgb}{0.00, 0.00, 0.00}
\definecolor{dark-magenta}{rgb}{.5,0,.5}
\definecolor{myblack}{rgb}{0,0,0}
\definecolor{darkgray}{gray}{0.5}
\definecolor{lightgray}{gray}{0.75}
%%%%%%%%%%%%%%%%%%%%%%
%%%%%%%%%%%%%%%%%%%%%%%%%%%%
%  for importing pictures  %
%%%%%%%%%%%%%%%%%%%%%%%%%%%%
\usepackage[pdftex]{graphicx}
\usepackage{epstopdf}
% \usepackage{graphicx}
%% page setup %%
\setlength{\textheight}{20.8truecm}
\setlength{\textwidth}{14.8truecm}
\marginparwidth  0truecm
\oddsidemargin   01truecm
\evensidemargin  01truecm
\marginparsep    0truecm
\renewcommand{\baselinestretch}{1.1}
%% new commands %%
\newcommand{\bigno}{\bigskip\noindent}
\newcommand{\ds}{\displaystyle}
\newcommand{\medno}{\medskip\noindent}
\newcommand{\smallno}{\smallskip\noindent}
\newcommand{\nin}{\noindent}
\newcommand{\ts}{\textstyle}
\newcommand{\rr}{\mathbb{R}}
\newcommand{\p}{\partial}
\newcommand{\zz}{\mathbb{Z}}
\newcommand{\cc}{\mathbb{C}}
\newcommand{\ci}{\mathbb{T}}
\newcommand{\ee}{\varepsilon}

\def\refer #1\par{\noindent\hangindent=\parindent\hangafter=1 #1\par}
%% equation numbers %%
\renewcommand{\theequation}{\thesection.\arabic{equation}}
%% new environments %%
%\swapnumbers
\theoremstyle{plain}  % default
\newtheorem{theorem}{Theorem}
\newtheorem{proposition}{Proposition}
\newtheorem{lemma}{Lemma}
\newtheorem{corollary}{Corollary}
\newtheorem{conjecture}[subsection]{conjecture}
\theoremstyle{definition}
\newtheorem{definition}{Definition}
%
\begin{document}
%\begin{titlepage}
\title{An Exposition of Einstein's ``On the Brownian Motion of Small Suspended
  Particles''}
\author{David Karapetyan }
\address{Department of Mathematics  \\
  University  of Notre Dame\\
  Notre Dame, IN 46556 }
\date{\today}
%
\maketitle
%
%
\parindent0in
\parskip0.1in
%
%\end{titlepage}
%
%
%
\section{Calculating the Osmotic Pressure of Small Particles}
\setcounter{equation}{0}
Consider a tank filled with a volume V of a liquid separated by a
permeable membrane. Dissolve $k$ grams of a non-electrolyte molecule
into part $V^*$ of total volume $V$. Then the membrane will be
subjected to osmotic pressure 
\begin{equation*}
  \begin{split}
    \Pi = \frac{kRT}{NV^*}
    \label{osmotic-pressure}
  \end{split}
\end{equation*}
by the non-electrolyte solution, where
\begin{equation*}
  \begin{split}
    & R = 0.08206 L \cdot atm \cdot mol^{-1} \cdot K^{-1}
    \; \text{(the gas constant)} 
    \\
    & \text{T is the absolute temperature (in Kelvins)}
    \\
    & N = 6.02214179 \times 10^{23} mol^{-1} \; \text{(Avogadro's
      Number)}.
    \label{osmotic-pressure-symbols}
  \end{split}
\end{equation*}
Suppose that, instead of having $k$ grams of dissolved molecules, we had $k$ grams of
suspended molecules; then the classical theory of thermodynamics suggests that no
osmotic pressure would be exerted on the membrane. The author casts doubt on this
conclusion, and will show, using the molecular-kinetic theory of heat, that suspended
bodies will produce the same amount of osmotic pressure as an equal number of dissolved
molecules (assuming the absence of gravity and outside forces).
\\
\hrule
{\bf Remark}. This is one of the three extraordinary insights Einstein includes in
this paper. His contemporaries assumed that the classical theory of thermodynamics
provided a more than adequate view of Brownian motion, and focused instead on
sharpening microscopic observations of thermodynamic results. In earlier work, Felix
Exner assumed both an equipartition of energy between the molecules of the liquid
and the suspended particles, and elastic collisions
between all atoms in the system. Using these two
assumptions, he was able to estimate the average kinetic energy of the molecules in
solution, which in turn allowed him to estimate the mean velocity
of the particles (using the formula $K = \frac{1}{2}mv^2$); however, his results
weren't in agreement with those of his contemporaries. This
should come as no surprise; a random variable undergoing a Brownian path has no
derivative; hence measuring the velocity of a particle undergoing Brownian motion in an
experiment will yield wildly different results with each iteration. As we shall soon
see, Einstein finds a workaround, using his molecular-kinetic theory of osmotic
pressure as a frame.
\\
\hrule
\section{Osmotic Pressure Derived Using the Molecular-Kinetic Theory of Heat}
Let $p_1,p_2,\dots,p_n$ be a sequence of state variables of our system for all atoms
in our tank (including our non-electrolytes). Recall that pressure, temperature,
entropy, enthalpy, and position coordinates and their derivatives are
all examples of state variables. We assume that our state variables will
include all cartesian coordinates of the atoms in question. Furthermore, suppose 
\begin{equation}
  \begin{split}
    \p_t p_v = \phi_v(p_1,p_2,\dots,p_n) \qquad (v=1,2,\dots,n).
    \label{derivative-state-variable}
  \end{split}
\end{equation}
Without loss of generality, we assume $p_1,p_2,\dots,p_j$ are cartesian coordinates of
all the atoms in our system. By \eqref{derivative-state-variable} we know their
velocity at time $t$; hence, our state variables completely determine the system in
question. Since we have assumed beforehand that no outside forces are acting on our system, we must have 
\begin{equation}
  \begin{split}
    \sum_{p_v} \p_t^2 p_v = 0
    \label{energy-conserved-accelerations-cancel}
  \end{split}
\end{equation}
(for example, if our state variables were merely coordinates, then the above asserts
that the sum of the directional accelerations of all particles is zero). Applying the theory
of classical thermodynamics with \eqref{derivative-state-variable} and
\eqref{energy-conserved-accelerations-cancel} we obtain
\begin{equation*}
  \begin{split}
    S = \frac{\bar{E}}{T} + \frac{RT}{N}\ln \int_{p_v} e^{-\frac{EN}{RT}}dp_1\dots dp_n
    \label{entropy-equation}
  \end{split}
\end{equation*}
where 
\begin{equation*}
  \begin{split}
    & T \; \; \text{is the absolute temperature (in Kelvins) of the system}
    \\
    & \bar{E} \; \; \text{is the energy of the system at time} \; \; t
    \\
    & E \; \; \text{is the energy as a function of all possible states}
    \; \; p_v.
    \label{entropy-equation-defs}
  \end{split}
\end{equation*}
We'd like to calculate 
\begin{equation*}
  \begin{split}
    B \doteq \frac{RT}{N}\ln \int_{p_v} e^{-\frac{EN}{RT}}dp_1\dots dp_n.
    \label{B}
  \end{split}
\end{equation*}
As it stands, it is exceedingly
difficult to do so; we aim to simplify our task. Suppose we have $k$
suspended particles $s_1, s_2, \dots, s_k$, with coordinates $(x_1, y_1, z_1)$, $(x_2, y_2, z_2)$, $\dots$,
$(x_k, y_k, z_k)$, respectively. Enclose each suspended particle $s_i$ in a small cubic region $dx_i dy_i dz_i$ such that the union of these cubic regions covers $V^*$. Next, note
that $E$ does not depend on the cartesian coordinates of the atoms and suspended
particles in our system; hence we can write
\begin{equation*}
  dB = dx_1 dy_1\dots dz_k \cdot J
\end{equation*}
where $J$ is independent of the $dx_i dy_i dz_i$. Furthermore, it is well-defined (i.e.\
independent of the positions of the regions we choose to encapsulate the suspended
particles). Hence, we have
\begin{equation*}
  \begin{split}
    B & = \frac{RT}{N} \ln
    \int_{V^*} \left[ J \right] dx_1dy_1dz_1\dots dx_k dy_k dz_k
    \\
    & = \frac{RT}{N} \ln \left[ J(V^*)^k \right]
    \\
    & = \frac{RT}{N}(\ln J + k \ln V^*)
    \label{simplified-B}
  \end{split}
\end{equation*}
from which we obtain
\begin{equation*}
  \begin{split}
    \Pi & = \frac{dB}{dV^*}
    \\
    & = \frac{kRT}{NV^*}.
    \label{pressure-from-molecular-heat}
  \end{split}
\end{equation*}
\\
\hrule
{\bf Remark.} Notice that Einstein has derived the formula for osmotic pressure
without the use of classical thermodynamics. If we assume that the suspended particles
are spherical in shape, then their diffusion coefficient can be derived using the first
law of thermodynamics and the formula for osmotic pressure. The result is 
\begin{equation}
  \begin{split}
    D = \frac{RT}{6 \pi v L N}
    \label{diffusion-coefficient}
  \end{split}
\end{equation}
where $L$ is the radius of the suspended particles, and $v$ is the coefficient of
viscosity of the liquid.
\\
\hrule
\section{Understanding Brownian Motion, and How It Can Be Used To Derive Avogadro's
  Number}
Suppose we have $k$ suspended particles present in a liquid, as before. We now make the
fundamental assumption that the x-coordinates of the particles distribute in a Gaussian
manner, i.e.\ for a time interval $\tau$, the number of particles that will undergo a displacement along the x-axis lying
between $y$ and $y + dy$ can be
expressed as
\begin{equation*}
  \begin{split}
    dk = k \phi(y) d\phi(y)
    \label{1}
  \end{split}
\end{equation*}
where 
\begin{equation*}
  \begin{split}
    \phi_{\sigma, u}(y) \doteq \frac{1}{\sigma \sqrt{2 \pi}} e^{-\frac{(y-u)^2}{2 \sigma^2}}
  \end{split}
\end{equation*}
which we recall is the probability density function of a normal
distribution. Citing \eqref{energy-conserved-accelerations-cancel} as motivation,
we set $u = 0$, obtaining 
\begin{equation}
  \begin{split}
    \phi_{\sigma, 0}(y) \doteq \frac{1}{\sigma \sqrt{2 \pi}} e^{-\frac{y^2}{2 \sigma^2}}.
    \label{2}
  \end{split}
\end{equation}
Now, let $f(x,t): \rr^3 \times \rr \to \rr$ represent the number of particles per unit volume. Then
the number of particles enclosed in a rectangular region with height and width equal
to $1$ and length $dx$ parallel to the $x$-axis is simply $f(x,t) dx$. Hence we have
\begin{equation}
  \begin{split}
    f(x,t + \tau) dx = dx \cdot \int_{y=-\infty}^{y=\infty} f(x+y, t) \phi(y) dy.
    \label{3}
  \end{split}
\end{equation}
Furthermore, we can use a Taylor expansion to write
\begin{equation}
  \begin{split}
    f(x, t + \tau) \approx f(x,t) + \tau \frac{\p f}{\p t}(x,t)
    \label{4}
  \end{split}
\end{equation}
and
\begin{equation}
  \begin{split}
    f(x + y, t) \approx f(x,t) + y \frac{\p f(x,t)}{\p x} +
    \frac{y^2}{2!}\frac{\p^2 f(x,t)}{\p x^2}.
    \label{5}
  \end{split}
\end{equation}
Substituting \eqref{5} and \eqref{4} into \eqref{3}, cancelling $dx$, and dropping the
``approximately equal'' sign in favor of equality, we obtain
\begin{equation}
  \begin{split}
    f(x, t ) + \tau \frac{\p f(x,t)}{\p t}
    & = f(x,t) \int_{-\infty}^{\infty} \phi(y)
    dy
    \\
    & + \frac{\p f(x,t)}{\p x} \int_{-\infty}^{\infty} y \phi(y) dy +
    \frac{\p^2 f(x,t)}{\p x^2} \int_{-\infty}^{\infty} \frac{y^2}{2}\phi(y) dy.
    \label{6}
  \end{split}
\end{equation}
We now aim to simplify \eqref{6}. First, note that 
\begin{equation*}
  \begin{split}
    \int_{-\infty}^{\infty} \phi(y) dy = 1
    \label{7}.
  \end{split}
\end{equation*}
Furthermore, the integrand of the second term of \eqref{6} is an odd function; hence
the second term vanishes. Incorporating these changes we obtain
\begin{equation*}
  \begin{split}
    \cancel{f(x, t )} + \tau \frac{\p f(x,t)}{\p t}
    = \cancel{f(x,t)} +
    \frac{\p^2 f(x,t)}{\p x^2} \int_{-\infty}^{\infty} \frac{y^2}{2}\phi(y) dy.
    \label{8}
  \end{split}
\end{equation*}
Dividing through by $\tau$ and setting 
\begin{equation*}
  \frac{1}{\tau}\int_{-\infty}^{\infty} \frac{y^2}{2}\phi(y) dy = D
\end{equation*}
we obtain the famous heat equation
\begin{equation}
  \begin{split}
    \frac{\p f}{\p t} = D \frac{\p^2 f}{\p x^2}.
    \label{9}
  \end{split}
\end{equation}
Notice now that our previous work extends
without loss of generality over all of $\rr^3$ (we had chosen the simple case of
Brownian dispersion along the x-axis for clarity). To solve \eqref{9}, we need to
identify the initial conditions. Assume that our
non-electroylytic particles are initially suspended at the origin. Then for $x
\lessgtr 0$ and $t=0$ we have
\begin{equation}
  \label{10}
  f(x,t) = 0
\end{equation}
(all particles are still located at the origin at $t=0$; hence the
density of particles outside the origin at time $t=0$ must be zero). Furthermore, since
all particles must be contained in our tank, we have
\begin{equation}
  \begin{split}
    \int_{-\infty}^{\infty} f(x,t) dx = k.
    \label{11}
  \end{split}
\end{equation}
Hence \eqref{9}, \eqref{10}, and \eqref{11} determine an initial value problem with 
solution
\begin{equation}
  \begin{split}
    f(x,t) = \frac{k}{\sqrt {4 \pi Dt}} e^{-\frac{x^2}{4Dt}}.
    \label{12}
  \end{split}
\end{equation}
Recalling \eqref{2}, we conclude from \eqref{12} that
\begin{equation}
  \begin{split}
    f(x,t)
    = k \phi_{\sqrt{2Dt}, 0}(x).
    \label{13}
  \end{split}
\end{equation}
Hence, the variance $\sigma^2$ of the dispersion of the particles is precisely
$2Dt$. 
\\
\hrule
{\bf Remark:} Note that if we choose $D=\frac{1}{2}$ we obtain that the variance of
the dispersion of the particles is $t$; recall that this was one of our fundamental
assumptions when developing the theory of Brownian motion in class. Thus, Einstein has
developed the theory of Brownian motion using the molecular-kinetic framework
introduced earlier. Note how he has presented an alternative, and more effective way
of understanding the motion of particles in a system than by merely trying to measure
their velocities (which turns out to be useless in a system of particles following
Brownian paths, since random variables following Brownian paths have no derivatives).
Einstein's brilliant idea of modelling how the particles spread, on average, leads to a
simple and elegant method for measuring Avogadro's number:
\\
\hrule
By recalling
\eqref{diffusion-coefficient} and substituting in the equality 
\begin{equation*}
  \sigma^2 = 2Dt
\end{equation*}
we can rearrange terms to obtain
\begin{equation}
  N = \frac{t}{\sigma^2} \cdot \frac{RT}{3 \pi v L}.
\end{equation}
Hence the problem of measuring the number of atoms in $k$ grams of our non-electrolytic
molecule is reduced to measuring the variance of the displacement of the particles at
time $t$ and the osmotic pressure generated on the semi-permeable membrane.
\\
\hrule









\end{document}


