%%%%%%%%%%%%%%%%%%%%%%%%%%%%%%%%%%%%%%%%%%%
%   
%                  HR Peakon-antipeakon Interaction      
%                 Alex Himonas & David Karapetyan
%
%%%%%%%%%%%%%%%%%%%%%%%%%%%%%%%%%%%%%%%%%%%%

\documentclass[12pt,reqno]{amsart}

\usepackage{amscd}
\usepackage{amsfonts}
\usepackage{mathtools}
\usepackage{amssymb}
\usepackage[alphabetic, msc-links, abbrev]{amsrefs}
\usepackage{cancel}
%use \cancel{c}  OR  \cancelto{0}{c}
%  \usepackage{ulem}
% use the "sout" tag to "strike through" text
\usepackage{amsthm}
\usepackage{fancyhdr}
\usepackage{latexsym}
\usepackage{yhmath}
\setcounter{secnumdepth}{1} %number only sections, not subsections
\hypersetup{colorlinks=true,
linkcolor=blue,
citecolor=blue,
urlcolor=blue,
}
\synctex=1
\numberwithin{equation}{section}  %eliminate need for keeping track of counters
\numberwithin{figure}{section}
\setlength{\parindent}{0in} %no indentation of paragraphs after section title
\synctex=1




%%%%%%%%%%%%%%%%%%%%%%
\usepackage{color}

\definecolor{Red}{rgb}{1.00, 0.00, 0.00}
\definecolor{DarkGreen}{rgb}{0.00, 1.00, 0.00}
\definecolor{Blue}{rgb}{0.00, 0.00, 1.00}
\definecolor{Cyan}{rgb}{0.00, 1.00, 1.00}
\definecolor{Magenta}{rgb}{1.00, 0.00, 1.00}
\definecolor{DeepSkyBlue}{rgb}{0.00, 0.75, 1.00}
\definecolor{DarkGreen}{rgb}{0.00, 0.39, 0.00}
\definecolor{SpringGreen}{rgb}{0.00, 1.00, 0.50}
\definecolor{DarkOrange}{rgb}{1.00, 0.55, 0.00}
\definecolor{OrangeRed}{rgb}{1.00, 0.27, 0.00}
\definecolor{DeepPink}{rgb}{1.00, 0.08, 0.57}
\definecolor{DarkViolet}{rgb}{0.58, 0.00, 0.82}
\definecolor{SaddleBrown}{rgb}{0.54, 0.27, 0.07}
\definecolor{Black}{rgb}{0.00, 0.00, 0.00}
\definecolor{dark-magenta}{rgb}{.5,0,.5}
\definecolor{myblack}{rgb}{0,0,0}
\definecolor{darkgray}{gray}{0.5}
\definecolor{lightgray}{gray}{0.75}
%%%%%%%%%%%%%%%%%%%%%%






%%%%%%%%%%%%%%%%%%%%%%%%%%%%§ § 
%  For Importing Pictures  %
%%%%%%%%%%%%%%%%%%%%%%%%%%%%


\usepackage[pdftex]{graphicx}
\usepackage{epstopdf}


%% Page Setup %%

\setlength{\textheight}{21.8truecm}
\setlength{\textwidth}{15.0truecm}
\marginparwidth  0truecm
\oddsidemargin   01truecm
\evensidemargin  01truecm
\marginparsep    0truecm
\renewcommand{\baselinestretch}{1.1}

%% New Commands %%

\newcommand{\bigno}{\bigskip\noindent}
\newcommand{\ds}{\displaystyle}
\newcommand{\medno}{\medskip\noindent}
\newcommand{\smallno}{\smallskip\noindent}
\newcommand{\nin}{\noindent}
\newcommand{\ts}{\textstyle}

\newcommand{\rr}{\mathbb{R}}
\newcommand{\p}{\partial}
\newcommand{\zz}{\mathbb{Z}}
\newcommand{\cc}{\mathbb{C}}
\newcommand{\ci}{\mathbb{T}}


\newcommand{\tor}{\mathbb{T}}
\newcommand{\ee}{\varepsilon}
\newcommand{\vp}{\varphi}

\def\makeautorefname#1#2{\expandafter\def\csname#1autorefname\endcsname{#2}}
\makeautorefname{equation}{Equation}
\makeautorefname{footnote}{footnote}
\makeautorefname{item}{item}
\makeautorefname{figure}{Figure}
\makeautorefname{table}{Table}
\makeautorefname{part}{Part}
\makeautorefname{appendix}{Appendix}
\makeautorefname{chapter}{Chapter}
\makeautorefname{section}{Section}
\makeautorefname{subsection}{Section}
\makeautorefname{subsubsection}{Section}
\makeautorefname{paragraph}{Paragraph}
\makeautorefname{subparagraph}{Paragraph}
\makeautorefname{theorem}{Theorem}
\makeautorefname{theo}{Theorem}
\makeautorefname{thm}{Theorem}
\makeautorefname{addendum}{Addendum}
\makeautorefname{add}{Addendum}
\makeautorefname{maintheorem}{Main theorem}
\makeautorefname{corollary}{Corollary}
\makeautorefname{lemma}{Lemma}
\makeautorefname{sublemma}{Sublemma}
\makeautorefname{proposition}{Proposition}
\makeautorefname{property}{Property}
\makeautorefname{scholium}{Scholium}
\makeautorefname{step}{Step}
\makeautorefname{conjecture}{Conjecture}
\makeautorefname{question}{Question}
\makeautorefname{definition}{Definition}
\makeautorefname{notation}{Notation}
\makeautorefname{remark}{Remark}
\makeautorefname{remarks}{Remarks}
\makeautorefname{example}{Example}
\makeautorefname{algorithm}{Algorithm}
\makeautorefname{axiom}{Axiom}
\makeautorefname{case}{Case}
\makeautorefname{claim}{Claim}
\makeautorefname{assumption}{Assumption}
\makeautorefname{conclusion}{Conclusion}
\makeautorefname{condition}{Condition}
\makeautorefname{construction}{Construction}
\makeautorefname{criterion}{Criterion}
\makeautorefname{exercise}{Exercise}
\makeautorefname{problem}{Problem}
\makeautorefname{solution}{Solution}
\makeautorefname{summary}{Summary}
\makeautorefname{operation}{Operation}
\makeautorefname{observation}{Observation}
\makeautorefname{convention}{Convention}
\makeautorefname{warning}{Warning}
\makeautorefname{note}{Note}
\makeautorefname{fact}{Fact}


\def\refer #1\par{\noindent\hangindent=\parindent\hangafter=1 #1\par}

%% Equation Numbers %%

\renewcommand{\theequation}{\thesection.\arabic{equation}}

%% New Environments %%

%\swapnumbers
\theoremstyle{plain}  % default
\newtheorem{theorem}{Theorem}
\newtheorem{proposition}{Proposition}
\newtheorem{lemma}{Lemma}
\newtheorem{corollary}{Corollary}
\newtheorem{conjecture}[subsection]{Conjecture}

\theoremstyle{definition}
\newtheorem{definition}{Definition}

\begin{document}

%\topmargin0.1truecm

%%% Title %%%

%\begin{titlepage}
  \title{ HR Peakon-Antipeakon Interactions}

  \author{{\it Alex Himonas \& David Karapetyan}}
  %\author{Alex Himonas}
  %\address{Department of Mathematics  \\
  %       University of Notre Dame     \\
  %         Notre Dame, IN 46556}




  \date{October 22, 2010}




  \maketitle
  \markboth{HR Peakon-Antipeakon Interactions}
  {Alex  Himonas    \& David Karapetyan}


  \parindent0in
  \parskip0.1in

%\end{titlepage}



%%%%%%%%%%%%%%%%%%%%%%%%
%
%      Introduction
%
%%%%%%%%%%%%%%%%%%%%%%%%%%%%%%%%%%%%%%%%%%%%



\section{Objective}
%
%
Let the hyperelastic-rod (HR) initial value problem (ivp) be written in the following local form
%
%
%
\begin{gather}
  \label{HR-1}
  \left( 1- \p_x^2 \right)\left( \p_t u + \frac{\gamma}{2} \p_x(u^2) \right) +
  \p_x\left[ \frac{3-\gamma}{2}u^{2}  + \frac{\gamma}{2}  \left( \p_x u  \right)^2\right] = 0,
  \\
  \label{hr-eq-init-data}
  u(x,0) = u_0(x), \quad x \in \ci \ \text{or} \ \rr, \quad t \in \rr.
\end{gather}
%
%
%
%
We wish to prove the following.
%
%
%%%%%%%%%%%%%%%%%%%%%%%%%%%%%%%%%%%%%%%%%%%%%%%%%%%%%
%
%
%				 Main Theorem
%
%
%%%%%%%%%%%%%%%%%%%%%%%%%%%%%%%%%%%%%%%%%%%%%%%%%%%%%
%
%
\begin{theorem}
  The HR ivp is ill-posed for $1 < s < 3/2$. More precisely, for $\ee >0$, there
  exists a solution $u \in C([0, T): H^s))$ to \eqref{HR-1}
  with existence time $T \le \ee$, such that
  %
  %
  \begin{equation*}
    \begin{split}
      \lim_{t \to T^{-}} \|u(t)\|_{H^s} = \infty \quad \text{and} \quad
      \|u(0)\|_{H^s} \le \ee.
    \end{split}
  \end{equation*}
  %
  %
\end{theorem}
Following \cite{Byers-2006-Existence-time-for-the-Camassa-Holm}, we tackle the
non-periodic and periodic cases separately.
%
%
%%%%%%%%%%%%%%%%%%%%%%%%%%%%%%%%%%%%%%%%%%%%%%%%%%%%%
%
%
%                Traveling Wave Solutions CH and HR
%
%
%%%%%%%%%%%%%%%%%%%%%%%%%%%%%%%%%%%%%%%%%%%%%%%%%%%%%
%
%
\section{Traveling Wave Solutions} 
\label{sec:trav-wav-soln}
We want to show that the following function
%
\begin{equation}
  \label{trav-wav-HR-trav}
  u(x,t)= p e^{-| x + q |}, 
\end{equation}
%
is a traveling wave solution for the HR equation for suitable $p=p(t),
q=q(t)$. We begin with the the definition
%
\begin{equation}
  \label{sign-funct}
  \sigma(x) \doteq \text{sign}(x) = \frac{|x|}{x} 
  = 
  \begin{cases}
    1, \; \; \; \; \; x > 0 \\
    -1, \; \; x < 0.
  \end{cases}
\end{equation}
%
We have the formula
% 
%
\begin{equation}
  \label{sign-deriv1}
  \p_x e^{-|x|}= -\sigma(x) e^{-|x|}.
\end{equation}
%
Also,  we have 
%
\begin{equation}
  \label{sign-deriv2}
  \p_x\sigma(x-q) = 2\delta_q,
  \quad
  \text{and}
  \quad
  \p_x\sigma(x+q) = 2\delta_{-q}.
\end{equation}
%
We verify only the first relation. The second is similar.
For this we choose a test function $\varphi(x)$ and we have
%
\begin{align*}
  \langle
  \p_x\sigma(x-q), \varphi(x)
  \rangle
  &=
  -\langle
  \sigma(x-q),  \p_x\varphi(x)
  \rangle
  \\
  &=
  -
  \int_\rr
  \sigma(x-q) \varphi'(x) dx 
  \\
  &=
  -
  \int_{-\infty}^q
  \sigma(x-q) \varphi'(x) dx 
  -
  \int_q^\infty
  \sigma(x-q) \varphi'(x) dx 
  \\
  &=
  \int_{-\infty}^q \varphi'(x) dx 
  -
  \int_q^\infty \varphi'(x) dx 
  \\
  &=
  \varphi(x)\Big|_{-\infty}^q
  -
  \varphi(x)\Big|_q^\infty 
  \\
  &=
  \varphi(q)- \varphi(-\infty) 
  -
  [\varphi(\infty)- \varphi(q) ]
  \\
  &=
  \varphi(q)- 0
  -
  [0- \varphi(q) ]
  \\
  &=
  2\varphi(q)
  \\
  &=
  \langle
  \delta_q, \varphi(x)
  \rangle.
\end{align*}
{\bf  Computing the term $\p_t u-\p_{x}^2\p_t u$:}  Using the above formulas we get
%
\begin{align*}
  &\p_t u
  =
  p'e^{-|x+q|} 
  -
  pq'\sigma(x+q)e^{-|x+q|}.
\end{align*}
%
Therefore, we have 
%
\begin{align*}
  \p_x\p_t u
  = 
  &-p' \sigma(x+q) e^{-|x+q|}
  + 
  pq'e^{-|x+q|} 
  -
  2pq' \delta_{-q}
\end{align*}
%
and 
%
\begin{align*}
  \p_x^2\p_t u
  &=
  p'e^{-|x+q|}
  -
  pq'\sigma(x+q)e^{-|x+q|}
  -2p' \delta_{-q} - 2pq'\delta_{-q}' 
\end{align*}
%
We now observe that in fact
%
\begin{align}
  \label{deriv-t-term-trav-wav}
  u_t - u_{txx} = 2p' \delta_{-q} + 2pq'\delta_{-q}' 
\end{align}
%
{\bf Computing the Term $\displaystyle \frac{3-\gamma}{2} \p_x (u^{2})$ and $\displaystyle \frac{\gamma}{2} \p_x (u^{2})$.}
We have
%
%
%
%
\begin{gather}
  \notag \frac{3-\gamma}{2}\p_x (u^{2}) 
  = -(3-\gamma) p^{2}e^{-2| x+q |} \sigma(x+q),
  \\
  \label{deriv-u-sq-HR-trav}
  \frac{\gamma}{2}\p_x (u^{2}) 
  = -\gamma p^{2}e^{-2| x+q |} \sigma(x+q). 
\end{gather}
%
{\bf Computing the term $\displaystyle \frac{\gamma}{2} \p_x^3 (u^2)$.}
Taking one derivative of the right hand side of \eqref{deriv-u-sq-HR-trav}, we obtain 
%
\begin{equation*}
  \begin{split}
    \frac{\gamma}{2}\p_x^2 (u^{2})
    = \gamma p^{2}\left[ 2 e^{-2| x+q |} -2 \delta_{-q}  
    \right].
  \end{split}
\end{equation*}
%
and hence
\begin{equation}
  \label{deriv-trip-u-sq-HR-trav}
  \begin{split}
    \frac{\gamma}{2}\p_x^3 (u^{2})
    & = \gamma p^{2}\left[ -4 e^{-2| x+q |} \sigma(x+q) -2 \delta_{-q}'
    \right].  
  \end{split}
\end{equation}
%
{\bf Computing the term $\displaystyle \frac{\gamma}{2}\p_x [(\p_x u)^2]$.}
Since 
%
\begin{align*}
  \p_x u = -p\sigma(x+q) e^{-|x+q|} 
\end{align*}
%
we have
%
\begin{align*}
  (\p_x u)^2
  &= 
  p^2 e^{-2|x+q|}.
\end{align*}
%
%
Therefore,
%
%
\begin{equation}
  \begin{split}
    \label{deriv-deriv-u-sq2-HR-trav}
    \frac{\gamma}{2}\p_x (\p_x u)^2
    = -\gamma p^2 e^{-2 | x+q |} \sigma(x+q). 
  \end{split}
\end{equation}
%
{\bf Substituting into the HR equation.}
%
Applying \eqref{deriv-t-term-trav-wav}, \eqref{deriv-u-sq-HR-trav},
\eqref{deriv-trip-u-sq-HR-trav}, and \eqref{deriv-deriv-u-sq2-HR-trav} to the
the HR equation, we
obtain
%
%
%
\begin{gather}
  \label{ode-term-1-HR-trav}
  2p'\delta_{-q} 
  \\
  \label{ode-term-2-HR-trav}
  + \delta_{-q}'  \left [2 \gamma p^{2}  + 2pq' \right ]
  \\
  \label{problem-term-HR-trav}
  +3(\gamma -1)p^{2} e^{-2| x+q |}\sigma(x+q) 
\end{gather}
%
Notice that for $\gamma \neq 1$, we do not have an exact solution of form
$\eqref{trav-wav-HR-trav}$. 
%
%
\section{Peakon-Antipeakon Traveling Wave Solutions}
\label{sec:peakon-antipeakon}
We want to show that the following function
%
\begin{equation}
  \label{peakon-anti}
  u(x,t)= p e^{-| x + q |} - p e^{-| x-q |},
\end{equation}
%
where $p=p(t)$ and  $q=q(t)$ are positive functions of $t$,
is a peakon anti-peakon weak traveling wave solution if
%
\begin{align}
  &p' = p^2 e^{-2q}, \\
  &q' = p (e^{-2q} -1).
\end{align}
%
%
%
%
{\bf  Computing the term $\p_t u-\p_{x}^2\p_t u$:}  Using the above formulas we get
%
\begin{align*}
  &\p_t u
  =
  p'e^{-|x+q|} 
  - 
  p'e^{-|x-q|}
  -
  pq'\sigma(x+q)e^{-|x+q|}
  -
  pq'\sigma(x-q)e^{-|x-q|}.
\end{align*}
%
Therefore, we have 
%
\begin{align*}
  \p_x\p_t u
  = 
  &-p' \sigma(x+q) e^{-|x+q|}
  +
  p'\sigma(x-q)e^{-|x-q|} 
  + 
  pq'e^{-|x+q|} 
  +
  pq' e^{-|x-q|}\\
  &-
  2pq' \delta_{-q}
  -2pq'\delta_q
\end{align*}
%
and 
%
\begin{align*}
  \p_x^2\p_t u
  &=
  p'e^{-|x+q|}
  -
  p'e^{-|x-q|}
  -
  pq'\sigma(x+q)e^{-|x+q|}
  -
  pq'\sigma(x-q)e^{-|x-q|} \\
  &-2p' \delta_{-q} + 2p'\delta_q - 2pq'\delta_{-q}' - 2pq'\delta_q'
\end{align*}
%
We now observe that in fact
%
\begin{align}
  \label{deriv-t-term}
  u_t - u_{txx} = 2p' \delta_{-q} - 2p'\delta_q + 2pq'\delta_{-q}' + 2pq'\delta_q'
\end{align}
%
\medno
{\bf Computing the Term $\displaystyle \frac{3-\gamma}{2} \p_x (u^{2})$ and
$\displaystyle \frac{\gamma}{2} \p_x (u^{2})$.}
We have
%
%
\begin{equation*}
  \begin{split}
    u^{2} = p^{2}\left( e^{-2| x+q |} + e^{-2| x-q |} - 2e^{-| x+q
    |}e^{-| x-q |} \right).
  \end{split}
\end{equation*}
%
%
To compute the weak derivative of $u^{2}$, we need to compute the weak
derivative of $e^{-| x+q |}e^{-| x-q |}$. That is
%
%
%
%
\begin{equation*}
  \begin{split}
    & <\p_x\left( e^{-| x+q |}e^{-| x-q |}\right), \vp(x)>
    \\
    & = - <e^{-| x+q |} e^{| x-q |}, \vp'(x)>
    \\
    & = - \int_{\rr}e^{-| x+q |}e^{-| x-q |}\vp'(x) dx
    \\
    & = - \left[ \int_{-\infty}^{-q}e^{-| x+q |}e^{-| x-q |} \vp'(x) dx +
    \int_{-q}^{q}e^{-| x+q |}e^{-| x-q |} \vp'(x) dx +
    \int_{q}^{\infty}e^{-| x+q |}e^{-| x-q |}\vp'(x) dx \right]
    \\
    & = -\left[ \int_{-\infty}^{-q}e^{x+q}e^{x-q} \vp'(x) dx +
    \int_{-q}^{q}e^{-(x+q)}e^{x-q}\vp'(x) dx +
    \int_{q}^{\infty}e^{-(x+q)}e^{-(x-q)} \vp'(x) dx \right]
    \\
    & = -\left[ \int_{-\infty}^{-q}e^{2x}\vp'(x) dx +
    \int_{-q}^{q}e^{-2q}\vp'(x) dx + \int_{q}^{\infty}e^{-2x} \vp'(x) dx \right]
    \\
    & = -\left\{ \cancel{e^{2x} \vp(x) \Big |_{-\infty}^{-q}} -2 \int_{-\infty}^{-q}
    e^{2x} \vp(x) dx + \cancel{e^{-2q}\left[ \vp(q) - \vp(-q)
    \right]} \right.
    \\
    & \left . \phantom{=} + \cancel{e^{-2x}\vp(x) \Big |_{q}^{\infty}} + 2
    \int_{q}^{\infty}e^{-2x}\vp(x) dx \right\}
    \\
    & = 2 <e^{2x}\chi_{(-\infty, -q)}, \vp(x)>
    - 2 <e^{-2x}\chi_{(q, \infty)}, \vp(x)>.
  \end{split}
\end{equation*}
%
%
Therefore,
%
%
\begin{equation}
  \label{deriv-u-sq}
  \begin{split}
    & \frac{3-\gamma}{2}\p_x (u^{2}) 
    \\
    & = (3-\gamma) p^{2}\left\{ - e^{-2| x+q |} \sigma(x+q) -  e^{-2| x-q
    |}\sigma(x-q) - 2e^{2x}\chi_{(-\infty, -q)} + 2e^{-2x} \chi_{(q, \infty)}
    \right \},
    \\
    & \frac{\gamma}{2}\p_x (u^{2}) 
    \\
    & = \gamma p^{2}\left\{ - e^{-2| x+q |} \sigma(x+q) -  e^{-2| x-q
    |}\sigma(x-q) - 2e^{2x}\chi_{(-\infty, -q)} + 2e^{-2x} \chi_{(q, \infty)}
    \right \}.
  \end{split}
\end{equation}
%
{\bf Computing the term $\displaystyle \frac{\gamma}{2} \p_x^3 (u^2)$.}
It will be enough  
to compute the weak second derivatives of all
bracketed terms in \eqref{deriv-u-sq}. First, we note that
%
%
%
%
%
\begin{equation*}
  \begin{split}
    & \p_x \left[ e^{-2| x+q |} \sigma(x+q) \right]
    \\
    & = -2e^{-2| x+q |} + \left[ e^{-2| -q^{+} + q |} \sigma(-q^{+} + q) -
    e^{-2| -q^{-} + q |} \sigma(-q^{-} + q) \right]\delta_{-q}
    \\
    & = -2e^{-2| x+q |} + 2 \delta_{-q}
  \end{split}
\end{equation*}
%
%
which implies
%
%
\begin{equation}
  \label{7aa}
  \begin{split}
    \p_x^2 \left[ e^{-2| x+q |} \sigma(x+q) \right] = 4e^{-2| x+q |}
    \sigma(x+q) + 2 \delta'_{-q}.
  \end{split}
\end{equation}
%
%
Similarly, 
%
%
\begin{equation}
  \label{8aa}
  \begin{split}
    \p_x^{2}\left[ e^{-2| x-q |}\sigma(x-q) \right] = 4e^{-2| x-q
    |}\sigma(x-q) + 2 \delta'_{q}.
  \end{split}
\end{equation}
%
Furthermore, we have
\begin{equation*}
  \begin{split}
    & \p_x(e^{2x} \chi_{(-\infty, -q)})
    \\
    & = 2e^{2x} \chi_{(-\infty, -q)} + \left[ e^{-2q^{+}}\chi_{(-\infty, -q)}
    (-q^+) - e^{-2q^{-}} \chi_{(-\infty, -q)} (-q^{-}) \right]\delta_{-q}
    \\
    & = 2e^{2x} \chi_{(-\infty, -q)} - e^{-2q} \delta_{-q}
  \end{split}
\end{equation*}
%
%
which gives
%
%
\begin{equation}
  \label{5aa}
  \begin{split}
    \p_x^{2}(e^{2x} \chi_{(-\infty, -q)}) = 4e^{2x} \chi_{(-\infty, -q)} -
    2e^{-2q} \delta_{-q} - e^{-2q} \delta'_{-q}.
  \end{split}
\end{equation}
%
%
We also have
%
%
\begin{equation*}
  \begin{split}
    \p_x (e^{-2x} \chi_{(q, \infty)})
    & = -2e^{-2x} \chi_{(q, \infty)} + \left[ e^{-2q^{+}} \chi_{(q,
    \infty)}(q^{+}) - e^{-2q^{-}} \chi_{(q, \infty)}(q^{-})
    \right]\delta_{q}
    \\
    & = -2e^{-2x} \chi_{(q, \infty)} + e^{-2q} \delta_{q}
  \end{split}
\end{equation*}
%
%
and so
%
%
\begin{equation}
  \label{6aa}
  \begin{split}
    \p_x^2 (e^{-2x} \chi_{(q, \infty)})
    & = 4e^{-2x} \chi_{(q, \infty)} - 2e^{-2q} \delta_{q} + e^{-2q}
    \delta'_{q}.
  \end{split}
\end{equation}
%
%
%
%
From \eqref{deriv-u-sq}-\eqref{8aa}, we conclude that
%
%
\begin{equation}
  \label{deriv-trip-u-sq}
  \begin{split}
    \frac{\gamma}{2}\p_x^3 (u^{2})
    & = \gamma p^{2}\left[ -4 e^{-2| x+q |} \sigma(x+q) - 4e^{-2| x-q
    |}\sigma(x-q) -2 (\delta_{-q}' + \delta_{q}') \right.
    \\
    & - 8e^{2x} \chi_{(-\infty, -q)} + 4e^{-2q} \delta_{-q} +
    2e^{-2q} \delta_{-q}'
    \\
    & \left. + 8e^{-2x}\chi_{(q, \infty)} -4 e^{-2q}
    \delta_{q} + 2e^{-2q} \delta_{q}' \right].
  \end{split}
\end{equation}
%
%
{\bf Computing the term $\displaystyle \frac{\gamma}{2}\p_x [(\p_x u)^2]$.}
Since 
%
\begin{align*}
  u_x = -p\sigma(x+q) e^{-|x+q|} + p\sigma(x-q)e^{-|x-q|}
\end{align*}
%
we have
%
\begin{align*}
  (\p_x u)^2
  &= 
  p^2 e^{-2|x+q|} + p^2e^{-2|x-q|} \\
  &-2p^2\sigma(x+q) \sigma(x-q) e^{-|x+q|} e^{-|x-q|}.
\end{align*}
%
To compute   the weak derivative of $(\p_x u)^2$ we need 
to compute the weak derivative of 
$\sigma(x+q) \sigma(x-q) e^{-|x+q|} e^{-|x-q|}$, which we do next.
For  $\varphi$ a test function we have
%
%
%
\begin{align*}
  &\langle
  \p_x[
  \sigma(x+q) \sigma(x-q) e^{-|x+q|} e^{-|x-q|}], \varphi(x)
  \rangle=
  \\
  &=
  -
  \langle
  \sigma(x+q) \sigma(x-q) e^{-|x+q|} e^{-|x-q|}, \p_x\varphi(x)
  \rangle
  \\
  &=
  -\int_\rr
  \Big[
  \sigma(x+q) \sigma(x-q) e^{-|x+q|} e^{-|x-q|} 
  \Big]
  \varphi'(x) dx \\
  &=
  - \int_{-\infty}^{-q}
  (-1)(-1) e^{-(-(x+q))} e^{-(-(x-q))} \varphi'(x) dx \\
  &-
  \int_{-q}^q
  (+1)(-1) e^{-(x+q)} e^{-(-(x-q))} \varphi'(x) dx \\
  &-
  \int_q^{\infty} (+1)(+1) e^{-(x+q)} e^{-(x-q)} \varphi'(x) dx \\
  &=
  - e^{-2q} \varphi(-q)
  +2 \int_{-\infty}^{-q} e^{2x} \varphi(x) dx  \\
  &
  +e^{-2q} \varphi(q) 
  -e^{-2q}  \varphi(-q) \\
  &+
  e^{-2q}\varphi(q) 
  -2 \int_q^{\infty} e^{-2x} \varphi(x) dx \\
  &=
  2 <
  e^{2x} \chi_{(-\infty, -q)}, \varphi(x)> 
  -2  < e^{-2x} \chi_{(q, \infty)}, \varphi(x)> \\
  &
  + 2 e^{-2q}\left( < \delta_q, \varphi(x)>
  - < \delta_{-q}, \varphi(x)>\right).
\end{align*}
%
Therefore,
%
%
\begin{equation}
  \begin{split}
    \label{deriv-deriv-u-sq2}
    \frac{\gamma}{2}\p_x (\p_x u)^2
    = \gamma p^2 & \left\{ - e^{-2 | x+q |} \sigma(x+q) -  e^{-2| x-q
    |}\sigma(x-q) \right.
    \\
    & \left. - 2 e^{2x} \chi_{(-\infty, -q)} 
    + 2 e^{-2x} \chi_{(q, \infty)}
    - 2e^{-2q}(\delta_{q} - \delta_{-q}) \right \}
  \end{split}
\end{equation}
%
%
%
%
%
%
\medno
{\bf Substituting into the HR equation.}
Applying	\eqref{deriv-t-term},
\eqref{deriv-u-sq}, \eqref{deriv-trip-u-sq}, and \eqref{deriv-deriv-u-sq2} to the
HR equation \eqref{HR-1}, we
obtain
%
%
%
\begin{gather}
  \label{ode-term-1}
  \delta_{q}\left [(4-2\gamma)p^{2}e^{-2q} - 2p' \right ]
  + \delta_{-q} \left [2p' - (4-2 \gamma)p^{2}e^{-2q} \right ]
  \\
  \label{ode-term-2}
  + \delta_{q}'\left [2\gamma p^2 - 2 \gamma p^2e^{-2q} + 2 pq' \right ]
  + \delta_{-q}'  \left [2 \gamma p^{2} - 2 \gamma p^{2}e^{-2q} + 2pq' \right ]
  \\
  \label{problem-term}
  3(\gamma -1)p^{2}\left[ e^{-2| x+q |}\sigma(x+q) + e^{-2| x-q |}
  \sigma(x-q) \right.
  \\
  \notag
  + \left. 2e^{2x} \chi_{(-\infty, -q)} - 2e^{-2x} \chi_{(q, \infty)} \right ].
\end{gather}
%
Setting 
\begin{equation}
  \label{dif-eq}
  \begin{split}
    & p' =\frac{4-2\gamma}{2} p^2 e^{-2q}, 
    \\
    & q'=p\gamma(e^{-2q} -1).
  \end{split}
\end{equation}
we see that terms \eqref{ode-term-1} and \eqref{ode-term-2} vanish. 
%
\textbf{TERM \eqref{problem-term} PREVENTS US FROM GETTING A USEFUL ODE FOR $\gamma
\neq1$.}
%
\\
\\
Things are no better if we take
%
%
\begin{equation*}
  \begin{split}
    u(x,t) = p e^{-c| x + q |} - p e^{-c| x-q |}
  \end{split}
\end{equation*}
%
%
for some $c \in \rr$. To see why, we compute the term $\p_t u-\p_{x}^2\p_t u$.
Following our earlier computations, we get 
%
\begin{align*}
  &\p_t u
  =
  p'e^{-c|x+q|} 
  - 
  p'e^{-c|x-q|}
  -
  cpq'\sigma(x+q)e^{-c|x+q|}
  -
  cpq'\sigma(x-q)e^{-c|x-q|}.
\end{align*}
%
Therefore, we have 
%
\begin{align*}
  \p_x\p_t u
  = 
  &-cp' \sigma(x+q) e^{-c|x+q|}
  +
  cp'\sigma(x-q)e^{-c|x-q|} 
  + 
  c^2pq'e^{-c|x+q|} 
  +
  c^2pq' e^{-c|x-q|}\\
  &-
  2cpq' \delta_{-q}
  -2cpq'\delta_q
\end{align*}
%
and 
%
\begin{align*}
  \p_x^2\p_t u
  &=
  c^2p'e^{-c|x+q|}
  -
  c^2p'e^{-c|x-q|}
  -
  c^3pq'\sigma(x+q)e^{-c|x+q|}
  -
  c^3pq'\sigma(x-q)e^{-c|x-q|} \\
  &-2cp' \delta_{-q} + 2cp'\delta_q - 2cpq'\delta_{-q}' - 2cpq'\delta_q'.
\end{align*}
%
We now observe that in fact
%
\begin{align}
  u_t - u_{txx} &= 2p' \delta_{-q} - 2p'\delta_q + 2pq'\delta_{-q}' + 2pq'\delta_q'
  \\
  \label{prob-term}
  & + (1-c^2)p'e^{-c| x+q |} - (1 -c^2)p'e^{-c| x-q |}
  \\
  & -c(1-c^{2})pq'\sigma(x+q)e^{-c| x+q |} - c(1-c^{2})pq' \sigma(x-q)e^{-c| x-q
  |}.
\end{align}
We have no hope that \eqref{prob-term} will be cancelled by remaining terms of
the HR equation, since none of them will have a $p'$ component.
%
%
%
%%%%%%%%%%%%%%%%%%%%%%%%%%%%%%%%%%%%%%%%%%%%%%%%%%%%%
%
%
%				 A Simpler Computation, To check last section
%
%
%%%%%%%%%%%%%%%%%%%%%%%%%%%%%%%%%%%%%%%%%%%%%%%%%%%%%
%
%
\section{A Simpler Computation, To Double Check
\autoref{sec:peakon-antipeakon}.}	
We consider the function \eqref{peakon-anti} as before, and assume $x >q$. Then
%
%
\begin{equation*}
  \begin{split}
    u(x,t) & = p\left[ e^{-(x+q)} - e^{-(x-q)} \right]
    \\
    & = p e^{-x}\left[ e^{-q} - e^{q} \right].
  \end{split}
\end{equation*}
%
%
{\bf Computing the term $ \displaystyle (1-\p_x^2) \p_t u$.}
We have
%
%
\begin{equation*}
  \begin{split}
    \p_t u
    & = pe^{-x}\left[ -q'e^{-q} - q' e^{q} \right] + p'e^{-x}\left[
    e^{-q}-e^{q}
    \right]
    \\
    & = -pe^{-x}q'\left[ e^{-q} + e^{q} \right] + p'e^{-x}\left[ e^{-q} -
    e^{q}
    \right]
    \\
    & = e^{-x}\left( e^{-q}-e^{q} \right)\left( p' - pq' \right).
  \end{split}
\end{equation*}
%
%
Therefore,
%
%
\begin{equation}
  \label{pseudo-dtu}
  \begin{split}
    \left( 1- \p_x^2 \right) \p_t u = 0.
  \end{split}
\end{equation}
%
%
{\bf Computing the term $ \displaystyle \left( 1- \p_x^2 \right)\left[
\frac{\gamma}{2} \p_x(u^2) \right] $.}
We have
%
%
\begin{equation*}
  \begin{split}
    u^{2} = p^{2}e^{-2x}\left[ e^{-2q} + e^{2q}-2 \right]
  \end{split}
\end{equation*}
%
%
and so
%
%
\begin{equation*}
  \begin{split}
    \p_x(u^{2}) = -2p^{2}e^{-2x}\left[ e^{-2q} + e^{2q} -2 \right]
  \end{split}
\end{equation*}
%
%
from which we obtain
%
%
\begin{equation*}
  \begin{split}
    \frac{\gamma}{2} \p_{x}(u^{2}) = - \gamma p^{2}e^{-2x}\left[ e^{-2q} +
    e^{2q} -2
    \right].
  \end{split}
\end{equation*}
%
%
Furthermore,
%
%
\begin{equation*}
  \begin{split}
    \p_x(u^{2}) = -8p^{2}e^{-2x}\left[ e^{-2q} + e^{2q}-2 \right]
  \end{split}
\end{equation*}
%
%
from which we obtain
%
%
\begin{equation*}
  \begin{split}
    \frac{\gamma}{2} \p_x^3(u^{2}) = -4 \gamma p^{2}e^{-2x}\left[ e^{-2q}+
    e^{2q} -2
    \right].
  \end{split}
\end{equation*}
%
%
Therefore,
%
%
\begin{equation}
  \label{pseudo-dx-u-sq}
  \begin{split}
    \left( 1 - \p_x^2 \right)\left[ \frac{\gamma}{2} \p_x(u^{2}) \right] =
    3\gamma p^{2} e^{-2x}\left[ e^{-2q} + e^{2q}-2 \right].
  \end{split}
\end{equation}
%
%
{\bf Computing the term $ \displaystyle \p_x \left[ \frac{3-\gamma}{2}u^{2} \right] $.}
We have
%
%
\begin{equation}
  \label{pseudo-dx-3-gam-u-sq}
  \begin{split}
    \p_x \left[ \frac{3-\gamma}{2}u^{2} \right] = -(3- \gamma)p^{2}e^{-2x}\left[
    e^{-2q} + e^{2q}-2 \right].
  \end{split}
\end{equation}
%
%
{\bf Computing the term $\displaystyle \p_x \left[ \frac{\gamma}{2}(\p_x
u)^{2} \right]$.}
We have
%
%
\begin{equation*}
  \begin{split}
    \p_x u = -p e^{-x}\left[ e^{-q} - e^{q} \right]
  \end{split}
\end{equation*}
%
%
and so
%
%
\begin{equation*}
  \begin{split}
    (\p_x u)^{2} = p^{2} e^{-2x}\left[ e^{-2q} + e^{2q} -2 \right].
  \end{split}
\end{equation*}
%
%
Therefore,
%
%
%
%
\begin{equation}
  \label{del-x-delx-u-sq}
  \begin{split}
    \p_x\left[ \frac{\gamma}{2}(\p_x u )^{2}\right] = - \gamma p^{2}e^{-2x}\left[
    e^{-2q} + e^{2q} -2 \right].
  \end{split}
\end{equation}
%
%
Substituting \eqref{pseudo-dtu}-\eqref{del-x-delx-u-sq} into the left-hand-side
of the HR equation yields
%
%
\begin{gather*}
  3 \gamma p^{2}e^{-2x}\left[ e^{-2q} + e^{2q} -2 \right]
  - (3 -\cancel{\gamma})p^{2}e^{-2x}\left[ e^{-2q} + e^{2q}-2 \right]
  \\
  - \cancel{\gamma
  p^{2}e^{-2x}\left[ e^{-2q} + e^{2q}-2 \right]}
\end{gather*}
%
%
or
%
%
\begin{equation}
  \label{fund-relation-simp-case}
  \begin{split}
    3(\gamma -1)p^{2}e^{-2x}\left[ e^{-2q} + e^{2q}-2 \right].
  \end{split}
\end{equation}
%
%
Hence, when $\gamma \neq 1$, and $p$ and $q$ are non-trivial, there are no solutions of form \eqref{peakon-anti}
to the HR equation. Note that when $x > q$, \eqref{ode-term-1} and
\eqref{ode-term-2} vanish (since $\delta_{c}(x) = 0$ for all $x \neq c$), while \eqref{problem-term}
simplifies to \eqref{fund-relation-simp-case}.
%
%
%
%
%%%%%%%%%%%%%%%%%%%%%%%%%%%%%%%%%%%%%%%%%%%%%%%%%%%%%
%
%
%                Another Weak Deriv Computation
%
%
%%%%%%%%%%%%%%%%%%%%%%%%%%%%%%%%%%%%%%%%%%%%%%%%%%%%%
%
%
\section{Difference of Traveling Wave Solutions Computation} 
\label{sec:dif-trav-wave-comp}
Let $u(x,t), v(x,t)$ be solutions to the HR ivp. Then $u -v$ is a solution if
and only if
%
%
\begin{equation*}
  \begin{split}
    0 & = \p_t(u -v) - \p_t \p_x^{2}(u -v) + 3(u-v)\p_x(u-v)
    \\
    & - 2 \gamma
    \p_x(u-v)\p_x^{2}(u-v) - \gamma (u-v)\p_x^{3}(u-v)
    \\
    & = -3(u \p_x v + v \p_x u) + 2\gamma(\p_x u \p_x^{2}v + \p_x v \p_x^2 u) +
    \gamma(u \p_x^3 v + v \p_x^3 u).
  \end{split}
\end{equation*}
%
%
or, equivalently,
%
%
\begin{equation*}
  \begin{split}
    0
    & = -3 \p_x(uv) + 2 \gamma \p_x\left( \p_x u \p_x v \right)
    \\
    & + \gamma \left[ \p_x(u \p_x^2v - \int \p_x u \p_x^2 v dx + v \p_x^2 u - \int
    \p_x v \p_x^2 u dx) \right]
    \\
    & = -3 \p_x(uv) + 2 \gamma \p_x (\p_x u \p_x v)
    \\
    & + \gamma\left[ \p_x (u \p_x^2 v - \int \p_x(\p_x u \p_x v) + v \p_x^2 u) \right]
    \\
    & = - 3 \p_x(uv) + 2 \gamma \p_x(\p_x u \p_xv)
    \\
    & + \gamma \p_x(u \p_x^2 v - \p_x u \p_x v + v \p_x^2 u)
    \\
    & = - 3 \p_x(uv) + \gamma \p_x(u \p_x^2 v + \p_x u \p_x v + v \p_x^2 u).
  \end{split}
\end{equation*}
%
%
or, equivalently (by integration)
%
%
%
\begin{equation*}
  \begin{split}
    f(t)=- 3 (uv) + \gamma (u \p_x^2 v + \p_x u \p_x v + v \p_x^2 u).
  \end{split}
\end{equation*}
%
%
%Letting $u(x,t) = p(t)e^{-| x-q |}, v(x,t)=p(t)e^{-| x+q |}$, we see that
%%
%%
%\begin{equation*}
  %\begin{split}
    %& - 3 \p_x(uv) + \gamma \p_x(u \p_x^2 v + \p_x u \p_x v + v \p_x^2 u)
    %\\
    %& = e^{-| x+q |}e^{-| x-q |}\left[ \sigma(x+q)\sigma(x-q) -1 \right]
    %\\
    %& = 
    %\begin{cases}
      %0, & \qquad x \le -q, 
      %\\
      %-2e^{-2q}, & \qquad q < x < q,
      %\\
      %0, & \qquad x \ge q.
    %\end{cases}
  %\end{split}
%\end{equation*}
%
%
%
%%%%%%%%%%%%%%%%%%%%%%%%%%%%%%%%%%%%%%%%%%%%%%%%%%%%%
%
%
%				 The non-periodic Case
%
%
%%%%%%%%%%%%%%%%%%%%%%%%%%%%%%%%%%%%%%%%%%%%%%%%%%%%%
%
%
\section{The Non-Periodic Case for CH}
For any $\lambda \in \rr$, $b \in \rr \setminus \{0\}$, the differential system \eqref{dif-eq} has the
solution
%
%
\begin{equation}
  \label{dif-eq-soln}
  \begin{split}
    & p(t) = - \lambda \coth (\lambda t - b),
    \\
    & q(t) = \ln \cosh (\lambda t - b).
  \end{split}
\end{equation}
%
%
Hence, for $0 \le t < T$ where $T = b / \lambda$,
%
%
\begin{equation}
  \label{soln-CH}
  \begin{split}
    u(x,t)=-\lambda \coth(\lambda t -b)\left[ e^{-| x+ \ln \cosh(\lambda t -b)|} 
    - e^{-| x - \ln \cosh(\lambda t -b) |} \right]
  \end{split}
\end{equation}
%
%
is a solution to the CH equation. Next, we will show that for $1 <
s < 3/2 $, the $H^s$ decay of the bracketed term as $t \to b / \lambda$ is
dominated by the corresponding blowup of $p(t)$. First, we will need the
following. 
%
%
%
%
%
%%%%%%%%%%%%%%%%%%%%%%%%%%%%%%%%%%%%%%%%%%%%%%%%%%%%%
%
%
%				Norm Blowup
%
%
%%%%%%%%%%%%%%%%%%%%%%%%%%%%%%%%%%%%%%%%%%%%%%%%%%%%%
%
%
\begin{lemma}
  \label{lem:norm-blowup}
  %
  %
  For $r < s < 3/2$,
  %
  %
  \begin{equation}
    \label{norm-blowup}
    \begin{split}
      \lim_{t \to T^{-}}
      \frac{\|u(t)\|_{H^{r}(\rr)}}{\|u(t)\|_{H^{s}(\rr)}} = 0.
    \end{split}
  \end{equation}
\end{lemma}
%
%
{\bf Proof.} Using the definition of the Fourier transform and a change of
variable, we obtain
%
%
\begin{equation}
  \label{1h}
  \begin{split}
    \widehat{e^{-| x-q |}}(\xi)
    & = \int_{\rr} e^{-ix \xi} e^{-| x-q |}dx
    \\
    & = \int_{\rr} e^{-i(x+q)\xi} e^{-| x |} dx
    \\
    & = e^{-iq \xi} \int_{\rr} e^{-ix \xi} e^{-| x |}dx.
  \end{split}
\end{equation}
%
%
Next, we write
%
%
\begin{equation*}
  \begin{split}
    \int_{\rr}e^{-ix \xi} e^{-| x |}dx = \int_{0}^{\infty}e^{-ix \xi}
    e^{-x} dx + \int_{-\infty}^{0}e^{-ix \xi} e^{x} dx.
  \end{split}
\end{equation*}
%
%
Making the change of variable $x \to -x$ in the second integral, we obtain
%
%
\begin{equation*}
  \begin{split}
    \int_{\rr}e^{-ix \xi} e^{-| x |}dx
    & = \int_{0}^{\infty} \left( e^{-ix \xi} + e^{ix \xi} \right)e^{-x} dx
    \\
    & = 2 \int_{0}^{\infty} \cos(\xi x)e^{-x}dx.
  \end{split}
\end{equation*}
%
%
Integration by parts then gives
%
%
\begin{equation}
  \label{2h}
  \begin{split}
    \int_{\rr}e^{-ix \xi} e^{-| x |}dx = \frac{2}{\xi^2 + 1}.
  \end{split}
\end{equation}
%
%
Hence, from \eqref{1h} and \eqref{2h}, we obtain
%
%
\begin{equation*}
  \begin{split}
    \widehat{e^{-| x-q |}} (\xi) = \frac{2 e^{-iq \xi}}{ \xi^2 + 1}
  \end{split}
\end{equation*}
%
%
and so
%
%
\begin{equation*}
  \begin{split}
    \widehat{e^{-| x + q |} - e^{-| x-q |}} (\xi) = \frac{2 \left( e^{i q \xi} -
    e^{-i q \xi}
    \right)}{ \xi^2 + 1}.
  \end{split}
\end{equation*}
%
%
Therefore
%
%
\begin{equation*}
  \begin{split}
    \|e^{-| x+q |} - e^{-| x-q |} \|^2_{H^s(\rr)}
    & = \int_{\rr} \left( 1 + \xi^2 \right)^s  \left |
    \frac{2\left( e^{i q \xi} - e^{-i q \xi} \right)}{\xi^2 + 1} \right |^2 d
    \xi
    \\
    & = 4 \int_{\rr}\left( 1 + \xi^2 \right)^{s-2} |e^{i q \xi} - e^{-i q \xi}|^2
    d \xi
    \\
    & = 8 \int_{\rr} \left( 1 + \xi^2 \right)^{s-2} \sin^2(q \xi) d \xi.
  \end{split}
\end{equation*}
%
%
The change of variable $q \xi \to \xi'$ then gives
%
%
\begin{equation*}
  \begin{split}
    \frac{8}{q} \int_{\rr} \left( 1 + \frac{\xi'^2}{q^2} \right)^{s-2}
    \sin^2\xi' d \xi'
  \end{split}
\end{equation*}
or
\begin{equation}
  \label{3h}
  \begin{split}
    8 q^{3 - 2s} \int_{\rr} \left( q^2 + \xi'^2 \right)^{s-2} \sin^2\xi' d \xi'. 
  \end{split}
\end{equation}
%
%
Therefore,
%
%
\begin{equation*}
  \begin{split}
    \lim_{t \to T^{-}}
    \frac{\|u(t)\|_{H^r(\rr)}}{\|u(t)\|_{H^{s}(\rr)}}
    & = \lim_{t \to T^{-}}
    \frac{ p(t) \left[ 8 q^{3 - 2r} \int_{\rr} \left( q^2 + \xi'^2 \right)^{r-2}
    \sin^2\xi'
    d \xi' \right]^{1/2}}{ p(t) \left[8 q^{3 - 2s}
    \int_{\rr} \left( q^2 + \xi'^2 \right)^{s-2} \sin^2\xi'd
    \xi'\right]^{1/2}}
    \\
    & \le \lim_{t \to T^{-}} q^{s-r}, \qquad r < s < 3/2
    \\
    & = 0,
  \end{split}
\end{equation*}
%
%
%
%
concluding the proof. \qquad \qedsymbol
\\
\\
The blowup of the $H^s$ norm of $u(t)$ for $1 < s < 3/2$ is an immediate
corollary. More precisely, we have the following.
%
%
%%%%%%%%%%%%%%%%%%%%%%%%%%%%%%%%%%%%%%%%%%%%%%%%%%%%%
%
%
%				Corollary 1
%
%
%%%%%%%%%%%%%%%%%%%%%%%%%%%%%%%%%%%%%%%%%%%%%%%%%%%%%
%
%
\begin{corollary}
  \label{cor:h1-cons-rest-blowup}
  For $b \neq 0$ and $1 < s < 3/2$,
  \begin{equation*}
    \begin{split}
      \lim_{t \to T^{-}} \|u(t)\|_{H^s(\rr)} = \infty
    \end{split}
  \end{equation*}
  where $u(x,t)$ is defined in \eqref{soln-CH}.
\end{corollary}
%
%
{\bf Proof.} Since the $H^1$ norm of solutions to the CH equation are conserved, and since
$u_0$ is bounded when $b \neq 0$, taking $r = 1$ and $s >1$ in
\autoref{lem:norm-blowup} completes the proof. \qquad \qedsymbol
%
%
%
%
%%%%%%%%%%%%%%%%%%%%%%%%%%%%%%%%%%%%%%%%%%%%%%%%%%%%%
%
%
%				Hs bound for peakon antipeakon
%
%
%%%%%%%%%%%%%%%%%%%%%%%%%%%%%%%%%%%%%%%%%%%%%%%%%%%%%
%
%
\begin{corollary}
  \label{cor:peakon-antipeakon-Hs-bound}
  Let $ \ee >0$ and $ 1/2< s < 3/2$. If $q \le c_s \ee$, then 
  %
  %
  \begin{equation}
    \label{peakon-antipeakon-Hs-bound}
    \begin{split}
      \|e^{-| x+ q |} - e^{-| x- q |} \|_{H^{s}(\rr)} < \ee^{3/2 - s}.
    \end{split}
  \end{equation}
  Here $c_s$ is a constant depending only on $s$. 
  %
  %
\end{corollary}
%
%
%
{\bf Proof.}
Repeating the calculations of \autoref{lem:norm-blowup}, we obtain %
%
\begin{equation*}
  \begin{split}
    \|e^{-| x+q |} - e^{-| x-q |} \|^2_{H^s(\rr)}
    & = 8 \int_{\rr} \left( 1 + \xi^2 \right)^{s-2} \sin^2(q \xi) d \xi
  \end{split}
\end{equation*}
%
%
which we estimate by splitting the domain of integration. We have
%
%
%
\begin{equation*}
  \begin{split}
    \int_{| \xi| \le 1/q } (1 + \xi^2)^{s-2} \sin^2(q \xi)d \xi
    & \le \int _{| \xi| \le 1/q }\xi^{2s-4} \sin^2(q \xi)d \xi
    \\
    & \lesssim q^{2} \int_{0}^{1/q} \xi^{2s-2} d \xi
    \\
    & = \frac{q}{2s-1} \left[ \xi^{2s-1} \right] \Big |_0^{1/q}
    \\
    & = \frac{q^{3-2s}}{2s-1}, \qquad s > 1/2
  \end{split}
\end{equation*}
%
%
and
%
%
\begin{equation*}
  \begin{split}
    \int_{| \xi | > 1/q }(1 + \xi^2)^{s-2} \sin^2(q \xi) d \xi
    & \lesssim \int_{1/q}^{\infty} \xi^{2s-4} d \xi
    \\
    & = \frac{1}{2s-3} \left[ \xi^{2s-3} \right] \Big |_{1/q}^{\infty}
    \\
    & = \frac{q^{3-2s}}{3-2s}, \qquad s < 3/2
  \end{split}
\end{equation*}
%
%
Taking square roots, and setting $c_s = \min \left\{ \sqrt{|2s-1|},
\sqrt{|3-2s|} \right\}$ concludes the proof.
\qquad \qedsymbol
\\
\\
Next, fix $0<\ee<1$, and set $b = c_s \ee$. Then
%
%
\begin{equation*}
  \begin{split}
    1 < \cosh(-b) = \frac{e^{-b}+ e^{b}}{2} < e^{b} = e^{c_s \ee^{3/2 - s}}
  \end{split}
\end{equation*}
%
%
and hence by \autoref{cor:peakon-antipeakon-Hs-bound},
%
%
%
%
\begin{equation}
  \label{peak-antipeak-eps-bound}
  \begin{split}
    \|e^{-| x + \ln \cosh (-b) |} - e^{-| x - \ln \cosh(-b) |}\|_{H^{s}(\rr)} <
    \ee^{3/2 - s}.
  \end{split}
\end{equation}
%
%
Furthermore,
%
%
\begin{equation}
  \label{coth-bound}
  \begin{split}
    -\coth(-b) = \frac{e^{-2b + 1}}{e^{-2b - 1}} = \frac{1 +
    e^{2b}}{e^{2b}-1} = \frac{1 + e^{2c_{s} \ee}}{e^{2c_{s}\ee} -1} \le
    \frac{2}{e^{\ee} -1}, \qquad \ee <<1.
  \end{split}
\end{equation}
%
%
Let $\eta_s = (3/2- s)/2$, and set $\lambda = c_{s}(e^{\ee}-1)\ee^{\eta_s + s-3/2}$. Then it follows from
\eqref{peak-antipeak-eps-bound} and \eqref{coth-bound} that the corresponding solution
$u(x,t)$ to the CH equation has lifespan
%
%
\begin{equation}
  \label{result-part-1}
  \begin{split}
    T = \frac{c_{s}\ee}{c_{s}(e^{\ee}-1)\ee^{\eta_s + s -3/2}} = \frac{\ee^{5/2
    - s - \eta_s }}{\ee
    + \ee^2/2! + \ee^3/3! + \cdots} < \ee^{3/2 -s - \eta_s} = \ee^{\eta_s}
  \end{split}
\end{equation}
%
%
with  
\begin{equation}
  \label{result-part-2}
  \begin{split}
    \|u_0\|_{H^s(\rr)}< c_{s}(e^{\ee}-1)\ee^{\eta_s + s -3/2} \cdot \frac{2}{e^{\ee}-1}
    \cdot \ee^{3/2 - s} < 2 \ee^{\eta_s}.  
  \end{split}
\end{equation}
%
%
Since $\ee$ can be chosen independently of $\eta_s$, collecting
\eqref{result-part-1}, \eqref{result-part-2}, and
\autoref{cor:h1-cons-rest-blowup} completes the proof of the theorem in the
non-periodic case. \quad \qedsymbol
%
%
%
\bibliography{/Users/davidkarapetyan/math/bib-files/references}
%
%
%
%
%
% \bib, bibdiv, biblist are defined by the amsrefs package.
%\begin{bibdiv}
  %\begin{biblist}

    %\bib{Byers-2006-Existence-time-for-the-Camassa-Holm}{article}{
    %author={Byers, Peter},
    %title={Existence time for the {C}amassa-{H}olm equation and the critical
    %{S}obolev index},
    %date={2006},
    %ISSN={0022-2518},
    %journal={Indiana Univ. Math. J.},
    %volume={55},
    %number={3},
    %pages={941\ndash 954},
    %url={http://dx.doi.org/10.1512/iumj.2006.55.2710},
    %review={\MR{MR2244592 (2007k:35395)}},
    %}

  %\end{biblist}
%\end{bibdiv}

\end{document}







