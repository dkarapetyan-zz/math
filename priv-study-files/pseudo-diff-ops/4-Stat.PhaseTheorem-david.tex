\documentclass[12pt,reqno]{amsart}
\usepackage{amscd}
\usepackage{amsfonts}
\usepackage{amsmath}
\usepackage{amssymb}
\usepackage{amsthm}
\usepackage{appendix}
\usepackage{fancyhdr}
\usepackage{latexsym}
\usepackage{pdfsync}
\usepackage{cancel}
\usepackage{amsxtra}
\usepackage[colorlinks=true, pdfstartview=fitv, linkcolor=blue,
citecolor=blue, urlcolor=blue]{hyperref}
\input epsf
\input texdraw
\input txdtools.tex
\input xy
\xyoption{all}
%%%%%%%%%%%%%%%%%%%%%%
\usepackage{color}
\definecolor{red}{rgb}{1.00, 0.00, 0.00}
\definecolor{darkgreen}{rgb}{0.00, 1.00, 0.00}
\definecolor{blue}{rgb}{0.00, 0.00, 1.00}
\definecolor{cyan}{rgb}{0.00, 1.00, 1.00}
\definecolor{magenta}{rgb}{1.00, 0.00, 1.00}
\definecolor{deepskyblue}{rgb}{0.00, 0.75, 1.00}
\definecolor{darkgreen}{rgb}{0.00, 0.39, 0.00}
\definecolor{springgreen}{rgb}{0.00, 1.00, 0.50}
\definecolor{darkorange}{rgb}{1.00, 0.55, 0.00}
\definecolor{orangered}{rgb}{1.00, 0.27, 0.00}
\definecolor{deeppink}{rgb}{1.00, 0.08, 0.57}
\definecolor{darkviolet}{rgb}{0.58, 0.00, 0.82}
\definecolor{saddlebrown}{rgb}{0.54, 0.27, 0.07}
\definecolor{black}{rgb}{0.00, 0.00, 0.00}
\definecolor{dark-magenta}{rgb}{.5,0,.5}
\definecolor{myblack}{rgb}{0,0,0}
\definecolor{darkgray}{gray}{0.5}
\definecolor{lightgray}{gray}{0.75}
%%%%%%%%%%%%%%%%%%%%%%
%%%%%%%%%%%%%%%%%%%%%%%%%%%%
%  for importing pictures  %
%%%%%%%%%%%%%%%%%%%%%%%%%%%%
\usepackage[pdftex]{graphicx}
\usepackage{epstopdf}
% \usepackage{graphicx}
%% page setup %%
\setlength{\textheight}{20.8truecm}
\setlength{\textwidth}{14.8truecm}
\marginparwidth  0truecm
\oddsidemargin   01truecm
\evensidemargin  01truecm
\marginparsep    0truecm
\renewcommand{\baselinestretch}{1.1}
%% new commands %%
\newcommand{\bigno}{\bigskip\noindent}
\newcommand{\ds}{\displaystyle}
\newcommand{\medno}{\vskip0.1in\noindent}
\newcommand{\smallno}{\vskip0.1in\noindent}
\newcommand{\ts}{\textstyle}
\newcommand{\rr}{\mathbb{R}}
\newcommand{\p}{\partial}
\newcommand{\zz}{\mathbb{Z}}
\newcommand{\cc}{\mathbb{C}}
\newcommand{\ci}{\mathbb{T}}
\newcommand{\ee}{\varepsilon}
\def\refer #1\par{\noindent\hangindent=\parindent\hangafter=1 #1\par}
%% equation numbers %%
\renewcommand{\theequation}{\thesection.\arabic{equation}}
%% new environments %%
%\swapnumbers
\theoremstyle{plain}  % default
\newtheorem{theorem}{Theorem}
\newtheorem{proposition}{Proposition}
\newtheorem{lemma}{Lemma}
\newtheorem{corollary}{Corollary}
\newtheorem{conjecture}[subsection]{conjecture}
\theoremstyle{definition}
\newtheorem{definition}{Definition}
\newcommand{\ve}{\varepsilon} 
\newcommand{\nin}{\noindent}
\newcommand{\oL}{\bar L}
\newcommand{\vph}{\varphi}
\begin{document}
\title{Pseudodifferential and Fourier Integral Operators}
\author{Alex Himonas, {\it Summer 2009}}
\maketitle
\setcounter{section}{3}
\section{Stationary Phase Theorem}
\nin
We shall study the behavior of integrals of the form
\begin{equation}
	\label{1}
	I(x, \lambda) = \int e^{i \lambda \vph (x,y)} \ a(x, y, \lambda) dy 
\end{equation}
for $ \lambda $ large.  Here $ \vph (x,y) $ is a $ C^\infty $ real-valued function
on $ X \times Y$, where $ X $ is open in $ \Bbb R^\ell $ and $ Y $ is open in $ \Bbb
R^n$, and $ a(x,y, \lambda) $ is a symbol in the class $ S^m(X \times Y \times
\Bbb R^+) $ for some $ m \in \Bbb R$.  We shall assume that the symbol $ a(x,y,
\lambda) $ is compactly supported in $ y$: i.e. there exists a compact set $
K^\prime $ in $ Y $ such that
\begin{equation}
	\label{2}
	a(x, y, \lambda) = 0 \text{ if } y \notin K^\prime, \ x \in X, \ \lambda \in \Bbb
R^+. 
\end{equation}
First we shall consider the case when $ \vph $ is never stationary in $ y$.
\vskip0.1in
\nin{\bf I. $\vph^\prime_y (x,y) \ne 0 $ on $X \times Y$:}  Then 
\begin{equation}
	\label{3}
	I(x, \lambda) \in S^{- \infty} (X \times \Bbb R^+). \ 
\end{equation}
In fact, if we define $ L $ by
$$L = \frac{1}{\lambda} \frac{ < \vph^\prime_y (x,y), D_y
>}{|\vph^\prime_y (x,y)|^2}$$
then we obtain that
$$L(e^{i \lambda \vph (x,y)} ) = e^{i \lambda \vph (x,y)} \text{ and } L: S^m
\longrightarrow S^{m-1}.$$ 
Now we have
$$I(x, \lambda) = \int L(e^{i \lambda \vph (x,y)}) a(x,y, \lambda) dy, $$
and if we integrate by parts we obtain
$$I(x, \lambda) = \int e^{i \lambda \vph (x,y)} \ {}^tLa(x,y, \lambda) dy.$$
If $ k \in \Bbb N $ and we repeat the above step $ k $ times then we obtain
$$I(x, \lambda) =\int e^{i \lambda \vph (x,y)} ({}^tL)^k a(x,y, \lambda) dy. $$
By the formula for $ L $ and the fact that $ a \in S^m $ we obtain that
$$|I(x, \lambda )| \le C_{\vph, a, k}\  \lambda^{m-k}.$$
Since similar estimates hold for $ \partial^\alpha_x \partial^\beta_\lambda
a(x,\lambda)$, we obtain that $ I(x, \lambda) \in S^{- \infty} (X \times \Bbb
R^+)$, and this completes the proof of \eqref{3}. \vskip0.1in
\nin
Next we shall consider a special, but very important, stationary case.
\vskip0.1in
\nin
{\bf II. Quadratic Case:}  In this case
\begin{equation}
	\label{4}
	\vph (x,y) = \frac{1}{2} < Q(x) y, y >  \ 
\end{equation}
where $ Q(x) $ is an $ n \times n,$ invertible, real, symmetric matrix with
coefficients in $ C^\infty(X)$, and $ Y = \Bbb R^n$.  To study this case we shall
need the following lemma.
\begin{lemma}
	\label{lem1}
	If $ A $ is an $ n \times n $ real, symmetric, and invertible
matrix, then $ e^{\frac{i}{2} <Ay,y>} $ belongs in $ \mathcal S^\prime(\Bbb R^n) $
with its Fourier transform given by the formula
\begin{equation}
	\label{5}
	\mathcal F \left ( e^{\frac{i}{2}<Ay,y>} \right ) (\eta) = \frac{(2
\pi)^{\frac{n}{2}}}{\sqrt{|\text{det}A|}} \ e^{i \frac{\pi}{4} \text{sgn}A} \ e^{-
\frac{i}{2} <A^{-1} \eta, \eta>}, \ 
\end{equation}
where  sgn $A$ = (number of the positive eigenvalues of $ A$) - (number of the
negative eigenvalues of $ A$). 
\end{lemma}
\vskip0.1in
\nin
 Before proving the above lemma we shall
describe the behavior of $ I(x, \lambda)$.
\nin
\begin{proposition}  If $ \vph $ is as in \eqref{4} then $ I(x, \lambda)  \in 
S^{m-\frac{n}{2}} (X \times \Bbb R^+)$ and 
$$I(x, \lambda) \sim \left ( \frac{2 \pi}{\lambda}\right )^{\frac{n}{2}}
\frac{1}{\sqrt{|\text{det} \; Q(x)|}} \ e^{i \frac{\pi}{4} \text{sgn}Q(x)}
\sum^\infty_{k=0} \ \frac{\lambda^{-k}}{k!} R^k (a(x,y, \lambda)) \big |_{y=0},
$$
where $ R $ is the following second order pdo
$$R = R(x, dy) = \frac{i}{2} < Q^{-1} (x) \frac{\partial}{\partial y},
\frac{\partial}{\partial y} >.$$
\end{proposition}
\nin
{\bf Proof.}  We have
\begin{equation*}
	\begin{split}
I(x, \lambda)
&= \int_{\Bbb R^n} e^{i \frac{\lambda}{2}<Q(x)y,y>} \ a(x, y,
\lambda) dy \\
& \overset{\text{Parseval}}{\underset{\text{in y}}{=}}
\ \frac{1}{(2 \pi)^n} \int_{\Bbb
R^n} \overline{\mathcal F_y (e^{-i \frac{\lambda}{2} <Q(x)y, y>}}(\eta) \
\hat a(x, \eta, \lambda) d\eta\\
&= \frac{1}{(2 \pi)^n} \int_{\Bbb R^n} \frac{(2
\pi)^{\frac{n}{2}}}{\sqrt{|\text{det} \lambda Q(x)|}} \ e^{i \frac{\pi}{4}
\text{sgn}Q(x)} \ e^{- \frac{i}{2 \lambda} <Q(x)^{-1} \eta, \eta>} \ \hat a(x,
\eta, \lambda) d \eta\\
&= \frac{(2\pi)^{\frac{n}{2}}}{\sqrt{|\text{det}Q(x)|}} \
e^{i \frac{\pi}{4} \text{sgn}Q(x)} \ \lambda^{- \frac{n}{2}} J(x, \lambda)
\end{split}
\end{equation*}
where
\begin{equation}
	\label{6}J(x, \lambda) =\frac{1}{(2 \pi)^n}  \int_{\Bbb R^n} e^{- \frac{i}{2 \lambda} <Q^{-1} (x) \eta,
\eta>} \ \hat a(x, \eta, \lambda) d \eta. \ 
\end{equation}
Now we shall need the following elementary lemma. 
\begin{lemma}  For any $ N \in \Bbb N $ we have
	\begin{equation}
		\label{7}
		e^{is} = \sum^{N-1}_{k=0} \ \frac{(is)^k}{k!} + r_N (s), \ s \in \Bbb R, \ 
	\end{equation}
with the remainder $ r_N $ satisfying the estimate
$$|\partial^j_s r_N (s) | \le \frac{|s|^{N-j}}{(N-j)!} \ \text{ for all } 0 \le j \le N. $$
\end{lemma}
{\bf Proof.}  By Taylor's theorem we have
$$e^{is} = \sum^{N-1}_{k=0} \frac{(is)^k}{k!} + \frac{s^N}{N^!} \ i^N e^{its}, \
t \in [0, 1].$$
Since
$$r_N(s) = \frac{s^N}{N!}\  i^N e^{its}$$
for $ j = 0 $ we obtain the desired estimate.  For $ j > 0 $ we differentiate first
and then we estimate to obtain the desired estimate. 
Now if in \eqref{7} we let $ s =
\frac{-1}{2\lambda} <Q^{-1}(x)\eta, \eta>$ then we have
$$J(x, \lambda) = \sum^{N-1}_{k=0} J_k(x, \lambda) + R_N (x, \lambda) $$
where
$$J_k(x, \lambda) = \frac{1}{k!} \frac{1}{(2\pi)^n} \int_{\Bbb R^n} \left (
\frac{-1}{2 \lambda} <Q^{-1}(x)\eta, \eta> \right )^k
 \hat a(x,\eta,\lambda) d\eta$$ and
 \begin{equation}
	 \label{8}
	 R_N(x, \lambda) =\frac{1}{(2\pi)^n}
 \int_{\Bbb R^n} r_N \left ( - \frac{1}{2 \lambda} <Q^{-1} (x)
\eta, \eta > \right ) \hat a (x, \eta,\lambda) d \eta.  
\end{equation}
Since
$$\frac{1}{(2 \pi)^n} \int \eta^\alpha \hat a(x, \eta, \lambda) d \eta =
D^\alpha_y a(x,y, \lambda) \bigg |_{y=0}$$
we obtain that
\begin{equation*}
	\begin{split}
J_k(x, \lambda) &= \frac{\lambda^{-k}}{k!} \left ( - \frac{i}{2} <Q^{-1} (x)
D_y, D_y> \right )^k a(x, y, \lambda) \bigg |_{y=0}\\
&= \frac{\lambda^{-k}}{k!} R^k (a(x, y,\lambda)) \bigg |_{y=0}.
\end{split}
\end{equation*}
Since $ J_k \in S^{m-k} (X \times \Bbb R^+) $ formula \eqref{5} will be proved if we
show
\begin{equation}
	\label{9}
	J(x, \lambda) \sim \sum^\infty_{k=0} J_k (x, \lambda) \text{ in } S^m (X
\times \Bbb R^+). \
\end{equation}
For this we have to show that for each $ N $ \
$$R_N = J - \sum^{N-1}_{k=0} J_k \in S^{m-N} (X \times \Bbb R^+). $$
And for this we need to show that
\begin{equation}
	\label{10}
	|\partial^\alpha_x \partial^\beta_\lambda R_N (x, \lambda) | \le C_{K, \alpha,
\beta} \lambda^{M-N-\beta}, \ x \in K, \lambda \in \Bbb R^+ 
\end{equation}
Since $ J_k \in S^{m-k} $  it suffices to prove \eqref{10} only for  $ |\alpha| + \beta \le N$. 
We shall do it only for $ \alpha = \beta = 0$.  First we observe that by
\eqref{8}
$$r_N \left ( - \frac{1}{2 \lambda} <Q^{-1} (x) \eta, \eta > \right ) \le C
\lambda^{-N} |\eta|^{2N}.$$
Then for any $ M \in \Bbb N $ we have
\begin{align*}
|(1 + |\eta|^2)^M \hat a(x, \eta, \lambda) | &= | \int_{\Bbb R^n} e^{-iy\eta}
(1-\Delta_y)^M a(x, y, \lambda) dy |\\
&\le C_M (a) (1 + \lambda)^m \end{align*}
Therefore 
$$|\hat a(x, \eta, \lambda)| \le C_M (1 + \lambda)^m (1 + |\eta|^2)^{-M}$$
and hence
\begin{align*}
|R_N(x, \lambda)| &\le C \lambda^{-N} \int_{\Bbb R^n} |\eta|^{2N} (1 +
\lambda)^m (1 + |\eta|^2)^{-M} d \eta\\
&\le C^{\prime \prime} \lambda^{m-N}. \end{align*}
For the rest we use
$$\left | \partial^j_s r_N (s) \right | \le \frac{|s|^{N-j}}{(N-j)!}. $$
\vskip0.1in
\nin
{\bf III. The Non-degenerate Stationary Case.}  In this case we assume that at a
given point $ (x_0, y_0) \in X \times Y $ the following hold:
\begin{equation}
	\label{11}
	\vph^\prime_y (x_0, y_0) = 0,
\end{equation}
\begin{equation}
	\label{12}
	\vph''_y (x_0, y_0)=\left \{ \frac{\partial^2 \vph}{\partial y_j
\partial y_k } (x_0, y_0) \right \}, \text{the Hessian,  is invertible.}
\end{equation}
Then by the implicit function theorem we can solve the equation $
\vph^\prime_y (x,y) = 0 $ for $ y $ in terms of $ x$.  More precisely there
exists open sets $ U $ in $ X $ and $ V $ in $ Y $ and a $ C^\infty $ function $
\gamma: U \longrightarrow V $ such that
\begin{equation}
	\label{13}
	\vph^\prime_y (x, \gamma(x)) = 0, \ x \in U 
\end{equation}
\begin{equation}
	\label{14}
\gamma(x_0) = y_0. 
\end{equation}
We have that $ \gamma(x) $ is the critical point of the function $ y \longmapsto
\vph (x,y)$.  Also $ U $ and $ V $ can be chosen such that 
\begin{equation}
	\label{15}
	\vph''_y(x, \gamma (x)) = \left \{ \frac{\partial^2 \vph}{\partial y_j \partial y_k} (x,
\gamma (x)) \right \} \text{ is invertible for } x \in U. 
\end{equation}
By applying Taylor's Theorem around the point $ y = \gamma (x) $ we obtain
$$\vph(x, y) = \vph (x, \gamma(x)) + \frac{1}{2} <B(x,y) (y-\gamma (x)), \ y -
\gamma (x) >$$
where $ B(x, y) $ is a smooth matrix.  Now by applying Morse's Lemma we can
change variables $ y \longrightarrow z $ such that $ \vph (x,y) = \vph(x,
\gamma(x)) + \frac{1}{2} <\vph''_y(x, \gamma (x)) z, z>$.  More precisely we have the following
lemma.
\vskip0.1in
\nin
{\bf Morse's Lemma with parameter.}  Let $ \vph (x,y) $ be such that
\eqref{11} and
\eqref{12} hold.  Then there exist open sets $ U $ in $ X $ and $ V $ in $ Y $ and a $
C^\infty $ function
$$h: U \times V \longrightarrow \Bbb R^n$$
such that
\vskip0.1in
\nin
{\bf 1.} For each $x \in U$  the map $y \in V \longmapsto z = h(x,y)
\in \Bbb R^n$ \text{ is a diffeomorphism from} $V$  into its image, which is
called the {\bf Morse diffeomorphism}.
\vskip0.1in
\nin
{\bf 2.} For each $x$ the Morse diffeomorphism maps the critical
point $\gamma (x)$ to $0$; i.e.
\begin{equation}
	\label{19}
	h(x, \gamma (x)) = 0,
\end{equation}
\begin{equation}
	\label{20}
	\vph (x,y) = \vph (x, \gamma (x)) + 
\frac{1}{2} <\vph''_y(x, \gamma (x)) z, z>, x \in U, y \in V
\end{equation}
and
\begin{equation}
	\frac{\partial z}{\partial y} (x, \gamma (x)) \doteq h^\prime_y (x, \gamma
(x)) = I, \ x \in U. 
\end{equation}
\vskip0.1in
\nin
{\bf Proof.}  See H\"ormander:  FIO, I, Acta Math., Lemma 3.2.3.
\vskip0.1in
\nin
Now let us assume that the symbol $ a(x, y, \lambda) $ is supported in $ U
\times V $ where the conclusions of Morse's Lemma hold; i.e. there exist a
compact set $ K_h $ in $ X \times Y $ such that
$$a(x, y, \lambda) = 0 \text{ if } (x, y) \notin K_h. $$
Then we write the integral $ I(x, \lambda) $ in the form
$$I(x, \lambda) = e^{i \lambda \vph(x, \gamma (x))} \int_Y e^{i \lambda [\vph
(x,y) - \vph (x, \gamma (x))]} a(x, y, \lambda) dy $$
Then by \eqref{20} we obtain
\begin{align}
e^{-i \lambda \vph (x, \gamma (x))} I(x, \lambda) &= \int_Y e^{i
\frac{\lambda}{2} <\vph''_y(x, \gamma (x)) h(x,y), h(x, y)>} a(x, y, \lambda) dy\\
&= \int_W e^{i \frac{\lambda}{2} <\vph''_y(x, \gamma (x)) z, z>} 
a(x, g(x, z), \lambda)
|\text{det } \frac{\partial g}{\partial z} (x, z) |dz, \end{align}
where the map $ g: U \times W \longrightarrow V $ is the inverse of the Morse
diffeomorphism $ y \in V \longmapsto z = h(x, y) \in W$.  Now we apply the
quadratic Case II to obtain the following theorem.
\begin{theorem}  If $ \vph, \gamma,  h, g $ and $ a $ are as above,
then 
$e^{- i \lambda \vph(x, \gamma(x))} I(x, \lambda) 
\in S^{m-\frac n 2 }(X\times \Bbb R^+)$,
 and the
following asymptotic expansion holds:
$$e^{- i \lambda \vph(x, \gamma(x))} I(x, \lambda) \sim \left ( \frac{2
\pi}{\lambda} \right )^{\frac{n}{2}} \frac{ e^{i \frac{\pi}{4}
\text{sgn}\vph''_y(x, \gamma (x))}}
{\sqrt{|\text{det}\vph''_y(x, \gamma (x))|}} \ \sum^\infty_{k=0}
\frac{\lambda^{-k}}{k!} R^k (\tilde a) \bigg |_{z=0},  $$
where
$$R(x, \partial_z) = \frac{i}{2} <[\vph''_y]^{-1}(x, \gamma (x)) \partial_z, \
\partial_z>$$ and
$$\tilde a(x, z, \lambda) = a(x, g(x, z), \lambda ) | \text{det } \frac{\partial
g}{\partial z} (x, z) |.$$
In particular we have 
$$e^{- i \lambda \vph(x, \gamma(x))} I(x, \lambda)
 - \left ( \frac{2
\pi}{\lambda} \right )^{\frac{n}{2}} \frac{ e^{i \frac{\pi}{4}
\text{sgn}\vph''_y(x, \gamma (x))}}
{\sqrt{|\text{det}\vph''_y(x, \gamma (x))|}}  
 a(x,\gamma (x), \lambda)\in S^{m-\frac n 2 -1}(X\times \Bbb R^+).  $$
\end{theorem}
\bigskip 
\nin
\centerline{\bf Proof of Lemma \ref{lem1}}
The proof of Lemma \ref{lem1} is reduced by an orthogonal transformation
to  the following lemma in one dimension.
\begin{lemma}  Let $ \alpha \ge 0 $ and $ \beta \in
\Bbb R$.  Then
\begin{equation}
	\label{50}
	\mathcal F \left ( e^{- \frac{\alpha + i \beta}{2} x^2} \right ) (\xi) =
\sqrt{\frac{2 \pi}{\alpha + i \beta}} \ e^{- \frac{1}{2(\alpha + i \beta)}
\xi^2}, 
\end{equation}
where $ \mathcal F $ denotes the Fourier transform, and $ \sqrt{\alpha + i
\beta } $ is the natural branch of the square root in right half-plane.  In
particular, for $ \alpha = 0 $ and $ \beta \ne 0 $ we obtain
\begin{equation}
	\label{51}
	\mathcal F \left ( e^{- \frac{i \beta}{2} x^2} \right ) (\xi) = \sqrt{\frac{2
\pi}{|\beta|}} \ e^{-i \frac{\pi}{4} \text{sgn} \beta} \ e^{\frac{i}{2 \beta}
\xi^2}. 
\end{equation}
\end{lemma}
\nin
{\bf Proof.}  If $ \alpha > 0 $ then the function
$$\vph_{\alpha + i \beta} (x) = e^{- \frac{\alpha + i \beta}{2} x^2}$$
belongs to $ \mathcal S(\Bbb R)$.  Then $ \widehat \vph_{\alpha + i \beta}
(\xi) $ is also in $ \mathcal S (\Bbb R) $ and we can obtain formula
\eqref{50} by the
definition of the Fourier transform and a deformation of the
integration.  Here we use an alternative method.  The function $ \vph =
\vph_{\alpha + i \beta} $ satisfies the ode
$$\vph^\prime + (\alpha + i \beta) x \vph = 0. $$
Then by taking Fourier transform we obtain the ode $ i \xi \hat \vph
(\xi) + i(\alpha + i \beta) \frac{d \hat \vph}{d \xi} (\xi) = 0 $ or
$$\frac{d \hat \vph}{d \xi} + \frac{\xi}{\alpha + i \beta} \hat \vph = 0,$$
or
$$\frac{d}{d \xi} \left ( e^{\frac{\xi^2}{2(\alpha + i \beta)}} \hat \vph
\right ) = 0$$
or
$$\hat \vph (\xi) = c e^{- \frac{1}{2 (\alpha + i \beta)} \xi^2}. $$
To compute $ c $ we have
$$c = \hat \vph (0) = \int_{\Bbb R} \vph (x) dx = \int_\Bbb R e^{-
\frac{\alpha + i \beta}{2} x^2} dx = 2 \int^\infty_0 e^{- \frac{\alpha + i
\beta}{2} x^2} dx. $$
Now by applying the well-known trick we obtain
\begin{align*}
c^2 &= 4 \int^\infty_0 \int^\infty_0 e^{- \frac{\alpha + i \beta}{2} (x^2
+ y^2)} dx dy\\
&= 4 \int^{\frac{\pi}{2}}_0 \int^\infty_0 e^{- \frac{\alpha + i \beta}{2}
r^2} r dr d\theta\\
&= 2 \pi \int^\infty_0 e^{- \frac{\alpha + i \beta}{2} r^2} r dr\\
&= \frac{2 \pi}{\alpha + i \beta} \left [ - e^{ - \frac{\alpha + i \beta}{2}
r^2} \right ]^\infty_0 = \frac{2 \pi}{\alpha + i \beta}.
\end{align*}
Therefore $ c = \sqrt{\frac{2 \pi}{\alpha + i \beta}} $ and this proves
formula \eqref{50} for $ \alpha > 0$.
\vskip0.1in
\nin
\underbar{$\alpha  =  0, \beta \ne 0  $ { \bf case:}}  \  Then we
consider the sequence $ \vph_{\frac{1}{n} + i \beta}, \ n = 1, 2, \dots$ 
By using the dominated convergence theorem we obtain that
$$\langle \vph_{\frac{1}{n} + i \beta}, \psi \rangle = \langle \vph_{i
\beta}, \psi \rangle, \ \forall \psi \in \mathcal S (\Bbb R). $$
Therefore $ \vph_{\frac{1}{n} + i \beta} $ converges to $ \vph_{i \beta}
$ in the topology of $ \mathcal S^\prime ( \Bbb R)$.  Since the Fourier
transform $ \mathcal F: \mathcal S^\prime (\Bbb R) \longrightarrow \mathcal
S^\prime (\Bbb R) $ is continuous, we obtain that $ \hat
\vph_{\frac{1}{n} + i \beta} \longrightarrow \hat \vph_{i \beta}$.  By
formula \eqref{50} and the dominated convergence theorem we have that $
\hat \vph_{\frac{1}{n} + i \beta}, \beta \ne 0 $ converges to
$$\sqrt{\frac{2 \pi}{i \beta}} \ e^{- \frac{1}{2 i \beta} \xi^2}, $$
which must be equal to $ \hat \vph_{i \beta} (\xi) $ by the uniqueness
of the limit in $ \mathcal S^\prime (\Bbb R)$.  Since
$$\sqrt{i \beta} =
\begin{cases} \sqrt{\beta} e^{i \frac{\pi}{4}} \text{ if }
\beta > 0\\
\\
\sqrt{|\beta|} e^{- i \frac{\pi}{4}} \text{ if } \beta < 0, \end{cases}$$
we obtain formula \eqref{51} for $ \alpha = 0, \beta \ne 0$.  This completes
the proof of the lemma.
\end{document}
                                                                                                                                                                   

                                                                                                                                                                                                                                                                            
 

