\documentclass[12pt,reqno]{amsart}
\usepackage{amscd}
\usepackage{amsfonts}
\usepackage{amsmath}
\usepackage{amssymb}
\usepackage{amsthm}
\usepackage{appendix}
\usepackage{fancyhdr}
\usepackage{latexsym}
\usepackage{pdfsync}
\usepackage{cancel}
\usepackage{amsxtra}
\usepackage[colorlinks=true, pdfstartview=fitv, linkcolor=blue,
citecolor=blue, urlcolor=blue]{hyperref}
\input epsf
\input texdraw
\input txdtools.tex
\input xy
\xyoption{all}
%%%%%%%%%%%%%%%%%%%%%%
\usepackage{color}
\definecolor{red}{rgb}{1.00, 0.00, 0.00}
\definecolor{darkgreen}{rgb}{0.00, 1.00, 0.00}
\definecolor{blue}{rgb}{0.00, 0.00, 1.00}
\definecolor{cyan}{rgb}{0.00, 1.00, 1.00}
\definecolor{magenta}{rgb}{1.00, 0.00, 1.00}
\definecolor{deepskyblue}{rgb}{0.00, 0.75, 1.00}
\definecolor{darkgreen}{rgb}{0.00, 0.39, 0.00}
\definecolor{springgreen}{rgb}{0.00, 1.00, 0.50}
\definecolor{darkorange}{rgb}{1.00, 0.55, 0.00}
\definecolor{orangered}{rgb}{1.00, 0.27, 0.00}
\definecolor{deeppink}{rgb}{1.00, 0.08, 0.57}
\definecolor{darkviolet}{rgb}{0.58, 0.00, 0.82}
\definecolor{saddlebrown}{rgb}{0.54, 0.27, 0.07}
\definecolor{black}{rgb}{0.00, 0.00, 0.00}
\definecolor{dark-magenta}{rgb}{.5,0,.5}
\definecolor{myblack}{rgb}{0,0,0}
\definecolor{darkgray}{gray}{0.5}
\definecolor{lightgray}{gray}{0.75}
%%%%%%%%%%%%%%%%%%%%%%
%%%%%%%%%%%%%%%%%%%%%%%%%%%%
%  for importing pictures  %
%%%%%%%%%%%%%%%%%%%%%%%%%%%%
\usepackage[pdftex]{graphicx}
\usepackage{epstopdf}
% \usepackage{graphicx}
%% page setup %%
\setlength{\textheight}{20.8truecm}
\setlength{\textwidth}{14.8truecm}
\marginparwidth  0truecm
\oddsidemargin   01truecm
\evensidemargin  01truecm
\marginparsep    0truecm
\renewcommand{\baselinestretch}{1.1}
%% new commands %%
\newcommand{\bigno}{\vskip0.1in\noindent}
\newcommand{\ds}{\displaystyle}
\newcommand{\medno}{\vskip0.1in\noindent}
\newcommand{\smallno}{\vskip0.1in\noindent}
\newcommand{\ts}{\textstyle}
\newcommand{\rr}{\mathbb{R}}
\newcommand{\p}{\partial}
\newcommand{\zz}{\mathbb{Z}}
\newcommand{\cc}{\mathbb{C}}
\newcommand{\ci}{\mathbb{T}}
\newcommand{\ee}{\varepsilon}
\def\refer #1\par{\noindent\hangindent=\parindent\hangafter=1 #1\par}
%% equation numbers %%
\renewcommand{\theequation}{\thesection.\arabic{equation}}
%% new environments %%
%\swapnumbers
\theoremstyle{plain}  % default
\newtheorem{theorem}{Theorem}
\newtheorem{proposition}{Proposition}
\newtheorem{lemma}{Lemma}
\newtheorem{corollary}{Corollary}
\newtheorem{conjecture}[subsection]{conjecture}
\theoremstyle{definition}
\newtheorem{definition}{Definition}
\newcommand{\ve}{\varepsilon} 
\newcommand{\nin}{\noindent}
\newcommand{\oL}{\bar L}
\newcommand{\vph}{\varphi}
\begin{document}
\title{Pseudodifferential and Fourier Integral Operators}
\author{Alex Himonas, {\it Summer 2009}}
\maketitle
\setcounter{section}{6}
\setcounter{theorem}{0}
\section{Microlocal Analysis}
\vskip0.1in
\noindent
Let $ u \in \mathcal E^\prime (\rr^n)$.  A simple version of the Paley-Wiener
theorem states that $ u \in C^\infty (\rr^n) $ if and only if for every $ N \in
\Bbb N $ there exists $ C_N > 0 $ such
\begin{equation}
	\label{1}
	|\hat u(\xi)| \le C_N (1+ |\xi|)^{-N}, \ \xi \in \rr^n. 
\end{equation}
Therefore if $ u $ is not in $ C_0^\infty $ then \eqref{1} must fail
for at least a certain
ray $ \lambda \xi_0, \xi_0 \ne 0, \lambda \in \Bbb R$, while it may hold near
other rays.  This motivates the following refined, and very useful in the theory
of PDE's, notion of smoothness.
\smallskip
\begin{definition}
	\label{def1}
	Let $ X $ be an open set in $ \rr^n, u \in
	\mathcal{D}^\prime (X) $ and $ (x_0, \xi_0) \in X \times \dot \rr^n$. 
	It is said that $ u $ is $ C^\infty $ at $ (x_0, \xi_0)$, or
	that $ (x_0, \xi_0) $ does not belong to the {\bf wave front} of $ u$,
	and write $ (x_0, \xi_0)  \notin WF(u)$, if there exists a function
	$ \varphi_0 \in C^\infty_0 (X) $ with $\varphi_0 (x_0) \neq 0 $ and
	an open cone $ \Gamma_0 $ containing $ \xi_0 $ such that for any $ N
	\in \Bbb N $ there exists $ C_N $ with
\begin{equation}
	\label{2}
	|\widehat {\varphi_0u} (\xi)| \le C_N (1 + |\xi|)^{-N}, \
\xi \in \Gamma_0. 
\end{equation}
The wave front of $u $ is defined as
$$WF(u) = \{(x_0, \xi_0) \in X \times \dot \rr^n: \eqref{2} \text{ does not hold\}}.
$$
By its definition $ WF(u) $ is a {\bf closed }and {\bf conic} set in $ X \times
\dot \rr^n$.
\end{definition}
\vskip0.1in
\noindent
Next we state another equivalent definition of $ WF(u)$.
\noindent
\smallskip
\begin{proposition}
	\label{prop1}
	$ (x_0, \xi_0) \notin WF(u) $ if and only if there exist
an open set $ V $ containing $ x_0 $ and a conic open set $ \Gamma $
containing  $ \xi_0 $ such that for any $ \varphi \in C^\infty_0 (V) $
\begin{equation}
	\label{200}
	| \widehat{\varphi u} (\xi) | \le C_{\varphi, N} (1 + |\xi|)^{-N}, \ \xi \in
\Gamma.
\end{equation}
\end{proposition}
\nin
{\bf Proof.}  Let $ (x_0, \xi_0) \notin WF(u) $ and $ \varphi_0 $ as in Definition
\ref{def1}.  Define
$$V = \{ x \in X: |\varphi_0 (x)| > \frac{1}{2} |\varphi_0 (x_0 )|\}.$$
Then $ V $ is a neighborhood of $ x_0 $ and for any $ \vph \in C^\infty_0 (V) $
we have
$$\varphi u= (\varphi \varphi^{-1}_0) (\varphi_0 u). $$
If we let
$$\psi = \varphi \varphi^{-1}_0 \text{ and } v = \varphi_0 u $$
 then we obtain
 \begin{equation*}
\begin{split}
\widehat{\varphi u} (\xi) &= \hat \psi * \hat v(\xi)\\
&= \int_{\rr^n} \hat \psi (\xi - \eta) \hat v(\eta) d \eta
\end{split}
\end{equation*}
Therefore
\begin{equation*}
\begin{split}
|\widehat{\varphi u} (\xi)| &\le \int_{\Gamma_0} |\hat \psi (\xi - \eta)| |\hat v
(\eta)| d \eta + \int_{\rr^n - \Gamma_0} |\hat \psi (\xi - \eta) |\hat v(\eta)
|d \eta\\
&\doteq I_1 (\xi) + I_2 (\xi) \end{split}
\end{equation*}
\noindent
{\bf Estimation of $\bold  I_{\bold 1}.$}  Since
$$(1 + |\xi|)^N \le (1 + |\xi - \eta|)^N (1 + |\eta|)^N$$
we obtain that
\begin{equation*}
\begin{split}
(1 + |\xi|)^N I_1 (\xi) &\le \int_{\Gamma_0} |\hat \psi (\xi - \eta)| (1 + |\xi -
\eta|)^N |\hat v(\eta)| (1 + |\eta|)^N d\eta\\
& \overset{\eqref{2}}{\le} C_N \int_{\Gamma_0} |\hat \psi (\xi - \eta) | (1 + |\xi -
\eta|)^N d\eta\\
&\le C_N \int_{\rr^n} |\hat \psi (\theta)| (1 + |\theta|)^N d \theta \le
C^\prime_N \end{split}
\end{equation*}
since $ \hat \psi \in \mathcal S(\rr^n)$.  Therefore
\begin{equation}
	\label{4}
	I_1 (\xi) \le C^\prime_N (1 + |\xi|)^{-N}, \ \forall \xi \in \rr^n. 
\end{equation}
{\bf Estimation of $\bold I_{\bold 2}.$}  Since $ v \in \mathcal E^\prime (X) $ there
exists $ M > 0 $ such that
$$| \hat v(\xi)| \le C(1 + |\xi|)^M, \ \xi \in \rr^n. $$
Therefore
$$I_2 (\xi) \le C \int_{\rr^n - \Gamma_0} |\hat \psi (\xi - \eta)| (1 +
|\eta|)^M d \eta.$$
Now we choose an open cone $ \Gamma $ containing $ \xi_0 $ such that
\begin{equation*}
	\Gamma \cap S^{n-1} \subset \subset \Gamma_0 \cap S^{n-1}
\end{equation*}
%$$\Gamma \cap S^{n-1} \subset \subset \Gamma_0 \cap S^{n-1}$$
%\vskip 3.0 in
%\hskip .5 in 
%\def\picture#1 by #2 (#3){
%\vbox to #2{
%\hrule width #1 height 0pt depth 0pt
%\vfill
%\special{#3}}}
%\def\scaledpicture #1 by #2 (#3 scaled #4){{
%\dimen0=#1 \dimen1=#2
%\divide\dimen0 by 1000 \multiply\dimen0 by #4
%\divide\dimen1 by 1000 \multiply\dimen1 by #4
%\picture \dimen0 by \dimen1 (#3 scaled #4)}}
%\def\finder{\scaledpicture 85.4pc by 57pc
%(screen0 scaled 500)}
%\vskip 1.5 in
%\hskip 1.5 in
%\special{picture small1}
%\bigskip
%\noindent
Then there exists $ \varepsilon > 0 $ such that
$$|\xi - \eta| \ge \varepsilon |\xi|, \ \xi \in \Gamma, \ \eta \in \rr^n -
\Gamma_0.$$
In fact, let $ \varepsilon > 0$ be such that
$$ \left | \frac{\xi}{|\xi|} - \eta \right | \ge \varepsilon, \ \xi \in \Gamma, \ \eta
\in \rr^n - \Gamma_0. $$
Then
$$|\xi - |\xi| \eta| \ge \varepsilon |\xi|$$
and if we replace $ \eta $ with $ \frac{\eta}{|\xi|} $ we obtain the desired
inequality. 
\vskip0.1in
\noindent
If we also use the fact that $ \hat \psi $ satisfies inequalities
similar to \eqref{1} then for $ \xi \in \Gamma $ and any $ N \in \Bbb N $ we obtain
$$|I_2 (\xi)| \le C^{\prime \prime}_N \int_{|\xi - \eta| \ge \varepsilon |\xi|} (1 +
|\xi - \eta|)^{-N} (1 + |\eta|)^M d \eta.$$
Since again
$$(1 + |\eta|)^M \le (1 + |\xi - \eta|)^M (1 + |\xi|)^M$$
we obtain
\begin{equation*}
\begin{split}
I_2 (\xi) &\le C^{\prime \prime}_N (1 + |\xi|)^M \int_{|\xi - \eta| \ge
\varepsilon |\xi|} (1 + |\xi - \eta|)^{-N+M} d \eta\\
&= C^{\prime \prime}_N (1 + |\xi|)^M \int_{|\theta| \ge \varepsilon |\xi|} (1 +
|\theta|)^{-N+M} d \theta \end{split}
\end{equation*}
Now we use spherical coordinates
$$\theta = \rho \omega, \text{ where } \omega = \frac{\theta}{|\theta|}, $$
and we have
\begin{equation*}
\begin{split}
\int_{|\theta| \ge \varepsilon |\xi|} (1 + |\theta|)^{-N+M} d \theta &= C
\int^\infty_{\varepsilon |\xi|} (1 + \rho)^{-N+M} \rho^{n-1} d \rho\\
&\le C \frac{(1 + \rho)^{-N+M+n}}{-N +M +n} \bigg |^\infty_{\varepsilon |\xi|}\\
&= \frac{C}{N-M-n} (1 + \varepsilon  |\xi|)^{-N+M+n} ,\end{split}
\end{equation*}
if $ -N + M < -n -1$.  Therefore
$$I_2 (\xi) \le C^{\prime \prime \prime}_N (1 + |\xi|)^{-N+2M+n}, \ \xi \in
\Gamma. $$ 
Now if we replace $ N $ with $ N + 2M + n $ we obtain the estimate
\begin{equation}
	\label{5}
	I_2 (\xi) \le C_N (1 + |\xi|)^{-N}, \ \xi \in\Gamma
\end{equation}
By \eqref{4} and \eqref{5} we obtain \eqref{200}. 
The converse follows immediately,  and this
completes the proof of Proposition \ref{prop1}.
\vskip0.1in
\noindent
In the proof of Proposition \ref{prop1} we proved also the following lemma.
\vskip0.1in
\noindent
\begin{lemma}  If $ v \in \mathcal E^\prime (X) $ is such that
$$| \hat v(\xi)| \lesssim (1 + |\xi|)^{-N}, \ \xi \in \Gamma_0$$
for some open cone $ \Gamma_0, $ then for any open cone $ \Gamma \subset
\subset \Gamma_0 $ and any $ \psi \in \mathcal S(\rr^n) $ we obtain that
$$|\hat \psi \ast \hat v (\xi) | \lesssim (1 + |\xi|)^{-N}, \ \xi \in \Gamma. $$
\end{lemma}
\vskip0.1in
\noindent
\begin{proposition}
	\label{prop2}
	If $ Pr_1 $ is the projection of $ X \times \dot \rr^n $
into $ X $ then for any $ u \in \mathcal D^\prime (X) $
$$Pr_1 \bigg (WF(u) \bigg ) = \sin g \text{ supp } u.$$   
\end{proposition}
\nin
{\bf Proof.}  The inclusion
$$Pr_1\bigg (WF(u)\bigg ) \subset  \sin g \text{ supp } u$$
is easy.  To prove the other direction we will assume that
\begin{equation}
	\label{1'}
	x_0 \notin Pr_1 (WF(u)) 
\end{equation}
and we shall show that
\begin{equation}
	\label{2'}
	x_0 \notin \sin g \text{ supp } u. 
\end{equation}
By \eqref{1'} we have that $ (x_0, \xi) \notin WF(u) $ for every $ \xi \in S^{n-1}$.  By
applying Proposition \ref{1} at each $ (x_0, \xi) $ and using the compactness of $
S^{n-1} $ we conclude that there exist 
$$ \xi_1, \dots, \xi_k, \ \  V_1, \dots, V_k \text{ and } \Gamma_1, \dots,
\Gamma_k $$
 such that 
\vskip0.1in
\noindent
$\bullet \quad V_j $ is open containing $ \xi_0.$
\vskip0.1in
\noindent
$\bullet $ \quad $ \Gamma_j $ is open containing $ \xi_0 $ and $ \cup^k_{j=1}
\Gamma_j = \dot \rr^n. $
\vskip0.1in
\noindent
$\bullet$ \quad For any $ \varphi_j \in C^\infty_0 (V_j), \ |\widehat{\varphi_j u}
(\xi)| \lesssim (1 + |\xi|)^{-N}, \ \xi \in \Gamma_j. $
\vskip0.1in
\noindent
If we choose
$$V = V_1 \cap \cdots \cap V_k$$
and $ \varphi \in C^\infty_0 (V) $ with $ \varphi (x_0) \ne 0 $ then we obtain
$$|\widehat{\varphi u(\xi)}| \lesssim (1 + |\xi|)^{-N}, \xi \in \rr^n $$
which implies that $ \varphi u \in C^\infty_0 (X)$.  Since $ \varphi (x_0) \ne 0 $
this implies that $ u $ is $ C^\infty $ at $ x_0$, which is the desired
relation \eqref{2'}.
\vskip0.1in
\noindent
\begin{proposition}
	\label{prop3}
	Let $ u \in \mathcal D^\prime (X) $ and $ (x_0, \xi_0) \in X
\times  \dot \rr^n$. Then $ \  (x_0, \xi_0) \notin WFu $ if and only if there
exist: \vskip0.1in
\noindent
$\bullet \quad \varphi \in C^\infty_0 (X) \text{ with } \varphi (x_0) \ne 0 $
\vskip0.1in
\noindent
$\bullet \quad \chi \in C^\infty (\dot \rr^n) $ with $ \chi(\xi_0) \ne 0 $ and
homogeneous of degree 0 
\vskip0.1in
\noindent
such that
$$\chi (D) \varphi u \in C^\infty (\rr^n). $$
{\bf Recall: } $ \chi (D) \varphi $ is the  $\psi$do of  order  zero
defined by
\begin{equation}
	\label{1''}
\begin{split}
\chi(D) \varphi u(x) &= \int_{\rr^n} e^{ix \xi} \chi(\xi) \widehat{\varphi u}
(\xi) d \xi\\
&\overset{\text{osc.}}{=} \int_{\rr^n} \int_{\rr^n} e^{i(x-y) \xi}
\chi(\xi) \varphi (y) u (y) dy d \xi \end{split}
\end{equation}
The symbol of $ \chi(D) \varphi $ is
$$p(x, y, \xi) = \varphi (y)\chi(\xi). $$
\end{proposition}
\nin
{\bf Proof.}  If $ (x_0, \xi_0) \notin WFu $ then by the definition there is $
\varphi \in C^\infty_0 (X), \ \varphi (x_0) \ne 0$, and an open cone $ \Gamma,
\xi_0 \in \Gamma $ such that
$$\left | \widehat{\varphi u} (\xi) \right | \lesssim (1 + |\xi|)^{-N}, \ \xi \in
\Gamma$$ 
Then if we choose $ \chi (\xi) $ supported in $ \Gamma $ then
$$\chi(\xi) \widehat{\varphi u} (\xi) \in S^{- \infty} (\rr^n)$$
and therefore by \eqref{1''} $ \chi(D) \varphi u $ is in $ C^\infty (\rr^n)$.
\vskip0.1in
\noindent
{\bf Conversely: } Let $ \chi $ and $ \varphi $ be as in the statement of the
proposition and such that
\begin{equation}
	\label{2''}
	\chi(D) \varphi u \in C^\infty (\rr^n). 
\end{equation}
We would like to show that
\begin{equation}
	\label{3''}
	\chi(\xi) \widehat{\varphi u} (\xi) \text{ is rapidly decaying in } \rr^n.
\end{equation}
Define
$$v = \varphi u \in \mathcal E^\prime (\rr^n) \text{ and } T \in \mathcal S^\prime
(\rr^n) \text{ with } \hat T = \chi. $$
Then $ T $ is homogeneous of degree $ -n $ and
\begin{equation}
	\label{4''}
	\chi \widehat{\varphi u} = \hat T \hat v = \widehat{T \ast v}
\end{equation}
To show \eqref{3''} it suffices to show that for any $ \alpha $ 
$$\xi^\alpha \chi(\xi) \hat v(\xi) \text{ is bounded.}$$
Since
$$\xi^\alpha \chi(\xi) \hat v(\xi) = \widehat{D^\alpha (T\ast v)} (\xi) $$
it suffices to show that
\begin{equation}
	\label{5''}
	D^\alpha (T\ast v) \in L^1 (\rr^n). 
\end{equation}
Since $ T \ast v = \chi(D) \varphi u \in C^\infty (\rr^n) $ it suffices to study
the behavior of $ D^\alpha (T\ast v) (x) $ for $ |x| \longrightarrow \infty$.  If $
|x| $ is large and since the support of $ v $ is compact we have
\begin{equation}
	\label{6''}
	(D^\alpha T \ast v) (x) = <v(y), D^\alpha T(x-y)> 
\end{equation}
Since $ v \in \mathcal E^\prime (\rr^n) $ there exist $ K $ a compact set, and $ k
\in \Bbb N $ such that
\begin{equation}
	\label{7''}
	|<v, \psi>| \le C \sup_{\substack{\\
	y \in K\\ |\beta| \le k}}  |D^\beta \psi(y)| 
\end{equation}
By \eqref{6''} and \eqref{7''} we obtain
\begin{equation*}
\begin{split} 
	|(D^\alpha T \ast v)(x)| &\le C \substack{\sup\\
y \in K\\
|\beta| \le k}  |D^{\alpha + \beta} T(x-y)|\\
&\le C^\prime |x-y|^{-n -|\alpha| - |\beta|}\\
&\le C^{\prime \prime} |x|^{-n - |\alpha| - |\beta|}\\
&\le C |x|^{-n - |\alpha|} \end{split}
\end{equation*}
Since for $ |\alpha | \ge 1 $ the function $  |x|^{-n-|\alpha|} $ is integrable for $ |x|
> r > 0 $, and therefore we obtain \eqref{5''}. 
This completes the proof of Proposition \ref{prop3}.
\vskip0.1in
\begin{theorem}
	\label{thm1}
	Let $ P \in \Psi^m(X)$.  Then 
	\begin{equation}
		\label{1a}
		WF (Pu) \subset WF(u), \ \forall u \in \mathcal E^\prime (X). 
	\end{equation}
\end{theorem}
\noindent
{\bf Remark:}  If $ P $ is also properly supported then \eqref{1a} holds for all $ u \in
\mathcal D^\prime (X)$.
\vskip0.1in
\noindent
{\bf Proof.}  Let $ (x_0, \xi_0) \notin WF(u)$.  We shall show that $ (x_0,
\xi_0) \notin WF(Pu)$.  Since $ (x_0, \xi_0) \notin WF(u) $ by Proposition
\ref{prop3}
there exist $ \varphi \in C^\infty_0 (X), $ with $ \varphi (x) = 1 $ near $ x_0$,
and $ \chi \in C^\infty (\dot \rr^n) $ homogeneous of degree 0 with $ \chi =
1 $ near $ \xi_0 $ such that
\begin{equation}
	\label{2a}
	\chi(D) \varphi u \in C^\infty (\rr^n). 
\end{equation}
If $ \chi^\prime (x, D) \in \Psi^0 $ is a properly supported $ \psi$do with $ \chi
(D) - \chi^\prime (x, D) \in \Psi^{- \infty},$ then \eqref{2a} implies
\begin{equation}
	\label{3a}
	P\chi^\prime \varphi u \in C^\infty (X). 
\end{equation}
Now choose $ \varphi_1 \in C^\infty_0 (X) $ such that $ \varphi_1 \varphi =
\varphi_1 $ and $ \varphi_1 (x_0) \ne 0 $ and $ \chi_1 \in C^\infty(\dot 
\rr^n) $ such that $ \chi_1 \chi = \chi_1$ and $\chi_1 (\xi_0) \ne 0$.  To show
that $ (x_0, \xi_0) \notin WF(Pu) $ it suffices to show the following claim.
\vskip0.1in
\noindent
{\bf Claim:}  $ \chi_1 (D) \varphi_1 Pu \in C^\infty (\rr^n). $
\vskip0.1in
\noindent
In fact, we have
\begin{equation}
	\label{4a}
	\chi_1 (D) \varphi_1 Pu = \chi_1 (D) \varphi_1 P \chi^\prime (x, D) \varphi u
+ \chi_1 (D) \varphi_1 P(1 - \chi^\prime(x, D) \varphi)u. 
\end{equation}
By \eqref{3a} we have
\begin{equation}
	\label{5a}
	\chi_1 (D) \varphi_1 P \chi^\prime (x, D) \varphi u \in C^\infty (\rr^n).
\end{equation}
The total symbol of the operator $ \chi_1 (D) \varphi_1 P(1 - \chi^\prime (x, D)
\varphi) $ has complete symbol equal to zero since
$$\chi_1 (\xi) (1 - \chi(\xi) \varphi(x)) = 0 $$
Therefore
\begin{equation}
	\label{6a}
	\chi_1 (D) \varphi_1 P(1 - \chi^\prime (x, D) \varphi) u \in C^\infty (\Bbb
R^n). 
\end{equation}
The claim follows from \eqref{4a}-\eqref{6a} and this completes the proof
of the theorem.
\medskip
\noindent
\begin{theorem}
	\label{thm2}
	Let $ P \in \Psi^m (X) $ and $ (x_0, \xi_0) \in X \times
\dot \rr^n$.  If there exists an open set $ U \subset X $ with $ x_0 \in U $
and an open cone $ \Gamma \subset \dot \rr^n $ with $ \xi_0 \in \Gamma $
such that for any $ N \in \Bbb N $ and $ \alpha \in \Bbb N_0^n$
\begin{equation}
	\label{1b}
	(1 + |\xi|)^N \partial^\alpha_x \sigma (P) (x, \xi) \text{ is bounded for } (x,
\xi) \in U \times \Gamma 
\end{equation}
then
\begin{equation}
	\label{2b}
	(x_0, \xi_0) \notin WF (Pu), \ \forall u \in \mathcal E^\prime (X). 
\end{equation}
\end{theorem}
\noindent
{\bf Remark.}  If $ P  $ is properly supported then \eqref{2b} holds for any $ u \in \mathcal
D^\prime (X)$.
\vskip0.1in
\noindent
{\bf Proof.}  Let $ \varphi \in C^\infty_0 (U) $ with $ \varphi (x_0) \ne 0$, and
$ \chi \in C^\infty (\dot \rr^n) $ with supp$\chi \cap S^{n-1} \subset
\subset \Gamma \cap S^{n-1}, \ \chi $ homogeneous of degree 0 and $ \chi
(\xi_0) \ne 0$.  To show that $ (x_0, \xi_0) \notin WF(Pu) $ it suffices to show
that
\begin{equation}
	\label{3b}
	\chi(D) \varphi P \in \Psi^{- \infty} (X)
\end{equation}
since then $ \chi (D) \varphi Pu \in C^\infty (\rr^n)$.  In fact we have
$$\sigma (\chi(D)) = \chi(\xi), \ \sigma (\varphi) = \varphi (x) $$
By the composition formula
$$\sigma (\chi(D) \varphi) (x, \xi) \sim \sum_{\beta !} \frac{1}{\beta !}
\partial^\beta_\xi \chi (\xi) D^\beta_x \varphi (x) $$
and
\begin{equation*}
\begin{split} 
\sigma (\chi(D) \varphi P) (x, \xi) &\sim \sum_{\alpha !} \frac{1}{\alpha !}
\partial^\alpha_\xi \sigma (\chi(D) \varphi) (x, \xi) D^\alpha_x \sigma (P) (x,
\xi) \\
&\sim \sum_{\alpha, \beta} \frac{1}{\alpha ! \beta !} \partial^\alpha_\xi \left (
\partial^\beta_\xi \chi (\xi) D^\beta_x \varphi (x) \right ) D^\alpha_x \sigma
(P) (x, \xi) \end{split}
\end{equation*}
By the assumption \eqref{1b} every term in the sum is in $ S^{- \infty} (U \times \dot
\rr^n)$.  Therefore \eqref{3b} holds and this completes the proof of the theorem.
\begin{theorem}
	\label{thm3}
	Let $ P \in \Psi^m (X)$.  If $ P $ is elliptic then
	\begin{equation}
		\label{1c}
		WF(u) \subset WF(Pu), \ \forall u \in \mathcal E^\prime (X). 
	\end{equation}
\noindent
{\bf Remark.}  If $ P $ is also properly supported then \eqref{1c} holds for any $ u \in
\mathcal D^\prime (X)$.
\end{theorem}
\vskip0.1in
\noindent
{\bf Proof.}  We will need the following proposition:
\begin{proposition}
	\label{prop4}
If $P \in \Psi^m (X)$ then
$$P = P' + R,$$
where $P' $ is properly supported, and $ R $ is smoothing.
\end{proposition}
\nin
{\bf Proof.} Let $\{\psi_j\}$ be a locally finite partition of unity on $X$;
i.e. $\psi_j \in C_0^\infty (X)$, 
$1 = \sum_{j=1}^{\infty} \psi_j$, and locally finite.
Then define
$$P' u (x) = \sum_{\text{supp} \psi_j \cap \text{supp} \psi_k \ne 0} \psi_j (x)
P(\psi_ku).$$ 
The  operator $P'$ is properly supported, and the symbol of $P - P'$ is given
by 
$$\sum_{\text{supp} \psi_j \cap \text{supp} \psi_k = 0} \psi_j (x) p(x, y, \xi)
\psi_k (y)$$ 
which is equal to zero in a neighborhood of the diagonal of $X \times X$. Thus
$P-P'$ is smoothing. $\qquad \Box$
\vskip0.1in
\nin
Hence, by Propisition \ref{prop4} there exists a properly supported $ Q \in
\Psi^{-m} (X) $ such that
$$QP = I + R, \ R \in \Psi^{- \infty} (X).$$
Therefore for any $ u \in \mathcal E^\prime (X) $ we obtain
$$u = QPu - Ru. $$
Since $ WF (Ru) = \phi $ we obtain
$$WF(u) = WF(QPu) \subset WF(Pu), $$
and this completes the proof of the theorem.
\noindent
\begin{corollary}
	\label{cor1}
	If $ P \in \Psi^m(X) $ is elliptic then
\begin{equation}
	\label{2c}
	WF(u) = WF(Pu), \ \forall u \in \mathcal E^\prime (X). 
\end{equation}
\end{corollary}
\nin
{\bf Remark:}  The converse of Corollary \ref{cor1} is not true. For example, the
operator in $ \Bbb R^2 $ defined by
$$P = \frac{\partial}{\partial x_1} + ix_1^2 \frac{\partial}{\partial x_2}$$
satisfies \eqref{2c} while it is not elliptic on the set
$$\Sigma = \{(0, x_2; 0, \xi_2): \ \xi_2 \ne 0 \}.$$
Another example is the operator
$$P = \left ( \frac{\partial}{\partial x_1} \right )^2 + \left ( x^k_1
\frac{\partial}{\partial x_2} \right )^2, \ k \in \Bbb N $$
which satisfies \eqref{2c} but is not elliptic on $ \Sigma $ too.
\medskip
\noindent
\begin{definition}  $ P \in \Psi^m (X) $ is called hypoelliptic in $ X $ if
$$WF(u) = WF(Pu), \ \forall u \in \mathcal E^\prime (X).$$
\end{definition}
\medskip
\begin{theorem}
	\label{thm4}
	Let $ (x_0, \xi_0) \in X \times \dot \rr^n$, and $ P \in
\Psi^m (X) $ such that the principal symbol $ \sigma_m(P) $ satisfies
\vskip0.1in
{\bf 1.} $ \sigma_m (P) (x, \xi) $ is homogeneous of degree $ m $ in $\xi.$
\vskip0.1in
{\bf 2.} $ \sigma_m (P) (x_0, \xi_0) \ne 0$, i.e. $ P $ is elliptic at $ (x_0, \xi_0)$
\vskip0.1in
\noindent
Then for any $ u \in \mathcal E^\prime (X) $ we have
\begin{equation}
	\label{3c}
	(x_0, \xi_0) \notin WF (Pu) \Longrightarrow (x_0, \xi_0) \notin WF(u). 
\end{equation}
\end{theorem}
\nin
{\bf Proof.}  Let $ (x_0, \xi_0) \notin WF(Pu)$.  Then by Theorem
\ref{thm3} 
\begin{equation}
	\label{4c}
	(x_0, \xi_0) \notin WF (P^* Pu). 
\end{equation}
Now we observe that the principal symbol of $ P^* P $ is
\begin{equation}
	\label{5c}
	\sigma_{2m} (P^*P) = |\sigma_m (P) (x, \xi) |^2. 
\end{equation}
Next we shall modify $ P^*P $ to be elliptic everywhere.  By \eqref{1} and
\eqref{2} there
exist $ U $ open $ \subset X, \ x_0 \in U $ and $ \Gamma $ open $ \subset
\dot \rr^n\  \xi_0 \in \Gamma $ such that
\begin{equation}
	\label{6}
	\sigma_m (P) (x, \xi) \ne 0, \ x \in U, \ \xi \in \Gamma. 
\end{equation}
Now we choose
\vskip0.1in
\noindent
$\bullet \ \chi \in C^\infty (\rr^n), \ 0 \le \chi \le 1$, homogeneous of
degree $ 0, \ \chi(\xi) = 0 $ near $ \xi_0 $ and $ \chi(\xi) = 1 $ in $ S^{n-1} \cap
(\dot \rr^n - \Gamma), $
\vskip0.1in
\noindent
and we define $ Q \in \Psi^m (U) $ by
\begin{equation}
	\label{7c}
	\sigma(Q) (x, \xi) = \sigma (P^*P) + |\xi|^{2m} \chi(\xi). 
\end{equation}
Then by \eqref{7c}
\begin{equation}
	\label{8c}
	\sigma_{2m} (Q) (x, \xi) = |\sigma_m (P) (x, \xi)|^2 + |\xi|^{2m} \chi(\xi).
\end{equation}
By \eqref{7c} and the definition of $ \chi $ we obtain
\begin{equation}
	\label{9c}
	\sigma_{2m} (Q) (x, \xi) \ne 0, \ x \in U, \ \xi \in \dot \rr^n, 
\end{equation}
i.e. $ Q $ is elliptic in $ U$.  Also by \eqref{7c} we obtain that
\begin{equation}
	\label{10c}
	\sigma(Q - P^*P) = 0 \text{ near } (x_0, \xi_0). 
\end{equation}
By \eqref{10c} and Theorem \ref{thm3} we obtain that
\begin{equation}
	\label{11c}
	(x_0, \xi_0) \notin WF((Q - P^*P)u) 
\end{equation}
Then by \eqref{4c} and \eqref{11c} 
\begin{equation}
	\label{12c}
	(x_0, \xi_0) \notin WF(Qu). 
\end{equation}
Since by \eqref{9c} $ Q $ is elliptic,  relation \eqref{12c} implies that $ (x_0, \xi_0) \notin
WF(u)$, and this completes the proof of the theorem.
\noindent
\begin{corollary}
	\label{cor2}
	If $ P \in \Psi^m (X) $ is elliptic at $ (x_0, \xi_0) \in X \times
\dot \rr^n $ then for any $ u \in \mathcal E^\prime (X)$
$$(x_0, \xi_0) \notin WF(u) \Longleftrightarrow (x_0, \xi_0) \notin WF(Pu), $$
or equivalently
$$(x_0, \xi_0) \in WF(u) \Longleftrightarrow (x_0, \xi_0) \in WF(Pu). $$
\end{corollary}
\nin
Corollary \ref{cor2} motivates the following definition of the wave front, which is
stated as a theorem.
\vskip0.1in
\begin{theorem}
	\label{thm5}
	For any $ u \in \mathcal E^\prime (X) $
	\begin{equation}
		\label{1d}
		WF(u) = \bigcap_{\substack{  P \in \Psi^0(X)\\
		\sigma_0 (P) \text{ homo. of deg. 0 in } \xi\\  Pu \in C^\infty
		(X)}}
		\left \{ (x, \xi) \in X \times \dot
\rr^n: \ \sigma_0 (P) (x, \xi) = 0 \right \} 
\end{equation}
\end{theorem}
\noindent
{\bf Remark.}  If $ u \in \mathcal D^\prime (X) $ then in \eqref{1d}
$ P $ should also be
properly supported.
\vskip0.1in
\noindent
{\bf Proof.} Relation \eqref{1d} is equivalent to 
\begin{equation}
	\label{2d}
	(X \times \dot \rr^n) - WF(u) = \bigcup_{\substack{  P \in \Psi^0(X)\\
		\sigma_0 (P) \text{ homo. of deg. 0 in } \xi\\  Pu \in C^\infty
		(X)}}
		\left \{ (x, \xi) \in X \times \dot
\rr^n: \ \sigma_0 (P) (x, \xi) \neq 0 \right \} 
\end{equation}
\noindent
{\bf Proof of $\boldsymbol \subset$:}  Let $ (x_0, \xi_0) \notin WF(u)$. Then by
Proposition \ref{prop3} there exist $ \varphi \in C^\infty_0 (X), \ \varphi (x_0 ) \ne 0$,
and $ \chi \in C^\infty (\dot \rr^n)$, homogeneous of degree 0 with $ \chi
(\xi_0) \ne 0 $ such that
$$\chi(D) \varphi u \in C^\infty (X). $$
Then $ P = \chi(D) \varphi \in \Psi^0 (X) $ with $ \sigma_0 (P) = \chi(\xi)
\varphi (x) $ homogeneous of degree 0 in $ \xi$, and $ \sigma_0 (P) (x_0, \xi_0)
\ne 0$.  Therefore $ (x_0, \xi_0) $ is in the right-hand side of \eqref{2d}.
\vskip0.1in
\noindent
{\bf Proof of $\boldsymbol  \supset$:} Let $ (x_0, \xi_0) $ be in the right-hand
side of \eqref{2d}.  Then there exists $ P \in \Psi^0 (X) $ which is elliptic at $ (x_0,
\xi_0) $ with $ (x_0, \xi_0) \notin WF(Pu)$.  Then by Theorem \ref{thm5}
we obtain that $
(x_0, \xi_0) \notin WF (u) $ and this completes the proof of the theorem.
\medskip
\begin{corollary}
	\label{cor3}
	The Wave Front is invariant under $ C^\infty $ changes of
variables.
\end{corollary}
\medskip
\noindent
{\bf Proof.}  It follows by Theorem \ref{thm5} and the fact $ \psi$do's are invariant
under change of variables.  We recall that if $ X, Y $ are open
sets in $ \rr^n $ and
$$F: \ X \longrightarrow Y \text{ a } C^\infty \text{diffeomorphism}$$
then the following formula holds
$$\xi = {}^tF'(x) \eta. $$
\noindent
{\bf Remark.}  Due to Corollary \ref{cor3}, the notion of Wave Front can be well
defined on manifolds.
\vskip0.1in
\nin
Next we shall describe the wave front of a distribution defined by an
oscillatory integral (see $\S$2).
\medskip
\begin{theorem}
	\label{thm6}
	Let $ X $ be an open subset of $ \rr^n, \ \varphi (x,
\theta) $ be a phase in $ X \times \dot \rr^N$, and $ a(x, \theta) \in \mathcal S^m
(X \times \rr^N)$.  If $ I(a, \varphi) $ is the distribution defined by
$$I(a, \varphi) \overset{\text{osc.}}{=} \int_{\rr^N} \ e^{i \varphi (x,
\theta)} a(x, \theta) d \theta $$
then
\begin{equation}
	\label{1e}
	WF(I(a, \varphi))\subset \Lambda_\varphi
\end{equation}
where
\begin{equation}
	\label{2e}
	\Lambda_\varphi = \{ (x, \varphi^\prime_x (x, \theta)): \varphi_\theta' (x,
\theta) = 0 \text{ for some } \theta \ne 0 \text{ and } x \in X \}. 
\end{equation}
\end{theorem}
\nin
{\bf Proof.}  Let $ (x_0, \xi_0) \in (X \times \dot \rr^n) -
\Lambda_\varphi$, and $ \psi \in C^\infty_0 (X) $ with $ \psi(x_0) \ne 0$.  We
have
\begin{equation*}
\begin{split}
\widehat{\psi I(a, \varphi)} (\xi) &= <\psi I(a, \varphi),\  e^{-i \cdot \xi}>\\
&\overset{\text{osc.}}{=} \int_{\rr^n} \int_{\Bbb R^N} \ e^{i(\varphi(x,
\theta) - x \xi)} \psi(x) a (x, \theta) d \theta dx \end{split}
\end{equation*}
We would like to show that if $ \psi $ is supported near $ x_0 $ there exists an
open cone $ \Gamma_0, \xi_0 \in \Gamma_0$, such that
\begin{equation}
	\label{3e}
	\left | \widehat{\psi I(a, \varphi)} (\xi) \right | \le C_N (1 + |\xi|)^{-N}, \ \xi \in
\Gamma_0. 
\end{equation}
It can be shown easily that $ \Lambda_\varphi $ is closed conic set in $ X
\times \dot \rr^n$.  Since $ (x_0, \xi_0) \notin \Lambda_\varphi $ there
exists $ U $ open $\subset X, \ x_0 \in U $ and $ \Gamma $ open in $ \dot
\rr^n, \ \xi_0 \in \Gamma $ such that
$$(U \times \Gamma) \cap \Lambda_\varphi = \emptyset .$$
Therefore if we choose $ \Gamma_0 \subset \subset \Gamma, \ \xi_0 \in
\Gamma_0 $,  then there exists $ \varepsilon > 0 $ such that
\begin{equation}
	\label{4e}
	|\xi - \varphi'_x (x, \theta)| \ge \varepsilon (|\xi| + |\theta|), \ \xi \in
\Gamma_0, \ x \in U, \ \theta \in \dot \rr^N. 
\end{equation}
By \eqref{4c} we obtain
$$\left | \frac{\partial}{\partial x} (\varphi(x, \theta) - x\xi) \right | \ge
\varepsilon (|\xi| + |\theta|), \ \xi \in \Gamma_0, \ x \in U, \ \theta \in \dot 
\rr^N.$$
Now if we choose $ \psi \in C^\infty_0 (U) $ with $ \psi (x_0) \ne 0 $ and
$$L = \frac{1}{|\varphi^\prime_x (x, \theta) - \xi|^2} \ \sum^n_{j=1} \left (
\frac{\partial \varphi}{\partial x_j} - \xi_j \right ) D_{x_j}$$
\centerline{$\cdots$}
{\bf Example 1.}  If $ \varphi(x, y, \xi) = (x - y) \cdot \xi $ then $ \varphi_\xi = x
- y = 0 $ if $ x = y$.  Also we have $ \varphi_{x,y} = (\xi, - \xi)$.  Therefore for
the kernel of a $ \psi$do in $ X $ we have
$$\Lambda_\varphi = \{(x, x; \xi, - \xi): \ x \in X, \ \xi \in \dot \rr^n\}$$
which is the conormal bundle of $ \Delta.$
\vskip0.1in
\noindent
{\bf Example 2:}  If $ \varphi = (x - y) \cdot \xi + t|\xi| $ then 
$$ \varphi_\xi = x- y + t \frac{\xi}{|\xi|} = 0 \text{ if } |x-y| = |t|.$$
Also we have $ \varphi_{x,y} = (\xi, - \xi)$.  Therefore
$$\Lambda_\varphi = \{(x, y; \xi,  - \xi): |x-y| =|t|, \ \xi \in \dot \rr^n\}$$
\vskip0.1in
\centerline{\bf .......}
%
\end{document}



                                                                                                                                                                   

                                                                                                                                                                                                                                                                            
 

