\documentclass[12pt,reqno]{amsart}
\usepackage{amscd}
\usepackage{amsfonts}
\usepackage{amsmath}
\usepackage{amssymb}
\usepackage{amsthm}
\usepackage{appendix}
\usepackage{fancyhdr}
\usepackage{latexsym}
\usepackage{pdfsync}
\usepackage{cancel}
\usepackage{amsxtra}
\usepackage[colorlinks=true, pdfstartview=fitv, linkcolor=blue,
citecolor=blue, urlcolor=blue]{hyperref}
\input epsf
\input texdraw
\input txdtools.tex
\input xy
\xyoption{all}
%%%%%%%%%%%%%%%%%%%%%%
\usepackage{color}
\definecolor{red}{rgb}{1.00, 0.00, 0.00}
\definecolor{darkgreen}{rgb}{0.00, 1.00, 0.00}
\definecolor{blue}{rgb}{0.00, 0.00, 1.00}
\definecolor{cyan}{rgb}{0.00, 1.00, 1.00}
\definecolor{magenta}{rgb}{1.00, 0.00, 1.00}
\definecolor{deepskyblue}{rgb}{0.00, 0.75, 1.00}
\definecolor{darkgreen}{rgb}{0.00, 0.39, 0.00}
\definecolor{springgreen}{rgb}{0.00, 1.00, 0.50}
\definecolor{darkorange}{rgb}{1.00, 0.55, 0.00}
\definecolor{orangered}{rgb}{1.00, 0.27, 0.00}
\definecolor{deeppink}{rgb}{1.00, 0.08, 0.57}
\definecolor{darkviolet}{rgb}{0.58, 0.00, 0.82}
\definecolor{saddlebrown}{rgb}{0.54, 0.27, 0.07}
\definecolor{black}{rgb}{0.00, 0.00, 0.00}
\definecolor{dark-magenta}{rgb}{.5,0,.5}
\definecolor{myblack}{rgb}{0,0,0}
\definecolor{darkgray}{gray}{0.5}
\definecolor{lightgray}{gray}{0.75}
%%%%%%%%%%%%%%%%%%%%%%
%%%%%%%%%%%%%%%%%%%%%%%%%%%%
%  for importing pictures  %
%%%%%%%%%%%%%%%%%%%%%%%%%%%%
\usepackage[pdftex]{graphicx}
\usepackage{epstopdf}
% \usepackage{graphicx}
%% page setup %%
\setlength{\textheight}{20.8truecm}
\setlength{\textwidth}{14.8truecm}
\marginparwidth  0truecm
\oddsidemargin   01truecm
\evensidemargin  01truecm
\marginparsep    0truecm
\renewcommand{\baselinestretch}{1.1}
%% new commands %%
\newcommand{\bigno}{\vskip0.1in\noindent}
\newcommand{\ds}{\displaystyle}
\newcommand{\medno}{\vskip0.1in\noindent}
\newcommand{\smallno}{\vskip0.1in\noindent}
\newcommand{\ts}{\textstyle}
\newcommand{\rr}{\mathbb{R}}
\newcommand{\p}{\partial}
\newcommand{\zz}{\mathbb{Z}}
\newcommand{\cc}{\mathbb{C}}
\newcommand{\ci}{\mathbb{T}}
\newcommand{\ee}{\varepsilon}
\def\refer #1\par{\noindent\hangindent=\parindent\hangafter=1 #1\par}
%% equation numbers %%
\renewcommand{\theequation}{\thesection.\arabic{equation}}
%% new environments %%
%\swapnumbers
\theoremstyle{plain}  % default
\newtheorem{theorem}{Theorem}
\newtheorem{proposition}{Proposition}
\newtheorem{lemma}{Lemma}
\newtheorem{corollary}{Corollary}
\newtheorem{conjecture}[subsection]{conjecture}
\theoremstyle{definition}
\newtheorem{definition}{Definition}
\newcommand{\ve}{\varepsilon} 
\newcommand{\nin}{\noindent}
\newcommand{\oL}{\bar L}
\newcommand{\vph}{\varphi}
\begin{document}
\title{Pseudodifferential and Fourier Integral Operators}
\author{Alex Himonas, {\it Summer 2009}}
\maketitle
\setcounter{section}{5}
\section{Applications of the Asymptotic Expansion Formula (AEF)}
\vskip0.2in
\nin
{\bf 1. A characterization of $ \psi$do's:}
\begin{theorem}
	\label{thm1}
	Let $ P: C^\infty_0 (X) \longrightarrow C^\infty (X) $
be a continuous linear operator.  Then $ P \in \Psi^m (X) $ if and only if
\begin{equation}
	\label{1}
	e^{-ix \xi} P(u(y) e^{iy \xi}) (x) \in S^m (X \times \Bbb R^n), \ \forall u \in
C^\infty_0 (X). 
\end{equation}
If in addition $ P $ is proper then $ P \in \Psi^m(X) $  if  and only if
\begin{equation}
	\label{2}
	e^{-ix \xi} P(e^{i y \xi} )(x) \in S^m(X \times \Bbb R^n). 
\end{equation}
\end{theorem}
\nin
{\bf Proof.} Let $ P \in \Psi^m(X)$.  If in the AEF we
let
 $$\psi(y, \xi) = y  \xi \text{ and } a(y) = u(y) \in S^0$$
 then we obtain \eqref{1}.  If $ P $ is properly supported and in the AEF we let
$$\psi (y, \xi) = y  \xi \text{ and } a(y) = 1 $$
then we obtain \eqref{2}.
\vskip0.1in
\nin
{\bf Conversely:}  Let $ P: C^\infty(X) \longrightarrow C^\infty (X) $ be a proper
continuous operator.  Then for any $ u \in C^\infty_0 (X) $ we write
$$u(x) = \frac{1}{(2 \pi)^n} \int_{\Bbb R^n} e^{ix \xi} \hat u (\xi) d \xi,$$
and by the continuity of $ P $ we can pass $ P $ inside the integral to obtain
\begin{equation}
	\begin{split}
Pu(x) &= \frac{1}{(2\pi)^n} \int_{\Bbb R^n} P(e^{ix \xi}) \hat u (\xi) d \xi \\
&= \frac{1}{(2 \pi)^n} \int_{\Bbb R^n} e^{i x \xi} \left [ e^{-ix \xi} P(e^{ix \xi})
\right ] \hat u (\xi) d \xi, \end{split}
\end{equation}
and by \eqref{2} we conclude that $ P \in \Psi^m (X)$.
\vskip0.1in
\nin
{\bf $ \bold P $ not Proper:}  If $ P $ is not proper then we choose a locally finite
partition of unity $ \{\alpha_j\} $ on $ X$; i.e.
$$1 = \sum^\infty_{j=1} \alpha_j (x) \text{ on } X$$
and on any compact set $ K \subset X $ only finite many $ \alpha_j $ are not $ 0.$
Also we choose another locally finite partition of unity $ \{\beta_j\} $ on $ X $
such that 
$$\beta_j = 1 \text{ on supp } \alpha_j.$$
Now for any $ u \in C^\infty_0 (X) $ we write
\begin{equation*}
\begin{split}
(\alpha_j u) (x) &= \frac{1}{(2 \pi)^n} \int_{\Bbb R^n} e^{i x \xi}
\widehat{\alpha_j u} (\xi) d \xi\\
&= \frac{1}{(2 \pi)^n} \int_{\Bbb R^n} \beta_j (x) e^{i x \xi} \widehat{\alpha_j
u} (\xi) d \xi. \end{split}
\end{equation*}
By the continuity of $ P $ we pass $ P $ inside the integral to obtain
\begin{equation*}
\begin{split}
P(\alpha_j u) (x) &= \frac{1}{(2 \pi)^n} \int_{\Bbb R^n} P(\beta_j (x) e^{i x \xi})
\widehat{\alpha_j u} (\xi) d \xi\\
&= \frac{1}{(2 \pi)^n} \int_{\Bbb R^n} \int_X e^{i(x-y)\xi} \left [ e^{-ix \xi}
P(\beta_j (\cdot) e^{i \cdot \xi} ) (x) \right ] a_j (y) u (y) dy d\xi \end{split}
\end{equation*}
Since by \eqref{1} we have
$$p_j(x, \xi) \doteq e^{-ix \xi} P(\beta_j (\cdot) e^{i \cdot \xi}) (x) \in S^m (X
\times \Bbb R^n)$$
and since
\begin{equation*}
\begin{split}
Pu(x) &= \sum P(\alpha_j u)(x)\\
&= \frac{1}{(2 \pi)^n} \int_{\Bbb R^n} \int_X e^{i(x-y) \xi} \left ( \sum p_j(x,
\xi) \alpha_j (y) \right ) u(y) dy d \xi \end{split}
\end{equation*}
we conclude that $ P \in \Psi^m(X) $ with symbol
$$p(x, y, \xi) = \sum p_j (x, \xi) \alpha_j (y).$$
\vskip0.1in
\nin
{\bf 2. Definition of Complete Symbol}
\begin{theorem}
	\label{thm2}
	If $ P \in \Psi^m(X) $ and properly supported, 
then there exists a
unique symbol $ \sigma_P \in S^m (X \times \Bbb R^n) $,
called the {\bf  Complete Symbol} of $P$, which is given by the formula
$$\sigma_P (x, \xi) = e^{-i x \xi} P(e^{iy \xi} )(x), $$
and  is  such that 
\begin{equation}
	\label{3}
	Pu(x) = \frac{1}{(2 \pi)^n} \int_{\Bbb R^n} e^{i x \xi} \sigma_P (x, \xi) \hat
	u(\xi) d \xi, \ \forall u \in \mathcal{S}(\Bbb R^n).
\end{equation}
If $ p(x, y, \xi) $ is a symbol for $ P $ then
\begin{equation}
	\label{4}
	\sigma_P (x, \xi) \sim \sum_\alpha \frac{1}{\alpha !} \partial^\alpha_\xi
D^\alpha_y p(x, y, \xi) \big |_{y=x}. 
\end{equation}
\end{theorem}
\nin
{\bf Proof.}  By Theorem \ref{thm1} we obtain \eqref{3}. 
Now from the AEF applied with $ \psi (x, \xi) = x \xi $ and $ a = 1 $ we obtain
$$e^{-ix \xi} P(e^{-i y \xi})(x) \sim \sum_{\alpha, \beta} \frac{1}{\alpha ! \beta !}
\left ( \partial^{\alpha + \beta}_\xi D^\beta_y p \right ) (x, x, \xi) D^\alpha_y
(e^{ir(x, y, \xi)})_{\big |_{y=x}}$$
where here $ r(x, y, \xi) = y\xi - x \xi - <y-x, \xi> = 0$.  Therefore we obtain
formula \eqref{4}.
\vskip0.1in
\noindent
To show the uniqueness of $ \sigma_p $ we assume that there exist another
symbol $ a \in S^m (X \times \Bbb R^n) $ such that \eqref{3} holds. If
$$b(x, \xi) = \sigma_P (x, \xi) - a(x, \xi) $$
then we would have
$$\int_{\Bbb R^n} e^{ix \xi}b(x, \xi)  \hat u (\xi) d \xi = 0, \ \forall u
\in \mathcal{S}(\Bbb R^n). $$
But for fixed $ x \in X $ the function $ e^{ix \xi} b(x, \xi) \in \mathcal S^\prime(\Bbb
R^n)$.  By the last integral we must have
$$e^{ix \xi} b(x, \xi) = 0, \ \forall x \in X, \ \xi \in \Bbb R^n.$$
Therefore $ b \equiv 0$.  This completes the proof of Theorem \ref{thm2}.
\vskip0.1in
\nin
{\bf 3. The Complete Symbol of the Transpose and the Adjoint:}
\vskip0.1in
\nin
We have seen that if $ P \in \Psi^m(X) $ with amplitude $ p(x, y, \xi) $ then the
following hold:
\vskip0.1in
\noindent
{\bf 1.}  The transpose $ {}^tP $ of $ P $ has amplitude $ p(y, x, - \xi)$.
\vskip0.1in
\noindent
{\bf 2.}  The adjoint $ P^* $ of $ P $ has amplitude $ \overline{p(y, x, \xi)}$.
\vskip0.1in
\nin
\begin{theorem}
	\label{thm3}
	If $ P \in \Psi^m(X) $ and proper with complete symbol
$ \sigma_p(x, \xi) $ then the complete symbols of ${}^t P $ and $ P^* $ have the
following asymptotic expansions
\begin{equation}
	\label{5}
	\sigma_{{}^tP} (x, \xi) \sim \sum \frac{(-1)^{|\alpha|}}{\alpha !} \left (
\partial^\alpha_\xi D^\alpha_x \sigma_P \right ) (x, - \xi) 
\end{equation}
\begin{equation}
	\label{6}
\sigma_{P^*} (x, \xi) \sim \sum \frac{1}{\alpha !} \partial^\alpha_\xi
D^\alpha_x \overline{\sigma_P (x, \xi)}.
\end{equation}
\end{theorem}
\nin
{\bf Proof.}  It follows by the above two facts and the AEF. 
\vskip0.2in
\nin
{\bf 4. Composition of two Proper $ \psi$do's}
\begin{theorem}
	\label{thm4}
	If $ P_j \in \Psi^{m_j}, \ j = 1, 2, $ are two properly
supported $\psi$do's then
$ P_1 P_2 \in \Psi^{m_1 + m_2} (X) $ with complete symbol admitting the
asymptotic expansion
\begin{equation}
	\label{7}
	\sigma_{P_1P_2} (x, \xi) \sim \sum_\alpha \frac{1}{\alpha !}
\partial^\alpha_\xi \sigma_{P_1} (x, \xi) D^\alpha_x \sigma_{P_2} (x, \xi).
\end{equation}
\end{theorem}
\nin
{\bf Proof.}  We have
\begin{equation*}
	\begin{split}
e^{-ix \xi} P_1 P_2 (e^{iy \xi})(x) &= e^{-ix \xi}  P_1 \left [ P_2 \left
(e^{iz \xi} \right )(y) \right ](x)\\
&= e^{-ix \xi} P_1 \big  [ e^{iy \xi} \{e^{-iy \xi} 
P_2 (e^{i z \xi}) (y) \big ](x)\\ 
&= e^{-ix \xi} P_1 \left [e^{iy \xi} \sigma_{P_2} (y, \xi) \right ] (x) \in
S^{m_1 + m_2} (X \times \Bbb R^n).
\end{split}
\end{equation*}
The last relation follows by the AEF.  Therefore by Theorem \ref{thm1}, $ P_1 P_2 \in
\Psi^{m_1 + m_2}(X)$.  Formula \eqref{7} follows by expanding the last quantity by
using the AEF.  This completes the proof of Theorem \ref{thm4}.
\vskip0.1in
\noindent
{\bf 5. Commutators}
\vskip0.1in
\begin{theorem}
	\label{thm5}
	Let $ P_j \in \Psi^{m_j} (X), \ j = 1, 2,$ be
properly supported.  Then
\vskip0.1in
\noindent
{\bf 1.}  $ \sigma_{P_1 P_2} - \sigma_{P_1} \sigma_{P_2} \in S^{m_1 + m_2 -1}
(X \times \Bbb R^n) $.
\vskip0.1in
\noindent
{\bf 2.}  The commutator $ [P_1, P_2] \doteq P_1 P_2 - P_2 P_1 \in \Psi^{m_1
+ m_2 -1} (X) $ with
$$\sigma_{[P_1, P_2]} - \frac{1}{i} \{ \sigma_{P_1}, \sigma_{P_2} \} \in
\Psi^{m_1 + m_2 -2} (X), $$
where for two functions $ f(x, \xi) $ and $ g(x, \xi) $ 
$$\{f, g\} = \sum^n_{j=1} \frac{\partial f}{\partial \xi_j} \frac{\partial
g}{\partial x_j} - \frac{\partial f}{\partial x_j} \frac{\partial g}{\partial \xi_j}, $$
denotes their
Poisson  bracket.
\end{theorem}
\nin
{\bf Proof.}  It follows by the above results.
\vskip0.1in
\noindent
{\bf 6.  Change of Variables}
\vskip0.1in
\nin
Let $ X, Y $ be two open sets in $ \Bbb R^n $ and
$$F: X \longrightarrow Y $$ a diffeomorphism, and let $ F_* $ be the map $$v \in
C^\infty_0 (Y) \xrightarrow{F_*} F_* (v) = v_0 F \in C^\infty_0
(X). $$ Also let $ P \in \Psi^m(X) $ which with no loss of generality we shall
assume to be proper.  Then by using the following diagram $$\CD C^\infty(X)
@>P>> C^\infty (X)\\ @AAF_*A    @VVF^{-1}_*V\\ C^\infty (Y) @>Q>> C^\infty
(Y) \endCD$$ we define the operator $ Q : C^\infty (Y) \longrightarrow C^\infty
(Y) $ by the following formula $$Q = F^{-1}_* P F_*$$
\begin{theorem}  The operator $ Q \in \Psi^m (Y) $ with complete symbol
having the asymptotic expansion:
$$\sigma_Q(y, \eta) \sim \sum \frac{1}{\alpha !} \left ( \partial^\alpha_\xi
\sigma_P \right ) (F^{-1} (y), {}^tF^\prime (F^{-1} (y)) \eta) 
 D^\alpha_z (e^{ir(F^{-1} (y), z, \eta})
\bigg |_{z=F^{-1} (y)}$$ 
where
$$r(x, z, \eta) = F(z) \eta - F(x) \eta - {}^tF^\prime (x) \eta \cdot (z - x). $$
\end{theorem}
\nin
{\bf Proof.}  By Theorem \ref{thm1} we must show that
\begin{equation}
	\label{2'}
	q(y, \xi) \doteq e^{-iy \eta} Q(e^{iy \eta} ) \in S^m (Y \times \Bbb R^n).
\end{equation}
We have
$$q(y, \eta) = e^{-iy \eta} P(e^{iF(x) \eta}) F^{-1} (y) $$
and $ q(y, \eta) \in S^m (X \times \Bbb R^n) $ if
$$f(x, \eta) \doteq e^{iF(x) \eta} P(e^{iF(\cdot) \eta}) (x) \in S^m (X \times
\Bbb R^n).$$
Now we apply the AEF with
$$\psi (x, \eta) = F(x) \eta \text{ and } a(y, \eta) = 1, $$
since $ \psi^\prime_x (x, \eta) = F^\prime_x (x) \eta \ne 0 $ for $ \eta \ne 0$.
Therefore we obtain that $ f(x, \eta) \in S^m(X \times \Bbb R^n) $ and
$$f(x, \eta) \sim \sum \frac{1}{\alpha !} \left ( \partial^\alpha_\xi \sigma_P
\right ) (x, {}^tF^\prime (x) \eta) \cdot D^\alpha_y (e^{ir(x, y, \eta}) \bigg
|_{y=x}$$ where
$$r(x, y, \eta) = F(y) \eta - F(x) \eta - {}^tF^\prime (x) \eta \cdot  (y - x). $$
%
\end{document}
%
%                                                                                                                                                                   
%
%                                                                                                                                                                                                                                                                            
% 
%
