\documentclass[12pt,reqno]{amsart}
\usepackage{amscd}
\usepackage{amsfonts}
\usepackage{amsmath}
\usepackage{amssymb}
\usepackage{amsthm}
\usepackage{appendix}
\usepackage{fancyhdr}
\usepackage{latexsym}
\usepackage{pdfsync}
\usepackage{cancel}
\usepackage{amsxtra}
\usepackage[colorlinks=true, pdfstartview=fitv, linkcolor=blue,
citecolor=blue, urlcolor=blue]{hyperref}
\input epsf
\input texdraw
\input txdtools.tex
\input xy
\xyoption{all}
%%%%%%%%%%%%%%%%%%%%%%
\usepackage{color}
\definecolor{red}{rgb}{1.00, 0.00, 0.00}
\definecolor{darkgreen}{rgb}{0.00, 1.00, 0.00}
\definecolor{blue}{rgb}{0.00, 0.00, 1.00}
\definecolor{cyan}{rgb}{0.00, 1.00, 1.00}
\definecolor{magenta}{rgb}{1.00, 0.00, 1.00}
\definecolor{deepskyblue}{rgb}{0.00, 0.75, 1.00}
\definecolor{darkgreen}{rgb}{0.00, 0.39, 0.00}
\definecolor{springgreen}{rgb}{0.00, 1.00, 0.50}
\definecolor{darkorange}{rgb}{1.00, 0.55, 0.00}
\definecolor{orangered}{rgb}{1.00, 0.27, 0.00}
\definecolor{deeppink}{rgb}{1.00, 0.08, 0.57}
\definecolor{darkviolet}{rgb}{0.58, 0.00, 0.82}
\definecolor{saddlebrown}{rgb}{0.54, 0.27, 0.07}
\definecolor{black}{rgb}{0.00, 0.00, 0.00}
\definecolor{dark-magenta}{rgb}{.5,0,.5}
\definecolor{myblack}{rgb}{0,0,0}
\definecolor{darkgray}{gray}{0.5}
\definecolor{lightgray}{gray}{0.75}
%%%%%%%%%%%%%%%%%%%%%%
%%%%%%%%%%%%%%%%%%%%%%%%%%%%
%  for importing pictures  %
%%%%%%%%%%%%%%%%%%%%%%%%%%%%
\usepackage[pdftex]{graphicx}
\usepackage{epstopdf}
% \usepackage{graphicx}
%% page setup %%
\setlength{\textheight}{20.8truecm}
\setlength{\textwidth}{14.8truecm}
\marginparwidth  0truecm
\oddsidemargin   01truecm
\evensidemargin  01truecm
\marginparsep    0truecm
\renewcommand{\baselinestretch}{1.1}
%% new commands %%
\newcommand{\bigno}{\bigskip\noindent}
\newcommand{\ds}{\displaystyle}
\newcommand{\medno}{\medskip\noindent}
\newcommand{\smallno}{\smallskip\noindent}
\newcommand{\ts}{\textstyle}
\newcommand{\rr}{\mathbb{R}}
\newcommand{\p}{\partial}
\newcommand{\zz}{\mathbb{Z}}
\newcommand{\cc}{\mathbb{C}}
\newcommand{\ci}{\mathbb{T}}
\newcommand{\ee}{\varepsilon}
\def\refer #1\par{\noindent\hangindent=\parindent\hangafter=1 #1\par}
%% equation numbers %%
\renewcommand{\theequation}{\thesection.\arabic{equation}}
%% new environments %%
%\swapnumbers
\theoremstyle{plain}  % default
\newtheorem{theorem}{Theorem}
\newtheorem{proposition}{Proposition}
\newtheorem{lemma}{Lemma}
\newtheorem{corollary}{Corollary}
\newtheorem{conjecture}[subsection]{conjecture}
\theoremstyle{definition}
\newtheorem{definition}{Definition}
\newcommand{\ve}{\varepsilon} 
\newcommand{\nin}{\noindent}
\newcommand{\oL}{\bar L}
\newcommand{\vph}{\varphi}
\begin{document}
\title{Pseudodifferential and Fourier Integral Operators}
\author{Alex Himonas, {\it Summer 2009}}
\maketitle
\setcounter{section}{1}
\section{Oscillatory Integrals}
\nin
We have seen that a partial differential operator of 
order $m,\  P = P(x,D)$  can be
written in the form
\begin{equation}
	\begin{split}
		P u(x)&= \frac{1}{(2\pi)^n}\int_{\rr^n} e^{ix\xi} p(x,\xi)\hat u(\xi) d\xi\\
&= \frac{1}{(2\pi)^n} \int_{\rr^n}e^{ix\xi} p(x,\xi)\left(\int_{\rr^n}e^{-iy\xi}u(y)dy\right) d\xi\\
&=\frac{1}{(2\pi)^n} \int_{\rr^n}
 \int_{\rr^n}e^{i(x-y)\cdot\xi}p(x,\xi)u(y)dy
d\xi, 
\end{split}
\end{equation}
where $p(x,\xi)$ is the symbol of $P$.
The last integral is an example of an oscillatory integral and it makes sense only
as an iterated integral.  Below we shall express it as an absolutely convergent
integral.  There are two methods for
 this and we shall demonstrate them for the case
where $ p = 1; $   i.e.  the identity operator,  and in the dimension 1.
\vskip0.1in
\nin
{\bf Integration by Parts Method:}  For $ u \in C^\infty_0 (\rr) $ let
\begin{equation}
	\begin{split}
		Pu(x) = \frac{1}{2\pi} \iint\limits_{\rr  \rr}
e^{i(x-y)\xi}u(y)dyd\xi\quad (=u(x)).
		\label{2}
	\end{split}
\end{equation}
Here we  think of the variable $ x $ as a {\bf parameter}.
Let $\chi \in C^\infty_0(\rr)$ such that $\chi=1$ near $0$
and let 
$$L=\frac{1-\chi(\xi)}{-\xi}D_y+ \chi(\xi).$$
Then
$$L(e^{i(x-y)\cdot\xi})=e^{i(x-y)\xi},$$
and we have
\begin{equation*}
\begin{split}
Pu(x)
&= \iint\limits_{ \rr  \rr} L(e^{ i(x-y)\xi})u(y)dyd\xi
\\
& \overset{\text{integration}}{\underset{\text{by parts}}{=}} \iint\limits_{\rr  \rr} e^{i(x-y)\xi} \left(\frac{1-\chi(\xi)}{\xi}
D_y u (y)+ \chi(\xi)u(y)\right) dy d\xi \\
&= \iint\limits_{\rr  \rr} L(e^{i(x-y)\xi})\underbrace{
{\left (\frac{1-\chi(\xi)}{\xi} D_y u (y) +  \chi(\xi)u(y) \right )}}_{^t
Lu} dy d\xi\\ 
&
\overset{\text{integration}}{\underset{\text{by parts}}{=}}
\iint\limits_{\rr  \rr} e^{i(x-y)\xi} \underbrace{
\left [\frac{(1-\chi(\xi))^2}{\xi^2}D^2_y u + 2
\frac{\chi(\xi)(1-\chi(\xi))}{\xi} D_y u + \chi^2(\xi) u (y) \right
]}_{(^tL)^2u}
d\xi .
\end{split}
\end{equation*}
The last integral is absolutely convergent and it is equal with the original iterated
integral. Thus we can define 
\begin{equation}
	\begin{split}
		Pu(x) = \frac{1}{2\pi} \iint\limits_{\rr  \rr} e^{i(x-y)\xi} ({}^tL)^2
u(y)dyd\xi \quad (=u(x)).  
		\label{3}
	\end{split}
\end{equation}
\vskip0.1in
\nin
{\bf Approximation of the Symbol Method}.  Let $ \chi\in
C^\infty_0 (\rr)$ with $\chi =1$ near $0$.  Then we can write
\begin{equation}
	\begin{split}
		Pu(x)=\lim_{j\to\infty} \frac{1}{2\pi}\iint\limits_{\rr  \rr}
e^{i(x-y)} \chi (\xi/j) u (y)dyd\xi \quad (=u(x))  
		\label{4}
	\end{split}
\end{equation}
i.e. we can write the iterated integral as limit of absolutely convergent 
integrals. \medskip
\nin
Next we shall consider more general integrals of the form
$$I_\vph(pu) \doteq \iint e^{i\vph(x,\xi)} p (x,\xi) u (x) dxd\xi,u\in C^\infty_0(X),$$
where $p\in S^m(X \times \rr^N)$ and $\vph$ is a phase.
\bigskip
\begin{definition}  A phase $\vph(x,\xi)$ is a real valued function in
$C^\infty(X \times \dot \rr^N), \ \dot \rr^N =\rr^N-0$,  which is 
positively homogeneous of degree 1 in $\xi$, i.e.
$$\vph(x,t\xi)=t\vph(x,\xi),\ \ t>0$$
and
$$d_{x,\xi } \vph(x,\xi)\ne 0, \ \ x \in X,\ \xi \in \dot \rr^N.$$
\medskip
\end{definition}
{\bf Example}:  $\vph (x,\xi) =x\xi.$
\vskip0.1in
\nin
Next we will generalize the construction of the vector field $L$ used above.
\begin{lemma}  If $\vph(x,\xi)$ is a phase on $X \times \dot \rr^N$ then there 
exists a first order differential operator
\begin{equation}
	\begin{split}
		L=\sum^N_{j=1}a_j(x,\xi)\frac{\partial}{\partial \xi_j} +
\sum^n_{j=1}b_j(x,\xi)\frac{\partial}{\partial x_j}+c(x,\xi) 
		\label{5}
	\end{split}
\end{equation}
such that
$$L(e^{i\vph})=e^{i\vph}$$
and with $a_j\in S^0,b_j,c \in S^{-1}$.
\end{lemma}
\vskip0.1in
\nin
{\bf Proof.}  If $L$ is of the form \eqref{5} then we must have
$$e^{-i\vph}L(e^{i\vph})= i\sum^N_{j=1} a_j \frac{\partial \vph}{\partial \xi_j} +
i \sum^n_{j=1} b_j \frac{\partial \vph}{\partial x_j} + c(x,\xi) =1.$$
 Let $ \chi \in C^\infty_0(\rr^N)$ with $ \chi = 1 $ near $\xi =0$, and choose 
$$a_j = \frac{(1-\chi)|\xi|^2}{i \Phi} \ \frac{\partial \vph}{\partial \xi_j} \in S^0  
\ b_j = \frac{1-\chi}{i \Phi} \ \frac{\partial \vph}{\partial x_j}\in S^{-1}
 \text{ and
}  c = \chi \in S^{-\infty}$$ 
where 
$$\Phi(x,\xi) = |\xi|^2|\vph_\xi|^2+|\vph_x|^2.$$
Then $L$ is the desired operator.
\vskip0.1in
\nin
{\bf Remark}:  The transpose ${}^tL$ of $L$ is given by
$${}^tL = -\sum^N_{j=1}a_j\frac{\partial}{\partial
\xi_j}-\sum^n_{j=1}b_j\frac{\partial}{\partial x_j}+c'(x, \xi),$$
and  for any $m$
$${}^tL:S^m \longrightarrow S^{m-1}.$$
If $p \in S^m$ with $ m < -N$ then
\begin{equation}
	\begin{split}
		I_\vph(pu) \doteq \int_{\rr^N} \int_{\rr^n} e^{i\vph(x,\xi)} p(x,\xi) u (x)
dx d\xi,\ u \in C^\infty_0(X)
		\label{6}
	\end{split}
\end{equation}
is an absolutely convergent integral and for fixed $ u $ the linear form
\begin{equation}
	\begin{split}
		p \in S^m \longmapsto I_\vph(pu) \in \Bbb C, \ m < -N, \text{ is continuous on }
S^m.
		\label{7}
	\end{split}
\end{equation}
\nin
\begin{theorem}  Let $\vph (x,\xi)$ be a phase in $ X \times \dot \rr^N$.
Then:  
\vskip0.1in
\nin
{\bf 1.} The linear form \eqref{7} extends uniquely into a linear form on
$S^\infty(X \times \rr^N)$ which is continuous on $S^m$ for any given $m$.  
This extension is called an {\bf oscillatory integral}.  If $k \in \Bbb N$ with
$k>m+N$ then
\begin{equation}
	\begin{split}
		I_\vph(pu) = \int_{\rr^N} \int_{\rr^n}  e^{i\vph(x,\xi)}
({}^tL)^k [p(x,\xi)u(x)]dx d\xi
		\label{8}
	\end{split}
\end{equation}
and this integral is absolutely convergent.
\vskip0.1in
\nin
{\bf 2.}   If $ \chi \in C^\infty_0 (\rr^N)$ with $\chi =1$ near $0$ then
$$I_\vph(pu) = \lim_{j\to \infty} \int_{\rr^N} \int_{\rr^n} e^{i\vph(x,\xi)}
\chi (\xi/j) p(x,\xi) u (x) dx d\xi$$
and each integral is absolutely convergent.
\vskip0.1in
\nin
{\bf 3.}  For fixed $p\in S^m$ the functional
$$u \in C^\infty_0 (X) \longmapsto I_\vph (pu) \in \Bbb C$$
is a distribution on $X$ of order $\le k$ for $k>m+N$.
\end{theorem}
\vskip0.1in
\nin
{\bf Proof.}  If $m<-N$ then for any $k \in \Bbb N_0$ we obtain \eqref{8} by
integrating by parts.  If $m$ is arbitrary and we choose $k$ such that $k > m + N$ 
then we define $ I_\vph (pu) $ by \eqref{8}, which is an absolutely convergent integral
since $$({}^tL)^k [p(x,\xi) u (x)] \in S^{m-k}$$
with $m - k < -N$ and it is compactly supported in $x$.  Thus we obtain an 
extension  of $I_\vph(pu)$ on $S^\infty$.  For fixed $m$ this extension is
continuous in $S^m$ since for fixed $u$
$$ S^m
\xrightarrow[\text{continuous}]{(^tL)^k}
S^{m-k} \xrightarrow[\text{continuous}]{I_\vph (\bullet u)}\cc.$$
To prove (2) and the uniqueness of the extension we notice that by Proposition 
1 $$\chi (\xi/j)p(x,\xi) \longrightarrow p(x,\xi) \text{ in }
S^{m+\ve}, \ \ve > 0.$$  
Since $ \chi (\xi/j) p(x,\xi) \in S^{-\infty}$ we have
$$\int_{\rr^N } \int_{\rr^n} e^{i\vph(x,\xi)} \chi  (\xi/j )p(x,\xi) u(x)dx
d\xi = \int_{\rr^N} \int_{\rr^n}  e^{i\vph(x,\xi)}  ({}^tL)^k[ \chi (\xi/ j) p
(x,\xi) u (x)] dx d\xi .$$  Since
$$({}^tL)^k \left [\chi (\xi/ j) p(x,\xi) u (x)\right ] \longrightarrow
({}^tL)^k[p(x,\xi)u(x)] \text{ in } S^{m-k+\ve}$$
\vskip0.1in
\nin
if we choose $ k $  such that  $m-k+\ve < -N$ then
\begin{equation*}
	\begin{split}
\int_{\rr^N} \int_{\rr^n}  e^{i\vph(x,\xi)} ({}^tL)^k [ \chi (\xi/j)p(x,\xi) u
(x)] dx d\xi \xrightarrow[j\to \infty]{} &\int_{\rr^N} \int_{\rr^n}  e^{i\vph (x,\xi)}
({}^tL)^k [p (x,\xi) u (x)]dx d\xi \\
 &= I_\vph (pu).
\end{split}
\end{equation*}
Therefore (2) holds.
The uniqueness of the extension $I_\vph(\bullet u)$ from $S^{-\infty}$ to
$S^m$  follows from (2) since all extensions must agree on 
$\left \{\chi (\xi /j) p(x,\xi)\right \}$.
It remains to prove (3).  If $p$ is fixed in $S^m$ and $k>m+N$ then for any 
compact set $ K\subset X$ and $u\in C_0^\infty(K)$  we have
$$|({}^tL)^k[p(x,\xi) u (x)]|\le C_1 \sup_{\substack{ |\alpha| \le k \\ x
\in K}} |\partial^\alpha u(x)|(1+|\xi|)^{m-k},\  u \in C^\infty_0(K).$$
Since $m - k < -N$ we have
$$|I_\vph(pu)| \le \int_{x \in K} \int_{\xi \in \rr^N} C_1
\sup_{\substack{|\alpha| \le k \\ x \in K}} |\partial^\alpha u (x)|(1+|\xi|)^{m-k} dx d\xi
\le C_2 \sup_{\substack{|\alpha| \le k\\x \in K}}  |\partial^\alpha u(x)|$$
Therefore $I_\vph(p \bullet) \in  \mathcal{D'}(X)$ of order $\le k$.
\vskip0.1in
\nin
{\bf Example}:  If $\vph(x,\xi)=x \cdot \xi, n=N$ and $p=1$ then
$$
\int_{\rr^n} \int_{\rr^n} e^{ix\xi} u(x) dx d\xi = \int_{\rr^n} 
\hat u(-\xi)d\xi = (2\pi)^nu(0)
 =(2\pi)^n<\delta, u>.$$
\nin
We know that the order of $\delta$ is $0$ but the last theorem says that the
order of $\delta$ is $\le k, \ k >n$.
\medskip
\nin
Next we shall consider oscillatory integrals depending on {\bf parameters}.
\begin{proposition}  Let $X$  an open  subset of $\rr^n$, $\  Y$ an open
subset  of $\rr^\ell$, and $\vph$ be a phase function on $X \times Y \times
\dot \rr^N$ such that $$d_{x,\xi} \vph(x,y,\xi) \ne 0 \text{ on } X \times Y
\times \dot \rr^N.$$
 If for $p \in S^m(X \times Y \times \rr^N)$ and
$u(x,y) \in C^\infty_0(X \times Y)$ we let
$$F(y) \overset{\text{osc.}}{=} \iint e^{i\vph(x,y,\xi)} p(x,y,\xi)u(x,y)dxd\xi$$
with $ L = L(x,y,\xi,\partial_x, \partial_\xi),$
then $ F \in C^\infty_0 (Y) $ with 
\begin{equation}
	\begin{split}
		\partial^\gamma_y F \overset{\text{osc.}}{=} \iint
\partial^\gamma_y (e^{i\vph} pu) dx d\xi  
		\label{1a}
	\end{split}
\end{equation}
and
\begin{equation}
	\begin{split}
		\int_Y F(y)dy \overset{\text{osc. }}{=}
 \iiint e^{i\vph(x,y,\xi)}p(x,y,\xi)u(x,y)dxdyd\xi.
		\label{2a}
	\end{split}
\end{equation}
\end{proposition}
\nin
{\bf Proof of \ref{2a}.} Let  $L(x,y,\xi, \partial_x, \partial_\xi)$ constructed as
before with 
$$a_j \in S^0 (X \times Y \times \rr^N),\  b_j, c \in S^{-1}(X\times Y
\times \rr^N).$$
If $k > m+N$ then
$$F(y) = \iint e^{i\vph(x,y,\xi)} ({}^tL)^k [p(x,y,\xi) u (x,y)] dx d\xi$$
is an absolutely convergent integral.  Since $F(y)$ is bounded on supp$_y
u(x,\cdot)$ we have
\begin{equation*}
	\begin{split}
\int_Y F(y)dy &= \int_Y \left ( \iint e^{i\vph(x,y,\xi)}({}^tL)^k
[p(x,y,\xi)u(x,y)]dxd\xi \right ]dy\\ 
& \overset{\text{Fubini}}{=} \iiint e^{i\vph(x,y,\xi)}({}^tL)^k [p(x,y,\xi) u (x,y)]dx dy d\xi\\
&\overset{\text{osc.}}{=} \iiint e^{i\vph(x,y,\xi)}p(x,y,\xi) u (x,y) dx dyd\xi.
\end{split}
\end{equation*}
{\bf Proof of \ref{1a}.} Let $ \chi(\xi)$ as before and 
$$F_j(y) = \iint e^{i \vph(x,y,\xi)} \chi (\xi/ j) p (x,y,\xi) u (x,y) dx
d\xi.$$ Then
\begin{equation*}
	\begin{split}
		\partial^{\gamma}_y F_j(y) &=\iint \partial^{\gamma}_y
[e^{i\varphi(x,y,\xi)}p(x,y,\xi)u(x,y)] \chi (\xi/ j) dxd \xi \\
&=\iint e^{i\varphi (x,y, \xi)} \chi (\xi/ j) q (x,y, \xi) d x d
\xi , 
\end{split}
\end{equation*}
where
\begin{equation*}
	\begin{split}
		q(x,y, \xi) = e^{-i \varphi (x,y, \xi)} \partial^{\gamma}_y [e^{i \varphi (x,y, \xi)}
p(x,y, \xi) u(x,y)].
	\end{split}
\end{equation*}
Thus $q \in S^{m+\gamma} (X \times Y \times \rr^N).$
We have
$$\partial^\gamma_y F_j(y) \xrightarrow[\text{in } y ]{\text{unif.}} \int \int
e^{i\vph} qu \; dxd\xi \overset{\text{osc.} }{=} \int \int \partial^\gamma_y
(e^{i\vph}pu) d x d\xi, \text{ and }   F_j(y) \xrightarrow[\text{in } y ]{\text{unif.}} F(y).$$
Therefore $F \in C^\infty (Y)$ and we can pass differentiation inside the
oscillatory integral:
\begin{corollary} Let $\varphi(x, \xi)$ be a phase function such that $d_{\xi}
\varphi(x, \xi) \ne 0$ on $X \times \dot\rr^N$ and $ p \in S^\infty$. Then
$$F(x) \overset{\text{osc.}}{=} \int_{\rr^N} e^{i \varphi(x, \xi)} p(x, \xi)
d\xi \in C^\infty(X)$$
Note in this case $L = L(x, \xi, \partial_\xi).$
\end{corollary}
\begin{proposition}  Let $\varphi(x, \xi)$ be a phase function  on $X \times
\dot \rr^n$ and $p \in S^m (X \times \rr^n)$ and $T$ be the distribution
defined by 
$$T = \int_{\rr^N} e^{i \varphi(x, \xi)} p(x, \xi) d\xi.$$
Then
\vskip0.1in
\nin 
{\bf 1)} Singsupp $T \subset C \doteq \{x \in X : $ 
there exists $ \xi \ne 0 \ $ 
such that $   d_\xi \varphi(x, \xi) = 0 \}$
\vskip0.1in
\nin
{\bf 2)} Let  $C_\varphi = \{(x, \xi) \in X \times \dot  \rr^N: \ d_\xi
\ \varphi(x, \xi) = 0\}.$
\vskip0.1in
\nin
If $p$ vanishes on $C_\varphi$ then $T \in C^\infty$ (X).
\end{proposition}
\nin
{\bf Examples:}
\vskip0.1in
\nin
{\bf 1)}
\begin{equation*}
	\begin{split}
		\text{ dim } X &= \text{ dim } Y = n \\
\varphi(x,y, \xi) &= (x - y) \cdot \xi,\  d_\xi \varphi = x - y \\
C &= \text{diag} (X \times Y)
\end{split}
\end{equation*}
\nin
{\bf 2)}
\begin{equation*}
	\begin{split}
\varphi(x,y,t,\xi) &= (x - y) \cdot \xi + t|\xi|\\
d_\xi \varphi &= x - y + t \frac{\xi}{|\xi|}\\
C &= \{(x,y) \in  X \times Y: \
|x - y| = |t|\}
\end{split}
\end{equation*}
\end{document}
