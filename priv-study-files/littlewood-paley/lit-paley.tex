%
\documentclass[12pt,reqno]{amsart}
\usepackage{amssymb}
\usepackage{appendix}
\usepackage[showonlyrefs=true]{mathtools} %amsmath extension package
\usepackage{cancel}  %for cancelling terms explicity on pdf
\usepackage{yhmath}   %makes fourier transform look nicer, among other things
\usepackage{tikz}
\usepackage{framed}  %for framing remarks, theorems, etc.
\usepackage{enumerate} %to change enumerate symbols
\usepackage[margin=2.5cm]{geometry}  %page layout
\setcounter{tocdepth}{1} %must come before secnumdepth--else, pain
\setcounter{secnumdepth}{1} %number only sections, not subsections
%\usepackage[pdftex]{graphicx} %for importing pictures into latex--pdf compilation
\numberwithin{equation}{section}  %eliminate need for keeping track of counters
%\numberwithin{figure}{section}
\setlength{\parindent}{0in} %no indentation of paragraphs after section title
\renewcommand{\baselinestretch}{1.1} %increases vert spacing of text
%
\usepackage{hyperref}
\hypersetup{colorlinks=true,
linkcolor=blue,
citecolor=blue,
urlcolor=blue,
}
\usepackage[alphabetic, initials, msc-links]{amsrefs} %for the bibliography; uses cite pkg. Must be loaded after hyperref, otherwise doesn't work properly (conflicts with cref in particular)
\usepackage{cleveref} %must be last loaded package to work properly
%
%
\newcommand{\ds}{\displaystyle}
\newcommand{\ts}{\textstyle}
\newcommand{\nin}{\noindent}
\newcommand{\rr}{\mathbb{R}}
\newcommand{\nn}{\mathbb{N}}
\newcommand{\zz}{\mathbb{Z}}
\newcommand{\cc}{\mathbb{C}}
\newcommand{\ci}{\mathbb{T}}
\newcommand{\zzdot}{\dot{\zz}}
\newcommand{\wh}{\widehat}
\newcommand{\p}{\partial}
\newcommand{\ee}{\varepsilon}
\newcommand{\vp}{\varphi}
\newcommand{\wt}{\widetilde}
%
%
%
%
\newtheorem{theorem}{Theorem}[section]
\newtheorem{lemma}[theorem]{Lemma}
\newtheorem{corollary}[theorem]{Corollary}
\newtheorem{claim}[theorem]{Claim}
\newtheorem{prop}[theorem]{Proposition}
\newtheorem{proposition}[theorem]{Proposition}
\newtheorem{no}[theorem]{Notation}
\newtheorem{definition}[theorem]{Definition}
\newtheorem{remark}[theorem]{Remark}
\newtheorem{examp}{Example}[section]
\newtheorem {exercise}[theorem] {Exercise}
%
%makes proof environment bold instead of italic
\newcommand{\uol}{u^\omega_\lambda}
\newcommand{\lbar}{\bar{l}}
\renewcommand{\l}{\lambda}
\newcommand{\R}{\mathbb R}
\newcommand{\RR}{\mathcal R}
\newcommand{\al}{\alpha}
\newcommand{\ve}{q}
\newcommand{\tg}{{tan}}
\newcommand{\m}{q}
\newcommand{\N}{N}
\newcommand{\ta}{{\tilde{a}}}
\newcommand{\tb}{{\tilde{b}}}
\newcommand{\tc}{{\tilde{c}}}
\newcommand{\tS}{{\tilde S}}
\newcommand{\tP}{{\tilde P}}
\newcommand{\tu}{{\tilde{u}}}
\newcommand{\tw}{{\tilde{w}}}
\newcommand{\tA}{{\tilde{A}}}
\newcommand{\tX}{{\tilde{X}}}
\newcommand{\tphi}{{\tilde{\phi}}}
\synctex=1
\begin{document}
\title{Littlewood-Paley Decompositions}
\author{David Karapetyan}
\address{Department of Mathematics  \
  University  of Notre Dame\
    Notre Dame, IN 46556 }
    \date{\today}
    %
    \maketitle
    %
    %
    %
    %
    %
    %
    \section{The Trichotomy Formula}
Observe that
%
%
\begin{equation*}
\begin{split}
  P_{k}(fg) = \sum_{k', k'' \in \zz} P_{k}\left[ (P_{k'}f)(P_{k''}g) \right].
\end{split}
\end{equation*}
%
%
We will now show that most of the terms in this sum are zero. Notice that $P_{k'}f$ and $P_{k''}g$ are supported in phase space in the annuli
%
%
\begin{gather*}
  A_{1} = \left\{ \xi \in \rr: 2^{k'-1} \le | \xi | \le 2^{k' + 1} \right\}
  \\
  A_{2} = \left\{ \xi \in \rr: 2^{k''-1} \le | \xi | \le 2^{k'' + 1} \right\}
\end{gather*}
%
%
Since $(P_{k'}f)(P_{k''}g) = \wh{P_{k'}f} * \wh{P_{k''}g}$, it follows that 
$(P_{k'}f)(P_{k''}g)$ has support in $A_{1} + A_{2}$ in phase space.
In fact, we can write
%
%
\begin{equation*}
\begin{split}
  A_{1} + A_{2} & = \left\{ \xi \in \rr: 2^{k'-1} + 2^{k'' -1} \le 2^{k'+1} + 2^{k''+1}  \right\} \cup \left\{ | \xi_{1} - \xi_{2} |: \xi_{1} \in A_{1}^{+}, \xi_{2} \in A_{2}^{+} \right\} 
  \\
  & \doteq B_{1} \cup B_{2}
\end{split}
\end{equation*}
%
%
where
\begin{gather*}
  A_{1}^{+} = \left\{ \xi \in \rr: 2^{k'-1} \le  \xi  \le 2^{k' + 1} \right\}
  \\
  A_{2}^{+} = \left\{ \xi \in \rr: 2^{k''-1} \le  \xi  \le 2^{k'' + 1} \right\}.
\end{gather*}
Now, $P_{k} h$ has support in 
%
%
\begin{equation*}
\begin{split}
  V = \left\{ \xi \in \rr: 2^{k-1} \le | \xi |  \le 2^{k+1}\right\}
\end{split}
\end{equation*}
%
%
It follows that for $P_{k}\left[ (P_{k'}f)(P_{k''}g) \right ]$ to be nonzero, we must have 
  %
  %
  \begin{equation*}
  \begin{split}
    (B_{1} \cup B_{2}) \cap V \neq \emptyset
  \end{split}
  \end{equation*}
  %
  %
  \begin{enumerate}[a)]
  \item{}
    \label{ita}
    If $B_{1} \cap V \neq \emptyset$, then $2^{k-1} \le 2^{k'+1} + 2^{k'' + 1}$ and $2^{k'-1} + 2^{k''-1} \le 2^{k+1}$. This implies $k-2  \le k' \le k + 2$ or $k-2 \le k''\le k + 2$.
  \item{} 
    \label{itb}
    If $B_{2} \cap V \neq \emptyset$, then there exist $\xi_{1}, \xi_{2}$ such that
    %
    %
    \begin{equation*}
    \begin{split}
      2^{k-1} \le | \xi_{1} - \xi_{2} | \le 2^{k+1}
    \end{split}
    \end{equation*}
    %
    %
   where
   \begin{gather*}
     2^{k' -1} \le \xi_{1} \le 2^{k'+1}
     \\
     2^{k'' -1 } \le \xi_{2} \le 2^{k'' + 1}.
   \end{gather*}
  
 
%
\vspace{1cm}
\begin{center}
\begin{tikzpicture}[scale=1.5]
% Draw thin grid lines with color 40% gray + 60% white
% Draw x and y axis lines
  \draw [->] (0,0) node [below] {$2^{k'-1}$} -- (3,0) node [below] {$2^{k' + 1}$};
\draw [->] (6,0) node [below] {$2^{k''-1}$} -- (9,0) node [below] {$2^{k'' + 1}$};
\draw (1.5,0) node [above] {$\xi_{1}$};
\draw (7.5,0) node [above] {$\xi_{2}$};
\end{tikzpicture}
\end{center}
Hence, we must have
%
%
\begin{equation*}
\begin{split}
  | 2^{k' + 1} - 2^{k''-1} | \le 2^{k+1}.
\end{split}
\end{equation*}
%
%
which for $k' < k-2$ implies $k-2 \le k'' \le k+2$. Similarly, for $k'' < k-2$, we
obtain $k -2 \le k' \le k+2$. If $k' > k+2$ or $k'' > k+2$, then $|k' - k''| < 2$.% 
%
%
\end{enumerate}
%
If neither \eqref{ita} and \eqref{itb} hold, then $P_{k} [(P_{k' f} P_{k''g})]=0$. Therefore, 
%
%
%
%
\begin{equation*}
\begin{split}
  P_{k}(fg) 
  & = \sum_{k', k'' \in \zz} P_{k}\left[ (P_{k'}f)(P_{k''}g) \right ]
  \\
  & = P_{k} \big [ (P_{k-2 \le \cdot \le k+2}f)(P_{k-2 \le \cdot \le k+2}g) 
  \\
  & + (P_{k-2 \le \cdot \le k+2}f)(P_{< k-2}g) 
  \\
  & + (P_{< k-2}f)(P_{k-2 \le \cdot \le k+2}g) 
  \\
  & + \sum_{k', k'' > k+2 : \ |k' - k''|< 2}(P_{k'}f)(P_{k''}g) \big ]
\end{split}
\end{equation*}
%
%
which is commonly referred to as the \emph{Littlewood-Paley trichotomy}. For more details, see Klainerman's or Tao's notes.
%
%
%
%
%
    %\nocite{*}
    %\bibliography{/Users/davidkarapetyan/math/bib-files/references}
    \end{document}
