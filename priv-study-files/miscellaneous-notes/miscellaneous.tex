%
\documentclass[12pt,reqno]{amsart}
\usepackage{amssymb}
\usepackage{appendix}
\usepackage[showonlyrefs=true]{mathtools} %amsmath extension package
\usepackage{cancel}  %for cancelling terms explicity on pdf
\usepackage{yhmath}   %makes fourier transform look nicer, among other things
\usepackage{framed}  %for framing remarks, theorems, etc.
\usepackage{enumerate} %to change enumerate symbols
\usepackage[margin=2.5cm]{geometry}  %page layout
\setcounter{tocdepth}{1} %must come before secnumdepth--else, pain
\setcounter{secnumdepth}{1} %number only sections, not subsections
%\usepackage[pdftex]{graphicx} %for importing pictures into latex--pdf compilation
\numberwithin{equation}{section}  %eliminate need for keeping track of counters
%\numberwithin{figure}{section}
\setlength{\parindent}{0in} %no indentation of paragraphs after section title
\renewcommand{\baselinestretch}{1.1} %increases vert spacing of text
%
\usepackage{hyperref}
\hypersetup{colorlinks=true,
linkcolor=blue,
citecolor=blue,
urlcolor=blue,
}
\usepackage[alphabetic, initials, msc-links]{amsrefs} %for the bibliography; uses cite pkg. Must be loaded after hyperref, otherwise doesn't work properly (conflicts with cref in particular)
\usepackage{cleveref} %must be last loaded package to work properly
%
%
\newcommand{\ds}{\displaystyle}
\newcommand{\ts}{\textstyle}
\newcommand{\nin}{\noindent}
\newcommand{\rr}{\mathbb{R}}
\newcommand{\nn}{\mathbb{N}}
\newcommand{\zz}{\mathbb{Z}}
\newcommand{\cc}{\mathbb{C}}
\newcommand{\ci}{\mathbb{T}}
\newcommand{\zzdot}{\dot{\zz}}
\newcommand{\wh}{\widehat}
\newcommand{\p}{\partial}
\newcommand{\ee}{\varepsilon}
\newcommand{\vp}{\varphi}
\newcommand{\wt}{\widetilde}
%
%
%
%
\newtheorem{theorem}{Theorem}[section]
\newtheorem{lemma}[theorem]{Lemma}
\newtheorem{corollary}[theorem]{Corollary}
\newtheorem{claim}[theorem]{Claim}
\newtheorem{prop}[theorem]{Proposition}
\newtheorem{proposition}[theorem]{Proposition}
\newtheorem{no}[theorem]{Notation}
\newtheorem{definition}[theorem]{Definition}
\newtheorem{remark}[theorem]{Remark}
\newtheorem{examp}{Example}[section]
\newtheorem {exercise}[theorem] {Exercise}
%
\makeatletter \renewenvironment{proof}[1][\proofname] {\par\pushQED{\qed}\normalfont\topsep6\p@\@plus6\p@\relax\trivlist\item[\hskip\labelsep\bfseries#1\@addpunct{.}]\ignorespaces}{\popQED\endtrivlist\@endpefalse} \makeatother%
%makes proof environment bold instead of italic
\newcommand{\uol}{u^\omega_\lambda}
\newcommand{\lbar}{\bar{l}}
\renewcommand{\l}{\lambda}
\newcommand{\R}{\mathbb R}
\newcommand{\RR}{\mathcal R}
\newcommand{\al}{\alpha}
\newcommand{\ve}{q}
\newcommand{\tg}{{tan}}
\newcommand{\m}{q}
\newcommand{\N}{N}
\newcommand{\ta}{{\tilde{a}}}
\newcommand{\tb}{{\tilde{b}}}
\newcommand{\tc}{{\tilde{c}}}
\newcommand{\tS}{{\tilde S}}
\newcommand{\tP}{{\tilde P}}
\newcommand{\tu}{{\tilde{u}}}
\newcommand{\tw}{{\tilde{w}}}
\newcommand{\tA}{{\tilde{A}}}
\newcommand{\tX}{{\tilde{X}}}
\newcommand{\tphi}{{\tilde{\phi}}}
\synctex=1
\begin{document}
\title{Private Miscellaneous Notes}
\author{David Karapetyan}
\address{Department of Mathematics  \\
    University  of Notre Dame\\
        Notre Dame, IN 46556 }
        \date{\today}
        %
        \maketitle
        %
        %
        %
        %
        %
        %
        \section{}
        \label{sec:ll}
        
\begin{lemma}
Let $X$ be Banach, and $L(X)$ be the space of continuous linear functionals on
$X$, equipped with the operator norm. Suppose 
%
%
\begin{gather*}
  x_{k} \xrightarrow{X} x,
  \\
  T_{k} \xrightarrow{L(X)} T.
\end{gather*}
%
%
Then
\begin{gather*}
  T_{k} x_{k} \to Tx.
\end{gather*}
%
\label{lem:diag}
\end{lemma}
%
%
\begin{proof}[Proof of Lemma \ref{lem:diag}]
  Since $x_{k} \xrightarrow{X} x$, there exists $N > 0$ such that for $k > N$,
  we have $\|x -x_{k} \|_{X} \le 1$. Furthermore, if we denote $B_{X}(R) \doteq
  \left\{ x \in X: \| x \|_{X} \le 1 \right\}$, then $T_{k}
  \xrightarrow{L(X)} T$ is by definition equivalent to 
  %
  %
  \begin{equation*}
  \begin{split}
    \sup_{B_{X}(1)}  | (T - T_{k})(x) | \to 0.
  \end{split}
  \end{equation*}
  %
  %
  Hence, for $k > N$, 
  %
  %
  \begin{equation*}
  \begin{split}
    | Tx - T_{k}x_{k} |
    & = | Tx - T_{k}x + T_{k}x - T_{k}x_{k} |
    \\
    & \le | (T - T_{k})x | + | T_{k}(x -x_{k}) |
    \\
    & \le | (T - T_{k})x |
    + \| x - x_{k}\|_{X} \sup_{y \in B_{x}(1)} | T_{k}y |
    \\
    & \to 0
  \end{split}
  \end{equation*}
  %
  %
  which completes the proof.
\end{proof}



        %\nocite{*}
        %\bibliography{/Users/davidkarapetyan/Documents/math/}

        \end{document}
