\chapter{Point-Set Topology}
\section{Topological Spaces}
\begin{definition}
	\label{def:top}
	A \emph{topology} $\mathcal{T}$ on $X$ is a collection of subsets of $X$,
	including $X$ and $\emptyset$ that is closed under unions and finite
	intersections.

	We call $X$ equipped with a topology $\mathcal{T}$ a \emph{topological space},
	denoted by $(X, \mathcal{T})$. If $U \in \mathcal{T}$, we say $U$ is an \emph{open} set.

	Let $\mathcal{T}, \mathcal{T'}$ be topologies on $X$. If $\mathcal{T} \subset
	\mathcal{T'}$, then we say $\mathcal{T}$ and $\mathcal{T'}$ are
	\emph{comparable topologies}, and that $\mathcal{T'}$ is a \emph{finer} (or
	\emph{weaker}) topology and
	$\mathcal{T}$ is a \emph{coarser} (or \emph{stronger}) topology. 

\end{definition}
\begin{examples} $ $
	\begin{enumerate}	
		\item $X = \left\{ a,b,c \right\}, \ \mathcal{T} = \left\{
				\left\{
				a,b,c \right\}, \left\{
			b \right\}, \left\{ b,c \right\}\right\}$.
		\item $\left( X, \left\{ X, \emptyset \right\} \right)$. This is
			known as the 
			\emph{trivial topology}.
		\item $\left( X, \mathrm{P}(X) \right)$, known as the \emph{discrete
			topology}.
	\end{enumerate}
\end{examples}
\section{Basis for a Topology}
\begin{definition}
	\label{def:basis}
	A basis for a topology $\mathcal{T}$ is a collection $\mathcal{B}$ of subsets of 
	$\mathcal{T}$ such that
	\begin{enumerate}
		\item For each $x \in X$, there exists a basis element $B$ containing
			$x$. 
		\item If $x \in B_{1} \cap B_{2}$, then there exists a basis element
			$B_{3}$ such that $x \in B_{3}$ and $B_{3} \subset B_{1} \cap
			B_{2}$.
	\end{enumerate}
\end{definition}
Intuitively, a basis is a subset of our topology that is closed under
intersections, but not necessarily unions. However, the set of all possible
unions of members of our basis generates our topology!
Hence, a basis can be thought of as the ``skeleton'' of a topology.
We now make this rigorous.
\begin{theorem}
	\label{thm:basis-gen-topology}
	Let $\mathcal{B}$ be a basis for a set $X$. Then the collection $\mathcal{T}$ generated
	by collecting all possible unions of elements of $\mathcal{B}$ is a topology
	on $X$
\end{theorem}
\begin{proof}
	Proving closure under unions follows by construction of $\mathcal{T}$.

\end{proof}
<++>
\begin{example}
	If $X$ is any set, the collection of all one-point subsets of $X$
	is a basis for the discrete topology on $X$.
\end{example}

