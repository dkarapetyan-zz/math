\chapter{Point-Set Topology}
\section{Topological Spaces}
\begin{definition}
\label{def:top}
A \emph{topology} $\mathcal{T}$ on $X$ is a collection of subsets of $X$,
including $X$ and $\emptyset$ that is closed under unions and finite
intersections.

We call $X$ equipped with a topology $\mathcal{T}$ a \emph{topological space},
denoted by $(X, \mathcal{T})$. If $U \in \mathcal{T}$, we say $U$ is an
\emph{open} set.

Let $\mathcal{T}, \mathcal{T'}$ be topologies on $X$. If $\mathcal{T} \subset
\mathcal{T'}$, then we say $\mathcal{T}$ and $\mathcal{T'}$ are \emph{comparable
topologies}, and that $\mathcal{T'}$ is a \emph{finer} (or \emph{weaker})
topology and $\mathcal{T}$ is a \emph{coarser} (or \emph{stronger}) topology.
\end{definition}
\begin{example} $ $
\begin{enumerate}
  \item $X = \left\{ a,b,c \right\}, \ \mathcal{T} = \left\{
  \left\{
  a,b,c \right\}, \left\{
  b \right\}, \left\{ b,c \right\}\right\}$.
  \item $\left( X, \left\{ X, \emptyset \right\} \right)$. This is
  known as the
  \emph{trivial topology}.
  \item $\left( X, \mathrm{P}(X) \right)$, known as the \emph{discrete
  topology}.
\end{enumerate}
\end{example}
\section{Basis for a Topology}
\begin{definition}
\label{def:basis}
A basis for a topology $\mathcal{T}$ is a collection $\mathcal{B}$ of subsets of
$\mathcal{T}$ such that
\begin{enumerate}
  \item $\mathcal{B}$ is a cover for $X$. That is, $\bigcup_{B \in
  \mathcal{B}} B = X$.
  \item If $x \in B_{1} \cap B_{2}$, then there exists a basis element
  $B_{3}$ such that $x \in B_{3}$ and $B_{3} \subset B_{1} \cap
  B_{2}$.
\end{enumerate}
\end{definition}
\begin{example}
If $X$ is any set, the collection of all one-point subsets of $X$
is a basis for the discrete topology on $X$.
\end{example}
Intuitively, a basis is a subset of our topology that is closed under
intersections, but not necessarily unions. However, the set of all possible
unions of members of our basis generates our topology!
Hence, a basis can be thought of as the ``skeleton'' of a topology.
We now make this rigorous.
\begin{theorem}
\label{thm:basis-gen-topology}
Let $\mathcal{B}$ be a basis for a set $X$. Then the collection $\mathcal{T}$
generated by collecting all possible unions of elements of $\mathcal{B}$ is
a topology on $X$
\end{theorem}
\begin{proof}
Closure under unions follows by construction of $\mathcal{T}$. Suppose
$U_{1}, U_{2} \in \mathcal{T}$. To complete the proof, it will be enough to
show that $U_{1} \cap U_{2} \in \mathcal{T}$; closure under arbitrary finite
intersections will then follow by induction. Suppose $x \in U_{1} \cap
U_{2}$. By construction of $\mathcal{T}$, there exist $B_{1} \subset U_{1}$
and $B_{2} \subset U_{2}$ containing $x$ (i.e.\ unions of basis elements
generate $U_{1}$ and $U_{2}$). Since $\mathcal{B}$ is a basis, there exists
$B_{3} \subset B_{1} \cap B_{2}$ containing $x$, and so $U_{1} \cap U_{2}
\subset B_{3}$. However, since $B_{3} \subset U_{1} \cap U_{2}$, and $B_{1}
\subset U_{1}$ and $B_{2} \subset U_{2}$, we must also have $B_{3} \subset
U_{1} \cap U_{2}$, from which we conclude that $B_{3} = U_{1} \cap U_{2}$.
Since $\mathcal{T}$ is generated via unions of elements in $\mathcal{B}$, we
conclude that $U_{1} \cap U_{2} \in \mathcal{T}$.
\end{proof}
We have seen how to generate a topology from a basis. We now examine how to
generate a basis from a topology.
\begin{theorem}
\label{thm:basis-from-top}
Let $(X, \mathcal{T})$ be a topological space. Suppose $\mathcal{C}$ is a collection of
open sets of $X$ such that for each open set $U \in X$ and each
$x \in U$, there is an element $C_{x} \in \mathcal{C}$ such
that $x \in C_{x}
\subset U$. Then $\mathcal{C}$ is a basis for the topology on $X$.
\end{theorem}
\begin{proof}
First, observe that for a given open $U$, we have
$\cup_{x \in U}C_{x} = U$. Hence, unions of elements of $\mathcal{C}$
generates $\mathcal{T}$.

It remains to show that $\mathcal{C}$ is a basis.
Since our basis elements are open, by definition their finite intersections
are open, and so by definition we have that for $C_{x}^{(1)}$ and
$C_{x}^{(2)}$ containing $x$, there exists $C_{x}^{(3)} \subset
C_{x}^{(1)} \cap C_{x}^{(2)}$ containing $x$.
\end{proof}
\begin{lemma}
\label{lem:finer-coarser-basis}
Let $\mathcal{B}, \mathcal{B'}$ be bases for the the respective topologies
$\mathcal{T}, \mathcal{T'}$ on $X$. Then the following are equivalent:
\begin{enumerate}
  \item\label{it:finer} $\mathcal{T'}$ is finer than $\mathcal{T}$.
  \item\label{it:basis-subset} For each $x \in X$ and each basis element $B_{x} \in \mathcal{B}$
  containing $x$, there is a basis element $B' \in \mathcal{B'}$ such that
  $x \in B' \subset B$.
\end{enumerate}
\end{lemma}
\begin{proof}
Assume \cref{it:finer}. Then $\mathcal{T} \subset \mathcal{T'}$.
Since unions over basis elements generate topologies,
it follows that:
\begin{enumerate}
  \item There exist basis elements $B_{x}' \in \mathcal{B'}$ and
  $B_{x} \in \mathcal{B}$ containing $x$.
  \item $B_{x}' \cap B_{x}$ is open in $\mathcal{T'}$.
\end{enumerate}
Hence there exists some basis
element $\tilde{B}_{x}' \subset B_{x}' \cap B_{x}$ in $\mathcal{B'}$, and so
$\tilde{B}_{x}' \subset B_{x}$.

To prove the reverse direction, observe that for fixed $A \in \mathcal{T}$ there
exists a collection $\left\{ B_{x} \right\}_{x \in A}$ of basis elements in
$\mathcal{B}$ such that $A = \cup_{x \in A} B_{x}$.
Assuming \cref{it:basis-subset}, for each $B_{x}$, there exists $B_{x}'$ in
$\mathcal{B'}$ such that $B_{x}' \subset B_{x}$. Collecting these $B_{x}'$, it
follows that $A = \cup_{x \in A} B_{x}'$, and hence
$\mathcal{T} \subset \mathcal{T'}$.
\end{proof}
\begin{example}
Let $\mathcal{B}$ be the collection of all open intervals on the real line.
Then this is a basis for the topology typically used on the real line,
which we call the \emph{standard topology} on the real line.
\end{example}
\begin{definition}
\label{def:sub-basis}
A \emph{subbasis} $\mathcal{S}$ for a topology on $X$ is a collection of subsets of
$X$ whose union is $X$. The topology generated by $S$ is defined to be
the collection $\mathcal{T}$ of all unions and finite intersections of members
of $\mathcal{S}$.
\end{definition}
It follows by construction that if $\mathcal{S}$ generates a topology
$\mathcal{T}$, then the collection $\mathcal{B}$ of finite
intersections of elements in $\mathcal{S}$ is a basis for $\mathcal{T}$.
\section{Important Topologies}
\subsection{The Order Topology}
\begin{definition}
\label{def:order_top}
Let $X$ be a set with more than one element, and equipped with an order
relation $<$. Then we define the \emph{order topology} $\mathcal{T}$ as the topology
generated by the following basis $\mathcal{B}$, consisting of:
\begin{enumerate}
  \item\label{it:open-intervals} All intervals $(a,b) \doteq \left\{ x \in X:
  a < x < b
  \right\}$ in $X$.
  \item\label{it:ray1} All intervals of the form $[a_{0}, b) \doteq \left\{
  x \in X: a_{0} \le x < b \right\}$, where $a_{0}$
  is the smallest element (if any) of $X$.
  \item\label{it:ray2} All intervals of form $(a, b_{0}] \doteq \left\{
  x \in X: a < x \le b_{0} \right\}$,
  where $b_{0}$ is the largest
  element (if any) of $X$.
\end{enumerate}
\end{definition}
\begin{remark}
\label{rem:rr-top}
If $X = \mathbf{R^{n}}$, $\mathcal{B}$ consisting only of \cref{it:open-intervals}
generates $\mathcal{T}$. This is \emph{not} true If $X =
\mathbf{R} \cup
\{-\infty, \infty\}$.
\end{remark}
\begin{example}
$ $
\begin{enumerate}
  \item Consider $\mathbf{R^{2}}$ equipped with dictionary
  order. Then its order topology is generated by the open intervals
  $\left( (a,b), (c,d) \right) = \left\{ (x,y): a < x < c \ \text{and} \ b <
  y < d\right\}$
  \item The order topology on $\mathbf{Z}_{+}$ coincides with the discrete topology.
\end{enumerate}
\end{example}
\begin{definition}
\label{def:rays-subbasis}
If $X$ is an ordered set, then given $a \in X$, the sets
\begin{equation*}
(a, \infty), (-\infty, a), [a, \infty) \ \text{and} \  (-\infty, a]
\end{equation*}
are called \emph{rays}. Observe that that finite intersections of the
rays generate the basis for the order topology on $X$. Hence, the rays form a subbasis for the order topology on $X$.
\end{definition}
\subsection{The Finite Box Topology}
\begin{definition}
\label{def:box_topology}
Let $(X, \mathcal{T}_{X}$ and $(Y, \mathcal{T}_{Y})$ be topological spaces.
Then the \emph{box topology} on $X \times Y$ is defined to be the
topology having as basis the collection of all sets of form
\(U \times V \doteq \left\{ (x,y): x \in U \subset
\mathcal{T}_{X}, y \in V \subset \mathcal{T}_{Y}  \right\}\).
\end{definition}
However, we can find a subset of this basis that is also a basis for the box
topology:
\begin{theorem}
\label{thm:box-basis}
If $\mathcal{B}$ is a basis for $\mathcal{T}_{X}$, and
$\mathcal{C}$ is a basis for $\mathcal{T}_{Y}$, then
\begin{equation*}
\mathcal{D} = \left\{ B \times C: B \in \mathcal{B} \ \text{and} \ C \in
\mathcal{C} \right\}
\end{equation*}
is a basis for the box topology on $X \times Y$.
\end{theorem}
\begin{proof}
Let $U \in \mathcal{T}_{X}$ and $V \in \mathcal{T}_{Y}$.
Then $U, V$ are generated by unions and finite intersections via
elements of bases $\mathcal{B}$ and $\mathcal{C}$, respectively.
Hence, we can write
$U = \{\cup_{i=1}^{\infty} B_{i}\} \cap_{i=1}^{k} \hat{B}_{i}$
and $V = \{\cup_{i=1}^{\infty} C_{i}\} \cap_{i=1}^{k} \hat{C}_{i}$.

It follows from D'Morgan's laws that
\begin{align*}
U \times V & = \left\{ (x,y): x \in \{\cup_{i=1}^{\infty} B_{i}\}
\cap_{i=1}^{k} \hat{B}_{i}, y \in \{\cup_{i=1}^{\infty} C_{i}\} \cap_{i=1}^{k}
\hat{C}_{i} \right\} \\
& = \bigcup_{i=1}^{\infty} \left\{ (x,y): x \in B_{i} \cap_{i=1}^{k}
\hat{B}_{i}, y \in C_{i} \cap_{i=1}^{k} \hat{C}_{i} \right\}\\
& = \bigcup_{i=1}^{\infty} \left\{ (x,y): x \in B_{i}, y \in C_{i} \right \}
\cap \bigcup_{i=1}^{k} \left \{ (x,y): x \in \hat{B}_{i}, y \in \hat{C}_{i}
\right\}
\end{align*}
Observe that we have generated $U \times V$ using unions and finite
intersections of members of $\mathcal{D}$. It follows that the topology
generated by $\mathcal{D}$ is finer than the box topology.
However, it is also coarser, since any member of $\mathcal{D}$ lives in the
box topology. We conclude that the box topology and the topology
generated by $\mathcal{D}$ are identical.
\end{proof}
We now work towards constructing a subbasis for the box topology.
\begin{definition}
\label{def:projection}
For sets $X_{1},X_{2}, \cdots, X_{n}$, the \emph{projection of
$X_{1} \times \cdots \times X_{n}$ onto $X_{i}$} is given by
\begin{align*}
& \pi_{i}: X_{1} \times \cdots \times X_{n} \to X_{i} \\
& \pi_{i}(x_{1}, \cdots, x_{n}) = x_{i}.
\end{align*}
\end{definition}
Observe that if $U$ is open in $X_{i}$, then $\pi_{i}(U) = X_{1} \times \cdots
\times X_{i-1} \times U \times X_{i+1} \times \cdots \times X_{n}$, which is
also open in the box topology. In fact, the projection maps are the
simplest example of \emph{continuous maps}, an idea which we shall return to
later. Intuitively, the projection maps ``project'' a topological space onto its
$X_{1}$ to $X_{n}$ axes. It seems natural then that there should be a way to
reconstruct the space from the projections and the inverse of the projection
maps. In fact, we have the following:
\begin{theorem}
\label{thm:proj-subbasis}
The collection
\begin{equation*}
S = \bigcup_{i=1}^{n} \left\{ \pi_{i}^{-1}(B): B \in \mathcal{B}_{X_{i}} \right\}
\end{equation*}
is a subbasis for the box topology on $\prod_{i=1}^{n} X_{i}$.
\end{theorem}
\begin{proof}
Recall that basis elements are open. Fix $1 \le j \le n$, and observe that
\begin{align*}
\prod_{i=1}^{n} X_{i}
&= \bigcup_{B \in
\mathcal{B}_{X_{j}}}X_{1} \times X_{2} \times \cdots \times B \times
\cdots \times X_{k} \\
&= \bigcup_{B \in \mathcal{B}_{X_{j}}} \pi_{j}^{-1}(B)
\end{align*}
Since $j$ was chosen arbitrarily, it follows that
\begin{align*}
\prod_{i=1}^{n} X_{i} = \bigcup_{i=1}^{n} \bigcup_{B \in
\mathcal{B}_{X_{i}}} \pi_{i}^{-1}(B).
\end{align*}
Therefore, $S$ is a cover for $\prod_{i=1}^{n} X_{i}$, and hence a subbasis
for some topology on it.
Lastly, we have
\begin{align*}
B_{1} \times \cdots \times B_{n} = \bigcap_{i=1}^{n} \pi_{i}^{-1}(B_{i}).
\end{align*}
Therefore, by \cref{thm:box-basis}, the basis for the box topology is
a subset of the collection of finite intersections of $S$, and so the topology
generated by $S$ is a superset of the box topology. However, every member
of $S$ belongs to the box topology, and the topology generated by $S$ is
also a subset of the box topology.

We conclude that $S$ is a subbasis for the box topology.
\end{proof}
\begin{remark}
\label{rem:can-sub-open-sets}
It follows from the structure of the proof that
\begin{equation*}
S' = \bigcup_{i=1}^{n} \left\{ \pi^{-1}(U): U \in \mathcal{T}_{X_{i}}  \right\}
\end{equation*}
is also a subbasis for the box topology on $\prod_{i=1}^{n} X_{i}$, which
can be useful when constructing a subbasis for a box topology where
one wishes to bypass finding the bases of the individual components.
However, since $S \subset S'$, the result of the theorem is stronger.
\end{remark}
\subsection{The Finite Product Toplogy}
\begin{definition}
\label{def:infinite-cartesian}
Let \(J\) be an infinite indexing set, and \(\left\{ X_{\alpha}
\right\}_{\alpha \in J}\) a collection of topological space. Then we
define \(\prod_{\alpha \in J} X_{\alpha}\) to be the space of
functions \(\hat{x}: J \to \cup_{\alpha \in J} X_{\alpha}\), where
\(\hat{x}(\alpha) = x_{\alpha} \in X_{\alpha}\). We denote
\(\hat{x}\) itself by the symbol \( (x_{\alpha})_{\alpha \in J} \); that is,
we think of \(\hat{x}\) as an infinitely large ``coordinate''
in \(\prod_{\alpha \in J} X_{\alpha}\) space.
\end{definition}
\begin{definition}
\label{def:product-toplogy}
Let \(J\) be an indexing set, possibly uncountable, and
\(\left\{ X_{\alpha} \right\}_{\alpha \in J}\) a
collection of topological spaces. Then the \emph{product topology}
on \(\prod_{\alpha \in J} X_{\alpha}\) is defined to be the topology generated
by the subbasis \(S = \left\{ \pi_{\alpha}^{-1}(U): U \in X_{\alpha}
\right\}_{\alpha \in J}\).
\end{definition}
\begin{remark}
\label{rem:coincide-box-toplogy}
Observe that the subbasis for the box topology and the product topology
coincides for finite \(J\), and so the two topologies are equivalent for
finite cartesian products. In general, the product topology is coarser than
the box topology. Henceforth, we shall exclusively put the product topology on
cartesian products. The reasoning for this will become evident later.
\end{remark}
\subsection{The Infinite Product and Box Topologies}
\begin{theorem}
\label{thm:prod-topo-inf-basis}
Let \(\left\{ X_{\alpha} \right\}_{\alpha}\) be a collection of sets, where
\(J\) is an indexing set. The
basis for the product topology on \(\prod_{\alpha \in J} X_{\alpha}\)
consists of sets of the form \(\prod U_{\alpha}\), where \(U_{\alpha}\)
is open in \(X_{\alpha}\) for each \(\alpha\) and \(U_{\alpha} =
X_{\alpha}\) except for finitely many values of \(\alpha\).
\end{theorem}
\begin{proof}
Let \(\mathcal{B}\) be the basis for the product topology generated by the
subbasis for the product topology. Then for given \(B \in \mathcal{B}\), there
exists an index \(I \subset J\) and  \(\{U_{\alpha_{i}}\}_{i \in I} \subset
\mathcal{B}\) such that
\(B = \bigcap_{i \in I}\pi_{\alpha_{i}}^{-1}(U_{\beta_{i}})\). Then, \(x = (x_{\alpha}) \in B\)
if and only if its \(\beta_{\text{ith}}\) coordinate is in \(U_{\beta_{i}}\), for
\(i \in I\). Since the remaining coordinates are not restricted, they
can live in \emph{all} of \(X\). Hence,
\(B = \prod_{\alpha \in J} U_{\alpha}\), where \(U_{\alpha} = X_{\alpha}\)
if \(\alpha \neq \beta_{1}, \cdots, \beta_{n}\).
\end{proof}
\begin{remark}
\label{rem:prod-top-coarser-box}
In particular, it is clear from the proof above that, while the product
topology and the box topology coincide for finite cartesian products, for
infinite cartesian products the product topology is coarser than the box
topology.
\end{remark}
\subsection{The Subspace Topology}
\begin{definition}
\label{def:subspace-top}
Let $(X, \mathcal{T_{X}})$ be a topological space. Then $Y \subset X$,
equipped with the topology $\mathcal{T}_{Y} = \left\{ Y \cap U: U \in
\mathcal{T}_{X} \right\}$ is called a \emph{subspace} of $X$.
\end{definition}
\begin{lemma}
\label{lem:basis-subspace}
If $\mathcal{B}_{X}$ is a basis for $X$, and $Y \subset X$, then
\begin{equation*}
\mathcal{B}_{Y} = \left\{ B \cap Y: B \in \mathcal{B} \right\}
\end{equation*}
is a basis for the subspace topology on $Y$.
\end{lemma}
\begin{proof}
Every element of $\mathcal{B}_{Y}$ is in $\mathcal{T}_{Y}$, and so
the topology generated by $\mathcal{B}_{Y}$ is a subset of
$\mathcal{T}_{Y}$. Furthermore, every element of $\mathcal{T}_{Y}$
is generated via unions of members of $\mathcal{B}_{Y}$, and so
the topology generated by $\mathcal{B}_{Y}$ is a superset of
$\mathcal{T}_{Y}$.

It follows that these two topologies coincide. We now show that $\mathcal{B}$
is a basis. First, observe that since $\mathcal{B}$ is a
basis for $X$, there exist members in $\mathcal{B}$ whose union is $X$. Since
$X \cap Y = Y$, it follows that $\mathcal{B}_{Y}$ covers $Y$. Secondly,
for given $B_{1}, B_{2} \in \mathcal{B}$, there exists $B_{3} \in \mathcal{B}$
such that $B_{3} \subset B_{1} \cap B_{2}$. It follows that
$\left( B_{3} \cap Y  \right)\subset (B_{1} \cap Y) \cap (B_{2} \cap Y)$.
\end{proof}
It is not always the case that an element of the subspace topology on $Y \subset
X$ is open in $X$. However, we have the following:
\begin{lemma}[Transitivity of Openness]
\label{lem:open-criteria-subspace}
Let $Y$ be a subspace of $X$. If $U$ is open in $Y$, and $Y$ is open in
$X$, then $U$ is open in $X$.
\end{lemma}
\begin{proof}
Assume $U$ is open in $Y$. Then $U = Y \cap V$ for some $V \in
\mathcal{T}_{X}$. Also assume $Y$ is open in $X$. Then $Y \in \mathcal{T}_{X}$.
It follows that $U$ is formed via the intersection of two open sets in
$X$. It follows that $U$ is open in $X$.
\end{proof}
\begin{remark}
Let $A$ and $B$ be subspaces of topological spaces $X$ and $Y$, respectively.
Observe that the product topology on $X \times Y$, restricted to sets consisting only
of members in $A \times B$, coincides with the subspace topology
on $A \times B$ induced from $X \times Y$.
\end{remark}
\begin{remark}
Let $X$ be a set equipped with an order topology, and $Y \subset X$.
Then the order topology on $Y$ need not be the same as the subspace topology on $Y$.
\par
For example, let $Y = (0,1) \cup \left\{ 2 \right\}$, and $X = \mathbf{R}$. The set $\left\{ 2 \right\} = (3/2, 5/2) \cap Y$ is open in the subspace
topology. However, it is \emph{not} open in the order topology, since any
interval in $X$ containing $\left\{2\right\}$ is not open in the induced order
topology on $Y$.
\par
In general, the order topology induced on a set $Y \subset X$ is coarser than the
subspace topology on $Y$. To see this, note that the induced order topology is
generated by the restriction of elements in the basis of the order
topology on $X$ to elements of $Y$. That is,
$\mathcal{B}_{Y} = \mathcal{B}| Y$. Hence, if $U \in \mathcal{B}_{Y}$, then
$U = V | Y$ for some $V \in \mathcal{B}$. But, we can also write $U = V \cap Y$,
and so $U$ lives in the subspace topology as well.
\par
If $Y$ is a \emph{convex} subset of $X$, then the order topology and the subspace
topology do coincide. Recall that a subset $Y \subset X$ with induced order
topology from $X$ is \emph{convex} if for any points $a,b$ in $Y$ with $a < b$,
the interval $(a,b)$ is in $Y$.
\end{remark}
\section{Closed Sets and Limit Points}
\subsection{Closed Sets}
\begin{definition}
\label{def:closed-sets}
We say that a subset $A$ of a topological space $X$ is \emph{closed} if
$A^{c} \doteq X \setminus A$ is open.
\end{definition}
\begin{example} $ $
\begin{enumerate}
  \item $[a,b] \subset \mathbf{R}$ is closed, since
  $[a,b]^{\complement} = (-\infty, a) \cup (b, \infty)$ is open.
  \item In the discrete topology on $X$, every set is open, which implies
  that every set is closed as well.
\end{enumerate}
\end{example}
\begin{lemma}
\label{lem:equiv-closed-def-top}
Let $X$ be a topological space. Then
\begin{enumerate}
  \item $\emptyset$ and $X$ are closed.
  \item Arbitrary intersections of closed sets are closed.
  \item Finite unions of closed sets are closed.
\end{enumerate}
\end{lemma}
\begin{proof}
Applying DeMorgan's laws to an arbitrary closed collection
$\left\{ A_{\alpha} \right\}_{\alpha \in J}$, we obtain
\begin{align*}
\left[ \bigcap_{\alpha \in J}A_{\alpha} \right]^{\complement} & = \bigcup_{\alpha \in
J} A_{\alpha}^{\complement} \\
\left[ \bigcup_{\alpha \in J}A_{\alpha} \right]^{\complement} & = \bigcap_{\alpha \in
J} A_{\alpha}^{\complement}
\end{align*}
Since $A_{\alpha}^{\complement}$ is open, and topologies are closed under
unions and finite intersections, the result follows.
\end{proof}
\begin{remark}
\label{rem:equiv-closed-def-top}
It follows that we could have defined topological spaces in terms of closed
sets, and defined open sets as their complement.
\end{remark}
\begin{definition}
\label{def:closed-set-subspace}
Let $Y$ be a subspace of a toplogical space $X$. Then we say $A$
is \emph{closed} in $Y$ if $A$ is closed in the subspace topology of
$Y$.
\end{definition}
\begin{lemma}
\label{lem:closed-iff}
Let $Y$ be a subspace of $X$. Then $A$ is closed in $Y$ if and only if
it is the intersection of a closed set of $X$ with $Y$.
\end{lemma}
\begin{proof}
Suppose $A$ is closed in $Y$. Then $A = (Y \cap U)^{\complement} = Y \setminus
(Y \cap U)$ for some set $U$ that is open in $X$.
But $Y \setminus (Y \cap U) = Y \cap (Y \cap U)^{\complement} = Y \cap
U^{\complement}$.
\end{proof}
\begin{lemma}[Transitivity of Closures]
Let $Y$ be a subspace of $X$. If $A$ is closed in $Y$ and $Y$ is closed in
$X$, then $A$ is closed in $X$.
\end{lemma}
\begin{proof}
It is analagous to the proof of \cref{lem:open-criteria-subspace}.
\end{proof}
\begin{example}
$[1,2]$ is closed in $[0,3]$. But
$[0,3]$ is closed in $\mathbf{R}$. Hence,
$[1,t]$ is closed in $\mathbf{R}$.
\end{example}
\begin{definition}
\label{def:interior}
Let $A$ be a subset of a topological space $X$. The \emph{interior}
$A^{\circ}$ of
$A$ is defined to be the union of all open sets contained in $A$.
The \emph{closure} $\bar{A}$ of $A$ is the intersection of all closed
sets containing $A$.
\end{definition}
\begin{remark}
\label{rem:intuition-closure}
We can think of $\bar{A} \setminus A^{\circ}$ as the ``boundary''
of $A$. Observe that if $A$ is open, $A = A^{\circ}$, and if $A$ is closed,
$A = \bar{A}$.
\end{remark}
Next, we examine how closures propagate to subspaces.
\begin{theorem}
\label{thm:closure-prop}
Let $Y$ be a subspace of $X$, and $A \subset Y$. Then
$\overline{A_{Y}} = \overline{A_{X}} \cap Y$.
\end{theorem}
\begin{proof}
$\overline{A_{X}}$ is closed in $X$, so $\overline{A_{X}} \cap Y$ is closed in $Y$
by \cref{lem:closed-iff}. But $A \subset \overline{A_{X}} \cap Y$, and by
definition, $\overline{A_{Y}} = \bigcap\left\{ B \subset Y: B \supset A \
\text{and} \
B \ \text{is closed}\right\}$. Hence, $\overline{A_{Y}} \subset
\overline{A_{X}} \cap Y$.
\par
To prove the reverse inclusion, we first note that
$\overline{A_{Y}} = C \cap Y$ for some closed $C \subset Y$.
It follows that $A \subset C$. Since $\overline{A_{X}}$ is the intersection of
all closed sets in $X$, $\overline{A_{X}} \subset C$. Hence,
$(\overline{A_{X}}\cap Y) \subset (C \cap Y) = \overline{A_{Y}}$.
\end{proof}
\begin{definition}
\label{def:intersects}
We say a set $A$ \emph{intersects} $B$ if $A \cap B \neq \emptyset$.
\end{definition}
\begin{theorem}
\label{thm:closure-intersect-equivs}
Let $A$ be a subset of a topological space $X$. Then we have the following:
\begin{enumerate}
  \item $x \in \bar{A}$ $\iff$  every open set of $U$ containing
  $x$ intersects $A$.
  \item $x \in \bar{A}$ $\iff$ every basis element of $B$ containing
  $x$ intersects $A$.
\end{enumerate}
\end{theorem}
\begin{example} $ $
\begin{enumerate}
  \item If $X = \mathbf{R}, A = (0,1]$, then $\overline{A_{X}} = [0,1]$.
  \item If $B = \left\{ 1/n, n \in \mathbf{Z}_{+} \right\}$, then
  $\bar{B} = \left\{ 0 \right\} \cup B$.
  \item $\bar{Q}_{\mathbf{R}} = \mathbf{R}$
  \item Let $Y = (0,1]$, viewed as a subspace of $\mathbf{R}$. Then
  $\overline{(0,1/2)}_{\mathbf{R}} = [0,1/2]$, while
  $\overline{(0,1/2)}_{\mathbf{Y}} = [0,1/2]$
\end{enumerate}
\end{example}

\subsection{Limit Points}
We have a different way of describing closure.
\begin{definition}
If $A$ is a subset of a topological space $X$ and $x \in A$, we say $x$
is a \emph{limit point} of $A$ if every neighborhood of $x$ intersects
$A$ at some point other than $x$ itself. Equivalently,
$x$ is a limit point if $x \in \overline{A \setminus \left\{ x \right\}}$
\end{definition}
\begin{example}
Consider the interval \((0,1)\) as a subset of \(\mathbf{R}\). Then \(\bar{A} =
[0,1]\), and \(\left\{ 0,1 \right\}\) is the set of limit points.
\end{example}
\begin{theorem}
\label{thm:limit-points-def-closedness}
Let \(A\) be a subset of a topological space \(X\), and let \(A'\) denote the
set of limit points of \(A\). Then
\begin{equation*}
\bar{A} = A \cup A'.
\end{equation*}
\begin{proof}
Observe that if \(x \in A'\), then, by definition, every neighborhood of
\(x\) intersects \(A\setminus \left\{ x \right\}\). Hence, \(x \in
\bar{A}\), by \cref{thm:closure-intersect-equivs}, and so
\(A' \subset \bar{A}\), from which it follows that \(A \cup A' \subset
\bar{A}\).
\par
To prove the reverse inclusion, we assume \(x \in \bar{A}\). If \(x \in A\),
if \(x \in A\), then it is trivial that \(x \in A \cup \bar{A}\), so suppose
\(x \not \in A\). Since \(x \in \bar{A}\), by
\cref{thm:closure-intersect-equivs} every neighborhood \(U\) of \(x\)
intersects \(A\). By assumption, \(x\) does not belong to the intersection,
and so every neighborhood of \(x\) intersects \(A\setminus \left\{ x
\right\} \). That is, \(x \in A'\), and so \(x \in A \cup A'\).
\end{proof}
\end{theorem}
\begin{corollary}
\label{cor:subset-closed-iff-lim-points}
A subset of a topological space is closed if and only if it contains all of
its limit points.
\end{corollary}
\begin{proof}
A set \(A\) is closed if and only if \(A = \bar{A}\). But this is true
if and only if \( A' \subset A\), since \(\bar{A} = A \cup A'\).
\end{proof}
\begin{definition}
\label{def:conv-limit}
We say a sequence \(\left\{ x_{n} \right\}_{n}\) of points in \(X\)
\emph{converges} to a point \(x \in X\) if for each neighborhood
\(U\) of \(x\), there exists \(N \equiv N(x, U)\) such that
\(x_{n} \subset U\)
for all \(n \ge N\).
\end{definition}
\begin{remark}
\label{rem:uniqueness-conv-r}
In the standard topology on \(\mathbf{R}\), a sequence converges to at most
one point. However, in general, a sequence can have more than one limit.
Consider  \(\mathbf{R}\), equipped with the trivial topology. Then the
sequence \(\left\{ 1/2, 1/4, 1/8, \cdots \right\}\) converges to \emph{every}
point in \(\mathbf{R}\). In fact, any sequence converges to every point in
\(R\)! Heuristically, the finer a topology, the more it allows us to
``distinguish'' one sequence from another, and the better the odds a sequence
converges to at most one limit. However, if the topology
is too strong, there is a chance a given sequence will not converge in it.
Indeed, if the topology is strong enough, no sequence will converge.
For example, no sequence in \(\mathbf{R}\) converges in the discrete
topology.
\end{remark}
\begin{definition}
\label{def:hausdorff}
A topological space \(X\) is called \emph{Hausdorff} if it \emph{separates}
points, i.e.\ for distinct points
\(x_{1}, x_{2} \in X\), there exist disjoint neighborhoods \(U_{1}, U_{2}\)
of \(x_{1}, x_{2}\), respectively.
\end{definition}
\begin{theorem}
\label{thm:hausdorff-limit-uniqueness}
If \(X\) is a Hausdorff space, then a sequence of points in \(X\) converges
to at most one point in \(X\).
\end{theorem}
\begin{proof}
Choose a sequence \(\left\{ x_{n} \right\}_{n} \subset X\) and
distinct \(x,y\) in \(X\), and suppose \(x_{n} \to x\). Since \(X\)
is Hausdorff, there exist open sets \(U, V\) separating \(x\) and \(y\),
respectively. By the definition of limit points,
there exists \(N \equiv N(U, x)\) such that
\(x_{n} \in U\) for \(n \ge N\). Since \(U\) and \(V\) are disjoint, it
follows that \(x_{n} \not \in V\) for \(n \ge N\), and so
\(x_{n} \not \to y\).
\end{proof}
\begin{theorem}
\label{thm:hausdorff-inherit}
We have the following:
\begin{enumerate}
  \item A simply ordered set is Hausdorff.
  \item A product of Hausdorff spaces is Hausdorff.
  \item A subspace of a Hausdorff space is Hausdorff.
\end{enumerate}
\end{theorem}
\section{Continuous Functions}
\begin{definition}
\label{def:cont-fnc}
Let \(X\) and \(Y\) be topological spaces. Then we say
\(f: X \to Y\) is \emph{continuous} if for each open \(V \in Y\),
\(U = f^{-1}(V)\) is open in \(X\).
\end{definition}
\begin{remark}
Observe that the continuity of \(f\) ultimately depends on the topologies of
\(X\) and \(Y\). For example, if \(X\) is equipped with the discrete topology,
then any map from it is continuous automatically. If \(Y\) is equipped with
the trivial topology, then any map to it is continuous automatically.  As a
heuristic, the finer \(X\) is, we will have fewer continuous maps from it to
other spaces. Analogously, the coarser \(Y\) is, the more continuous maps to
it we will have.
\end{remark}
If \(Y\) has a basis \(\mathcal{B}\), then to show continuity of a map
\(f: X \to Y\), it is enough to show that \(f^{-1}(B)\) is open for all
\(B \in \mathcal{B}\). This follows from the observation that, for \(V\) open in
\(Y\), we have \(V = \cup_{\alpha \in J} B_{\alpha}\),
and so \(f^{-1}(V) = f^{-1}(\cup_{\alpha \in J}) B_{\alpha} = \cup_{\alpha \in
J} f^{-1}(B_{\alpha})\).
\begin{theorem}
\label{thm:equiv-cont}
Let \(X\) and \(Y\) be topological spaces, and \(f: X \to Y\). Then the
following are equivalent:
\begin{enumerate}
  \item \(f\) is continuous.
  \item For every subset \(A \subset X\), \(f(\bar{A}) \subset
  \overline{f(A)}\)
  \item For every closed \(B \in Y\), \(f^{-1}(B)\) is closed in \(X\).
  \item{\emph{Continuity via \(\varepsilon/\delta\)}.} For each \(x \in X\) and each neighborhood \(V\) of \(f(x)\), there is
  a neighborhood \(U\) of \(x\) such that \(f(U) \subset V\).
\end{enumerate}
\end{theorem}
\begin{definition}
\label{def:cont-at-a-point}
Fix \(x \in X\). If for each neighrbood \(V\) of \(f(x)\),
there is a neighborhood \(U\) of \(x\) such that
\(f(U) \subset V\), then we say \(f\) is \emph{continuous at \(x\)}.
\end{definition}
\subsection{Homeomorphisms}
\begin{definition}
\label{def:homeo}
Let \(X\) and \(Y\) be topological spaces, and  \(f: X \to Y\) bijective.
If \(f\) and \(f^{-1}\) are continuous, we say \(f\) is a
\emph{homeomorphisms}.
\end{definition}
\begin{theorem}
\label{thm:equiv-spaces}
If there exists a bijection \(f\) between topological spaces \(X\) and
\(Y\), then \(X\) and \(Y\) are topologically equivalent, in the sense that
\(U\) is open in \(X\) if and only if \(f(U)\) is open in \(Y\).
\end{theorem}
\begin{proof}
If \(f^{-1}\) is continuous, then for each open \(U \subset X\),
\(f^{-1}(f^{-1}(U))\) is open in \(Y\). But \(f^{-1} \circ f^{-1} = f\),
and so \(f(U)\) is open in \(Y\). For the reverse direction, we note that if
\(f(U)\) is open, then by continuity, \(U = f^{-1}(f(U)) \) is open.
\end{proof}
\begin{remark}
Hence, a homeomorphic map between topological spaces \(X\) and \(Y\) gives a
bijective correspondence between not just the members of the sets \(X\) and
\(Y\), but also between the members of their topologies.  \par If \(f: X \to
Y\) is injective and continuous, then the same map is homeomorphic from \(X\)
to \(f(X)\), equipped with the subspace topology. We say \(f\)
\emph{topologically embeds} \(X\) in \(Y\); that is, there is a bijective
correspondence between \(X\) and a subspace of \(Y\).
\end{remark}
\begin{example}
Every linear function \(f: \mathbf{R} \to \mathbf{R}\) is a homeomorphism. As
an example, take \(f(x): 5x + 2\). Then \(f^{-1}\) exists and is given by
\(f^{-1}(x) = (x-2)/5\).
\end{example}
However, functions can be continuous without being homeomorphisms.
\begin{example}
Let \(f: [0,1) \to \mathbf{S}\) be a map given by \(f(t) = (\cos 2\pi t, \sin 2 \pi
t)\). Then \(f\) is continuous, but \(f^{-1}\) is not, since
\(f([0, 1/4))\) is not open in \(\mathbf{S}\).
\end{example}
\subsection{Constructing Continuous Functions}
\begin{theorem}
\label{thm:building-cont-fncs}
Let \(X,Y,Z\) be toplogical spaces.
\begin{enumerate}[(a)]
  \item{The Constant Function.}\label{it:cf} If \(f: X \to Y\) maps all of \(X\)
  into a single point in \(Y\), then \(f\) is continuous.
  \item{Identity.}\label{it:id} If \(A\) is a subspace of \(X\), then the identity function
  \(\mathrm{I}: A \to X\) is continuous.
  \item{Composites.}\label{it:comp} If \(f: X \to Y\) and \(g: Y \to Z\) are continuous, then
  \(g \circ f: X \to Z\) is continuous.
  \item{Restrictions of Domain.}\label{it:rd} If \(f: X \to Y\) is continuous, and
  \(A \subset X\) is a subspace, then \(f | A: A \to Y \) is continuous.
  \item{Restrictions of Target.}\label{it:rt} If \(Z \subset Y\) is a subspace
  with \(f(X) \subset Z\), then \(g: X \to Z \), \(g \equiv f\) is
  continuous.
  \item {Extensions.}\label{it:ex} If \(Z\) is a topological space having \(Y\) as a
  subspace, then the extension \(h: X \to Z\) is continuous.
  %TODO: Fix definition
  \item{Pasting.}\label{it:pasting} The map \(f: X \to Y\) is continuous if and only if
  \(X\) can be written as the union of open sets \(U_{\alpha}\) such that
  \(f|U_{\alpha}\) is continuous for each \(\alpha\).
\end{enumerate}
\end{theorem}
\begin{proof}
\Cref{it:cf} follows immediately from the fact that \(X\) is open and
\(f(y_{0}) = X \) for the single point \(y_{0} \in f(X)\).
\par
\cref{it:id} follows from the observation that, for open \(U \in X\),
\(\mathrm{I}^{-1}(U) = U \cap A\), which is open in \(A\).
\par
\cref{it:comp}  follows from the observation that if \(U\) is open in
\(Z\), then \(g^{-1}(U)\) is open in \(Y\), and \(f^{-1}(g^{-1}(U))\) is open
in \(X\). But \(f^{-1}(g^{-1}(u)) = (g \circ f)^{-1}(U)\).
\par
To prove \cref{it:ex}, we first write \(X = \cap_{\alpha} U_{\alpha}\).
Not that if \(V\) is open in \(Y\), then
\(f^{-1}(V) \cap U_{\alpha} = (f | U_{\alpha})^{-1}(V)\) is open, by
continuity of \(f | U_{\alpha}\). Hence,
\(f^{-1}(V) = \cup_{\alpha}(f^{-1}(V) \cap U_{\alpha}) = \cup_{\alpha}
(f | U_{\alpha})^{-1}(V)\) is open in \(X\).
\end{proof}
\begin{theorem}
\label{thm:pasting-lemma}
Let \(X = A \cup B\), where \(A,B\) are closed in \(X\). Let \(f: A \to Y\)
and \(g: B \to Y\) be continuous functions. If \(f(x) = g(x)\) for every
\(x \in A \cap B\), then can be combined  to give a continuous function
\(h: X \to Y\), where
\begin{align*}
h(x) & = f(x), \ x \in A \\
h(x) & = g(x), \ x \in B.
\end{align*}
\end{theorem}
\begin{example}
\(h: \mathbf{R} \to \mathbf{R}\), given by
\begin{equation*}
h(x) = \begin{cases}
x, \ & x \le 0 \\
x/2, \ & x \ge 0
\end{cases}
\end{equation*}
is continuous, while \(\ell(x): \mathbf{R} \to \mathbf{R}\), given by
\begin{equation*}
\ell(x) = \begin{cases}
x-2, \ & x < 0 \\
x+2, \ & x \ge 0
\end{cases}
\end{equation*}
is not.
\end{example}
\begin{theorem}
\label{thm:product-top-cont-via-proj}
Let \(f: A \to X \times Y\) be given by \(f(a) = (f_{1}(a), f_{2}(a))\),
where \(f_{1}: A \to X\) and \(f_{2}: A \to Y\).
Then \(f\) is continuous if and only if \(f_{1}\) and \(f_{2}\) are
continuous.
\end{theorem}
\begin{proof}
Pick an open \(U_{1} \in X\) and an open \(U_{2} \in Y\).
Then if \(f_{1}\) and \(f_{2}\) are continuous,
\(f^{-1}(U) = f^{-1}(U_{1} \times U_{2}) =
(f_{1}(U_{1}), f_{2}(U_{2}))\) is open in the product topology.
We conclude that \(f\) is continuous.

For the reverse direction, we assume \(f\) is continuous, and recall that
projection maps are continuous, and compositions of continuous maps are
continuous. Therefore, \(\pi_{1}(f) = f_{1}\)
and \(\pi_{2}(f) = f_{2}\) are continuous.
\end{proof}
\begin{theorem}
Let $f: A \to \prod_{\alpha \in J}X_\alpha$ be given by
\begin{equation*}
f(a) = (f_{\alpha})
\end{equation*}
where $f_{\alpha}: A \to X_{\alpha}$. Let $\prod X_{\alpha}$
have the product topology. Then $f$ is continuous if and only if
each $f_{\alpha}$ is continuous.
\end{theorem}
\begin{proof}
The forward direction is analogous to that of the proof of
\cref{thm:product-top-cont-via-proj}.
%TODO Reverse Direction
\end{proof}
\begin{example}
The theorem fails for the box topology. For example, consider
$\mathbf{R}^{\omega} = \prod_{n \in \mathbf{Z_{+}} \mathbf{R}}$
and $f: \mathbf{R} \to \mathbf{R}^{\omega}$ given by
$f(t) = (t, t, \ldots, t, \ldots)$. We can rewrite this as
$f(t) = (f_{0}(t), f_{1}(t), \ldots, f_{n}(t), \ldots)$, where
the $f_i: \mathbf{R} \to \mathbf{R}$ are given by $f_{i}(t) = t$.
Clearly the $f_{i}$ are continuous. Next, note that
$B = (-1,1) \times (-1/2, 1/2), \ldots, (-1/n, 1/n), \ldots$ is open in the box
topology. However, $f^{-1}(B)$ is not open in $\mathbf{R}$.
If it were, then for sufficiently small $\delta > 0$, $(-\delta, \delta)
\subset f^{-1}(B)$, which would imply the false statement $f((-\delta, \delta))
\subset B$.
\end{example}
\section{Metrics}
\begin{definition}
\label{def:metric}
A metric on a set \(X\) is a function \(\mathrm{d}: X \times X \to \mathbf{R}\)
with the followiong properties:
\begin{enumerate}
  \item{Non-negativity} \(\mathrm{d}(x,y) \ge 0, \ \forall x,y \in X\)
  \item{Identity of Indiscernibles} \(\mathrm{d}(x,y) = 0 \iff x=y\)
  \item{Symmetry} \(\mathrm{d}(x,y) = \mathrm{d}(y,x) \ \forall x,y \in X\)
  \item{Triangle Inequality} \(\mathrm{d}(x,z) \le \mathrm{d}(x,y) +
  \mathrm{d}(y,z)\).
\end{enumerate}
\end{definition}
\begin{definition}
\label{def:metric-topology}
If \(\mathrm{d}\) is a metric on a set \(X\), then for fixed \(\varepsilon >
0\),  the collection of \emph{\(\varepsilon\)-balls} \(B_{d}(x, \varepsilon)
\doteq \left\{ y \in X:
\mathrm{d}(x,y) < \varepsilon \right\}\) is a basis, and generates toe
\emph{metric topology} on \(X\).
\end{definition}
\begin{example} \(\)
\begin{enumerate}
  \item The \emph{standard metric} on \( \mathbf{R} \)
  is given by \( \mathrm{d}(x,y) =
  |x-y| \). It induces the same topology as the order topology,
  since
  \( (a,b) = \mathbf((a+b)/2), \varepsilon = (b-a)/2 \) and,
  conversely, \( \mathbf(x,\varepsilon) =
  (x - \varepsilon, x + \varepsilon) \).
  \item Given a set \(X\), define
  \begin{equation*}
\mathrm{d}(x,y) = \begin{cases}
1, & x \neq y \\
0, & x = y.
\end{cases}
\end{equation*}
Then for fixed \(x \in X\) and any \(r \ge 0\), we have
\(\mathbf{B}(x, r) = \left\{ x \right\}\). Hence, this metric
induces the discrete topology.
\end{enumerate}
\end{example}
In general, it is easier to work with a metric directly, rather than the
topological sets that it induces. A natural question therefore is to ask whether
all topologies are induced by some metric.
It turns out, the answer in general is no, but we will leave a
more detailed of this for a later time. If a topology is induced by a metric, we
say that the topology is \emph{metrizable}.
\begin{definition}
\label{def:bounded-wrt-metric}
Let \(X\) be a metric space equipped with a metric \textrm{d}.
We say \(A \subset X\) is \emph{bounded} if there exists \(M \ge 0\)
such that \(\mathrm{d}(a_{1}, a_{2}) \le M \ \forall a_{1}, a_{2} \in A\).
If \(A\) is bounded and nonempty, then we define
\(\mathrm{diam}A = \sup\left\{ \mathrm{d}(a_{1}, a_{2}): a_{1}, a_{2} \in A
\right\}\).
\end{definition}
\begin{definition}
\label{def:euclidean-metric}
Given \(x = (x_{1}, x_{2}, \cdots, x_{n}) \in \mathbf{R}^{n}\),
and the notation \(\| x \| \doteq (x_{1}^{2} + \cdots +
x_{n}^{2})^{1/2}\), the \emph{euclidean metric} is given by
\(\mathrm{d}(x,y) = \| x-y \| = \left[ (x_{1} - y_{1})^{2} + \cdots +
(x_{n} - y_{n})^{2} \right]^{1/2}\).
The \emph{square metric} is given by
\(\rho(x, y) = \max\left\{ | x_{1} - y_{1}|, \cdots, |x_{n} - y_{n}|
\right\}\).
\end{definition}
\begin{theorem}
\label{thm:induce-product-top}
The topologies on \(\mathbf{R}^{n}\) induced by the square and euclidean
metrics coincide with the product topology on \(\mathbf{R}^{n}\).
\end{theorem}
To prove the theorem, we shall need a series of lemmas:
\begin{lemma}
\label{lem:finer-top-in-balls}
Let \(\mathcal{T}\) and \(\mathcal{T'}\) be two topologies on \(X\)
induced by the metrics \textrm{d} and \textrm{d'}, respectively.
Then \(\mathcal{T'}\) is finer than \(\mathcal{T}\) if and only if for each
\(x \in X\) and each \(\varepsilon > 0\), \(\exists \delta > 0\) such that
\(\mathbf{B}_{\mathrm{d'}}(x, \delta) \subset
\mathbf{B}_{\mathrm{d}}(x, \varepsilon)\).
\end{lemma}
\begin{definition}
\label{def:equiv-metric}
We say two metrics \(\mathrm{d_{1}}, \mathrm{d_{2}}\) on a set \(X\)
are \emph{equivalent} if there exist constants \(\alpha, \beta \in
\mathbf{R}\) such that
\begin{equation*}
\alpha \mathrm{d_{2}}(x,y) \le \mathrm\delta_{1}(x,y) \le \beta
\mathrm{d_{2}}(x,y)
\end{equation*}
for all \(x, y \in X\).
\end{definition}
\begin{lemma}
\label{lem:equiv-metric-equiv-top}
Equivalent metrics on a set \(X\) induce the same topology.
\end{lemma}
\begin{proof}
We have
\begin{equation*}
\alpha \mathrm{d_{2}}(x,y) \implies \mathbf{B}_{\alpha
\mathrm{d_{2}}}(x, \varepsilon) \subset
\mathbf{B}_{\mathrm{d_{1}}}(x, \varepsilon) \implies
\mathbf{B}_{\mathrm{d_{2}}}(x, \alpha \varepsilon) \subset
\mathbf{B}_{\mathrm{d_{1}}}(x, \varepsilon)
\end{equation*}
which by \cref{lem:finer-top-in-balls} implies that the topology generated by
\(\mathrm{d_{2}}\) is finer than that generated by \(\mathrm{d_{1}}\).
A similar argument shows that if \(d_{1}(x,y) \le \beta
\mathrm{d_{2}}(x,y)\), then the topology generated by \(\mathrm{d_{1}}\) is
finer than that generated by \(\mathrm{d_{2}}\).
\end{proof}
\begin{proof}[Proof of \cref{thm:induce-product-top}]
First, observe that
\begin{equation*}
\rho(x,y) \le \mathrm{d}(x,y) \le n \rho(x,y)
\end{equation*}
Hence, it will be enough to show that the product topology and the topology
generated by the square metric are the same. We shall do so by showing that
the square topology is both finer and coarser than the product topology.

Proceeding, let \(B = (a_{1}, b_{1}) \times \cdots \times (a_{n},
b_{n})\) be a basis element of the product topology. Let \(x  =
(x_{1}, x_{2}, \cdots, x_{n})\), where \(x_{i} =
(a_{i} + b_{i})/2\). Then there exist \(\varepsilon_{i}\) such that
\( (x_{i} - \varepsilon_{i}), x_{i} + \varepsilon_{i} \subset (a_{i},
b_{i})\). Let \(\varepsilon = \min\left\{ \varepsilon_{i}
\right\}_{i=1}^{n}\). Then \(\mathbf{B}_{\rho}(x, \varepsilon) \subset B\),
and hence the square topology is finer than the product topology.

For the reverse direction, we note that if we had taken the
\(\varepsilon_{i}\) to be sufficiently large, then
\( (x_{i} - \varepsilon_{i}, x_{i} + \varepsilon_{i}) \supset (a_{i},
b_{i})\). Generating \(\varepsilon\) by taking the maximum over these
\(\varepsilon_{i}\), we obtain \(\mathbf{B}_{\rho}(x, \varepsilon) \supset
B\), and hence the square topology is coarser than the product topology.
\end{proof}
Next, we'd like to come up with a metric for \( \mathbf{R}^{\omega} \).
A first guess might be inspired by the euclidean metric; namely
\begin{equation*}
\mathrm{d}(x,y) = \left[\sum_{n=1}^{\infty}(x_i - y_i)^{2} \right]^{1/2}
\end{equation*}
Unfortunately, this function is not guaranteed to be finite. A similar problem
rules out the function
\begin{equation*}
\rho(x,y) = \sup \{ |x_n - y_n| \}.
\end{equation*}
\begin{definition}
The uniform metric on \( \mathbf{R}^{\omega} \) is given by
\begin{equation*}
\mathbf{d}(x,y) = \sup \min \left\{ 1, |x_n - y_n \right\}.
\end{equation*}
\end{definition}
\begin{theorem}
The uniform topology on \( \mathbf{R}^J \) is finer than the product
topology if $J$ is finite. If \( J \) is inifinite, then they are different
topologies, in the sense that one is not coarser or finer than the other.
\end{theorem}
%TODO Write proof
\begin{theorem}
Let \( A \) be a subspace of \( X \), equipped with a metric \( \mathrm{d} \).
Then
\begin{enumerate}
  \item\label{it:restriction-metric-equiv-order} The metric,
		restricted to \( A
  \) defines a topology on \( A \) identical to the subspace topology.
  \item\label{it:metric-is-hausdorff} \( X \) is Hausdorff.
  \item\label{it:prod-top-metrizable} The product topologies on \( \mathbf{X}^n
  \) and \( \mathbf{X}^\omega \) are metrizable.
\end{enumerate}
%TODO Write proof
\end{theorem}
\begin{proof}
To prove \cref{it:restriction-metric-equiv-order}, let 
\( \mathcal{B} \) be the basis for \( X \) consisting of open balls
and \( \mathcal{B'} \) be the basis for \( A \) consisting of all open balls
lying within \( A \); that is, the basis induced by the metric \( \mathrm{d} \)
restricted to \( A \). Since any ball in \( A \) under \( \mathrm{d}|A \) is a
ball in \( X \) under \( \mathrm{d} \), we have  \( \mathcal{B'} \subset
\mathcal{B} \). To prove the the reverse direction, let \( S \) be a set in the
subspace topology. Let \( \delta > 0 \). Then \( S = \cup_{x \in S}
B_{\mathrm{d}|A}(x, \delta) \), completing the proof of
\cref{it:restriction-metric-equiv-order}.
\par
To prove \cref{it:metric-is-hausdorff}, let \( x, y \in X: x \neq y \),
and \( \delta = \mathrm{d}(x,y) \). Then the balls \( B(x, \delta/4) \)
and \( B(y, \delta/4) \) contain \( x \) and \( y \), respectively, and are
disjoint.
\par
Lastly, to prove \cref{it:prod-top-metrizable}, we ask that the reader verify
that the metric
\begin{equation*}
\mathrm{d}  = \sum_{m \in J} \frac{\mathrm{d}_{m}(x,y)}{2^{m}(1 +
\mathrm{d}_{m}(x,y))}
\end{equation*}
induces the product topology on \(X^{n}\), when \(|J| = n\), and on
\( X^{\omega} \), when \( |J| = \omega \).
%TODO Complete proof yourself 
\end{proof}
