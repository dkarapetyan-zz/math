\chapter{Point-Set Topology}
\section{Topological Spaces}
\begin{definition}
	\label{def:top}
	A \emph{topology} $\mathcal{T}$ on $X$ is a collection of subsets of $X$,
	including $X$ and $\emptyset$ that is closed under unions and finite
	intersections.

	We call $X$ equipped with a topology $\mathcal{T}$ a \emph{topological space},
	denoted by $(X, \mathcal{T})$. If $U \in \mathcal{T}$, we say $U$ is an \emph{open} set.

	Let $\mathcal{T}, \mathcal{T'}$ be topologies on $X$. If $\mathcal{T} \subset
	\mathcal{T'}$, then we say $\mathcal{T}$ and $\mathcal{T'}$ are
	\emph{comparable topologies}, and that $\mathcal{T'}$ is a \emph{finer} (or
	\emph{weaker}) topology and
	$\mathcal{T}$ is a \emph{coarser} (or \emph{stronger}) topology.
\end{definition}
\begin{example} $ $
	\begin{enumerate}
		\item $X = \left\{ a,b,c \right\}, \ \mathcal{T} = \left\{
				\left\{
				a,b,c \right\}, \left\{
			b \right\}, \left\{ b,c \right\}\right\}$.
		\item $\left( X, \left\{ X, \emptyset \right\} \right)$. This is
			known as the
			\emph{trivial topology}.
		\item $\left( X, \mathrm{P}(X) \right)$, known as the \emph{discrete
			topology}.
	\end{enumerate}
\end{example}
\section{Basis for a Topology}
\begin{definition}
	\label{def:basis}
	A basis for a topology $\mathcal{T}$ is a collection $\mathcal{B}$ of subsets of
	$\mathcal{T}$ such that
	\begin{enumerate}
		\item $\mathcal{B}$ is a cover for $X$. That is, $\bigcup_{B \in
			\mathcal{B}} B = X$.
		\item If $x \in B_{1} \cap B_{2}$, then there exists a basis element
			$B_{3}$ such that $x \in B_{3}$ and $B_{3} \subset B_{1} \cap
			B_{2}$.
	\end{enumerate}
\end{definition}
\begin{example}
	If $X$ is any set, the collection of all one-point subsets of $X$
	is a basis for the discrete topology on $X$.
\end{example}
Intuitively, a basis is a subset of our topology that is closed under
intersections, but not necessarily unions. However, the set of all possible
unions of members of our basis generates our topology!
Hence, a basis can be thought of as the ``skeleton'' of a topology.
We now make this rigorous.
\begin{theorem}
	\label{thm:basis-gen-topology}
	Let $\mathcal{B}$ be a basis for a set $X$. Then the collection $\mathcal{T}$
	generated by collecting all possible unions of elements of $\mathcal{B}$ is
	a topology on $X$
\end{theorem}
\begin{proof}
	Closure under unions follows by construction of $\mathcal{T}$. Suppose
	$U_{1}, U_{2} \in \mathcal{T}$. To complete the proof, it will be enough to
	show that $U_{1} \cap U_{2} \in \mathcal{T}$; closure under arbitrary finite
	intersections will then follow by induction. Suppose $x \in U_{1} \cap
	U_{2}$. By construction of $\mathcal{T}$, there exist $B_{1} \subset U_{1}$
	and $B_{2} \subset U_{2}$ containing $x$ (i.e.\ unions of basis elements
	generate $U_{1}$ and $U_{2}$). Since $\mathcal{B}$ is a basis, there exists
	$B_{3} \subset B_{1} \cap B_{2}$ containing $x$, and so $U_{1} \cap U_{2}
	\subset B_{3}$. However, since $B_{3} \subset U_{1} \cap U_{2}$, and $B_{1}
	\subset U_{1}$ and $B_{2} \subset U_{2}$, we must also have $B_{3} \subset
	U_{1} \cap U_{2}$, from which we conclude that $B_{3} = U_{1} \cap U_{2}$.
	Since $\mathcal{T}$ is generated via unions of elements in $\mathcal{B}$, we
	conclude that $U_{1} \cap U_{2} \in \mathcal{T}$.
\end{proof}
We have seen how to generate a topology from a basis. We now examine how to
generate a basis from a topology.
\begin{theorem}
	\label{thm:basis-from-top}
	Let $(X, \mathcal{T})$ be a topological space. Suppose $\mathcal{C}$ is a collection of
	open sets of $X$ such that for each open set $U \in X$ and each
	$x \in U$, there is an element $C_{x} \in \mathcal{C}$ such
	that $x \in C_{x}
	\subset U$. Then $\mathcal{C}$ is a basis for the topology on $X$.
\end{theorem}
\begin{proof}
	First, observe that for a given open $U$, we have
	$\cup_{x \in U}C_{x} = U$. Hence, unions of elements of $\mathcal{C}$
	generates $\mathcal{T}$.

	It remains to show that $\mathcal{C}$ is a basis.
	Since our basis elements are open, by definition their finite intersections
	are open, and so by definition we have that for $C_{x}^{(1)}$ and
	$C_{x}^{(2)}$ containing $x$, there exists $C_{x}^{(3)} \subset
	C_{x}^{(1)} \cap C_{x}^{(2)}$ containing $x$.
\end{proof}
\begin{lemma}
	\label{lem:finer-coarser-basis}
	Let $\mathcal{B}, \mathcal{B'}$ be bases for the the respective topologies
	$\mathcal{T}, \mathcal{T'}$ on $X$. Then the following are equivalent:
	\begin{enumerate}
		\item\label{it:finer} $\mathcal{T'}$ is finer than $\mathcal{T}$.
		\item\label{it:basis-subset} For each $x \in X$ and each basis element $B_{x} \in \mathcal{B}$
			containing $x$, there is a basis element $B' \in \mathcal{B'}$ such that
			$x \in B' \subset B$.
	\end{enumerate}
\end{lemma}
\begin{proof}
	Assume \cref{it:finer}. Then $\mathcal{T} \subset \mathcal{T'}$.
	Since unions over basis elements generate topologies,
	it follows that:
	\begin{enumerate}
		\item There exist basis elements $B_{x}' \in \mathcal{B'}$ and
			$B_{x} \in \mathcal{B}$ containing $x$.
		\item $B_{x}' \cap B_{x}$ is open in $\mathcal{T'}$.
	\end{enumerate}
	Hence there exists some basis
	element $\tilde{B}_{x}' \subset B_{x}' \cap B_{x}$ in $\mathcal{B'}$, and so
	$\tilde{B}_{x}' \subset B_{x}$.

	To prove the reverse direction, observe that for fixed $A \in \mathcal{T}$ there exists a collection $\left\{ B_{x} \right\}_{x \in A}$ of basis elements in $\mathcal{B}$ such that $A = \cup_{x
	\in A} B_{x}$.
	Assuming \cref{it:basis-subset}, for each $B_{x}$, there exists $B_{x}'$ in $\mathcal{B'}$ such that $B_{x}' \subset B_{x}$.
	Collecting these $B_{x}'$, it follows that $A = \cup_{x \in A} B_{x}'$, and hence
	$\mathcal{T} \subset \mathcal{T'}$.
\end{proof}
\begin{example}
	Let $\mathcal{B}$ be the collection of all open intervals on the real line.
	Then this is a basis for the topology typically used on the real line,
	which we call the \emph{standard topology} on the real line.
\end{example}
\begin{definition}
	\label{def:sub-basis}
	A \emph{subbasis} $\mathcal{S}$ for a topology on $X$ is a collection of subsets of
	$X$ whose union is $X$. The topology generated by $S$ is defined to be
	the collection $\mathcal{T}$ of all unions and finite intersections of members
	of $\mathcal{S}$.
\end{definition}
It follows by construction that if $\mathcal{S}$ generates a topology
$\mathcal{T}$, then the collection $\mathcal{B}$ of finite
intersections of elements in $\mathcal{S}$ is a basis for $\mathcal{T}$.
\section{Important Topologies}
\subsection{The Order Topology}
\begin{definition}
	\label{def:order_top}
	Let $X$ be a set with more than one element, and equipped with an order
	relation $<$. Then we define the \emph{order topology} $\mathcal{T}$ as the topology
	generated by the following basis $\mathcal{B}$, consisting of:
	\begin{enumerate}
		\item\label{it:open-intervals} All intervals $(a,b) \doteq \left\{ x \in X:
				a < x < b
			\right\}$ in $X$.
		\item\label{it:ray1} All intervals of the form $[a_{0}, b) \doteq \left\{
			x \in X: a_{0} \le x < b \right\}$, where $a_{0}$
			is the smallest element (if any) of $X$.
		\item\label{it:ray2} All intervals of form $(a, b_{0}] \doteq \left\{
			x \in X: a < x \le b_{0} \right\}$,
			where $b_{0}$ is the largest
			element (if any) of $X$.
	\end{enumerate}
\end{definition}
\begin{remark}
	\label{rem:rr-top}
	If $X = \mathbf{R^{n}}$, $\mathcal{B}$ consisting only of \cref{it:open-intervals}
	generates $\mathcal{T}$. This is \emph{not} true If $X =
	\mathbf{R} \cup
	\{-\infty, \infty\}$.
\end{remark}
\begin{example}
	$ $
	\begin{enumerate}
		\item Consider $\mathbf{R^{2}}$ equipped with dictionary
			order. Then its order topology is generated by the open intervals
			$\left( (a,b), (c,d) \right) = \left\{ (x,y): a < x < c \ \text{and} \ b <
			y < d\right\}$
		\item The order topology on $\mathbf{Z}_{+}$ coincides with the discrete topology.
	\end{enumerate}
\end{example}
\begin{definition}
	\label{def:rays-subbasis}
	If $X$ is an ordered set, then given $a \in X$, the sets
	\begin{equation*}
		(a, \infty), (-\infty, a), [a, \infty) \ \text{and} \  (-\infty, a]
	\end{equation*}
	are called \emph{rays}. Observe that that finite intersections of the
	rays generate the basis for the order topology on $X$. Hence, the rays form a subbasis for the order topology on $X$.
\end{definition}
\subsection{The Product Topology}
\begin{definition}
	\label{def:product_topology}
	Let $(X, \mathcal{T}_{X}$ and $(Y, \mathcal{T}_{Y})$ be topological spaces.
		Then the \emph{product topology} on $X \times Y$ is defined to be the
		topology having as basis the collection of all sets of form
		\(U \times V \doteq \left\{ (x,y): x \in U \subset
		\mathcal{T}_{X}, y \in V \subset \mathcal{T}_{Y}  \right\}\).
\end{definition}
However, we can find a subset of this basis that is also a basis for the product
topology:
\begin{theorem}
	\label{thm:product-basis}
	If $\mathcal{B}$ is a basis for $\mathcal{T}_{X}$, and
	$\mathcal{C}$ is a basis for $\mathcal{T}_{Y}$, then
	\begin{equation*}
		\mathcal{D} = \left\{ B \times C: B \in \mathcal{B} \ \text{and} \ C \in
		\mathcal{C} \right\}
	\end{equation*}
	is a basis for the product topology on $X \times Y$.
\end{theorem}
\begin{proof}
	Let $U \in \mathcal{T}_{X}$ and $V \in \mathcal{T}_{Y}$.
	Then $U, V$ are generated by unions and finite intersections via
	elements of bases $\mathcal{B}$ and $\mathcal{C}$, respectively.
	Hence, we can write
	$U = \{\cup_{i=1}^{\infty} B_{i}\} \cap_{i=1}^{k} \hat{B}_{i}$
	and $V = \{\cup_{i=1}^{\infty} C_{i}\} \cap_{i=1}^{k} \hat{C}_{i}$.

	It follows from D'Morgan's laws that
	\begin{align*}
		U \times V & = \left\{ (x,y): x \in \{\cup_{i=1}^{\infty} B_{i}\}
			\cap_{i=1}^{k} \hat{B}_{i}, y \in \{\cup_{i=1}^{\infty} C_{i}\} \cap_{i=1}^{k}
		\hat{C}_{i} \right\} \\
		& = \bigcup_{i=1}^{\infty} \left\{ (x,y): x \in B_{i} \cap_{i=1}^{k}
		\hat{B}_{i}, y \in C_{i} \cap_{i=1}^{k} \hat{C}_{i} \right\}\\
		& = \bigcup_{i=1}^{\infty} \left\{ (x,y): x \in B_{i}, y \in C_{i} \right \}
		\cap \bigcup_{i=1}^{k} \left \{ (x,y): x \in \hat{B}_{i}, y \in \hat{C}_{i}
		\right\}
	\end{align*}
	Observe that we have generated $U \times V$ using unions and finite
	intersections of members of $\mathcal{D}$. It follows that the topology
	generated by $\mathcal{D}$ is finer than the product topology.
	However, it is also coarser, since any member of $\mathcal{D}$ lives in the
	product topology. We conclude that the product topology and the topology
	generated by $\mathcal{D}$ are identical.
\end{proof}
We now work towards constructing a subbasis for the product topology.
\begin{definition}
	\label{def:projection}
	For sets $X_{1},X_{2}, \cdots, X_{n}$, the \emph{projection of
	$X_{1} \times \cdots \times X_{n}$ onto $X_{i}$} is given by
	\begin{align*}
		& \pi_{i}: X_{1} \times \cdots \times X_{n} \to X_{i} \\
		& \pi_{i}(x_{1}, \cdots, x_{n}) = x_{i}.
	\end{align*}
\end{definition}
Observe that if $U$ is open in $X_{i}$, then $\pi_{i}(U) = X_{1} \times \cdots
\times X_{i-1} \times U \times X_{i+1} \times \cdots \times X_{n}$, which is
also open in the product topology. In fact, the projection maps are the
simplest example of \emph{continuous maps}, an idea which we shall return to
later. Intuitively, the projection maps ``project'' a topological space onto its
$X_{1}$ to $X_{n}$ axes. It seems natural then that there should be a way to
reconstruct the space from the projections and the inverse of the projection
maps. In fact, we have the following:
\begin{theorem}
	\label{thm:proj-subbasis}
	The collection
	\begin{equation*}
		S = \bigcup_{i=1}^{n} \left\{ \pi_{i}^{-1}(B): B \in \mathcal{B}_{X_{i}} \right\}
	\end{equation*}
	is a subbasis for the product topology on $\prod_{i=1}^{n} X_{i}$.
\end{theorem}
\begin{proof}
	Recall that basis elements are open. Fix $1 \le j \le n$, and observe that
	\begin{align*}
		\prod_{i=1}^{n} X_{i}
		&= \bigcup_{B \in
		\mathcal{B}_{X_{j}}}X_{1} \times X_{2} \times \cdots \times B \times
		\cdots \times X_{k} \\
		&= \bigcup_{B \in \mathcal{B}_{X_{j}}} \pi_{j}^{-1}(B)
	\end{align*}
	Since $j$ was chosen arbitrarily, it follows that
	\begin{align*}
		\prod_{i=1}^{n} X_{i} = \bigcup_{i=1}^{n} \bigcup_{B \in
		\mathcal{B}_{X_{i}}} \pi_{i}^{-1}(B).
	\end{align*}
	Therefore, $S$ is a cover for $\prod_{i=1}^{n} X_{i}$, and hence a subbasis
	for some topology on it.
	Lastly, we have
	\begin{align*}
		B_{1} \times \cdots \times B_{n} = \bigcap_{i=1}^{n} \pi_{i}^{-1}(B_{i}).
	\end{align*}
	Therefore, by \cref{thm:product-basis}, the basis for the product topology is
	a subset of the collection of finite intersections of $S$, and so the topology
	generated by $S$ is a superset of the product topology. However, every member
	of $S$ belongs to the product topology, and the topology generated by $S$ is
	also a subset of the product topology.

	We conclude that $S$ is a subbasis for the product topology.
\end{proof}
\begin{remark}
	\label{rem:can-sub-open-sets}
	It follows from the structure of the proof that
	\begin{equation*}
		S' = \bigcup_{i=1}^{n} \left\{ \pi^{-1}(U): U \in \mathcal{T}_{X_{i}}  \right\}
	\end{equation*}
	is also a subbasis for the product topology on $\prod_{i=1}^{n} X_{i}$, which
	can be useful when constructing a subbasis for a product topology where
	one wishes to bypass finding the bases of the individual components.
	However, since $S \subset S'$, the result of the theorem is stronger.
\end{remark}
\subsection{The Subspace Topology}
\begin{definition}
	\label{def:subspace-top}
	Let $(X, \mathcal{T_{X}})$ be a topological space. Then $Y \subset X$,
	equipped with the topology $\mathcal{T}_{Y} = \left\{ Y \cap U: U \in
	\mathcal{T}_{X} \right\}$ is called a \emph{subspace} of $X$.
\end{definition}
\begin{lemma}
	\label{lem:basis-subspace}
	If $\mathcal{B}_{X}$ is a basis for $X$, and $Y \subset X$, then
	\begin{equation*}
		\mathcal{B}_{Y} = \left\{ B \cap Y: B \in \mathcal{B} \right\}
	\end{equation*}
	is a basis for the subspace topology on $Y$.
\end{lemma}
\begin{proof}
	Every element of $\mathcal{B}_{Y}$ is in $\mathcal{T}_{Y}$, and so
	the topology generated by $\mathcal{B}_{Y}$ is a subset of
	$\mathcal{T}_{Y}$. Furthermore, every element of $\mathcal{T}_{Y}$
	is generated via unions of members of $\mathcal{B}_{Y}$, and so
	the topology generated by $\mathcal{B}_{Y}$ is a superset of
	$\mathcal{T}_{Y}$.

	It follows that these two topologies coincide. We now show that $\mathcal{B}$
	is a basis. First, observe that since $\mathcal{B}$ is a
	basis for $X$, there exist members in $\mathcal{B}$ whose union is $X$. Since
	$X \cap Y = Y$, it follows that $\mathcal{B}_{Y}$ covers $Y$. Secondly,
	for given $B_{1}, B_{2} \in \mathcal{B}$, there exists $B_{3} \in \mathcal{B}$
	such that $B_{3} \subset B_{1} \cap B_{2}$. It follows that
	$\left( B_{3} \cap Y  \right)\subset (B_{1} \cap Y) \cap (B_{2} \cap Y)$.
\end{proof}
It is not always the case that an element of the subspace topology on $Y \subset
X$ is open in $X$. However, we have the following:
\begin{lemma}[Transitivity of Openness]
	\label{lem:open-criteria-subspace}
	Let $Y$ be a subspace of $X$. If $U$ is open in $Y$, and $Y$ is open in
	$X$, then $U$ is open in $X$.
\end{lemma}
\begin{proof}
	Assume $U$ is open in $Y$. Then $U = Y \cap V$ for some $V \in
	\mathcal{T}_{X}$. Also assume $Y$ is open in $X$. Then $Y \in \mathcal{T}_{X}$.
	It follows that $U$ is formed via the intersection of two open sets in
	$X$. It follows that $U$ is open in $X$.
\end{proof}
\begin{remark}
	Let $A$ and $B$ be subspaces of topological spaces $X$ and $Y$, respectively.
	Observe that the product topology on $X \times Y$, restricted to sets consisting only
	of members in $A \times B$, coincides with the subspace topology
	on $A \times B$ induced from $X \times Y$.
\end{remark}
\begin{remark}
	Let $X$ be a set equipped with an order topology, and $Y \subset X$.
	Then the order topology on $Y$ need not be the same as the subspace topology on $Y$.
	\par
	For example, let $Y = (0,1) \cup \left\{ 2 \right\}$, and $X = \mathbf{R}$. The set $\left\{ 2 \right\} = (3/2, 5/2) \cap Y$ is open in the subspace
	topology. However, it is \emph{not} open in the order topology, since any
	interval in $X$ containing $\left\{2\right\}$ is not open in the induced order
	topology on $Y$.
	\par
	In general, the order topology induced on a set $Y \subset X$ is coarser than the
	subspace topology on $Y$. To see this, note that the induced order topology is
	generated by the restriction of elements in the basis of the order
	topology on $X$ to elements of $Y$. That is,
	$\mathcal{B}_{Y} = \mathcal{B}| Y$. Hence, if $U \in \mathcal{B}_{Y}$, then
	$U = V | Y$ for some $V \in \mathcal{B}$. But, we can also write $U = V \cap Y$,
	and so $U$ lives in the subspace topology as well.
	\par
	If $Y$ is a \emph{convex} subset of $X$, then the order topology and the subspace
	topology do coincide. Recall that a subset $Y \subset X$ with induced order
	topology from $X$ is \emph{convex} if for any points $a,b$ in $Y$ with $a < b$,
	the interval $(a,b)$ is in $Y$.
\end{remark}
\section{Closed Sets and Limit Points}
\subsection{Closed Sets}
\begin{definition}
	\label{def:closed-sets}
	We say that a subset $A$ of a topological space $X$ is \emph{closed} if
	$A^{c} \doteq X \setminus A$ is open.
\end{definition}
\begin{example} $ $
	\begin{enumerate}
		\item $[a,b] \subset \mathbf{R}$ is closed, since
			$[a,b]^{\complement} = (-\infty, a) \cup (b, \infty)$ is open.
		\item In the discrete topology on $X$, every set is open, which implies
			that every set is closed as well.
	\end{enumerate}
\end{example}
\begin{lemma}
	\label{lem:equiv-closed-def-top}
	Let $X$ be a topological space. Then
	\begin{enumerate}
		\item $\emptyset$ and $X$ are closed.
		\item Arbitrary intersections of closed sets are closed.
		\item Finite unions of closed sets are closed.
	\end{enumerate}
\end{lemma}
\begin{proof}
	Applying DeMorgan's laws to an arbitrary closed collection
	$\left\{ A_{\alpha} \right\}_{\alpha \in J}$, we obtain
	\begin{align*}
		\left[ \bigcap_{\alpha \in J}A_{\alpha} \right]^{\complement} & = \bigcup_{\alpha \in
		J} A_{\alpha}^{\complement} \\
		\left[ \bigcup_{\alpha \in J}A_{\alpha} \right]^{\complement} & = \bigcap_{\alpha \in
		J} A_{\alpha}^{\complement}
	\end{align*}
	Since $A_{\alpha}^{\complement}$ is open, and topologies are closed under
	unions and finite intersections, the result follows.
\end{proof}
\begin{remark}
	\label{rem:equiv-closed-def-top}
	It follows that we could have defined topological spaces in terms of closed
	sets, and defined open sets as their complement.
\end{remark}
\begin{definition}
	\label{def:closed-set-subspace}
	Let $Y$ be a subspace of a toplogical space $X$. Then we say $A$
	is \emph{closed} in $Y$ if $A$ is closed in the subspace topology of
	$Y$.
\end{definition}
\begin{lemma}
	\label{lem:closed-iff}
	Let $Y$ be a subspace of $X$. Then $A$ is closed in $Y$ if and only if
	it is the intersection of a closed set of $X$ with $Y$.
\end{lemma}
\begin{proof}
	Suppose $A$ is closed in $Y$. Then $A = (Y \cap U)^{\complement} = Y \setminus
	(Y \cap U)$ for some set $U$ that is open in $X$.
	But $Y \setminus (Y \cap U) = Y \cap (Y \cap U)^{\complement} = Y \cap
	U^{\complement}$.
\end{proof}
\begin{lemma}[Transitivity of Closures]
	Let $Y$ be a subspace of $X$. If $A$ is closed in $Y$ and $Y$ is closed in
	$X$, then $A$ is closed in $X$.
\end{lemma}
\begin{proof}
	It is analagous to the proof of \cref{lem:open-criteria-subspace}.
\end{proof}
\begin{example}
	$[1,2]$ is closed in $[0,3]$. But
	$[0,3]$ is closed in $\mathbf{R}$. Hence,
	$[1,t]$ is closed in $\mathbf{R}$.
\end{example}
\begin{definition}
	\label{def:interior}
	Let $A$ be a subset of a topological space $X$. The \emph{interior}
	$A^{\circ}$ of
	$A$ is defined to be the union of all open sets contained in $A$.
	The \emph{closure} $\bar{A}$ of $A$ is the intersection of all closed
	sets containing $A$.
\end{definition}
\begin{remark}
	\label{rem:intuition-closure}
	We can think of $\bar{A} \setminus A^{\circ}$ as the ``boundary''
	of $A$. Observe that if $A$ is open, $A = A^{\circ}$, and if $A$ is closed,
	$A = \bar{A}$.
\end{remark}
Next, we examine how closures propagate to subspaces.
\begin{theorem}
	\label{thm:closure-prop}
	Let $Y$ be a subspace of $X$, and $A \subset Y$. Then
	$\overline{A_{Y}} = \overline{A_{X}} \cap Y$.
\end{theorem}
\begin{proof}
	$\overline{A_{X}}$ is closed in $X$, so $\overline{A_{X}} \cap Y$ is closed in $Y$
	by \cref{lem:closed-iff}. But $A \subset \overline{A_{X}} \cap Y$, and by
	definition, $\overline{A_{Y}} = \bigcap\left\{ B \subset Y: B \supset A \
		\text{and} \
	B \ \text{is closed}\right\}$. Hence, $\overline{A_{Y}} \subset
	\overline{A_{X}} \cap Y$.
	\par
	To prove the reverse inclusion, we first note that
	$\overline{A_{Y}} = C \cap Y$ for some closed $C \subset Y$.
	It follows that $A \subset C$. Since $\overline{A_{X}}$ is the intersection of
	all closed sets in $X$, $\overline{A_{X}} \subset C$. Hence,
	$(\overline{A_{X}}\cap Y) \subset (C \cap Y) = \overline{A_{Y}}$.
\end{proof}
\begin{definition}
	\label{def:intersects}
	We say a set $A$ \emph{intersects} $B$ if $A \cap B \neq \emptyset$.
\end{definition}
\begin{theorem}
	\label{thm:closure-intersect-equivs}
	Let $A$ be a subset of a topological space $X$. Then we have the following:
	\begin{enumerate}
		\item $x \in \bar{A}$ $\iff$  every open set of $U$ containing
			$x$ intersects $A$.
		\item $x \in \bar{A}$ $\iff$ every basis element of $B$ containing
			$x$ intersects $A$.
	\end{enumerate}
\end{theorem}
\begin{example} $ $
	\begin{enumerate}
		\item If $X = \mathbf{R}, A = (0,1]$, then $\overline{A_{X}} = [0,1]$.
		\item If $B = \left\{ 1/n, n \in \mathbf{Z}_{+} \right\}$, then
			$\bar{B} = \left\{ 0 \right\} \cup B$.
		\item $\bar{Q}_{\mathbf{R}} = \mathbf{R}$
		\item Let $Y = (0,1]$, viewed as a subspace of $\mathbf{R}$. Then
			$\overline{(0,1/2)}_{\mathbf{R}} = [0,1/2]$, while
			$\overline{(0,1/2)}_{\mathbf{Y}} = [0,1/2]$
	\end{enumerate}
\end{example}

\subsection{Limit Points}
We have a different way of describing closure.
\begin{definition}
	If $A$ is a subset of a topological space $X$ and $x \in A$, we say $x$
	is a \emph{limit point} of $A$ if every neighborhood of $x$ intersects
	$A$ at some point other than $x$ itself. Equivalently,
	$x$ is a limit point if $x \in \overline{A \setminus \left\{ x \right\}}$
\end{definition}
\begin{example}
	Consider the interval \((0,1)\) as a subset of \(\mathbf{R}\). Then \(\bar{A} =
	[0,1]\), and \(\left\{ 0,1 \right\}\) is the set of limit points.
\end{example}
\begin{theorem}
	\label{thm:limit-points-def-closedness}
	Let \(A\) be a subset of a topological space \(X\), and let \(A'\) denote the
	set of limit points of \(A\). Then
	\begin{equation*}
		\bar{A} = A \cup A'.
	\end{equation*}
	\begin{proof}
		Observe that if \(x \in A'\), then, by definition, every neighborhood of
		\(x\) intersects \(A\setminus \left\{ x \right\}\). Hence, \(x \in
		\bar{A}\), by \cref{thm:closure-intersect-equivs}, and so
		\(A' \subset \bar{A}\), from which it follows that \(A \cup A' \subset
		\bar{A}\).
		\par
		To prove the reverse inclusion, we assume \(x \in \bar{A}\). If \(x \in A\),
		if \(x \in A\), then it is trivial that \(x \in A \cup \bar{A}\), so suppose
		\(x \not \in A\). Since \(x \in \bar{A}\), by
		\cref{thm:closure-intersect-equivs} every neighborhood \(U\) of \(x\)
		intersects \(A\). By assumption, \(x\) does not belong to the intersection,
	and so every neighborhood of \(x\) intersects \(A\setminus \left\{ x
	\right\} \). That is, \(x \in A'\), and so \(x \in A \cup A'\).
	\end{proof}
\end{theorem}
\begin{corollary}
	\label{cor:subset-closed-iff-lim-points}
	A subset of a topological space is closed if and only if it contains all of
	its limit points.
\end{corollary}
\begin{proof}
	A set \(A\) is closed if and only if \(A = \bar{A}\). But this is true
	if and only if \( A' \subset A\), since \(\bar{A} = A \cup A'\).
\end{proof}
\begin{definition}
	\label{def:conv-limit}
	We say a sequence \(\left\{ x_{n} \right\}_{n}\) of points in \(X\)
	\emph{converges} to a point \(x \in X\) if for each neighborhood
	\(U\) of \(x\), there exists \(N \equiv N(x, U)\) such that
	\(x_{n} \subset U\)
	for all \(n \ge N\).
\end{definition}
\begin{remark}
	\label{rem:uniqueness-conv-r}
	In the standard topology on \(\mathbf{R}\), a sequence converges to at most
	one point. However, in general, a sequence can have more than one limit.
	Consider  \(\mathbf{R}\), equipped with the trivial topology. Then the
	sequence \(\left\{ 1/2, 1/4, 1/8, \cdots \right\}\) converges to \emph{every}
	point in \(\mathbf{R}\). In fact, any sequence converges to every point in
	\(R\)! Heuristically, the finer a topology, the more it allows us to
	``distinguish'' one sequence from another, and the better the odds a sequence
	converges to at most one limit. However, if the topology
	is too strong, there is a chance a given sequence will not converge in it.
	Indeed, if the topology is strong enough, no sequence will converge.
	For example, no sequence in \(\mathbf{R}\) converges in the discrete
	topology.
\end{remark}
\begin{definition}
	\label{def:hausdorff}
	A topological space \(X\) is called \emph{Hausdorff} if it \emph{separates}
	points, i.e.\ for distinct points
	\(x_{1}, x_{2} \in X\), there exist disjoint neighborhoods \(U_{1}, U_{2}\)
	of \(x_{1}, x_{2}\), respectively. 
\end{definition}
\begin{theorem}
	\label{thm:hausdorff-limit-uniqueness}
	If \(X\) is a Hausdorff space, then a sequence of points in \(X\) converges
	to at most one point in \(X\).  
\end{theorem}
\begin{proof}
	Choose a sequence \(\left\{ x_{n} \right\}_{n} \subset X\) and
	distinct \(x,y\) in \(X\), and suppose \(x_{n} \to x\). Since \(X\)
	is Hausdorff, there exist open sets \(U, V\) separating \(x\) and \(y\),
	respectively. By the definition of limit points,
	there exists \(N \equiv N(U, x)\) such that
	\(x_{n} \in U\) for \(n \ge N\). Since \(U\) and \(V\) are disjoint, it
	follows that \(x_{n} \not \in V\) for \(n \ge N\), and so
	\(x_{n} \not \to y\). 
\end{proof}
\begin{theorem}
	\label{thm:hausdorff-inherit}
	We have the following:
	\begin{enumerate}
		\item A simply ordered set is Hausdorff.
		\item A product of Hausdorff spaces is Hausdorff.
		\item A subspace of a Hausdorff space is Hausdorff.
	\end{enumerate}
\end{theorem}
