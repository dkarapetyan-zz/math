\chapter{Set Theory}
\section{Sets}
Consider the following collection of objects
\begin{equation*}
\begin{split}
A & = \left\{ 1,2,3,\ldots \right\}, \,
B = \left\{ a, b, b, d \right\}, \,  \\
C & = \left\{ a,b,c,d,e \right\}, \,
D = \left\{ x: x \ \text{is an even integer} \right\}.
\end{split}
\end{equation*}
Note that $A$ have infinitely many objects, while $B$ and $C$ have finitely
many.
\begin{definition}
We call collections of infinitely many objects an \emph{infinite set}, and a
collection of finitely many objects a \emph{finite set}. Furthermore,
observe that $B$ nests in $C$. We say that $B$ is \emph{contained} in $C$,
or a \emph{subset} of $C$, and write $B \subset C$. Since $B$ does not nest
in $A$, it is \emph{not contained} in $A$, denoted $B \not \subset A$. It is
always true that a set is a subset of itself. Since $B$ is contained in $C$
and strictly smaller than $C$ (that is, not $C$ itself), we call $B$ a
\emph{proper subset} of $C$, denoted $B \subsetneq C$.

Using this notation, we define
\begin{equation*}
\begin{split}
A \cup B & = \left\{ x: x \in A \ \text{or} \ x \in B \right\} \\
A \bigcap B & = \left\{ x: x \in A \ \text{and} \ x \in B \right\} \\
\bigcup_{A \in \Omega} A & = \left\{ x: x \in A \ \text{for at least one A} \in \Omega
\right\} \\
\bigcap_{A \in \Omega} A & = \left\{ x: x \in A \ \text{for every A} \in \Omega \right\}
\\
A \times B & = \left\{ (a,b): a \in A \ \text{and} \ b \in B. \right\}
\\
A \times B \times C & = \left\{ (a,b,c): a \in A, b \in B,  \ \text{and} \ c \in C. \right\}
\end{split}
\end{equation*}
\end{definition}
\begin{example}
With $A = \{1,3,5\}$, $B  = \{2,4,5,6\}$,
and $C = \{2,4,5\}$, we have:
\begin{enumerate}
  \item $A \cup B = \{1,2,3,4,5,6\}$
  \item	$A \cap B = \{5\}$
 \item	$A \cap C  = \{\} \doteq \emptyset$
\item $A \setminus B  \doteq
\{ x: x \in A \ \text{and} \ x \not \in B\} $
\end{enumerate}
\end{example}
Observe that we don't have repeats in sets, and that order doesn't matter.
It is left as an exercise to prove the following property of sets:
\begin{lemma}
We have
\begin{equation*}
\begin{split}
A \cap (B \cup C) &= (A \cap B)  \cup (A \cap C) \\
A \cup (B \cap C) &= (A \cup B)  \cap (A \cup C)
\end{split}
\end{equation*}
\end{lemma}
\section{Logic}
\begin{definition}
If a hypothesis $P$ implies a conlcusion $Q$, then we say $P \implies Q$.
If statements $P \implies Q$ and $Q \implies P$ are both true, then we
write $P \iff Q$, and say that $P$ and $Q$ are \emph{equivalent}.
Furthermore, $P$ is true if and only if the logical complement $\neg P$ of $P$ is
false.
\end{definition}
\begin{lemma}[Contrapositive]\label{lem:contrapositive}
If $P \implies Q $, then $\neg Q \implies \neg P$. That is
\begin{equation*}
\begin{split}
( P \implies Q )  \implies \left( \neg Q  \implies \neg P\right)
\end{split}
\end{equation*}
\end{lemma}
\begin{example} $ $
\begin{enumerate}
  \item \text{$a$ is an integer $\implies a$ is a
  real number}
  \item \text{$a$ is a real number $\not \implies a$ is an
  integer. Counterexample: $a =4.5$}
\item A statement $P \implies Q$ and its contrapositive are equivalent.\end{enumerate}
\end{example}
\begin{corollary}
$\neg$ (For every $x \in A$, statement $p$ holds)
$\iff$ For at least one $x \in A$, statement $p$ does not hold.
\end{corollary}
\section{Functions}
We define a function $f: A \to B$ to be a rule of assignment. That is,
each point $a$ in the \emph{domain} $A$ is associated with a unique point in
the \emph{range} $B$. We denote this unique point $f(a)$, and say that
$a$ is \emph{mapped} to $b$. We define the \emph{image} of $f$ on $A$ to be
$f(A) \doteq \left\{ b \in B: b = f(a) \text{ for at least one } a \in A
\right\}$.
\begin{example}
General notation is to first specify the domain and range $f$ associates with each
other, followed by the association rule itself. For example,
\begin{equation*}
\begin{split}
& f: \mathbf{R} \to \mathbf{R}, \quad \text{(specify this line first)}\\
& f(x) = x^{3}.
\end{split}
\end{equation*}
\end{example}
We say $f$ is \emph{surjective} from $A$ to $B$ if its image on $A$ is all of
$B$. We say $f$ is \emph{injective} if $f(a) = f(b) \implies a = b$. That is,
for each $b \in B$, there exists only one $a \in A$ such that $f(a) = b$.
If $f$ is injective and surjective, we say that it is \emph{bijective}.
If $f$ is bijective, then we can define its \emph{inverse}
\begin{equation*}
\begin{split}
& f^{-1}: B \to A
\\
& f^{-1}(b) = a | f(a) = b
\end{split}
\end{equation*}
Observe that the inverse of $f$ is well-defined, since $f$ is injective (i.e. \
there is one and only one $a \in A$ such that $f(a) = b$). It follows that
an injective map $f$ from $A \to B$ may not be bijective; however, it is
bijective from $A$ to its image $f(A)$. If $B_{0} \subset B$ and $f: A \to B$ is
bijective, then its
\emph{pre-image} is defined to be $f^{-1}(B_{0}) = \left\{ a \in A: f(a) \in
B_{0} \right\}$.
\section{Relations}
\subsection{Introduction}

\begin{definition}
A \emph{relation} on a set $A$ is a subset $C \subset A \times A$.
If two elemebts $x, y \in A$ are related by $C$, then $(x,y) \in C$ and
we say $x$ is related to $y$, or $xCy$.
\end{definition}
\begin{example}
$\left\{ (x,y) \in \mathbf{R} \times \mathbf{R}: x<y \right\}$ is a relation on $\mathbf{R}$.
In fact, it is a special kind of relation, called the usual order relation.
\end{example}

\subsection{Order Relations}
\begin{definition}
A \emph{usual order relation } on a set $A$ is defined to have the following
properties:
\begin{enumerate}[(i)]
  \item{Comparability} For every $x,y \in A$ where $x \neq y$, either $xCy$ or $yCx$.
  \item{Non-reflexivity} For no $x \in A$ does the relation $xCx$ hold (non-reflexivity).
  \item{Transitivity} If $xCy$ and $yCz$, then $xCz$.
\end{enumerate}
General notation for the usual order relation on $\mathbf{R}$ is $<$.
\end{definition}
\begin{remark}\label{rem:}
Observe that equality is \emph{not} an ordering relation on $\mathbf{R}$. Non-reflexivity
fails.
\end{remark}
\begin{example}
On $\mathbf{R} \times \mathbf{R}$, define $xCy$ if $x^{2} < y^{2}$. It's an ordering
relation on $\mathbf{R}$ (with some bizarre results for negative numbers).
\end{example}
Observe that for any set $X$ equipped with an order relation $<$, we can
construct subsets of $X$ using the order relation. Indeed, two special
subsets are the \emph{open} and \emph{closed} intervals,
$(a,b)\doteq \left\{ x \in X: a < x < b \right\}$ and $[a,b] \doteq \left\{ x
\in X: a \le x \le b
\right\}$, respectively.

Let $A_{0}  \subset A$. We say $b \in A_{0}$ is the \emph{largest element} of
$A_{0}$ if $b \in A_{0}$ and $a \le b$ for all $a \in A_{0}$, and the
\emph{smallest element} if $a \ge b$ for all $a \in A_{0}$.

We say $A_{0}$ is \emph{bounded above} if there exists $x \in A$ such that
$a \le x$ for all $a \in A_{0}$, and \emph{bounded below} if there exists $y \in
A$ such that $a \ge y$ for all $a \in A_{0}$. We call $x$ and $y$ an \emph{upper
bound} and \emph{lower bound}, respectively. If a set of
upper bounds has a least element, we call it the \emph{supremum}of $a_{0}$, and
denote it $\sup A_{0}$. If a set of lower bounds has a greatest elment, we call
it the \emph{infimum} of $a_{0}$, and denote it $\inf A_{0}$. Note that neither
$\sup A_{0}$ nor $\inf A_{0}$ need belong to $A_{0}$.
\begin{example}
$ \sup (a,b) = b \not \in (a,b), \quad \inf (a,b) = a \not \in (a,b)$
\end{example}

\begin{definition}[Least Upper Bound Property]
An ordered set $A$ is said to have the \emph{least upper bound property} if for
every $\emptyset \neq A_{0} \subset A$ that is bounded above in $A$ has its
supremum in $A$. Similarly, it is said to have the \emph{greatest lower bound
property} if each $\emptyset \neq A_{0} \subset A$ that is bounded below in $A$ has
its infimum in $A$.
\end{definition}
\begin{example}
$B = (-1,0) \cup (0,1)$ has neither the the least upper bound nor the
greatest lower bound property. To see
this, consider $B_{0} = \{ -1/2n: n \in Z_{+}\} \subset B$. Then
$\sup B_{0} = 0 \not \in B$. Similarly, $\inf -B_{0} = 0 \not \in B$.
\end{example}

\subsection{Equivalence Relations}

If order relations can be seen as a generalization of the typical order relation
$\mathbf{R}$ enjoys, then the analogue for equality on $\mathbf{R}$ are \emph{equivalence
classes}.
\begin{definition}
An \emph{equivalence relation} on a set $A$ is defined to be a relation on $A$
with the following properties:
\begin{enumerate}[(i)]
  \item{Reflexivity} $xCx$ for every $a \in A$.
  \item{Symmetry} If $xCy$, then $yCx$.
  \item{Transitivity} If $xCy$ and $yCz$, then $xCz$.
\end{enumerate}
If $a \in A$ and $C$ is an equivalence relation on $A$, then
the subset $E_{x} \doteq \left\{ x \in A: xCa  \right\}$ is called
an \emph{equivalence class}.

We typically denote equivalence relations by $\sim$.
\end{definition}
\begin{lemma}\label{lem:}
Two equivalence classes $E_{x}$ and $E_{y}$ on a set $A$ are
either disjoint or equal.
\end{lemma}
\begin{proof}
Suppose $E_{x} \cap E_{y}$ is nonempty. Then there is some $z \in A$
such that $z \in E_{x} \cap E_{y}$. But then $z \in E_{x}$ and
$z \in E_{y}$. Hence, $x \sim z$ and $z \sim y$. By transitivity, it follows
that $x \sim y$. But this implies $E_{x} = E_{y}$ (Exercise).
\end{proof}
\begin{example}
Let $A \doteq C(\mathbf{R})$, and let $sim$ be the equivalence relation differs at
only finitely many points'. Then $E_{f} = {g \in C(\mathbf{R}): f=g \ \text{except at
finitely many points}}$.
\end{example}
Observe that equivalence relations ``break apart'' sets into disjoint chunks. We call
these chunks \emph{partitions}.
\begin{definition}
A \emph{partition} of a set $A$  is a collection of disjoint nonempty subsets of
$A$ such that their union is $A$.
\end{definition}
Observe that $\bigcup_{x \in A}  E_{x}  = A$ for any equivalence relation on
$A$. That is, equivalence relations are partitions. The converse is also true:
Given any partition on $A$, there is exactly one equivalence relation from
which it is derived (Exercise: prove this).

\section{Construction of the Real Numbers, and Integers from Real Numbers}
\begin{definition}[Construction of the Real Numbers]
Assume there exists a set called $\mathbf{R}$ with the binary operations $+: \mathbf{R} \times
\mathbf{R} \to \mathbf{R}$, $\cdot: \mathbf{R} \times \mathbf{R} \to \mathbf{R}$, and order relation $<$
such that the following properties hold for all $x,y,z \in \mathbf{R}$:
\begin{description}
\item[Algebraic Properties]\
\begin{enumerate}[(i)]
  \item{Associativity:} $(x + y) + z = x + (y+z), \quad (x \cdot y) \cdot z = x
  \cdot (y \cdot z)$
  \item{Symmetry:} $x + y = y + x, \quad x \cdot y = y \cdot x$
  \item{Existence of Identity:} There exists an identity for $+$ (denoted $0$) and
  an identity for $\cdot$ (denoted $1$). Recall that, by definition, $x \in U$
  is an identity for a binary operator $f: U\times U \to V$ if $f(x,y) = y$ for
  all $y \in U$.
  \item{Existence of Inverse:} There exists a unique inverse
  with respect to $+$ for each $x \in \mathbf{R}$, and a unique inverse, with respect
  to $\cdot$, for reach $0 \neq x \in \mathbf{R}$. Recall that, by definition,
  $y \in U$ is an inverse for $x \in \mathbf{R}$, with respect to a binary operator $f:
  U \times U \to V$, if an identity I exists in $\mathbf{R}$ and $f(x,y) = \mathrm{I}$.
  \item{Distributativity:} $x \cdot (y+z) = (x \cdot y) + (x \cdot z)$.
  Observe that distributativity is the only property above that belongs
  exclusively to $\cdot$.
\end{enumerate}
\item[Order Properties]\
\begin{enumerate}[(vi)]
  \item{Continuity:}
  If $x<y$, then there exists an element $z$ such that $x <z < y$.
  \item{Closure Under Limits:} $<$ has the least upper bound property.
  \item{Preservation of Order Under Binary Operation:} If $x < y$, then $x + z < y
  + z$. If $x < y$ and $0 < z$, then $x \cdot z < y \cdot z$.
\end{enumerate}
\end{description}
\end{definition}
\begin{definition}[The Construction of the Integers]
Let $S_{1} = \{1\}$,
$S_{2} = \{1, 1+1\}$, \dots, $S_{n} = \{1, 1+1, \dots, 1+1+1+\dots+1\}$
be subsets of our newly constructed $\mathbf{R}$. Then
\begin{equation*}
\begin{split}
\mathbf{Z}_{+} \doteq \bigcup_{i=1}^{\infty} S_{i}.
\end{split}
\end{equation*}
Let $\mathbf{Z}_{-1} \subset \mathbf{R}$ be the set of inverses of $\mathbf{Z}$. Then
\begin{equation*}
\begin{split}
\mathbf{Z} \doteq \mathbf{Z} \cup \mathbf{Z}_{-1} \cup \{0\}.
\end{split}
\end{equation*}
\end{definition}
\begin{definition}[Induction]
\end{definition}
\section{Cartesian Products}
Let $\Omega$ be a collection of sets. We ask: how do we list its elements?
If $\Omega$ is finitely large or ``countably large'', we can write
$\Omega = \{ A_{1}, A_{2}, \dots, A_{n}\}$, or
$\Omega = \{ A_{1}, A_{2}, \dots, A_{n}, \dots\}$, respectively.
Then we call $\{ 1,\dots, n\}$ and $\{1, \dots, n, \dots\}$ \emph{indexing
sets}.

However, what if $\Omega$ is uncountable? In this case, we find a set $J$ with
cardinality $|\Omega|$, and define, using the axiom of choice, a bijective \emph{indexing function} $f: J
\to \Omega$ where $f(\alpha) = A_{\alpha}$. That is, $f$ maps each element of
$J$ to a unique element of $\Omega$. Again invoking the axiom of choice,
we define
\begin{equation*}
\begin{split}
& \bigcup_{\alpha \in J} A_{\alpha} = \{ x: \text{for at least one $\alpha \in
J$, $x \in A_{\alpha}$}\} \\
& \bigcap_{\alpha \in J} A_{\alpha} = \{ x: \text{for every $\alpha \in
J$, $x \in A_{\alpha}$}\}
\end{split}
\end{equation*}
If $\Omega$ is a finite set or an infinite ``countable set'', i.e., like $\mathbf{Z}$,
we can define the \emph{cartesian product} of its elements as
\begin{equation*}
\begin{split}
& \prod_{i=1}^{m} A_{i} \doteq A_{1} \times A_{2} \times \dots \times A_{m} \\
& \prod_{i =1}^{\infty} A_{i} \doteq A_{1} \times A_{2} \times \dots \times
A_{m} \times \dots,
\end{split}
\end{equation*}
respectively. The elements of the finite and countable products are known as
$m$-tuples, and $\omega$-tuples, respectively.
If $\Omega$ is uncountable, then for an indexing set $J$ with the same
cardinality, we can define the cartesian product of the elements of $\Omega$
as
\begin{equation*}
\begin{split}
\prod_{\alpha \in J} A_{\alpha} = \left\{ f: J \to \bigcup_{\alpha} A_{\alpha} \mid
\forall \alpha \in J, \, f(\alpha) \in X_{\alpha}\right\}.
\end{split}
\end{equation*}
Observe that for this definition, the functions $f$ are analogues of the
$m$-tuples and $\omega$-tuples in the finite and countable cases.
We also definte the projection operator
\begin{equation*}
\begin{split}
& \pi_{\alpha}: \prod_{\alpha \in J} A_{\alpha} \to \bigcup_{\alpha}
A_{\alpha} \\
& \pi_{\alpha}(f) = f(\alpha)
\end{split}
\end{equation*}
In the finite and countable cases, this definition reduces to returning
the $\alpha$-th coordinate of a tuple.
\section{Finite Sets}
\begin{definition}
A set $A$ is called \emph{finite} is there exists a bijective
$f: A \to \{ 1, \dots, n\}$ for some $n$. If $A$ does not contain any
elements, then we say $A$ is \emph{empty}, and write $A \doteq \emptyset$.
\end{definition}
Observe that a bijective map from a finite $A$ to $\{1, \dots, n\}$ is not
unique; in fact, there are $n$! permutations of $\{1, \dots, n\}$, and so $n$! possible
bjiective maps.
\begin{theorem}\label{thm:subset-bij}
Assume there exists a bijective $f: A \to \{1, \dots, n\}$, where $n \in
\mathbf{Z}_{+}$. Let $B \subsetneqq A$. Then there does not exist a bijection $g: B
\to \{1, \dots, n\}$, but provided $B \neq \emptyset$, there exists a
bijection $h: B \to \{1, \dots, m\}$ for some $m < n$.
\end{theorem}
\begin{proof}
For the case $n=1$, we must have $B = \emptyset$, and so the theorem
trivially holds. For $n>1$, we let $g = f |_{B}$. Then $g: B \to f(B)$ is bijective. But
$f(B) = f(A) \setminus f(A \setminus B) = S \subsetneqq \{1, \dots, n\}$.
Hence, the elements of $S$ can be indexed: we give the first element the
$1$ index, the second element the $2$ index, and so on, down to the last
element, which we give the $m$ index, where $m < n$ is the size of $S$.
Then the function $h: S \to \{1, \dots, m\}$ given by $h(s_{i}) = i$
is a bijection. Then $h \circ g: B \to \{1, \dots, m\}$ is bijective.
\end{proof}
\begin{corollary}
If $B \subsetneqq A$, then $|B| < |A| < \infty$.
\end{corollary}
\begin{corollary}
Finite unions and finite cartesian product of finite sets are finite.
\end{corollary}
\begin{corollary}
The cardinality of a finite set $A$ is uniquely determined by $A$. That is,
if $f: A \to \{1, \dots, n\}$ is a bijection, there cannot exist a
$g: A \to \{ 1, \dots, m\}$, $m < n$, that is a bijection.
\end{corollary}
\section{Countable and Uncountable Sets}
\begin{definition}
A set is called \emph{countably infinite} if there exists a bijective
$f: A \to \mathbf{Z}_{+}$.
\end{definition}

\begin{claim}
$\mathbf{Z}$ is countable.
\end{claim}
\begin{proof} We define
\begin{equation*}
\begin{split}
& f: \mathbf{Z} \to \mathbf{Z}_{+}
\\
& f(n) = \begin{cases}
2n, \quad & n > 0 \\
2n+1, \quad & n \ge 0.
\end{cases}
\end{split}
\end{equation*}
This is a bijective function, with
\begin{equation*}
\begin{split}
f^{-1}(k) =
\begin{cases}
k/2, \quad & k \ \text{is even} \\
(1-k)/2, \quad & k \ \text{is odd}
\end{cases}
\end{split}
\end{equation*}
\end{proof}
\begin{claim}
$\mathbf{Z}_{+} \times \mathbf{Z}_{+}$ is countable.
\end{claim}
\begin{proof}Let $A = \left\{(x,y) \in \mathbf{Z}_{+} \times
Z_{+}: y \le x\right\}$
and define
\begin{alignat*}{2}
& f: \mathbf{Z}_{+} \times \mathbf{Z}_{+} \to A, \quad && g: A \to
\mathbf{Z}_{+} \\
& f(x,y) = (x + y-1, y),  \quad && g(x,y) = \frac{1}{2}(x-1)x + y.
\end{alignat*}
It is easy to check that $f$ and $g$ are bijective. Since the composition of
bijective functions is bijective, we conclude that $f \circ g: \mathbf{Z}_{+}
\times
\mathbf{Z}_{+} \to \mathbf{Z}_{+}$ is bijective.
\end{proof}

Observe that it has taken a lot of work to prove countability so far.
In order to simplify proofs of countability, we use the following theorem.
\begin{theorem}\label{thm:proof-countability}
Let $B \neq \emptyset$. Then the following are equivalent:
\begin{enumerate}
  \item\label{it:1} $B$ is countable
  \item\label{it:2} There exists a surjective $f: \mathbf{Z}_{+} \to B$
  \item\label{it:3} There exists an injective $g: B \to \mathbf{Z}_{+}$.
\end{enumerate}
\end{theorem}
To prove this, we shall need the following lemma:

\begin{lemma}\label{lem:subset_z_count}
If $A \subset \mathbf{Z}_{+}$, then $A$ is countable.
\end{lemma}
\begin{proof}[Proof of \cref{lem:subset_z_count}]
If $A$ is finite, the proof is trivial. Hence, assume that $A$ is infinite.
Our strategy will be to ``rewrite'' $A$ such that setting up a bijection
between it and $\mathbf{Z}_{+}$ will be easy.

Proceeding, set $a_1 = \inf A$, $a_{2} = \inf \{A \setminus {a_{1}}\}$,
$a_{3} = \inf \left\{ A \setminus \left\{ a_{1}, a_{2} \right\} \right\}$,
and recursively define $a_{n} = \inf \left\{ A \setminus \left\{
a_{1}, \dots,
a_{n-1} \right\}\right\}$. We claim that $\bigcup_{n} a_{n} = A$. By
construction, it is clear that $\bigcup_{n} a_{n} \subset A$. It remains to show
that $A \subset \bigcup_{n} a_{n} \doteq h(n)$. Pick $a \in A$. Since
$h(n)$ is a strictly increasing function, there exists a smallest $n$ such that
$h(n) > a$. It follows that
$a \not \in A \setminus\left\{ a_{1}, \dots, a_{n-1} \right\}$. It follows that
$a \in \left\{ a_{1}, \dots, a_{n-1} \right\}$, and so
$\bigcup_{n} a_{n} = A$.

It is clear that $f: B \to
\mathbf{Z}_{+}$ given by $f(a_{i}) = i$ is a bijection.

\end{proof}

\begin{proof}[Proof of \cref{thm:proof-countability}]
We use a standard idea when proving equivalences. It is enough to show that
\[~\eqref{it:1}~\implies~\eqref{it:2}~\implies~\eqref{it:3}~\implies~\eqref{it:1}.\]

To prove that~\eqref{it:1}~$\implies$~\eqref{it:2}, we observe that by
definition, if $B$ is countably infinite, there exists a bijective
(and, hence, surjective) map  $f: \mathbf{Z}_{+} \to B$.
If $B$ is finite, then by definition there exists a bijective map
$f: \left\{ 1, \dots, n \right\} \to B$. We extend this map to a surjective
map $\bar{f}: \mathbf{Z}_{+} \to B$ by definining $\bar{f} \equiv f$ on $B$,
and $\bar{f}(x) = 1$ for all $x \in \mathbf{Z}_{+} \setminus \left\{ 1, \dots, n
\right\}$.

To prove that~\eqref{it:2}~$\implies$~\eqref{it:3}, let
$f:\mathbf{Z}_{+} \to B$ be surjective. Define $g: B \to \mathbf{Z}_{+}$
where $g(b) = \inf \{f^{-1}(b) \}$. Observe that the infimum is always
realized at some $c \in \mathbf{Z}_{+}$. Since $f$ is a well-defined
function, it follows that $g$ must be injective.

Lastly, to prove that~\eqref{it:3}~$\implies$~\eqref{it:1}, we first note that
if $B$ is finite, we are done. We assume that $B$ is infinite. Since $g$ is
injective, $g(B) \subset \mathbf{Z}_{+}$ is infinite. Applying
\cref{lem:subset_z_count} concludes the proof.

\end{proof}

\begin{corollary}
The subset of a countable set is countable.
\end{corollary}

To see how useful \cref{thm:proof-countability} is, we note that it implies that
$\mathbf{Z}_{+} \times \mathbf{Z}_{+}$ is countably infinite, via the injectiveness of $f:
\mathbf{Z}_{+} \times \mathbf{Z}_{+} \to \mathbf{Z}$ given by $f(n,m) = 2^{n} 3^{m}$. Note that this
map is \emph{not} surjective, but that this is inconsequential, thanks to
\cref{thm:proof-countability}.

Similarly, $\mathbb{Q}_{+}$ is countably infinite, via the injectiveness
of $f: \mathbf{Z}_{+} \to \mathbf{Z}_{+}$ given by $g(n,m) = n/m$.
\begin{theorem}\label{thm:cantor}
The set $\{ 0,1 \}^{\omega}
\doteq \left\{ 0,1 \right\} \times
\left\{ 0,1 \right\} \times \dots$ is uncountable.
\end{theorem}
\begin{proof}
Our proof utilizes Cantor's famous diagonalization argument.
Suppose $g: Z_{+} \to \left\{ 0,1 \right\}^{\omega}$. We want to show that
$g$ can't be surjective.

Observe that $g(\mathbf{Z}_{+}) = \left\{ g(1), g(2), \dots, g(k), \dots \right\}$.
Let $g(k)_{k}$ denote the $kth$ coordinate of $g(k)$, and define
$y = (y_{1}, y_{2}, \dots, y_{n}, \dots) \in \left\{ 0,1 \right\}^{\omega}$,
where
\begin{equation*}
\begin{split}
y_{i} = g(i)_{i}^{\mathsf{c}}.
\end{split}
\end{equation*}
Intuitively, if we view $g(\mathbf{Z}_{+})$ as an infinite sized matrix with rows
$g(i)$, then $y$ is the bitwise complement of its diagonal. It follows that
$y \neq g(i)$, for any $i$, since it differs from $g(i)$ in at least one
coordinate (namely, its coordinate on the diagonal of the matrix
$g(\mathbf{Z}_{+})$).
\end{proof}
\begin{corollary}
For any infinite set $A$, its power set $\mathrm{P}(A)$ is uncountable.
\end{corollary}
\begin{proof}
Without loss of generality, we can assume $A$ is countable. That is, if $A$
were uncountable, then using the axiom of choice we could construct a
countable subset $\left\{ a_{i} \right\}_{i=1}^{\infty}$. Since
$P(\left\{ a_{i} \right\}_{i=1}^{\infty}) \subset P(A)$, proving the theorem
would reduce to the case where $A$ was countable.

Proceeding, we assume $A = \left\{ a_{i} \right\}_{i}^{\infty}$.
Then there exists a bijective map $f: P(A) \to \left\{ 0,1 \right\}^{\omega}$,
where
\begin{equation*}
\begin{split}
\pi_{i}(f(S)) = \begin{cases}
1 \ \text{if} \  a_{i} \in S \\
0 \ \text{if} \ a_{i} \not \in S.
\end{cases}
\end{split}
\end{equation*}
Assume $P(A)$ is countable. Then there exists a bijection $g: P(A) \to
\mathbf{Z}_{+}$. But then $g\circ f^{-1}: \left\{ 0,1 \right\}^{\omega} \to
\mathbf{Z}_{+}$ is bijective, contradicting \cref{thm:cantor}.
We conclude that $P(A)$ is uncountable.
\end{proof}
\section{Axiom of Choice}
\begin{theorem}
Let A be a set. Then the following are equivalent:
\begin{enumerate}
  \item\label{it:inj} There exists an injective $f: \mathbf{Z}_{+} \to A$.
  \item\label{it:bij-prop} There exists a bijection of $A$ with a proper subset of itself.
  \item\label{it:inf} $A$ is infinite
\end{enumerate}
\label{thm:inj-mapping}
\end{theorem}
\begin{proof}
To prove \cref{it:inj} $\implies$ \cref{it:bij-prop}, we assume there exists an injective $f: \mathbf{Z}_{+} \to A$, where $f(n) =
a_{n}$, and let $B = f(\mathbf{Z}_{+})$. Define
\begin{equation*}
\begin{split}
& g: A \to A \setminus \left\{ a_{1} \right\} \\
& g(x) = \begin{cases}
a_{n+1}, \quad x = a_{n} \in B \\
x, \quad x \in A \setminus B.
\end{cases}
\end{split}
\end{equation*} It is easy to show that this is a bijection.

To prove \cref{it:bij-prop} $\implies$ \cref{it:inf}, we shall prove the
contrapositive. Assume $A$ is finite. Then any proper subset of $A$ will have
fewer elements than $A$. Hence, any map from $A$ to this proper subset must
have at least two elements from $A$ that map to the same output, violating
injectivity.

To prove \cref{it:inf} $\implies$ \cref{it:inj}, we assume $A$ is infinite.
If $A$ is countable, then by the definition of countability, we are done.
Hence, we assume $A$ is uncountable. We would like to construct our injection
$f: \mathbf{Z}_{+} \to A$ by indexing the values of $A$; however, there does
\emph{not} exist a bijection from $\mathbf{Z}_{+}$ to $A$! What will our starting
point be for our recursive algorithm from which we ``build'' $f$? How do we
choose it?
\begin{remark}[Axiom of Choice]
Given a collection $\Omega$ of disjoint nonempty sets, we assume that there exists a set $C$ consisting of exactly one element from each element of $\Omega$.
\end{remark} Invoking the axiom of choice, we recursively define
\begin{equation*}
\begin{split}
& f: \mathbf{Z}_{+} \to A \\
& f(n) = \begin{cases}
a_{1}, & \quad n = 1 \\
a_{n} \in A \setminus
\bigcup_{i=1}^{n-1} a_{i}, & \quad n > 1.\end{cases}
\end{split}
\end{equation*}
This function is injective, completing the proof.
\end{proof}
\section{Well-Ordered Sets}
\begin{definition}
A set $A$ equipped with an order relation $<$ is \emph{well-ordered}
if every non-empty subset of $A$ has a smallest element.
\label{def:well-ordered}
\end{definition}

\begin{example}
$(\mathbf{Z}_{+}, <)$ is well-ordered.
\end{example}
\begin{example}
$(\left\{ 1,2 \right\} \times \mathbf{Z}_{+}, <_{\text{dict}})$ is well-ordered,
where $<_{\text{dict}}$ denotes \emph{dictionary order}. That is,
$(a,b) < (a', b')$ if either $a < a'$ or, in the case of equality,
$b < b'$.
\end{example}
\begin{example}
Any $A \subset \left( \mathbf{Z}_{+} \right)^{n}$ is well-ordered, for any $n \ge 1$, under
dictionary order.
\end{example}
\begin{remark}
$(\mathbf{Z}_{+})^{\omega}$ is \emph{not} well-ordered under dictionary order.
To see this, note that $A = (\mathbf{Z}_{+})^{\omega} \setminus (1, 1, \dots, 1,
\dots)$ does not have a least element. To prove this, assume that it does
have a least element, which we denote $\tilde{a}$.
Then it must have a $2$ in its $i$ coordinate, for some $i$.
But then there exists $a \in A$ with a $1$ in the $i-1$ coordinate, and
so $a < \tilde{a}$, a contradiction.
\end{remark}
\begin{theorem}
Every non-empty finite ordered set has the order type of a section of
$\mathbf{Z}_{+}$, and hence is well-ordered.
\end{theorem}
\begin{proof}
Since $A$ is finite, it has $n$ elements. One of these is the smallest element
under the $A$ order relation, which we label $a_{1}$. Then $a_{2} = \min\left\{
A \setminus a_{1} \right\}$ is the next smallest, and, in general, $a_{i} = \min
\left\{ A \setminus \left\{ a_{1}, \dots, a_{i} \right\} \right\}$ is the $i$-th
smallest. There exists a natural bijection $f: \left\{ a_{1}, \dots, a_{n}
\right\} \to \mathbf{Z}_{+}$, given by $f(a_{i}) = i$, that is order-preserving,
and so $A, <_{A}$ can be viewed as a slice of $(\mathbf{Z}_{+}, <)$.
\end{proof}
We saw that $(\mathbf{Z}_{+})^{\omega}$ is not well-ordered via dictionary
order.  However, this leads us to a natural question: is there an order
relation that makes it well-ordered?

\begin{theorem}[Zermelo]
For every set $A$, there exists a well-ordering.
\end{theorem}
The proof of this is outside the scope of this course. However, it is equivalent
to the axiom of choice.
\begin{corollary}
There exists an uncountable, well-ordered set.
\end{corollary}
\section{Maximum Principle}
We saw the the axiom of choice is equivalent to the the statement that
every set can be made well-ordered. In fact, it is also equivalent to the
maximum principle, which we will discuss more of later after suitable
preparation.
\begin{definition}
Given a set $A$, a relation $<$ on $A$ is called a
\emph{strict simple ordering} or \emph{strict total ordering}
on $A$ if it obeys
\begin{enumerate}
  \item{Non-reflexivity} The relation $a < a$ never holds.
  \item{Transitivity} If $a < b$ and $b < c$, then $a < c$.
  \item{Totality} Either $a < b$ or $b < a$ for every $a,b \in A$.
\end{enumerate}
If totality does not hold, we say $<$ is a \emph{strict partial ordering}.
\end{definition}
\begin{theorem}[Maximum Principle]
Let $A$ be a set equipped with a strict partial order $<$. Then there exists $B
\subset A$ such that $B$ is strictly totally ordered under $<$ restricted to
$B$, and such that for any $C \subset A$ of which $B$ is a proper subset, $C$
is not totally ordered under $<$.
\end{theorem}
More important is Zorn's lemma, which is equivalent to both the maximum
principle and the axiom of choice.
\begin{definition}
\label{def:zorn-def}
Let $A$ be a set equipped with a strict partial order $<$. If $B \subset A$,
we we say $c \in A$ is an \emph{upper bound} on $B$ if for every $b \in B$,
either $b = c$ or $b < c$. The \emph{maximal element} of $A$ is an element
$m \in A$ such that for no element $a$ of $A$ does the relation $m < a$ hold.
\end{definition}
\begin{theorem}[Zorn's Lemma]
Let $A$ be a set that is strictly partially ordered. If every totally ordered
subset of $A$ has an upper bound in $A$, then $A$ has a maximal element.
\end{theorem}
