%!TEX root = ../topology.tex

\chapter{Set Theory}
\section{Sets}
Consider the following collection of objects
\begin{equation*}
\begin{split}
	A & = \left\{ 1,2,3,\ldots \right\} \\
	B &= \left\{ a, b, b, d \right\} \\
	C & = \left\{ a,b,c,d,e \right\}.
\end{split}
\end{equation*}
Note that $A$ have infinitely many objects, while $B$ and $C$ have finitely
many. 
\begin{definition}
We call collections of infinitely many objects an \emph{infinite set},
and a collection of finitely many objects a \emph{finite set}. 
Furthermore, observe that $B$ nests in $C$. We say that $B$ is
\emph{contained} in $C$, or a \emph{subset} of $C$, and write
$B \subset C$. Since $B$ does not nest in $A$, it is \emph{not contained} in
$A$, denoted $B \not \subset A$. It is always true that a set is a subset of
itself. Since $B$ is
contained in $C$ and strictly smaller
than $C$ (that is, not $C$ itself), we call $B$ a \emph{proper subset} of $C$, denoted $B \subsetneq C$.
\end{definition}
We can also specify sets as follows:
\begin{equation*}
\begin{split}
	B = \left\{ x: x \text{ is an even integer} \right\}.
\end{split}
\end{equation*}
\begin{notation}
Using this notation, we define
\begin{equation*}
\begin{split}
	A \cup B & = \left\{ x: x \in A \text{ or } x \in B \right\} \\
  A \bigcap B & = \left\{ x: x \in A \text{ and } x \in B \right\} \\
	\bigcup_{A \in \Omega} A & = \left\{ x: x \in A \text{ for at least one A} \in \Omega
	\right\} \\
\bigcap_{A \in \Omega} A & = \left\{ x: x \in A \text{ for every A} \in \Omega \right\}
\end{split}
\end{equation*}
\end{notation}
\begin{example}
\begin{equation*}
\begin{split}
	A & = \{1,3,5\} \\
	B & = \{2,4,5,6\} \\
	A \cup B &= \{1,2,3,4,5,6\} \\
	A \cap B & = \{5\} \\
	C & = \{2,4,5\} \\
	A \cap C & = \{\} \doteq \emptyset \\
	A \setminus B & \doteq \{ x: x \in A \text{ and } x \not \in B\}
\end{split}
\end{equation*}
Observe that we don't have repeats in sets, and that order doesn't matter.

It is left as an exercise to prove the following property of sets:
\begin{lemma}
\begin{equation*}
\begin{split}
	A \cap (B \cup C) &= (A \cap B)  \cup (A \cap C) \\
	A \cup (B \cap C) &= (A \cup B)  \cap (A \cup C) 
\end{split}
\end{equation*}
\end{lemma}

\end{example}
\section{Logic} 
\begin{notation}
	If a hypothesis $P$ implies a conlcusion $Q$, then we say $P \implies Q$.
\end{notation}
\begin{lemma}[Contrapositive]\label{lem:contrapositive}
	If $P \implies Q $ , then $ \text{not } Q \implies \text{not } P$. That is
	\begin{equation*}
	\begin{split}
		( P \implies Q )  \implies \left( (\text{not } Q)  \implies \text{not }
		P\right) 
	\end{split}
	\end{equation*}
\end{lemma}
\begin{corollary}
	If $ (\text{ not } Q)  \implies (\text{not }  P ) $ , then
	\begin{equation*}
	\begin{split}
		\left[  \text{not } Q   \implies \text{not }  P\right] \implies (P
		\implies Q) 
	\end{split}
	\end{equation*}
\end{corollary}

