%!TEX root = ../topology.tex

\chapter{Set Theory}
\section{Sets}
Consider the following collection of objects
\begin{equation*}
\begin{split}
	A & = \left\{ 1,2,3,\ldots \right\}, \,
	B = \left\{ a, b, b, d \right\}, \,  \\
	C & = \left\{ a,b,c,d,e \right\}, \,
  D = \left\{ x: x \text{ is an even integer} \right\}.
\end{split}
\end{equation*}
Note that $A$ have infinitely many objects, while $B$ and $C$ have finitely
many. 
\begin{definition}
We call collections of infinitely many objects an \emph{infinite set},
and a collection of finitely many objects a \emph{finite set}. 
Furthermore, observe that $B$ nests in $C$. We say that $B$ is
\emph{contained} in $C$, or a \emph{subset} of $C$, and write
$B \subset C$. Since $B$ does not nest in $A$, it is \emph{not contained} in
$A$, denoted $B \not \subset A$. It is always true that a set is a subset of
itself. Since $B$ is
contained in $C$ and strictly smaller
than $C$ (that is, not $C$ itself), we call $B$ a \emph{proper subset} of $C$, denoted $B \subsetneq C$.
\end{definition}
\begin{notation}
Using this notation, we define
\begin{equation*}
\begin{split}
	A \cup B & = \left\{ x: x \in A \text{ or } x \in B \right\} \\
  A \bigcap B & = \left\{ x: x \in A \text{ and } x \in B \right\} \\
	\bigcup_{A \in \Omega} A & = \left\{ x: x \in A \text{ for at least one A} \in \Omega
	\right\} \\
\bigcap_{A \in \Omega} A & = \left\{ x: x \in A \text{ for every A} \in \Omega \right\}
\\
 A \times B & = \left\{ (a,b): a \in A \text{ and } b \in B. \right\}
\\
A \times B \times C & = \left\{ (a,b,c): a \in A, b \in B,  \text{ and } c \in C. \right\}
\end{split}
\end{equation*}
\end{notation}
\begin{example}
\begin{equation*}
\begin{split}
	A & = \{1,3,5\} \\
	B & = \{2,4,5,6\} \\
	A \cup B &= \{1,2,3,4,5,6\} \\
	A \cap B & = \{5\} \\
	C & = \{2,4,5\} \\
	A \cap C & = \{\} \doteq \emptyset \\
	A \setminus B & \doteq \{ x: x \in A \text{ and } x \not \in B\} \\
	\rr^{2} & = \rr \times \rr \\
	\rr^{3} & = \rr \times \rr \times \rr
\end{split}
\end{equation*}
\end{example}
Observe that we don't have repeats in sets, and that order doesn't matter.
It is left as an exercise to prove the following property of sets:
\begin{lemma}
\begin{equation*}
\begin{split}
	A \cap (B \cup C) &= (A \cap B)  \cup (A \cap C) \\
	A \cup (B \cap C) &= (A \cup B)  \cap (A \cup C) 
\end{split}
\end{equation*}
\end{lemma}
\section{Logic} 
\begin{notation}
	If a hypothesis $P$ implies a conlcusion $Q$, then we say $P \implies Q$.
\end{notation}
\begin{lemma}[Contrapositive]\label{lem:contrapositive}
	If $P \implies Q $ , then $ \text{not } Q \implies \text{not } P$. That is
	\begin{equation*}
	\begin{split}
		( P \implies Q )  \implies \left( \text{not } Q  \implies \text{not }
		P\right) 
	\end{split}
	\end{equation*}
\end{lemma}
\begin{corollary}
	If $ \text{ not } Q  \implies \text{not }  P  $ , then
	\begin{equation*}
	\begin{split}
		\left(  \text{not } Q   \implies \text{not }  P\right) \implies (P
		\implies Q) 
	\end{split}
	\end{equation*}
\end{corollary}
\begin{example} 
	\begin{equation*}
	\begin{split}
		& a \text{ is an integer} \implies  \text{ is a real number} \\
		& a \text{ is a real number } \not \implies  \text{ is an integer.}
		\text{Counterexample:} a =4.5
	\end{split}
	\end{equation*}
\end{example}
\begin{notation}
	If statements $P \implies Q$ and $Q \implies P$ are both true, then we 
	write $P \iff Q$, and say that $P$ and $Q$ are \emph{equivalent}.
\end{notation}
\begin{example}
	A statement $P \implies Q$ and its contrapositive are equivalent.
\end{example}
We now discuss logical negation. Observe that
\begin{equation*}
\begin{split}
	& \text{not(For every $x \in A$, statement $p$ holds)} 
	\\
	& \iff \text{For at least one $x \in A$, statement $p$ does not hold.}
\end{split}
\end{equation*}
\section{Functions}
We define a function $f: A \to B$ to be a rule of assignment. That is,
each point $a$ in the \emph{domain} $A$ is associated with a unique point in
the \emph{range} $B$. We denote this unique point $f(a)$, and say that
$a$ is \emph{mapped} to $b$. We define the \emph{image} of $f$ on $A$ to be
$f(A) \doteq \left\{ b \in B: b = f(a) \text{ for at least one } a \in A
\right\}$.
\begin{example}
	General notation is to first specify the domain and range $f$ associates with each
	other, followed by the association rule itself. That is, as an example,
	\begin{equation*}
	\begin{split}
	& f: \rr \to \rr, \quad \text{(specify this line first)}\\
	& f(x) = x^{3}.
	\end{split}
	\end{equation*}
\end{example}
We say $f$ is \emph{surjective} from $A$ to $B$ if its image on $A$ is all of
$B$. We say $f$ is \emph{injective} if $f(a) = f(b) \implies a = b$. That is,
for each $b \in B$, there exists only one $a \in A$ such that $f(a) = b$.
If $f$ is injective and surjective, we say that it is \emph{bijective}.
If $f$ is bijective, then we can define its \emph{inverse} 
\begin{equation*}
\begin{split}
	& f^{-1}: B \to A
	\\
	& f^{-1}(b) = a | f(a) = b
\end{split}
\end{equation*}
Observe that the inverse of $f$ is well-defined, since $f$ is injective (i.e.
there is one and only one $a \in A$ such that $f(a) = b$). It follows that
an injective map $f$ from $A \to B$ may not be bijective; however, it is
bijective from $A$ to its image $f(A)$. If $B_{0} \subset B$ and $f: A \to B$ is
bijective, then its
\emph{pre-image} is defined to be $f^{-1}(B_{0}) = \left\{ a \in A: f(a) \in
B_{0} \right\}$.
\section{Relations}
\subsection{Introduction} 

\begin{definition}
	A \emph{relation} on a set $A$ is a subset $C \subset A \times A$.
	If two elemebts $x, y \in A$ are related by $C$, then $(x,y) \in C$ and 
	we say $x$ is related to $y$, or $xCy$.
\end{definition}
\begin{example}
	$\left\{ (x,y) \in \rr \times \rr: x<y \right\}$ is a relation on $\rr$.
	In fact, it is a special kind of relation, called the usual order relation.
\end{example}

\subsection{Order Relations} 
\begin{definition}
 A \emph{usual order relation } on a set $A$ is defined to have the following
 properties:
 \begin{enumerate}[(i)]
	 \item{Comparability} For every $x,y \in A$ where $x \neq y$, either $xCy$ or $yCx$.
	 \item{Non-reflexivity} For no $x \in A$ does the relation $xCx$ hold (non-reflexivity).
		 \item{Transitivity} If $xCy$ and $yCz$, then $xCz$.
 \end{enumerate}
\end{definition}
\begin{remark}\label{rem:}
	Observe that equality is \emph{not} an ordering relation on $\rr$. Non-reflexivity
	fails.
\end{remark}
\begin{example}
	On $\rr \times \rr$, define $xCy$ if $x^{2} < y^{2}$. It's an ordering
	relation on $\rr$ (with some bizarre results for negative numbers).
\end{example}
\begin{notation}
	General notation for the usual order relation on $\rr$ is $<$.
\end{notation}
Observe that for any set $X$ equipped with an order relation $<$, we can
construct subsets of $X$ using the order relation. Indeed, two special
subsets are the \emph{open} and \emph{closed} intervals,
$(a,b)\doteq \left\{ x \in X: a < x < b \right\}$ and $[a,b] \doteq \left\{ x
	\in X: a \le x \le b
\right\}$, respectively.

Let $A_{0}  \subset A$. We say $b \in A_{0}$ is the \emph{largest element} of
$A_{0}$ if $b \in A_{0}$ and $a \le b$ for all $a \in A_{0}$, and the 
\emph{smallest element} if $a \ge b$ for all $a \in A_{0}$.

We say $A_{0}$ is \emph{bounded above} if there exists $x \in A$ such that
$a \le x$ for all $a \in A_{0}$, and \emph{bounded below} if there exists $y \in
A$ such that $a \ge y$ for all $a \in A_{0}$. We call $x$ and $y$ an \emph{upper
bound} and \emph{lower bound}, respectively. If a set of
upper bounds has a least element, we call it the \emph{supremum}of $a_{0}$, and
denote it $\sup A_{0}$. If a set of lower bounds has a greatest elment, we call
it the \emph{infimum} of $a_{0}$, and denote it $\inf A_{0}$. Note that neither
$\sup A_{0}$ nor $\inf A_{0}$ need belong to $A_{0}$.
\begin{example}
	$ \sup (a,b) = b \not \in (a,b), \quad \inf (a,b) = a \not \in (a,b)$
\end{example}

\begin{definition}[Least Upper Bound Property]
An ordered set $A$ is said to have the \emph{least upper bound property} if for
every $\emptyset \neq A_{0} \subset A$ that is bounded above in $A$ has its
supremum in $A$. Similarly, it is said to have the \emph{greatest lower bound
property} if each $\emptyset \neq A_{0} \subset A$ that is bounded below in $A$ has 
its infimum in $A$.
\end{definition}
\begin{example}
	$B = (-1,0) \cup (0,1)$ has neither the the least upper bound nor the
	greatest lower bound property. To see
	this, consider $B_{0} = \{ -1/2n: n \in Z_{+}\} \subset B$. Then
	$\sup B_{0} = 0 \nin B$. Similarly, $\inf -B_{0} = 0 \nin B$.
\end{example}

\subsection{Equivalence Relations} 

If order relations can be seen as a generalization of the typical order relation
$\rr$ enjoys, then the analogue for equality on $\rr$ are \emph{equivalence
classes}.
\begin{definition}
	An \emph{equivalence relation} on a set $A$ is defined to be a relation on $A$
	with the following properties:
	\begin{enumerate}[(i)]
		\item{Reflexivity} $xCx$ for every $a \in A$.
		\item{Symmetry} If $xCy$, then $yCx$.
		\item{Transitivity} If $xCy$ and $yCz$, then $xCz$.
	\end{enumerate}
	If $a \in A$ and $C$ is an equivalence relation on $A$, then 
	the subset $E_{x} \doteq \left\{ x \in A: xCa  \right\}$ is called
	an \emph{equivalence class}.
\end{definition}
\begin{notation}
	We typically denote equivalence relations by $\sim$.
\end{notation}
\begin{lemma}\label{lem:}
	Two equivalence classes $E_{x}$ and $E_{y}$ on a set $A$ are 
	either disjoint or equal.
\end{lemma}
\begin{proof}
	Suppose $E_{x} \cap E_{y}$ is nonempty. Then there is some $z \in A$
	such that $z \in E_{x} \cap E_{y}$. But then $z \in E_{x}$ and
	$z \in E_{y}$. Hence, $x \sim z$ and $z \sim y$. By transitivity, it follows
	that $x \sim y$. But this implies $E_{x} = E_{y}$ (Exercise).		
\end{proof}
\begin{example}
	Let $A \doteq C(\rr)$, and let $sim$ be the equivalence relation 'differs at
	only finitely many points'. Then $E_{f} = {g \in C(\rr): f=g \text{ except at
	finitely many points}}$.
\end{example}
Observe that equivalence relations ``break apart'' sets into disjoint chunks. We call
these chunks \emph{partitions}.
\begin{definition}
A \emph{partition} of a set $A$  is a collection of disjoint nonempty subsets of
$A$ such that their union is $A$.
\end{definition}
Observe that $\bigcup_{x \in A}  E_{x}  = A$ for any equivalence relation on
$A$. That is, equivalence relations are partitions. The converse is also true:
Given any partition on $A$ , there is exactly one equivalence relation from
which it is derived (Exercise: prove this).

\section{Construction of the Real Numbers, and Integers from Real Numbers} 
\begin{definition}[Construction of the Real Numbers]
Assume there exists a set called $\rr$ with the binary operations $+: \rr \times
\rr \to \rr$, $\cdot: \rr \times \rr \to \rr$, and order relation $<$
such that the following properties hold for all $x,y,z \in \rr$:
\begin{description}
	\item[Algebraic Properties]\
\begin{enumerate}[(i)]
	\item{Associativity:} $(x + y) + z = x + (y+z), \quad (x \cdot y) \cdot z = x
		\cdot (y \cdot z)$
\item{Symmetry:} $x + y = y + x, \quad x \cdot y = y \cdot x$
\item{Existence of Identity:} There exists an identity for $+$ (denoted $0$) and
	an identity for $\cdot$ (denoted $1$). Recall that, by definition, $x \in U$
	is an identity for a binary operator $f: U\times U \to V$ if $f(x,y) = y$ for
	all $y \in U$.
\item{Existence of Inverse:} There exists a unique inverse
	with respect to $+$ for each $x \in \rr$, and a unique inverse, with respect
	to $\cdot$, for reach $0 \neq x \in \rr$. Recall that, by definition,
	$y \in U$ is an inverse for $x \in \rr$, with respect to a binary operator $f:
	U \times U \to V$, if an identity I exists in $\rr$ and $f(x,y) = \mathrm{I}$.  
\item{Distributativity:} $x \cdot (y+z) = (x \cdot y) + (x \cdot z)$.
Observe that distributativity is the only property above that belongs
exclusively to $\cdot$.
\end{enumerate}
\item[Order Properties]\
	\begin{enumerate}[(vi)]
\item{Continuity:}
If $x<y$, then there exists an element $z$ such that $x <z < y$.
\item{Closure Under Limits:} $<$ has the least upper bound property.
\item{Preservation of Order Under Binary Operation:} If $x < y$, then $x + z < y
	+ z$. If $x < y$ and $0 < z$, then $x \cdot z < y \cdot z$.
\end{enumerate}
\end{description}
\end{definition}
\begin{definition}[The Construction of the Integers]
Let $S_{1} = \{1\}$,
$S_{2} = \{1, 1+1\}, \cdots, S_{n} = \{1 + 1+1, 1+1+1, ..., 1+1+1...+1\}$
be subsets of our newly constructed $\rr$. Then
\begin{equation*}
\begin{split}
\zz_{+} \doteq \bigcup_{i=1}^{\infty} S_{i}.
\end{split}
\end{equation*}
Let $\zz_{-1} \subset \rr$ be the set of inverses of $\zz$. Then
\begin{equation*}
\begin{split}
	\zz \doteq \zz \cup \zz_{-1} \cup \{0\}.
\end{split}
\end{equation*}
\end{definition}
\begin{definition}[Induction]
\end{definition}
\section{Cartesian Products} 
Let $\Omega$ be a collection of sets. We ask: how do we list its elements?
If $\Omega$ is finitely large or ``countably large'', we can write
$\Omega = \{ A_{1}, A_{2}, \cdots, A_{n}\}$, or
$\Omega = \{ A_{1}, A_{2}, \cdots, A_{n}, \cdots\}$, respectively.
Then we call $\{ 1,\cdots, n\}$ and $\{1, \cdots, n, \cdots\}$ \emph{indexing
sets}.

However, what if $\Omega$ is uncountable? In this case, we find a set $J$ with
cardinality $|\Omega|$, and define, using the axiom of choice, a bijective \emph{indexing function} $f: J
\to \Omega$ where $f(\alpha) = A_{\alpha}$. That is, $f$ maps each element of
$J$ to a unique element of $\Omega$. Again invoking the axiom of choice,
we define
\begin{equation*}
\begin{split}
	& \bigcup_{\alpha \in J} A_{\alpha} = \{ x: \text{for at least one $\alpha \in
J$, $x \in A_{\alpha}$}\} \\
& \bigcap_{\alpha \in J} A_{\alpha} = \{ x: \text{for every $\alpha \in
J$, $x \in A_{\alpha}$}\}
\end{split}
\end{equation*}
If $\Omega$ is a finite set or an infinite ``countable set'', i.e., like $\zz$, 
we can define the \emph{cartesian product} of its elements as
\begin{equation*}
\begin{split}
	& \prod_{i=1}^{m} A_{i} \doteq A_{1} \times A_{2} \times \cdots \times A_{m} \\
	& \prod_{i =1}^{\infty} A_{i} \doteq A_{1} \times A_{2} \times \cdots \times
	A_{m} \times \cdots,
\end{split}
\end{equation*}
respectively. The elements of the finite and countable products are known as
$m$-tuples, and $\omega$-tuples, respectively.
If $\Omega$ is uncountable, then for an indexing set $J$ with the same
cardinality, we can define the cartesian product of the elements of $\Omega$
as 
\begin{equation*}
\begin{split}
	\prod_{\alpha \in J} A_{\alpha} = \left\{ f: J \to \bigcup_{\alpha} A_{\alpha} \mid
	\forall \alpha \in J, \, f(\alpha) \in X_{\alpha}\right\}.
\end{split}
\end{equation*}
Observe that for this definition, the functions $f$ are analogues of the
$m$-tuples and $\omega$-tuples in the finite and countable cases.
We also definte the projection operator
\begin{equation*}
\begin{split}
	& \pi_{\alpha}: \prod_{\alpha \in J} A_{\alpha} \to \bigcup_{\alpha}
	A_{\alpha} \\
	& \pi_{\alpha}(f) = f(\alpha)
\end{split}
\end{equation*}
In the finite and countable cases, this definition reduces to returning
the $\alpha$-th coordinate of a tuple.
\section{Finite Sets} 
\begin{definition}
	A set $A$ is called \emph{finite} is there exists a bijective
	$f: A \to \{ 1, \cdots, n\}$ for some $n$. If $A$ does not contain any
	elements, then we say $A$ is \emph{empty}, and write $A \doteq \emptyset$.
\end{definition}
Observe that a bijective map from a finite $A$ to $\{1, \cdots, n\}$ is not unique; in fact, there are $n!$ permutations of $\{1, \cdots, n\}$, and so $n!$ possible
bjiective maps. 
\begin{theorem}\label{thm:subset-bij}
	Assume there exists a bijective $f: A \to \{1, \cdots, n\}$, where $n \in
	\zz_{+}$. Let $B \subsetneqq A$. Then there does not exist a bijection $g: B
	\to \{1, \cdots, n\}$, but provided $B \neq \emptyset$, there exists a
	bijection $h: B \to \{1, \cdots, m\}$ for some $m < n$.	
\end{theorem}
\begin{proof}
For the case $n=1$, we must have $B = \emptyset$, and so the theorem
trivially holds. For $n>1$, we let $g = f |_{B}$. Then $g: B \to f(B)$ is bijective. But
$f(B) = f(A) \setminus f(A \setminus B) = S \subsetneqq \{1, \cdots, n\}$.
Hence, the elements of $S$ can be indexed: we give the first element the
$1$ index, the second element the $2$ index, and so on, down to the last
element, which we give the $m$ index, where $m < n$ is the size of $S$. 
Then the function $h: S \to \{1, \cdots, m\}$ given by $h(s_{i}) = i$
is a bijection. Then $h \circ g: B \to \{1, \cdots, m\}$ is bijective.
\end{proof}
\begin{corollary}
	If $B \subsetneqq A$, then $|B| < |A| < \infty$.
\end{corollary}
\begin{corollary}
	Finite unions and finite cartesian product of finite sets are finite.
\end{corollary}
\begin{corollary}
	The cardinality of a finite set $A$ is uniquely determined by $A$. That is,
	if $f: A \to \{1, \cdots, n\}$ is a bijection, there cannot exist a 
	$g: A \to \{ 1, \cdots, m\}$, $m < n$, that is a bijection.
\end{corollary}
\section{Countable and Uncountable Sets} 
\begin{definition}
A set is called \emph{countably infinite} if there exists a bijective
	$f: A \to \zz_{+}$. 
\end{definition}

\begin{claim}
	$\zz$ is countable.
\end{claim}
To see this, we define 
\begin{equation*}
\begin{split}
	& f: \zz \to \zz_{+}
	\\
	& f(n) = \begin{cases}
	2n, \quad n > 0 \\
	2n+1, \quad n \ge 0
	\end{cases}
\end{split}
\end{equation*}
This is a bijective function, with 
\begin{equation*}
\begin{split}
	f^{-1}(k) = 
	\begin{cases}
		k/2, \quad k \text{ is even} \\
		(1-k)/2, \quad k \text{ is odd}
	\end{cases}
\end{split}
\end{equation*}
\begin{claim}
	$\zz_{+} \times \zz_{+}$ is countable.		 
\end{claim}
To see this, let $A = \left\{(x,y) \in \zz_{+} \times Z_{+}: y \le x\right\}$
and define
\begin{equation*}
\begin{split}
	& f: \zz_{+} \times \zz_{+} \to A, \quad g: A \to \zz_{+} \\
	& f(x,y) = (x + y-1, y), \quad g(x,y) = \frac{1}{2}(x-1)x + y		
\end{split}
\end{equation*}
It is easy to check that $f$ and $g$ are bijective. Since the composition of
bijective functions is bijective, we conclude that $f \circ g: \zz_{+} \times
\zz_{+} \to \zz_{+}$ is bijective.

Observe that it has taken a lot of work to prove countability so far. 
In order to simplify proofs of countability, we use the following theorem.
\begin{theorem}\label{thm:proof-countability}
	Let $B \neq \emptyset$. Then the following are equivalent:
	\begin{enumerate}
		\item \label{it:1} $B$ is countable
		\item \label{it:2} There exists a surjective $f: \zz_{+} \to B$
		\item \label{it:3} There exists an injective $g: B \to \zz_{+}$. 
	\end{enumerate}
\end{theorem}
To prove this, we shall need the following lemma:

\begin{lemma}\label{lem:subset_z_count}
	If $A \subset \zz_{+}$, then $A$ is countable.
\end{lemma}
\begin{proof}[Proof of \cref{lem:subset_z_count}]
	If $A$ is finite, the proof is trivial. Hence, assume that $A$ is infinite.
	Our strategy will be to ``rewrite'' $A$ such that setting up a bijection
	between it and $\zz_{+}$ will be easy.

	Proceeding, set $a_1 = \mathrm{inf} A$, $a_{2} = \mathrm{inf} \{A \setminus {a_{1}}\}$,
	$a_{3} = \mathrm{inf} \left\{ A \setminus \left\{ a_{1}, a_{2} \right\} \right\}$,
	and recursively define $a_{n} = \mathrm{inf} \left\{ A \setminus \left\{
		a_{1}, \dots,
a_{n-1} \right\}\right\}$. We claim that $\bigcup_{n} a_{n} = A$. By
construction, it is clear that $\bigcup_{n} a_{n} \subset A$. It remains to show
that $A \subset \bigcup_{n} a_{n} \doteq h(n)$. Pick $a \in A$. Since
$h(n)$ is a strictly increasing function, there exists a smallest $n$ such that
$h(n) > a$. It follows that
$a \not \in A \setminus\left\{ a_{1}, \dots, a_{n-1} \right\}$. It follows that
$a \in \left\{ a_{1}, \dots, a_{n-1} \right\}$, and so
$\bigcup_{n} a_{n} = A$. 

It is clear that $f: B \to
\zz_{+}$ given by $f(a_{i}) = i$ is a bijection.
	
\end{proof}

\begin{proof}[Proof of \cref{thm:proof-countability}]
	We use a standard idea when proving equivalences. It is enough to show
	that \eqref{it:1} $\implies$ \eqref{it:2} $implies$ \eqref{it:3} $implies$
	\eqref{it:1}.

	To prove that \eqref{it:1} $\implies$ \eqref{it:2}, we observe that by
	definition, if $B$ is countably infinite, there exists a bijective
	(and, hence, surjective) map  $f: \zz_{+} \to B$. 
	If $B$ is finite, then by definition there exists a bijective map
	$f: \left\{ 1, \dots, n \right\} \to B$. We extend this map to a surjective
	map $\bar{f}: \zz_{+} \to B$ by definining $\bar{f} \equiv f$ on $B$,
	and $\bar{f}(x) = 1$ for all $x \in \zz_{+} \setminus \left\{ 1, \dots, n
	\right\}$.

	To prove that \eqref{it:2} $\implies$ \eqref{it:3}, let
	$f:\zz_{+} \to B$ be surjective. Define $g: B \to \zz_{+}$
	where $g(b) = \mathrm{inf} \{f^{-1}(b) \}$. Observe that the infimum is always
	realized at some $c \in \zz_{+}$. Since $f$ is a well-defined
	function, it follows that $g$ must be injective.

	Lastly, to prove that \eqref{it:3} $\implies$ \eqref{it:1}, we first note that
	if $B$ is finite, we are done. We assume that $B$ is infinite. Since $g$ is
	injective, $g(B) \subset \zz_{+}$ is infinite. Applying
	\cref{lem:subset_z_count} concludes the proof.

\end{proof}

\begin{corollary}
	The subset of a countable set is countable.	
\end{corollary}

To see how useful \cref{thm:proof-countability} is, we note that it implies that
$\zz_{+} \times \zz_{+}$ is countably infinite, via the injectiveness of $f:
\zz_{+} \times \zz_{+} \to \zz$ given by $f(n,m) = 2^{n} 3^{m}$. Note that this
map is \emph{not} surjective, but that this is inconsequential, thanks to 
\cref{thm:proof-countability}.

Similarly, $\mathbb{Q}_{+}$ is countably infinite, via the injectiveness
of $f: \zz_{+} \to \zz_{+}$ given by $g(n,m) = n/m$.
\begin{theorem}\label{thm:cantor}
	The set $\left\{ 0,1 \right\}^{\omega} \doteq \left\{ 0,1 \right\} \times
	\left\{ 0,1 \right\} \times \dots$ is uncountable.
\end{theorem}
\begin{proof}
	Our proof utilizes Cantor's famous diagonalization argument. 
	Suppose $g: Z_{+} \to \left\{ 0,1 \right\}^{\omega}$. We want to show that
	$g$ can't be surjective. 

	Observe that $g(\zz_{+}) = \left\{ g(1), g(2), \dots, g(k), \dots \right\}$.
	Let $g(k)_{k}$ denote the $kth$ coordinate of $g(k)$, and define
	$y = (y_{1}, y_{2}, \dots, y_{n}, \dots) \in \left\{ 0,1 \right\}^{\omega}$,
	where
	\begin{equation*}
	\begin{split}
		y_{i} = g(i)_{i}^{\mathsf{c}}.
	\end{split}
	\end{equation*}
	Intuitively, if we view $g(\zz_{+})$ as an infinite sized matrix with rows
	$g(i)$, then $y$ is the bitwise complement of its diagonal. It follows that
	$y \neq g(i)$, for any $i$, since it differs from $g(i)$ in at least one
	coordinate (namely, its coordinate on the diagonal of the matrix
		$g(\zz_{+})$. 
\end{proof}
\begin{corollary}
	For any infinite set $A$, its power set $\mathrm{P}(A)$ is uncountable.
\end{corollary}
\begin{proof}
	Without loss of generality, we can assume $A$ is countable. That is, if $A$
	were uncountable, then using the axiom of choice we could construct a
	countable subset $\left\{ a_{i} \right\}_{i=1}^{\infty}$. Since 
	$P(\left\{ a_{i} \right\}_{i=1}^{\infty}) \subset P(A)$, proving the theorem
	would reduce to the case where $A$ was countable.

	Proceeding, we assume $A = \left\{ a_{i} \right\}_{i}^{\infty}$.
	Then there exists a bijective map $f: P(A) \to \left\{ 0,1 \right\}^{\omega}$,
	where 
	\begin{equation*}
	\begin{split}
		\pi_{i}(f(S)) = \begin{cases}
			1 \text{ if } a_{i} \in S \\
			0 \text{ if } a_{i} \not \in S.
		\end{cases}
	\end{split}
	\end{equation*}
	Assume $P(A)$ is countable. Then there exists a bijection $g: P(A) :
	\zz_{+}$. But then $g\circ f^{-1}: \left\{ 0,1 \right\}^{\omega} \to
	\zz_{+}$ is bijective, contradicting \cref{thm:cantor}.
	We conclude that $P(A)$ is uncountable.
\end{proof}

