\documentclass[12pt]{article}
\usepackage{amsthm}
\usepackage{amssymb} %for symbols like lesssim
\usepackage{mathtools} %amsmath extension package
%\mathtoolsset{showonlyrefs}
\usepackage{cancel}  %for canceling terms explicitly on pdf
\usepackage{enumerate} %to change enumerate symbols
\usepackage[margin=2.5cm]{geometry}  %page layout
\usepackage{hyperref}
\hypersetup{colorlinks=true,
	linkcolor=blue,
	citecolor=green,
	filecolor=magenta,
	urlcolor=cyan
	linkbordercolor=Blue
	citebordercolor=Violet
	filebordercolor=Red
}
\usepackage{microtype} %improve spacing
\usepackage[alphabetic, initials, msc-links]{amsrefs} %for the bibliography; 
%uses cite pkg. Must be loaded after hyperref, otherwise doesn't work properly 
%(conflicts with cref in particular)
\usepackage{cleveref}
\newcommand{\rr}{\mathbb{R}}
\newcommand{\nn}{\mathbb{N}}
\newcommand{\zz}{\mathbb{Z}}
\newcommand{\cc}{\mathbb{C}}
\newcommand{\ci}{\mathbb{T}}
\newcommand{\dollar}{\$}
\newcommand{\wh}{\widehat}
\newcommand{\p}{\partial}
\newcommand{\ee}{\varepsilon}
\newcommand{\vp}{\varphi}
\newcommand{\wt}{\widetilde}
\newcommand{\filter}{\mathcal{F}}
\newcommand{\borel}{\mathcal{B}}
\newcommand{\prob}{\mathbb{P}}
\newcommand{\var}{\text{Var}}
\newcommand{\covar}{\text{Covar}}
\theoremstyle{plain}
\newtheorem{theorem}{Theorem}
\newtheorem{lemma}[theorem]{Lemma}
\newtheorem{corollary}[theorem]{Corollary}
\newtheorem{proposition}[theorem]{Proposition}
\theoremstyle{definition}
\newtheorem*{definition}{Definition}
\newtheorem*{example}{Example}
\newtheorem*{exercise}{Exercise}
\theoremstyle{remark}
\newtheorem*{remark}{Remark}
\numberwithin{equation}{section}  %eliminate need for keeping track of counters
\begin{document}
\title{Probability and Finance Notes}
\author{David Karapetyan}
\date{}
\maketitle
\newpage
\setcounter{tocdepth}{2}
\setcounter{secnumdepth}{2}
\tableofcontents
\newpage
\section{Outcomes and Events, and Likelihood}
Throughout this course, we seek to rigorously describe the likelihood of
outcomes of an experiment, or, more generally, combinations of outcomes of an 
experiment.
For example, we would like to know the likelihood of the Dodgers
winning the pennant, or the likelihood that stock of a company will go up
in a week. We have used words to describe these events---since we are doing
mathematics, we would like to translate everything to numbers, without losing
any of the meaning.
\begin{example}
	A fair coin is tossed. There are two possible outcomes: heads (denoted by 
	$0$) 
	and tails (denoted by $1$). We call $ \left\{ 0,1 \right\} $ the 
	\emph{outcome
		space} or \emph{sample space}. The \emph{event space} consists of
	\begin{enumerate}[(a)]
		\item Outcome is heads or tails
		\item Outcome is heads and tails
		\item Outcome is heads
		\item Outcome is tails
	\end{enumerate}
	which we translate, via numbers, to the set $ \left\{ \left\{ 0 \right\} 
		\cup
		\left\{ 1 \right\}, \left\{ 0 \right\} \cap \left\{ 1 \right\} , \left\{ 0
		\right\} , \left\{ 1 \right\} \right \}$.
	Observe that the event ``Heads or tails'' is the union of the events
	``Outcome is Heads'' and ``Outcome is tails'', and ``Heads and tails'' is
	their intersection.
\end{example}	

\begin{example}
	A die with six distinct faces is thrown. The outcome space is
	$ \left\{ 1, 2, 3, 4, 5, 6 \right\} $. Some possible events are
	\begin{enumerate}[(a)]
		\item The outcome is even
		\item The outcome is odd
		\item The outcome is greater than $2$
		\item The outcome is less than $2$.
	\end{enumerate}
	These events, respectively, are expressed via the sets
	$ \left\{ 2, 4, 6 \right\} , \left\{ 1,3,5 \right\} , \left\{ 3, 4, 5, 6
	\right\}$, and $\left\{ 1 \right\}$.
	Observe that there are many other possible outcomes.
\end{example}
Observe that the larger the outcome space, the more complex
our event space may become. Also observe that the event
``The outcome is even'' is the \emph{complement} (with respect to the
outcome space) of
the event ``The outcome is odd''. That is,
${\left\{ 2,4,6 \right\}}^c
= \left\{ 1,3,5 \right\}$.

Motivated by these two examples, we seek to generalize our notion of
event space such that it completely models all possible events
that can arise from a given outcome space. We have the following.
\begin{definition}
	Let $\Omega$ be an outcome space. Then a \emph{$\sigma$-algebra}
	or \emph{filter} on $\Omega$, denoted by $\filter(\Omega)$ is defined to be 
	a
	set of subsets of $\Omega$ that is closed under countable
	unions and complements.
	More precisely
	\begin{enumerate}[(a)]
		\item $ A_i \in  \sigma \implies \cup_i A_i \in \sigma$
		\item $A_i \in \sigma \implies A_i^c \in \sigma$
	\end{enumerate}
\end{definition}
\begin{example}
	Suppose we are tossing a fair coin repeatedly until the first tail shoes up,
	and wish to know the number of tosses we must use. Then our outcome space
	is \\ $ \left\{ \left\{ 0 \right\} , \left\{ 0,1 \right\} , \left\{ 0,0,1
		\right\},\ldots \right\}$. This is an infinite outcome space.
\end{example}
\begin{example} The following are all filters of $\Omega$:
	\begin{enumerate}[(a)]
		\item $\left\{ \emptyset, \Omega \right\}$
		\item $\left\{ \emptyset, A, A^c, \Omega \right\}$
		\item The power set of $\Omega$
	\end{enumerate}
\end{example}
Observe that a natural outcome space to use is the real line, or subsets
thereof, and to build appropriate filters on top of it. We will return
to this idea later.

\section{Discrete Probability}
\subsection{Unconditional Probability}
Suppose we roll a $6$-sided die $N$ times. Let $N(A)$ denote the number of $4's$
rolled in $N$ tries. One can observe that
\begin{align*}
	\lim_{N \to \infty} N(A)/N = 1/6.
\end{align*}
A similar phenomenon holds for many other real-world events.

\begin{definition}
	Let $N(A)$ denote the number of ``successes'' $S$ in $N$ trials. If the
	limit $N(A)/N \doteq P$ exists, we say $S$ occurs with probability $P$.
\end{definition}
Observe that $ 0 \le P \le 1$, by definition.
Next, let $A$ and $B$ be two disjoint events. Then
it is easy to check that
\begin{align*}
	N(A \cup B) = N(A) + N(B)
\end{align*}
and that, in general (i.e.\ for possibly disjoint $A$, $B$),
\begin{align*}
	N(A \cup B) = N(A) + N(B) - N(A \cap B).
\end{align*}
Furthermore, since $\Omega = A \cup A^c$, it follows immediately
that $N(\Omega) = N$. A similar argument shows $N(\emptyset) = 0$.
This discussion motivates the following.
\begin{definition}
	A \emph{probability measure} on $(\Omega, \filter)$ is a continuous function
	$\prob: \filter \to [0,1]$ which satisfies
	\begin{enumerate}[(a)]
		\item
			$\prob(\Omega) = 1$, $\prob(\emptyset) = 0$
		\item
			For disjoint $A_i$, $\prob(\cup_i^n A_i) = \sum_{i = 1}^n 
			\prob(A_i)$.
	\end{enumerate}
\end{definition}
Observe that our construction immediately rules out silly probability
measures such as $P \equiv 0$ or $P \equiv 1$. Furthermore,
we can show the following, whose proof we leave to the reader.
\begin{lemma}
	Let $\prob$ be a probability measure on $(\Omega, \filter)$
	\begin{enumerate}[(a)]
		\item $\prob(A^c) = 1 - \prob(A)$
		\item $\prob(A \cup B) = P(A) + P(B) - P(A \cap B)$.
		\item If $B \subset A$, then $P(A) \ge P(B)$.
		\item \emph{(Set Continuity)}.
			If $\cup_i^\infty A_i = A$ and
			$\cap B_i^\infty = B$, with $A_i \subset A_j$ and $B_j \subset B_i$ 
			for $j > i$,
			then
			\begin{align*}
				& \lim \prob(A_i) = \prob(\lim \cup_i^n A_i) = \prob(A) \\
				& \lim \prob(B_i) = \prob(\lim \cap_i^n B_i) = \prob(B) \\
			\end{align*}
	\end{enumerate}
\end{lemma}
A last technical distinction: the empty-set denotes the event ``nothing
occurs'' and has probability $0$, by definition. However, there are many
events in a given sigma-algebra that have probability $0$ but that are not the
empty set.

\begin{example}What is the probability that, using a fair coin, one
	flips only heads in infinitely many tries? 
\end{example}
Letting $H_N$ denote the event
$N$ heads in the first $N$ tries, set continuity yields
continuity,
\begin{align*}
	P(N_\infty) = \lim_{N \to \infty} \prob(N_H) = \lim_{N \to \infty} 2^{-N} = 
	0.
\end{align*}
\subsection{Conditional Probability}
We now wish to tackle statement that often occur in practice,
such as ``What is the probability that it will snow today, \emph{given}
that the sky is grey today'', or ``What is the probability that a mystery word
is `zebra', \emph{given} that the first three letters are
`z', `e', `b'.
Observe that when we are given information, it allows us to adjust the
probability of the event we are interested in. We wish to express this
mathematically. Motivated by the ``zebra'' example, we have the following.
\begin{definition}
	Let $\Omega, \filter$ be a filter, and $A, B \in \filter$ be events, where
	$\prob(B) > 0$. Then $\prob(A | B)$ is defined to be the probability of 
	$A$, given that $A \in \filter_B$, where $\filter_B
	\doteq \left\{ U \in \filter: B \cap U \neq \emptyset \right\}$.
\end{definition}
Imagine that we roll a die, and wish to know the probability that
we have rolled a $4$.
To compute this, we let $A$ denote the event that we roll a $4$. Then in the 
last section we saw
that
\begin{align*}
	\prob(A) = \lim_{N \to \infty} N(A)/N.
\end{align*}
Suppose now that we are given the event $B$, which is the event that we haven't
rolled a $6$.
What's the probability now that we have rolled a $4$?
Stated a bit differently, how can we modify our ratio above to reflect the
new value? We simply discard all trials whose rolls gave a $6$, as this is 
clearly
impossible with the new die. Of course, we must also discard the number of times
we rolled a $6$. Putting this all together, we obtain
\begin{align*}
	\prob(A | B) & = \lim_{N \to \infty} N(A \cap B)/(N -
	\text{number of occurrences of $6$})
	\\
	& = \lim_{N \to \infty} \frac{N(A \cap B)}{N}
	\times \frac{1}{1 - (\text{number of occurrences of $6$})/N}
	\\
	& = \frac{\prob(A \cap B)}{(1 - \prob(B^c))}
	\\
	& = \frac{\prob(A \cap B)}{\prob(B)}.
\end{align*}
Motivated by this, we have the following definition.
\begin{definition}
	Let $A, B$ be events, and suppose $B$ occurs. Then
	\begin{align*}
		\prob(A | B) \doteq \frac{\prob(A \cap B)}{\prob(B)}
	\end{align*}
\end{definition}
Observe that it is implicit in the definition that $\prob(B) > 0$, otherwise
$B$ does not occur.
\begin{example}
	A family has two children. What is the probability that both are boys,
	given that at least one of them is a boy?
\end{example}
\begin{example}
	A family has two children. What is the probability that both are boys,
	given that the youngest is a boy?
\end{example}
Observe that if no information is given, the probability of two boys is $1/4$.
Hence, the conditions given in the example above improve the probability.
Sometimes, given information may reduce the probability of an event.
If the given information has no effect on the probability of an event,
we say that the event and the information are \emph{independent}. We will
discuss this more in the upcoming lectures.

Often, when information is given, it makes computing probabilities easier.
Consequently, the following result is extremely useful.
\begin{lemma}
	Let $ {\left\{ B_i \right\}}_{i=1}^n$ be a
	partition of $\Omega$. Then
	for any $A \in \filter$
	\begin{align*}
		\prob(A) = \sum_{i = 1}^n \prob(A | B_i) \prob(B_i).
	\end{align*}
\end{lemma}
the above lemma has the limitation that sometimes $\prob(A |
B)$ is difficult to compute as well. In such situations, 
it is often easier to compute $\prob(B | A)$. Fortunately, we have the
following.
\begin{lemma}[Bayes' Theorem] For events $A, B$, we have
	\begin{align*}
		\prob(A | B) = \frac{\prob(B | A) \prob (A)}{\prob(B)}
	\end{align*}
\end{lemma}
\begin{example}
	We are given two urns, each containing an assortment of colored balls.
	Urn $I$ contains two white and three blue balls, and urn $II$ contains
	three white and four blue balls. A ball is drawn at random from urn $I$
	and put into urn $II$, and then a ball is picked at random from urn $II$.
	What is the probability that it is blue?
\end{example}
\begin{example}
	You are on ``Let's Make a Deal''. There are three doors, with a new 
	convertible
	behind one, and a goat behind each of the others. You pick door number $I$.
	To tease you, Monty opens door number $II$, revealing a goat, then offers 
	to let
	you switch your choice to door number $III$. Should you?
\end{example}
\begin{example}[Symmetric Random Walk]
	Las Vegas has decided to offer a new game, with $50/50$ odds
	of winning or losing in each iteration. The player begins with $\dollar k$
	and the house begins with $ \dollar N$, where $N \gg k$. If the player wins 
	in
	a round, he wins a dollar; otherwise, he loses a dollar. The player adopts 
	the strategy
	to continue playing until either he or the casino is bankrupted.
	What the probability that the player goes bankrupt?
\end{example}
\begin{proof}[Solution]
	Let $A$ denote the event that the player is eventually bankrupted,
	and let $B$ denote victory for the player in the first trial.
	Then
	\begin{align*}
		\prob_k(A)
		& = \prob_k(A | B) \prob(B) + \prob_k(A | B^c) \prob(B^c)
		\\
		& \approx \frac{p_{k+1}}{2} + \frac{p_{k-1}}{2}
	\end{align*}
	We have the boundary conditions $p_0 = 1$ and $p_N = 0$, which we combine 
	with
	the above \emph{difference equation} to obtain
	\begin{align*}
		\prob_k(A) = 1 - k/N.
	\end{align*}
\end{proof}
Observe that if we start off with $k \approx N$, the player stands a chance to
bankrupt the casino! This is one of the reasons all games in Vegas have odds
favoring the house. Also, the above example motivates casinos to have a lot of
cash on hand, so that the amount you have at any given time is dwarfed by
comparison. If you are a high roller with a bankroll rivaling the casino's,
the casino will try to get you to play a game that give you terrible odds over
the long run (for example, craps). If you decide to play blackjack (the game
offering the best odds), the casino will try to distract you with drinks,
entertainment, women/men, etc. It usually works.
\begin{example}[False Positives]
	A rare disease affects one person in $10^5$. A test for the disease
	is wrong with probability $1/100$; that is, it is positive with probability
	$1/100$ for someone who is in fact healthy, and negative
	with probability $1/100$ for someone who is in fact ill. What is the 
	probability
	that you have the disease given that you took the test and it is positive?
\end{example}
\begin{proof}[Solution]
	Let $A$ be the event that we have the disease, and $B$ be the event
	that the test if positive. Then we apply
	Bayes' Theorem to obtain
	\begin{align*}
		\prob(A | B) & = \frac{\prob(B | A) \prob(A)}{\prob(B)}
		\\
		& = \frac{\prob(B | A) \prob(A)}{\prob(B|A) \prob(A) + \prob(B | A^c)
			\prob(A^c)}
		\\
		& = \frac{99/100 \times 1/10^5}{99/100 \times 1/10^5 + 1/100 \times 
			(10^5 -
			1)/10^5}
		\\
		& \approx 1/1000.
	\end{align*}
\end{proof}
Moral: don't freak out if your doctor says you \emph{might} have
cancer. Take the test again.
\subsection{Independence}
\begin{definition}
	We say two events $A, B$, $\prob(B) > 0$ are \emph{independent} if 
	\[\prob(A |
		B) = \prob(A),\] or, equivalently, that \[\prob(A \cap B) = \prob(A) 
		\prob(B).\] More generally,
	a family $ {\left\{ A_i \right\}}_{i = 1}^n $ is
	\emph{independent} if 
	\[ \prob(\cap_{i = 1}^n A_i) = \prod_{i = 1}^n \prob(A_i).\]
\end{definition}
\begin{example}
	What is the probability that Connie's first child will be a masculine child,
	given that a neighboring mobster recently had a boy?
\end{example}
\begin{example}
	What is the probability of flipping heads with a fair coin on the $10th$ 
	trial,
	given that heads is flipped on all previous trials?
\end{example}
\section{Translating Outcomes to Numbers}
As we have seen, the sample space $\Omega$, equipped with an associated filter
$\filter$ and probability measure $\prob: \filter \to [0,1]$ 
are the linchpins of our theory of probability. However, in practice,
abstract outcomes $\omega \in \Omega$ and sets $E \in \filter$ are
cumbersome to work with. We would like to ``translate'' our analysis
on $(\Omega, \filter, \prob)$ to $(\rr, \borel_{\rr}, dx)$,
where $\borel_{\rr}$ denotes the Borel filter on $\rr$ (the smallest 
filter containing all the open sets in $\rr$), and $dx$ is
Lebesgue measure. This translation will be given by continuous functions from
$\Omega$, equipped with its filter structure, to $\rr$,
equipped with the Borel filter. We will call such functions 
\emph{random variables}.
\subsection{Random Variables} 
\begin{definition}
	A \emph{random variable} is a function $X: \Omega \to \rr$ with the property
	that $X^{-1}(U) \in \filter$ for every open $U \in \rr$. If the range of $X$
	is a countable subset of $\rr$, we say $X$
	is a \emph{discrete random variable}. If $\Omega$ is countable, we say that 
	$\left\{ \Omega, \filter, \prob \right\}$ is a \emph{discrete probability 
		space}.
\end{definition}
As shorthand, we shall often denote $\{\omega: X(\omega) \in B\}$ by
$\{X \in B$\}.
\begin{remark}
	Observe that, for a discrete random variable $X: \Omega \to [\alpha_1,
	\ldots, \alpha_n]$,  $n \le \infty$, 
	\begin{align*}
		\prob(\Omega) = \sum_{i =1}^n \prob(X = x_i).
	\end{align*}
\end{remark}
Lastly, we say two random variables $X$, $Y$ are \emph{independent} if
$\{X \in B_1\}$ and $\{Y \in B_2\}$ are independent for all Borel sets
$B_1, B_2$. 
\begin{example}
	Suppose we are interested in studying a single flip of a fair coin.
	Let
	$\Omega = \{H, T\}$ be the outcomes of a coin flip, and
	$\filter = \{\emptyset, \Omega, H, T \}$. Then $X: \Omega \to \rr$, given by
	$X(H) = 0$, $X(T) = 1$ is a discrete random variable.
\end{example}
\begin{example}
	A traveler is lost in the woods, and starts walking aimlessly, but never
	west. 
	We can assign numbers to the directions he takes. 
	Then $\Omega$ is the set of all possible directions (North, South, East
	and the \emph{continuum} of values in between these). Then
	$X: \Omega \to \rr$ given by $X(North) = 1$, $X(South) = -1$, $X(East) = 0$
	and $F_{X} \doteq \left\{ X^{-1}(U): U \in \rr \right\}$ is a continuous 
	random
	variable.
\end{example}
\begin{definition}
	For a space $\left\{ \Omega, \filter \right\}$, we say that a measure $u$ is
	\emph{absolutely continuous} respect to a measure $v$, denoted $u \ll v$, if
	$v(E) = 0$ implies $u(E) = 0$ for every $E \in \filter$.
\end{definition}
\begin{theorem}[Radon-Nikodym]
	Let $u,v: \Omega \to \rr^n$ be measures on $\filter_{\Omega}$, with $u \ll 
	v$.
	Then there exists a positive $f: \Omega \to \rr^n$ such that
	\begin{align*}
		u(E) = \int_{E} f(\omega) dv(\omega)
	\end{align*}
	for every $E \in \filter$.
\end{theorem}
The proof of this theorem lies outside the scope of this course, but can be
found in almost every graduate textbook on analysis. 

We now apply the Radon-Nikodym theorem to define continuous random variables.
\begin{definition}
	Let $X$ be a random variable on a probability space $(\Omega, \filter, 
	\prob)$. 
	The \emph{distribution measure} of $X$ is the probability measure $\mu_X:
	\borel_X \to [0,1]$ given by $\mu_X(B) = \prob(\{\omega\}: X(\omega) \in 
	B)$.
\end{definition}
\begin{definition}
	A random variable $X$ is discrete if and only if it takes values in a 
	countable
	subset of $\rr$, and continuous if and only if $\mu_X \ll dm$, where $dm$
	denotes Lebesgue measure on $\rr^n$. By Radon-Nikodym, its 
	\emph{distribution
		function} $F(x) \doteq \prob(X \le x)$ is given by
	\begin{align*}
		F(x) = \int_{-\infty}^x f(z) \,dz.
	\end{align*}
	for some unique, positive $f(z)$, which we call \emph{probability density}
	function of $X$.
	If $X$ is discrete, we call
	$f(x) \doteq \prob(X = x)$ its \emph{mass function}.
\end{definition}
Lastly, we remark that there exist random variables which are neither
discrete nor continuous, but rather a mixture of the two.
\begin{example}[A Random Variable that is Neither Discrete Nor Continuous]
	A coin is tossed. We assign the event ``lands heads'' the number $-1$, and 
	if it lands tails, we toss a rod, and assign ``lands tails'' how far the 
	rod has
	landed from us. In this case, our random variable $X$ is neither continuous 
	nor
	discrete: it has a point mass at $X = H$, but is continuous otherwise.
	Observe that $X | \text{lands tails}$ and $X | \text{lands heads}$ are
	continuous and discrete, respectively.
\end{example}
\subsection{Common Discrete Distributions}
\subsubsection{Bernoulli Distribution}
We begin with the most basic distribution, from which
we are able to derive a multitude of others.
\begin{definition}
	Let $X$ be the random variable denoting the number of successes
	in $n$ independent trials, where $p$ is the probability of success in
	an individual trial. Then
	\begin{equation*}
		b(k, n, p) \doteq \mathbb{P}(X = k) = \binom{n}{k} p^k {(1 - p)}^{n-k}
	\end{equation*}
	is called the \emph{Bernoulli distribution} of $k$ successes
	in $n$ independent trials, where $p$ is the probability of success on an
	individual trial. The special case
	$p = 1/2$ is called a \emph{binomial distribution}.
\end{definition}
\begin{example}
	Let $X$ denote the number of heads one obtains from $n$ flips of a
	two-sided, non-weighted coin. Then $X$ is binomially distributed.
	If the coin is weighted, then $X$ follows a Bernoulli distribution.
\end{example}
\subsubsection{Poisson Distribution}
We wish now to study events that are distributed sparsely in time.
Suppose we have a large number $n$ of independent trials, with
success in each individual trial unlikely. In order to have a sparse
distribution of events that is nontrivial (i.e.\ we have no successes),
we need to assume that there is a positive
probability that we achieve at least one success in $n$ trials. Hence, we 
assume the expected number of
successes $\lambda \doteq np$ to be nonzero.

Observe that with the above assumptions, we obtain
\begin{equation*}
	\binom{n}{k} = \frac{n!}{(n-k)! k!} \approx \frac{n^k}{k!}
\end{equation*}
and so
\begin{align*}
	P(X = k)
	& = \binom{n}{k} p^k {(1 - p)}^{n-k} \\
	& \approx \frac{n^k p^k q^{n-k}}{k!} \\
	& = \frac{\lambda^k q^{n-k}}{k!} \\
	& = \frac{\lambda^k {(1-p)}^{n-k}}{k!} \\
	& = \frac{\lambda^k {(1-\lambda/n)}^{n-k}}{k!} \\
	& \approx \frac{\lambda^k {(1-\lambda/n)}^{n}}{k!} , \quad n >> 1\\
	& \approx \frac{\lambda^k e^{-\lambda}}{k!}
\end{align*}
\begin{definition}
	We call
	\begin{align*}
		p(k, \lambda) \doteq \frac{\lambda^k e^{-\lambda}}{k!}
	\end{align*}
	the \emph{Poisson distribution} of $k$ successes in a large number
	of trials, where, on average, we expect $\lambda$ successes,
	where $\lambda$ is much less than the number of trials.
	That is, the trials have the feature that successes are
	sparsely distributed.
\end{definition}
We will now extend this result. Assume that in the interval $[0,1]$,
we have $n$ equally spaced points, representing trials. Split this interval 
into two pieces,
$[0,t]$ and $[t, 1]$. Then we have $nt$ trials in interval $[0,t]$ and
$n - nt$ trials in interval $[t,1]$. Then the average number of successes
in $[0,t]$ is given by $\lambda_t \doteq n t p = \lambda t$. Repeating
our preceding computation, but with $\lambda$ replaced by $\lambda_t$
and $n$ by $nt$, we obtain the following.
\begin{theorem}
	Let $\lambda$ be the average number of successes in the interval $[0,1]$.
	The probability of finding exactly $k$ successes  in the subinterval 
	$[0,t]$,
	$t \le 1$, is given by
	\begin{align*}
		p(k, \lambda t) = \frac{{(\lambda t)}^{k} e^{-\lambda t}}{k!}
	\end{align*}
	with average $\lambda t$.
\end{theorem}
\begin{exercise}
	Generalize this result to intervals of arbitrary length.
\end{exercise}
\subsubsection{Negative Binomial Distribution}
For an experiment, it is often important to know just how many trials
one needs to achieve a certain number of successes. To tackle this problem,
we ask a simpler question: out of a total of $n$ independent trials with
either success or failure as the outcome, what is the probability
that the $r'th$ success occurs at the $v'th$ trial, where $r \le v \le n$?
We reason this out as follows: there is a total of $\binom{v-1}{r-1}$ ways to
arrange $r-1$ successes out of $v-1$ trials, each with associated probability
$p^{r-1}{(1 - p)}^{(v-1) - (r-1)} = p^{r-1}q^{v-r}$. Right after the $v-1$ 
trial, we would like to
have the $r'th$ success. This success has associated probability $p$. Putting
everything together, we obtain the following.
\begin{theorem}
	Let $X$ denote the number of successes achieved after $v$ independent 
	trials of an experiment,
	where we assume a success occurs at trial $v$, and where the probability
	of success if $p$. Then the probability that $X=r$, where $r \le v$, is 
	given by
	\begin{equation*}
		p(r, v, p) \doteq \mathbb{P}(X = r) = \binom{v-1}{r-1} p^{r} q^{v-r}
	\end{equation*}
	\begin{definition}
		We call $X$ a random variable with a \emph{negative binomial 
			distribution}.
	\end{definition}
\end{theorem}
\subsection{Common Continuous Distributions}
\subsubsection{Uniform Distribution}
\begin{definition}
	Let $X$ be a continuous random variable on a probability space $\Omega, 
	\filter,
	\prob$ such that the minimum and maximum values of $X$ are $a,b \in \rr$,
	respectively, and $\prob(c \le X \le d ) = \prob(c + t \le X \le d + t)$
	for all $c, d$ in $[a,b]$  and $t \in \rr$ such that $c+t, d+t \in [a,b]$.
	Then we say $X$ is \emph{uniformly distributed} in $[a,b]$. It is easy to 
	check
	that $X$ has density  
	\begin{align*}
		f(x) = \begin{cases}
			x/(b-a), \quad & x \in [a,b] \\
			0 \quad &\text{otherwise}
		\end{cases}
	\end{align*}
\end{definition}
\subsubsection{Normal Distribution}
\begin{definition}
	Let $X$ be a continuous random variable on a probability space $\Omega, 
	\filter,
	\prob$. We say $X$ is \emph{normally distributed} with standard deviation 
	$\sigma$ and
	mean $\mu$ if $X$ has density
	\begin{align*}
		f(x) = \frac{1}{\sigma \sqrt{2 \pi}} e^{-\frac{{(x - \mu)}^2}{2 
				\sigma^2}}.
	\end{align*}
	If $\mu = 0$ and $\sigma =1$, we say $X$ is \emph{standard normally
		distributed}.
\end{definition}
\section{Joint Distributions}
Up to now, we have considered random variables taking values in $\rr$.
It is easy to extend this theory to random variables taking values in $\rr^n$:
our distribution of $X$ is then given by $\mu_X(B) =
\prob(X^{-1}(B))$ for every Borel set $B \in
\rr^2$. Using our intuition, we ask if our distribution function should be of
\begin{align*}
	F(x) = \int_{-\infty}^{x} f(s) \, ds
\end{align*}
where now $ds$ denotes Lebesgue measure on $\rr^2$ and $x = (x_1, x_2) \in
\rr^2$. However, what do we mean when we say $x \le y$ for $x, y \in \rr^2$? 

First, observe that if ${X}: \Omega \to \rr^n$, then $X = (X_1, X_2,
\ldots, X_n)$, where $X_i: \Omega \to \rr$. Hence, our study of $X$ will be
simplified if we can study it component-wise. Motivated by this, we define
$x \le y$ if and only if $x_i \le y_i$. 
\begin{definition}
	Let ${X} = (X_1, X_2, \ldots, X_n)$. Then the \emph{joint distribution}
	$F_X: \rr^n \to \rr$
	is defined as $F_X(x) = \prob({X} \le x)$.
	\begin{lemma}
		Let $F_X$ be a joint distribution. Then
		\begin{enumerate}[(a)]
			\item
				$\lim_{x \to \infty} F_X(x) = 1$ and $\lim_{x \to -\infty}
				F_X(x) = 0$
			\item
				If $x \le y$, then $F_X(x) \le F_X(y)$.
			\item
				$F(x)$ is continuous from above.
		\end{enumerate}
	\end{lemma}
	If all the $X_i$ are discrete, we say $X$ has a \emph{joint discrete
		distribution}. If all the $X_i$ are continuous, we say $X$ has a 
	\emph{joint continuous distribution}. If the distribution is jointly 
	discrete,
	then for $x = (x_1, x_2, \ldots, x_n)$,
	$f(x) = \prob(X_1 = x_1, X_2, = x_2,
	\ldots, X_n = x_n)$. If the distribution is jointly continuous, then by
	Radon-Nikodym
	\begin{align*}
		F(x) = \int_{-\infty}^x f(s) \, ds = \int_{-\infty}^{x_1} \ldots
		\int_{-\infty}^{x_n} f(x_1, x_2, \ldots x_n) \, ds_1 \, ds_2 \ldots \, 
		ds_n.
	\end{align*}

	One of the primary applications of joint distributions is in finding the
	distributions and associated density functions of sums and products of 
	random
	variables.
	\begin{example}
		Let $X$ and $Y$ be continuous random variables, with associated 
		densities
		$f_X(x)$ and $f_Y(y)$, respectively. What are the density functions of
		$Z = X+Y$ and $Z = XY$? What if $X$ and $Y$ are independent?
	\end{example}
	Recall that a continuous random variable $X: \Omega \to \rr^n$
	is defined to be a random variable whose distribution
	\begin{align*}
		u_X(B) \doteq \prob(X \in B)
	\end{align*}
	is absolutely continuous with respect to Lebesgue measure on $\rr^n$. 
	Hence, by
	the Radon-Nikodym theorem
	\begin{align*}
		F_Z(z) = \prob(Z \le z) = \prob((X,Y) \in B_z) 
		& = \int_{B_z} f \, dm
		\\
		& = \int_{B_y} \int_{B_x} f(x,y) \, dx dy
	\end{align*}
	Applying the observation that $B_z
	= \{(x,y):
	-\infty < x < \infty, y \le z-x \}$, we obtain the distribution function 
	\begin{align*}
		F_Z(z) = \int_{-\infty}^{\infty} \int_{-\infty}^{z-x} f_{(X,Y)}(x,y) dy 
		dx.
	\end{align*}
	Differentiating with respect to $z$, we arrive
	at the density
	\begin{align*}
		f_Z(z) = \int_{-\infty}^{\infty} f_{(X,Y)}(x,z-x) dx.
	\end{align*}
	If $X$ and $Y$ are independent, then $f_{(X,Y)}(x, y) = f_X(x) f_Y(y)$, 
	giving
	\begin{align*}
		f_Z(z) = \int_{-\infty}^{\infty} f_X(x) f_Y(z-x) dx = f_X * f_Y(z).
	\end{align*}
	Similarly, for $Z = XY$,  the distribution
	function is given by
	\begin{align*}
		F_Z(z) = \int_{-\infty}^{\infty} \int_{-\infty}^{z/x} f_{(X,Y)}(x,y) dy 
		dx
	\end{align*}
	with density
	\begin{align*}
		f_Z(z) = \int_{-\infty}^{\infty} \frac{1}{x} f_{(X,Y)}(x,z/x) dx.
	\end{align*}
	If $X$ and $Y$ are independent, this density becomes
	\begin{align*}
		f_Z(z) = \int_{-\infty}^{\infty} \frac{1}{x} f_X(x) f_Y(z/x) dx.
	\end{align*}
	\section{Expectation}
	Let $X: \Omega \to [\alpha_1, \ldots, \alpha_k]$, $k < \infty$ be a random
	variable denoting the outcome of some experiment, and let $X_1, \ldots, 
	X_n$ be independent trials
	of the same experiment, where \\ $X_i:
	\Omega \to [\alpha_1, \ldots, \alpha_k]$. Then we 
	call
	\begin{align*}
		E(X) \doteq \lim_{n \to \infty} \frac{1}{n} \sum_{i =1}^n X_i \doteq 
		\mu 
	\end{align*}
	the \emph{expectation}, or \emph{mean}, or $X$. If the limit does not exist,
	or $\mu = \pm \infty$, we say that the expectation does not exist.
\end{definition}
\begin{example}
	Let $X$ be a discrete random variable with mass function 
	\begin{align*}
		f(k) = \begin{cases}
			A k^{-2},  \quad &k\in \zz^{+} 
			\\
			0, \quad & \text{otherwise}
		\end{cases}
	\end{align*}
	(Observe that for $f$ to be a mass function, $A$ must be chosen such that
	$\sum_{k = 1}^{\infty} k^{-2} = 1$).
	Then it is easy to check that
	\begin{align*}
		E(X) = \sum_{k=1}^{\infty} k A k^{-2} = A \sum_{k=1}^{\infty} k^{-1} = 
		\infty.
	\end{align*}

\end{example}
Observe that we can write
\begin{align*}
	\frac{1}{n} \sum_{i = 1}^n X_i
	& = \frac{\alpha_i (\#\alpha_i) + \cdots +
		\alpha_n (\#\alpha_n)}{n}
	\to \alpha_1 \prob(\alpha_1) + \cdots + \alpha_n \prob(\alpha_n).
\end{align*}
This motivates the following.
\begin{definition}
	The \emph{expectation} of a random variable $X$ with mass function $f$ is
	\begin{align*}
		E(X) = \sum_{x \in \rr: f(x) > 0} x f(x), \quad \text{provided} \quad
		\sum_{x \in \rr: f(x) > 0} |x
		f(x) | <
		\infty
	\end{align*}
	If $X$ is continuous with density $f(x)$, then
	\begin{align*}
		E(X) = \int_{-\infty}^{\infty} x f(x) \, dx, \quad \text{provided} \quad
		\int_{-\infty}^{\infty} |x f(x)| \, dx < \infty.
	\end{align*}
\end{definition}
\begin{lemma}
	If $g: \rr \to \rr$ is continuous, then
	\begin{align*}
		E(g(X)) = \sum_{x \in \rr: f(x) > 0} g(x) f(x) \longleftrightarrow
		\int_{-\infty}^{\infty} g(x) f(x) \, dx
	\end{align*}
\end{lemma}
\begin{theorem}[Properties of Expectation]
	Let $X$ be a random variable. Then
	\begin{enumerate}[(a)]
		\item $E(a X + b Y) = a X(X) + b E(Y)$.
		\item E (1) = 1
		\item If $X,Y$ are independent, then $E(XY) = E(X)E(Y)$.
	\end{enumerate}
\end{theorem}
\begin{proof}
	We shall only do the discrete case; the proof for the continuous case is
	analogous.
	\begin{enumerate}[(a)]
		\item This follows immediately from the linearity of sums and integrals.
		\item
			We have $E(1) = \sum x f(x) = \sum_{x = 1} x f(x) = 1$.
		\item
			Let $Z = XY$. Then
			\[E(Z) = \sum_{z} z f_Z(z) = \sum_{x,y} x y f_{(X,Y)}(x,y) = \sum_x 
				x f_X(x)
				\sum_y y f_Y(y) = E(X) E(Y).\]
	\end{enumerate}
\end{proof}
Having defined expectation, we can use it to define the
\emph{variance} of a random variable $X$
\begin{align*}
	E({[X - E(X)]}^2),
\end{align*}
the \emph{standard deviation}
\begin{align*}
	\sigma_X \doteq \sqrt{\var(X)}
\end{align*}
and \emph{covariance} of random variables $X, Y$
\begin{align*}
	\covar(X,Y) \doteq E([X - E(X)][Y - E(Y)])
\end{align*}
The standard deviation of a random variable $X$ is the average spread of the
trials of an experiment away from the mean, and is always positive. 

In practice, we consider the normalized covariance of two random variables,
commonly called the \emph{correlation coefficient}
\begin{align*}
	\rho(X,Y) = \frac{\covar(X,Y)}{\sigma_X \sigma_Y}
\end{align*}
The correlation coefficient has the following important property.
\begin{theorem}
	For random variables $X, Y$, $|\rho(x,y)| \le 1$, where $|\rho(x,y)| = 1$ 
	if and
	only if $aX + bY = 1$ almost surely, for some $a,b \in \rr$. 
\end{theorem}
\begin{proof}
	It is a standard application of Cauchy-Schwarz.
\end{proof}
From this theorem, we see that the correlation index of two random variables is
a measure of the linear dependence of one random variable on another. If the
index is negative, a rise in one random variables implies a fall in the other,
and vice verse. If the coefficient is positive, then the variables rise or fall
together.
\begin{definition}
	We say two random variables $X$ and $Y$ are \emph{correlated} 
	if $E(XY) = E(X)E(Y)$.
\end{definition}
Observe that independent random variables are uncorrelated. However,
lack of correlation does not, in general, imply independence. Heuristically, 
this is
because correlation is only a measure of the degree of linear dependence
between variables, and does not capture higher order (for example, quadratic)
types of dependence. 
\begin{example}
	Let $X$ be a continuous random variable with the standard normal 
	distribution,
	and let $Y = X^2$. Clearly $X$ and $Y$ are dependent. However,
	\begin{align*}
		E(X^3) = \frac{1}{\sqrt{2 \pi}}\int_{-\infty}^{\infty} x^3 e^{-x^2/2} 
		\, dx = 0
	\end{align*}
	by anti-symmetry, and so
	\begin{align*}
		E(XY) = E(X^3) = 0 = E(X) E(Y).
	\end{align*}
	which implies $X, Y$ are uncorrelated.
\end{example}
\section{Conditional Distributions and Expectation}
\begin{definition}
	The \emph{conditional distribution function} of $Y$ given $X = x$,
	written $F_{Y|X}(\cdot | x)$ is defined by
	\begin{align*}
		F_{Y|X}(y|x) = \prob (Y \le y | X = x).
	\end{align*}
	In the discrete case, the \emph{conditional mass function}
	is given by
	\begin{align*}
		f_{Y|X}(y|x) = \prob(Y = y | X = x).
	\end{align*}
\end{definition}
Observe that, in the discrete case
\begin{align*}
	f_{(X,Y)}(x,y)
	& = \prob_{\filter \times \filter}(X \le x, Y \le y)
	\\
	& = \prob_{\filter} (X \le x \cap Y \le y)
	\\
	& = \prob_{\filter}(X \le x | Y \le y) \prob_{\filter}(Y \le y)
	\\
	& = f_{X|Y}	(x,y) f_y(y)	
\end{align*}
and so
\begin{align*}
	f_{X|Y}(x|y) = \frac{f_{(X,Y)}(x,y)}{f_Y(y)}
\end{align*}
and similarly
\begin{align*}
	f_{Y|X}(y|x) = \frac{f_{(X,Y)}(x,y)}{f_X(x)}.
\end{align*}
We will heretofore suppress the distinction between $\prob_{\filter \times
	\filter}$ and $\prob_{\filter}$, for the sake of clarity.

Observe that in
the continuous case, we have some difficulties,
since we are not allowed to condition on null events. However, we may consider
\begin{align*}
	\prob(Y \le y , x \le X \le x + h)
	& = \prob(Y \le y | x \le X \le x +h) \prob(x \le X \le x + h)
\end{align*}
which we can rewrite, using Taylor series, as
\begin{align*}
	\prob(Y \le y | x \le X \le x + h)
	& = \frac{F_{(X,Y)}(x+h,y ) - F_{(X,Y)}(x,y )}{F_X(x+h) - F_X(x)}
	\\
	& = \frac{hf_{(X,Y)}(x,y) + O(h^2)}{hf_X(x) + O(h^2)}
	\\
	& = \frac{f_{(X,Y)}(x,y) + O(h)}{f_X(x) + O(h)}
\end{align*}
Letting $h \to 0$, we obtain
\begin{align*}
	f_{Y|X}(y|x) = \frac{f_{(X,Y)}(x,y)}{f_X(x)}.
\end{align*}
Similarly, 
\begin{align*}
	f_{X|Y}(x|y) = \frac{f_{(X,Y)}(x,y)}{f_Y(y)}.
\end{align*}
Thus, the formulas for conditional distributions
functions in both the discrete and continuous cases are identical.

Since $Y|X =x$ is a random variable, we can write $\psi(x) = E(Y | X =x)$.
Then $\psi(X)$ is a random variable, which we call the expectation of $Y$ given
$X$, also written as $E(Y|X)$.
\begin{theorem}
	For any two random variable $X,Y: \Omega \to \rr^n$,
	\begin{align*}
		E(E(Y | X)) = E(Y)
	\end{align*}
	\begin{proof}
		For the discrete case, we have
		\begin{align*}
			E(\psi(X)) 
			& = \sum_{x} \psi(x) f_x(x)
			\\
			& = \sum_{x,y} y f_{Y|X}(y|x) f_X(x)
			\\
			& = \sum_{x,y} y f_{(x,y)}
			\\
			& = \sum_y y f_Y(y)
			\\
			& = E(Y).
		\end{align*}
		The proof for the continuous case is analogous. 
	\end{proof}
\end{theorem}
\begin{theorem}
	Let $X,Y$ be random variables as before, and $g(x)$ be a function such that
	$E(g(X))< \infty$. Then
	\begin{align*}
		E(E(Y|X)g(X)) = E(Yg(X)).
	\end{align*}
\end{theorem}
\begin{proof}
	It is almost identical to that of the preceding theorem.
\end{proof}
The two theorems above are immensely useful in solving a wide variety of
discrete and non-discrete problems. Heuristically, they allow us to divide a
difficult probability problem into several simpler problems.
\begin{example}
	Suppose we flip a fair coin, and assign $-1$ points to tails, and $+1$ 
	points to
	heads. We play a game where we flip a coin repeatedly until we obtain
	$-1$ points, or $2$ points. What is the expected number of flips before the
	game terminates? 
\end{example}
Let $X_i, -1 \le i \le 2$ denote the number of flips it will take for the game
to terminate if we begin with $i$ points. Then we have
\begin{align*}
	E(X_0) & = E(X_0 | \text{first flip heads})\prob(\text{first flip heads}) +
	E(X_0 | \text{first flip tails}) \prob(\text{first flip tails})
	\\
	& = (E(X_1) + 1) \frac{1}{2} + (E(X_{-1}) + 1) \frac{1}{2}
	\\
	& = \frac{1}{2}(E(X_1) + E(X_{-1})) + 1
	\\
	& = \frac{1}{2}E(X_1) + 1
	\\
	& = \frac{1}{2}[E(X_1 | \text{flip heads}) \frac{1}{2} + E(X_1 | \text{flip
		tails}) \frac{1}{2}] + 1
	\\
	& = \frac{1}{4}[E(X_1 | \text{flip heads}) + E(X_1 | \text{flip
		tails})] + 1
	\\
	& = \frac{1}{4} [ E(X_2) + 1  + E(X_0) + 1] + 1
	\\
	& = \frac{1}{4}[ E(X_0) + 2] + 1
\end{align*}
which gives $E(X_0) = 2$.
\begin{example}
	A hen lays $N$ eggs, where $N$ has a Poisson distribution with parameter
	$\lambda$. An egg hatches with probability $p$, independently  of the other
	eggs. Let $K$ be the number of chicks. Compute $E(K|N)$, $E(K)$, and $E(N | 
	K)$.
\end{example}
To compute $E(K|N)$, we must first recall the Poisson distribution
\begin{align*}
	f_N(n) = \frac{\lambda^n e^{-\lambda}}{n!}
\end{align*}
Given $N$ eggs, the probability that $K$ out of $N$ eggs hatch obeys a 
Bernoulli distribution, and so 
\begin{align*}
	f_{K|N}(k|n) = \binom{n}{k} p^k {(1-p)}^{n-k}
\end{align*}
Now, $E(K | N = n) = pn$, and so $E(K | N) = pN$. Furthermore, $E(K) = E(E(K
| N	)) = E(pN) = p \lambda$.
To compute $E(N | K)$, we apply Bayes Theorem to obtain
\begin{align*}
	f_{N|K}(n|k) & = \frac{f_{K|N}(k|n)f_N(n)}{f_K(k)}
	\\
	& = \begin{cases}
		\frac{\binom{n}{k}p^k {(1-p)}^{n-k} \lambda^n
			e^{-\lambda}/n!}{\sum_{m \ge k}
			\binom{m}{k} p^k {(1 - p)}^{m-k} (\lambda^m e^{-\lambda})/m!}
		, \quad & n \ge k
		\\
		0, \quad & n<k
	\end{cases}
\end{align*}
where we obtained the denominator via the observation
\begin{align*}
	\prob(K = k) & = \prob(K = k \cap N \ge k) 
	\\
	& = \sum_{m \ge k} \prob(K =
	k \cap N = m) 
	\\
	& = \sum_{m \ge k} \prob(K = k | N = m) \prob(N = m)
	\\
	& = \sum_{m \ge k} f_{K|N}(k, m) f_N(m)
	\\
	& = \sum_{m \ge k}
	\binom{m}{k} p^k {(1 - p)}^{m-k} (\lambda^m e^{-\lambda})/m!.
\end{align*}
Simplifying, we obtain
\begin{align*}
	f_{N|K}(n|k) = 
	\begin{cases}
		\frac{{(q\lambda)}^{n-k} e^{-q\lambda}}{(n-k)!}, \quad & n \ge k
		\\
		0, \quad & n<k
	\end{cases}
\end{align*}
and so 
\begin{align*}
	E(N | K = k) & = \sum_{n \ge k} n f_{N|K}(n | k)
	\\
	& = \sum_{n \ge 0} (n + k) \frac{{(q \lambda)}^n e^{-q\lambda}}{n!}
	\\
	& = k \sum_{n \ge 0} \frac{{(q \lambda)}^n e^{-q\lambda}}{n!} + \sum_{n \ge 
		0}
	n \frac{{(q\lambda)}^n e^{-q\lambda}}{n!} 
	\\
	& = k + q\lambda.
\end{align*}
Hence, $E(N|K) = K + q \lambda$.
\section{Risk}
\begin{definition}
	A \emph{riskless asset} is an asset whose future value can be precisely 
	determined. If an asset is not riskless, we say it is a \emph{risky asset} 
\end{definition}
The idea of a riskless asset is a theoretical construction. No asset in the 
world is without some risk. However, the primary example of a (nearly) riskless 
asset is a bond of a stable government. Examples of risky assets include stock 
in a limited liability company (especially those on the NASDAQ or outside the 
Fortune 500),
and real estate.

\begin{theorem}[Axioms of Market Efficiency]
	In an efficient market, we have the following: 
	\begin{enumerate}
		\item 
			In a free market, all available information about an asset is 
			contained in its price. In other words, the price of an asset 
			reflects its risk.
		\item 
			The past price of a stock has no influence on its present price. 
			More precisely,
			stock movement is a martingale.
	\end{enumerate}
\end{theorem}
We will assume these axioms throughout the course. Of course, they are not true 
in general, otherwise there would be no bargains in the market, and investors 
like Warren Buffett would not have had nearly as much success as they have had. 
However, the axiom does hold for the average investor in general. More 
precisely, by the time the average investor hears about a deal, the deal 
vanishes (that is, quicker, savvier investors like Buffett swoop in before 
them), which in turn makes a deal less of a bargain (demand for it raises the 
price). 

Rephrased differently, our competition with each other allows stock prices to 
stabilize. Observe further that the riskiness of a stock is related to how high 
its price is.
The more expensive a stock, the more you have to lose if you a purchase a share 
and the company goes under. 

Since the two axioms hold for the overwhelming majority of investors, of what 
use
is finance to them? 
\begin{remark}
	The purpose of finance is not so much to beat the market,
	but to relate the prices of assets to each other, in order to reduce our 
	exposure to risk. This is known as \emph{hedging}, and will be discussed in 
	detail in the upcoming lectures.  
\end{remark}
\begin{example}
	Generally, when the price of pork goes up, the price of beef goes down, 
	since
	they are goods that can be substituted for one another. If I have a 
	portfolio
	consisting of pork, I am in danger of losing a considerable amount of money
	if the price of pork goes down. To hedge against this, I also purchase beef.
	Now, if the price of pork goes down, my gains from my beef investment will
	offset them. However, if the price of pork goes up, I will lose money on my
	beef investment. What is of note is that the potential ``swing'', or
	variance, around my expected gains from my portfolio has been diminished by
	\emph{hedging} our pork purchase with a beef purchase.
\end{example}
\subsection{Introduction to Hedging and Risk Diversification}
Suppose there is a contract on the market, which guarantees $100$ is we flip 
heads with a fair coin, and $0$ otherwise. We would like to find what the fair 
price is for the contract. 

Our expected winnings is $50$ dollars. However, people are naturally \emph{risk 
	averse}, especially when large sums of money are involved. Hence, we are 
willing to pay $\le \$50$. Now suppose another contract exists that pays out 
$100$ if we flip tails, and $0$ otherwise. Since both contracts above exist on 
the market, we can buy both and be guaranteed $100$. Hence, each contract must 
be worth $50$ dollars \emph{when the other is available}.  The ability to hedge 
has eliminated the effects of our natural risk aversion!. 

A closely related idea is \emph{diversification}. 
\begin{example}
	Suppose now that there are $N$ independent coin flips $X_i$, each paying 
	$100/N$ for heads, and $0$ for tails. Let $X \doteq X_1 + X_2 + \cdots + 
	X_n$. Then $E(X) = 50$, as in the single coin example above. However
	\begin{align*}
		\text{Var}(X)
		& = E({[X - 50]}^2)
		\\
		& = E(X^2) - 2500
		\\
		& = \sum_{i,j: i \neq j} E(X_i)E(X_j) + \sum_{i=1}^N E(X_i^2) - 2500 
		\\
		& = \sum_{i,j: i \neq j} 2500/N^2 + \sum_{i=1}^{N} 5000/N^2 - 2500
		\\
		& = (N^{2} - N)(2500/N^2) + 5000/N - 2500
		\\
		& = 2500/N 
	\end{align*}
	and so
	\begin{align*}
		\sigma(X) = 50/N \to 0.
	\end{align*}
	Observe, that for the single coin case, $N = 1$ and $\sigma(X) = 50$.
	Hence, spreading one's investments around, or \emph{diversifying} allows 
	them to
	control variance and, hence, control risk. However, a consequence of our 
	analysis is that the more diversified a portfolio, the less chance that one 
	will make a great deal of money (i.e.\ deviate far from the mean).  
\end{example}
\subsection{Assets}
We now give a brief overview of some of the most common assets.

\subsubsection{Stocks}
Most shares traded on the market are from \emph{limited liability} companies.
What this means is that if the company goes bankrupt, the investor loses their 
investment, but can't be sued if the company is sued. The investor makes money 
from the stock via both dividends paid out, and by the price of the share on 
the market.

\subsubsection{Bonds} 
The most typical is a $30$ year bond, with coupons (i.e.\ interest payments) 
issued every year. The principal is returned at the end of the $30$ years.
They are  not entirely riskless---interest rates may change, diminishing the 
value of the bond on the open market. This is reflected by the \emph{yield} of 
the bond, where we define \begin{equation*}
	\text{yield} = \frac{\text{interest payment}}{\text{price of bond}}.
\end{equation*}
Since bonds from stable governments are close to riskless, their yield is 
typically low (around $3\%$). Bonds from unstable governments are very risky 
(good luck getting General Sibanda to return your principal once he takes over 
Zimbabwe), and so have high yields.
\subsubsection{Corporate Bonds}
These are loans to a company, and their riskiness varies, depending on the 
stability of the company. 
\subsubsection{Derivatives}
These are any instrument whose value is determined by the value of another 
asset, commonly called the \emph{underlying}. At their core, derivatives are 
instruments that allow us to offset the risk of an asset by adopting the 
``opposite'' risk.
\begin{example}
	Suppose a firm will be paid in yen one year into the future,and will want 
	to exchange the yen into dollars immediately upon payment.
	If the exchange rate dips in one year, the firm will lose money. To lessen 
	the risk of this occurring, the firm can purchase a \emph{forward 
		contract}. That is, it can agree today to a fixed exchange rate for a fixed 
	amount of yen one year into the future. Observe that if the exchange rate 
	is greater one year from now than that fixed in the contract, the firm 
	loses potential profits. However, if the exchange rate is less one year 
	from now than that in the contract, the firm has guarded itself from great 
	losses. 
\end{example}
If the amount of yen paid at the end of the year will be highly variable, the 
firm can enter into an \emph{option}, which is a contract to exchange 
\emph{any} quantity of yen at a fixed interest rate, in the future. However, 
unlike forwards, options are \emph{not-binding}. That is, the firm would have 
the option to not exercise the contract if it is not variable to a year from 
now.
\begin{definition}
	A \emph{call option} is a contract whose purchaser has the option to buy an 
	asset at an agreed upon rate in the future. A \emph{put option} is the a 
	contract whose purchaser has the option to sell an asset at an agreed upon 
	rate in the future. A \emph{vanilla option} is either an \emph{American 
		option} or \emph{European option}---an American option can be exercised any 
	day before an agreed upon day in the future, while a European option can 
	only be exercised on one agreed upon day in the future.
\end{definition}
The pricing of options is a difficult problem, which we shall tackle in future 
lectures.\section{Arbitrage}
We motivate our discussion of \emph{arbitrage} via an example. 
\begin{example}
	Suppose $1$ British pound is worth $1.5$ dollars,  $1$ dollar is worth 
	$100$ yen, 
	and $1$ pound is worth $149$ yen.
	We now construct a strategy to make a profit \emph{without risk of losses} 
	as follows: 
	We borrow $149$ yen, immediately exchange it for $1$ pound, which we 
	exchange for $1.5$ dollars. Then we exchange our $1.5$ dollars for $150$ 
	yen, which we use to pay back our debt of $149$ yen. All these trades are 
	executed instantaneously, allowing us to avoid interest costs. Therefore, 
	our 
	our net profit is $150 - 149 = 1$. This type of investment opportunity is 
	known as an
	\emph{arbitrage opportunity}.
\end{example}
We remark that arbitrage opportunities do exist in the market, but they don't 
exist for long, due to their extreme desirability. That is, once enough people 
find out about the opportunity in the example, it raises the number of people 
demanding to exchange yen for pounds, lowering the value of yen relative to 
pounds until a $150:1$ yen to pound ratio is reached, upon which the arbitrage 
opportunity disappears.  

Since arbitrage opportunities are rare and vanish quickly, we will assume 
absence of arbitrage in our models. Furthermore, we shall adopt all the 
following
assumptions when building models of the financial market:

\begin{enumerate}[(i)]
	\item No market moving. That is, one can buy and sell any amount of an asset
		without affecting its price. Many prop shops attempt to violate this by 
		artificially moving prices, thereby constructing an arbitrage 
		opportunity. This has adverse effects for the economy and market 
		efficiency.   
	\item Liquidity---we can buy or sell any amount of an asset. Observe that 
		Wall St.\  provides this. Indeed, any stable stock exchange does. 
	\item We are allowed to short assets.
	\item Fractional quantities of shares can be purchased. This is motivated 
		by the idea that if there are $10^{6}$ shares of a particular stock 
		available, then $1/10^6 \sim 0.5/10^{6}$.
	\item No transaction costs, commonly known as the \emph{bid-offer spread}. 
		This is of course false in the real world, but we assume it to simplify our 
		models. Observe that our assumption introduces a negligible error if the 
		transaction cost is small relative to the value of the asset being 
		exchanged.
\end{enumerate}
\subsection{Mathematical Definition of Arbitrage, and its Corollaries}
\begin{definition}
	A portfolio is said to be an \emph{arbitrage portfolio} if today it is
	of $x$ value, and in the future it has zero probability of being $<x$ in 
	value,
	and nonzero probability of being $>x$ in value. 
\end{definition}
\begin{theorem}[Monotonicity Theorem]
	Consider two portfolios $A, B$ with values $A(t)$ and $B(t)$, respectively. 
	If $A(T) \ge B(T)$ in every state of the world at time $T$, then
	$A(t) \ge B(t)$ for every $t < T$. In addition, if we know that
	$A(T) > B(T)$ in some states of the world at time $T$, then
	$A(t) > B(t)$ for all $t < T$. 
\end{theorem}
\begin{proof}
	Consider portfolio $C$ constructed by going long $A$, and short $B$. Then
	$C(T) = A(T) - B(T) \ge 0$ in all states of the world at time $T$. 
	By the no-arbitrage principle, we must have $C(t) \ge 0$. \qedhere
\end{proof}
\begin{corollary}
	A vanilla call option or put option is always of positive value before 
	expiry.
\end{corollary}
\begin{proof}
	Let $A$ be a portfolio that is long a call option, and $B$ be an empty
	portfolio.  Then $B$ has zero value for all time, while $A(T) = {(S_{T}
		- K)}_{+} \ge 0$ in all states of the market at time $T$. Hence, $A(T) \ge
	B(T)$ in all states, and hence $A(t) \ge B(t)$ by monotonicity.
\end{proof}
\begin{corollary}[Weighted Symmetry of Randomness]
	If two portfolios $P$ and $Q$ are equal at some $t < T$ and $P(T) > Q(T)$
	in some world states, then $Q(T) > P(T)$ in others. 
\end{corollary}
\begin{proof}
	Assume $P(T) > Q(T)$ in some world states evolving from time $t$, where
	$P(t) = Q(t)$. We proceed by contradiction. Suppose $P(T) < Q(T)$ does not 
	hold
	in any world state. Then $P(T) \ge Q(T)$ in all world states. By 
	monotonicity,
	$P(t) > Q(t)$, which is a contradiction.
\end{proof}
\subsection{Arbitrage-Free Pricing}
\subsubsection{Forward Exchange Rates}
Suppose a firm wants to enter a contract to exchange up to $1$
dollar for yen a year from now at an exchange rate $K'$. We
wish to compute a non-arbitrage inducing exchange rate $K'$ for the
contract. 
Assume we start off with net worth $0$ dollars. Recall that we assume that 
there are no transaction costs, so we are allowed to enter the forward without 
a fee. 
Next, observe that we can hedge a long position in the forward by
shorting $1/(1+r)$ in bonds, which we then exchange at today's exchange rate
$K$ for yen.\footnote{The way to find the appropriate short is to ask yourself: 
	If
	the market exchange rate in a year's time is bigger than $K'$,
	which currency has benefited? Why?  We lose out on having agreed to
	a forward exchange rate $K$, but can benefit by having invested in
	the currency that weakened over the course of the year.
}

We then invest this in the Japanese bond market, which accrues interest at rate 
$d$. We then must exchange this back for dollars at year end at the forward 
exchange rate $1/K'$, and we use the proceeds to pay back our debt from 
shorting, which is now $1$. Hence, in all possible states of the world in one 
year, our net worth in dollars will be 
\begin{equation*}
	\begin{split}
		\frac{K}{K'} \left( \frac{1+d}{1+r} \right) -  1.
	\end{split}
\end{equation*}
Since we started with $0$ dollars, we must end up with $0$ dollars, otherwise 
an arbitrage opportunity is present. Therefore,
\begin{equation*}
	\begin{split}
		\frac{K}{K'} \left( \frac{1+d}{1+r} \right) -  1=0
	\end{split}
\end{equation*}
which gives\footnote{Observe that we did not need to even consider stocks
	and derivatives in our hedging argument. Perhaps their existence would
	create an even bigger arbitrage than the one we obtain above 
	if~\eqref{forward-cont} does not hold. However, this is irrelevant---the 
	existence of bonds and forwards in a market is enough to 
	force~\eqref{forward-cont} to hold.
}
\begin{equation}
	\label{forward-cont}
	\begin{split}
		K' = K\left( \frac{1+d}{1+r} \right). 
	\end{split}
\end{equation}
\begin{exercise}
	Show that our exchange of exactly $1$, and our strategy to short $1/(1+r)$ 
	is somewhat artificial: we can short $a/(1+r)$, where $0 < a < 1$, and 
	obtain the same
	value for $K'$.
\end{exercise}
Hence, we have shown that going long a forward contract is equivalent
to the bond shorting strategy we have outlined above. The strategy of pricing 
an asset by decomposing it into instruments whose price can be computed is 
known as \emph{replication}.
\subsubsection{Options}
Recall that a European call option grants the owner the right, but not the 
obligation, to purchase an asset at price $K$ at some point in the future, but 
not before. 
Let $S$ be the value of the underlying at expiry. Then at expiry, the 
\emph{pay-off} of the option is precisely
\begin{equation*}
	\begin{split}
		{(S - K)}_{+} = \text{max}(S - K, 0)
	\end{split}
\end{equation*}
Similarly, for a put option we have pay-off
\begin{equation*}
	\begin{split}
		{(K - S)}_{+} = \text{max}(K - S, 0)
	\end{split}
\end{equation*}
\begin{theorem}[Put-call parity]
	If a call option, with price $C$, and a put option, with price $P$, and a 
	forward contract, of price $F$, have the same strike and expiry, then
	\begin{equation*}\label{thm:put-call}
		\begin{split}
			C - P = F
		\end{split}
	\end{equation*}
\end{theorem}
\begin{proof}
	Observe that if we adopt a portfolio long a call option, and short a put,
	both with strike price $K$, then the pay off of the portfolio at expiry,
	\emph{in all states of the world}, is $S-K -C + P$, where $C$, and $P$ are
	the prices of the call and put options at starting time $t = 0$,
	respectively. If we go long a forward contract with strike price $K$ and
	cost $F$, this will have pay-off $S-K - F$ at expiry in all states of the
	world. 
	Next, observe that, at expiry, $(S - K - F) - (S - K - C + P) = C - P - F$.
	If $C - P - F \ge 0$, then by the monotonicity theorem, a portfolio long a
	call option, short a put, and short a forward has value $\ge 0$ in all
	states. If $C - P - F > 0$, we have an arbitrage opportunity; that is, we 
	can short a call-put portfolio, invest the proceeds in a forward, and 
	always be guaranteed a profit.
	Similarly, if $C - P - F < 0$, then we can short a forward, go long a 
	call-put portfolio, and always be guaranteed a profit. Hence, by the no 
	arbitrage principle, we
	conclude that we must have $C - P - F = 0$, or $C - P = F$. 
\end{proof}
\begin{theorem}[Bounds on Option Prices]\label{thm:bound-op-prices}
	Let $C_t$ be the price of a call option on a non-dividend paying stock, 
	$S_t$, with
	expiry $T$ and strike $K$. Let $Z_t$ be the price of a zero-coupon bond 
	with maturity $T$, at which time
	it is worth $\$1$. 
	Then we have
	\begin{equation*}
		\begin{split}
			S_{t} > C_{t} > S_{t} - K Z_{t}.
		\end{split}
	\end{equation*}
	In particular, if the interest rate is non-negatiave, then $C_{t} >
	S_{t} - K$ for $t < T$. 
\end{theorem}
\begin{proof}
	At expiry the stock is worth $S_{T}$ and the option is worth
	$\text{max}(S_{T} - K, 0)$. Hence, the stock is always worth at least as 
	much
	as the option in all world states at time $T$. By the monotonicity theorem, 
	it follows
	that $S_{t} > C_{t}$. 

	To establish the lower bound, consider the portfolio long one option and 
	$K$ zero-coupon bonds which
	mature at time $T$. Then at expiry, our portfolio is worth
	\begin{equation*}
		\begin{split}
			{(S_{T} - K)}_{+} + K =  \text{max}(S_{T}, K) \ge S_{T}
		\end{split}
	\end{equation*}
	Observe that our analysis implies that in some world states at time $T$, 
	our portfolio is worth as much as the stock, and in others it is worth 
	more. By the monotonicity theorem, this is true for all $t < T$, and so
	\begin{equation*}
		\begin{split}
			C_{t} + K Z_{t} > S_{t} 
		\end{split}
	\end{equation*}
	or
	\begin{equation*}
		\begin{split}
			C_{t} > S_{t} - K Z_{t}
		\end{split}
	\end{equation*}
	completing the proof.
\end{proof}
\begin{remark}\label{rem:nev-ex}
	In particular, if the interest rate is non-negative, then $Z_{t} < 1$ for 
	any $t < T$, and so
	$C_{t} > S_{t} - KZ_{t} > S_{t} - K$.
	This implies that, before expiry, an option is always 
	worth more than the value that would be obtained by exercising it today.
\end{remark}
This remark, in conjunction with the above theorem, implies the following.
\begin{corollary}\label{cor:am-eu-equiv}
	If interest rates are non-negative, then a European call option and an 
	American call option with the same
	non-dividend paying underlying stock with the same strike and expiry, are 
	of equal value.
\end{corollary}
\begin{proof}
	Recall that an American option and a European option are nearly identical,
	with the caveat that an American option can be exercised any time before 
	expiry.
	Hence, we are paying an additional premium to have this right.
	However, by the previous theorem, the American option is worth the most at 
	expiry.
	Hence, it is only exercised at expiry. Therefore, the right to exercise
	the American option at any time before $T$ has no value, and so the premium
	must be zero. Hence, the American option and European option have the same 
	value at all times.
\end{proof}
\begin{definition}
	Let $S$ be a stock that is fixed in price.
	We say an option $C = C(K, T, S, t)$  is \emph{time-homogeneous} 
	if $C(K, T, S, t) = C(K, T-t, S, 0)$
\end{definition}
Intuitively, time-homogeneity tells us that, as long as the market remains
stable, the cost of an option is influenced by its duration, and not the point
in time which we evaluate it. In terms of price, it makes no difference if we
evaluate an option starting in January with expiry in July, or an identical
option starting in July with expiry in December.  The prices will be the same.
\begin{theorem}[The Greeks]
	Let $C(K, T, S(t), t)$ denote a call option on an underlying $S$, with 
	strike $K$ and expiry $T$. Then
	\begin{enumerate}[(i)]
		\item\label{itm:first} $C(K) = C(K, T, S(t), t)$ is a strictly decreasing Lipschitz continuous
			function of $K$, with Lipschitz constant $Z(t)$, where $Z$ is a non-coupon
			bearing bond with $Z(T) = 1$. Furthermore,
			\begin{equation*}
				\begin{split}
					-Z(t) < \frac{\p C}{\p K}(K) < 0 
					\ \text{if}\  \frac{\p C}{\p K}(K) \  \text{exists} 
				\end{split}
			\end{equation*}

		\item\label{itm:second} $C(K)$ is strictly convex in $K$. Furthermore,
			\begin{equation*}
				\begin{split}
					\frac{\p^{2} C}{\p
						k^{2}}(K) > 0 \ \text{if} \ \frac{\p^{2} C}{\p
						k^{2}}(K) \text{exists}
				\end{split}
			\end{equation*}
		\item\label{itm:third}
			$C(T) = C(K, T, S(t), t)$ is a strictly increasing function of
			$T$ when interest rates are non-negative.
		\item\label{itm:fourth}
			$C(t) = C(K, T, S(t), t)$ is a strictly decreasing function of
			$t$. Furthermore, 
			\begin{equation*}
				\begin{split}
					\frac{\p C}{\p t}(t) < 0 \ \text{if} 
					\  \frac{\p C}{\p t}(t) \ \text{exists}
				\end{split}
			\end{equation*}
		\item\label{itm:fifth}
			$C(S) = C(K, T, S(t), t)$ is a strictly increasing,
			Lipschitz continuous function of $S$. Furthermore,
			\begin{equation*}
				\begin{split}
					\frac{\p C}{\p S}(S) > 0 \ \text{if}
					\ \frac{\p C}{\p S}(S) \ \text{exists}
				\end{split}
			\end{equation*}
	\end{enumerate}
\end{theorem} 

\begin{proof}[Proof of \cref{itm:first}]
	Assume $K_{2} > K_{1}$. Then $C(K_{2}, T, S(t), t) < C(K_{1}, T, S(t), t)$
	in all states. Hence, by monotonicity, $C(K_{2}, t) < C(K_{1}, t)$, and so
	$C$ is  strictly decreasing with respect to $K$. Next, observe that
	\begin{equation*}
		\begin{split}
			C(K_{2}) + (K_{2} - K_{1})Z(T)
			& = (K_{2} - K_{1}) + {[S(T) - K_{2}]}_{+}
			\\
			& 
			\ge {[S(T) - K_{1}]}_{+}
		\end{split}
	\end{equation*}
	Hence, by monotonicity,
	\begin{equation*}
		\begin{split}
			C(K_{2}, T, S(t), t) + (K_{2} - K_{1})Z(t) \ge C(K_{1}, T, S(t), t)
		\end{split}
	\end{equation*}
	or
	\begin{equation*}
		\begin{split}
			C(K_{1}, T, S(t), t) - C(K_{2}, T, S(T), t) \le (K_{2} - K_{1})Z(t).
		\end{split}
	\end{equation*}
	This establishes the Lipschitz continuity. Lastly, dividing both sides by 
	$(K_{2} - K_{1})$ and invoking the mean-value theorem
	gives the desired bounds on $\p C/ \p K$, if it exists.
\end{proof}
\begin{proof}[Proof of \cref{itm:second}]
	Observe that applying the identity ${(A + B)}_{+} \le A_{+} + B_{+}$
	gives
	\begin{equation*}
		\begin{split}
			& \theta{(S_{T} - K_{2})}_{+} + (1 - \theta)(S_{T} - K_{1})
			- {(S_{T} - [\theta K_{2} + (1 - \theta)K_{1}])}_{+}
			\\
			& = \theta{(S_{T} - K_{2})}_{+} + (1 - \theta){(S_{T} - K_{1})}_{+}
			- {\left( S_{T} - [\theta K_{2} + (1 - \theta)K_{1}] \right)}_{+}
			\\
			& \ge {[\theta(S_{T} - K_{2}) + (1 - \theta)(S_{T} - K_{1})]}_{+}
			- {\left( S_{T} - \left[ \theta K_{2} + (1 - \theta)K_{1} \right]
				\right)}_{+}
			\\
			& = {\left[ S_{T} - [\theta K_{2} + (1 - \theta)K_{1}] \right]}_{+}
			- {\left( S_{T} - \left[ \theta K_{2} + (1 - \theta)K_{1} \right]
				\right)}_{+}
			\\
			& = 0.
		\end{split}
	\end{equation*}
	Observe that the lower bound established above is \emph{strict}
	in some states of the world. Hence, by monotonicity
	\begin{equation*}
		\begin{split}
			\theta C(K_{2}, t) + (1 - \theta)C(K_{1}) > C(\theta K_{2} + (1 -
			\theta)K_{1}).
		\end{split}
	\end{equation*}
	Lastly, if $\p^2 C/ \p K^2$ exists, then it is positive by the second
	derivate test.
\end{proof}
\begin{proof}[Proof of \cref{itm:third}]
	Let $T_{1} < T_{2}$, fix $t \le T_{1}$, and
	adopt the notation
	$C_{1} = C(K, T_{1}, S(t), t)$, $C_{2} = C(K, T_{2}, S(t), t)$.  
	By \cref{cor:am-eu-equiv}, $C_{1}$ and $C_{2}$ are each equal in value to
	an American option with the same underlying, strike, and expiry. Hence,
	without loss of generality we
	may treat $C_{2}$ as an American option. Next, observe
	that $C_{2}(T_{1}) > C_{1}(T_{1}) = \text{max}\{S(T_{1})-K, 0\}$
	by \cref{thm:bound-op-prices} and the remark following it.
	By monotonicity, $C_{2}(t) >  
	C_{1}(t)$, completing the proof. 
\end{proof}
\begin{proof}[Proof of \cref{itm:fourth}]
	By time homogeneity and \cref{itm:third}, respectively
	\begin{equation*}
		\begin{split}
			C(K, T, S(0), t) = C(K, T-t, S(t), 0) < C(K, T, S(t), 0)
		\end{split}
	\end{equation*}
	completing the proof.
\end{proof}
\begin{proof}[Proof of \cref{itm:fifth}]
	The proof is analogous to that of \cref{itm:first}. Proceeding,
	let $S_{1} < S_{2}$ be two fixed stock prices. Then $C(K, T, S, T) <
	C(K, T, S_{2}, T)$, so by monotonicity
	$C(K, T, S, t) < C(K, T, S_{2}, t)$ for all $t < T$. Hence,
	$C(S)$ is a strictly decreasing function of $S$. Next, observe that
	\begin{equation*}
		\begin{split}
			C(S_{1}) + (S_{2} - S_{1})Z(T) 
			& = {[S_{1} - K]}_{+} + S_{2} - S_{1}
			\\
			& \ge {[S_{2} - K]}_{+}
			\\
			& = C(S_{2})
		\end{split}
	\end{equation*}
	so by monotonicity
	\begin{equation*}
		\begin{split}
			C(K, T, S_{1}, t) + (S_{2} - S_{1})Z(t) \ge C(K, T, S_{2}, t)
		\end{split}
	\end{equation*}
	or
	\begin{equation*}
		\begin{split}
			C(K, T, S_{2}, t) - C(k, T, S_{1}, t) \le (S_{2} - S_{1})Z(t).
		\end{split}
	\end{equation*}
	This establishes the Lipschitz continuity. Lastly, dividing both sides by
	$(S_{2} - S_{1})$ and invoking the mean-value theorem
	gives the desired bounds on $\p C/ \p S$, if it exists.
\end{proof}
\begin{remark}[The Federal Reserve]
	Observe that concavity in $S$. More precisely, from
	\cref{thm:bound-op-prices} we have
	\begin{equation*}
		\begin{split}
			KZ_{t} > S_{t} - C_{t}
		\end{split}
	\end{equation*}
	Concavity in $S$ tells us that, for fixed $t$, there exists $N$ such that for
	$S > N$, $C_{t}$ quickly diminishes to zero value. Hence, from
	\cref{thm:bound-op-prices}, we see that
	\begin{equation*}
		\begin{split}
			KZ_{t} > S_{t} - C_{t} \approx S_{t}, \quad S \gg 1
		\end{split}
	\end{equation*}
	Hence, for fixed $K$, we see that, for $S \gg 1$, the lower bound for $Z_{t}$
	increases as $S$ increases.  In short, for low strikes $K$, high stock prices
	indicate high interest rates. Similarly, convexity in $K$ implies that, for
	high strikes, interest rates decrease as strike increases. If interest rates
	are set low, and strike is considerably higher, then stock price is high.

	If the Federal Reserve feels inflation setting on, it can increase interest
	rates,
	which in turn lowers either strike prices or stock price.  If strike price is
	lowered, options become the better investment long term than the stock, which
	in turn lowers \emph{its} price, due to fall in demand. Hence, a sufficiently
	high interest rate will result in a fall of stock price. This is one of the
	main reasons why the Federal Reserve manipulates interest rates.
\end{remark}
	\section{The Continuous Black-Scholes Formula}
In order to accurately price an option, our model must take into account the infinitelymany states the future can take. To simplify our discussion, we shall first restrict our model to futures with finitely many states, and zero-interest. 

Proceeding, suppose we wish to price an option today with an underlying worth $100$,
and expiry one day into the future. Furthermore, suppose that we know
that the underlying will tomorrow be worth $110$ with probability $p$,
and $90$ tomorrow with probability $1-p$, where $p \neq 0,1$. 
\begin{remark}
	Observe that if $p=1$, then we have an arbitrage opportunity, since
	we can go short $\$100$ worth in bonds, purchase the stock at $100$, which then will increase to $110$, after which we can pay back our bond debt and be left with 
	a profit of $10$, risk free. A similar arbitrage exists for the case when $p=0$.
\end{remark}
\subsection{The Non-Arbitrage Approach}
Consider a portfolio $P$
that is long $\delta$ shares of the underlying, and short the option.
Then
\begin{equation*}
\begin{split}
P(0) & = \delta S(0) + C_{0} - C(0)
\\
& P(T) = \begin{cases}
	110 \delta + C_{0} - 10, \quad & \text{in some states}
	\\
	90 \delta + C_{0}, \quad & \text{in others}.
\end{cases}
\end{split}
\end{equation*}
Observe that if $P(T)$ were the same in all states, monotonicity would force
$P(t) = P(T)$ for all $t \le T$, and hence, elminate risk. For this to occur,
we must have
\begin{equation*}
\begin{split}
110 \delta + C_{0} - 10 = 90 \delta + C_{0}
\end{split}
\end{equation*}
which gives $\delta = 1/2$. Hence, for $\delta = 1/2$, we have
$P(0) = P(T)$, which by substitution gives us
\begin{equation*}
\begin{split}
\frac{100}{2} = \frac{90}{2} + C_{0}
\end{split}
\end{equation*}
or $C_{0} = 5$. Hence, arbitrage considerations have once again
allowed us to price an option. Observe furthermore that we were able
to price the option independently of the probability $0 < p < 1$. 
Hence, it is reasonable to ask whether we can choose a probability
$0 < p' < 1$ that somehow simplifies our computations.
\subsection{The Risk Neutral Approach}
From our arbitrage considerations, we see that, for suitably chosen
$\delta$, we had $P(0) = P(T)$. Then of course, for this $\delta$, we must
have $E(P(0)) = E(P(T))$. Let us weaken our assumptions a little by allowing
arbitrary $\delta$, while still assuming $E(P(0)) = E(P(T))$. 
Now, $E(P(0)) = P(0) = 100 \delta$, so
\begin{equation*}
\begin{split}
(110 \delta + C_{0} - 10) p + (90 \delta + C_{0}) (1- p) = 100 \delta
\end{split}
\end{equation*}
which reduces to
\begin{equation*}
\begin{split}
20 \delta p - 10p + C_{0} = 10 \delta
\end{split}
\end{equation*}
Observe that for $p = 1/2$, the dependence on $\delta$ vanishes, and we obtain
$C_{0} = 5$. We call $p = 1/2$ the \emph{risk-neutral} probability.

Does such a probability exist in the general case?
\begin{theorem}
\label{thm:risk-neutral}
Consider a two state, no interest world for an asset $S$. Suppose $S(0) = s$
and 
\begin{equation*}
\begin{split}
S(T) = \begin{cases}
	s + u, \quad & \text{with probability }p \\
	s - d, \quad & \text{with probability }1-p
\end{cases}
\end{split}
\end{equation*}
where $u = d \ge 0$. Suppose the option is worth $\alpha$ in the ``up'' state
and $\beta$ in the ``down'' state.
Then the price of a call option $C(0)$ with underlying
$S$ and expiry $T$ is equal to the expected price of the option at time $T$
with $p = 1/2$. That is, \[ C(0) = 1/2(\alpha + \beta). \]
\end{theorem}
\begin{proof}
Construct a portfolio by buying $\delta$
shares of $S$ and shorting a call option with underlying $S$.
Then, to obtain a portfolio without risk, we must find $\delta$ such that
our portfolio has the same value in either the ``up'' or ``down'' states.
Hence, we set 
\begin{equation*}
\begin{split}
s + u \delta - \alpha + C_{0} = (s - d) \delta - \beta + C_{0}
\end{split}
\end{equation*}
which gives
\begin{equation*}
\begin{split}
	\delta = \frac{\alpha - \beta}{u + d}
\end{split}
\end{equation*}
By monotonicity, we must have $P(0) = P(T)$,
and so
\begin{equation*}
\begin{split}
& \frac{s(\alpha - \beta)}{u + d} = (s + u) \delta - \alpha + C_{0} \\
& \frac{s(\alpha - \beta)}{u + d} = (s - d) \delta - \beta + C_{0}. 
\end{split}
\end{equation*}
Summing these two equations, substituting in $\delta$, and simplifying gives
\begin{equation}
	\label{rn-drift}
\begin{split}
	C_{0} & =
	\frac{1}{2} \left [ \frac{\alpha - \beta}{u + d} (d - u) \right ]
	+ \frac{1}{2}(\alpha + \beta)
	\\
	& = \frac{1}{2}(\alpha + \beta)
\end{split}
\end{equation}
which gives $C_{0} = 1/2(\alpha + \beta)$ for $u = d$, completing the proof.
\end{proof}
\begin{remark}
Observe that if $u \neq d$, we can obtain still obtain a risk-neutral $p'$
by setting the right-hand-side of~\eqref{rn-drift} equal to 
\[p' \alpha + (1-p')\beta \] and solving for $p'$. 
\end{remark}
\subsection{Pricing the Two-Step Model}
As before, we assume zero-interest. Let an underlying $S$ be worth
$100$ today ($t=0$), with up and down states of $\pm 5$ tomorrow ($t=1$), respectively.  We let each of these states, in turn, have up and down states of
$\pm 5$ the following day. Our option will have a strike price of $100$
and expiry $T = 2$ two days from now. 

We can see this two step model as a one step model: in the up and down states
tomorrow, the option will have some value. Applying 
\cref{thm:risk-neutral}, we can then take the risk-neutral 
expectation over these two values to obtain the cost of the option today.
That is
\begin{equation*}
\begin{split}
& C_{105}(0) = E(C_{105}(1)) = \frac{1}{2}{(110 - 100)}_{+} +
\frac{1}{2}{(100 - 100)}_{+} = 5
\\
& C_{95}(0) = E(C_{95}(1)) = \frac{1}{2}{(100 - 100)}_{+} +
\frac{1}{2}{(90 - 100)}_{+} = 0
\end{split}
\end{equation*}
and so
\begin{equation*}
\begin{split}
C_{100}(0) = E(C_{100}(1))
& = \frac{1}{2}C_{105}(0) + \frac{1}{2} C_{95}(0)
\\
& = 2.5.
\end{split}
\end{equation*}
Note that this is nothing but the expected value
of $C(T)$ over all possible two-step paths; that is
\begin{equation*}
\begin{split}
2.5 = C_{100}(0) = \frac{{(110-100)}_{+} + 2 {(100 - 100)}_{+} + {(90 - 100)}_{+}}{4}
\end{split}
\end{equation*}
Let $C$ be an option with underlying $S$, whose price follows
a tree with $N$ steps, with a choice of two
directions for each step (``up'' and ``down'').
Since the price of the option depends solely on the price of the
underlying, we may adopt the notation $C(t) = f(S(t))$. Then
an induction shows that
\begin{equation*}
\begin{split}
C_{0} = E(C(T)) =  \prob(S = S_{0} + Nu)
\end{split}
\end{equation*}
\end{document}
