%
\documentclass[12pt,reqno]{amsart}
\usepackage{amssymb}
\usepackage{cancel}  %for cancelling terms explicity on pdf
\usepackage{yhmath}   %makes fourier transform look nicer, among other things
\usepackage{framed}  %for framing remarks, theorems, etc.
%\usepackage{showkeys}  %shows source equation labels on the pdf
\usepackage[margin=3cm]{geometry}  %page layout
%\usepackage[pdftex]{graphicx} %for importing pictures into latex--pdf compilation
\setcounter{secnumdepth}{1} %number only sections, not subsections
\usepackage{hyperref}
\hypersetup{colorlinks=true,
	linkcolor=blue,
	citecolor=blue,
	urlcolor=blue,
}
\synctex=1
\numberwithin{equation}{section}  %eliminate need for keeping track of counters
\numberwithin{figure}{section}
\setlength{\parindent}{0in} %no indentation of paragraphs after section title
\renewcommand{\baselinestretch}{1.1} %increases vert spacing of text
%
%
\newcommand{\ds}{\displaystyle}
\newcommand{\ts}{\textstyle}
\newcommand{\nin}{\noindent}
\newcommand{\rr}{\mathbb{R}}
\newcommand{\nn}{\mathbb{N}}
\newcommand{\zz}{\mathbb{Z}}
\newcommand{\cc}{\mathbb{C}}
\newcommand{\ci}{\mathbb{T}}
\newcommand{\zzdot}{\dot{\zz}}
\newcommand{\wh}{\widehat}
\newcommand{\p}{\partial}
\newcommand{\ee}{\varepsilon}
\newcommand{\vp}{\varphi}
%
%
\theoremstyle{plain}  
\newtheorem{theorem}{Theorem}
\newtheorem{proposition}{Proposition}
\newtheorem{lemma}{Lemma}
\newtheorem{corollary}{Corollary}
\newtheorem{claim}{Claim}
\newtheorem{conjecture}[subsection]{conjecture}
%
\theoremstyle{definition}
\newtheorem{definition}{Definition}
%
\theoremstyle{remark}
\newtheorem{remark}{Remark}
\newtheorem{example}{Example}
%
%
%
\def\makeautorefname#1#2{\expandafter\def\csname#1autorefname\endcsname{#2}}
\makeautorefname{equation}{Equation}
\makeautorefname{footnote}{footnote}
\makeautorefname{item}{item}
\makeautorefname{figure}{Figure}
\makeautorefname{table}{Table}
\makeautorefname{part}{Part}
\makeautorefname{appendix}{Appendix}
\makeautorefname{chapter}{Chapter}
\makeautorefname{section}{Section}
\makeautorefname{subsection}{Section}
\makeautorefname{subsubsection}{Section}
\makeautorefname{paragraph}{Paragraph}
\makeautorefname{subparagraph}{Paragraph}
\makeautorefname{theorem}{Theorem}
\makeautorefname{theo}{Theorem}
\makeautorefname{thm}{Theorem}
\makeautorefname{addendum}{Addendum}
\makeautorefname{add}{Addendum}
\makeautorefname{maintheorem}{Main theorem}
\makeautorefname{corollary}{Corollary}
\makeautorefname{lemma}{Lemma}
\makeautorefname{sublemma}{Sublemma}
\makeautorefname{proposition}{Proposition}
\makeautorefname{property}{Property}
\makeautorefname{scholium}{Scholium}
\makeautorefname{step}{Step}
\makeautorefname{conjecture}{Conjecture}
\makeautorefname{question}{Question}
\makeautorefname{definition}{Definition}
\makeautorefname{notation}{Notation}
\makeautorefname{remark}{Remark}
\makeautorefname{remarks}{Remarks}
\makeautorefname{example}{Example}
\makeautorefname{algorithm}{Algorithm}
\makeautorefname{axiom}{Axiom}
\makeautorefname{case}{Case}
\makeautorefname{claim}{Claim}
\makeautorefname{assumption}{Assumption}
\makeautorefname{conclusion}{Conclusion}
\makeautorefname{condition}{Condition}
\makeautorefname{construction}{Construction}
\makeautorefname{criterion}{Criterion}
\makeautorefname{exercise}{Exercise}
\makeautorefname{problem}{Problem}
\makeautorefname{solution}{Solution}
\makeautorefname{summary}{Summary}
\makeautorefname{operation}{Operation}
\makeautorefname{observation}{Observation}
\makeautorefname{convention}{Convention}
\makeautorefname{warning}{Warning}
\makeautorefname{note}{Note}
\makeautorefname{fact}{Fact}
%
\begin{document}
\title{The Banach Space ODE Theorem}
\author{David Karapetyan}
\address{Department of Mathematics  \\
	University  of Notre Dame\\
		Notre Dame, IN 46556 }
		\date{06/08/2011}
		%
		\maketitle
		%
		%
		%
		%
		%
		%
%
%
\section{Introduction}
\begin{definition}
	\label{def:diff-simp}
	Let $X$ be Banach space, $(a,b) \subset \rr$ an open interval, and
	consider the map $f: (a,b) \to X$.
	Then $f$ is \emph{differentiable at $t_0$} if there exists a map
	$f': (a,b) \to X$ such that 
	%
	%
	%
	%
	\begin{equation}
		\label{diff-limit-simp}
		\begin{split}
			\lim_{h \to 0} \| \frac{f(t_0+ h) - f(t_0) 
			 }{h} - f'(t_0) \|_X = 0.
		\end{split}
	\end{equation}
	%
	%
	We call $f'(t_0)$ the \emph{derivative of $f$ at $t_0$}.
	If \eqref{diff-limit-simp}
	holds for all $t_0 \in (a,b)$, then we say $f$ is \emph{differentiable in
	$(a,b)$}, and call $f'$ the
	\emph{derivative of $f$ in $(a,b)$}.  
\end{definition}
%
%
%
%%%%%%%%%%%%%%%%%%%%%%%%%%%%%%%%%%%%%%%%%%%%%%%%%%%%%
%
%
%				Notation
%
%
%%%%%%%%%%%%%%%%%%%%%%%%%%%%%%%%%%%%%%%%%%%%%%%%%%%%%
%
%
%
%
\begin{framed}
\begin{example}
Let $f: (a,b) \to \rr^2$ be defined by $f(t) = (1, t^2)$. Then $Df(1)$ maps
$(a,b)$ to $\rr$ by the relation $Df(1)(t) = (0, 2t) = t[0, 2]$. Hence, $Df(1)$ is the $1
\times 2$ matrix $[0,2]$, which we may also view as the point $(0,2) \in \rr^2$.
That is, $f'(1) = (0,2)$. Similarly, $f'(t) = (0,2t)$.
\end{example}
\end{framed}


Our goal will be to prove the following.
%
%
%%%%%%%%%%%%%%%%%%%%%%%%%%%%%%%%%%%%%%%%%%%%%%%%%%%%%
%
%
%				Existence Theorem
%
%
%%%%%%%%%%%%%%%%%%%%%%%%%%%%%%%%%%%%%%%%%%%%%%%%%%%%%
%
%
\begin{theorem}[Banach Space ODE Theorem]
	\label{ode-thm}
	Let $X$ be Banach, $E \subset X$ open, $u_0 \in E$, and let $(-a, a)$ be an
	open interval in $\rr$. If $f: (-a, a) \times E \to X$ satisfies the
	inequality
	%
	%
	\begin{equation}
		\label{stronger-ode}
		\begin{split}
			\|f(t, x) - f(t, y) \|_X \le c \|x - y\|_X, \qquad \forall t \in (-a, a),
			\qquad \forall x, y \in E,
		\end{split}
	\end{equation}
	%
	%
	then for sufficiently small $h > 0$ there exists a unique
	differentiable map \\ $u: (-h, h) \to E$ such that for all $t \in (-h, h)$
	%
	%
	\begin{gather}
    \label{ode-thm-eq}
			u'(t) = f(t, u(t)),
			\\
      \label{ode-thm-init-data}
			u(0) = u_0.
	\end{gather}
\end{theorem}
%
%
Notice that \autoref{ode-thm} is a 
generalization of the following classical result.
%
%
%%%%%%%%%%%%%%%%%%%%%%%%%%%%%%%%%%%%%%%%%%%%%%%%%%%%%
%
%
%				 Picard-Lindelof
%
%
%%%%%%%%%%%%%%%%%%%%%%%%%%%%%%%%%%%%%%%%%%%%%%%%%%%%%
%
%
\begin{theorem}[Picard-Lindel\"{o}f]
	Let $f(t, x)$ be a continuous function on $(- a, a) \times (- b,
	b)$ such that
	%
	%
	\begin{equation*}
		\begin{split}
			| f(t, x) - f(t, y) | \le c| x - y |, \quad \forall t \in (-a, a),
			\quad \forall x,y \in (- b, b).
		\end{split}
	\end{equation*}
	%
	%
	Then for any $\xi \in (-b, b)$ there exists an interval $(-h, h)
	\subset (-a, a)$ such that the initial value problem
	%
	%
	\begin{gather}
			\frac{dx}{dt} = f(t, x(t)),
			\\
			x(0) = \xi 
	\end{gather}
	%
	%
	has a unique solution $x(t)$ for all $t \in (-h, h)$.
	%
	%
	\end{theorem}
To prove \autoref{ode-thm}, we will first define
integration over Banach spaces, which we will use to
rewrite \eqref{ode-thm-eq} as an integral equation. A Banach fixed point
argument will then complete the proof. These notes are heavily influenced by a
number of excellent books, among them Dieudonn{\'e}
\cite{Dieudonne_1969_Foundations-of-}, Yosida \cite{Yosida:1980fk},
Folland \cite{Folland_1999_Real-analysis}, Jost
\cite{Jost-1998-Postmodern-analysis}, Rudin \cite{Rudin:1976uq}, and Knapp
\cite{Knapp:2005rm}-\cite{Knapp:2005yg}.
%
%
%%%%%%%%%%%%%%%%%%%%%%%%%%%%%%%%%%%%%%%%%%%%%%%%%%%%%
%
%
%				 Differentiaion in Banach Spaces
%
%
%%%%%%%%%%%%%%%%%%%%%%%%%%%%%%%%%%%%%%%%%%%%%%%%%%%%%
%
%
%%%%%%%%%%%%%%%%%%%%%%%%%%%%%%%%%%%%%%%%%%%%%%%%%%%%%
%
%
%				 Integration in Banach Spaces
%
%
%%%%%%%%%%%%%%%%%%%%%%%%%%%%%%%%%%%%%%%%%%%%%%%%%%%%%
%
%
\section{Integration in Banach Spaces}
\label{sec:int-banach}
%
%
%%%%%%%%%%%%%%%%%%%%%%%%%%%%%%%%%%%%%%%%%%%%%%%%%%%%%
%
%
%				Definition of step functions
%
%
%
%
%
%
%%%%%%%%%%%%%%%%%%%%%%%%%%%%%%%%%%%%%%%%%%%%%%%%%%%%%
%
%
%				Definition of step function
%
%
%%%%%%%%%%%%%%%%%%%%%%%%%%%%%%%%%%%%%%%%%%%%%%%%%%%%%
%
%
\begin{definition}
	A mapping $f:[a,b] \to X$ is a \emph{step-function} if there exists a disjoint
	partitioning $\left\{ (t_{j}, t_{j+1}) \right\}$ of $[a,b]$ such that
	%
	%
	\begin{equation}
		\label{step-function}
		\begin{split}
			f(t)=\sum_{j=0}^{n-1} {\alpha_j} \chi_{(t_{j}, t_{j+1})}.
		\end{split}
	\end{equation}
	%
	%
\end{definition}
%
%
\begin{framed}
\begin{example}
	The map $f:[-1,1] \to L^2(\rr)$ given by 
	%
	%
	\begin{equation*}
		\begin{split}
			f(t) = 
			\begin{cases}
				 e^{-x^2},  \quad & -1 \le t\le0 \\
				 e^{-2x^2},  \quad & \phantom - 0 < t \le 1 \\
				 0,  \quad & \phantom - \text{otherwise}
			\end{cases}
		\end{split}
	\end{equation*}
	%
	%
	is a step-function.
	\end{example}
\end{framed}
%
%
\begin{definition}[Banach Space Integration for Step Functions]
	Let $(a,b)$ be an open interval in $\rr$, $X$ a Banach space, $f: [a,b] \to
	X$ a step function as in \eqref{step-function}. 
	Then 
	%
	\begin{equation*}
		\begin{split}
      \int_a^b f(t) dt \doteq \sum_{j=0}^{n-1} \alpha_j (t_{j+1} - t_{j}).
    \end{split}
	\end{equation*}
	%
	%
%
\end{definition}
%
%
Note that this definition is independent of the partitioning we use for $f$, due
to the following lemma.
%
%
%%%%%%%%%%%%%%%%%%%%%%%%%%%%%%%%%%%%%%%%%%%%%%%%%%%%%
%
%
%                Indpendence of Partioning
%
%
%%%%%%%%%%%%%%%%%%%%%%%%%%%%%%%%%%%%%%%%%%%%%%%%%%%%%
%
%
\begin{lemma}
For step functions $f, g$, we have
%
%
\begin{equation*}
\begin{split}
  \int_{a}^{b} f(t) dt - \int_{a}^{b} g(t) dt = \int_{a}^{b}\left[ f(t) - g(t) \right]dt.
\end{split}
\end{equation*}
%
%
\label{lem:indep-part}
\end{lemma}
%
%
\begin{proof}
%
Let
%
\begin{equation*}
\begin{split}
  & f = \sum_{j=1}^{m}\alpha_{j} \chi_{(t_{j}, t_{j+1})}
  \\
  & g = \sum_{k =1}^{n}
  \beta_{k} \chi_{(s_{k}, s_{k+1})}
\end{split}
\end{equation*}
%
and assume without loss of generality that $n \ge m$. 
Then
%
%
\begin{equation*}
\begin{split}
  f(t) - g(t) = \sum_{p=1}^{m+n} \gamma_{p} \chi_{(y_{p}, y_{p+1})} 
\end{split}
\end{equation*}
%
%
where
\begin{equation*}
  \begin{split}
    \gamma_{p} \chi_{(y_{p}, y_{p+1}) } = 
    \begin{cases}
      \alpha_{p} \chi_{(t_{p}, t_{p+1})}
, \quad & 1 \le p \le m \\
-\beta_{p-m} \chi_{(s_{p-m}, s_{p-m+1})}, \quad & m < p \le m+n.
\end{cases}
\end{split}
\end{equation*}
Then
%
%
%
\begin{equation*}
\begin{split}
  \int_{a}^{b}\left[ f(t) - g(t) \right]dt
  & = \sum_{p=1}^{m+n}
  \gamma_{p}(t_{p+1} - t_{p})
  \\
  & = \sum_{p=1}^{m} \gamma_{p}(t_{p+1} - t_{p}) +
  \sum_{p=1}^{m+n}\gamma_{p}(t_{p+1} - t_{p})
  \\
  & = \sum_{p=1}^{m} \alpha_{p}(t_{p+1} - t_{p}) +
  \sum_{p=m+1}^{m+n}-\beta_{p-m}(s_{p-m+1} - s_{p-m})
  \\
  & = \sum_{j=1}^{m}\alpha_{j}(t_{j+1} - t_{j}) -
  \sum_{k=1}^{n}\beta_{k}(s_{k+1} - s_{k})
  \\
  & = \int_{a}^{b}f(t)dt - \int_{a}^{b} g(t) dt.
\end{split}
\end{equation*}
%
%
%
%
\end{proof}
%
We are now interested in extending integration to a larger class of functions. We will need the
following. %
%
\begin{lemma}
	\label{lem:dense}
	Let $[a, b] \subset \rr$ be a closed interval in $\rr$, $X$ a Banach space, and
	suppose the function $f:[a,b] \to X$ is continuous . Then
	there exists a sequence of step functions $\left\{ f_n \right\}$
	such that $\|f - f_n\|_{L^\infty( [a,b], X)} \to 0$. 
\end{lemma}
%
%
\begin{proof}
  Fix $n \in \mathbb{N}$. Since $f$ is continuous, for every $t \in
\left[ a,b \right]$ there is an open set $V_t \in [a,b]$ such that
$\|f(x) - f(y)) \|_X \le 1/n$ for all $x, y \in V_t$. Note that $\left\{ V_t
\right\}_{t \in \left[ a,b \right]}$ is an open cover of $[a,b]$. We assume
without loss of generality that this cover is disjoint (since for any two
intervals $A$ and $B$ which overlap, we can replace $A$ with $A \setminus
B$ and still remain with an open cover of $[a,b]$). 
Due to the compactness of $[a,b]$, we can extract
a finite subcover $\left\{ (t_j, t_{j +1}) \right\}$ of $[a,b]$. This subcover
will be disjoint, since the original covering of $[a,b]$ was assumed to be
disjoint. Furthermore,
observe that if $t \in (t_{j}, t_{j+1})$,
then $(t_{j}, t_{j+1}) \in V_{t}$. For each interval $(t_j, t_{j +1})$
choose any $t_{j}^* \in (t_{j}, t_{j +1})$, and define the step-function
%
%
\begin{equation*}
	\begin{split}
		f_n(t) =
		\begin{cases}
      f(t_{j}^{*}), \quad & t \in (t_{j},t_{j+1}) \subset V_{t}
		\\
		0 & \text{otherwise}.
	\end{cases}
	\end{split}
\end{equation*}
%
Fix $t \in [a,b]$. By construction, $t \in V_{t}$ and
$t \in (t_{j}, t_{j+1})$ for some $j$. Hence, picking $s \in (t_{j},
t_{j+1})$, we have $s \in V_{t}$ and so
%
%
\begin{equation*}
	\begin{split}
		\|f(t)-f_n(t)\|_X
    & = \| f(t) - f(t_{j}^{*}) + f(t_{j}^{*}) - f_{n}(t)  \|_{X}
    \\
    & \le \| f(t) - f(t_{j}^{*}) \|_{X} + \| f(t_{j}^{*}) - f_{n}(t) \|_{X}
    \\
    & = \| f(t) - f(t_{j}^{*}) \|_{X}
    \\
    & \le \frac{1}{n}.
	\end{split}
\end{equation*}
%
%
Since $t$ was chosen arbitrarily, the proof is complete. 
\end{proof}
%
\begin{framed}
\begin{remark}
  \label{rem:unif}
If $f_{n}: [a,b] \to X$ is any sequence of step functions converging pointwise
in $t$ to $f$ in $X$, it converges \emph{uniformly} in $t$, due to the
compactness of $[a,b]$. 
\end{remark}
\end{framed}
%
\begin{definition}[Banach Space Integration for Continuous Functions]
  \label{def:banach}
  Let $X$ be a Banach space, $f: [a,b] \to X$ continuous, and $f_{n} \to f$ in
  $L^{\infty}([a,b], X)$. Then 
  %
  %
  \begin{equation*}
  \begin{split}
    \int_{a}^{b} f(t) dt \doteq \lim_{n \to \infty} \int_{a}^{b} f_{n}(t) dt
  \end{split}
  \end{equation*}
  %
  %
\end{definition}
%
where the convergence is in the topology of $X$. We now show that this
definition for Banach space integration is
well-defined. More precisely, we will first show that $\int_{a}^{b}
f_{n}(t) dt$ is Cauchy in $X$. Then, we will show the convergence is independent
of the step functions chosen. Proceeding, we first prove the following lemma.
%
%
%
%
%%%%%%%%%%%%%%%%%%%%%%%%%%%%%%%%%%%%%%%%%%%%%%%%%%%%%
%
%
%                tri inequality
%
%
%%%%%%%%%%%%%%%%%%%%%%%%%%%%%%%%%%%%%%%%%%%%%%%%%%%%%
%
%
\begin{lemma}
  For a step function $f: [a,b] \to X$, we have
  %
  %
  \begin{equation*}
  \begin{split}
    \| \int_{a}^{b} f(t) dt \|_{X} \le \int_{a}^{b} \| f(t) \|_{X} dt.
  \end{split}
  \end{equation*}
  %
  %
\label{lem:tri-ineq-int}
\end{lemma}
%
%
%
%
\begin{proof}
%
%
\begin{equation*}
\begin{split}
  \| \int_{a}^{b} f(t) dt \|_{X} &
  = \| \sum_{j=1}^{m}\alpha_{j} (t_{j+1} - t_{j})
  \|_{X}
  \\
  & \le  \sum_{j=1}^{m} \| \alpha_{j} \|_{X} (t_{j+1} - t_{j})
  \\
  & = \int_{a}^{b} \| f(t) \|_{X} dt.
\end{split}
\end{equation*}
%
%
\end{proof}
%
%
%
%
%%%%%%%%%%%%%%%%%%%%%%%%%%%%%%%%%%%%%%%%%%%%%%%%%%%%%
%
%
%                uniqueness
%
%
%%%%%%%%%%%%%%%%%%%%%%%%%%%%%%%%%%%%%%%%%%%%%%%%%%%%%
%
Definition \ref{def:banach} is then well-defined due to the following two
lemmas.
%
%
%
%
%%%%%%%%%%%%%%%%%%%%%%%%%%%%%%%%%%%%%%%%%%%%%%%%%%%%%
%
%
%                conv
%
%
%%%%%%%%%%%%%%%%%%%%%%%%%%%%%%%%%%%%%%%%%%%%%%%%%%%%%
%
%
\begin{lemma}[Convergence] Let $f_{n}: [a,b] \to X$ be a sequence of simple
  functions converging to $f$ in $L^{\infty}([a,b], X)$. Then
  %
  %
  \begin{equation*}
  \begin{split}
    \int_{a}^{b}f_{n}(t) dt
  \end{split}
  \end{equation*}
  %
  %
  is Cauchy in $X$. 
\label{lem:conv}
\end{lemma}
%
%
%
%
\begin{proof}
  \begin{equation}
    \begin{split}
  \| \int_{a}^{b} f_{n}(t) dt - \int_{a}^{b} f_{m}(t) dt \|_{X}
  & = \| \int_{a}^{b}\left[ f_{n}(t) - f_{m}(t) \right]dt \|_{X}, \quad \text{Lemma
  \ref{lem:indep-part}}
  \\
  & \le \int_{a}^{b} \| f_{n}(t) - f_{m}(t) \|_{X} dt,
  \quad \phantom{m} \text{Lemma \ref{lem:tri-ineq-int}}
  \\
  & \le \int_{a}^{b} (\| f(t) - f_{n}(t) \|_{X} + \| f(t) - f_{m}(t) \|_{X})dt.
\end{split}
\end{equation}
%
Fix $\ee$. By Remark \ref{rem:unif}, we see that there exists $N$ such
that for $m, n > N$
%
%
\begin{equation*}
\begin{split}
& \| f(t) - f_{n}(t) \|_{X} \le \ee, \quad \forall t \in [a, b].
\\
& \| f(t) - f_{m}(t) \|_{X} \le \ee, \quad \forall t \in [a, b].
\end{split}
\end{equation*}
Hence,
%
%
\begin{equation*}
\begin{split}
\| \int_{a}^{b} f_{n}(t) dt - \int_{a}^{b} f_{m}(t) dt \|_{X} \le 2(b-a) \ee.
\end{split}
\end{equation*}
%
Since $\ee$ can be chosen arbitrarily small, the proof is complete.
\end{proof}
%
%
\begin{lemma}[Independence]
  Let $f_{n}(t)$, $g_{n}(t) : [a,b] \to X$
  be two step functions converging to $f$ in $L^{\infty}([a,b], X)$.
  If
  %
  %
  \begin{equation*}
  \begin{split}
    & \int_{a}^{b} f_{n}(t) dt \xrightarrow{X} F
    \\
    & \int_{a}^{b} g_{n}(t) dt \xrightarrow{X} G
  \end{split}
  \end{equation*}
  %
  %
  then
  %
  %
  \begin{equation*}
  \begin{split}
    F =G.
  \end{split}
  \end{equation*}
  %
  %
\label{lem:well-def}
\end{lemma}
%
%
%
%
%
%
\begin{proof}
%
%
\begin{equation*}
\begin{split}
  \| F - G \|_{X} & = \| \int_{a}^{b} f(t) dt - \int_{a}^{b} g(t) dt \|_{X}
  \\
  & = \| \int_{a}^{b}\left[ f(t) - g(t) \right]dt \|_{X}, \quad \text{Lemma
  \ref{lem:indep-part}}
  \\
  & \le \int_{a}^{b} \| f(t) - g(t) \|_{X} dt,
  \quad \phantom{m} \text{Lemma \ref{lem:tri-ineq-int}}
  \\
  & \le \int_{a}^{b} (\| f(t) - f_{n}(t) \|_{X} + \| f(t) - g_{n}(t) \|_{X})dt.
\end{split}
\end{equation*}
%
%
Fix $\ee$. By Remark \ref{rem:unif}, we see that there exists $N$ such
that for $n > N$
%
%
\begin{equation*}
\begin{split}
& \| f(t) - f_{n}(t) \|_{X} \le \ee, \quad \forall t \in [a, b]
\\
& \| f(t) - g_{n}(t) \|_{X} \le \ee, \quad \forall t \in [a, b].
\end{split}
\end{equation*}
%
%
Hence
%
%
\begin{equation*}
\begin{split}
\| F - G \|_{X} \le 2(b-a) \ee.
\end{split}
\end{equation*}
%
%
Since $\ee$ can be chosen arbitrarily small, we see that we must have $F = G$.
\end{proof}
%
%
We conclude the section by showing that Banach space integration and
differentiation are inverse operations.
%
%
%%%%%%%%%%%%%%%%%%%%%%%%%%%%%%%%%%%%%%%%%%%%%%%%%%%%%
%
%
%                FTC
%
%
%%%%%%%%%%%%%%%%%%%%%%%%%%%%%%%%%%%%%%%%%%%%%%%%%%%%%
%
%
\begin{lemma}[Fundamental Theorem of Calculus]
Let $f \in C\left( [a,b], X \right)$, where $X$ is a Banach space. For $a \le x
\le b$, let
%
%
\begin{equation*}
\begin{split}
  F(t) = \int_{a}^{t}f(s)ds.
\end{split}
\end{equation*}
%
%
Then the map $t \mapsto F(t)$ is Lipschitz continuous from $[a,b]$ to $X$.
Furthermore, $F$ is differentiable at all $t_{0} \in [a,b]$, and
%
%
\begin{equation}
  \label{fund-thm-calc-diff}
\begin{split}
  F'(t_{0})= f(t_{0}).
\end{split}
\end{equation}
\label{lem:fund-thm-calc}
\end{lemma}
%
%
%
%
\begin{proof}
We first establish the continuity of the map $t \mapsto F(t)$. Assume without
loss of generality that $t_{1} \le t_{2}$. We have
%
%
\begin{equation*}
\begin{split}
  \| F(t_{1}) - F(t_{2}) \|_{X}
  & = \| \int_{a}^{t_{1}}f(s)ds -
  \int_{a}^{t_{2}} f(s)ds \|_{X}
  \\
  & = \| \int_{t_{1}}^{t_{2}} f(s) ds \|_{X}
  \\
  & \le \int_{t_{1}}^{t_{2}}\| f(s) \|_{X} ds.
\end{split}
\end{equation*}
%
%
Since $[a,b]$ is compact and $f$ is continuous, we have $\| f(s) \|_{X} \le M$
for all $s \in [a,b]$. Hence
%
%
\begin{equation*}
\begin{split}
  \| F(t_{1}) - F(t_{2}) \| \le M(t_{2} - t_{1}).
\end{split}
\end{equation*}
%
%
which establishes the Lipschitz continuity of the map $t \mapsto F(t)$. To
establish \eqref{fund-thm-calc-diff}, we apply the definition of Banach space
differentiation given in \eqref{diff-limit-simp} and see that
%
%
\begin{equation*}
\begin{split}
  \| \frac{F(t_{0} + h) - F(t_{0})}{h}  - f(t_{0}) \|_{X}
  & = \| \frac{\int_{a}^{t_{0}+h}f(s)ds - \int_{a}^{t_{0}}f(s)ds}{h} -
  f(t_{0}) \|_{X}
  \\
  & = \| \int_{a}^{t_{0}+h}\frac{f(s)}{h}ds  -
  \int_{t_{0}}^{t_{0} + h} \frac{f(t_{0})}{h} ds \|_{X}
  \\
  & = \frac{1}{h} \| \int_{t_{0}}^{t_{0} + h}[f(s) - f(t_{0})]ds \|_{X}
  \\
  & \le \frac{1}{h} \int_{t_{0}}^{t_{0} + h} \| f(s) - f(t_{0})
  \|_{L^{\infty}\left( [t_{0}, t_{0} + h], X \right)}
  \\
  & = \| f(s) - f(t_{0})
  \|_{L^{\infty}\left( [t_{0}, t_{0} + h], X \right)}
  \\
  & = \lim_{h \to 0} \sup_{0 \le k \le 1} \| f(t_{0} + kh) - f(t_{0})
  \|_{X}.
\end{split}
\end{equation*}
%
%
Note that since $f$ is continuous on $[a,b]$, and $[a,b]$ is compact, it follows
that $f$ is uniformly continuous on $[a,b]$. Hence, for fixed $\delta > 0$,
there exists $\ee_{\delta} > 0$ such that for $h < \ee_{\delta}$, we have
%
%
\begin{equation*}
\begin{split}
  \| f(t_{0} + kh) - f(t_{0}) \|_{X} < \delta, \ \text{for all } \  0 \le k \le 1.
\end{split}
\end{equation*}
%
%
Hence
%
%
%
\begin{equation*}
\begin{split}
  \sup_{0 \le k \le 1} \| f(t_{0} + kh) - f(t_{0}) \|_{X} < \delta, \quad h <
  \ee_{\delta}.
\end{split}
\end{equation*}
%
%
Since $\delta$ can be chosen arbitrarily small, we conclude that
%
\begin{equation*}
\begin{split}
  \lim_{h \to 0} \sup_{0 \le k \le 1} \| f(t_{0} + kh) - f(t_{0})
  \|_{X} = 0,
\end{split}
\end{equation*}
%
%
concluding the proof.
\end{proof}
%
%
%
%
%
%
%
%
%%%%%%%%%%%%%%%%%%%%%%%%%%%%%%%%%%%%%%%%%%%%%%%%%%%%%
%
%
%				 Contraction Mapping Theorem
%
%
%%%%%%%%%%%%%%%%%%%%%%%%%%%%%%%%%%%%%%%%%%%%%%%%%%%%%
%
%
\section{Contraction Mapping Lemma}
Let $\left( X, d \right)$ be a metric space. A mapping $T: X \to X$ is called a
\emph{contraction} if there exists an $\alpha$, $0 < \alpha <1$ such that
%
%
\begin{equation*}
	\begin{split}
		d(Tx, Ty) \le \alpha d(x,y), \qquad \forall x, y \in X.
	\end{split}
\end{equation*}
%
%
\begin{framed}
\begin{remark}
	Observe that a contraction mapping is always continuous.
\end{remark}
\end{framed}
%
%
\begin{framed}
\begin{example}
	The function $T(x) = 0.1x^2$ defines a contraction mapping in the metric space
	$X = [-4, 4]$ with distance $d(x,y) = |x-y|$. 
\end{example}
\end{framed}
	%
	%
	%
	%
	%%%%%%%%%%%%%%%%%%%%%%%%%%%%%%%%%%%%%%%%%%%%%%%%%%%%%
	%
	%
	%				Banach Space Fixed Point Theorem 
	%
	%
	%%%%%%%%%%%%%%%%%%%%%%%%%%%%%%%%%%%%%%%%%%%%%%%%%%%%%
	%
	%
	\begin{lemma}[Contraction Mapping Lemma]
		\label{lem:fixed-point}
	Let $(X,d)$ be a complete metric space, and $T: X \to X$ a contraction
	mapping. Then $T$ has a unique fixed point in $X$. That is, there is a unique
	point $x^* \in X$ such that $Tx^* = x^*$. Furthermore, if $x_0$ is any point
  in $X$, and we define the sequence $x_{n+1} = Tx_n$, then $x_n \xrightarrow{X} x^*$ as $n
	\to \infty$.
	\end{lemma}
	%
	%
  \begin{proof} First we show uniqueness. If $x^*$ and $x^{**}$ are two fixed
	points, then
	%
	%
	\begin{equation*}
		\begin{split}
			d(x^*, x^{**}) = d(Tx^*, Tx^{**}) \le \alpha d(x^*, x^{**}) \implies d(x^*,
			x^{**}) = 0 \implies x^* = x^{**}.
		\end{split}
	\end{equation*}
	%
	%
To prove existence, we observe that since $X$ is complete it suffices to show
that $x_n$ is Cauchy. A repeated application of the
contraction inequality gives
%
%
\begin{equation*}
	\begin{split}
		d\left( x_{n+1},x_n \right)
		& = d\left( Tx_n, Tx_{n-1} \right)
		\\
		& \le \alpha d\left( x_n, x_{n-1} \right)
		\\
		& \le \alpha^2 d\left( x_{n-1}, x_{n-2} \right)
		\\
		& \cdots
		\\
		& \le \alpha^n d\left( x_1, x_0 \right).
	\end{split}
\end{equation*}
%
%
Hence
%
%
\begin{equation*}
\begin{split}
  d\left( x_{n+k},x_n \right)
  & \le (\alpha^{n } +\alpha^{n+1} + \cdots +
  \alpha^{n+k-2} + \alpha^{n+k-1})d(x_{1}, x_{0}) 
  \\
  & \to 0 \ \text{as} \ n \to \infty
\end{split}
\end{equation*}
%
%
since $0 < \alpha < 1$. 
\end{proof}
%
%%%%%%%%%%%%%%%%%%%%%%%%%%%%%%%%%%%%%%%%%%%%%%%%%%%%%
%
%
%				 Proof of ODE Theorem
%
%
%%%%%%%%%%%%%%%%%%%%%%%%%%%%%%%%%%%%%%%%%%%%%%%%%%%%%
%
%
\section{Proof of \hyperref[ode-thm]{Banach Space ODE Theorem}}
	%
	%
	Since $f(t)$ is Lipschitz continuous on $E$ for $t \in (-a, a)$, it is
	continuous on $E$. Hence, using the theory of Banach space integration
  developed earlier and the fundamental theorem of calculus,
  one can check that for $t \in (-a, a)$, the initial value problem
  \eqref{ode-thm-eq}-\eqref{ode-thm-init-data} is equivalent to the
	integral equation
	%
	%
	\begin{equation*}
		\begin{split}
			u(t) = u_0 + \int_0^t f(s, u(s) ) \ ds.
		\end{split}
	\end{equation*}
	%
	%
	%
	%
	Let $[-h, h] \subset (-a, a)$ be a closed interval,
	and choose $r$ such that
	$$B_r(u_0) \doteq \left\{ u \in X: \|u - u_0\|_X \le r \right\}$$
	is a subset of $E$. Define 
	$$V_h = \left\{ \text{all maps} \; \;  v: [-h, h]
		\to B_r(u_0)\right\}.$$ Then $V_h$ is a Banach
		space under the norm $$| | | v | | | = \sup_{s \in [-h, h]} | |v(s) | |_{X}.$$ Let $T$ be a map
	acting on $V_h$ via the relation $$Tv(t) = u_0 + \int_0^t f(s, v(s) ) \ ds.$$
	Recalling \autoref{lem:fixed-point}, we see that to complete the proof it will be enough to show that $T$ is a contraction on
	$V_h$, for suitably small $h >0$. First, note that for $v \in V_h$, we have
	%
	%
\begin{equation}
	\label{cont-map-into}
	\begin{split}
		| | | Tv - u_0| | |
		& = | | | \int_0^t f(s, v(s) ) \ ds | | |
		\\
		& \le  \int_0^t | | | f(s, v(s) ) | | | ds
		\\
		& \le   \int_0^t M_1 ds
		\\
		& =  t M_1
		\\
		& \le  h M_1.
	\end{split}
\end{equation}
%
%
Choosing $h \le r/M_1$, it follows that $T: V_h \to V_h$. 
%Since $[-h, h] \subset (-a, a)$, we can find a compact set $C$ such that $(-h,
%h) \subset C \subset (-a, a)$. Since $f$ is a continuous function of two
%variables, $f( (C \times K) )$ is compact, and so \eqref{cont-map-into} gives
%the estimate
%%
%%
%\begin{equation*}
%	\begin{split}
%		| | | Tv | | | \le \|u_0\|_X + Mh < \infty
%	\end{split}
%\end{equation*}
%
%
Next, note that for any two points $v_1, v_2 \in
V_h$, we have
%
%
\begin{equation}
	\label{cont-part-1}
	\begin{split}
		| | | Tv_2 - Tv_1 | | | 
		& = | | | \int_0^t \left[ f(s, v_2(s) ) - f(s, v_1 (s) ) \right]ds | | |
		\\
		& \le \int_0^t | | | f(s, v_2(s) ) - f(s, v_1 (s) ) | | | ds
		\\
		& = \int_0^t \sup_{s \in [-h, h]} \|f(s, v_2(s) ) - f(s, v_1 (s) )\|_X \ ds
		\\
		& \le \int_0^t M_2 \sup_{s \in [-h, h]} \| v_2(s) - v_1 (s)\|_X \
		ds
		\end{split}
\end{equation}
%
%
where the last step follows from \eqref{stronger-ode}.
Hence
\begin{equation*}
	\begin{split}
		| | | Tv_2 - Tv_1 | | | 
		& \le M_2 h \sup_{s \in [-h, h]} \| v_2(s)  - v_1 (s) \|_{ X}
		\\
		& = M_2 h   | | | v_2 - v_1 | | |_X.  
	\end{split}
\end{equation*}
%
%
Restricting $h < 1/(M_2)$, we obtain
\begin{equation*}
	\begin{split}
		| | | Tv_2 - Tv_1 | | | & \le c | | | v_2 - v_1 | | |_X, \qquad c <1. 
	\end{split}
\end{equation*}
Hence, for $h < \min\left\{r/M_1, 1/M_2  \right\}$, $T$ is a contraction on
$V_h$.  This completes \\ the proof. \qed
%
%
\appendix
\section{Differentiation in Banach Spaces}
\begin{definition}
	\label{def:diff}
	Let $X,Y$ be Banach spaces, and consider the map $f: X \to Y$.
	Then $f$ is differentiable at $x_0$ if there
	is a continuous linear map $Df(x_0): X \to Y$ such that
	%
	%
	%
	%
	\begin{equation}
		\label{diff-limit}
		\begin{split}
			\lim_{h \to 0} \frac{\|f(x_0+ h) - f(x_0) -
			Df(x_0)(h) \|_Y}{\|h\|_{X}} = 0
		\end{split}
	\end{equation}
	%
	%
	This map, which we call the \emph{total derivative} of $f$ at $t_0$, is 
	unique. If $Df(x_0)$ exists for all $x_0 \in X$,
	then we say that $f$ is
	\emph{differentiable} in $X$. If $f$ is differentiable in $X$, and 
	$\|Df(x_0 + h) - Df(x_0) \|_Y \to 0$ as $\|h\|_{X} \to 0$ for all $x_0 \in X$,
	then we say that $f$ is \emph{continuously differentiable in $X$}. 	
\end{definition}
	%
  \begin{framed}
	\begin{remark}
		\label{rem:usual-diff}
		Consider the map $f: \rr \to \rr$. Then \autoref{def:diff} 
		coincides with the usual definitions of
		differentiability and continuous differentiability for maps
		from $\rr $ to $\rr$.
		To see this, assume $f$ is differentiable at $x_0$.
		Let us first verify that the map  $D(x_0): \rr ~\to~\rr$
		defined by $Df(x_0)(x) =
		xf'(x_0)$ is the total derivative of $f$. Assume without loss of generality
		that $h \to 0^+$. Then
		%
		%
		\begin{equation}
			\label{2}
			\begin{split}
				 \lim_{h \to 0^+} \frac{| f( x_0 + h) - f(x_0) -
				Df(x_0)(h) |}{|h|}
				 & = \lim_{h\to 0^+} \left |\frac{f(x_0+h) -
				f(x_0)}{h} - \frac{Df(x_0)(h)}{h}  \right |
				\\
				 & =\lim_{h \to 0^+} \left |\frac{f(x_0+h) - f(x_0)}{h} -
				f'(x_0) \right | = 0.
			\end{split}
		\end{equation}
		%
		%
%
%
For any $a \in \rr$, define the map $T_{a} \in L(\rr , \rr)$ by
$T_{a}(x) = ax$. Then the map $a \mapsto T_a$ is an isometric
isomorphism from $\rr$ to $L( \rr, \rr)$. Hence, 
we may identify $f'(x_0)$ with $T_{f'(x_0)} = Df(x_0)$,
which by \eqref{2} is the total derivative of $f$.  Using this identification,
it follows that $f$ is continuously
differentiable in $\rr$ if and only if $f'$ is a continuous function of
$x_0$, for all $x_0 \in \rr$. \qed
%
\end{remark}
\end{framed}
%
%
In general, for an arbitrary Banach space $X$, open interval $(a,b) \subset
\rr$, and map $f:(a,b) \to X$, we can identify $Df(x_0)$ with an
element of $X$ via the following.
%
%
%%%%%%%%%%%%%%%%%%%%%%%%%%%%%%%%%%%%%%%%%%%%%%%%%%%%%
%
%
%			Lemma Isometry	
%
%
%%%%%%%%%%%%%%%%%%%%%%%%%%%%%%%%%%%%%%%%%%%%%%%%%%%%%
%
%
\begin{lemma}
	\label{lem:isometry} Let $(a,b) \subset \rr$ be an open interval, $X$ a Banach
	space, with $x \in X$. Define the map $T_x \in L\left ( (a,b) , X \right )$ by
	$T_x(t_0) = x t_0$. Then the map $x \mapsto T_x$ is an
	isometric isomorphism from
	$X$ to $L((a,b) , X)$. 
\end{lemma}
%
%
\begin{proof} Note that 
%
%
\begin{equation*}
	\begin{split}
		| | | T_x | | |
		& = \sup_{|t_0| = 1} \| T_x (t_0) \|_X
		= \| x t_0\|_X
		= \|x\|_X.
	\end{split}
\end{equation*}
%
%
Hence, the map $x \mapsto T_x$ is an isometry from $X$ into $L((a,b),
X)$. It remains to show that it is onto. Let $U \in L( (a,b), X)$. Then
by linearity
%
%
\begin{equation*}
	\begin{split}
		U(t_0) = U(1)t_0. 
	\end{split}
\end{equation*}
%
%
Hence, $U = T_{U(1)}$, completing the proof. 
\end{proof}
%
%
Applying the lemma, we see that if $Df(t_0)$ exists for a map $f: (a,b) \to X$,
then it can be viewed as an
element of $X$. Similarly, if $Df(t_0)$ exists for all $t_0 \in (a,b)$, then
we may view the map $t_0 \to Df(t_0)$ as an
element of $L( (a,b), X)$. Hence, for a
map $f:(a,b) \to X$, we see that Definition \ref{def:diff-simp} is an equivalent
reformulation of \autoref{def:diff}. 
Lastly, we include the following.
%
%
\begin{proposition}
		(Chain Rule) Let $X,Y,Z$ be Banach spaces, $g: Y \to Z$ and $f:
		X \to Y$ continuously differentiable maps. Then for all $x_0,
		x \in X$ we have \begin{equation*} (g \circ f)' (x_0) =
			g'(f(x_0)) \circ (f'(x_0)).
		\end{equation*} 
	\end{proposition}
  \begin{proof} Choose $0<\ee<1$. Since 
			\begin{equation*}
				\lim_{s \to 0} \frac{\|f (x_0 + s) - f(x_0) -
				f'(x_0)(s)\|}{\|s\|} = 0
			\end{equation*}
			we can find $\delta > 0$ such that 
			\begin{equation*}
				\frac{\|f (x_0 + s) - f(x_0) -
				f'(x_0)(s)\|}{\|s\|} < \ee
			\end{equation*}
			for $\|s\| < \delta$. This implies
			\begin{equation*}
				\|f(x_0 + s) - f(x_0) - f'(x_0)(h) \| < \ee \|s\|.
			\end{equation*}
			Hence, we can write
			\begin{equation*}
				f(x_0 + s) = f(x_0) + f'(x_0)(s) + O_1(s)
			\end{equation*}
			with $\|O_1(s)\| < \ee \|s\|$. Similarly, for $\|t\| <
			\delta$ we have 
			\begin{equation*}
				g(x_0 + t) = g(x_0) + g'(x_0)(t)+ O_2(t)
			\end{equation*}
			with $\|O_2(t)\| < \ee \|t\|$. In addition, since $f'(x_0)$
			and $g'(f(x_0))$ are continous linear operators, we have
			\begin{equation*}
				\begin{split}
					&\|f'(x_0)(s) \| < a \|s \|,
					\\
					&\|g'(x_0)(t) \| < b \|t \|
				\end{split}
			\end{equation*}
			for some constants $a , b$. Hence
			\begin{equation*}
				\|f'(x_0)(s) + O_1(s) \| < (a + 1) \|s\|
			\end{equation*}
			for $\|s\| < \delta$. Then for $\|s \| <
			\delta/(a+1)$ we have
			\begin{equation}
				\label{key_estimate}
				\begin{split}
					&\| O_2 [f'(x_0)(h) + O_1(s)] \| < \ee(a+1)\|s \|,
					\\
					& \|g'(f(x_0))(O_1(s))\| < b \ee \|s\|.
				\end{split}
			\end{equation}
			Hence
			\begin{equation*}
				\begin{split}
					h(x_0 + s) &= g(f(x_0 +s)
					\\
					&= g(f(x_0) + f'(x_0)(s) + O_1(s))
					\\
					&= g(f(x_0)) + g'(f(x_0))(f'(x_0)(s) + O_1(s)) 
					+ O_2(f'(x_0)(s) + O_1(s))
					\\
					&= g(f(x_0)) + g'(f(x_0))f'(x_0)(s) +
					g'(f(x_0))(O_1(s)) + O_2(f'(x_0)(s) + O_1(s))
				\end{split}
			\end{equation*}
			Applying \eqref{key_estimate}, we conclude that
			\begin{equation*}
				\begin{split}
				&h(x_0 + s) = g(f(x_0)) + g'(f(x_0))(f'(x_0)(s)) +
				O_3(s), \; 
			\text{where}
				\\
				&\|O_3(s)\| < (a + b + 1) \ee \|s\|
			\end{split}
			\end{equation*}
     completing the proof. 
   \end{proof}
%
%
\providecommand{\bysame}{\leavevmode\hbox to3em{\hrulefill}\thinspace}
\providecommand{\MR}{\relax\ifhmode\unskip\space\fi MR }
% \MRhref is called by the amsart/book/proc definition of \MR.
\providecommand{\MRhref}[2]{%
  \href{http://www.ams.org/mathscinet-getitem?mr=#1}{#2}
}
\providecommand{\href}[2]{#2}
\begin{thebibliography}{Kna05b}

\bibitem[Die69]{Dieudonne_1969_Foundations-of-}
J.~Dieudonn{\'e}, \emph{Foundations of modern analysis}, 1969.

\bibitem[Fol99]{Folland_1999_Real-analysis}
Gerald~B. Folland, \emph{Real analysis}, 1999.

\bibitem[Jos98]{Jost-1998-Postmodern-analysis}
J{\"u}rgen Jost, \emph{Postmodern analysis}, 1998.

\bibitem[Kna05a]{Knapp:2005rm}
A.W. Knapp, \emph{Advanced real analysis}, Birkhauser, 2005.

\bibitem[Kna05b]{Knapp:2005yg}
\bysame, \emph{Basic real analysis}, Birkhauser, 2005.

\bibitem[Rud76]{Rudin:1976uq}
Walter Rudin, \emph{Principles of mathematical analysis}, third ed.,
  McGraw-Hill Book Co., New York, 1976, International Series in Pure and
  Applied Mathematics. \MR{0385023 (52 \#5893)}

\bibitem[Yos80]{Yosida:1980fk}
K{\^o}saku Yosida, \emph{Functional analysis}, sixth ed., Grundlehren der
  Mathematischen Wissenschaften [Fundamental Principles of Mathematical
  Sciences], vol. 123, Springer-Verlag, Berlin, 1980. \MR{617913 (82i:46002)}

\end{thebibliography}
%
%
%\bibliography{/Users/davidkarapetyan/math/bib-files/references.bib}
%\bibliographystyle{amsalpha}
\end{document}
