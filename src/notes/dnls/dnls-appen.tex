%
\section{Proofs of Lemmas and Estimates}
\begin{proof}[Proof of \cref{1lem:cutoff-loc-soln}]
%
%
\begin{equation*}
	\begin{split}
		\lim_{t_{n} \to t} \|u(\cdot, t) - u(\cdot, t_{n})\|_{\dot{H}^s(\ci)} 
		& = \lim_{t_{n} \to t} \|\psi(t) u(\cdot, t) - \psi(t_n) u(\cdot,
		t_{n})\|_{\dot{H}^s(\ci)} 
		\\
		& = \lim_{t_n \to t} \left[ \sum_{n \in \zzdot}| n |
		^{2s} | \psi(t)  \wh{u}(n, t) - \psi(t_n) \wh{ u}(n, t_n) |^2 \right]^{1/2}
		\\
		& = \lim_{t_n \to t} \left[ \sum_{n \in \zzdot} | n |^{2s} | \int_{\rr} (e^{it \tau} - e^{it_{n} \tau}) \wh{\psi u}(n,
		\tau) d \tau |^2 \right]^{1/2}.
	\end{split}
\end{equation*}
		It is clear that
		%
		%
		\begin{equation*}
			\begin{split}
				| n |
				^{2s} | \int_{\rr} (e^{it \tau} - e^{it_{n}\tau}) \wh{\psi u}(n, \tau) d \tau |^2 
		& \le 4  | n |^{2s} \left ( \int_{\rr} |\wh{\psi u}(n, \tau)| d \tau
		\right )^2 
	\end{split}
\end{equation*}
and 
%
%
\begin{equation*}
	\begin{split}
 \sum_{n \in \zzdot} | n |^{2s} \left ( \int_{\rr} |\wh{\psi u}(n, \tau)| d \tau
		\right ) ^2 
		& = \| |n |^s \wh{\psi u}\|_{\dot{\ell}_n^2 L_\tau^1}
		\\
		& \le \|\psi u \|_{Y^s}^2 
	\end{split}
\end{equation*}
which is bounded by assumption.
Applying dominated convergence completes the proof. 
\end{proof}
%
%
%\begin{proof}[Proof of \cref{1lem:schwartz-mult}]
%Note that
%%
%%
%\begin{equation*}
	%\begin{split}
		%\wh{\psi f}\left( n, \tau \right)
		%& = \wh{\psi}(\cdot) * \wh{f}(n,
		%\cdot)(\tau)
		%= \int_\rr \wh{\psi}(\tau_1) \wh{f} \left( n, \tau - \tau_1 \right) 
		%d\tau_1
	%\end{split}
%\end{equation*}
%%
%%
%and hence
%%
%%
%\begin{equation}
	%\label{11b}
	%\begin{split}
		%\|\psi f\|_{\dot{X}^s} 
		%& = \left( \sum_{n \in \zzdot} |n|^{2s} \int_\rr \left( 1 + | \tau -
		%n^{m} | \right) | \int_\rr \wh{\psi}(\tau_1) \wh{f}\left( n, \tau -
		%\tau_1
		%\right)  d \tau_1 d \tau |^2 \right)^{1/2}
		%\\
		%& \le \left( \sum_{n \in \zzdot} |n|^{2s} \int_\rr \left( 1 + | \tau -
		%n^{m }
		%|
		%\right) \left( \int_\rr \wh{\psi}\left( \tau_1 \right) \wh{f}\left( n,
		%\tau - \tau_1
		%\right)  d \tau_1 d \tau \right)^2 \right)^{1/2}.
	%\end{split}
%\end{equation}
%%
%%
%Using the relation
%%
%%
%\begin{equation*}
	%\begin{split}
		%1 + | \tau - n^{m } |
		%& = 1 + | \tau + \tau_1 - n^{m} |
		%\\
		%& \le 1 + | \tau_1 | + | \tau - \tau_1 - n^{m} |
		%\\
		%& \le \left( 1 + | \tau_1 | \right)\left( 1 + | \tau - \tau_1 -
		%n^{m} | \right)
	%\end{split}
%\end{equation*}
%%
%%
%we obtain
%%
%%
%\begin{equation*}
	%\begin{split}
		%\eqref{11b}
		%& \le \left( \sum_{n \in \zzdot} |n|^{2s} \right.
		%\\
		%& \times \left . \int_\rr \left(
		%\int_\rr \left( 1 + | \tau_1 | \right)^{1/2} | \wh{\psi}(\tau_1) |
		%\left( 1 + | \tau - \tau_1 - n^{m} | \right)^{1/2} \wh{f}\left( n, \tau
		%- \tau_1
		%\right)d \tau_1
		%\right)^2 d \tau \right)^{1/2}
	%\end{split}
%\end{equation*}
%%
%%
%which by Minkowski's inequality is bounded by
%%
%%
%\begin{equation}
	%\label{12b}
	%\begin{split}
		%& \left( \sum_{n \in \zzdot} |n|^{2s}  \right.
		%\\
		%& \times \left. \left( \int_\rr \left[ \int_\rr
		%\left( 1 + | \tau_{1} | \right) | \wh{\psi}(\tau_1) |^2 \left( 1 + |
		%\tau - \tau_1 - n^{m} |
		%\right) | \wh{f}\left( n, \tau - \tau_1 \right) |^2 d \tau_1 
		%\right]^{1/2} d \tau \right)^2 \right)^{1/2}.
	%\end{split}
%\end{equation}
%%
%%
%Using the change of variable $\tau - \tau_1 \to \lambda$ gives
%%
%%
%\begin{equation*}
	%\begin{split}
		%\eqref{12b}
		%& = \left( \sum_{n \in \zzdot} |n|^{2s}\right.
		%\\
		%& \times \left.  \left( \int_\rr \left[
		%\int_\rr \left( 1 + | \tau_1 | \right) | \wh{\psi}\left( \tau_1
		%\right) |^2 \left( 1 + | \lambda - n^{m} | \right) | \wh{f} \left( n,
		%\lambda
		%\right)|^2 d \tau_1 \right]^{1/2} d \lambda \right)^2 \right)^{1/2}
		%\\
		%& =  \left( \sum_{n \in \zzdot} |n|^{2s} \right.
		%\\
		%& \times \left. \left( \int_\rr \left( 1 + |
		%\tau_1 |
		%\right)^{1/2} | \wh{\psi}(\tau_1) | d \tau_1 \left[ \int_\rr \left( 1 + |
		%\lambda - n^{m} |
		%\right) | \wh{f}\left( n, \lambda \right) |^2 d \lambda \right]^{1/2}
		%\right)^2 \right)^{1/2}
		%\\
		%& = c_{\psi} \left( \sum_{n \in \zzdot} |n|^{2s} \left( \left[ \int_\rr
		%\left( 1 + | \lambda - n^{m} | \right) | \wh{f}\left( n, \lambda
		%\right) |^2 d \lambda
		%\right]^{\cancel{1/2}} \right)^{\cancel{2}} \right)^{1/2}
		%\\
		%& = c_{\psi} \|f\|_{\dot{X}^s},
	%\end{split}
%\end{equation*}
%%
%%
%concluding the proof. 
%\end{proof}
%
%
%
%
%
%
\begin{proof}[Proof of \cref{1lem:number-theory1}]
First note that
%
\begin{equation*}
		| - n^m + n_1^m + n_2^m|
		 = 3 | n | |n_1 | |n_2 |.
\end{equation*}
%
%
Hence, it will be enough to show that for $c \ge 0$
%
%
\begin{equation*}
	\begin{split}
		| n | |n_1 | |n_2 | \gtrsim | n |^{\frac{2 + c}{2}}| n_1
		|^{\frac{2-c}{2}}| n_2 |^{\frac{2-c}{2}}
	\end{split}
\end{equation*}
%
%
or, dividing through on both sides by $|n| | n_1 | | n_2 |$ and rearranging terms
%
%
\begin{equation*}
	\begin{split}
		| n |^{c/2} \lesssim | n_1 |^{c/2} | n_2 |^{c/2}.
	\end{split}
\end{equation*}
%
%
But
%
%
\begin{equation*}
	\begin{split}
		| n |^{c/2} &= | n_1 + n_2 |^{c/2}
		\\
		& \le (| n_1 | + |n_2|)^{c/2} 
		\\
		& \le (2\max\{|
		n_1 |, | n_2 |)^{c/2}
		\\
		& \le (2|
		n_1 | | n_2 |)^{c/2}
		\\
		& = 2^{c/2} | n_1 |^{c/2} | n_2 |^{c/2}
	\end{split}
\end{equation*}
completing the proof.
\end{proof}
%
%
%
%
\begin{proof}[Proof of \cref{1lem:number-theory}] Define
%
\begin{equation*}
	\begin{split}
		| - n^{m} + n_1^{m} + n_2^{m }|
		& = | n_{1}^{m} - n^{m} + (n-n_{1})^{m}| 
		\\
		& \doteq f(n).
	\end{split}
\end{equation*}
%
%
For fixed $n_1$, the absolute minima
of $f(n)$ occurs at $n = 1+n_{1}$ ($n = n_1$ is not available by assumption). Next, note that
%
%
\begin{equation*}
	\begin{split}
		f(1+ n_{1}) = | n_{1}^{m} - (1 + n_{1})^m + 1 |
		& = | (1 + n_{1} )^{m} - n_{1}^{m} -1 |.
	\end{split}
\end{equation*}
We now seek a lower bound for the right hand side. By symmetry we may assume
$n_1 >0$ without loss of generality.
%
%
\begin{framed}
\begin{remark}
	By the term ``symmetry'', we mean that
	\begin{equation*}
	\begin{split}
	| [1 + (-n_1)]^m - (-n_1)^m -1 |
	& = | (1 - n_1)^m + n_1^m -1 |
	\\
	& = | (1 + p_1)^m + (-p_1)^m -1 |, \qquad p_1 = -n_1
	\\
	& = | (1 + p_1)^m - (p_1)^m -1 |.
	\end{split}
\end{equation*}
%
%
\end{remark}
\end{framed}
%
%
Then 
%
%
\begin{equation*}
	\begin{split}
	| (1 + n_{1} )^{m} - n_{1}^{m} -1 |
	& = | \sum_{1 \le k \le m-1} c_{k} n_1^{k}|, \qquad \{c_k\} \in
	\mathbb{N}\setminus 0
	 \\
	 & = \sum_{1 \le k \le m-1} c_{k} n_1^{k}
	 \\
	 & \ge c_{m-1}  n_1^{m-1}
	 \\
	 & = c_{m-1}  n_1^{c} n_1^{m-1-c}
	 \\
	 & \gtrsim (1 + n_1)^{c}  n_1^{m-1-c}
	 \\
	 & = n^{c} n_1^{m-1-c}. 
 \end{split}
\end{equation*}
%
%
Since we assumed $n_1 >0$ without loss of generality, it follows that 
%
%
\begin{equation*}
	\begin{split}
		f(n) \gtrsim |n|^{c} | n_1 |^{m-1-c}. 
	\end{split}
\end{equation*}
%
%
But since $f(n)$ is symmetric in $n_1$ and $n_2$, a similar argument shows that
%
%
\begin{equation*}
	\begin{split}
		f(n) \gtrsim |n|^{c} | n_2 |^{m-1-c}. 
	\end{split}
\end{equation*}
%
%
Therefore,
%
%
\begin{equation*}
	\begin{split}
		f(n) \gtrsim | n |^{c}| n_1 |^{\frac{m-1-c}{2}} | n_2 |^{\frac{m-1-c}{2}}
	\end{split}
\end{equation*}
%
%
completing the proof. 
\end{proof}
%
%
\begin{proof}[Proof of \cref{1lem:calc}]
%
%
%
By the change of variable $\theta \mapsto a/2 + x$, we have
%
%
\begin{equation*}
	\begin{split}
		\int_{\rr} \frac{1}{(1 + | \theta |)(1 + | a - \theta |)}d \theta
	= \int_{\rr} \frac{1}{(1 + |  a/2 + x |)(1 + | a/2 - x |)}d x.
	\end{split}
\end{equation*}
%
%
Hence, it suffices to show that
%
%
\begin{equation*}
	\begin{split}
		\int_{\rr} \frac{1}{(1 + | a - \theta |)(1 + | a + \theta |)}d \theta
		\lesssim \frac{\log(2 + | a |)}{1 + | a |}.
	\end{split}
\end{equation*}
%
%
Let us leave the case $a = 0$ for last. By symmetry, the cases $a<0$ and $a >0$
are equivalent. Hence, to cover the case $a \neq0$, we may assume
without loss of generality that $a >0$.
%
%
Then
\begin{equation}
	\label{1a1}
	\begin{split}
		& \int_{\rr} \frac{1}{(1 + | a - \theta |)(1 + | a + \theta |)}d \theta
		\\
		& = \int_{| \theta| \le a+1 } \frac{1}{(1 + | a - \theta |)(1 + | a + \theta
		|)}d \theta + \int_{| \theta| \ge a+1 } \frac{1}{(1 + | a - \theta |)(1 + |
		a + \theta |)}d \theta.
	\end{split}
\end{equation}
Estimating the second integral of \eqref{1a1}, we have
\begin{equation*}
	\begin{split}
		& \int_{| \theta| \ge a+1 } \frac{1}{(1 + | a - \theta |)(1 + | a + \theta
		|)}d \theta 
		\\
		& = \int_{\theta \ge a + 1} \frac{1}{(1 + \theta-a)(1 + \theta+a)} d \theta
		+ \int_{\theta \le -a -1} \frac{1}{(1 + \theta - a) (1 + \theta + a)}d \theta
		\\
		& = \frac{1}{2a} \int_{\theta \ge a + 1} \left[ \frac{1}{1 + \theta -a} -
		\frac{1}{1 + \theta+a} \right] d \theta
		+ \frac{1}{2a} \int_{\theta \le -a-1} \left[ \frac{1}{1 + \theta+a}
		-\frac{1}{1 + \theta -a} \right] d \theta
		\\
		& = \frac{1}{a} \log(1+a)
		\\
		& \lesssim \frac{\log(2 + |a|)}{1 + | a |}.
	\end{split}
\end{equation*}
To evaluate the first integral of \eqref{1a1}, we split into the cases $a \le \theta \le
a+1$, $-a \le \theta \le 0$, $0 \le \theta \le a$, and $a \le \theta \le a+1$.
However, note that 
%
%
\begin{equation*}
	\begin{split}
		& \int_{a}^{a+1} \frac{1}{(1 + | a - \theta |)(1 + | a + \theta |)}d \theta =
		\int_{-a-1}^{-a} \frac{1}{(1 + | a - \theta |)(1 + | a + \theta |)}d \theta,
		\\
		& \int_{0}^{a} \frac{1}{(1 + | a - \theta |)(1 + | a + \theta |)}d \theta =
		\int_{-a}^{0} \frac{1}{(1 + | a - \theta |)(1 + | a + \theta |)}d \theta.
	\end{split}
\end{equation*}
%
%
Therefore, we need only consider the cases $a \le \theta \le a+1$ and $0 \le
\theta \le a$. For the case $a \le \theta \le a+1$, we have have
%
%
\begin{equation*}
	\begin{split}
		\int_{a}^{a+1} \frac{1}{(1 + | a-\theta |)(1 + | a + \theta |)}d \theta
		& = \int_{a}^{a+1} \frac{1}{(1 + \theta -a)(1 + a + \theta)}d \theta
		\\
		& = \frac{1}{2a} \int_{a}^{a+1} \left[ \frac{1}{1 + \theta -a} -
		\frac{1}{1 + \theta + a}  \right]d \theta
		\\
		& =\frac{1}{2a} \log\left( \frac{1 + \theta -a}{1 + \theta + a} \right) \Big
		|_a^{a+1}
		\\
		& = \frac{1}{2a} \log\left( \frac{2a+1}{a+1} \right)
		\\
		& \lesssim\frac{\log 2}{2a}
		\\
		& \lesssim \frac{\log(2 + | a |)}{1 + | a |}.
	\end{split}
\end{equation*}
%
%
while for the case $0 \le \theta \le a$, we have
%
%
\begin{equation*}
	\begin{split}
		\int_{0}^{a} \frac{1}{(1 + | a - \theta |)(1 + | a + \theta |)}d \theta
		& = \int_{0}^{a} \frac{1}{(1 +  a - \theta )(1 +  a + \theta )}d \theta
		\\
		& = \frac{1}{2(1 + a)} \int_{0}^{a} \left[ \frac{1}{1 + a - \theta} +
		\frac{1}{1 + a + \theta} \right]d \theta
		\\
		& = \frac{1}{2(1 + a)} \log \left( \frac{1 + a + \theta}{1 + a - \theta}
		\right) \Big |_{0}^{a}
		\\
		& = \frac{\log\left( 1 + 2a \right)}{2\left( 1 + a \right)}
		\\
		& \lesssim \frac{\log(2 + | a |)}{1 + | a |}.
	\end{split}
\end{equation*}
%
%
This completes the proof for the case $a \neq 0$. Lastly, for the case
$a =0$, we use dominated convergence and our preceding work to
conclude that
%
%
\begin{equation*}
	\begin{split}
		\int_{\rr} \frac{1}{(1 + | \theta|)^2} d \theta
		& = \lim_{a \to 0}
		\int_{\rr} \frac{1}{(1 + | a - \theta |)(1 + | a + \theta |)}d \theta
		\\
		& \lesssim \lim_{a \to 0} \frac{\log(2 + | a |)}{1 + | a |}
		\\
		& =  \log 2
		\\
		& = \frac{\log(2 + | 0 |)}{1 + | 0 |} 
	\end{split}
\end{equation*}
%
which completes the proof.
%
\end{proof}
%
\begin{proof}[Conservation of the $L_x^2$ norm] 
We have
%
%
\begin{equation*}
	\begin{split}
		\frac{d}{dt} \int_\ci | u |^2  dx
		& = \int_\ci \frac{d}{dt} | u |^2  dx
		\\
		& = \int_\ci \frac{d}{dt} \left( u \overline{u} \right)  dx
		\\
		& = \int_\ci \left( u \p_t \overline{u} + \overline{u} \p_t u \right) dx
		\\
		& = \int_\ci \left( u \overline{\p_t u} + \overline{u} \p_t u \right)dx.
	\end{split}
\end{equation*}
%
%
Substituting in $\p_t u = i\left( \p_x^{m} u + | u |^2 u \right)$ we obtain
%
%
\begin{equation*}
	\begin{split}
		& \int_{\ci} \left\{ u\left[ -i\left( \p_x^{m} \overline{u} + | u |^2
		\overline{u} \right) \right] + \overline{u}\left[ i\left( \p_x^{m} u + | u
		|^2 u \right) \right] \right\}dx
		\\
		& = \int_\ci \left[ -iu \p_x^{m} \overline{u} - i| u |^4 + i \overline{u}
		\p_x^{m} u + i | u |^4 \right]dx
		\\
		& = i \int_{\ci}\left( \overline{u} \p_x^{m} u - u \p_x^{m } \overline{u}
		\right)dx.
	\end{split}
\end{equation*}
%
%
Integrating by parts $m/2$ times and using
the spatial periodicity of $u$, the right
hand side simplifies to
%
%
\begin{equation*}
	\begin{split}
		i \int_\ci \left( \p_x^{m/2} \overline{u} \p_x^{m/2} u - \p_x^{m/2} u
		\p_x^{m/2 } 
		\overline{u} \right) dx = 0.
	\end{split}
\end{equation*}
%
%
Therefore, the $L_x^2(\ci)$ norm of solutions to the dNLS is conserved. 
\end{proof}
