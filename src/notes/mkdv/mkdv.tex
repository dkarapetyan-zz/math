\chapter{Well Posedness for the mKDV}
%
%
%
%
\section{Introduction}
We consider the modified Korteweg-de Vries (mKDV) initial value problem (ivp)
%
%
\begin{gather}
	\label{mmKDV-eq}
	\p_t u + \p_x^{m} u + \lambda u \p_x u = 0,
	\\
	\label{mmKDV-init-data}
	u(x,0) = u_0(x), \quad x \in \ci, \ t \in \rr.
\end{gather}
%
%
where $m \in \{3, 5, 7,\dots \}$ and $\lambda \in \{-1, 1\}$.
%
%
\begin{definition}
	We say that the mKDV ivp \eqref{mmKDV-eq}-\eqref{mmKDV-init-data} is
	\emph{locally well posed} in
	$X$ if 
	\begin{enumerate}
		\item For every $\vp(x) \in
	B_R$ there exists $T>0$ depending on $R$ and a unique function
	\\
	$u \in C([-T, T],
	X)$ satisfying \eqref{mmKDV-eq} for all $t \in [-T, T]$. 
\item The flow map $u_0 \mapsto u(t)$ is locally uniformly continuous. That is, if $u_0
	\in B_R$, $\{u_{0,n}\} \subset B_R$, and 
	$\|u_0 - u_{0, n} \|_{H^{s}(\ci)} \to 0$, then there exists $T >0$ depending
	on $R$ such that $\|u(\cdot, t) - u_{n}(\cdot,t) \|_{X} \to
	0$ for $t \in [-T, T]$. 
	\end{enumerate}
	%Otherwise, we say that the mKDV ivp is \emph{ill-posed}.
\end{definition}
%
%
We are now prepared to state the following result.
%
%%%%%%%%%%%%%%%%%%%%%%%%%%%%%%%%%%%%%%%%%%%%%%%%%%%%%
%
%
%				 Well Posedness Theorem
%
%
%%%%%%%%%%%%%%%%%%%%%%%%%%%%%%%%%%%%%%%%%%%%%%%%%%%%%
%
%
\begin{theorem}
	\label{mthm:main}
	The mKDV ivp is well-posed in $\dot{H}^s(\ci)$ for $s \ge \frac{1-m}{4}$.  
\end{theorem}
%
%
%%%%%%%%%%%%%%%%%%%%%%%%%%%%%%%%%%%%%%%%%%%%%%%%%%%%%
%
%
%				Outline
%
%
%%%%%%%%%%%%%%%%%%%%%%%%%%%%%%%%%%%%%%%%%%%%%%%%%%%%%
%
%
\section{Outline of the Proof of Main Theorem}
%
%
%
%
%
We first derive a weak formulation of the mKDV ivp. 
Let $\ci = [0, 2 \pi]$, and use
the following notation for the Fourier transform
%
%
%
%
\begin{equation}
	\label{mfour-trans-pde}
	\begin{split}
		\widehat{f}(n) = \int_{\ci} e^{-ix n} f(x) \, dx.
	\end{split}
\end{equation}
Let $w(x,t) = u \p_x u$. Applying 
the Fourier transform to the mKDV ivp in the space variable we obtain 
%
%
\begin{gather*}
	\p_t \widehat{u}(n, t) = (-1)^{\frac{m-3}{2}}i n^m \widehat{u}(n, t) - \lambda i  
	\widehat{w} (n, t),
	\\
	\widehat{u} (n,0) = \widehat{\vp}(n)
\end{gather*}
%
%
which is a globally well-defined relation in $t$ 
and $n$. Note that by time reversal, we similarly have 
\begin{gather*}
	-\p_t \widehat{u}(n, -t) = (-1)^{\frac{m-3}{2}}i n^m \widehat{u}(n, -t) - \lambda i  
	\widehat{w} (n, -t),
\end{gather*}
or
\begin{gather*}
	\p_t \widehat{u}(n, -t) = (-1)^{\frac{m-1}{2}}i n^m \widehat{u}(n, -t) + \lambda i  
	\widehat{w} (n, -t).
\end{gather*}
Since the sign of $\lambda$ plays no role in the proof of local well-posedness,
we now assume $\frac{m-1}{2}$ to be even without loss of generality. 
Multiplying \eqref{mfour-trans-pde} by the integrating factor $e^{itn^m}$ then yields
%%
%%
\begin{equation*}
	\begin{split}
		\left[ e^{ it n^m} \widehat{u}(n) \right]_t = i
		 e^{ it n^m} \widehat{w} (n, t).	
	\end{split}
\end{equation*}
%
%
Integrating from $0$ to $t$, we obtain
%
%
\begin{equation*}
	\begin{split}
		\wh{u}(n, t) = \wh{\vp}(n) e^{- it n^m} + i  
		\int_0^t e^{ i(t' - t) n^m} \wh{w}(n, t') \ 
		dt'.
	\end{split}
\end{equation*}
%
%
Therefore, by Fourier inversion 
%
%
\begin{equation}
	\label{mmKDV-integral-form}
	\begin{split}
		u(x,t) & = \sum_{n \in \zz} \wh{\vp}(n) e^{i\left( xn - t n^m 
		\right)} 
		\\
		& + i \sum_{n \in \zz} \int_0^t e^{i\left[ xn + \left( t' - t 
		\right) n^m \right]} \wh{w}(n, t') \ dt'.
	\end{split}
\end{equation}
%
%
Then it is immediate that \eqref{mmKDV-integral-form} is a weaker 
restatement of the Cauchy-problem \eqref{mmKDV-eq}-\eqref{mmKDV-init-data}, 
since by construction any classical solution of the mKDV 
ivp is a solution to \eqref{mmKDV-integral-form}. 
\\
\\
%
%
We now derive an integral 
equation global in $t$ and equivalent to \eqref{mmKDV-integral-form} for $t 
\in [-T, T]$. Let $\psi(t)$ be a cutoff function symmetric about the 
origin such that $\psi(t) = 1$ for $|t| \le T$ and $\text{supp} \, \psi 
= [-2T, 2T ]$. Multiplying the right hand side of expression
$\eqref{mmKDV-integral-form}$ by $\psi(t)$, we obtain
%
%
\begin{equation}
	\begin{split}
		\label{mcutoff-int-eq}
		u(x, t)
		& = \frac{1}{2 \pi} \psi(t) \sum_{n \in \zz} e^{i(xn - t n^m)} \widehat{\vp}(n) 
		\\
		& + \frac{i }{2 \pi} \psi(t) \int_0^t \sum_{n \in \zz} 
		e^{i\left[ xn + (t - t')n^m \right]} \wh{w}(n, t') \ dt'.
	\end{split}
\end{equation}
%
%
Noting that $e^{i\left( xn + tn^m \right)}$ 
does not depend on $t'$, we may rewrite
%
%
\begin{equation}
	\label{mpre-prim-int-form}
	\begin{split}
		& \frac{i }{2 \pi} \psi(t) \int_0^t \sum_{n \in \zz} 
		e^{i\left[ xn + (t - t') n^m \right]} \wh{w}(n, t') \ dt'
		\\
		& = \frac{i}{2 \pi} \psi(t) \sum_{n \in \zz} e^{i\left( xn + t 
		 n^m 
		\right)} \int_0^t e^{- it'n^m} \wh{w}(n, t') \ dt'.
	\end{split}
\end{equation}
%%
%%
We remark that this is a \emph{global} relation in $t$. Therefore, by Fourier 
inversion
%
%
%
%
%
%
%
\begin{equation*}
	\begin{split}
		\text{rhs of} \; \eqref{mpre-prim-int-form}
		& = \frac{i}{4 \pi^2} \psi (t) \sum_{n \in \zz} e^{i\left( xn + t 
		 n^m
		\right)} \int_0^t \int_\rr e^{it'\left( \tau - n^m \right) }
		\wh{w}(n, \tau) d \tau dt'
		\\
		& = \frac{i}{4 \pi^2} \psi(t) \sum_{n \in \zz} \int_\rr 
		e^{i\left( xn + tn^m \right)} \frac{e^{it\left( \tau - n^m 
		\right)}-1}{\tau - n^m} \wh{w}(n, \tau) d \tau
	\end{split}
\end{equation*}
%
%
where the last step follows from integration. Substituting
into \eqref{mcutoff-int-eq} we obtain
%
%
\begin{equation}
	\begin{split}
		\label{mcutoff-int-eq-2}
		u(x, t)
		& = \frac{1}{2 \pi} \psi(t) \sum_{n \in \zz} e^{i(xn - tn^m)} \widehat{\vp}(n) 
		\\
		& + \frac{1}{4 \pi^2} \psi(t) \sum_{n \in \zz} \int_\rr
		e^{i(xn + t n^m)} \frac{e^{it(\tau - n^m)}- 1}{\tau - n^m} 
		\wh{w}(n, \tau) \ d \tau.
	\end{split}
\end{equation}
%
%
%
%
%
Next, we localize near the singular curve $\tau =  n^m$.  Multiplying the
summand of the second term of \eqref{mcutoff-int-eq-2} by $1 + \psi(\tau -
n^m) - \psi(\tau -
n^m) $ and
rearranging terms, we have
%
%
\begin{equation*}
	\begin{split}
		 u(x, t)
		& = \frac{1}{2 \pi} \psi(t) \sum_{n \in \zz} e^{i(xn + t n^{m 
		})} \widehat{\vp}(n) 
		\\
		& + \frac{1}{4 \pi^2} \psi(t) \sum_{n \in \zz} \int_\rr e^{ixn}  
		e^{it \tau} \frac{ 1 - \psi(\tau - n^m) 
		}{\tau - n^m} \wh{w}(n, \tau) \ d \tau
		\\
		& - \frac{1}{4 \pi^2} \psi(t) \sum_{n \in \zz} \int _\rr e^{i(xn + 
		t n^m)}
		 \frac{1- \psi(\tau - n^m)}{\tau - n^m} \wh{w}(n, \tau) \ d \tau
		\\
		& + \frac{1}{4 \pi^2} \psi(t) \sum_{n \in \zz} \int_\rr
		e^{i(xn + t n^m)}
		\frac{\psi(\tau - n^m)\left[ e^{it(\tau - n^m)}-1 
		\right]}{\tau - n^m} \wh{w}(n, \tau) \ d \tau
	\end{split}
\end{equation*}
%
%
which by a power series expansion of $[e^{it(\tau - n^m)}-1]$ simplifies  
to
%
%
\begin{align}
	\label{mmain-int-expression-0}
	& u(x, t) 
		\\
		\label{mmain-int-expression-1}
		& = \frac{1}{2 \pi} \psi(t) \sum_{n \in \zz} e^{i(xn + tn^{m 
		})} \widehat{\vp}(n) 
		\\
		\label{mmain-int-expression-2}
		& + \frac{1}{4 \pi^2} \psi(t) \sum_{n\in \zz} \int_\rr e^{ixn}  
		e^{it \tau} \frac{ 1 - \psi(\tau -  n^m) 
		}{\tau -  n^m} \wh{w}(n, \tau) \ d \tau
		\\
		\label{mmain-int-expression-3}
		& - \frac{1}{4 \pi^2} \psi(t) \sum_{n\in \zz} \int_\rr e^{i(xn + 
		t n^m)}
		 \frac{1- \psi(\tau -  n^m)}{\tau -  n^m} \wh{w}(n, \tau) \ d \tau
		\\
		\label{mmain-int-expression-4}
		& + \frac{1}{4 \pi^2} \psi(t) \sum_{k \ge 1} \frac{i^k t^k}{k!}
		\sum_{n \in \zz} \int_\rr e^{i(xn + t n^m )}
		\psi(\tau -  n^m) (\tau -  n^m)^{k-1} \wh{w}(n, \tau)  
		\\
		& \doteq T(u) \notag
\end{align}
%
%
where $T = T_{\vp}$. We now introduce the following spaces. 
%
\begin{definition}
	Denote $\dot{Y}^s$ to be the space of all
	functions $u$ on $\ci \times \rr$ with
	bounded norm
\begin{equation}
	\label{mY-s-norm}
	\begin{split}
		\|u\|_{\dot{Y}^s} = \|u\|_{\dot{X}^s} + \||n|^s \wh{ u}\|_{ \dot{\ell}^2_n L^1_\tau }
	\end{split}
\end{equation}
%
%
%
%
where
%
\begin{equation}
	\label{mX^s-norm}
	\begin{split}
		& \|u\|_{\dot{X}^s}
		= \left ( \sum_{n\in \zz} |n|^{2s} \int_\rr \left ( 1 + | 
		\tau - n^m \right ) | \wh{u} ( n, \tau ) |^2
		\right )^{1/2}
	\end{split}
\end{equation}
and
%
%
\begin{equation}
	\label{mE-norm}
	\| | n |^{s} \wh{u}\|_{ \dot{\ell}^2_n L^1_\tau } = \left[ \sum_{n \in \zzdot}| n |^{2s} \left(
	\int_{\rr}| \wh{u}(n, \tau) |d \tau \right)^{2} \right]^{1/2}.
\end{equation}
%
%
%
%
\end{definition}
The $\dot{Y}^s$ spaces have the following important property, whose proof
is provided in the appendix.
\begin{lemma}
	\label{mlem:cutoff-loc-soln}
	Let $\psi(t)$ be a smooth cutoff function with $\psi(t) =1$ for $t \in [-T, T]$. If
	$\psi(t)u(x,t) \in \dot{Y}^s$, then $u \in C([-T, T], \dot{H}^s(\ci))$.
\end{lemma}
%
%
We will 
show that for initial data $\vp \in \dot{H}^s(\ci)$, $T$ is a contraction on $B_M 
\subset \dot{Y}^s$, where $B_M$ is the ball centered at the origin of radius $M = 
M_{\vp}> 0$, by estimating the $\dot{Y}^s$
norm of \eqref{mmain-int-expression-1}-\eqref{mmain-int-expression-4}. The 
Picard fixed point theorem will
then yield a unique solution to
\eqref{mmain-int-expression-0}-\eqref{mmain-int-expression-4}. An application of
\cref{mlem:cutoff-loc-soln} will then imply the existence of a unique, local
solution $u \in C([-T, T], \dot{H}^s(\ci))$ to the mKDV ivp which coincides with the solution to
\eqref{mmain-int-expression-0}-\eqref{mmain-int-expression-4} on the interval $[-T, T]$. Local Lipschitz continuity of the flow map will follow
from estimates used to establish the contraction mapping. %
%
%%%%%%%%%%%%%%%%%%%%%%%%%%%%%%%%%%%%%%%%%%%%%%%%%%%%%
%
%
%			Proof of Theorem	
%
%
%%%%%%%%%%%%%%%%%%%%%%%%%%%%%%%%%%%%%%%%%%%%%%%%%%%%%
%
%
\section{Proof of Main Theorem}
%
%
%
%%%%%%%%%%%%%%%%%%%%%%%%%%%%%%%%%%%%%%%%%%%%%%%%%%%%%
%
%
%		Estimation of Integral Equality Part 1		
%
%
%%%%%%%%%%%%%%%%%%%%%%%%%%%%%%%%%%%%%%%%%%%%%%%%%%%%%
%
%
%
%
\subsection{Estimate for \eqref{mmain-int-expression-1}.}
%
%
Letting $f(x,t) = \psi(t) \sum_{n \in \zz} e^{i(xn + tn^m)} 
\wh{\vp}(n)$, we have $\wh{f}(n,t) = \psi(t) \wh{\vp}(n) e^{itn^m}$,
from which we obtain
%
%
\begin{equation}
	\label{mfourier-trans-calc}
	\begin{split}
		\wh{f}(n, \tau)
		& = \wh{\vp}(n) \int_\rr e^{-it( \tau - n^m)} 
		\psi(t) \ d t
		= \wh{\psi}(\tau - n^m) \wh{\vp}(n).
	\end{split}
\end{equation}
%
%
%
%
%
%
Since $\wh{\psi}(\xi)$ is Schwartz for $|\xi| \ge T$, we see that 
%
%
\begin{equation}
	\begin{split}
	\label{mmain-int1-est}
		\|\eqref{mmain-int-expression-1}\|_{\dot{Y}^s}
		& = \left (  \sum_{n\in \zz} |n|^{2s} \int_\rr \left( 1 + | \tau - n^m 
		| \right )
		| \wh{\psi}(\tau - n^m) \wh{\vp}(n) |^2 d \tau \right)^{1/2} 
		\\
		& + \left[ \sum_{n \in \zz }| n |^{2s} \left( \int_{\rr} |
		\wh{\psi}(\tau - n^m)\wh{\vp}(n) | d \tau
		\right)^{2} \right]^{1/2}
		\\
		& \le c_{\psi}
		\|\vp\|_{\dot{H}^s(\ci)}.
	\end{split}
\end{equation}
%
%
%
%
\subsection{Estimate for \eqref{mmain-int-expression-2}.}
We now need the following lemma, whose proof is provided in the appendix.
%
%
%%%%%%%%%%%%%%%%%%%%%%%%%%%%%%%%%%%%%%%%%%%%%%%%%%%%%
%
%
%			Schwartz Multiplier	
%
%
%%%%%%%%%%%%%%%%%%%%%%%%%%%%%%%%%%%%%%%%%%%%%%%%%%%%%
%
%
\begin{lemma}
\label{mlem:schwartz-mult}
	For $\psi \in S(\rr)$,
%
%
\begin{equation}
	\label{mschwartz-mult}
	\begin{split}
		\|\psi f \|_{\dot{X}^s} \le c_{\psi} \|f \|_{\dot{X}^s}.
	\end{split}
\end{equation}
%
%
\end{lemma}
%
%
Hence,
%
%
\begin{equation}
	\label{mmain-int2-est-X-s-part}
	\begin{split}
		\|\eqref{mmain-int-expression-2}\|_{\dot{X}^s} 
		& \lesssim 
		\left( \| \sum_{n \in \zz} e^{ixn} \int_\rr 
		e^{it \tau} \frac{ 1 - \psi (\tau - n^m ) 
		}{\tau - n^m} \wh{w}(n, \tau) \ 
		d \tau\|_{\dot{X}^s} \right)^{1/2}
		\\
		& =  \left( \sum_{n \in \zz} |n|^{2s} \int_\rr
		(1 + |\tau - n^m|) \left | \frac{1 - \psi(\tau - n^{2 
		})}{\tau - n^m} 
		\wh{w}(n, \tau) \right |^2 \ d 
		\tau \right)^{1/2}
		\\
		& \le \left( \sum_{n \in \zz} |n|^{2s} \int_{| \tau - n^m| \ge 1}
		(1 + |\tau - n^m|) \frac{|\wh{w}(n, \tau)|^2 }{|\tau - n^m|^2} 
		\ d 
		\tau \right)^{1/2}
		\\
		& \lesssim  \left( \sum_{n \in 
		\zz} |n|^{2s} \int_\rr
		\frac{|\wh{w}(n, \tau) |^2}{1+ |\tau - 
		n^m|} 
		 \ d \tau 
		\right)^{1/2}
		\\
		& \lesssim  \|u\|_{\dot{X}^s}^2
	\end{split}
\end{equation}
%
%
where the last two steps follow from the inequality 
%
\begin{equation}
	\label{mone-plus-ineq}
	\begin{split}
		\frac{1}{|\tau - n^m| } \le \frac{2}{1 + |\tau - n^m| }, 
		\qquad |\tau - n^m| \ge 1
	\end{split}
\end{equation}
%
%
and the following bilinear estimate, whose proof we leave for later.
%
%%%%%%%%%%%%%%%%%%%%%%%%%%%%%%%%%%%%%%%%%%%%%%%%%%%%%
%
%
%				 Bilinear Estimates
%
%
%%%%%%%%%%%%%%%%%%%%%%%%%%%%%%%%%%%%%%%%%%%%%%%%%%%%%
%
%
\begin{proposition}
	\label{mprop:prim-bilin-est}
	For any $s \ge \frac{1-m}{4}$ we have
	\begin{equation}
		\label{mprim-bilin-est}
		\left( \sum_{n \in \dot{\zz}} |n|^{2s} \int_\rr
		\frac{|\wh{w_{fg}}(n, \tau) |^2}{1+ |\tau - 
		n^m| } 
		 \ d \tau 
		\right)^{1/2}
		\lesssim \|f\|_{\dot{X}^s} \|g\|_{\dot{X}^s}
	\end{equation}
	where $w_{fg}(x,t)$ = $\p_x(fg)(x,t)$.
\end{proposition}
Furthermore,
%
%
%
%
\begin{equation}
	\label{mmain-int-expression-2-Y-s-part}
	\begin{split}
		\| | n |^{s} \wh{\eqref{mmain-int-expression-2}} \|_{\dot{\ell}^2_n L^1_\tau}
		& \lesssim 
		\left[ \sum_{n \in \zz}|n|^{2s} \left(
		\int_{\rr}\frac{1 - \psi(\tau - n^m)}{\tau - n^m} \wh{w}(n, \tau) d
		\tau \right)^{2} \right]^{1/2}
		\\
		& \lesssim \|f\|_{X^s} \|g\|_{X^s}
	\end{split}
\end{equation}
%
%
where the last step follows from \eqref{mone-plus-ineq} and
the following bilinear estimate.
%
%%%%%%%%%%%%%%%%%%%%%%%%%%%%%%%%%%%%%%%%%%%%%%%%%%%%%
%
%
%				Second trilinear Estimate 
%
%
%%%%%%%%%%%%%%%%%%%%%%%%%%%%%%%%%%%%%%%%%%%%%%%%%%%%%
%
%
\begin{proposition}
\label{mprop:bilinear-estimate2}
For any $s \ge \frac{1-m}{4}$ we have
%
%
\begin{equation}
	\label{mbilinear-estimate2}
	\begin{split}
		\left( \sum_{n \in \zzdot} |n|^{2s}  \left ( \int_\rr 
		\frac{|\wh{w_{fg}}(n, \tau) |}{1 + | \tau - n^m |}
		 \ d\tau \right)^2  \right)^{1/2} \lesssim \|f\|_{\dot{X}^s} \|g\|_{\dot{X}^s}.
	\end{split}
\end{equation}
\end{proposition}
%
%
Combining \eqref{mmain-int2-est-X-s-part} and
\eqref{mmain-int-expression-2-Y-s-part}, we conclude that
%
%
%
%
\begin{equation}
	\label{mmain-int2-est}
	\begin{split}
		\|\eqref{mmain-int-expression-2}\|_{\dot{Y}^s} \le c_{\psi}\|f\|_{\dot{X}^s} \|g\|_{\dot{X}^s}.
	\end{split}
\end{equation}
%
%
\subsection{Estimate for \eqref{mmain-int-expression-3}.}
Letting $$f(x,t) = \psi(t) \sum_{n \in \zzdot} e^{i\left( xn + tn^m \right)} 
\int_\rr \frac{1 - \psi\left( \lambda - n^m \right)}{\lambda - n^m} 
\wh{w} \left( n, \lambda \right) \ d \lambda,$$ we have
%
%
\begin{equation*}
	\begin{split}
		& \wh{f^x}(n, t) = \psi(t) e^{itn^m} \int_\rr
		\frac{1 - \psi\left( \lambda - n^m \right)}{\lambda - n^m} 
		\wh{w}(n, \lambda) \ d \lambda
	\end{split}
\end{equation*}
and
\begin{equation*}
	\begin{split}
		 \wh{f}\left( n, \tau \right)
		 & = \int_\rr e^{-it\left( \tau - n^m 
		\right)} \psi(t) \int_\rr \frac{1 - \psi\left( 
		\lambda - n^m 
		\right)}{\lambda - n^m} \wh{w}(n, \lambda) \ d \lambda d \tau
		\\
		& = \wh{\psi}\left( \tau - n^m \right) \int_\rr 
		\frac{1 - \psi\left( 
		\lambda - n^m 
		\right)}{\lambda - n^m} \wh{w}(n, \lambda) \ d \lambda.
	\end{split}
\end{equation*}
Therefore,
%
%
\begin{equation*}
	\begin{split}
		& \| \eqref{mmain-int-expression-3} \|_{\dot{X}^s} 
		\\
		& = \left( \sum_{n \in \zzdot} |n|^{2s} \int_\rr \left( 1 + | \tau - n^{m
		} \right ) | | \wh{\psi}\left( \tau - n^m \right) |^2 \ d \tau
		\right.
		\\
		& \times \left . |
		\int_\rr \frac{1 - \psi\left( \lambda - n^m \right)}{\lambda -
		n^m} \wh{w}(n, \lambda) \ d \lambda |^2  \right)^{1/2}
		\\
		& \lesssim \left( \sum_{n \in \zzdot} |n|^{2s} | \int_\rr
		\frac{1 - \psi\left( \lambda - n^m \right)}{\lambda - n^m}
		\wh{w}(n, \lambda) \ d\lambda |^2 \right)^{1/2}
		\\
		& \le \left( \sum_{n \in \zzdot} |n|^{2s}  \left ( \int_\rr
		\frac{1 - \psi\left( \lambda - n^m \right)}{|\lambda - n^m|}
		|\wh{w}(n, \lambda) | \ d\lambda \right )^2 \right)^{1/2}
		\\
		& \le \left( \sum_{n \in \zzdot} |n|^{2s}  \left ( \int_{| \lambda - 
		n^m | \ge 1}
		\frac{|\wh{w}(n, \lambda) | }{|\lambda - n^m|}
		\ d\lambda \right )^2 \right)^{1/2}.
	\end{split}
\end{equation*}
%
%
Applying estimate \eqref{mone-plus-ineq} then gives
%
%%
\begin{equation}
	\label{mmain-int3-est-X-s-part}
	\begin{split}
		\| \eqref{mmain-int-expression-3} \|_{\dot{X}^s}
		& \lesssim \left( \sum_{n \in \zzdot} |n|^{2s}  \left ( \int_\rr
		\frac{|\wh{w}(n, \lambda)| }{1 + |\lambda - n^m|}
		 \ d\lambda \right )^2 \right)^{1/2}
		 \\
		& \lesssim \|u\|_{\dot{X}^s}^2
	\end{split}
\end{equation}
%
%%
where the last step follows from \cref{mprop:bilinear-estimate2}.
Furthermore, 
%
%
\begin{equation}
	\label{mmain-int-estimate-3-Y-s-part}
	\begin{split}
		\| | n |^s \wh{\eqref{mmain-int-expression-3}}\|_{\dot{\ell}^2_n L^1_\tau}
		& = \left[ \sum_{n \in \zzdot} |n|^{2s} \int_{\rr} |
		\wh{\psi}(\tau - n^m) |^{2} \left( \int_{\rr}\frac{1 - \psi(\lambda -
		n^m)}{\lambda - n^m} \wh{w}(n, \lambda) d \lambda \right)^{2} d \tau
		\right]^{1/2}
		\\
		& \le c_{\psi} \left[ \sum_{n \in \zzdot} |n|^{2s} \left(
		\int_{\rr} \frac{1 - \psi(\lambda - n^m)}{\lambda - n^m}
		\wh{w}(n, \lambda) d \lambda
		\right)^{2}\right]^{1/2}
		\\
		& \le 2 c_{\psi} \left[ \sum_{n \in \zzdot} |n|^{2s} \left(
		\int_{\rr} \frac{\wh{w}(n, \lambda) }{1 + |\lambda - n^m|}
		d \lambda
		\right)^{2}\right]^{1/2}
		\\
		& \lesssim \|f\|_{\dot{X}^s} \|g\|_{\dot{X}^s} 
	\end{split}
\end{equation}
%
%
where the last two steps follow from \eqref{mone-plus-ineq} and
\cref{mprop:bilinear-estimate2}, respectively. Combining
\eqref{mmain-int3-est-X-s-part} and \eqref{mmain-int-estimate-3-Y-s-part}, we
conclude that
%
%
\begin{equation}
	\label{mmain-int3-est}
	\begin{split}
		\|\eqref{mmain-int-expression-3}\|_{\dot{Y}^s} 
		\lesssim \|f\|_{\dot{X}^s} \|g\|_{\dot{X}^s}.
	\end{split}
\end{equation}
%
%
%
\subsection{Estimate for \eqref{mmain-int-expression-4}.}
Note that
%
%
\begin{equation}
	\label{m1n}
	\begin{split}
		\eqref{mmain-int-expression-4} \simeq \sum_{k \ge 1}
		\frac{i^k}{k!}g_k(x,t)
	\end{split}
\end{equation}
%
%
where 
%
%
\begin{equation*}
	\begin{split}
		& g_k(x,t) = t^k \psi(t) \sum_{n \in \zzdot} e^{i\left( xn + tn^m
		\right)} h_k(n),
		\\
		& h_k(n) = \int_\rr \psi \left( \tau - n^m \right) \cdot \left(
		\tau - n^m \right)^{k -1} \wh{w}(n, \tau) \ d \tau.
	\end{split}
\end{equation*}
%
%
Hence
%
%
\begin{equation*}
	\begin{split}
		\wh{g_k^x}(n, t) = t^{k} \psi(t) e^{i t n^m} h_k(n)
	\end{split}
\end{equation*}
%
%
which gives
%
%
\begin{equation*}
	\begin{split}
		\wh{g_k}(n, \tau)
		& = h_k(n) \int_\rr e^{-it\left( \tau - n^m \right)}
		t^{k}\psi(t) \ dt
		\\
		& = h_k(n) \wh{t^{k}\psi(t)} \left( \tau - n^m \right).
	\end{split}
\end{equation*}
%
%
Applying this to \eqref{m1n}, we obtain
%
%
\begin{equation}
	\label{m2n}
	\begin{split}
		\|\eqref{mmain-int-expression-4}\|_{\dot{X}^s} 
		& \simeq \left( \sum_{n \in \zzdot} |n|^{2s} \int_\rr \left( 1 + | \tau -
		n^m
		|
		\right) | \wh{\sum_{k \ge 1} \frac{i^k}{k!}g_k(x,t)} |^2 \ d \tau
		\right)^{1/2}
		\\
		& \le \sum_{k \ge 1} \frac{1}{k!}\left( \sum_{n \in \zzdot} |n|^{2s}
		\int_\rr \left( 1 + | \tau - n^m | \right) | \wh{g_k}(n, \tau) |^2 \
		d \tau \right)^{1/2}
		\\
		& = \sum_{k \ge 1} \frac{1}{k!} \left( \sum_{n \in \zzdot} |n|^{2s}
		\int_\rr \left( 1 + | \tau - n^m | \right) | h_k(n) \wh{t^k
		\psi(t)} \left( \tau - n^m \right) |^2 \ d \tau \right)^{1/2}
		\\
		& = \sum_{k \ge 1} \frac{1}{k!} \left( \sum_{n \in \zzdot} |n|^{2s} |
		h_k(n) |^2 \int_\rr \left( 1 + | \tau - n^m | \right) | \wh{t^k
		\psi(t)} \left( \tau - n^m \right) |^2 \ d \tau \right)^{1/2}.
	\end{split}
\end{equation}
%
%
Notice that for fixed $n$, the change of variable $\tau - n^m \to \tau'$
gives
%
%
\begin{equation}
	\label{m3n}
	\begin{split}
		\int_\rr \left( 1 + | \tau - n^m | \right) | \wh{t^{k}
		\psi(t)}\left( \tau - n^m \right) |^2 \ d \tau
		& = \int_\rr \left( 1 + |\tau'| \right) | \wh{t^k \psi(t)}(\tau') |^2 \
		d \tau'
		\\
		& \le \int_\rr \left( 1 + |\tau'| \right)^2 | \wh{t^k \psi(t)}(\tau')
		|^2 \ d \tau'
		\\
		& \lesssim \int_\rr \left( 1 + | \tau' |^2 \right) | \wh{t^{k}
		\psi(t)}(\tau') |^2 \ d \tau'
		\\
		& = \|t^k \psi(t) \|_{H^1(\rr)}^2.
	\end{split}
\end{equation}
%
%
But
%
%
\begin{equation}
	\label{m4n}
	\begin{split}
		\|t^k \psi(t) \|_{H^1(\rr)}^2
		& = \left( \|t^k \psi(t)\|_{L^2(\rr)} + \|\p_t \left( t^k \psi(t)
		\right)\|_{L^2(\rr)} \right)^2
		\\
		& \lesssim \|t^{k}\psi(t) \|_{L^2(\rr)}^2 + \|\p_t \left (t^{k}
		\psi(t) \right )\|_{L^2(\rr)}^2
		\\
		& \le \|t^k \psi(t) \|_{L^2(\rr)}^2 + \|t^k \p_t \psi(t)
		\|_{L^2(\rr)}^2 + \|k t^{k -1} \psi(t) \|_{L^2(\rr)}^2
		\\
		& = c_{\psi} + c_{\psi}' + k^2 c_{\psi}''
		\\
		& \lesssim k^2.
	\end{split}
\end{equation}
%
%
Hence, applying \eqref{m3n} and \eqref{m4n} to \eqref{m2n}, we obtain
%
%%
\begin{equation}
	\label{m5n}
	\begin{split}
		\|\eqref{mmain-int-expression-4} \|_{\dot{X}^s}
		& \lesssim
		\sum_{k \ge 1} \frac{k}{k!} \left( \sum_{n \in \zzdot} |n|^{2s} | h_k(n) |^2 
		\right)^{1/2}
		\\
		& \le \sum_{k \ge 1} \frac{k}{k!}
		 \sup_{k \ge 1} \left( \sum_{n \in \zzdot} |n|^{2s} | 
		h_k(n) |^2 \right)^{1/2}
		\\
		& = \sum_{k \ge 1} \frac{k}{k!}  \sup_{k \ge 1} 
		\left( \sum_{n \in \zzdot} |n|^{2s} \int_\rr 
		\psi\left( \tau - n^m \right) \cdot \left( \tau - n^m 
		\right)^{k -1} \wh{w}(n, \tau) \ d \tau \right)^{1/2}.
	\end{split}
\end{equation}
%
%%
Recall that $\text{supp} \, |\psi| \subset [0, T ]$. Pick $T \le 1$. 
Then $| \psi\left( \tau - n^m \right) \cdot \left( \tau - n^m \right)^{k 
-1} | \le \chi_{| \tau - n^m | \le 1}$ for all $k \ge 1$. Hence, \eqref{m5n} gives
%
%%
\begin{equation*}
	\begin{split}
		\|\eqref{mmain-int-expression-4} \|_{\dot{X}^s} 
		& \lesssim \sum_{k \ge 1} \frac{k}{k!}  \left( \sum_{n \in \zzdot} | 
		\int_{| \tau - n^m  |\le 1} | \wh{w}(n, \tau) \ d \tau |^2 
		\right)^{1/2}
	\end{split}
\end{equation*}
%
%%
which by the inequality
%
%%
\begin{equation*}
	\begin{split}
		\frac{1 + | \tau - n^m |}{1 + | \tau  - n^m |} \le 
		\frac{2}{1 + | \tau - n^m |}, \qquad | \tau - n^m  | \le 1
	\end{split}
\end{equation*}
%
%%
implies
%
%%
\begin{equation}
\label{mmain-int4-est-X-s-part}
	\begin{split}
		\|\eqref{mmain-int-expression-4}\|_{\dot{X}^s}
		& \lesssim \left( \sum_{n \in \zzdot} | \int_{| \tau - n^m| \le 1 }
		\frac{\wh{w}(n, \tau)}{1 + | \tau - n^m |} \ d \tau |^2 
		\right)^{1/2}
		\\
		& \le \left( \sum_{n \in \zzdot} | \int_\rr
		\frac{\wh{w}(n, \tau)}{1 + | \tau - n^m |} \ d \tau |^2 
		\right)^{1/2} \\
		& \le \left( \sum_{n \in \zzdot} \left( \int_\rr 
		\frac{|\wh{w}(n, \tau)|}{1 + | \tau - n^m |}  \ d \tau  \right)^2
		\right)^{1/2} \\
		& \lesssim \|u\|_{\dot{X}^s}^2
	\end{split}
\end{equation}
%
%%
where the last step follows from \cref{mprop:bilinear-estimate2}. Similarly,
we have
%
%
\begin{equation}
\label{mmain-int4-est-Y-s-part}
	\begin{split}
		\| n| |^s \wh{\eqref{mmain-int-expression-4}}\|_{\dot{\ell}^2_n L^1_\tau}
		& \simeq \left[ \sum_{n \in
		\zzdot}|n|^{2s} \left( \int_{\rr} | \sum_{k \ge 1}
		\wh{\frac{i^{k}}{k!}g_{k}(x,t)(n, \tau)} |d \tau \right)^{2} \right]^{1/2}
		\\
		& \le \sum_{k \ge 1} \frac{1}{k!} \left[ \sum_{n \in \zzdot} (1 + | n
		|)^{2s} \left( \int_{\rr} | \wh{g}(n, \tau) | d \tau \right)^{2}
		\right]^{1/2}
		\\
		& = \sum_{k \ge 1} \frac{1}{k!} \left[ \sum_{n \in \zzdot} (1 + | n
		|)^{2s} | h_{k}(n) |^2 \left( \int_{\rr} | \wh{t^{k} \psi(t)}(\tau -
		n^m) |d \tau \right)^{2} \right]^{1/2}
		\\
		& = c_{\psi} \sum_{k \ge 1} \frac{1}{k!} \left[ \sum_{n \in \zzdot} (1 + | n
		|)^{2s} | h_{k}(n) |^2 \right]^{1/2}
		\\
		& \lesssim \|u\|_{\dot{X}^s}^2
	\end{split}
\end{equation}
%
%
where the last step follows from the computations starting from \eqref{m5n}
through \eqref{mmain-int4-est-X-s-part}.
Combining \eqref{mmain-int4-est-X-s-part} and \eqref{mmain-int4-est-Y-s-part}, we
have
%
%
\begin{equation}
\label{mmain-int4-est}
	\begin{split}
		\|\eqref{mmain-int-expression-4}\|_{\dot{Y}^s} \lesssim \|u\|_{\dot{X}^s}^2.
	\end{split}
\end{equation}
%
%
Collecting estimates \eqref{mmain-int1-est}, \eqref{mmain-int2-est}, 
\eqref{mmain-int3-est}, and \eqref{mmain-int4-est}, and recalling 
\eqref{mmain-int-expression-1}-\eqref{mmain-int-expression-4}, we see that
$$\|Tu\|_{\dot{Y}^s} \le c_\psi \left( \|\vp \|_{\dot{H}^s(\ci)} + \|u\|_{\dot{X}^s}^2 \right )$$ 
which by the inequality $\|u\|_{\dot{X}^s} \le \|u\|_{\dot{Y}^s}$ yields the following.
%%
%%%%%%%%%%%%%%%%%%%%%%%%%%%%%%%%%%%%%%%%%%%%%%%%%%%%%
%
%% Contraction Proposition
%				 
%%%%%%%%%%%%%%%%%%%%%%%%%%%%%%%%%%%%%%%%%%%%%%%%%%%%%%
%%
%%
%
\begin{proposition}
\label{mprop:contraction}
Let $s \ge \frac{1-m}{4}$. Then
%
%%
\begin{equation*}
	\begin{split}
		\|Tu\|_{\dot{Y}^s} \le c_\psi \left( \|\vp \|_{\dot{H}^s(\ci)} + \|u\|_{\dot{Y}^s}^2 
		\right).
	\end{split}
\end{equation*}
%
%%
\end{proposition}
We will now use \cref{mprop:contraction} to prove local well-posedness for the 
mKDV ivp. Let $c = c_{\psi}$. For given $\vp$, we may choose $\psi$ such
that 
%
%%
\begin{equation*}
	\begin{split}
		\|\vp\|_{\dot{H}^s(\ci)} \le \frac{3}{16c^2}.
	\end{split}
\end{equation*}
%
%%
Then if $\|u\|_{\dot{Y}^s} \le \frac{1}{4c}$, we have
%
%%
\begin{equation*}
	\begin{split}
		\|T u \|_{\dot{Y}^s} 
		& \le c \left[ \frac{3}{16c^2} + \left( 
		\frac{1}{4c} \right)^2 \right]
		=  \frac{1}{4c}.
	\end{split}
\end{equation*}
%
%%
Hence, $T=T_{\vp}$ maps the ball $B\left( 0, \frac{1}{4c} \right) \subset \dot{Y}^s$ into 
itself. Next, note that
%
%%
\begin{equation*}
	\begin{split}
		Tu - Tv = \eqref{mmain-int-expression-2} + \eqref{mmain-int-expression-3} 
		+ \eqref{mmain-int-expression-4}
	\end{split}
\end{equation*}
%
%%
where now $w = \frac{1}{2} \p_x (u^2 - v^2)$. Rewriting
%
%%
\begin{equation*}
	\begin{split}
	\p_x(u^2 - v^2)	
		& = \p_x[(u-v)(u+v)]
		\end{split}
\end{equation*}
%
%%
and repeating the arguments used to estimate 
\eqref{mmain-int-expression-2}-\eqref{mmain-int-expression-4} (in the
bilinear estimates, we now set $f=u-v$ and $g = u+v$), we obtain
%
%%
%%
\begin{equation}
	\label{m20a}
	\begin{split}
		\|Tu - Tv \|_{\dot{Y}^s}  
		& \le c_\psi \|u -v\|_{\dot{Y}^s} \|u + v \|_{\dot{Y}^s}
		\\
		& \le c_\psi \|u -v\|_{\dot{Y}^s} (\|u\|_{\dot{Y}^s}+ \|v \|_{\dot{Y}^s}).
	\end{split}
\end{equation}
%
%%
If $u, v \in B(0, \frac{1}{4c}) \subset \dot{Y}^s$, it follows that
%
%%
\begin{equation}
	\label{m21a}
	\begin{split}
		\|Tu - Tv \|_{\dot{Y}^s}
		& \le c \|u -v \|_{\dot{Y}^s} \left( \frac{1}{4c} + 
		\frac{1}{4c} \right)
		\\
		& = \frac{1}{2} \|u -v \|_{\dot{Y}^s}. 
	\end{split}
\end{equation}
%
%%
We conclude that $T = T_{\vp}$ is a contraction on the ball $B(0, 
\frac{1}{4c}) \subset \dot{Y}^s$. A Picard iteration and application of 
\cref{mlem:cutoff-loc-soln} then yield a unique, local
solution to the mKDV ivp \eqref{mmKDV-eq}-\eqref{mmKDV-init-data}.
\begin{definition}
	We say that the flow map $u_0 \mapsto u(t)$ is \emph{locally Lipschitz} in a Banach
	space $X$ if for
	$$u_0, v_0 \in B_R \doteq \{f: \|f\|_X < R\},$$ there exist $C, T>0$
	depending on $R$ such that $\|u(\cdot, t) - v(\cdot, t)
	\|_X \le C \|u_{0} - v_0 \|_{X}$ for $t \in [-T, T]$. We
	say the flow map is \emph{locally uniformly
	continuous} in $X$ if for
	$u_0, v_0 \in B_R$ there exists $T >0$ depending on $R$ such that for
	$t \in [-T, T]$, $\|u(\cdot, t) - v(\cdot, t) \|_{X} \to
	0$ if $\|u_0 - v_0 \|_{H^{s}(\ci)} \to 0$. 
\end{definition}
%
%
Clearly any locally Lipschitz flow map is locally uniformly continuous. 
Next, we shall establish local Lipschitz continuity in $\dot{Y}^s$ of the flow
map. Let $\vp_1, \vp_2 \subset \dot{H}^s(\ci)$ be given. Choose $\psi$ such that
$\vp_1, \vp_2 \subset B(0, \frac{15}{64c^{3}})$.  Then there exist $u_1, u_2 \in
\dot{Y}^s$ such that $u_1 = T_{\vp_1}$, $u_2 = T_{\vp_2}$, and so
%
%
\begin{equation}
	\label{mgen-1a}
	\begin{split}
		T_{\vp_1}(u) - T_{\vp_2}(v)
		& = \frac{1}{2\pi} \psi(t) \sum_{n \in
		\zzdot}e^{i\left( xn + tn^m \right)} \wh{\vp_1 - \vp_2}(n)
		\\
		& + \eqref{mmain-int-expression-2} + \eqref{mmain-int-expression-3} +
		\eqref{mmain-int-expression-4} 
	\end{split}
\end{equation}
%
%
where $w = \frac{1}{2} \p_x (u^2 - v^2)$. Using an argument similar to \eqref{mfourier-trans-calc}-\eqref{mmain-int1-est},
we obtain
%
%
\begin{equation}
	\label{mgen-2a}
	\begin{split}
		\| \frac{1}{2\pi} \psi(t) \sum_{n \in
		\zzdot}e^{i\left( xn + tn^m \right)} \wh{\vp_1 - \vp_2}(n)\|_{\dot{Y}^s}
		\le c_\psi \|\vp_{1} - \vp_{2}\|_{\dot{Y}^s}.
	\end{split}
\end{equation}
%
%
Therefore, from \eqref{m21a}-\eqref{mgen-2a}, we obtain
%
%
\begin{equation*}
	\begin{split}
		\|u -v \|_{\dot{Y}^s} = \|T_{\vp_1}(u) - T_{\vp_2}(v) \|_{\dot{Y}^s} \le c_\psi
		\|\vp_{1} - \vp_{2} \|_{\dot{H}^s\left( \ci \right)} +
		\frac{1}{2} \|u -v \|_{\dot{Y}^s}
	\end{split}
\end{equation*}
%
%
which implies
%
%
\begin{equation*}
	\begin{split}
		\frac{1}{2} \|u-v\|_{\dot{Y}^s} \le c_\psi \|\vp_1 - \vp_2 \|_{\dot{H}^s(\ci)}
	\end{split}
\end{equation*}
%
%
or
%
%
\begin{equation*}
	\begin{split}
		\|u -v \|_{\dot{Y}^s} \le 2 c_\psi \|\vp_1 - \vp_2 \|_{\dot{H}^s(\ci)}.
	\end{split}
\end{equation*}
%
%
Applying \cref{mlem:cutoff-loc-soln}, we then obtain
%
%
	 %
	 %
	 \begin{equation*}
		 \begin{split}
			\|u(\cdot, t) -v(\cdot, t) \|_{\dot{H}^s(\ci)} \le 2 c_\psi \|\vp_1 -
			\vp_2 \|_{\dot{H}^s(\ci)}, \qquad t \in [-T, T].
		 \end{split}
	 \end{equation*}
	 %
	 %
Hence, the flow map of the mKDV ivp is locally Lipschitz continuous in
$\dot{H}^s(\ci)$. This
concludes the proof of \cref{mthm:main}. \qquad \qedsymbol
%
%
%
%
\section{Proof of First Bilinear Estimate}
Note first that $|\wh{w_{fg}}(n, \tau) |  = | n\wh{f} *  \wh{g} 
(n, \tau)|$. From this and the conservation of mass, it follows that
%
%
\begin{equation}
	\label{mnon-lin-rep}
	\begin{split}
		| \wh{w_{fg}}(n, \tau)|
		& = | \sum_{\substack{n_1 \neq 0, n_2 \neq 0 \\n_1 +n_2 =n}}  \int_{\tau_1 + \tau_2 = \tau}n\wh{f}\left( n_1,  \tau_1 
\right) \wh{g}\left( n_2, \tau_2  
\right) d \tau_1 d \tau_2 |
\\
& = | \sum_{\substack{n_1 \neq0, n_2 \neq 0 \\n_1 + n_2 =n}}  \int_{\tau_1 + \tau_2 = \tau}n\wh{f}\left( n_1,  \tau_1 
\right) \wh{g}\left( n_2, \tau_2  
\right) d \tau_1 d \tau_2 | 
\\
& \le \sum_{\substack{n_1 \neq0, n_2 \neq 0 \\n_1 + n_2 =n}}   \int_{\tau_1 + \tau_2 = \tau}| n | \times | \wh{f}\left( n_1, \tau_1 
\right) | \times  | \wh{g}\left( n_2, \tau_2 
\right) |   d \tau_1 d \tau_2  
\\
& = \sum_{\substack{n_1 \neq0, n_2 \neq 0 \\n_1 + n_2 =n}} \int_{\tau_1 + \tau_2 = \tau}| n | \times \frac{c_f\left( n_1, \tau_1 
\right)}{|n_1|^s \left( 1 + | \tau_1 - n_1^m | \right)^{1/2}}
\\
& \times \frac{c_{g}\left( n_2, \tau_2 \right)}{|n_2|^s\left( 1 + | \tau_2 -  n_2^m| 
\right)^{1/2}}
  \ d \tau_1 d \tau_2 
\end{split}
\end{equation}
%
%
where 
%
%
\begin{equation*}
	\begin{split}
		c_h(n, \tau) =
		\begin{cases}
			|n|^s \left( 1 + | \tau - n^m |  
			\right)^{1/2} | \wh{h}\left( n, \tau \right) |, \qquad & n \neq 0
		\\
		0, \qquad & n = 0.
	\end{cases}
	\end{split}
\end{equation*}
%
%
From our work above, it follows that 
%
%
\begin{equation}
	\label{mconvo-est-starting-pnt}
	\begin{split}
		 & |n|^s \left( 1 + | \tau - n^m | \right)^{-1/2} | \wh{w_{fg}}\left( 
		n, \tau \right) |
		\\
		& \le \left( 1 + | \tau - n^m | \right)^{-1/2}
		\sum_{\substack{n_1 \neq0, n_2 \neq 0 \\n_1 + n_2 =n}} \int_{\tau_1 + \tau_2 = \tau}\frac{|n|^{s+1}}{|n_1|^s | n_2|^s} 
		\times \frac{c_f(n_1, \tau_1)}{\left( 1 + | \tau_1 - n_1^m | 
		\right)^{1/2}}
		\\
		& \times
		\frac{c_g(n_2, \tau_2)}{\left( 1 + | \tau_2 - n_2^m | 
		\right)^{1/2}}\ d \tau_1 d \tau_2.
	\end{split}
\end{equation}
%
%
Unlike the NLS, we must use the smoothing properties of the
principal symbol $\tau - n^m$ regardless of the choice of $s$, since the quantity
%
%
\begin{equation}
	\label{mconvo-multiplier}
	\begin{split}
		\frac{|n|^{s+1}}{|n_1|^s |n_2|^s }
	\end{split}
\end{equation}
%
%
blows up in general, due to the presence of the extra power of $|n|$ coming from the derivative on
the nonlinearity. To utilize the smoothing effects of the principal symbol, we
first note that 
$$| \tau - n^m - \left( \tau_1 - n_1^m 
+ \tau_2 - n_2^m  \right ) | = | - n^m + n_1^m +
n_2^m| \doteq d_m(n_1, n_2).$$ We will need the following two lemmas, whose
proofs are provided in the appendix.
%
%
%
\begin{lemma}
	\label{mlem:number-theory1}
	Let $n=n_1 + n_2$ and suppose that $n, n_1, n_2\neq
	0$. Then for any integer $c \ge 0$
%
%
\begin{equation}
	\begin{split}
		\label{mnumber-theory1}
		d_3(n_1,n_2) \ge 2^{-c/2} | n |^{\frac{2+c}{2}} | n_{1}
		|^{\frac{2-c}{2}}| n_2 |^{\frac{2-c}{2}}.
	\end{split}
\end{equation}
%
%
\end{lemma}
%
%
%
%
%
%
\begin{lemma}
	\label{mlem:number-theory}
	Let $n=n_1 + n_2$ and suppose that $n, n_1, n_2\neq
	0$. Then for any integer  $m \ge 3$
%
%
\begin{equation}
	\begin{split}
		\label{mnumber-theory}
		d_m(n_1,n_2) \ge b_{m, c } 
		|n|^{c/2} |n_1|^{\frac{m-1-c}{2}} | n_2 |^{\frac{m-1-c}{2}}
		\end{split}
\end{equation}
%
%
where the constant $b_{m,c}$ depends only on $m$ and $c$. 
\end{lemma}
%
%
%
%\begin{remark}
%	The case $-1/2 \le s \le 0$ is delicate, and must be treated differently from
%	the case $s < -1/2$ in order to obtain the optimal well-posedness results.
%	This is the motivation for having two instead of one number theory lemma.
%\end{remark}
%%
%
Let us proceed with the case $m=3$ first; we will then generalize to arbitrary
odd $m \ge 3$. By the pigeonhole principle we must have one of the 
following.
%
%
\begin{align}
	\label{mpigeon-case-1}
	& |\tau - n^3| \ge \frac{d_m(n_1, n_2)}{3} 
		 \\
		\label{mpigeon-case-2}
		& | \tau_1 - n_1^3 | \ge \frac{d_m(n_1, n_2)}{3} 
		 \\
		\label{mpigeon-case-3}
		& | \tau_2 - n_2^3 | \ge \frac{d_m(n_1, n_2)}{3}.
		\end{align}
%
%
By the symmetry of the convolution, it will be enough to consider only
\eqref{mpigeon-case-1} and \eqref{mpigeon-case-2}.
%
%
%
\subsection{Case \eqref{mpigeon-case-1}.} 
Applying \cref{mlem:number-theory1}, we have, for nonzero $ n, n_1, n_2 $
%
%%
\begin{equation}
	\label{mconvo-deriv-bound}
	\begin{split}
		& \frac{|n|^{s+1}}{|n_1|^s 
		| n_2|^s}
		\times
		\frac{1}{(1 + | \tau -n^3 |)^{1/2}}
		\\
		& \lesssim | n |^{s+1}| n_1 |^{-s}| n_2 |^{-s} \times | n
		|^{-\frac{2+c}{4}}| n_1 |^{-\frac{2-c}{4}}| n_2 |^{-\frac{2-c}{4}} 
		\\
		& = | n |^{\frac{4s +2 -c}{4}} | n_1 |^{\frac{-4s -2 +c}{4}} | n_2
		|^{\frac{-4s -2 +c}{4}}
		\\
		& \le 1, \qquad s \ge -1/2.
	\end{split}  
\end{equation}
%
%
\begin{framed}
\begin{remark}
	\label{mrem:s-val}
	The last line follows from the following reasoning: Set $(4s + 2 -c) = 0$
or, equivalently, $-4s -2 +c = 0$. Then for any $c \ge 0$ such that $c = 4s+2$
the left hand side of
\eqref{mconvo-deriv-bound} is bounded by $1$. Of course such a $c$ exists, as
long as $s \ge -1/2$.
\end{remark}
\end{framed}
%
%
%
Hence, recalling \eqref{mconvo-est-starting-pnt} and applying estimates 
\eqref{mpigeon-case-1} and \eqref{mconvo-deriv-bound}, we obtain
%
%
\begin{equation}
	\label{mnon-lin-rep-with-bound}
	\begin{split}
		& |n|^s \left( 1 + | \tau - n^3 | \right)^{-1/2} | 
		\wh{w_{fg}}(n, \tau) | 
		\\
		& \lesssim \sum_{\substack{n_1 \neq0, n_2 \neq 0 \\n_1 + n_2 =n}} \int_{\tau_1 + \tau_2 = \tau}\frac{c_f(n_1, \tau_1)}{\left( 1 + | 
		\tau_1 -  n_1^3| \right)^{1/2}}
		\times \frac{c_g\left( n_2, \tau_2\right)}{\left( 1 + | \tau_2 -n_2^3|
		\right)^{1/2}}
		\\
		& = \wh{C_f C_g}(n, \tau)
	\end{split}
\end{equation}
%
%
where
\begin{equation*}
	\begin{split}
		C_h(x,t) =
		\left[ \frac{c_h(n, \tau)}{\left( 1 + | \tau - n^3 | 
		\right)^{1/2}}\right]^\vee .	
	\end{split}
\end{equation*}
%
%
%
Therefore, from \eqref{mnon-lin-rep-with-bound}, Plancherel, and generalized 
H\"{o}lder, we obtain
%
%
\begin{equation}
	\label{mgen-holder-bound}
	\begin{split}
		& \| |n|^s \left( 1 + | \tau - n^3 | \right )^{-1/2}  \wh{w_{fg}}\left( 
		n, \tau \right) \|_{L^2(\ci \times \rr)}
		\\
		& \lesssim \|\wh{C_f C_g }\left( n, \tau \right) 
		\|_{L^2\left( \zzdot \times \rr \right)}
		\\
		& \simeq \|C_f C_g \|_{L^2\left( \ci \times \rr \right)}
		\\
		& \le \|C_f \|_{L^4(\ci \times \rr)} \|C_g \|_{L^4(\ci \times \rr)}.
	\end{split}
\end{equation}
%
We now need the following Fourier multiplier estimate, whose proof can be found
in~\cite{Himonas:2001db}.
%
\begin{lemma}
	\label{mlem:four-mult-est-L4}
	Let $(x, t) \in \ci \times \rr $ and $(n, \tau) \in \zz \times \rr$ be 
	the dual variables. Let $v$ be a positive even integer. Then there is a 
	constant $c_v > 0$ such that
%
%
\begin{equation}
	\label{mfour-mult-est-L4}
	\begin{split}
		\|f\|_{L^4(\ci \times \rr)} \le c_v \|\left( 1 + | \tau - n^v | 
		\right)^\frac{v+1}{4v} \wh{f} \|_{L^2( \zz \times \rr)}
	\end{split}
\end{equation}
for every test function $f(x, t)$. 
%
%
%
%
\end{lemma}
From the lemma, we see that
%
%
\begin{equation}
	\label{mfour-mult-conseq}
	\begin{split}
		\|C_h\|_{L^4(\ci \times \rr)} 
		& \lesssim \|(1 + | \tau - n^3 |)^{1/2} \wh{C_h}
		\|_{L^2(\zz \times \rr)}
		\\
		& = \|c_{h} \|_{L^2(\zz \times \rr)} 
		\\
		& = \|h \|_{\dot{X}^s}. 
	\end{split}
\end{equation}
%
%
Applying this to \eqref{mgen-holder-bound} we
conclude that
\begin{equation*}
	\begin{split}
		\| |n|^s \left( 1 + | \tau - n^3 | \right ) ^{-1/2} \wh{w_{fg}}\left( 
		n, \tau \right) \|_{L^2(\zzdot \times \rr)}
		& \lesssim \|f\|_{\dot{X}^s} \|g\|_{\dot{X}^s}.
	\end{split}
\end{equation*}
%
%
%
\subsection{Case \eqref{mpigeon-case-2}.}
Applying \cref{mlem:number-theory1}, we have, for nonzero $ n, n_1, n_2 $
\begin{equation}
	\label{mconvo-deriv-bound-2}
	\begin{split}
		& \frac{|n|^{s+1}}{|n_1|^s 
		| n_2|^s}
		\times
		\frac{1}{(1 + | \tau -n^3 |)^{1/2}}
		\\
		& \lesssim | n |^{s+1}| n_1 |^{-s}| n_2 |^{-s} \times | n
		|^{-\frac{2+c}{4}}| n_1 |^{-\frac{2-c}{4}}| n_2 |^{-\frac{2-c}{4}} 
		\\
		& = | n |^{\frac{4s +2 -c}{4}} | n_1 |^{\frac{-4s -2 +c}{4}} | n_2
		|^{\frac{-4s -2 +c}{4}}
		\\
		& \le 1, \qquad s \ge -1/2
	\end{split}  
\end{equation}
%
%
where the last line follows from \cref{mrem:s-val}.
%
%
Hence, recalling \eqref{mconvo-est-starting-pnt} and applying estimate 
\eqref{mconvo-deriv-bound-2}, we obtain
%
%
\begin{equation}
	\label{m1f}
	\begin{split}
		& |n|^s  \left( 1 + | \tau - n^3 | \right)^{-1/2}| \wh{w_{fg}}\left( 
		n, \tau \right) |
		\\
		& \lesssim 
		\left( 1 + | \tau - n^3 | \right)^{-1/2}\sum_{\substack{n_1 \neq0, n_2 \neq 0 \\n_1 + n_2 =n}} \int_{\tau_1 + \tau_2 = \tau}		c_f(n_1, \tau_1)
		\times
		\frac{c_g(n_2, \tau_2)}{\left( 1 + | \tau_2 - n_2^3 | 
		\right)^{1/2}} 
		\\
		& = \left( 1 + | \tau - n^3 | \right)^{-1/2} \wh{\overset{\sim}{C_f} C_g}
	\end{split}
\end{equation}
%
%%
where
%
%
\begin{equation*}
	\begin{split}
		\overset{\sim}{C_h}(x,t) = \left[ c_h(n, \tau) \right]^\vee.
	\end{split}
\end{equation*}
%
%
Hence
%
%%
\begin{equation}
	\label{m3f}
	\begin{split}
		& \| |n|^s \left( 1 + | \tau - n^3 | \right)^{-1/2} \wh{w_{fg}}(n, \tau) 
		\|_{L^2(\zzdot \times \rr)}
		\\
		& \lesssim \|\left( 1 + | \tau - n^3 | \right)^{-1/2} 
		\wh{\overset{\sim}{C_f} C_g } \|_{L^2(\zzdot \times \rr)}
		\\
		& =  \|\left( 1 + | \tau - n^3 | \right)^{-1/2} 
		\wh{\overset{\sim}{C_f} C_g } \|_{L^2(\zz \times \rr)}
		\\
		& \lesssim  \|\overset{\sim}{C_f} C_g  \|_{L^{4/3}(\ci \times \rr)}
	\end{split}
\end{equation}
%
%%
where the last step follows by dualizing \cref{mlem:four-mult-est-L4}. More
precisely, we have the following.
\begin{corollary}
	\label{mcor:four-mult-est-L4}
	Let $(x, t) \in \ci \times \rr $ and $(n, \tau) \in \zz \times \rr$ be 
	the dual variables. Let $v$ be a positive even integer. Then there is a 
	constant $c_v > 0$ such that
%
%
\begin{equation}
	\label{mfour-mult-est-L4*}
	\begin{split}
		\| \left( 1 + | \tau - n^v | 
		\right)^{-\frac{v+1}{4v}}
		\wh{f}\|_{L^2(\zz \times \rr)} \le c_v \|f \|_{L^{4/3}( \ci \times \rr)}.
	\end{split}
\end{equation}
%
%
\end{corollary}
%
Applying H\"{o}lder's inequality to the right hand side of
\eqref{m3f}, we obtain the bound
%
%%
\begin{equation}
	\label{m4f}
	\begin{split}
		\|\overset{\sim}{C_f} \|_{L^2(\ci \times \rr)} \|C_g \|_{L^4\left( \ci 
		\times \rr 
		\right)}. 
	\end{split}
\end{equation}
%
%%
By Plancherel we have
%
%%
%
%%
\begin{equation}
	\label{m5f}
	\begin{split}
		\|\overset{\sim}{C_f} \|_{L^2(\ci \times \rr)}
		& \simeq \|c_f\|_{L^2(\zz \times \rr)}
		\\
		& = \|f \|_{\dot{X}^s}
	\end{split}
\end{equation}
%
%%
while \eqref{mfour-mult-conseq} gives
%
%
\begin{equation}
	\label{m6f}
	\begin{split}
		\|C_g \|_{L^4(\ci \times \rr)} \lesssim \|g\|_{\dot{X}^s}.
	\end{split}
\end{equation}
%
%
We conclude from \eqref{m3f}-\eqref{m6f} that
%
%
\begin{equation*}
	\begin{split}
		\| |n|^s \left( 1 + | \tau - n^3 | \right)^{-1/2} \wh{w_{fg}}(n, \tau) 
		 \|_{L^2(\zzdot \times \rr)}
		 \lesssim \|f\|_{\dot{X}^s} \|g\|_{\dot{X}^s}
	\end{split}
\end{equation*}
%
%
\subsection{Generalizing to arbitrary odd $m >3$.}
%
%
Since $$| \tau - n^m - \left( \tau_1 - n_1^m 
+ \tau_2 - n_2^m  \right ) | = | - n^m + n_1^m +
n_2^m|,$$ by and
the pigeonhole principle we must have one of the 
following.
%
%
\begin{align}
	\label{mpigeon-case-1-gen}
	& |\tau - n^m| \ge \frac{d(n_1, n_2)}{3} 	\\
		\label{mpigeon-case-2-gen}
		& | \tau_1 - n_1^m | \ge \frac{d(n_1, n_2)}{3},		\\
		\label{mpigeon-case-3-gen}
		& | \tau_2 - n_2^m | \ge \frac{d(n_1, n_2)}{3}.
	\end{align}
%
%
By the symmetry of the convolution, it will be enough to consider only
\eqref{mpigeon-case-1-gen} and \eqref{mpigeon-case-2-gen}.
%
%
%
\subsection{Case \eqref{mpigeon-case-1-gen}}
Applying \cref{mlem:number-theory}, we have, for nonzero $ n, n_1, n_2 $
%
%%
\begin{equation}
	\label{mconvo-deriv-bound-gen-case2}
	\begin{split}
		& \frac{|n|^{s+1}}{|n_1|^s 
		| n_2|^s}
		\times
		\frac{1}{(1 + | \tau -n^m |)^{1/2}}
		\\
		& \lesssim | n |^{s+1}| n_1 |^{-s}| n_2 |^{-s} \times | n
		|^{-\frac{c}{2}}| n_1 |^{-\frac{m-1-c}{4}}| n_2 |^{-\frac{m-1-c}{4}} 
		\\
		& = | n |^{\frac{2s+2 -c}{2}} | n_1 |^{\frac{-4s -m + 1+ c}{4}} | n_2
		|^\frac{-4s -m + 1+ c}{4}
		\\
		& \le 1, \qquad s \ge \frac{1-m}{4}.
	\end{split}  
\end{equation}
%
%
\begin{framed}
\begin{remark}
	\label{mrem:gen-s-val}
	 The last line follows from the following reasoning: Set $(2s + 2 -c) \le
0$ and $-4s -m +1 +c \le 0$. Then we want to find $c \ge 0$ such that $2s +2 \le c \le
4s + m-1$ or 
%
%
\begin{equation}
	\label{malgebra-ineq}
	\begin{split}
		2 \le c - 2s \le 2s + m-1.
	\end{split}
\end{equation}
%
%
Note that $c=0$ satisfies \eqref{malgebra-ineq} for $\frac{1-m}{4} \le s \le
-1$. Furthermore, $c = 4 + 4s$ satisfies \eqref{malgebra-ineq} for $s \ge -1$ ($c$ must be non-negative) and $m \ge 5$. 
\end{remark}
\end{framed}
%
Hence, from \eqref{mconvo-est-starting-pnt} and
\eqref{mconvo-deriv-bound-gen-case2},
we obtain 
\begin{equation}
	\label{mconvo-est-starting-pnt-gen-case2}
	\begin{split}
		 & |n|^s \left( 1 + | \tau - n^m | \right)^{-1/2} | \wh{w_{fg}}\left( 
		n, \tau \right) |
		\\
		& \le \left( 1 + | \tau - n^m | \right)^{-1/2}
		\sum_{\substack{n_1 \neq0, n_2 \neq 0 \\n_1 + n_2 =n}} \int_{\tau_1 + \tau_2 = \tau}\frac{|n|^{s+1}}{|n_1|^s | n_2|^s} 
		\times \frac{c_f(n_1, \tau_1)}{\left( 1 + | \tau_1 - n_1^m | 
		\right)^{1/2}}
		\\
		& \times
		\frac{c_g(n_2, \tau_2)}{\left( 1 + | \tau_2 - n_2^m | 
		\right)^{1/2}}\ d \tau_1 d \tau_2
		\\
		& \lesssim \sum_{\substack{n_1 \neq0, n_2 \neq 0 \\n_1 + n_2 =n}} \int_{\tau_1 + \tau_2 = \tau}\frac{c_f(n_1, \tau_1)}{\left( 1 + | \tau_1 - n_1^m | 
		\right)^{1/2}} \times
		\frac{c_g(n_2, \tau_2)}{\left( 1 + | \tau_2 - n_2^m | 
		\right)^{1/2}}\ d \tau_1 d \tau_2, \qquad s \ge \frac{1-m}{4}
		\\
		& = \wh{{C_f} C_g}.
	\end{split}
\end{equation}
Therefore, from \eqref{mconvo-est-starting-pnt-gen-case2}, Plancherel, and generalized 
H\"{o}lder, we obtain
%
%
\begin{equation}
	\label{mgen-holder-bound-case2}
	\begin{split}
		& \| |n|^s \left( 1 + | \tau - n^m | \right )^{-1/2}  \wh{w_{fg}}\left( 
		n, \tau \right) \|_{L^2(\ci \times \rr)}
		\\
		& \lesssim \|\wh{C_f C_g }\left( n, \tau \right) 
		\|_{L^2\left( \zzdot \times \rr \right)}
		\\
		& \simeq \|C_f C_g \|_{L^2\left( \ci \times \rr \right)}
		\\
		& \le \|C_f \|_{L^4(\ci \times \rr)} \|C_g \|_{L^4(\ci \times \rr)}.
	\end{split}
\end{equation}
%
From \cref{mlem:four-mult-est-L4}, we see that
%
%
\begin{equation}
	\label{mfour-mult-conseq-gen-case2}
	\begin{split}
		\|C_\sigma\|_{L^4(\ci \times \rr)} 
		& \lesssim \|(1 + | \tau - n^m |)^{1/2} \wh{C_\sigma}
		\|_{L^2(\zz \times \rr)}
		\\
		& = \|c_{\sigma} \|_{L^2(\zz \times \rr)} 
		\\
		& = \|\sigma \|_{\dot{X}^s}. 
	\end{split}
\end{equation}
%
%
Applying this to \eqref{mgen-holder-bound-case2} we
conclude that
\begin{equation*}
	\begin{split}
		\| |n|^s \left( 1 + | \tau - n^m | \right ) ^{-1/2} \wh{w_{fg}}\left( 
		n, \tau \right) \|_{L^2(\zzdot \times \rr)}
		& \lesssim \|f\|_{\dot{X}^s} \|g\|_{\dot{X}^s}.
	\end{split}
\end{equation*}
%
%
%
\subsection{Case \eqref{mpigeon-case-2-gen}}
We have for nonzero $ n, n_1, n_2 $
%
%%
\begin{equation}
	\label{mconvo-deriv-bound-gen}
	\begin{split}
		& \frac{|n|^{s+1}}{|n_1|^s 
		| n_2|^s}
		\times
		\frac{1}{(1 + | \tau_1 -n_1^m |)^{1/2}}
		\\
		& \lesssim | n |^{s+1}| n_1 |^{-s}| n_2 |^{-s} \times | n
		|^{-\frac{c}{2}}| n_1 |^{-\frac{m-1-c}{4}}| n_2 |^{-\frac{m-1-c}{4}} 
		\\
		& = | n |^{\frac{2s+2 -c}{2}} | n_1 |^{\frac{-4s -m + 1+ c}{4}} | n_2
		|^\frac{-4s -m + 1+ c}{4}
		\\
		& \le 1, \qquad s \ge \frac{1-m}{4}.
	\end{split}  
\end{equation}
%
%
where the last line follows from \cref{mrem:gen-s-val}.
Hence, from \eqref{mconvo-est-starting-pnt} and \eqref{mconvo-deriv-bound-gen},
we obtain 
\begin{equation}
	\label{mconvo-est-starting-pnt-gen}
	\begin{split}
		 & |n|^s \left( 1 + | \tau - n^m | \right)^{-1/2} | \wh{w_{fg}}\left( 
		n, \tau \right) |
		\\
		& \le \left( 1 + | \tau - n^m | \right)^{-1/2}
		\sum_{\substack{n_1 \neq0, n_2 \neq 0 \\n_1 + n_2 =n}} \int_{\tau_1 + \tau_2 = \tau}\frac{|n|^{s+1}}{|n_1|^s | n_2|^s} 
		\times \frac{c_f(n_1, \tau_1)}{\left( 1 + | \tau_1 - n_1^m | 
		\right)^{1/2}}
		\\
		& \times
		\frac{c_g(n_2, \tau_2)}{\left( 1 + | \tau_2 - n_2^m | 
		\right)^{1/2}}\ d \tau_1 d \tau_2
		\\
		& \lesssim \left( 1 + | \tau - n^m | \right)^{-1/2}
		\sum_{\substack{n_1 \neq0, n_2 \neq 0 \\n_1 + n_2 =n}} \int_{\tau_1 + \tau_2
		= \tau} c_f(n_1, \tau_1) \times
		\frac{c_g(n_2, \tau_2)}{\left( 1 + | \tau_2 - n_2^m | 
		\right)^{1/2}}\ d \tau_1 d \tau_2
		\\
		& = \left( 1 + | \tau - n^m | \right)^{-1/2}
\wh{\overset{\sim}{C_f} C_g}.
	\end{split}
\end{equation}
%
%%
%
%
Hence
%
%%
\begin{equation}
	\label{m3f-gen}
	\begin{split}
		& \| |n|^s \left( 1 + | \tau - n^m | \right)^{-1/2} \wh{w_{fg}}(n, \tau) 
		\|_{L^2(\zzdot \times \rr)}
		\\
		& \lesssim \|\left( 1 + | \tau - n^m | \right)^{-1/2} 
		\wh{\overset{\sim}{C_f} C_g } \|_{L^2(\zzdot \times \rr)}
		\\
		& =  \|\left( 1 + | \tau - n^m | \right)^{-1/2} 
		\wh{\overset{\sim}{C_f} C_g } \|_{L^2(\zz \times \rr)}
		\\
		& \lesssim  \|\overset{\sim}{C_f} C_g  \|_{L^{4/3}(\ci \times \rr)}
	\end{split}
\end{equation}
%
%%
where the last step follows from \cref{mcor:four-mult-est-L4}.
%
%
Applying H\"{o}lder's inequality to the right hand side of
\eqref{m3f-gen}, we obtain the bound
%
%%
\begin{equation}
	\label{m4f-gen}
	\begin{split}
		\|\overset{\sim}{C_f} \|_{L^2(\ci \times \rr)} \|C_g \|_{L^4\left( \ci 
		\times \rr 
		\right)}. 
	\end{split}
\end{equation}
%
%%
By Plancherel we have
%
%%
%
%%
\begin{equation}
	\label{m5f-gen}
	\begin{split}
		\|\overset{\sim}{C_f} \|_{L^2(\ci \times \rr)}
		& \simeq \|c_f\|_{L^2(\zz \times \rr)}
		\\
		& = \|f \|_{\dot{X}^s}
	\end{split}
\end{equation}
%
%%
while \cref{mlem:four-mult-est-L4} gives
%
%
\begin{equation}
	\label{mfour-mult-conseq-gen}
	\begin{split}
		\|C_h\|_{L^4(\ci \times \rr)} 
		& \lesssim \|(1 + | \tau - n^m |)^{1/2} \wh{C_h}
		\|_{L^2(\zz \times \rr)}
		\\
		& = \|c_{h} \|_{L^2(\zz \times \rr)} 
		\\
		& = \|h \|_{\dot{X}^s}. 
	\end{split}
\end{equation}
%
%
We conclude from \eqref{m3f-gen}-\eqref{mfour-mult-conseq-gen} that
%
%
\begin{equation*}
	\begin{split}
		\| |n|^s \left( 1 + | \tau - n^m | \right)^{-1/2} \wh{w_{fg}}(n, \tau) 
		 \|_{L^2(\zzdot \times \rr)}
		 \lesssim \|f\|_{\dot{X}^s} \|g\|_{\dot{X}^s}.
	\end{split}
\end{equation*}
%
%
%
%
\section{Proof of Second Bilinear Estimate}
Recall that for the NLS, one obtains one trilinear estimate as a corollary of
another. Using this as motivation, let us see if we can obtain
\cref{mprop:bilinear-estimate2} as a corollary of
\cref{mprop:prim-bilin-est}. By
duality, it suffices to show that
%
%%
\begin{equation}
	\label{mduality-est}
	\begin{split}
	|	\sum_{n \in \zzdot}  |n|^{s}
		a_n \int_{\rr} \frac{|\wh{w_{fg}}(n, \tau)|}{1 
		+ | \tau - n^m |} \ d \tau | \lesssim \|f\|_{\dot{X}^s} \|g\|_{\dot{X}^s}
		\|a_n \|_{\ell^2}, \qquad s \ge -1/2.
	\end{split}
\end{equation}
%
%%
By the triangle inequality 
and Cauchy-Schwartz,
%
%%
\begin{equation}
	\label{m1m}
	\begin{split}
		& | \sum_{n \in \zzdot} |n|^{s} a_n
		\int_{\rr}\frac{| \wh{w_{fg}}(n, \tau) |}{(1 + | \tau - n^m |)} \ d \tau |
		\\
		& \le \sum_{n \in \zzdot} \int_{\rr} \frac{| a_n |}{\left( 1 + 
		| \tau - n^m |
		\right)^{1/2 + \eta}} \times \frac{| n|^s  |
		\wh{w_{fg}}(n, \tau) |}{\left( 
		1 + | \tau - n^m | \right)^{1/2 - \eta}} \ d \tau
		\\
		& \le \left( \sum_{n \in \zzdot} | a_{n} |^2\int_{\rr} \frac{1}{\left( 1 + |
		\tau - n^m | \right)^{1 + 2 \eta}} \ d \tau  
		\right)^{1/2} 
		\left ( \sum_{n \in \zzdot}\int_{\rr} \frac{|n|^{2s} | \wh{w_{fg}}(n, \tau) 
		|^2}{\left( 1 + | \tau - n^m | \right)^{1 -2 \eta}}\ d \tau 
		\right)^{1/2}.
	\end{split}
\end{equation}
%
%%
Applying the change of variable $\tau - n^m
\mapsto \tau'$ we obtain  
%%
%
\begin{equation*}
	\begin{split}
		& \left( \sum_{n \in \zzdot} | a_{n} |^2\int_{\rr} \frac{1}{\left( 1 + | \tau -
		n^m | \right)^{1 + 2 \eta}} \ d \tau  
		\right)^{1/2} 
		\\
		& = \left ( \sum_{n \in \zzdot}
		| a_n |^2 
		\int_{\rr} \frac{1}{\left( 1 + | \tau' | \right)^{1 + 2 \eta}} \ d 
		\tau \right)^{1/2}
		\\
		& \simeq \|a_n\|_{\ell^2}, \qquad \eta >0.
		\end{split}
\end{equation*}
However, if we assume $\eta >0$, then
we cannot use \cref{mprop:prim-bilin-est} to bound
\begin{equation*}
	\begin{split}
		\left ( \sum_{n \in \zzdot}\int_{\rr} \frac{|n|^{2s} | \wh{w_{fg}}(n, \tau) 
		|^2}{\left( 1 + | \tau - n^m | \right)^{1 - 2\eta}}\ d \tau
		\right)^{1/2}. 
	\end{split}
\end{equation*}
%%
%%
\begin{framed}
\begin{remark}
Hence, unlike the NLS, we have not been able to obtain a second bilinear
estimate as a corollary from the first. Heuristically, this is due to the
derivative in nonlinearity, which is not present in the NLS nonlinearity.
However, one can obtain \eqref{mbilinear-estimate2} for $s>\frac{1-m}{4}$ as a
corollary of \cref{mprop:prim-bilin-est} by using the ideas
above and by modifying the proof of \cref{mprop:prim-bilin-est} slightly (i.e.,
showing that if $b = \frac{1}{2}^-$, then \eqref{mprim-bilin-est} holds for
$s\ge \frac{1-m}{4}^+$). To show that \eqref{mbilinear-estimate2} holds for the
case $s=1/2$, we will have to resort to Kenig-Ponce-Vega~\cite{Kenig:1996yn} techniques.
\end{remark}
\end{framed}
%
%
Proceeding, note that by duality, to prove \cref{mprop:bilinear-estimate2} it
suffices to show \eqref{mduality-est} for $s \ge \frac{1-m}{4}$. By the symmetry of the convolution, we
consider only cases \eqref{mpigeon-case-1} and \eqref{mpigeon-case-2}.
%
%
\subsection{Case \eqref{mpigeon-case-1}.} Assume $s \ge \frac{1-m}{4}$. Then from 
\eqref{mconvo-est-starting-pnt-gen-case2} we have
%
%
\begin{equation}
	\label{mgen-smoothing-ineq}
	\begin{split}
		& |n|^s \left( 1 + | \tau - n^m | \right)^{-1/2} | 
		\wh{w_{fg}}(n, \tau) | 
		\\
		& \lesssim \sum_{\substack{n_1 \neq0, n_2 \neq 0 \\n_1 + n_2 =n}} \int_{\tau_1 + \tau_2 = \tau}\frac{c_f(n_1, \tau_1)}{\left( 1 + | 
		\tau_1 -  n_1^m| \right)^{1/2}}
		\times \frac{c_g\left( n_2, \tau_2\right)}{\left( 1 + | \tau_2 -n_2^m|
		\right)^{1/2}}.
	\end{split}
\end{equation}
%
%
From the triangle inequality and \eqref{mgen-smoothing-ineq}, we have
%
%
\begin{equation*}
	\begin{split}
	 |\eqref{mduality-est}|
	& \lesssim \sum_{n \in \zzdot} |a_{n}| \int_{\rr} \sum_{\substack{n_1 \neq 0, n_2 \neq 0
		\\ n_1 +n_2 =n}} \int_{\tau_1 + \tau_2 = \tau} c_f(n_1, \tau_1)
		c_g(n_2, \tau_2)
		\\
		& \times \frac{1}{(1 + | \tau - n^m |)^{1/2}(1 + |
		\tau_{1}-n_{1}^m |)^{1/2}(1 + | \tau-n_{2}^m |^{1/2})} d \tau_1 d \tau_2
		d \tau
	\end{split}
\end{equation*}
%
%
which by Cauchy-Schwartz is bounded by
%
%
\begin{equation}
	\label{m10g}
	\begin{split}
		& \sum_{n \in \zzdot} |a_n| \int_{\rr} \left(  \sum_{\substack{n_1 \neq 0, n_2
		\neq 0 \\n_1 +n_2 =n}} \int_{\tau_1 + \tau_2 = \tau} c_{f}^{2}(n_1, \tau_1)
		c_{g}^{2} (n_2, \tau_2) d \tau_1 d \tau_2 \right)^{1/2} 
		\\
		& \times \left( \sum_{\substack{n_1 \neq 0, n_2 \neq 0 \\n_1 +n_2 =n}}
		\int_{\tau_1 + \tau_2 = \tau} \frac{1}{(1 + | \tau - n^m |)(1 + | \tau_{1}-n_{1}^m |)(1 + |
		\tau_2 -n_{2}^m |)} d \tau_1 d \tau_2
		\right)^{1/2} d \tau.
	\end{split}
\end{equation}
%
%
Applying Cauchy-Schwartz again, \eqref{m10g} is bounded by
%
%
\begin{align}
	\notag
		& \|\left( \sum_{\substack{n_1 \neq 0, n_2 \neq 0 \\n_1 +n_2 =n}}\int_{\tau_1 + \tau_2 = \tau} c_{f}^{2}(n_1, \tau_1)
		c_{g}^{2} (n_2, \tau_2) d \tau_1 d \tau_2 \right)^{1/2} \|_{L^{2}(\zz \times
		\rr)}
		\\
		\notag
		& \times  \|a_{n}
		\left( \sum_{\substack{n_1 \neq 0, n_2 \neq 0 \\n_1 +n_2
		=n}}\int_{\tau_1 + \tau_2 = \tau} \frac{1}{ (1 + | \tau - n^m |)(1 + |
		\tau_{1}-n_{1}^m |)(1 + | \tau_2 -n_{2}^m |)} d \tau_1 d \tau_2
		\right)^{1/2} \|_{L^2(\zz \times \rr)}
		\\
		\notag
		& = \|f\|_{\dot{X}^s} \|g\|_{\dot{X}^s}
		\\
		\label{mholder-term}
		& \times 
		\|a_{n}
		\left( \sum_{\substack{n_1 \neq 0, n_2 \neq 0 \\n_1 +n_2
		=n}}\int_{\tau_1 + \tau_2 = \tau} \frac{1}{ (1 + | \tau - n^m |)(1 + |
		\tau_{1}-n_{1}^m |)(1 + | \tau_2 -n_{2}^m |)} d \tau_1 d \tau_2
		\right)^{1/2} \|_{L^2(\zz \times \rr)}.
\end{align}
%
Applying H{\"o}lder then gives
%
%
\begin{equation*}
	\begin{split}
		& \eqref{mholder-term}
		 \le \| a_{n} \|_{\ell^2}
		\\
		& \times \left( \sup_{n \neq 0} \int_{\rr}
		\sum_{\substack{n_1 \neq 0, n_2 \neq 0 \\n_1 +n_2 =n}} \int_{\tau_1 + \tau_2
		= \tau} \frac{1}{ (1 + | \tau - n^m |)(1 + |
		\tau_{1}-n_{1}^m |)(1 + | \tau_2 -n_{2}^m |)} d \tau_1 d \tau_2 d \tau
		\right)^{1/2}.
	\end{split}
\end{equation*}
%
%
Hence, to complete the proof for case \eqref{mpigeon-case-1}, it will be enough
to show that 
%
%
%
%
\begin{equation*}
	\begin{split}
		 \sup_{n \neq 0} \int_{\rr}
		\sum_{\substack{n_1 \neq 0, n_2 \neq 0 \\n_1 +n_2 =n}} \int_{\tau_1 + \tau_2
		= \tau} \frac{1}{ (1 + | \tau - n^m |)(1 + |
		\tau_{1}-n_{1}^m |)(1 + | \tau_2 -n_{2}^m |)} d \tau_1 d \tau_2 d \tau <\infty
	\end{split}
\end{equation*}
%
%
or, equivalently, that
%
%
\begin{equation}
	\label{m12g}
	\begin{split}
		\sup_{n \neq 0} \sum_{\substack{n_1 \neq 0, n_2 \neq 0 \\n_1 +n_2 =n}} \int_{\rr}
		\int_\rr  \frac{1}{(1 + | \tau - n^m |)(1 + | \tau_1 - n_{1}^m |)(1 + | \tau - \tau_1 -
		n_2^m |)} d \tau_1 d \tau < \infty.
	\end{split}
\end{equation}
%
%
Following Kenig~\cite{Kenig:1996yn}, we now need the following Calculus lemma.
%
%
%%%%%%%%%%%%%%%%%%%%%%%%%%%%%%%%%%%%%%%%%%%%%%%%%%%%%
%
%
%				 Calculus Lemma
%
%
%%%%%%%%%%%%%%%%%%%%%%%%%%%%%%%%%%%%%%%%%%%%%%%%%%%%%
%
%
\begin{lemma}
	\label{mlem:calc}
 %
 %
 \begin{equation}
	 \label{mcalc}
	 \begin{split}
		 \int_{\rr} \frac{1}{(1 + | \theta |)(1 + | a - \theta |)} d \theta \lesssim
		 \frac{\log(2 + | a |)}{1 + | a |}.
	 \end{split}
 \end{equation}
 %
 %
 \end{lemma}
%
%
Applying the lemma with $\theta = \tau_1 - n_1^m$ and $a = \tau - n_1^m -
n_2^m$, we see that
%
%
\begin{equation*}
	\begin{split}
	\int_{\rr}
		\int_\rr  \frac{1}{(1 + | \tau - n^m |)(1 + | \tau - \tau_1 -
		n_2^m |)} d \tau_1 d \tau \lesssim \frac{\log(2 + | \tau - n_{1}^m -
		n_{2}^m |)}{1 + | \tau - n_{1}^m - n_{2}^m |}.
	\end{split}
\end{equation*}
%
%
%
Hence, the left hand side of \eqref{m12g} is bounded by
%
\begin{equation*}
	\begin{split}
		\sup_{n \neq 0} \sum_{\substack{n_1 \neq 0, n_2 \neq 0 \\n_1 +n_2 =n}}
		\int_{\rr} \frac{\log(2 + | \tau - n_{1}^m -
		n_{2}^m |)}{(1 + | \tau - n_{1}^m - n_{2}^m |)(1 + | \tau - n^m |)}
		d \tau	
	\end{split}
\end{equation*}
%
%
or, equivalently, by
%
%
\begin{equation}
	\label{m13g}
	\begin{split}
		\sup_{n \neq 0} \sum_{n_1 \neq 0} \int_{\rr} \frac{\log(2 + | \tau -
		n_{1}^m - (n - n_1)^m |)}{(1 + | \tau - n_{1}^m - (n - n_{1})^m |)(1
		+ | \tau - n^m |)} d \tau.
	\end{split}
\end{equation}
%
%
%
Now, note that 
$$ |\tau - n^m| \ge \frac{d_m(n_1, n_2)}{3} \gtrsim
| n_1 n_2 |^{(m-1)/2},$$ where the right hand side follows from
\cref{mlem:number-theory} with $c=0$. Hence, \eqref{m13g} is bounded by a constant times
%
%
%
%
\begin{equation}
	\label{m14g}
	\begin{split}
		& \sup_{n \neq 0} \sum_{n_1 \neq 0}
		\frac{1}{| n_1 n_2 |^{(\frac{1}{2} + \eta)(m-1)/2}} \int_{\rr} \frac{\log(2 + | \tau - n_{1}^m -
		(n - n_1)^m |)}{(1 + | \tau - n_{1}^m - (n - n_{1})^m
		|)(1 + | \tau - n^m |)^{\frac{1}{2}-\eta}}
		d \tau
		\\
		& \le \sup_{n \neq 0} \sum_{n_1 \neq 0}
		\frac{1}{| n_1 n_2 |^{(\frac{1}{2} + \eta)(m-1)/2}} 	\\
		& \times \sup_{n \neq 0} \sum_{n_1 \neq 0}
		\int_{\rr} \frac{\log(2 + | \tau
		- n_{1}^m - (n - n_1)^m |)}{(1 + | \tau - n_{1}^m - (n - n_{1})^m
		|)(1 + | \tau - n^m |)^{\frac{1}{2}-\eta}}
		d \tau
	\end{split}
\end{equation}
%
for any $\eta >0$.
Observe that for the first sum, the supremum is attained at $n=1$.
%
%
\begin{framed}
\begin{remark}
To see this,
write $n_1 n_2 = n_1(n-n_1) \doteq f(n)$ and note that $|f(n)|$ has a global
minimum at $n=n_1$. Furthermore, $f(n)$ is strictly
increasing (if $n_1>0$) or strictly decreasing (if $n_1 <0$).
\end{remark}
\end{framed}
%
%
%
But then $n_2 = 1 - n_1$, and so $| n_1 n_2 | \gtrsim | n_1 |^2$. Furthermore, we know that 
for any $\ee > 0$, we have $\log (2 + | a |) \le c_{\ee}(1 + | a
|)^{\ee}$. Hence, we bound \eqref{m14g} by
%
%
%
%
\begin{equation*}
	\begin{split}
		c_{\ee}  \sum_{n_1 \neq 0} \frac{1}{|n_1|^{(\frac{1}{2} + \eta)(m-1)}}
		\sup_{n \neq 0} \sum_{n_1 \neq 0} \int_{\rr} \frac{1}{(1 +
		| \tau - n_{1}^m - (n - n_{1})^m |)^{1- \ee}(1 + | \tau - n^m
		|)^{\frac{1}{2}-\eta}} d \tau
	\end{split}
\end{equation*}
%
%
%
which due to the estimate
%
%
\begin{equation}
	\label{m16g}
	\begin{split}
		(1 + | \tau - n^m |)
		& = 1 + \frac{1}{4}| \tau - n^m | + \frac{3}{4}| \tau - n^m |
		\\
		& \ge 1 + \frac{1}{4}| \tau - n^m | + \frac{3}{4} \times
		\frac{1}{3}d(n_1,n_2)
		\\
		& = 1 + \frac{1}{4}| \tau - n^m | + \frac{1}{4}| n^m - n_1^m - (n -
		n_1)^m |
		\\
		& \ge \frac{1}{4}| \tau - n_1^m - (n - n_1)^m |
	\end{split}
\end{equation}
%
%
is bounded by
%
%
\begin{equation}
	\label{m15g}
	\begin{split}
		& 4 c_{\ee} \sum_{n_1 \neq 0} \frac{1}{| n_1 |^{(\frac{1}{2} +
		\eta)(m-1)}} 	\sup_{n \neq 0} \sum_{n_1 \neq 0}	\int_{\rr} \frac{1}{(1 + |
		\tau - n_{1}^m - (n - n_{1})^m |)^{\frac{3}{2}-\ee - \eta}} d \tau
		\\
		& \lesssim \sum_{n_! \neq 0} \frac{1}{| n_1 |^{(\frac{1}{2} +
		\eta)(m-1)}} 		\qquad (\text{for} \ \eta \ \text{sufficiently small})
		\\
		& < \infty, \qquad (\text{since} \ m \ge 3). \qquad \qed
	\end{split}
\end{equation}
%
%
%
%
%
%
\subsection{Case \eqref{mpigeon-case-2}.} From the triangle inequality and
\eqref{mgen-smoothing-ineq}, we see that for $s \ge \frac{1-m}{4}$ we have
%
%
%
%
%
%
\begin{equation}
	\begin{split}
		& | \sum_{n \neq 0} \int_{\rr} a_n |n|^s \left( 1 + | \tau - n^m | \right)^{-1} | 
		\wh{w_{fg}}(n, \tau) | d \tau |
		\\
		& \lesssim \sum_{n \neq 0}  \int_{\rr} |a_{n}| (1+ | \tau - n^m |)^{-1} \wh{\overset{\sim}{C_f} C_g} d
		\tau
	\\	
	& = \sum_{n \neq 0} \int_{\rr} |a_{n}| (1+ | \tau - n^m |)^{-5/8} (1 + | \tau - n^m
	|)^{-3/8} \wh{\overset{\sim}{C_f} C_g} d
		\tau
		\\
		& \le \|a_{n} (1 + | \tau - n^m |)^{-5/8}\|_{L^2(\zz \times \rr)}  \| (1 +
		| \tau - n^m |)^{-3/8} \wh{\overset{\sim}{C_f} C_g}  \|_{L^2(\zz \times
		\rr)}
		\end{split}
\end{equation}
%
%
where the last step follows from Cauchy-Schwartz. Applying the change of
variable $\tau - n^{m } \mapsto \tau'$ we obtain  %
%%
\begin{equation*}
	\begin{split}
		\|a_{n} (1 + | \tau - n^m |)^{-5/8}\|_{L^2(\zz \times \rr)} 
		& = \left( \sum_{n \in \zz} | a_{n} |^2\int_{\rr} \frac{1}{\left( 1 + | \tau -
		n^{m } | \right)^{5/4}} \ d \tau  
		\right)^{1/2} 
		\\
		& = \left ( \sum_{n \in \zz}
		| a_n |^2 
		\int_{\rr} \frac{1}{\left( 1 + | \tau' | \right)^{5/4}} \ d 
		\tau' \right)^{1/2}
		\\
		& \simeq \|a_n\|_{\ell^2}
		\end{split}
\end{equation*}
%
%
%
while \eqref{m3f-gen}-\eqref{mfour-mult-conseq-gen} yields the bound
%
%
\begin{equation*}
	\begin{split}
	\| (1 + | \tau - n^m |)^{-3/8} \wh{\overset{\sim}{C_f} C_g}  \|_{L^2(\zz
	\times \rr)} \lesssim \|f\|_{\dot{X}^s} \|g\|_{\dot{X}^s}
	\end{split}
\end{equation*}
%
%
completing the proof. \qquad \qedsymbol
%
%
%
\section{Proofs of Lemmas and Estimates}
\begin{proof}[Proof of \cref{mlem:cutoff-loc-soln}]
%
%
\begin{equation*}
	\begin{split}
		\lim_{t_{n} \to t} \|u(\cdot, t) - u(\cdot, t_{n})\|_{\dot{H}^s(\ci)} 
		& = \lim_{t_{n} \to t} \|\psi(t) u(\cdot, t) - \psi(t_n) u(\cdot,
		t_{n})\|_{\dot{H}^s(\ci)} 
		\\
		& = \lim_{t_n \to t} \left[ \sum_{n \in \zzdot}| n |
		^{2s} | \psi(t)  \wh{u}(n, t) - \psi(t_n) \wh{ u}(n, t_n) |^2 \right]^{1/2}
		\\
		& = \lim_{t_n \to t} \left[ \sum_{n \in \zzdot} | n |^{2s} | \int_{\rr} (e^{it \tau} - e^{it_{n} \tau}) \wh{\psi u}(n,
		\tau) d \tau |^2 \right]^{1/2}.
	\end{split}
\end{equation*}
		It is clear that
		%
		%
		\begin{equation*}
			\begin{split}
				| n |
				^{2s} | \int_{\rr} (e^{it \tau} - e^{it_{n}\tau}) \wh{\psi u}(n, \tau) d \tau |^2 
		& \le 4  | n |^{2s} \left ( \int_{\rr} |\wh{\psi u}(n, \tau)| d \tau
		\right )^2 
	\end{split}
\end{equation*}
and 
%
%
\begin{equation*}
	\begin{split}
 \sum_{n \in \zzdot} | n |^{2s} \left ( \int_{\rr} |\wh{\psi u}(n, \tau)| d \tau
		\right ) ^2 
		& = \| |n |^s \wh{\psi u}\|_{\dot{\ell}_n^2 L_\tau^1}
		\\
		& \le \|\psi u \|_{Y^s}^2 
	\end{split}
\end{equation*}
which is bounded by assumption.
Applying dominated convergence completes the proof. 
\end{proof}
%
%
\begin{proof}[Proof of \cref{mlem:schwartz-mult}]
Note that
%
%
\begin{equation*}
	\begin{split}
		\wh{\psi f}\left( n, \tau \right)
		& = \wh{\psi}(\cdot) * \wh{f}(n,
		\cdot)(\tau)
		= \int_\rr \wh{\psi}(\tau_1) \wh{f} \left( n, \tau - \tau_1 \right) 
		d\tau_1
	\end{split}
\end{equation*}
%
%
and hence
%
%
\begin{equation}
	\label{m1b}
	\begin{split}
		\|\psi f\|_{\dot{X}^s} 
		& = \left( \sum_{n \in \zzdot} |n|^{2s} \int_\rr \left( 1 + | \tau -
		n^{m} | \right) | \int_\rr \wh{\psi}(\tau_1) \wh{f}\left( n, \tau -
		\tau_1
		\right)  d \tau_1 d \tau |^2 \right)^{1/2}
		\\
		& \le \left( \sum_{n \in \zzdot} |n|^{2s} \int_\rr \left( 1 + | \tau -
		n^{m }
		|
		\right) \left( \int_\rr \wh{\psi}\left( \tau_1 \right) \wh{f}\left( n,
		\tau - \tau_1
		\right)  d \tau_1 d \tau \right)^2 \right)^{1/2}.
	\end{split}
\end{equation}
%
%
Using the relation
%
%
\begin{equation*}
	\begin{split}
		1 + | \tau - n^{m } |
		& = 1 + | \tau + \tau_1 - n^{m} |
		\\
		& \le 1 + | \tau_1 | + | \tau - \tau_1 - n^{m} |
		\\
		& \le \left( 1 + | \tau_1 | \right)\left( 1 + | \tau - \tau_1 -
		n^{m} | \right)
	\end{split}
\end{equation*}
%
%
we obtain
%
%
\begin{equation*}
	\begin{split}
		\eqref{m1b}
		& \le \left( \sum_{n \in \zzdot} |n|^{2s} \right.
		\\
		& \times \left . \int_\rr \left(
		\int_\rr \left( 1 + | \tau_1 | \right)^{1/2} | \wh{\psi}(\tau_1) |
		\left( 1 + | \tau - \tau_1 - n^{m} | \right)^{1/2} \wh{f}\left( n, \tau
		- \tau_1
		\right)d \tau_1
		\right)^2 d \tau \right)^{1/2}
	\end{split}
\end{equation*}
%
%
which by Minkowski's inequality is bounded by
%
%
\begin{equation}
	\label{m2b}
	\begin{split}
		& \left( \sum_{n \in \zzdot} |n|^{2s}  \right.
		\\
		& \times \left. \left( \int_\rr \left[ \int_\rr
		\left( 1 + | \tau_{1} | \right) | \wh{\psi}(\tau_1) |^2 \left( 1 + |
		\tau - \tau_1 - n^{m} |
		\right) | \wh{f}\left( n, \tau - \tau_1 \right) |^2 d \tau_1 
		\right]^{1/2} d \tau \right)^2 \right)^{1/2}.
	\end{split}
\end{equation}
%
%
Using the change of variable $\tau - \tau_1 \to \lambda$ gives
%
%
\begin{equation*}
	\begin{split}
		\eqref{m2b}
		& = \left( \sum_{n \in \zzdot} |n|^{2s}\right.
		\\
		& \times \left.  \left( \int_\rr \left[
		\int_\rr \left( 1 + | \tau_1 | \right) | \wh{\psi}\left( \tau_1
		\right) |^2 \left( 1 + | \lambda - n^{m} | \right) | \wh{f} \left( n,
		\lambda
		\right)|^2 d \tau_1 \right]^{1/2} d \lambda \right)^2 \right)^{1/2}
		\\
		& =  \left( \sum_{n \in \zzdot} |n|^{2s} \right.
		\\
		& \times \left. \left( \int_\rr \left( 1 + |
		\tau_1 |
		\right)^{1/2} | \wh{\psi}(\tau_1) | d \tau_1 \left[ \int_\rr \left( 1 + |
		\lambda - n^{m} |
		\right) | \wh{f}\left( n, \lambda \right) |^2 d \lambda \right]^{1/2}
		\right)^2 \right)^{1/2}
		\\
		& = c_{\psi} \left( \sum_{n \in \zzdot} |n|^{2s} \left( \left[ \int_\rr
		\left( 1 + | \lambda - n^{m} | \right) | \wh{f}\left( n, \lambda
		\right) |^2 d \lambda
		\right]^{\cancel{1/2}} \right)^{\cancel{2}} \right)^{1/2}
		\\
		& = c_{\psi} \|f\|_{\dot{X}^s},
	\end{split}
\end{equation*}
%
%
concluding the proof. 
\end{proof}
%
%
%
%
%
%
\begin{proof}[Proof of \cref{mlem:number-theory1}] First note that
%
\begin{equation*}
		| - n^m + n_1^m + n_2^m|
		 = 3 | n | |n_1 | |n_2 |.
\end{equation*}
%
%
Hence, it will be enough to show that for $c \ge 0$
%
%
\begin{equation*}
	\begin{split}
		| n | |n_1 | |n_2 | \gtrsim | n |^{\frac{2 + c}{2}}| n_1
		|^{\frac{2-c}{2}}| n_2 |^{\frac{2-c}{2}}
	\end{split}
\end{equation*}
%
%
or, dividing through on both sides by $|n| | n_1 | | n_2 |$ and rearranging terms
%
%
\begin{equation*}
	\begin{split}
		| n |^{c/2} \lesssim | n_1 |^{c/2} | n_2 |^{c/2}.
	\end{split}
\end{equation*}
%
%
But
%
%
\begin{equation*}
	\begin{split}
		| n |^{c/2} &= | n_1 + n_2 |^{c/2}
		\\
		& \le (| n_1 | + |n_2|)^{c/2} 
		\\
		& \le (2\max\{|
		n_1 |, | n_2 |)^{c/2}
		\\
		& \le (2|
		n_1 | | n_2 |)^{c/2}
		\\
		& = 2^{c/2} | n_1 |^{c/2} | n_2 |^{c/2}
	\end{split}
\end{equation*}
%
which completes the proof. 
\end{proof}
%
%
%
\begin{proof}[Proof of \cref{mlem:number-theory}] Define
%
\begin{equation*}
	\begin{split}
		| - n^{m} + n_1^{m} + n_2^{m }|
		& = | n_{1}^{m} - n^{m} + (n-n_{1})^{m}| 
		\\
		& \doteq f(n).
	\end{split}
\end{equation*}
%
%
For fixed $n_1$, the absolute minima
of $f(n)$ occurs at $n = 1+n_{1}$ ($n = n_1$ is not available by assumption). Next, note that
%
%
\begin{equation*}
	\begin{split}
		f(1+ n_{1}) = | n_{1}^{m} - (1 + n_{1})^m + 1 |
		& = | (1 + n_{1} )^{m} - n_{1}^{m} -1 |.
	\end{split}
\end{equation*}
We now seek a lower bound for the right hand side. By symmetry we may assume
$n_1 >0$ without loss of generality.
%
%
\begin{framed}
\begin{remark}
	By the term ``symmetry'', we mean that
	\begin{equation*}
	\begin{split}
	| [1 + (-n_1)]^m - (-n_1)^m -1 |
	& = | (1 - n_1)^m + n_1^m -1 |
	\\
	& = | (1 + p_1)^m + (-p_1)^m -1 |, \qquad p_1 = -n_1
	\\
	& = | (1 + p_1)^m - (p_1)^m -1 |.
	\end{split}
\end{equation*}
%
%
\end{remark}
\end{framed}
%
%
Then 
%
%
\begin{equation*}
	\begin{split}
	| (1 + n_{1} )^{m} - n_{1}^{m} -1 |
	& = | \sum_{1 \le k \le m-1} c_{k} n_1^{k}|, \qquad \{c_k\} \in
	\mathbb{N}\setminus 0
	 \\
	 & = \sum_{1 \le k \le m-1} c_{k} n_1^{k}
	 \\
	 & \ge c_{m-1}  n_1^{m-1}
	 \\
	 & = c_{m-1}  n_1^{c} n_1^{m-1-c}
	 \\
	 & \gtrsim (1 + n_1)^{c}  n_1^{m-1-c}
	 \\
	 & = n^{c} n_1^{m-1-c}. 
 \end{split}
\end{equation*}
%
%
Since we assumed $n_1 >0$ without loss of generality, it follows that 
%
%
\begin{equation*}
	\begin{split}
		f(n) \gtrsim |n|^{c} | n_1 |^{m-1-c}. 
	\end{split}
\end{equation*}
%
%
But since $f(n)$ is symmetric in $n_1$ and $n_2$, a similar argument shows that
%
%
\begin{equation*}
	\begin{split}
		f(n) \gtrsim |n|^{c} | n_2 |^{m-1-c}. 
	\end{split}
\end{equation*}
%
%
Therefore,
%
%
\begin{equation*}
	\begin{split}
		f(n) \gtrsim | n |^{c}| n_1 |^{\frac{m-1-c}{2}} | n_2 |^{\frac{m-1-c}{2}}
	\end{split}
\end{equation*}
%
%
completing the proof. 
%
\end{proof}
%
%
\begin{proof}[Proof of \cref{mlem:calc}]
%
%
%
By the change of variable $\theta \mapsto a/2 + x$, we have
%
%
\begin{equation*}
	\begin{split}
		\int_{\rr} \frac{1}{(1 + | \theta |)(1 + | a - \theta |)}d \theta
	= \int_{\rr} \frac{1}{(1 + |  a/2 + x |)(1 + | a/2 - x |)}d x.
	\end{split}
\end{equation*}
%
%
Hence, it suffices to show that
%
%
\begin{equation*}
	\begin{split}
		\int_{\rr} \frac{1}{(1 + | a - \theta |)(1 + | a + \theta |)}d \theta
		\lesssim \frac{\log(2 + | a |)}{1 + | a |}.
	\end{split}
\end{equation*}
%
%
Let us leave the case $a = 0$ for last. By symmetry, the cases $a<0$ and $a >0$
are equivalent. Hence, to cover the case $a \neq0$, we may assume
without loss of generality that $a >0$.
%
%
Then
\begin{equation}
	\label{ma1}
	\begin{split}
		& \int_{\rr} \frac{1}{(1 + | a - \theta |)(1 + | a + \theta |)}d \theta
		\\
		& = \int_{| \theta| \le a+1 } \frac{1}{(1 + | a - \theta |)(1 + | a + \theta
		|)}d \theta + \int_{| \theta| \ge a+1 } \frac{1}{(1 + | a - \theta |)(1 + |
		a + \theta |)}d \theta.
	\end{split}
\end{equation}
Estimating the second integral of \eqref{ma1}, we have
\begin{equation*}
	\begin{split}
		& \int_{| \theta| \ge a+1 } \frac{1}{(1 + | a - \theta |)(1 + | a + \theta
		|)}d \theta 
		\\
		& = \int_{\theta \ge a + 1} \frac{1}{(1 + \theta-a)(1 + \theta+a)} d \theta
		+ \int_{\theta \le -a -1} \frac{1}{(1 + \theta - a) (1 + \theta + a)}d \theta
		\\
		& = \frac{1}{2a} \int_{\theta \ge a + 1} \left[ \frac{1}{1 + \theta -a} -
		\frac{1}{1 + \theta+a} \right] d \theta
		+ \frac{1}{2a} \int_{\theta \le -a-1} \left[ \frac{1}{1 + \theta+a}
		-\frac{1}{1 + \theta -a} \right] d \theta
		\\
		& = \frac{1}{a} \log(1+a)
		\\
		& \lesssim \frac{\log(2 + |a|)}{1 + | a |}.
	\end{split}
\end{equation*}
To evaluate the first integral of \eqref{ma1}, we split into the cases $a \le \theta \le
a+1$, $-a \le \theta \le 0$, $0 \le \theta \le a$, and $a \le \theta \le a+1$.
However, note that 
%
%
\begin{equation*}
	\begin{split}
		& \int_{a}^{a+1} \frac{1}{(1 + | a - \theta |)(1 + | a + \theta |)}d \theta =
		\int_{-a-1}^{-a} \frac{1}{(1 + | a - \theta |)(1 + | a + \theta |)}d \theta,
		\\
		& \int_{0}^{a} \frac{1}{(1 + | a - \theta |)(1 + | a + \theta |)}d \theta =
		\int_{-a}^{0} \frac{1}{(1 + | a - \theta |)(1 + | a + \theta |)}d \theta.
	\end{split}
\end{equation*}
%
%
Therefore, we need only consider the cases $a \le \theta \le a+1$ and $0 \le
\theta \le a$.  For the case $a \le \theta \le a+1$ we have
%
%
\begin{equation*}
	\begin{split}
		\int_{a}^{a+1} \frac{1}{(1 + | a-\theta |)(1 + | a + \theta |)}d \theta
		& = \int_{a}^{a+1} \frac{1}{(1 + \theta -a)(1 + a + \theta)}d \theta
		\\
		& = \frac{1}{2a} \int_{a}^{a+1} \left[ \frac{1}{1 + \theta -a} -
		\frac{1}{1 + \theta + a}  \right]d \theta
		\\
		& =\frac{1}{2a} \log\left( \frac{1 + \theta -a}{1 + \theta + a} \right) \Big
		|_a^{a+1}
		\\
		& = \frac{1}{2a} \log\left( \frac{2a+1}{a+1} \right)
		\\
		& \lesssim\frac{\log 2}{2a}
		\\
		& \lesssim \frac{\log(2 + | a |)}{1 + | a |}.
	\end{split}
\end{equation*}
%
%
while for the case $0 \le \theta \le a$
%
%
\begin{equation*}
	\begin{split}
		\int_{0}^{a} \frac{1}{(1 + | a - \theta |)(1 + | a + \theta |)}d \theta
		& = \int_{0}^{a} \frac{1}{(1 +  a - \theta )(1 +  a + \theta )}d \theta
		\\
		& = \frac{1}{2(1 + a)} \int_{0}^{a} \left[ \frac{1}{1 + a - \theta} +
		\frac{1}{1 + a + \theta} \right]d \theta
		\\
		& = \frac{1}{2(1 + a)} \log \left( \frac{1 + a + \theta}{1 + a - \theta}
		\right) \Big |_{0}^{a}
		\\
		& = \frac{\log\left( 1 + 2a \right)}{2\left( 1 + a \right)}
		\\
		& \lesssim \frac{\log(2 + | a |)}{1 + | a |}.
	\end{split}
\end{equation*}
%
%
This completes the proof for the case $a \neq 0$. Lastly, for the case
$a =0$, we use dominated convergence and our preceding work to
conclude that
%
%
\begin{equation*}
	\begin{split}
		\int_{\rr} \frac{1}{(1 + | \theta|)^2} d \theta
		& = \lim_{a \to 0}
		\int_{\rr} \frac{1}{(1 + | a - \theta |)(1 + | a + \theta |)}d \theta
		\\
		& \lesssim \lim_{a \to 0} \frac{\log(2 + | a |)}{1 + | a |}
		\\
		& =  \log 2
		\\
		& = \frac{\log(2 + | 0 |)}{1 + | 0 |}. 
	\end{split}
\end{equation*}
%
This completes the proof. 
\end{proof}
%
