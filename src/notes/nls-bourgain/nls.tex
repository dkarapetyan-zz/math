%
\documentclass[12pt,reqno]{amsart}
\usepackage{amssymb}
\usepackage{appendix}
\usepackage[showonlyrefs=true]{mathtools} %amsmath extension package
\usepackage{cancel}  %for cancelling terms explicity on pdf
\usepackage{yhmath}   %makes fourier transform look nicer, among other things
\usepackage{framed}  %for framing remarks, theorems, etc.
\usepackage{enumerate} %to change enumerate symbols
\usepackage[margin=2.5cm]{geometry}  %page layout
\setcounter{tocdepth}{1} %must come before secnumdepth--else, pain
\setcounter{secnumdepth}{1} %number only sections, not subsections
%\usepackage[pdftex]{graphicx} %for importing pictures into latex--pdf compilation
\numberwithin{equation}{section}  %eliminate need for keeping track of counters
%\numberwithin{figure}{section}
\setlength{\parindent}{0in} %no indentation of paragraphs after section title
\renewcommand{\baselinestretch}{1.1} %increases vert spacing of text
%
\usepackage{hyperref}
\hypersetup{colorlinks=true,
linkcolor=blue,
citecolor=blue,
urlcolor=blue,
}
\usepackage[alphabetic, initials, msc-links]{amsrefs} %for the bibliography; uses cite pkg. Must be loaded after hyperref, otherwise doesn't work properly (conflicts with cref in particular)
\usepackage{cleveref} %must be last loaded package to work properly
%
%
\newcommand{\ds}{\displaystyle}
\newcommand{\ts}{\textstyle}
\newcommand{\nin}{\noindent}
\newcommand{\rr}{\mathbb{R}}
\newcommand{\nn}{\mathbb{N}}
\newcommand{\zz}{\mathbb{Z}}
\newcommand{\cc}{\mathbb{C}}
\newcommand{\ci}{\mathbb{T}}
\newcommand{\zzdot}{\dot{\zz}}
\newcommand{\wh}{\widehat}
\newcommand{\p}{\partial}
\newcommand{\ee}{\varepsilon}
\newcommand{\vp}{\varphi}
\newcommand{\wt}{\widetilde}
%
%
%
%
\newtheorem{theorem}{Theorem}[section]
\newtheorem{lemma}[theorem]{Lemma}
\newtheorem{corollary}[theorem]{Corollary}
\newtheorem{claim}[theorem]{Claim}
\newtheorem{prop}[theorem]{Proposition}
\newtheorem{proposition}[theorem]{Proposition}
\newtheorem{no}[theorem]{Notation}
\newtheorem{definition}[theorem]{Definition}
\newtheorem{remark}[theorem]{Remark}
\newtheorem{examp}{Example}[section]
\newtheorem {exercise}[theorem] {Exercise}
%
\makeatletter \renewenvironment{proof}[1][\proofname] {\par\pushQED{\qed}\normalfont\topsep6\p@\@plus6\p@\relax\trivlist\item[\hskip\labelsep\bfseries#1\@addpunct{.}]\ignorespaces}{\popQED\endtrivlist\@endpefalse} \makeatother%
%makes proof environment bold instead of italic
\renewcommand{\cref}{\Cref}
\newcommand{\uol}{u^\omega_\lambda}
\newcommand{\lbar}{\bar{l}}
\renewcommand{\l}{\lambda}
\newcommand{\R}{\mathbb R}
\newcommand{\RR}{\mathcal R}
\newcommand{\al}{\alpha}
\newcommand{\ve}{q}
\newcommand{\tg}{{tan}}
\newcommand{\m}{q}
\newcommand{\N}{N}
\newcommand{\ta}{{\tilde{a}}}
\newcommand{\tb}{{\tilde{b}}}
\newcommand{\tc}{{\tilde{c}}}
\newcommand{\tS}{{\tilde S}}
\newcommand{\tP}{{\tilde P}}
\newcommand{\tu}{{\tilde{u}}}
\newcommand{\tw}{{\tilde{w}}}
\newcommand{\tA}{{\tilde{A}}}
\newcommand{\tX}{{\tilde{X}}}
\newcommand{\tphi}{{\tilde{\phi}}}
\synctex=1
\begin{document}
\title{Well-Posedness for the NLS}
\author{Alex Himonas, David Karapetyan, and Gerson Petronilho}
\address{Department of Mathematics  \\
University  of Notre Dame\\
Notre Dame, IN 46556 }
\address{Department of Mathematics \\
University  of Notre Dame\\
Notre Dame, IN 46556 }
\address{Departamento de Matemática \\
Universidade Federal de São
Carlos \\
Rodovia Washington Luiz, Km 235, São Carlos, SP,
13565-905, Brasil}
        \date{03/22/2011}
        %
        \maketitle
        %
        %
        %
        %
        %
        %
\section{Introduction}
				  We consider the cubic nonlinear Schr\"{o}dinger (NLS) 
				  initial value problem (ivp)
%
%
\begin{gather}
	\label{nNLS-eq}
	i \p_t u + \p_x^{2} u \pm |u|^2 u =0,
		\\
		\label{nNLS-init-data}
		u(x,0) = \vp(x) \in H^s(\ci), \ \ t \in \rr, \ \ x \in \ci.
\end{gather}
%
%
For the local well-posedness of the cubic NLS,
the sign in front of the nonlinearity plays no significant role.
Hence, without loss of generality we assume a positive sign.
Following the method of Bourgain~\cite{Bourgain-Fourier-transfo-1},
\cite{Bourgain-Fourier-transfo}, we first derive a
weak formulation of the NLS ivp. Let
$\ci = [0, 2 \pi]$, and use
the following notation for the Fourier transform
%
%
%
%
\begin{equation}
	\label{nfour-trans-pde}
	\begin{split}
    \widehat{f}(n) = \int_{\ci} e^{-ix n} f(x) \, dx.
	\end{split}
\end{equation}
Applying the spatial Fourier transform to the NLS ivp we obtain
%
%
\begin{gather}
  \label{nfour}
  \p_t \widehat{u}(n, t) = -i n^2 \widehat{u}(n, t) + i  
  \widehat{w} (n, t)
	\\
  \label{nfour-init-data}
	\widehat{u} (n,0) = \widehat{\vp}(n).
\end{gather}
Multiplying \eqref{nfour} by the integrating factor $e^{itn^2}$ then yields
%%
%%
\begin{equation*}
	\begin{split}
		\left[ e^{ it n^2} \widehat{u}(n) \right]_t = i
		 e^{ it n^2} \widehat{w} (n, t).	
	\end{split}
\end{equation*}
%
%
Integrating from $0$ to $t$, we obtain
%
%
\begin{equation*}
	\begin{split}
		\wh{u}(n, t) = \wh{\vp}(n) e^{- it n^2} + i  
		\int_0^t e^{ i(\tau - t) n^2} \wh{w}(n, \tau) \ 
		d\tau.
	\end{split}
\end{equation*}
%
%
Therefore, by Fourier inversion 
%
%
\begin{equation}
	\label{nNLS-integral-form}
	\begin{split}
		u(x,t) & = \frac{1}{2\pi} \sum_{n \in \zz} \wh{\vp}(n) e^{i\left( xn - t n^2 
		\right)} 
		\\
    & + \frac{i}{2 \pi} \sum_{n \in \zz} \int_0^t e^{i\left[ xn + \left( \tau - t 
		\right) n^2 \right]} \wh{w}(n, \tau) \ d\tau.
	\end{split}
\end{equation}
%
%
Note that \eqref{nNLS-integral-form} is a weaker 
restatement of the Cauchy-problem \eqref{nNLS-eq}-\eqref{nNLS-init-data}, 
since by construction any classical solution of the NLS 
ivp is a solution to \eqref{nNLS-integral-form}. 
%
%
We now derive an integral 
equation global in $t$ and equivalent to \eqref{nNLS-integral-form} for $t 
\in [-1, 1]$. Let $\psi(t)$ be a cutoff function symmetric about the 
origin such that $\psi(t) = 1$ for $|t| \le 1$ and $\text{supp} \, \psi 
= [-2, 2 ]$. Multiplying both sides of expression
$\eqref{nNLS-integral-form}$ by $\psi(t)$, we obtain
%
%
\begin{equation}
	\begin{split}
		\label{ncutoff-int-eq}
    \psi u(x, t)
		& = \frac{1}{2 \pi} \psi(t) \sum_{n \in \zz} e^{i(xn - t n^{2})} \widehat{\vp}(n) 
		\\
		& + \frac{i }{2 \pi} \psi(t) \int_0^t \sum_{n \in \zz} 
		e^{i\left[ xn + (t - \tau)n^2 \right]} \wh{w}(n, \tau) \ d\tau.
	\end{split}
\end{equation}
%
%
Noting that $e^{i\left( xn + tn^{2} \right)}$ 
does not depend on $\tau$, we may rewrite the second term
%
%
\begin{equation}
	\label{npre-prim-int-form}
	\begin{split}
		& \frac{i }{2 \pi} \psi(t) \int_0^t \sum_{n \in \zz} 
		e^{i\left[ xn + (t - \tau) n^2 \right]} \wh{w}(n, \tau) \ d\tau
		\\
		& = \frac{i}{2 \pi} \psi(t) \sum_{n \in \zz} e^{i\left( xn + t 
		 n^{2} 
		\right)} \int_0^t e^{- i\tau n^{2}} \wh{w}(n, \tau) \ d\tau.
	\end{split}
\end{equation}
%%
%%
We remark that this is a \emph{global} relation in $t$. Therefore, by Fourier 
inversion
%
%
%
%
%
%
%
\begin{equation*}
	\begin{split}
		\text{rhs of} \; \eqref{npre-prim-int-form}
		& = \frac{i}{4 \pi^2} \psi (t) \sum_{n \in \zz} e^{i\left( xn + t 
		 n^2
		\right)} \int_0^t \int_\rr e^{i\tau\left( \lambda - n^2 \right) }
		\wh{w}(n, \lambda) d \lambda d\tau
		\\
		& = \frac{1}{4 \pi^2} \psi(t) \sum_{n \in \zz} \int_\rr 
		e^{i\left( xn + tn^2 \right)} \frac{e^{it\left( \lambda - n^2 
		\right)}-1}{\lambda - n^2} \wh{w}(n, \lambda) d \lambda
	\end{split}
\end{equation*}
%
%
where the last step follows from Fubini and integration. Substituting
into \eqref{ncutoff-int-eq} we obtain
%
%
\begin{equation}
	\begin{split}
		\label{ncutoff-int-eq-2}
    \psi u(x, t)
		& = \frac{1}{2 \pi} \psi(t) \sum_{n \in \zz} e^{i(xn - tn^{2})} \widehat{\vp}(n) 
		\\
		& + \frac{1}{4 \pi^2} \psi(t) \sum_{n \in \zz} \int_\rr
		e^{i(xn + t n^2)} \frac{e^{it(\lambda - n^2)}- 1}{\lambda - n^2} 
		\wh{w}(n, \lambda) \ d \lambda.
	\end{split}
\end{equation}
%
%
%
Next, we localize near the singular curve $\lambda =  n^2$.  Multiplying the
summand of the second term of \eqref{ncutoff-int-eq-2} by $1 + \psi(\lambda -
n^2) - \psi(\lambda -
n^2) $ and
rearranging terms, we have
%
%
\begin{equation*}
	\begin{split}
    \psi u(x, t)
		& = \frac{1}{2 \pi} \psi(t) \sum_{n \in \zz} e^{i(xn + t n^{2 
		})} \widehat{\vp}(n) 
		\\
		& + \frac{1}{4 \pi^2} \psi(t) \sum_{n \in \zz} \int_\rr e^{ixn}  
		e^{it \lambda} \frac{ 1 - \psi(\lambda - n^2) 
		}{\lambda - n^2} \wh{w}(n, \lambda) \ d \lambda
		\\
		& - \frac{1}{4 \pi^2} \psi(t) \sum_{n \in \zz} \int _\rr e^{i(xn + 
		t n^2)}
		 \frac{1- \psi(\lambda - n^2)}{\lambda - n^2} \wh{w}(n, \lambda) \ d \lambda
		\\
		& + \frac{1}{4 \pi^2} \psi(t) \sum_{n \in \zz} \int_\rr
		e^{i(xn + t n^2)}
		\frac{\psi(\lambda - n^2)\left[ e^{it(\lambda - n^2)}-1 
		\right]}{\lambda - n^2} \wh{w}(n, \lambda) \ d \lambda
	\end{split}
\end{equation*}
%
%
which by a power series expansion of $[e^{it(\lambda - n^2)}-1]$ simplifies  
to
%
%
\begin{align}
	\label{nmain-int-expression-0}
  & \psi u(x, t) 
		\\
		\label{nmain-int-expression-1}
		& = \frac{1}{2 \pi} \psi(t) \sum_{n \in \zz} e^{i(xn + tn^{2 
		})} \widehat{\vp}(n) 
		\\
		\label{nmain-int-expression-2}
		& + \frac{1}{4 \pi^2} \psi(t) \sum_{n\in \zz} \int_\rr e^{ixn}  
		e^{it \lambda} \frac{ 1 - \psi(\lambda -  n^2) 
		}{\lambda -  n^2} \wh{w}(n, \lambda) \ d \lambda
		\\
		\label{nmain-int-expression-3}
		& - \frac{1}{4 \pi^2} \psi(t) \sum_{n\in \zz} \int_\rr e^{i(xn + 
		t n^2)}
		 \frac{1- \psi(\lambda -  n^2)}{\lambda -  n^2} \wh{w}(n, \lambda) \ d \lambda
		\\
		\label{nmain-int-expression-4}
		& + \frac{1}{4 \pi^2} \psi(t) \sum_{k \ge 1} \frac{i^k t^k}{k!}
		\sum_{n \in \zz} \int_\rr e^{i(xn + t n^2 )}
		\psi(\lambda -  n^2) (\lambda -  n^2)^{k-1} \wh{w}(n, \lambda)  
		\\
		& \doteq T(u) \notag
\end{align}
%
%
where $T = T_{\vp, \psi}$. 
\begin{definition}
  Given a Banach space $X$, denote $B_{X}(R) \doteq \left\{ f: \| f \|_{X} < R
  \right\}$. We say that the NLS ivp
  \eqref{nNLS-eq}-\eqref{nNLS-init-data} is
	\emph{locally well posed in
  $H^s(\ci)$ for small data} if 
	\begin{enumerate}
    \item For every $\vp(x) \in B_{H^{s}(\ci)}(R)$, $R$ sufficiently small
      there exists a unique $u \in C([-1,
      1], H^s(\ci))$ satisfying
      \eqref{nmain-int-expression-0}-\eqref{nmain-int-expression-4}.
    \item
      The data to solution map $u_0 \mapsto u(t)$ is uniformly continuous from
      $B_{H^{s}(\ci)}(R)$ 
      to $C(\left[ -1, 1 \right], H^s(\ci))$. That is, if
      $\{u_{0,n} \}, \{v_{0,n}\} \subset B_{H^{s}(\ci)}(R)$ such that $\|u_{0,n} -
      v_{0,n} \|_{H^{s}(\ci)} \to 0$, then 
      $\sup_{t \in [-1, 1]}
      \|u_{n}(\cdot, t) - v_{n}(\cdot, t) \|_{H^s(\ci)} \to 0$.
  \end{enumerate}
	Otherwise, we say that the NLS ivp is \emph{ill-posed}.
\end{definition}
%
\begin{definition}
  Let $\mathcal{Y}$ be the space of functions $F(\cdot)$ such that
  \begin{enumerate}[(i)]
   \item{$F: \ci \times \rr \to \cc$ }.
   \item{ $F(x, \cdot) \in \mathcal{S}(\rr)$ for each $x \in \ci$}.
   \item{ $F(\cdot, t) \in C^{\infty}(\ci)$for each $t \in \rr$}.
  \end{enumerate}
  Let $Y^{s}$ denote the completion of $\mathcal{Y}$ with
  respect to the norm
  %
  %
  \begin{equation}
	\label{nY-s-norm}
	\begin{split}
		\|u\|_{Y^s} = \|u\|_{X^s} + \|\wh{u}\|_{ \ell^2_n L^1_\lambda }
	\end{split}
\end{equation}
  %
    %
    where
\begin{equation}
	\label{nX^s-norm}
	\begin{split}
		& \|u\|_{X^s}
		= \left ( \sum_{n\in \zz} \left (1 + |n| \right )^{2s} \int_\rr \left ( 1 + | 
		\lambda - n^{2} | \right ) | \wh{u} ( n, \lambda ) |^2
		\right )^{1/2}
	\end{split}
\end{equation}
and
%
%
\begin{equation}
	\label{nE-norm}
	\|\wh{u}\|_{ \ell^2_n L^1_\lambda } = \left[ \sum_{n \in \zz}(1 + | n |)^{2s} \left(
	\int_{\rr}| \wh{u}(n, \lambda) |d \lambda \right)^{2} \right]^{1/2}.
\end{equation}
    %
  \end{definition}
%
The $Y^s$ spaces embed continuously into $C([-1,1]), H^{s}(\ci)$. More
specifically, we have the following, whose proof
is provided in the appendix.
\begin{lemma}
	\label{nlem:cutoff-loc-soln}
	Let $\psi(t)$ be a smooth cutoff function with $\psi(t) =1$ for $t \in [-1,
  1]$. If
  $\psi(t)u(x,t) \in Y^s$, then $u \in C([-1, 1], H^s(\ci))$.
\end{lemma}

We are now prepared to state the main result of this paper.
%
%
%
%
%%%%%%%%%%%%%%%%%%%%%%%%%%%%%%%%%%%%%%%%%%%%%%%%%%%%%
%
%
%	Main Result				
%
%
%%%%%%%%%%%%%%%%%%%%%%%%%%%%%%%%%%%%%%%%%%%%%%%%%%%%%
%
%
\begin{theorem}
\label{nthm:main}
The initial value problem 
\eqref{nNLS-eq}-\eqref{nNLS-init-data} is locally well-posed for small data in $H^s(\ci)$ for $s \ge
0$.
%
\end{theorem} 
%
%
%
%
%
%
%%%%%%%%%%%%%%%%%%%%%%%%%%%%%%%%%%%%%%%%%%%%%%%%%%%%%
%
%
%			Proof of Theorem	
%
%
%%%%%%%%%%%%%%%%%%%%%%%%%%%%%%%%%%%%%%%%%%%%%%%%%%%%%
%
%
\section{Proof of Main Theorem}
%
%
To prove well-posedness for the NLS ivp we we will 
show that for initial data $\vp \in B_{H^{s}(\ci)}(R)$, $R$ sufficiently small,
$T$ is a contraction on
$B_{Y^{s}}(M_{R})$ by estimating the $Y^s$
norm of \eqref{nmain-int-expression-1}-\eqref{nmain-int-expression-4}. The 
Picard fixed point theorem will
then yield a unique solution in $Y^{s}$ to
$u = Tu$. An application of
\cref{nlem:cutoff-loc-soln} will then imply $u \in Y^{s} \cap C([-1, 1],
H^s(\ci))$, and hence 
$\psi u = Tu$ for $| t | \le 1$.
Lipschitz continuity of the data to solution map (and hence, uniform
continuity) will follow from estimates used to establish the contraction
mapping. 
%
%%%%%%%%%%%%%%%%%%%%%%%%%%%%%%%%%%%%%%%%%%%%%%%%%%%%%
%
%
%		Estimation of Integral Equality Part 1		
%
%
%%%%%%%%%%%%%%%%%%%%%%%%%%%%%%%%%%%%%%%%%%%%%%%%%%%%%
%
%
%
%
\subsection{Estimate for \eqref{nmain-int-expression-1}.}
%
%
Letting $f(x,t) = \psi(t) \sum_{n \in \zz} e^{i(xn + tn^{2})} 
\wh{\vp}(n)$, we have \\ $\wh{f}(n,t) = \psi(t) \wh{\vp}(n) e^{itn^{2}}$,
from which we obtain
%
%
\begin{equation}
	\label{nfourier-trans-calc}
	\begin{split}
		\wh{f}(n, \lambda)
		& = \wh{\vp}(n) \int_\rr e^{-it( \lambda - n^{2})} 
		\psi(t) \ d t
    = \wh{\psi}(\lambda - n^{2}) \wh{\vp}(n).
	\end{split}
\end{equation}
%
%
%
%
%
%
Therefore
%
\begin{equation}
	\begin{split}
	\label{nmain-int1-est}
		\|\eqref{nmain-int-expression-1}\|_{Y^s}
		& \simeq \left (  \sum_{n\in \zz} \left (1 + |n| \right )^s \int_\rr \left( 1 + | \lambda - n^{2} 
		| \right )
    | \wh{\psi}(\lambda - n^{2}) \wh{\vp}(n) |^2 d \lambda \right)^{1/2} 
		\\
		& + \left[ \sum_{n \in \zz }\left( 1 + | n | \right)^{2s} \left( \int_{\rr} |
    \wh{\psi}(\lambda - n^{2})\wh{\vp}(n) | d \lambda
		\right)^{2} \right]^{1/2}
		\\
    & = c_{\psi}
		\|\vp\|_{H^s(\ci)}.
	\end{split}
\end{equation}
%
%
%
%
\subsection{Estimate for \eqref{nmain-int-expression-2}.}
We now need the following lemma, whose proof is provided in the appendix.
%
%
%%%%%%%%%%%%%%%%%%%%%%%%%%%%%%%%%%%%%%%%%%%%%%%%%%%%%
%
%
%			Schwartz Multiplier	
%
%
%%%%%%%%%%%%%%%%%%%%%%%%%%%%%%%%%%%%%%%%%%%%%%%%%%%%%
%
%
\begin{lemma}
\label{nlem:schwartz-mult}
	For $\psi \in S(\rr)$,
%
%
\begin{equation}
	\label{nschwartz-mult}
	\begin{split}
		\|\psi f \|_{Y^s} \le c_{\psi} \|f \|_{Y^s}.
	\end{split}
\end{equation}
%
%
\end{lemma}
%
%
Hence,
%
%
\begin{equation}
  \label{nyu}
	\begin{split}
		\|\eqref{nmain-int-expression-2}\|_{Y^s} 
    & \le c_{\psi}
		\| \sum_{n \in \zz} e^{ixn} \int_\rr 
		e^{it \lambda} \frac{ 1 - \psi (\lambda - n^{2} ) 
		}{\lambda - n^{2}} \wh{w}(n, \lambda) \ 
		d \lambda\|_{Y^s}.
			\end{split}
\end{equation}
%
We first estimate
%
%
\begin{equation}
\label{nmain-int2-est-X-s-part}
\begin{split}
  & \| \sum_{n \in \zz} e^{ixn} \int_\rr 
		e^{it \lambda} \frac{ 1 - \psi (\lambda - n^{2} ) 
		}{\lambda - n^{2}} \wh{w}(n, \lambda) \ 
		d \lambda\|_{X^s}
		\\
    & = \left( \sum_{n \in \zz} \left (1 + |n| \right )^{2s} \int_\rr
		(1 + |\lambda - n^{2}|) \left | \frac{1 - \psi(\lambda - n^{2 
		})}{\lambda - n^{2}} 
		\wh{w}(n, \lambda) \right |^2 \ d 
		\lambda \right)^{1/2}
		\\
		& \le \left( \sum_{n \in \zz} \left (1 + |n| \right )^{2s} \int_{| \lambda - n^{2}| \ge 1}
		(1 + |\lambda - n^{2}|) \frac{|\wh{w}(n, \lambda)|^2 }{|\lambda - n^{2}|^2} 
		\ d 
		\lambda \right)^{1/2}
		\\
		& \lesssim  \left( \sum_{n \in 
		\zz} \left (1 + |n| \right )^{2s} \int_\rr
		\frac{|\wh{w}(n, \lambda) |^2}{1+ |\lambda - 
		n^{2}|} 
		 \ d \lambda 
		\right)^{1/2}
		\\
		& \lesssim  \|u\|_{X^s}^3
\end{split}
\end{equation}
%
%
%
where the last two steps follow from the inequality 
%
\begin{equation}
	\label{none-plus-ineq}
	\begin{split}
		\frac{1}{|\lambda - n^{2}| } \le \frac{2}{1 + |\lambda - n^{2}| }, 
		\qquad |\lambda - n^{2}| \ge 1
	\end{split}
\end{equation}
%
%
and the following trilinear estimate, whose proof we leave for later.
%
%
%%%%%%%%%%%%%%%%%%%%%%%%%%%%%%%%%%%%%%%%%%%%%%%%%%%%%
%
%
%				Proposition
%
%
%%%%%%%%%%%%%%%%%%%%%%%%%%%%%%%%%%%%%%%%%%%%%%%%%%%%%
%
%
\begin{proposition}
\label{nprop:trilinear-est}
	%
	%
	For any $s \ge 0$ and $b \ge 3/8$, we have
	\begin{equation}
		\left( \sum_{n \in \zz} \left (1 + |n| \right )^{2s} \int_\rr
		\frac{|\wh{w_{fgh}}(n, \lambda) |^2}{\left (1+ |\lambda - 
    n^{2}| \right )^{2b}} 
		 \ d \lambda 
		\right)^{1/2}
		\lesssim \|f\|_{X^s} \|g\|_{X^s}\|h\|_{X^s}
	\end{equation}
	where $w_{fgh}(x,t)$ = $fg \bar h (x,t)$.
%
%
%
%
\end{proposition}
%
%
Furthermore,
%
%
%
%
\begin{equation}
	\label{nmain-int-expression-2-Y-s-part}
	\begin{split}
    & \| \wh{\eqref{nmain-int-expression-2}}\|_{\ell^{2}_{n}L^{1}_{\lambda}}
		\\
    & \simeq \left[ \sum_{n \in \zz}(1 + | n |)^{2s} \left(
		\int_{\rr}\frac{|1 - \psi(\lambda - n^{2})|}{|\lambda - n^{2}|} |\wh{w}(n, \lambda)| d
		\lambda \right)^{2} \right]^{1/2}
		\\
    & \le \left[ \sum_{n \in \zz}(1 + | n |)^{2s} \left(
    \int_{| \lambda - n^{2} | \ge 1 }\frac{|\wh{w}(n, \lambda)|}{|\lambda - n^{2}|}  d
		\lambda \right)^{2} \right]^{1/2}
    \\
    & \lesssim \left[ \sum_{n \in \zz}(1 + | n |)^{2s} \left(
    \int_{\rr}\frac{|\wh{w}(n, \lambda)|}{1 + |\lambda - n^{2}|}  d
		\lambda \right)^{2} \right]^{1/2}
    \\
		& \lesssim \|f\|_{X^s} \|g\|_{X^s}\|h\|_{X^s}
	\end{split}
\end{equation}
%
%
where the last two steps follow from \eqref{none-plus-ineq} and the following
corollary to \cref{nprop:trilinear-est}.
%
%
%%%%%%%%%%%%%%%%%%%%%%%%%%%%%%%%%%%%%%%%%%%%%%%%%%%%%
%
%
%				Second trilinear Estimate 
%
%
%%%%%%%%%%%%%%%%%%%%%%%%%%%%%%%%%%%%%%%%%%%%%%%%%%%%%
%
%
\begin{corollary}
\label{ncor:trilinear-estimate2}
	For $s \ge 0$ we have
%
%
\begin{equation}
	\label{ntrilinear-estimate2}
	\begin{split}
		\left( \sum_{n \in \zz} \left (1 + |n| \right )^{2s}  \left ( \int_\rr 
		\frac{|\wh{w_{fgh}}(n, \lambda) |}{1 + | \lambda - n^{2} |}
		 \ d\lambda \right)^2  \right)^{1/2} \lesssim \|f\|_{X^s} \|g\|_{X^s}\|h\|_{X^s}.
	\end{split}
\end{equation}
\end{corollary}
%
%
Combining \eqref{nyu}, \eqref{nmain-int2-est-X-s-part}, and
\eqref{nmain-int-expression-2-Y-s-part}, we conclude that
%
%
%
%
\begin{equation}
	\label{nmain-int2-est}
	\begin{split}
		\|\eqref{nmain-int-expression-2}\|_{Y^s} \le c_{\psi}\|f\|_{X^s} \|g\|_{X^s}\|h\|_{X^s}.
	\end{split}
\end{equation}
%
%
\subsection{Estimate for \eqref{nmain-int-expression-3}.}
Letting $$f(x,t) = \psi(t) \sum_{n \in \zz} e^{i\left( xn + tn^{2} \right)} 
\int_\rr \frac{1 - \psi\left( \lambda - n^{2} \right)}{\lambda - n^{2}} 
\wh{w} \left( n, \lambda \right) \ d \lambda,$$ we have
%
%
\begin{equation*}
	\begin{split}
		& \wh{f^x}(n, t) = \psi(t) e^{itn^{2}} \int_\rr
		\frac{1 - \psi\left( \lambda - n^{2} \right)}{\lambda - n^{2}} 
		\wh{w}(n, \lambda) \ d \lambda
	\end{split}
\end{equation*}
and
\begin{equation*}
	\begin{split}
		 \wh{f}\left( n, \lambda \right)
		 & = \int_\rr e^{-it\left( \lambda - n^{2} 
		\right)} \psi(t) \int_\rr \frac{1 - \psi\left( 
		\lambda - n^{2} 
		\right)}{\lambda - n^{2}} \wh{w}(n, \lambda) \ d \lambda d \lambda
		\\
    & = \wh{\psi}\left( \lambda - n^{2} \right) \int_\rr 
		\frac{1 - \psi\left( 
		\lambda - n^{2} 
		\right)}{\lambda - n^{2}} \wh{w}(n, \lambda) \ d \lambda.
	\end{split}
\end{equation*}
Therefore,
%
%
\begin{equation}
  \label{niu}
	\begin{split}
		& \| \eqref{nmain-int-expression-3} \|_{X^s} 
		\\
		& \simeq \left( \sum_{n \in \zz} \left (1 + |n| \right )^{2s} \int_\rr \left( 1 + | \lambda - n^{2
    } \right ) | | \wh{\psi}\left( \lambda - n^{2} \right) |^2 \ d \lambda
		\right.
		\\
		& \times \left . |
		\int_\rr \frac{1 - \psi\left( \lambda - n^{2} \right)}{\lambda -
		n^{2}} \wh{w}(n, \lambda) \ d \lambda |^2  \right)^{1/2}
		\\
    & \le c_{\psi}\left( \sum_{n \in \zz} \left (1 + |n| \right )^{2s} | \int_\rr
		\frac{1 - \psi\left( \lambda - n^{2} \right)}{\lambda - n^{2}}
		\wh{w}(n, \lambda) \ d\lambda |^2 \right)^{1/2}
		\\
		& \simeq \left( \sum_{n \in \zz} \left (1 + |n| \right )^{2s}  \left ( \int_\rr
		\frac{1 - \psi\left( \lambda - n^{2} \right)}{|\lambda - n^{2}|}
		|\wh{w}(n, \lambda) | \ d\lambda \right )^2 \right)^{1/2}
		\\
		& \le \left( \sum_{n \in \zz} \left (1 + |n| \right )^{2s}  \left ( \int_{| \lambda - 
		n^{2} | \ge 1}
		\frac{|\wh{w}(n, \lambda) | }{|\lambda - n^{2}|}
		\ d\lambda \right )^2 \right)^{1/2}.
	\end{split}
\end{equation}
%
%
Applying estimate \eqref{none-plus-ineq}, we bound this by
%
%%
\begin{equation}
	\label{nmain-int3-est-X-s-part}
	\begin{split}
		& 2 \left( \sum_{n \in \zz} \left (1 + |n| \right )^{2s}  \left ( \int_\rr
		\frac{|\wh{w}(n, \lambda)| }{1 + |\lambda - n^{2}|}
		 \ d\lambda \right )^2 \right)^{1/2}
		 \\
		& \lesssim \|u\|_{X^s}^3
	\end{split}
\end{equation}
%
%%
where the last step follows from \cref{ncor:trilinear-estimate2}.
Furthermore, 
%
%
\begin{equation}
	\label{nmain-int-estimate-3-Y-s-part}
	\begin{split}
    \|\wh{\eqref{nmain-int-expression-3}}\|_{\ell^{2}_{n}L^{1}_{\lambda}}
		& \simeq \left[ \sum_{n \in \zz} (1 + | n |)^{2s} \int_{\rr} |
    \wh{\psi}(\lambda - n^{2}) |^{2} \left( \int_{\rr}\frac{1 - \psi(\lambda -
		n^{2})}{\lambda - n^{2}} \wh{w}(n, \lambda) d \lambda \right)^{2} d \lambda
		\right]^{1/2}
		\\
		& \le c_{\psi} \left[ \sum_{n \in \zz} (1 + | n |)^{2s} \left(
		\int_{\rr} \frac{1 - \psi(\lambda - n^{2})}{\lambda - n^{2}}
		\wh{w}(n, \lambda) d \lambda
		\right)^{2}\right]^{1/2}
		\\
		& \le 2 c_{\psi} \left[ \sum_{n \in \zz} (1 + | n |)^{2s} \left(
		\int_{\rr} \frac{\wh{w}(n, \lambda) }{1 + |\lambda - n^{2}|}
		d \lambda
		\right)^{2}\right]^{1/2}
		\\
		& \lesssim \|f\|_{X^s} \|g\|_{X^s} \|h\|_{X^s}
	\end{split}
\end{equation}
%
%
where the last two steps follow from \eqref{none-plus-ineq} and
\cref{ncor:trilinear-estimate2}, respectively. Combining \eqref{niu},
\eqref{nmain-int3-est-X-s-part}, and \eqref{nmain-int-estimate-3-Y-s-part}, we
conclude that
%
%
\begin{equation}
	\label{nmain-int3-est}
	\begin{split}
		\|\eqref{nmain-int-expression-3}\|_{Y^s} 
    \le c_{\psi} \|f\|_{X^s} \|g\|_{X^s} \|h\|_{X^s}.
	\end{split}
\end{equation}
%
%
%
\subsection{Estimate for \eqref{nmain-int-expression-4}.}
Note that
%
%
\begin{equation}
	\label{n1n}
	\begin{split}
		\eqref{nmain-int-expression-4} \simeq \sum_{k \ge 1}
		\frac{i^k}{k!}g_k(x,t)
	\end{split}
\end{equation}
%
%
where 
%
%
\begin{equation*}
	\begin{split}
		& g_k(x,t) = t^k \psi(t) \sum_{n \in \zz} e^{i\left( xn + tn^{2}
		\right)} h_k(n),
		\\
		& h_k(n) = \int_\rr \psi \left( \lambda - n^{2} \right) \cdot \left(
		\lambda - n^{2} \right)^{k -1} \wh{w}(n, \lambda) \ d \lambda.
	\end{split}
\end{equation*}
%
%
Hence
%
%
\begin{equation*}
	\begin{split}
		\wh{g_k^x}(n, t) = t^{k} \psi(t) e^{i t n^{2}} h_k(n)
	\end{split}
\end{equation*}
%
%
which gives
%
%
\begin{equation*}
	\begin{split}
		\wh{g_k}(n, \lambda)
		& = h_k(n) \int_\rr e^{-it\left( \lambda - n^{2} \right)}
		t^{k}\psi(t) \ dt
		\\
		& = h_k(n) \wh{t^{k}\psi(t)} \left( \lambda - n^{2} \right).
	\end{split}
\end{equation*}
%
%
Applying this to \eqref{n1n}, we obtain
%
%
\begin{equation}
	\label{n2n}
	\begin{split}
		\|\eqref{nmain-int-expression-4}\|_{X^s} 
		& \simeq \left( \sum_{n \in \zz} \left (1 + |n| \right )^{2s} \int_\rr \left( 1 + | \lambda -
		n^{2}
		|
		\right) | \wh{\sum_{k \ge 1} \frac{i^k}{k!}g_k(x,t)} |^2 \ d \lambda
		\right)^{1/2}
		\\
		& \le \sum_{k \ge 1} \frac{1}{k!}\left( \sum_{n \in \zz} \left (1 + |n| \right )^{2s}
		\int_\rr \left( 1 + | \lambda - n^{2} | \right) | \wh{g_k}(n, \lambda) |^2 \
		d \lambda \right)^{1/2}
		\\
		& = \sum_{k \ge 1} \frac{1}{k!} \left( \sum_{n \in \zz} \left (1 + |n| \right )^{2s}
		\int_\rr \left( 1 + | \lambda - n^{2} | \right) | h_k(n) \wh{t^k
		\psi(t)} \left( \lambda - n^{2} \right) |^2 \ d \lambda \right)^{1/2}
		\\
		& = \sum_{k \ge 1} \frac{1}{k!} \left( \sum_{n \in \zz} \left (1 + |n| \right )^{2s} |
		h_k(n) |^2 \int_\rr \left( 1 + | \lambda - n^{2} | \right) | \wh{t^k
		\psi(t)} \left( \lambda - n^{2} \right) |^2 \ d \lambda \right)^{1/2}.
	\end{split}
\end{equation}
%
%
Notice that for fixed $n$, the change of variable $\lambda - n^{2} = \lambda'$
gives
%
%
\begin{equation}
	\label{n3n}
	\begin{split}
		\int_\rr \left( 1 + | \lambda - n^{2} | \right) | \wh{t^{k}
		\psi(t)}\left( \lambda - n^{2} \right) |^2 \ d \lambda
		& = \int_\rr \left( 1 + |\lambda'| \right) | \wh{t^k \psi(t)}(\lambda') |^2 \
		d \lambda'
		\\
		& \le \int_\rr \left( 1 + |\lambda'| \right)^2 | \wh{t^k \psi(t)}(\lambda')
		|^2 \ d \lambda'
		\\
		& \lesssim \int_\rr \left( 1 + | \lambda' |^2 \right) | \wh{t^{k}
		\psi(t)}(\lambda') |^2 \ d \lambda'
		\\
		& = \|t^k \psi(t) \|_{H^1(\rr)}^2.
	\end{split}
\end{equation}
%
%
But
%
%
\begin{equation}
	\label{n4n}
	\begin{split}
		\|t^k \psi(t) \|_{H^1(\rr)}^2
		& = \left( \|t^k \psi(t)\|_{L^2(\rr)} + \|\p_t \left( t^k \psi(t)
		\right)\|_{L^2(\rr)} \right)^2
		\\
		& \lesssim \|t^{k}\psi(t) \|_{L^2(\rr)}^2 + \|\p_t \left (t^{k}
		\psi(t) \right )\|_{L^2(\rr)}^2
		\\
		& \le \|t^k \psi(t) \|_{L^2(\rr)}^2 + \|t^k \p_t \psi(t)
		\|_{L^2(\rr)}^2 + \|k t^{k -1} \psi(t) \|_{L^2(\rr)}^2
		\\
		& = c_{\psi} + c_{\psi}' + c_{\psi}''k^2 
		\\
    & \lesssim c_{\psi} k^2.
	\end{split}
\end{equation}
%
%
Hence, applying \eqref{n3n} and \eqref{n4n} to \eqref{n2n}, we obtain
%
%%
\begin{equation}
	\label{n5n}
	\begin{split}
		\|\eqref{nmain-int-expression-4} \|_{X^s}
		& \lesssim
    c_{\psi}\sum_{k \ge 1} \frac{k}{k!} \left( \sum_{n \in \zz} \left (1 + |n| \right )^{2s} | h_k(n) |^2 
		\right)^{1/2}
		\\
		& \lesssim \sum_{k \ge 1} \frac{1}{(k-1)!}
		\times \sup_{k \ge 1} \left( \sum_{n \in \zz} \left (1 + |n| \right )^{2s} | 
		h_k(n) |^2 \right)^{1/2}
		\\
		& = \sum_{k \ge 1} \frac{1}{(k-1)!}
    \\
    & \times \sup_{k \ge 1} 
		\left( \sum_{n \in \zz} \left (1 + |n| \right )^{2s} |\int_\rr 
		\psi\left( \lambda - n^{2} \right) \cdot \left( \lambda - n^{2} 
    \right)^{k -1} \wh{w}(n, \lambda) \ d \lambda|^{2} \right)^{1/2}.
    	\end{split}
\end{equation}
%
%%
Recall that $0 \le \psi \le 1, \text{ supp} \, \psi \subset [-2,2 ]$. 
Hence, we bound the
right hand side of \eqref{n5n} by
%
%%
\begin{equation*}
	\begin{split}
    & \sum_{k \ge 1} \frac{2^{k-1}}{(k-1)!} \times \left( \sum_{n \in \zz} (1 + | n |)^{2s}| 
		\int_{| \lambda - n^{2}  |\le 1}  \wh{w}(n, \lambda) \ d \lambda |^2 
		\right)^{1/2}
    \\
    & = e^{2} \left( \sum_{n \in \zz} (1 + | n |)^{2s}| 
		\int_{| \lambda - n^{2}  |\le 1}  \wh{w}(n, \lambda) \ d \lambda |^2 
		\right)^{1/2}
    \\
    & \le e^{2} \left[ \sum_{n \in \zz} (1 + | n |)^{2s}\left (  
		\int_{| \lambda - n^{2}  |\le 1} | \wh{w}(n, \lambda) | \ d \lambda \right ) ^2 
		\right]^{1/2}
	\end{split}
\end{equation*}
%
%%
which by the inequality
%
%%
\begin{equation*}
	\begin{split}
		\frac{1 + | \lambda - n^{2} |}{1 + | \lambda  - n^{2} |} \le 
		\frac{2}{1 + | \lambda - n^{2} |}, \qquad | \lambda - n^{2}  | \le 1
	\end{split}
\end{equation*}
%
%%
is bounded by 
%
%%
\begin{equation}
\label{nmain-int4-est-X-s-part}
	\begin{split}
    & 2e^{2} \left[ \sum_{n \in \zz} (1 + | n |)^{2s}\left ( \int_\rr
		\frac{|\wh{w}(n, \lambda)|}{1 + | \lambda - n^{2} |} \ d \lambda \right ) ^2 
		\right]^{1/2} \\
		& \lesssim \|u\|_{X^s}^3
	\end{split}
\end{equation}
%
%%
where the last step follows from \cref{ncor:trilinear-estimate2}. Similarly,
we have
%
%
\begin{equation}
\label{nmain-int4-est-Y-s-part}
	\begin{split}
    \|\wh{\eqref{nmain-int-expression-4}}\|_{ \ell^2_n L^1_\lambda }
		& \simeq \left[ \sum_{n \in
		\zz}(1 + | n |)^{2s} \left( \int_{\rr} | \sum_{k \ge 1}
		\wh{\frac{i^{k}}{k!}g_{k}(x,t)(n, \lambda)} |d \lambda \right)^{2} \right]^{1/2}
		\\
		& \le \sum_{k \ge 1} \frac{1}{k!} \left[ \sum_{n \in \zz} (1 + | n
    |)^{2s} \left( \int_{\rr} | \wh{g_{k}}(n, \lambda) | d \lambda \right)^{2}
		\right]^{1/2}
		\\
		& = \sum_{k \ge 1} \frac{1}{k!} \left[ \sum_{n \in \zz} (1 + | n
		|)^{2s} | h_{k}(n) |^2 \left( \int_{\rr} | \wh{t^{k} \psi(t)}(\lambda -
		n^{2}) |d \lambda \right)^{2} \right]^{1/2}
		\\
		& = c_{\psi} \sum_{k \ge 1} \frac{1}{k!} \left[ \sum_{n \in \zz} (1 + | n
		|)^{2s} | h_{k}(n) |^2 \right]^{1/2}
		\\
		& \lesssim \|u\|_{X^s}^{3}
	\end{split}
\end{equation}
%
%
where the last step follows from the computations starting from \eqref{n5n}
through \eqref{nmain-int4-est-X-s-part}.
Combining \eqref{nmain-int4-est-X-s-part} and \eqref{nmain-int4-est-Y-s-part}, we
have
%
%
\begin{equation}
\label{nmain-int4-est}
	\begin{split}
    \|\eqref{nmain-int-expression-4}\|_{Y^s} \le c_{\psi} \|u\|_{X^s}^{3}.
	\end{split}
\end{equation}
%
%
Collecting estimates \eqref{nmain-int1-est}, \eqref{nmain-int2-est}, 
\eqref{nmain-int3-est}, and \eqref{nmain-int4-est}, and recalling 
\eqref{nmain-int-expression-1}-\eqref{nmain-int-expression-4}, we see that
$$\|Tu\|_{Y^s} \le c_{\psi} \left( \|\vp \|_{H^s(\ci)} + \|u\|_{X^s}^3 \right )$$ 
which by the inequality $\|u\|_{X^s} \le \|u\|_{Y^s}$ yields the following.
%%
%%%%%%%%%%%%%%%%%%%%%%%%%%%%%%%%%%%%%%%%%%%%%%%%%%%%%
%
%% Contraction Proposition
%				 
%%%%%%%%%%%%%%%%%%%%%%%%%%%%%%%%%%%%%%%%%%%%%%%%%%%%%%
%%
%%
%
\begin{proposition}
\label{nprop:contraction}
	Let $s \ge0$. Then
%
%%
\begin{equation*}
	\begin{split}
		\|Tu\|_{Y^s} \le c_{\psi} \left( \|\vp \|_{H^s(\ci)} + \|u\|_{Y^s}^3 
		\right).
	\end{split}
\end{equation*}
%
%%
\end{proposition}
%
%
\subsection{Existence and Uniqueness} 
\label{ssec:exis-uniq}
We will now use \cref{nprop:contraction} to prove local well-posedness for the 
NLS ivp. Let $c = c_{\psi}^{1/2}$. Suppose that the intial data is small, that
is, it satisfies
%
%%
\begin{equation}
	\begin{split}
		\|\vp\|_{H^s(\ci)} \le \frac{15}{64c^3}.
	\end{split}
\end{equation}
%
%%
Then if $$\|u\|_{Y^s} \le \frac{1}{4c}$$ we have
%
%%
\begin{equation*}
	\begin{split}
		\|T u \|_{Y^s} 
		& \le c^2 \left[ \frac{15}{64c^3} + \left( 
		\frac{1}{4c} \right)^3 \right]
		=  \frac{1}{4c}.
	\end{split}
\end{equation*}
%
%%
Hence, $T=T_{\vp, \psi}$ maps the ball $B_{Y^{s}}\left( \frac{1}{4c} \right)$ into 
itself. Next, note that
%
%%
\begin{equation*}
	\begin{split}
		Tu - Tv = \eqref{nmain-int-expression-2} + \eqref{nmain-int-expression-3} 
		+ \eqref{nmain-int-expression-4}
	\end{split}
\end{equation*}
%
%%
where now $w = u | u |^2 - v | v |^{2}$. Rewriting
%
%%
\begin{equation*}
	\begin{split}
		u | u |^{2} - v | v |^{2}
		& = | u |^2 \left( u -v \right) + v\left( | u 
		|^2 - | v |^2
		\right)
		\\
		& = u \bar u \left( u -v \right) + v u \bar u - v v \bar v
		\\
		& = u \bar u \left( u - v \right) + v \bar u\left( u - v \right) + v 
		\bar u v - v v \bar v
		\\
		& = u \bar u \left( u -v \right) + v \bar u\left( u - v \right) + v v 
		\left( \overline{u -v} \right),
	\end{split}
\end{equation*}
%
%%
the triangle inequality and linearity of the Fourier transform give
%
%%
\begin{equation*}
	\begin{split}
		| \wh{w}(n, \lambda) | = | \mathcal{F}(u | u |^2 - v| v |^2) |
		& \le | \wh{u \overline{u} \left (u -v \right )} | +
		| \wh{v \overline{u} (u -v)} | + |\wh{v v 
		(\overline{u-v})}|
		\\
		& = | \wh{w_1} | + | \wh{w_2} | + | \wh{w_3} |
	\end{split}
\end{equation*}
%
%%
where
%
%%
\begin{equation*}
	\begin{split}
		w_1 = u \bar u \left( u -v \right), \qquad w_2 = v \bar u \left( u -v 
		\right), \qquad w_3 = v v \left( \overline{u -v} \right).
	\end{split}
\end{equation*}
%
%%
Hence, $Tu - Tv = \sum_{\ell=1}^{3} 
T_\ell(u, v)$, where
\begin{align}
	\label{nmain-int-exp-mod1}
	& \frac{i}{4 \pi^2} \psi(t) \sum_{n\in \zz} \int_\rr e^{ixn}  
		e^{it \lambda} \frac{ 1 - \psi(\lambda - n^{2}) 
		}{\lambda - n^{2}} \wh{w_\ell}(n, \lambda) \ d \lambda
		\\
		\label{nmain-int-exp-mod2}
		- & \frac{i}{4 \pi^2} \psi(t) \sum_{n\in \zz} \int_\rr e^{i(xn + 
		tn^{2})}
		 \frac{1- \psi(\lambda - n^{2})}{\lambda - n^{2}} \wh{w_\ell}(n, \lambda) \ d \lambda
		\\
		\label{nmain-int-exp-mod3}
		+ & \frac{i}{4 \pi^2} \psi(t) \sum_{k \ge 1} \frac{i^k t^k}{k!}
		\sum_{n \in \zz} \int_\rr e^{i(xn + tn^{2} )}
		\psi(\lambda - n^{2}) (\lambda - n^{2})^{k-1} \wh{w_\ell}(n, \lambda)  
		\\
		\doteq & T_\ell(u). \notag
\end{align}
Repeating the arguments used to estimate 
\eqref{nmain-int-expression-2}-\eqref{nmain-int-expression-4}, we obtain
%
%%
\begin{equation*}
	\begin{split}
    & \|T_1\|_{Y^s} \le c_{\psi} \|u -v \|_{Y^s} \|u\|^2_{Y^s}
		\\
    & \|T_2\|_{Y^s} \le c_{\psi} \|u -v \|_{Y^s} \|u\|_{Y^s} \|v\|_{Y^s}
		\\
    & \|T_3\|_{Y^s} \le c_{\psi} \|u -v \|_{Y^s} \|v\|_{Y^s}^2.
	\end{split}
\end{equation*}
%
%%
Therefore,
%
%%
\begin{equation}
	\label{n20a}
	\begin{split}
    \|Tu - Tv \|_{Y^s} = & \| \sum_{\ell =1}^{3} T_\ell(u, v) \|_{Y^s}
		\\
    & \le c_{\psi} \|u -v \|_{Y^s} \left( \|u\|_{Y^s}^2 + 
		\|u\|_{Y^s} \|v\|_{Y^s} + \|v\|_{Y^s}^2 \right)
		\\
		& \le c_{\psi} \|u -v\|_{Y^s} \left( \|u\|_{Y^s} + \|v\|_{Y^s} \right)^2
		\\
		& = c^2 \|u -v\|_{Y^s} \left( \|u\|_{Y^s} + \|v\|_{Y^s} \right)^2.
	\end{split}
\end{equation}
%
%%
If $u, v \in B_{Y^{s}}(\frac{1}{4c})$, it follows that
%
%%
\begin{equation}
	\label{n21a}
	\begin{split}
		\|Tu - Tv \|_{Y^s}
		& \le c^2 \|u -v \|_{Y^s} \left( \frac{1}{4c} + 
		\frac{1}{4c} \right)^2
		\\
		& = \frac{1}{4} \|u -v \|_{Y^s}. 
	\end{split}
\end{equation}
%
%%
We conclude that $T = T_{\vp, \psi}$ is a contraction on the ball
$B_{Y^{s}}(\frac{1}{4c})$. A Picard iteration, coupled with
\cref{nlem:cutoff-loc-soln} then yields a unique solution $u \in Y^{s} \cap
C(\left[ -1, 1 \right], H^{s}(\ci))$ to the integral equation $\psi u = Tu$.
%
%
%
\subsection{Lipschitz Continuity} 
\label{ssec:lip-cont-flow-map}
We now establish Lipschitz continuity of the data to solution map from
\\ $B_{H^{s}(\ci)}(\frac{15}{64c^{3}})$ to $C(\left[ -1, 1 \right], H^{s}(\ci))$.
Assume $\vp_1, \vp_2
\subset B_{H^s(\ci)}(\frac{15}{64c^{3}})$.
Then by the preceding arguments there exist $u, v \in Y^s$ such that 
$u = T_{\vp_1, \psi}$, $v = T_{\vp_2, \psi}$, and so
%
%
\begin{equation*}
	\begin{split}
		T_{\vp_1, \psi}(u) -
    T_{\vp_2, \psi}(v) = \frac{1}{2\pi} \psi(t) \sum_{n \in
		\zz}e^{i\left( xn + tn^{2} \right)} \wh{\vp_1 - \vp_2}(n) + \sum_{\ell=1
    }^{3} T_{\ell}(u).
	\end{split}
\end{equation*}
%
%
Using an argument similar to \eqref{nfourier-trans-calc}-\eqref{nmain-int1-est},
we obtain
%
%
\begin{equation*}
	\begin{split}
		\| \frac{1}{2\pi} \psi(t) \sum_{n \in
		\zz}e^{i\left( xn + tn^{2} \right)} \wh{\vp_1 - \vp_2}(n)\|_{Y^s}
		\le c_{\psi} \|\vp_{1} - \vp_{2}\|_{Y^s}.
	\end{split}
\end{equation*}
%
%
Hence, \eqref{n20a}-\eqref{n21a} gives
%
%
\begin{equation*}
	\begin{split}
    \sum_{\ell=1}^{3} T_{\ell}(u,v) \le \frac{1}{4}\|u-v\|_{Y^s}.
	\end{split}
\end{equation*}
%
%
Therefore,
%
%
\begin{equation*}
	\begin{split}
    \|u - v \|_{Y^s} \le c_{\psi}
		\|\vp_{1} - \vp_{2} \|_{H^{s}\left( \ci \right)}\| +
		\frac{1}{4} \|u -v \|_{Y^s}
	\end{split}
\end{equation*}
%
%
which implies
%
%
\begin{equation*}
	\begin{split}
    \frac{3}{4} \| u-  v\|_{Y^s} \le c_{\psi} \|\vp_1 - \vp_2 \|_{H^s(\ci)}
	\end{split}
\end{equation*}
%
%
or
%
%
\begin{equation*}
	\begin{split}
    \| u -  v \|_{Y^s} \le \frac{4}{3} c_{\psi} \|\vp_1 - \vp_2
    \|_{H^{s}(\ci)}
	\end{split}
\end{equation*}
%

%
which by \cref{nlem:cutoff-loc-soln} gives
%
%
	 %
	 %
	 \begin{equation}
     \label{lip-hs}
		 \begin{split}
       \sup_{t \in \left[ -1, 1 \right]}\|u(\cdot, t) -v(\cdot, t) \|_{H^s(\ci)} \le \frac{4}{3} c_{\psi} \|\vp_1 -
			\vp_2 \|_{H^{s}(\ci)}.
		 \end{split}
	 \end{equation}
	 %
   Furthermore, since $u = T_{\vp_1, \psi}$, $v = T_{\vp_2, \psi}$,
   \cref{nlem:cutoff-loc-soln} implies
   $\psi u = T_{\vp_1, \psi}$, $\psi v = T_{\vp_2, \psi}$ for $| t | \le 1$.
   Hence, the data to solution map of the NLS ivp is Lipschitz continuous from
   $B_{H^{s}(\ci)}( \frac{15}{64c^{3}})$ to  $C(\left[ -1, 1 \right],
   H^{s}(\ci))$. Since this implies uniform continuity of the data to solution map from
   $B_{H^{s}(\ci)}( \frac{15}{64c^{3}})$ to  $C(\left[ -1, 1 \right],
   H^{s}(\ci))$, the proof of \cref{nthm:main} is complete. \qquad
   \qedsymbol
%
%
%
%
\section{Proof of Trilinear Estimate}
%
%
%
%
%
%
Note first that $|\wh{w_{fgh}}(n, \lambda) |  = | \wh{f} * ( \wh{g} 
* \wh{\bar h})(n, \lambda)|$ and $| \wh{\bar{h}}(n, \lambda) | = |\overline{ \wh{\overline{h}} 
}(n, \lambda)| = | \wh{h}(-n, -\lambda) |$. It follows that
%
%
\begin{equation}
	\label{nnon-lin-rep}
	\begin{split}
		& | \wh{w_{fgh}}(n, \lambda)|
    \\
    & = |  \sum_{n_{1} } \int_{\lambda_{1}} \sum_{n_{2}}
    \int_{\lambda_{2}} \wh{f}\left( n
    -n_1,  \lambda - \lambda_1 \right) \wh{g}\left( n_{1} - n_2, \lambda_{1} - \lambda_2  
\right) \wh{\bar h}\left( n_2, \lambda_2 \right) d \lambda_2 d \lambda_1 |
\\
& \le \sum_{n_{1} } \int_{\lambda_{1}} \sum_{n_{2}}
\int_{\lambda_{2}}  | \wh{f}\left( n - n_1, \lambda - \lambda_1 
\right) | \times  | \wh{g}\left( n_1 - n_2, \lambda_1 - \lambda_2 
\right) | \times | \wh{\bar h}\left( n_{2}, \lambda_{2} \right) | d \lambda_2 d \lambda_1
\\
& = \sum_{n_{1} } \int_{\lambda_{1}} \sum_{n_{2}}
\int_{\lambda_{2}}  | \wh{f}\left( n - n_1, \lambda - \lambda_1 
     \right) | \times | \wh{g}\left( n_{1} - n_2, \lambda_{1} - \lambda_2 
\right) | \times | \wh{h}\left( -n_{2}, - \lambda_2 \right) | d \lambda_2 d \lambda_1
\\
& = \sum_{n_{1} } \int_{\lambda_{1}} \sum_{n_{2}}
\int_{\lambda_{2}} \frac{c_f\left( n - n_1, \lambda - \lambda_1 
\right)}{\left (1 + |n - n_{1}| \right )^s \left( 1 + | \lambda - \lambda_1 - (n - n_{1})^{2} | \right)^{b}}
\\
& \times \frac{c_{g}\left( n_1 - n_2, \lambda_1 - \lambda_2 \right)}{\left (1 + |n_1 -
n_2| \right ) 
^s\left( 1 + | \lambda_1 - \lambda_2 -  (n_1 - n_2)^{2}| 
\right)^{b}}
 \times \frac{c_{h}\left( -n_{2}, -\lambda_2 \right)}{\left (1 + |n_{2}| \right ) ^s\left( 1 + | 
\lambda_2 + n_{2}^{2} | \right)^{b}} \ d \lambda_2 d \lambda_1
\end{split}
\end{equation}
%
%
where
%
%
\begin{equation*}
	\begin{split}
		c_\sigma(n, \lambda) = \left (1 + |n| \right ) ^s \left( 1 + | \lambda - n^{2} |  
		\right)^{b} | \wh{\sigma}\left( n, \lambda \right) | .
	\end{split}
\end{equation*}
%
%
Hence
%
%
\begin{equation}
	\label{nconvo-est-starting-pnt}
	\begin{split}
		 & \left (1 + |n| \right )^s \left( 1 + | \lambda - n^{2} | \right)^{-b} | \wh{w_{fgh}}\left( 
		n, \lambda \right) |
		\\
		& \le \left( 1 + | \lambda - n^{2} | \right)^{-b}
    \sum_{n_{1} } \int_{\lambda_{1}} \sum_{n_{2}}
    \int_{\lambda_{2}}     \\
    & \frac{\left (1 + |n| \right )^s}{\left (1 +
		|n - n_{1}| \right )^s \left (1 + | n_1 - n_2| \right )^s \left (1 + |n_{2}| \right )^s} 
		\times \frac{c_f(n - n_{1}, \lambda - \lambda_1)}{\left( 1 + | \lambda - \lambda_1 - (n - n_{1})^{2} | 
		\right)^{b}}
		\\
		& \times
		\frac{c_g(n_1 - n_2, \lambda_1 - \lambda_2)}{\left( 1 + | \lambda_1 - \lambda_2 - (n_1 - n_2)^{2} | 
		\right)^{b}} \times
		\frac{c_h(-n_{2}, -\lambda_2)}{\left( 1 + | \lambda_2 + n_{2}^{2} | 
		\right)^{b}}\ d \lambda_2 d \lambda_1 .
	\end{split}
\end{equation}
%
%
For $s \ge 0$, observe that
%
%
\begin{equation}
	\label{nderiv-bound-easy-s}
	\begin{split}
		\frac{\left (1 + |n| \right ) ^s}{\left (1 + |n - n_{1}| \right ) ^s \left (1 + |n_1 - n_2| \right ) ^s \left (1 + |n_2| \right ) ^s} 
		\le 3^{s}
	\end{split}
\end{equation}
%
%
by the following lemma, whose proof is provided in the appendix.
%
%
\begin{lemma}
\label{nlem:splitting}
	For $\nu \ge 0$ we have
%
%
\begin{equation}
	\label{nsplitting}
	\begin{split}
    \left ( 1 + |a +b + c| \right)^{\nu} \le 3^{ \nu}
    \left(1 + | a | \right)^{\nu} \left(
    1 + | b | \right)^{\nu} \left( 1 + | c | \right)^{\nu}.
	\end{split}
\end{equation}
%
%
\end{lemma}
%
%
Hence, from \eqref{nconvo-est-starting-pnt} and \eqref{nderiv-bound-easy-s}, we 
obtain
%
\begin{equation*}
	\begin{split}
		& \left (1 + |n| \right )^s \left( 1 +  | \lambda - n^{2}  | \right)^{-b} | 
		\wh{w_{fgh}}\left( n, \lambda \right) | 
		\\
    & \lesssim  \frac{1}{\left( 1 +
		| \lambda - n^{2}| 
		\right)^{b}}  
		\times
    \sum_{n_{1} } \int_{\lambda_{1}} \sum_{n_{2}}
    \int_{\lambda_{2}} \frac{c_f\left( n - n_{1}, \lambda - \lambda_1 
		\right)}{\left (1 + |n - n_{1}| \right )^s \left( 1 + | \lambda - \lambda_1 - (n - n_{1})^{2} |
		\right)^{b}}
		\\
		& \times \frac{c_{g}\left( n_1 - n_2, \lambda_1 - \lambda_2 \right)}{\left (1 + |n_1 - n_2| \right ) 
		^s\left( 1 + | \lambda_1 - \lambda_2 -  (n_1 - n_2)^{2}| 
		\right)^{b}}
    \times \frac{c_{h}\left( -n_2, -\lambda_2 \right)}{\left (1 + |n_2| \right )
    ^s \left( 1 + | \lambda_2 + n_2^{2} | \right)^{b}} \ d \lambda_2 d \lambda_1 
    \\
		& = \left( 1 + | \lambda - n^{2} | \right)^{-b}
		\wh{C_f C_{g} C^+_{h}} \left( n, \lambda \right)
	\end{split}
\end{equation*}
%
%
where
%
%
\begin{equation*}
	\begin{split}
		C_\sigma(x, t) = \left[ \frac{c_\sigma\left( n, \lambda \right)}{\left( 
		1 + | \lambda - n^{2} | \right)^{b}} \right]^\vee,
		\ \ C^+_\sigma(x, t) = \left[ \frac{c_\sigma\left( -n, -\lambda \right)}{\left( 
		1 + | \lambda + n^{2} | \right)^{b}} \right]^\vee.
	\end{split}
\end{equation*}
%
%
Therefore
%
%
\begin{equation}
	\label{ngen-holder-pre-estimate}
	\begin{split}
		& \| \left( 1 + |n | \right)^s
		\left( 1 + | \lambda - n^{2} | \right)^{-b} \wh{w_{fgh}}(n, 
		\lambda)		
		\|_{L^2(\zz \times \rr)}
		\\
		& \lesssim \| \left( 1 + | \lambda - n^{2} | \right)^{-b}
		\wh{C_f C_{g} C^+_{h}} \|_{L^2(\zz \times \rr)}.
	\end{split}
\end{equation}
%
We now require the following multiplier estimate, whose proof can be found in
\cite{Bourgain-Fourier-transfo-1}. 
%
%
%%%%%%%%%%%%%%%%%%%%%%%%%%%%%%%%%%%%%%%%%%%%%%%%%%%%%
%
%
%			Four Mult Est	
%
%
%%%%%%%%%%%%%%%%%%%%%%%%%%%%%%%%%%%%%%%%%%%%%%%%%%%%%
%
%
%
%
%
%
%
%
\begin{lemma}
	\label{nlem:four-mult-est-L4}
	Let $(x, t) \in \ci \times \rr $ and $(n, \lambda) \in \zz \times \rr$ be 
	the dual variables. Then there is a 
	constant $c > 0$ such that
%
%
\begin{equation}
	\label{nfour-mult-est-L4*}
	\begin{split}
    \| \left( 1 + | \lambda - n^{2} | 
		\right)^{-\frac{3}{8}}
		\wh{f}\|_{L^2(\zz \times \rr)} \le c \|f \|_{L^{4/3}( \ci \times \rr)}
	\end{split}
\end{equation}
for every test function $f(x, t)$. 
%
%
\end{lemma}
%
A proof of a more general result can be found in
\cite{Himonas-Misiolek-2001-A-priori-estimates-for-Schrodinger} and
\cite{Himonas_Misiolek-A-priori-estima}.
Applying \cref{ncor:four-mult-est-L4} and generalized H\"{o}lder to the 
right-hand-side of \eqref{ngen-holder-pre-estimate} gives
%
%
\begin{equation}
	\label{ngen-holder-piece-1}
	\begin{split}
		\|\left( 1 + | \lambda - n^{2} | \right)^{-b} \wh{C_f C_{ 
		g } C^+_{h}}\|_{L^2(\zz \times \rr)}
		& \lesssim  \|C_f C_{g} C^+_{h} \|_{L^{4/3}(\ci \times \rr)}
		\\
		& \le \|C_f \|_{L^4(\ci \times \rr)} \|C_{g}\|_{L^4(\ci \times \rr)} 
		\|C^+_{h}\|_{L^4(\ci \times \rr)}.
	\end{split}
\end{equation}
%
%
Note that a change of variable gives
%
%
\begin{equation*}
	\begin{split}
		C_\sigma^+(x, t)
		& = \sum_{n \in \zz} \int_\rr e^{i(nx +  \lambda t)} \frac{c_\sigma\left( -n, -\lambda \right)}{\left( 
		1 + | \lambda + n^{2} | \right)^{b}} \ d \lambda
		\\
		& = - \sum_{n \in \zz} \int_\rr e^{-i(nx +   \lambda t )}
		\frac{c_\sigma\left( n, \lambda \right)}{\left( 
		1 + | \lambda - n^{2} | \right)^{b}} \ d \lambda
	\end{split}
\end{equation*}
%
%
and so
%
%
\begin{equation*}
	\begin{split}
		C_\sigma^+(-x, -t) = -C_\sigma(x, t).
	\end{split}
\end{equation*}
%
%
We will now need the following dual estimate of
\cref{nlem:four-mult-est-L4}.
%
\begin{corollary}
	\label{ncor:four-mult-est-L4}
	Let $(x, t) \in \ci \times \rr $ and $(n, \lambda) \in \zz \times \rr$ be 
	the dual variables. Then there is a 
	constant $c > 0$ such that
%
%
\begin{equation}
	\label{nfour-mult-est-L4}
	\begin{split}
    \|f\|_{L^4(\ci \times \rr)} \le c \|\left( 1 + | \lambda - n^{2} | 
		\right)^\frac{3}{8} \wh{f} \|_{L^2( \zz \times \rr)}
	\end{split}
\end{equation}
for every test function $f(x, t)$. 
%
%
%
%
\end{corollary}
%
%
Recalling that $L^4(\ci \times \rr)$ is invariant under the transformation $(x, 
t) \mapsto (-x,-t)$ and applying 
\cref{ncor:four-mult-est-L4}, we obtain
%
%
\begin{equation}
	\label{nC-sig-estimate}
	\begin{split}
		\| C^+_\sigma \|_{L^4(\ci \times \rr)} = \|C_\sigma \|_{L^4(\ci \times \rr)} 
		& \lesssim \|\left( 1 + | \lambda - n^{2} | 
		\right)^{3/8} \left( 1 + | \lambda - n^{2} | 
		\right)^{-b} c_\sigma \|_{L^2(\zz \times \rr)}
		\\
		& \le \|c_\sigma \|_{L^2(\zz \times \rr)}, \qquad b \ge 3/8 
    \\
		& = \|\sigma\|_{X^s}.
	\end{split}
\end{equation}
%
%
We conclude from \eqref{ngen-holder-pre-estimate}, \eqref{ngen-holder-piece-1}, 
and \eqref{nC-sig-estimate} that 
%
%
%
%
\begin{equation*}
	\begin{split}
		\| \left( 1 + |n | \right)^s \left( 1 + | \lambda - n^{2} | \right)^{-b} \wh{w_{fgh}} 
		(n, \lambda) \|_{L^2(\zz \times \rr)} \lesssim 
		\|f\|_{X^s}\|g\|_{X^s}\|h\|_{X^s}, \qquad b \ge 3/8.
	\end{split}
\end{equation*}
%
%
%
%
%
%
%
\section{Proof of Corollary to Trilinear Estimate}
By duality, it suffices to show that 
%
%%
\begin{equation*}
	\begin{split}
		| \sum_{n \in \zz} \left (1 + |n| \right )^{s}
		a_n \int_{\rr} \frac{|\wh{w_{fgh}}(n, \lambda)|}{1 
		+ | \lambda - n^{2} |} \ d \lambda | \lesssim \|f\|_{X^s} \|g\|_{X^s} \|h\|_{X^s}
		\|a_n \|_{\ell^2}
	\end{split}
\end{equation*}
%
%%
for $\{a_n\} \in \ell^2$. By the triangle inequality 
and Cauchy-Schwartz,
%
%%
\begin{equation}
	\label{n1m}
	\begin{split}
		& | \sum_{n \in \zz} \left (1 + |n| \right )^{s} a_n
		\int_{\rr}\frac{| \wh{w_{fgh}}(n, \lambda) |}{1 + | \lambda - n^{2} |} \ d \lambda |
		\\
		& \le \sum_{n \in \zz} \int_{\rr} \frac{| a_n |}{\left( 1 + 
		| \lambda - n^{2} |
		\right)^{1/2 + \eta}} \cdot \frac{\left( 1 + | n| \right)^s  |
		\wh{w_{fgh}}(n, \lambda) |}{\left( 
		1 + | \lambda - n^{2} | \right)^{1/2 - \eta}} \ d \lambda
		\\
		& \le \left( \sum_{n \in \zz} | a_{n} |^2\int_{\rr} \frac{1}{\left( 1 + | \lambda - n^{2} | \right)^{1 + 2 \eta}} \ d \lambda  
		\right)^{1/2} 
		\left ( \sum_{n \in \zz}\int_{\rr} \frac{\left (1 + |n| \right )^{2s} | \wh{w_{fgh}}(n, \lambda) 
		|^2}{\left( 1 + | \lambda - n^{2} | \right)^{1 - 2 \eta}}\ d \lambda 
		\right)^{1/2}.
	\end{split}
\end{equation}
%
%%
Restrict $\eta \in (0, 1/8)$. Applying the change of variable $\lambda - n^{2}
= \lambda'$ we obtain  %
%%

\begin{equation*}
	\begin{split}
		& \left( \sum_{n \in \zz} | a_{n} |^2\int_{\rr} \frac{1}{\left( 1 + | \lambda -
		n^{2} | \right)^{1 + 2 \eta}} \ d \lambda  
		\right)^{1/2} 
		\\
		& = \left ( \sum_{n \in \zz}
		| a_n |^2 
		\int_{\rr} \frac{1}{\left( 1 + | \lambda' | \right)^{1 + 2 \eta}} \ d 
		\lambda \right)^{1/2}
		\\
		& \simeq \|a_n\|_{\ell^2}
		\end{split}
\end{equation*}
while \cref{nprop:trilinear-est} gives the bound
\begin{equation*}
	\begin{split}
		\left ( \sum_{n \in \zz}\int_{\rr} \frac{\left (1 + |n| \right )^{2s} | \wh{w_{fgh}}(n, \lambda) 
		|^2}{\left( 1 + | \lambda - n^{2} | \right)^{1 - 2 \eta}}\ d \lambda 
		\right)^{1/2} \lesssim \|f\|_{X^s} \|g\|_{X^s} \|h\|_{X^s}
	\end{split}
\end{equation*}
%
%%
completing the proof.
\qquad \qedsymbol
%
%
%
%
%
%
%
\appendix
\section{}
%
%
\begin{proof}[Proof of \cref{nlem:cutoff-loc-soln}]
%
%
\begin{equation}
  \label{ndm}
	\begin{split}
		\lim_{t_{k} \to t} \|u(\cdot, t) - u(\cdot, t_{k})\|_{H^s(\ci)} 
    & = \lim_{t_{k} \to t} \|\psi(t) u(\cdot, t) - \psi(t_{k}) u(\cdot, t_{k})\|_{H^s(\ci)} 
		\\
		& = \lim_{t_{k} \to t} \left[ \sum_{n}\left( 1 + | n |
    \right)^{2s} | \psi(t)  \wh{u}(n, t) - \psi(t_{k}) \wh{ u}(n, t_{k}) |^2 \right]^{1/2}
		\\
		& = \lim_{t_{k} \to t} \left[ \sum_{n} \left( 1 + | n |
    \right)^{2s} | \int_{\rr} (e^{it \lambda} - e^{it_{k} \lambda})
    \wh{\psi u}(n,
		\lambda) d \lambda |^2 \right]^{1/2}.
	\end{split}
\end{equation}
First note that
%
%
%
%
\begin{equation*}
\begin{split}
& \lim_{t_{k} \to t}  | \int_{\rr} (e^{it \lambda} - e^{it_{k} \lambda})
    \wh{\psi u}(n,
		\lambda) d \lambda |^2 
    \\
    = 
     & \lim_{t_{k} \to t}  \int_{\rr} (e^{it \lambda} - e^{it_{k} \lambda})
    \wh{\psi u}(n,
    \lambda) d \lambda \times \lim_{t_{k} \to t} \overline{\int_{\rr} (e^{it \lambda} - e^{it_{k} \lambda})
    \wh{\psi u}(n,
    \lambda) d \lambda }  
    \\
    = 
    &  \lim_{t_{k} \to t}  \int_{\rr} (e^{it \lambda} - e^{it_{k} \lambda})
    \wh{\psi u}(n,
    \lambda) d \lambda \times \lim_{t_{k} \to t} \int_{\rr} (e^{-it \lambda} - e^{-it_{k} \lambda})
    \overline{\wh{\psi u}}(n,
    \lambda) d \lambda.   
    \end{split}
\end{equation*}
%
%
But for fixed $n$ 
%
%
\begin{equation*}
\begin{split}
|(e^{it \lambda} - e^{it_{k} \lambda})  
\wh{\psi u}(n, \lambda) | \le 2 |\wh{\psi u}(n, \lambda) |
\end{split}
\end{equation*}
%
%
and
%
%
%
\begin{equation*}
\begin{split}
  \int_{\rr} |2 \wh{\psi u}(n, \lambda) | d \lambda < \infty.
\end{split}
\end{equation*}
%
%
Hence, by dominated convergence
%
%
\begin{equation*}
\begin{split}
\lim_{t_{k} \to t}  \int_{\rr} (e^{it \lambda} - e^{it_{k} \lambda})
    \wh{\psi u}(n,
    \lambda) d \lambda =  \int_{\rr} \lim_{t_{k} \to t} (e^{it \lambda} - e^{it_{k} \lambda})
    \wh{\psi u}(n,
    \lambda) d \lambda = 0. 
\end{split}
\end{equation*}
%
%
Similarly, 
%
%
%
\begin{equation*}
\begin{split}
\lim_{t_{k} \to t} \int_{\rr} (e^{-it \lambda} - e^{-it_{k} \lambda})
    \overline{\wh{\psi u}}(n,
    \lambda) d \lambda  =\int_{\rr}  \lim_{t_{k} \to t} (e^{-it \lambda} - e^{-it_{k} \lambda})
    \overline{\wh{\psi u}}(n,
    \lambda) d \lambda  = 0.
\end{split}
\end{equation*}
%
%
Hence
%
%
%
\begin{equation}
  \label{ngh}
\begin{split}
  \lim_{t_{k} \to t} | \int_{\rr} (e^{it \lambda} - e^{it_{k} \lambda})
    \wh{\psi u}(n,
		\lambda) d \lambda |^2 = 0.
\end{split}
\end{equation}
%
%
		Furthermore,
    %
    %
    \begin{equation*}
    \begin{split}
      (1 + | n |)^{2s} | \int_{\rr} \left( e^{it\lambda} - e^{it_{k} \lambda} \right)
      \wh{\psi u}(n, \lambda) d \lambda|^{2} \le 4 (1 + | n |)^{2s} \left(
      \int_{\rr} | \wh{\psi u}(n, \lambda)  | d \lambda
      \right)^{2}
    \end{split}
    \end{equation*}
    %
    %
    and
		%
		%
		\begin{equation*}
			\begin{split}
         \sum_{n}  \left( 1 + | n |
        \right)^{2s} \left ( \int_{\rr} |\wh{\psi u}(n, \lambda)| d \lambda
        \right )^2  
        & = \|\wh{\psi u}\|_{\ell^{2}_{n}L^{1}_{\lambda}}^2
		\\
		& \le \|\psi u \|_{Y^{s}}^2 
	\end{split}
\end{equation*}
which is bounded by assumption. Therefore, applying dominated convergence and
\eqref{ngh}, we
obtain 
%
%
\begin{equation*}
\begin{split}
  \text{rhs of \eqref{ndm}} = \left[ \sum_{n} \left( 1 + | n |
    \right)^{2s} \lim_{t_{k} \to t} | \int_{\rr} (e^{it \lambda} - e^{it_{k} \lambda})
    \wh{\psi u}(n,
		\lambda) d \lambda |^2 \right]^{1/2} = 0
\end{split}
\end{equation*}
%
%
completing the proof. 
\end{proof}
%
%
\begin{proof}[Proof of \cref{nlem:schwartz-mult}]
Note that
%
%
\begin{equation*}
	\begin{split}
		\wh{\psi f}\left( n, \lambda \right)
		& = \wh{\psi}(\cdot) * \wh{f}(n,
		\cdot)(\lambda)
		= \int_\rr \wh{\psi}(\lambda_1) \wh{f} \left( n, \lambda - \lambda_1 \right) 
		d\lambda_1
	\end{split}
\end{equation*}
%
%
and hence
%
%
\begin{equation}
	\label{n1b}
	\begin{split}
		\|\psi f\|_{X^s} 
		& = \left( \sum_{n \in \zz} \left (1 + |n| \right )^{2s} \int_\rr \left( 1 + | \lambda -
		n^{2} | \right) | \int_\rr \wh{\psi}(\lambda_1) \wh{f}\left( n, \lambda -
		\lambda_1
		\right)  d \lambda_1 d \lambda |^2 \right)^{1/2}
		\\
		& \le \left( \sum_{n \in \zz} \left (1 + |n| \right )^{2s} \int_\rr \left( 1 + | \lambda -
		n^{2}
		|
		\right) \left( \int_\rr |\wh{\psi}\left( \lambda_1 \right) | |\wh{f}\left( n,
		\lambda - \lambda_1
		\right) |  d \lambda_1 d \lambda \right)^2 \right)^{1/2}.
	\end{split}
\end{equation}
%
%
Using the relation
%
%
\begin{equation*}
	\begin{split}
		1 + | \lambda - n^{2} |
    & = 1 + | \lambda - \lambda_1 + \lambda_{1} - n^{2} |
		\\
		& \le 1 + | \lambda_1 | + | \lambda - \lambda_1 - n^{2} |
		\\
		& \le \left( 1 + | \lambda_1 | \right)\left( 1 + | \lambda - \lambda_1 -
		n^{2} | \right)
	\end{split}
\end{equation*}
%
%
we obtain
%
%
\begin{equation*}
	\begin{split}
		\eqref{n1b}
		& \le \left( \sum_{n \in \zz} \left (1 + |n| \right )^{2s} \right.
		\\
		& \times \left . \int_\rr \left(
		\int_\rr \left( 1 + | \lambda_1 | \right)^{1/2} | \wh{\psi}(\lambda_1) |
		\left( 1 + | \lambda - \lambda_1 - n^{2} | \right)^{1/2} \wh{f}\left( n, \lambda
		- \lambda_1
		\right)d \lambda_1
		\right)^2 d \lambda \right)^{1/2}
	\end{split}
\end{equation*}
%
%
which by Minkowski's inequality is bounded by
%
%
\begin{equation}
	\label{n2b}
	\begin{split}
		& \left( \sum_{n \in \zz} \left (1 + |n| \right )^{2s}  \right.
		\\
		& \times \left. \left( \int_\rr \left[ \int_\rr
		\left( 1 + | \lambda_{1} | \right) | \wh{\psi}(\lambda_1) |^2 \left( 1 + |
		\lambda - \lambda_1 - n^{2} |
		\right) | \wh{f}\left( n, \lambda - \lambda_1 \right) |^2 d \lambda 
    \right]^{1/2} d \lambda_{1} \right)^2 \right)^{1/2}.
	\end{split}
\end{equation}
%
%
Using the change of variable $\lambda - \lambda_1 = \tau$ gives
%
%
\begin{equation*}
	\begin{split}
		\eqref{n2b}
		& = \left( \sum_{n \in \zz} \left (1 + |n| \right )^{2s}\right.
		\\
		& \times \left.  \left( \int_\rr \left[
		\int_\rr \left( 1 + | \lambda_1 | \right) | \wh{\psi}\left( \lambda_1
		\right) |^2 \left( 1 + | \tau - n^{2} | \right) | \wh{f} \left( n,
		\tau
    \right)|^2 d \tau \right]^{1/2} d \lambda_{1} \right)^2 \right)^{1/2}
		\\
		& =  \left( \sum_{n \in \zz} \left (1 + |n| \right )^{2s} \right.
		\\
		& \times \left. \left( \int_\rr \left( 1 + |
		\lambda_1 |
		\right)^{1/2} | \wh{\psi}(\lambda_1) | d \lambda_1 \left[ \int_\rr \left( 1 + |
		\tau - n^{2} |
		\right) | \wh{f}\left( n, \tau \right) |^2 d \tau \right]^{1/2}
		\right)^2 \right)^{1/2}
		\\
		& = c_{\psi} \left( \sum_{n \in \zz} \left (1 + |n| \right )^{2s} \left( \left[ \int_\rr
		\left( 1 + | \tau - n^{2} | \right) | \wh{f}\left( n, \tau
		\right) |^2 d \tau
		\right]^{\cancel{1/2}} \right)^{\cancel{2}} \right)^{1/2}
		\\
		& = c_{\psi} \|f\|_{X^s}.
	\end{split}
\end{equation*}
%
Also, by Young's inequality we have the estimate 
%
%
\begin{equation*}
\begin{split}
  \|\wh{\psi f}\|_{\ell^{2}_{n} L^{1}_{\lambda}} 
  & = \left[ \sum_{n \in \zz} \left (1 + |n| \right )^{2s} \left (
  \int_{\rr} | \wh{\psi}(\cdot) * \wh{f}(n, \cdot)(\lambda) | d \lambda  \right ) ^2 \right]^{1/2}
  \\
  & \le  \left[ \sum_{n \in \zz} (1 + | n |)^{2s} \left( \int_{\rr} |
    \wh{\psi}(\lambda) | d \lambda  \times \int_{\rr} | \wh{f}(n, \lambda) | d \lambda
    \right)^{2}\right]^{1/2}
  \\
  & = c_{\psi} \| \wh{f} \|_{\ell^{2}_{n} L^{1}_{\lambda}}
\end{split}
\end{equation*}
%
%
%
concluding the proof. 
\end{proof}
%
%
\begin{proof}[Proof of \cref{nlem:splitting}] We have
%
%
\begin{equation}
	\label{n6a}
	\begin{split}
		1 + | a + b + c| 
		& \le 1 + | a | + | b | + | c |
		\\
		& \le 1 + | a | + 1 + | b | + 1 + | c |
		\\
		& \le 3\left( \max\{1+| a |, 1+| b |, 1+ | c | \}\right)
		\\
		& \le 3 \left( 1 + | a | \right)\left( 1 + | b | \right) \left( 1 + |
		c |
		\right).
	\end{split}
\end{equation}
%
%
Raising both sides of expression $\eqref{n6a}$ to the power $\nu \ge 0$ completes 
the proof. 
\end{proof}
%
%
%
%\bibliography{/Users/davidkarapetyan/math/bib-files/references}
% \bib, bibdiv, biblist are defined by the amsrefs package.
\begin{bibdiv}
\begin{biblist}

\bib{Bourgain-Fourier-transfo-1}{article}{
      author={Bourgain, J},
       title={Fourier transform restriction phenomena for certain lattice
  subsets and applications to nonlinear evolution equations. i. schr{\"o}dinger
  equations},
        date={1993},
     journal={Geom. Funct. Anal.},
      volume={3},
      number={2},
       pages={107\ndash 156},
  url={http://www.ams.org.proxy.library.nd.edu/mathscinet-getitem?mr=1209299},
}

\bib{Bourgain-Fourier-transfo}{article}{
      author={Bourgain, J},
       title={Fourier transform restriction phenomena for certain lattice
  subsets and applications to nonlinear evolution equations. ii. the
  kdv-equation},
        date={1993},
     journal={Geom. Funct. Anal.},
      volume={3},
      number={3},
       pages={209\ndash 262},
  url={http://www.ams.org.proxy.library.nd.edu/mathscinet-getitem?mr=1215780},
}

\bib{Himonas-Misiolek-2001-A-priori-estimates-for-Schrodinger}{article}{
      author={Himonas, A~Alexandrou},
      author={Misiolek, Gerard},
       title={A priori estimates for schr{\"o}dinger type multipliers},
        date={2001},
     journal={Illinois J. Math.},
      volume={45},
      number={2},
       pages={631\ndash 640},
  url={http://www.ams.org.proxy.library.nd.edu/mathscinet-getitem?mr=1878623},
}

\bib{Himonas_Misiolek-A-priori-estima}{article}{
      author={Himonas, A~Alexandrou},
      author={Misiolek, Gerard},
       title={A priori estimates for schr{\"o}dinger type multipliers},
        date={2001},
     journal={Illinois J. Math.},
      volume={45},
      number={2},
       pages={631\ndash 640},
  url={http://www.ams.org.proxy.library.nd.edu/mathscinet-getitem?mr=1878623},
}

\end{biblist}
\end{bibdiv}
\end{document}
