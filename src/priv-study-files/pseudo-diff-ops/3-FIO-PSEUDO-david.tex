\documentclass[12pt,reqno]{amsart}
\usepackage{amscd}
\usepackage{amsfonts}
\usepackage{amsmath}
\usepackage{amssymb}
\usepackage{amsthm}
\usepackage{appendix}
\usepackage{fancyhdr}
\usepackage{latexsym}
\usepackage{pdfsync}
\usepackage{cancel}
\usepackage{amsxtra}
\usepackage[colorlinks=true, pdfstartview=fitv, linkcolor=blue,
citecolor=blue, urlcolor=blue]{hyperref}
\input epsf
\input texdraw
\input txdtools.tex
\input xy
\xyoption{all}
%%%%%%%%%%%%%%%%%%%%%%
\usepackage{color}
\definecolor{red}{rgb}{1.00, 0.00, 0.00}
\definecolor{darkgreen}{rgb}{0.00, 1.00, 0.00}
\definecolor{blue}{rgb}{0.00, 0.00, 1.00}
\definecolor{cyan}{rgb}{0.00, 1.00, 1.00}
\definecolor{magenta}{rgb}{1.00, 0.00, 1.00}
\definecolor{deepskyblue}{rgb}{0.00, 0.75, 1.00}
\definecolor{darkgreen}{rgb}{0.00, 0.39, 0.00}
\definecolor{springgreen}{rgb}{0.00, 1.00, 0.50}
\definecolor{darkorange}{rgb}{1.00, 0.55, 0.00}
\definecolor{orangered}{rgb}{1.00, 0.27, 0.00}
\definecolor{deeppink}{rgb}{1.00, 0.08, 0.57}
\definecolor{darkviolet}{rgb}{0.58, 0.00, 0.82}
\definecolor{saddlebrown}{rgb}{0.54, 0.27, 0.07}
\definecolor{black}{rgb}{0.00, 0.00, 0.00}
\definecolor{dark-magenta}{rgb}{.5,0,.5}
\definecolor{myblack}{rgb}{0,0,0}
\definecolor{darkgray}{gray}{0.5}
\definecolor{lightgray}{gray}{0.75}
%%%%%%%%%%%%%%%%%%%%%%
%%%%%%%%%%%%%%%%%%%%%%%%%%%%
%  for importing pictures  %
%%%%%%%%%%%%%%%%%%%%%%%%%%%%
\usepackage[pdftex]{graphicx}
\usepackage{epstopdf}
% \usepackage{graphicx}
%% page setup %%
\setlength{\textheight}{20.8truecm}
\setlength{\textwidth}{14.8truecm}
\marginparwidth  0truecm
\oddsidemargin   01truecm
\evensidemargin  01truecm
\marginparsep    0truecm
\renewcommand{\baselinestretch}{1.1}
%% new commands %%
\newcommand{\bigno}{\bigskip\noindent}
\newcommand{\ds}{\displaystyle}
\newcommand{\medno}{\medskip\noindent}
\newcommand{\smallno}{\smallskip\noindent}
\newcommand{\ts}{\textstyle}
\newcommand{\rr}{\mathbb{R}}
\newcommand{\p}{\partial}
\newcommand{\zz}{\mathbb{Z}}
\newcommand{\cc}{\mathbb{C}}
\newcommand{\ci}{\mathbb{T}}
\newcommand{\ee}{\varepsilon}
\def\refer #1\par{\noindent\hangindent=\parindent\hangafter=1 #1\par}
%% equation numbers %%
\renewcommand{\theequation}{\thesection.\arabic{equation*}}
%% new environments %%
%\swapnumbers
\theoremstyle{plain}  % default
\newtheorem{theorem}{Theorem}
\newtheorem{proposition}{Proposition}
\newtheorem{lemma}{Lemma}
\newtheorem{corollary}{Corollary}
\newtheorem{conjecture}[subsection]{conjecture}
\theoremstyle{definition}
\newtheorem{definition}{Definition}
\newcommand{\ve}{\varepsilon} 
\newcommand{\nin}{\noindent}
\newcommand{\oL}{\bar L}
\newcommand{\vph}{\varphi}
\begin{document}
\title{Pseudodifferential and Fourier Integral Operators}
\author{Alex Himonas, {\it Summer 2009}}
\maketitle
\setcounter{section}{2}
\section{Fourier Integral and Pseudodifferential Operators}
\nin
Let $X, Y$ be open subsets of $\Bbb R^n$ and $\varphi (x,y, \xi)$ be a phase on $X
\times Y \times \Bbb R^N, $
i.e.
$$d_{x, y, \xi}\  \varphi(x,y, \xi) \ne 0 \; \; \text{on} \; \; X \times Y \times
\dot{\Bbb R^N}.$$ 
If $ p\in S^m(X \times Y \times \Bbb R^N)$ then the operator
$$P : C^\infty_0 (Y) \longrightarrow \mathcal D'(X)$$
defined by $u \in C^\infty_0 (Y) \longmapsto Pu$ with
$$<Pu,v> \overset{osc.}{=} \iiint e^{i(x,y,\xi)} p(x,y,\xi)u(y)
v(x)dydxd \xi$$ 
is called a {\bf Fourier integral operator (FIO)} with phase $\vph$ and symbol $p$.
\medskip
\nin
Note that:
$$|<Pu,v> |\le c (\sup_{\substack{|\alpha| \le k \\ x \in K}} |D^\alpha u|)
(\sup_{\substack{|\alpha| \le k\\ y\in K}} |D^\alpha v|)$$
i.e. $P$ is continuous.
\medskip
\nin
If $u \in C^\infty_0 (Y)$ then $Pu \in \mathcal D'(X)$ is defined by
$$Pu(x) = \int_{R^N} \int_Y e^{i\vph(x,y,\xi)} p(x,y,\xi) u(y)dyd\xi.$$
The {\bf kernel} of $P$ is given by the distribution $K_P(x,y) \in \mathcal D'(X \times
Y)$ defined by 
$$K_P(x,y)= \int_{\Bbb R^N}e^{i\vph(x,y,\xi)}p(x,y,\xi)d\xi.$$
The {\bf transpose}  ${}^tP$ of $P$ is defined by
$$\langle Pu(x),v(x)\rangle = \langle u(y),{}^tP v(y)\rangle,\ u\in
C^\infty_0(Y),\ v\in C^\infty_0(X).$$
We have that
$${}^tP:C^\infty_0 \xrightarrow{cont.} \mathcal D'(Y)$$
with 
$${}^tPv(y) = \int_{\Bbb R^N} \int_Xe^{i\vph(x,y,\xi)}p(x,y,\xi)v(x)dxd\xi.$$
Note that
$${}^t({}^tP)=P.$$
\begin{theorem}
The following hold:
\vskip0.1in
\nin {\bf 1)} $\text{ sing supp } K_P \subset\{(x,y) \in X \times Y:
\exists \xi \in \dot{\Bbb R}^N \; \text{with} \;\; d_\xi \vph(x,y,\xi) = 0\}$
\vskip0.1in
\nin
{\bf 2)} If $d_{y,\xi} \ \vph(x,y,\xi) \ne 0$ on $X\times Y \times
\dot{\Bbb R}^N$ then $$P:C^\infty_0 (Y) \xrightarrow{cont.} C^\infty(X)$$
with
$$(Pu)(x) \overset{osc.}{=} \int_{\Bbb R^N} \int_{\Bbb R^n} e^{i\vph
(x,y,\xi)} p(x,y,\xi)u(y)dyd\xi,\  u \in C^\infty_0(Y).$$    
\smallskip
\nin
{\bf 3)}  If $d_{x,\xi}\ \vph(x,y,\xi)\ne 0$ on $X \times Y \times
\dot{\Bbb R}^N$ then
$${}^tP:C^\infty_0(X) \xrightarrow{cont.} C^\infty(Y)$$
and $P$ extends as a continuous operator from $\mathcal E'(Y)$ into $\mathcal D'(X)$; i.e.
$$P:  \mathcal E' (Y) \xrightarrow{cont.}  \mathcal D'(X).$$
\end{theorem}
\nin
{\bf Proof.} It follows from the properties of the oscillatory
integrals depending on parameters.$ \qquad \Box$
\vskip0.2in
\begin{definition}  If $\vph(x,y,\xi)=(x-y) \cdot \xi$ on $X\times X \times
\Bbb R^n$ and $p(x,y,\xi) \in S^m(X \times X \times \Bbb R^n)$ then  a Fourier
integral operator with phase $\vph$ and symbol $p$ is called a
{\bf  pseudodifferential} operator $(\psi do)$ on $X, $ i.e. 
$$P: C^\infty_0(X) \xrightarrow{ cont.}  C^\infty(X)$$
with
$$(Pu) (x) \overset{osc.}{=} \frac{1}{(2\pi)^n}\int_{\Bbb R^n} \int_{\Bbb R^n}
e^{i(x-y)\xi} p(x,y,\xi) u(y) dyd\xi,\ u\in
C^\infty_0(X).$$
\end{definition}
\nin
This is because $d_{y,\xi}\ \vph(x,y,\xi) = 
- (\xi,y-x)\ne 0$ on $X \times X \times \dot{\Bbb R}^n$.  
Since we also have $d_{x,\xi}\ \vph =- (\xi,x-y) \ne 0$ on $X \times X
\times \dot{\Bbb R}^n,\ P$ extends continuously from $\mathcal E'(X)$ into $\mathcal D'(X)$.
The kernel of $P$ is 
$$K_P(x,y) =  \frac{1}{(2\pi)^n}\int_{\Bbb R^n} e^{i(x-y)\xi} p(x,y,\xi)d\xi$$
with
$$\text{sing supp } K_P \subset \text{diag } (X \times X).$$
The transpose of $P$ is given by
$$({}^tPu)(x)\overset{osc.}{=} \iint\limits_{\Bbb R^n \Bbb R^n} e^{i(x-y)\xi}
p(y,x, -\xi)u(y) dyd\xi.$$
\medskip
\nin
{\bf Examples:}
\vskip0.1in
\nin
{\bf 1)} \quad   $\vph = (x-y)\xi, \ p=1$.  Then
 $$(Iu)(x) = \frac{1}{(2\pi)^n} \underset{\Bbb R^n \Bbb R^n}{\iint} e^{i(x-y)\xi}
u(y)dyd\xi$$
is the identity operator.
\vskip0.1in
\nin
{\bf 2)} \quad $\vph = (x-y)\xi, \ p(x,\xi) = \sum\limits_{|\alpha| \le m}
a_\alpha (x) \xi^\alpha,\  a_\alpha \in C^\infty(X).$
Then $P$ is a linear partial differential operator with symbol $p(x,\xi)$.
\vskip0.1in
\nin
{\bf 3)} \quad Let  $X$ be an open set in $\Bbb R^q$ and $Y$ be an open set in
$\Bbb R^n $, and $f: X \longrightarrow Y$ be a $C^\infty$ function. 
If $u \in C^\infty_0 (Y)$ then  its  pullback by $f$ is defined by
\begin{align*}
f^*(u)(x) &= u(f(x))\\
&= \frac{1}{(2\pi)^n} \underset{\Bbb R^n \Bbb R^n}{{\iint}} e^{i(f(x) -y) \cdot
\xi} u(y)dyd\xi
\end{align*}
and it is a Fourier integral operator with symbol $p=1$ and phase
$$\vph(x,y,\xi) = (f(x) - y)  \xi.$$
Since $d_{y,\xi} \ \vph =-(\xi, y)\ne 0$ on $X \times Y \times \dot{\Bbb
R}^n$
we have that $f^*: C^\infty_0 (Y)\longrightarrow  C^\infty_0(X)$.
Since  
$$d_{x,\xi} \ \vph = (f^t_x(x)\xi,f(x)-y)\ne 0\text{ on }X \times Y \times
\dot{\Bbb R}^n $$
if ${}^tf_x$ is one-to-one, or equivalently if for each fixed $x$ 
$$f'_x(x): \Bbb R^q
\xrightarrow{onto} \Bbb R^n,$$ we have that $f^*$ extends continuously from
$$\mathcal E'(Y) \longrightarrow \mathcal D'(X),$$
since $f^*={}^t({}^tf^*)$.
\vskip0.1in
\nin
{\bf 4.)}\quad{\bf  Trace:}\quad  Let $q<n$ and $Y$ open
$\subset \Bbb R^n, \ X = Y \cap \Bbb R^q \ne \emptyset$ and
$f:X \longrightarrow Y$
with $x \overset{f}{\longmapsto} (x,0) \in \Bbb R^n$.
Then $f^*$ is the trace operator with phase
$$\vph(x,y,\xi) = ((x,0) - y)  \xi.$$
{\bf 5)} \quad   Let the following initial problem
for the wave operator in $\Bbb R^{n+1}$:
\begin{equation*}
	\begin{split}
	\begin{cases} (\partial^2_t-\Delta_x)u(x,t) = 0\\ 
u(0,x) = f(x),\ f \in \mathcal
S(\Bbb R^n)\\ 
\frac{\partial u}{\partial t} (0,x) = 0.
\end{cases} 
\end{split}
\end{equation*}
By taking the Fourier transform in $x$ we obtain 
\begin{equation*}
	\begin{split}
		\begin{cases} \partial^2_t \hat u + |\xi|^2
			\hat u = 0\\ \hat u (0,\xi) = \hat f(\xi).
		\end{cases}
		\end{split}
	\end{equation*}
Thus
$$\hat u(t,\xi) = A(\xi) \cos (t|\xi|) + B(\xi) \sin (t|\xi|).$$
Since $A(\xi) = \hat f(\xi)$,  and $B=0$
$$\hat u(t,\xi) = \hat f(\xi) \frac{e^{it|\xi|} + e^{-it|\xi|}}{2}.$$
Therefore
\begin{align*}
u(t,\xi) &= \frac{1}{(2\pi)^n} \int_{\Bbb R^n} e^{ix\xi}\hat f(\xi) 
\frac{e^{it|\xi|} + e^{-it|\xi|}}{2} d\xi\\
&= \frac{1}{(2\pi)^n} \iint e^{i(x-y)\xi+it|\xi|} \frac 12 f(y)dyd\xi +
\frac{1}{(2\pi)^n} \iint e^{i(x-y)\xi-it|\xi|} \frac 12
f(y)dyd\xi.\end{align*}
\nin
Thus for fixed $t$ we obtain two Fourier integral operators with phases
$$\vph_+(x,y,\xi) = (x-y) \xi + t|\xi|,\quad \text { and }
 \quad \vph_-(x,y,\xi) = (x-y) \xi - t|\xi|.$$
\bigskip
\begin{proposition} Pseudodifferential operators are pseudolocal; i.e
$$\text{sing supp } Pu \subset \text{sing supp } u,\quad u \in \mathcal
E'(X).$$
\end{proposition}
\nin
{\bf Proof.}
Let $x_0 \notin $ sing supp $ u.$
Since sing supp $ u $ is compact there exist open sets $U$ and $V$ such that 
$x_0 \in U$, sing supp $u \subset V$, and
$$(U \times V) \cap \text{ sing supp } K_P = \emptyset$$
Let $\varphi \in C_0^\infty (V)$ with $\varphi \equiv 1$ on sing supp $ u$.
Then
\begin{align*}
Pu &=P(\varphi u + (1 - \varphi)u) \\
&=P(\varphi u) + P ((1- \varphi)u).
\end{align*}
Since $(1 - \varphi) u \in C_0^\infty (X)$ we have $P((1 - \varphi)u) \in C^\infty$.
Since
$$K_P|_{U\times V} \in C^\infty$$
we have $P(\varphi u)|_U \in C^\infty (U)$.
Thus $x_0 \notin $ sing supp $Pu$, and this completes the proof of the
Proposition.
\vskip0.1in
\centerline{\bf Notation}
\vskip0.1in
\nin
The following notation is used in the literature:
\begin{align*}
\Psi^m(X) \text{ or }L^m(X) &= \text{Pseudodifferential operators with 
symbol in}\; S^m (X
\times X \times \Bbb R^n) \\ 
\Psi^\infty(X) \text{ or }L^\infty(X) &= \cup_m \Psi^m \\
\Psi^{-\infty}(X) \text{ or }L^{-\infty}(X) &= \cap_m \Psi^m
\end{align*} 
Also 
\begin{align*}
R(X) &= \text{ operators with
kernel } K(x,y) \in C^\infty (X\times X)\\ &\Updownarrow \\
&=\text{ Linear continuous maps from }  \mathcal E'(X) \longrightarrow C^\infty (X)
\end{align*} 
\begin{lemma}  $ R(X) = \Psi^{- \infty} (X)$.
\end{lemma}
\nin
{\bf Proof.}  If $P\in \Psi^{-\infty} (X)$ then for any $k \in \Bbb N$ its
kernel is
$$K(x,y) = \frac{1}{(2 \pi)^n} \int e^{i(x-y)\xi} p(x,y, \xi)d\xi$$
with $p \in S^{-n-1-k}$. Therefore $K \in C^k (X \times X)$ for any $k$, which
implies that $K \in C^\infty (X \times X)$. Conversely let
 $K \in C^\infty (X \times X)$. Then choose
$\varphi(\xi)\in C_0^\infty (\Bbb R^n)$ with
$$\frac{1}{(2\pi)^n} \int_{\Bbb R^n}  \varphi(\xi)d\xi=1.$$
Then for $u \in C_0^\infty (\Bbb R^n)$ we
have 
\begin{align*}
Pu(x) &=\int K(x,y)u(y)dy = \frac{1}{(2\pi)^n} \iint \varphi(\xi) K (x,y)u(y)dy d\xi
\\ 
&=\frac{1}{(2\pi)^n} \iint e^{i(x-y)\xi} [e^{-i(x-y) \cdot \xi}\  \varphi(\xi)
K(x,y)] u(y) dy d\xi
\end{align*}
and
$$e^{-i(x-y)\cdot \xi}\  \varphi(\xi)k(x,y) \in S^{-\infty}(X \times X \times \Bbb
R^n).$$
Therefore  $P\in\Psi^{-\infty}(X)$, and this completes the proof of the
Lemma. $\qquad \Box$
\vskip0.2in
\centerline{\bf Properly Supported Pseudodifferential Operators}
\begin{definition} Let $P\in \Psi^m (X)$ with distribution kernel $K(x,y)$. 
Then  $P$ is {\bf properly supported} if the projections
$$pr_1 : \text{supp } K \longrightarrow X$$
and
$$pr_2 : \text{supp } K \longrightarrow X$$
are proper maps; i.e if $F$ is compact in $X$ then
$$(F \times X) \cap \text{ supp } K \text{ and } (X \times F) \cap \text{ supp } K$$
are compact in $X$.
\end{definition}
\nin
This is equivalent to:
\vskip0.1in
\nin
{\bf 1.} For  any  compact set $ F$ of $X$ there exists another compact
set  $F' $ of $X$ such that 
$$ \text{ supp }u\subset F \  \Longrightarrow \text{ supp }Pu\subset F'
\text{ and  if }  u =0\text{ on }F'\text{ then } Pu = 0  \text{ on  }F$$
and {\bf (1)} is equivalent to:
\vskip0.1in
\nin
{\bf 2.} For  any  compact set $ F$ of $X$ there exists another compact
set  $F' $ of $X$ such that 
$$ \text{ supp }u\subset F \  \Longrightarrow \text{ supp }Pu\subset F'$$
and
$$\text{ supp }u\subset F \  \Longrightarrow \text{ supp }{}^tPu\subset F'.$$
\nin
{\bf Examples:}
\vskip0.1in
\nin
{\bf 1.)}\quad If $P$ is a  pdo then it is proper. 
\vskip0.1in
\nin
{\bf 2.)}\quad Convolution operators are not properly supported. For example, if  $f \in C^\infty (\Bbb R) $ then 
$$ (Pu) (x) =\int f (x-y) u (y) dy $$
is not properly supported if $ f\not\equiv 0$. If say,
$f(x) = e^{-x^2}$  then supp  $K = \Bbb R^2.$
\begin{proposition} If $P \in \Psi^m (X)$ then
$$P = P' + R,$$
where $P' $ is properly supported, and $ R $ is smoothing.
\end{proposition}
\nin
{\bf Proof.} Let $\{\psi_j\}$ be a locally finite partition of unity on $X$;
i.e. $\psi_j \in C_0^\infty (X)$, 
$1 = \sum_{j=1}^{\infty} \psi_j$, and locally finite.
Then define
$$P' u (x) = \sum_{\text{supp} \psi_j \cap \text{supp} \psi_k \ne 0} \psi_j (x)
P(\psi_ku).$$ 
The  operator $P'$ is properly supported, and the symbol of $P - P'$ is given
by 
$$\sum_{\text{supp} \psi_j \cap \text{supp} \psi_k = 0} \psi_j (x) p(x, y, \xi)
\psi_k (y)$$ 
which is equal to zero in a neighborhood of the diagonal of $X \times X$. Thus
$P-P'$ is smoothing. $\qquad \Box$
\vskip0.1in
\begin{proposition} If $P \in \Psi^m (X)$ is properly supported then:
	\begin{enumerate}
		\item $P: C^\infty_0 (X) \xrightarrow{cont.} C_0^\infty (X)$
\item $P: C^\infty (X) \xrightarrow{cont.} C^\infty (X)$
\item $P: \mathcal E' (X) \xrightarrow{cont.} \mathcal E' (X)$
\item $P: \mathcal D'(X) \xrightarrow{cont.} \mathcal D' (X)$
\end{enumerate}
Also the same conclusions hold  for ${}^tP$.
\end{proposition}
\nin
{\bf Proof.} It follows from the definition and
 transposition. $\qquad \Box$
 \vskip0.1in
\noindent
{\bf Remark:  } If $P$ and  $Q$ are two properly supported
 $\psi$ do's then $P \circ Q$ and $Q \circ P$ are well defined.
\end{document}
                                                                                                                                                                   

                                                                                                                                                                                                                                                                            
 

