%
\documentclass[12pt,reqno]{amsart}
\usepackage{amssymb}
\usepackage{appendix}
\usepackage[showonlyrefs=true]{mathtools} %amsmath extension package
\usepackage{cancel}  %for cancelling terms explicity on pdf
\usepackage{yhmath}   %makes fourier transform look nicer, among other things
\usepackage{framed}  %for framing remarks, theorems, etc.
\usepackage{enumerate} %to change enumerate symbols
\usepackage[margin=2.5cm]{geometry}  %page layout
\setcounter{tocdepth}{1} %must come before secnumdepth--else, pain
\setcounter{secnumdepth}{1} %number only sections, not subsections
%\usepackage[pdftex]{graphicx} %for importing pictures into latex--pdf compilation
\numberwithin{equation}{section}  %eliminate need for keeping track of counters
%\numberwithin{figure}{section}
%
\usepackage{hyperref}
\hypersetup{colorlinks=true,
linkcolor=blue,
citecolor=blue,
urlcolor=blue,
}
\usepackage[alphabetic, initials, msc-links]{amsrefs} %for the bibliography; uses cite pkg. Must be loaded after hyperref, otherwise doesn't work properly (conflicts with cref in particular)
%
%
\newcommand{\ds}{\displaystyle}
\newcommand{\ts}{\textstyle}
\newcommand{\nin}{\noindent}
\newcommand{\rr}{\mathbb{R}}
\newcommand{\nn}{\mathbb{N}}
\newcommand{\zz}{\mathbb{Z}}
\newcommand{\cc}{\mathbb{C}}
\newcommand{\ci}{\mathbb{T}}
\newcommand{\zzdot}{\dot{\zz}}
\newcommand{\wh}{\widehat}
\newcommand{\p}{\partial}
\newcommand{\ee}{\varepsilon}
\newcommand{\vp}{\varphi}
\newcommand{\wt}{\widetilde}
%
%
%
%
\newtheorem{theorem}{Theorem}[section]
\newtheorem{lemma}[theorem]{Lemma}
\newtheorem{corollary}[theorem]{Corollary}
\newtheorem{claim}[theorem]{Claim}
\newtheorem{prop}[theorem]{Proposition}
\newtheorem{proposition}[theorem]{Proposition}
\newtheorem{no}[theorem]{Notation}
\newtheorem{definition}[theorem]{Definition}
\newtheorem{remark}[theorem]{Remark}
\newtheorem{examp}{Example}[section]
\newtheorem {exercise}[theorem] {Exercise}
%
%\makeatletter \renewenvironment{proof}[1][\proofname] {\par\pushQED{\qed}\normalfont\topsep6\p@\@plus6\p@\relax\trivlist\item[\hskip\labelsep\bfseries#1\@addpunct{.}]\ignorespaces}{\popQED\endtrivlist\@endpefalse} \makeatother%
%makes proof environment bold instead of italic
\newcommand{\uol}{u^\omega_\lambda}
\newcommand{\lbar}{\bar{l}}
\renewcommand{\l}{\lambda}
\newcommand{\R}{\mathbb R}
\newcommand{\RR}{\mathcal R}
\newcommand{\al}{\alpha}
\newcommand{\ve}{q}
\newcommand{\tg}{{tan}}
\newcommand{\m}{q}
\newcommand{\N}{N}
\newcommand{\ta}{{\tilde{a}}}
\newcommand{\tb}{{\tilde{b}}}
\newcommand{\tc}{{\tilde{c}}}
\newcommand{\tS}{{\tilde S}}
\newcommand{\tP}{{\tilde P}}
\newcommand{\tu}{{\tilde{u}}}
\newcommand{\tw}{{\tilde{w}}}
\newcommand{\tA}{{\tilde{A}}}
\newcommand{\tX}{{\tilde{X}}}
\newcommand{\tphi}{{\tilde{\phi}}}
\synctex=1
\begin{document}
\title{Private Miscellaneous Notes}
\author{David Karapetyan}
        %
        \maketitle
        %
        %
        %
        %
        %
        %
        \section{}
        \label{sec:ll}
        
\begin{lemma}
Let $X$ be Banach, and $L(X)$ be the space of continuous linear functionals on
$X$, equipped with the operator norm. Suppose 
%
%
\begin{gather*}
  x_{k} \xrightarrow{X} x,
  \\
  T_{k} \xrightarrow{L(X)} T.
\end{gather*}
%
%
Then
\begin{gather*}
  T_{k} x_{k} \to Tx.
\end{gather*}
%
\label{lem:diag}
\end{lemma}
%
%
\begin{proof}[Proof of Lemma~\ref{lem:diag}]
  Since $x_{k} \xrightarrow{X} x$, there exists $N > 0$ such that for $k > N$,
  we have $\|x -x_{k} \|_{X} \le 1$. Furthermore, if we denote $B_{X}(R) \doteq
  \left\{ x \in X: \| x \|_{X} \le 1 \right\}$, then $T_{k}
  \xrightarrow{L(X)} T$ is by definition equivalent to 
  %
  %
  \begin{equation*}
  \begin{split}
    \sup_{B_{X}(1)}  | (T - T_{k})(x) | \to 0.
  \end{split}
  \end{equation*}
  %
  %
  Hence, for $k > N$, 
  %
  %
  \begin{equation*}
  \begin{split}
    | Tx - T_{k}x_{k} |
    & = | Tx - T_{k}x + T_{k}x - T_{k}x_{k} |
    \\
    & \le | (T - T_{k})x | + | T_{k}(x -x_{k}) |
    \\
    & \le | (T - T_{k})x |
    + \| x - x_{k}\|_{X} \sup_{y \in B_{x}(1)} | T_{k}y |
    \\
    & \to 0
  \end{split}
  \end{equation*}
  %
  %
  which completes the proof.
\end{proof}

\begin{lemma}
If $E(X_{n}) \to a$ and $Var(X_{n}) \to 0$, then $X_{n} \xrightarrow{L^{2}(\Omega)}a$. That is, \\ $E[(X_{n} -a)^{2}] \to 0$.
\label{lem:}
\end{lemma}

\begin{proof}
Observe that $E[(X_{n}- a )^{2}] = E(X_{n}^{2}) - 2aE(X_{n}) + a^{2}$. Assume that $E(X_{n}) \to a$ and $Var(X_{n}) \to 0$; that is, $E(X_{n}) \to a$ and $[EX_{n}^{2}] - (E X_{n})^{2} \to 0$. Then
\begin{equation*}
\begin{split}
  E[(X_{n} - a)^{2}] & = E(X_{n}^{2}) - 2aE(X_{n}) + a^{2}
  \\
  & = (EX_{n})^{2} - 2a E(X_{n}) + a^{2}
  \\
  & \to a^{2} - 2a +  a^{2} = 0 
\end{split}
\end{equation*}
which completes the proof.
\end{proof}

\begin{lemma}
If $\Delta_{n}(t) \xrightarrow{L^{2}(\Omega)} \Delta$, then for any continuous $f: \rr \to \rr$, there exists a subsequence $\{ \Delta_{n_{k}} \}$ such that $f(\Delta_{n_{k}}) \xrightarrow{L^{2}(\Omega)} f(\Delta(t))$.
\label{lem:}
\end{lemma}
\begin{proof}
Since $\Delta_{n}(t) \xrightarrow{L^{2}(\Omega)} \Delta(t)$, there exists a subsequence $\left \{ \Delta_{n_{k}} \right \}$ such that $\Delta_{n_{k}}(t) \xrightarrow{a.e.} \Delta(t)$. In fact, we can say more: $| \Delta(t) | \le C$. Hence, for $k$ sufficiently large, $| \Delta_{n_{k}}(t) | \le 2c$. Reindexing, we see that we must have $f(\Delta_{n_{k}}(t)) \xrightarrow{L^{\infty}(\Omega) f(\Delta)}$. But this implies $f(\Delta_{n_{k}}(t)) \xrightarrow{L^{2}(\Omega)}$, since our probability measure $\mathbb{P}$ is sigma-finite. That is,

\begin{equation*}
\begin{split}
  \int_{\Omega} [f(\Delta(t)) - f(\Delta_{n_{k}}(t))]^{2} d \mathbb{P} 
  & \le \| f(\Delta(t)) - f(\Delta_{n_{k}}(t)) \|^{2}_{L^{\infty}} \int_{\Omega} d \mathbb{P}
  \\
  & =  \| f(\Delta(t)) - f(\Delta_{n_{k}}(t)) \|^{2}_{L^{\infty}}
  \\
  & \to 0
\end{split}
\end{equation*}
completing the proof.
\end{proof}

\begin{lemma}
Let $X(t), Y(t)$ be two random variables. Suppose $X(t)$ has quadratic variation $a=a(t)$ from $0$ to $t$, and $Y(t)$ has zero quadratic variation from $0$ to $t$. Then $Z(t) = X(t) + Y(t)$ has quadratic variation $a = a(t)$.
\label{lem:}
\end{lemma}

\begin{proof}
Let $0 = t_{1} < t_{2} < \ldots < t_{n} = t$ be a partition of $[0, t]$. Then
\begin{equation*}
\begin{split}
[Z, Z](t) = \lim_{\| II \| \to 0} \sum_{k=1}^{n} [Z(t_{k+1}) - Z(t_{k})]^{2}.
\end{split}
\end{equation*}
But
\begin{equation*}
\begin{split}
  & \sum_{k=1}^{n} [Z(t_{k+1}) - Z(t_{k})]^{2}
  \\
  & = \sum_{k=1}^{n} [X(t_{k+1}) - X(t_{k}) + Y(t_{k+1}) - Y(t_{k})]^{2}
  \\
  & = \sum_{k=1}^{n}[X(t_{k+1}) - X(t_{k})]^{2} + 
  \sum_{k=1}^{n}[Y(t_{k+1}) - Y(t_{k})]^{2} + 2 \sum_{k=1}^{n} [X(t_{k+1}) - X(t_{k})] [Y(t_{k+1}) - Y(t_{k})]
  \\
 & \to a(t) + 2 \lim_{\| II \| \to 0} \sum_{k=1}^{n} [X(t_{k+1}) - X(t_{k})][Y(t_{k+1}) - Y(t_{k})]
 \\
& \to a(t) \ \ \text{by Cauchy Schwartz} 
\end{split}
\end{equation*}
completing the proof.
\end{proof}

Let us now show how one can use the Ito Calculus to solve SDEs. 
Let us return to the SDE for geometric Brownian motion
\begin{equation*}
\begin{split}
dS = S(t) \alpha(t) dt + S(t) \sigma(t) dW(t) 
\end{split}
\end{equation*}
which we rewrite as
\begin{equation*}
\begin{split}
  \frac{dS}{S(t)} = \alpha(t) dt + \sigma(t) dW(t).
\end{split}
\end{equation*}
Observe that it is the random noise in this model that makes it difficult to solve(if $\sigma = 0$, then the resulting equation is an ODE and can be easily solved explicitly.) Hence, we first restrict our attention to
\begin{equation}
  \label{pure-noise}
\begin{split}
\frac{dS(t)}{S(t)} = \sigma(t) dW(t).
\end{split}
\end{equation}

By the Ito Calculus,

\begin{equation*}
\begin{split}
d \log | S(t) |
& = \frac{dS}{S(t)} -\frac{1}{2[S(t)]^{2}} \cdot [dS, dS](t)
\\
& = \frac{dS(t)}{S(t)}  -\frac{1}{2[S(t)]^{2}} \cdot \sigma^{2}(t) S^{2}(t)
\\
& = \frac{dS(t)}{S(t)} - \frac{1}{2} \sigma^{2}(t) dt
\\
& = \sigma(t) d W(t) - \frac{1}{2} \sigma^{2}(t) dt
\end{split}
\end{equation*}
where the last line follows by substituting relation \eqref{pure-noise}. Therefore,
\begin{equation*}
\begin{split}
  \log | S(t) | - \log | S(0) | = \int_{0}^{t} \sigma(s) dW(s) -\frac{1}{2} \int_{0}^{t} \sigma^{2}(s) ds.
\end{split}
\end{equation*}
We conclude that
\begin{equation*}
\begin{split}
S(t) = S_{0} \exp[\int_{0}^{t} \sigma(s) dW(s) - \frac{1}{2} \int_{0}^{t} \sigma^{2}(s) ds].
\end{split}
\end{equation*}

\begin{lemma}
Consider the function $f(x) = 4x^3 - 3x$, and define $f_n$ to be
the composition of $f$ $n$ times with itself. Show that
for any given $\ee > 0$, there exist $a,b \in (0,1)$ at most $\ee$ apart
such that $f_n(a)$ converges but $f_n(b)$ diverges
\end{lemma}
\begin{proof}
We first compute $f'(x) = 12x^2 - 3$. Setting this equal first
to $1$, then $1- 12\ee$, we obtain candidates $c = \sqrt{1/3}$ and $d =
\sqrt{1/3 - \ee}$, respectively, with difference 
\begin{align*}
\sqrt{1/3} - \sqrt{1/3 - \ee} = \frac{\ee}{\sqrt{1/3} + \sqrt{1/3 - \ee}} \le
\ee \sqrt{3}.
\end{align*}

Next, by chain rule
\begin{align*}
f_{n+1}'(x) = f_n'(f(x)) f'(x), \quad f_0 \equiv 1
\end{align*}
for all integers $n \ge 0$.
From this relation, a simple induction 
using the observation that $f(1) = 1$ and $f'(1) = 9 > 1$
gives
\begin{align*}
f_{n+1}'(1) \ge 1, \quad \forall n\ge0.
\end{align*}
Running another induction using this fact gives
\begin{enumerate}
\item
$f'(x)$ is an increasing function for $x \ge (1/4)^{1/3}$ 
\item
$f'(c)=1$ and $f'(d) = 1 - 12 \ee$
\item
$f(c) = 1$ and $f(d) \approx 1$.
\end{enumerate}
shows that for our choice of $c$ and $d$ above
(which satisfy $f'(c) = 1$ and $|f'(d)| < 1$), we have
\begin{align*}
& f_{n}'(c) \ge 1 \\
& | f_{n}'(d) |  < 1
\end{align*}
for all $n \ge 0$. 
It immediately follows that $f_n(c)$ diverges and $f_n(d)$ converges
as $n \to \infty$. Hence, the proof reduces to finding $c$ and $d$ 
that are at most $\ee$ apart.


To do so, we compute $f'(x) = 12x^3 - 3$. Setting this equal first
to $1$, then $1- 12\ee$, we obtain $c = \pm \sqrt{1/3}$ and $d = 1/3 - \ee$,
respectively, with difference 
\begin{align*}
\sqrt{1/3} - \sqrt{1/3 - \ee} = \frac{\ee}{\sqrt{1/3} + \sqrt{1/3 - \ee}} \le
\ee \sqrt{3}.
\end{align*}
which completes the proof. (For clarity, one can replace $\ee$ with
$\eta/\sqrt{3}$ in the above computations). 
\end{proof}
        %\nocite{*}
        %\bibliography{/Users/davidkarapetyan/Documents/math/}

        \end{document}
