%
\documentclass[12pt,reqno]{amsart}
\usepackage{amscd}
\usepackage{amsfonts}
\usepackage{amsmath}
\usepackage{amssymb}
\usepackage{amsthm}
\usepackage{appendix}
\usepackage{fancyhdr}
\usepackage{latexsym}
\usepackage{cancel}
\usepackage{amsxtra}
\synctex=1
\usepackage[colorlinks=true, pdfstartview=fitv, linkcolor=blue,
citecolor=blue, urlcolor=blue]{hyperref}
%%%%%%%%%%%%%%%%%%%%%%
\usepackage{color}
\definecolor{red}{rgb}{1.00, 0.00, 0.00}
\definecolor{darkgreen}{rgb}{0.00, 1.00, 0.00}
\definecolor{blue}{rgb}{0.00, 0.00, 1.00}
\definecolor{cyan}{rgb}{0.00, 1.00, 1.00}
\definecolor{magenta}{rgb}{1.00, 0.00, 1.00}
\definecolor{deepskyblue}{rgb}{0.00, 0.75, 1.00}
\definecolor{darkgreen}{rgb}{0.00, 0.39, 0.00}
\definecolor{springgreen}{rgb}{0.00, 1.00, 0.50}
\definecolor{darkorange}{rgb}{1.00, 0.55, 0.00}
\definecolor{orangered}{rgb}{1.00, 0.27, 0.00}
\definecolor{deeppink}{rgb}{1.00, 0.08, 0.57}
\definecolor{darkviolet}{rgb}{0.58, 0.00, 0.82}
\definecolor{saddlebrown}{rgb}{0.54, 0.27, 0.07}
\definecolor{black}{rgb}{0.00, 0.00, 0.00}
\definecolor{dark-magenta}{rgb}{.5,0,.5}
\definecolor{myblack}{rgb}{0,0,0}
\definecolor{darkgray}{gray}{0.5}
\definecolor{lightgray}{gray}{0.75}
%%%%%%%%%%%%%%%%%%%%%%
%%%%%%%%%%%%%%%%%%%%%%%%%%%%
%  for importing pictures  %
%%%%%%%%%%%%%%%%%%%%%%%%%%%%
\usepackage[pdftex]{graphicx}
\usepackage{epstopdf}
% \usepackage{graphicx}
%% page setup %%
\setlength{\textheight}{20.8truecm}
\setlength{\textwidth}{14.8truecm}
\marginparwidth  0truecm
\oddsidemargin   01truecm
\evensidemargin  01truecm
\marginparsep    0truecm
\renewcommand{\baselinestretch}{1.1}
%% new commands %%
\newcommand{\tf}{\tilde{f}}
\newcommand{\ti}{\tilde}
\newcommand{\bigno}{\bigskip\noindent}
\newcommand{\ds}{\displaystyle}
\newcommand{\medno}{\medskip\noindent}
\newcommand{\smallno}{\smallskip\noindent}
\newcommand{\nin}{\noindent}
\newcommand{\ts}{\textstyle}
\newcommand{\rr}{\mathbb{R}}
\newcommand{\p}{\partial}
\newcommand{\zz}{\mathbb{Z}}
\newcommand{\cc}{\mathbb{C}}
\newcommand{\ci}{\mathbb{T}}
\newcommand{\ee}{\varepsilon}
\newcommand{\vp}{\varphi}
\def\autorefer #1\par{\noindent\hangindent=\parindent\hangafter=1 #1\par}
%% equation numbers %%
\renewcommand{\theequation}{\thesection.\arabic{equation}}
%% new environments %%
%\swapnumbers
\theoremstyle{plain}  % default
\newtheorem{theorem}{Theorem}
\newtheorem{proposition}{Proposition}
\newtheorem{lemma}{Lemma}
\newtheorem{corollary}{Corollary}
\newtheorem{claim}{Claim}
\newtheorem{remark}{Remark}
\newtheorem{conjecture}[subsection]{conjecture}
\newtheorem{definition}{Definition}
\def\makeautorefname#1#2{\expandafter\def\csname#1autorefname\endcsname{#2}}
\makeautorefname{equation}{Equation}
\makeautorefname{footnote}{footnote}
\makeautorefname{item}{item}
\makeautorefname{figure}{Figure}
\makeautorefname{table}{Table}
\makeautorefname{part}{Part}
\makeautorefname{appendix}{Appendix}
\makeautorefname{chapter}{Chapter}
\makeautorefname{section}{Section}
\makeautorefname{subsection}{Section}
\makeautorefname{subsubsection}{Section}
\makeautorefname{paragraph}{Paragraph}
\makeautorefname{subparagraph}{Paragraph}
\makeautorefname{theorem}{Theorem}
\makeautorefname{theo}{Theorem}
\makeautorefname{thm}{Theorem}
\makeautorefname{addendum}{Addendum}
\makeautorefname{addend}{Addendum}
\makeautorefname{add}{Addendum}
\makeautorefname{maintheorem}{Main theorem}
\makeautorefname{mainthm}{Main theorem}
\makeautorefname{corollary}{Corollary}
\makeautorefname{corol}{Corollary}
\makeautorefname{coro}{Corollary}
\makeautorefname{cor}{Corollary}
\makeautorefname{lemma}{Lemma}
\makeautorefname{lemm}{Lemma}
\makeautorefname{lem}{Lemma}
\makeautorefname{sublemma}{Sublemma}
\makeautorefname{sublem}{Sublemma}
\makeautorefname{subl}{Sublemma}
\makeautorefname{proposition}{Proposition}
\makeautorefname{proposit}{Proposition}
\makeautorefname{propos}{Proposition}
\makeautorefname{propo}{Proposition}
\makeautorefname{prop}{Proposition}
\makeautorefname{proposition}{Proposition}
\makeautorefname{property}{Property}
\makeautorefname{proper}{Property}
\makeautorefname{scholium}{Scholium}
\makeautorefname{step}{Step}
\makeautorefname{conjecture}{Conjecture}
\makeautorefname{conject}{Conjecture}
\makeautorefname{conj}{Conjecture}
\makeautorefname{question}{Question}
\makeautorefname{questn}{Question}
\makeautorefname{quest}{Question}
\makeautorefname{ques}{Question}
\makeautorefname{qn}{Question}
\makeautorefname{definition}{Definition}
\makeautorefname{defin}{Definition}
\makeautorefname{defi}{Definition}
\makeautorefname{def}{Definition}
\makeautorefname{dfn}{Definition}
\makeautorefname{notation}{Notation}
\makeautorefname{nota}{Notation}
\makeautorefname{notn}{Notation}
\makeautorefname{remark}{Remark}
\makeautorefname{rema}{Remark}
\makeautorefname{rem}{Remark}
\makeautorefname{rmk}{Remark}
\makeautorefname{rk}{Remark}
\makeautorefname{remarks}{Remarks}
\makeautorefname{rems}{Remarks}
\makeautorefname{rmks}{Remarks}
\makeautorefname{rks}{Remarks}
\makeautorefname{example}{Example}
\makeautorefname{examp}{Example}
\makeautorefname{exmp}{Example}
\makeautorefname{exam}{Example}
\makeautorefname{exa}{Example}
\makeautorefname{algorithm}{Algorith}
\makeautorefname{algo}{Algorith}
\makeautorefname{alg}{Algorith}
\makeautorefname{axiom}{Axiom}
\makeautorefname{axi}{Axiom}
\makeautorefname{ax}{Axiom}
\makeautorefname{case}{Case}
\makeautorefname{claim}{Claim}
\makeautorefname{clm}{Claim}
\makeautorefname{assumption}{Assumption}
\makeautorefname{assumpt}{Assumption}
\makeautorefname{conclusion}{Conclusion}
\makeautorefname{concl}{Conclusion}
\makeautorefname{conc}{Conclusion}
\makeautorefname{condition}{Condition}
\makeautorefname{condit}{Condition}
\makeautorefname{cond}{Condition}
\makeautorefname{construction}{Construction}
\makeautorefname{construct}{Construction}
\makeautorefname{const}{Construction}
\makeautorefname{cons}{Construction}
\makeautorefname{criterion}{Criterion}
\makeautorefname{criter}{Criterion}
\makeautorefname{crit}{Criterion}
\makeautorefname{exercise}{Exercise}
\makeautorefname{exer}{Exercise}
\makeautorefname{exe}{Exercise}
\makeautorefname{problem}{Problem}
\makeautorefname{problm}{Problem}
\makeautorefname{probm}{Problem}
\makeautorefname{prob}{Problem}
\makeautorefname{solution}{Solution}
\makeautorefname{soln}{Solution}
\makeautorefname{sol}{Solution}
\makeautorefname{summary}{Summary}
\makeautorefname{summ}{Summary}
\makeautorefname{sum}{Summary}
\makeautorefname{operation}{Operation}
\makeautorefname{oper}{Operation}
\makeautorefname{observation}{Observation}
\makeautorefname{observn}{Observation}
\makeautorefname{obser}{Observation}
\makeautorefname{obs}{Observation}
\makeautorefname{ob}{Observation}
\makeautorefname{convention}{Convention}
\makeautorefname{convent}{Convention}
\makeautorefname{conv}{Convention}
\makeautorefname{cvn}{Convention}
\makeautorefname{warning}{Warning}
\makeautorefname{warn}{Warning}
\makeautorefname{note}{Note}
\makeautorefname{fact}{Fact}
%
\begin{document}
%\begin{titlepage}
\title{Research Prospectus }
\author{David Karapetyan}
\address{Department of Mathematics  \\
         University  of Notre Dame\\
		          Notre Dame, IN 46556 }
				  \date{11/12/09}
				  %
				  \maketitle
				  %
				  %
				  \parindent0in
				  \parskip0.1in
				  %
				  %\end{titlepage}
				  %
				  %
				  %
				  \section{Research Goals}
				  \setcounter{equation}{0}
				  My research goals are the following:
				  %
				  %
				  \begin{enumerate}
					  \item Understand the ``Burgers world'' of PDE's.
					  \item Understand the ``KDV world'' of PDE's.
					  \item Conduct numerical experimentation with
						  equations from the ``Burgers'' and ``KDV''
						  worlds.
					  \item Study PDE's arising from problems in
						  financial mathematics. 
				  \end{enumerate}
				  %
				  %
				  I am currently studying a family of nonlinear evolution
				  equations of form
%
%
\begin{gather}
	\label{mch}
		 \p_t u + \gamma_0 \p_x^3 + \gamma_1 u \p_x u + (1 - \p_x^2)^{-1}
		\p_x \left[ \gamma_2 u^2 + \gamma_3 (\p_x u)^2 \right] = 0,
		\\
		u(x,0) = \vp(x), \quad t \in \rr, \quad x \in \ci.
		\label{mch-init-data}
	\end{gather}
%
%
Define the multi-index $\gamma = (\gamma_0, \gamma_1, \gamma_2, \gamma_3)$.
Note that when $\gamma = (1,1,2, 1/2)$ we
obtain the modified Camassa-Holm (mCH) equation, which has previously been
studied by Himonas, Misiolek, Byers, Gorsky, and Olson. Assuming $\gamma_0
\neq 0$, we have the following:
%
%
%
%%%%%%%%%%%%%%%%%%%%%%%%%%%%%%%%%%%%%%%%%%%%%%%%%%%%%
%
%
%	Well-posedness Theorem				
%
%
%%%%%%%%%%%%%%%%%%%%%%%%%%%%%%%%%%%%%%%%%%%%%%%%%%%%%
%
%
%
%
%
%
\begin{theorem}
	\label{thm:him-gorsky-key-thm}
	For any $s \ge 1/2$ the initial value problem
	\eqref{mch}-\eqref{mch-init-data} is locally well-posed for
	sufficiently small initial data $\vp \in H^s(\ci)$. More precisely, if
	$\|\vp\|_{H^s}$ is sufficiently small then the initial value problem
	\eqref{mch}-\eqref{mch-init-data} has a unique local solution in the
	space $Y^s$ of all $L^2$ functions $u: \ci \times \rr \to \rr$ with
	finite norm:
	%
	%
%
%
\begin{equation}
	\label{soln-space-norm}
	\begin{split}
		\|u\|_{Y_s} = \|u\|_{X_s} + \left( \sum_{n \in \zz} |n|^{2s} \left(
		\int_\rr |\widehat{u}(n, \lambda)|d \lambda \right)^2 \right)^{1/2},
	\end{split}
\end{equation}
%
%
where
%
%
%
%
\begin{equation}
	\label{norm-piece}
	\begin{split}
		\|u\|_{X_s} = \left( \sum_{n \in \zz} |n|^{2s} \int_\rr (1 + |\lambda
		- n^3|) |\widehat{u}(n, \lambda) |^2 d \lambda \right)^{1/2},
	\end{split}
\end{equation}
%
%
and solutions depend continuously on the initial data.
%
%
\end{theorem}
%
%
%
My current project consists of a series of questions and tasks 
regarding the i.v.p.
\eqref{mch}-\eqref{mch-init-data} and \autoref{thm:him-gorsky-key-thm}:
%
%
\begin{enumerate}
		%
		%
	\item Understand the proof of \autoref{thm:him-gorsky-key-thm} and
		present it in the PDE seminar.
	\item Can \autoref{thm:him-gorsky-key-thm} be improved, i.e. is the
		equation well-posed for $s > s_0$ for $s_0 < 1/2$?
	\item What is the regularity of the data-to-solution map?
	\item What are the conditions for ill-posedness?
	\item Conduct numerical experimentation to see how different values of
		$\gamma$ (which can give rise to the KDV, CH, DP,  HR, and other
		equations) influence the properties of \eqref{mch}.
	\item If $\gamma_0 = \ee \to 0$, then is it true that
		$\eqref{mch}_\gamma$ i.v.p solutions $\to CH_\gamma$ i.v.p
		solutions (i.e. Lax-Livermore, KDV-like convergence)? 
	\item Anything else that is natural to ask.
		%
		%
\end{enumerate}








				  \end{document}


