%
\documentclass[12pt,reqno]{amsart}
\usepackage{amscd}
\usepackage{amsfonts}
\usepackage{amsmath}
\usepackage{amssymb}
\usepackage{amsthm}
\usepackage{appendix}
\usepackage{fancyhdr}
\usepackage{latexsym}
\usepackage{cancel}
\usepackage{amsxtra}
\synctex=1
\usepackage[colorlinks=true, pdfstartview=fitv, linkcolor=blue,
citecolor=blue, urlcolor=blue]{hyperref}
\input epsf
\input texdraw
\input txdtools.tex
\input xy
\xyoption{all}
%%%%%%%%%%%%%%%%%%%%%%
\usepackage{color}
\definecolor{red}{rgb}{1.00, 0.00, 0.00}
\definecolor{darkgreen}{rgb}{0.00, 1.00, 0.00}
\definecolor{blue}{rgb}{0.00, 0.00, 1.00}
\definecolor{cyan}{rgb}{0.00, 1.00, 1.00}
\definecolor{magenta}{rgb}{1.00, 0.00, 1.00}
\definecolor{deepskyblue}{rgb}{0.00, 0.75, 1.00}
\definecolor{darkgreen}{rgb}{0.00, 0.39, 0.00}
\definecolor{springgreen}{rgb}{0.00, 1.00, 0.50}
\definecolor{darkorange}{rgb}{1.00, 0.55, 0.00}
\definecolor{orangered}{rgb}{1.00, 0.27, 0.00}
\definecolor{deeppink}{rgb}{1.00, 0.08, 0.57}
\definecolor{darkviolet}{rgb}{0.58, 0.00, 0.82}
\definecolor{saddlebrown}{rgb}{0.54, 0.27, 0.07}
\definecolor{black}{rgb}{0.00, 0.00, 0.00}
\definecolor{dark-magenta}{rgb}{.5,0,.5}
\definecolor{myblack}{rgb}{0,0,0}
\definecolor{darkgray}{gray}{0.5}
\definecolor{lightgray}{gray}{0.75}
%%%%%%%%%%%%%%%%%%%%%%
%%%%%%%%%%%%%%%%%%%%%%%%%%%%
%  for importing pictures  %
%%%%%%%%%%%%%%%%%%%%%%%%%%%%
\usepackage[pdftex]{graphicx}
\usepackage{epstopdf}
% \usepackage{graphicx}
%% page setup %%
\setlength{\textheight}{20.8truecm}
\setlength{\textwidth}{14.8truecm}
\marginparwidth  0truecm
\oddsidemargin   01truecm
\evensidemargin  01truecm
\marginparsep    0truecm
\renewcommand{\baselinestretch}{1.1}
%% new commands %%
\newcommand{\tf}{\tilde{f}}
\newcommand{\ti}{\tilde}
\newcommand{\bigno}{\bigskip\noindent}
\newcommand{\ds}{\displaystyle}
\newcommand{\medno}{\medskip\noindent}
\newcommand{\smallno}{\smallskip\noindent}
\newcommand{\nin}{\noindent}
\newcommand{\ts}{\textstyle}
\newcommand{\rr}{\mathbb{R}}
\newcommand{\p}{\partial}
\newcommand{\zz}{\mathbb{Z}}
\newcommand{\cc}{\mathbb{C}}
\newcommand{\ci}{\mathbb{T}}
\newcommand{\ee}{\varepsilon}
\newcommand{\vp}{\varphi}
\def\autorefer #1\par{\noindent\hangindent=\parindent\hangafter=1 #1\par}
%% equation numbers %%
\renewcommand{\theequation}{\thesection.\arabic{equation}}
%% new environments %%
%\swapnumbers
\theoremstyle{plain}  % default
\newtheorem{theorem}{Theorem}
\newtheorem{proposition}{Proposition}
\newtheorem{lemma}{Lemma}
\newtheorem{corollary}{Corollary}
\newtheorem{claim}{Claim}
\newtheorem{remark}{Remark}
\newtheorem{conjecture}[subsection]{conjecture}
\newtheorem{definition}{Definition}
\def\makeautorefname#1#2{\expandafter\def\csname#1autorefname\endcsname{#2}}
\makeautorefname{equation}{Equation}
\makeautorefname{footnote}{footnote}
\makeautorefname{item}{item}
\makeautorefname{figure}{Figure}
\makeautorefname{table}{Table}
\makeautorefname{part}{Part}
\makeautorefname{appendix}{Appendix}
\makeautorefname{chapter}{Chapter}
\makeautorefname{section}{Section}
\makeautorefname{subsection}{Section}
\makeautorefname{subsubsection}{Section}
\makeautorefname{paragraph}{Paragraph}
\makeautorefname{subparagraph}{Paragraph}
\makeautorefname{theorem}{Theorem}
\makeautorefname{theo}{Theorem}
\makeautorefname{thm}{Theorem}
\makeautorefname{addendum}{Addendum}
\makeautorefname{addend}{Addendum}
\makeautorefname{add}{Addendum}
\makeautorefname{maintheorem}{Main theorem}
\makeautorefname{mainthm}{Main theorem}
\makeautorefname{corollary}{Corollary}
\makeautorefname{corol}{Corollary}
\makeautorefname{coro}{Corollary}
\makeautorefname{cor}{Corollary}
\makeautorefname{lemma}{Lemma}
\makeautorefname{lemm}{Lemma}
\makeautorefname{lem}{Lemma}
\makeautorefname{sublemma}{Sublemma}
\makeautorefname{sublem}{Sublemma}
\makeautorefname{subl}{Sublemma}
\makeautorefname{proposition}{Proposition}
\makeautorefname{proposit}{Proposition}
\makeautorefname{propos}{Proposition}
\makeautorefname{propo}{Proposition}
\makeautorefname{prop}{Proposition}
\makeautorefname{proposition}{Proposition}
\makeautorefname{property}{Property}
\makeautorefname{proper}{Property}
\makeautorefname{scholium}{Scholium}
\makeautorefname{step}{Step}
\makeautorefname{conjecture}{Conjecture}
\makeautorefname{conject}{Conjecture}
\makeautorefname{conj}{Conjecture}
\makeautorefname{question}{Question}
\makeautorefname{questn}{Question}
\makeautorefname{quest}{Question}
\makeautorefname{ques}{Question}
\makeautorefname{qn}{Question}
\makeautorefname{definition}{Definition}
\makeautorefname{defin}{Definition}
\makeautorefname{defi}{Definition}
\makeautorefname{def}{Definition}
\makeautorefname{dfn}{Definition}
\makeautorefname{notation}{Notation}
\makeautorefname{nota}{Notation}
\makeautorefname{notn}{Notation}
\makeautorefname{remark}{Remark}
\makeautorefname{rema}{Remark}
\makeautorefname{rem}{Remark}
\makeautorefname{rmk}{Remark}
\makeautorefname{rk}{Remark}
\makeautorefname{remarks}{Remarks}
\makeautorefname{rems}{Remarks}
\makeautorefname{rmks}{Remarks}
\makeautorefname{rks}{Remarks}
\makeautorefname{example}{Example}
\makeautorefname{examp}{Example}
\makeautorefname{exmp}{Example}
\makeautorefname{exam}{Example}
\makeautorefname{exa}{Example}
\makeautorefname{algorithm}{Algorith}
\makeautorefname{algo}{Algorith}
\makeautorefname{alg}{Algorith}
\makeautorefname{axiom}{Axiom}
\makeautorefname{axi}{Axiom}
\makeautorefname{ax}{Axiom}
\makeautorefname{case}{Case}
\makeautorefname{claim}{Claim}
\makeautorefname{clm}{Claim}
\makeautorefname{assumption}{Assumption}
\makeautorefname{assumpt}{Assumption}
\makeautorefname{conclusion}{Conclusion}
\makeautorefname{concl}{Conclusion}
\makeautorefname{conc}{Conclusion}
\makeautorefname{condition}{Condition}
\makeautorefname{condit}{Condition}
\makeautorefname{cond}{Condition}
\makeautorefname{construction}{Construction}
\makeautorefname{construct}{Construction}
\makeautorefname{const}{Construction}
\makeautorefname{cons}{Construction}
\makeautorefname{criterion}{Criterion}
\makeautorefname{criter}{Criterion}
\makeautorefname{crit}{Criterion}
\makeautorefname{exercise}{Exercise}
\makeautorefname{exer}{Exercise}
\makeautorefname{exe}{Exercise}
\makeautorefname{problem}{Problem}
\makeautorefname{problm}{Problem}
\makeautorefname{probm}{Problem}
\makeautorefname{prob}{Problem}
\makeautorefname{solution}{Solution}
\makeautorefname{soln}{Solution}
\makeautorefname{sol}{Solution}
\makeautorefname{summary}{Summary}
\makeautorefname{summ}{Summary}
\makeautorefname{sum}{Summary}
\makeautorefname{operation}{Operation}
\makeautorefname{oper}{Operation}
\makeautorefname{observation}{Observation}
\makeautorefname{observn}{Observation}
\makeautorefname{obser}{Observation}
\makeautorefname{obs}{Observation}
\makeautorefname{ob}{Observation}
\makeautorefname{convention}{Convention}
\makeautorefname{convent}{Convention}
\makeautorefname{conv}{Convention}
\makeautorefname{cvn}{Convention}
\makeautorefname{warning}{Warning}
\makeautorefname{warn}{Warning}
\makeautorefname{note}{Note}
\makeautorefname{fact}{Fact}
%
\begin{document}
%\begin{titlepage}
\title{Blowup of $C^2$ Solutions for a Modified Burgers Equation }
\author{David Karapetyan}
\address{Department of Mathematics  \\
University  of Notre Dame\\
Notre Dame, IN 46556 }
\date{11/16/09}
%
\maketitle
%
%
\parindent0in
\parskip0.1in
%
%\end{titlepage}
%
%
%
\setcounter{equation}{0}
We consider a modified Burgers equation (mB)
%
%
\begin{gather}
	\label{mbeq}
	\p_t u + u \p_x u - \gamma \p_x^2 u = 0,
	\\
	\label{mbeq-init-cond}
	u(0,x) = u_0(x), \quad x \in \rr,  \ \ t \in \rr
\end{gather}
%
%
and prove the following result:
%
%
%%%%%%%%%%%%%%%%%%%%%%%%%%%%%%%%%%%%%%%%%%%%%%%%%%%%%
%
%
%			Main Theorem	
%
%
%%%%%%%%%%%%%%%%%%%%%%%%%%%%%%%%%%%%%%%%%%%%%%%%%%%%%
%
%
\begin{theorem}
	\label{thm:main}
	The compactly supported $C^2$ solutions of the mB Cauchy problem
	\eqref{mbeq}-\eqref{mbeq-init-cond} blow up on or
	before
	%
	%
	\begin{equation}
		\label{lifespan}
		\begin{split}
			T = -\frac{1}{\gamma} \ln\left( \frac{H_0}{|K|}\left[
			\frac{H_0}{|K|} - \frac{2\gamma}{\sup_{x \in K} x^2}
			\right]^{-1} \right)
		\end{split}
	\end{equation}
	%
	%
	provided that
	%
	%
	\begin{equation}
		\label{blowup-cond}
		\begin{split}
			H_0 = \int_K \p_x u(0, x(0)) \ dx < 0
		\end{split}
	\end{equation}
	%
	%
	where $K= K(t)$ is the support of the solution $u(t,x(t))$ and $|K|$ is the
	volume of $K$.
\end{theorem}
%
%
{\bf Proof.}
%
%
Multiplying both sides of \eqref{mbeq} by $x$ and integrating gives
%
%
%
%
\begin{equation*}
	\begin{split}
		\int_K (\p_t u + u \p_x u) \cdot x \ dx - \gamma \int_K	\p_x^2 u \cdot 
		x \ dx = 0.
	\end{split}
\end{equation*}
%
%
Integrating by parts and recalling that $u$ is compactly supported, we 
obtain
%
%
\begin{equation}
	\label{before-char}
	\begin{split}
		\frac{1}{2} \int_K x^2 \p_x (\p_t u + u \p_x u) \ dx - \gamma \int_K 
		\p_x u \ dx = 0.
	\end{split}
\end{equation}
%
%
Since \eqref{mbeq} is quasi-linear and the real line is 
non-characteristic, we can find characteristic curves $x(t)$
extending from the real line into the $tx$ plane along which $u(t, x(t))$ is 
constant. Hence, \eqref{before-char} can be viewed as a function of $t$ 
along the characteristic curves. That is
\begin{equation}
	\label{after-char}
	\begin{split}
		\frac{1}{2} \int_K x^2 \p_x (\p_t u(t, x(t)) + u(t, x(t)) \p_x u(t, 
		x(t)) \ dx - \gamma \int_K 
		\p_x u(t,x(t)) \ dx = 0.
	\end{split}
\end{equation}

Next, note that since $u \in C^2$ a priori, $\p_x^2 u$ is continuous. Hence, we can 
find a $T >0$ sufficiently small such that $ \p_x^2$ which by the mean value theorem for integrals takes the form
%
%
\begin{equation*}
	\begin{split}
		\frac{\bar{x}^2}{2} \int_K \p_x (\p_t u + u \p_x u) \ dx - \gamma 
		\int_K \p_x u \ dx = 0	
	\end{split}
\end{equation*}
%
%
where $\bar{x} = \bar{x}(t) \in K(t)$. Expanding the $\p_x (\p_t u + u \p_x 
u)$ term yields
%
%
\begin{equation*}
	\begin{split}
		\frac{\bar{x}^2}{2} \int_K \left[ \p_t \p_x u + u \p_x^2 u + (\p_x 
		u)^2 \right]  dx - \gamma \int_K \p_x u \ dx = 0.	
	\end{split}
\end{equation*}
%
%
\begin{equation*}
	\begin{split}
		& \frac{\bar{x}^2}{2} \left [ \p_t H(t) +  \int_K u(t,x(t)) 
		\p_x^2 u(t,x(t)) \ dx + \int_K \left \{\p_x u(t,x(t)) \right \}^2  dx 
		\right ]
		- \gamma H(t) = 0
	\end{split}
\end{equation*}
where
%
%
\begin{equation*}
	\label{H-def}
	\begin{split}
		H(t) \doteq \int_K \p_x u(t, x(t)) \ dx.
	\end{split}
\end{equation*}
%
%
Since $u(t, x(t))$ is a function only of $t$, we can pull the operator $u(t, x(t)) \p_x$ 
outside of the first integral to obtain
\begin{equation}
	\begin{split}
		\label{bar-x-to-sup}
		& \frac{\bar{x}^2}{2} \left [ \p_t H(t) + u(t, x(t)) \p_x H(t) +
		\int_K \left \{\p_x u(t,x(t)) \right \}^2  dx 
		\right ]
		- \gamma H(t) = 0.
	\end{split}
\end{equation}
Note that since $u \in C^2(K)$ a priori, we have $H \in C^1(0, T)$. 
Furthermore, by continuity and the bound $H_0 < 0$, we have 
$H(t) <0$ for all non-negative $t$ in a sufficiently small neighborhood of 
$0$. Hence, \eqref{bar-x-to-sup} implies
%
%
\begin{equation*}
	\begin{split}
		& \frac{\bar{x}^2}{2} \left [ \p_t H(t) + u(t, x(t)) \p_x H(t) +
		\int_K \left \{\p_x u(t,x(t)) \right \}^2  dx 
		\right ] = \gamma H(t) \le 0
	\end{split}
\end{equation*}
%
%
which gives
\begin{equation}
	\label{after-char-est}
	\begin{split}
		& \frac{\sup x^2}{2} \left [ \p_t H(t) + u(t, x(t)) \p_x H(t) +
		\int_K \left \{\p_x u(t,x(t)) \right \}^2  dx 
		\right ] 		 \\
		& - \gamma H(t) \le 0
	\end{split}
\end{equation}
where $\sup x^2 = \sup_{x \in K} x^2$. We wish to obtain an ODE in $H(t)$ 
from \eqref{after-char-est}. By Cauchy-Schwartz, we have 
%
%
\begin{equation*}
	\begin{split}
		| H(t) |
		& \le \left( \int_K \left[  
		\p_x u(t, x(t,x)) \right]^2 \ dx
		\right)^{1/2} \left( \int_K 1 \ dx \right)^{1/2}
		\\
		& = |K|^{1/2} \left( \int_K \left[  
		\p_x u(t, x(t,x)) \right]^2 \ dx
		\right)^{1/2} 
	\end{split}
\end{equation*}
%
%
which implies
%
%
\begin{equation}
	\label{cauchy-schwartz-est}
	\begin{split}
		\frac{H(t)^2}{|K|} \le \int_K \left[ \p_x u(t, x(t)) \right]^2 dx .
	\end{split}
\end{equation}
%
%
Applying \eqref{cauchy-schwartz-est} to \eqref{after-char-est}, we obtain the ODE-like 
expression
%
%
\begin{equation}
	\label{ode}
	\frac{dH}{dt} + u(t, x(t)) \p_x H + 
	\frac{H^2}{|K|} - \frac{2\gamma}{\sup x^2}H \le 0.
\end{equation}
%
%
%
%
%
%


\end{document}


