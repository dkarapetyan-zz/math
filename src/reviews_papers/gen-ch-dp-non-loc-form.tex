%
\documentclass[12pt,reqno]{amsart}
\usepackage{amssymb}
\usepackage{cancel}  %for cancelling terms explicity on pdf
\usepackage{yhmath}   %makes fourier transform look nicer, among other things
\usepackage[alphabetic, msc-links]{amsrefs} %for the bibliography; uses cite pkg
%\usepackage{showkeys}  %shows source equation labels on the pdf
\usepackage[margin=3cm]{geometry}  %page layout
%\usepackage[pdftex]{graphicx} %for importing pictures into latex--pdf compilation
\setcounter{secnumdepth}{1} %number only sections, not subsections
\hypersetup{colorlinks=true,
linkcolor=blue,
citecolor=red,
urlcolor=red,
}
\synctex=1
\numberwithin{equation}{section}  %eliminate need for keeping track of counters
\numberwithin{figure}{section}
\setlength{\parindent}{0in} %no indentation of paragraphs after section title
\renewcommand{\baselinestretch}{1.1} %increases vert spacing of text
%
\newcommand{\ds}{\displaystyle}
\newcommand{\ts}{\textstyle}
\newcommand{\nin}{\noindent}
\newcommand{\rr}{\mathbb{R}}
\newcommand{\nn}{\mathbb{N}}
\newcommand{\zz}{\mathbb{Z}}
\newcommand{\cc}{\mathbb{C}}
\newcommand{\ci}{\mathbb{T}}
\newcommand{\zzdot}{\dot{\zz}}
\newcommand{\wh}{\widehat}
\newcommand{\p}{\partial}
\newcommand{\ee}{\varepsilon}
\newcommand{\vp}{\varphi}
%
%
\theoremstyle{plain}  
\newtheorem{theorem}{Theorem}
\newtheorem{proposition}{Proposition}
\newtheorem{lemma}{Lemma}
\newtheorem{corollary}{Corollary}
\newtheorem{claim}{Claim}
\newtheorem{conjecture}[subsection]{conjecture}
%
\theoremstyle{definition}
\newtheorem{definition}{Definition}
%
\theoremstyle{remark}
\newtheorem{remark}{Remark}
%
%
%
\def\makeautorefname#1#2{\expandafter\def\csname#1autorefname\endcsname{#2}}
\makeautorefname{equation}{Equation}
\makeautorefname{footnote}{footnote}
\makeautorefname{item}{item}
\makeautorefname{figure}{Figure}
\makeautorefname{table}{Table}
\makeautorefname{part}{Part}
\makeautorefname{appendix}{Appendix}
\makeautorefname{chapter}{Chapter}
\makeautorefname{section}{Section}
\makeautorefname{subsection}{Section}
\makeautorefname{subsubsection}{Section}
\makeautorefname{paragraph}{Paragraph}
\makeautorefname{subparagraph}{Paragraph}
\makeautorefname{theorem}{Theorem}
\makeautorefname{theo}{Theorem}
\makeautorefname{thm}{Theorem}
\makeautorefname{addendum}{Addendum}
\makeautorefname{add}{Addendum}
\makeautorefname{maintheorem}{Main theorem}
\makeautorefname{corollary}{Corollary}
\makeautorefname{lemma}{Lemma}
\makeautorefname{sublemma}{Sublemma}
\makeautorefname{proposition}{Proposition}
\makeautorefname{property}{Property}
\makeautorefname{scholium}{Scholium}
\makeautorefname{step}{Step}
\makeautorefname{conjecture}{Conjecture}
\makeautorefname{question}{Question}
\makeautorefname{definition}{Definition}
\makeautorefname{notation}{Notation}
\makeautorefname{remark}{Remark}
\makeautorefname{remarks}{Remarks}
\makeautorefname{example}{Example}
\makeautorefname{algorithm}{Algorithm}
\makeautorefname{axiom}{Axiom}
\makeautorefname{case}{Case}
\makeautorefname{claim}{Claim}
\makeautorefname{assumption}{Assumption}
\makeautorefname{conclusion}{Conclusion}
\makeautorefname{condition}{Condition}
\makeautorefname{construction}{Construction}
\makeautorefname{criterion}{Criterion}
\makeautorefname{exercise}{Exercise}
\makeautorefname{problem}{Problem}
\makeautorefname{solution}{Solution}
\makeautorefname{summary}{Summary}
\makeautorefname{operation}{Operation}
\makeautorefname{observation}{Observation}
\makeautorefname{convention}{Convention}
\makeautorefname{warning}{Warning}
\makeautorefname{note}{Note}
\makeautorefname{fact}{Fact}
%
\begin{document}
\title{The Non-Local Form of Two Generalized Camassa-Holm Equations }
\author{David Karapetyan}
\address{Department of Mathematics  \\
         University  of Notre Dame\\
				          Notre Dame, IN 46556 }
									\date{07/20/2010}
									%
									\maketitle
									%
									%
									%
									%
									%
									%
									\section{The Non-local Form for a Model Containing Both CH and
									DP}
We consider the equation
%
%
\begin{equation}
	\label{gen-CH}
	\begin{split}
		u_{t} - u_{xxt} = -\p_x(2ku + \frac{m}{2}u^{2}) + a u_{x}u_{xx} +
		buu_{xxx}
	\end{split}
\end{equation}
%
%
where $b>0$ and $a, k, m$ are arbitrary constants. Next we rewrite
\eqref{gen-CH} in its non-local form.
%
%
\begin{equation*}
	\begin{split}
		(1-\p_x^2)u_{t}
		& = -2k u_{x} - muu_{x} + \frac{a}{2}\p_x u_{x}^2 + b \left[
		\p_x(uu_{xx}) - u_xu_{xx} \right]
		\\
		& = - 2ku_{x} - muu_{x} + \frac{a}{2}\p_x u_{x}^2 + b \p_x(u u_{xx}) -
		\frac{b}{2}\p_x u_{x}^2
		\\
		& = - 2ku_{x} - mu u_{x} + \left( \frac{a}{2} - \frac{b}{2} \right)\p_x
		u_x^2 + b \p_x \left[ \p_x(u u_{x}) - u_{x}^2 \right]
		\\
		& = -2k u_x - mu u_{x} + \left( \frac{a}{2} - \frac{3b}{2} \right)\p_x
		u_{x}^2 + b \p_x^2(u u_{x})
		\\
		& = -2ku_{x} - muu_{x} + \left( \frac{a}{2} - \frac{3b}{2} \right) \p_x
		u_{x}^2 + \frac{b}{2}\p_x^{3}u^{2}
		\\
		& = -2ku_{x} - \frac{m}{2} \p_x u^{2} + \frac{b}{2}\p_x^3 u^{2} + \left(
		\frac{a}{2} - \frac{3b}{2} \right)\p_x u_x^2
		\\
		& = -2k u_{x} - \frac{b}{2}\p_x u^2 + \frac{b-m}{2} \p_x u^{2} +
		\frac{b}{2} \p_x^3 u^2 + \left( \frac{a}{2} - \frac{3b}{2} \right)\p_x
		u_{x}^2
		\\
		& = -\frac{b}{2} \p_x (1-\p_x^2)u^{2} + \p_x \left[ \frac{b-m}{2}u^{2} +
		\left( \frac{a}{2} - \frac{3b}{2} \right )u_{x}^2 - 2ku \right]
	\end{split}
\end{equation*}
%
%
which implies
%
%
\begin{equation}
	\label{gen-CH-non-local}
	\begin{split}
		u_{t} = -buu_{x} - (1- \p_x^2)^{-1} \p_x \left[ \frac{m-b}{2}u^{2} + \left(
		\frac{3b}{2} - \frac{a}{2} \right)u_{x}^2 + 2k u\right].
	\end{split}
\end{equation}
%
%
Note that 
%
%
\begin{equation*}
	\begin{split}
		 m=4, a=3, b=1 \qquad & \text{gives CH}
		\\
		k = 0, m=3, a=2, b=1 \qquad & \text{gives DP}.
	\end{split}
\end{equation*}
%
%
\section{The Non-Local Form for a Generalized Camassa-Holm Equation with
High-Order Non-linearity}
%
%
We consider the equation
%
%
\begin{equation}
	\label{high-order-non-lin-eq}
	\begin{split}
		u_{t} - u_{xxt} = -2k u_{x} - a u^{m}u_{x} + \p_x \left( \frac{n
		u^{n-1}u_{x}^2}{2} + u^{n}u_{xx} \right) + \beta \p_x u_{x}^{2N-1}
	\end{split}
\end{equation}
%
%
where $m \ge 1, n \ge 1$, and $N \ge 1$ are natural numbers, and $a, k$, and
$\beta \ge 0$ are constants. Next we rewrite \eqref{high-order-non-lin-eq} in
its non-local form.
%
%
\begin{equation*}
	\begin{split}
		(1 - \p_x^2)u_{t}
		& = -2ku_x - a u^{m}u_{x} + \p_x \left( \frac{nu^{n-1}u_{x}^2}{2} +
		u^{n}u_{xx} \right) + \beta \p_x u_{x}^{2N-1}
		\\
		& = -2k u_{x} - nu^{m}u_{x} + (n-a)u^{m}u_{x} + \p_x \left(
		\frac{nu^{n-1}u_{x}^2}{2} + u^{n}u_{xx} \right)
		\\
		& = -\frac{n}{m+1}\p_x u^{m+1} + \frac{n-a}{m+1}\p_x u^{m+1} +
		\frac{1}{n+1} \p_x^3 u^{n+1} - \frac{1}{2} \p_x (u^{n-1}u_{x}^2) - 2ku_x
		\\
		& = \p_x \left( -\frac{n}{m+1} u^{m+1} + \frac{1}{n+1} \p_x^{2}u^{n+1}
		\right) + \p_x \left[ \frac{n-a}{m+1}u^{m+1} -
		\frac{1}{2}u^{n-1}u_{x}^2 - 2k \p_x u \right]
	\end{split}
\end{equation*}
%
%
which implies
%
%
\begin{equation}
	\label{high-nl-non-local}
	\begin{split}
		u_t
		& = \p_x (1-\p_x^2)^{-1} \left( -\frac{n}{m+1} u^{m+1} + \frac{1}{n+1} \p_x^{2}u^{n+1}
		\right)
		\\
		& + \p_x (1-\p_x^2)^{-1} \left[ \frac{n-a}{m+1}u^{m+1} -
		\frac{1}{2}u^{n-1}u_{x}^2 - 2k \p_x u \right].
	\end{split}
\end{equation}
%
%
Note that 
%
%
\begin{equation*}
	\begin{split}
		a=3, m=1, n=1, \beta=0 \qquad \text{gives CH}.
	\end{split}
\end{equation*}
%
%
In this case, the first term of \eqref{high-nl-non-local} is simply the 
``Burgers'' term; that is
%
%
\begin{equation*}
	\begin{split}
		& \p_x (1-\p_x^2)^{-1} \left( -\frac{n}{m+1} u^{m+1} + \frac{1}{n+1}
		\p_x^{2}u^{n+1} \right)
		\\
		& = \p_x (1-\p_x^2)^{-1} \left( -\frac{1}{2}u^{2} + \frac{1}{2} \p_x^2 u^{2} \right)
		\\
		& = -\p_x \cancel{(1-\p_x^2)^{-1}} \cancel{(1-\p_x^2)} \left( \frac{1}{2}u^{2} \right) 
		\\
		& = -u \p_x u.
	\end{split}
\end{equation*}
%
%



									%\nocite{*}
									%\bibliography{/Users/davidkarapetyan/Documents/math/}

									\end{document}


