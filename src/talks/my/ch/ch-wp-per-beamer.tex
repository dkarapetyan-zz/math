\documentclass{beamer}
\let\Tiny=\tiny
\usepackage{amscd}
\usepackage{amsfonts}
\usepackage{amsmath}
\usepackage{amssymb}
\usepackage{amsthm}
\usepackage{fancyhdr}
\usepackage{latexsym}
\input epsf
\input texdraw
\input txdtools.tex
\input xy
\xyoption{all}
\usepackage{color}

\definecolor{Red}{rgb}{1.00, 0.00, 0.00}
\definecolor{DarkGreen}{rgb}{0.00, 1.00, 0.00}
\definecolor{Blue}{rgb}{0.00, 0.00, 1.00}
\definecolor{Cyan}{rgb}{0.00, 1.00, 1.00}
\definecolor{Magenta}{rgb}{1.00, 0.00, 1.00}
\definecolor{DeepSkyBlue}{rgb}{0.00, 0.75, 1.00}
\definecolor{DarkGreen}{rgb}{0.00, 0.39, 0.00}
\definecolor{SpringGreen}{rgb}{0.00, 1.00, 0.50}
\definecolor{DarkOrange}{rgb}{1.00, 0.55, 0.00}
\definecolor{OrangeRed}{rgb}{1.00, 0.27, 0.00}
\definecolor{DeepPink}{rgb}{1.00, 0.08, 0.57}
\definecolor{DarkViolet}{rgb}{0.58, 0.00, 0.82}
\definecolor{SaddleBrown}{rgb}{0.54, 0.27, 0.07}
\definecolor{Black}{rgb}{0.00, 0.00, 0.00}
\definecolor{dark-magenta}{rgb}{.5,0,.5}
\definecolor{myblack}{rgb}{0,0,0}
\definecolor{darkgray}{gray}{0.5}
\definecolor{lightgray}{gray}{0.75}
%
\newcommand{\bigno}{\bigskip\noindent}
\newcommand{\ds}{\displaystyle}
\newcommand{\medno}{\medskip\noindent}
\newcommand{\smallno}{\smallskip\noindent}
\newcommand{\nin}{\noindent}
\newcommand{\ts}{\textstyle}
\newcommand{\rr}{\mathbb{R}}
\newcommand{\p}{\partial}
\newcommand{\zz}{\mathbb{Z}}
\newcommand{\cc}{\mathbb{C}}
\newcommand{\ci}{\mathbb{T}}
\newcommand{\tor}{\mathbb{T}}
\newcommand{\ee}{\varepsilon}
\newcommand{\wh}{\widehat}
\newcommand{\weak}{\rightharpoonup}

\def\refer #1\par{\noindent\hangindent=\parindent\hangafter=1 #1\par}

%% Equation Numbers %%

\renewcommand{\theequation}{\thesection.\arabic{equation}}





%%%%%%%%%%%%%%%%%%%%%%

\date{\today}
\title{Well-posedness of the Camassa-Holm Equation on the Torus}
\author{\it David Karapetyan}
\begin{document}
 %
 \begin{frame}
	 \titlepage
 \end{frame}
 
 \begin{frame}
	 \frametitle{Abstract}
	 We consider the Cauchy problem for the Camassa-Holm equation
	 \begin{equation*} 
		 \label{hr}
		 \partial_t u + u\partial_x u + (1-\p_x^2)^{-1} \p_x \left
		 [u^2 + (\p_x u)^2 \right ] = 0,
		 \,\,
		 \ x \in  \mathbb{T},  \;\;  \ t \in \mathbb{R},
	 \end{equation*}
	 \begin{equation*} 
		 \label{hr-data} 
		 u(x, 0) = u_0 (x)
	 \end{equation*}
%
and prove the following result:
\end{frame}

\begin{frame}
%
%
%
\begin{theorem}
	\vskip0.1in
\label{hr-wp}
If $u_0(x) \in  H^s(\ci)$ for some $s >3/2$,  then there is  a $T>0$
depending only on  $\|u_0\|_{H^s}$ such that there exists a unique
function $u(x, t)$ solving  the Cauchy problem
in the sense of distributions with  $u \in C([0, T]; H^s)$.
The solution $u$ depends continuously  on the initial data $u_0$
in the sense that the mapping of the initial data to the solution 
is continuous from the Sobolev space $H^s$ to the space $C([0, T]; H^s)$.
Furthermore, the  lifespan (the maximal existence time)
 is greater than 
%
     \begin{equation*}
   T
   \doteq
   \frac{1}{2c_s}
   \frac{1}{ \|u_0 \|_{H^s(\ci)}},
 \end{equation*}
%
where $c_s$  is a constant depending only on $s$.
Also, we have 
%
  \begin{equation*}
   \label{u-u0-Hs-bound}
\|u(t)\|_{H^s(\ci))}
  \le
  2
  \|u_0 \|_{H^s(\ci)},
  \quad
  0\le t \le T.
   \end{equation*}
  %
\end{theorem}
\end{frame}

\section{The Banach Space ODE Theorem  } 
 \setcounter{equation}{0}

 \begin{frame}
	 \frametitle{The Banach Space ODE Theorem}
%
Our main tool will be the following ODE theorem in Banach spaces:
%
\vskip0.1in
\begin{theorem}
	\label{ode_theorem}
	Let  $Y$  be a Banach space and $X\subset Y$ be an open subset.
	Define $f: J \times X\to Y$ to be a continuously differentiable
	map, where $J \subset \rr$.  Then for any $t_{0}
	\in J$ and $x_{0} \in X$ there exists an
	open ball $I \subset J$ and a unique differentiable mapping $u:I
   \to Y$ such that for all $t \in I$,  $u'(t) = f(t, u(t))$
	and $u(t_{0}) = x_{0}.$
\end{theorem}
\end{frame}

\begin{frame}
	We cannot apply the ODE Theorem for the Camassa-Holm equation as is; for
	arbitrary $u\in H^s(\ci)$, we cannot guarantee $u\p_x u \in H^s(\ci)$.
	Omitting our domain $\ci$ henceforth for clarity, we consider a mollified version of the CH equation:
	\begin{lemma}
		\label{mollified_ch}
Let $s > \frac{3}{2}$ and $f:H^s \to H^s$ be given by 
\begin{equation*}
f_\ee(u) = J_\varepsilon ( J_\varepsilon u \cdot  \partial_x J_\varepsilon u)
+ (1-\p_x^2)^{-1} \p_x \left
[u^2 + \p_x u^2 \right ] 
\end{equation*}
for each $u \in H^s$.  Then $f_\ee$  is a continuously differentiable map.
\end{lemma}

\dots

Applying the lemma in conjunction with the ODE theorem, we see that 
\begin{equation*}
	\p_t u= f_\ee (u)
\end{equation*}
admits a unique family $\{u_\ee\}_\ee$ of solutions.

\end{frame}

\section{Estimates for Life-Span}
\begin{frame}
	\frametitle{Estimates for Life-Span of $u_\ee$}
We will derive the following estimate for our solution $u_\ee$:
  %
\begin{equation*} 
\frac 12
\frac{d}{dt}
  \|u_\ee(t)\|_{H^{s}(\ci)}^2
\le
c_s
 \|u_\ee(t)\|_{H^{s}(\ci)}^3,
 \quad
|t| \le T_\ee.
\end{equation*}
Applying  the operator $D^s$ to  both sides of our equation
then  multiplying the resulting equation by $D^s u_\ee$
and integrating it for $x\in\ci$ gives the following:
\end{frame}

\begin{frame}
	
%
\begin{equation*} 
	\begin{split}
\frac 12
\frac{d}{dt} \|u_\ee \|_{H^s}^2
=
&-
\int_{\ci} D^s(J_\ee u_\ee \partial_x J_\ee u_\ee) \cdot  
D^s J_\ee u_\ee 
\\
&+ D^s u_\ee D^s\p_x(1-\p_x^2)^{-1} \left [u_\ee^2 + \frac12 (\p_x
	 u_\ee)^2 \right] 
	 \, dx
 \end{split}
\end{equation*}
%

For the non-local term, we can arrange


\begin{equation*}
	\int D^s\p_x(1-\p_x^2)^{-1} \left [u_\ee^2 + \frac12 (\p_x
	u_\ee)^2 \right] \le c_s \|u_\ee\|_{H^s}^3.
 \end{equation*}
 \end{frame}
 \begin{frame}
 Set $v=J_\ee u_\ee$; then for the remaining mollified Burgers term we have
\begin{equation*} 
	\begin{split}
\label{B-moli-int-v}
- \int_{\ci} D^s (J_{\ee} u_{\ee} \p_x J_\ee u_\ee) \cdot D^s
J_{\ee}u_\ee \; dx
= &-\int_\ci
     D^s(v \partial_x v) \cdot   D^s v \, dx
\\
=& -\int_\ci
\big[ 
D^s(v\p_x v)  -  v D^s (\p_xv)
\big]
 D^s v   \, dx
 \\
&
       -\int_\ci
  v D^s (\p_xv)
 D^sv\, dx.
 \end{split}
\end{equation*}
%
%
%
To estimate the middle term we use the Kato-Ponce estimate; the remaining terms
are handled using Cauchy-Schwartz and the algebra property. The result is
\begin{equation*} 
- \int_{\ci} D^s (J_{\ee} u_{\ee} \p_x J_\ee u_\ee) \cdot D^s
J_{\ee}u_\ee \; dx \le c_s \|u_\ee\|_{H^s}^3
 \end{equation*}
 \end{frame}
 \begin{frame}
 Combining the estimates for the local and the non-local term of our mollified
 I.V.P we obtain the desired differential inequality
  %
\begin{equation*} 
\frac 12
\frac{d}{dt}
  \|u_\ee(t)\|_{H^{s}(\ci)}^2
\le
c_s
 \|u_\ee(t)\|_{H^{s}(\ci)}^3,
 \quad
|t| \le T_\ee.
\end{equation*}
Differentiating the left hand side, simplifying and applying some
elementary calculus, we see the $\{u_\ee\}_\ee$ have common lifespan
\begin{equation}
	T \doteq \frac{1}{2 c_s} \frac{1}{\|u_0\|_{H^s(\ci)}}
\end{equation}
with 
\begin{equation}
	\|u_\ee (t)\|_{H^s(\ci)} \le 2 \|u_0\|_{H^s(\ci)}.
\end{equation}

 \end{frame}
 \section{Existence of Solution}
 \begin{frame}

 \frametitle{Existence of Solution}
 \vskip0.1in
 We will now show that our family of $\{u_\ee\}$ has a convergent subsequence
 whose limit $u$ solves the Camassa-Holm I.V.P. 

 Let
 $$
 I= [0, T].
 $$
By our previous work, we have 
%
\begin{equation*}
\label{C-1-fam}
\{u_\ee\}\subset C(I; H^{s})\cap C^1(I; H^{s-1})
\end{equation*}
%
and bounded. Therefore,  
 %
 \begin{equation*}
\label{Lip-1-fam}
\{u_\ee\}\subset L^{\infty}(I; H^{s})\cap Lip(I; H^{s-1})
\end{equation*}
%
and bounded. Observe that  $L^\infty (I, H^s) $
is the dual space of $L^1(I, H^{-s})$; therefore by Alaoglu 
the family $\{u_\ee\}$ is compact in the weak$^*$ topology, and hence
converges weakly to some $u \in L^{\infty}(I, H^s) \; \bigcap\; Lip(I,
H^{s-1})$.  While helpful, this is not enough; we will soon prove that
\begin{equation*}
\label{strong-conv}
u_{\ee_n}\longrightarrow u
\quad
\text{ in } \,\,   C(I, C^1)
\end{equation*}
\end{frame}
\begin{frame}
which, when used with our weak convergence result, will yield a unique solution
$u$ to our I.V.P. To prove convergence in $C(I,C^1)$, we will need the
following:
%%%%%%%%%%%%%%%%%%%%%%%%%%%
%
%
%                 Interpolation Lemma
%
%
%%%%%%%%%%%%%%%%%%%%%%%%%%%

\begin{lemma}
	\vskip0.1in
(Interpolation)  Let  $s > \frac{3}{2}$.
If $f \in C(I, H^s) \cap C^1(I, H^{s-1})$
then $f \in C^\sigma (I, H^{s- \sigma})$ for  $0 < \sigma < 1$.
\end{lemma}
%
%
\vskip0.1in
Using this lemma it can be shown that the family $\{u_\ee\}$ is equicontinuous
in $C(I,C^1)$. To apply Arzela-Ascoli, we recall that our family is uniformly
bounded, and note that the inclusion map is a
compact operator from $H^s(\ci) \to H^{s-\sigma}(\ci)$; choosing sufficiently
small $\sigma$ and applying the Sobolev Imbedding Theorem, we conclude
$\{u_\ee(t)\}_\ee$ is compact in $C^1(\ci)$. 
\end{frame}
\begin{frame}
	Hence, we have
	\begin{equation*}
		u_\ee \overset{*}{\rightharpoonup} u
	\end{equation*}
	and 
	\begin{equation*}
		u_\ee \to u \; \; \text{in} \; \; C(I, C^1)
	\end{equation*}
	from which it follows that u is the unique solution
	to our Camassa-Holm I.V.P.
 \end{frame}
 \begin{frame}
	 \section{Uniqueness}
	 \frametitle{Uniqueness}
	Using an $H^1$ energy estimate, the following can be shown:
	\vskip0.1in
	\begin{theorem}
		Let $v \in C(I, H^s)$ be a solution to the Camassa-Holm I.V.P with
		initial data $v(0)= v_0$, and let $\{u_\ee\}_\ee$ be a family of
		solutions to the mollified I.V.P with common initial data
		$u_\ee(0)=v_0$. Then
		\begin{equation*}
			\|v -u_\ee\|_{H^1} \to 0
		\end{equation*}
	\end{theorem}
 \end{frame}
 \begin{frame}
 \section{Continuous Dependence}
     \frametitle{Continuous Dependence}
 \begin{theorem}
	 \vskip0.1in
     The solution map from initial data $u_0 \in H^s$ to solution $u\in
	 C(I,H^s)$
     of the Camassa-Holm equation is a continous map.
 \end{theorem}
 \end{frame}
 
 \section{Thank You}
 \begin{frame}
	 \frametitle{\center{Thank You For Coming}}
\end{frame} 
 \end{document}


