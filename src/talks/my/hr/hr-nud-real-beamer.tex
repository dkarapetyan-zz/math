\documentclass{beamer}
%\let\Tiny=\tiny
\usepackage{amscd}
\usepackage{amsfonts}
\usepackage{amsmath}
\usepackage{amssymb}
\usepackage{amsthm}
\usepackage{fancyhdr}
\usepackage{latexsym}
\usepackage{lmodern}
\synctex=1
\input epsf
\input texdraw
\input txdtools.tex
\input xy
\xyoption{all}
\usepackage{color}

\definecolor{Red}{rgb}{1.00, 0.00, 0.00}
\definecolor{DarkGreen}{rgb}{0.00, 1.00, 0.00}
\definecolor{Blue}{rgb}{0.00, 0.00, 1.00}
\definecolor{Cyan}{rgb}{0.00, 1.00, 1.00}
\definecolor{Magenta}{rgb}{1.00, 0.00, 1.00}
\definecolor{DeepSkyBlue}{rgb}{0.00, 0.75, 1.00}
\definecolor{DarkGreen}{rgb}{0.00, 0.39, 0.00}
\definecolor{SpringGreen}{rgb}{0.00, 1.00, 0.50}
\definecolor{DarkOrange}{rgb}{1.00, 0.55, 0.00}
\definecolor{OrangeRed}{rgb}{1.00, 0.27, 0.00}
\definecolor{DeepPink}{rgb}{1.00, 0.08, 0.57}
\definecolor{DarkViolet}{rgb}{0.58, 0.00, 0.82}
\definecolor{SaddleBrown}{rgb}{0.54, 0.27, 0.07}
\definecolor{Black}{rgb}{0.00, 0.00, 0.00}
\definecolor{dark-magenta}{rgb}{.5,0,.5}
\definecolor{myblack}{rgb}{0,0,0}
\definecolor{darkgray}{gray}{0.5}
\definecolor{lightgray}{gray}{0.75}
%
\newcommand{\bigno}{\bigskip\noindent}
\newcommand{\ds}{\displaystyle}
\newcommand{\medno}{\medskip\noindent}
\newcommand{\smallno}{\smallskip\noindent}
\newcommand{\nin}{\noindent}
\newcommand{\ts}{\textstyle}
\newcommand{\rr}{\mathbb{R}}
\newcommand{\p}{\partial}
\newcommand{\zz}{\mathbb{Z}}
\newcommand{\cc}{\mathbb{C}}
\newcommand{\ci}{\mathbb{T}}
\newcommand{\tor}{\mathbb{T}}
\newcommand{\ee}{\varepsilon}
\newcommand{\wh}{\widehat}
\newcommand{\weak}{\rightharpoonup}
\newcommand{\vp}{\varphi}
%
%
\newtheorem{proposition}{Proposition}
\newtheorem{claim}{Claim}
\newtheorem{remark}{Remark}
\newtheorem{conjecture}[subsection]{conjecture}

\def\refer #1\par{\noindent\hangindent=\parindent\hangafter=1 #1\par}

%% Equation Numbers %%

\renewcommand{\theequation}{\thesection.\arabic{equation}}





%%%%%%%%%%%%%%%%%%%%%%

\date{}
\title{Sharp Well-Posedness for the Hyperelastic Rod Equation}
\author{\it David Karapetyan}
\begin{document}
 %
 \begin{frame}
	 \titlepage
 \end{frame}

\section*{Table of Contents}
\begin{frame}
	\frametitle{Table Of Contents}
	\tableofcontents
\end{frame}

\section{Introduction}
 \begin{frame}
	 \frametitle{Introduction}
	 We consider the Cauchy problem for the non-periodic Hyperelastic Rod (HR) equation
	 \begin{equation*} 
		 \label{hr}
		 \partial_t u + \gamma u\partial_x u + (1-\p_x^2)^{-1} \p_x \left
		 [\frac{3- \gamma}{2} u^2 + \frac{\gamma}{2}(\p_x u)^2 \right ] = 0,
		 \ x \in  \mathbb{R},  \ t \in \mathbb{R},
	 \end{equation*}
	 \begin{equation*} 
		 \label{hr-data} 
		 u(x, 0) = u_0 (x)
	 \end{equation*}
%
and prove the following result.
\end{frame}
\begin{frame}
%
%
%
\begin{theorem}
\label{hr-non-unif-dependence}
Let $\gamma$ be a nonzero constant. Then 
the data-to-solution map $u(0) \mapsto u(t)$ of the Cauchy-problem
for the HR equation is not uniformly continuous
from any bounded subset of  $H^s$ into $C([-T, T], H^s)$
for $s>1$ on the line.
%
\end{theorem}
\end{frame}
%
%
\section{Strategy}
\begin{frame}
	\frametitle{Strategy}
We begin by outlining the method of the proof,
as it has been applied for the case $\gamma=1$ in the work of
Himonas and Kenig. 
We will show that there there exist two sequences of solutions 
$u_n(t)$
and $v_n(t)$ in $C([-T, T], H^s)$ such that
\pause
%
%
%
%
\begin{equation*}
\label{h-s-bdd}
\| u_n(t)  \|_{H^s}
+
\| v_n(t)  \|_{H^s}
\lesssim
1,
\end{equation*}
%
%
%
%
%
\begin{equation*}
\label{zero-limit-at-0}
\lim_{n\to\infty}
\|
u_n(0)
-
v_n(0)
\|_{H^s}
=
0,
\end{equation*}
%
%
%
%
and
%
%
%
%
\begin{equation*}
\label{bdd-away-from-0}
\liminf_{n\to\infty}
\|
u_n(t)
-
v_n(t)
\|_{H^s}
\gtrsim
|\sin ( \gamma t)|,
\quad
| \gamma t|\le 1.
\end{equation*}%
%
%
\pause
We accomplish this in two steps.
First, we will construct two sequences of approximate solutions
satisfying the above properties.
Then, we will construct two sequences of actual solutions 
coinciding with the approximate solutions at time zero.
The key point of this method is that 
the difference between solutions and approximate solutions
must decay.
\end{frame}
%
%
\section{Well-Posedness Theorem}
\begin{frame}
	\frametitle{Well-Posedness}
We will require the following.
\begin{theorem}
\label{hr-wp}
If $u_0(x) \in  H^s$ for some $s >3/2$,  then there is  a $T>0$
depending only on  $\|u_0\|_{H^s}$ such that there exists a unique
function $u(x, t)$ solving  the HR Cauchy problem
in the sense of distributions with  $u \in C([0, T]; H^s)$.
The solution $u$ depends continuously on the initial data $u_0$
in the sense that the mapping of the initial data to the solution 
is continuous from the Sobolev space $H^s$ to the space $C([0, T]; H^s)$.
Furthermore, the  lifespan (the maximal existence time)
 is greater than 
%
     \begin{equation*}
   T
   \doteq
   \frac{1}{2c_s}
   \frac{1}{ \|u_0 \|_{H^s(\ci)}},
 \end{equation*}
%
where $c_s$  is a constant depending only on $s$.
Also, we have 
%
  \begin{equation*}
   \label{u-u0-Hs-bound}
\|u(t)\|_{H^s(\ci))}
  \le
  2
  \|u_0 \|_{H^s(\ci)},
  \quad
  0\le t \le T.
   \end{equation*}
  %
\end{theorem}
\end{frame}
%
%
%
\section{Approximate solutions on the Line}
\begin{frame}
	\frametitle{Approximate Solutions on the Line}
Following Himonas and Kenig, our approximate solutions
\\ $u^{\omega, \lambda} = u^{\omega,
\lambda}(x,t)$ will
consist of a low frequency and a high frequency part,
i.e.
%
%
%
%
\begin{equation*}
\label{apple1}
u^{\omega,\lambda} = u_\ell + u^h
\end{equation*}
%
%
%
%
where $\omega$ is in a bounded set of $\rr$ and $\lambda > 0$. The high frequency part is given by 
\pause
%
%
%
%
\begin{equation*}
\begin{split}
u^h = u^{h,\omega,\lambda}(x,t) =
\lambda^{-\frac{\delta}{2} -s}
\phi \left (\frac{x}{\lambda^\delta}\right )
\cos(\lambda x - \gamma \omega t)
\end{split}
\end{equation*}
%
%
%
%
where $\phi$ is a $C^\infty$ cut-off function such that
%
%
%
%
\begin{equation*}
\phi = \begin{cases}
1, &\text{if $|x|<1$,} \\
0, &\text{if $|x| \ge 2,$} \end{cases}
\end{equation*}
%
%
%
%
\pause
and by the existence portion of well-posedness,
we let the low frequency part $u_\ell = u_{l,
\omega, \lambda}(x,t)$ be the unique solution to the Cauchy problem
\end{frame}
%
%
\begin{frame}
\begin{align*}
& \p_t u_\ell = -\gamma u_\ell \p_x u_\ell -
\Lambda^{-1} \left[ \frac{3-\gamma}{2}(u_\ell)^2 +
\frac{\gamma}{2} \left( \p_x u_\ell \right)^2
\right],
\\
& u_\ell(x,0) = \omega \lambda^{-1} \tilde{\phi} \left(
\frac{x}{\lambda^{\delta}}
\right), \quad x \in \rr, \quad t \in \rr
\end{align*}
%
%
%
%
where $\tilde{\phi}$ is a $C^{\infty}_0(\rr)$ function such that
%
%
%
%
\begin{equation*}
\label{apple1***}
\tilde{\phi}(x) = 1 \; \;  \text{if} \; \;
x \in \text{supp} \; \phi.
\end{equation*}
\end{frame}
%
%
\section{Error of Approximate Solutions on the Line}
\begin{frame}
	\frametitle{Error of Approximate Solutions on the Line}
Substituting the
approximate solution $u^{\omega, \lambda} = u_\ell + u^h$ into the HR
equation, we see that the error
$E$ of our approximate solution is given by
%
%
\begin{equation*}
E=E_1 + E_2 + \dots + E_8
\end{equation*}
%
%
\pause
where
%
%
\begin{equation*}
\label{all_errors_together}
\begin{split}
E_1 & = \gamma \lambda^{1 -\frac{\delta}{2}-s}  \left[ u_\ell(x,0) - u_\ell(x,t)
\right] \phi\left(
\frac{x}{\lambda^ \delta}
\right)\sin(\lambda x - \gamma \omega t),
\\
E_2 & = \gamma \lambda^{-\frac{3\delta}{2}-s}
u_\ell(x,t) \cdot \phi'\left( \frac{x}{\lambda^\delta} \right)\cos\left( \lambda
x - \gamma \omega t
\right),
\\
E_3 & = \gamma u^h \p_x u_\ell, \; \; E_4 = \gamma u^h \p_x u^h, \ E_5  = 
 \frac{3-\gamma}{2} \Lambda^{-1} \left[  \left( u^h \right)^2 \right], \\
E_6 & = (3- \gamma)\Lambda^{-1}
  \left[ u_\ell u^h \right], \  E_7 = \frac{\gamma}{2} \Lambda^{-1} \left[ 
 \left(
\p_x u^h \right)^2 \right ], \; \;
\\
E_8 & = \gamma \Lambda^{-1} \left[  \p_x u_\ell \p_x u^h \right]
.
\end{split}
\end{equation*}
%
%
\end{frame}
%
%
%
\begin{frame}
%
%
\begin{proposition}
Let $1<\delta<2$. Then for $s > 1$, bounded $\omega$, and
$\lambda >>1$ we are assured the decay of the error $E$ of the
approximate solutions to the HR equation. Specifically
%
%
%
\begin{equation*}
\label{E-est}
\|E(t)\|_{H^1(\rr)} \lesssim \lambda^{\frac{\delta}{2} -s}, \quad |t| \le 
T.
\end{equation*}
%
%
%
\end{proposition}
%
%
\end{frame}
%
%
\section{Construction of Solutions on the Line}
\begin{frame}
	\frametitle{Construction of Solutions on the Line}
We wish now to estimate the difference between approximate and actual 
solutions to
the HR ivp with common initial data. Let
$u_{\omega,\lambda}(x,t)$ be the unique solution to the HR equation
with initial data $u^{\omega,\lambda}(x,0)$. That is,
$u_{\omega,\lambda}$ solves the initial value problem
\begin{gather*}
 \p_t u_{\omega,\lambda} = - \gamma u_{\omega,\lambda} \p_x 
u_{\omega,\lambda} - \Lambda^{-1} \left[
\frac{3- \gamma}{2}\left( u_{\omega,\lambda} \right)^2 + 
\frac{\gamma}{2}\left(
\p_x u_{\omega,\lambda} \right)^2
\right], 
\\
 u_{\omega,\lambda}(x, 0) = u^{\omega,\lambda}(x,0) = \omega \lambda^{-1}
\tilde{\phi} \left( \frac{x}{\lambda^\delta} \right)
+ \lambda^{-\frac{\delta}{2} -s}
\phi\left( \frac{x}{\lambda^\delta} \right) \cos(\lambda x).
\end{gather*}
%
%
%
We will now prove that the $H^1(\rr)$ norm of the difference decays: 
%
\end{frame}
%%%%%%%%%%%%%%%%%%%%%%%%%%%%%%%%%%
%
%
%
%
%
%    : H^1 bound_for_difference-of-approx-and-actual-soln
%
%
%
%
%
%
%%%%%%%%%%%%%%%%%%%%%%%%%%%%%%%%%
%
\begin{frame}
%
%
\begin{proposition}
\label{applelem:bound_for_difference-of-approx-and-actual-soln}
%
Let $v = u^{\omega,\lambda} - u_{\omega,\lambda}$, with $\lambda >>1$.
Then, for $s > 1$ and $1<\delta<2$ we have
%
%
\begin{equation*} \|
v(t)
\|_{H^1(\rr)}
\lesssim \lambda^{\frac{\delta}{2} -s}, \quad
|t| \le T.
\end{equation*}
%
%
\end{proposition}
%
%
\pause
{\bf   Proof.}
First we observe that $v$ satisfies 
%
%
\begin{equation*}
\begin{split}
\p_t v & = E + \gamma(v \p_x v - v \p_x u^{\omega,\lambda} - 
u^{\omega,\lambda} \p_x v) \\
& + \Lambda^{-1}  \left[ \frac{3-
\gamma}{2}v^2 + \frac{\gamma}{2}\left( \p_x v \right)^2 - \left(
3 - \gamma \right)u^{\omega,\lambda} v -
\gamma \p_x u^{\omega,\lambda} \p_x v \right].
\end{split}
\end{equation*}
%%
\pause
It follows  that
\end{frame}
\begin{frame}
		\begin{equation*}
		\label{appleenergy-est*}
		\begin{split}
			&\frac{1}{2} \frac{d}{dt} \|v\|_{H^1(\rr)}^2  
			\\
		& =  \int_{\rr} \left[ v(1-\p_x^2)E \right]dx
		\\
		& - \gamma \int_{\rr} \left[ v(1-\p_x^2)(v\p_x u^{\omega,\lambda} + u^{\omega,\lambda} \p_x v) \right]dx
		\\
		&- \int_{\rr}\left[ \left( 3-\gamma \right)v \p_x\left( u^{\omega,\lambda}v \right) + \gamma v
		\p_x \left( \p_x u^{\omega,\lambda} \p_x v \right)\right]dx
		\\
		&+  \int_{\rr}
		\left[ \gamma v \left( 1-\p_x^2 \right)\left( v \p_x v \right) + v
		\p_x \left( \frac{3-\gamma}{2} v^2 + \frac{\gamma}{2}\left( \p_x v \right)^2
		\right) \right . +  v \p_x^2 \p_t v 
		\\
		& + \p_x v \p_t \p_x v\bigg]dx.
	\end{split}
\end{equation*}
\end{frame}
%
\begin{frame}
Noting that the the last integral can be rewritten as 
\begin{equation*}
	\begin{split}
	\int_{\rr} \left[ \p_x (v^3) - \gamma \p_x (v^2 \p_x^2 v) + \p_x\left( v \p_t
	\p_x v
	\right) \right]dx  = 0
\end{split}
\end{equation*}
%
we can simplify to obtain
%
%
\begin{equation*}
\label{appleenergy-est}
\begin{split}
\frac{1}{2} \frac{d}{dt} \|v\|_{H^1(\rr)}^2  
& = 
 \int_{\rr} \left[ v(1-\p_x^2)E \right]dx\\
 &-
 \gamma \int_{\rr} \left[ v(1-\p_x^2)(v\p_x u^{\omega,\lambda} + 
u^{\omega,\lambda} \p_x v) \right]dx
\\
&- \int_{\rr}\left[ \left( 3-\gamma \right)v \p_x\left( u^{\omega,\lambda}v 
\right) + \gamma v
\p_x \left( \p_x u^{\omega,\lambda} \p_x v \right)\right]dx.
\end{split}
\end{equation*}
%
%
\pause
We now estimate the three integrals on the right-hand side. Integrating by parts and applying Cauchy-Schwartz,  
we obtain
%
%
%
%
%
\begin{equation*}
\begin{split}
\label{appleenergy-estimate-best}
\frac{d}{dt} \|v(t)\|_{H^1(\rr)}^2
& \lesssim \left( \|u^{\omega,\lambda}\|_{L^\infty(\rr)} + \|
\p_x u^{\omega,\lambda} \|_{L^\infty(\rr)} + \|\p_x^2 u^{\omega,\lambda} 
\|_{L^\infty (\rr)} \right)
\\
& \times \|v\|_{H^1(\rr)}^2 + \|v\|_{H^1(\rr)} \|E\|_{H^1(\rr)}.
\end{split}
\end{equation*}
%
%
Assume $\lambda >>1$. A straightforward calculation of derivatives yields
%
%
\begin{equation*}
\begin{split}
\|u^h\|_{L^\infty(\rr)} + \|\p_x u^h\|_{L^\infty(\rr)} + \|\p_x^2
u^h\|_{L^\infty(\rr)} \lesssim \lambda^{- \frac{\delta}{2} - s +2 }.
\label{apple53}
\end{split}
\end{equation*}
%
%
\end{frame}
%
%
\begin{frame}
Furthermore, by the Sobolev Imbedding Theorem and our well-posedness energy 
estimate, we have
%
%
%
%
\begin{equation*}
\begin{split}
\|u_\ell\|_{L^\infty(\rr)} + \|\p_x u_\ell \|_{L^\infty(\rr)} + \|\p_x^2
u_\ell\|_{L^\infty(\rr)}
& \le c_s \|u_\ell\|_{H^3(\rr)} 
 \lesssim \lambda^{-1 + \frac{\delta}{2}}. 
\label{apple55}
\end{split}
\end{equation*}
%
%
Hence
%
%
\begin{equation*}
\begin{split}
\|u^{\omega,\lambda}\|_{L^\infty(\rr)} + \|\p_x 
u^{\omega,\lambda}\|_{L^\infty(\rr)} + \|\p_x^2
u^{\omega,\lambda}\|_{L^\infty(\rr)}
& \lesssim \lambda^{-\rho_s}, \quad |t| \le T
\label{apple56}
\end{split}
\end{equation*}
%
%
where $\rho_s = \text{min} \Big\{ \frac{\delta}{2} + s -2, \; 1-
\frac{\delta}{2} \Big\}$.  Note that for $s>1$, we can assure $\rho_s > 0$
by choosing a suitable $1<\delta<2$.
Therefore, 
%
\begin{equation*}
\label{apple58}
\frac{d}{dt} \|v(t)\|_{H(\rr)}^2 \lesssim \lambda^{-\rho_s}
\|v\|_{H^1(\rr)}^2 + \lambda^{\frac{\delta}{2} -s}
\|v \|_{H^1(\rr)}, \quad |t| \le T
\end{equation*}
%
%
and applying Gronwall's Inequality completes the proof. \qquad \qedsymbol%
%
%
\end{frame}

\section{Non-Uniform Dependence for $s>1$ on the Line}
\begin{frame}
	\frametitle{Non-Uniform Dependence for $s>1$ on the Line}

Let $u_{\pm 1,\lambda}$ be solutions to the HR ivp with initial 
data $u^{\pm 1,
\lambda}(0)$. We wish to show that the $H^s$ norm of the difference of $u_{\pm 1,
\lambda}$ and the associated approximate solution $u^{\pm 1,\lambda}$
decays as $\lambda \to \infty$. Note that
\pause
%
%
\begin{equation*}
\begin{split}
\label{apple62}
 \|u^{\pm 1, \lambda}(t)\|_{H^{2s-1}(\rr)}
 & \le \|u_{\ell, \pm 1, \lambda}\|_{H^{2s-1}(\rr)}
\\
& +
\| \lambda^{-\frac{\delta}{2} -s} \phi \left(
\frac{x}{\lambda^\delta} \right) \cos(\lambda x \mp \gamma \omega t)
\|_{H^{2s-1}(\rr)}
\\
& \lesssim \lambda^{s-1}, \quad |t| \le T
\end{split}
\end{equation*}
%
%
and
%
\begin{equation*}
\begin{split}
\|u_{\pm 1,\lambda} (t) \|_{H^{2s-1}(\rr)}
& \le 2 \|u^{\pm 1,\lambda}(0) \|_{H^{2s-1}(\rr)}, \quad
|t| \le T.
\label{apple60}
\end{split}
\end{equation*}
%
%
%
%
\pause
Hence
%
\begin{equation*}
\begin{split}
\|u^{\pm 1, \lambda}(t) - u_{\pm 1, \lambda}(t) \|_{H^{2s-1}(\rr)}
\lesssim \lambda^{s-1}, \quad |t| \le T.
\label{apple63}
\end{split}
\end{equation*}
%
%
\end{frame}

\begin{frame}

Furthermore, by the proposition we just proved 
%
%
\begin{equation*}
\begin{split}
\|u^{\pm 1, \lambda}(t) - u_{\pm 1, \lambda} \|_{H^1(\rr)} \lesssim
\lambda^{\frac{\delta}{2} -s}, \quad |t| \le T.
\label{apple64}
\end{split}
\end{equation*}
%
%
%
\pause
%
%
%
%
%
Interpolating using 
the inequality
\begin{equation*}
\label{apple403}
\|\psi \|_{H^s (\rr)} \leq  (\| \psi \|_{H^1 (\rr)} \| \psi
\|_{H^{2s-1}(\rr)})^\frac12
\end{equation*}
%
%
gives
%
%
\begin{equation*}
\begin{split}
\|u^{\pm 1, \lambda}(t) - u_{\pm 1, \lambda}(t)
\|_{H^s(\rr)}
\lesssim \lambda^{\frac{\delta -2}{4}}, \quad |t| \le T.
\label{apple65}
\end{split}
\end{equation*}
%
%
Next, we will use this estimate to prove non-uniform
dependence when $s > 1$.
%%%%%%%%%%%%% Behavior at time  t = 0  %%%%%%%%%%%% 

%
\end{frame}

\begin{frame}
	\frametitle{Behavior at Time $t=0$}  We have
%
%
%
%
\begin{equation*}
\begin{split}
\|u_{1,\lambda}(0) - u_{-1,\lambda}(0) \|_{H^s(\rr)} & = \|u^{1,\lambda}(0) 
- u^{-1,\lambda}(0) \|_{H^s(\rr)}
\\
& = 2 \lambda^{-1} \| \tilde{\phi}\left( \frac{x}{\lambda^\delta} \right) 
\|_{H^s(\rr)}
\\
& = 2
\lambda^{\frac{\delta}{2}-1} \|\tilde{\phi} \|_{H^s(\rr)} \to 0
\; \; \text{as} \; \; \lambda \to \infty.
\end{split}
\end{equation*}


\end{frame}
%
%

\begin{frame}
	\frametitle{ Behavior at time  $t>0$}
%
%
%%%%%%%%%%%%%% Behavior at time  t >0  %%%%%%%%%%%% 
%  
%

Using the reverse triangle inequality, we 
have
%
%
%
%
%
\begin{equation*} \label{appleHR-slns-differ-t-pos}
\begin{split}
\|
u_{1,\lambda}(t)
-
u_{- 1,\lambda}(t)
\|_{H^s(\rr)}
&
\ge
\|
u^{1,\lambda}(t)
-
u^{- 1,\lambda}(t)
\|_{H^s(\rr)}
\\
& -
\|
u^{1,\lambda}(t)
-
u_{1,\lambda}(t)
\|_{H^s(\rr)}
\\
& -
\|
-u^{-1,\lambda}(t)
+
u_{-1,\lambda}(t)
\|_{H^s(\rr)} .
\end{split}
\end{equation*}
%
%
%
%
from which it follows that
%
%
%
%
\begin{equation*} \label{appleHR-slns-differ-t-pos-est}
\|
u_{1,\lambda}(t)
-
u_{- 1,\lambda}(t)
\|_{H^s(\rr)}
\ge
\|
u^{1,\lambda}(t)
-
u^{- 1,\lambda}(t)
\|_{H^s(\rr)}
-
c \lambda^{\frac{\delta - 2}{4}}
\end{equation*}
%
%
where c is a positive, non-zero constant. Letting $\lambda$ go to $\infty$ 
then yields
%
%
%
\begin{equation*} \label{appleHR-slns-to-ap-est}
\liminf_{n\to\infty}
\|
u_{1,\lambda}(t)
-
u_{- 1,\lambda}(t)
\|_{H^s(\rr)}
\ge
\liminf_{n\to\infty}
\|
u^{1,\lambda}(t)
-
u^{- 1,\lambda}(t)
\|_{H^s(\rr)}.
\end{equation*}
%
%
%
\end{frame}

\begin{frame}
%
Using the identity $$
\cos \alpha -\cos \beta
=
-2
\sin(\frac{\alpha + \beta}{2})
\sin(\frac{\alpha - \beta}{2})
$$
gives
%
%
\begin{equation*}
\label{apple80}
\begin{split}
u^{1,\lambda}(t)
-
u^{- 1,\lambda}(t)
& =
u_{\ell,1,\lambda}(t) - u_{\ell,-1,\lambda}(t)
\\
& + 
2\lambda^{-\frac{\delta}{2}-s}
\phi\left( \frac{x}{\lambda^\delta} \right)\sin(\lambda x) \sin(\gamma t).
\end{split}
\end{equation*}
%
%
Now, by our well-posedness energy estimate we have
%
%
\begin{equation*}
\begin{split}
\|u_{\ell,1,\lambda}(t) - u_{\ell,-1,\lambda}(t)\|_{H^s(\rr)} \lesssim
\lambda^{-1 + \frac{\delta}{2}}.
\end{split}
\end{equation*}
%
%
Hence, applying the reverse triangle inequality we 
obtain
%
%
\begin{equation*} 
\begin{split}
& \|
u^{1,\lambda}(t)
-
u^{- 1,\lambda}(t)
\|_{H^s(\rr)}
\\
& \ge 2 \lambda^{-\frac{\delta}{2}-s} \|\phi\left(
\frac{x}{\lambda^\delta} \right) \sin(\lambda x) \|_{H^s(\rr)} |\sin \gamma 
t|
- \|u_{\ell,-1,\lambda}(t) - u_{\ell,1,\lambda}(t)\|_{H^s(\rr)} \\
& \gtrsim \lambda^{-\frac{\delta}{2}-s} \|\phi\left(
\frac{x}{\lambda^\delta} \right ) \sin(\lambda x) \|_{H^s(\rr)} |\sin 
\gamma t| -
\lambda^{-1 + \frac{\delta}{2}}.
\end{split}
\end{equation*}
%
%
%
\end{frame}

\begin{frame}
%
Letting $\lambda$ go to $\infty$,  
gives
%
%
%
%
\begin{equation*} \label{apple91}
\liminf_{\lambda \to\infty}
\|
u^{1,\lambda}(t)
-
u^{- 1,\lambda}(t)
\|_{H^s(\rr)}
\gtrsim
|\sin \gamma t|, \quad |t| \le T.
\end{equation*}
%
%
This completes 
the proof of non-uniform dependence on the line. \qquad \qedsymbol
\end{frame}


\begin{thebibliography}{HKM09}

		\providecommand{\bysame}{\leavevmode\hbox to3em{\hrulefill}\thinspace}
\providecommand{\MR}{\relax\ifhmode\unskip\space\fi MR }
% \MRhref is called by the amsart/book/proc definition of \MR.
\providecommand{\MRhref}[2]{%
  \href{http://www.ams.org/mathscinet-getitem?mr=#1}{#2}
}
\providecommand{\href}[2]{#2}


\bibitem[HK09]{Himonas_2009_Non-uniform-dep}
Alex Himonas and Carlos~E. Kenig, \emph{Non-uniform dependence on initial data
  for the ch equation on the line}, Differential Integral Equations \textbf{22}
  (2009), no.~3-4, 201--224.


\bibitem[Tay91]{Taylor_1991_Pseudodifferent}
Michael~Eugene Taylor, \emph{Pseudodifferential operators and nonlinear {PDE}},
  Progress in Mathematics, vol. 100, Birkh{\"a}user Boston Inc., Boston, MA,
  1991. 


\end{thebibliography}


%\bibliographystyle{amsalpha}
%\bibliography{/Users/davidkarapetyan/Documents/math/references.bib}
%\nocite{*}

 \end{document}


