\documentclass[12 pt]{article}
\usepackage{amsmath,amsfonts,amsthm,amscd}
\newtheorem{thm}{Theorem}[section]
\newtheorem{pro}[thm]{Proposition}
\newtheorem{cor}[thm]{Corollary}
\newtheorem{lem}[thm]{Lemma}
\newtheorem{conj}[thm]{Conjecture}
\newtheorem{ques}[thm]{Question}
\newtheorem{statement}[thm]{Statement}
\newtheorem{claim}[thm]{Claim}
\newtheorem{ex}[thm]{Example}
\newtheorem{rem}[thm]{Remark}
\newtheorem{defn}[thm]{Definition}
\newtheorem{prob}[thm]{Problem}
\newtheorem{quot}[]{Result}
\newcommand{\dime}{\operatorname{dim}}
\newcommand{\Obj}{\operatorname{Obj}}
\newcommand{\Mor}{\operatorname{Mor}}
\newcommand{\chan}{\operatorname{char}}
\newcommand{\degr}{\operatorname{deg}}
\newcommand{\expo}{\operatorname{exp}}
\newcommand{\spant}{\operatorname{span}}
\newcommand{\res}{\operatorname{res}}
\newcommand{\Term}{\operatorname{Term}}
\newcommand{\Init}{\operatorname{Init}}
\newcommand{\spec}{\operatorname{spec}}
\newcommand{\Endo}{\operatorname{End}}
\newcommand{\Supp}{\operatorname{Supp}}
\newcommand{\Order}{\operatorname{ord}}
\newcommand{\kerl}{\operatorname{Ker}}
\newcommand{\Img}{\operatorname{Image}}
\newcommand{\sine}{\operatorname{sin}}

\begin{document}

\centerline{\bf MATH 280: Homework 4 }

\bigskip

\noindent 1. \\
(a) Recall that in the bisection method with initial interval $[a_0,b_0]$, one
has that the $n$th midpoint $c_n$ and root $r$ of $f(x)=0$ satisfy $$ |c_n - r |
\leq \frac{1}{2} (b_n-a_n) = \frac{1}{2^{n+1}} (b_0-a_0).
$$ Let $\epsilon > 0$. If $N$ is the number of steps that must be taken in the
bisection method to guarantee that $|r-c_N| \leq \epsilon$, show that $$ N \geq
\frac{\log(b_0-a_0)-\log(\epsilon)}{\log(2)} - 1.
$$ (b) Assuming $a_0 > 0$, find a similar inequality as in part (a) for the
number $N$ of steps that must be taken in the bisection method to ensure the
root is found with relative accuracy $\leq \epsilon$, i.e., $|\frac{r-c_N}{r}|
\leq \epsilon$. In your final expression, you should replace any occurrence of
$r$ with either $a_0$ or $b_0$ (whichever works correctly) as apriori $r$ is not
known.

\medskip

\noindent 2. Consider $f(x)=x^2-4x\sin(x) + \sin^2(x)$. Note $r=0$ is a root of
$f(x)=0$.
We wish to find a {\bf positive} root $r$ of $f(x)=0$ using the bisection
method. In this problem you are free to use a calculator/computer to compute the
evaluations required but otherwise will do the bisection method "by hand". \\
(a) Find suitable values $a_0, b_0 > 0$ to choose as a starting interval
$[a_0,b_0]$ for the bisection method.  \\
(b) Use the bisection method to find a {\bf positive} root accurate to two
decimal digits. In each step state the value of the evaluation at the midpoint
$f(c_n)$ and which half interval you keep clearly. Also state clearly the root's
value up to two decimal digits.

\medskip
\noindent 3. Let $q > 0$ be a fixed positive real number and $f(x)=x^2-q$.
Notice the positive root of $f(x)$ is $\sqrt{q}$.
\\
(a) Find the Newton iteration function $N_f(x) = x - \frac{f(x)}{f'(x)}$ for
this example. \\
(b) For what real $x$ is this Newton function not defined? Show that if $x > 0$
then $N_f(x) > 0$ and $x<0$ then $N_f(x)<0$.
Based on your results, which "seeds" $x_0$ lead to values where $N_f(x)$ is not
defined when $N_f$ is applied iteratively? 

\medskip

\noindent 4. If Newton's method is used to compute a root of $f(x)=x^3-2$
starting with $x_0=1$, what is $x_2$? How close is $(x_2)^3$ to $2$?

\medskip

\noindent 5. (a) Show that for $f(x)=x^2$, the secant method will yield the
recursion $$ x_{n+1} = \frac{x_n x_{n-1}}{x_n + x_{n-1}}.
$$ Explain why if $x_0, x_1 > 0$ then $x_n > 0$ for all $n$. \\
(b) Show that if $x_0, x_1 > 0$ then $x_{n+1} < x_n$ for all $n \geq 1$. Thus
$x_1, x_2, x_3, \dots$ is a decreasing sequence, bounded below by zero and hence
the monotone convergence theorem says it has a limit $\alpha$. Explain why
$\alpha$ must satisfy $$ \alpha = \frac{\alpha^2}{2\alpha} $$ if it were nonzero
and use this to deduce $\alpha$. What do your results say about the convergence
of the secant method to a root in this instance?

\medskip

\noindent 6. Let $f(x) = x^{3} - 2$.  Code two routines, one that implements
the bisection method
\emph{recursively}, and another that implements it \emph{iteratively} for
$f(x)$. Use both routines to iterate to the root $x = \sqrt{2}$.
\end{document}


