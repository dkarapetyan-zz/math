\documentclass[12 pt]{article}
\usepackage{amsmath,amsfonts,amsthm,amscd}
\newtheorem{thm}{Theorem}[section]
\newtheorem{pro}[thm]{Proposition}
\newtheorem{cor}[thm]{Corollary}
\newtheorem{lem}[thm]{Lemma}
\newtheorem{conj}[thm]{Conjecture}
\newtheorem{ques}[thm]{Question}
\newtheorem{statement}[thm]{Statement}
\newtheorem{claim}[thm]{Claim}
\newtheorem{ex}[thm]{Example}
\newtheorem{rem}[thm]{Remark}
\newtheorem{defn}[thm]{Definition}
\newtheorem{prob}[thm]{Problem}
\newtheorem{quot}[]{Result}
\newcommand{\dime}{\operatorname{dim}}
\newcommand{\Obj}{\operatorname{Obj}}
\newcommand{\Mor}{\operatorname{Mor}}
\newcommand{\chan}{\operatorname{char}}
\newcommand{\degr}{\operatorname{deg}}
\newcommand{\expo}{\operatorname{exp}}
\newcommand{\spant}{\operatorname{span}}
\newcommand{\res}{\operatorname{res}}
\newcommand{\Term}{\operatorname{Term}}
\newcommand{\Init}{\operatorname{Init}}
\newcommand{\spec}{\operatorname{spec}}
\newcommand{\Endo}{\operatorname{End}}
\newcommand{\Supp}{\operatorname{Supp}}
\newcommand{\Order}{\operatorname{ord}}
\newcommand{\kerl}{\operatorname{Ker}}
\newcommand{\Img}{\operatorname{Image}}
\newcommand{\sine}{\operatorname{sin}}

\begin{document}

\centerline{\bf MATH 280: Homework 3. }

\bigskip

\noindent
1. \\
(a) Find the binary expansion of the number $\frac{1}{10}$. \\
(b) If the number $\frac{1}{10}$ is correctly rounded to the nearest 
left-shifted normalized binary number with 23 digit mantissa i.e., 
$(1.a_1a_2a_3 \dots a_{23})_2 \times 2^m$, what is the absolute 
roundoff error? What is the relative roundoff error?

\medskip

\noindent
2. For this question, consider a machine that works with normalized numbers with
$5$ decimal digits as its machine numbers. Thus $0.12345 \times 10^3$ or
$-0.98725 \times 10^{-1}$ etc. Let $fl(x)$ denote the nearest machine number on this
machine to the real number $x$. (If the two nearest machine numbers are of equal
distance, let us round up say). \\ (a) Find examples of real numbers $x$ and $y$
such that $fl(x+y)$ is not equal to $fl(fl(x)+fl(y))$. \\ (b) Find examples of
real numbers $x$ and $y$ such that $fl(xy)$ is not equal to $fl(fl(x) fl(y))$.
\\ (c) Find examples of machine numbers $x$, $y$ and $z$ such that $$fl(fl(xy)
z) \neq fl(x fl(yz)).$$Thus "machine multiplication is not associative". Recall
that real numbers under actual multiplications have the "associative property"
$$(xy)z=x(yz).$$




\medskip


\noindent
3. Let $x$ and $y$ be positive normalized binary machine numbers such that $x > y$ and $(1-\frac{y}{x}) = 0.0001$. Use Theorem 1 on Page 57 of the book 
(Theorem on Loss of Precision) to compute the most and the least number of significant binary bits lost in the subtraction $x-y$.

\medskip

\noindent
4. Suggest a way to avoid loss of significance due to subtracting close quantities in each of the following calculations: \\
(a) $\log(x)-\log(y)$  when $x$ is close to $y$ in value. \\
(b) $\sqrt{1+x^2} - x$ when $x$ is very large. \\
(c) $e^x - e$ when $x$ is close to $1$. 

\medskip

\noindent
5. Write a program that executes the following two functions: The first function prints three variables, all holding the number $-5.147$, but declared as int, double, and unsigned int, respectively. The second function takes a floating point number as an input, and prints the absolute and relative errors when the decimal portion is "chopped" off. Make sure to test for edge cases (for example, how does the second function respond when we evaluate at $0.0$?). 
 
\end{document}


