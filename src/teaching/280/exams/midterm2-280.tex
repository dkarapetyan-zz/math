\documentclass[12pt, oneside]{amsart}
\usepackage{graphicx}
\usepackage[nohead, margin=0.5in]{geometry}
\usepackage{enumerate}
%\geometry{left=0.5in,right=0.5in,top=0.3in,bottom=0.5in} 
\usepackage{listings}
\pagestyle{empty}
%Next are definitions of "\one" etc. 
%This is an easy way of assigning points to your questions and to 
%your table all at once. 
\newcommand{\one}{20}
\newcommand{\two}{20}
\newcommand{\three}{20}
\newcommand{\four}{20}
\newcommand{\five}{20}
\newcommand{\rr}{\mathbb{R}}
\synctex=1

\begin{document}


%feel free to change the title page
%

\begin{center}
    \hrulefill\\
    {\bf \textsf{\raisebox{-0.10cm}{Fall 2013: MATH 280} \hspace{\fill} 
            \raisebox{-0.10cm}{Numerical Analysis} \hspace{\fill}
            \raisebox{-0.10cm}{David Karapetyan}}}\\
    \hrulefill\\
    {\large \rule{0cm}{1.2cm} \textsf{Wednesday 11/13/2013} \hfill
        \textsf{Midterm} \hfill  \textsf{75 minutes}}\\
    {\large\rule{0cm}{1.2cm}\textsf{Name: \framebox[2.9in]{\rule{0cm}{0.8cm}} 
            \hspace{\fill}
            Student ID: \framebox[2.1in]{\rule{0cm}{0.8cm}}}}\\
\end{center}
\vspace{0.8cm}

\noindent
{\bf \textsf{Instructions.}}

\begin{enumerate}
    \item Attempt all questions.   
    \item Show all the steps of your work clearly.  
    \item Good luck 
        %The method (reasoning) used to 
        %obtain an answer is worth more than the answer itself.   
\end{enumerate}

\vfill

%The \rule commands create vertical space, which makes things sit nicely in 
%vertical way in boxes of table below.

\begin{center}
    {\large
        \begin{tabular}{|c|c|c|}
            \hline
            \rule[-0.3cm]{0cm}{1cm}
            \textsf{Question} & \textsf{Points} &  \textsf{Your Score} \\
            \hline
            \hline
            \rule[-0.3cm]{0cm}{1cm}
            \textsf{Q1} & \one &\\
            \hline
            \rule[-0.3cm]{0cm}{1cm}
            \textsf{Q2} & \two &\\
            \hline
            \rule[-0.3cm]{0cm}{1cm}
            \textsf{Q3} & \three &\\
            \hline
            \rule[-0.3cm]{0cm}{1cm}
            \textsf{Q4} & \four &\\
            \hline
            \rule[-0.3cm]{0cm}{1cm}
            \textsf{TOTAL} & 80 & \\
            \hline
        \end{tabular}
    } 

\end{center}

\vfill


\newpage
\noindent
\textbf{Q1}. \\ \\ 
Let $A$ be a square matrix, and let $\sigma(A)$ denote its spectrum. Prove that
if $\sigma(A) < 1$, then $I-A$ is invertible, and that the Neumann series of $A$ converges in any norm to $(I - A)^{-1}$. \\
\newpage

\noindent
\textbf{Q2}.\\ \\ 
\begin{enumerate}[a)]
\item State the definitions of the $L^1$, $L^2$, and $L^\infty$ norms on $\mathbb{C^n}$.
\vspace{2in}
\item State the definition of the operator norm for $n \times n$ matrices.
\vspace{2in}
\item Let $A : \mathbb{C}^n \to \mathbb{C}^n$ be an $n \times n$ matrix. Prove that
$ \| A \|_{\infty} = \sup_i \sum_{j = 1}^n | a_{ij} | $.
\end{enumerate}

\newpage

\noindent
\textbf{Q3}. \\ \\ 
Let $A$ be an $n \times n$ matrix with distinct eigenvalues, $n, M \in
\mathbb{N}$, and $x \in \mathbb{C}^n$. For arbitrary $v \in \mathbb{C}$, let
$v_1$ denote its first coordinate. Consider the algorithm 
\vspace{0.1in}\\
\noindent
input $n,A,x,M$ \\
for $k=1,2,\ldots,M$ do \\
$\phantom{bobo}y \leftarrow Ax$ \\
$\phantom{bobo}r \leftarrow y_1/x_1$ \\
$\phantom{bobo} x \leftarrow y$ \\
end \\
return $r$ 
\vspace{0.3in} \\
\noindent
As $M \to \infty$, what does $r$ converge to? Please supply a proof. 
\newpage
\noindent
\textbf{Q4}. \\ \\ 
\noindent
Find an $LU$ factorization for the matrix
\begin{equation*}
A = \begin{bmatrix}
60 & 30 & 20 \\
30 & 20 & 15 \\
20 & 15 & 12 \\
\end{bmatrix}
\end{equation*}
where $L$ has unit diagonal. Note: guessing will not be awarded points---your work should make extensive use of the diagonal entries of $A$. 




\end{document}



