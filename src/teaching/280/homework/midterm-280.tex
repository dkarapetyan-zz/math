\documentclass[12pt, oneside]{amsart}
\usepackage{graphicx}
\usepackage[nohead, margin=0.5in]{geometry}
\usepackage{enumerate}
%\geometry{left=0.5in,right=0.5in,top=0.3in,bottom=0.5in} 

\pagestyle{empty}
%Next are definitions of "\one" etc. 
%This is an easy way of assigning points to your questions and to 
%your table all at once. 
\newcommand{\one}{20}
\newcommand{\two}{20}
\newcommand{\three}{20}
\newcommand{\four}{20}
\newcommand{\five}{20}
\newcommand{\rr}{\mathbb{R}}
\synctex=1

\begin{document}


%feel free to change the title page
%

\begin{center}
    \hrulefill\\
    {\bf \textsf{\raisebox{-0.10cm}{Fall 2013: MATH 280} \hspace{\fill} 
            \raisebox{-0.10cm}{Numerical Analysis} \hspace{\fill}
            \raisebox{-0.10cm}{David Karapetyan}}}\\
    \hrulefill\\
    {\large \rule{0cm}{1.2cm} \textsf{Wednesday 10/02/2013} \hfill
        \textsf{Midterm} \hfill  \textsf{75 minutes}}\\
    {\large\rule{0cm}{1.2cm}\textsf{Name: \framebox[2.9in]{\rule{0cm}{0.8cm}} 
            \hspace{\fill}
            Student ID: \framebox[2.1in]{\rule{0cm}{0.8cm}}}}\\
\end{center}
\vspace{0.8cm}

\noindent
{\bf \textsf{Instructions.}}

\begin{enumerate}
    \item Attempt all questions.   
    \item Show all the steps of your work clearly.  
    \item Good luck 
        %The method (reasoning) used to 
        %obtain an answer is worth more than the answer itself.   
\end{enumerate}

\vfill

%The \rule commands create vertical space, which makes things sit nicely in 
%vertical way in boxes of table below.

\begin{center}
    {\large
        \begin{tabular}{|c|c|c|}
            \hline
            \rule[-0.3cm]{0cm}{1cm}
            \textsf{Question} & \textsf{Points} &  \textsf{Your Score} \\
            \hline
            \hline
            \rule[-0.3cm]{0cm}{1cm}
            \textsf{Q1} & \one &\\
            \hline
            \rule[-0.3cm]{0cm}{1cm}
            \textsf{Q2} & \two &\\
            \hline
            \rule[-0.3cm]{0cm}{1cm}
            \textsf{Q3} & \three &\\
            \hline
            \rule[-0.3cm]{0cm}{1cm}
            \textsf{Q4} & \four &\\
            \hline
            \rule[-0.3cm]{0cm}{1cm}
            \textsf{Q5} & \five &\\
            \hline
            \rule[-0.3cm]{0cm}{1cm}

            \textsf{TOTAL} & 100 & \\
            \hline
        \end{tabular}
    } 

\end{center}

\vfill


\newpage
\noindent
\textbf{Q1}. \\ \\ 
State the asymptotic order of the following, with proof.  
\\

\noindent
(a) $(n+10^7)/(n^2)$ 

\vspace{2in}

\noindent
(b) $(n+1)/(\sqrt{n})$ 

\vspace{2in}

\noindent
(c) $\cos(x)-1$ 

\vspace{2in}

\noindent
(d) $e^{1/x}$ 
\newpage
\noindent
\textbf{Q2}.\\ \\ 
Consider the recurrence 
$$
x_{n+2}=x_{n+1}+x_n
$$
for $n \geq 0$. \\

\noindent
(a) If $E$ is the shift operator on the vector space of complex sequences, and $\hat{x}=(x_0,x_1,x_2,\dots)$, rewrite the recurrence in the form 
$p(E)\hat{x}=\hat{0}$. What is the characteristic polynomial $p(x)$? 

\vspace{1.5 in}

\noindent
(b) Find the roots of the characteristic polynomial and use them to describe the general solution to this recurrence. 

\vspace{1.5in}

\noindent
(c) If $x_0=0$ and $x_1=1$, find $x_2, x_3, x_4, x_5$ directly from the recurrence. \\

\vspace{1.5in}

\noindent
(d) Find the particular solution to this recurrence when $x_0=0$ and $x_1=1$. In particular express $x_n$ as a function of $n$.

\newpage
\noindent
\textbf{Q3}. \\ \\ 

\noindent
(a) Find the binary expansion of the base $10$ number $39$.

\vspace{2in}

\noindent
(b) Find the binary expansion of the base $10$ number $3/8$. 

\vspace{2in}

\noindent
(c) Find the octal (base $8$) expansion of the base $10$ number $113$. 

\vspace{2in}

\noindent
(d) Find the hexadecimal (base $16$) expansion of the base $10$ number $274$. 

\newpage
\noindent
\textbf{Q4}. \\ \\ 
\noindent
a) Let $f(x)=x^5$ and $x_0=10$ with absolute error at most $\delta_x=0.2$. Compute $f'(x_0)$ and the condition number and use them to estimate 
the absolute and relative error in $y_0=f(x_0)$. 

\vspace{4in}

\noindent
b) After $n$ iterations of the bisection method for $f$, starting in the interval $[-1, 1]$, what candidate root do you arrivate at? What upper bound for the absolute error does the bisection method guarantee us? What is the actual absolute error?  

\newpage
\noindent

\noindent \textbf{Q5}. \\ \\ 
\noindent
Write a C or C++ program implementing Newton's method for the function $f(x) = x^3 -7$, with initial starting point $x_0 = 2$.    
\vspace{2in}




\end{document}



