\documentclass[12 pt]{article}
\usepackage{amsmath,amsfonts,amsthm,amscd}
\newtheorem{thm}{Theorem}[section]
\newtheorem{pro}[thm]{Proposition}
\newtheorem{cor}[thm]{Corollary}
\newtheorem{lem}[thm]{Lemma}
\newtheorem{conj}[thm]{Conjecture}
\newtheorem{ques}[thm]{Question}
\newtheorem{statement}[thm]{Statement}
\newtheorem{claim}[thm]{Claim}
\newtheorem{ex}[thm]{Example}
\newtheorem{rem}[thm]{Remark}
\newtheorem{defn}[thm]{Definition}
\newtheorem{prob}[thm]{Problem}
\newtheorem{quot}[]{Result}
\newcommand{\dime}{\operatorname{dim}}
\newcommand{\Obj}{\operatorname{Obj}}
\newcommand{\Mor}{\operatorname{Mor}}
\newcommand{\chan}{\operatorname{char}}
\newcommand{\degr}{\operatorname{deg}}
\newcommand{\expo}{\operatorname{exp}}
\newcommand{\spant}{\operatorname{span}}
\newcommand{\res}{\operatorname{res}}
\newcommand{\Term}{\operatorname{Term}}
\newcommand{\Init}{\operatorname{Init}}
\newcommand{\spec}{\operatorname{spec}}
\newcommand{\Endo}{\operatorname{End}}
\newcommand{\Supp}{\operatorname{Supp}}
\newcommand{\Order}{\operatorname{ord}}
\newcommand{\kerl}{\operatorname{Ker}}
\newcommand{\Img}{\operatorname{Image}}
\newcommand{\sine}{\operatorname{sin}}

\begin{document}

\centerline{\bf MATH 280: Homework 7 (Written=20 points). }
\centerline{\bf Due in class, Monday, Nov 5}

\bigskip

\noindent
[6 points]1. \\ Recall on $\mathbb{R}^n$ we have a few different norms. For example
$$
||x||_{\infty}=max\{ |x_i| | 1 \leq i \leq n \}
$$
$$
||x||_{1}=\sum_{i=1}^n |x_i| 
$$
and the Euclidean norm 
$$
||x||_{2}=\sqrt{\sum_{i=1}^n x_i^2}.
$$
(a) Show that $$||x||_{\infty} \leq ||x||_2 \leq ||x||_1$$ for all vectors $x \in \mathbb{R}^n$. \\

\noindent
(b) Show that $$||x||_1 \leq n ||x||_{\infty}$$ and $$||x||_2 \leq \sqrt{n} ||x||_{\infty}$$ for all $x \in \mathbb{R}^n$. \\

\noindent
(c) Two norms $||\cdot||, ||\cdot||'$ are equivalent if there exist constants $0 < c \leq C < \infty$ such that 
$$c||x|| \leq ||x||' \leq C||x||$$ for all $x$. You have shown in parts (a) and (b) that $||\cdot||_1, ||\cdot||_2$ and $||\cdot||_{\infty}$ 
are all equivalent norms. Recall we say a sequence $x_n$ converges to $\alpha$ in the metric derived from $|| \cdot ||$ if 
$$ \lim_{n \to \infty} ||x_n - \alpha|| = 0.$$ 
Given a sequence $x_n$ explain why if $x_n$ converges to $\alpha$ in a norm $|| \cdot ||$, it also converges to $\alpha$ 
in any equivalent norm.

\medskip

\noindent
[4 points]2. \\ 
Let $\mathbb{A} = \begin{bmatrix} 1 & 2 & 3 \\ 0 & 1 & 2 \\ 0 & 1 & 1 \end{bmatrix}$. \\
(a)Compute $||\mathbb{A}||_{\infty}$ and $\kappa_{\infty}(\mathbb{A})$ the condition number of $\mathbb{A}$ computed 
using the matrix norm $||\cdot ||_{\infty}$ subordinate to the vector norm $|| \cdot ||_{\infty}$. \\

\noindent
(b) If $b=(1,0,2)^T$ and $b'=(1.01,-0.02, 2.03)^T$ and $\mathbb{A}x=b, \mathbb{A}x'=b'$, 
what is an upper bound for $||x-x'||_{\infty}$? (Give an upper bound that does not require you to solve for $x$ and $x'$.) What is an upper bound for $\frac{||x-x'||}{||x||}$?

\medskip

\noindent
[4 points]3. \\
(a) Let $\mathbb{B}= \begin{bmatrix} 1.2 & 0.4 & -0.3 \\ 0.2 & 0.9 & -0.1 \\ 0.1 & 0.3 & 0.8 \end{bmatrix}$.
Find $\mathbb{A}=\mathbb{I} - \mathbb{B}$ and compute $||\mathbb{A}||_{\infty}$. 
Use this to explain why $\mathbb{B}^{-1}$ exists and give a Neumann series (matrix power series) that converges to $\mathbb{B}^{-1}$. \\

\noindent
(b) Let $\mathbb{C}$ be an invertible square matrix. Explain why $(\lambda \mathbb{C})^{-1} = \frac{1}{\lambda} \mathbb{C}^{-1}$ for any nonzero 
$\lambda$. 

\noindent
(c) Use (a) and (c) to quickly find a matrix power series for the inverse of $\mathbb{C}=\begin{bmatrix} 12 & 4 & -3 \\ 2 & 9 & -1 \\ 1 & 3 & 8 \end{bmatrix}$. Notice that often 
though a matrix might not directly have a small enough norm to use Neumann series, it can be scaled to one that does.

\medskip

\noindent
[6 points]4. \\
Consider the system of linear equations given by:
$$
\begin{bmatrix} 3 & 1 & 1 \\ 1 & 3 & -1 \\ 3 & 1 & -5 \end{bmatrix}
\begin{bmatrix} x_1 \\ x_2 \\ x_3 \end{bmatrix}
=
\begin{bmatrix} 5 \\ 3 \\ -1 \end{bmatrix}
$$
(a) In the Jacobi Iterative Method, find the splitting matrix $\mathbb{Q}$ for this system and compute 
$\mathbb{I} - \mathbb{Q}^{-1} \mathbb{A}$. Write down the corresponding recursion system and 
find $|| \mathbb{I}-\mathbb{Q}^{-1}\mathbb{A}||_{\infty}$. What if anything does Theorem 1 in section 4.6 say in this situation? \\

\noindent
(b) In the Gauss-Seidel Method, find the splitting matrix $\mathbb{Q}$ for this system and compute 
$\mathbb{I} - \mathbb{Q}^{-1} \mathbb{A}$. Write down the corresponding recursion system and 
find $|| \mathbb{I}-\mathbb{Q}^{-1}\mathbb{A}||_{\infty}$. What if anything does Theorem 1 in section 4.6 say in this situation?\\

\noindent
(c) In the Richardson Iterative Method, find the splitting matrix $\mathbb{Q}$ for this system and compute 
$\mathbb{I} - \mathbb{Q}^{-1} \mathbb{A}$. Write down the corresponding recursion system and 
find $|| \mathbb{I}-\mathbb{Q}^{-1}\mathbb{A}||_{\infty}$. What if anything does Theorem 1 in section 4.6 say in this situation?

\end{document}


