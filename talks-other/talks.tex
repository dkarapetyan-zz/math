%
\documentclass[12pt,reqno]{amsart}
\usepackage{amssymb}
\usepackage{appendix}
\usepackage[showonlyrefs=true]{mathtools} %amsmath extension package
\usepackage{cancel}  %for cancelling terms explicity on pdf
\usepackage{yhmath}   %makes fourier transform look nicer, among other things
\usepackage{framed}  %for framing remarks, theorems, etc.
\usepackage{enumerate} %to change enumerate symbols
\usepackage[margin=2.5cm]{geometry}  %page layout
\setcounter{tocdepth}{1} %must come before secnumdepth--else, pain
\setcounter{secnumdepth}{1} %number only sections, not subsections
%\usepackage[pdftex]{graphicx} %for importing pictures into latex--pdf compilation
\numberwithin{equation}{section}  %eliminate need for keeping track of counters
%\numberwithin{figure}{section}
\setlength{\parindent}{0in} %no indentation of paragraphs after section title
\renewcommand{\baselinestretch}{1.1} %increases vert spacing of text
%
\usepackage{hyperref}
\hypersetup{colorlinks=true,
  linkcolor=blue,
  citecolor=blue,
  urlcolor=blue,
}
\usepackage[alphabetic, initials, msc-links]{amsrefs} %for the bibliography; uses cite pkg. Must be loaded after hyperref, otherwise doesn't work properly (conflicts with cref in particular)
\usepackage{cleveref} %must be last loaded package to work properly
%
%
\newcommand{\ds}{\displaystyle}
\newcommand{\ts}{\textstyle}
\newcommand{\nin}{\noindent}
\newcommand{\rr}{\mathbb{R}}
\newcommand{\nn}{\mathbb{N}}
\newcommand{\zz}{\mathbb{Z}}
\newcommand{\cc}{\mathbb{C}}
\newcommand{\ci}{\mathbb{T}}
\newcommand{\zzdot}{\dot{\zz}}
\newcommand{\wh}{\widehat}
\newcommand{\p}{\partial}
\newcommand{\ee}{\varepsilon}
\newcommand{\vp}{\varphi}
\newcommand{\wt}{\widetilde}
%
%
%
%
\newtheorem{theorem}{Theorem}[section]
\newtheorem{lemma}[theorem]{Lemma}
\newtheorem{corollary}[theorem]{Corollary}
\newtheorem{claim}[theorem]{Claim}
\newtheorem{prop}[theorem]{Proposition}
\newtheorem{proposition}[theorem]{Proposition}
\newtheorem{no}[theorem]{Notation}
\newtheorem{definition}[theorem]{Definition}
\newtheorem{remark}[theorem]{Remark}
\newtheorem{examp}{Example}[section]
\newtheorem {exercise}[theorem] {Exercise}
%
%makes proof environment bold instead of italic
\newcommand{\uol}{u^\omega_\lambda}
\newcommand{\lbar}{\bar{l}}
\renewcommand{\l}{\lambda}
\newcommand{\R}{\mathbb R}
\newcommand{\RR}{\mathcal R}
\newcommand{\al}{\alpha}
\newcommand{\ve}{q}
\newcommand{\tg}{{tan}}
\newcommand{\m}{q}
\newcommand{\N}{N}
\newcommand{\ta}{{\tilde{a}}}
\newcommand{\tb}{{\tilde{b}}}
\newcommand{\tc}{{\tilde{c}}}
\newcommand{\tS}{{\tilde S}}
\newcommand{\tP}{{\tilde P}}
\newcommand{\tu}{{\tilde{u}}}
\newcommand{\tw}{{\tilde{w}}}
\newcommand{\tA}{{\tilde{A}}}
\newcommand{\tX}{{\tilde{X}}}
\newcommand{\tphi}{{\tilde{\phi}}}
\synctex=1
\begin{document}
%\address{Department of Mathematics  \\
    %University  of Notre Dame\\
	%Notre Dame, IN 46556 }
	%
	%
	%
	%
\section{Antononopoulos Mitsotakis Boussinesq Approximations}
\begin{enumerate}
  \item Numerical solution of the ``classical'' Boussinesq system,
    preprint 2009
  \item D.C. Antonopoulos and V.D. Notes on error estimates for Galerkin
    approximations of the ``classical'' Boussinesq system and related
    hyperbolic problems, preprint 2010 arxiv
  \item D.C. Antonopoulos, D.E. Mitsotakis and V.D. Galerkin, approximations of
    periodic solutions of Boussinesq systems, preprint 2010
\end{enumerate}
\section{Asymptotic Linear Stability of Solitary water waves--Tzvetkov}
Weinstein's argument (Stability of KDV solitons). Doesn't work for KP
equation. Merle and Vega prove stability of KDV solitons in
$L^{2}$ using the Miura transform and the Galilei invariance. Tzvetkov
applied Merle-Vega's method to KP-II\@. Got stability of $\vp_{c}(x - 2ct)$
soliton.
	%
	%
\section{Transverse Instability for KP-I and NLS equations---Stefanov}
KP-I
       %
       %
\begin{equation*}
  \begin{split}
    & (u_{t} + u_{xxx} + (f(u))_{x})_{x} - u_{yy} =0
    \\
    & (u_{t} + u_{xxx} + (f(u))_{x})_{x} + u_{yy} = 0.
  \end{split}
\end{equation*}
       %
       %
Johnson and Zumbrun--periodic waves of gKdV equation, 2-D perturbations.
Haragus--transverse spectral stability.
\section{Stability of the $\mu$-CH peakons---Liu}
Asymptotic stability for CH is open.
\section{Erdogan and Tzirakis}
       %
       %
       %%%%%%%%%%%%%%%%%%%%%%%%%%%%%%%%%%%%%%%%%%%%%%%%%%%%%
       %
       %
       %                
       %
       %
       %%%%%%%%%%%%%%%%%%%%%%%%%%%%%%%%%%%%%%%%%%%%%%%%%%%%%
       %
       %
\begin{theorem}
  Let $s > -1/2$ and $s_{1} < \min(3s+1, s+1)$. Then, given $H^{s}$, we
  have
	 %
	 %
  \begin{equation*}
    \begin{split}
      \| u(t) - e^{-t \p_{x}^{3}}g \|_{H^{s_{1}}} \le C (\| g
      \|_{H^{s}}) \langle t \rangle^{\alpha(s)},
    \end{split}
  \end{equation*}
	 %
	 %
  for some $\alpha(s) > 0$.
  \label{thm:main3}
\end{theorem}
       %
       %
Christ studying smoothing $FL^{p} \to FL^{q}$. Other theorem in paper
similar to that one in paper.
Techniques in Tzirakis paper allows you to update uniqueness for $s \ge 0$.
Very powerful.  Michael Christ disproved being able to update uniqueness
for $s < 0$ via counterexample. Unpublished paper, on his website.
       %
       %
       %%%%%%%%%%%%%%%%%%%%%%%%%%%%%%%%%%%%%%%%%%%%%%%%%%%%%
       %
       %
       %                blowup complex KDV
       %
       %
       %%%%%%%%%%%%%%%%%%%%%%%%%%%%%%%%%%%%%%%%%%%%%%%%%%%%%
       %
       %
\section{Blowup of the Complex KDV--Bona, Weissler} 
\label{sec:kdv-blowup}
       %
       %
       %%%%%%%%%%%%%%%%%%%%%%%%%%%%%%%%%%%%%%%%%%%%%%%%%%%%%
       %
       %
       %                him
       %
       %
       %%%%%%%%%%%%%%%%%%%%%%%%%%%%%%%%%%%%%%%%%%%%%%%%%%%%%
       %
       %
\section{CH Equation--Alex Himonas} 
\label{sec:him}
Integrable equations--Fokas-Fuchssteiner, Gel'fand-Dorfman



	%
	%
	%
\section{The Camassa Holm Equation}
\label{sec:ch}
\subsection{Origins}
\label{ssec:origins}
\begin{enumerate}
  \item
    Fokas and Fuchssteiner--Hereditary symmetries ('81). Wrong coefficients.
  \item Camassa-Holm--Shallow water theory ('93).
\end{enumerate}

	%
\begin{equation*}
  \begin{split}
    \text{KDV}: \ \
    u_{t} + 3 uu_{x} + \alpha u_{xxx} + \beta u_{x} = 0, \quad \alpha,
    \beta \neq 0.
  \end{split}
\end{equation*}
	%
	%
For Camassa-Holm, have one extra term on left. Also,
instead of zero on right hand side, have
	%
	%
\begin{equation*}
  \begin{split}
    \text{CH}: \ \
    u_{t} + - u_{xxt} + 3 uu_{x} + \alpha u_{xxx} + \beta u_{x} = uu_{xxx} + 2
    u_{x} u_{xx}, \quad \alpha \neq 0.
  \end{split}
\end{equation*}
	%
	%
If $\alpha, \beta \neq 0$ have smooth solitons (look very similar to KDV
solitons). \emph{But is it physical}?
\begin{enumerate}
  \item Dullin, Cottwald, and Holm ('03)--yes, it is physical. 
  \item Constantin (09')--yes, it is physical.
  \item Mikhailov, Bhatt--On the Inconsistency of the Camassa-Holm
    equation\ldots('10)--No, it isn't.
\end{enumerate}
More interesting: dispersionless equations. Set $\alpha \to 0, \beta \to
0$, or $u \mapsto \tilde{u} = u + u_{0}$, $x \mapsto \tilde{x} = x -
ct$, and $t \mapsto \tilde{t} = t$. Rewrite dispersionless equations as
	%
	%
\begin{equation*}
  \begin{split}
    & m_{t} + um_{x} + 2 u_{x}m =0,
    \\
    & \text{"momentum"} m = u - u_{xx}
  \end{split}
\end{equation*}
	%
	%
	%
	%
\begin{remark}
  Can think of this as a geodesic equation. $u$ is like a Lagrangian
  variable, $m$ is like a Hamiltonian variable.
\end{remark}
	%
	%


\subsection{Peakons} 
\label{ssec:peakons}
\begin{enumerate}
  \item Single peakon: $u = ce^{-| x -ct |}, m = 2 c \delta(x -ct)$.
  \item Multi-peakons: linear combination of single peakons. $u =
    \sum_{j=1}^{N} p_{j}e^{-| x - q_{j}(t) |}, m = 2 \sum_{j
    =1}^{N} p_{j}(t) \delta(x - q_{j}(t))$.
\end{enumerate}

\subsection{Bi-Hamiltonian Structure and Recursion Operator} 
\label{ssec:bi-ham}
CH better rewritten as 
	%
	%
$u = g*m$ with $g(x) = \frac{1}{2}e^{-| x |}$. How about positions
$q_{j}$ and amplitudes $p_{j}$? They satisfy a canonical finite dimensional
Hamiltonian system:%
	%
\begin{equation*}
  \begin{split}
    & \frac{dq_{j}}{dt} = \frac{\p H}{\p p_{j}}
    \\
    & \frac{d _{p_{j}}}{d_{t}} = -\frac{\p H}{d q_{j}}, j = 1, \cdots N
  \end{split}
\end{equation*}
	%
	%
where
	%
	%
\begin{equation*}
  \begin{split}
    H = \frac{1}{2} \sum_{j,k} p_{j} p_{k} e^{-| q_{j} - q_{k} |}
  \end{split}
\end{equation*}
	%
	%
Motion of the peakons is a geodesic flow. CH equation itself is a
geodesic equation. Rewriting Hamilton's equations is to write
%
%
\begin{equation*}
  \begin{split}
    \frac{d}{dt} (\vec{q}/ \vec{p}) = J \Delta H
  \end{split}
\end{equation*}
%
%
where 
%
%
\begin{equation*}
  \begin{split}
    J = 
    \begin{bmatrix}
      0 & I
      \\
      -I & 0
    \end{bmatrix}
  \end{split}
\end{equation*}
%
%
As a geodesic equation, CH comes from $L = \frac{1}{2}\|u \|_{H^{1}}^{2}$. If you
switch to the Hamiltonian version, then $m = \frac{\delta L}{\delta u} = u -
u_{xx}$ (Legendre Transform).
\begin{enumerate}
  \item CH has the Hamiltonian form
    %
    %
    \begin{equation*}
      \begin{split}
	\frac{\p m}{\p t} = B_{0} \frac{\p H}{\p m}
      \end{split}
    \end{equation*}
    %
    %
    with %
    %
    \begin{equation*}
      \begin{split}
	B_{0} = -(m D_{x} + D_{x}m) \text{it's skew symmetric}
      \end{split}
    \end{equation*}
    %
    %
    and %
    %
    \begin{equation*}
      \begin{split}
	H = L = \frac{1}{2} \int m g*m dx
      \end{split}
    \end{equation*}
    %
    %
    Gradient defind by
    %
    %
    \begin{equation*}
      \begin{split}
	\langle \frac{\delta H}{\delta m}, v  \rangle  = \frac{d}{d \ee} H(m + \ee
	v) |_{\ee = 0}, \forall v
      \end{split}
    \end{equation*}
    %
    %
    In this case, $H = \frac{1}{2} \langle m, g*m \rangle $ and this implies
    $\frac{\delta H}{\delta m} = g*m =u$.
  \item But also:
    %
    %
    \begin{equation*}
      \begin{split}
	\frac{\p m}{\p t} = B_{1} \frac{\delta \tilde{H}}{\delta m}
      \end{split}
    \end{equation*}
    %
    %
    where %
    %
    \begin{equation*}
      \begin{split}
	& B_{1} = -D_{x}(1 - D_{x}^{2}), \\
	& \tilde{H} = \frac{1}{2} \int (u u_{x}^{2} + u^{3})dx
      \end{split}
    \end{equation*}
    %
    %
    Then 
    %
    %
    \begin{equation*}
      \begin{split}
	\frac{\delta \tilde{H}}{\delta u}
	& = \frac{1}{2} u_{x}^{2} + \frac{3}{2}
	u^{2} - (u u_{x})_{x}
	\\
	& = -\frac{1}{2} u_{x}^{2} + u u_{xx} + \frac{3}{2} u^{2}.
      \end{split}
    \end{equation*}
    %
    %
    Now
    %
    %
    \begin{equation*}
      \begin{split}
	m = M[u] \implies \frac{\delta \tilde{H}}{\delta u} = M'[u] + \frac{\delta
	  \tilde{H}}{\delta m}. 
	\end{split}
      \end{equation*}
    %
    %
  \end{enumerate}
  To recap, what we have shown is that CH has two Hamiltonian structures.
  $B_{0}$ and $B_{1}$ are compatible, i.e. $\lambda_{0} B_{0} +
  \lambda_{1} + B_{1}$ is Hamiltonian $\forall (\lambda_{0}, \lambda_{1})$. This
  Bi-Hamiltonian compatibility condition is one definition for integrability,
  since from this we can create infinitely many symmetries based on recursion
  operator
%
%
  \begin{equation*}
    \begin{split}
      &  R = B_{1} B_{0}^{-1}
      \\
      & \frac{\p m}{\p t_{n}} = R^{n} \frac{\p m}{\p t}
    \end{split}
  \end{equation*}
%
%
  where %
%
  \begin{equation*}
    \begin{split}
      \frac{\p}{\p t_{n}} \ \text{commutes with} \ \frac{\p}{\p t} \ \forall n
    \end{split}
  \end{equation*}
%
%
  \subsection{Lax Pairs and Reciprocal transfer} 
%
%
  \section{Camassa-Holm II}
%
%
  \begin{equation*}
    \begin{split}
      & B_{0} \frac{\delta H_{-1}}{\delta m} = B_{1} \frac{\delta H_{-3}}{\delta m} =
      -(u m_{x} + 2 u_{x}m)
      \\
      & B_{0} \frac{\delta H_{-1}}{\delta m} = B_{1} \frac{\delta H_{-3}}{\delta m} =
      -m_{x}
      \\
      & B_{0} \frac{\delta C}{\delta m} = B_{1} \frac{\delta H_{1}}{\delta m} =0
      \\
      & = B_{0} \frac{\delta H_{3}}{\delta m} = B_{1} \frac{\delta C}{\delta m} =
      -\frac{1}{2} (D_{x} - D_{x}^{3})m^{-1/2} \ \text{Harvy Dym}
      \\
      & B_{1} \frac{\delta H_{3}}{\delta m} = \cdots \text{5th order flow}
    \end{split}
  \end{equation*}
%
%
  Recall that %
%
  \begin{equation*}
    \begin{split}
      & R = B_{1} B_{0}
      \\
      & H_{-3} \frac{1}{2} \int (u u_{x}^{2} + u^{3}) dx
      \\
      & H_{-1} - \frac{1}{2} \int m g*m dx
      \\
      & C = \int m^{1/2} dx
      \\
      & H_{1} = \int m dx
      \\
      & H_{3} = \int (-1/16 m^{-5/2} m_{x}^{2} - \frac{1}{4} m^{-1/2})dx
    \end{split}
  \end{equation*}
%
%
  Now
%
%
  \begin{equation*}
    \begin{split}
      & dX = m^{1/2}dx - m^{1/2} u dt
      \\
      & dT = dt
      \\
      &\text{CH eqn} \ \ V_{T} = -p_{x}
    \end{split}
  \end{equation*}
%
%
  where %
%
  \begin{equation*}
    \begin{split}
      & p(x, T) = m^{1/2}(x,t)
      \\
      & v(X, T) = \frac{-p_{xx}}{2p} + \frac{1}{4} \frac{p_{x}^{2}}{p^{2}} -
      \frac{1}{2p^{2}}
    \end{split}
  \end{equation*}
%
%
  These are called reciprocal transformations---can apply to
  any equation with conservation laws. Weights of CH/Harry Dyn hiererarchy. Can do
  counting definition: each
  $x$ derivative counts as $+1$, take the highest that appears. $m$ has weight
  $0$. $u$ has weight $-2$, since $u = (1 - D_{x}^{2})^{-1}m$. 
  \subsection{Other integral peakon-type equations?} 
  \label{ssec:lax-pair}
  \label{ssec:peak-type}
  \begin{enumerate}
    \item{Degasperis-Procesi}
      $m_{t} + u m_{x} + 3 u_{x} m = 0$, \quad $m = u - u_{xx}, u = g*m, g =
      \frac{1}{2}e^{-| x |}$.
    \item{Novikov}
      $m_{t} + u^{2} m_{x} + 3 u u_{x} m =0$
    \item{Qiao[Fokas]}
      $m_{t} + (m(u^{2}- u_{x}^{2})) =0$.
  \end{enumerate}
  \subsection{b-family of equations} 
  \label{ssec:b-family}
  $m_{t} + u m_{x} + b u_{x} m = 0$. According to all sorts of tests or
  integrability, only $b=2$ (CH) and $b=3$ (DP) are integrable. Still a nice
  family of equations which for every member admits multi-peakon solutions.
%
%
  \begin{equation*}
    \begin{split}
      u = \sum_{j=1}^{N}p_{j}(t) g(x - q_{j}(t)), \quad m = \sum_{j=1}^{N}
      p_{j} \delta(x - q_{j})
    \end{split}
  \end{equation*}
%
%
  Equations of motion for peakons (``pulsons for abitrary g'', symmetric kernel):
%
%
  \begin{equation*}
    \begin{split}
      & \frac{dq_{j}}{dt} = \sum_{k} p_{k} g (q_{j} - q_{k})
      \\
      & \frac{dp_{j}}{dt} = -(b-1) \sum_{k} p_{j} p_{k} g'(q_{j} - q_{k}).
    \end{split}
  \end{equation*}
%
%
  If $b=2$, these equations are canonical Hamiltonian equations with Hamiltonian
%
%
  \begin{equation*}
    \begin{split}
      h = \frac{1}{2} \sum p_{j} p_{k} g(q_{j} - q_{k})
    \end{split}
  \end{equation*}
%
%
  for arbitrary symmetric $g$. Now, b-family with arbitrary choice of $g$ can be
  written in the following way.%
%
  \begin{equation*}
    \begin{split}
      \frac{\p m}{\p t = } B \frac{\delta H}{\delta m}
    \end{split}
  \end{equation*}
%
%
  where $H = \int m dx$ and (up to scaling)
  $B = -m^{1 - 1/b} D_{x} m^{1/b} \wh{G} m^{1.b}
  D_{x} m^{1 - 1/b}$ where $B$ is a skew symmetric operator, and $\wh{G}$ is
  convolution with $G(x) = \int_{0}^{x} g(s) dt$. 
  Now, not necessarily Hamiltonian, since may not satisfy Jacobi identity.        
  The skew bracket defined by $B$ reduces to a skew bracket for $p_{j},
  q_{k}$ on the level of the pulson solutions.
%
%
  \begin{equation*}
    \begin{split}
      & \left\{ q_{j}, q_{k} \right\} = G(q_{j} - q_{k}),
      \\
      & \left\{ q_{j}, p_{k} \right\} = (b-1) G'(q_{j} - q_{k})p_{k}
      \\
      & \left\{ p_{j}, p_{k} \right\} = -(b-1)^{2} G''(q_{j} - q_{k})p_{j}
      p_{k}.
    \end{split}
  \end{equation*}
%
%
  Jacobi identity:
%
%
  \begin{equation*}
    \begin{split}
  %\left \{ \left\{ f, g \right\}, h\} + \left\{ \left\{ h,f \right\},g \right\}
  %+ \left\{ \left\{ g,h \right\}, f \right\} = 0 \forall f, g, h
    \end{split}
  \end{equation*}
%
%
  Need to check 
%
%
  \begin{equation*}
    \begin{split}
      & \left\{ \left\{ q_{j}, q_{k} \right\}, q_{l} \right\} + \text{cyclic} = 0
      \\
      & \left\{ \left\{ q, q \right\}, p \right\} + \text{cyclic} = 0
      \\
      & \left\{ \left\{ q, p \right\}, p \right\} + \text{cyclic} = 0
    \end{split}
  \end{equation*}
%
%
  Functional equation for $G$:
%
%
  \begin{equation*}
    \begin{split}
      & G'(\alpha)(G(\beta) + G(\gamma)) + \text{cyclic} = 0
      \\
      &  \text{whenever} \ \alpha + \beta + \gamma =0
    \end{split}
  \end{equation*}
%
%
  The general odd solution belonging to $C^{1}(\rr)\setminus 0$ is
%
%
  \begin{equation*}
    \begin{split}
      G(x) = A \text{signum} x(1 - e^{-B| x |})
    \end{split}
  \end{equation*}
%
%
  Now, if $B \to 0, A = \frac{1}{B}$, you get $G(x) = cx$, which implies
  $m_{t} + k m_{x} = 0, K = A \int m(y) dy$. If $B \to \infty, A=1, G(x) = A
  \text{signum} x$, then $m_{t} + kmm_{x}=0$. Holm and Staley did numerics on the
  b-family. Found that for $b \ge 1$, you get stable trains of multi-peakons. For
  example
%
%
  \begin{equation*}
    \begin{split}
      u_{t} + u^{3} u_{x} + u_{xxx} = 0
    \end{split}
  \end{equation*}
%
%
  seemingly has many solitons (stable), yet not integrable. $b < 1$, get leftons,
  $-1 < b < 1$ ragged type.
  \subsection{Lax pairs for CH} 
  \label{ssec:lax-pair-CH}
  \begin{definition}
    A Lax pair is a pair of linear equations with a spectral parameter $\lambda \in
    \cc$, whose compatibility yields the nonlinear equation of interest. 
  \end{definition}
  For a PDE in $1+1$ dimensions (indept variables $x$ and $t$), the Lax pair
  consists of an $x$ part and a $t$ part.

  CH scalar Lax pair:
%
%
  \begin{equation*}
    \begin{split}
      & \text{x part}: \psi_{xx} + (m \lambda - \frac{1}{4}) \psi = 0.
      \\
      & \text{t part}: \psi_{t} + (u + \frac{1}{2} \lambda^{-1})\psi_{x} -
      \frac{1}{2}u_{x} \psi = 0
    \end{split}
  \end{equation*}
%
%
  \begin{definition}
    Compatibility: $\psi_{xxt} = \psi_{txx} \forall \psi, \lambda$. 
  \end{definition}
  Let's compute:
%
%
  \begin{equation*}
    \begin{split}
      \psi_{xxt}
      & = \left( (\frac{1}{4} - m \lambda)\psi \right)_{t}
      \\
      & = -m_{t} \lambda \psi + \left( \frac{1}{4} - m \lambda \right)
      \psi_{t}
      \\
      & = -m_{t} \lambda \psi + \left( \frac{1}{4} - m \lambda \right)\left(
      -(u + \frac{1}{2} \lambda^{-1}\psi_{x} + \frac{1}{2} u_{x} \psi)
      \right)
      \\
      & = \left[ -m_{t} \lambda + (\frac{1}{4}- m \lambda)\frac{1}{2}u_{x} \right]\psi
      - (\frac{1}{4} - m \lambda)(u + \frac{1}{2}\lambda^{-1})\psi_{x}
    \end{split}
  \end{equation*}
%
%
  while
%
%
  \begin{equation*}
    \begin{split}
      \psi_{txx}
      & = \left[ (u + \frac{1}{2} \lambda^{-1})\psi_{xx} + u_{x}
	\psi_{x} - \frac{1}{2} u_{x} \psi_{x} - \frac{1}{2} u_{xx} \psi
      \right]_{x}
      \\
      & = \left[ (u + \frac{1}{2}\lambda^{-1})(\frac{1}{4} - m \lambda)\psi
      + \frac{1}{2} u_{x} \psi_{x} - \frac{1}{2} u_{xx} \psi \right]_{x}
      \\
      & = \left( \left[ (u + \frac{1}{2} \lambda^{-1})(\frac{1}{4} - m \lambda) -
      \frac{1}{2} u_{xx} \right]\psi + \frac{1}{2} u_{x} \psi_{x} \right)_{x}a
      \\
      & = \frac{1}{2} u_{x}(\frac{1}{4} - m \lambda)
      \\
      & = \frac{1}{2}u_{x}\left( \frac{1}{4} - m \lambda \right)\psi + \left[
	\frac{1}{2}u_{xx} - \frac{1}{2} u_{xx} + (u +
	\frac{1}{2}\lambda^{-1})(\frac{1}{4} - m \lambda)
      \right]\psi_{x}
      \\
      & + u_{x}\left( (\frac{1}{4} - m \lambda) - m_{x} \lambda (u +
      \frac{1}{2} \lambda^{-1}) - \frac{1}{2} u_{xxx}\right)\psi
    \end{split}
  \end{equation*}
%%
  Some mistake in the computation--but idea is above.
%
  For a shortcut, set $V = m \lambda - \frac{1}{4}, A = u +
  \frac{1}{2}\lambda^{-1}, B = -\frac{1}{2}u_{x} = -\frac{1}{2}A_{x}$. 
  Then 
%
%
  \begin{equation*}
    \begin{split}
      & \psi_{xx} + V \psi = 0
      \\
      & \psi_{t} + A \psi_{x} - \frac{1}{2} A_{x} \psi = 0
    \end{split}
  \end{equation*}
%
%
  and so
%
%
  \begin{equation*}
    \begin{split}
      \psi_{xxt}
      & = -(V \psi)_{t}
      \\
      & = -V_{t} \psi - V \psi_{t}
      \\
      & = -V_{t} \psi + V\left( -A \psi_{x} + \frac{1}{2} A_{x} \psi \right)
      \\
      & = \left( -V_{t} + \frac{1}{2} A_{x}V \right)\psi - A V \psi_{x}
    \end{split}
  \end{equation*}
%
%
  while
%%
%%
  \begin{equation*}
    \begin{split}
      \psi_{txx}
      & = \left( -A \psi_{x} + \frac{1}{2}A_{x} \psi \right)_{xx}
      \\
      & = \left( -A \psi_{xx} - A_{x} \psi_{x} + \frac{1}{2} A_{x} \psi_{x} +
      \frac{1}{2} A_{xx} \psi \right)_{x}
      \\
      & = \left( AV \psi - \frac{1}{2} A_{x} \psi_{x} + \frac{1}{2} A_{xx}\psi
      \right)_{x}
      \\
      & = \left[ \left( \frac{1}{2}A_{xx} + AV \right)\psi - \frac{1}{2}A_{x}
      \psi_{x} \right]_{x}
      \\
      & = \psi_{txx} = \left( \frac{1}{2}A_{xxx} + A_{xx}V + A V_{x} \right)\psi 
      \\
      & + \left( \frac{1}{2}A_{xx} + AV - \frac{1}{2}A_{xx} \right)\psi_{x} -
      \frac{1}{2} A_{x} \psi_{xx}
      \\
      & = \left( \frac{1}{2}A_{xxx} + A_{x}V + AV_{x} + AV_{x} +
      \frac{1}{2}A_{x}V \right) \psi + AV \psi_{x}
    \end{split}
  \end{equation*}
%%
%%
  Then answer should be
%
%
  \begin{equation*}
    \begin{split}
      -V_{t} + \left( \frac{1}{2}D_{x}^{3} + 2V D_{x} + V_{x}\right)A = 0
    \end{split}
  \end{equation*}
%%
%%
  Expanding, get
%
%
  \begin{equation*}
    \begin{split}
      & \text{coefficients of $\lambda$}:  m_{t} + 2mu_{x} + m_{x} u = 0
      \\
      &  \text{coefficiients of $\lambda^{0}$}: \frac{1}{2} u_{xxx} - \frac{1}{2}
      u_{x} + \frac{1}{2} m_{x} =  0.
    \end{split}
  \end{equation*}
%%
%%
  \subsection{Matrix Lax Pair for CH} 
  \label{ssec:matrix-lax}
%
%
  \begin{equation*}
    \begin{split}
      & \p_{x} \Psi = U \Psi
      \\
      & \p_{t} \Psi = V \Psi, \quad \text{where} \ \Psi = 
      \begin{bmatrix}
	\psi & \tilde{\psi}
	\\
	\psi_{n} & \tilde{\psi_{n}}
      \end{bmatrix}
    \end{split}
  \end{equation*}
%%
%%
%%
%%
  \begin{equation*}
    \begin{split}
      U = 
      \begin{bmatrix}
	0 & 1
	\\
	-m\lambda + \frac{1}{4} & 0
      \end{bmatrix}
    \end{split}
  \end{equation*}
%
%
  and 
%
%
  \begin{equation*}
    \begin{split}
      V = 
      \begin{bmatrix}
	\frac{1}{2} u_{x} & -u - \frac{1}{2} \lambda^{-1}
	\\
	um\lambda + \frac{1}{4}u - \frac{1}{8}\lambda^{-1} & -\frac{1}{2}u_{x}
      \end{bmatrix}
    \end{split}
  \end{equation*}
%
%
  Compatibility $\Psi_{xt} = \Psi_{tx}$ or $\left[ \p_x - U, \p_t -V \right]=0$
  and $U_{t} - V_{x} + \left[ U, V \right]=0$. Now, can you use the lax pair to
  find the bi-hamiltonian structure? Suppose $\psi, \tilde{\psi}$ be two
  independent solutions of $\psi_{xx} + (m \lambda - \frac{1}{4})\psi = 0$. Let
  $P = \psi \tilde{\psi}$. $\implies P_{xxx} + 4(m \lambda -
  \frac{1}{4})P_{x} + 2 m_{x} \lambda P = 0$. $\implies \left[ (D_{x}^{3} - D_{x})
  + \lambda(4m D_{x} + 2 m_{x})\right]P = 0$. $\implies (B_{1} - 2 \lambda
  B_{0})P = 0$. Taking $\lambda \to \infty$, and doing asymptotic expansion, we
  get $P \sim P_{o} +
  \lambda^{-1} + \lambda^{-2}P_{2} +\cdots$, where $P_{0} \alpha m^{-1/2} \alpha
  \frac{\delta c}{\delta m}, P_{1} \alpha \frac{\delta H_{3}}{\delta m}$.
  Another way to get bi-hamiltonian structure is to take $\rho = (\log
  \psi)_{x}$ and $\rho_{t} = \left[ (\log \psi)_{t} \right]_{x}$, i.e.
  $\rho_{t} = F$. 
%
%
%%%%%%%%%%%%%%%%%%%%%%%%%%%%%%%%%%%%%%%%%%%%%%%%%%%%%
%
%
%                Peakonomics
%
%
%%%%%%%%%%%%%%%%%%%%%%%%%%%%%%%%%%%%%%%%%%%%%%%%%%%%%
%
%
  \section{Peakonomics} 
  \label{sec:peakonomics}
  Collaborators: Johnny Kelsey, Francesca Medda, Alan Wilson. Interested in
  pattern formation in economic systems. Looked at a model of Krugman (Nobel
  Economics '08). Wrote book called ``The Self-Organizing Economy'' ('96).
  Modeling: main variable is $\lambda (x,t)$, called the ``business density''.
  It's the density of how much economic activity is going on. 
%
%
  The $\lambda$, in CH notation,
  is actually the $m$, or momentum density. Another quantity
  $P(x,t)$, called ``market potential''. In CH notation, this is $u$, or velocity
  field. $G(x) = Ae^{-r_{1} | x |} - Be^{-r_{2}| x |}$. In CH notation,
  $g(x) = \frac{1}{2}e^{-| x |}$. Also, $P = G* \lambda = \int G(x-y) \lambda
  (y,t) dy$, where integral is over $\rr$ or $\ci$. In CH, $u = g*m$.
  Now %
%
  \begin{equation*}
    \begin{split}
      & P(x,t) = \int \left( A e^{-r_{1}| x-y |} - Be^{-r_{2}| x-y |}  \right)
      \lambda(y,t) dt \quad \text{first term with A centripetal, second centrifugal}
      \\
      & \frac{r_{1}}{r_{2}} > \frac{A}{B} > \frac{r_{2}}{r_{1}} \quad \text{all
      params positive}
    \end{split}
  \end{equation*}
%
%
  $G(x)$ looks like mexican hat--symmetric about origin, triangle that dips down
  into negative $y$ range, then slowly twists back up to $y=0$. Another quantity
  is $\bar{P}(t) = \int P(x,t) \lambda(x,t) dx$. Now, Knigman's model:
%
%
  \begin{equation*}
    \begin{split}
      \frac{d \lambda}{dt} = \gamma \lambda (P - \bar{P}), \quad \gamma >0 \quad
      \text{Set $\gamma=1$ henceforth}
    \end{split}
  \end{equation*}
%
%
  Implicit assumption
%
%
  \begin{equation*}
    \begin{split}
      \int \lambda(x,t)dx=1
    \end{split}
  \end{equation*}
%
%
  Better definition of the model: Let $\lambda$, $P$ be as before, but let
  $\bar{P}(t) = \frac{\int P(x,t) \lambda(x,t)dx}{\int \lambda(x,t)dx}$.
  $C = \int \lambda(x,t) dx$ with $\bar{P} = \int P \lambda dx / C$ the model is
  dimensionally correct.
%
%
%
%
%%%%%%%%%%%%%%%%%%%%%%%%%%%%%%%%%%%%%%%%%%%%%%%%%%%%%
%
%
%                conservatino of probability
%
%
%%%%%%%%%%%%%%%%%%%%%%%%%%%%%%%%%%%%%%%%%%%%%%%%%%%%%
%
%
  \begin{lemma}
%
    We have the conservation of probability
%
    \begin{equation*}
      \begin{split}
	\frac{dC}{dt} =0.
      \end{split}
    \end{equation*}
%
%
    \label{lem:cons-prob}
  \end{lemma}
%
%
  So $C(t) = C(0) \forall t$. 
%
%
%
%
%%%%%%%%%%%%%%%%%%%%%%%%%%%%%%%%%%%%%%%%%%%%%%%%%%%%%
%
%
%                
%
%
%%%%%%%%%%%%%%%%%%%%%%%%%%%%%%%%%%%%%%%%%%%%%%%%%%%%%
%
%
  \begin{lemma}
%
%
    \begin{equation*}
      \begin{split}
	\frac{d \bar{P}}{dt} = 2 \text{var}(P) \ge 0
      \end{split}
    \end{equation*}
%
%
    The general model is as before, 
%
%
    \begin{equation*}
      \begin{split}
	\frac{\p \lambda}{\p t} = \lambda (P - \bar{P}) 
      \end{split}
    \end{equation*}
%
%
    with $\bar{P}$ redefined accordingly; $G$ could be different.
    \label{lem:deriv-barP}
  \end{lemma}
%
%
  Analysis easier on the circle. Linearize around constant solution
  $\lambda(x,t) = 1/L$, where $\ci = [-L/2, L/2]$. Do perturbation analysis.
  Get ODEs for Fourier modes, which grow or decay. (Turing Mechanism gives another
  example).
%
%
%
%
%%%%%%%%%%%%%%%%%%%%%%%%%%%%%%%%%%%%%%%%%%%%%%%%%%%%%
%
%
%                main thm
%
%
%%%%%%%%%%%%%%%%%%%%%%%%%%%%%%%%%%%%%%%%%%%%%%%%%%%%%
%
%
  \begin{theorem}
    Krugman's model has exact solutions given by a sum of point masses
%
%
    \begin{equation*}
      \begin{split}
	\lambda(x,t) = \sum_{j=1}^{N} a_{j}(t) \delta(x - x_{j})
      \end{split}
    \end{equation*}
%
%
    where $x_{j} = \text{const} \ \forall t$ and $a_{j}$ evolve according to 
%
%
    \begin{equation*}
      \begin{split}
	\frac{d a_{j}}{dt} = a_{j}\left[ \sum_{k=1}^{N} a_{k} G(x_{j}, x_{k}) -
	\frac{1}{C} \sum_{k, \ell} a_{k}a_{\ell} G(x_{k} - x_{\ell}) \right] 
      \end{split}
    \end{equation*}
%
%
    with $C = \sum a_{j} = \text{const}$.
    \label{thm:main-theorem}
  \end{theorem}
%
%
  In other words, in model, everything gets sucked into a single point--Krugman
  was not looking for this\ldots wanted pockets of economic activity to form
  everywhere. Integration is performed using Riemann sums\ldots and derivatives
  are handled using finite differences (Method of Lines). 
%
%
%
%%%%%%%%%%%%%%%%%%%%%%%%%%%%%%%%%%%%%%%%%%%%%%%%%%%%%
%
%
%                Somos Sequences
%
%
%%%%%%%%%%%%%%%%%%%%%%%%%%%%%%%%%%%%%%%%%%%%%%%%%%%%%
%
%
  \section{Somos Sequences} 
  \label{sec:somos}
  Graham Everest introduced Hone. Michael Somos (early '80's). 
%
%
  \begin{equation*}
    \begin{split}
      x_{n+4}x_{n} = x_{n+3}x_{n+1} + x_{n+2}^{2}.
    \end{split}
  \end{equation*}
%
%
  Initial data $x_{0}, x_{1}, x_{2}, x_{3} =1$. Relation above gives
  \\ $1,1,1,1,2,3,7,23,59,314,1529,8209,83313,\cdots$. 
%
%
%
%%%%%%%%%%%%%%%%%%%%%%%%%%%%%%%%%%%%%%%%%%%%%%%%%%%%%
%
%
%                
%
%
%%%%%%%%%%%%%%%%%%%%%%%%%%%%%%%%%%%%%%%%%%%%%%%%%%%%%
%
%
  \begin{proposition}
    With these initial data, $x_{n} \in \zz \ \ \forall n \in \zz$. 
    \label{prop:somos}
  \end{proposition}
%
%
  \subsection{Prehistory} 
  \label{ssec:prehistory}
%
%
  \begin{equation*}
    \begin{split}
      F_{n+1} = F_{n} + F_{n-1}, \quad F_{0}=0, F_{1}=1
    \end{split}
  \end{equation*}
%
%
  But for Fibonacci sequence, $(F_{n})$ is a divisibility sequence, i.e.
  $F_{n} \vert F_{m}$ whenever $n \vert m$. Similarly, Lucas sequences.
  M. Ward ($'40$).
%
%
  \begin{equation*}
    \begin{split}
      x_{n+4} x_{n} = \alpha x_{n+3}x_{n+1} + \beta x_{n+2}^{2}
    \end{split}
  \end{equation*}
%
%
  Take $x_{1} =1, x_{2},x_{3}, x_{4} \in \zz, x_{2} \vert x_{4}, \ \text{and} \
  \alpha = x_{2}^{2}, \beta = -x_{1} x_{3}$. 
%
%
%%%%%%%%%%%%%%%%%%%%%%%%%%%%%%%%%%%%%%%%%%%%%%%%%%%%%
%
%
%                Ward
%
%
%%%%%%%%%%%%%%%%%%%%%%%%%%%%%%%%%%%%%%%%%%%%%%%%%%%%%
%
%
  \begin{theorem}[Ward]
%
%
    \begin{equation*}
      \begin{split}
	x_{n} \in \zz \ \forall n \in \zz \ \text{and} \ x_{n} \vert x_{m} \
	\text{whenever} \ n \vert m.
      \end{split}
    \end{equation*}
%
%
%
%
%
%
    \label{thm:main}
  \end{theorem}
%
%
  \begin{proof}
    Use the fact that %
%
    \begin{equation*}
      \begin{split}
	x_{n} = \frac{\sigma(nv)}{\sigma(v)^{n^{2}}}
      \end{split}
    \end{equation*}
%
%
    where $\sigma(z) = \sigma(z, g_{2}, g_{3})$ is the Weirstrauss $\delta$ function
    associated with the elliptic curve
%
%
    \begin{equation*}
      \begin{split}
	E: y^{2}=4x^{3} - g_{2}x - g_{3}
      \end{split}
    \end{equation*}
%
%
  \end{proof}
  Hilbert's $10$th problem--undecidability proved by Matiyasevich ('70)
  using linear recurrence sequences.
  Question: Why does Somos-4 give integers? Answer: Because the recurrence has the
  Laurent property. General Somos-4:
%
%
  \begin{equation*}
    \begin{split}
      x_{n+4}x_{n} = \alpha x_{n+3} x_{n+1} \beta x_{n+2}^{2}
    \end{split}
  \end{equation*}
%
%
  The Laurent property for the above syas that $x_{n} \in \zz[x_{0}^{\pm 1},
  x_{1}^{\pm 1}, \cdots, x_{3}^{\pm 1}, \alpha, \beta ]$ for all $n \in \zz$. 
%
%
  \begin{proof}
    Do modular arithmetic in $\rr$, using induction with any $4$ adjacent terms
    coprime. For more details, see David Gale ``Tracking the Automatic Ant''.
  \end{proof}
%
%
  Hirota-Miwa KDV transformation reduces KDV to Somos-4. Look it up.
%
%
%%%%%%%%%%%%%%%%%%%%%%%%%%%%%%%%%%%%%%%%%%%%%%%%%%%%%
%
%
%                Liu
%
%
%%%%%%%%%%%%%%%%%%%%%%%%%%%%%%%%%%%%%%%%%%%%%%%%%%%%%
%
%
  \section{Liu-Gromov Witten Theory} 
  \label{sec:liu}
  Considering infinite system of differential equations--also called integrable
  hierarchy. 
  \begin{gather*}
    u = u(x)
    \\
    R_{n+1}(u, u_{x}, u_{xx}, \cdots)
    \\
    \frac{d}{dx} R_{n+1}(u) = \frac{1}{2n+1} \left( u_{x} + 2u \frac{d}{dx} +
    \frac{1}{4}\left( \frac{d}{dx} \right)^{3} \right) R_{n}(u)
    \\
    R_{1}(u) = u
    \\
    R_{2}(u) = \frac{1}{2} u^{2} + \frac{1}{12} u_{xx}.
  \end{gather*}
  n-th eqn in KDV hierarcy
%
%
  \begin{gather*}
    u = u(t,x)
    \\
    u_{t} =  \frac{d}{dx}R_{n+1}, \quad n = 0,1,2,\cdots
    \\
    n = 0, \quad u_{t} = u_{x}
    \\
    n=1, \quad u_{t} = uu_{x} + \frac{1}{12}u_{xxx}, \quad \text{KDV}
    \\
    n=2, \quad \text{and so on}
  \end{gather*}
  We say
  \begin{gather*}
    Z = Z(x, t_{0}, t_{1}, t_{2}, \cdots)
  \end{gather*}
  is a $\tau$-function of KDV hierarchy if
  \begin{gather*}
    u(x, t_{0}, t_{1}, \cdots) \doteq \frac{\p^{2} \log Z}{\p x^{2}}
  \end{gather*}
  is a solution for the $n$-th equation in KDV hierarchy if $x =x$, $t =
  t_{n}$. 
  Define $M_{g,k}$ = space of genus-g stable curves with $k$ marked points.
  Basically Riemann surface with $k$ punctures. Deligne-Mumford compactification $ M_{g,n} \subset \bar{M_{g,k}}$.
  \begin{gather*}
    \bar{M_{g,k}} = \left\{ (C; x_{1}, \cdots, x_{k}) \right\}, \quad C \
    \text{curve} 
  \end{gather*}
  Line bundle $T_{x_{2}}^{*}C \subset E_{2}$.
  \begin{gather*}
    \langle \tau_{n_{1}}, \cdots, \tau_{n_{k}} \rangle_{g} \doteq
    \int_{\bar{M_{g,k}}} c_{1}(E_{1})^{n_{1}} \times \cdots \times
    c_{1}(E_{k})^{n_{k}}
  \end{gather*}
  \begin{gather*}
    F_{g}(t_{0}, t_{1}, \cdots) = \sum_{k \ge 0} \frac{1}{k!} \sum_{n_{1}, \ldots,
    n_{k} \ge 0} t_{n_{1}} \times\cdots\times t_{n_{k}} \langle \tau_{n_{1}}
    \times\cdots \times \tau_{n_{k}} \rangle_{g}.
  \end{gather*}
%
%
%%%%%%%%%%%%%%%%%%%%%%%%%%%%%%%%%%%%%%%%%%%%%%%%%%%%%
%
%
%                
%
%
%%%%%%%%%%%%%%%%%%%%%%%%%%%%%%%%%%%%%%%%%%%%%%%%%%%%%
%
%
  \begin{theorem}[Witten's Conjecture]
    \begin{gather*}
      Z = e^{\sum_{g \ge 0} F_{g}(t)}
    \end{gather*}
    is a $\tau$-function of KDV hierarchy. $t_{0} \simeq x$
    \label{thm:witt}
  \end{theorem}
%
%
  $\bar{M_{0,3}} = \left\{ pt \right\}$, 
  \begin{gather*}
    \langle \tau_{0} \tau_{0} \tau_{0} \rangle  = \int_{\bar{M_{0,3}}} 1 =1
  \end{gather*}
  Witten's conjecture determines all
  \begin{gather*}
    \langle \tau_{n_{1}}, \cdots, \tau_{n_{k}} \rangle. 
  \end{gather*}
  Kontesevich proved the conjecture. Reduces difficult problem of computing $\tau$
  functions to calculus\ldots don't even need to know $\tau$ functions now, thanks
  to Witten conjecture. Define
  \begin{gather*}
    L_{-1} = -\frac{\p}{\p t_{0}} + \frac{1}{2} t_{0}^{2} + \sum_{i=1}^{\infty}
    t_{i+1} \frac{\p}{\p t_{2}}
  \end{gather*}
  String equation: $L_{-1} Z = 0$.
	%\nocite{*}
	%\bibliography{/Users/davidkarapetyan/Documents/math/}
  \section{Notion of Integrability}
  \label{sec:}
  Consider equations of form
%
%
  \begin{equation*}
    \begin{split}
      (1 - \p_x^{2})u_{t} = P(u, u_{x}, u_{xx}, u_{xxx}, \cdots).
    \end{split}
  \end{equation*}
%
%
%
%
%%%%%%%%%%%%%%%%%%%%%%%%%%%%%%%%%%%%%%%%%%%%%%%%%%%%%
%
%
%                thm
%
%
%%%%%%%%%%%%%%%%%%%%%%%%%%%%%%%%%%%%%%%%%%%%%%%%%%%%%
%
%
  \begin{theorem}[Mikhailov-Novikov, 2002]
    If equation (b-family)
%
%
    \begin{equation*}
      \begin{split}
	m_{t} = b m u_{x} + u m_{x}, m = (1 - \p_x^{2})u = u - u_{xx}
      \end{split}
    \end{equation*}
%
%
    possesses an ``infinite hierarchy of (quasi) local higher symmetries'', i.e.\
    integrability, then
    $b =2$ (CH) or $b =3$ (DP)
    \label{thm:main-theorem2}
  \end{theorem}
%
%
%
%
%%%%%%%%%%%%%%%%%%%%%%%%%%%%%%%%%%%%%%%%%%%%%%%%%%%%%
%
%
%                
%
%
%%%%%%%%%%%%%%%%%%%%%%%%%%%%%%%%%%%%%%%%%%%%%%%%%%%%%
%
%
  \begin{theorem}[Novikov, 2009]
    The equation
%
%
    \begin{equation*}
      \begin{split}
	& (1 - \ee^{2} \p_{x}^{2})u_{t} = c_{1} uu_{x} + \ee [c_{2} u u_{xx} +
	c_{3} u_{x}^{2}] + \ee^{2}[c_{4} uu_{xx} + c_{5} u_{x}u_{xxx}]
	\\
	& +
	\ee^{3}[c_{6} u u_{xxxx} + c_{7} u_{x} u_{xxx} + c_{8}u_{xx}^{2}] +
	\ee^{4}[c_{9} u u _{xxxxx} + c_{10} u_{x} u_{xx} + c_{11} u_{xx}u_{xxx}]
      \end{split}
    \end{equation*}
%
    with at least one of the following conditions not satisfied.
    \begin{enumerate}
      \item{$c_{2} = 0$}
      \item{$c_{6} = 0$ }
      \item{$c_{1} + c_{4} = 0$}
    \end{enumerate}
    is integrable then it is one in the following list of $10$ equations.
    \begin{enumerate}
      \item{CH }
      \item{DP}
      \item{$(1 - \ee^{2} \p_x^{2}) \p_{t} u = \p_{x} [(4 - \ee^{2}
	\p_{x}^{2})u]^{2}$}
      \item{And so on}
    \end{enumerate}
%
    \label{thm:integ}
  \end{theorem}
%
%
%
%%%%%%%%%%%%%%%%%%%%%%%%%%%%%%%%%%%%%%%%%%%%%%%%%%%%%
%
%
%                
%
%
%%%%%%%%%%%%%%%%%%%%%%%%%%%%%%%%%%%%%%%%%%%%%%%%%%%%%
%
%
  \begin{theorem}[Novikov]
    If $P(\cdots)$ is of weight -1 (each derivative is +1 weight, $\ee$ is -1 weight)
    ind satisfies some conditions implies $8$ more integrable equations.
  \end{theorem}
%
%
%
%
%%%%%%%%%%%%%%%%%%%%%%%%%%%%%%%%%%%%%%%%%%%%%%%%%%%%%
%
%
%                
%
%
%%%%%%%%%%%%%%%%%%%%%%%%%%%%%%%%%%%%%%%%%%%%%%%%%%%%%
%
%
  \begin{theorem}[Novikov]
    If $P(\cdots)$ has weight=1 and cubic nonlinearities + condition
    implies 10 integrable equations and one of them is NEW, and called the Novikov
    equation.
%
%
    \begin{equation*}
      \begin{split}
	(1 - \p_{x}^{2})u_{t} = -4u^{2} u_{x} + 3 u u_{x} u_{xx} + u^{2} u_{xxx}. 
      \end{split}
    \end{equation*}
%
    It has 
    \begin{enumerate}
      \item{Infinitely many symmetries}
      \item{(Hone et al) Lax pair}
      \item{Infinitely many conserved quantitites}
      \item{Bihamiltonian structure}
      \item{??Can be solved using the inverse scattering method??}
    \end{enumerate}
%
  \end{theorem}
%
%
%
%%%%%%%%%%%%%%%%%%%%%%%%%%%%%%%%%%%%%%%%%%%%%%%%%%%%%
%
%
%                
%
%
%%%%%%%%%%%%%%%%%%%%%%%%%%%%%%%%%%%%%%%%%%%%%%%%%%%%%
%
%
  \begin{theorem}[Degasperis-Procesi]
%
%
    \begin{equation*}
      \begin{split}
	u_{t} + c_{0} u_{x} + \gamma u_{xxx} = (c_{1} u^{2} + c_{2} u_{x}^{2} +
	c_{3} u_{xx})_{x}
      \end{split}
    \end{equation*}
%
%
    is integrable implies
    \begin{enumerate}
      \item{CH, or}
      \item{DP, or}
      \item{KDV}
    \end{enumerate}
  \end{theorem}
%
%
%
%
%%%%%%%%%%%%%%%%%%%%%%%%%%%%%%%%%%%%%%%%%%%%%%%%%%%%%
%
%
%                
%
%
%%%%%%%%%%%%%%%%%%%%%%%%%%%%%%%%%%%%%%%%%%%%%%%%%%%%%
%
%
  \begin{theorem}[Khesim-Misiolek]
    \ldots geodesic equation on some object is intregrable implies
    \begin{enumerate}
      \item{KDV, or}
      \item{CH, or}
      \item{Hunter-Saxton}
    \end{enumerate}
%
  \end{theorem}
%
%
%
%
%%%%%%%%%%%%%%%%%%%%%%%%%%%%%%%%%%%%%%%%%%%%%%%%%%%%%
%
%
%                
%
%
%%%%%%%%%%%%%%%%%%%%%%%%%%%%%%%%%%%%%%%%%%%%%%%%%%%%%
%
%
  \begin{theorem}[Fokas-Fuchsteiner]
    \ldots a family is integrable implies
    \begin{enumerate}
      \item{KDV, or}
      \item{CH}
    \end{enumerate}
  \end{theorem}
%
%
%
%
  \begin{framed}
    \begin{remark}
      Different researchers have different machinery. Spits out some equations, not
      others.
    \end{remark}
  \end{framed}
%
%
%
%
%%%%%%%%%%%%%%%%%%%%%%%%%%%%%%%%%%%%%%%%%%%%%%%%%%%%%
%
%
%                
%
%
%%%%%%%%%%%%%%%%%%%%%%%%%%%%%%%%%%%%%%%%%%%%%%%%%%%%%
%
%
  \begin{theorem}
    The Novikov equation is well-posed in $H^{S}$ if $s > 3/2$ and dependence on
    initial data is continuous.
  \end{theorem}
%
%
%
%
%%%%%%%%%%%%%%%%%%%%%%%%%%%%%%%%%%%%%%%%%%%%%%%%%%%%%
%
%
%                
%
%
%%%%%%%%%%%%%%%%%%%%%%%%%%%%%%%%%%%%%%%%%%%%%%%%%%%%%
%
%
  \begin{theorem}
    The dependence is not better than continuous.
  \end{theorem}
%
%
%
%
  \begin{framed}
%
%
    \begin{remark}
      Novikov equation has PEAKON SOLUTIONS and CONSERVES $H^{1}$. First equation
      with cubic nonlinearity that has peakon solutions. 
    \end{remark}
%
%
  \end{framed}
%
%
%
%
%%%%%%%%%%%%%%%%%%%%%%%%%%%%%%%%%%%%%%%%%%%%%%%%%%%%%
%
%
%				Midwest PDE Seminar
%
%
%%%%%%%%%%%%%%%%%%%%%%%%%%%%%%%%%%%%%%%%%%%%%%%%%%%%%
%
%
  \section{Midwest PDE Seminar} 
  \label{sec:mpde}
  \subsection{Ponce--Benjamin Ono} 
  \label{ssec:ponce-benono}
  \begin{itemize}
    \item{}
      Molinet-Pilot--survey of Benjamin Ono
    \item{}
      Canton-uniqueness of traveling waves to BO equation
  \end{itemize}
  Main Tools:
  \begin{enumerate}
    \item{}
      Ap condition Muckenhoupt. Hunt-Muckenhoupt-Whedden Theorem--Hilbert transform bounded $L^{p}$ iff Ap condition holds.
    \item{}
      Calderon Commutator Theorem for Hilbert Transforms
    \item{}
      Sobolev space non-fourier transform definition gives
  %
  %
      \begin{equation*}
	\begin{split}
	  \| D^{b}(fg) \|_{2} \lesssim \| fD^{b}g \|_{2} + \| gD^{b}f \|_{2}.
	\end{split}
      \end{equation*}
  \end{enumerate}
  %
  %
  \subsection{Colliander} 
  \label{ssec:col-web}
  \begin{enumerate}
    \item{} Noether's Theorem--big deal relating conserved quantitites to integrability. See Tao's book as well.
    \item{} KDV-Jams ``Sharp Global\ldots'' one of best articles for learning I-method. For more info, check his website (sent it to you via email).
  \end{enumerate}
    %
    %
    %%%%%%%%%%%%%%%%%%%%%%%%%%%%%%%%%%%%%%%%%%%%%%%%%%%%%
    %
    %
    %				Numerics
    %
    %
    %%%%%%%%%%%%%%%%%%%%%%%%%%%%%%%%%%%%%%%%%%%%%%%%%%%%%
    %
    %
  \section{Numerics} 
  \label{sec:numerics}
  \subsection{Xu} 
  \begin{enumerate}
    \item{}
      Cockburn and Shu--TUB Runge Kutta, 1989
    \item{}
      Waburton, Haggstrom
    \item{}
      Liu, Shu, Tadmor, $L^{p}$ Stability analysis, 2009.
    \item{}
      Z.L Xu, Shu, Xu--Hierarchical reconstruction with up to, 2009
  \end{enumerate}
  \section{Ming Chen Oct 2012}
  \begin{enumerate}
    \item{}
      Long wave abcd system:
      \begin{equation*}
	\begin{split}
	  & \eta_{t} + u_{x} + (\eta u)_{x} + au_{xxx} - b\eta_{txx} =0
	\end{split}
      \end{equation*}
  \end{enumerate}
  Conserved quantitites:
  \begin{equation*}
    \begin{split}
      & H(\eta, u) = \frac{1}{2} \int_{\rr}[-c \eta_{x}^{2} - av_{x}^{2} + \eta^{2} + (1 + \eta)u^{2}] dx,
      \\
      & I = \int_{\rr} (\eta u + b \eta_{x} u_{x}) dx, \quad \text{impulse}
      \\
      & \int u dx, \quad \int_{\rr} \eta dx.
    \end{split}
  \end{equation*}
  We assume $b = d$.
  \\
  $\bullet$ Chen-Nguyen-Sun, 2009. Hamilton-impulse minimizers when $a,c<0$.
  \\
  $\bullet$ Chen-Nguyen-Sun, 2011. Quadratic-cubic splitting. Also show all solitary waves are smooth, with decay at infinity $H^{\infty}$.
  \begin{itemize}
    \item
      Ground state $\vec{u} = (\eta, u)$  is an $H^{1} \times H^{1}$ solitary wave solution which minimizes $S_{w} = H - $.
    \item
      Result comes by ignoring cubic term (weak convergence to 0) via a refined Fatou's lemma (Brezis-Lieb).
    \item Ground state theorem Bao-Chen-Liu.
    \item{}
      Current state, assume $a, c < 0$ for well-posedness. $L^{2}$ critical for certain $a,b,c,d$.
    \item{}
      Saut has results on local regularity.
  \end{itemize}
  \begin{itemize}
    \item{}
      What's the connection with Hamiltonian-impulse soliary waves?
    \item{}
      Made smallness assumption on smallness speed--can we get rid of it?
    \item{}
      Big problem: uniqueness of ground state?
    \item{}
      Stability?
    \item{}
      Symmetry of solitary waves? Only proved symmetry for ground states.
  \end{itemize}
\section{Greenblatt}
\begin{itemize}
\item{}
Look at oscillatory integral
\begin{equation*}
\begin{split}
& I(\lambda) = \int_{\rr^{n}} e^{i \lambda S(x_{1}, x_{2}, \ldots, x_{n})} \vp(x_{1}, \cdots, x_{n}) dx
\\
& S(x) = \text{phase functions analytic},
\\
& S(0) = 0, 
\\
& \vp = \text{cutoff function near 0}, 
\\
& \lambda=\text{real parameter}
\end{split}
\end{equation*}
\item{}
Want best estimate
\begin{gather*}
  | I(\lambda) | \le C| \lambda |^{-\ee}
\end{gather*}
\item{}
Want to relate $\ee$ to $S(x)$, i.e. it's nice properties.
\item Definitions:
\\
$S(x) = \sum_{\alpha} s_{\alpha} x^{\alpha}$. Taylor Expansion.
\\
Given $\alpha$ for which $s_{\alpha}\neq 0$, let $Q_\alpha = \left\{ x \in \rr^{n}: x \ge \alpha_{i} \forall i \right\}$.
\\
$N(S) = \text{Newton polyhedron of S}= \text{convex hull of all} Q_{\alpha}$.
  \begin{theorem}[Varchenko]
    Under nondegeneracy condition,
    \begin{gather*}
      | I(\lambda) | \le C| \lambda |^{-1/d(S)} | \ln(\lambda) |^{n-k-1}
    \end{gather*}
    where $k$ is the dimension of face of $N(S)$ intersecting line $(t, t, \cdots, t, t)$, where
    \begin{gather*}
      d(S) = \sup_{t}: (t, t, \cdots, t, t) \nin N(S).
    \end{gather*}
    \item{}
      Nondegeneracy Condition: Given any compact face $F$ of $N(S)$, let $S_{F}(	x = \sum_{\alpha \in F} s_{\alpha} x^{\alpha}$. Condition $\grad S_{F} \neq 0$ on $(\rr - \left\{ 0 \right\})^{n}$ for all $F$. 
\end{theorem}
\item{}
  Questions: Can you relax nondegeneracy condition? If not, are there other Newton polyhedrom based statements? Reasons to think this can be done: Phong/Stein/Sturm proved multinlinear operator analogues, didn't require nondegen cond.
  \item{}
    Algorithm: Divide neighborhood of origin into horns as follows: One horn $H_{i}$ corresponds to each compact face $F_{i}$. On $H_{i}$, you have if $v, v' \in F$, $c_{i}^{-1} \le \frac{| x^{v} |}{x^{v_{'}}} \le C_{i}$. And if $v \in F_{i}$, $v' \nin F_{i}$, then $| x^{v'} | < \frac{1}{C_{i}^{N}}| X^{v} |$, $N >>0$. 
\end{itemize}
\section{Erdogan \\ Mathematical Problems Arising from Nonlinear Fiber Optics}

\begin{itemize}
  \item 
$u = complex amplitutde, \\
t = position along cable, x = Retarded time, d = dispersion along the cable$. 

    \begin{gather*}
    iu_{t} + d(t) u_{xx} + | u |^{2} u = 0
    \\
    u(0,x) = \phi(x), \quad x \in \rr.
    \end{gather*}
\end{itemize}
\begin{itemize}
  \item Way to Solve--split step method--solve linear Schrodinger and purely nonlinear schrodinger numerically--solution converges to solution to NLS.
    \item{}
    Shape of the pulse remains the same in solution for purely nonlinear NLS.
    \item{}
    Solitons do not change their shape for NLS during evolution.
    \item{}
    Zakharov, Manakov (99). Erdogan, Zharnitsky, Commun. Math. Phys 2008.
    \item{}
    Tzirakis, Zharnitsky, Near-linear dynamics in KDV with periodic boundary conditions.
\end{itemize}
  \end{document}
