\documentclass{beamer}
\setbeamersize{text margin left=0.4cm, text margin right=0.4cm}
%\usetheme{Boadilla}
\usetheme{Madrid}
\setbeamertemplate{theorems}[numbered]
\setbeamercovered{transparent}
\usepackage{amscd}
\usepackage{amsfonts}
\usepackage{amsmath}
\usepackage{amssymb}
\usepackage{amsthm}
\usepackage{fancyhdr}
\usepackage{latexsym}
\usepackage{lmodern}
\usepackage{cancel}
\usepackage{hyperref}
\synctex=1    
\numberwithin{equation}{section}
\newcommand{\bigno}{\bigskip\noindent}
\newcommand{\ds}{\displaystyle}
\newcommand{\medno}{\medskip\noindent}
\newcommand{\smallno}{\smallskip\noindent}
\newcommand{\nin}{\noindent}
\newcommand{\ts}{\textstyle}
\newcommand{\rr}{\mathbb{R}}
\newcommand{\p}{\partial}
\newcommand{\zz}{\mathbb{Z}}
\newcommand{\cc}{\mathbb{C}}
\newcommand{\ci}{\mathbb{T}}
\newcommand{\tor}{\mathbb{T}}
\newcommand{\ee}{\varepsilon}
\newcommand{\wh}{\widehat}
\newcommand{\weak}{\rightharpoonup}
\newcommand{\vp}{\varphi}
\newtheorem{proposition}{Proposition}
\newtheorem{claim}{Claim}
\newtheorem{remark}{Remark}
\newtheorem{conjecture}[subsection]{conjecture}
%% Equation Numbers %%
%%%%%%%%%%%%%%%%%%%%%%
\title[Ill-posedness for g-Boussinesq]{Ill-posedness results for generalized Boussinesq equations}
\author[Geba, Himonas, Karapetyan]{Dan Geba (University of Rochester), \\ Alex Himonas (University of Notre Dame), \\ David Karapetyan (University of Rochester)}
\institute[UR, ND]{}
\date{}
\begin{document}
\begin{frame}
\titlepage
\end{frame}
\section{Generalized Boussinesq: Introduction}
\begin{frame}
  \frametitle{Generalized Boussinesq equation}
\begin{equation*}
\left\{
\begin{array}{l}
u_{tt}-u_{xx}+u_{xxxx}+(f(u))_{xx}\,=\,0, \qquad u=u(t,x): \mathbb{R}_+\times M \to \mathbb{R},\\
\\
u(0,x)\,=\,u_0(x),\qquad u_t(0,x)\,=\,u_1(x),\\
\end{array}\right.
\label{main}
\end{equation*}
\pause
\begin{itemize}
  \item $f(u)\,\simeq\,\pm\,|u|^{p-1}\,u$ is a pure power nonlinearity with $p>1$,
and $M=\mathbb{R}$ (the non-periodic case) or $M=\mathbb{T}$ (the periodic case).
\end{itemize}
\pause
\end{frame}
\begin{frame}
  \begin{itemize}
    \item  Similar type of  equation, with $f(u)=4u^3-6u^5$, derived by Falk, Laedke, and Spatschek in study of shape-memory alloys. 
      \pause
    \item For $f(u) =  u^{2}$, one recovers the classical ``good'' Boussinesq equation, which is a water wave model. 
\pause
    \item Non-periodic case, Bona and Sachs proved local well-posedness (LWP) for $f\in C^\infty(\mathbb{R})$, $f(0)=0$, and $(u_0,u_1)\in H^s \times H^{s-2}$, $s>5/2$. 
  \end{itemize}
    \end{frame}
    \begin{frame}
      \begin{itemize}
    \item The LWP was then improved for pure power nonlinearities by Tsutsumi and Matahashi, who demonstrated that this holds for $u_0\in H^1$ and $u_1=\phi_{xx}$, with $\phi\in H^1$. 
      \pause
    \item Followed by Linares, who proved LWP for $(u_0,u_1)=(g, h_x)$ when either $(g,h) \in H^1\times L^2$ and $p>1$ or $(g,h) \in L^2 \times \dot{H}^{-1}$  and $1<p\leq 5$. Furthermore, one has global well-posedness (GWP) in the former setting if $\|g\|_{H^1}+\|h\|_{L^2}$ is sufficiently small. 
      \pause
    \item Farah showed LWP for $u_0\in H^s$ and $u_1=\phi_{xx}$, with $\phi\in H^s$, when
\[
p>1 \quad \text{and} \quad s\geq \max \left\{0,\,\frac{1}{2}-\frac{2}{p-1}\right\}. \]
\end{itemize}
\end{frame}
\begin{frame}
  \begin{itemize}
\item For the defocusing problem $f(u)= -|u|^{p-1}\,u$, where $p\geq 3$ is an odd integer, Farah and Linares and Farah and Wang, respectively, proved GWP for 
\[
u(0)\in H^s, \quad u_t(0)=h_x , \quad h \in H^{s-1}, \quad s> 1-\frac{2}{3(p-1)}.
\]
\pause 
\item For the periodic problem there are considerably fewer results. In fact, to our knowledge, the only LWP result is due to Grillakis and Fang, who proved this for $(u_0,u_1)\in H^s \times H^{s-2}$,  with 
\[
s\geq \max \left\{0,\,\frac{1}{2}-\frac{1}{p-1}\right\}. \]
\pause
\item They also showed that, for
\[
f(u)\,=\,\lambda |u|^{q-1}u - |u|^{r-1}u, \quad 1<q<r, \quad \lambda\in \mathbb{R},\] 
GWP holds if $(u_0,u_1)\in H^1\times H^{-1}$. 
\end{itemize}
\end{frame}
\begin{frame}
  \begin{itemize}
\item Observe that there is no ill-posedness (IP) result known for the generalized Boussinesq equation. 
\end{itemize}
\end{frame}
\begin{frame}
  \frametitle{Main Results--Using Bejanaru-Tao Framework}
  \begin{itemize}
    \item If $p$ is an integer and $f(u)=\pm\, u^p$, then the generalized Boussinesq equation can be rewritten in the form 
\begin{equation*}
u\,=\,L(u_0,u_1)\,+\,N(u,u, \ldots, u),
\label{LN}
\end{equation*}
where $(u_0, u_1)$ is an initial data lying in a data space $D$, $u$ takes values in a solution space $S$, $L: D \to S$ is a linear operator, and $N:S^{p} \to S$ is a $p$-linear operator, both of which are densely defined.  

\end{itemize}
\end{frame}
\begin{frame}
  \begin{itemize}
\item If  $(D,\|\cdot \|_D)$ and $(S,\|\cdot \|_S)$ are Banach spaces satisfying 
\begin{gather*}
\|L(u_0,u_1)\|_S \lesssim \|(u_0,u_1)\|_D \\ \|N(u_{1}, u_{2}, \cdots, u_{p})\|_S \lesssim \| u_{1} \|_{S}\cdots\| u_{p} \|_{S}, 
\label{estim}
\end{gather*}
then for $ \|(u_0,u_1)\|_D\ll 1$, we have the absolutely convergent sum
\begin{equation*}
u\,=\,\sum_{n=0}^{\infty} A_{n}(u_0,u_1), 
\label{series}
\end{equation*}
\pause
where the nonlinear maps $A_{p}: D\to S$ are defined recursively by
\begin{equation*}
  \begin{split}
& A_{0}(u_0,u_1)=0, \qquad 
  A_1(u_0,u_1)\,=\,L(u_0,u_1), \qquad 
  \\
  & A_{n}(u_0,u_1)\,=\,\sum_{\stackrel{n_{1}, \ldots, n_{p} \ge 0}{n_{1} + \ldots n_{p} = n}} N(A_{n_1}(u_0,u_1),\cdots, A_{n_{p}}(u_0,u_1)), \qquad (\forall)n\geq 2.
\label{An}
\end{split}
\end{equation*}
\end{itemize}
\end{frame}
\begin{frame}
  \begin{itemize}
    \item It is also proved that if the solution series is continuous, as a function of the initial data, in coarser topologies than the ones given by $\| \cdot\|_D$ and $\| \cdot \|_S$, then so is each of the nonlinear maps $A_{n}$. 
      \pause
    \item
      Using this fact, we can prove an IP result if 
i) a \textbf{quantitative well-posedness} pair $(D,S)$ for which one has the continuous solution map 
\begin{equation*}
(u_0,u_1) \in (B_1,\|\cdot \|_D)\,\longrightarrow\,u\in (B_2,\|\cdot \|_S),
\end{equation*} 
where $B_1\subset D$ and $B_2 \subset S$ are two balls centered at the origin;

ii) two weaker norms, $\| \cdot\|_{D'}$ and $\| \cdot\|_{S'}$ and a sequence $\left(u^N_0,u^N_1\right)_N\subset D$ of initial data satisfying
\pause
\begin{align*}
\limsup_{N\to \infty} \|\big(u^N_0,&u^N_1\big)\|_D\,\ll\,1, \quad \lim_{N\to \infty} \|\left(u^N_0,u^N_1\right)\|_{D'}\,=\,0,
\\
&\sup_N \|A_{p}\left(u^N_0,u^N_1\right)\|_{S'}\,\sim\,1.
\label{cont}
\end{align*}
\end{itemize}
\end{frame}
\begin{frame}
  \begin{itemize}
    \item Indeed, these demonstrate that 
\begin{equation*}
A_{p}: (B_1,\|\cdot \|_{D'})\,\longrightarrow\,(B_2,\| \cdot\|_{S'})
\end{equation*} 
is not continuous at the origin, which then implies that the same is true for the solution map in these coarser  topologies.
\pause
\item A crucial fact used in this approach is that the first of the two conditions guarantees that the solutions corresponding to the sequence of initial data $\left(u^N_0,u^N_1\right)_N$, for $N$ sufficiently large, will all exist for a fixed amount of time, which is independent of $N$. 
\end{itemize}
\end{frame}
\begin{frame}
\begin{remark}
Initially, our condition for failure of continuity was
\begin{equation*}
  \sup_N \frac{\|A_p\left(u^N_0,u^N_1\right)\|_{S'}}{\|\left(u^N_0,u^N_1\right)\|_{D'}^{p}}\,=\,\infty.
\end{equation*}
However, this in fact is sufficient only to establish the failure of a $p$-linear estimate on $A_{p}$. 
\end{remark}
\pause
We now have all the ingredients to state our IP results:
\end{frame}
\begin{frame}
\begin{theorem}
Let the g-Boussinesq Cauchy problem, with $p>1$ an integer and $f(u)=\pm\, u^p$, be quantitatively well-posed for
\[
D\,=\,H^s \times H^{s-2}(M) \quad \text{and} \quad S\,\subseteq \,C([0,T]; H^s(M)), \]
where  $M=\mathbb{R}$ or $\mathbb{T}$ and $T=T(B_1)$ is the time of existence for a solution whose initial data is in $B_1$. Let $s'< \min\{s_p, s\}$,
where
\begin{equation*}
s_p\,=\,\left\{
\begin{array}{l}
-2/p, \quad \text{for} \quad p \ \text{odd},\\
\\
-1/p, \quad \text{for} \quad p \ \text{even}.\\
\end{array}\right.
\label{sp}
\end{equation*}
Then, if a solution $u$ with associated initial data $(u_{0}, u_{1}) \in H^{s'} \times H^{s'}$ exists for g-Boussinesq, it cannot be obtained via a Picard iteration on any continuously embedded space $X \subset C([0, T]; H^{s'})$. For the Boussinesq, we obtain the stronger result of failure of continuity of the data to solution map at the origin for $s' < -1/2$.
\end{theorem}
\end{frame}
\begin{frame}
\begin{remark}
For the ``good'' Boussinesq equation (i.e., $p=2$ and $f(u)=u^2$), our theorem recovers the sharp IP result of Kishimoto, $s<-1/2$. We believe our proof has the advantage that it is more natural, as it doesn't go through  the associated Schrodinger equation and, hence, it doesn't require an extra scaling argument.  
\end{remark}
\end{frame}
\begin{frame}
  \frametitle{The Proof}
\begin{equation*}
A_p(u_0,u_1)\,=\,N(L(u_0,u_1),L(u_0,u_1), \ldots, L(u_0,u_1)),
\label{Ap}
\end{equation*}
\pause
where
\begin{equation*}
\widehat{L(u_0,u_1)}(t,\xi)\,=\,\cos(t \lambda(\xi))\, \widehat{u}_0(\xi)+\frac{\sin(t \lambda(\xi))}{\lambda(\xi)} \,\widehat{u}_1(\xi)
\label{L}
\end{equation*}
\pause
and
\begin{equation*}
\widehat{N(u_{1},u_{2}, \ldots, u_{p})}(t,\xi)\,=\,\int_0^t\,\frac{\sin((t-s) \lambda(\xi))}{\lambda(\xi)}\, \xi^2\, \widehat{v_{1}  v_{2}  \ldots v_{p}}(s,\xi)\,ds,
\label{N}
\end{equation*}
with $\lambda(\xi)=\sqrt{\xi^2+\xi^4}$. 
\end{frame}
\begin{frame}
Next, our goal is to construct a sequence $\left(u^N_0,u^N_1\right)_N\subset H^s\times H^{s-2}$ of initial data which violates boundedness of the p-linear operator $A_p$ at the origin, when $s'< \min\{s_p, s\}$. It needs to satisfy
\pause
\begin{equation*}
\limsup_{N\to \infty} \|u_0^N\|_{H^s} + \|u_0^N\|_ {H^{s-2}}\,\ll\,1, \qquad
\lim_{N\to \infty} \|u_0^N\|_{H^{s'}} + \|u_0^N\|_ {H^{{s'}-2}}\,=\,0,
\label{ivhs}
\end{equation*}
\pause
and 
\begin{equation*}
\sup_N \frac{\|A_{p}(u_0^N,u_1^N)\|_{C([0,T],H^{s'})}}{(\|u_0^N\|_{H^{s'}} + \|u_0^N\|_ {H^{{s'}-2}})^{p}}\,=\,\infty.
\label{ratio}
\end{equation*}
\end{frame}
\begin{frame}
The general profile for our initial data, in both cases, is
\begin{align*}
\widehat{u^N_0}(\xi)\,&=\,\frac{r}{N^{s}}\,\left(\varphi_{A_N}(\xi)\,+\,\varphi_{-A_N}(\xi)\right), \\
\widehat{u^N_1}(\xi)\,&=\,-i \frac{r}{N^{s}}\,\lambda(\xi)\,\left(\varphi_{A_N}(\xi)\,-\,\varphi_{-A_N}(\xi)\right),\label{u1n}
\end{align*}
where $r>0$ is a positive parameter, $(A_N)_N$ is a sequence of subsets of $\mathbb{R}$ or $\mathbb{T}$ to be specified later, and $\varphi_A$ is the characteristic function of the set $A$. 
\end{frame}
\begin{frame}
We obtain first that
\begin{gather*}
\widehat{L(u^N_0,u^N_1)}(t,\xi)\,=\, \frac{r}{N^{s}}\,\left(e^{-it\lambda(\xi)}\varphi_{A_N}(\xi)\,+\,e^{it\lambda(\xi)}\varphi_{-A_N}(\xi)\right)\,
\\
=\,\frac{r}{N^{s}}\,e^{\mp it\lambda(\xi)}\varphi_{\pm A_N}(\xi),
\end{gather*}
where we assumed, in the last expression, a silent summation convention for $\pm$. Moreover, the sign convention is the one suggested by the above notation, i.e., if $\xi\in \pm \,A_N$, then the corresponding exponent is $\mp \,is\lambda(\xi)$.
\end{frame}
\begin{frame}

Subsequently, we are led to
\begin{align*}
\widehat{A_p(u^N_0,u^N_1)}(t,\xi)\,
& =\, \frac{r^p \xi^2}{N^{ps}\lambda(\xi)}\int_0^t \,\sin((t-s) \lambda(\xi))\,\cdot
\bigg[\int_{\mathbb{R}^{p-1}}\varphi_{\pm A_N}(\xi-\sum_{j=1}^{p-1} \eta_j)&\cdot e^{\mp is\lambda(\xi-\sum_{j=1}^{p-1}\eta_j)}\,
\\
& \cdot \prod_{j=1}^{p-1}\varphi_{\pm A_N}(\eta_j)\cdot e^{\mp is\lambda(\eta_j)}\,d\eta_1\ldots d\eta_{p-1}\bigg]ds.
\end{align*}
\end{frame}
\begin{frame}
We continue by recording an integral calculus result
\begin{align*}
\int_0^t\,\sin(\alpha(t-s))\,e^{i\beta s}\,ds\,&=\,\frac{-\alpha}{\beta^2-\alpha^2}(\cos (\beta t)-\cos (\alpha t))\\
&+\,i\left[\frac{-\alpha}{\beta^2-\alpha^2}\sin (\beta t) + \frac{\beta}{\beta^2-\alpha^2}\sin (\alpha t)\right],
\\
& \alpha\neq \beta.
\label{calc}
\end{align*}
\end{frame}
\begin{frame}
\frametitle{Outline of main ideas in the argument} 
We will localize first the initial data $(u^N_0,u^N_1)$ at frequency $N$, i.e.,
\[
\eta\in \pm \,A_N\quad \Longrightarrow \quad |\eta| \approx N,\]
\pause
and we will measure the output of $A_p$ only at frequency $\xi \approx 1$. This allows us to argue that
\begin{equation*}
\aligned
\|A_{p}(u_0^N,u_1^N)\|_{C([0,T], H^{s'})}\,\geq\, \|A_{p}(u_0^N,u_1^N)\|_{C([0,T], H^{s'}(\xi \approx 1))}\\
\gtrsim\, \|A_{p}(u_0^N,u_1^N)\|_{C([0,T], L^2(\xi \approx 1))}.
\endaligned
\label{aphs}
\end{equation*}
\end{frame}
\begin{frame}
Secondly, 
\begin{equation*}
\alpha=\lambda(\xi) \quad \text{and} \quad \beta= -\,\epsilon_1\lambda(a_1)-\epsilon_2\lambda(a_2)-\cdots-\epsilon_p\lambda(a_p),
\label{ab}
\end{equation*}
where
\begin{equation*}
\xi=  \epsilon_1 a_1+\epsilon_2 a_2+ \cdots+\epsilon_p a_p, \qquad \epsilon_j=\pm \,1, \ a_j\in A_N, \ (\forall)1\leq j\leq p.\label{xi}
\end{equation*}
\pause
The key point in the proof is the construction of the sequence of subsets $(A_N)_N$ such that
\begin{equation*}
 |\beta|\,=\, \left\{
\begin{array}{l}
\ \beta \approx N^{2}, \quad \text{for} \quad p \ \text{odd},\\
\\
-\beta \approx N, \quad \text{for} \quad p \ \text{even}.\\
\end{array}\right.
\end{equation*}
\end{frame}
\begin{frame}
\frametitle{Argument for $p$ even} 
We split the discussion into two cases, depending on the parity of $p$. First, for $p$ even, the choice for the sequence $(A_N)_N$ is 
\[
A_N\,=\,[N,N+1], \quad (\forall) N \geq 1.
\]
\pause
Then, for $\xi \in \left[\frac{1}{4},\frac{1}{2}\right]$ and $N$ sufficiently large, precisely half of the terms in the representation for $\xi$ above have the coefficient $\epsilon =1$. Otherwise
\[
\left|\epsilon_1 a_1+\epsilon_2 a_2+ \cdots+\epsilon_p a_p\right|\,\gtrsim\,N.\]
This leads to
\pause
\begin{equation*}
\|A_{p}(u_0^N,u_1^N)\|_{C([0,T], H^{s'})}\,\gtrsim\,  \frac{1}{N^{ps+1}}\,\sup_{\stackrel{t \in [0,T]}{\xi \in \left[\frac{1}{4},\frac{1}{2}\right]}} \,|\sin(\lambda(\xi)t)|.
\label{aphs2}
\end{equation*} 
\end{frame}
\begin{frame}
Using $A_N\,=\,[N,N+1]$, also get
\begin{equation*}
 \|u_0^N\|_{H^{s'}} + \|u_0^N\|_ {H^{{s'}-2}}\,\approx\, r\,N^{s'-s},\qquad (\forall) s'\in \mathbb{R},
\label{idhs} 
\end{equation*}
\pause
Finally,  
\begin{equation*}
\frac{\|A_{p}(u_0^N,u_1^N)\|_{C([0,T],H^{s'})}}{(\|u_0^N\|_{H^{s'}} + \|u_0^N\|_ {H^{{s'}-2}})^{p}}\,\gtrsim\, \frac{1}{N^{ps' +1}}\end{equation*}
thus finishing the argument for the case $p$ even.
\end{frame}
\begin{frame}
  \frametitle{$p$ odd}
The first remark we want to make here is that the previous choice for $A_N$ (i.e., $A_N\,=\,[N,N+1]$) doesn't work in this case, because, as $p$ is odd, 
\[
\left|\epsilon_1 a_1+\epsilon_2 a_2+ \cdots+\epsilon_p a_p\right|\,\approx\,N, 
\]
for all possible representations with $\epsilon_j=\pm \,1$, $a_j\in A_N$,  and $1\leq j\leq p$. 

\pause
Instead, we pick
\[
A_N\,=\,\tilde{A}_N\,\cup\,\tilde{A}_{2N}\,=\,\left[N+\frac{3(p-1)}{2p^2},N+\frac{3(p+2)}{2p^2}\right]\,\cup\, \left[2N, 2N+\frac{3}{p^2}\right]
\]
\pause
the idea being that we can create something comparable to $1$ with $p/3$ triplets $(a,b,c)$, where $a$, $b \in \tilde{A}_N$ and $c \in \tilde{A}_{2N}$.
\end{frame}
\begin{frame}
  \begin{center}
    \Large{Thank you}
\end{center}
\end{frame}
\end{document}
