\documentclass[12pt]{amsart}
\usepackage[margin=1cm]{geometry}  %page layout
\usepackage{outline}
\synctex=1
\pagestyle{empty}

\begin{document}
\title{Triple Your Latexing Speed}
\maketitle
\begin{outline}
\item{\bf Introduction---The Custom Latex-Suite \\
  (http://davidkarapetyan.com/computing.php)} 
  \begin{outline}
  \item{Skim}
    \begin{outline}
    \item{More advanced than Preview}
      \begin{outline}
      \item{Supports source-pdf sync (like in TexShop)}
      \item{Note taking feature}
      \item{Snapshots---multiple copies of pdf}
      \end{outline}

    \end{outline}
  \item{Excalibur}
    \begin{outline}
    \item{Advanced spell-checker. Will ignore math environments,
      latex commands, etc. when spell-checking}
    \end{outline}
  \item{Bibdesk}
    \begin{outline}
    \item{Reference manager}
    \item{Can use to easily include reference in a latex
      document, as well as to manage pdfs of articles,
      books, etc.}
    \end{outline}
  \item{MacVim (most important)} 
    \begin{outline}
    \item{Advanced text editor (for writing code)}
    \item{Very customizable, i.e. TexShop on steroids}
      \begin{outline}
      \item{Can be extended by plugins}
      \item{Latex-suite is a plugin, with my
        personal customizations}
      \end{outline}

    \end{outline}


  \end{outline}

\item{\bf MacVim}
  \begin{outline}
  \item{ Documentation at http://www.vim.org/}
  \item{Based on vim, a popular Unix/Linux text editor}
  \item{ Great tutorial--open Terminal and type 'vimtutor' }
  \item{Supports MODAL EDITING }
    \begin{outline}
    \item{Three modes--command, insert, and visual. Gives insane
      power and speed }
      \begin{outline}
      \item{Insert }
      \item{ Command. Discuss cursor movements hjkl}
      \item{ Visual}

      \end{outline}
    \item{ Contrast with TexShop--no modes. Must use mouse to move
      around}

    \end{outline}
  \item{ Configuration}
    \begin{outline}
    \item{Unix/Linux like directory structure. Demonstrate }
    \item{ \$HOME/.vimrc---global configuration file}
    \item{ Plugins located in \$HOME/.vim directory}

    \end{outline}


  \end{outline}
\item{\bf Latex-Suite}
  \begin{outline}
  \item{ Documentation at http://vim-latex.sourceforge.net/}
  \item{ Combination of plugins and personal code}
  \item{ Commands}
    \begin{outline}
    \item{Compile }
    \item{ View--forward and reverse search}
    \end{outline}
  \item{ Menus}
  \item{ Reference insertion with \textbackslash eqref\{F5 and \textbackslash ref\{F5}
  \item{ Macros (super useful}
    \begin{outline}
    \item{Via snipmate--a plugin }
    \item{ Equation environment, paper templates, beamer template,
      etc. Demonstrate use of placeholders}
    \item{ View all others, and add your own, by opening \\
      \$HOME/.vim/bundle/snipmate.vim/snippets/tex.snippets}
    \end{outline}



  \end{outline}
\item{ \bf Excalibur}
  \begin{outline}
  \item{To spell check, just press F1 from within MacVim. That's
    it!} 
  \end{outline}
\item{\bf Bibdesk }
  \begin{outline}
  \item{Create file, and load in some references. Save}
  \item{ Bottom of source code--point to location of
    references file}
  \item{ Type \textbackslash cite\{F9 --- can now load in any references you
    want}
  \end{outline}
\item{ \bf More Features}
  \begin{outline}
  \item{Consult README, which you will find in the \\
    http://davidkarapetyan.com/computing.php download }
  \end{outline}
\end{outline}





\end{document}
