%
\documentclass[12pt,reqno]{amsart}
\usepackage{amssymb}
\usepackage{appendix}
\usepackage{cancel}  %for cancelling terms explicity on pdf
%\usepackage{showkeys}
\usepackage{framed}  %for framing remarks, theorems, etc.
\usepackage{enumerate} %to change enumerate symbols
\usepackage[margin=2.5cm]{geometry}  %page layout
\setcounter{tocdepth}{1} %must come before secnumdepth--else, pain
\setcounter{secnumdepth}{1} %number only sections, not subsections
%\usepackage[pdftex]{graphicx} %for importing pictures into latex--pdf compilation
\numberwithin{equation}{section}  %eliminate need for keeping track of counters
%\numberwithin{figure}{section}
\setlength{\parindent}{0in} %no indentation of paragraphs after section title
\renewcommand{\baselinestretch}{1.1} %increases vert spacing of text
%
\usepackage{hyperref}
\hypersetup{colorlinks=true,
linkcolor=blue,
citecolor=blue,
urlcolor=blue,
}
%\usepackage[alphabetic, initials, msc-links]{amsrefs} %for the bibliography; uses cite pkg. Must be loaded after hyperref, otherwise doesn't work properly (conflicts with cref in particular)
%
%
\newcommand{\ds}{\displaystyle}
\newcommand{\ts}{\textstyle}
\newcommand{\nin}{\noindent}
\newcommand{\rr}{\mathbb{R}}
\newcommand{\nn}{\mathbb{N}}
\newcommand{\zz}{\mathbb{Z}}
\newcommand{\cc}{\mathbb{C}}
\newcommand{\ci}{\mathbb{T}}
\newcommand{\zzdot}{\dot{\zz}}
\newcommand{\wh}{\widehat}
\newcommand{\p}{\partial}
\newcommand{\ee}{\varepsilon}
\newcommand{\vp}{\varphi}
\newcommand{\wt}{\widetilde}
%
%
%
%
\newtheorem{theorem}{Theorem}[section]
\newtheorem{lemma}[theorem]{Lemma}
\newtheorem{corollary}[theorem]{Corollary}
\newtheorem{claim}[theorem]{Claim}
\newtheorem{prop}[theorem]{Proposition}
\newtheorem{proposition}[theorem]{Proposition}
\newtheorem{no}[theorem]{Notation}
\newtheorem{definition}[theorem]{Definition}
\newtheorem{remark}[theorem]{Remark}
\newtheorem{examp}{Example}[section]
\newtheorem {exercise}[theorem] {Exercise}
%
\makeatletter \renewenvironment{proof}[1][\proofname] {\par\pushQED{\qed}\normalfont\topsep6\p@\@plus6\p@\relax\trivlist\item[\hskip\labelsep\bfseries#1\@addpunct{.}]\ignorespaces}{\popQED\endtrivlist\@endpefalse} \makeatother%
%makes proof environment bold instead of italic
\newcommand{\uol}{u^\omega_\lambda}
\newcommand{\lbar}{\bar{l}}
\renewcommand{\l}{\lambda}
\newcommand{\R}{\mathbb R}
\newcommand{\RR}{\mathcal R}
\newcommand{\al}{\alpha}
\newcommand{\ve}{q}
\newcommand{\tg}{{tan}}
\newcommand{\m}{q}
\newcommand{\N}{N}
\newcommand{\ta}{{\tilde{a}}}
\newcommand{\tb}{{\tilde{b}}}
\newcommand{\tc}{{\tilde{c}}}
\newcommand{\tS}{{\tilde S}}
\newcommand{\tP}{{\tilde P}}
\newcommand{\tu}{{\tilde{u}}}
\newcommand{\tw}{{\tilde{w}}}
\newcommand{\tA}{{\tilde{A}}}
\newcommand{\tX}{{\tilde{X}}}
\newcommand{\tphi}{{\tilde{\phi}}}
\synctex=1
\begin{document}
\title{Notes on Molinet Paper}
\author{Dan Geba, Alex Himonas, and David Karapetyan}
\address{Department of Mathematics  \\
    University  of Notre Dame\\
        Notre Dame, IN 46556 }
        \date{\today}
        %
        \maketitle
        %
        %
        %
        %
        %
        %
        \section{Notions of Well-Posedness}
        We consider the modified 
        Kortewew-de Vries (mKDV) ivp
        \begin{gather}
          \label{mkdv}
          u_{t} + u_{xxx} + u^{2} u_{x} = 0
          \\
          u(x,0) = u_{0}(x), \ x \in \ci, \ t \in \rr.
          \label{mkdv-init}
        \end{gather}
        viewing $w(x,t) \doteq u^{2} u_{x}$ as a forcing term.
        Taking the spatial Fourier transform of its linearization
        %
        %
        \begin{gather*}
          \label{lin-mkdv}
          u_{t} + u_{xxx} =0
          \\
          u(x,0) = u_{0}(x)
          \label{lin-mkdv-init}
        \end{gather*}
        %
        %
 %
 %
 we obtain
 \begin{equation*}
 \begin{split}
   \wh{u_{t}}(n, t) =  i n^{3}\wh{u}(n,t) 
   \\
   \wh{u}(0) = \wh{u_{0}}(n).
 \end{split}
 \end{equation*}
 %
 %
 This admits the unique solution
 %
 %
 \begin{equation*}
 \begin{split}
   \wh{u}(n,t) = \wh{u_{0}}(n) e^{i n^{3}t}
 \end{split}
 \end{equation*}
 %
 %
 which by Fourier inversion gives
 %
 %
 \begin{equation*}
 \begin{split}
   u(x,t) = \sum_{n \in \zz} \wh{u_{0}}(n) e^{i\left( nx + n^{3}t \right)} \doteq S(t)u_{0}(x).  \end{split}
 \end{equation*}
 %
 %
Hence, by Duhamel's principle, 
%
%
\begin{equation}
  \label{int-reform-mkdv}
\begin{split}
  u(t) = S(t)u_{0} + \int_{0}^{t} S(t- \tau) w(\tau) d \tau, \quad w =
  -u^{2}u_{x}
\end{split}
\end{equation}
%
%
is the integral formulation of of the mKDV ivp. 
        \subsection{Kenig-Ponce-Vega Well-Posedness} 
        \label{ssec:kpv-wp}
	We say that the mKDV ivp is
	\emph{locally well posed} in a Banach function space
	$X$ if 
	\begin{enumerate}
		\item For every $u_{0}(x) \in
	B_R$ there exists $T>0$ depending on $R$ and a function
	\\
	$u \in C([0, T],
  X)$ satisfying \eqref{int-reform-mkdv} for all $t \in [0, T]$.
\item The solution $u(x,t)$ is unique in a subspace $Y \hookrightarrow
  C([0, T], X)$.  
\item The data to solution map $u_0 \mapsto u$ is locally uniformly continuous. That is,
  if$\{u_{0,n}\} \subset B_R$, and 
  $\|u_0 - u_{0, n} \|_{X} \to 0$, then $\sup_{t \in [0,T]} \|u(\cdot, t) - u_{n}(\cdot,t) \|_{X} \to
	0$.
	\end{enumerate}
  \subsection{Classical Hadamard Well-Posedness} 
  \label{ssec:hadamard-wp}
   We say that the mKDV ivp
   is
	\emph{locally well posed} in a Banach function space $X$ if
	\begin{enumerate}
    \item For every $u_{0}(x)$ there exists $T>0$ depending on $u_{0}(x)$
      and a function $u \in C([0, T],
  X)$ satisfying \eqref{int-reform-mkdv} for all $t \in [0, T]$.
\item The solution $u(x,t)$ is unique in $C([0,T], X)$.  
    \item
      The data to solution map $u_0 \mapsto u$ is continuous. That
      is, given a sequence $\{u_{0,n} \} \in X$
      such that $$\|u_{0} - u_{0,n} \|_{X} \to 0$$
      with corresponding solutions $u_{n} \in
      C([0,
      \delta_{n}])$ and $u \in C([0, \delta_{\infty}])$
      then there exists $0 < \delta \le \inf\left\{
      \delta_{n}, \delta_{\infty} \right\}$ such that 
      $$\sup_{t \in [0, \delta]} \|u(\cdot, t) - u_{n}(\cdot, t) \|_{X} \to 0 
      .$$
  \end{enumerate}
  \subsection{Kappeler-Topalov \cite{Kappeler:2006uq} Well-Posedness} 
  \label{ssec:kap-wp}
  We first cite the following result of Bourgain \cite{Bourgain-Fourier-transfo} used
  by Kappeler and Topalov.
  %
  %
  %%%%%%%%%%%%%%%%%%%%%%%%%%%%%%%%%%%%%%%%%%%%%%%%%%%%%
  %
  %
  %                bona thm
  %
  %
  %%%%%%%%%%%%%%%%%%%%%%%%%%%%%%%%%%%%%%%%%%%%%%%%%%%%%
  %
  %
  \begin{theorem}[Bourgain]
    For initial data $u_{0}(x) \in H^{s}(\ci)$, $1\le s \le \infty$, 
    there exists a unique global solution $u \in C(\rr, H^{s}(\ci))$ to
    \eqref{int-reform-mkdv}. Furthermore, the data to solution map $u_{0}
    \mapsto u$ is Lipschitz.
    \label{thm:bour}
  \end{theorem}
  %
  %
  %
  %
    %
  %
  %
  %
  %
  %
  %
  %
  We say that a continuous curve $\gamma: (a,b) \to H^{\beta}(\ci)$ with $0 \in
  (a,b)$ and $\gamma(0) = u_{0}$ is a solution of the mKDV in
  $H^{\beta}(\ci)$ with initial data $u_{0}$ iff for any $C^{\infty}$ sequence
  of initial data $\left\{ u_{0,n} \right\}$ converging to $u_{0}$ in
  $H^{\beta}(\ci)$ and for any $t \in [a,b]$, the sequence of emanating
  solutions $u_{n}$ of the mKDV equation satisfies $u_{n}(t) \to
  \gamma(t)$ in $H^{\beta}(\ci)$.
  %
  %
  %
  \begin{framed}
  \begin{remark}
    Uniqueness of the solution in $H^{\beta}$ follows from the definition. We
    note that continuous dependence on initial data follows from the Lipschitz
    continuity of the data to solution map for $H^{s}$ initial data, $s \ge 1$.
    To see this, assume $v_{0,n} \subset C^{\infty}$,
  $v_{0,n} \to v_{0}$ in $H^{\beta}$, and $v_{n}(t) \to v(t)$ in $H^{\beta}$ for
  $t \in [a,b]$. Then
  %
  %
  \begin{equation}
    \label{yuu}
  \begin{split}
    \|u(t) - v(t) \|_{\beta} 
    & = \|u(t) \pm u_{n}(t) \pm v_{n}(t) - v(t) \|_{\beta}
    \\
    & \le \| u(t) - u_{n}(t) \|_{\beta} + \| u_{n}(t) - v_{n}(t) \|_{\beta} + \|
    v(t) -
    v_{n}(t) \|_{\beta}
    \\
    & \le \| u(t) - u_{n}(t) \|_{\beta} + C_{\beta} \| u_{0,n} - v_{0,n}
    \|_{\beta} + \| v(t) -
    v_{n}(t) \|_{\beta}.
  \end{split}
  \end{equation}
  %
  For fixed $t \in [a,b]$ and $\ee > 0$, there exists $N$ such that
  \eqref{yuu} implies
  %
  %
  %
  \begin{equation*}
  \begin{split}
  \|u(t) - v(t) \|_{\beta} 
       \le 2 \ee +  C_{\beta} \| u_{0,n} - v_{0,n} \|_{\beta}, \quad n > N.
     \end{split}
  \end{equation*}
  %
  %
  Hence, solutions to the mKDV (in the sense of Kappeler-Topalov) depend
  continuously on the initial data for $\beta \ge 1$.
  \label{rem:uniqueness}
  \end{remark}
  %
  %
  \end{framed}
  %
  %
\begin{framed}
  %
  %
  \begin{remark}
    Tao and company \cite{Colliander:2003kx} were able to lower the global well-
    posedness result for mKDV to $s \ge 1/2$, where their definition of
    well-posedness is that of Kenig-Ponce-Vega. Hence, in
    fact, we obtain continuous dependence on initial data for $s \ge 1/2$.
    However, Kappeler and Topalov \cite{Kappeler:2005kx}
    were able to show existence (and hence,
    uniqueness) of solutions for
    $s \ge 0$, using the definition of existence we have defined in this
    section. This begs the question: is there continous dependence on initial
    data for $0 \le s
    < 1/2$? To my knowledge, this is an open problem.
    \label{rem:tao}
  \end{remark}
  %
  %
  \end{framed}

  

  
  \section{Outline of the Molinet Paper \cite{Molinet:2011uq}}
        \label{sec:outline}
        \begin{theorem}
          \label{thm:cont-fail}
        There exists $T > 0$ such that the data to solution map $u_{0} \mapsto u$ for
        the mKDV is not continuous at any $u_{0} \in H^{\infty}$ from $L^{2}(\ci)$
        equipped with the weak topology to $D'\left([0, T] \times \ci
        \right)$. More precisely, if 
        \begin{equation}
          \langle u_{0,n}, \vp \rangle  \to \langle u_{0}, \vp \rangle  \ 
          \text{for all} \ \vp \in L^{2}(\ci)
        \end{equation}
        then there exists $\psi \in C^{\infty}_{0}(\ci \times \left[ 0, T
        \right])$ such that
                %
        %
        \begin{gather*}
           \langle u_{n}, \psi \rangle  \not \to \langle u, \psi \rangle 
          \\
          (\text{i.e.} \ u_{n} \not \to u \ \text{in} \ D'\left( \ci \times [0,T]
          \right)).
        \end{gather*}
        %
        %
        \end{theorem}
    %
    %
    %%%%%%%%%%%%%%%%%%%%%%%%%%%%%%%%%%%%%%%%%%%%%%%%%%%%%
    %
    %
    %                ill-pos
    %
    %
    %%%%%%%%%%%%%%%%%%%%%%%%%%%%%%%%%%%%%%%%%%%%%%%%%%%%%
    %
    In fact, the failure of convergence is severe: as we shall see, it fails  
    for any test function.
    Assuming Theorem \ref{thm:cont-fail} for now, we obtain the following
    corollary.
    %
    \begin{corollary}[Ill-Posedness]
   The mKDV ivp is ill-posed for $s < 0$.  
    \label{prop:ill-pos}
    \end{corollary}
    %
    %
    %
    %
    \begin{proof}
      Let 
      $\left\{ u_{0,n} \right\} \subset H^{\infty}$ be a sequence such that $u_{0,n}
      \rightharpoonup u_{0}$ in $L^{2}(\ci)$. By Banach-Steinhaus, $\left\{
      u_{0,n}
      \right\}$ is uniformly bounded in $L^{2}$.
      %
      %
      Since the
      inclusion $L^{2}(\ci) \subset H^{s}(\ci)$ is compact for $s<0$, there
      exists a subsequence $u_{0,n_{k}}$ such that
      %
      %
      \begin{equation*}
      \begin{split}
        u_{0,n_{k}} \xrightarrow{H^{s}} v_{0}.
      \end{split}
      \end{equation*}
      %
      %
      By the uniqueness of the limit in $D'(\ci)$, we see that $v_{0} = u_{0}$. 
      Hence,
      %
      %
      \begin{equation*}
      \begin{split}
        u_{0,n_{k}} \xrightarrow{H^{s}}u_{0}, \quad s<0.
      \end{split}
      \end{equation*}
      %
      %
      If existence fails for $s<0$ in the sense of
      Kappeler-Topalov, we are done. So assume
      existence holds. Furthermore, assume continuity of the data to solution map for
      holds from $H^{s}$ to $C([0, T], H^{s})$, $s < 0$. Then
      %
      %
      \begin{equation}
        \label{s-conv}
      \begin{split}
      u_{n_{k}} \xrightarrow{C\left( [0,T], H^{s} \right)} u.
      \end{split}
      \end{equation}
      %
      This implies
      %
      %
      \begin{equation*}
      \begin{split}
        u_{n_{k}} \xrightarrow{D'(\ci \times [0, T])} u
      \end{split}
      \end{equation*}
      %
      since for any $\phi(x,t) \in C^{\infty}\left( \ci \times [0,T] \right)$,
      we have
      %
      %
      \begin{equation*}
      \begin{split}
        | \int_{\ci \times [0,T]} (u - u_{n_{k}}) \phi(x,t) d (x \times t) |
        & \le  \int_{\ci \times [0,T]} | (u - u_{n_{k}}) \phi(x,t)| d (x \times t) 
        \\
        & = \int_{\ci} \int_{[0,T]} | u - u_{n_{k}} | | \phi | dt dx
        \\
        & \le \| u - u_{n_{k}} \|_{L^{\infty}\left( [0,T], H^{s} \right)} \|
        \phi \|_{L^{\infty}\left( [0,T], H^{-s} \right)}
        \\
        & \simeq \| u - u_{n_{k}} \|_{L^{\infty}\left( [0,T], H^{s} \right)} \to
        0.
      \end{split}
      \end{equation*}
      %
      %
     This contradicts Theorem \ref{thm:cont-fail}.
      %
      %
    \end{proof}
    %
    %
%
%
%%%%%%%%%%%%%%%%%%%%%%%%%%%%%%%%%%%%%%%%%%%%%%%%%%%%%
%
%
%                Proof of thm
%
%
%%%%%%%%%%%%%%%%%%%%%%%%%%%%%%%%%%%%%%%%%%%%%%%%%%%%%
%
%
\section{Proof of Theorem \ref{thm:cont-fail}}
\label{sec:pf-thm}
Let $u_{0} \in H^{\infty}$ be a non constant. Set
%
%
\begin{equation}
  \label{ill-pos-init-data}
\begin{split}
  u_{0,n} = u_{0} + k \cos(nx), \quad k\in \rr, k \neq 0.
\end{split}
\end{equation}
%
Note that the sequence $\left\{ u_{0,n} \right\}$ is uniformly bounded in $L^{\infty}$.
Furthermore, it has the following convergence properties.
%
%
%
%
%%%%%%%%%%%%%%%%%%%%%%%%%%%%%%%%%%%%%%%%%%%%%%%%%%%%%
%
%
%                
%
%
%%%%%%%%%%%%%%%%%%%%%%%%%%%%%%%%%%%%%%%%%%%%%%%%%%%%%
%
%
\begin{lemma}
We have
%
%
\begin{gather}
  u_{0,n}\overset{L^{2}}{\rightharpoonup} u_{0}
  \\
  \| u_{0,n} \|_{L^{2}}^{2} \to \| u_{0} \|_{L^{2}}^{2} + \pi k^{2}.
\end{gather}
\label{lem:conv-prop-init}
\end{lemma}
%
%
%
%
Now, do the corresponding solutions $u_{n}(t)$ converge in some sense to
$u(t)$? This question is
the central issue of the paper, and is resolved in the following.
%
%
%%%%%%%%%%%%%%%%%%%%%%%%%%%%%%%%%%%%%%%%%%%%%%%%%%%%%
%
%
%                key prop
%
%
%%%%%%%%%%%%%%%%%%%%%%%%%%%%%%%%%%%%%%%%%%%%%%%%%%%%%
%
%
\begin{proposition}
  Let $\left\{ u_{0,n} \right\} \subset
  H^{\infty}$ be such that and $u_{0,n} \overset{L^{2}}{\rightharpoonup} u_{0}$.
  Then there exists a weak solution solution $u \in C(\rr,
  L^{2})$ solving the ivp
  %
  %
  \begin{gather}
    \label{n-mkdv}
    u_{t} + u_{xxx} + \left( u^{2} - \int_{\ci} u^{2} dx \right)u_{x} = 0,
    \\
    u(0) = u_{0}(x)
    \label{n-mkdv-init}
  \end{gather}
  %
  %
  and a subsequence $\left\{ u_{n_{k}} \right\}$ of solutions to the ivp
  \begin{gather}
    u_{t} + u_{xxx} + \left( u^{2} - \int_{\ci} u^{2} dx \right)u_{x} = 0,
    \\
    u(0) = u_{0,n}(x)
  \end{gather}
%
%
  such that for $T > 0$ sufficiently small
  %
  %
  \begin{equation*}
  \begin{split}
    \sup_{t \in [0, T]} \langle u_{n_{k}}(t), \vp \rangle \to \langle
    u(t), \vp
    \rangle, \quad \vp \in L^{2}(\ci).
  \end{split}
  \end{equation*}
  %
  %
\label{prop:key-paper}
\end{proposition}
%
%
%
%
%\begin{framed}
%%
%%
%\begin{remark}
  %In fact, $u_{n_{k}} \in C(\rr, H^{\infty})$, due to Bourgain's global well-posedness
  %result for mKDV \eqref{thm:bour}.
  %It follows that $u_{n_{k}} \in C^{\infty}(\rr, H^{\infty})$.
  %To see this, we note that for any solution to the mKDV equation, its
  %derivative in $t$ exists if it is thrice differentiable in $x$. 
  %Since $u_{n,k}$ is $C^{\infty}$ in $x$, it is $C^{\infty}$ in $t$. Hence,
  %$u_{n_{k}}$ is a classical solution to the mKDV. 
%\label{rem:smooth-soln}
%\end{remark}
%%
%%
%\end{framed}
%
%
%
%
%
%
Next, we choose a sequence of initial data as in \eqref{ill-pos-init-data},
convering to $u_{0}$ weakly in $L^{2}$.  
Then by Proposition \ref{prop:key-paper}, there exists a sequence
$u_{n} \in C(\rr, L^{2})$ of corresponding solutions to the normalized mKDV, and
a subsequence 
\begin{equation*}
  u_{n_{k}}(t) \overset{L^{2}}{\rightharpoonup} u(t) \ \text{for
all} \ t \in \rr.
\end{equation*}
Next, note that if $u_{n_{k}}$ is a solution to the
normalized mKDV, then  
%
\begin{equation}
  \label{mkdv-sol}
\begin{split}
  v_{n_{k}}(x,t) \doteq u_{n,k}\left( x - t \| u_{0,n} \|^{2}_{L^{2}}, t \right)
\end{split}
\end{equation}
%
%
is a solution to mKDV.  
\begin{framed}
  \begin{remark}
    To see this, we plug in
$v_{n}(x,t)$ into the mKDV and get, via the chain rule
%
%
\begin{equation*}
\begin{split}
  & \p_{t} v_{n}(x,t) + \p_{x} v_{n} \cdot -\| u_{0,n} \|_{L^{2}}^{2} +
  \p_{x}^{3} v_{n}(x,t) + v_{n}^{2}(x,t) \p_{x} v_{n}(x,t)
  \\
  & = \p_{t} v_{n}(x,t) + \p_{x}^{3} v_{n}(x,t) + \left( v_{n}^{2}(x,t) -
  \int_{\ci} u_{0,n}^{2} dx \right)
  \\
  & = \p_{t} u_{n}(x - t \| u_{0} \|^{2}_{L^{2}},t) + \p_{x}^{3} u(x - t \|
  u_{0} \|^{2}_{L^{2}},t) + \left[ u^{2}(x - t \| u_{0},t) \|^{2}_{L^{2}}) -
  \int_{\ci} u^{2}_{0,n} dx \right]
  \\
  & = 0
\end{split}
\end{equation*}
%
%
due to conservation of the $L^{2}$ norm for solution to the normalized mKDV (see
appendix).
\end{remark}
\end{framed}
%
%
Since $v_{n_{k}}(0) = u_{0,n_{k}} \in
H^{\infty}$, Theorem \ref{thm:bour} implies that $v_{n_{k}}\in C^{\infty}(\rr,
H^{\infty})$. %
Hence $u_{n_{k}} \in C^{\infty}(\rr, H^{\infty})$. Having done
enough introductory work, we now prove ill-posedness via contradiction. 
Assume that the data to soluion map at $u_{0}$ is continuous from
$L^{2}$ equipped with the weak topology into $D'\left( \ci \times [0,T]
\right)$. Since $v_{n}(0) = u_{0,n} \overset{L^{2}}{\rightharpoonup}
u_{0}$, we must have %
%
\begin{equation*}
\begin{split}
  v_{n_{k}} \to v \ \text{in} \ D'\left( \ci \times [0,T] \right).
\end{split}
\end{equation*}
%
%
Therefore, we have
%
%
%
%
%
%
\begin{equation}
  \label{fgg}
\begin{split}
  u_{n_{k}}(x - t \| u_{0,n_{k}} \|_{L^{2}}^{2}, t) \to v(x,t) \ \text{in} \
  D'\left( \ci \times [0,T] \right).
\end{split}
\end{equation}
%
We now need the following.
%
%
%%%%%%%%%%%%%%%%%%%%%%%%%%%%%%%%%%%%%%%%%%%%%%%%%%%%%
%
%
%                diagonal conv
%
%
%%%%%%%%%%%%%%%%%%%%%%%%%%%%%%%%%%%%%%%%%%%%%%%%%%%%%
%
%
%
%
%
%
%
%
%
%
%%%%%%%%%%%%%%%%%%%%%%%%%%%%%%%%%%%%%%%%%%%%%%%%%%%%%
%
%
%                conv
%
%
%%%%%%%%%%%%%%%%%%%%%%%%%%%%%%%%%%%%%%%%%%%%%%%%%%%%%
%
%
\begin{lemma}
  Suppose $c_{n} \xrightarrow{\rr} c$, and $u_{n}(x,t), u(x,t)$ as above. Then
  for $t \in [0, T]$, there exists a subsequence $u_{n_{k}}(x,t)$ such that   
  %
  \begin{equation*}
  \begin{split}
    u_{n_{k}}(x + c_{n_{k}}, t)\overset{L^{2}}{\rightharpoonup} u(x + c, t).
  \end{split}
  \end{equation*}
  %
  %
%
%
\label{lem:key-conv}
\end{lemma}
%
%
%
%
Hence, by \eqref{fgg} and Lemma \ref{lem:key-conv}, we have
%
%
\begin{equation*}
\begin{split}
& u_{n_{k}}(x - t \| u_{0,n_{k}} \|_{L^{2}}^{2}, t) \to v(x,t) \ \text{in} \
  D'\left( \ci \times [0,T] \right),
  \\
  & u_{n_{k}}(x - t \| u_{0,n_{k}} \|_{L^{2}}^{2}, t) 
  \overset{L^{2}}{\rightharpoonup}
  u(x - t [\| u_{0} \|_{L^{2}}^{2} + \pi k^{2}], t).
\end{split}
\end{equation*}
%
%
Hence, by the uniqueness of the limit
%
%
\begin{equation*}
\begin{split}
  u(x,t) = v(x + t [\|u_{0}  \|_{L^{2}}^{2} + \kappa^{2} \pi],t) \ \text{on} \left[
  0,T \right].
\end{split}
\end{equation*}
%
%
Note that $v(0) = u_{0} \in H^{\infty}$, and so $v(t) \in H^{\infty}$ for
$0 \le t \le T$ by Theorem \ref{thm:bour}. Therefore, $u(x,t)$ is a classical solution of the normalized mKDV for $t \in
[0,T]$ which also solves
%
%
\begin{equation*}
\begin{split}
  u_{t} + u_{xxx} + \left( u^{2} - \int_{\ci} u^{2} dx  - k^{2} \pi
  \right)u_{x}.
\end{split}
\end{equation*}
%
%
Subtracting normalized mKDV from the above, we obtain
%
%
\begin{equation*}
\begin{split}
  k^{2} \pi u_{x} = 0
\end{split}
\end{equation*}
%
%
which implies $u(0)$ is constant, which is a contradiction. \qed
    

\appendix
\section{Conserved Quantities for the g-KDV and Normalized mKDV Needed in the Paper}
We consider the generalized Korteweg-de Vries (g-KDV) initial value problem
%
%
\begin{gather}
  u_{t} + u_{xxx} + (u^{k})_{x} = 0, \quad k \in \mathbb{N}
  \label{gkdv}
  \\
  u(0) = u_{0}(x) \in \ci \ \text{or} \ \rr
  \label{gkdv-init}
\end{gather}
%
%
and establish the following conservation laws. The computations are formal, and
rely on certain smoothness and integrability properties of $u(x,t)$, which can
be established via the well-posedness theory for the g-KDV. 
\subsection{Conservation of Mass} 
\label{ssec:c-mass}
Integrating both sides of \eqref{gkdv} and apply the fundamental theorem of
calculus to get
%
%
\begin{equation*}
\begin{split}
  \frac{d}{dt} \int u dx & = - \int u_{xxx} dx + \int (u^{k})_{x}  = 0
\end{split}
\end{equation*}
%
%
where we are deliberately ambiguous about the limits of integration. 
Note that to establish this conservation law for the normalized mkdv, we need
only establish that
%
%
\begin{equation*}
\begin{split}
  \int \left ( \int u^{2} dx \right ) u_{x} dx = 0.
\end{split}
\end{equation*}
%
%
But 
\begin{equation*}
\begin{split}
  \int \left ( \int u^{2} dx \right ) u_{x} dx \simeq \int  u_{x} dx =
  0.
\end{split}
\end{equation*}
%
%
\subsection{Conservation of $L^{2}$}
\label{ssec:c-l2}
Multiplying both sides of \eqref{gkdv} by $u$, integrating, and applying
integration by parts and the
fundamental theorem of calculus, we get
%
%
\begin{equation*}
\begin{split}
  \frac{d}{dt} \int u^{2} 
  & = - \int u u_{xxx} dx - \int (u^{k})_{x} dx
  \\
  & = - \int u u_{xxx} dx
  \\
  & =  \int u_{x} u_{xx}dx
  \\
  & = \frac{1}{2} \int (u_{x}^{2})_{x} dx
  \\
  & = 0.
\end{split}
\end{equation*}
%
%
Note that to establish this conservation law for the normalized mkdv, we need
only establish that
%
%
\begin{equation*}
\begin{split}
  \int \left ( \int u^{2} dx \right ) u u_{x} dx = 0.
\end{split}
\end{equation*}
%
%
But 
\begin{equation*}
\begin{split}
  \int \left ( \int u^{2} dx \right ) u u_{x} dx \simeq \int  (u^{2})_{x} dx =
  0.
\end{split}
\end{equation*}
%
%
%
%
\section{Substituting Normalized mKDV for mKDV}
  When authors are proving results on mKDV, they often substitute the mKDV ivp
  with \eqref{n-mkdv}-\eqref{n-mkdv-init}. To see why, recall the mKDV ivp
  \eqref{mkdv}-\eqref{mkdv-init}. Set $u_{0} = \vp(x)$ and define $c =
  \int_{\ci} \vp^{2} dx$. Consider the ivp
  \begin{gather*}
    u_{t} + u_{xxx} + c u_{x} = 0,
    \\
    u(0) = \vp(x).
  \end{gather*}
 Taking the spatial Fourier transform in space, we obtain
 %
 %
 \begin{gather*}
   \wh{u_{t}}(n, t) = \left[ \wh{u}(n,t) \right]i\left( n^{3} - cn \right)
   \\
   \wh{u}(0) = \wh{\vp}(n).
 \end{gather*}
 %
 %
 This admits the unique solution
 %
 %
 \begin{equation*}
 \begin{split}
   \wh{u}(n,t) = \wh{\vp}(n) e^{i\left( n^{3} - cn \right)t}
 \end{split}
 \end{equation*}
 %
 %
 which by Fourier inversion gives
 %
 %
 \begin{equation*}
 \begin{split}
   u(x,t) = \sum_{n \in \zz} \wh{\vp}(n) e^{i\left[ nx + \left( n^{3} - cn
   \right)t \right]} \doteq S(t)\vp(x).
 \end{split}
 \end{equation*}
 %
 %
Now consider the integral equation
%
%
\begin{equation}
  \label{n-mkdv-like-int}
\begin{split}
  u(t) = S(t)\vp + \int_{0}^{t} S(t- \tau) w(\tau), \quad w =\left[
  -u^{2} + \int_{\ci} u^{2}(x,t) dx
  \right] u_{x}.
\end{split}
\end{equation}
%
%
By Duhamel's principle, this is the integral formulation of the ivp
\begin{gather*}
  u_{t} + u_{xxx} + c u_{x} + w = 0,
  \\
  u(x,0) = \vp(x)
\end{gather*}
or
\begin{gather*}
  u_{t} + u_{xxx} + c u_{x} = \left[ \int_{\ci} u^{2} dx - u^{2}
  \right]u_{x}
  \\
  u(x,0) = \vp(x)
\end{gather*}
which obeys the conservation law $\int_{\ci}u^{2} dx = \int_{\ci} \vp(x)^{2}
dx$. To see this, multiply both sides by $u$ and integrate; rearranging terms,
one obtains $\p_{t} \int_{\ci} u^{2} dx$ on one side, and vanishing terms on the
other. Hence, we conclude that any solution of \eqref{n-mkdv-like-int} is a
solution to the mKDV ivp. Hence, modification of the principal symbol by $cn$
allows us to replace the mKDV with a normalized version. The normalized version admits
some useful cancellations of terms when one rewrites it using fourier sums (see
\cite{Bourgain-Fourier-transfo} and \cite{Molinet:2011uq}). 
%
%
\section{Proofs of Lemmas}
%
%
%
\begin{proof}[Proof of Lemma \ref{lem:key-conv}]
First, note that $u_{n_{k}}(x + c_{n_{k}}, t)$
is Cauchy in the weak
$L^{2}$ topology, since  $u_{n_{k}}(x ,t)$ is Cauchy
and the $\{c_{n} \}$ are uniformly bounded in $\rr$. 
Next, by the conservation of mass we have 
%
%
%
%
\begin{equation*}
\begin{split}
  \lim_{k \to \infty} \int_{\ci} u_{n_{k}}(x + c_{n_{k}}, t) dx 
  & = \lim_{k \to \infty} \int_{\ci}
  u_{0,n_{k}}(x + c_{n_{k}}) dx
  \\
  & = \lim_{k \to \infty}\int_{\rr} u_{0,n_{k}}(x) \chi_{[c_{n_{k}}, 2 \pi +
  c_{n_{k}}]} dx.
\end{split}
\end{equation*}
%
%
Since the $u_{0,n_{k}}$ are uniformly bounded in $L^{\infty}$, by dominated
convergence this is equal to
%
%
\begin{equation*}
\begin{split}
  \int_{\rr} \lim_{k \to \infty} u_{0,n_{k}}(x) \chi_{[c_{n_{k}}, 2 \pi + c_{n_{k}}]}
  & = \int_{\rr} \lim_{k \to \infty} u_{0,n_{k}}(x) \chi_{[c_{k}, 2 \pi  +
  c_{k}]} dx
  \\
  & = \int_{c}^{2 \pi + c} \lim_{k \to \infty} u_{0,n_{k}}(x) dx
  \\
  & = \lim_{k \to \infty} \int_{c}^{2 \pi + c}  u_{0,n_{k}}(x) dx
  \\
  & = \lim_{k \to \infty} \int_{\ci} u_{0,n_{k}}(x+c) dx
  \\
  & = \int_{\ci} u_{0}(x +c) dx
\end{split}
\end{equation*}
where the last line follows from the fact that $u_{0,n_{k}}
\overset{L^{2}(\ci)}{\rightharpoonup} u_{0}$ and that $1 \in L^{2}(\ci)$.
Hence, invoking the conservation of mass one last time, we see that
%
%
\begin{equation*}
\begin{split}
  \lim_{k \to \infty} \int_{\ci} u_{n_{k}}(x + c_{n_{k}}, t) dx = \int_{\ci}
  u_{0}(x +c) dx = \int_{\ci} u(x +c, t) dx.
\end{split}
\end{equation*}
%
%
Therefore, by the uniqueness of the limit in weak $L^{2}$, the proof is complete. 
%
%
%
\end{proof}
% %
%
%
%
%
%
\newcommand{\etalchar}[1]{$^{#1}$}
\providecommand{\bysame}{\leavevmode\hbox to3em{\hrulefill}\thinspace}
\providecommand{\MR}{\relax\ifhmode\unskip\space\fi MR }
% \MRhref is called by the amsart/book/proc definition of \MR.
\providecommand{\MRhref}[2]{%
  \href{http://www.ams.org/mathscinet-getitem?mr=#1}{#2}
}
\providecommand{\href}[2]{#2}
\begin{thebibliography}{CKS{\etalchar{+}}03}

\bibitem[Bou93]{Bourgain-Fourier-transfo}
J. Bourgain, \emph{Fourier transform restriction phenomena for certain
  lattice subsets and applications to nonlinear evolution equations. ii. the
  kdv-equation}, Geom. Funct. Anal. \textbf{3} (1993), no.~3, 209--262.

\bibitem[CKS{\etalchar{+}}03]{Colliander:2003kx}
J. Colliander, M. Keel, G. Staffilani, H. Takaoka, and T. Tao, \emph{Sharp global
  well-posedness for kdv and modified kdv on {$\Bbb R$} and {$\Bbb T$}}, J.
  Amer. Math. Soc. \textbf{16} (2003), no.~3, 705--749 (electronic).

\bibitem[KT05]{Kappeler:2005kx}
T.~Kappeler and P.~Topalov, \emph{Global well-posedness of m{K}d{V} in
  {$L^2(\Bbb T,\Bbb R)$}}, Comm. Partial Differential Equations \textbf{30}
  (2005), no.~1-3, 435--449.

\bibitem[KT06]{Kappeler:2006uq}
\bysame, \emph{Global wellposedness of {K}d{V} in {$H^{-1}(\Bbb T,\Bbb R)$}},
  Duke Math. J. \textbf{135} (2006), no.~2, 327--360. 

\bibitem[Mol11]{Molinet:2011uq}
L. Molinet, \emph{Sharp ill-posedness results for the kdv and mkdv equations
  on the torus}.

\end{thebibliography}        %\nocite{*}
%\bibliography{/Users/davidkarapetyan/math/bib-files/references}
%\bibliographystyle{amsalpha}
        \end{document}
