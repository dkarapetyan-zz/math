%
\documentclass[12pt,reqno]{amsart}
\usepackage{amssymb}
\usepackage{appendix}
\usepackage[showonlyrefs=true]{mathtools} %amsmath extension package
\usepackage{cancel}  %for cancelling terms explicity on pdf
\usepackage{subfig}
\usepackage{yhmath}   %makes fourier transform look nicer, among other things
\usepackage{framed}  %for framing remarks, theorems, etc.
\usepackage{enumerate} %to change enumerate symbols
\usepackage[margin=2.5cm]{geometry}  %page layout
\setcounter{tocdepth}{1} %must come before secnumdepth--else, pain
\setcounter{secnumdepth}{1} %number only sections, not subsections
%\usepackage[pdftex]{graphicx} %for importing pictures into latex--pdf compilation
\numberwithin{equation}{section}  %eliminate need for keeping track of counters
%\numberwithin{figure}{section}
\setlength{\parindent}{0in} %no indentation of paragraphs after section title
\renewcommand{\baselinestretch}{1.1} %increases vert spacing of text
%
\usepackage{hyperref}
\hypersetup{colorlinks=true,
linkcolor=blue,
citecolor=blue,
urlcolor=blue,
}
\usepackage[alphabetic, initials, msc-links]{amsrefs} %for the bibliography; uses cite pkg. Must be loaded after hyperref, otherwise doesn't work properly (conflicts with cref in particular)
\usepackage{cleveref} %must be last loaded package to work properly
%
%
\newcommand{\ds}{\displaystyle}
\newcommand{\ts}{\textstyle}
\newcommand{\nin}{\noindent}
\newcommand{\rr}{\mathbb{R}}
\newcommand{\nn}{\mathbb{N}}
\newcommand{\zz}{\mathbb{Z}}
\newcommand{\cc}{\mathbb{C}}
\newcommand{\ci}{\mathbb{T}}
\newcommand{\zzdot}{\dot{\zz}}
\newcommand{\wh}{\widehat}
\newcommand{\p}{\partial}
\newcommand{\ee}{\varepsilon}
\newcommand{\vp}{\varphi}
\newcommand{\wt}{\widetilde}
%
\DeclareMathOperator{\sech}{sech}
\DeclareMathOperator{\csch}{csch}

%
%
%
%
\newtheorem{theorem}{Theorem}[section]
\newtheorem{lemma}[theorem]{Lemma}
\newtheorem{corollary}[theorem]{Corollary}
\newtheorem{claim}[theorem]{Claim}
\newtheorem{prop}[theorem]{Proposition}
\newtheorem{proposition}[theorem]{Proposition}
\newtheorem{no}[theorem]{Notation}
\newtheorem{definition}[theorem]{Definition}
\newtheorem{remark}[theorem]{Remark}
\newtheorem{examp}{Example}[section]
\newtheorem {exercise}[theorem] {Exercise}
%
\makeatletter \renewenvironment{proof}[1][\proofname] {\par\pushQED{\qed}\normalfont\topsep6\p@\@plus6\p@\relax\trivlist\item[\hskip\labelsep\bfseries#1\@addpunct{.}]\ignorespaces}{\popQED\endtrivlist\@endpefalse} \makeatother%
%makes proof environment bold instead of italic
\newcommand{\uol}{u^\omega_\lambda}
\newcommand{\lbar}{\bar{l}}
\renewcommand{\l}{\lambda}
\newcommand{\R}{\mathbb R}
\newcommand{\RR}{\mathcal R}
\newcommand{\al}{\alpha}
\newcommand{\ve}{q}
\newcommand{\tg}{{tan}}
\newcommand{\m}{q}
\newcommand{\N}{N}
\newcommand{\ta}{{\tilde{a}}}
\newcommand{\tb}{{\tilde{b}}}
\newcommand{\tc}{{\tilde{c}}}
\newcommand{\tS}{{\tilde S}}
\newcommand{\tP}{{\tilde P}}
\newcommand{\tu}{{\tilde{u}}}
\newcommand{\tw}{{\tilde{w}}}
\newcommand{\tA}{{\tilde{A}}}
\newcommand{\tX}{{\tilde{X}}}
\newcommand{\tphi}{{\tilde{\phi}}}
\synctex=1
\begin{document}
\title{Notes on the Local Ill-Posedness of the MKdV}
\author{David Karapetyan}
\address{Department of Mathematics  \\
    University  of Notre Dame\\
        Notre Dame, IN 46556 }
        \date{06/06/2011}
        %
        \maketitle
        %
        %
        %
        %
        %
        %
        \section{The Fourier Transform of Solitons}
        \label{sec:comps}

        The generalized Korteweg-de Vries (gKDV) ivp
        %
        %
        \begin{gather*}
          u_{t} + u_{xxx} + u^{n}u_{x} = 0, 
          \\
          u(x,0) = u_{0}(x), \quad t, \ x \in \rr, \ n \in \zz^{+}
        \end{gather*}
        %
        %
        admits a one parameter $(k)$ family of soliton solutions
        %
        %
        \begin{equation*}
        \begin{split}
          u_{n}(x,t,k) = \left( \frac{n+2}{2} \right)^{1/n} k^{2/n}
          \sech^{2/n}\left[ \frac{n}{2}(kx - k^{3}t) \right], \quad 0 < k <
          \infty.
        \end{split}
        \end{equation*}
        %
        %
        We wish to compute the Fourier transform of 
        \begin{equation}
          \label{kdv-soliton}
          \begin{split}
            u_{1}(x, t, k) =
        \frac{3}{2} k^{2} \sech^{2}\left[ \frac{1}{2}(kx - k^{3}t)
        \right]
      \end{split}
    \end{equation}
    and 
        \begin{equation}
          \label{mkdv-soliton}
            u_{2}(x, t, k) = \sqrt{2} k \sech\left( kx -
        k^{3}t \right).
      \end{equation}
                

\subsection{The Fourier Transform of $\sech(x)$.} 
\label{ssec:four-sech}
        %
        %
        %
        We wish to find an explicit
        formula for
        %
        %
        \begin{equation}
          \label{sech-four-init}
        \begin{split}
          \wh{\sech(x)}(\xi) = 2 \int_{- \infty}^{ \infty} e^{-i x \xi}
          \frac{1}{e^{x} + e^{-x}} dx 
        \end{split}
        \end{equation}
        %
        %
        by computing the integral on the right hand side using contour
        integration in the complex plane and the residue theorem.
        Let $D = D(R)$ be a rectangle in the complex plane with vertices at radius
        $ -R, R, R + i \pi, -R + i \pi$, and let $C = C(R) = C_{\text{right}}
        \cup C_{\text{top}}\cup C_{\text{left}} \cup C_{\text{[-R, R]}}$
        be the contour tracing the rectangle counter-clockwise. Note that along
        contour $C_{\text{right}}$, which we parametrize by $z = R + it, \ 0 \le
        t \le \pi$, we have the
        bound
        %
        %
        \begin{equation*}
        \begin{split}
          | e^{-iz \xi} \frac{1}{e^{z} + e^{-z}} | = \frac{e^{\xi t}}{| e^{R + it} +
          e^{-R -it}|} \lesssim \frac{e^{\xi t}}{e^{R}}
        \end{split}
        \end{equation*}
        %
        %
        due to the (global in $t$) estimate
        %
        %
        \begin{equation}
          \label{key-abs-est}
        \begin{split}
          | e^{R + it} + e^{-R -it} |
          & = | e^{R}(\cos t + i \sin t) + e^{-R}(\cos t - i \sin t) |
          \\
          & = | (e^{R} + e^{-R})\cos t + (e^{R} - e^{-R})i \sin t |
          \\
          & = \sqrt{e^{2R} + 2(\cos^{2} t - \sin^{2} t ) + e^{-2R}}
          \\
          & \ge \sqrt{e^{2R} -2}, \qquad R \ge \frac{1}{2}\log 2
          \\
          & \ge \frac{1}{2}e^{R}, \qquad R \ge \log 2.
        \end{split}
        \end{equation}
        Similarly, along $C_{\text{left}}$, which we parametrize by $z = -R +
        it, \ 0 \le t \le \pi$, we have the bound
        %
        %
        \begin{equation*}
        \begin{split}
          | e^{-iz \xi} \frac{1}{e^{z} + e^{-z}} | = \frac{e^{\xi t}}{| e^{-R + it} +
          e^{R -it}}| \lesssim \frac{e^{\xi t}}{e^{R}}
        \end{split}
        \end{equation*}
        via \eqref{key-abs-est}, with $-t$ substituted for $t$. 
        Hence, for $R \ge \log 2$
        %
        %
        \begin{equation}
          \label{right-decay}
        \begin{split}
          | \int_{C_{\text{left}} \cup C_{\text{right}}}
          e^{-iz \xi} \frac{1}{e^{z} + e^{-z}} dz |
        \lesssim \frac{1}{e^{R}} \int_{0}^{\pi} e^{t \xi} dt \to 0.
     \end{split}
   \end{equation}
        %
        %
       % 
        %
        Lastly, using the change of variable $z = t + \pi i$, $-R \le t \le R$,
        we have
        %
        %
        \begin{equation}
          \label{top-exp}
        \begin{split}
          \int_{C_{\text{top}}}e^{-iz \xi} \frac{1}{e^{z} + e^{-z}} dz
          & = \int_{R}^{-R} e^{-i(t + \pi i)\xi} 
          \frac{1}{e^{t + \pi i} + e^{-t - \pi i}} dt
          \\
          & = e^{ \pi \xi}  \int_{R}^{-R} e^{-it \xi} \frac{1}{-e^{t} -
          e^{-t}} dt
          \\
          & = e^{ \pi \xi}  \int_{-R}^{R} e^{-it \xi} \frac{1}{e^{t} +
          e^{-t}} dt
          \\
          & = e^{ \pi \xi} \wh{\frac{1}{e^{t} + e^{-t}}}(\xi).
        \end{split}
        \end{equation}
        %
        while
        %
        %
        \begin{equation*}
        \begin{split}
          \int_{C_{[-R, R]}}e^{-iz \xi} \frac{1}{e^{z} + e^{-z}} dz
        & = \int_{-R}^{R}e^{-it \xi} \frac{1}{e^{t} + e^{-t}} dt
        \to \wh{ \frac{1}{e^{t} + e^{-t}}}(\xi).
        \end{split}
        \end{equation*}

        %
        Next, note that $$\frac{e^{-i z \xi}}{e^{z} +
        e^{-z}}$$ has a singularity in $D$ only at the
        point $z = \pi i/2$. Therefore, applying the residue theorem gives
        %
        %
        \begin{equation}
          \label{1h}
        \begin{split}
          2 \pi i \, \text{Res}
          \left( \frac{ e^{-iz \xi}}{e^{z} + e^{-z}}, \frac{\pi i}{2}
          \right)
          & = \lim_{R \to \infty} \int_{C(R)} e^{-i z \xi}
          \frac{1}{e^{z} + e^{-z}} dz 
          \\
          & = (1 + e^{ \pi \xi})
          \wh{\frac{1}{e^{t} + e^{-t}}}(\xi).
        \end{split}
        \end{equation}
        %
        %
        %
        For $f(z)$ with a pole of order $n$ at $c$, we have the formula
        %
        %
        \begin{equation}
          \label{res-form}
        \begin{split}
          \text{Res}(f, c) =  \frac{1}{(n-1)!} \lim_{z \to c}
          \frac{d}{dz^{n-1}} \left[ (z-c)^{n} f(z) \right].
        \end{split}
        \end{equation}
        %
        If $f(z) = g(z)/h(z)$ has a pole or order $1$ at $c$, then the above
        simplifies to
        %
        %
        \begin{equation}
          \label{res-form-1-order}
        \begin{split}
          \text{Res}(f,c) = \frac{g(c)}{h'(c)}.
        \end{split}
        \end{equation}
        %
        %
        %
        Hence, applying \eqref{res-form-1-order}, we compute
        %
        %
        %
        \begin{equation*}
        \begin{split}
          \text{Res} \left( \frac{ e^{-iz \xi}}{e^{z} + e^{-z}},
          \frac{\pi i}{2} \right)
          & = \frac{e^{-iz \xi} \vert_{z = \pi i/2}}{(e^{z} + e^{-z})'
          \vert_{z = \pi i/2}}
          \\
          & = - \frac{i e^{ \pi \xi/2}}{2}.
        \end{split}
        \end{equation*}
        %
        Substituting into \eqref{1h}, we obtain
        %
        %
        %
        %
        %
        \begin{equation*}
        \begin{split}
          2 \pi i \left( - \frac{ie^{ \pi \xi/2}}{2} \right) =  (1 +
          e^{ \pi \xi}) \wh{\frac{1}{e^{t} + e^{-t}}}(\xi)
        \end{split}
        \end{equation*}
        %
        %
        or
        \begin{equation*}
        \begin{split}
          \wh{\frac{1}{e^{t} + e^{-t}}}(\xi) 
          & =
           \pi \frac{e^{ \pi \xi /2}}{e^{ \pi \xi} + 1}
          \\
          & = \frac{ \pi}{2}  \sech(\pi \xi).
        \end{split}
        \end{equation*}
        Plugging this into \eqref{sech-four-init}, we see that
        %
        %
        \begin{equation}
          \label{sech-four-fin}
        \begin{split}
        \wh{\sech(x)}(\xi) = \pi \sech(\pi \xi).
        \end{split}
        \end{equation}
        %
        %
        Applying the equalities
        \begin{equation}
          \label{four-scale}
        \begin{split}
        \wh{f(kx)}(\xi) = \frac{1}{k}\wh{f}(\xi/k)
        \end{split}
        \end{equation}
        %
        and
        %
        %
        \begin{equation}
          \label{time-trans}
        \begin{split}
          \wh{f(x-ct)}(\xi) = e^{-ict \xi} \wh{f}(\xi)
        \end{split}
        \end{equation}
        to \eqref{sech-four-fin}, we conclude that  
        %
        %
        \begin{equation}
          \label{mkdv-sol-four}
        \begin{split}
          \wh{u_{2}}(\xi, t, k) =
          \sqrt{2} \pi e^{-ik^{3}t \xi} \sech \left (\frac{\pi
          \xi}{2k} \right ).
        \end{split}
        \end{equation}
        %
                %
\subsection{The Fourier Transform of $\sech^{2}(x)$.} 
        \label{ssec:sech-sq-four}
        Next, we prove that
        %
        %
        \begin{equation*}
        \begin{split}
          \wh{\sech^{2}(x)}(\xi) = \xi \csch(\xi).
        \end{split}
        \end{equation*}
        %
        %
        Observe that
        %
        %
        \begin{equation*}
        \begin{split}
          \sech^{2}x = \frac{4}{(e^{x} + e^{-x})^{2}}.
        \end{split}
        \end{equation*}
        %
        %
        %
        %
        We wish to find an explicit
        formula for
        %
        %
        \begin{equation*}
        \begin{split}
          \wh{\frac{1}{(e^{x} + e^{-x})^{2}}}(\xi)
          = \int_{- \infty}^{ \infty} e^{-ix \xi}
          \frac{1}{(e^{x}+e^{-x})^{2}} dx
        \end{split}
        \end{equation*}
        %
        %
        by computing the integral on the right hand side using contour
        integration in the complex plane and the residue theorem.
        Let $D = D(R)$ be a rectangle in the complex plane with vertices at radius
        $ -R, R, R + i \pi, -R + i \pi$, and let $C = C(R) = C_{\text{right}}
        \cup C_{\text{top}}\cup C_{\text{left}} \cup C_{\text{[-R, R]}}$
        be the contour tracing the rectangle counter-clockwise. Note that along
        contour $C_{\text{right}}$, which we parametrize by $z = R + it, \ 0 \le
        t \le \pi$, we have the
        bound
        %
        %
        \begin{equation*}
        \begin{split}
          | e^{-iz \xi} \frac{1}{(e^{z} + e^{-z})^{2}}|
          = \frac{e^{\xi t}}{| e^{R + it} +
          e^{-R -it}|^{2}} \lesssim \frac{e^{\xi t}}{e^{2R}}
        \end{split}
        \end{equation*}
        %
        %
        due to \eqref{key-abs-est}.
        Similarly, along $C_{\text{left}}$, which we parametrize by $z = -R +
        it, \ 0 \le t \le \pi$, we have the bound
        %
        %
        \begin{equation*}
        \begin{split}
          | e^{-iz \xi} \frac{1}{(e^{z} + e^{-z})^{2}} | =
          \frac{e^{\xi t}}{| e^{-R + it} +
          e^{R -it}|^{2}} \lesssim \frac{e^{\xi t}}{e^{2R}}
        \end{split}
        \end{equation*}
        via \eqref{key-abs-est}, with $-t$ substituted for $t$. 
        Hence, for $R \ge \log 2$
        %
        %
        \begin{equation}
          \label{right-decay-sech-sq}
        \begin{split}
          | \int_{C_{\text{left}} \cup C_{\text{right}}}
          e^{-iz \xi} \frac{1}{(e^{z} + e^{-z})^{2}} dz |
        \lesssim \frac{1}{e^{2R}} \int_{0}^{\pi} e^{t \xi} dt \to 0.
     \end{split}
   \end{equation}
        %
        %
       % 
        %
        Using the change of variable $z = t + \pi i$, we have
        %
        %
        \begin{equation}
          \label{top-exp-sechsq}
        \begin{split}
          \int_{C_{\text{top}}}e^{-iz \xi} \frac{1}{(e^{z} + e^{-z})^{2}} dz
          & = \int_{R}^{-R} e^{-i(t + \pi i)\xi} 
          \frac{1}{(e^{t + \pi i} + e^{-t - \pi i})^{2}} dt
          \\
          & = e^{ \pi \xi}  \int_{R}^{-R} e^{-it \xi} \frac{1}{(e^{t} +
          e^{-t})^{2}} dt
          \\
          & = -e^{ \pi \xi}  \int_{-R}^{R} e^{-it \xi} \frac{1}{(e^{t} +
          e^{-t})^{2}} dt
          \\
          & = -e^{ \pi \xi} \wh{\frac{1}{(e^{t} + e^{-t})^{2}}}(\xi).
        \end{split}
        \end{equation}
        %
        %
        while
        %
        %
        \begin{equation*}
        \begin{split}
          \int_{C_{[-R, R]}}e^{-iz \xi} \frac{1}{(e^{z} + e^{-z})^{2}} dz
        & = \int_{-R}^{R}e^{-it \xi} \frac{1}{(e^{t} + e^{-t})^{2}} dt
        \to \wh{ \frac{1}{(e^{t} + e^{-t})^{2}}}(\xi).
        \end{split}
        \end{equation*}
        %
        %
        %
        Next, note that $e^{-i z \xi}/(e^{z} + e^{-z})^{2}$
        has a singularity in $D$ only at the
        point $z = \pi i/2$ (in fact, it is a pole of order $2$).
        Therefore, by the residue theorem
        %
        %
        \begin{equation*}
        \begin{split}
          2 \pi i \, \text{Res} \left[ \frac{ e^{-i z \xi}}{(e^{z} +
          e^{-z})^{2}}, \frac{\pi i}{2} \right]
          & = \lim_{R \to \infty} \int_{C(R)} e^{-i z \xi}
          \frac{1}{(e^{z} + e^{-z})^{2}} dz 
          \\
          & = (1 - e^{ \pi \xi})
          \wh{\frac{1}{(e^{t} + e^{-t})^{2}}}(\xi).
        \end{split}
        \end{equation*}
        %
        %
        %
        Applying \eqref{res-form} and the equality
        %
        %
        \begin{equation*}
        \begin{split}
          \lim_{z \to c} \frac{df}{dz} \Big |_{z} = \lim_{z \to 0}
          \frac{df}{dz}\Big |_{z + c} = \lim_{z \to 0} \frac{d}{dz}f(z + c)
        \end{split}
        \end{equation*}
        %
        %
        we compute

        %
        %
        %
        \begin{equation*}
        \begin{split}
         & \text{Res} \left( \frac{e^{-iz \xi}}{(e^{z} + e^{-z})^{2}}, \frac{\pi i}{2} 
        \right)
        \\
        & = \lim_{z \to \pi i/2} \left[ (z - \pi i /2 )^{2} \frac{e^{-iz
        \xi}}{(e^{z} + e^{-z})^{2}} \right]'
        \\
        & = \lim_{z \to 0} \left[ z^{2} \frac{e^{-i(z + \pi i /2)
        \xi}}{(e^{z + \pi i/2} + e^{-z - \pi i /2})^{2}} \right]'
        \\
        & = -  e^{ \pi \xi/2} \lim_{z \to 0} \left[ z^{2}
        \frac{e^{-i z\xi}}{(e^{z} - e^{-z })^{2}} \right]'
        \\
        & = - e^{ \pi \xi/2} \lim_{z \to 0} \left[ 
        \frac{ - i \xi e^{-iz \xi} z^{2} (e^{z} - e^{-z})^{2} - 2z^{2} e^{-iz
        \xi}(e^{z} + e^{-z})(e^{z} - e^{-z})}{(e^{z} - e^{-z})^{4}} +
        \frac{2z e^{-i z \xi}}{(e^{z} - e^{-z})^{2}}
        \right ]
        \\
        &  = - e^{ \pi \xi/2} \lim_{z \to 0} \left[ \frac{ - i \xi
        z^{2}}{(e^{z} - e^{-z})^{2}} - \frac{4z^{2}}{(e^{z} - e^{-z})^{3}} +
        \frac{2z}{(e^{z} - e^{-z})^{2}}
        \right ]
      \end{split}
        \end{equation*}
        %
        which gives
        %
        %
        \begin{equation*}
        \begin{split}
        & - e^{ \pi \xi/2} \lim_{z \to 0} \left[ \frac{ - i \xi
        z^{2}}{e^{-2z}(e^{2z} - 1)^{2}} - \frac{4z^{2}}{e^{-3z}(e^{2z} - 1)^{3}} +
        \frac{2z}{e^{-2z}(e^{2z} - 1)^{2}}
        \right ]
        \\
        & = - e^{ \pi \xi/2} \lim_{z \to 0} \left[ \frac{ - i \xi
        \cancel{z^{2}}}{\cancel{(e^{z} - 1)^{2}}(e^{z} + 1)^{2}}
        - \frac{4 \cancel{z^{2}}}{(e^{z} -
        1)^{\cancel{3}}(e^{z} + 1)^{3}} +
        \frac{2\cancel{z}}{(e^{z} - 1)^{\cancel{2}}(e^{z}+1)^{2}}
        \right ]
        \\
        & = \frac{i \xi}{4} e^{\pi \xi /2}
        - e^{ \pi \xi/2} \lim_{z \to 0} \left[ -\frac{1}{2(e^{z} -1)} +
        \frac{1}{2(e^{z}-1)} \right]
        \\
        & = \frac{i \xi}{4} e^{\pi \xi /2}.
        \end{split}
        \end{equation*}
        %
        %
        %
        Hence,  
        %
        %
        %
        \begin{equation*}
        \begin{split}
          2 \pi i\left( \frac{i\xi}{4} e^{\pi \xi/2} \right) = \left (1 -
          e^{ \pi \xi} \right ) \wh{\frac{1}{(e^{t} + e^{-t})^{2}}}(\xi)
        \end{split}
        \end{equation*}
        or
        %
        %
        \begin{equation*}
        \begin{split}
          \wh{\frac{4}{(e^{t} + e^{-t})^{2}}}(\xi) = \pi \xi \frac{2  e^{ \pi \xi
          /2}}{e^{ \pi \xi} -1}.
        \end{split}
        \end{equation*}
        %
        %
        Therefore, 
        %
        %
        \begin{equation*}
        \begin{split}
          \wh{\sech^{2}(x)}(\xi)
          & = \pi \xi \csch( \pi \xi/2).
        \end{split}
        \end{equation*}
        %
        %
        %
        %
        Applying \eqref{four-scale} and \eqref{time-trans}, we conclude that
        %
        %
                %
        %
        \begin{equation}
          \label{kdv-sol-four}
        \begin{split}
          \wh{u_{1}}(\xi, t, k) = \frac{3 \pi }{2} e^{-ik^{3}t \xi} \xi
          \csch \left (\frac{\pi \xi}{2k} \right ).
        \end{split}
        \end{equation}
        %
        %
In fact, using the method of residues, one can compute $\wh{u_{n}}(\xi, t, k)$
        for arbitrary $n$. However, this is tedious and computationally
        intensive.  In the next section, we investigate a more efficient method
        for computing $\wh{u_{n}}(\xi, t, k)$. 
        %
        \section{An Alternative Computation} 
        %
        %
        Observe that
        %
        %
        \begin{equation*}
        \begin{split}
          \wh{\sech^{2/n}\left( \frac{nkx}{2} \right)}(\xi) =
          \int_{-\infty}^{\infty}e^{-ix\xi}\left( \frac{e^{nkx/2}}{1 +
          e^{nkx}} \right)^{2/n} dx.
        \end{split}
        \end{equation*}
        %
        %
        Using the change of variable $t = e^{nkx}$ gives
        %
        %
        \begin{equation*}
        \begin{split}
          & \int_{0}^{\infty} e^{-i \frac{\log t}{nk} \xi} \left(
          \frac{t^{1/2}}{1 + t} \right)^{2/n} \frac{dt}{nkt}
          \\
          & = \frac{1}{nk} \int_{0}^{\infty} \frac{t^{\frac{1}{n} -1 -\frac{i
          \xi}{nk}}}{(1 + t)^{\frac{2}{n}}} dt
        \end{split}
        \end{equation*}
        %
        %
       where the integral is equal to the beta function
       %
       %
       \begin{equation*}
       \begin{split}
         B(x,y) = \int_{0}^{\infty} \frac{t^{x-1}}{(1 + t)^{x+y}} dt, \quad
         \text{Re}(x) >0, \ \text{Re}(y) > 0
       \end{split}
       \end{equation*}
       %
       %
       with 
       %
       %
       \begin{equation*}
       \begin{split}
       & x = \frac{1}{n} - \frac{i \xi}{nk}, 
       \\
       & y = \frac{1}{n} + \frac{i \xi}{nk}.
       \end{split}
       \end{equation*}
       %
       %
       Therefore
       %
       %
       \begin{equation*}
       \begin{split}
       \wh{\sech^{2/n}\left( \frac{nkx}{2} \right)}(\xi) =
       \frac{1}{nk} B\left( \frac{1}{n} - \frac{i \xi}{nk}, \frac{1}{n} + \frac{i \xi}{nk}
       \right).
       \end{split}
       \end{equation*}
       %
       %
       Recall that
       %
       %
       \begin{equation*}
       \begin{split}
         B(x,y) = \frac{\Gamma(x) \Gamma(y)}{\Gamma(x+y)}, \quad
         \text{Re}(x) > 0, \ \text{Re}(y) > 0
       \end{split}
       \end{equation*}
       %
       %
       where $\Gamma(z)$ is the gamma function
       %
       %
       \begin{equation*}
       \begin{split}
         \Gamma(z) = \int_{0}^{\infty} t^{z-1} e^{-t} dt, \quad
         \text{Re}(z) > 0.
       \end{split}
       \end{equation*}
       %
       %
       Hence,
       %
       %
       \begin{equation*}
         \begin{split}
           \wh{\sech^{2/n}\left( \frac{nkx}{2} \right)}(\xi)
           & =
           \frac{1}{nk}
           \frac{\Gamma \left (\frac{1}{n} - \frac{i \xi}{nk} \right ) \Gamma \left (
           \frac{1}{n} + \frac{i \xi}{nk} \right )}{\Gamma \left (\frac{2}{n} \right )}
           \\
           & = 
           \frac{| \Gamma \left (\frac{1}{n} + \frac{i \xi}{nk} \right)|^{2}}
           {nk \Gamma \left (\frac{2}{n} \right )}
         \end{split}
       \end{equation*}
       %
       where the last step follows from the equality.
       %
       %
       \begin{equation*}
       \begin{split}
         \Gamma(\bar{z}) = \overline{\Gamma(z)}.
       \end{split}
       \end{equation*}
       %
       %
       We conclude that
       %
       \begin{equation*}
         \begin{split}
           \wh{u_{n}}(\xi,t,k) = e^{-ik^{3}t \xi}
           \left( \frac{n+2}{2} \right)^{1/n} \frac{k^{2/n -1}}{n}
           \frac{|\Gamma \left (\frac{1}{n} + \frac{i \xi}{nk} \right)|^{2}}
           {\Gamma \left (\frac{2}{n}
           \right )}.
         \end{split}
       \end{equation*}
       %
       %
       %%%%%%%%%%%%%%%%%%%%%%%%%%%%%%%%%%%%%%%%%%%%%%%%%%%%%
       %
       %
       %                Ill posedness
       %
       %
       %%%%%%%%%%%%%%%%%%%%%%%%%%%%%%%%%%%%%%%%%%%%%%%%%%%%%
       %
       %
       \section{Ill-Posedness for mKDV} 
       \label{sec:ill-pos-mkdv}
       The modified Korteweg-de Vries equation (mkDV)
       \begin{gather}
          u_{t} + u_{xxx} + u^{2}u_{x} = 0, 
          \label{mkdv}
          \\
          u(x,0) = u_{0}(x), \quad x, t \in \rr
          \label{mkdv-data}
        \end{gather}
        is given by setting $n =2$ in gKDV. We shall prove the following result.
        %
        %
        %%%%%%%%%%%%%%%%%%%%%%%%%%%%%%%%%%%%%%%%%%%%%%%%%%%%%
        %
        %
        %                ill-pos-thm
        %
        %
        %%%%%%%%%%%%%%%%%%%%%%%%%%%%%%%%%%%%%%%%%%%%%%%%%%%%%
        %
        %
        \begin{theorem}
          The mKDV initial value problem \eqref{mkdv}-\eqref{mkdv-data} is
          locally ill-posed in $H^{s}, \ s<-1/2$. That is, for any $T > 0$, 
          existence, uniqueness, or continuity of the data-to-solution map fails
          for $0 < t < T$. 
        \label{thm:mkdv-ill-pos}
        \end{theorem}
        %
        %
        First some preliminaries. Let 
        %
        %
        \begin{equation}
          \label{seq-init-data}
        \begin{split}
          u_{\ee, 0}(x) \doteq \frac{\sqrt{2}}{\ee} \sech\left( \frac{x}{\ee}
          \right).
        \end{split}
        \end{equation}
        %
        %
        Then by \eqref{mkdv-soliton}, $u_{\ee, 0}(x)$ has an associated soliton
        solution
        %
        %
        \begin{equation}
          \label{seq-sol}
        \begin{split}
          u_{\ee}(x,t) = \frac{\sqrt{2}}{\ee} \sech\left( \frac{x}{ \ee} -
          \frac{t}{ \ee^{3}} \right).
        \end{split}
        \end{equation}
        %
        %
        We wish to study the convergence of
        $\eqref{seq-init-data}-\eqref{seq-sol}$ in $H^{s}$ as $\ee \to 0$. We
        have the following. 
        %
        %
        %%%%%%%%%%%%%%%%%%%%%%%%%%%%%%%%%%%%%%%%%%%%%%%%%%%%%
        %
        %
        %                init-data-strong-conv
        %
        %
        %%%%%%%%%%%%%%%%%%%%%%%%%%%%%%%%%%%%%%%%%%%%%%%%%%%%%
        %
        %
        \begin{lemma}
          For $s< -1/2$,
          %
          %
          \begin{equation*}
          \begin{split}
            u_{\ee, 0}(x) \xrightarrow{H^{s}} \sqrt{2} \pi \delta(x).
          \end{split}
          \end{equation*}
          \label{lem:seq-data-strong-conv}
        \end{lemma}

          %
          %
          %
          \begin{proof}
            By \eqref{mkdv-sol-four},
            %
            %
            \begin{equation*}
            \begin{split}
              \wh{u_{\ee, 0}}(\xi) = \sqrt{2} \pi \sech\left( \frac{\ee \pi
              \xi}{2} \right).
            \end{split}
            \end{equation*}
            %
            %
            Hence,
            %
            %
            \begin{equation}
              \label{strong-conv-eq}
            \begin{split}
              \| u_{\ee, 0}(\cdot) -
              \sqrt{2} \pi \delta(\cdot) \|_{H^{s}}
              & = \int_{\rr} (1 + \xi^{2})^{s} | \wh{u_{\ee, 0}}(\xi) -
              \wh{\sqrt{2} \pi \delta(\cdot)}(\xi) |^{2} d \xi
              \\
              & = \int_{\rr} (1 + \xi^{2})^{s} | \sqrt{2} \pi \sech\left(
              \frac{\ee \pi \xi}{2}
              \right) - \sqrt{2} \pi |^{2} d \xi
              \\
              & = 2 \pi^{2} \int_{\rr} (1 + \xi^{2})^{s} |\sech\left( \frac{\ee \pi
              \xi}{2}
              \right) -1  |^{2} d \xi.
            \end{split}
            \end{equation}
            %
            %
            Note that
            %
            %
            \begin{equation*}
            \begin{split}
              (1 + \xi^{2})^{s} |\sech\left( \frac{\ee \pi \xi}{2} \right ) -1
              |^{2}
              \le 2(1 + \xi^{2})^{s} \in L^{1}, \quad s<-1/2.
            \end{split}
            \end{equation*}
            %
            %
            Hence, applying dominated convergence to \eqref{strong-conv-eq}, we
            obtain
            %
            %
            \begin{equation*}
            \begin{split}
              \lim_{\ee \to 0}
              \| u_{\ee, 0}(\cdot) -
              \sqrt{2} \pi \delta(\cdot) \|_{H^{s}}
              & = 2 \pi^{2} \lim_{\ee \to 0}
              \int_{\rr} (1 + \xi^{2})^{s} |\sech\left( \frac{\ee \pi
              \xi}{2}
              \right) -1  |^{2} d \xi
              \\
              & = 2 \pi^{2}  
              \int_{\rr} \lim_{\ee \to 0}
              (1 + \xi^{2})^{s} |\sech\left( \frac{\ee \pi \xi}{2}
              \right) -1  |^{2} d \xi
              \\
              & = 0. 
            \end{split}
            \end{equation*}
            %
            %
          \end{proof}
          %
          %
          %
          On the following page, we have provided graphs of
          $u_{\ee,0}(x)$ with $\ee = 1$, $\ee = 0.1$, and $\ee = 0.01$
          respectively. It is clear that the graphs are converging to some
          multiple of $\delta(x)$ as $\ee$ becomes small.
          \newpage
\begin{figure}[!ht]
  \begin{center}
  \vspace{-30mm}
  \subfloat{\includegraphics[scale=0.4]{sech-ep-1}}
\hspace{-20mm}
\subfloat{\includegraphics[scale=0.4]{sech-ep-01}}
\\
\vspace{-50mm}
\subfloat{\includegraphics[scale=0.4]{sech-ep-001}}
\end{center}
\end{figure}
%
%
\vspace{-30mm}
%
\begin{framed}
  \begin{remark}
    We have no analogue of Lemma \ref{lem:seq-data-strong-conv} for the KDV. To
    see this, consider the sequence of initial data 
    %
    %
    \begin{equation*}
    \begin{split}
      u_{\ee,0}(x) = \frac{3}{2\ee^{2}}\sech^{2}\left(
      \frac{x}{2 \ee} \right).
    \end{split}
    \end{equation*}
    %
    %
    From \eqref{kdv-soliton}, we see that for KDV the associated soliton
    solutions are
    %
    %
    \begin{equation*}
    \begin{split}
      u_{\ee}(x,t) = \frac{3}{2\ee^{2}}
      \sech^{2}\left( \frac{x}{2 \ee} - \frac{t}{\ee^{3}}
        \right).
    \end{split}
    \end{equation*}
    %
    From \eqref{kdv-sol-four}
    %
    %
    \begin{equation*}
    \begin{split}
      \wh{u_{\ee, 0}}(\xi) = \frac{3 \pi}{2} \xi \csch\left( \frac{\pi \xi
      \ee}{2} \right).
    \end{split}
    \end{equation*}
    %
    %
    Leting
    $k \in \mathbb{N}$, $c \in \rr$ be arbitrary, we see that for any $s \in \rr$,
    %
    %
    \begin{equation*}
    \begin{split}
      \| u_{\ee,0}(\cdot) - c \delta^{(k)} \|_{H^{s}}^{2}
      & = \int_{\rr}
      (1 + \xi^{2})^{s} | \wh{u_{\ee,0}}(\xi) - \wh{c \delta^{(k)}}(\xi)
      |^{2} d \xi
      \\
      & =  \int_{\rr} (1 + \xi^{2})^{s} | \frac{3 \pi}{2} \xi \csch\left(
      \frac{\pi \xi \ee}{2} \right) - c(i \xi)^{k}
       | d \xi
       \\
       & \to \infty 
    \end{split}
    \end{equation*}
    %
    as $\ee \to 0$.
    Hence the method of proof used to show ill-posedness for
    mKDV does not work for KDV. 
    %
    
    %
    \end{remark}
  \end{framed}
          We have a weaker result for the convergence of the associated soliton
          solutions $u_{\ee}(x,t)$.
        %
        %
        %%%%%%%%%%%%%%%%%%%%%%%%%%%%%%%%%%%%%%%%%%%%%%%%%%%%%
        %
        %
        %                weak-conv-solns
        %
        %
        %%%%%%%%%%%%%%%%%%%%%%%%%%%%%%%%%%%%%%%%%%%%%%%%%%%%%
        %
        %
        \begin{lemma}
        For any $s \in \rr$,
        %
        %
        \begin{equation*}
        \begin{split}
          u_{\ee}(x,t) \rightharpoonup 0 \ \text{in} \ H^{s} \ \text{for} \ t>
          0.
        \end{split}
        \end{equation*}
        %
        %
        \label{lem:weak-conv}
        \end{lemma}
        %
        %
        %
        %
        \begin{proof}
          For any test function $\vp(x) \in C_{0}^{\infty}(\rr)$, we have
          %
          %
          \begin{equation*}
          \begin{split}
            \langle u_{\ee}, \vp \rangle 
            & = \int_{\rr} u_{\ee}(x,t) \vp(x) dx
            \\
            & = \frac{\sqrt{2}}{\ee} \int_{\rr} \vp(x) \sech\left( \frac{x}{\ee}
            - \frac{t}{\ee^{3}}
            \right) dx.
          \end{split}
          \end{equation*}
          %
          %
          Using the change of variable $y = x/\ee - t/\ee^{3}$, we obtain
          %
          %
          \begin{equation*}
          \begin{split}
            \langle u_{\ee}, \vp \rangle = \sqrt{2} \int_{\rr} \vp\left( \ee y +
            \frac{t}{\ee^{2}} 
            \right)\sech(y) dy.
          \end{split}
          \end{equation*}
          %
          %
          %
          Since
          %
          %
          \begin{equation*}
          \begin{split}
            | \vp\left( \ee y + \frac{t}{\ee^{2}} \right) \sech y | \le
            c_{\vp} \sech y \in L^{1}
          \end{split}
          \end{equation*}
          %
          %
           we conclude via dominated convergence that
          %
          %
          \begin{equation*}
          \begin{split}
            \lim_{\ee \to 0} \langle u_{\ee}, \vp \rangle  
            & = \sqrt{2} \int_{\rr} \lim_{\ee \to 0} \vp\left( \ee y +
            \frac{t}{\ee^{2}} \right) \sech y \ dy
            \\
            & = 0, \qquad t > 0
          \end{split}
          \end{equation*}
          %
          %
          since $\vp$ is compactly supported.
        \end{proof}
        %
        %
        %
        %
        \subsection{Proof of Theorem \ref{thm:mkdv-ill-pos}} 
        \label{ssec:pf-thm}
        Let $u_{0}(x) = \sqrt{2} \pi \delta(x)$, and fix $s < -1/2$.
        Then $u_{0} \in H^{s}$. If there does not exist a unique $u(x,t)$
        solving \eqref{mkdv} locally in time with $u(x, 0) = u_{0}(x)$, then we
        are done. So assume otherwise. The proof will be complete if we can show
        that there exists a sequence $u_{\ee}(x,t) \in H^{s}$, $0 \le t \le T$
        such that
        %
        %
        \begin{equation*}
        \begin{split}
          & u_{\ee, 0}(x) \to u_{0}(x) \ \text{in} \ H^{s}, \ \text{but}
          \\
          & u_{\ee}(x,t) \not \to u(x,t) \ \text{in} \ H^{s} \ \text{for} \ t>0.
        \end{split}
        \end{equation*}
        %
        %
        Let 
        %
        %
        \begin{equation*}
        \begin{split}
          u_{\ee, 0}(x) = \frac{\sqrt{2}}{\ee} \sech\left( \frac{x}{\ee}
          \right). 
        \end{split}
        \end{equation*}
        %
        %
        Then by Lemma \ref{lem:seq-data-strong-conv}, $u_{\ee, 0} \to
        u_{0}$ strongly in $H^{s}$, $s < -1/2$. Recall that the $u_{\ee, 0}$ have
        the associated solutions
        %
        %
        \begin{equation*}
        \begin{split}
          u_{\ee}(x,t) = \frac{\sqrt{2}}{\ee} \sech\left( \frac{x}{ \ee} -
          \frac{t}{ \ee^{3}} \right).
        \end{split}
        \end{equation*}
        %
        %
        If the $u_{\ee}(x,t)$ do not converge strongly in $H^{s}$ for $t > 0$, we
        are done. So assume that they do converge. Then by Lemma
        \ref{lem:weak-conv}, we must have
        %
        %
        \begin{equation*}
        \begin{split}
          u_{\ee}(x,t) \to 0 \ \text{in} \ H^{s} \ \text{for} \ t > 0.
        \end{split}
        \end{equation*}
        %
        %
        But the $0$ solution to mKDV cannot be the solution associated with
        initial data $u_{0}(x) = \sqrt{2} \pi \delta(x)$, since $0(x,t)
        \vert_{t = 0} = 0 \neq \sqrt{2} \pi \delta(x)$. The proof is complete.
        \qed
        %
        %
        \appendix
        \section{Soliton Computations for KDV and mKDV}
        We first compute the solitons for mKDV. Substituting the ansatz
        %
        %
        \begin{equation*}
        \begin{split}
        u(x,t) = f(x-ct)
        \end{split}
        \end{equation*}
        %
        %
        into the mKDV equation gives
        %
        %
        \begin{equation*}
        \begin{split}
          -cf' + f''' + f^{2}f' = 0.
        \end{split}
        \end{equation*}
        %
        %
        Integrating once, we get
        %
        %
        \begin{equation*}
        \begin{split}
          -cf + f'' + \frac{1}{3}f^{3} -a = 0.
        \end{split}
        \end{equation*}
        %
        %
        Multiplfying by $f'$ and integrating gives
        %
        %
        \begin{equation*}
        \begin{split}
          -\frac{c}{2}f^{2} + \frac{1}{2}(f')^{2} + \frac{1}{12}f^{4} -af -b =
          0
        \end{split}
        \end{equation*}
        %
        or
        %
        %
        \begin{equation*}
        \begin{split}
          (f')^{2} = -\frac{1}{6}f^{4} + cf^{2} + 2af + 2b.
        \end{split}
        \end{equation*}
        %
        %
        We assume a priori that $f$ is a soliton. Then $f, f', f'' \to 0$ as $x
        \to \infty$. This implifes $a=0$, and so
        %
        %
        \begin{equation}
          \label{mkdv-ode-with-const}
        \begin{split}
          (f')^{2} = -\frac{1}{6}f^{4} + cf^{2} + 2bf
        \end{split}
        \end{equation}
        %
        %
        or
        %
        %
        \begin{equation*}
        \begin{split}
          \frac{(f')^{2}}{f} = -\frac{1}{6}f^{3} + cf + 2b.
        \end{split}
        \end{equation*}
        %
        %
        But by L'H{\^o}pital's rule
        %
        %
        %
        \begin{equation*}
        \begin{split}
          \lim_{x \to \infty} \frac{(f')^{2}}{f} = \lim_{x \to \infty} \frac{2
          f' f''}{f'} = 2 \lim_{x \to \infty} f'' = 0.
        \end{split}
        \end{equation*}
        %
        %
        Coupling this with the fact that $f \to 0$ as $x \to \infty$, we see
        that we must have $b = 0$. Hence \eqref{mkdv-ode-with-const} reduces to
        %
        %
        \begin{equation*}
        \begin{split}
          (f')^{2} = \frac{1}{6}f^{2}(6c - f^{2}).
        \end{split}
        \end{equation*}
        %
        %
        Assume $| f | \le \sqrt{6c}$. Then
        %
        %
        \begin{equation*}
        \begin{split}
          f' = \sqrt{\frac{1}{6}} f \left( 6c - f^{2} \right)^{1/2}
        \end{split}
        \end{equation*}
        %
        %
        or
        %
        %
        \begin{equation*}
        \begin{split}
          \frac{df}{f\left( 6c - f^{2} \right)^{1/2}} = \sqrt{\frac{1}{6}}dx
        \end{split}
        \end{equation*}
        %
        %
        Integrating both sides and applying the formula
        %
        %
        \begin{equation*}
        \begin{split}
          \int \frac{1}{t(a^{2} - t^{2})^{1/2}}dt = -\frac{1}{a} \log | \frac{a
          + (a^{2}-t^{2})^{1/2}}{t}
          |, \quad t^{2} < a^{2}
        \end{split}
        \end{equation*}
        %
        %
       gives
       %
       %
       \begin{equation*}
       \begin{split}
         -\frac{1}{\sqrt{6c}} \log | \frac{\sqrt{6c} + (6c - f^{2})^{1/2}}{f} |
         = \sqrt{\frac{1}{6}} + C
       \end{split}
       \end{equation*}
       %
       %
       or
       %
       %
       \begin{equation*}
       \begin{split}
         \frac{\sqrt{6c} + (6c - f^{2})^{1/2}}{f} = D e^{- \sqrt{c} x}.
       \end{split}
       \end{equation*}
       %
       %
       Let 
       \begin{equation}
         \label{sub}
         f(x) = \sqrt{6c} \sech v.
       \end{equation}
       Then 
       %
       %
       \begin{equation*}
       \begin{split}
         \frac{\sqrt{6c} + (6c - 6c \sech^{2}v)^{1/2}}{\sqrt{6c} \sech v} =
         D e^{- \sqrt{c} x}
       \end{split}
       \end{equation*}
       %
       %
       or 
       %
       %
       \begin{equation*}
       \begin{split}
         \frac{1 + \tanh v}{\sech v} = De^{-\sqrt{c} x}
       \end{split}
       \end{equation*}
       %
       %
       or
       %
       %
       \begin{equation*}
       \begin{split}
         \cosh v + \sinh v = De^{-\sqrt{c} x}.
       \end{split}
       \end{equation*}
       %
       %
       Using the definitions of $\cosh v$ and $\sinh v$, we see that
       %
       %
       \begin{equation*}
       \begin{split}
         \frac{e^{v} + e^{-v}}{2} + \frac{e^{v} - e^{-v}}{2}  = De^{-\sqrt{c} x}
       \end{split}
       \end{equation*}
       %
       %
       or
       %
       %
       \begin{equation*}
       \begin{split}
         e^{v} = De^{-\sqrt{c} x}. 
       \end{split}
       \end{equation*}
       %
       %
       Letting $D =1$, we see that
       %
       %
       \begin{equation*}
       \begin{split}
         \sech v =  \frac{2}{e^{v} + e^{-v}} = \frac{2}{e^{- \sqrt{c}x} +
         e^{\sqrt{c}x}} = \sech{\sqrt{c} x}.
       \end{split}
       \end{equation*}
       %
       %
       Substituting back into \eqref{sub}, we obtain
       %
       %
       \begin{equation*}
       \begin{split}
         f(x) = \sqrt{6 c} \sech{\sqrt{c} x}.
       \end{split}
       \end{equation*}
       %
       %
       %
       %
       %
       %

        




        %\nocite{*}
        %\bibliography{/Users/davidkarapetyan/Documents/math/}

        \end{document}
