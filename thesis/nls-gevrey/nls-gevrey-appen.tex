\begin{proof}[Proof of \cref{plem:cutoff-loc-soln}]
%
%
\begin{equation*}
  \begin{split}
    \lim_{t_{n} \to t} \|u(\cdot, t) - u(\cdot, t_{n})\|_{H^s(\ci)} 
    & = \lim_{t_{n} \to t} \|\psi(t) u(\cdot, t) - \psi(t_n) u(\cdot, t_{n})\|_{H^s(\ci)} 
    \\
    & = \lim_{t_n \to t} \left[ \sum_{n \in \zz}\left( 1 + | n |
    \right)^{2s} | \psi(t)  \wh{u}(n, t) - \psi(t_n) \wh{ u}(n, t_n) |^2 \right]^{1/2}
    \\
    & = \lim_{t_n \to t} \left[ \sum_{n \in \zz} \left( 1 + | n |
    \right)^{2s} | \int_{\rr} (e^{it \tau} - e^{it_{n} \tau}) \wh{\psi u}(n,
    \tau) d \tau |^2 \right]^{1/2}.
  \end{split}
\end{equation*}
    It is clear that
    %
    %
    \begin{equation*}
      \begin{split}
        \left( 1 + | n |
        \right)^{2s} | \int_{\rr} (e^{it \tau} - e^{it_{n}\tau}) \wh{\psi u}(n, \tau) d \tau |^2 
    & \le 4  \left( 1 + | n |
    \right)^{2s} \left ( \int_{\rr} |\wh{\psi u}(n, \tau)| d \tau
    \right )^2 
  \end{split}
\end{equation*}
and 
%
%
\begin{equation*}
  \begin{split}
 \sum_{n \in \zz} \left( 1 + | n |
    \right)^{2s} \left ( \int_{\rr} |\wh{\psi u}(n, \tau)| d \tau
    \right ) ^2 
    \le \|\psi u \|_{Y^s}^2 
  \end{split}
\end{equation*}
which is bounded by assumption.
Applying dominated convergence completes the proof. 
\end{proof}
%
%
\begin{proof}[Proof of \cref{plem:schwartz-mult}]
Note that
%
%
\begin{equation*}
	\begin{split}
		\wh{\psi f}\left( n, \tau \right)
		& = \wh{\psi}(\cdot) * \wh{f}(n,
		\cdot)(\tau)
		= \int_\rr \wh{\psi}(\tau_1) \wh{f} \left( n, \tau - \tau_1 \right) 
		d\tau_1
	\end{split}
\end{equation*}
%
%
and hence
%
%
\begin{equation}
	\label{p1b}
	\begin{split}
		\|\psi f\|_{X^s} 
		& = \left( \sum_{n \in \zz} \left (1 + |n| \right )^{2s} \int_\rr \left( 1 + | \tau -
		n^{m} | \right) | \int_\rr \wh{\psi}(\tau_1) \wh{f}\left( n, \tau -
		\tau_1
		\right)  d \tau_1 d \tau |^2 \right)^{1/2}
		\\
		& \le \left( \sum_{n \in \zz} \left (1 + |n| \right )^{2s} \int_\rr \left( 1 + | \tau -
		n^{m }
		|
		\right) \left( \int_\rr \wh{\psi}\left( \tau_1 \right) \wh{f}\left( n,
		\tau - \tau_1
		\right)  d \tau_1 d \tau \right)^2 \right)^{1/2}.
	\end{split}
\end{equation}
%
%
Using the relation
%
%
\begin{equation*}
	\begin{split}
		1 + | \tau - n^{m } |
		& = 1 + | \tau + \tau_1 - n^{m} |
		\\
		& \le 1 + | \tau_1 | + | \tau - \tau_1 - n^{m} |
		\\
		& \le \left( 1 + | \tau_1 | \right)\left( 1 + | \tau - \tau_1 -
		n^{m} | \right),
	\end{split}
\end{equation*}
%
%
we obtain
%
%
\begin{equation*}
	\begin{split}
		\eqref{p1b}
		& \le \left( \sum_{n \in \zz} \left (1 + |n| \right )^{2s} \right.
		\\
		& \times \left . \int_\rr \left(
		\int_\rr \left( 1 + | \tau_1 | \right)^{1/2} | \wh{\psi}(\tau_1) |
		\left( 1 + | \tau - \tau_1 - n^{m} | \right)^{1/2} \wh{f}\left( n, \tau
		- \tau_1
		\right)d \tau_1
		\right)^2 d \tau \right)^{1/2}
	\end{split}
\end{equation*}
%
%
which by Minkowski's inequality is bounded by
%
%
\begin{equation}
	\label{p2b}
	\begin{split}
		& \left( \sum_{n \in \zz} \left (1 + |n| \right )^{2s}  \right.
		\\
		& \times \left. \left( \int_\rr \left[ \int_\rr
		\left( 1 + | \tau_{1} | \right) | \wh{\psi}(\tau_1) |^2 \left( 1 + |
		\tau - \tau_1 - n^{m} |
		\right) | \wh{f}\left( n, \tau - \tau_1 \right) |^2 d \tau_1 
		\right]^{1/2} d \tau \right)^2 \right)^{1/2}.
	\end{split}
\end{equation}
%
%
Using the change of variable $\tau - \tau_1 \to \lambda$ gives
%
%
\begin{equation*}
	\begin{split}
		\eqref{p2b}
		& = \left( \sum_{n \in \zz} \left (1 + |n| \right )^{2s}\right.
		\\
		& \times \left.  \left( \int_\rr \left[
		\int_\rr \left( 1 + | \tau_1 | \right) | \wh{\psi}\left( \tau_1
		\right) |^2 \left( 1 + | \lambda - n^{m} | \right) | \wh{f} \left( n,
		\lambda
		\right)|^2 d \tau_1 \right]^{1/2} d \lambda \right)^2 \right)^{1/2}
		\\
		& =  \left( \sum_{n \in \zz} \left (1 + |n| \right )^{2s} \right.
		\\
		& \times \left. \left( \int_\rr \left( 1 + |
		\tau_1 |
		\right)^{1/2} | \wh{\psi}(\tau_1) | d \tau_1 \left[ \int_\rr \left( 1 + |
		\lambda - n^{m} |
		\right) | \wh{f}\left( n, \lambda \right) |^2 d \lambda \right]^{1/2}
		\right)^2 \right)^{1/2}
		\\
		& = c_{\psi} \left( \sum_{n \in \zz} \left (1 + |n| \right )^{2s} \left( \left[ \int_\rr
		\left( 1 + | \lambda - n^{m} | \right) | \wh{f}\left( n, \lambda
		\right) |^2 d \lambda
		\right]^{\cancel{1/2}} \right)^{\cancel{2}} \right)^{1/2}
		\\
		& = c_{\psi} \|f\|_{X^s},
	\end{split}
\end{equation*}
%
%
which proves \eqref{pschwartz-mult-piece-1}.
Estimating 
%
%
\begin{equation*}
\begin{split}
  \| \psi f \|_{E}^{2}
  & = \sum_{n \in \zz} | n |^{2s} \left( \int_{\rr} |
  \wh{\psi f}(n, \tau)
  | d \tau \right)^{2}
  \\
  & = \sum_{n \in \zz} | n |^{2s} \left( \int_{\rr} \wh{\psi} * \wh{f}
  (n, \tau) d \tau \right)^{2}
  \\
  & \le \| \wh{\psi} \|_{L^1}^{2} \sum_{n \in \zz} | n |^{2s} \left(
  \int_{\rr} \wh{f}(n, \tau) d \tau
  \right)^{2} \quad \text{(Young's Inequality)}
  \\
  & = c_{\psi} \| f \|_{E}^2.
\end{split}
\end{equation*}
%
%
and taking square roots of both sides gives \eqref{pschwartz-mult-piece-2}. Combining
\eqref{pschwartz-mult-piece-1} and \eqref{pschwartz-mult-piece-2} we obtain
\eqref{pschwartz-mult}, completing the proof.  
\end{proof}
%
%
\begin{proof}[Proof of \cref{plem:splitting}] We have
%
%
\begin{equation}
	\label{p6a}
	\begin{split}
		1 + | a + b + c| 
		& \le 1 + | a | + | b | + | c |
		\\
		& \le 1 + | a | + 1 + | b | + 1 + | c |
		\\
		& \le 3\left( \max\{1+| a |, 1+| b |, 1+ | c | \}\right)
		\\
		& \le 3 \left( 1 + | a | \right)\left( 1 + | b | \right) \left( 1 + |
		c |
		\right), \qquad a, b, c \in \zz.
	\end{split}
\end{equation}
%
%
Raising both sides of expression $\eqref{p6a}$ to the $v$ power completes 
the proof. 
\end{proof}
%
%
\section{Scaling and Invariances}
The cubic NLS has a number of invariances. 
\begin{enumerate}[(i)]
  \item{ $u_{\gamma} = \gamma u(\gamma x, \gamma^{2} t)$ with critical Sobolev index
    $s_{c} = -1/2$}.
    \label{pinvar-it-1}
  \item{ $u_{a,b}(x,t) = u(x -a, t - b), a \in \rr, b \in \rr$}.
    \label{pinvar-it-2}
  \item{ $u_{\theta}(x,t) = e^{i \theta} u(x, t), \theta \in \rr$}.
    \label{pinvar-it-3}
  \item{ $u_{c}(x,t) = e^{icx - i| c^{2} |t}u(x -2tc, t), c \in \rr$}.
    \label{pinvar-it-4}
\end{enumerate}
%
%
\begin{proof}[Proof of \eqref{pinvar-it-1}]
Let $u(x, t)$ be a solution to the $NLS$ equation, that is
%
$$
NLS(u)=
i \p_t u + \p_x^{2} u  + | u |^{2} u  = 0
$$
%
We would like to find the constants
$a, b, c$ such that
\[
u_\gamma (x, t) = \gamma^a u(\gamma^b x, \gamma^c t)
\]
is also a solution to $NLS$.  Since 
$$
NLS(u_\gamma)=
i \gamma^{a + c} \p_t u + \gamma^{a + 2b} \p_x^{2} u  + \gamma^{3a} | u |^{2} u 
$$
we see that $u_\gamma$ is a $NLS$ solution only if
$$
a=b, c = 2a.
$$
Setting $a=1$ completes the proof. 
To find the critical Sobolev index, we compute
%
%
\begin{equation}
\begin{split}
  \| u_{\gamma} \|_{\dot{H}^s(\ci)} 
  & = \gamma \| u(\gamma x, \gamma^2 t) \|_{\dot{H}^{s}(\ci)}
  \\
  & = \gamma \left( \int_{\rr} | \xi |^{2s} | \wh{u(\gamma x,
  \gamma^{2} t)}^x (\xi, t)| \right)^{1/2}.
\end{split}
\label{pcrit-ind-comp}
\end{equation}
%
But
%
%
\begin{equation*}
\begin{split}
  \wh{u(\gamma x, \gamma^{2}t)^x}(\xi, t)
  & = \int_{\rr}e^{-i\xi x}u(\gamma x, \gamma^2 t) dx
  \\
  & = \frac{1}{\gamma} \int_{\rr}e^{-i \frac{n}{\gamma} x'}u(x',
  \gamma^{2} t) dx'
  \\
  & = \frac{1}{\gamma} \wh{u(\cdot, \gamma^{2}t)}(\frac{\xi}{\gamma})
\end{split}
\end{equation*}
%
%
Substituting back into \eqref{pcrit-ind-comp}, we obtain
%
%
\begin{equation*}
\begin{split}
  \| u_{\gamma} \|_{\dot{H}^s(\rr)} 
  & = \gamma\left( \int_{\rr} | \xi |^{2s} |
  \frac{1}{\gamma}\wh{u(\cdot, \gamma^{2}t)}(\frac{\xi}{\gamma}) |^2 d \xi
  \right)^{1/2}
  \\
  & = \left( \int_{\rr}| \xi |^{2s} | \wh{u(\cdot,
  \gamma^{2}t)}(\frac{\xi}{\gamma}) |^2 d \xi  \right)^{1/2}
  \\
  & = \left( \int_{\rr} | \gamma \xi' |^{2s} 
  \wh{u(\cdot, \gamma^{2}t)}(\xi') |^2 \gamma d \xi
  \right)^{1/2}
  \\
  & = \gamma^{s + 1/2} \|u(\cdot, \gamma^{2}t) \|_{\dot{H}^s (\ci)}.
\end{split}
\end{equation*}
%
%
Therefore, $\| u_{\gamma(0)} \|_{\dot{H}^s(\rr)} = \gamma^{s + 1/2} \|
u_{0} \|_{\dot{H}^{s}(\rr)}$, and so $s=-1/2$ is the critical Sobolev index.
\end{proof}
%
%
\begin{proof}[Proof of \eqref{pinvar-it-2}] Let $u(x, t)$ be a solution to the $NLS$ equation.
We would like to find constants
$a, b \in \rr$ such that $$u_{a,b}(x,t) = u(x -a, t - b)$$ is a solution the NLS
equation. Since
%
%
\begin{equation*}
\begin{split}
  NLS(u_{a,b}) = i \p_t u + \p_x^{2} u  + | u |^{2} u 
\end{split}
\end{equation*}
%
%
we see that $u_{a,b}$ is a solution to the NLS equation for all $a, b \in \rr$.
This completes the proof.
\end{proof}
%
%
\begin{proof}[Proof of \eqref{pinvar-it-3}]
Let $u(x, t)$ be a solution to the $NLS$ equation.
We would like to find a constant
$c \in \cc$ such that
\[
u_c (x, t) = c u(x, t)
\]
is also a solution to $NLS$.  Since 
$NLS(u_{c}) = c i \p_{t} u  + c \p_{x}^{2} u + |c|^2c | u |^{2}u$, we obtain
$|c|^{2} = 1$, or $c = e^{i \theta}$, for any $\theta \in \rr$.
\end{proof}
%
%
\begin{proof}[Proof of \eqref{pinvar-it-4}]
Pending. Want to find a clever (i.e. natural)
way of coming up with the galilean invariance, rather than just verifying the relation.
\end{proof}
%
%
%
