\begin{proof}[Proof of \cref{nlem:cutoff-loc-soln}]
%
%
\begin{equation}
  \label{ndm}
	\begin{split}
		\lim_{t_{k} \to t} \|u(\cdot, t) - u(\cdot, t_{k})\|_{H^s(\ci)} 
    & = \lim_{t_{k} \to t} \|\psi_{\delta}(t) u(\cdot, t) - \psi_{\delta}(t_{k}) u(\cdot, t_{k})\|_{H^s(\ci)} 
		\\
		& = \lim_{t_{k} \to t} \left[ \sum_{n}\left( 1 + | n |
    \right)^{2s} | \psi_{\delta}(t)  \wh{u}(n, t) - \psi_{\delta}(t_{k}) \wh{ u}(n, t_{k}) |^2 \right]^{1/2}
		\\
		& = \lim_{t_{k} \to t} \left[ \sum_{n} \left( 1 + | n |
    \right)^{2s} | \int_{\rr} (e^{it \tau} - e^{it_{k} \tau})
    \wh{\psi_{\delta} u}(n,
		\tau) d \tau |^2 \right]^{1/2}.
	\end{split}
\end{equation}
First note that
%
%
%
%
\begin{equation*}
\begin{split}
& \lim_{t_{k} \to t}  | \int_{\rr} (e^{it \tau} - e^{it_{k} \tau})
    \wh{\psi_{\delta} u}(n,
		\tau) d \tau |^2 
    \\
    = 
     & \lim_{t_{k} \to t}  \int_{\rr} (e^{it \tau} - e^{it_{k} \tau})
    \wh{\psi_{\delta} u}(n,
    \tau) d \tau \times \lim_{t_{k} \to t} \overline{\int_{\rr} (e^{it \tau} - e^{it_{k} \tau})
    \wh{\psi_{\delta} u}(n,
    \tau) d \tau }  
    \\
    = 
    &  \lim_{t_{k} \to t}  \int_{\rr} (e^{it \tau} - e^{it_{k} \tau})
    \wh{\psi_{\delta} u}(n,
    \tau) d \tau \times \lim_{t_{k} \to t} \int_{\rr} (e^{-it \tau} - e^{-it_{k} \tau})
    \overline{\wh{\psi_{\delta} u}}(n,
    \tau) d \tau.   
    \end{split}
\end{equation*}
%
%
But for fixed $n$ 
%
%
\begin{equation*}
\begin{split}
|(e^{it \tau} - e^{it_{k} \tau})  
    \wh{\psi_{\delta} u}(n, \tau) | \le 2 |\wh{\psi_{\delta} u(n, \tau)} |
\end{split}
\end{equation*}
%
%
and
%
%
%
\begin{equation*}
\begin{split}
  \int_{\rr} |2 \wh{\psi_{\delta} u(n, \tau)} | d \tau < \infty.
\end{split}
\end{equation*}
%
%
Hence, by dominated convergence
%
%
\begin{equation*}
\begin{split}
\lim_{t_{k} \to t}  \int_{\rr} (e^{it \tau} - e^{it_{k} \tau})
    \wh{\psi_{\delta} u}(n,
    \tau) d \tau =  \int_{\rr} \lim_{t_{k} \to t} (e^{it \tau} - e^{it_{k} \tau})
    \wh{\psi_{\delta} u}(n,
    \tau) d \tau = 0. 
\end{split}
\end{equation*}
%
%
Similarly, 
%
%
%
\begin{equation*}
\begin{split}
\lim_{t_{k} \to t} \int_{\rr} (e^{-it \tau} - e^{-it_{k} \tau})
    \overline{\wh{\psi_{\delta} u}}(n,
    \tau) d \tau  =\int_{\rr}  \lim_{t_{k} \to t} (e^{-it \tau} - e^{-it_{k} \tau})
    \overline{\wh{\psi_{\delta} u}}(n,
    \tau) d \tau  = 0.
\end{split}
\end{equation*}
%
%
Hence
%
%
%
\begin{equation}
  \label{ngh}
\begin{split}
  \lim_{t_{k} \to t} | \int_{\rr} (e^{it \tau} - e^{it_{k} \tau})
    \wh{\psi_{\delta} u}(n,
		\tau) d \tau |^2 = 0.
\end{split}
\end{equation}
%
%
		Furthermore,
    %
    %
    \begin{equation*}
    \begin{split}
      (1 + | n |)^{2s} | \int_{\rr} \left( e^{it\tau} - e^{it_{k} \tau} \right)
      \wh{\psi_{\delta}u}(n, \tau) d \tau|^{2} \le 4 (1 + | n |)^{2s} \left(
      \int_{\rr} | \wh{\psi_{\delta} u}(n, \tau)  | d \tau
      \right)^{2}
    \end{split}
    \end{equation*}
    %
    %
    and
		%
		%
		\begin{equation*}
			\begin{split}
         \sum_{n}  \left( 1 + | n |
        \right)^{2s} \left ( \int_{\rr} |\wh{\psi_{\delta} u}(n, \tau)| d \tau
        \right )^2  
        & = \|\wh{\psi_{\delta} u}\|_{\ell^{2}_{n}L^{1}_{\tau}}^2
		\\
		& \le \|\psi_{\delta} u \|_{X_{s,b}}^2 
	\end{split}
\end{equation*}
which is bounded by assumption. Therefore, applying dominated convergence and
\eqref{ngh}, we
obtain 
%
%
\begin{equation*}
\begin{split}
  \text{rhs of \eqref{ndm}} = \left[ \sum_{n} \left( 1 + | n |
    \right)^{2s} \lim_{t_{k} \to t} | \int_{\rr} (e^{it \tau} - e^{it_{k} \tau})
    \wh{\psi_{\delta} u}(n,
		\tau) d \tau |^2 \right]^{1/2} = 0
\end{split}
\end{equation*}
%
%
completing the proof. 
\end{proof}
%
%
\begin{proof}[Proof of \cref{nlem:schwartz-mult}]
Note that
%
%
\begin{equation*}
	\begin{split}
		\wh{\psi f}\left( n, \tau \right)
		& = \wh{\psi}(\cdot) * \wh{f}(n,
		\cdot)(\tau)
		= \int_\rr \wh{\psi}(\tau_1) \wh{f} \left( n, \tau - \tau_1 \right) 
		d\tau_1
	\end{split}
\end{equation*}
%
%
and hence
%
%
\begin{equation}
	\label{n1b}
	\begin{split}
		\|\psi f\|_{X^s} 
		& = \left( \sum_{n \in \zz} \left (1 + |n| \right )^{2s} \int_\rr \left( 1 + | \tau -
		n^{m} | \right) | \int_\rr \wh{\psi}(\tau_1) \wh{f}\left( n, \tau -
		\tau_1
		\right)  d \tau_1 d \tau |^2 \right)^{1/2}
		\\
		& \le \left( \sum_{n \in \zz} \left (1 + |n| \right )^{2s} \int_\rr \left( 1 + | \tau -
		n^{m }
		|
		\right) \left( \int_\rr |\wh{\psi}\left( \tau_1 \right) | |\wh{f}\left( n,
		\tau - \tau_1
		\right) |  d \tau_1 d \tau \right)^2 \right)^{1/2}.
	\end{split}
\end{equation}
%
%
Using the relation
%
%
\begin{equation*}
	\begin{split}
		1 + | \tau - n^{m } |
    & = 1 + | \tau - \tau_1 + \tau_{1} - n^{m} |
		\\
		& \le 1 + | \tau_1 | + | \tau - \tau_1 - n^{m} |
		\\
		& \le \left( 1 + | \tau_1 | \right)\left( 1 + | \tau - \tau_1 -
		n^{m} | \right)
	\end{split}
\end{equation*}
%
%
we obtain
%
%
\begin{equation*}
	\begin{split}
		\eqref{n1b}
		& \le \left( \sum_{n \in \zz} \left (1 + |n| \right )^{2s} \right.
		\\
		& \times \left . \int_\rr \left(
		\int_\rr \left( 1 + | \tau_1 | \right)^{1/2} | \wh{\psi}(\tau_1) |
		\left( 1 + | \tau - \tau_1 - n^{m} | \right)^{1/2} \wh{f}\left( n, \tau
		- \tau_1
		\right)d \tau_1
		\right)^2 d \tau \right)^{1/2}
	\end{split}
\end{equation*}
%
%
which by Minkowski's inequality is bounded by
%
%
\begin{equation}
	\label{n2b}
	\begin{split}
		& \left( \sum_{n \in \zz} \left (1 + |n| \right )^{2s}  \right.
		\\
		& \times \left. \left( \int_\rr \left[ \int_\rr
		\left( 1 + | \tau_{1} | \right) | \wh{\psi}(\tau_1) |^2 \left( 1 + |
		\tau - \tau_1 - n^{m} |
		\right) | \wh{f}\left( n, \tau - \tau_1 \right) |^2 d \tau_1 
		\right]^{1/2} d \tau \right)^2 \right)^{1/2}.
	\end{split}
\end{equation}
%
%
Using the change of variable $\tau - \tau_1 = \lambda$ gives
%
%
\begin{equation*}
	\begin{split}
		\eqref{n2b}
		& = \left( \sum_{n \in \zz} \left (1 + |n| \right )^{2s}\right.
		\\
		& \times \left.  \left( \int_\rr \left[
		\int_\rr \left( 1 + | \tau_1 | \right) | \wh{\psi}\left( \tau_1
		\right) |^2 \left( 1 + | \lambda - n^{m} | \right) | \wh{f} \left( n,
		\lambda
		\right)|^2 d \tau_1 \right]^{1/2} d \lambda \right)^2 \right)^{1/2}
		\\
		& =  \left( \sum_{n \in \zz} \left (1 + |n| \right )^{2s} \right.
		\\
		& \times \left. \left( \int_\rr \left( 1 + |
		\tau_1 |
		\right)^{1/2} | \wh{\psi}(\tau_1) | d \tau_1 \left[ \int_\rr \left( 1 + |
		\lambda - n^{m} |
		\right) | \wh{f}\left( n, \lambda \right) |^2 d \lambda \right]^{1/2}
		\right)^2 \right)^{1/2}
		\\
		& = c_{\psi} \left( \sum_{n \in \zz} \left (1 + |n| \right )^{2s} \left( \left[ \int_\rr
		\left( 1 + | \lambda - n^{m} | \right) | \wh{f}\left( n, \lambda
		\right) |^2 d \lambda
		\right]^{\cancel{1/2}} \right)^{\cancel{2}} \right)^{1/2}
		\\
		& = c_{\psi} \|f\|_{X^s}.
	\end{split}
\end{equation*}
%
Also, by Young's inequality we have the estimate 
%
%
\begin{equation*}
\begin{split}
  \|\wh{\psi f}\|_{\ell^{2}_{n} L^{1}_{\tau}} 
  & = \left[ \sum_{n \in \zz} \left (1 + |n| \right )^{2s} \left (
  \int_{\rr} | \wh{\psi}(\cdot) * \wh{f}(n, \cdot)(\tau) | d \tau  \right ) ^2 \right]^{1/2}
  \\
  & \le  \left[ \sum_{n \in \zz} (1 + | n |)^{2s} \left( \int_{\rr} |
    \wh{\psi}(\tau) | d \tau  \times \int_{\rr} | \wh{f}(n, \tau) | d \tau
    \right)^{2}\right]^{1/2}
  \\
  & = c_{\psi} \| \wh{f} \|_{\ell^{2}_{n} L^{1}_{\tau}}
\end{split}
\end{equation*}
%
%
%
concluding the proof. 
\end{proof}
%
%
\begin{proof}[Proof of \cref{nlem:splitting}] We have
%
%
\begin{equation}
	\label{n6a}
	\begin{split}
		1 + | a + b + c| 
		& \le 1 + | a | + | b | + | c |
		\\
		& \le 1 + | a | + 1 + | b | + 1 + | c |
		\\
		& \le 3\left( \max\{1+| a |, 1+| b |, 1+ | c | \}\right)
		\\
		& \le 3 \left( 1 + | a | \right)\left( 1 + | b | \right) \left( 1 + |
		c |
		\right).
	\end{split}
\end{equation}
%
%
Raising both sides of expression $\eqref{n6a}$ to the $v$ power completes 
the proof. 
\end{proof}
%
%
%
\section{Remarks about mNLS and Related Equations}
%
\begin{proof}[Why \cref{nprop:trilinear-est} fails when dealing
with nonlinearity $\frac{1}{3} \p_x u^3$]
%
%
%
Recalling \eqref{nnon-lin-rep}, we have
\begin{equation}
	\begin{split}
		| \wh{w_{fgh}}(n, \tau)|
    & = | \sum_{n_1, n_2, n_3 = n}  \int_{\tau_{1} + \tau_{2} + \tau_{3} = \tau}
    n \wh{f}\left( n_1,  \tau_1 
\right) \wh{g}\left( n_2, \tau_2  
\right) \wh{h}\left( n_3, \tau_3 \right) d \tau_1 d \tau_2 d \tau_3 |
\\
& \le \sum_{n_1, n_2, n_3 = n}  \int_{\tau_{1} + \tau_{2} + \tau_{3} = \tau}
| n | \times | \wh{f}\left( n_1, \tau_1 
\right) | \times  | \wh{g}\left( n_2, \tau_2 
\right) | \times | \wh{ h}\left( n_3, \tau_3 \right) | d \tau_1 d \tau_2 d 
\tau_3
\\
& = \sum_{n_1, n_2, n_3 = n}  \int_{\tau_{1} + \tau_{2} + \tau_{3} = \tau} \frac{| n |c_f\left( n_1, \tau_1 
\right)}{\left (1 + |n_1| \right )^s \left( 1 + | \tau_1 - n_1^{m} | \right)^{b}}
\\
& \times \frac{c_{g}\left( n_2, \tau_2 \right)}{\left (1 + |n_2| \right ) 
^s\left( 1 + | \tau_2 -  n_2^{m }| 
\right)^{b}}
 \times \frac{c_{h}\left( n_3, \tau_3 \right)}{\left (1 + |n_3| \right ) ^s\left( 1 + | 
\tau_3 - n_3^{m } | \right)^{b}} \ d \tau_1 d \tau_2 d \tau_3
\end{split}
\end{equation}
where 
%
%
\begin{equation*}
	\begin{split}
		c_\sigma(n, \tau) = \left (1 + |n| \right ) ^s \left( 1 + | \tau - n^{m } |  
		\right)^{b} | \wh{\sigma}\left( n, \tau \right) | .
	\end{split}
\end{equation*}
%
%
Hence
%
%
\begin{equation*}
	\begin{split}
		 & \left (1 + |n| \right )^s \left( 1 + | \tau - n^{m } | \right)^{-b} | \wh{w_{fgh}}\left( 
		n, \tau \right) |
		\\
		& \le \left( 1 + | \tau - n^{m } | \right)^{-b}
		\sum_{n_1, n_2, n_3 = n}  \int_{\tau_{1} + \tau_{2} + \tau_{3} = \tau}
    \frac{\left (1 + |n| \right )^s}{\left (1 +
		|n_1| \right )^s \left (1 + | n_2| \right )^s \left (1 + |n_3| \right )^s} 
		\\
    & \times \frac{c_f(n_1, \tau_1)}{\left( 1 + | \tau_1 - n_1^{m } | 
		\right)^{b}}
		\times
		\frac{c_g(n_2, \tau_2)}{\left( 1 + | \tau_2 - n_2^{m } | 
		\right)^{b}} \times
		\frac{c_h(n_3, \tau_3)}{\left( 1 + | \tau_3 - n_3^{m } | 
		\right)^{b}}\ d \tau_1 d \tau_2 d \tau_3.
	\end{split}
\end{equation*}
%
%
For $s \ge 0$, observe that the quantity 
%
%
\begin{equation}
	\label{nunbounded-quan}
	\begin{split}
		\frac{| n | \left (1 + |n| \right ) ^s}{\left (1 + |n_1| \right ) ^s \left (1 + |n_2| \right ) ^s \left (1 + |n_3| \right ) ^s} 
	\end{split}
\end{equation}
is unbounded (take $n_1 = n_2 = 0$ and $n_3$ arbitrarily large). Hence
we hope that the  principal symbol $\tau - n^m$ offers enough
decay to give control of \eqref{nunbounded-quan}. Following the Bourgain
approach we consider the quantity
%
%
\begin{equation*}
	\begin{split}
		| \tau - n^{m} - \left( \tau_{1} - n_{1}^m + \tau_{2} - n_{2}^m +
		\tau_{3} - n_{3}^m \right) | = |n_{1}^m + n_2^m + n_3^m - n^m|
	\end{split}
\end{equation*}
%
%
and seek a lower bound that is a function of $n$. No such bound exists (this
becomes evident if one sets $n_1 = n_2$). In the case of the KDV, to prove well-posedness we need to bound
\begin{equation}
	\label{nKDV-bound-term}
	\begin{split}
		\frac{| n | \left (1 + |n| \right ) ^s}{\left (1 + |n_1| \right ) ^s \left (1 + |n_2| \right ) ^s} 
	\end{split}
\end{equation}
where $n_1 + n_2 = n$. 
Consider 
%
%
\begin{equation}
  \label{nah}
	\begin{split}
		| \tau - n^{3} - \left( \tau_{1} - n_{1}^3 + \tau_{2} - n_{2}^3 \right) | = |n_{1}^3 + n_2^3 - n^3|
	\end{split}
\end{equation}
where  $\tau_1 + \tau_2 = \tau$. Unlike the mNLS with derivative nonlinearity
considered above, in the case of the KDV we can use the conservation of mass to
\emph{exclude} the pathological cases $n_1=0$ and $n_2=0$, allowing us to obtain
a lower bound of $|n|^{2}$ for \eqref{nah}.
By the pigeonhole principle, we then have three
cases, each with the lower bound
%
%
\begin{equation*}
	\begin{split}
		\frac{1}{| \tau_{i} - n_{i}^{m} |^{b}} \gtrsim \frac{1}{|n|^{b}}	
	\end{split}
\end{equation*}
%
%
where $\tau_0 =\tau, n_0 = n$. 
Hence, we must set $b \ge 1$ to offset the $|n|$ in the numerator of 
\eqref{nKDV-bound-term}.
Lastly, in the case of the mNLS, we are able to bound 
\begin{equation*}
	\begin{split}
    \frac{\left (1 + |n| \right ) ^s}{\left (1 + |n_1| \right ) ^s \left (1 +
    |n_2| \right ) ^s \left (1 + |n_3| \right ) ^s}, \quad s \ge 0 
	\end{split}
\end{equation*}
without relying on any potential smoothing from the principal symbol.
This is due to the
absence of a $|n|$ term in the
numerator. A consequence is that we have more freedom in how
we choose $b$. In fact, we can choose $b$ all the way down to $3/8$, but no
lower, since we must have $b \ge 3/8$ in order to be able to apply
\cref{ncor:four-mult-est-L4}. 
\end{proof}
%
%
\begin{proof}[Conservation of the $L_x^2$ norm.] 
We have
%
%
\begin{equation*}
	\begin{split}
		\frac{d}{dt} \int_\ci | u |^2  dx
		& = \int_\ci \frac{d}{dt} | u |^2  dx
		\\
		& = \int_\ci \frac{d}{dt} \left( u \overline{u} \right)  dx
		\\
		& = \int_\ci \left( u \p_t \overline{u} + \overline{u} \p_t u \right) dx
		\\
		& = \int_\ci \left( u \overline{\p_t u} + \overline{u} \p_t u \right)dx.
	\end{split}
\end{equation*}
%
%
Substituting in $\p_t u = i\left( \p_x^{m} u + | u |^2 u \right)$ we obtain
%
%
\begin{equation*}
	\begin{split}
		& \int_{\ci} \left\{ u\left[ -i\left( \p_x^{m} \overline{u} + | u |^2
		\overline{u} \right) \right] + \overline{u}\left[ i\left( \p_x^{m} u + | u
		|^2 u \right) \right] \right\}dx
		\\
		& = \int_\ci \left[ -iu \p_x^{m} \overline{u} - i| u |^4 + i \overline{u}
		\p_x^{m} u + i | u |^4 \right]dx
		\\
		& = i \int_{\ci}\left( \overline{u} \p_x^{m} u - u \p_x^{m } \overline{u}
		\right)dx.
	\end{split}
\end{equation*}
%
%
Integrating by parts $m/2$ times and using
the spatial periodicity of $u$, the right
hand side simplifies to
%
%
\begin{equation*}
	\begin{split}
    i (-1)^{m/2}\int_\ci \left( \p_x^{m/2} \overline{u} \p_x^{m/2} u - \p_x^{m/2} u
		\p_x^{m/2 } 
		\overline{u} \right) dx = 0.
	\end{split}
\end{equation*}
%
%
Therefore, the $L_x^2(\ci)$ norm of solutions to the mNLS is conserved. 
\end{proof}
%
%
\begin{proof}[Why Assuming Mean Initial Data is Problematic]
Recall the NLS ivp
%
%
\begin{equation*}
	\begin{split}
		&i \p_t u = \p_x^2 u - | u |^2 u,
		\\
		& u(x,0) = \vp(x).
	\end{split}
\end{equation*}
%
%
This is equivalent to the ivp
%
%
\begin{gather*}
		 i \p_t [u - \wh{\vp}(0)]
		  = -\p_x^2 [u - \wh{\vp}(0)] - [u -
		\wh{\vp}(0)][u - \wh{\vp}(0)][\bar{u} - \bar{\wh{\vp}}(0)]
		\\
		- 2| u |^2
		\wh{\vp}(0) + \bar{u}\left[ \wh{\vp}(0) \right]^2 - u^{2}
		\bar{\wh{\vp}}(0) + 2 u | \wh{\vp}(0) |^2 - | \wh{\vp}(0) |^2
		\wh{\vp}(0),
		\\
		u(x,0) = \vp(x) - \wh{\vp}(0)
\end{gather*}
or
%
%
\begin{gather*}
		 i \p_t u 
		  = -\p_x^2 u - [u -
		\wh{\vp}(0)][u - \wh{\vp}(0)][\bar{u} - \bar{\wh{\vp}}(0)]
		\\
		- 2| u |^2
		\wh{\vp}(0) + \bar{u}\left[ \wh{\vp}(0) \right]^2 - u^{2}
		\bar{\wh{\vp}}(0) + 2 u | \wh{\vp}(0) |^2 - \boxed{| \wh{\vp}(0) |^2
		\wh{\vp}(0)},
		\\
		u(x,0) = \vp(x) - \wh{\vp}(0).
\end{gather*}
%
The boxed term is problematic. 
\end{proof}
%
\section{Classical Well-Posedness for the mNLS}
%
%
%%%%%%%%%%%%%%%%%%%%%%%%%%%%%%%%%%%%%%%%%%%%%%%%%%%%%
%
%
%			Alternate WP Theorem	
%
%
%%%%%%%%%%%%%%%%%%%%%%%%%%%%%%%%%%%%%%%%%%%%%%%%%%%%%
%
%
%
%
We have introduced the spaces $Y_s$ in part because well-posedness
in $H^s(\ci)$ for the mNLS becomes problematic as $s$ becomes small. 
On the other hand, for $s > 1/2$, well-posedness in $H^s(\ci)$ is a direct 
consequence of the algebra property of Sobolev spaces and the fact that the operator 
$e^{it \p_x^2}$ isometrically preserves Sobolev spaces. Stated more 
precisely, we have the following result:
%
%
\begin{proposition}
  Let $B_R \doteq \{f \in H^s : \|f\|_{H^s} < R \}$.
  Then the generalized mNLS ivp
\begin{gather}
  \label{general-mNLS-eq}
    i \p_t v = - \p_x^2 u - \lambda |u|^{\alpha -1} u, \ \ \alpha > 
    1, \lambda > 1
    \\
    \label{general-mNLS-init-data}
    u(x,0) = \vp(x), \ \ t \in \rr, \ \ x \in \ci \ \text{or} \ \rr
\end{gather}
  is locally well-posed in $H^s$ for $s > 1/2$ for 
  sufficiently small initial data $\vp \in B_R$, where the lifespan $T$ 
  satisfies 
%
%
\begin{equation*}
  \begin{split}
    T < 1/c
  \end{split}
\end{equation*}
%
%
for some constant $c = c(s, \lambda, \alpha, R, \vp)$.
\end{proposition}
%
%
\begin{proof} We will only provide a proof on the circle; the case on 
the line is nearly identical. The key ingredient
will be to establish that $L$ is a 
contraction on $C([-T, T], B_R)$. For the sake of clarity, we let $H^s_x 
= H^s_x(\ci)$. Let $e^{it \p_x^2}: \mathcal{E}'(\ci) \to 
\mathcal{E}'(\ci)$ be an operator defined by  
%
%
\begin{equation}
  \label{unit-op}
  \begin{split}
    e^{it \p_x^2} f(x) = \left[ e^{(-1)^j i t n^2} \wh{f}(n)
    \right]^{\vee} = 
    \sum_{n \in \zz} e^{i(nx + (-1)^j it n^2)} \wh{f}(n).
  \end{split}
\end{equation}
%
%
First, note that $e^{it \p_x^2}$ 
is unitary on $H^s(\ci)$; that is
%
%
\begin{equation}
  \label{unitary-op}
  \begin{split}
    \|e^{it \p_x^2} f \|_{H^s_x} & = \sum_{n \in \zz} |e^{(-1)^j it n^2} 
    \wh{f}(n)|^2 (1 + n^2)^s  
    \\
    & = \sum_{n \in \zz} |\wh{f}(n)|^2 (1 + n^2)^s 
    \\
    & = \|f\|_{H^s_x}.
  \end{split}
\end{equation}

Rewriting \eqref{general-mNLS-eq}-\eqref{general-mNLS-init-data} in its 
integral form
%
%
\begin{equation}
  \label{mNLS-int-form-with-op}
  \begin{split}
    u(x,t) = e^{it \p_x^2} \vp + i \lambda \int_0^t e^{i(t - 
    t')\p_x^2} |u|^{\alpha -1} u(x, t') \ dt' 
  \end{split}
\end{equation}
%
%
and applying the triangle inequality, Minkowski's inequality, and 
\eqref{unitary-op}, we obtain
%
%
\begin{equation}
  \label{bound-for-L}
  \begin{split}
    & \|Lu\|_{L^\infty_t[-T, T] H^s_x}
    \\
    & \le \|e^{t \p_x^2}
    \vp\|_{L^\infty_t[-T, T] H^s_x} + \|i \lambda \int_0^t e^{i(t - 
    t')\p_x^2} |u|^{\alpha -1} u(x, t') \ dt' \|_{L^\infty_t[-T, T] 
    H^s_x} 
    \\
    & \le \|\vp\|_{H^s_x} + |\lambda| \int_0^T \|e^{i(t 
    -t')\p_x^2} |u|^{\alpha -1} u \|_{L^\infty_t[-T, T] H^s_x} \ 
    dt'
    \\
    & = \|\vp\|_{H^s} + T |\lambda| \|u^\alpha \|_{L^\infty_t[-T, T] H^s_x}.
  \end{split}
\end{equation}
%
%
We now need the following lemma, whose proof can be found in Taylor 
\cite{Taylor_1991_Pseudodifferent}:
%
%
%
\begin{lemma}
  \label{lem:algebra-prop}
  The Sobolev space $H^s$ is an algebra for $s>1/2$. More precisely, 
%
%
\begin{equation}
  \label{algebra-prop}
  \begin{split}
    \|fg\|_{H^s} \le c_s \|f\|_{H^s} \|g\|_{H^s}.
  \end{split}
\end{equation}
%
%
%
\end{lemma}
%
%
Applying \cref{lem:algebra-prop} to estimate \eqref{bound-for-L} gives
%
%
%
%
\begin{equation}
  \label{Tu-space-bound}
  \begin{split}
    \|Lu\|_{L^\infty_t[-T, T] H^s_x}
    & \le \|\vp\|_{H^s_x} + Tc_s | \lambda| \|u\|_{L^\infty_t[-T, T] 
    H^s_x}^\alpha
  \end{split}
\end{equation}
%
%
and since $u \in C([-T, T], B_R)$ a priori, it follows
that for sufficiently small $\vp$ and $T = T(s, \lambda, \alpha, R, \vp)$ we must 
have $Lu \in L^\infty([-T, T], B_R)$. To improve the regularity of 
$Lu$, let $\{t_n\} \subset [-T, T]$ and suppose that $t_n \to t \in [-T, 
T]$. Then
%
%
\begin{equation}
  \label{befo-dom}
  \begin{split}
    & \lim_{n \to \infty} \|Lu(\cdot, t) - Lu(\cdot, t_n)\|_{H^s_x} 
    \\
    & = \lim_{n \to \infty} \| \left \{ i \lambda \int_0^{t - t_n} e^{i(t  
    - t') \p_x^2} \left [|u|^{\alpha -1}u( \cdot, t') \right ]
    \ dt'\right \} \|_{H^s_x}
    \\
    & \le |\lambda|
    \lim_{n \to \infty}  \left \{  \int_0^{t - t_n} \| e^{i(t  
    - t') \p_x^2} \left [ |u|^{\alpha -1}u( \cdot, t') \right ]  
    \|_{H^s_x} \ dt' \right \}
    \\
    & = |\lambda|
    \lim_{n \to \infty}  \left[  \int_\rr \chi_{[0, t-t_n]}
    \| u^{\alpha}(\cdot, t') \|_{H^s_x} \ dt' \right ]
  \end{split}
\end{equation}
%
%
where the last step follows from \eqref{unitary-op}. 
By the algebra property and our a priori
assumption $u \in C([-T, T], B_R )$, we have 
%
%
\begin{equation*}
  \begin{split}
    \chi_{[0,t-t_n]}	\|u^\alpha(\cdot, t')\|_{H^s_x}
    \lesssim \chi_{[0,t-t_n]} 	\|u(\cdot, t')\|_{H^s_x}^\alpha 
    \le \chi_{[0,T]}\|u\|^\alpha_{L^\infty_t(\rr) H^s_x} \in 
    L^1_t(\rr).
    \end{split}
\end{equation*}
%
%
%
%
Hence, applying dominated 
convergence to \eqref{befo-dom}, we may pass the limit inside the integral,  
giving
%
\begin{equation*}
  \begin{split}
    \lim_{n \to \infty} \|Lu(\cdot, t)  - Lu(\cdot, t_n)\|_{H^s_x} 
    \le |\lambda|
    \int_\rr \lim_{n \to \infty} \left [ \chi_{[0, t-t_n]}
    \| u^{\alpha}(\cdot, t') \|_{H^s_x} \ dt' \right ]
    = 0
  \end{split}
\end{equation*}
%
%
which implies $Lu \in C([-T, T], B_R)$. Furthermore, for 
$u, v \in C([-T, T], B_R)$, we have
%
%
\begin{equation*}
  \begin{split}
    & \|Lu-Lv\|_{L^\infty_t[-T, T] H^s_x}
    \\
    & = \|i \lambda \int_0^t e^{i(t 
    -t')\p_x^2} (|u|^{\alpha - 1}u -|v|^{\alpha -1} v ) \ dt'
    \|_{L^\infty_t[-T, T] H^s_x}
    \\
    & \le |\lambda| \int_0^T \||u^{\alpha-1 }| u - | v^{\alpha - 1}| v
    \|_{L^\infty_t[-T, T] H^s_x} \ dt'
    \\
    & = |\lambda| T \cdot \|(u-v)(|u^{\alpha -1}| + |v^{\alpha -1}|) 
    + |u^{\alpha -1}|v 
    + u |v^{\alpha -1}| \|_{L^\infty_t[-T, T] H^s_x}
  \end{split}
\end{equation*}
%
%
which by the triangle inequality and algebra property simplifies to
%
%
\begin{equation}
  \label{L-contract}
  \begin{split}
    & \|Lu-Lv\|_{L^\infty_t[-T, T] H^s_x}
    \\
    & \le  T c_s |\lambda| \cdot \big [ \|u-v\|_{L^\infty_t[-T, T]
    H^s_x}(\|u\|^{\alpha -1}_{L^\infty_t[-T, T] H^s_x} +
    \|v\|^{\alpha -1}_{L^\infty_t[-T, T] H^s_x})
    \\
    & + \|u\|^{\alpha-1}_
    {L^\infty_t[-T, T] H^s_x} \|v\|_{L^\infty_t[-T, T] H^s_x}
    + \|u\|_
    {L^\infty_t[-T, T] H^s_x} \|v\|^{\alpha -1}_
    {L^\infty_t[-T, T] H^s_x} \big ]
    \\
    & \le T c_s |\lambda| \cdot \left[  2R^{\alpha -1} 
    \|u -v\|_{L^\infty_t[-T, T] H^s_x} + 2R^{\alpha} \right]
    \\
    & \le T c' \|u -v \|_{L^\infty_t[-T, T] 
    H^s_x}
  \end{split}
\end{equation}
%
%
where $c' = c'(s, \lambda, \alpha, R)$ is a constant.
Since it was established earlier that $T = T(s, \lambda, \alpha, R, \vp)$,  
we conclude that for $$T < 1/c,  \qquad c = c(s, \lambda, \alpha, R, 
\vp),$$ $L$ is a 
contraction on $C([-T, T], B_R)$. 
\end{proof}

