\chapter{Well Posedness for the KNLS}
%
%
\section{Introduction}
We consider the mixed nonlinear Schr{\"o}dinger and Korteweg-de Vries (KNLS) initial value problem (ivp)
%
%
\begin{gather}
	\label{lKNLS-eq}
	i\p_t u + \p_x^{m} u + \lambda u \p_x u = 0,
	\\
	\label{lKNLS-init-data}
	u(x,0) = u_0(x), \quad x \in \ci, t \in \rr
\end{gather}
%
%
where $m \ge 4$ is an even integer and $\lambda \in \{-1, 1\}$.
%
%
\begin{definition}
	We say that the KNLS ivp \eqref{lKNLS-eq}-\eqref{lKNLS-init-data} is
	\emph{locally well posed} in
	$X$ if 
	\begin{enumerate}
		\item For every $\vp(x) \in
	B_R$ there exists $T>0$ depending on $R$ and a unique function
	\\
	$u \in C([-T, T],
	X)$ satisfying \eqref{lKNLS-eq} for all $t \in [-T, T]$. 
\item The flow map $u_0 \mapsto u(t)$ is locally uniformly continuous. That is, if $u_0
	\in B_R$, $\{u_{0,n}\} \subset B_R$, and 
	$\|u_0 - u_{0, n} \|_{H^{s}(\ci)} \to 0$, then there exists $T >0$ depending
	on $R$ such that $\|u(\cdot, t) - u_{n}(\cdot,t) \|_{X} \to
	0$ for $t \in [-T, T]$. 
	\end{enumerate}
	Otherwise, we say that the KNLS ivp is \emph{ill-posed}.
\end{definition}
%
%
We are now prepared to state the following result.

%%%%%%%%%%%%%%%%%%%%%%%%%%%%%%%%%%%%%%%%%%%%%%%%%%%%%
%
%
%				 Well Posedness Theorem
%
%
%%%%%%%%%%%%%%%%%%%%%%%%%%%%%%%%%%%%%%%%%%%%%%%%%%%%%
%
%
\begin{theorem}
	\label{lthm:main}
	The KNLS is well-posed in $\dot{H}^s(\ci)$ for $s \ge (2-m)/4$.  
\end{theorem}
%
%
%%%%%%%%%%%%%%%%%%%%%%%%%%%%%%%%%%%%%%%%%%%%%%%%%%%%%
%
%
%				Outline
%
%
%%%%%%%%%%%%%%%%%%%%%%%%%%%%%%%%%%%%%%%%%%%%%%%%%%%%%
%
%
\section{Outline of the Proof of Main Theorem}
%
%
%
%
%
We first derive a weak formulation of the KNLS ivp. 
Let $\ci = [0, 2 \pi]$, and use
the following notation for the Fourier transform
%
%
%
%
\begin{equation}
	\label{lfour-trans-pde}
	\begin{split}
		\widehat{f}(n) = \int_{\ci} e^{-ix n} f(x) \, dx.
	\end{split}
\end{equation}
Let $w(x,t) = u \p_x u$. Applying 
the Fourier transform to the KNLS ivp in the space variable we obtain 
%
%
\begin{gather*}
	\p_t \widehat{u}(n, t) = (-1)^{m/2}i n^m \widehat{u}(n, t) + \lambda i  
	\widehat{w} (n, t),
	\\
	\widehat{u} (n,0) = \widehat{\vp}(n)
\end{gather*}
%
%
which is a globally well-defined relation in $t$ 
and $n$. Note that by time reversal, we similarly have 
\begin{gather*}
	-\p_t \widehat{u}(n, -t) = (-1)^{\frac{m}{2}}i n^m \widehat{u}(n, -t) + \lambda i  
	\widehat{w} (n, -t),
\end{gather*}
or
\begin{gather*}
	\p_t \widehat{u}(n, -t) = (-1)^{\frac{m+2}{2}}i n^m \widehat{u}(n, -t) - \lambda i  
	\widehat{w} (n, -t).
\end{gather*}
Since the sign of $\lambda$ plays no role in the proof of local well-posedness,
we now assume $m/2$ to be odd without loss of generality. 
Multiplying \eqref{lfour-trans-pde} by the integrating factor $e^{itn^m}$ then yields
%%
%%
\begin{equation*}
	\begin{split}
		\left[ e^{ it n^m} \widehat{u}(n) \right]_t = i
		 e^{ it n^m} \widehat{w} (n, t).	
	\end{split}
\end{equation*}
%
%
Integrating from $0$ to $t$, we obtain
%
%
\begin{equation*}
	\begin{split}
		\wh{u}(n, t) = \wh{\vp}(n) e^{- it n^m} + i  
		\int_0^t e^{ i(t' - t) n^m} \wh{w}(n, t') \ 
		dt'.
	\end{split}
\end{equation*}
%
%
Therefore, by Fourier inversion 
%
%
\begin{equation}
	\label{lKNLS-integral-form}
	\begin{split}
		u(x,t) & = \sum_{n \in \zz} \wh{\vp}(n) e^{i\left( xn - t n^m 
		\right)} 
		\\
		& + i \sum_{n \in \zz} \int_0^t e^{i\left[ xn + \left( t' - t 
		\right) n^m \right]} \wh{w}(n, t') \ dt'.
	\end{split}
\end{equation}
%
%
Then it is immediate that \eqref{lKNLS-integral-form} is a weaker 
restatement of the Cauchy-problem \eqref{lKNLS-eq}-\eqref{lKNLS-init-data}, 
since by construction any classical solution of the KNLS 
ivp is a solution to \eqref{lKNLS-integral-form}. 
\\
\\
%
%
We now derive an integral 
equation global in $t$ and equivalent to \eqref{lKNLS-integral-form} for $t 
\in [-T, T]$. Let $\psi(t)$ be a cutoff function symmetric about the 
origin such that $\psi(t) = 1$ for $|t| \le T$ and $\text{supp} \, \psi 
= [-2T, 2T ]$. Multiplying the right hand side of expression
$\eqref{lKNLS-integral-form}$ by $\psi(t)$, we obtain
%
%
\begin{equation}
	\begin{split}
		\label{lcutoff-int-eq}
		u(x, t)
		& = \frac{1}{2 \pi} \psi(t) \sum_{n \in \zz} e^{i(xn - t n^{m })} \widehat{\vp}(n) 
		\\
		& + \frac{i }{2 \pi} \psi(t) \int_0^t \sum_{n \in \zz} 
		e^{i\left[ xn + (t - t')n^m \right]} \wh{w}(n, t') \ dt'.
	\end{split}
\end{equation}
%
%
Noting that $e^{i\left( xn + tn^{m } \right)}$ 
does not depend on $t'$, we may rewrite
%
%
\begin{equation}
	\label{lpre-prim-int-form}
	\begin{split}
		& \frac{i }{2 \pi} \psi(t) \int_0^t \sum_{n \in \zz} 
		e^{i\left[ xn + (t - t') n^m \right]} \wh{w}(n, t') \ dt'
		\\
		& = \frac{i}{2 \pi} \psi(t) \sum_{n \in \zz} e^{i\left( xn + t 
		 n^{m } 
		\right)} \int_0^t e^{- it'n^{m }} \wh{w}(n, t') \ dt'.
	\end{split}
\end{equation}
%%
%%
We remark that this is a \emph{global} relation in $t$. Therefore, by Fourier 
inversion
%
%
%
%
%
%
%
\begin{equation*}
	\begin{split}
		\text{rhs of} \; \eqref{lpre-prim-int-form}
		& = \frac{i}{4 \pi^2} \psi (t) \sum_{n \in \zz} e^{i\left( xn + t 
		 n^m
		\right)} \int_0^t \int_\rr e^{it'\left( \tau - n^m \right) }
		\wh{w}(n, \tau) d \tau dt'
		\\
		& = \frac{i}{4 \pi^2} \psi(t) \sum_{n \in \zz} \int_\rr 
		e^{i\left( xn + tn^m \right)} \frac{e^{it\left( \tau - n^m 
		\right)}-1}{\tau - n^m} \wh{w}(n, \tau) d \tau
	\end{split}
\end{equation*}
%
%
where the last step follows from integration. Substituting
into \eqref{lcutoff-int-eq} we obtain
%
%
\begin{equation}
	\begin{split}
		\label{lcutoff-int-eq-2}
		u(x, t)
		& = \frac{1}{2 \pi} \psi(t) \sum_{n \in \zz} e^{i(xn - tn^{m })} \widehat{\vp}(n) 
		\\
		& + \frac{1}{4 \pi^2} \psi(t) \sum_{n \in \zz} \int_\rr
		e^{i(xn + t n^m)} \frac{e^{it(\tau - n^m)}- 1}{\tau - n^m} 
		\wh{w}(n, \tau) \ d \tau.
	\end{split}
\end{equation}
%
%
%
%
%
Next, we localize near the singular curve $\tau =  n^m$.  Multiplying the
summand of the second term of \eqref{lcutoff-int-eq-2} by $1 + \psi(\tau -
n^m) - \psi(\tau -
n^m) $ and
rearranging terms, we have
%
%
\begin{equation*}
	\begin{split}
		 u(x, t)
		& = \frac{1}{2 \pi} \psi(t) \sum_{n \in \zz} e^{i(xn + t n^{m 
		})} \widehat{\vp}(n) 
		\\
		& + \frac{1}{4 \pi^2} \psi(t) \sum_{n \in \zz} \int_\rr e^{ixn}  
		e^{it \tau} \frac{ 1 - \psi(\tau - n^m) 
		}{\tau - n^m} \wh{w}(n, \tau) \ d \tau
		\\
		& - \frac{1}{4 \pi^2} \psi(t) \sum_{n \in \zz} \int _\rr e^{i(xn + 
		t n^m)}
		 \frac{1- \psi(\tau - n^m)}{\tau - n^m} \wh{w}(n, \tau) \ d \tau
		\\
		& + \frac{1}{4 \pi^2} \psi(t) \sum_{n \in \zz} \int_\rr
		e^{i(xn + t n^m)}
		\frac{\psi(\tau - n^m)\left[ e^{it(\tau - n^m)}-1 
		\right]}{\tau - n^m} \wh{w}(n, \tau) \ d \tau
	\end{split}
\end{equation*}
%
%
which by a power series expansion of $[e^{it(\tau - n^m)}-1]$ simplifies  
to
%
%
\begin{align}
	\label{lmain-int-expression-0}
	& u(x, t) 
		\\
		\label{lmain-int-expression-1}
		& = \frac{1}{2 \pi} \psi(t) \sum_{n \in \zz} e^{i(xn + tn^{m 
		})} \widehat{\vp}(n) 
		\\
		\label{lmain-int-expression-2}
		& + \frac{1}{4 \pi^2} \psi(t) \sum_{n\in \zz} \int_\rr e^{ixn}  
		e^{it \tau} \frac{ 1 - \psi(\tau -  n^m) 
		}{\tau -  n^m} \wh{w}(n, \tau) \ d \tau
		\\
		\label{lmain-int-expression-3}
		& - \frac{1}{4 \pi^2} \psi(t) \sum_{n\in \zz} \int_\rr e^{i(xn + 
		t n^m)}
		 \frac{1- \psi(\tau -  n^m)}{\tau -  n^m} \wh{w}(n, \tau) \ d \tau
		\\
		\label{lmain-int-expression-4}
		& + \frac{1}{4 \pi^2} \psi(t) \sum_{k \ge 1} \frac{i^k t^k}{k!}
		\sum_{n \in \zz} \int_\rr e^{i(xn + t n^m )}
		\psi(\tau -  n^m) (\tau -  n^m)^{k-1} \wh{w}(n, \tau)  
		\\
		& \doteq T(u) \notag
\end{align}
%
%
where $T = T_{\vp}$. We now introduce the following spaces. 

\begin{definition}
	Denote $\dot{Y}^s$ to be the space of all
	functions $u$ on $\ci \times \rr$ with
	bounded norm
\begin{equation}
	\label{lY-s-norm}
	\begin{split}
		\|u\|_{\dot{Y}^s} = \|u\|_{\dot{X}^s} + \|n^s \wh{ u}\|_{ \dot{\ell}^2_n L^1_\tau }
	\end{split}
\end{equation}
%
%
%
%
where
%
\begin{equation}
	\label{lX^s-norm}
	\begin{split}
		& \|u\|_{\dot{X}^s}
		= \left ( \sum_{n\in \zz} |n|^{2s} \int_\rr \left ( 1 + | 
		\tau - n^{m } \right ) | \wh{u} ( n, \tau ) |^2
		\right )^{1/2}
	\end{split}
\end{equation}
and
%
%
\begin{equation}
	\label{lE-norm}
	\|n^s \wh{u}\|_{ \dot{\ell}^2_n L^1_\tau } = \left[ \sum_{n \in \zzdot}| n |^{2s} \left(
	\int_{\rr}| \wh{u}(n, \tau) |d \tau \right)^{2} \right]^{1/2}.
\end{equation}
%
%
%
%
\end{definition}
The $\dot{Y}^s$ spaces have the following important property, whose proof
is provided in the appendix.
\begin{lemma}
	\label{llem:cutoff-loc-soln}
	Let $\psi(t)$ be a smooth cutoff function with $\psi(t) =1$ for $t \in [-T, T]$. If
	$\psi(t)u(x,t) \in \dot{Y}^s$, then $u \in C([-T, T], \dot{H}^s(\ci))$.
\end{lemma}
%
%
We will 
show that for initial data $\vp \in \dot{H}^s(\ci)$, $T$ is a contraction on $B_M 
\subset \dot{Y}^s$, where $B_M$ is the ball centered at the origin of radius $M = 
M_{\vp}> 0$, by estimating the $\dot{Y}^s$
norm of \eqref{lmain-int-expression-1}-\eqref{lmain-int-expression-4}. The 
Picard fixed point theorem will
then yield a unique solution to
\eqref{lmain-int-expression-0}-\eqref{lmain-int-expression-4}. An application of
\cref{llem:cutoff-loc-soln} will then imply the existence of a unique, local
solution $u \in C([-T, T], \dot{H}^s(\ci))$ to the KNLS ivp which coincides with the solution to
\eqref{lmain-int-expression-0}-\eqref{lmain-int-expression-4} on the interval $[-T, T]$. Local Lipschitz continuity of the flow map will follow
from estimates used to establish the contraction mapping. %
%
%%%%%%%%%%%%%%%%%%%%%%%%%%%%%%%%%%%%%%%%%%%%%%%%%%%%%
%
%
%			Proof of Theorem	
%
%
%%%%%%%%%%%%%%%%%%%%%%%%%%%%%%%%%%%%%%%%%%%%%%%%%%%%%
%
%
\section{Proof of Main Theorem}
%
%
%
%%%%%%%%%%%%%%%%%%%%%%%%%%%%%%%%%%%%%%%%%%%%%%%%%%%%%
%
%
%		Estimation of Integral Equality Part 1		
%
%
%%%%%%%%%%%%%%%%%%%%%%%%%%%%%%%%%%%%%%%%%%%%%%%%%%%%%
%
%
%
%
\subsection{Estimate for \eqref{lmain-int-expression-1}.}
%
%
Letting $f(x,t) = \psi(t) \sum_{n \in \zz} e^{i(xn + tn^{m})} 
\wh{\vp}(n)$, we have $\wh{f}(n,t) = \psi(t) \wh{\vp}(n) e^{itn^{m}}$,
from which we obtain
%
%
\begin{equation}
	\label{lfourier-trans-calc}
	\begin{split}
		\wh{f}(n, \tau)
		& = \wh{\vp}(n) \int_\rr e^{-it( \tau - n^{m})} 
		\psi(t) \ d t
		= \wh{\psi}(\tau - n^{m}) \wh{\vp}(n).
	\end{split}
\end{equation}
%
%
%
%
%
%
Since $\wh{\psi}(\xi)$ is Schwartz for $|\xi| \ge T$, we see that 
%
%
\begin{equation}
	\begin{split}
	\label{lmain-int1-est}
		\|\eqref{lmain-int-expression-1}\|_{\dot{Y}^s}
		& = \left (  \sum_{n\in \zz} |n|^s \int_\rr \left( 1 + | \tau - n^{m} 
		| \right )
		| \wh{\psi}(\tau - n^{m}) \wh{\vp}(n) |^2 d \tau \right)^{1/2} 
		\\
		& + \left[ \sum_{n \in \zz }\left( 1 + | n | \right)^{2s} \left( \int_{\rr} |
		\wh{\psi}(\tau - n^{m})\wh{\vp}(n) | d \tau
		\right)^{2} \right]^{1/2}
		\\
		& \le c_{\psi}
		\|\vp\|_{\dot{H}^s(\ci)}.
	\end{split}
\end{equation}
%
%
%
%
\subsection{Estimate for \eqref{lmain-int-expression-2}.}
We now need the following lemma, whose proof is provided in the appendix.
%
%
%%%%%%%%%%%%%%%%%%%%%%%%%%%%%%%%%%%%%%%%%%%%%%%%%%%%%
%
%
%			Schwartz Multiplier	
%
%
%%%%%%%%%%%%%%%%%%%%%%%%%%%%%%%%%%%%%%%%%%%%%%%%%%%%%
%
%
\begin{lemma}
\label{llem:schwartz-mult}
	For $\psi \in S(\rr)$,
%
%
\begin{equation}
	\label{lschwartz-mult}
	\begin{split}
		\|\psi f \|_{\dot{X}^s} \le c_{\psi} \|f \|_{\dot{X}^s}.
	\end{split}
\end{equation}
%
%
\end{lemma}
%
%
Hence,
%
%
\begin{equation}
	\label{lmain-int2-est-X-s-part}
	\begin{split}
		\|\eqref{lmain-int-expression-2}\|_{\dot{X}^s} 
		& \lesssim 
		\left( \| \sum_{n \in \zz} e^{ixn} \int_\rr 
		e^{it \tau} \frac{ 1 - \psi (\tau - n^{m} ) 
		}{\tau - n^{m}} \wh{w}(n, \tau) \ 
		d \tau\|_{\dot{X}^s} \right)^{1/2}
		\\
		& =  \left( \sum_{n \in \zz} |n|^{2s} \int_\rr
		(1 + |\tau - n^{m}|) \left | \frac{1 - \psi(\tau - n^{2 
		})}{\tau - n^{m}} 
		\wh{w}(n, \tau) \right |^2 \ d 
		\tau \right)^{1/2}
		\\
		& \le \left( \sum_{n \in \zz} |n|^{2s} \int_{| \tau - n^{m }| \ge 1}
		(1 + |\tau - n^{m}|) \frac{|\wh{w}(n, \tau)|^2 }{|\tau - n^{m }|^2} 
		\ d 
		\tau \right)^{1/2}
		\\
		& \lesssim  \left( \sum_{n \in 
		\zz} |n|^{2s} \int_\rr
		\frac{|\wh{w}(n, \tau) |^2}{1+ |\tau - 
		n^{m}|} 
		 \ d \tau 
		\right)^{1/2}
		\\
		& \lesssim  \|u\|_{\dot{X}^s}^3
	\end{split}
\end{equation}
%
%
where the last two steps follow from the inequality 
%
\begin{equation}
	\label{lone-plus-ineq}
	\begin{split}
		\frac{1}{|\tau - n^{m}| } \le \frac{2}{1 + |\tau - n^{m}| }, 
		\qquad |\tau - n^{m}| \ge 1
	\end{split}
\end{equation}
%
%
and the following bilinear estimate, whose proof we leave for later.
%
%%%%%%%%%%%%%%%%%%%%%%%%%%%%%%%%%%%%%%%%%%%%%%%%%%%%%
%
%
%				 Bilinear Estimates
%
%
%%%%%%%%%%%%%%%%%%%%%%%%%%%%%%%%%%%%%%%%%%%%%%%%%%%%%
%
%
\begin{proposition}
	\label{lprop:prim-bilin-est}
	For any $b \ge 1/2$, $s \ge \frac{b(3-m)}{2}$ we have
	\begin{equation}
		\left( \sum_{n \in \dot{\zz}} |n|^{2s} \int_\rr
		\frac{|\wh{w_{fg}}(n, \tau) |^2}{\left (1+ |\tau - 
		n^{m}| \right )^{2b}} 
		 \ d \tau 
		\right)^{1/2}
		\lesssim \|f\|_{\dot{X}^s} \|g\|_{\dot{X}^s}
	\end{equation}
	where $w_{fg}(x,t)$ = $\p_x(fg)(x,t)$.
\end{proposition}
%\nocite{*}
%\bibliography{/Users/davidkarapetyan/Documents/math/}
Furthermore,
%
%
%
%
\begin{equation}
	\label{lmain-int-expression-2-Y-s-part}
	\begin{split}
		\|\wh{\eqref{lmain-int-expression-2}} \|_{\dot{\ell}^2_n L^1_\tau}
		& \lesssim \left( \| \sum_{n \in \zz} e^{ixn} \int_\rr 
		e^{it \tau} \frac{ 1 - \psi (\tau - n^{m} ) 
		}{\tau - n^{m}} \wh{w}(n, \tau) \ 
		d \tau\|_{\dot{\ell}^2_n L^1_\tau} \right)^{1/2}
		\\
		& = \left[ \sum_{n \in \zz}|n|^{2s} \left(
		\int_{\rr}\frac{1 - \psi(\tau - n^{m})}{\tau - n^{m}} \wh{w}(n, \tau) d
		\tau \right)^{2} \right]^{1/2}
		\\
		& \lesssim \|f\|_{X^s} \|g\|_{X^s}
	\end{split}
\end{equation}
%
%
where the last step follows from the following bilinear estimate.
%
%%%%%%%%%%%%%%%%%%%%%%%%%%%%%%%%%%%%%%%%%%%%%%%%%%%%%
%
%
%				Second trilinear Estimate 
%
%
%%%%%%%%%%%%%%%%%%%%%%%%%%%%%%%%%%%%%%%%%%%%%%%%%%%%%
%
%
\begin{proposition}
\label{lprop:bilinear-estimate2}
For any $s \ge \frac{3-m}{4}$ we have
%
%
\begin{equation}
	\label{ltrilinear-estimate2}
	\begin{split}
		\left( \sum_{n \in \zzdot} |n|^{2s}  \left ( \int_\rr 
		\frac{|\wh{w_{fg}}(n, \tau) |}{1 + | \tau - n^{m } |}
		 \ d\tau \right)^2  \right)^{1/2} \lesssim \|f\|_{\dot{X}^s} \|g\|_{\dot{X}^s}.
	\end{split}
\end{equation}
\end{proposition}
%
%
Combining \eqref{lmain-int2-est-X-s-part} and
\eqref{lmain-int-expression-2-Y-s-part}, we conclude that
%
%
%
%
\begin{equation}
	\label{lmain-int2-est}
	\begin{split}
		\|\eqref{lmain-int-expression-2}\|_{\dot{Y}^s} \le c_{\psi}\|f\|_{\dot{X}^s} \|g\|_{\dot{X}^s}.
	\end{split}
\end{equation}
%
%
\subsection{Estimate for \eqref{lmain-int-expression-3}.}
Letting $$f(x,t) = \psi(t) \sum_{n \in \zzdot} e^{i\left( xn + tn^{m} \right)} 
\int_\rr \frac{1 - \psi\left( \lambda - n^{m} \right)}{\lambda - n^{m}} 
\wh{w} \left( n, \lambda \right) \ d \lambda,$$ we have
%
%
\begin{equation*}
	\begin{split}
		& \wh{f^x}(n, t) = \psi(t) e^{itn^{m}} \int_\rr
		\frac{1 - \psi\left( \lambda - n^{m} \right)}{\lambda - n^{m}} 
		\wh{w}(n, \lambda) \ d \lambda
	\end{split}
\end{equation*}
and
\begin{equation*}
	\begin{split}
		 \wh{f}\left( n, \tau \right)
		 & = \int_\rr e^{-it\left( \tau - n^{m} 
		\right)} \psi(t) \int_\rr \frac{1 - \psi\left( 
		\lambda - n^{m} 
		\right)}{\lambda - n^{m}} \wh{w}(n, \lambda) \ d \lambda d \tau
		\\
		& = \wh{\psi}\left( \tau - n^{m} \right) \int_\rr 
		\frac{1 - \psi\left( 
		\lambda - n^{m} 
		\right)}{\lambda - n^{m}} \wh{w}(n, \lambda) \ d \lambda.
	\end{split}
\end{equation*}
Therefore,
%
%
\begin{equation*}
	\begin{split}
		& \| \eqref{lmain-int-expression-3} \|_{\dot{X}^s} 
		\\
		& = \left( \sum_{n \in \zzdot} |n|^{2s} \int_\rr \left( 1 + | \tau - n^{m
		} \right ) | | \wh{\psi}\left( \tau - n^{m } \right) |^2 \ d \tau
		\right.
		\\
		& \times \left . |
		\int_\rr \frac{1 - \psi\left( \lambda - n^{m } \right)}{\lambda -
		n^{m }} \wh{w}(n, \lambda) \ d \lambda |^2  \right)^{1/2}
		\\
		& \lesssim \left( \sum_{n \in \zzdot} |n|^{2s} | \int_\rr
		\frac{1 - \psi\left( \lambda - n^{m } \right)}{\lambda - n^{m }}
		\wh{w}(n, \lambda) \ d\lambda |^2 \right)^{1/2}
		\\
		& \le \left( \sum_{n \in \zzdot} |n|^{2s}  \left ( \int_\rr
		\frac{1 - \psi\left( \lambda - n^{m } \right)}{|\lambda - n^{m }|}
		|\wh{w}(n, \lambda) | \ d\lambda \right )^2 \right)^{1/2}
		\\
		& \le \left( \sum_{n \in \zzdot} |n|^{2s}  \left ( \int_{| \lambda - 
		n^{m } | \ge 1}
		\frac{|\wh{w}(n, \lambda) | }{|\lambda - n^{m }|}
		\ d\lambda \right )^2 \right)^{1/2}.
	\end{split}
\end{equation*}
%
%
Applying estimate \eqref{lone-plus-ineq} then gives
%
%%
\begin{equation}
	\label{lmain-int3-est-X-s-part}
	\begin{split}
		\| \eqref{lmain-int-expression-3} \|_{\dot{X}^s}
		& \lesssim \left( \sum_{n \in \zzdot} |n|^{2s}  \left ( \int_\rr
		\frac{|\wh{w}(n, \lambda)| }{1 + |\lambda - n^{m }|}
		 \ d\lambda \right )^2 \right)^{1/2}
		 \\
		& \lesssim \|u\|_{\dot{X}^s}^3
	\end{split}
\end{equation}
%
%%
where the last step follows from \cref{lprop:bilinear-estimate2}.
Furthermore, 
%
%
\begin{equation}
	\label{lmain-int-estimate-3-Y-s-part}
	\begin{split}
		\|\eqref{lmain-int-expression-3}\|_{\dot{\ell}^2_n L^1_\tau}
		& = \left[ \sum_{n \in \zzdot} |n|^{2s} \int_{\rr} |
		\wh{\psi}(\tau - n^{m}) |^{2} \left( \int_{\rr}\frac{1 - \psi(\lambda -
		n^{m})}{\lambda - n^{m}} \wh{w}(n, \lambda) d \lambda \right)^{2} d \tau
		\right]^{1/2}
		\\
		& \le c_{\psi} \left[ \sum_{n \in \zzdot} |n|^{2s} \left(
		\int_{\rr} \frac{1 - \psi(\lambda - n^{m})}{\lambda - n^{m}}
		\wh{w}(n, \lambda) d \lambda
		\right)^{2}\right]^{1/2}
		\\
		& \le 2 c_{\psi} \left[ \sum_{n \in \zzdot} |n|^{2s} \left(
		\int_{\rr} \frac{\wh{w}(n, \lambda) }{1 + |\lambda - n^{m}|}
		d \lambda
		\right)^{2}\right]^{1/2}
		\\
		& \lesssim \|f\|_{\dot{X}^s} \|g\|_{\dot{X}^s} 
	\end{split}
\end{equation}
%
%
where the last two steps follow from \eqref{lone-plus-ineq} and
\cref{lprop:bilinear-estimate2}, respectively. Combining
\eqref{lmain-int3-est-X-s-part} and \eqref{lmain-int-estimate-3-Y-s-part}, we
conclude that
%
%
\begin{equation}
	\label{lmain-int3-est}
	\begin{split}
		\|\eqref{lmain-int-expression-3}\|_{\dot{Y}^s} 
		\lesssim \|f\|_{\dot{X}^s} \|g\|_{\dot{X}^s}.
	\end{split}
\end{equation}
%
%
%
\subsection{Estimate for \eqref{lmain-int-expression-4}.}
Note that
%
%
\begin{equation}
	\label{l1n}
	\begin{split}
		\eqref{lmain-int-expression-4} \simeq \sum_{k \ge 1}
		\frac{i^k}{k!}g_k(x,t)
	\end{split}
\end{equation}
%
%
where 
%
%
\begin{equation*}
	\begin{split}
		& g_k(x,t) = t^k \psi(t) \sum_{n \in \zzdot} e^{i\left( xn + tn^{m}
		\right)} h_k(n),
		\\
		& h_k(n) = \int_\rr \psi \left( \tau - n^{m } \right) \cdot \left(
		\tau - n^{m } \right)^{k -1} \wh{w}(n, \tau) \ d \tau.
	\end{split}
\end{equation*}
%
%
Hence
%
%
\begin{equation*}
	\begin{split}
		\wh{g_k^x}(n, t) = t^{k} \psi(t) e^{i t n^{m }} h_k(n)
	\end{split}
\end{equation*}
%
%
which gives
%
%
\begin{equation*}
	\begin{split}
		\wh{g_k}(n, \tau)
		& = h_k(n) \int_\rr e^{-it\left( \tau - n^{m } \right)}
		t^{k}\psi(t) \ dt
		\\
		& = h_k(n) \wh{t^{k}\psi(t)} \left( \tau - n^{m } \right).
	\end{split}
\end{equation*}
%
%
Applying this to \eqref{l1n}, we obtain
%
%
\begin{equation}
	\label{l2n}
	\begin{split}
		\|\eqref{lmain-int-expression-4}\|_{\dot{X}^s} 
		& \simeq \left( \sum_{n \in \zzdot} |n|^{2s} \int_\rr \left( 1 + | \tau -
		n^{m }
		|
		\right) | \wh{\sum_{k \ge 1} \frac{i^k}{k!}g_k(x,t)} |^2 \ d \tau
		\right)^{1/2}
		\\
		& \le \sum_{k \ge 1} \frac{1}{k!}\left( \sum_{n \in \zzdot} |n|^{2s}
		\int_\rr \left( 1 + | \tau - n^{m } | \right) | \wh{g_k}(n, \tau) |^2 \
		d \tau \right)^{1/2}
		\\
		& = \sum_{k \ge 1} \frac{1}{k!} \left( \sum_{n \in \zzdot} |n|^{2s}
		\int_\rr \left( 1 + | \tau - n^{m } | \right) | h_k(n) \wh{t^k
		\psi(t)} \left( \tau - n^{m } \right) |^2 \ d \tau \right)^{1/2}
		\\
		& = \sum_{k \ge 1} \frac{1}{k!} \left( \sum_{n \in \zzdot} |n|^{2s} |
		h_k(n) |^2 \int_\rr \left( 1 + | \tau - n^{m } | \right) | \wh{t^k
		\psi(t)} \left( \tau - n^{m } \right) |^2 \ d \tau \right)^{1/2}.
	\end{split}
\end{equation}
%
%
Notice that for fixed $n$, the change of variable $\tau - n^{m } \to \tau'$
gives
%
%
\begin{equation}
	\label{l3n}
	\begin{split}
		\int_\rr \left( 1 + | \tau - n^{m } | \right) | \wh{t^{k}
		\psi(t)}\left( \tau - n^{m } \right) |^2 \ d \tau
		& = \int_\rr \left( 1 + |\tau'| \right) | \wh{t^k \psi(t)}(\tau') |^2 \
		d \tau'
		\\
		& \le \int_\rr \left( 1 + |\tau'| \right)^2 | \wh{t^k \psi(t)}(\tau')
		|^2 \ d \tau'
		\\
		& \lesssim \int_\rr \left( 1 + | \tau' |^2 \right) | \wh{t^{k}
		\psi(t)}(\tau') |^2 \ d \tau'
		\\
		& = \|t^k \psi(t) \|_{H^1(\rr)}^2.
	\end{split}
\end{equation}
%
%
But
%
%
\begin{equation}
	\label{l4n}
	\begin{split}
		\|t^k \psi(t) \|_{H^1(\rr)}^2
		& = \left( \|t^k \psi(t)\|_{L^2(\rr)} + \|\p_t \left( t^k \psi(t)
		\right)\|_{L^2(\rr)} \right)^2
		\\
		& \lesssim \|t^{k}\psi(t) \|_{L^2(\rr)}^2 + \|\p_t \left (t^{k}
		\psi(t) \right )\|_{L^2(\rr)}^2
		\\
		& \le \|t^k \psi(t) \|_{L^2(\rr)}^2 + \|t^k \p_t \psi(t)
		\|_{L^2(\rr)}^2 + \|k t^{k -1} \psi(t) \|_{L^2(\rr)}^2
		\\
		& = c_{\psi} + c_{\psi}' + k^2 c_{\psi}''
		\\
		& \lesssim k^2.
	\end{split}
\end{equation}
%
%
Hence, applying \eqref{l3n} and \eqref{l4n} to \eqref{l2n}, we obtain
%
%%
\begin{equation}
	\label{l5n}
	\begin{split}
		\|\eqref{lmain-int-expression-4} \|_{\dot{X}^s}
		& \lesssim
		\sum_{k \ge 1} \frac{k}{k!} \left( \sum_{n \in \zzdot} |n|^{2s} | h_k(n) |^2 
		\right)^{1/2}
		\\
		& \le \sum_{k \ge 1} \frac{k}{k!}
		\cdot \sup_{k \ge 1} \left( \sum_{n \in \zzdot} |n|^{2s} | 
		h_k(n) |^2 \right)^{1/2}
		\\
		& = \sum_{k \ge 1} \frac{k}{k!} \cdot \sup_{k \ge 1} 
		\left( \sum_{n \in \zzdot} |n|^{2s} \int_\rr 
		\psi\left( \tau - n^{m } \right) \cdot \left( \tau - n^{m } 
		\right)^{k -1} \wh{w}(n, \tau) \ d \tau \right)^{1/2}.
	\end{split}
\end{equation}
%
%%
Recall that $\text{supp} \, |\psi| \subset [0, T ]$. Pick $T \le 1$. 
Then $| \psi\left( \tau - n^{m } \right) \cdot \left( \tau - n^{m } \right)^{k 
-1} | \le \chi_{| \tau - n^{m } | \le 1}$ for all $k \ge 1$. Hence, \eqref{l5n} gives
%
%%
\begin{equation*}
	\begin{split}
		\|\eqref{lmain-int-expression-4} \|_{\dot{X}^s} 
		& \lesssim \sum_{k \ge 1} \frac{k}{k!} \cdot \left( \sum_{n \in \zzdot} | 
		\int_{| \tau - n^{m}  |\le 1} | \wh{w}(n, \tau) \ d \tau |^2 
		\right)^{1/2}
	\end{split}
\end{equation*}
%
%%
which by the inequality
%
%%
\begin{equation*}
	\begin{split}
		\frac{1 + | \tau - n^{m } |}{1 + | \tau  - n^{m } |} \le 
		\frac{2}{1 + | \tau - n^{m } |}, \qquad | \tau - n^{m }  | \le 1
	\end{split}
\end{equation*}
%
%%
implies
%
%%
\begin{equation}
\label{lmain-int4-est-X-s-part}
	\begin{split}
		\|\eqref{lmain-int-expression-4}\|_{\dot{X}^s}
		& \lesssim \left( \sum_{n \in \zzdot} | \int_{| \tau - n^{m}| \le 1 }
		\frac{\wh{w}(n, \tau)}{1 + | \tau - n^{m } |} \ d \tau |^2 
		\right)^{1/2}
		\\
		& \le \left( \sum_{n \in \zzdot} | \int_\rr
		\frac{\wh{w}(n, \tau)}{1 + | \tau - n^{m } |} \ d \tau |^2 
		\right)^{1/2} \\
		& \le \left( \sum_{n \in \zzdot} \left( \int_\rr 
		\frac{|\wh{w}(n, \tau)|}{1 + | \tau - n^{m } |}  \ d \tau  \right)^2
		\right)^{1/2} \\
		& \lesssim \|u\|_{\dot{X}^s}^3
	\end{split}
\end{equation}
%
%%
where the last step follows from \cref{lprop:bilinear-estimate2}. Similarly,
we have
%
%
\begin{equation}
\label{lmain-int4-est-Y-s-part}
	\begin{split}
		\|\eqref{lmain-int-expression-4}\|_{\dot{\ell}^2_n L^1_\tau}
		& \simeq \left[ \sum_{n \in
		\zzdot}|n|^{2s} \left( \int_{\rr} | \sum_{k \ge 1}
		\wh{\frac{i^{k}}{k!}g_{k}(x,t)(n, \tau)} |d \tau \right)^{2} \right]^{1/2}
		\\
		& \le \sum_{k \ge 1} \frac{1}{k!} \left[ \sum_{n \in \zzdot} (1 + | n
		|)^{2s} \left( \int_{\rr} | \wh{g}(n, \tau) | d \tau \right)^{2}
		\right]^{1/2}
		\\
		& = \sum_{k \ge 1} \frac{1}{k!} \left[ \sum_{n \in \zzdot} (1 + | n
		|)^{2s} | h_{k}(n) |^2 \left( \int_{\rr} | \wh{t^{k} \psi(t)}(\tau -
		n^{m}) |d \tau \right)^{2} \right]^{1/2}
		\\
		& = c_{\psi} \sum_{k \ge 1} \frac{1}{k!} \left[ \sum_{n \in \zzdot} (1 + | n
		|)^{2s} | h_{k}(n) |^2 \right]^{1/2}
		\\
		& \lesssim \|u\|_{\dot{X}^s}^{3}
	\end{split}
\end{equation}
%
%
where the last step follows from the computations starting from \eqref{l5n}
through \eqref{lmain-int4-est-X-s-part}.
Combining \eqref{lmain-int4-est-X-s-part} and \eqref{lmain-int4-est-Y-s-part}, we
have
%
%
\begin{equation}
\label{lmain-int4-est}
	\begin{split}
		\|\eqref{lmain-int-expression-4}\|_{\dot{Y}^s} \lesssim \|u\|_{\dot{X}^s}^{3}.
	\end{split}
\end{equation}
%
%
Collecting estimates \eqref{lmain-int1-est}, \eqref{lmain-int2-est}, 
\eqref{lmain-int3-est}, and \eqref{lmain-int4-est}, and recalling 
\eqref{lmain-int-expression-1}-\eqref{lmain-int-expression-4}, we see that
$$\|Tu\|_{\dot{Y}^s} \le c_\psi \left( \|\vp \|_{\dot{H}^s(\ci)} + \|u\|_{\dot{X}^s}^3 \right )$$ 
which by the inequality $\|u\|_{\dot{X}^s} \le \|u\|_{\dot{Y}^s}$ yields the following.
%%
%%%%%%%%%%%%%%%%%%%%%%%%%%%%%%%%%%%%%%%%%%%%%%%%%%%%%
%
%% Contraction Proposition
%				 
%%%%%%%%%%%%%%%%%%%%%%%%%%%%%%%%%%%%%%%%%%%%%%%%%%%%%%
%%
%%
%
\begin{proposition}
\label{lprop:contraction}
Let $s \ge \frac{3-m}{4}$. Then
%
%%
\begin{equation*}
	\begin{split}
		\|Tu\|_{\dot{Y}^s} \le c_\psi \left( \|\vp \|_{\dot{H}^s(\ci)} + \|u\|_{\dot{Y}^s}^3 
		\right).
	\end{split}
\end{equation*}
%
%%
\end{proposition}
We will now use \cref{lprop:contraction} to prove local well-posedness for the 
KNLS ivp. Let $c = c_{\psi}^{1/2}$. For given $\vp$, we may choose $\psi$ such
that 
%
%%
\begin{equation*}
	\begin{split}
		\|\vp\|_{\dot{H}^s(\ci)} \le \frac{15}{64c^3}.
	\end{split}
\end{equation*}
%
%%
Then if $\|u\|_{\dot{Y}^s} \le \frac{1}{4c}$, we have
%
%%
\begin{equation*}
	\begin{split}
		\|T u \|_{\dot{Y}^s} 
		& \le c^2 \left[ \frac{15}{64c^3} + \left( 
		\frac{1}{4c} \right)^3 \right]
		=  \frac{1}{4c}.
	\end{split}
\end{equation*}
%
%%
Hence, $T=T_{\vp}$ maps the ball $B\left( 0, \frac{1}{4c} \right) \subset \dot{Y}^s$ into 
itself. Next, note that
%
%%
\begin{equation*}
	\begin{split}
		Tu - Tv = \eqref{lmain-int-expression-2} + \eqref{lmain-int-expression-3} 
		+ \eqref{lmain-int-expression-4}
	\end{split}
\end{equation*}
%
%%
where now $w = u | u |^2 - v | v |^{2}$. Rewriting
%
%%
\begin{equation*}
	\begin{split}
		u | u |^{2} - v | v |^{2}
		& = | u |^2 \left( u -v \right) + v\left( | u 
		|^2 - | v |^2
		\right)
		\\
		& = u \bar u \left( u -v \right) + v u \bar u - v v \bar v
		\\
		& = u \bar u \left( u - v \right) + v \bar u\left( u - v \right) + v 
		\bar u v - v v \bar v
		\\
		& = u \bar u \left( u -v \right) + v \bar u\left( u - v \right) + v v 
		\left( \overline{u -v} \right)
	\end{split}
\end{equation*}
%
%%
the triangle inequality and linearity of the Fourier transform then give
%
%%
\begin{equation*}
	\begin{split}
		| \wh{w}(n, \tau) | = | \mathcal{F}(u | u |^2 - v| v |^2) |
		& \le | \wh{u \overline{u} \left (u -v \right )} | +
		| \wh{v \overline{u} (u -v)} | + |\wh{v v 
		(\overline{u-v})}|
		\\
		& \doteq | \wh{w_1} | + | \wh{w_2} | + | \wh{w_3} |
	\end{split}
\end{equation*}
%
%%
where
%
%%
\begin{equation*}
	\begin{split}
		w_1 = u \bar u \left( u -v \right), \qquad w_2 = v \bar u \left( u -v 
		\right), \qquad w_3 = v v \left( \overline{u -v} \right).
	\end{split}
\end{equation*}
%
%%
Hence, $Tu - Tv = \sum_{\ell=1, 2, 3} 
T_\ell(u, v)$, where
\begin{align}
	\label{lmain-int-exp-mod1}
	& \frac{1}{4 \pi^2} \psi(t) \sum_{n\in \zzdot} \int_\rr e^{ixn}  
		e^{it \tau} \frac{ 1 - \psi(\tau - n^{m}) 
		}{\tau - n^{m}} \wh{w_\ell}(n, \tau) \ d \tau
		\\
		\label{lmain-int-exp-mod2}
		- & \frac{1}{4 \pi^2} \psi(t) \sum_{n\in \zzdot} \int_\rr e^{i(xn + 
		tn^{m})}
		 \frac{1- \psi(\tau - n^{m})}{\tau - n^{m}} \wh{w_\ell}(n, \tau) \ d \tau
		\\
		\label{lmain-int-exp-mod3}
		+ & \frac{1}{4 \pi^2} \psi(t) \sum_{k \ge 1} \frac{i^k t^k}{k!}
		\sum_{n \in \zzdot} \int_\rr e^{i(xn + tn^{m} )}
		\psi(\tau - n^{m}) (\tau - n^{m})^{k-1} \wh{w_\ell}(n, \tau)  
		\\
		\doteq & T_\ell(u). \notag
\end{align}
Repeating the arguments used to estimate 
\eqref{lmain-int-expression-2}-\eqref{lmain-int-expression-4}, we obtain
%
%%
\begin{equation*}
	\begin{split}
		& \|T_1\|_{\dot{Y}^s} \le c_\psi \|u -v \|_{\dot{Y}^s} \|u\|^2_{\dot{Y}^s}
		\\
		& \|T_2\|_{\dot{Y}^s} \le c_\psi \|u -v \|_{\dot{Y}^s} \|u\|_{\dot{Y}^s} \|v\|_{\dot{Y}^s}
		\\
		& \|T_3\|_{\dot{Y}^s} \le c_\psi \|u -v \|_{\dot{Y}^s} \|v\|_{\dot{Y}^s}^2.
	\end{split}
\end{equation*}
%
%%
Therefore,
%
%%
\begin{equation}
	\label{l20a}
	\begin{split}
		\|Tu - Tv \|_{\dot{Y}^s} = & \| \sum T_\ell(u, v) \|_{\dot{Y}^s}
		\\
		& \le c_\psi \|u -v \|_{\dot{Y}^s} \left( \|u\|_{\dot{Y}^s}^2 + 
		\|u\|_{\dot{Y}^s} \|v\|_{\dot{Y}^s} + \|v\|_{\dot{Y}^s}^2 \right)
		\\
		& \le c_\psi \|u -v\|_{\dot{Y}^s} \left( \|u\|_{\dot{Y}^s} + \|v\|_{\dot{Y}^s} \right)^2
		\\
		& = c^2 \|u -v\|_{\dot{Y}^s} \left( \|u\|_{\dot{Y}^s} + \|v\|_{\dot{Y}^s} \right)^2.
	\end{split}
\end{equation}
%
%%
If $u, v \in B(0, \frac{1}{4c}) \subset \dot{Y}^s$, it follows that
%
%%
\begin{equation}
	\label{l21a}
	\begin{split}
		\|Tu - Tv \|_{\dot{Y}^s}
		& \le c^2 \|u -v \|_{\dot{Y}^s} \left( \frac{1}{4c} + 
		\frac{1}{4c} \right)^2
		\\
		& = \frac{1}{4} \|u -v \|_{\dot{Y}^s}. 
	\end{split}
\end{equation}
%
%%
We conclude that $T = T_{\vp}$ is a contraction on the ball $B(0, 
\frac{1}{4c}) \subset \dot{Y}^s$. A Picard iteration and application of 
\cref{llem:cutoff-loc-soln} then yield a unique, local
solution to the KNLS ivp \eqref{lKNLS-eq}-\eqref{lKNLS-init-data}.
\begin{definition}
	We say that the flow map $u_0 \mapsto u(t)$ is \emph{locally Lipschitz} in a Banach
	space $X$ if for
	$$u_0, v_0 \in B_R \doteq \{f: \|f\|_X < R\},$$ there exist $C, T>0$
	depending on $R$ such that $\|u(\cdot, t) - v(\cdot, t)
	\|_X \le C \|u_{0} - v_0 \|_{X}$ for $t \in [-T, T]$. We
	say the flow map is \emph{locally uniformly
	continuous} in $X$ if for
	$u_0, v_0 \in B_R$ there exists $T >0$ depending on $R$ such that for
	$t \in [-T, T]$, $\|u(\cdot, t) - v(\cdot, t) \|_{X} \to
	0$ if $\|u_0 - v_0 \|_{H^{s}(\ci)} \to 0$. 
\end{definition}
%
%
Clearly any locally Lipschitz flow map is locally uniformly continuous. 
Next, we shall establish local Lipschitz continuity in $\dot{Y}^s$ of the flow
map. Let $\vp_1, \vp_2 \subset \dot{H}^s(\ci)$ be given. Choose $\psi$ such that
$\vp_1, \vp_2 \subset B(0, \frac{15}{64c^{3}})$.  Then there exist $u_1, u_2 \in
\dot{Y}^s$ such that $u_1 = T_{\vp_1}$, $u_2 = T_{\vp_2}$, and so
%
%
\begin{equation*}
	\begin{split}
		T_{\vp_1}(u) - T_{\vp_2}(v) = \frac{1}{2\pi} \psi(t) \sum_{n \in
		\zzdot}e^{i\left( xn + tn^{m} \right)} \wh{\vp_1 - \vp_2}(n) + \sum_{\ell
		= 1,2,3} T_{\ell}(u).
	\end{split}
\end{equation*}
%
%
Using an argument similar to \eqref{lfourier-trans-calc}-\eqref{lmain-int1-est},
we obtain
%
%
\begin{equation*}
	\begin{split}
		\| \frac{1}{2\pi} \psi(t) \sum_{n \in
		\zzdot}e^{i\left( xn + tn^{m} \right)} \wh{\vp_1 - \vp_2}(n)\|_{\dot{Y}^s}
		\le c_\psi \|\vp_{1} - \vp_{2}\|_{\dot{Y}^s} 
	\end{split}
\end{equation*}
%
%
Hence, \eqref{l20a}-\eqref{l21a} gives
%
%
\begin{equation*}
	\begin{split}
		\sum_{\ell=1,2,3} T_{\ell}(u,v) \le \frac{1}{4}\|u-v\|_{\dot{Y}^s}.
	\end{split}
\end{equation*}
%
%
Hence,
%
%
\begin{equation*}
	\begin{split}
		\|u -v \|_{\dot{Y}^s} = \|T_{\vp_1}(u) - T_{\vp_2}(v) \|_{\dot{Y}^s} \le c_\psi
		\|\vp_{1} - \vp_{2} \|_{\dot{H}^s\left( \ci \right)}\| +
		\frac{1}{4} \|u -v \|_{\dot{Y}^s}
	\end{split}
\end{equation*}
%
%
which implies
%
%
\begin{equation*}
	\begin{split}
		\frac{3}{4} \|u-v\|_{\dot{Y}^s} \le c_\psi \|\vp_1 - \vp_2 \|_{\dot{H}^s(\ci)}
	\end{split}
\end{equation*}
%
%
or
%
%
\begin{equation*}
	\begin{split}
		\|u -v \|_{\dot{Y}^s} \le \frac{4}{3} c_\psi \|\vp_1 - \vp_2 \|_{\dot{H}^s(\ci)}.
	\end{split}
\end{equation*}
%
%
Applying \cref{llem:cutoff-loc-soln}, we then obtain
%
%
	 %
	 %
	 \begin{equation*}
		 \begin{split}
			\|u(\cdot, t) -v(\cdot, t) \|_{\dot{H}^s(\ci)} \le \frac{4}{3} c_\psi \|\vp_1 -
			\vp_2 \|_{\dot{H}^s(\ci)}, \qquad t \in [-T, T].
		 \end{split}
	 \end{equation*}
	 %
	 %
Hence, the flow map of the KNLS ivp is locally Lipschitz continuous in
$\dot{H}^s(\ci)$. This
concludes the proof of \cref{lthm:main}. \qquad \qedsymbol
%
%
%
%
\section{Proof of Bilinear Estimate}
Note first that $|\wh{w_{fg}}(n, \tau) |  = | n\wh{f} *  \wh{g} 
(n, \tau)|$. It follows that
%
%
\begin{equation}
	\label{lnon-lin-rep}
	\begin{split}
		| \wh{w_{fg}}(n, \tau)|
		& = | \sum_{n_1, n_2,}  \int n\wh{f}\left( n_1,  \tau_1 
\right) \wh{g}\left( n_2, \tau_2  
\right) d \tau_1 d \tau_2 |
\\
& = | \sum_{n_1 \neq 0, n_2 \neq 0}  \int n\wh{f}\left( n_1,  \tau_1 
\right) \wh{g}\left( n_2, \tau_2  
\right) d \tau_1 d \tau_2 | \qquad \text{(due to conservation of mass)}
\\
& \le \sum_{n_1 \neq 0, n_2 \neq 0}   \int | n | \times | \wh{f}\left( n_1, \tau_1 
\right) | \times  | \wh{g}\left( n_2, \tau_2 
\right) | \times  d \tau_1 d \tau_2  
\\
& = \sum_{n_1 \neq 0, n_2 \neq 0} \int | n | \times \frac{c_f\left( n_1, \tau_1 
\right)}{|n_1|^s \left( 1 + | \tau_1 - n_1^{m} | \right)^{b}}
\\
& \times \frac{c_{g}\left( n_2, \tau_2 \right)}{|n_2|^s\left( 1 + | \tau_2 -  n_2^{m }| 
\right)^{b}}
  \ d \tau_1 d \tau_2 
\end{split}
\end{equation}
%
%
where $n = n_1 + n_2$, $\tau = \tau_1 + \tau_2$, and 
%
%
\begin{equation*}
	\begin{split}
		c_\sigma(n, \tau) =
		\begin{cases}
			|n|^s \left( 1 + | \tau - n^{m } |  
		\right)^{b} | \wh{\sigma}\left( n, \tau \right) |, \qquad & n \neq 0
		\\
		0, \qquad & n = 0.
	\end{cases}
	\end{split}
\end{equation*}
%
%
For clarity of notation, let  $\sum_{n_1, n_2}$ denote $\sum_{n_1 \neq 0, n_2
\neq 0}$. From our work above, it follows that 
%
%
\begin{equation}
	\label{lconvo-est-starting-pnt}
	\begin{split}
		 & |n|^s \left( 1 + | \tau - n^{m } | \right)^{-b} | \wh{w_{fg}}\left( 
		n, \tau \right) |
		\\
		& \le \left( 1 + | \tau - n^{m } | \right)^{-b}
		\sum_{n_1, n_2} \int \frac{|n|^{s+1}}{|n_1|^s | n_2|^s} 
		\times \frac{c_f(n_1, \tau_1)}{\left( 1 + | \tau_1 - n_1^{m } | 
		\right)^{b}}
		\\
		& \times
		\frac{c_g(n_2, \tau_2)}{\left( 1 + | \tau_2 - n_2^{m } | 
		\right)^{b}}\ d \tau_1 d \tau_2.
	\end{split}
\end{equation}
%
%
Unlike the mNLS, we must use the smoothing properties of the
principal symbol $\tau - n^m$ regardless of the choice of $s$, since the quantity
%
%
\begin{equation}
	\label{lconvo-multiplier}
	\begin{split}
		\frac{|n|^{s+1}}{|n_1|^s |n_2|^s }
	\end{split}
\end{equation}
%
%
blows up in general, due to the presence of the extra power of $|n|$ coming from the derivative on
the nonlinearity. To utilize the smoothing effects of the principal symbol, we
will need the following two lemmas, whose
proofs are provided in the appendix.
%
%
%
\begin{lemma}
	\label{llem:number-theory1}
	Let $n=n_1 + n_2$ and suppose that $n, n_1, n_2\neq
	0$. Then for any integer $c \ge 0$
%
%
\begin{equation}
	\begin{split}
		\label{lnumber-theory1}
		| - n^{3} + n_1^3 + n_2^3| \ge 2^{-c/2} | n |^{\frac{2+c}{2}} | n_{1}
		|^{\frac{2-c}{2}}| n_2 |^{\frac{2-c}{2}}.
	\end{split}
\end{equation}
%
%
\end{lemma}
%
%
\begin{remark}
	In~\cite{Bourgain-Fourier-transfo}, Bourgain obtains the lower bound $n^2$ for
	the left hand side of \eqref{lnumber-theory}. This is too coarse an estimate,
	as we shall see.
\end{remark}
%
%
%
%
\begin{lemma}
	\label{llem:number-theory}
	Let $n=n_1 + n_2$ and suppose that $n, n_1, n_2\neq
	0$. Then for any integer  $m \ge 3$
%
%
\begin{equation}
	\begin{split}
		\label{lnumber-theory}
		| - n^{m} + n_1^{m} + n_2^{m }| \ge b_{m, c } 
		|n|^{c/2} |n_1|^{\frac{m-1-c}{2}} | n_2 |^{\frac{m-1-c}{2}}
		\end{split}
\end{equation}
%
%
where the constant $b_{m,c}$ depends only on $m$ and $c$. 
\end{lemma}
%
%
%
%\begin{remark}
%	The case $-1/2 \le s \le 0$ is delicate, and must be treated differently from
%	the case $s < -1/2$ in order to obtain the optimal well-posedness results.
%	This is the motivation for having two instead of one number theory lemma.
%\end{remark}
%
%
%
Since $$| \tau - n^{m} - \left( \tau_1 - n_1^{m} 
+ \tau_2 - n_2^{m }  \right ) | = | - n^{m} + n_1^{m} +
n_2^{m }|,$$ by \cref{llem:number-theory} and
the pigeonhole principle we must have one of the 
following.
%
%
\begin{align}
	\label{lpigeon-case-1}
	& |\tau - n^{m }| \ge \frac{c_m}{3} |n|^2 | n_1 |^{m-3} 		\\
		\label{lpigeon-case-2}
	    & | \tau_1 - n_1^{m } | \ge \frac{c_m}{3} |n|^2 | n_1 |^{m-3} ,  
		\\
		\label{lpigeon-case-3}
		& | \tau_2 - n_2^{m } | \ge
		\frac{c_m}{3} |n|^2 | n_1 |^{m-3}.  
\end{align}
%
%
By the symmetry of the convolution, it will be enough to consider only
\eqref{lpigeon-case-1} and \eqref{lpigeon-case-2}.
%
%
%
\subsection{Subcase \eqref{lpigeon-case-1}.} 
We shall need the following, whose proof is provided in the appendix.
%
%
\begin{lemma}
\label{llem:splitting}
	For $k \ge 0$ and $a, b \in {\zz}$, we have
%
%
\begin{equation}
	\label{lsplitting}
	\begin{split}
		\left ( 1 + |a +b | \right)^k \le 2^k \left(1 + | a | \right)^k \left(
		1 + | b | \right)^k.
	\end{split}
\end{equation}
%
%
\end{lemma}
%
Applying the lemma, we obtain
%
%
%
%%
\begin{equation*}
	\begin{split}
		\frac{|n|^{s+1}}{|n_1|^s | n_2|^s } 
		& =\frac{|n_1|^{-s} |n_2|^{-s}
		}{|n|^{-s}}
		\\
		& = \frac{| n_1|^{-s} | n - n_1 |^{-s} }{ | n|^{-s-1}} 
		\\
		& \lesssim \frac{|n_1|^{-s} \cancel{|n|^{-s} }|n_1 |
		^{-s}}{ |n|
		^{\cancel{-s}-1}}
		\\
		& = |n| | n_1 |^{-2s}
	\end{split}
\end{equation*}
%
%%
which in conjunction with \eqref{lpigeon-case-1} implies
%
%%
\begin{equation}
	\label{lconvo-deriv-bound}
	\begin{split}
		\frac{|n|^s}{|n_1|^s 
		| n_2|^s}
		\times
		\frac{1}{1 + | \tau -n^{m} |^{b}}
		& \lesssim  |n| |n_{1} |^{-2s} \times |n|^{-2b} |n_{1}|^{-b(m-3)} 
		\\
		& \lesssim 1, \qquad b\ge 1/2, \ s \ge \frac{b(3-m)}{2}.
	\end{split}  
\end{equation}
%
%
\begin{remark}
	Note that when $m=2$, we have $|-n^{m} + n_{1}^{m} + n_{2}^{m}| = 2| n_1 |
	|n_2|$. Applying the pigeonhole principle as before, we seek to bound 
	%
	%
	\begin{equation*}
		\begin{split}
			\frac{| n |^{s+1}}{| n_1 |^s |n_2|^s} \times \frac{1}{(| n_1 | |n_2
			|)^{1/2}} = \frac{| n |^{s+1}}{|n_{1}|^{s + 1/2}| n_2 |^{s+1/2}}.
		\end{split}
	\end{equation*}
	%
	%
	However, this quantity blows up (simply take $n=1$ and $n_2 \to \infty$).
	Hence, the KDV dispersive techniques fail for the case $m=2$. 
\end{remark}
%
Hence, recalling \eqref{lconvo-est-starting-pnt} and applying estimates 
\eqref{lpigeon-case-1} and \eqref{lconvo-deriv-bound}, we obtain
%
%
\begin{equation}
	\label{lnon-lin-rep-with-bound}
	\begin{split}
		& |n|^s \left( 1 + | \tau - n^{m } | \right)^{b} | 
		\wh{w_{fg}}(n, \tau) | 
		\\
		& \lesssim \sum_{n_1,n_2} \int \frac{c_f(n_1, \tau_1)}{\left( 1 + | 
		\tau_1 -  n_1^{m }| \right)^{b}}
		\times \frac{c_g\left( n_2, \tau_2\right)}{\left( 1 + | \tau_2 -n_2^{m }|
		\right)^{b}}
		\\
		& = \wh{C_f C_g}(n, \tau)
	\end{split}
\end{equation}
%
%
where
\begin{equation*}
	\begin{split}
		C_\sigma(x,t) =
		\left[ \frac{c_\sigma(n, \tau)}{\left( 1 + | \tau - n^{m } | 
		\right)^{b}}\right]^\vee .	
	\end{split}
\end{equation*}

%
%
Therefore, from \eqref{lnon-lin-rep-with-bound}, Plancherel, and generalized 
H\"{o}lder, we obtain
%
%
\begin{equation}
	\label{lgen-holder-bound}
	\begin{split}
		& \| |n|^s \left( 1 + | \tau - n^{m } | \right ) ^{b} \wh{w_{fg}}\left( 
		n, \tau \right) \|_{L^2(\ci \times \rr)}
		\\
		& \lesssim \|\wh{C_f C_g }\left( n, \tau \right) 
		\|_{L^2\left( \zzdot \times \rr \right)}
		\\
		& \simeq \|C_f C_g \|_{L^2\left( \ci \times \rr \right)}
		\\
		& \le \|C_f \|_{L^4(\ci \times \rr)} \|C_g \|_{L^4(\ci \times \rr)}.
	\end{split}
\end{equation}
%
We now need the following Fourier multiplier estimate, whose proof can be found
in \cite{Himonas-Misiolek-2001-A-priori-estimates-for-Schrodinger}.
%
\begin{lemma}
	\label{llem:four-mult-est-L4}
	Let $(x, t) \in \ci \times \rr $ and $(n, \tau) \in \zz \times \rr$ be 
	the dual variables. Let $v$ be a positive even integer. Then there is a 
	constant $c_v > 0$ such that
%
%
\begin{equation}
	\label{lfour-mult-est-L4}
	\begin{split}
		\|f\|_{L^4(\ci \times \rr)} \le c_v \|\left( 1 + | \tau - n^v | 
		\right)^\frac{v+1}{4v} \wh{f} \|_{L^2( \zz \times \rr)}
	\end{split}
\end{equation}
for every test function $f(x, t)$. 
%
%
%
%
\end{lemma}
From the lemma, we see that
%
%
\begin{equation}
	\label{lfour-mult-conseq}
	\begin{split}
		\|C_\sigma\|_{L^4(\ci \times \rr)} 
		& \lesssim \|(1 + | \tau - n^m |)^{\frac{m+1}{4m}} \wh{C_\sigma}
		\|_{L^2(\zz \times \rr)}
		\\
		& = \|c_{\sigma} \|_{L^2(\zz \times \rr)} \qquad (\text{Since} \ \frac{m+1}{4m} \le 1/2 )
		\\
		& = \|\sigma \|_{\dot{X}^s}. 
	\end{split}
\end{equation}
%
%
Applying this to \eqref{lgen-holder-bound} we
conclude that
\begin{equation*}
	\begin{split}
		\| |n|^s \left( 1 + | \tau - n^{m } | \right ) ^{-b} \wh{w_{fg}}\left( 
		n, \tau \right) \|_{L^2(\zzdot \times \rr)}
		& \lesssim \|f\|_{\dot{X}^s} \|g\|_{\dot{X}^s}.
	\end{split}
\end{equation*}
%
%
%
\subsection{Subcase \eqref{lpigeon-case-2}.}
Using a similar argument to that in Subcase \eqref{lpigeon-case-1}, we obtain
%
%
\begin{equation}
	\label{l1f}
	\begin{split}
		 & |n|^s \left( 1 + | \tau - n^{m } | \right)^{b} | \wh{w_{fg}}\left( 
		n, \tau \right) |
		\\
		& \lesssim \left( 1 + | \tau - n^{m } | \right)^{b}
		\sum_{n_1, n_2} \int
		c_f(n_1, \tau_1)
		\times
		\frac{c_g(n_2, \tau_2)}{\left( 1 + | \tau_2 - n_2^{m } | 
		\right)^{b}} 
		\\
		& = \left( 1 + | \tau - n^{m } | \right)^{b} \wh{\overset{\sim}{C_f} C_g}.
	\end{split}
\end{equation}
%
%%
where
%
%
\begin{equation*}
	\begin{split}
		\overset{\sim}{C_\sigma}(x,t) = \left[ c_\sigma(n, \tau) \right]^\vee.
	\end{split}
\end{equation*}
%
%
Hence
%
%%
\begin{equation}
	\label{l3f}
	\begin{split}
		& \| |n|^s \left( 1 + | \tau - n^{m } | \right)^{b} \wh{w_{fg}}(n, \tau) 
		\|_{L^2(\zzdot \times \rr)}
		\\
		& \lesssim \|\left( 1 + | \tau - n^{m} | \right)^{b} 
		\wh{\overset{\sim}{C_f} C_g } \|_{L^2(\zzdot \times \rr)}
		\\
		& =  \|\left( 1 + | \tau - n^{m} | \right)^{b} 
		\wh{\overset{\sim}{C_f} C_g } \|_{L^2(\zz \times \rr)}
		\\
		& \lesssim  \|\overset{\sim}{C_f} C_g  \|_{L^{4/3}(\ci \times \rr)}
	\end{split}
\end{equation}
%
%%
where the last step follows by dualizing \cref{llem:four-mult-est-L4}. More
precisely, we have the following.
\begin{corollary}
	\label{lcor:four-mult-est-L4}
	Let $(x, t) \in \ci \times \rr $ and $(n, \tau) \in \zzdot \times \rr$ be 
	the dual variables. Let $v$ be a positive even integer. Then there is a 
	constant $c_v > 0$ such that
%
%
\begin{equation}
	\label{lfour-mult-est-L4*}
	\begin{split}
		\| \left( 1 + | \tau - n^v | 
		\right)^{-\frac{v+1}{4v}}
		\wh{f}\|_{L^2(\zz \times \rr)} \le c_v \|f \|_{L^{4/3}( \ci \times \rr)}.
	\end{split}
\end{equation}
%
%
\end{corollary}
%
Applying H\"{o}lder's inequality to the right hand side of
\eqref{l3f}, we obtain the bound
%
%%
\begin{equation}
	\label{l4f}
	\begin{split}
		\|\overset{\sim}{C_f} \|_{L^2(\ci \times \rr)} \|C_g \|_{L^4\left( \ci 
		\times \rr 
		\right)}. 
	\end{split}
\end{equation}
%
%%
By Plancherel we have
%
%%
%
%%
\begin{equation}
	\label{l5f}
	\begin{split}
		\|\overset{\sim}{C_f} \|_{L^2(\ci \times \rr)}
		& \simeq \|c_f\|_{L^2(\zz \times \rr)}
		\\
		& = \|f \|_{\dot{X}^s}
	\end{split}
\end{equation}
%
%%
while \eqref{lfour-mult-conseq} gives
%
%
\begin{equation}
	\label{l6f}
	\begin{split}
		\|C_g \|_{L^4(\ci \times \rr)} \lesssim \|g\|_{\dot{X}^s}.
	\end{split}
\end{equation}
%
%
We conclude from \eqref{l3f}-\eqref{l6f} that
%
%
\begin{equation*}
	\begin{split}
		\| |n|^s \left( 1 + | \tau - n^{m } | \right)^{-b} \wh{w_{fg}}(n, \tau) 
		 \|_{L^2(\zzdot \times \rr)}
		 \lesssim \|f\|_{\dot{X}^s} \|g\|_{\dot{X}^s}
	\end{split}
\end{equation*}
%
%
which completes the proof.  \qquad \qedsymbol
%
%

\section{Proof of Second Bilinear Estimate}
Recall that for the mNLS, one obtains one trilinear estimate as a corollary of
another. Using this as motivation, let us see if we can obtain
\cref{lprop:bilinear-estimate2} as a corollary of
\cref{lprop:prim-bilin-est}. By
duality, it suffices to show that
%
%%
\begin{equation}
	\label{lduality-est}
	\begin{split}
		\sum_{n \in \zzdot}  |n|^{s}
		a_n \int_{\rr} \frac{|\wh{w_{fg}}(n, \tau)|}{1 
		+ | \tau - n^{m } |} \ d \tau \lesssim \|f\|_{\dot{X}^s} \|g\|_{\dot{X}^s}
		\|a_n \|_{\ell^2}, \qquad s \ge (3-m)/4 
	\end{split}
\end{equation}
%
%%
By the triangle inequality 
and Cauchy-Schwartz,
%
%%
\begin{equation}
	\label{l1m}
	\begin{split}
		& | \sum_{n \in \zzdot} |n|^{s} a_n
		\int_{\rr}\frac{| \wh{w_{fg}}(n, \tau) |}{(1 + | \tau - n^{m } |)} \ d \tau |
		\\
		& \le \sum_{n \in \zzdot} \int_{\rr} \frac{| a_n |}{\left( 1 + 
		| \tau - n^{m } |
		\right)^{1/2 + \eta}} \times \frac{| n|^s  |
		\wh{w_{fg}}(n, \tau) |}{\left( 
		1 + | \tau - n^{m } | \right)^{1/2 - \eta}} \ d \tau
		\\
		& \le \left( \sum_{n \in \zzdot} | a_{n} |^2\int_{\rr} \frac{1}{\left( 1 + |
		\tau - n^{m } | \right)^{1 + 2 \eta}} \ d \tau  
		\right)^{1/2} 
		\left ( \sum_{n \in \zzdot}\int_{\rr} \frac{|n|^{2s} | \wh{w_{fg}}(n, \tau) 
		|^2}{\left( 1 + | \tau - n^{m } | \right)^{1 -2 \eta}}\ d \tau 
		\right)^{1/2}.
	\end{split}
\end{equation}
%
%%
Applying the change of variable $\tau - n^{m }
\mapsto \tau'$ we obtain  
%%

\begin{equation*}
	\begin{split}
		& \left( \sum_{n \in \zzdot} | a_{n} |^2\int_{\rr} \frac{1}{\left( 1 + | \tau -
		n^{m } | \right)^{1 + 2 \eta}} \ d \tau  
		\right)^{1/2} 
		\\
		& = \left ( \sum_{n \in \zzdot}
		| a_n |^2 
		\int_{\rr} \frac{1}{\left( 1 + | \tau' | \right)^{1 + 2 \eta}} \ d 
		\tau \right)^{1/2}
		\\
		& \simeq \|a_n\|_{\ell^2}, \qquad \eta >0.
		\end{split}
\end{equation*}
However, if we assume $\eta >0$, then
we cannot use \cref{lprop:prim-bilin-est} to bound
\begin{equation*}
	\begin{split}
		\left ( \sum_{n \in \zzdot}\int_{\rr} \frac{|n|^{2s} | \wh{w_{fg}}(n, \tau) 
		|^2}{\left( 1 + | \tau - n^{m } | \right)^{1 - 2\eta}}\ d \tau
		\right)^{1/2}. 
	\end{split}
\end{equation*}
%%
%%
\begin{remark}
Hence, unlike the mNLS, we have not been able to obtain a second bilinear
estimate as a corollary from the first. Heuristically, this is due to the
derivative in nonlinearity, which is not present in the mNLS nonlinearity,
affording us the ``wiggle room''  of a 1/4 derivative for the mNLS (i.e. in the case
of the mNLS, its analogue of \cref{lprop:prim-bilin-est} holds for $b \ge
3/8$.)
\end{remark}
%
%
Let us proceed in a different fashion. By duality, it suffices to show
\eqref{lduality-est}. By the symmetry of the convolution, we consider only cases
\eqref{lpigeon-case-1} and \eqref{lpigeon-case-2}.
%
%
\subsection{Case \eqref{lpigeon-case-1}.} In Progress
%
%
%Recalling \eqref{lnon-lin-rep-with-bound}, we have
%\begin{equation}
%	\begin{split}
%		 \int_{\rr} a_n |n|^s \left( 1 + | \tau - n^{m } | \right)^{b} | 
%		\wh{w_{fg}}(n, \tau) | d \tau
%	 & \lesssim \int_{\rr} a_n \wh{C_f C_g}(n, \tau) d \tau
%		\\
%		& \le \|a_n\|_{\ell^2} \|\wh{C_f C_g}(n, \tau)\|_{L^2(\zz \times \rr)}
%		\\
%		& \simeq \|a_n\|_{\ell^2} \|C_f C_g\|_{L^2(\ci \times \rr)}
%		\\
%		& \le \|a_n\|_{\ell^2} \|C_f\|_{L^4(\ci \times \rr)} \|C_g\|_{L^4(\ci \times \rr)}
%	\end{split}
%\end{equation}
%%
%%
%where the last three steps follow from H{\"o}lder's inequality and
%Plancherel. Applying \eqref{lfour-mult-conseq} then completes the proof for this
%case.
%
%
\subsection{Case \eqref{lpigeon-case-2}.} Recalling \eqref{l3f}, we have
%
\begin{equation}
	\begin{split}
		& \sum_{n \neq 0} \int_{\rr} a_n |n|^s \left( 1 + | \tau - n^{m } | \right)^{-1} | 
		\wh{w_{fg}}(n, \tau) | d \tau
		\\
		& \le \sum_{n \neq 0}  \int_{\rr} a_{n} (1+ | \tau - n^{m} |)^{-1} \wh{\overset{\sim}{C_f} C_g} d
		\tau
	\\	
	& = \sum_{n \neq 0} \int_{\rr} a_{n} (1+ | \tau - n^{m} |)^{-5/8} (1 + | \tau - n^{m}
	|)^{-3/8} \wh{\overset{\sim}{C_f} C_g} d
		\tau
		\\
		& \le \|a_{n} (1 + | \tau - n^{m} |)^{-5/8}\|_{L^2(\zz \times \rr)}  \| (1 +
		| \tau - n^{m} |)^{-3/8} \wh{\overset{\sim}{C_f} C_g}  \|_{L^2(\zz \times
		\rr)}
		%\\
		%&\wh{\overset{\sim}{C_f} C_g}(n, \tau) d \tau
		%\\
		%& \le \|a_n\|_{\ell^2} \|\wh{\overset{\sim}{C_f} C_g}(n, \tau)\|_{L^2(\zz \times \rr)}
		%\\
		%& \simeq \|a_n\|_{\ell^2} \|\overset{\sim}{C_f} C_g\|_{L^2(\ci \times \rr)}
		%\\
		%& \le \|a_n\|_{\ell^2} \|\overset{\sim}{C_f}\|_{L^4(\ci \times \rr)} \|C_g\|_{L^4(\ci \times \rr)}
	\end{split}
\end{equation}
%
%
where the last step follows from Cauchy-Schwartz. A change of variable shows
that
%
%
\begin{equation*}
	\begin{split}
		\|a_{n} (1 + | \tau - n^{m} |)^{-5/8}\|_{L^2(\zz \times \rr)} \lesssim
		\|a_{n}\|_{\ell^2}
	\end{split}
\end{equation*}
%
%
while \eqref{l3f}-\eqref{l6f} yields the bound
%
%
\begin{equation*}
	\begin{split}
	\| (1 + | \tau - n^{m} |)^{-3/8} \wh{\overset{\sim}{C_f} C_g}  \|_{L^2(\zz
	\times \rr)} \lesssim \|f\|_{\dot{X}^s} \|g\|_{\dot{X}^s}
	\end{split}
\end{equation*}
%
%
completing the proof. \qquad \qedsymbol

%
%\begin{equation*}
%	\begin{split}
%				\\
%		& \le \|C_f\|_{L^4(\ci \times \rr)} \|C_g\|_{L^4(\ci \times \rr)}.
%	\end{split}
%\end{equation*}
%
%


