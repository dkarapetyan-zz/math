%
%
\chapter{H\"OLDER CONTINUITY FOR HR IN THE WEAK \\ TOPOLOGY}
%
%

%
%
%
%
\section{Proof of H\"older Continuity}
%
%
We note that the only significant difference between the proof of H\"older
continuity in the periodic and non-periodic cases is in the proof of Lemma~\ref{lem:frac-deriv}, which we have already addressed. Hence, we focus our
attention on the proof of H\"older continuity in the periodic case. 
%
%
\subsection{Region $\Omega_{1}$} 
\label{ssec:reg-m-imp}
%
%
Let $u_{0}(x), v_{0}(x)
\in B_{H^{s}}(R)$, $s > 3/2$ be two initial data. Then from
the well-posedness theory for HR \cite{Karapetyan:2010fk}, we
know that there exists unique corresponding solutions $u, v \in C(I,
B_{H^{s}}(2R))$ to HR\@.
Set $v=u-w$. Then $v$ solves the Cauchy-problem
%
%
\begin{align}
\label{uniqueness-exp}
& \p_t v
=  -\frac{\gamma}{2} \p_x [v(u + w)] 
\\
\notag
& \phantom{\p_t v = }-\p_x (1 - \p_{x}^{2})^{-1} \left\{
\frac{3-\gamma}{2}[v(u+w)] + \frac{\gamma}{2}[\p_x v \cdot \p_x (u+w)]
\right\},
\\
& v(x,0) = u_{0}(x) - v_{0}(x).
\label{uniqueness-init-data}
\end{align}
%
%
%
Let
\begin{equation*}
D^{m} \doteq (1 - \p_x^2)^{m/2}, \quad m \in \rr.
\end{equation*}
%
Applying $D^r$ to both sides of \eqref{uniqueness-exp}, then 
multiplying both sides by $D^r v$ and integrating, we obtain
%
%
\begin{equation}
\begin{split}
\frac{1}{2} \frac{d}{dt} \|v\|_{H^r}^2
= & -\frac{\gamma}{2} \int_{\ci} D^r \p_x [v(u+w)] \cdot
D^r v \ dx
\\
& - \frac{3-\gamma}{2} \int_{\ci}  D^{r -2}
\p_x[v(u+w)] \cdot
D^r v \ dx  
\\
& - \frac{\gamma}{2} \int_{\ci} D^{r 
-2} \p_x [ \p_x v
\cdot \p_x (u+w)]\cdot D^r v \ dx.
\label{2v-iu}
\end{split}
\end{equation}
We now estimate \eqref{2v-iu} in parts.
%
\subsubsection{Estimate of Integral 1} 
Note that
%
%
\begin{equation}
\begin{split}
& \left |  -\frac{\gamma}{2} \int_{\ci} D^r \p_x [v(u+w)] \cdot
D^r v \ dx \right |
\\
& =
\left |
-\frac{\gamma}{2} \int_{\ci} \left[ D^r \p_x, \ u+w \right]v \cdot
D^r v \ dx - \frac{\gamma}{2} \int_{\ci} (u+w) D^r
\p_x v \cdot D^r v\ dx
\right | \\
& \lesssim \left |
\int_{\ci} \left[ D^r \p_x, \ u+w \right]v \cdot
D^r v \ dx \right |
+ \left | \int_{\ci} (u+w) D^r \p_x v
\cdot D^r v\
dx \right |.
\label{4v-iu}
\end{split}
\end{equation}
%
%
Observe that integrating by parts gives
%
%
\begin{equation}
\begin{split}
\left | \int_{\ci} (u+w) D^r \p_x v \cdot
D^r v \ dx \right |
\le \|\p_x (u+w)\|_{L^\infty}
\|v\|_{H^r}^2.
\label{4'v-iu}
\end{split}
\end{equation}
%
%
%
%

An application of 
Cauchy-Schwartz and Lemma~\ref{cor1} then yields 
%
%
\begin{equation}
\begin{split}
\left | \int_{\ci} [D^r \p_x, \ u+w] v
\cdot D^r v \ dx \right |
& \lesssim \|u+w\|_{H^s} 
\|v\|_{H^r}^2.
\label{120v}
\end{split}
\end{equation}
%
%
Combining \eqref{4'v-iu} and \eqref{120v} and applying the Sobolev embedding 
theorem, we obtain the estimate
%
%
\begin{equation}
\begin{split}
\left |  -\frac{\gamma}{2} \int_{\ci} D^r \p_x [v(u+w)] \cdot
D^r v \ dx \right |
\lesssim \|u+w\|_{H^s} \|v\|_{H^r}^2, \quad s > 3/2, \ -1 \le r \le s-1.
\label{20v}
\end{split}
\end{equation}
%
%
%
%
%
\subsubsection{Estimate of Integral 2}
%
%
Applying Cauchy-Schwartz and Lemma~\ref{lem:frac-deriv}, we obtain
%
%
%
\begin{equation*}
\begin{split}
\left | - \frac{3-\gamma}{2} \int_{\ci}  D^{r -2}
\p_x[v(u+w)] \cdot
D^r v \ dx  \right |
& \lesssim \|u+w\|_{H^{r -1}} \|v\|_{H^r}^2
\end{split}
\end{equation*}
%
%
which implies
\begin{equation}
\begin{split}
\left | - \frac{3-\gamma}{2} \int_{\ci}  D^{r -2}
\p_x[v(u+w)] \cdot
D^r v \ dx  \right |
& \lesssim \|u+w\|_{H^{s}} \|v\|_{H^r}^2
\label{3vj}
\end{split}
\end{equation}
%
for $s > 3/2, \ r \le s, \ \text{and} \ s + r \ge 2$.

%
\subsubsection{Estimate of Integral 3} We first apply
Cauchy-Schwartz to obtain
%
%
\begin{equation*}
\begin{split}
\left | - \frac{\gamma}{2} \int_{\ci} D^{r 
-2} \p_x [ \p_x v
\cdot \p_x (u+w)]\cdot D^r v \ dx \right | 
\lesssim 
\|[\p_x v \cdot \p_x (u+w)] \|_{H^{r -1}}
\|v\|_{H^r}.
\end{split}
\end{equation*}
%
%
Applying Lemma~\ref{lem:frac-deriv} and the inequality $\| f_{x}
\|_{H^{m-1}} \le \| f \|_{H^{m}}$,  we conclude that
%
\begin{equation}
\begin{split}
\left | - \frac{\gamma}{2} \int_{\ci} D^{r 
-2} \p_x [ \p_x v
\cdot \p_x (u+w)]\cdot D^r v \ dx \right | 
\lesssim \|u+w \|_{H^{s}}
\|v\|_{H^r}^2
\label{3'v-iu}
\end{split}
\end{equation}
%
%
for $s > 3/2, \ r \le s, \ \text{and} \ s + r \ge 2$.
%
%
%
%
Grouping \eqref{20v}-\eqref{3'v-iu}, we obtain
%
%
\begin{equation*}
\begin{split}
\frac{1}{2} \frac{d}{dt}
\|v\|_{H^r}^2
& \lesssim \|u+w\|_{H^s}
\|v\|_{H^r}^2, \quad | t | < T
\\
& \le 4R \| v \|_{H^{r}}^{2}.
\label{9v-iu}
\end{split}
\end{equation*}
%
%
%
%
%
Letting $y(t) = \| v \|^{2}_{H^{r}}$, we obtain
%
%
%
\begin{equation*}
\begin{split}
\frac{dy}{dt} \le cy
\end{split}
\end{equation*}
%
where $c = c(s, r, R) > 0$. Hence
%
%
\begin{equation*}
\begin{split}
y(t) \le y(0) e^{ct}, \quad | t | < T
\end{split}
\end{equation*}
%
%
which implies
%
%
\begin{equation*}
\begin{split}
y(t) \le y(0) e^{cT}.
\end{split}
\end{equation*}
%
%
Substituting back in for $y$, we see that
%
%
\begin{equation*}
\begin{split}
\| v \|_{H^{r}}^{2} \le \| v(0) \|^{2}_{H^{r}} e^{cT}
\end{split}
\end{equation*}
%
%
or
%
%
\begin{equation}
\label{lip-ineq}
\begin{split}
& \| u(t) - w(t) \|_{H^{r}} \le C \| u_{0} - w_{0} \|_{H^{r}}, 
\\
& \text{for} \ | t | < T,
\ s > 3/2, \ -1 \le r \le s-1, \ s + r \ge 2.
\end{split}
\end{equation}
%
Hence, in region $\Omega_{1}$, the data-to-solution map is locally Lipschitz from
$B_{H^{s}}(R)$ (measured with the $H^{r}$
norm) to $C([-T, T], H^{r})$, with Lipschitz constant $C = C(s, r, R)$.
%
%
%
%
%
%
%
%
%
%
\subsection{Region $\Omega_{2}$} 
\label{ssec:case-4}
%
We have the estimate
\begin{equation}
\label{fgh}
\begin{split}
\| u(t) - w(t) \|_{H^{r}}
& < \|u(t) - w(t) \|_{H^{2-s}}.
\end{split}
\end{equation}
%
We see that \eqref{lip-ineq} is valid for $r = 2-s$, $3/2 < s \le 3$.
Hence, applying \eqref{lip-ineq} to \eqref{fgh}, we obtain 
%
%
%
%
\begin{equation*}
\begin{split}
\| u(t) - w(t) \|_{H^{r}}
\lesssim \|u_{0} - w_{0} \|_{H^{2-s}}.
\end{split}
\end{equation*}
%
%
%
%
%
Applying the interpolation estimate
%
%%%%%%%%%%%%%%%%%%%%%%%%%%%%%%%%%%%%%%%%%%%%%%%%%%%%%
%
%
%                interp
%
%
%%%%%%%%%%%%%%%%%%%%%%%%%%%%%%%%%%%%%%%%%%%%%%%%%%%%%
%
%
%
\begin{equation}
  \label{interp}
\begin{split}
\| f \|_{H^{m}} \le \| f \|_{H^{m_{1}}}^{(m_{2}-m)/(m_{2} - m_{1})} \| f
\|_{H^{m_{2}}}^{(m -m_{1})/(m_{2} - m_{1})}, \quad m_{1} < m < m_{2}
\end{split}
\end{equation}
with $m_{1} =r$, $m = 2-s$, and $m_{2} = s$ (notice
$m_{2} > m$ for $s > 1$), we bound 
%
%
\begin{equation*}
\begin{split}
\| u_{0} - w_{0} \|_{H^{2-s}} 
& \le \| u_{0} - w_{0} \|_{H^{r}}^{\frac{2(s-1)}{s-r}} \| u_{0} - w_{0}
\|_{H^{s}}^{\frac{2-s-r}{s-r}}
\\
&  \lesssim \| u_{0} - w_{0} \|_{H^{r}}^{\frac{2(s-1)}{s-r}}.
\end{split}
\end{equation*}
%
We conclude that
%
%
\begin{equation*}
\begin{split}
\| u(t) - w(t) \|_{H^{r}} \lesssim \|u_{0} - w_{0} \|_{H^{r}}^{\frac{2(s-1)}{s-r}}.
\end{split}
\end{equation*}
%
%
%
%
\subsection{Region $\Omega_{3}$} 
\label{ssec:case-2}
%
%
Applying \eqref{interp} with $m_{1} = s-1$, $m =r$ and $m_{2} = s$, and 
using the estimate
%
%
\begin{equation*}
\begin{split}
\|u - w \|_{H^{s}} \le 4R
\end{split}
\end{equation*}
%
%
we obtain
%
%
\begin{equation}
\label{pre-lip-ap}
\begin{split}
\| u(t) - w(t) \|_{H^{r}} & \lesssim \| u(t) - w(t) \|_{H^{s-1}}^{s-r} \|u(t)
- w(t)\|_{H^{s}}^{1-s+r}
\\
& \simeq \| u(t) - w(t) \|_{H^{s-1}}^{s-r}.
\end{split}
\end{equation}
%
%
We see that \eqref{lip-ineq} is valid for  $r = s-1$, $s \ge 3/2$. Hence,
applying \eqref{lip-ineq} to \eqref{pre-lip-ap} gives
%
%
\begin{equation*}
\begin{split}
\| u(t) - w(t) \|_{H^{r}} & \lesssim  \|u_{0} - w_{0}\|_{H^{s-1}}^{s-r} 
\\
& \le
\|u_{0} - w_{0}\|_{H^{r}}^{s-r}.
\end{split}
\end{equation*}
%
This completes the proof of Theorem~\ref{thm:main-thm}. \qed
%
%
%
%
%
%
