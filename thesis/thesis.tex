\documentclass[12pt,reqno]{amsbook}
%
%
\makeatletter  %all this is so that don't indent subsections in amsbook
\renewcommand{\subsection}{\@startsection
{subsection}%                   % the name
{2}%                         % the level
{0mm}% the indent
{.5\linespacing\@plus.7\linespacing}%  % the before skip
{-.5em}%          % the after skip
{\normalfont\bfseries}} % the style
%
\renewcommand{\subsubsection}{\@startsection{subsubsection}{3}%
{0mm}{-.5em}%
{\normalfont\itshape}}
\makeatother
%
\usepackage[utf8]{inputenc}
\usepackage[T1]{fontenc} %otherwise have font issues
\usepackage{lmodern} %this with above font packages give us all possible diacritics
\usepackage{amsmath}
\usepackage{amsthm}
\usepackage{amssymb}
\usepackage[showonlyrefs=true]{mathtools} %amsmath extension package
\usepackage{cancel}  %for cancelling terms explicity on pdf
\usepackage{yhmath}   %makes fourier transform look nicer, among other things
\usepackage{framed}  %for framing remarks, theorems, etc.
\usepackage{appendix}
\synctex=1
\usepackage{enumerate} %to change enumerate symbols
\usepackage[margin=2.5cm]{geometry}  %page layout
\setcounter{tocdepth}{1} %must come before secnumdepth--else, pain
\setcounter{secnumdepth}{1} %number only sections, not subsections
%\usepackage[pdftex]{graphicx} %for importing pictures into latex--pdf compilation
\numberwithin{equation}{section}  %eliminate need for keeping track of counters
%\numberwithin{figure}{section}
\setlength{\parindent}{0in} %no indentation of paragraphs after section title
\renewcommand{\baselinestretch}{1.1} %increases vert spacing of text
%\renewcommand{\subsection}{\textbf} %temporary fix--subsection indents in
%amsbook, and want to disable it

%
\usepackage{hyperref}
\hypersetup{colorlinks=true,
linkcolor=blue,
citecolor=blue,
urlcolor=blue,
}
\usepackage[alphabetic, initials, msc-links]{amsrefs} %for the bibliography; uses cite pkg. Must be loaded after hyperref, otherwise doesn't work properly (conflicts with cref in particular)
\usepackage{cleveref} %must be last loaded package to work properly
\newtheorem{theorem}{Theorem}[chapter]
\newtheorem{lemma}[theorem]{Lemma}
\newtheorem{corollary}[theorem]{Corollary}
\newtheorem{claim}[theorem]{Claim}
\newtheorem{prop}[theorem]{Proposition}
\newtheorem{proposition}[theorem]{Proposition}
\newtheorem{no}[theorem]{Notation}
\newtheorem{definition}[theorem]{Definition}
\newtheorem{remark}[theorem]{Remark}
\newtheorem{examp}{Example}[section]
\newtheorem {exercise}[theorem] {Exercise}
\newcommand{\ds}{\displaystyle}
\newcommand{\ts}{\textstyle}
\newcommand{\nin}{\noindent}
\newcommand{\rr}{\mathbb{R}}
\newcommand{\nn}{\mathbb{N}}
\newcommand{\zz}{\mathbb{Z}}
\newcommand{\cc}{\mathbb{C}}
\newcommand{\ci}{\mathbb{T}}
\newcommand{\zzdot}{\dot{\zz}}
\newcommand{\wh}{\widehat}
\newcommand{\p}{\partial}
\newcommand{\ee}{\varepsilon}
\newcommand{\vp}{\varphi}
\newcommand{\wt}{\widetilde}
%
%
%
%
%
\makeatletter \renewenvironment{proof}[1][\proofname] {\par\pushQED{\qed}\normalfont\topsep6\p@\@plus6\p@\relax\trivlist\item[\hskip\labelsep\bfseries#1\@addpunct{.}]\ignorespaces}{\popQED\endtrivlist\@endpefalse} \makeatother%
%makes proof environment bold instead of italic
\newcommand{\uol}{u^\omega_\lambda}
\newcommand{\lbar}{\bar{l}}
\renewcommand{\l}{\lambda}
\newcommand{\R}{\mathbb R}
\newcommand{\RR}{\mathcal R}
\newcommand{\al}{\alpha}
\newcommand{\ve}{q}
\newcommand{\tg}{{tan}}
\newcommand{\m}{q}
\newcommand{\N}{N}
\newcommand{\ta}{{\tilde{a}}}
\newcommand{\tb}{{\tilde{b}}}
\newcommand{\tc}{{\tilde{c}}}
\newcommand{\tS}{{\tilde S}}
\newcommand{\tP}{{\tilde P}}
\newcommand{\tu}{{\tilde{u}}}
\newcommand{\tw}{{\tilde{w}}}
\newcommand{\tA}{{\tilde{A}}}
\newcommand{\tX}{{\tilde{X}}}
\newcommand{\tphi}{{\tilde{\phi}}}
\renewcommand{\qed}{\qquad \qedsymbol} %for proof env qed symbol
%\renewcommand \thepart{\Roman{part}} %change part labeling to Roman
%numeral--not needed for memoir
\renewcommand{\cref}{\Cref} %capitalize option for cleveref doesn't work on all
%computers


\begin{document}
\frontmatter
\title{Thesis}
\author{David Karapetyan}
%\address{Department of Mathematics, University of Notre Dame, Notre Dame, IN 46556}
%\email{dkarapet@nd.edu}

\date{}

\maketitle
%    Dedication.  If the dedication is longer than a line or two,
%    remove the centering instructions and the line break.
%\thispagestyle{empty}
%\vspace*{13.5pc}
%\begin{center}
  %Dedication text (use \\[2pt] for line break if necessary)
%\end{center}

%    Change page number to 6 if a dedication is present.
%\setcounter{page}{4}

%\thispagestyle{empty} %don't want page number for book cover
\tableofcontents

\mainmatter
\part{Strongly Dispersive Equations}
\chapter{Well-Posedness for the Cubic NLS }
			  %
				  %
				  \section{Introduction}
				  We consider the cubic nonlinear Schr\"{o}dinger (NLS) 
				  initial value problem (ivp)
%
%
\begin{gather}
	\label{NLS-eq}
		i \p_t u = - \p_x^2 u - |u|^{2} u,
		\\
		\label{NLS-init-data}
		u(x,0) = \vp(x), \ \ x \in \ci, \ \ t \in \rr.
\end{gather}
%
%
and prove the following result.
%
%
%
%
%%%%%%%%%%%%%%%%%%%%%%%%%%%%%%%%%%%%%%%%%%%%%%%%%%%%%
%
%
%	Main Result				
%
%
%%%%%%%%%%%%%%%%%%%%%%%%%%%%%%%%%%%%%%%%%%%%%%%%%%%%%
%
%
\begin{theorem}
	\label{thm:main}
	For any $s \ge 0$, the initial value problem 
	\eqref{NLS-eq}-\eqref{NLS-init-data} is locally well-posed for 
	initial data $\vp(x) \in H^s(\ci)$.
%
%
\end{theorem} 
%
%
%
%
%%%%%%%%%%%%%%%%%%%%%%%%%%%%%%%%%%%%%%%%%%%%%%%%%%%%%
%
%
%				Outline
%
%
%%%%%%%%%%%%%%%%%%%%%%%%%%%%%%%%%%%%%%%%%%%%%%%%%%%%%
%
%
\section{Outline of the Proof of Theorem}
%
%
%
%
%
We first derive a weak formulation of the NLS ivp. Without loss of 
generality we let $\ci = [-2B, 2B]$, and use
the following notation for the Fourier transform
%
%
%
%
\begin{equation*}
	\begin{split}
		\widehat{f}(n) = \int_{\ci} e^{-ix n} f(x) \ dx.
	\end{split}
\end{equation*}
Assume 
$u(x,t)$ is a classical solution of \eqref{NLS-eq}-\eqref{NLS-init-data}.
Then $u(\cdot ,   t): \rr \to C^2(\ci)$ and is $C^1$ in time; hence, applying 
the Fourier transform to the NLS ivp in the space variable we obtain 
%
%
\begin{gather*}
	\p_t \widehat{u}(n, t) = -i n^2 \widehat{u}(n, t) + i  
	\widehat{|u|^{2} u} (n, t),
	\\
	\widehat{u} (n,0) = \widehat{\vp}(n)
\end{gather*}
%
%
which is a globally well-defined relation in $t$ 
and $n$. Multiplication by the integrating factor $e^{itn^2}$ then yields
%%
%%
\begin{equation*}
	\begin{split}
		\left[ e^{it n^2} \widehat{u}(n) \right]_t = i
		 e^{it n^2} \widehat{|u|^{2} u} (n, t).	
	\end{split}
\end{equation*}
%
%
Integrating from $0$ to $t$, we obtain
%
%
\begin{equation*}
	\begin{split}
		\wh{u}(n, t) = \wh{\vp}(n) e^{-it n^2} + i  
		\int_0^t e^{i(t' - t)n^2} \wh{|u|^{2} u}(n, t') \ 
		dt'
	\end{split}
\end{equation*}
%
%
which by Fourier inversion yields 
%
%
\begin{equation}
	\label{NLS-integral-form}
	\begin{split}
		u(x,t) & = \sum_{n \in \zz} \wh{\vp}(n) e^{i\left( xn - t n^2 
		\right)} 
		 + i \sum_{n \in \zz} \int_0^t e^{i\left[ xn + \left( t' - t 
		\right)n^2 \right]} \wh{|u|^2 u}(n, t') \ dt'.
	\end{split}
\end{equation}
%
%
Here we remark that \eqref{NLS-integral-form} is a weaker 
restatement of the Cauchy-problem \eqref{NLS-eq}-\eqref{NLS-init-data}, 
since by construction any classical solution of the NLS ivp is a solution to 
\eqref{NLS-integral-form}. This motivates the following.



\begin{definition}
	%
	%
	We say $u(x,t)$ is a \emph{local solution} of the NLS ivp
\eqref{NLS-eq}-\eqref{NLS-init-data} if it satisfies the weak formulation 
\eqref{NLS-integral-form} for $t \in I \subset \rr$. If $I = \rr$, then we say 
$u$ is a \emph{global solution}. 
%
%
\end{definition}
%
%
Restrict $t \in [0, 2\delta]$, where $\delta = \pi/m_0$ for some $m_0 \in 
\mathbb{N}$. We now derive an integral 
equation global in $t$ and equivalent to \eqref{NLS-integral-form} for $t 
\in [0, \delta]$. Let $\psi_1(t)$ be a cutoff function symmetric about the 
origin such that $\psi_1(t) = 1$ for $|t| \le \delta$ and $\text{supp} \, \psi_1 
= 
[-2\delta, 2\delta ]$. Multiplying the right hand side of expression 
$\eqref{NLS-integral-form}$ by $\psi_1(t)$, we obtain
%
%
\begin{equation}
	\begin{split}
		\label{cutoff-int-eq}
		u(x, t)
		& = \frac{1}{2 \pi} \psi_1(t) \sum_{n \in \zz} e^{i(xn - tn^{2
		})} \widehat{\vp}(n) 
		\\
		& + \frac{i }{2 \pi} \psi_1(t) \int_0^t \sum_{n \in \zz} 
		e^{i\left[ xn + (t - t')n^2 \right]} \wh{w}(n, t') \ dt'
	\end{split}
\end{equation}
%
%
where $w(x, t) \doteq |u|^2 u$. Periodically extend $u(x, t)$ in the time 
variable, and assume a priori that $u \in 
L^4(\ci^2)$. 
\\
\hrule
Claim: Therefore, by Fourier inversion (makes sense if we view $w$ as a 
distribution, which we are permitted to do since $w \in L^{4/3}$) and 
Fubini-Tonelli (does not make sense to me--what guarantee 
do we have that $\wh{w} \in \ell^1$? \textbf{these questions have been resolved 
in mNLS document--in process of copying them over})
\\
\hrule
%
\begin{equation}
	\label{prim-int-form}
	\begin{split}
		& \frac{i }{2 \pi} \psi_1(t) \int_0^t \sum_{n \in \zz} 
		e^{i\left[ xn + (t - t')n^2 \right]} \wh{w}(n, t') \ dt'
		\\
		& = \frac{i }{4 \pi^2} \psi_1(t)
		\int_0^t 
		e^{it'(\tau - n^2)} \ dt' 
		\sum_{n, \tau \in \zz} e^{i(xn + tn^2)} \wh{w}(n, \tau) .
	\end{split}
\end{equation}
%%
%%
Since
%
%
\begin{equation*}
	\begin{split}
		\int_0^t e^{it'(\tau - n^2)} \ dt' = -i \frac{e^{it(\tau - n^{2  
		})}- 1}{\tau - n^2}
	\end{split}
\end{equation*}
%%
%%
expression \eqref{prim-int-form} gives
%
%
%
\begin{equation*}
	\begin{split}
		& \frac{i }{2 \pi} \psi_1(t) \int_0^t \sum_{n \in \zz} 
		e^{i\left[ xn + (t - t')n^2 \right]} \wh{w}(n, t') \ dt'
		\\
		& = \frac{1}{4 \pi^2} \psi_1(t) \sum_{n, \tau \in \zz} 
		e^{i(xn + tn^2)} \frac{e^{it(\tau - n^2)}- 1}{\tau - n^2} 
		\wh{w}(n, \tau)  
		\end{split}
\end{equation*}
%
%
which we substitute
into \eqref{cutoff-int-eq} to obtain
%
%
\begin{equation}
	\begin{split}
		\label{cutoff-int-eq-2}
		u(x, t)
		& = \frac{1}{2 \pi} \psi_1(t) \sum_{n \in \zz} e^{i(xn - tn^{2  
		})} \widehat{\vp}(n) 
		\\
		& + \frac{1}{4 \pi^2} \psi_1(t) \sum_{n, \tau \in \zz} 
		e^{i(xn + tn^2)} \frac{e^{it(\tau - n^2)}- 1}{\tau - n^2} \wh{w}(n, 
		\tau). 
	\end{split}
\end{equation}
%
%
Next, we localize near the singular curve $\tau = n^2$. Let $\psi_2(t)$ be a cutoff function symmetric about the origin such that 
$\psi_2(t) = 1$ for $|t| \le B$ and $\text{supp} \, \psi_2 = 
[-2B, 2 B]$.
Multiplying the summand of the second term of \eqref{cutoff-int-eq-2} by $1 \pm \psi_2(\tau - n^2)$ and rearranging 
terms, we have
%
%
\begin{equation*}
	\begin{split}
		 u(x, t)
		& = \frac{1}{2 \pi} \psi_1(t) \sum_{n \in \zz} e^{i(xn + tn^{2 
		})} \widehat{\vp}(n) 
		\\
		& + \frac{1}{4 \pi^2} \psi_1(t) \sum_{n \in \zz} e^{ixn} \sum_{\tau \in \zz} 
		e^{it \tau} \frac{ 1 - \psi_2(\tau - n^2) 
		}{\tau - n^2} \wh{w}(n, \tau) 
		\\
		& - \frac{1}{4 \pi^2} \psi_1(t) \sum_{n \in \zz} e^{i(xn + 
		tn^2)}
		\sum_{\tau \in \zz} \frac{1- \psi_2(\tau - n^2)}{\tau - n^2} \wh{w}(n, \tau) \ 
		d\tau
		\\
		& + \frac{1}{4 \pi^2} \psi_1(t) \sum_{n \in \zz}
		e^{i(xn + tn^2)}
		\sum_{\tau \in \zz} 
		\frac{\psi_2(\tau - n^2)\left[ e^{it(\tau - n^2)}-1 
		\right]}{\tau - n^2} \wh{w}(n, \tau) 
	\end{split}
\end{equation*}
%
%
which by a power series expansion of $[e^{it(\tau - n^2)}-1]$ simplifies  to
%
%
\begin{align}
	\label{main-int-expression-1}
	& u(x, t) \notag
		\\
		& = \frac{1}{2 \pi} \psi_1(t) \sum_{n \in \zz} e^{i(xn + tn^{2 
		j})} \widehat{\vp}(n) 
		\\
		\label{main-int-expression-2}
		& + \frac{1}{4 \pi^2} \psi_1(t) \sum_{n \in \zz} e^{ixn} \sum_{\tau \in \zz} 
		e^{it \tau} \frac{ 1 - \psi_2(\tau - n^2) 
		}{\tau - n^2} \wh{w}(n, \tau) 
		\\
		\label{main-int-expression-3}
		& - \frac{1}{4 \pi^2} \psi_1(t) \sum_{n \in \zz} e^{i(xn + 
		tn^2)}
		\sum_{\tau \in \zz} \frac{1- \psi_2(\tau - n^2)}{\tau - n^2} \wh{w}(n, \tau) 
		\\
		\label{main-int-expression-4}
		& + \frac{1}{8 B \pi^2} \psi_1(t) \sum_{k \ge 1} \frac{i^k (2Bt)^k}{k!}
		\sum_{n \in \zz} e^{i(xn + tn^2 )}
		\sum_{\tau \in \zz}	\psi_2 (\tau - n^2) \left(\frac{\tau - 
		n^2}{2B} \right)^{k -1} \wh{w}(n, \tau) 
		\\
		& \doteq T(u). \notag
\end{align}
where $T = T(\vp)$. Note that 
\eqref{main-int-expression-1}-\eqref{main-int-expression-4} is a global 
relation in $t$; furthermore, the fixed point solution $Tu=u$ gives rise to a 
local solution of the NLS ivp by simply restricting the time 
variable to 
the $[0, \delta]$ interval. Hence, we focus our attention on 
\eqref{main-int-expression-1}-\eqref{main-int-expression-4}. In 
Section 3 we will show that for initial data $\vp \in H^s(\ci)$, $T$ is 
a contraction on $B_M 
\subset L^4(\ci^2)$, where $B_M$ is the ball centered at 
the origin of radius $M = M(\vp) > 0$, by estimating the $L^4(\ci^2)$
norm of 
\eqref{main-int-expression-1}-\eqref{main-int-expression-4}. The Picard 
fixed point theorem and time restriction will
then yield a unique local solution to the NLS ivp in the time interval
$I = [0, \delta]$. Continuous 
dependence will follow from estimates used to establish the contraction 
mapping.  In Section 4 a similar argument for handling well-posedness for 
\eqref{NLS-integral-form} in $H^s(\ci)$, $s > 0$ will be discussed.
%In the 
%appendix we establish necessary conditions under which a local solution to 
%\eqref{NLS-integral-form} can be viewed as a distributional solution of 
%\eqref{NLS-eq}-\eqref{NLS-init-data}.
%
%
%
%
%%%%%%%%%%%%%%%%%%%%%%%%%%%%%%%%%%%%%%%%%%%%%%%%%%%%%
%
%
%	Case \ell^2			
%
%
%%%%%%%%%%%%%%%%%%%%%%%%%%%%%%%%%%%%%%%%%%%%%%%%%%%%%
%
%
\section{The case $\vp \in L^2(\ci)$}
\label{sec:s=0}
%
%
%
%
%%%%%%%%%%%%%%%%%%%%%%%%%%%%%%%%%%%%%%%%%%%%%%%%%%%%%
%
%
%		Estimation of Integral Equality Part 1		
%
%
%%%%%%%%%%%%%%%%%%%%%%%%%%%%%%%%%%%%%%%%%%%%%%%%%%%%%
%
%
\vskip0.1in
\subsection{Estimate for \eqref{main-int-expression-1}}
Letting $f(x,t) = \psi(t) \sum_{n \in \zz} e^{i(xn + tn^2)} 
\wh{\vp}(n)$, we have $\wh{f}(n,t) = \psi(t) \wh{\vp}(n) e^{itn^2}$,
from which we obtain
%
%
\begin{equation}
	\label{fourier-trans-calc}
	\begin{split}
		\wh{f}(n, \tau)
		& = \wh{\vp}(n) \int_\rr e^{-it( \tau - n^2}) 
		\wh{\psi_1}(\tau) \ d\tau
		= \wh{\psi_1}(\tau - n^2) \wh{\vp}(n).
	\end{split}
\end{equation}
%
%
We now require the following multiplier estimate, whose proof can be found in 
\cite{Bourgain-Fourier-transfo-1}. %
%
%%%%%%%%%%%%%%%%%%%%%%%%%%%%%%%%%%%%%%%%%%%%%%%%%%%%%
%
%
%			Fourier Multiplier Estimate	
%
%
%%%%%%%%%%%%%%%%%%%%%%%%%%%%%%%%%%%%%%%%%%%%%%%%%%%%%
%
%
\begin{lemma}[Bourgain]
	\label{lem:four-mult-est}
	Let $(x, t) \in \ci^2$ and let $(n, \tau) \in \zz^2$ 
	be the dual variables. Then for $h \in L^2(\ci^2)$,
%
\begin{equation}
	\label{four-mult-est}
	\begin{split}
		\|h\|_{L^4(\ci^2)} \le c \|(1 + |\tau - 
		n^2|)^{3/8} \, \wh{h}(n, \tau) \|_{\ell^2(\zz^2)}.
	\end{split}
\end{equation}
%
%
\end{lemma}
%
%
%
%
%
%
%
Applying \cref{lem:four-mult-est} and \eqref{schwartz-bound} to
\eqref{fourier-trans-calc}, we obtain
%
%
\begin{equation}
	\label{main-int-est-part-1}
	\begin{split}
		\|\eqref{main-int-expression-1}\|_{L^4(\ci^2)} 
		& \le c  \|(1 + |\tau - 
		n^2|)^{3/8} \wh{\psi}(\tau - n^2) 		\wh{\vp}(n) \|_{\ell^2(\zz^2)}
		\\
		& \le c \|(1 + |\xi|)  \wh{\psi_1}(\xi)\|_{L^\infty(\zz)} 
		\|\wh{\vp} \|_{\ell^2(\zz)}
		\\
		& \lesssim \|\vp \|_{L^2(\ci)}.
	\end{split}
\end{equation}
where the last step follows from Plancharel and the estimate
%
%
\begin{equation}
	\label{schwartz-bound}
	\begin{split}
		\|\left( 1 + |\lambda| \right)^k \wh{\psi_1}(\lambda) 
		\|_{L^\infty(\zz)} \le c_k , \qquad k\ge0
	\end{split}
\end{equation}
%
%
whose proof is provided in the appendix.
%
%
\vskip0.1in
%
%
%
%
%
%
\vskip0.1in
\textbf{Estimate for \eqref{main-int-expression-2}.}
We have
%
%
\begin{equation}
	\label{1ag}
	\begin{split}
		\|\eqref{main-int-expression-2}\|_{L^4(\ci^2)} 
		& \lesssim \| \sum_{n \in \zz} e^{ixn} \sum_{\tau \in
		\zz} e^{it \tau} \frac{1- \psi_2\left( \tau - n^2 \right)}{\tau - 
		n^2} \wh{w}\left( n, \tau \right) \|_{L^4(\ci^2)}
		\\
		& \lesssim  \|\left( 1 + |\tau - n^2| \right)^{3/8}
		\frac{1- \psi_2\left( \tau - n^2 \right)}{\tau - 
		n^2} \wh{w}(n, \tau) \|_{\ell^2(\zz^2)}
	\end{split}
\end{equation}
%
%
where the last step follows from \cref{lem:four-mult-est}. Applying the 
estimate
%
%
\begin{equation}
	\label{2ag}
	\begin{split}
		\left[ 1 - \psi_2\left( \tau - n^2 \right) \right]
		\left( \tau - n^2 \right)^{-1}
		& \le \left( 1 + 1 /B \right)\left( 1 + |\tau - n^2| 
		\right)^{-1}
		\\
		& = \left( 1 + 1/B \right)\left( 1 + B \right)^{-k }
		\left( 1 + B \right)^k \left( 1 + |\tau - n^2| \right)^{-1} 
				\\
		& \le \left( 1 + 1/B \right)\left( 1 + B \right)^{-k 
		} \left( 1 + |\tau - n^2| \right)^{k-1}
		\\
		& = \frac{\left( 1 + B \right)^{1-k }}{B} \left( 1 + |\tau - 
		n^2|
		\right)^{k - 1}
		\\
		& \le 2^{k -1} B^{-k } \left( 1 + |\tau - n^2| \right)^{k - 1}, 
		\qquad k \le 1
	\end{split}
\end{equation}
%
%
with $k = 1/4$ to \eqref{1ag}, we obtain 
%
%
\begin{equation}
	\label{3ag}
	\begin{split}
		\|\eqref{main-int-expression-2}\|_{L^4(\ci^2)}
		& \lesssim  
		B^{-1/4} \|\left( 1 + |\tau - n^2| \right)^{-3/8} \wh{w} (n, 
		\tau) \|_{\ell^2(\zz^2)}.
	\end{split}
\end{equation}
%
%
We now need the following dual estimate of 
\cref{lem:four-mult-est}, whose proof is provided in the appendix:
%
%
\begin{corollary}
	\label{cor:four-mult-est-dual}
Let $(n, \tau) \in \zz^2$ and let  $(x, t) \in \ci^2$
be the dual variables. Then for $h \in L^2(\ci^2)$
%
%
\begin{equation}
	\label{four-mult-est-dual}
	\|(1 + |\tau - 
	n^2|)^{-3/8} \wh{h}(n, \tau) \|_{\ell^2(\zz^2)} \le 
	c \|h\|_{L^{4/3}(\ci^2)}.
\end{equation}
%
%
\end{corollary}
%%
%%
Applying \cref{cor:four-mult-est-dual} to \eqref{3ag}, we conclude that 
%
%
\begin{equation}
	\label{main-int-est-2}
	\begin{split}
		\|\eqref{main-int-expression-2}\|_{L^4(\ci^2)} 
		& \le c B^{-1/4} \|w\|_{L^{4/3}(\ci^2)}
		\\
		& = c B^{-1/4} \|u^3\|_{L^{4/3}(\ci^2)}
		\\
		& = c B^{-1/4} \|u\|^3_{L^4(\ci^2)}.
	\end{split}
\end{equation}
%%
%%

\vskip0.1in
\textbf{Estimate for \eqref{main-int-expression-3}.} 
Letting $$f(x,t) = \psi(t) \sum_{n \in \zz} e^{i\left( xn + tn^2 \right)} 
\sum_{\lambda \in \zz} \frac{1 - \psi\left( \tau - n^2 \right)}{\tau - n^2} 
\wh{w} \left( n, \tau \right),$$ we have
%
%
\begin{equation*}
	\begin{split}
		& \wh{f^x}(n, t) = \psi(t) e^{itn^2} \sum_{\lambda \in \zz} 
		\frac{1 - \psi\left( \lambda - n^2 \right)}{\lambda - n^2} 
		\wh{w}(n, \lambda)
	\end{split}
\end{equation*}
and
\begin{equation*}
	\begin{split}
		 \wh{f}\left( n, \tau \right)
		 & = \int_\ci e^{-it\left( \tau - n^2 
		\right)} \psi(t) \sum_{\lambda \in \zz} \frac{1 - \psi\left( 
		\lambda - n^2 
		\right)}{\lambda - n^2} \wh{w}(n, \lambda)
		\\
		& = \wh{\psi_1}\left( \tau - n^2 \right) \sum_{\lambda \in \zz}
		\frac{1 - \psi\left( 
		\lambda - n^2 
		\right)}{\lambda - n^2} \wh{w}(n, \lambda).
	\end{split}
\end{equation*}
%
%
Therefore, applying \cref{lem:four-mult-est} and 
\eqref{schwartz-bound} gives 
%
%
\begin{equation}
	\label{4hh}
	\begin{split}
		\|f\|_{L^4(\ci^2)}
		& \lesssim \|\left( 1 + | \tau - n^2| \right)^{3/8} 
		\wh{\psi_1}\left( \tau - n^2 \right) \sum_{\lambda \in \zz} 
		\frac{1 - \psi_2\left( \lambda - n^2 \right)}{\lambda - n^2} 
		\wh{w}\left( n, \lambda \right) 
		\|_{\ell^2\left( \zz^2 \right)}
		\\
		& \lesssim \|\sum_{\lambda \in \zz}
		\frac{1 - \psi_2\left( \lambda - n^2 \right)}{\lambda - n^2} \wh{w} 
		\left( n, \lambda \right) \|_{\ell^2\left( \zz \right)}.
	\end{split}
\end{equation}
%
%
Next, note 
that 
%
%
\begin{equation}
	\label{apply-ortho}
	\begin{split}
		\|\sum_{\lambda \in \zz}
		\frac{1 - \psi_2\left( \lambda - n^2 \right)}{\lambda - n^2} \wh{w} 
		\left( n, \lambda \right) \|_{\ell^2\left( \zz \right)}
		& = \left( \sum_{n \in \zz}  | \sum_{\lambda \in \zz}
		\frac{1 - \psi_2\left( \lambda - n^2 \right)}{\lambda - n^2} \wh{w} 
		\left( n, \lambda \right) |^2 \right )^{1/2}
		\\
		& \lesssim \left( \sum_{n \in \zz} \sum_{\lambda \in \zz} |
		\frac{1 - \psi_2\left( \lambda - n^2 \right)}{\lambda - n^2} 
		\wh{w}\left( n, \lambda \right) |^2 
		\right)^{1/2} 
	\end{split}
\end{equation}
%
%
\hrule
Claim [Bourgain]: where the last step follows from orthogonality. 
\\
Preliminary Explanation: I understand now how Bourgain does this; the work is in 
my mNLS document. I am in the process of copying it over to this document.
\\
\hrule
%
%
But by \eqref{2ag} and \cref{cor:four-mult-est-dual}, we have
\begin{equation*}
	\begin{split}
		\left( \sum_{n, \tau \in \zz} |
		\frac{1 - \psi_2\left( \lambda - n^2 \right)}{\lambda - n^2} 
		\wh{w}\left( n, \lambda \right) |^2 
		\right)^{1/2}
		& = \|\frac{1 - \psi_2 \left( \lambda - n^2 
		\right)}{\lambda - n^2} \wh{w}\left( n, \lambda \right) 
		\|_{\ell^2(\zz^2)}
		\\
		& \lesssim B^{-5/8} \|\left( 1 + |\tau - n^2| \right)^{-3/8} 
		\wh{w}\left( n, \lambda \right) \|_{\ell^2(\zz^2)}
		\\
		& \lesssim B^{-5/8} \|w\|_{L^{4/3}(\ci^2)}
		\\
		& = B^{-5/8} \|u\|_{L^4(\ci^2)}^3.
	\end{split}
\end{equation*}
%
%
Substituting back into \eqref{4hh}, and letting $c$ absorb all superficial 
constants, we obtain
%
%
\begin{equation}
	\label{main-int-3-est}
	\begin{split}
		\|f\|_{L^4(\ci^2)} \le c \delta B^{-5/8} 
		\|u\|_{L^4(\ci^2)}^3.
	\end{split}
\end{equation}
%
%
\textbf{Estimate for \eqref{main-int-expression-4}.}
Noting that
%
%
\begin{equation*}
	\begin{split}
		|\sum_{k \ge 1} \frac{i^k (2Bt)^k}{k!}| = |e^{2iBt} - 1| \le 2,
	\end{split}
\end{equation*}
%
%
we obtain
%
%
\begin{equation}
	\label{10aa}
	\begin{split}
		\|\eqref{main-int-expression-4}\|_{L^4(\ci^2)} 
		\le 2 \sup_k \|f \|_{L^4(\ci^2)}
	\end{split}
\end{equation}
%
%
where $$f(x,t) = \psi_1(t) \sum_{n \in \zz} e^{i\left( xn + tn^2 \right)} 
		\sum_{\tau \in \zz} \psi_2\left( \tau - n^2 
		\right)\left (\frac{ \tau - n^2}{2B}\right)^{k - 1} \wh{w}\left( n, \tau 
		\right).$$
%
%
Calculating the Fourier transform of $f(x, t)$, we have
\begin{equation*}
	\begin{split}
		& \wh{f^x}(n, t) = \psi_1(t) e^{itn^2} \sum_{\tau \in \zz} 
		\psi_2\left (\frac{ \tau - n^2}{2B}\right)^{k - 1}
		\wh{w}\left( n, \tau \right)
	\end{split}
\end{equation*}
which gives
\begin{equation*}
	\begin{split}
		\wh{f}(n, \lambda)
		& = \int_\ci e^{-it\left( \lambda - n^2 \right)} 
		\psi_1(t) \ dt \sum_{\tau \in \zz} \psi_2
		\left (\frac{ \tau - n^2}{2B}\right)^{k - 1}
\wh{w}\left( n, \tau 
		\right)
		\\
		&  = \wh{\psi_1}\left( \lambda - n^2 \right) \sum_{\tau \in \zz} 
		\psi_2\left( \tau - n^2 \right)
		\left (\frac{ \tau - n^2}{2B}\right)^{k - 1}
		\wh{w} \left(n, \tau \right).
	\end{split}
\end{equation*}
%
%
Therefore, applying \cref{lem:four-mult-est} and 
\eqref{schwartz-bound} to \eqref{10aa} yields
%
%
\begin{equation}
	\label{main-int-4-est-prelim}
	\begin{split}
		& \|f\|_{L^4(\ci^2)} 
		\\
		& \le c \sup_{k \ge 1} \|\left( 1 + |\tau - n^2| 
		\right)^{3/8} \wh{\psi_1}\left( \tau - n^2 \right) \sum_{\lambda 
		\in \zz} \psi_2\left( \lambda - n^2 \right)\left( \frac{\lambda - 
		n^2}{2B} 
		\right)^{k - 1} \wh{w}\left( n, \lambda \right) \|_{\ell^2(\zz^2)}
		\\
		& \le c \delta \| \sum_{\lambda \in \zz} 
		\psi_2\left( \lambda - n^2 \right) \wh{w}\left( n, \lambda \right)\|_{\ell^2(\zz)}
		\\
		& = c \delta \left( \sum_{n \in \zz} | 
		\sum_{\lambda \in \zz} \psi_2\left( \lambda - n^2 \right) 
		\wh{w}\left( n, \lambda \right)|^2\right)^{1/2}
		\\
		& \lesssim \delta \left( \sum_{n \in \zz} \wh{w}(n, 
		n^2) \right)^{1/2}
	\end{split}
\end{equation}
%
%
\hrule
Claim [Bourgain]: where the last step follows by the definition of 
$\psi_2$. 
\\
Preliminary Explanation: Again, resolved in the mNLS document. It follows from a 
duality argument, coupled with the first trilinear estimate proved there. \\
\hrule
Applying \cref{cor:four-mult-est-dual}, and letting $c$ absorb all 
superficial constants, we conclude
that 
%
%
\begin{equation}
	\label{main-int-4-est}
	\begin{split}
		\|f\|_{L^4(\ci^2)}
		& \le c \delta B \|w\|_{L^{4/3}(\ci^2)}
		\\
		& \le c \delta B \|u^3\|_{L^{4/3}(\ci^2)}
		\\
		& \le c \delta B |u\|_{L^4(\ci^2)}^3.
	\end{split}
\end{equation}
%
%
Collecting estimates \eqref{main-int-est-part-1},\eqref{main-int-est-2}, 
\eqref{main-int-3-est}, and \eqref{main-int-4-est}, we obtain
%
%
\begin{equation}
	\label{1gh}
	\begin{split}
		\|Tu\|_{L^4(\ci^2)} \le c\left( \|\vp\|_{L^2(\ci)}
		+ B^{-1/4} \|u\|_{L^4(\ci^2)}^3 + \delta B \|u\|_{L^4(\ci^2)}^3 \right).
	\end{split}
\end{equation}
%
%
Let $B_M = \left\{ u \in L^4(\ci^2): \|u\|_{L^4(\ci^2)} \le 2c \|\vp 
\|_{L^2(\ci)} \right\}$. Setting $B = \delta^{-1/2}$ and restricting $u \in B_M$, we 
obtain %
%
\begin{equation*}
	\begin{split}
		\|Tu\|_{L^4(\ci^2)}
		& \le c \left[ \|\vp\|_{L^2(\ci)} + \left( \delta^{1/8} 
		+ \delta^{1/2} \right) M^3 \right]
		\\
		& \le c \left[ \|\vp\|_{L^2(\ci)} + 2 \delta^{1/2} M^3 \right]
		\\
		& \le c \left[ \|\vp\|_{L^2(\ci)} + 16c^3 \delta^{1/2} \|\vp\|_{L^2(\ci)}^3 
		\right].
	\end{split}
\end{equation*}
 %
%
Recalling that  $\delta = \pi/n, \ n \in \mathbb{N}$ and choosing $n$ large enough 
such that $\delta < 1/(10^6 c^6 \|\vp\|_{L^2(\ci)}^4)$ then gives
%
%
\begin{equation}
	\label{ball-to-ball}
	\begin{split}
		\|Tu\|_{L^4(\ci^2)} < 2c\|\vp\|_{L^2(\ci)} = M, \qquad u \in B_M.
	\end{split}
\end{equation}
%
%
Hence, $T: B_M \to B_M$. Furthermore, from our definition of $T$ in 
\eqref{main-int-expression-1}-\eqref{main-int-expression-4}, we obtain
%
%
\begin{equation*}
	\begin{split}
		Tu - Tv = \eqref{main-int-expression-2} + 
		\eqref{main-int-expression-3} + \eqref{main-int-expression-4}
	\end{split}
\end{equation*}
%
%
where now $w = u|u|^2 - v|v|^2$. By the triangle inequality and generalized
H\"{o}lder
%
%
\begin{equation}
	\label{gen-holder}
	\begin{split}
		\|u |u|^2 - v |v|^2\|_{4/3}
		& = \| |u|^2\left( u -v \right) + v\left( |u|^2 - |v|^2 
		\right)\|_{4/3}
		\\
		& \le \|u^2\left( u -v \right)\|_{4/3} + \|v\left( |u| + |v| 
		\right)\left( |u| - |v| \right) \|_{4/3}
		\\
		& \le \|u\|_4^2 \|u -v \|_4 + \|v\|_4  \| |u| + |v| \|_4 
		\| |u| - |v |\|_4 
		\\
		& \le \|u \|_4^2 \|u -v\|_4 + \|v\|_4\left( \|u\|_4 + \|v\|_4 
		\right) \|u -v \|_4
		\\
		& \le 2 \|u -v \|_4 \left(  \|u\|_4 + \|v\|_4 \right)^2.
	\end{split}
\end{equation}
%
%
Substituting $w = u|u|^2 - v|v|^2$ into the first line of \eqref{main-int-est-2}, 
\eqref{main-int-3-est}, and \eqref{main-int-4-est}, and estimating using  
\eqref{gen-holder}, we conclude that
%
%
\begin{equation*}
	\begin{split}
		\|Tu - Tv\|_{L^4(\ci^2)}
		& \le 2c\left( \delta B + B^{-1/4} 
		\right)\left( \|u\|_{L^4(\ci^2)} +
		\|v\|_{L^4(\ci^2)}\right)^2 \|u -v \|_{L^4(\ci^2)}
		\\
		& \le 8 c M^2(\delta B + B^{-1/4}) \|u - v\|_{L^4(\ci^2)}
		\\
		& \le 16cM^2 \delta^{1/2} \|u-v\|_{L^4(\ci^2)}
		\\
		& < 64 \cancel{c^3} \cancel{\|\vp\|_{L^2(\ci)}^2}
		\times \frac{1}{10^3 \cancel{c^3} \cancel{\|\vp\|_{L^2(\ci)}^2}}
		\|u-v\|_{L^4(\ci^2)}
	\end{split}
\end{equation*}
%
%
which yields the estimate
%
%
\begin{equation}
	\label{contract-est}
	\begin{split}
		\|Tu - Tv\|_{L^4(\ci^2)} < \frac{1}{2} \|u -v \|_{L^4(\ci^2)}.
	\end{split}
\end{equation}
%
%
By \eqref{ball-to-ball} and \eqref{contract-est}, we conclude that
$T = T(\vp)$ is a contraction on $B_M \subset L^4(\ci^2)$. 
%
%
%
%

\chapter{Well Posedness for the dNLS}
%
%
\section{Introduction}
We consider the derivative nonlinear Schr{\"o}dinger (dNLS) initial value problem (IVP)
%
%
\begin{gather}
	\label{1dNLS-eq}
	\p_t u + \p_x^{m} u + \lambda \p_x (|u|^2 u) = 0,
	\\
	\label{1dNLS-init-data}
	u(x,0) = u_0(x), \quad x \in \ci, \ t \in \rr.
\end{gather}
%
%
where $m \in \{3, 5, 7,\dots \}$ and $\lambda \in \{-1, 1\}$.
%
%
Following the arguments used for the mNLS, the dNLS can be rewritten in the
integral form %
%
\begin{equation}
	\label{1dNLS-integral-form}
	\begin{split}
		u(x,t) & = \sum_{n \in \zz} \wh{\vp}(n) e^{i\left( xn - t n^m 
		\right)} 
		\\
		& + i \sum_{n \in \zz} \int_0^t e^{i\left[ xn + \left( t' - t 
		\right) n^m \right]} \wh{w}(n, t') \ dt'.
	\end{split}
\end{equation}
%
%
where 
%
%
\begin{equation*}
\begin{split}
  w(x,t) = \frac{1}{2 \pi} \sum_{n \in \zz} n e^{inx}  \wh{| u |^{2} u}(n, t)
\end{split}
\end{equation*}
%
%
We then localize in time to obtain 
%
%
\begin{align}
	\label{1main-int-expression-0}
	& u(x, t) 
		\\
		\label{1main-int-expression-1}
		& = \frac{1}{2 \pi} \psi_{\delta}(t) \sum_{n \in \zz} e^{i(xn + tn^{m 
		})} \widehat{\vp}(n) 
		\\
		\label{1main-int-expression-2}
		& + \frac{1}{4 \pi^2} \psi_{\delta}(t) \sum_{n\in \zz} \int_\rr e^{ixn}  
		e^{it \tau} \frac{ 1 - \psi(\tau -  n^m) 
		}{\tau -  n^m} \wh{w}(n, \tau) \ d \tau
		\\
		\label{1main-int-expression-3}
		& - \frac{1}{4 \pi^2} \psi_{\delta}(t) \sum_{n\in \zz} \int_\rr e^{i(xn + 
		t n^m)}
		 \frac{1- \psi(\tau -  n^m)}{\tau -  n^m} \wh{w}(n, \tau) \ d \tau
		\\
		\label{1main-int-expression-4}
		& + \frac{1}{4 \pi^2} \psi_{\delta}(t) \sum_{k \ge 1} \frac{i^k t^k}{k!}
		\sum_{n \in \zz} \int_\rr e^{i(xn + t n^m )}
		\psi(\tau -  n^m) (\tau -  n^m)^{k-1} \wh{w}(n, \tau)  
		\\
		& \doteq T(u) \notag
\end{align}
%
%
where $T = T_{\vp, \psi, \delta}$. We now introduce the following spaces. 
%
\begin{definition}
	Denote $\dot{Y}^s$ to be the space of all
	functions $u$ on $\ci \times \rr$ with
	bounded norm
\begin{equation}
	\label{1Y-s-norm}
	\begin{split}
		\|u\|_{\dot{Y}^s} = \|u\|_{\dot{X}^s} + \||n|^s \wh{ u}\|_{ \dot{\ell}^2_n L^1_\tau }
	\end{split}
\end{equation}
%
%
%
%
where
%
\begin{equation}
	\label{1X^s-norm}
	\begin{split}
		& \|u\|_{\dot{X}^s}
		= \left ( \sum_{n\in \zz} |n|^{2s} \int_\rr \left ( 1 + | 
		\tau - n^m \right ) | \wh{u} ( n, \tau ) |^2
		\right )^{1/2}
	\end{split}
\end{equation}
and
%
%
\begin{equation}
	\label{1E-norm}
	\| \wh{u}\|_{ \dot{\ell}^2_n L^1_\tau } = \left[ \sum_{n \in \zzdot}| n |^{2s} \left(
	\int_{\rr}| \wh{u}(n, \tau) |d \tau \right)^{2} \right]^{1/2}.
\end{equation}
%
%
%
%
\end{definition}
The $\dot{Y}^s$ spaces have the following important property, whose proof
is provided in the appendix.
\begin{lemma}
	\label{1lem:cutoff-loc-soln}
  Let $\psi(t)$ be a smooth cutoff function with $\psi(t) =1$ for $t \in [-1,
  1]$, and define $\psi_{\delta}(t) = \psi(t/\delta)$. If
  $\psi_{\delta}(t)u(x,t) \in \dot{Y}^s$, then $u \in C([-\delta, \delta],
  \dot{H}^s(\ci))$.
\end{lemma}
%
We are now prepared to state our result.
%
%%%%%%%%%%%%%%%%%%%%%%%%%%%%%%%%%%%%%%%%%%%%%%%%%%%%%
%
%
%				 Well Posedness Theorem
%
%
%%%%%%%%%%%%%%%%%%%%%%%%%%%%%%%%%%%%%%%%%%%%%%%%%%%%%
%
%
\begin{theorem}
	\label{1thm:prim}
	The dNLS IVP is well-posed in $\dot{H}^s(\ci)$ for $s \ge \frac{1-m}{4}$.  
\end{theorem}
%
%
%%%%%%%%%%%%%%%%%%%%%%%%%%%%%%%%%%%%%%%%%%%%%%%%%%%%%
%
%
%				Outline
%
%
%%%%%%%%%%%%%%%%%%%%%%%%%%%%%%%%%%%%%%%%%%%%%%%%%%%%%
%
%
%
%
%
%
%
%
\begin{proof}
  It follows from the following bilinear estimates.
\end{proof}
%
%
%
%%%%%%%%%%%%%%%%%%%%%%%%%%%%%%%%%%%%%%%%%%%%%%%%%%%%%
%
%
%				 Bilinear Estimates
%
%
%%%%%%%%%%%%%%%%%%%%%%%%%%%%%%%%%%%%%%%%%%%%%%%%%%%%%
%
%
\begin{proposition}[First Bilinear Estimate]
	\label{1prop:prim-bilin-est}
	For any $s \ge \frac{1-m}{4}$ we have
	\begin{equation}
		\label{1prim-bilin-est}
		\left( \sum_{n \in \dot{\zz}} |n|^{2s} \int_\rr
		\frac{|\wh{w_{fg}}(n, \tau) |^2}{1+ |\tau - 
		n^m| } 
		 \ d \tau 
		\right)^{1/2}
		\lesssim \|f\|_{\dot{X}^s} \|g\|_{\dot{X}^s}
	\end{equation}
	where $w_{fg}(x,t)$ = $\p_x(fg)(x,t)$.
\end{proposition}
%
%%%%%%%%%%%%%%%%%%%%%%%%%%%%%%%%%%%%%%%%%%%%%%%%%%%%%
%
%
%				Second trilinear Estimate 
%
%
%%%%%%%%%%%%%%%%%%%%%%%%%%%%%%%%%%%%%%%%%%%%%%%%%%%%%
%
%
\begin{proposition}[Second Bilinear Estimate]
\label{1prop:bilinear-estimate2}
For any $s \ge \frac{1-m}{4}$ we have
%
%
\begin{equation}
	\label{1bilinear-estimate2}
	\begin{split}
		\left( \sum_{n \in \zzdot} |n|^{2s}  \left ( \int_\rr 
		\frac{|\wh{w_{fg}}(n, \tau) |}{1 + | \tau - n^m |}
		 \ d\tau \right)^2  \right)^{1/2} \lesssim \|f\|_{\dot{X}^s} \|g\|_{\dot{X}^s}.
	\end{split}
\end{equation}
\end{proposition}
%
%
%
%
%
\section{Proof of First Bilinear Estimate}
Note first that $|\wh{w_{fg}}(n, \tau) |  = | n\wh{f} *  \wh{g} 
(n, \tau)|$. From this and the conservation of mass, it follows that
%
%
\begin{equation}
	\label{1non-lin-rep}
	\begin{split}
		| \wh{w_{fg}}(n, \tau)|
		& = | \sum_{\substack{n_1 \neq 0, n_2 \neq 0 \\n_1 +n_2 =n}}  \int_{\tau_1 + \tau_2 = \tau}n\wh{f}\left( n_1,  \tau_1 
\right) \wh{g}\left( n_2, \tau_2  
\right) d \tau_1 d \tau_2 |
\\
& = | \sum_{\substack{n_1 \neq0, n_2 \neq 0 \\n_1 + n_2 =n}}  \int_{\tau_1 + \tau_2 = \tau}n\wh{f}\left( n_1,  \tau_1 
\right) \wh{g}\left( n_2, \tau_2  
\right) d \tau_1 d \tau_2 | 
\\
& \le \sum_{\substack{n_1 \neq0, n_2 \neq 0 \\n_1 + n_2 =n}}   \int_{\tau_1 + \tau_2 = \tau}| n | \times | \wh{f}\left( n_1, \tau_1 
\right) | \times  | \wh{g}\left( n_2, \tau_2 
\right) |   d \tau_1 d \tau_2  
\\
& = \sum_{\substack{n_1 \neq0, n_2 \neq 0 \\n_1 + n_2 =n}} \int_{\tau_1 + \tau_2 = \tau}| n | \times \frac{c_f\left( n_1, \tau_1 
\right)}{|n_1|^s \left( 1 + | \tau_1 - n_1^m | \right)^{1/2}}
\\
& \times \frac{c_{g}\left( n_2, \tau_2 \right)}{|n_2|^s\left( 1 + | \tau_2 -  n_2^m| 
\right)^{1/2}}
  \ d \tau_1 d \tau_2 
\end{split}
\end{equation}
%
%
where 
%
%
\begin{equation*}
	\begin{split}
		c_h(n, \tau) =
		\begin{cases}
			|n|^s \left( 1 + | \tau - n^m |  
			\right)^{1/2} | \wh{h}\left( n, \tau \right) |, \qquad & n \neq 0
		\\
		0, \qquad & n = 0.
	\end{cases}
	\end{split}
\end{equation*}
%
%
From our work above, it follows that 
%
%
\begin{equation}
	\label{1convo-est-starting-pnt}
	\begin{split}
		 & |n|^s \left( 1 + | \tau - n^m | \right)^{-1/2} | \wh{w_{fg}}\left( 
		n, \tau \right) |
		\\
		& \le \left( 1 + | \tau - n^m | \right)^{-1/2}
		\sum_{\substack{n_1 \neq0, n_2 \neq 0 \\n_1 + n_2 =n}} \int_{\tau_1 + \tau_2 = \tau}\frac{|n|^{s+1}}{|n_1|^s | n_2|^s} 
		\times \frac{c_f(n_1, \tau_1)}{\left( 1 + | \tau_1 - n_1^m | 
		\right)^{1/2}}
		\\
		& \times
		\frac{c_g(n_2, \tau_2)}{\left( 1 + | \tau_2 - n_2^m | 
		\right)^{1/2}}\ d \tau_1 d \tau_2.
	\end{split}
\end{equation}
%
%
Unlike the NLS, we must use the smoothing properties of the
principal symbol $\tau - n^m$ regardless of the choice of $s$, since the quantity
%
%
\begin{equation}
	\label{1convo-multiplier}
	\begin{split}
		\frac{|n|^{s+1}}{|n_1|^s |n_2|^s }
	\end{split}
\end{equation}
%
%
blows up in general, due to the presence of the extra power of $|n|$ coming from the derivative on
the nonlinearity. To utilize the smoothing effects of the principal symbol, we
first note that 
$$| \tau - n^m - \left( \tau_1 - n_1^m 
+ \tau_2 - n_2^m  \right ) | = | - n^m + n_1^m +
n_2^m| \doteq d_m(n_1, n_2).$$ We will need the following two lemmas, whose
proofs are provided in the appendix.
%
%
%
\begin{lemma}
	\label{1lem:number-theory1}
	Let $n=n_1 + n_2$ and suppose that $n, n_1, n_2\neq
	0$. Then for any integer $c \ge 0$
%
%
\begin{equation}
	\begin{split}
		\label{1number-theory1}
		d_3(n_1,n_2) \ge 2^{-c/2} | n |^{\frac{2+c}{2}} | n_{1}
		|^{\frac{2-c}{2}}| n_2 |^{\frac{2-c}{2}}.
	\end{split}
\end{equation}
%
%
\end{lemma}
%
%
%
%
%
%
\begin{lemma}
	\label{1lem:number-theory}
	Let $n=n_1 + n_2$ and suppose that $n, n_1, n_2\neq
	0$. Then for any integer  $m \ge 3$
%
%
\begin{equation}
	\begin{split}
		\label{1number-theory}
		d_m(n_1,n_2) \ge b_{m, c } 
		|n|^{c/2} |n_1|^{\frac{m-1-c}{2}} | n_2 |^{\frac{m-1-c}{2}}
		\end{split}
\end{equation}
%
%
where the constant $b_{m,c}$ depends only on $m$ and $c$. 
\end{lemma}
%
%
%
%\begin{remark}
%	The case $-1/2 \le s \le 0$ is delicate, and must be treated differently from
%	the case $s < -1/2$ in order to obtain the optimal well-posedness results.
%	This is the motivation for having two instead of one number theory lemma.
%\end{remark}
%%
%
Let us proceed with the case $m=3$ first; we will then generalize to arbitrary
odd $m \ge 3$. By the pigeonhole principle we must have one of the 
following.
%
%
\begin{align}
	\label{1pigeon-case-1}
	& |\tau - n^3| \ge \frac{d_m(n_1, n_2)}{3} 
		 \\
		\label{1pigeon-case-2}
		& | \tau_1 - n_1^3 | \ge \frac{d_m(n_1, n_2)}{3} 
		 \\
		\label{1pigeon-case-3}
		& | \tau_2 - n_2^3 | \ge \frac{d_m(n_1, n_2)}{3}.
		\end{align}
%
%
By the symmetry of the convolution, it will be enough to consider only
\eqref{1pigeon-case-1} and \eqref{1pigeon-case-2}.
%
%
%
\subsection{Case \ref{1pigeon-case-1}.} 
Applying \cref{1lem:number-theory1}, we have, for nonzero $ n, n_1, n_2 $
%
%%
\begin{equation}
	\label{1convo-deriv-bound}
	\begin{split}
		& \frac{|n|^{s+1}}{|n_1|^s 
		| n_2|^s}
		\times
		\frac{1}{(1 + | \tau -n^3 |)^{1/2}}
		\\
		& \lesssim | n |^{s+1}| n_1 |^{-s}| n_2 |^{-s} \times | n
		|^{-\frac{2+c}{4}}| n_1 |^{-\frac{2-c}{4}}| n_2 |^{-\frac{2-c}{4}} 
		\\
		& = | n |^{\frac{4s +2 -c}{4}} | n_1 |^{\frac{-4s -2 +c}{4}} | n_2
		|^{\frac{-4s -2 +c}{4}}
		\\
		& \le 1, \qquad s \ge -1/2.
	\end{split}  
\end{equation}
%
%
\begin{framed}
\begin{remark}
	\label{1rem:s-val}
	The last line follows from the following reasoning: Set $(4s + 2 -c) = 0$
or, equivalently, $-4s -2 +c = 0$. Then for any $c \ge 0$ such that $c = 4s+2$
the left hand side of
\eqref{1convo-deriv-bound} is bounded by $1$. Of course such a $c$ exists, as
long as $s \ge -1/2$.
\end{remark}
\end{framed}
%
%
%
Hence, recalling \eqref{1convo-est-starting-pnt} and applying estimates 
\eqref{1pigeon-case-1} and \eqref{1convo-deriv-bound}, we obtain
%
%
\begin{equation}
	\label{1non-lin-rep-with-bound}
	\begin{split}
		& |n|^s \left( 1 + | \tau - n^3 | \right)^{-1/2} | 
		\wh{w_{fg}}(n, \tau) | 
		\\
		& \lesssim \sum_{\substack{n_1 \neq0, n_2 \neq 0 \\n_1 + n_2 =n}} \int_{\tau_1 + \tau_2 = \tau}\frac{c_f(n_1, \tau_1)}{\left( 1 + | 
		\tau_1 -  n_1^3| \right)^{1/2}}
		\times \frac{c_g\left( n_2, \tau_2\right)}{\left( 1 + | \tau_2 -n_2^3|
		\right)^{1/2}}
		\\
		& = \wh{C_f C_g}(n, \tau)
	\end{split}
\end{equation}
%
%
where
\begin{equation*}
	\begin{split}
		C_h(x,t) =
		\left[ \frac{c_h(n, \tau)}{\left( 1 + | \tau - n^3 | 
		\right)^{1/2}}\right]^\vee .	
	\end{split}
\end{equation*}
%
%
%
Therefore, from \eqref{1non-lin-rep-with-bound}, Plancherel, and generalized 
H\"{o}lder, we obtain
%
%
\begin{equation}
	\label{1gen-holder-bound}
	\begin{split}
		& \| |n|^s \left( 1 + | \tau - n^3 | \right )^{-1/2}  \wh{w_{fg}}\left( 
		n, \tau \right) \|_{L^2(\ci \times \rr)}
		\\
		& \lesssim \|\wh{C_f C_g }\left( n, \tau \right) 
		\|_{L^2\left( \zzdot \times \rr \right)}
		\\
		& \simeq \|C_f C_g \|_{L^2\left( \ci \times \rr \right)}
		\\
		& \le \|C_f \|_{L^4(\ci \times \rr)} \|C_g \|_{L^4(\ci \times \rr)}.
	\end{split}
\end{equation}
%
Applying \cref{nlem:four-mult-est-L4}, we see that
%
%
\begin{equation}
	\label{1four-mult-conseq}
	\begin{split}
		\|C_h\|_{L^4(\ci \times \rr)} 
		& \lesssim \|(1 + | \tau - n^3 |)^{1/2} \wh{C_h}
		\|_{L^2(\zz \times \rr)}
		\\
		& = \|c_{h} \|_{L^2(\zz \times \rr)} 
		\\
		& = \|h \|_{\dot{X}^s}. 
	\end{split}
\end{equation}
%
%
Applying this to \eqref{1gen-holder-bound} we
conclude that
\begin{equation*}
	\begin{split}
		\| |n|^s \left( 1 + | \tau - n^3 | \right ) ^{-1/2} \wh{w_{fg}}\left( 
		n, \tau \right) \|_{L^2(\zzdot \times \rr)}
		& \lesssim \|f\|_{\dot{X}^s} \|g\|_{\dot{X}^s}.
	\end{split}
\end{equation*}
%
%
%
\subsection{Case \ref{1pigeon-case-2}.}
Applying \cref{1lem:number-theory1}, we have, for nonzero $ n, n_1, n_2 $
\begin{equation}
	\label{1convo-deriv-bound-2}
	\begin{split}
		& \frac{|n|^{s+1}}{|n_1|^s 
		| n_2|^s}
		\times
		\frac{1}{(1 + | \tau -n^3 |)^{1/2}}
		\\
		& \lesssim | n |^{s+1}| n_1 |^{-s}| n_2 |^{-s} \times | n
		|^{-\frac{2+c}{4}}| n_1 |^{-\frac{2-c}{4}}| n_2 |^{-\frac{2-c}{4}} 
		\\
		& = | n |^{\frac{4s +2 -c}{4}} | n_1 |^{\frac{-4s -2 +c}{4}} | n_2
		|^{\frac{-4s -2 +c}{4}}
		\\
		& \le 1, \qquad s \ge -1/2
	\end{split}  
\end{equation}
%
%
where the last line follows from \cref{1rem:s-val}.
%
%
Hence, recalling \eqref{1convo-est-starting-pnt} and applying estimate 
\eqref{1convo-deriv-bound-2}, we obtain
%
%
\begin{equation}
	\label{11f}
	\begin{split}
		& |n|^s  \left( 1 + | \tau - n^3 | \right)^{-1/2}| \wh{w_{fg}}\left( 
		n, \tau \right) |
		\\
		& \lesssim 
		\left( 1 + | \tau - n^3 | \right)^{-1/2}\sum_{\substack{n_1 \neq0, n_2 \neq 0 \\n_1 + n_2 =n}} \int_{\tau_1 + \tau_2 = \tau}		c_f(n_1, \tau_1)
		\times
		\frac{c_g(n_2, \tau_2)}{\left( 1 + | \tau_2 - n_2^3 | 
		\right)^{1/2}} 
		\\
		& = \left( 1 + | \tau - n^3 | \right)^{-1/2} \wh{\overset{\sim}{C_f} C_g}
	\end{split}
\end{equation}
%
%%
where
%
%
\begin{equation*}
	\begin{split}
		\overset{\sim}{C_h}(x,t) = \left[ c_h(n, \tau) \right]^\vee.
	\end{split}
\end{equation*}
%
%
Hence
%
%%
\begin{equation}
	\label{13f}
	\begin{split}
		& \| |n|^s \left( 1 + | \tau - n^3 | \right)^{-1/2} \wh{w_{fg}}(n, \tau) 
		\|_{L^2(\zzdot \times \rr)}
		\\
		& \lesssim \|\left( 1 + | \tau - n^3 | \right)^{-1/2} 
		\wh{\overset{\sim}{C_f} C_g } \|_{L^2(\zzdot \times \rr)}
		\\
		& =  \|\left( 1 + | \tau - n^3 | \right)^{-1/2} 
		\wh{\overset{\sim}{C_f} C_g } \|_{L^2(\zz \times \rr)}
		\\
		& \lesssim  \|\overset{\sim}{C_f} C_g  \|_{L^{4/3}(\ci \times \rr)}
	\end{split}
\end{equation}
%
%%
where the last step follows by dualizing \cref{nlem:four-mult-est-L4}. More
precisely, we have the following.
\begin{corollary}
	\label{1cor:four-mult-est-L4}
	Let $(x, t) \in \ci \times \rr $ and $(n, \tau) \in \zz \times \rr$ be 
	the dual variables. Let $v$ be a positive even integer. Then there is a 
	constant $c_v > 0$ such that
%
%
\begin{equation}
	\label{1four-mult-est-L4*}
	\begin{split}
		\| \left( 1 + | \tau - n^v | 
		\right)^{-\frac{v+1}{4v}}
		\wh{f}\|_{L^2(\zz \times \rr)} \le c_v \|f \|_{L^{4/3}( \ci \times \rr)}.
	\end{split}
\end{equation}
%
%
\end{corollary}
%
Applying H\"{o}lder's inequality to the right hand side of
\eqref{13f}, we obtain the bound
%
%%
\begin{equation}
	\label{14f}
	\begin{split}
		\|\overset{\sim}{C_f} \|_{L^2(\ci \times \rr)} \|C_g \|_{L^4\left( \ci 
		\times \rr 
		\right)}. 
	\end{split}
\end{equation}
%
%%
By Plancherel we have
%
%%
%
%%
\begin{equation}
	\label{15f}
	\begin{split}
		\|\overset{\sim}{C_f} \|_{L^2(\ci \times \rr)}
		& \simeq \|c_f\|_{L^2(\zz \times \rr)}
		\\
		& = \|f \|_{\dot{X}^s}
	\end{split}
\end{equation}
%
%%
while \eqref{1four-mult-conseq} gives
%
%
\begin{equation}
	\label{16f}
	\begin{split}
		\|C_g \|_{L^4(\ci \times \rr)} \lesssim \|g\|_{\dot{X}^s}.
	\end{split}
\end{equation}
%
%
We conclude from \eqref{13f}-\eqref{16f} that
%
%
\begin{equation*}
	\begin{split}
		\| |n|^s \left( 1 + | \tau - n^3 | \right)^{-1/2} \wh{w_{fg}}(n, \tau) 
		 \|_{L^2(\zzdot \times \rr)}
		 \lesssim \|f\|_{\dot{X}^s} \|g\|_{\dot{X}^s}
	\end{split}
\end{equation*}
%
%
\subsection{Generalizing to arbitrary odd \texorpdfstring{$m >3$}{m > 3}.}
%
%
Since $$| \tau - n^m - \left( \tau_1 - n_1^m 
+ \tau_2 - n_2^m  \right ) | = | - n^m + n_1^m +
n_2^m|,$$ by and
the pigeonhole principle we must have one of the 
following.
%
%
\begin{align}
	\label{1pigeon-case-1-gen}
	& |\tau - n^m| \ge \frac{d(n_1, n_2)}{3} 	\\
		\label{1pigeon-case-2-gen}
		& | \tau_1 - n_1^m | \ge \frac{d(n_1, n_2)}{3},		\\
		\label{1pigeon-case-3-gen}
		& | \tau_2 - n_2^m | \ge \frac{d(n_1, n_2)}{3}.
	\end{align}
%
%
By the symmetry of the convolution, it will be enough to consider only
\eqref{1pigeon-case-1-gen} and \eqref{1pigeon-case-2-gen}.
%
%
%
\subsection{Case \ref{1pigeon-case-1-gen}}
Applying \cref{1lem:number-theory}, we have, for nonzero $ n, n_1, n_2 $
%
%%
\begin{equation}
	\label{1convo-deriv-bound-gen-case2}
	\begin{split}
		& \frac{|n|^{s+1}}{|n_1|^s 
		| n_2|^s}
		\times
		\frac{1}{(1 + | \tau -n^m |)^{1/2}}
		\\
		& \lesssim | n |^{s+1}| n_1 |^{-s}| n_2 |^{-s} \times | n
		|^{-\frac{c}{2}}| n_1 |^{-\frac{m-1-c}{4}}| n_2 |^{-\frac{m-1-c}{4}} 
		\\
		& = | n |^{\frac{2s+2 -c}{2}} | n_1 |^{\frac{-4s -m + 1+ c}{4}} | n_2
		|^\frac{-4s -m + 1+ c}{4}
		\\
		& \le 1, \qquad s \ge \frac{1-m}{4}.
	\end{split}  
\end{equation}
%
%
\begin{framed}
\begin{remark}
	\label{1rem:gen-s-val}
	 The last line follows from the following reasoning: Set $(2s + 2 -c) \le
0$ and $-4s -m +1 +c \le 0$. Then we want to find $c \ge 0$ such that $2s +2 \le c \le
4s + m-1$ or 
%
%
\begin{equation}
	\label{1algebra-ineq}
	\begin{split}
		2 \le c - 2s \le 2s + m-1.
	\end{split}
\end{equation}
%
%
Note that $c=0$ satisfies \eqref{1algebra-ineq} for $\frac{1-m}{4} \le s \le
-1$. Furthermore, $c = 4 + 4s$ satisfies \eqref{1algebra-ineq} for $s \ge -1$ ($c$ must be non-negative) and $m \ge 5$. 
\end{remark}
\end{framed}
%
Hence, from \eqref{1convo-est-starting-pnt} and
\eqref{1convo-deriv-bound-gen-case2},
we obtain 
\begin{equation}
	\label{1convo-est-starting-pnt-gen-case2}
	\begin{split}
		 & |n|^s \left( 1 + | \tau - n^m | \right)^{-1/2} | \wh{w_{fg}}\left( 
		n, \tau \right) |
		\\
		& \le \left( 1 + | \tau - n^m | \right)^{-1/2}
		\sum_{\substack{n_1 \neq0, n_2 \neq 0 \\n_1 + n_2 =n}} \int_{\tau_1 + \tau_2 = \tau}\frac{|n|^{s+1}}{|n_1|^s | n_2|^s} 
		\times \frac{c_f(n_1, \tau_1)}{\left( 1 + | \tau_1 - n_1^m | 
		\right)^{1/2}}
		\\
		& \times
		\frac{c_g(n_2, \tau_2)}{\left( 1 + | \tau_2 - n_2^m | 
		\right)^{1/2}}\ d \tau_1 d \tau_2
		\\
		& \lesssim \sum_{\substack{n_1 \neq0, n_2 \neq 0 \\n_1 + n_2 =n}} \int_{\tau_1 + \tau_2 = \tau}\frac{c_f(n_1, \tau_1)}{\left( 1 + | \tau_1 - n_1^m | 
		\right)^{1/2}} \times
		\frac{c_g(n_2, \tau_2)}{\left( 1 + | \tau_2 - n_2^m | 
		\right)^{1/2}}\ d \tau_1 d \tau_2, \qquad s \ge \frac{1-m}{4}
		\\
		& = \wh{{C_f} C_g}.
	\end{split}
\end{equation}
Therefore, from \eqref{1convo-est-starting-pnt-gen-case2}, Plancherel, and generalized 
H\"{o}lder, we obtain
%
%
\begin{equation}
	\label{1gen-holder-bound-case2}
	\begin{split}
		& \| |n|^s \left( 1 + | \tau - n^m | \right )^{-1/2}  \wh{w_{fg}}\left( 
		n, \tau \right) \|_{L^2(\ci \times \rr)}
		\\
		& \lesssim \|\wh{C_f C_g }\left( n, \tau \right) 
		\|_{L^2\left( \zzdot \times \rr \right)}
		\\
		& \simeq \|C_f C_g \|_{L^2\left( \ci \times \rr \right)}
		\\
		& \le \|C_f \|_{L^4(\ci \times \rr)} \|C_g \|_{L^4(\ci \times \rr)}.
	\end{split}
\end{equation}
%
From \cref{nlem:four-mult-est-L4}, we see that
%
%
\begin{equation}
	\label{1four-mult-conseq-gen-case2}
	\begin{split}
		\|C_\sigma\|_{L^4(\ci \times \rr)} 
		& \lesssim \|(1 + | \tau - n^m |)^{1/2} \wh{C_\sigma}
		\|_{L^2(\zz \times \rr)}
		\\
		& = \|c_{\sigma} \|_{L^2(\zz \times \rr)} 
		\\
		& = \|\sigma \|_{\dot{X}^s}. 
	\end{split}
\end{equation}
%
%
Applying this to \eqref{1gen-holder-bound-case2} we
conclude that
\begin{equation*}
	\begin{split}
		\| |n|^s \left( 1 + | \tau - n^m | \right ) ^{-1/2} \wh{w_{fg}}\left( 
		n, \tau \right) \|_{L^2(\zzdot \times \rr)}
		& \lesssim \|f\|_{\dot{X}^s} \|g\|_{\dot{X}^s}.
	\end{split}
\end{equation*}
%
%
%
\subsection{Case \ref{1pigeon-case-2-gen}}
We have for nonzero $ n, n_1, n_2 $
%
%%
\begin{equation}
	\label{1convo-deriv-bound-gen}
	\begin{split}
		& \frac{|n|^{s+1}}{|n_1|^s 
		| n_2|^s}
		\times
		\frac{1}{(1 + | \tau_1 -n_1^m |)^{1/2}}
		\\
		& \lesssim | n |^{s+1}| n_1 |^{-s}| n_2 |^{-s} \times | n
		|^{-\frac{c}{2}}| n_1 |^{-\frac{m-1-c}{4}}| n_2 |^{-\frac{m-1-c}{4}} 
		\\
		& = | n |^{\frac{2s+2 -c}{2}} | n_1 |^{\frac{-4s -m + 1+ c}{4}} | n_2
		|^\frac{-4s -m + 1+ c}{4}
		\\
		& \le 1, \qquad s \ge \frac{1-m}{4}.
	\end{split}  
\end{equation}
%
%
where the last line follows from \cref{1rem:gen-s-val}.
Hence, from \eqref{1convo-est-starting-pnt} and \eqref{1convo-deriv-bound-gen},
we obtain 
\begin{equation}
	\label{1convo-est-starting-pnt-gen}
	\begin{split}
		 & |n|^s \left( 1 + | \tau - n^m | \right)^{-1/2} | \wh{w_{fg}}\left( 
		n, \tau \right) |
		\\
		& \le \left( 1 + | \tau - n^m | \right)^{-1/2}
		\sum_{\substack{n_1 \neq0, n_2 \neq 0 \\n_1 + n_2 =n}} \int_{\tau_1 + \tau_2 = \tau}\frac{|n|^{s+1}}{|n_1|^s | n_2|^s} 
		\times \frac{c_f(n_1, \tau_1)}{\left( 1 + | \tau_1 - n_1^m | 
		\right)^{1/2}}
		\\
		& \times
		\frac{c_g(n_2, \tau_2)}{\left( 1 + | \tau_2 - n_2^m | 
		\right)^{1/2}}\ d \tau_1 d \tau_2
		\\
		& \lesssim \left( 1 + | \tau - n^m | \right)^{-1/2}
		\sum_{\substack{n_1 \neq0, n_2 \neq 0 \\n_1 + n_2 =n}} \int_{\tau_1 + \tau_2
		= \tau} c_f(n_1, \tau_1) \times
		\frac{c_g(n_2, \tau_2)}{\left( 1 + | \tau_2 - n_2^m | 
		\right)^{1/2}}\ d \tau_1 d \tau_2
		\\
		& = \left( 1 + | \tau - n^m | \right)^{-1/2}
\wh{\overset{\sim}{C_f} C_g}.
	\end{split}
\end{equation}
%
%%
%
%
Hence
%
%%
\begin{equation}
	\label{13f-gen}
	\begin{split}
		& \| |n|^s \left( 1 + | \tau - n^m | \right)^{-1/2} \wh{w_{fg}}(n, \tau) 
		\|_{L^2(\zzdot \times \rr)}
		\\
		& \lesssim \|\left( 1 + | \tau - n^m | \right)^{-1/2} 
		\wh{\overset{\sim}{C_f} C_g } \|_{L^2(\zzdot \times \rr)}
		\\
		& =  \|\left( 1 + | \tau - n^m | \right)^{-1/2} 
		\wh{\overset{\sim}{C_f} C_g } \|_{L^2(\zz \times \rr)}
		\\
		& \lesssim  \|\overset{\sim}{C_f} C_g  \|_{L^{4/3}(\ci \times \rr)}
	\end{split}
\end{equation}
%
%%
where the last step follows from \cref{1cor:four-mult-est-L4}.
%
%
Applying H\"{o}lder's inequality to the right hand side of
\eqref{13f-gen}, we obtain the bound
%
%%
\begin{equation}
	\label{14f-gen}
	\begin{split}
		\|\overset{\sim}{C_f} \|_{L^2(\ci \times \rr)} \|C_g \|_{L^4\left( \ci 
		\times \rr 
		\right)}. 
	\end{split}
\end{equation}
%
%%
By Plancherel we have
%
%%
%
%%
\begin{equation}
	\label{15f-gen}
	\begin{split}
		\|\overset{\sim}{C_f} \|_{L^2(\ci \times \rr)}
		& \simeq \|c_f\|_{L^2(\zz \times \rr)}
		\\
		& = \|f \|_{\dot{X}^s}
	\end{split}
\end{equation}
%
%%
while \cref{nlem:four-mult-est-L4} gives
%
%
\begin{equation}
	\label{1four-mult-conseq-gen}
	\begin{split}
		\|C_h\|_{L^4(\ci \times \rr)} 
		& \lesssim \|(1 + | \tau - n^m |)^{1/2} \wh{C_h}
		\|_{L^2(\zz \times \rr)}
		\\
		& = \|c_{h} \|_{L^2(\zz \times \rr)} 
		\\
		& = \|h \|_{\dot{X}^s}. 
	\end{split}
\end{equation}
%
%
We conclude from \eqref{13f-gen}-\eqref{1four-mult-conseq-gen} that
%
%
\begin{equation*}
	\begin{split}
		\| |n|^s \left( 1 + | \tau - n^m | \right)^{-1/2} \wh{w_{fg}}(n, \tau) 
		 \|_{L^2(\zzdot \times \rr)}
		 \lesssim \|f\|_{\dot{X}^s} \|g\|_{\dot{X}^s}.
	\end{split}
\end{equation*}
%
%
%
%
\section{Proof of Second Bilinear Estimate}
Recall that for the NLS, one obtains one trilinear estimate as a corollary of
another. Using this as motivation, let us see if we can obtain
\cref{1prop:bilinear-estimate2} as a corollary of
\cref{1prop:prim-bilin-est}. By
duality, it suffices to show that
%
%%
\begin{equation}
	\label{1duality-est}
	\begin{split}
	|	\sum_{n \in \zzdot}  |n|^{s}
		a_n \int_{\rr} \frac{|\wh{w_{fg}}(n, \tau)|}{1 
		+ | \tau - n^m |} \ d \tau | \lesssim \|f\|_{\dot{X}^s} \|g\|_{\dot{X}^s}
		\|a_n \|_{\ell^2}, \qquad s \ge -1/2.
	\end{split}
\end{equation}
%
%%
By the triangle inequality 
and Cauchy-Schwartz,
%
%%
\begin{equation}
	\label{11m}
	\begin{split}
		& | \sum_{n \in \zzdot} |n|^{s} a_n
		\int_{\rr}\frac{| \wh{w_{fg}}(n, \tau) |}{(1 + | \tau - n^m |)} \ d \tau |
		\\
		& \le \sum_{n \in \zzdot} \int_{\rr} \frac{| a_n |}{\left( 1 + 
		| \tau - n^m |
		\right)^{1/2 + \eta}} \times \frac{| n|^s  |
		\wh{w_{fg}}(n, \tau) |}{\left( 
		1 + | \tau - n^m | \right)^{1/2 - \eta}} \ d \tau
		\\
		& \le \left( \sum_{n \in \zzdot} | a_{n} |^2\int_{\rr} \frac{1}{\left( 1 + |
		\tau - n^m | \right)^{1 + 2 \eta}} \ d \tau  
		\right)^{1/2} 
		\left ( \sum_{n \in \zzdot}\int_{\rr} \frac{|n|^{2s} | \wh{w_{fg}}(n, \tau) 
		|^2}{\left( 1 + | \tau - n^m | \right)^{1 -2 \eta}}\ d \tau 
		\right)^{1/2}.
	\end{split}
\end{equation}
%
%%
Applying the change of variable $\tau - n^m
\mapsto \tau'$ we obtain  
%%
%
\begin{equation*}
	\begin{split}
		& \left( \sum_{n \in \zzdot} | a_{n} |^2\int_{\rr} \frac{1}{\left( 1 + | \tau -
		n^m | \right)^{1 + 2 \eta}} \ d \tau  
		\right)^{1/2} 
		\\
		& = \left ( \sum_{n \in \zzdot}
		| a_n |^2 
		\int_{\rr} \frac{1}{\left( 1 + | \tau' | \right)^{1 + 2 \eta}} \ d 
		\tau \right)^{1/2}
		\\
		& \simeq \|a_n\|_{\ell^2}, \qquad \eta >0.
		\end{split}
\end{equation*}
However, if we assume $\eta >0$, then
we cannot use \cref{1prop:prim-bilin-est} to bound
\begin{equation*}
	\begin{split}
		\left ( \sum_{n \in \zzdot}\int_{\rr} \frac{|n|^{2s} | \wh{w_{fg}}(n, \tau) 
		|^2}{\left( 1 + | \tau - n^m | \right)^{1 - 2\eta}}\ d \tau
		\right)^{1/2}. 
	\end{split}
\end{equation*}
%%
%%
\begin{framed}
\begin{remark}
Hence, unlike the NLS, we have not been able to obtain a second bilinear
estimate as a corollary from the first. Heuristically, this is due to the
derivative in nonlinearity, which is not present in the NLS nonlinearity.
However, one can obtain \eqref{1bilinear-estimate2} for $s>\frac{1-m}{4}$ as a
corollary of \cref{1prop:prim-bilin-est} by using the ideas
above and by modifying the proof of \cref{1prop:prim-bilin-est} slightly (i.e.,
showing that if $b = \frac{1}{2}^-$, then \eqref{1prim-bilin-est} holds for
$s\ge \frac{1-m}{4}^+$). To show that \eqref{1bilinear-estimate2} holds for the
case $s=1/2$, we will have to resort to Kenig-Ponce-Vega~\cite{Kenig:1996yn} techniques.
\end{remark}
\end{framed}
%
%
Proceeding, note that by duality, to prove \cref{1prop:bilinear-estimate2} it
suffices to show \eqref{1duality-est} for $s \ge \frac{1-m}{4}$. By the symmetry of the convolution, we
consider only cases \eqref{1pigeon-case-1} and \eqref{1pigeon-case-2}.
%
%
\subsection{Case \ref{1pigeon-case-1}.} Assume $s \ge \frac{1-m}{4}$. Then from 
\eqref{1convo-est-starting-pnt-gen-case2} we have
%
%
\begin{equation}
	\label{1gen-smoothing-ineq}
	\begin{split}
		& |n|^s \left( 1 + | \tau - n^m | \right)^{-1/2} | 
		\wh{w_{fg}}(n, \tau) | 
		\\
		& \lesssim \sum_{\substack{n_1 \neq0, n_2 \neq 0 \\n_1 + n_2 =n}} \int_{\tau_1 + \tau_2 = \tau}\frac{c_f(n_1, \tau_1)}{\left( 1 + | 
		\tau_1 -  n_1^m| \right)^{1/2}}
		\times \frac{c_g\left( n_2, \tau_2\right)}{\left( 1 + | \tau_2 -n_2^m|
		\right)^{1/2}}.
	\end{split}
\end{equation}
%
%
From the triangle inequality and \eqref{1gen-smoothing-ineq}, we have
%
%
\begin{equation*}
	\begin{split}
	 |\eqref{1duality-est}|
	& \lesssim \sum_{n \in \zzdot} |a_{n}| \int_{\rr} \sum_{\substack{n_1 \neq 0, n_2 \neq 0
		\\ n_1 +n_2 =n}} \int_{\tau_1 + \tau_2 = \tau} c_f(n_1, \tau_1)
		c_g(n_2, \tau_2)
		\\
		& \times \frac{1}{(1 + | \tau - n^m |)^{1/2}(1 + |
		\tau_{1}-n_{1}^m |)^{1/2}(1 + | \tau-n_{2}^m |^{1/2})} d \tau_1 d \tau_2
		d \tau
	\end{split}
\end{equation*}
%
%
which by Cauchy-Schwartz is bounded by
%
%
\begin{equation}
	\label{110g}
	\begin{split}
		& \sum_{n \in \zzdot} |a_n| \int_{\rr} \left(  \sum_{\substack{n_1 \neq 0, n_2
		\neq 0 \\n_1 +n_2 =n}} \int_{\tau_1 + \tau_2 = \tau} c_{f}^{2}(n_1, \tau_1)
		c_{g}^{2} (n_2, \tau_2) d \tau_1 d \tau_2 \right)^{1/2} 
		\\
		& \times \left( \sum_{\substack{n_1 \neq 0, n_2 \neq 0 \\n_1 +n_2 =n}}
		\int_{\tau_1 + \tau_2 = \tau} \frac{1}{(1 + | \tau - n^m |)(1 + | \tau_{1}-n_{1}^m |)(1 + |
		\tau_2 -n_{2}^m |)} d \tau_1 d \tau_2
		\right)^{1/2} d \tau.
	\end{split}
\end{equation}
%
%
Applying Cauchy-Schwartz again, \eqref{110g} is bounded by
%
%
\begin{align}
	\notag
		& \|\left( \sum_{\substack{n_1 \neq 0, n_2 \neq 0 \\n_1 +n_2 =n}}\int_{\tau_1 + \tau_2 = \tau} c_{f}^{2}(n_1, \tau_1)
		c_{g}^{2} (n_2, \tau_2) d \tau_1 d \tau_2 \right)^{1/2} \|_{L^{2}(\zz \times
		\rr)}
		\\
		\notag
		& \times  \|a_{n}
		\left( \sum_{\substack{n_1 \neq 0, n_2 \neq 0 \\n_1 +n_2
		=n}}\int_{\tau_1 + \tau_2 = \tau} \frac{1}{ (1 + | \tau - n^m |)(1 + |
		\tau_{1}-n_{1}^m |)(1 + | \tau_2 -n_{2}^m |)} d \tau_1 d \tau_2
		\right)^{1/2} \|_{L^2(\zz \times \rr)}
		\\
		\notag
		& = \|f\|_{\dot{X}^s} \|g\|_{\dot{X}^s}
		\\
		\label{1holder-term}
		& \times 
		\|a_{n}
		\left( \sum_{\substack{n_1 \neq 0, n_2 \neq 0 \\n_1 +n_2
		=n}}\int_{\tau_1 + \tau_2 = \tau} \frac{1}{ (1 + | \tau - n^m |)(1 + |
		\tau_{1}-n_{1}^m |)(1 + | \tau_2 -n_{2}^m |)} d \tau_1 d \tau_2
		\right)^{1/2} \|_{L^2(\zz \times \rr)}.
\end{align}
%
Applying H{\"o}lder then gives
%
%
\begin{equation*}
	\begin{split}
		& \eqref{1holder-term}
		 \le \| a_{n} \|_{\ell^2}
		\\
		& \times \left( \sup_{n \neq 0} \int_{\rr}
		\sum_{\substack{n_1 \neq 0, n_2 \neq 0 \\n_1 +n_2 =n}} \int_{\tau_1 + \tau_2
		= \tau} \frac{1}{ (1 + | \tau - n^m |)(1 + |
		\tau_{1}-n_{1}^m |)(1 + | \tau_2 -n_{2}^m |)} d \tau_1 d \tau_2 d \tau
		\right)^{1/2}.
	\end{split}
\end{equation*}
%
%
Hence, to complete the proof for case \eqref{1pigeon-case-1}, it will be enough
to show that 
%
%
%
%
\begin{equation*}
	\begin{split}
		 \sup_{n \neq 0} \int_{\rr}
		\sum_{\substack{n_1 \neq 0, n_2 \neq 0 \\n_1 +n_2 =n}} \int_{\tau_1 + \tau_2
		= \tau} \frac{1}{ (1 + | \tau - n^m |)(1 + |
		\tau_{1}-n_{1}^m |)(1 + | \tau_2 -n_{2}^m |)} d \tau_1 d \tau_2 d \tau <\infty
	\end{split}
\end{equation*}
%
%
or, equivalently, that
%
%
\begin{equation}
	\label{112g}
	\begin{split}
		\sup_{n \neq 0} \sum_{\substack{n_1 \neq 0, n_2 \neq 0 \\n_1 +n_2 =n}} \int_{\rr}
		\int_\rr  \frac{1}{(1 + | \tau - n^m |)(1 + | \tau_1 - n_{1}^m |)(1 + | \tau - \tau_1 -
		n_2^m |)} d \tau_1 d \tau < \infty.
	\end{split}
\end{equation}
%
%
Following Kenig~\cite{Kenig:1996yn}, we now need the following Calculus lemma.
%
%
%%%%%%%%%%%%%%%%%%%%%%%%%%%%%%%%%%%%%%%%%%%%%%%%%%%%%
%
%
%				 Calculus Lemma
%
%
%%%%%%%%%%%%%%%%%%%%%%%%%%%%%%%%%%%%%%%%%%%%%%%%%%%%%
%
%
\begin{lemma}
	\label{1lem:calc}
 %
 %
 \begin{equation}
	 \label{1calc}
	 \begin{split}
		 \int_{\rr} \frac{1}{(1 + | \theta |)(1 + | a - \theta |)} d \theta \lesssim
		 \frac{\log(2 + | a |)}{1 + | a |}.
	 \end{split}
 \end{equation}
 %
 %
 \end{lemma}
%
%
Applying the lemma with $\theta = \tau_1 - n_1^m$ and $a = \tau - n_1^m -
n_2^m$, we see that
%
%
\begin{equation*}
	\begin{split}
	\int_{\rr}
		\int_\rr  \frac{1}{(1 + | \tau - n^m |)(1 + | \tau - \tau_1 -
		n_2^m |)} d \tau_1 d \tau \lesssim \frac{\log(2 + | \tau - n_{1}^m -
		n_{2}^m |)}{1 + | \tau - n_{1}^m - n_{2}^m |}.
	\end{split}
\end{equation*}
%
%
%
Hence, the left hand side of \eqref{112g} is bounded by
%
\begin{equation*}
	\begin{split}
		\sup_{n \neq 0} \sum_{\substack{n_1 \neq 0, n_2 \neq 0 \\n_1 +n_2 =n}}
		\int_{\rr} \frac{\log(2 + | \tau - n_{1}^m -
		n_{2}^m |)}{(1 + | \tau - n_{1}^m - n_{2}^m |)(1 + | \tau - n^m |)}
		d \tau	
	\end{split}
\end{equation*}
%
%
or, equivalently, by
%
%
\begin{equation}
	\label{113g}
	\begin{split}
		\sup_{n \neq 0} \sum_{n_1 \neq 0} \int_{\rr} \frac{\log(2 + | \tau -
		n_{1}^m - (n - n_1)^m |)}{(1 + | \tau - n_{1}^m - (n - n_{1})^m |)(1
		+ | \tau - n^m |)} d \tau.
	\end{split}
\end{equation}
%
%
%
Now, note that 
$$ |\tau - n^m| \ge \frac{d_m(n_1, n_2)}{3} \gtrsim
| n_1 n_2 |^{(m-1)/2},$$ where the right hand side follows from
\cref{1lem:number-theory} with $c=0$. Hence, \eqref{113g} is bounded by a constant times
%
%
%
%
\begin{equation}
	\label{114g}
	\begin{split}
		& \sup_{n \neq 0} \sum_{n_1 \neq 0}
		\frac{1}{| n_1 n_2 |^{(\frac{1}{2} + \eta)(m-1)/2}} \int_{\rr} \frac{\log(2 + | \tau - n_{1}^m -
		(n - n_1)^m |)}{(1 + | \tau - n_{1}^m - (n - n_{1})^m
		|)(1 + | \tau - n^m |)^{\frac{1}{2}-\eta}}
		d \tau
		\\
		& \le \sup_{n \neq 0} \sum_{n_1 \neq 0}
		\frac{1}{| n_1 n_2 |^{(\frac{1}{2} + \eta)(m-1)/2}} 	\\
		& \times \sup_{n \neq 0} \sum_{n_1 \neq 0}
		\int_{\rr} \frac{\log(2 + | \tau
		- n_{1}^m - (n - n_1)^m |)}{(1 + | \tau - n_{1}^m - (n - n_{1})^m
		|)(1 + | \tau - n^m |)^{\frac{1}{2}-\eta}}
		d \tau
	\end{split}
\end{equation}
%
for any $\eta >0$.
Observe that for the first sum, the supremum is attained at $n=1$.
%
%
\begin{framed}
\begin{remark}
To see this,
write $n_1 n_2 = n_1(n-n_1) \doteq f(n)$ and note that $|f(n)|$ has a global
minimum at $n=n_1$. Furthermore, $f(n)$ is strictly
increasing (if $n_1>0$) or strictly decreasing (if $n_1 <0$).
\end{remark}
\end{framed}
%
%
%
But then $n_2 = 1 - n_1$, and so $| n_1 n_2 | \gtrsim | n_1 |^2$. Furthermore, we know that 
for any $\ee > 0$, we have $\log (2 + | a |) \le c_{\ee}(1 + | a
|)^{\ee}$. Hence, we bound \eqref{114g} by
%
%
%
%
\begin{equation*}
	\begin{split}
		c_{\ee}  \sum_{n_1 \neq 0} \frac{1}{|n_1|^{(\frac{1}{2} + \eta)(m-1)}}
		\sup_{n \neq 0} \sum_{n_1 \neq 0} \int_{\rr} \frac{1}{(1 +
		| \tau - n_{1}^m - (n - n_{1})^m |)^{1- \ee}(1 + | \tau - n^m
		|)^{\frac{1}{2}-\eta}} d \tau
	\end{split}
\end{equation*}
%
%
%
which due to the estimate
%
%
\begin{equation}
	\label{116g}
	\begin{split}
		(1 + | \tau - n^m |)
		& = 1 + \frac{1}{4}| \tau - n^m | + \frac{3}{4}| \tau - n^m |
		\\
		& \ge 1 + \frac{1}{4}| \tau - n^m | + \frac{3}{4} \times
		\frac{1}{3}d(n_1,n_2)
		\\
		& = 1 + \frac{1}{4}| \tau - n^m | + \frac{1}{4}| n^m - n_1^m - (n -
		n_1)^m |
		\\
		& \ge \frac{1}{4}| \tau - n_1^m - (n - n_1)^m |
	\end{split}
\end{equation}
%
%
is bounded by
%
%
\begin{equation}
	\label{115g}
	\begin{split}
		& 4 c_{\ee} \sum_{n_1 \neq 0} \frac{1}{| n_1 |^{(\frac{1}{2} +
		\eta)(m-1)}} 	\sup_{n \neq 0} \sum_{n_1 \neq 0}	\int_{\rr} \frac{1}{(1 + |
		\tau - n_{1}^m - (n - n_{1})^m |)^{\frac{3}{2}-\ee - \eta}} d \tau
		\\
		& \lesssim \sum_{n_! \neq 0} \frac{1}{| n_1 |^{(\frac{1}{2} +
		\eta)(m-1)}} 		\qquad (\text{for} \ \eta \ \text{sufficiently small})
		\\
		& < \infty, \qquad (\text{since} \ m \ge 3). \qquad \qed
	\end{split}
\end{equation}
%
%
%
%
%
%
\subsection{Case \ref{1pigeon-case-2}.} From the triangle inequality and
\eqref{1gen-smoothing-ineq}, we see that for $s \ge \frac{1-m}{4}$ we have
%
%
%
%
%
%
\begin{equation}
	\begin{split}
		& | \sum_{n \neq 0} \int_{\rr} a_n |n|^s \left( 1 + | \tau - n^m | \right)^{-1} | 
		\wh{w_{fg}}(n, \tau) | d \tau |
		\\
		& \lesssim \sum_{n \neq 0}  \int_{\rr} |a_{n}| (1+ | \tau - n^m |)^{-1} \wh{\overset{\sim}{C_f} C_g} d
		\tau
	\\	
	& = \sum_{n \neq 0} \int_{\rr} |a_{n}| (1+ | \tau - n^m |)^{-5/8} (1 + | \tau - n^m
	|)^{-3/8} \wh{\overset{\sim}{C_f} C_g} d
		\tau
		\\
		& \le \|a_{n} (1 + | \tau - n^m |)^{-5/8}\|_{L^2(\zz \times \rr)}  \| (1 +
		| \tau - n^m |)^{-3/8} \wh{\overset{\sim}{C_f} C_g}  \|_{L^2(\zz \times
		\rr)}
		\end{split}
\end{equation}
%
%
where the last step follows from Cauchy-Schwartz. Applying the change of
variable $\tau - n^{m } \mapsto \tau'$ we obtain  %
%%
\begin{equation*}
	\begin{split}
		\|a_{n} (1 + | \tau - n^m |)^{-5/8}\|_{L^2(\zz \times \rr)} 
		& = \left( \sum_{n \in \zz} | a_{n} |^2\int_{\rr} \frac{1}{\left( 1 + | \tau -
		n^{m } | \right)^{5/4}} \ d \tau  
		\right)^{1/2} 
		\\
		& = \left ( \sum_{n \in \zz}
		| a_n |^2 
		\int_{\rr} \frac{1}{\left( 1 + | \tau' | \right)^{5/4}} \ d 
		\tau' \right)^{1/2}
		\\
		& \simeq \|a_n\|_{\ell^2}
		\end{split}
\end{equation*}
%
%
%
while \eqref{13f-gen}-\eqref{1four-mult-conseq-gen} yields the bound
%
%
\begin{equation*}
	\begin{split}
	\| (1 + | \tau - n^m |)^{-3/8} \wh{\overset{\sim}{C_f} C_g}  \|_{L^2(\zz
	\times \rr)} \lesssim \|f\|_{\dot{X}^s} \|g\|_{\dot{X}^s}
	\end{split}
\end{equation*}
%
%
completing the proof. \qquad \qedsymbol
%
%
\section{Proof of Ill-Posedness}
We adapt an argument from \cite{Burq:2002xd}. For $s<1/2$, $m \in \{4, 8, 12, \dots\}$, set
%
%
%
%
\begin{equation}
	\label{1ill-soln}
	\begin{split}
		u_{n}(x,t)=\frac{1}{2}n^{-s}e^{it\left( n^{m}+\frac{1}{4}n^{-2s+1}
		\right)}e^{inx}.
	\end{split}
\end{equation}
%
%
Then
%
%
\begin{equation*}
	\begin{split}
		& i \p_t u_{n}
		= -\frac{1}{2}n^{-s}\left( n^{m}+\frac{1}{4}n^{-2s+1} \right)e^{it\left(
		n^{m}+ \frac{1}{4}n^{-2s+1} \right)}e^{inx},
		\\
		& \p_x^{m}u_{n}  = \frac{1}{2}n^{-s+m}e^{it\left(
		n^{m}+\frac{1}{4}n^{-2s+1} \right)}e^{inx},
		\\
		& \p_x (| u_{n} |^{2}u_{n})  = \frac{1}{8}n^{-3s+1}e^{it\left(
		n^{m}+\frac{1}{4}n^{-2s+1} \right)}e^{inx}.
	\end{split}
\end{equation*}
%
%
Hence,
%
%
\begin{equation*}
	\begin{split}
		i \p_t u_{n} + \p_x^{m}u_{n} + \p_x(| u_{n} |^{2} u_{n})
		=0.
	\end{split}
\end{equation*}
%
%
Therefore, $u_{n}(x,t)$ solves the initial value problem
%
%
\begin{gather*}
	\begin{split}
		i \p_t u + \p_x^m u + \p_x (| u |^{2}u) = 0,
		\\
		u(x,0) = \frac{1}{2}n^{-s}e^{inx}.
	\end{split}
\end{gather*}
%
%
Next, we show that $u_{n}(\cdot, t) \in \dot{H}^{s}(\ci)$ for all $t \in \rr$.
First, we compute
%
%
\begin{equation*}
	\begin{split}
		\|e^{inx}\|_{H^{s}(\ci)}
		& =  \left[ \sum_{\xi \in \zz} | \xi ^{2s}
    | \wh{e^{in(\cdot)}}(\xi) |^{2} \right]^{1/2}
		\\
		& =  \left[ \sum_{\xi \in \zz} | \xi |^{2s} 
		\int_{\ci}e^{ix(n- \xi)}dx |^{2}\right]^{1/2}.
	\end{split}
\end{equation*}
%
%
Noting that
%
\begin{equation*}
	\begin{split}
		\int_{\ci}e^{ix(n - \xi)}dx =
		\begin{cases}0, \qquad & n \neq \xi
			\\
			2 \pi, \qquad & n = \xi
		\end{cases}
	\end{split}
\end{equation*}
%
%
we obtain
%
%
\begin{equation}
	\label{1oscill-bound}
	\begin{split}
		\|e^{inx}\|_{H^{s}(\ci)} & = 2 \pi| n |^{s}
	\end{split}
\end{equation}
%
%
and so
%
%
\begin{equation*}
	\begin{split}
		\|u_{n}(\cdot, t)\|_{H^s{(\ci)}} = \frac{1}{2}|n|^{-s}
		\|e^{inx}\|_{H^{s}(\ci)} \le \pi.
	\end{split}
\end{equation*}
%
%
Next, let
%
%
\begin{equation*}
	\begin{split}
		u_{k_{n}}(x,t) = k_{n}n^{-s}e^{it\left( n^{m} + k_{n}^2 n^{-2s+1}
		\right)}e^{inx}.
	\end{split}
\end{equation*}
%
%
Following our preceding computations, it is easy to show that $u_{n, k_{n}}$ is a solution to the IVP
%
%
\begin{equation}
	\label{1family-IVP}
	\begin{split}
		i\p_t u + \p_x^{m} + | u |^{2}u = 0,
		\\
		u(x,0) = k_{n}n^{-s}e^{inx}
	\end{split}
\end{equation}
%
%
and satisfies 
%
%
\begin{equation*}
	\begin{split}
		\|u_{n, k_{n}}(\cdot, t)\|_{H^{s}(\ci)} \le 2 \pi k_{n}.
	\end{split}
\end{equation*}
%
%
Furthermore, choosing $\{k_{n}\}_{n} \subset (0, 1/2)$ to be a family of
rational numbers converging to $k =1/2$, we have
%
%
\begin{equation*}
	\begin{split}
		\|u(x,0) - u_{n, k_{n}}(x, 0) \|_{H^s(\ci)} 
		& =
		\|\frac{1}{2}n^{-s}e^{inx} - k_{n}n^{-s}e^{inx} \|_{H^s(\ci)}
		\\
		& = | n |^{-s} \|e^{inx}(\frac{1}{2} - k_{n})\|_{H^s(\ci)}
		\\
		& = 2 \pi |\frac{1}{2} - k_{n}| \to 0
	\end{split}
\end{equation*}
%
%
and
%
%
\begin{equation*}
	\begin{split}
		& \|u_{n}(\cdot, t) - u_{n, k_{n}}(\cdot, t) \|_{H^{s}(\ci)}
		\\
		& = \|\frac{1}{2}n^{-s}e^{it\left( n^{m} + \frac{1}{4}n^{-2s+1}
		\right)}e^{inx} - k_{n}n^{-s}e^{it\left( n^{m} + k_{n}^{2}n^{-2s+1}
		\right)}e^{inx} \|_{H^{s}(\ci)}
		\\
		& = | n |^{-s} \|e^{it\left( n^{m} + \frac{1}{4}n^{-2s+1}
		\right)}e^{inx}\left( \frac{1}{2} - k_{n}e^{it\left(
		k_{n}^{2}n^{-2s+1}-\frac{1}{4}n^{-2s+1} \right)} \right)\|_{H^{s}(\ci)}
		\\
		& = | n^{-s} |\|e^{inx}\left( \frac{1}{2} - k_{n}e^{it\left(
		k_{n}^{2}n^{-2s+1} - \frac{1}{4}n^{-2s+1}
		\right)} \right) \|_{H^{s}(\ci)}
		\\
		& = 2 \pi
    | \frac{1}{2} - k_{n}e^{itn^{-2s+1}\left( k_{n}^{2}- \frac{1}{4}\right)} |
	\end{split}
\end{equation*}
%
%
where the last step follows from \eqref{1oscill-bound}. Hence, in order for uniform continuity of the flow map to hold, we must have
%
%
\begin{equation*}
	\begin{split}
		\lim_{n \to \infty}  k_{n} e^{itn^{-2s+1}\left( k_{n}^{2} -
		\frac{1}{4} \right)}  = \frac{1}{2}.
	\end{split}
\end{equation*}
%
%
But setting $k_{n} = \left( \frac{1}{4} + n^{2s-1 + \ee} \right)^{1/2}$ where
$0 < \ee < 1-2s$, we see that $k_n \to 1/2$ while
%
%
\begin{equation*}
	\begin{split}
		\lim_{n \to \infty} k_{n} e^{itn^{-2s+1}\left( k_{n}^{2} - \frac{1}{4}
		\right)} = \lim_{n \to \infty} k_{n} e^{itn^{\ee}} \neq \frac{1}{2}.
	\end{split}
\end{equation*}
%
\begin{framed}
\begin{remark}
	Notice that our choice for $\ee$ is possible only when $s < 1/2$.
	It is here that
	our a priori assumption of $s < 1/2$ plays a crucial role.
\end{remark}
\end{framed}

%
In fact, the above limit does not converge at all. This concludes the proof for
the case $m \in \{4, 8, 12, \dots \}$. For the case $m \in \{2, 6, 10, \dots \}$, we take
%
%
\begin{equation*}
	\begin{split}
		u_{n}(x,t) = \frac{1}{2}n^{-s}e^{it\left( -n^{m} + \frac{1}{4}n^{-2s+1}
		\right)}e^{inx},
		\\ u_{n, k_{n}}(x,t) = k_{n}n^{-s}e^{it\left( -n^{m} + k_{n}^{2}n^{-2s+1}
		\right)}e^{inx} 
	\end{split}
\end{equation*}
and then repeat the above arguments. \qquad \qedsymbol
%
%
\begin{framed}
\begin{remark}
	Note that this result implies that it will be impossible to use a Picard
	iteration type argument to prove existence and uniqueness of solutions to the
	dNLS IVP for $s<1/2$, since this technique would imply uniform
	continuity of the flow map.
\end{remark}
\end{framed}
%
%
\section{Proofs of Lemmas and Estimates}
\begin{proof}[Proof of \cref{1lem:cutoff-loc-soln}]
%
%
\begin{equation*}
	\begin{split}
		\lim_{t_{n} \to t} \|u(\cdot, t) - u(\cdot, t_{n})\|_{\dot{H}^s(\ci)} 
		& = \lim_{t_{n} \to t} \|\psi(t) u(\cdot, t) - \psi(t_n) u(\cdot,
		t_{n})\|_{\dot{H}^s(\ci)} 
		\\
		& = \lim_{t_n \to t} \left[ \sum_{n \in \zzdot}| n |
		^{2s} | \psi(t)  \wh{u}(n, t) - \psi(t_n) \wh{ u}(n, t_n) |^2 \right]^{1/2}
		\\
		& = \lim_{t_n \to t} \left[ \sum_{n \in \zzdot} | n |^{2s} | \int_{\rr} (e^{it \tau} - e^{it_{n} \tau}) \wh{\psi u}(n,
		\tau) d \tau |^2 \right]^{1/2}.
	\end{split}
\end{equation*}
		It is clear that
		%
		%
		\begin{equation*}
			\begin{split}
				| n |
				^{2s} | \int_{\rr} (e^{it \tau} - e^{it_{n}\tau}) \wh{\psi u}(n, \tau) d \tau |^2 
		& \le 4  | n |^{2s} \left ( \int_{\rr} |\wh{\psi u}(n, \tau)| d \tau
		\right )^2 
	\end{split}
\end{equation*}
and 
%
%
\begin{equation*}
	\begin{split}
 \sum_{n \in \zzdot} | n |^{2s} \left ( \int_{\rr} |\wh{\psi u}(n, \tau)| d \tau
		\right ) ^2 
		& = \| |n |^s \wh{\psi u}\|_{\dot{\ell}_n^2 L_\tau^1}
		\\
		& \le \|\psi u \|_{Y^s}^2 
	\end{split}
\end{equation*}
which is bounded by assumption.
Applying dominated convergence completes the proof. 
\end{proof}
%
%
%\begin{proof}[Proof of \cref{1lem:schwartz-mult}]
%Note that
%%
%%
%\begin{equation*}
	%\begin{split}
		%\wh{\psi f}\left( n, \tau \right)
		%& = \wh{\psi}(\cdot) * \wh{f}(n,
		%\cdot)(\tau)
		%= \int_\rr \wh{\psi}(\tau_1) \wh{f} \left( n, \tau - \tau_1 \right) 
		%d\tau_1
	%\end{split}
%\end{equation*}
%%
%%
%and hence
%%
%%
%\begin{equation}
	%\label{11b}
	%\begin{split}
		%\|\psi f\|_{\dot{X}^s} 
		%& = \left( \sum_{n \in \zzdot} |n|^{2s} \int_\rr \left( 1 + | \tau -
		%n^{m} | \right) | \int_\rr \wh{\psi}(\tau_1) \wh{f}\left( n, \tau -
		%\tau_1
		%\right)  d \tau_1 d \tau |^2 \right)^{1/2}
		%\\
		%& \le \left( \sum_{n \in \zzdot} |n|^{2s} \int_\rr \left( 1 + | \tau -
		%n^{m }
		%|
		%\right) \left( \int_\rr \wh{\psi}\left( \tau_1 \right) \wh{f}\left( n,
		%\tau - \tau_1
		%\right)  d \tau_1 d \tau \right)^2 \right)^{1/2}.
	%\end{split}
%\end{equation}
%%
%%
%Using the relation
%%
%%
%\begin{equation*}
	%\begin{split}
		%1 + | \tau - n^{m } |
		%& = 1 + | \tau + \tau_1 - n^{m} |
		%\\
		%& \le 1 + | \tau_1 | + | \tau - \tau_1 - n^{m} |
		%\\
		%& \le \left( 1 + | \tau_1 | \right)\left( 1 + | \tau - \tau_1 -
		%n^{m} | \right)
	%\end{split}
%\end{equation*}
%%
%%
%we obtain
%%
%%
%\begin{equation*}
	%\begin{split}
		%\eqref{11b}
		%& \le \left( \sum_{n \in \zzdot} |n|^{2s} \right.
		%\\
		%& \times \left . \int_\rr \left(
		%\int_\rr \left( 1 + | \tau_1 | \right)^{1/2} | \wh{\psi}(\tau_1) |
		%\left( 1 + | \tau - \tau_1 - n^{m} | \right)^{1/2} \wh{f}\left( n, \tau
		%- \tau_1
		%\right)d \tau_1
		%\right)^2 d \tau \right)^{1/2}
	%\end{split}
%\end{equation*}
%%
%%
%which by Minkowski's inequality is bounded by
%%
%%
%\begin{equation}
	%\label{12b}
	%\begin{split}
		%& \left( \sum_{n \in \zzdot} |n|^{2s}  \right.
		%\\
		%& \times \left. \left( \int_\rr \left[ \int_\rr
		%\left( 1 + | \tau_{1} | \right) | \wh{\psi}(\tau_1) |^2 \left( 1 + |
		%\tau - \tau_1 - n^{m} |
		%\right) | \wh{f}\left( n, \tau - \tau_1 \right) |^2 d \tau_1 
		%\right]^{1/2} d \tau \right)^2 \right)^{1/2}.
	%\end{split}
%\end{equation}
%%
%%
%Using the change of variable $\tau - \tau_1 \to \lambda$ gives
%%
%%
%\begin{equation*}
	%\begin{split}
		%\eqref{12b}
		%& = \left( \sum_{n \in \zzdot} |n|^{2s}\right.
		%\\
		%& \times \left.  \left( \int_\rr \left[
		%\int_\rr \left( 1 + | \tau_1 | \right) | \wh{\psi}\left( \tau_1
		%\right) |^2 \left( 1 + | \lambda - n^{m} | \right) | \wh{f} \left( n,
		%\lambda
		%\right)|^2 d \tau_1 \right]^{1/2} d \lambda \right)^2 \right)^{1/2}
		%\\
		%& =  \left( \sum_{n \in \zzdot} |n|^{2s} \right.
		%\\
		%& \times \left. \left( \int_\rr \left( 1 + |
		%\tau_1 |
		%\right)^{1/2} | \wh{\psi}(\tau_1) | d \tau_1 \left[ \int_\rr \left( 1 + |
		%\lambda - n^{m} |
		%\right) | \wh{f}\left( n, \lambda \right) |^2 d \lambda \right]^{1/2}
		%\right)^2 \right)^{1/2}
		%\\
		%& = c_{\psi} \left( \sum_{n \in \zzdot} |n|^{2s} \left( \left[ \int_\rr
		%\left( 1 + | \lambda - n^{m} | \right) | \wh{f}\left( n, \lambda
		%\right) |^2 d \lambda
		%\right]^{\cancel{1/2}} \right)^{\cancel{2}} \right)^{1/2}
		%\\
		%& = c_{\psi} \|f\|_{\dot{X}^s},
	%\end{split}
%\end{equation*}
%%
%%
%concluding the proof. 
%\end{proof}
%
%
%
%
%
%
\begin{proof}[Proof of \cref{1lem:number-theory1}]
First note that
%
\begin{equation*}
		| - n^m + n_1^m + n_2^m|
		 = 3 | n | |n_1 | |n_2 |.
\end{equation*}
%
%
Hence, it will be enough to show that for $c \ge 0$
%
%
\begin{equation*}
	\begin{split}
		| n | |n_1 | |n_2 | \gtrsim | n |^{\frac{2 + c}{2}}| n_1
		|^{\frac{2-c}{2}}| n_2 |^{\frac{2-c}{2}}
	\end{split}
\end{equation*}
%
%
or, dividing through on both sides by $|n| | n_1 | | n_2 |$ and rearranging terms
%
%
\begin{equation*}
	\begin{split}
		| n |^{c/2} \lesssim | n_1 |^{c/2} | n_2 |^{c/2}.
	\end{split}
\end{equation*}
%
%
But
%
%
\begin{equation*}
	\begin{split}
		| n |^{c/2} &= | n_1 + n_2 |^{c/2}
		\\
		& \le (| n_1 | + |n_2|)^{c/2} 
		\\
		& \le (2\max\{|
		n_1 |, | n_2 |)^{c/2}
		\\
		& \le (2|
		n_1 | | n_2 |)^{c/2}
		\\
		& = 2^{c/2} | n_1 |^{c/2} | n_2 |^{c/2}
	\end{split}
\end{equation*}
completing the proof.
\end{proof}
%
%
%
%
\begin{proof}[Proof of \cref{1lem:number-theory}] Define
%
\begin{equation*}
	\begin{split}
		| - n^{m} + n_1^{m} + n_2^{m }|
		& = | n_{1}^{m} - n^{m} + (n-n_{1})^{m}| 
		\\
		& \doteq f(n).
	\end{split}
\end{equation*}
%
%
For fixed $n_1$, the absolute minima
of $f(n)$ occurs at $n = 1+n_{1}$ ($n = n_1$ is not available by assumption). Next, note that
%
%
\begin{equation*}
	\begin{split}
		f(1+ n_{1}) = | n_{1}^{m} - (1 + n_{1})^m + 1 |
		& = | (1 + n_{1} )^{m} - n_{1}^{m} -1 |.
	\end{split}
\end{equation*}
We now seek a lower bound for the right hand side. By symmetry we may assume
$n_1 >0$ without loss of generality.
%
%
\begin{framed}
\begin{remark}
	By the term ``symmetry'', we mean that
	\begin{equation*}
	\begin{split}
	| [1 + (-n_1)]^m - (-n_1)^m -1 |
	& = | (1 - n_1)^m + n_1^m -1 |
	\\
	& = | (1 + p_1)^m + (-p_1)^m -1 |, \qquad p_1 = -n_1
	\\
	& = | (1 + p_1)^m - (p_1)^m -1 |.
	\end{split}
\end{equation*}
%
%
\end{remark}
\end{framed}
%
%
Then 
%
%
\begin{equation*}
	\begin{split}
	| (1 + n_{1} )^{m} - n_{1}^{m} -1 |
	& = | \sum_{1 \le k \le m-1} c_{k} n_1^{k}|, \qquad \{c_k\} \in
	\mathbb{N}\setminus 0
	 \\
	 & = \sum_{1 \le k \le m-1} c_{k} n_1^{k}
	 \\
	 & \ge c_{m-1}  n_1^{m-1}
	 \\
	 & = c_{m-1}  n_1^{c} n_1^{m-1-c}
	 \\
	 & \gtrsim (1 + n_1)^{c}  n_1^{m-1-c}
	 \\
	 & = n^{c} n_1^{m-1-c}. 
 \end{split}
\end{equation*}
%
%
Since we assumed $n_1 >0$ without loss of generality, it follows that 
%
%
\begin{equation*}
	\begin{split}
		f(n) \gtrsim |n|^{c} | n_1 |^{m-1-c}. 
	\end{split}
\end{equation*}
%
%
But since $f(n)$ is symmetric in $n_1$ and $n_2$, a similar argument shows that
%
%
\begin{equation*}
	\begin{split}
		f(n) \gtrsim |n|^{c} | n_2 |^{m-1-c}. 
	\end{split}
\end{equation*}
%
%
Therefore,
%
%
\begin{equation*}
	\begin{split}
		f(n) \gtrsim | n |^{c}| n_1 |^{\frac{m-1-c}{2}} | n_2 |^{\frac{m-1-c}{2}}
	\end{split}
\end{equation*}
%
%
completing the proof. 
\end{proof}
%
%
\begin{proof}[Proof of \cref{1lem:calc}]
%
%
%
By the change of variable $\theta \mapsto a/2 + x$, we have
%
%
\begin{equation*}
	\begin{split}
		\int_{\rr} \frac{1}{(1 + | \theta |)(1 + | a - \theta |)}d \theta
	= \int_{\rr} \frac{1}{(1 + |  a/2 + x |)(1 + | a/2 - x |)}d x.
	\end{split}
\end{equation*}
%
%
Hence, it suffices to show that
%
%
\begin{equation*}
	\begin{split}
		\int_{\rr} \frac{1}{(1 + | a - \theta |)(1 + | a + \theta |)}d \theta
		\lesssim \frac{\log(2 + | a |)}{1 + | a |}.
	\end{split}
\end{equation*}
%
%
Let us leave the case $a = 0$ for last. By symmetry, the cases $a<0$ and $a >0$
are equivalent. Hence, to cover the case $a \neq0$, we may assume
without loss of generality that $a >0$.
%
%
Then
\begin{equation}
	\label{1a1}
	\begin{split}
		& \int_{\rr} \frac{1}{(1 + | a - \theta |)(1 + | a + \theta |)}d \theta
		\\
		& = \int_{| \theta| \le a+1 } \frac{1}{(1 + | a - \theta |)(1 + | a + \theta
		|)}d \theta + \int_{| \theta| \ge a+1 } \frac{1}{(1 + | a - \theta |)(1 + |
		a + \theta |)}d \theta.
	\end{split}
\end{equation}
Estimating the second integral of \eqref{1a1}, we have
\begin{equation*}
	\begin{split}
		& \int_{| \theta| \ge a+1 } \frac{1}{(1 + | a - \theta |)(1 + | a + \theta
		|)}d \theta 
		\\
		& = \int_{\theta \ge a + 1} \frac{1}{(1 + \theta-a)(1 + \theta+a)} d \theta
		+ \int_{\theta \le -a -1} \frac{1}{(1 + \theta - a) (1 + \theta + a)}d \theta
		\\
		& = \frac{1}{2a} \int_{\theta \ge a + 1} \left[ \frac{1}{1 + \theta -a} -
		\frac{1}{1 + \theta+a} \right] d \theta
		+ \frac{1}{2a} \int_{\theta \le -a-1} \left[ \frac{1}{1 + \theta+a}
		-\frac{1}{1 + \theta -a} \right] d \theta
		\\
		& = \frac{1}{a} \log(1+a)
		\\
		& \lesssim \frac{\log(2 + |a|)}{1 + | a |}.
	\end{split}
\end{equation*}
To evaluate the first integral of \eqref{1a1}, we split into the cases $a \le \theta \le
a+1$, $-a \le \theta \le 0$, $0 \le \theta \le a$, and $a \le \theta \le a+1$.
However, note that 
%
%
\begin{equation*}
	\begin{split}
		& \int_{a}^{a+1} \frac{1}{(1 + | a - \theta |)(1 + | a + \theta |)}d \theta =
		\int_{-a-1}^{-a} \frac{1}{(1 + | a - \theta |)(1 + | a + \theta |)}d \theta,
		\\
		& \int_{0}^{a} \frac{1}{(1 + | a - \theta |)(1 + | a + \theta |)}d \theta =
		\int_{-a}^{0} \frac{1}{(1 + | a - \theta |)(1 + | a + \theta |)}d \theta.
	\end{split}
\end{equation*}
%
%
Therefore, we need only consider the cases $a \le \theta \le a+1$ and $0 \le
\theta \le a$. For the case $a \le \theta \le a+1$, we have have
%
%
\begin{equation*}
	\begin{split}
		\int_{a}^{a+1} \frac{1}{(1 + | a-\theta |)(1 + | a + \theta |)}d \theta
		& = \int_{a}^{a+1} \frac{1}{(1 + \theta -a)(1 + a + \theta)}d \theta
		\\
		& = \frac{1}{2a} \int_{a}^{a+1} \left[ \frac{1}{1 + \theta -a} -
		\frac{1}{1 + \theta + a}  \right]d \theta
		\\
		& =\frac{1}{2a} \log\left( \frac{1 + \theta -a}{1 + \theta + a} \right) \Big
		|_a^{a+1}
		\\
		& = \frac{1}{2a} \log\left( \frac{2a+1}{a+1} \right)
		\\
		& \lesssim\frac{\log 2}{2a}
		\\
		& \lesssim \frac{\log(2 + | a |)}{1 + | a |}.
	\end{split}
\end{equation*}
%
%
while for the case $0 \le \theta \le a$, we have
%
%
\begin{equation*}
	\begin{split}
		\int_{0}^{a} \frac{1}{(1 + | a - \theta |)(1 + | a + \theta |)}d \theta
		& = \int_{0}^{a} \frac{1}{(1 +  a - \theta )(1 +  a + \theta )}d \theta
		\\
		& = \frac{1}{2(1 + a)} \int_{0}^{a} \left[ \frac{1}{1 + a - \theta} +
		\frac{1}{1 + a + \theta} \right]d \theta
		\\
		& = \frac{1}{2(1 + a)} \log \left( \frac{1 + a + \theta}{1 + a - \theta}
		\right) \Big |_{0}^{a}
		\\
		& = \frac{\log\left( 1 + 2a \right)}{2\left( 1 + a \right)}
		\\
		& \lesssim \frac{\log(2 + | a |)}{1 + | a |}.
	\end{split}
\end{equation*}
%
%
This completes the proof for the case $a \neq 0$. Lastly, for the case
$a =0$, we use dominated convergence and our preceding work to
conclude that
%
%
\begin{equation*}
	\begin{split}
		\int_{\rr} \frac{1}{(1 + | \theta|)^2} d \theta
		& = \lim_{a \to 0}
		\int_{\rr} \frac{1}{(1 + | a - \theta |)(1 + | a + \theta |)}d \theta
		\\
		& \lesssim \lim_{a \to 0} \frac{\log(2 + | a |)}{1 + | a |}
		\\
		& =  \log 2
		\\
		& = \frac{\log(2 + | 0 |)}{1 + | 0 |} 
	\end{split}
\end{equation*}
%
which completes the proof.
%
\end{proof}
%
\begin{proof}[Conservation of the $L_x^2$ norm] 
We have
%
%
\begin{equation*}
	\begin{split}
		\frac{d}{dt} \int_\ci | u |^2  dx
		& = \int_\ci \frac{d}{dt} | u |^2  dx
		\\
		& = \int_\ci \frac{d}{dt} \left( u \overline{u} \right)  dx
		\\
		& = \int_\ci \left( u \p_t \overline{u} + \overline{u} \p_t u \right) dx
		\\
		& = \int_\ci \left( u \overline{\p_t u} + \overline{u} \p_t u \right)dx.
	\end{split}
\end{equation*}
%
%
Substituting in $\p_t u = i\left( \p_x^{m} u + | u |^2 u \right)$ we obtain
%
%
\begin{equation*}
	\begin{split}
		& \int_{\ci} \left\{ u\left[ -i\left( \p_x^{m} \overline{u} + | u |^2
		\overline{u} \right) \right] + \overline{u}\left[ i\left( \p_x^{m} u + | u
		|^2 u \right) \right] \right\}dx
		\\
		& = \int_\ci \left[ -iu \p_x^{m} \overline{u} - i| u |^4 + i \overline{u}
		\p_x^{m} u + i | u |^4 \right]dx
		\\
		& = i \int_{\ci}\left( \overline{u} \p_x^{m} u - u \p_x^{m } \overline{u}
		\right)dx.
	\end{split}
\end{equation*}
%
%
Integrating by parts $m/2$ times and using
the spatial periodicity of $u$, the right
hand side simplifies to
%
%
\begin{equation*}
	\begin{split}
		i \int_\ci \left( \p_x^{m/2} \overline{u} \p_x^{m/2} u - \p_x^{m/2} u
		\p_x^{m/2 } 
		\overline{u} \right) dx = 0.
	\end{split}
\end{equation*}
%
%
Therefore, the $L_x^2(\ci)$ norm of solutions to the dNLS is conserved. 
\end{proof}


\chapter{Well Posedness for the KDV}
%
%
%
%
%
\section{Introduction}
We consider the Korteweg-de Vries (KDV) initial value problem (ivp)
%
%
\begin{gather}
	\label{2KDV-eq}
	\p_t u + \p_x^{3} u + u \p_x u = 0,
	\\
	\label{2KDV-init-data}
	u(x,0) = u_0(x), \quad x \in \ci, \ t \in \rr.
\end{gather}
%
%
%
%
\begin{definition}
	We say that the KDV ivp \eqref{2KDV-eq}-\eqref{2KDV-init-data} is
	\emph{locally well posed} in
	$X$ if 
	\begin{enumerate}
		\item For every $\vp(x) \in
	B_R$ there exists $T>0$ depending on $R$ and a unique function
	\\
	$u \in C([-T, T],
	X)$ satisfying \eqref{2KDV-eq} for all $t \in [-T, T]$. 
\item The flow map $u_0 \mapsto u(t)$ is locally uniformly continuous. That is, if $u_0
	\in B_R$, $\{u_{0,n}\} \subset B_R$, and 
	$\|u_0 - u_{0, n} \|_{H^{s}(\ci)} \to 0$, then there exists $T >0$ depending
	on $R$ such that $\|u(\cdot, t) - u_{n}(\cdot,t) \|_{X} \to
	0$ for $t \in [-T, T]$. 
	\end{enumerate}
	Otherwise, we say that the KDV ivp is \emph{ill-posed}.
\end{definition}
%
%
We are now prepared to state the following result.

%%%%%%%%%%%%%%%%%%%%%%%%%%%%%%%%%%%%%%%%%%%%%%%%%%%%%
%
%
%				 Well Posedness Theorem
%
%
%%%%%%%%%%%%%%%%%%%%%%%%%%%%%%%%%%%%%%%%%%%%%%%%%%%%%
%
%
\begin{theorem}
	\label{2thm:main}
	The KDV is well-posed in $\dot{H}^s(\ci)$ for $s \ge -1/2$.  
\end{theorem}
%
%
%%%%%%%%%%%%%%%%%%%%%%%%%%%%%%%%%%%%%%%%%%%%%%%%%%%%%
%
%
%				Outline
%
%
%%%%%%%%%%%%%%%%%%%%%%%%%%%%%%%%%%%%%%%%%%%%%%%%%%%%%
%
%
\section{Outline of the Proof of Main Theorem}
%
%
%
%
%
We first derive a weak formulation of the KDV ivp. 
Let $\ci = [0, 2 \pi]$, and use
the following notation for the Fourier transform
%
%
%
%
\begin{equation*}
	\begin{split}
		\widehat{f}(n) = \int_{\ci} e^{-ix n} f(x) \, dx.
	\end{split}
\end{equation*}
Let $w(x,t) = u \p_x u$. Applying 
the Fourier transform to the KDV ivp in the space variable we obtain 
%
%
\begin{gather}
	\label{2four-trans-pde}
	\p_t \widehat{u}(n, t) = -i n^3 \widehat{u}(n, t) + \lambda i  
	\widehat{w} (n, t),
	\\
	\notag
	\widehat{u} (n,0) = \widehat{\vp}(n)
\end{gather}
%
%
which is a globally well-defined relation in $t$ 
and $n$. Multiplying \eqref{2four-trans-pde} by the integrating factor $e^{itn^3}$ then yields
%%
%%
\begin{equation*}
	\begin{split}
		\left[e^{it n^3} \widehat{u}(n) \right]_t = i
		 e^{it n^3} \widehat{w} (n, t).	
	\end{split}
\end{equation*}
%
%
Integrating from $0$ to $t$, we obtain
%
%
\begin{equation*}
	\begin{split}
		\wh{u}(n, t) = \wh{\vp}(n) e^{- it n^3} + i  
		\int_0^t e^{i(t' - t) n^3} \wh{w}(n, t') \ 
		dt'.
	\end{split}
\end{equation*}
%
%
Therefore, by Fourier inversion 
%
%
\begin{equation}
	\label{2KDV-integral-form}
	\begin{split}
		u(x,t) & = \sum_{n \in \zz} \wh{\vp}(n) e^{i\left( xn - t n^3 
		\right)} 
		\\
		& + i \sum_{n \in \zz} \int_0^t e^{i\left[ xn + \left( t' - t 
		\right) n^3 \right]} \wh{w}(n, t') \ dt'.
	\end{split}
\end{equation}
%
%
Then it is immediate that \eqref{2KDV-integral-form} is a weaker 
restatement of the Cauchy-problem \eqref{2KDV-eq}-\eqref{2KDV-init-data}, 
since by construction any classical solution of the KDV 
ivp is a solution to \eqref{2KDV-integral-form}. 
\\
\\
%
%
We now derive an integral 
equation global in $t$ and equivalent to \eqref{2KDV-integral-form} for $t 
\in [-T, T]$. Let $\psi(t)$ be a cutoff function symmetric about the 
origin such that $\psi(t) = 1$ for $|t| \le T$ and $\text{supp} \, \psi 
= [-2T, 2T ]$. Multiplying the right hand side of expression
$\eqref{2KDV-integral-form}$ by $\psi(t)$, we obtain
%
%
\begin{equation}
	\begin{split}
		\label{2cutoff-int-eq}
		u(x, t)
		& = \frac{1}{2 \pi} \psi(t) \sum_{n \in \zz} e^{i(xn - t n^3)} \widehat{\vp}(n) 
		\\
		& + \frac{i }{2 \pi} \psi(t) \int_0^t \sum_{n \in \zz} 
		e^{i\left[ xn + (t - t')n^3 \right]} \wh{w}(n, t') \ dt'.
	\end{split}
\end{equation}
%
%
Noting that $e^{i\left( xn + tn^3 \right)}$ 
does not depend on $t'$, we may rewrite
%
%
\begin{equation}
	\label{2pre-prim-int-form}
	\begin{split}
		& \frac{i }{2 \pi} \psi(t) \int_0^t \sum_{n \in \zz} 
		e^{i\left[ xn + (t - t') n^3 \right]} \wh{w}(n, t') \ dt'
		\\
		& = \frac{i}{2 \pi} \psi(t) \sum_{n \in \zz} e^{i\left( xn + t 
		 n^3 
		\right)} \int_0^t e^{- it'n^3} \wh{w}(n, t') \ dt'.
	\end{split}
\end{equation}
%%
%%
We remark that this is a \emph{global} relation in $t$. Therefore, by Fourier 
inversion
%
%
%
%
%
%
%
\begin{equation*}
	\begin{split}
		\text{rhs of} \; \eqref{2pre-prim-int-form}
		& = \frac{i}{4 \pi^2} \psi (t) \sum_{n \in \zz} e^{i\left( xn + t 
		 n^3
		\right)} \int_0^t \int_\rr e^{it'\left( \tau - n^3 \right) }
		\wh{w}(n, \tau) d \tau dt'
		\\
		& = \frac{i}{4 \pi^2} \psi(t) \sum_{n \in \zz} \int_\rr 
		e^{i\left( xn + tn^3 \right)} \frac{e^{it\left( \tau - n^3 
		\right)}-1}{\tau - n^3} \wh{w}(n, \tau) d \tau
	\end{split}
\end{equation*}
%
%
where the last step follows from integration. Substituting
into \eqref{2cutoff-int-eq} we obtain
%
%
\begin{equation}
	\begin{split}
		\label{2cutoff-int-eq-2}
		u(x, t)
		& = \frac{1}{2 \pi} \psi(t) \sum_{n \in \zz} e^{i(xn - tn^3)} \widehat{\vp}(n) 
		\\
		& + \frac{1}{4 \pi^2} \psi(t) \sum_{n \in \zz} \int_\rr
		e^{i(xn + t n^3)} \frac{e^{it(\tau - n^3)}- 1}{\tau - n^3} 
		\wh{w}(n, \tau) \ d \tau.
	\end{split}
\end{equation}
%
%
%
%
%
Next, we localize near the singular curve $\tau =  n^3$.  Multiplying the
summand of the second term of \eqref{2cutoff-int-eq-2} by $1 + \psi(\tau -
n^3) - \psi(\tau -
n^3) $ and
rearranging terms, we have
%
%
\begin{equation*}
	\begin{split}
		 u(x, t)
		& = \frac{1}{2 \pi} \psi(t) \sum_{n \in \zz} e^{i(xn + t n^{m 
		})} \widehat{\vp}(n) 
		\\
		& + \frac{1}{4 \pi^2} \psi(t) \sum_{n \in \zz} \int_\rr e^{ixn}  
		e^{it \tau} \frac{1 - \psi(\tau - n^3) 
		}{\tau - n^3} \wh{w}(n, \tau) \ d \tau
		\\
		& - \frac{1}{4 \pi^2} \psi(t) \sum_{n \in \zz} \int _\rr e^{i(xn + 
		t n^3)}
		 \frac{1- \psi(\tau - n^3)}{\tau - n^3} \wh{w}(n, \tau) \ d \tau
		\\
		& + \frac{1}{4 \pi^2} \psi(t) \sum_{n \in \zz} \int_\rr
		e^{i(xn + t n^3)}
		\frac{\psi(\tau - n^3)\left[ e^{it(\tau - n^3)}-1 
		\right]}{\tau - n^3} \wh{w}(n, \tau) \ d \tau
	\end{split}
\end{equation*}
%
%
which by a power series expansion of $[e^{it(\tau - n^3)}-1]$ simplifies  
to
%
%
\begin{align}
	\label{2main-int-expression-0}
	& u(x, t) 
		\\
		\label{2main-int-expression-1}
		& = \frac{1}{2 \pi} \psi(t) \sum_{n \in \zz} e^{i(xn + tn^{m 
		})} \widehat{\vp}(n) 
		\\
		\label{2main-int-expression-2}
		& + \frac{1}{4 \pi^2} \psi(t) \sum_{n\in \zz} \int_\rr e^{ixn}  
		e^{it \tau} \frac{1 - \psi(\tau -  n^3) 
		}{\tau -  n^3} \wh{w}(n, \tau) \ d \tau
		\\
		\label{2main-int-expression-3}
		& - \frac{1}{4 \pi^2} \psi(t) \sum_{n\in \zz} \int_\rr e^{i(xn + 
		t n^3)}
		 \frac{1- \psi(\tau -  n^3)}{\tau -  n^3} \wh{w}(n, \tau) \ d \tau
		\\
		\label{2main-int-expression-4}
		& + \frac{1}{4 \pi^2} \psi(t) \sum_{k \ge 1} \frac{i^k t^k}{k!}
		\sum_{n \in \zz} \int_\rr e^{i(xn + t n^3 )}
		\psi(\tau -  n^3) (\tau -  n^3)^{k-1} \wh{w}(n, \tau)  
		\\
		& \doteq T(u) \notag
\end{align}
%
%
where $T = T_{\vp}$. We now introduce the following spaces. 

\begin{definition}
	Denote $\dot{Y}^s$ to be the space of all
	functions $u$ on $\ci \times \rr$ with
	bounded norm
\begin{equation}
	\label{2Y-s-norm}
	\begin{split}
		\|u\|_{\dot{Y}^s} = \|u\|_{\dot{X}^s} + \|n^s \wh{u}\|_{\dot{\ell}^2_n L^1_\tau }
	\end{split}
\end{equation}
%
%
%
%
where
%
\begin{equation}
	\label{2Xs-norm}
	\begin{split}
		& \|u\|_{\dot{X}^s}
		= \left ( \sum_{n\in \zz} |n|^{2s} \int_\rr \left ( 1 + | 
		\tau - n^3 \right ) | \wh{u} ( n, \tau ) |^2
		\right )^{1/2}
	\end{split}
\end{equation}
and
%
%
\begin{equation}
	\label{2E-norm}
	\|n^s \wh{u}\|_{\dot{\ell}^2_n L^1_\tau } = \left[ \sum_{n \in \zzdot}| n |^{2s} \left(
	\int_{\rr}| \wh{u}(n, \tau) |d \tau \right)^{2} \right]^{1/2}.
\end{equation}
%
%
%
%
\end{definition}
The $\dot{Y}^s$ spaces have the following important property, whose proof
is provided in the appendix.
\begin{lemma}
	\label{2lem:cutoff-loc-soln}
	Let $\psi(t)$ be a smooth cutoff function with $\psi(t) =1$ for $t \in [-T, T]$. If
	$\psi(t)u(x,t) \in \dot{Y}^s$, then $u \in C([-T, T], \dot{H}^s(\ci))$.
\end{lemma}
%
%
We will 
show that for initial data $\vp \in \dot{H}^s(\ci)$, $T$ is a contraction on $B_M 
\subset \dot{Y}^s$, where $B_M$ is the ball centered at the origin of radius $M = 
M_{\vp}> 0$, by estimating the $\dot{Y}^s$
norm of \eqref{2main-int-expression-1}-\eqref{2main-int-expression-4}. The 
Picard fixed point theorem will
then yield a unique solution to
\eqref{2main-int-expression-0}-\eqref{2main-int-expression-4}. An application of
\cref{2lem:cutoff-loc-soln} will then imply the existence of a unique, local
solution $u \in C([-T, T], \dot{H}^s(\ci))$ to the KDV ivp which coincides with the solution to
\eqref{2main-int-expression-0}-\eqref{2main-int-expression-4} on the interval $[-T, T]$. Local Lipschitz continuity of the flow map will follow
from estimates used to establish the contraction mapping. %
%
%%%%%%%%%%%%%%%%%%%%%%%%%%%%%%%%%%%%%%%%%%%%%%%%%%%%%
%
%
%			Proof of Theorem	
%
%
%%%%%%%%%%%%%%%%%%%%%%%%%%%%%%%%%%%%%%%%%%%%%%%%%%%%%
%
%
\section{Proof of Main Theorem}
%
%
%
%%%%%%%%%%%%%%%%%%%%%%%%%%%%%%%%%%%%%%%%%%%%%%%%%%%%%
%
%
%		Estimation of Integral Equality Part 1		
%
%
%%%%%%%%%%%%%%%%%%%%%%%%%%%%%%%%%%%%%%%%%%%%%%%%%%%%%
%
%
%
%
\subsection{Estimate for \ref{2main-int-expression-1}.}
%
%
Letting $f(x,t) = \psi(t) \sum_{n \in \zz} e^{i(xn + tn^{3})} 
\wh{\vp}(n)$, we have $\wh{f}(n,t) = \psi(t) \wh{\vp}(n) e^{itn^{3}}$,
from which we obtain
%
%
\begin{equation}
	\label{2fourier-trans-calc}
	\begin{split}
		\wh{f}(n, \tau)
		& = \wh{\vp}(n) \int_\rr e^{-it( \tau - n^{3})} 
		\psi(t) \ d t
		= \wh{\psi}(\tau - n^{3}) \wh{\vp}(n).
	\end{split}
\end{equation}
%
%
%
%
%
%
Since $\wh{\psi}(\xi)$ is Schwartz for $|\xi| \ge T$, we see that 
%
%
\begin{equation}
	\begin{split}
	\label{2main-int1-est}
		\|\eqref{2main-int-expression-1}\|_{\dot{Y}^s}
		& = \left (  \sum_{n\in \zz} |n|^s \int_\rr \left( 1 + | \tau - n^{3} 
		| \right )
		| \wh{\psi}(\tau - n^{3}) \wh{\vp}(n) |^2 d \tau \right)^{1/2} 
		\\
		& + \left[ \sum_{n \in \zz }\left( 1 + | n | \right)^{2s} \left( \int_{\rr} |
		\wh{\psi}(\tau - n^{3})\wh{\vp}(n) | d \tau
		\right)^{2} \right]^{1/2}
		\\
		& \le c_{\psi}
		\|\vp\|_{\dot{H}^s(\ci)}.
	\end{split}
\end{equation}
%
%
%
%
\subsection{Estimate for \ref{2main-int-expression-2}.}
We now need the following lemma, whose proof is provided in the appendix.
%
%
%%%%%%%%%%%%%%%%%%%%%%%%%%%%%%%%%%%%%%%%%%%%%%%%%%%%%
%
%
%			Schwartz Multiplier	
%
%
%%%%%%%%%%%%%%%%%%%%%%%%%%%%%%%%%%%%%%%%%%%%%%%%%%%%%
%
%
\begin{lemma}
\label{2lem:schwartz-mult}
	For $\psi \in S(\rr)$,
%
%
\begin{equation}
	\label{2schwartz-mult}
	\begin{split}
		\|\psi f \|_{\dot{X}^s} \le c_{\psi} \|f \|_{\dot{X}^s}.
	\end{split}
\end{equation}
%
%
\end{lemma}
%
%
Hence,
%
%
\begin{equation}
	\label{2main-int2-est-X-s-part}
	\begin{split}
		\|\eqref{2main-int-expression-2}\|_{\dot{X}^s} 
		& \lesssim 
		\left( \| \sum_{n \in \zz} e^{ixn} \int_\rr 
		e^{it \tau} \frac{1 - \psi (\tau - n^{3} ) 
		}{\tau - n^{3}} \wh{w}(n, \tau) \ 
		d \tau\|_{\dot{X}^s} \right)^{1/2}
		\\
		& =  \left( \sum_{n \in \zz} |n|^{2s} \int_\rr
		(1 + |\tau - n^{3}|) \left | \frac{1 - \psi(\tau - n^{2 
		})}{\tau - n^{3}} 
		\wh{w}(n, \tau) \right |^2 \ d 
		\tau \right)^{1/2}
		\\
		& \le \left( \sum_{n \in \zz} |n|^{2s} \int_{| \tau - n^3| \ge 1}
		(1 + |\tau - n^{3}|) \frac{|\wh{w}(n, \tau)|^2 }{|\tau - n^3|^2} 
		\ d 
		\tau \right)^{1/2}
		\\
		& \lesssim  \left( \sum_{n \in 
		\zz} |n|^{2s} \int_\rr
		\frac{|\wh{w}(n, \tau) |^2}{1+ |\tau - 
		n^{3}|} 
		 \ d \tau 
		\right)^{1/2}
		\\
		& \lesssim  \|u\|_{\dot{X}^s}^3
	\end{split}
\end{equation}
%
%
where the last two steps follow from the inequality 
%
\begin{equation}
	\label{2one-plus-ineq}
	\begin{split}
		\frac{1}{|\tau - n^{3}| } \le \frac{2}{1 + |\tau - n^{3}| }, 
		\qquad |\tau - n^{3}| \ge 1
	\end{split}
\end{equation}
%
%
and the following bilinear estimate, whose proof we leave for later.
%
%%%%%%%%%%%%%%%%%%%%%%%%%%%%%%%%%%%%%%%%%%%%%%%%%%%%%
%
%
%				 Bilinear Estimates
%
%
%%%%%%%%%%%%%%%%%%%%%%%%%%%%%%%%%%%%%%%%%%%%%%%%%%%%%
%
%
\begin{proposition}
	\label{2prop:prim-bilin-est}
	For any $s \ge -1/2$ we have
	\begin{equation}
		\label{2prim-bilin-est}
		\left( \sum_{n \in \dot{\zz}} |n|^{2s} \int_\rr
		\frac{|\wh{w_{fg}}(n, \tau) |^2}{1+ |\tau - 
		n^{3}| } 
		 \ d \tau 
		\right)^{1/2}
		\lesssim \|f\|_{\dot{X}^s} \|g\|_{\dot{X}^s}
	\end{equation}
	where $w_{fg}(x,t)$ = $\p_x(fg)(x,t)$.
\end{proposition}
Furthermore,
%
%
%
%
\begin{equation}
	\label{2main-int-expression-2-Y-s-part}
	\begin{split}
		\|\wh{\eqref{2main-int-expression-2}} \|_{\dot{\ell}^2_n L^1_\tau}
		& \lesssim \left( \| \sum_{n \in \zz} e^{ixn} \int_\rr 
		e^{it \tau} \frac{1 - \psi (\tau - n^{3} ) 
		}{\tau - n^{3}} \wh{w}(n, \tau) \ 
		d \tau\|_{\dot{\ell}^2_n L^1_\tau} \right)^{1/2}
		\\
		& = \left[ \sum_{n \in \zz}|n|^{2s} \left(
		\int_{\rr}\frac{1 - \psi(\tau - n^{3})}{\tau - n^{3}} \wh{w}(n, \tau) d
		\tau \right)^{2} \right]^{1/2}
		\\
		& \lesssim \|f\|_{X^s} \|g\|_{X^s}
	\end{split}
\end{equation}
%
%
where the last step follows from the following bilinear estimate.
%
%%%%%%%%%%%%%%%%%%%%%%%%%%%%%%%%%%%%%%%%%%%%%%%%%%%%%
%
%
%				Second trilinear Estimate 
%
%
%%%%%%%%%%%%%%%%%%%%%%%%%%%%%%%%%%%%%%%%%%%%%%%%%%%%%
%
%
\begin{proposition}
\label{2prop:bilinear-estimate2}
For any $s \ge -1/2$ we have
%
%
\begin{equation}
	\label{2bilinear-estimate2}
	\begin{split}
		\left( \sum_{n \in \zzdot} |n|^{2s}  \left ( \int_\rr 
		\frac{|\wh{w_{fg}}(n, \tau) |}{1 + | \tau - n^3 |}
		 \ d\tau \right)^2  \right)^{1/2} \lesssim \|f\|_{\dot{X}^s} \|g\|_{\dot{X}^s}.
	\end{split}
\end{equation}
\end{proposition}
%
%
Combining \eqref{2main-int2-est-X-s-part} and
\eqref{2main-int-expression-2-Y-s-part}, we conclude that
%
%
%
%
\begin{equation}
	\label{2main-int2-est}
	\begin{split}
		\|\eqref{2main-int-expression-2}\|_{\dot{Y}^s} \le c_{\psi}\|f\|_{\dot{X}^s} \|g\|_{\dot{X}^s}.
	\end{split}
\end{equation}
%
%
\subsection{Estimate for \ref{2main-int-expression-3}.}
Letting $$f(x,t) = \psi(t) \sum_{n \in \zzdot} e^{i\left( xn + tn^{3} \right)} 
\int_\rr \frac{1 - \psi\left( \lambda - n^{3} \right)}{\lambda - n^{3}} 
\wh{w} \left( n, \lambda \right) \ d \lambda,$$ we have
%
%
\begin{equation*}
	\begin{split}
		& \wh{f^x}(n, t) = \psi(t) e^{itn^{3}} \int_\rr
		\frac{1 - \psi\left( \lambda - n^{3} \right)}{\lambda - n^{3}} 
		\wh{w}(n, \lambda) \ d \lambda
	\end{split}
\end{equation*}
and
\begin{equation*}
	\begin{split}
		 \wh{f}\left( n, \tau \right)
		 & = \int_\rr e^{-it\left( \tau - n^{3} 
		\right)} \psi(t) \int_\rr \frac{1 - \psi\left( 
		\lambda - n^{3} 
		\right)}{\lambda - n^{3}} \wh{w}(n, \lambda) \ d \lambda d \tau
		\\
		& = \wh{\psi}\left( \tau - n^{3} \right) \int_\rr 
		\frac{1 - \psi\left( 
		\lambda - n^{3} 
		\right)}{\lambda - n^{3}} \wh{w}(n, \lambda) \ d \lambda.
	\end{split}
\end{equation*}
Therefore,
%
%
\begin{equation*}
	\begin{split}
		& \| \eqref{2main-int-expression-3} \|_{\dot{X}^s} 
		\\
		& = \left( \sum_{n \in \zzdot} |n|^{2s} \int_\rr \left( 1 + | \tau - n^{m
		} \right ) | | \wh{\psi}\left( \tau - n^3 \right) |^2 \ d \tau
		\right.
		\\
		& \times \left . |
		\int_\rr \frac{1 - \psi\left( \lambda - n^3 \right)}{\lambda -
		n^3} \wh{w}(n, \lambda) \ d \lambda |^2  \right)^{1/2}
		\\
		& \lesssim \left( \sum_{n \in \zzdot} |n|^{2s} | \int_\rr
		\frac{1 - \psi\left( \lambda - n^3 \right)}{\lambda - n^3}
		\wh{w}(n, \lambda) \ d\lambda |^2 \right)^{1/2}
		\\
		& \le \left( \sum_{n \in \zzdot} |n|^{2s}  \left ( \int_\rr
		\frac{1 - \psi\left( \lambda - n^3 \right)}{|\lambda - n^3|}
		|\wh{w}(n, \lambda) | \ d\lambda \right )^2 \right)^{1/2}
		\\
		& \le \left( \sum_{n \in \zzdot} |n|^{2s}  \left ( \int_{| \lambda - 
		n^3 | \ge 1}
		\frac{|\wh{w}(n, \lambda) | }{|\lambda - n^3|}
		\ d\lambda \right )^2 \right)^{1/2}.
	\end{split}
\end{equation*}
%
%
Applying estimate \eqref{2one-plus-ineq} then gives
%
%%
\begin{equation}
	\label{2main-int3-est-X-s-part}
	\begin{split}
		\| \eqref{2main-int-expression-3} \|_{\dot{X}^s}
		& \lesssim \left( \sum_{n \in \zzdot} |n|^{2s}  \left ( \int_\rr
		\frac{|\wh{w}(n, \lambda)| }{1 + |\lambda - n^3|}
		 \ d\lambda \right )^2 \right)^{1/2}
		 \\
		& \lesssim \|u\|_{\dot{X}^s}^3
	\end{split}
\end{equation}
%
%%
where the last step follows from \cref{2prop:bilinear-estimate2}.
Furthermore, 
%
%
\begin{equation}
	\label{2main-int-estimate-3-Y-s-part}
	\begin{split}
		\|\eqref{2main-int-expression-3}\|_{\dot{\ell}^2_n L^1_\tau}
		& = \left[ \sum_{n \in \zzdot} |n|^{2s} \int_{\rr} |
		\wh{\psi}(\tau - n^{3}) |^{2} \left( \int_{\rr}\frac{1 - \psi(\lambda -
		n^{3})}{\lambda - n^{3}} \wh{w}(n, \lambda) d \lambda \right)^{2} d \tau
		\right]^{1/2}
		\\
		& \le c_{\psi} \left[ \sum_{n \in \zzdot} |n|^{2s} \left(
		\int_{\rr} \frac{1 - \psi(\lambda - n^{3})}{\lambda - n^{3}}
		\wh{w}(n, \lambda) d \lambda
		\right)^{2}\right]^{1/2}
		\\
		& \le 2 c_{\psi} \left[ \sum_{n \in \zzdot} |n|^{2s} \left(
		\int_{\rr} \frac{\wh{w}(n, \lambda) }{1 + |\lambda - n^{3}|}
		d \lambda
		\right)^{2}\right]^{1/2}
		\\
		& \lesssim \|f\|_{\dot{X}^s} \|g\|_{\dot{X}^s} 
	\end{split}
\end{equation}
%
%
where the last two steps follow from \eqref{2one-plus-ineq} and
\cref{2prop:bilinear-estimate2}, respectively. Combining
\eqref{2main-int3-est-X-s-part} and \eqref{2main-int-estimate-3-Y-s-part}, we
conclude that
%
%
\begin{equation}
	\label{2main-int3-est}
	\begin{split}
		\|\eqref{2main-int-expression-3}\|_{\dot{Y}^s} 
		\lesssim \|f\|_{\dot{X}^s} \|g\|_{\dot{X}^s}.
	\end{split}
\end{equation}
%
%
%
\subsection{Estimate for \ref{2main-int-expression-4}.}
Note that
%
%
\begin{equation}
	\label{21n}
	\begin{split}
		\eqref{2main-int-expression-4} \simeq \sum_{k \ge 1}
		\frac{i^k}{k!}g_k(x,t)
	\end{split}
\end{equation}
%
%
where 
%
%
\begin{equation*}
	\begin{split}
		& g_k(x,t) = t^k \psi(t) \sum_{n \in \zzdot} e^{i\left( xn + tn^{3}
		\right)} h_k(n),
		\\
		& h_k(n) = \int_\rr \psi \left( \tau - n^3 \right) \times \left(
		\tau - n^3 \right)^{k -1} \wh{w}(n, \tau) \ d \tau.
	\end{split}
\end{equation*}
%
%
Hence
%
%
\begin{equation*}
	\begin{split}
		\wh{g_k^x}(n, t) = t^{k} \psi(t) e^{i t n^3} h_k(n)
	\end{split}
\end{equation*}
%
%
which gives
%
%
\begin{equation*}
	\begin{split}
		\wh{g_k}(n, \tau)
		& = h_k(n) \int_\rr e^{-it\left( \tau - n^3 \right)}
		t^{k}\psi(t) \ dt
		\\
		& = h_k(n) \wh{t^{k}\psi(t)} \left( \tau - n^3 \right).
	\end{split}
\end{equation*}
%
%
Applying this to \eqref{21n}, we obtain
%
%
\begin{equation}
	\label{22n}
	\begin{split}
		\|\eqref{2main-int-expression-4}\|_{\dot{X}^s} 
		& \simeq \left( \sum_{n \in \zzdot} |n|^{2s} \int_\rr \left( 1 + | \tau -
		n^3
		|
		\right) | \wh{\sum_{k \ge 1} \frac{i^k}{k!}g_k(x,t)} |^2 \ d \tau
		\right)^{1/2}
		\\
		& \le \sum_{k \ge 1} \frac{1}{k!}\left( \sum_{n \in \zzdot} |n|^{2s}
		\int_\rr \left( 1 + | \tau - n^3 | \right) | \wh{g_k}(n, \tau) |^2 \
		d \tau \right)^{1/2}
		\\
		& = \sum_{k \ge 1} \frac{1}{k!} \left( \sum_{n \in \zzdot} |n|^{2s}
		\int_\rr \left( 1 + | \tau - n^3 | \right) | h_k(n) \wh{t^k
		\psi(t)} \left( \tau - n^3 \right) |^2 \ d \tau \right)^{1/2}
		\\
		& = \sum_{k \ge 1} \frac{1}{k!} \left( \sum_{n \in \zzdot} |n|^{2s} |
		h_k(n) |^2 \int_\rr \left( 1 + | \tau - n^3 | \right) | \wh{t^k
		\psi(t)} \left( \tau - n^3 \right) |^2 \ d \tau \right)^{1/2}.
	\end{split}
\end{equation}
%
%
Notice that for fixed $n$, the change of variable $\tau - n^3 \to \tau'$
gives
%
%
\begin{equation}
	\label{23n}
	\begin{split}
		\int_\rr \left( 1 + | \tau - n^3 | \right) | \wh{t^{k}
		\psi(t)}\left( \tau - n^3 \right) |^2 \ d \tau
		& = \int_\rr \left( 1 + |\tau'| \right) | \wh{t^k \psi(t)}(\tau') |^2 \
		d \tau'
		\\
		& \le \int_\rr \left( 1 + |\tau'| \right)^2 | \wh{t^k \psi(t)}(\tau')
		|^2 \ d \tau'
		\\
		& \lesssim \int_\rr \left( 1 + | \tau' |^2 \right) | \wh{t^{k}
		\psi(t)}(\tau') |^2 \ d \tau'
		\\
		& = \|t^k \psi(t) \|_{H^1(\rr)}^2.
	\end{split}
\end{equation}
%
%
But
%
%
\begin{equation}
	\label{24n}
	\begin{split}
		\|t^k \psi(t) \|_{H^1(\rr)}^2
		& = \left( \|t^k \psi(t)\|_{L^2(\rr)} + \|\p_t \left( t^k \psi(t)
		\right)\|_{L^2(\rr)} \right)^2
		\\
		& \lesssim \|t^{k}\psi(t) \|_{L^2(\rr)}^2 + \|\p_t \left (t^{k}
		\psi(t) \right )\|_{L^2(\rr)}^2
		\\
		& \le \|t^k \psi(t) \|_{L^2(\rr)}^2 + \|t^k \p_t \psi(t)
		\|_{L^2(\rr)}^2 + \|k t^{k -1} \psi(t) \|_{L^2(\rr)}^2
		\\
		& = c_{\psi} + c_{\psi}' + k^2 c_{\psi}''
		\\
		& \lesssim k^2.
	\end{split}
\end{equation}
%
%
Hence, applying \eqref{23n} and \eqref{24n} to \eqref{22n}, we obtain
%
%%
\begin{equation}
	\label{25n}
	\begin{split}
		\|\eqref{2main-int-expression-4} \|_{\dot{X}^s}
		& \lesssim
		\sum_{k \ge 1} \frac{k}{k!} \left( \sum_{n \in \zzdot} |n|^{2s} | h_k(n) |^2 
		\right)^{1/2}
		\\
		& \le \sum_{k \ge 1} \frac{k}{k!}
		 \sup_{k \ge 1} \left( \sum_{n \in \zzdot} |n|^{2s} | 
		h_k(n) |^2 \right)^{1/2}
		\\
		& = \sum_{k \ge 1} \frac{k}{k!}  \sup_{k \ge 1} 
		\left( \sum_{n \in \zzdot} |n|^{2s} \int_\rr 
		\psi\left( \tau - n^3 \right) \cdot \left( \tau - n^3 
		\right)^{k -1} \wh{w}(n, \tau) \ d \tau \right)^{1/2}.
	\end{split}
\end{equation}
%
%%
Recall that $\text{supp} \, |\psi| \subset [0, T ]$. Pick $T \le 1$. 
Then $| \psi\left( \tau - n^3 \right) \cdot \left( \tau - n^3 \right)^{k 
-1} | \le \chi_{| \tau - n^3 | \le 1}$ for all $k \ge 1$. Hence, \eqref{25n} gives
%
%%
\begin{equation*}
	\begin{split}
		\|\eqref{2main-int-expression-4} \|_{\dot{X}^s} 
		& \lesssim \sum_{k \ge 1} \frac{k}{k!}  \left( \sum_{n \in \zzdot} | 
		\int_{| \tau - n^{3}  |\le 1} | \wh{w}(n, \tau) \ d \tau |^2 
		\right)^{1/2}
	\end{split}
\end{equation*}
%
%%
which by the inequality
%
%%
\begin{equation*}
	\begin{split}
		\frac{1 + | \tau - n^3 |}{1 + | \tau  - n^3 |} \le 
		\frac{2}{1 + | \tau - n^3 |}, \qquad | \tau - n^3  | \le 1
	\end{split}
\end{equation*}
%
%%
implies
%
%%
\begin{equation}
\label{2main-int4-est-X-s-part}
	\begin{split}
		\|\eqref{2main-int-expression-4}\|_{\dot{X}^s}
		& \lesssim \left( \sum_{n \in \zzdot} | \int_{| \tau - n^{3}| \le 1 }
		\frac{\wh{w}(n, \tau)}{1 + | \tau - n^3 |} \ d \tau |^2 
		\right)^{1/2}
		\\
		& \le \left( \sum_{n \in \zzdot} | \int_\rr
		\frac{\wh{w}(n, \tau)}{1 + | \tau - n^3 |} \ d \tau |^2 
		\right)^{1/2} \\
		& \le \left( \sum_{n \in \zzdot} \left( \int_\rr 
		\frac{|\wh{w}(n, \tau)|}{1 + | \tau - n^3 |}  \ d \tau  \right)^2
		\right)^{1/2} \\
		& \lesssim \|u\|_{\dot{X}^s}^3
	\end{split}
\end{equation}
%
%%
where the last step follows from \cref{2prop:bilinear-estimate2}. Similarly,
we have
%
%
\begin{equation}
\label{2main-int4-est-Y-s-part}
	\begin{split}
		\|\eqref{2main-int-expression-4}\|_{\dot{\ell}^2_n L^1_\tau}
		& \simeq \left[ \sum_{n \in
		\zzdot}|n|^{2s} \left( \int_{\rr} | \sum_{k \ge 1}
		\wh{\frac{i^{k}}{k!}g_{k}(x,t)(n, \tau)} |d \tau \right)^{2} \right]^{1/2}
		\\
		& \le \sum_{k \ge 1} \frac{1}{k!} \left[ \sum_{n \in \zzdot} (1 + | n
		|)^{2s} \left( \int_{\rr} | \wh{g}(n, \tau) | d \tau \right)^{2}
		\right]^{1/2}
		\\
		& = \sum_{k \ge 1} \frac{1}{k!} \left[ \sum_{n \in \zzdot} (1 + | n
		|)^{2s} | h_{k}(n) |^2 \left( \int_{\rr} | \wh{t^{k} \psi(t)}(\tau -
		n^{3}) |d \tau \right)^{2} \right]^{1/2}
		\\
		& = c_{\psi} \sum_{k \ge 1} \frac{1}{k!} \left[ \sum_{n \in \zzdot} (1 + | n
		|)^{2s} | h_{k}(n) |^2 \right]^{1/2}
		\\
		& \lesssim \|u\|_{\dot{X}^s}^{3}
	\end{split}
\end{equation}
%
%
where the last step follows from the computations starting from \eqref{25n}
through \eqref{2main-int4-est-X-s-part}.
Combining \eqref{2main-int4-est-X-s-part} and \eqref{2main-int4-est-Y-s-part}, we
have
%
%
\begin{equation}
\label{2main-int4-est}
	\begin{split}
		\|\eqref{2main-int-expression-4}\|_{\dot{Y}^s} \lesssim \|u\|_{\dot{X}^s}^{3}.
	\end{split}
\end{equation}
%
%
Collecting estimates \eqref{2main-int1-est}, \eqref{2main-int2-est}, 
\eqref{2main-int3-est}, and \eqref{2main-int4-est}, and recalling 
\eqref{2main-int-expression-1}-\eqref{2main-int-expression-4}, we see that
$$\|Tu\|_{\dot{Y}^s} \le c_\psi \left( \|\vp \|_{\dot{H}^s(\ci)} + \|u\|_{\dot{X}^s}^3 \right )$$ 
which by the inequality $\|u\|_{\dot{X}^s} \le \|u\|_{\dot{Y}^s}$ yields the following.
%%
%%%%%%%%%%%%%%%%%%%%%%%%%%%%%%%%%%%%%%%%%%%%%%%%%%%%%
%
%% Contraction Proposition
%				 
%%%%%%%%%%%%%%%%%%%%%%%%%%%%%%%%%%%%%%%%%%%%%%%%%%%%%%
%%
%%
%
\begin{proposition}
\label{2prop:contraction}
Let $s \ge -1/2$. Then
%
%%
\begin{equation*}
	\begin{split}
		\|Tu\|_{\dot{Y}^s} \le c_\psi \left( \|\vp \|_{\dot{H}^s(\ci)} + \|u\|_{\dot{Y}^s}^3 
		\right).
	\end{split}
\end{equation*}
%
%%
\end{proposition}
We will now use \cref{2prop:contraction} to prove local well-posedness for the 
KDV ivp. Let $c = c_{\psi}^{1/2}$. For given $\vp$, we may choose $\psi$ such
that 
%
%%
\begin{equation*}
	\begin{split}
		\|\vp\|_{\dot{H}^s(\ci)} \le \frac{15}{64c^3}.
	\end{split}
\end{equation*}
%
%%
Then if $\|u\|_{\dot{Y}^s} \le \frac{1}{4c}$, we have
%
%%
\begin{equation*}
	\begin{split}
		\|T u \|_{\dot{Y}^s} 
		& \le c^2 \left[ \frac{15}{64c^3} + \left( 
		\frac{1}{4c} \right)^3 \right]
		=  \frac{1}{4c}.
	\end{split}
\end{equation*}
%
%%
Hence, $T=T_{\vp}$ maps the ball $B\left( 0, \frac{1}{4c} \right) \subset \dot{Y}^s$ into 
itself. Next, note that
%
%%
\begin{equation*}
	\begin{split}
		Tu - Tv = \eqref{2main-int-expression-2} + \eqref{2main-int-expression-3} 
		+ \eqref{2main-int-expression-4}
	\end{split}
\end{equation*}
%
%%
where now $w = u | u |^2 - v | v |^{2}$. Rewriting
%
%%
\begin{equation*}
	\begin{split}
		u | u |^{2} - v | v |^{2}
		& = | u |^2 \left( u -v \right) + v\left( | u 
		|^2 - | v |^2
		\right)
		\\
		& = u \bar u \left( u -v \right) + v u \bar u - v v \bar v
		\\
		& = u \bar u \left( u - v \right) + v \bar u\left( u - v \right) + v 
		\bar u v - v v \bar v
		\\
		& = u \bar u \left( u -v \right) + v \bar u\left( u - v \right) + v v 
		\left( \overline{u -v} \right)
	\end{split}
\end{equation*}
%
%%
the triangle inequality and linearity of the Fourier transform then give
%
%%
\begin{equation*}
	\begin{split}
		| \wh{w}(n, \tau) | = | \mathcal{F}(u | u |^2 - v| v |^2) |
		& \le | \wh{u \overline{u} \left (u -v \right )} | +
		| \wh{v \overline{u} (u -v)} | + |\wh{v v 
		(\overline{u-v})}|
		\\
		& \doteq | \wh{w_1} | + | \wh{w_2} | + | \wh{w_3} |
	\end{split}
\end{equation*}
%
%%
where
%
%%
\begin{equation*}
	\begin{split}
		w_1 = u \bar u \left( u -v \right), \qquad w_2 = v \bar u \left( u -v 
		\right), \qquad w_3 = v v \left( \overline{u -v} \right).
	\end{split}
\end{equation*}
%
%%
Hence, $Tu - Tv = \sum_{\ell=1, 2, 3} 
T_\ell(u, v)$, where
\begin{align}
	\label{2main-int-exp-mod1}
	& \frac{1}{4 \pi^2} \psi(t) \sum_{n\in \zzdot} \int_\rr e^{ixn}  
		e^{it \tau} \frac{1 - \psi(\tau - n^{3}) 
		}{\tau - n^{3}} \wh{w_\ell}(n, \tau) \ d \tau
		\\
		\label{2main-int-exp-mod2}
		- & \frac{1}{4 \pi^2} \psi(t) \sum_{n\in \zzdot} \int_\rr e^{i(xn + 
		tn^{3})}
		 \frac{1- \psi(\tau - n^{3})}{\tau - n^{3}} \wh{w_\ell}(n, \tau) \ d \tau
		\\
		\label{2main-int-exp-mod3}
		+ & \frac{1}{4 \pi^2} \psi(t) \sum_{k \ge 1} \frac{i^k t^k}{k!}
		\sum_{n \in \zzdot} \int_\rr e^{i(xn + tn^{3} )}
		\psi(\tau - n^{3}) (\tau - n^{3})^{k-1} \wh{w_\ell}(n, \tau)  
		\\
		\doteq & T_\ell(u). \notag
\end{align}
Repeating the arguments used to estimate 
\eqref{2main-int-expression-2}-\eqref{2main-int-expression-4}, we obtain
%
%%
\begin{equation*}
	\begin{split}
		& \|T_1\|_{\dot{Y}^s} \le c_\psi \|u -v \|_{\dot{Y}^s} \|u\|^2_{\dot{Y}^s}
		\\
		& \|T_2\|_{\dot{Y}^s} \le c_\psi \|u -v \|_{\dot{Y}^s} \|u\|_{\dot{Y}^s} \|v\|_{\dot{Y}^s}
		\\
		& \|T_3\|_{\dot{Y}^s} \le c_\psi \|u -v \|_{\dot{Y}^s} \|v\|_{\dot{Y}^s}^2.
	\end{split}
\end{equation*}
%
%%
Therefore,
%
%%
\begin{equation}
	\label{220a}
	\begin{split}
		\|Tu - Tv \|_{\dot{Y}^s} = & \| \sum T_\ell(u, v) \|_{\dot{Y}^s}
		\\
		& \le c_\psi \|u -v \|_{\dot{Y}^s} \left( \|u\|_{\dot{Y}^s}^2 + 
		\|u\|_{\dot{Y}^s} \|v\|_{\dot{Y}^s} + \|v\|_{\dot{Y}^s}^2 \right)
		\\
		& \le c_\psi \|u -v\|_{\dot{Y}^s} \left( \|u\|_{\dot{Y}^s} + \|v\|_{\dot{Y}^s} \right)^2
		\\
		& = c^2 \|u -v\|_{\dot{Y}^s} \left( \|u\|_{\dot{Y}^s} + \|v\|_{\dot{Y}^s} \right)^2.
	\end{split}
\end{equation}
%
%%
If $u, v \in B(0, \frac{1}{4c}) \subset \dot{Y}^s$, it follows that
%
%%
\begin{equation}
	\label{221a}
	\begin{split}
		\|Tu - Tv \|_{\dot{Y}^s}
		& \le c^2 \|u -v \|_{\dot{Y}^s} \left( \frac{1}{4c} + 
		\frac{1}{4c} \right)^2
		\\
		& = \frac{1}{4} \|u -v \|_{\dot{Y}^s}. 
	\end{split}
\end{equation}
%
%%
We conclude that $T = T_{\vp}$ is a contraction on the ball $B(0, 
\frac{1}{4c}) \subset \dot{Y}^s$. A Picard iteration and application of 
\cref{2lem:cutoff-loc-soln} then yield a unique, local
solution to the KDV ivp \eqref{2KDV-eq}-\eqref{2KDV-init-data}.
\begin{definition}
	We say that the flow map $u_0 \mapsto u(t)$ is \emph{locally Lipschitz} in a Banach
	space $X$ if for
	$$u_0, v_0 \in B_R \doteq \{f: \|f\|_X < R\},$$ there exist $C, T>0$
	depending on $R$ such that $\|u(\cdot, t) - v(\cdot, t)
	\|_X \le C \|u_{0} - v_0 \|_{X}$ for $t \in [-T, T]$. We
	say the flow map is \emph{locally uniformly
	continuous} in $X$ if for
	$u_0, v_0 \in B_R$ there exists $T >0$ depending on $R$ such that for
	$t \in [-T, T]$, $\|u(\cdot, t) - v(\cdot, t) \|_{X} \to
	0$ if $\|u_0 - v_0 \|_{H^{s}(\ci)} \to 0$. 
\end{definition}
%
%
Clearly any locally Lipschitz flow map is locally uniformly continuous. 
Next, we shall establish local Lipschitz continuity in $\dot{Y}^s$ of the flow
map. Let $\vp_1, \vp_2 \subset \dot{H}^s(\ci)$ be given. Choose $\psi$ such that
$\vp_1, \vp_2 \subset B(0, \frac{15}{64c^{3}})$.  Then there exist $u_1, u_2 \in
\dot{Y}^s$ such that $u_1 = T_{\vp_1}$, $u_2 = T_{\vp_2}$, and so
%
%
\begin{equation*}
	\begin{split}
		T_{\vp_1}(u) - T_{\vp_2}(v) = \frac{1}{2\pi} \psi(t) \sum_{n \in
		\zzdot}e^{i\left( xn + tn^{3} \right)} \wh{\vp_1 - \vp_2}(n) + \sum_{\ell
		= 1,2,3} T_{\ell}(u).
	\end{split}
\end{equation*}
%
%
Using an argument similar to \eqref{2fourier-trans-calc}-\eqref{2main-int1-est},
we obtain
%
%
\begin{equation*}
	\begin{split}
		\| \frac{1}{2\pi} \psi(t) \sum_{n \in
		\zzdot}e^{i\left( xn + tn^{3} \right)} \wh{\vp_1 - \vp_2}(n)\|_{\dot{Y}^s}
		\le c_\psi \|\vp_{1} - \vp_{2}\|_{\dot{Y}^s}.
	\end{split}
\end{equation*}
%
%
Hence, \eqref{220a}-\eqref{221a} gives
%
%
\begin{equation*}
	\begin{split}
		\sum_{\ell=1,2,3} T_{\ell}(u,v) \le \frac{1}{4}\|u-v\|_{\dot{Y}^s}.
	\end{split}
\end{equation*}
%
%
Therefore,
%
%
\begin{equation*}
	\begin{split}
		\|u -v \|_{\dot{Y}^s} = \|T_{\vp_1}(u) - T_{\vp_2}(v) \|_{\dot{Y}^s} \le c_\psi
		\|\vp_{1} - \vp_{2} \|_{\dot{H}^s\left( \ci \right)}\| +
		\frac{1}{4} \|u -v \|_{\dot{Y}^s}
	\end{split}
\end{equation*}
%
%
which implies
%
%
\begin{equation*}
	\begin{split}
		\frac{3}{4} \|u-v\|_{\dot{Y}^s} \le c_\psi \|\vp_1 - \vp_2 \|_{\dot{H}^s(\ci)}
	\end{split}
\end{equation*}
%
%
or
%
%
\begin{equation*}
	\begin{split}
		\|u -v \|_{\dot{Y}^s} \le \frac{4}{3} c_\psi \|\vp_1 - \vp_2 \|_{\dot{H}^s(\ci)}.
	\end{split}
\end{equation*}
%
%
Applying \cref{2lem:cutoff-loc-soln}, we then obtain
%
%
	 %
	 %
	 \begin{equation*}
		 \begin{split}
			\|u(\cdot, t) -v(\cdot, t) \|_{\dot{H}^s(\ci)} \le \frac{4}{3} c_\psi \|\vp_1 -
			\vp_2 \|_{\dot{H}^s(\ci)}, \qquad t \in [-T, T].
		 \end{split}
	 \end{equation*}
	 %
	 %
Hence, the flow map of the KDV ivp is locally Lipschitz continuous in
$\dot{H}^s(\ci)$. This
concludes the proof of \cref{2thm:main}. \qquad \qedsymbol
%
%
%
%
\section{Proof of First Bilinear Estimate}
Note first that $|\wh{w_{fg}}(n, \tau) |  = | n\wh{f} *  \wh{g} 
(n, \tau)|$. From this and the conservation of mass, it follows that
%
%
\begin{equation}
	\label{2non-lin-rep}
	\begin{split}
		| \wh{w_{fg}}(n, \tau)|
		& = | \sum_{\substack{n_1 \neq 0, n_2 \neq 0 \\n_1 +n_2 =n}}  \int_{\tau_1 + \tau_2 = \tau}n\wh{f}\left( n_1,  \tau_1 
\right) \wh{g}\left( n_2, \tau_2  
\right) d \tau_1 d \tau_2 |
\\
& = | \sum_{\substack{n_1 \neq0, n_2 \neq 0 \\n_1 + n_2 =n}}  \int_{\tau_1 + \tau_2 = \tau}n\wh{f}\left( n_1,  \tau_1 
\right) \wh{g}\left( n_2, \tau_2  
\right) d \tau_1 d \tau_2 | 
\\
& \le \sum_{\substack{n_1 \neq0, n_2 \neq 0 \\n_1 + n_2 =n}}   \int_{\tau_1 + \tau_2 = \tau}| n | \times | \wh{f}\left( n_1, \tau_1 
\right) | \times  | \wh{g}\left( n_2, \tau_2 
\right) |   d \tau_1 d \tau_2  
\\
& = \sum_{\substack{n_1 \neq0, n_2 \neq 0 \\n_1 + n_2 =n}} \int_{\tau_1 + \tau_2 = \tau}| n | \times \frac{c_f\left( n_1, \tau_1 
\right)}{|n_1|^s \left( 1 + | \tau_1 - n_1^3 | \right)^{1/2}}
\\
& \times \frac{c_{g}\left( n_2, \tau_2 \right)}{|n_2|^s\left( 1 + | \tau_2 -  n_2^3| 
\right)^{1/2}}
  \ d \tau_1 d \tau_2 
\end{split}
\end{equation}
%
%
where 
%
%
\begin{equation*}
	\begin{split}
		c_h(n, \tau) =
		\begin{cases}
			|n|^s \left( 1 + | \tau - n^3 |  
			\right)^{1/2} | \wh{h}\left( n, \tau \right) |, \qquad & n \neq 0
		\\
		0, \qquad & n = 0.
	\end{cases}
	\end{split}
\end{equation*}
%
%
From our work above, it follows that 
%
%
\begin{equation}
	\label{2convo-est-starting-pnt}
	\begin{split}
		 & |n|^s \left( 1 + | \tau - n^3 | \right)^{-1/2} | \wh{w_{fg}}\left( 
		n, \tau \right) |
		\\
		& \le \left( 1 + | \tau - n^3 | \right)^{-1/2}
		\sum_{\substack{n_1 \neq0, n_2 \neq 0 \\n_1 + n_2 =n}} \int_{\tau_1 + \tau_2 = \tau}\frac{|n|^{s+1}}{|n_1|^s | n_2|^s} 
		\times \frac{c_f(n_1, \tau_1)}{\left( 1 + | \tau_1 - n_1^3 | 
		\right)^{1/2}}
		\\
		& \times
		\frac{c_g(n_2, \tau_2)}{\left( 1 + | \tau_2 - n_2^3 | 
		\right)^{1/2}}\ d \tau_1 d \tau_2.
	\end{split}
\end{equation}
%
%
Unlike the NLS, we must use the smoothing properties of the
principal symbol $\tau - n^3$ regardless of the choice of $s$, since the quantity
%
%
\begin{equation}
	\label{2convo-multiplier}
	\begin{split}
		\frac{|n|^{s+1}}{|n_1|^s |n_2|^s }
	\end{split}
\end{equation}
%
%
blows up in general, due to the presence of the extra power of $|n|$ coming from the derivative on
the nonlinearity. To utilize the smoothing effects of the principal symbol, we will need the following, whose
proof is provided in the appendix.
%
%
\begin{lemma}
	\label{2lem:number-theory}
	Let $n=n_1 + n_2$ and suppose that $n, n_1, n_2\neq
	0$. Then for any integer $c \ge 0$
%
%
\begin{equation}
	\begin{split}
		\label{2number-theory}
		| - n^{3} + n_1^3 + n_2^3| \ge 2^{-c/2} | n |^{\frac{2+c}{2}} | n_{1}
		|^{\frac{2-c}{2}}| n_2 |^{\frac{2-c}{2}}.
	\end{split}
\end{equation}
%
%
\end{lemma}
%
%
\begin{remark}
	In~\cite{Bourgain:1993ju}, Bourgain obtains the lower bound $n^2$ for
	the left hand side of \eqref{2number-theory}. This is too coarse an estimate,
	as we shall see.
\end{remark}
%
%
Since $$| \tau - n^{3} - \left( \tau_1 - n_1^3 
+ \tau_2 - n_2^3  \right ) | = | - n^{3} + n_1^3 +
n_2^3|,$$ by \cref{2lem:number-theory} and
the pigeonhole principle we must have one of the 
following.
%
%
\begin{align}
	\label{2pigeon-case-1}
	& |\tau - n^3| \ge \frac{2^{-c/2}}{3} | n |^{\frac{2+c}{2}} | n_{1}
		|^{\frac{2-c}{2}}| n_2 |^{\frac{2-c}{2}}		\\
		\label{2pigeon-case-2}
		& | \tau_1 - n_1^3 | \ge \frac{2^{-c/2}}{3} | n |^{\frac{2+c}{2}} | n_{1}
		|^{\frac{2-c}{2}}| n_2 |^{\frac{2-c}{2}},  
		\\
		\label{2pigeon-case-3}
		& | \tau_2 - n_2^3 | \ge
		\frac{2^{-c/2}}{3} | n |^{\frac{2+c}{2}} | n_{1}
		|^{\frac{2-c}{2}}| n_2 |^{\frac{2-c}{2}}.  
\end{align}
%
%
By the symmetry of the convolution, it will be enough to consider only
\eqref{2pigeon-case-1} and \eqref{2pigeon-case-2}.
%
%
%
\subsection{Case \ref{2pigeon-case-1}.} 
We have, for nonzero $ n, n_1, n_2 $
%
%%
\begin{equation}
	\label{2convo-deriv-bound}
	\begin{split}
		& \frac{|n|^{s+1}}{|n_1|^s 
		| n_2|^s}
		\times
		\frac{1}{(1 + | \tau -n^{3} |)^{1/2}}
		\\
		& \lesssim | n |^{s+1}| n_1 |^{-s}| n_2 |^{-s} \times | n
		|^{-\frac{2+c}{4}}| n_1 |^{-\frac{2-c}{4}}| n_2 |^{-\frac{2-c}{4}} 
		\\
		& = | n |^{\frac{4s +2 -c}{4}} | n_1 |^{\frac{-4s -2 +c}{4}} | n_2
		|^{\frac{-4s -2 +c}{4}}
		\\
		& \le 1, \qquad s \ge -1/2.
	\end{split}  
\end{equation}
%
%
\begin{remark}
	\label{2rem:s-val}
	The last line follows from the following reasoning: Set $(4s + 2 -c) = 0$
or, equivalently, $-4s -2 +c = 0$. Then for any $c \ge 0$ such that $c = 4s+2$
the left hand side of
\eqref{2convo-deriv-bound} is bounded by $1$. Of course such a $c$ exists, as long as $s \ge -1/2$. 
\end{remark}
%
%
%
Hence, recalling \eqref{2convo-est-starting-pnt} and applying estimates 
\eqref{2pigeon-case-1} and \eqref{2convo-deriv-bound}, we obtain
%
%
\begin{equation}
	\label{2non-lin-rep-with-bound}
	\begin{split}
		& |n|^s \left( 1 + | \tau - n^3 | \right)^{-1/2} | 
		\wh{w_{fg}}(n, \tau) | 
		\\
		& \lesssim \sum_{\substack{n_1 \neq0, n_2 \neq 0 \\n_1 + n_2 =n}} \int_{\tau_1 + \tau_2 = \tau}\frac{c_f(n_1, \tau_1)}{\left( 1 + | 
		\tau_1 -  n_1^3| \right)^{1/2}}
		\times \frac{c_g\left( n_2, \tau_2\right)}{\left( 1 + | \tau_2 -n_2^3|
		\right)^{1/2}}
		\\
		& = \wh{C_f C_g}(n, \tau)
	\end{split}
\end{equation}
%
%
where
\begin{equation*}
	\begin{split}
		C_h(x,t) =
		\left[ \frac{c_h(n, \tau)}{\left( 1 + | \tau - n^3 | 
		\right)^{1/2}}\right]^\vee.	
	\end{split}
\end{equation*}

%
%
Therefore, from \eqref{2non-lin-rep-with-bound}, Plancherel, and generalized 
H\"{o}lder, we obtain
%
%
\begin{equation}
	\label{2gen-holder-bound}
	\begin{split}
		& \| |n|^s \left( 1 + | \tau - n^3 | \right )^{-1/2}  \wh{w_{fg}}\left( 
		n, \tau \right) \|_{L^2(\ci \times \rr)}
		\\
		& \lesssim \|\wh{C_f C_g }\left( n, \tau \right) 
		\|_{L^2\left( \zzdot \times \rr \right)}
		\\
		& \simeq \|C_f C_g \|_{L^2\left( \ci \times \rr \right)}
		\\
		& \le \|C_f \|_{L^4(\ci \times \rr)} \|C_g \|_{L^4(\ci \times \rr)}.
	\end{split}
\end{equation}
%
We now need the following Fourier multiplier estimate. 
\begin{lemma}
	\label{2lem:four-mult-est-L4}
	Let $(x, t) \in \ci \times \rr $ and $(k, \tau) \in \zz \times \rr$ be
	the dual variables. Let $v$ be a positive even integer, and fix $\ee >
	0$. Then for $b$ satisfying the relations $b \ge (\ee v + 1/2)/v$ and $b \ge (v+1)/(4v)$, we have
\begin{equation}
	\label{2four-mult-est-L4}
	\begin{split}
		\|f\|_{L^4(\ci \times \rr)} \le c_\ee \|\left( 1 + | \tau - n^v | 
		\right)^b \wh{f} \|_{L^2( \zz \times \rr)}
	\end{split}
\end{equation}
for every test function $f(x, t)$. 
%
%
%
%
\end{lemma}
\begin{framed}
%
%
\begin{remark}
  See \cite{Himonas:2007qf} for a similar theorem. The above theorem is a slight improvement. 
\end{remark}
%
%
\end{framed}
From the lemma, we see that
%
%
\begin{equation}
	\label{2four-mult-conseq}
	\begin{split}
		\|C_h\|_{L^4(\ci \times \rr)} 
		& \lesssim \|(1 + | \tau - n^3 |)^{1/2} \wh{C_h}
		\|_{L^2(\zz \times \rr)}
		\\
		& = \|c_{h} \|_{L^2(\zz \times \rr)} 
		\\
		& = \|h \|_{\dot{X}^s}. 
	\end{split}
\end{equation}
%
%
Applying this to \eqref{2gen-holder-bound} we
conclude that
\begin{equation*}
	\begin{split}
		\| |n|^s \left( 1 + | \tau - n^3 | \right ) ^{-1/2} \wh{w_{fg}}\left( 
		n, \tau \right) \|_{L^2(\zzdot \times \rr)}
		& \lesssim \|f\|_{\dot{X}^s} \|g\|_{\dot{X}^s}.
	\end{split}
\end{equation*}
%
%
%
\subsection{Case \ref{2pigeon-case-2}.}
Using a similar argument to that in Case \eqref{2pigeon-case-1}, we obtain
%
%
\begin{equation}
	\label{21f}
	\begin{split}
		& |n|^s  | \wh{w_{fg}}\left( 
		n, \tau \right) |
		\\
		& \lesssim 
		\sum_{\substack{n_1 \neq0, n_2 \neq 0 \\n_1 + n_2 =n}} \int_{\tau_1 + \tau_2 = \tau}		c_f(n_1, \tau_1)
		\times
		\frac{c_g(n_2, \tau_2)}{\left( 1 + | \tau_2 - n_2^3 | 
		\right)^{1/2}} 
		\\
		& = \wh{\overset{\sim}{C_f} C_g}
	\end{split}
\end{equation}
%
%%
where
%
%
\begin{equation*}
	\begin{split}
		\overset{\sim}{C_h}(x,t) = \left[ c_h(n, \tau) \right]^\vee.
	\end{split}
\end{equation*}
%
%
Hence
%
%%
\begin{equation}
	\label{23f}
	\begin{split}
		& \| |n|^s \left( 1 + | \tau - n^3 | \right)^{-1/2} \wh{w_{fg}}(n, \tau) 
		\|_{L^2(\zzdot \times \rr)}
		\\
		& \lesssim \|\left( 1 + | \tau - n^{3} | \right)^{-1/2} 
		\wh{\overset{\sim}{C_f} C_g } \|_{L^2(\zzdot \times \rr)}
		\\
		& =  \|\left( 1 + | \tau - n^{3} | \right)^{-1/2} 
		\wh{\overset{\sim}{C_f} C_g } \|_{L^2(\zz \times \rr)}
		\\
		& \lesssim  \|\overset{\sim}{C_f} C_g  \|_{L^{4/3}(\ci \times \rr)}
	\end{split}
\end{equation}
%
%%
where the last step follows by dualizing \cref{2lem:four-mult-est-L4}. More
precisely, we have the following.
\begin{corollary}
	\label{2cor:four-mult-est-L4}
	Let $(x, t) \in \ci \times \rr $ and $(n, \tau) \in \zzdot \times \rr$ be 
	the dual variables. Let $v$ be a positive even integer. Then there is a 
	constant $c_v > 0$ such that
%
%
\begin{equation}
	\label{2four-mult-est-L4st}
	\begin{split}
		\| \left( 1 + | \tau - n^v | 
		\right)^{-\frac{v+1}{4v}}
		\wh{f}\|_{L^2(\zz \times \rr)} \le c_v \|f \|_{L^{4/3}( \ci \times \rr)}.
	\end{split}
\end{equation}
%
%
\end{corollary}
%
Applying H\"{o}lder's inequality to the right hand side of
\eqref{23f}, we obtain the bound
%
%%
\begin{equation}
	\label{24f}
	\begin{split}
		\|\overset{\sim}{C_f} \|_{L^2(\ci \times \rr)} \|C_g \|_{L^4\left( \ci 
		\times \rr 
		\right)}. 
	\end{split}
\end{equation}
%
%%
By Plancherel we have
%
%%
%
%%
\begin{equation}
	\label{25f}
	\begin{split}
		\|\overset{\sim}{C_f} \|_{L^2(\ci \times \rr)}
		& \simeq \|c_f\|_{L^2(\zz \times \rr)}
		\\
		& = \|f \|_{\dot{X}^s}
	\end{split}
\end{equation}
%
%%
while \eqref{2four-mult-conseq} gives
%
%
\begin{equation}
	\label{26f}
	\begin{split}
		\|C_g \|_{L^4(\ci \times \rr)} \lesssim \|g\|_{\dot{X}^s}.
	\end{split}
\end{equation}
%
%
We conclude from \eqref{23f}-\eqref{26f} that
%
%
\begin{equation*}
	\begin{split}
		\| |n|^s \left( 1 + | \tau - n^3 | \right)^{-1/2} \wh{w_{fg}}(n, \tau) 
		 \|_{L^2(\zzdot \times \rr)}
		 \lesssim \|f\|_{\dot{X}^s} \|g\|_{\dot{X}^s}
	\end{split}
\end{equation*}
%
%
which completes the proof.  \qquad \qedsymbol
%
%

\section{Proof of Second Bilinear Estimate}
Recall that for the NLS, one obtains one trilinear estimate as a corollary of
another. Using this as motivation, let us see if we can obtain
\cref{2prop:bilinear-estimate2} as a corollary of
\cref{2prop:prim-bilin-est}. By
duality, it suffices to show that
%
%%
\begin{equation}
	\label{2duality-est}
	\begin{split}
		\sum_{n \in \zzdot}  |n|^{s}
		a_n \int_{\rr} \frac{|\wh{w_{fg}}(n, \tau)|}{1 
		+ | \tau - n^3 |} \ d \tau \lesssim \|f\|_{\dot{X}^s} \|g\|_{\dot{X}^s}
		\|a_n \|_{\ell^2}, \qquad s \ge -1/2.
	\end{split}
\end{equation}
%
%%
By the triangle inequality 
and Cauchy-Schwartz,
%
%%
\begin{equation}
	\label{21m}
	\begin{split}
		& | \sum_{n \in \zzdot} |n|^{s} a_n
		\int_{\rr}\frac{| \wh{w_{fg}}(n, \tau) |}{(1 + | \tau - n^3 |)} \ d \tau |
		\\
		& \le \sum_{n \in \zzdot} \int_{\rr} \frac{| a_n |}{\left( 1 + 
		| \tau - n^3 |
		\right)^{1/2 + \eta}} \times \frac{| n|^s  |
		\wh{w_{fg}}(n, \tau) |}{\left( 
		1 + | \tau - n^3 | \right)^{1/2 - \eta}} \ d \tau
		\\
		& \le \left( \sum_{n \in \zzdot} | a_{n} |^2\int_{\rr} \frac{1}{\left( 1 + |
		\tau - n^3 | \right)^{1 + 2 \eta}} \ d \tau  
		\right)^{1/2} 
		\left ( \sum_{n \in \zzdot}\int_{\rr} \frac{|n|^{2s} | \wh{w_{fg}}(n, \tau) 
		|^2}{\left( 1 + | \tau - n^3 | \right)^{1 -2 \eta}}\ d \tau 
		\right)^{1/2}.
	\end{split}
\end{equation}
%
%%
Applying the change of variable $\tau - n^3
\mapsto \tau'$ we obtain  
%%

\begin{equation*}
	\begin{split}
		& \left( \sum_{n \in \zzdot} | a_{n} |^2\int_{\rr} \frac{1}{\left( 1 + | \tau -
		n^3 | \right)^{1 + 2 \eta}} \ d \tau  
		\right)^{1/2} 
		\\
		& = \left ( \sum_{n \in \zzdot}
		| a_n |^2 
		\int_{\rr} \frac{1}{\left( 1 + | \tau' | \right)^{1 + 2 \eta}} \ d 
		\tau \right)^{1/2}
		\\
		& \simeq \|a_n\|_{\ell^2}, \qquad \eta >0.
		\end{split}
\end{equation*}
However, if we assume $\eta >0$, then
we cannot use \cref{2prop:prim-bilin-est} to bound
\begin{equation*}
	\begin{split}
		\left ( \sum_{n \in \zzdot}\int_{\rr} \frac{|n|^{2s} | \wh{w_{fg}}(n, \tau) 
		|^2}{\left( 1 + | \tau - n^3 | \right)^{1 - 2\eta}}\ d \tau
		\right)^{1/2}. 
	\end{split}
\end{equation*}
%%
%%
\begin{remark}
Hence, unlike the NLS, we have not been able to obtain a second bilinear
estimate as a corollary from the first. Heuristically, this is due to the
derivative in nonlinearity, which is not present in the NLS nonlinearity.
However, one can obtain \eqref{2bilinear-estimate2} for $s>1/2$ as a
corollary of \cref{2prop:prim-bilin-est} by using the ideas
above and by modifying the proof of \cref{2prop:prim-bilin-est} slightly (i.e.,
showing that if $b = \frac{1}{2}^-$, then \eqref{2prim-bilin-est} holds for
$s\ge-\frac{1}{2}^+$). To show that \eqref{2bilinear-estimate2} holds for the
case $s=1/2$, we will have to resort to Kenig-Ponce-Vega~\cite{Kenig:1996yn} techniques.
\end{remark}
%
%
Proceeding, note that by duality, to prove \cref{2prop:bilinear-estimate2} it
suffices to show \eqref{2duality-est} for $s \ge -1/2$. By the symmetry of the convolution, we
consider only cases \eqref{2pigeon-case-1} and \eqref{2pigeon-case-2}.
%
%
\subsection{Case \ref{2pigeon-case-1}.} From the triangle inequality and \eqref{2non-lin-rep-with-bound}, we have
%
%
\begin{equation*}
	\begin{split}
	 |\eqref{2duality-est}|
	& \lesssim \sum_{n \in \zzdot} |a_{n}| \int_{\rr} \sum_{\substack{n_1 \neq 0, n_2 \neq 0
		\\ n_1 +n_2 =n}} \int_{\tau_1 + \tau_2 = \tau} c_f(n_1, \tau_1)
		c_g(n_2, \tau_2)
		\\
		& \times \frac{1}{(1 + | \tau - n^{3} |)^{1/2}(1 + |
		\tau_{1}-n_{1}^{3} |)^{1/2}(1 + | \tau-n_{2}^{3} |^{1/2})} d \tau_1 d \tau_2
		d \tau
	\end{split}
\end{equation*}
%
%
which by Cauchy-Schwartz is bounded by
%
%
\begin{equation}
	\label{210g}
	\begin{split}
		& \sum_{n \in \zzdot} |a_n| \int_{\rr} \left(  \sum_{\substack{n_1 \neq 0, n_2
		\neq 0 \\n_1 +n_2 =n}} \int_{\tau_1 + \tau_2 = \tau} c_{f}^{2}(n_1, \tau_1)
		c_{g}^{2} (n_2, \tau_2) d \tau_1 d \tau_2 \right)^{1/2} 
		\\
		& \times \left( \sum_{\substack{n_1 \neq 0, n_2 \neq 0 \\n_1 +n_2 =n}}
		\int_{\tau_1 + \tau_2 = \tau} \frac{1}{(1 + | \tau - n^{3} |)(1 + | \tau_{1}-n_{1}^{3} |)(1 + |
		\tau_2 -n_{2}^{3} |)} d \tau_1 d \tau_2
		\right)^{1/2} d \tau.
	\end{split}
\end{equation}
%
%
Applying Cauchy-Schwartz again, \eqref{210g} is bounded by
%
%
\begin{align}
	\notag
		& \|\left( \sum_{\substack{n_1 \neq 0, n_2 \neq 0 \\n_1 +n_2 =n}}\int_{\tau_1 + \tau_2 = \tau} c_{f}^{2}(n_1, \tau_1)
		c_{g}^{2} (n_2, \tau_2) d \tau_1 d \tau_2 \right)^{1/2} \|_{L^{2}(\zz \times
		\rr)}
		\\
		\notag
		& \times  \|a_{n}
		\left( \sum_{\substack{n_1 \neq 0, n_2 \neq 0 \\n_1 +n_2
		=n}}\int_{\tau_1 + \tau_2 = \tau} \frac{1}{(1 + | \tau - n^{3} |)(1 + |
		\tau_{1}-n_{1}^{3} |)(1 + | \tau_2 -n_{2}^{3} |)} d \tau_1 d \tau_2
		\right)^{1/2} \|_{L^2(\zz \times \rr)}
		\\
		\notag
		& = \|f\|_{\dot{X}^s} \|g\|_{\dot{X}^s}
		\\
		\label{2holder-term}
		& \times 
		\|a_{n}
		\left( \sum_{\substack{n_1 \neq 0, n_2 \neq 0 \\n_1 +n_2
		=n}}\int_{\tau_1 + \tau_2 = \tau} \frac{1}{(1 + | \tau - n^{3} |)(1 + |
		\tau_{1}-n_{1}^{3} |)(1 + | \tau_2 -n_{2}^{3} |)} d \tau_1 d \tau_2
		\right)^{1/2} \|_{L^2(\zz \times \rr)}.
\end{align}
%
Applying H{\"o}lder then gives
%
%
\begin{equation*}
	\begin{split}
		& \eqref{2holder-term}
		 \le \| a_{n} \|_{\ell^2}
		\\
		& \times \left( \sup_{n \neq 0} \int_{\rr}
		\sum_{\substack{n_1 \neq 0, n_2 \neq 0 \\n_1 +n_2 =n}} \int_{\tau_1 + \tau_2
		= \tau} \frac{1}{(1 + | \tau - n^{3} |)(1 + |
		\tau_{1}-n_{1}^{3} |)(1 + | \tau_2 -n_{2}^{3} |)} d \tau_1 d \tau_2 d \tau
		\right)^{1/2}.
	\end{split}
\end{equation*}
%
%
Hence, to complete the proof for case \eqref{2pigeon-case-1}, it will be enough
to show that 
%
%
%
%
\begin{equation*}
	\begin{split}
		 \sup_{n \neq 0} \int_{\rr}
		\sum_{\substack{n_1 \neq 0, n_2 \neq 0 \\n_1 +n_2 =n}} \int_{\tau_1 + \tau_2
		= \tau} \frac{1}{(1 + | \tau - n^{3} |)(1 + |
		\tau_{1}-n_{1}^{3} |)(1 + | \tau_2 -n_{2}^{3} |)} d \tau_1 d \tau_2 d \tau <\infty
	\end{split}
\end{equation*}
%
%
or, equivalently, that
%
%
\begin{equation}
	\label{212g}
	\begin{split}
		\sup_{n \neq 0} \sum_{\substack{n_1 \neq 0, n_2 \neq 0 \\n_1 +n_2 =n}} \int_{\rr}
		\int_\rr  \frac{1}{(1 + | \tau - n^{3} |)(1 + | \tau_1 - n_{1}^{3} |)(1 + | \tau - \tau_1 -
		n_2^3 |)} d \tau_1 d \tau < \infty.
	\end{split}
\end{equation}
%
%
Following Kenig~\cite{Kenig:1996yn}, we now need the following Calculus lemma.
%
%
%%%%%%%%%%%%%%%%%%%%%%%%%%%%%%%%%%%%%%%%%%%%%%%%%%%%%
%
%
%				 Calculus Lemma
%
%
%%%%%%%%%%%%%%%%%%%%%%%%%%%%%%%%%%%%%%%%%%%%%%%%%%%%%
%
%
\begin{lemma}
	\label{2lem:calc}
 %
 %
 \begin{equation}
	 \label{2calc}
	 \begin{split}
		 \int_{\rr} \frac{1}{(1 + | \theta |)(1 + | a - \theta |)} d \theta \lesssim
		 \frac{\log(2 + | a |)}{1 + | a |}.
	 \end{split}
 \end{equation}
 %
 %
 \end{lemma}
%
%
Applying the lemma with $\theta = \tau_1 - n_1^3$ and $a = \tau - n_1^3 -
n_2^3$, we see that
%
%
\begin{equation*}
	\begin{split}
	\int_{\rr}
		\int_\rr  \frac{1}{(1 + | \tau - n^{3} |)(1 + | \tau - \tau_1 -
		n_2^3 |)} d \tau_1 d \tau \lesssim \frac{\log(2 + | \tau - n_{1}^{3} -
		n_{2}^{3} |)}{1 + | \tau - n_{1}^{3} - n_{2}^{3} |}.
	\end{split}
\end{equation*}
%
%
%
Hence, the left hand side of \eqref{212g} is bounded by
%
\begin{equation*}
	\begin{split}
		\sup_{n \neq 0} \sum_{\substack{n_1 \neq 0, n_2 \neq 0 \\n_1 +n_2 =n}}
		\int_{\rr} \frac{\log(2 + | \tau - n_{1}^{3} -
		n_{2}^{3} |)}{(1 + | \tau - n_{1}^{3} - n_{2}^{3} |)(1 + | \tau - n^{3} |)}
		d \tau	
	\end{split}
\end{equation*}
%
%
or, equivalently, by
%
%
\begin{equation}
	\label{213g}
	\begin{split}
		\sup_{n \neq 0} \sum_{n_1 \neq 0} \int_{\rr} \frac{\log(2 + | \tau -
		n_{1}^{3} - (n - n_1)^{3} |)}{(1 + | \tau - n_{1}^{3} - (n - n_{1})^{3} |)(1
		+ | \tau - n^{3} |)} d \tau.
	\end{split}
\end{equation}
%
%
Now by assumption, for $m \ge 3$
we have  $ |\tau - n^m| \ge
\frac{2^{-2s+1}}{3} | n |^{2s+2} | n_{1} |^{-2s}| n_2
|^{-2s}$. This follows from  \eqref{2number-theory}
with $c = 4s+2$, where the choice of $c$ is motivated by \cref{2rem:s-val}. Hence, \eqref{213g} is bounded by a constant times
%
%
%
%
\begin{equation}
	\label{214g}
	\begin{split}
		& \sup_{n \neq 0} \sum_{n_1 \neq 0}
		\frac{1}{|n|^{+(2s +2)}{|n_1 n_2 |^{+(-s)}}}\int_{\rr} \frac{\log(2 + | \tau
		- n_{1}^m - (n - n_1)^m |)}{(1 + | \tau - n_{1}^m - (n - n_{1})^m
		|)(1 + | \tau - n^m |)^{\frac{1}{2}-}}
		d \tau
		\\
		& \le \sup_{n \neq 0} \sum_{n_1 \neq 0}
		\frac{1}{|n|^{+(2s +2)}{|n_1 n_2 |^{+(-s)}}}		\\
		& \times \sup_{n \neq 0} \sum_{n_1 \neq 0}
		\int_{\rr} \frac{\log(2 + | \tau
		- n_{1}^m - (n - n_1)^m |)}{(1 + | \tau - n_{1}^m - (n - n_{1})^m
		|)(1 + | \tau - n^m |)^{\frac{1}{2}-}}
		d \tau
	\end{split}
\end{equation}
%
%
Observe that for the first sum, the supremum is attained at $n=1$. But then $n_2
= 1 - n_1$, and so $| n_1 n_2 | \gtrsim | n_1 |^2$. Furthermore, we know that 
for any $\ee > 0$, we have $\log (2 + | a |) \le c_{\ee}(1 + | a
|)^{\ee}$. Hence, we bound \eqref{214g} by
%
%
%
%
\begin{equation*}
	\begin{split}
		c_{\ee}  \sum_{n_1 \neq 0} \frac{1}{|n_1|^{+(-2s)}}
		\sup_{n \neq 0} \sum_{n_1 \neq 0} \int_{\rr} \frac{1}{(1 +
		| \tau - n_{1}^m - (n - n_{1})^m |)^{1- \ee}(1 + | \tau - n^m
		|)^{\frac{1}{2}-}} d \tau
	\end{split}
\end{equation*}

%
which due to the estimate
%
%
\begin{equation}
	\label{216g}
	\begin{split}
		(1 + | \tau - n^{3} |)
		& = 1 + \frac{1}{4}| \tau - n^{3} | + \frac{3}{4}| \tau - n^{3} |
		\\
		& \ge 1 + \frac{1}{4}| \tau - n^{3} | + \frac{3}{4} \times
		\frac{1}{3}| n | |n_1 | n - n_1 |
		\\
		& = 1 + \frac{1}{4}| \tau - n^{3} | + \frac{1}{4}| -n^3 + n_1^3 + (n -
		n_1)^3 |
		\\
		& \ge \frac{1}{4}| \tau -n^3 + n_1^3 + (n - n_1)^{3} |
	\end{split}
\end{equation}
%
%
is bounded by
%
%
\begin{equation}
	\label{215g}
	\begin{split}
		& 4 c_{\ee} \sum_{n_1 \neq 0} \frac{1}{|n_1|^{(-2s)+}} \sup_{n \neq 0} \sum_{n_1 \neq 0}\int_{\rr} \frac{1}{(1 +
		| \tau - n_{1}^{3} - (n - n_{1})^{3} |)^{\frac{3}{2}^- - \ee}} d \tau
		\\
		& \lesssim \sum_{n_1 \neq 0} \frac{1}{|n_{1}|^{(-2s)+}} < \infty, \qquad s \ge
		-1/2,
	\end{split}
\end{equation}
%
%
completing the proof. \qquad \qedsymbol
%
%
\subsection{Case \ref{2pigeon-case-2}.} Recalling \eqref{21f}, we have
%
\begin{equation}
	\begin{split}
		& \sum_{n \neq 0} \int_{\rr} a_n |n|^s \left( 1 + | \tau - n^3 | \right)^{-1} | 
		\wh{w_{fg}}(n, \tau) | d \tau
		\\
		& \lesssim \sum_{n \neq 0}  \int_{\rr} a_{n} (1+ | \tau - n^{3} |)^{-1} \wh{\overset{\sim}{C_f} C_g} d
		\tau
	\\	
	& = \sum_{n \neq 0} \int_{\rr} a_{n} (1+ | \tau - n^{3} |)^{-5/8} (1 + | \tau - n^{3}
	|)^{-3/8} \wh{\overset{\sim}{C_f} C_g} d
		\tau
		\\
		& \le \|a_{n} (1 + | \tau - n^{3} |)^{-5/8}\|_{L^2(\zz \times \rr)}  \| (1 +
		| \tau - n^{3} |)^{-3/8} \wh{\overset{\sim}{C_f} C_g}  \|_{L^2(\zz \times
		\rr)}
		%\\
		%&\wh{\overset{\sim}{C_f} C_g}(n, \tau) d \tau
		%\\
		%& \le \|a_n\|_{\ell^2} \|\wh{\overset{\sim}{C_f} C_g}(n, \tau)\|_{L^2(\zz \times \rr)}
		%\\
		%& \simeq \|a_n\|_{\ell^2} \|\overset{\sim}{C_f} C_g\|_{L^2(\ci \times \rr)}
		%\\
		%& \le \|a_n\|_{\ell^2} \|\overset{\sim}{C_f}\|_{L^4(\ci \times \rr)} \|C_g\|_{L^4(\ci \times \rr)}
	\end{split}
\end{equation}
%
%
where the last step follows from Cauchy-Schwartz. A change of variable shows
that
%
%
\begin{equation*}
	\begin{split}
		\|a_{n} (1 + | \tau - n^{3} |)^{-5/8}\|_{L^2(\zz \times \rr)} \lesssim
		\|a_{n}\|_{\ell^2}
	\end{split}
\end{equation*}
%
%
while \eqref{23f}-\eqref{26f} yields the bound
%
%
\begin{equation*}
	\begin{split}
	\| (1 + | \tau - n^{3} |)^{-3/8} \wh{\overset{\sim}{C_f} C_g}  \|_{L^2(\zz
	\times \rr)} \lesssim \|f\|_{\dot{X}^s} \|g\|_{\dot{X}^s}
	\end{split}
\end{equation*}
%
%
completing the proof. \qquad \qedsymbol
%
%
%
\section{Proofs of Lemmas and Estimates}
\begin{proof}[Proof of Cutoff Lemma]
%
%
\begin{equation*}
	\begin{split}
		\lim_{t_{n} \to t} \|u(\cdot, t) - u(\cdot, t_{n})\|_{\dot{H}^s(\ci)} 
		& = \lim_{t_{n} \to t} \|\psi(t) u(\cdot, t) - \psi(t_n) u(\cdot,
		t_{n})\|_{\dot{H}^s(\ci)} 
		\\
		& = \lim_{t_n \to t} \left[ \sum_{n \in \zzdot}| n |
		^{2s} | \psi(t)  \wh{u}(n, t) - \psi(t_n) \wh{u}(n, t_n) |^2 \right]^{1/2}
		\\
		& = \lim_{t_n \to t} \left[ \sum_{n \in \zzdot} | n |^{2s} | \int_{\rr} (e^{it \tau} - e^{it_{n} \tau}) \wh{\psi u}(n,
		\tau) d \tau |^2 \right]^{1/2}.
	\end{split}
\end{equation*}
		It is clear that
		%
		%
		\begin{equation*}
			\begin{split}
				| n |
				^{2s} | \int_{\rr} (e^{it \tau} - e^{it_{n}\tau}) \wh{\psi u}(n, \tau) d \tau |^2 
		& \le 4  | n |^{2s} \left ( \int_{\rr} |\wh{\psi u}(n, \tau)| d \tau
		\right )^2 
	\end{split}
\end{equation*}
and 
%
%
\begin{equation*}
	\begin{split}
 \sum_{n \in \zzdot} | n |^{2s} \left ( \int_{\rr} |\wh{\psi u}(n, \tau)| d \tau
		\right ) ^2 
		& = \|\wh{\psi u}\|_{\dot{\ell}_n^2 L_\tau^1}
		\\
		& \le \|\psi u \|_{Y^s}^2 
	\end{split}
\end{equation*}
which is bounded by assumption.
Applying dominated convergence completes the proof. 
\end{proof}
%
%
%
\begin{proof}[Proof of General Multiplier Estimate]
  We are motivated by a proof of the case $v=2$ by Tzvetkov, as outlined by Tao \cite{Tao:2006el}. Observe that
  %
  %
  \begin{equation*}
  \begin{split}
    \| u \|_{L^{4}_{x}L^{4}_{t}}^{2} 
    &= \| u^{2} \|_{L^{2}_{x}, L^{2}_{t}}
    \\
    & = \| (\sum_{M}u_{M})^{2} \|_{L^{2}_{x}L^{2}_{t}}
  \end{split}
  \end{equation*}
  %
  %
  where $M$ is a dyadic integer, and $u_{M}$ is the portion of $u$ localized to the spacetime frequency region $M \le \langle \tau - k^{2} \rangle \le 2M$. Now 
  %
  %
  \begin{equation*}
  \begin{split}
    | \left( \sum_{M} u_{M} \right)^{2} |
    & = | \sum_{M} u_{M} \sum_{M'}u_{M'} |
    \\
    & = \sum_{M, M'} u_{M} u_{M'}
    \\
    & \simeq \sum_{m \ge 0} \sum_{M} u_{M}u_{2^{m}M}.
  \end{split}
  \end{equation*}
  %
  \begin{framed}
For fixed $M$, we have
  %
  %
  \begin{equation*}
  \begin{split}
    \sum_{M'} u_{M} u_{M'} = \sum_{m \in \zz} u_{M} u_{2^{m}M}
  \end{split}
  \end{equation*}
  %
  %
  and so
  %
  %
  \begin{equation*}
  \begin{split}
    \sum_{M, M'} u_{M}u_{M'}
    & = \sum_{m \in \zz} \sum_{M} u_{M} u_{2^{m}M}
    \\
    & = 2 \sum_{m \ge 0} \sum_{M} u_{M}u_{2^{m}M}.
  \end{split}
  \end{equation*}
  %
\end{framed}
  %
  %
  Also,
  %
  %
  \begin{equation*}
  \begin{split}
    \| u \|^{2}_{X^{0, b}} \sim \sum_{M} M^{2b} \| u_{M} \|^{2}_{L^{2}_{x}L^{2}_{t}}.
  \end{split}
  \end{equation*}
  %
  %
  So, it suffices to show
%
%
\begin{equation*}
\begin{split}
  \| \sum_{m \ge 0} \sum_{M} u_{M} u_{2^{m}M} \|_{L^{2}_{x}L^{2}_{t}}
  \lesssim M^{2b} \| u_{M} \|_{L^{2}_{x}L^{2}_{t}}^{2}
\end{split}
\end{equation*}
%
%
or, by the triangle inequality, that
%
%
\begin{equation*}
\begin{split}
  \sum_{M} \| u_{M} u_{2^{m}M} \|_{L^{2}_{x}L^{2}_{t}}
\lesssim 2^{-\ee m} \sum_{M} M^{2b} \| u_{M} \|_{L^{2}_{x}L^{2}_{t}}^{2}
\end{split}
\end{equation*}
where $\ee > 0$ is some fixed constant.
But by Cauchy-Schwartz,
%
%
\begin{equation*}
\begin{split}
  \sum_{M} M^{b} \| u_{M} \|_{L^{2}_{x}L^{2}_{t}} (2^{m}M)^{b} \| u_{M} \|_{L^{2}_{x}L^{2}_{t}} 
  & \le (\sum_{M} M^{2b} \| u_{M} \|_{L^{2}_{x}L^{2}_{t}}^{2})^{1/2}
  (\sum_{M} (2^{m}M)^{2b} \| u_{2^{m}M} \|_{L^{2}_{x}L^{2}_{t}}^{2})^{1/2}
  \\
  & = \sum_{M} M^{2b} \| u_{M} \|_{L^{2}_{x}L^{2}_{t}}^{2}
\end{split}
\end{equation*}
%
%
and so it will suffice to show
\begin{equation*}
\begin{split}
\| u_{M} u_{2^{m}M} \|_{L^{2}_{x}L^{2}_{t}}
& \lesssim 2^{-\ee m} M^{b} \| u_{M} \|_{L^{2}_{x}L^{2}_{t}}
(2^{m} M)^{b} \| u_{2^{m} M} \|_{L^{2}_{x}L^{2}_{t}}
\\
& = 2^{m(b - \ee)} M^{2b} \| u_{M} \|_{L^{2}_{x}L^{2}_{t}}
\| u_{2^{m} M} \|_{L^{2}_{x}L^{2}_{t}}.
\end{split}
\end{equation*}
%
Now by Parseval, Cauchy-Schwartz, and H\"older
%
%
\begin{equation*}
\begin{split}
\| u_{M} u_{2^{m}M} \|_{L^{2}_{x}L^{2}_{t}}
& \simeq \| \int_{\rr} \sum_{k_{1}} \wh{u_{M}}(\tau - \tau_{1}, k - k_{1}) \wh{u_{2^{m}M}}(\tau_{1}, k_{1}) \|_{L^{2}_{\tau} \ell^{2}_{k}}
\\
& \le \| \left( \int_{\rr} \sum_{k_{1}} | \wh{u_{M}}(\tau - \tau_{1}, k - k_{1}) |^{2} | \wh{u_{2^{m}M}}(\tau_{1}, k_{1}) |^{2} d \tau_{1} \right)^{1/2} 
\\
& \times \left ( \int_{\rr} \sum_{k_{1}} \chi_{M}(\tau - \tau_{1}, k - k_{1}) \chi_{2^{m}M}(\tau_{1}, k_{1}) d \tau_{1} \right ) ^{1/2} \|_{\ell^{2}_{k} L^{2}_{\tau}}
\\
& \le \left ( \sup_{\tau \in \rr, k \in \zz} \int_{\rr} \sum_{k_{1}}
\chi_{M}(\tau - \tau_{1}, k - k_{1}) \chi_{2^{m}M}(\tau_{1}, k_{1}) d \tau_{1} \right )^{1/2}  
\\
& \times \| \left( \int_{\rr} \sum_{k_{1}} | \wh{u_{M}}(\tau - \tau_{1}, k - k_{1}) |^{2} | \wh{u_{2^{m}M}}(\tau_{1}, k_{1}) |^{2} d \tau_{1} \right)^{1/2} \|_{\ell^{2}_{k} L^{2}_{\tau}}. 
\end{split}
\end{equation*}
%
%
where $\chi_{N}(\lambda, n)$ is an indicator function with support on the set of pairs $(\lambda, n)$ satisfying $N \le \langle \lambda - n^{v} \rangle  \le 2N$.  But
%
%
\begin{equation*}
\begin{split}
  \| \left( \int_{\rr} \sum_{k_{1}} | \wh{u_{M}}(\tau - \tau_{1}, k - k_{1}) |^{2} | \wh{u_{2^{m}M}}(\tau_{1}, k_{1}) |^{2} d \tau_{1} \right)^{1/2} \|_{\ell^{2}_{k}L^{2}_{\tau}} 
  & = \| u_{M} \|_{\ell^{2}_{k}L^{2}_{\tau}} \| u_{2^{m} M} \|_{\ell^{2}_{k}L^{2}_{\tau}} 
\end{split}
\end{equation*}
%
%
and so we have reduced to proving that for fixed $k \in \zz, \tau \in \rr$
%
%
\begin{equation*}
\begin{split}
\int_{\rr} \sum_{k_{1}}
\chi_{M}(\tau - \tau_{1}, k - k_{1}) \chi_{2^{m}M}(\tau_{1}, k_{1}) d \tau_{1}  \lesssim 2^{m(2b - 2 \ee)} M^{4b}
\end{split}
\end{equation*}
%
%
where the bound does not depend upon $k$ or $\tau$. Now for fixed $k, \tau$, we invoke Fubini, the symmetry of the convolution, and the support of $\chi_{N}$ to obtain
%
%
\begin{equation*}
\begin{split}
\int_{\rr} \sum_{k_{1}}
\chi_{M}(\tau - \tau_{1}, k - k_{1}) \chi_{2^{m}M}(\tau_{1}, k_{1}) d \tau_{1}  
& = 
\sum_{k_{1}}\int_{\rr} 
\chi_{M}(\tau_{1}, k_{1}) \chi_{2^{m}M}(\tau - \tau_{1}, k - k_{1}) d \tau_{1} 
\\
& = \sum_{k_{1 \in S}} \int_{k_{1}^{v} + M}^{k_{1}^{v} + 2M} d \tau_{1}
\\
& = M | S |
\end{split}
\end{equation*}
%
%
where
%
%
\begin{equation*}
\begin{split}
  S =  \{ & k_{1} \in \zz: \text{for every} \ \tau_{1} \in \rr \ \text{satisfying} \ 
  k_{1}^{v} + M \le \tau_{1} \le k_{1}^{v} + 2M, \ \text{we have}
  \\
  & 2^{m} M \le \langle \tau - \tau_{1} - (k - k_{1})^{v} \rangle  \le 2^{m+1}M \}.
\end{split}
\end{equation*}
%
Therefore, to complete the proof, it will be enough to show that
%
%
\begin{equation*}
\begin{split}
  | S | \lesssim 2^{m(2b - 2 \ee)}M^{4b-1}.
\end{split}
\end{equation*}
%
Proceeding, we observe that we reduce to counting the number of $k_{1}$ such that
%
%
%
\begin{equation*}
\begin{split}
  2^{m} M \le | \tau - k_{1}^{v} - cM  - (k - k_{1})^{v}| \le 2^{m+1}M
\end{split}
\end{equation*}
%
for all $c \in [1,2]$.
This is equivalent to counting the number of $k_{1}$ satisfying one of two equations 
%
%
%
\begin{equation*}
\begin{split}
  & 2^{m}M + cM - \tau \le k_{1}^{v} + (k - k_{1})^{v} \le 2^{m+1}M + cM - \tau,
  \\
  & 
  - 2^{m}M - cM  + \tau \ge   k_{1}^{v} + (k - k_{1})^{v}  \ge -2^{m+1}M - cM + \tau
\end{split}
\end{equation*}
%
%
%
%
If $v$ is even, then $k_{1}^{v}$ and $(k - k_{1})^{v}$ are positive, so we can restrict our attention to counting the number of $k_{1}$ satisfying one of two equations
\begin{equation}
  \label{ijj}
\begin{split}
  & 2^{m}M + cM - \tau \le k_{1}^{v}  \le 2^{m+1}M + cM - \tau,
  \\
  & 
  - 2^{m}M - cM  + \tau \ge   k_{1}^{v}  \ge -2^{m+1}M - cM + \tau
\end{split}
\end{equation}
For this relation to be satisfied, $k_{1}^{v}$ must live in an interval of length $2^{m}M$. Observe that there is an upper bound of  $4 \times \lfloor 2^{m/v} M^{1/v} \rfloor$ on the number of choices of $k_{1}$ in this case, where $\lfloor \cdot \rfloor$ denotes the nearest integer. Restricting $v$ such that $1/v \le (2b - 2 \ee)$ and $1/v \le 4b-1$ completes the proof.
\end{proof}
%
\begin{framed} The case of odd $v$ results in infinitely many $k_{1}$ when $k=0$. Hence, the proof falls apart in this case.
\end{framed}
%
%
%
%
%
\begin{proof}[Proof of \cref{2lem:schwartz-mult}]
Note that
%
%
\begin{equation*}
	\begin{split}
		\wh{\psi f}\left( n, \tau \right)
		& = \wh{\psi}(\cdot) * \wh{f}(n,
		\cdot)(\tau)
		= \int_\rr \wh{\psi}(\tau_1) \wh{f} \left( n, \tau - \tau_1 \right) 
		d\tau_1
	\end{split}
\end{equation*}
%
%
and hence
%
%
\begin{equation}
	\label{19b}
	\begin{split}
		\|\psi f\|_{\dot{X}^s} 
		& = \left( \sum_{n \in \zzdot} |n|^{2s} \int_\rr \left( 1 + | \tau -
		n^{m} | \right) | \int_\rr \wh{\psi}(\tau_1) \wh{f}\left( n, \tau -
		\tau_1
		\right)  d \tau_1 d \tau |^2 \right)^{1/2}
		\\
		& \le \left( \sum_{n \in \zzdot} |n|^{2s} \int_\rr \left( 1 + | \tau -
		n^{m }
		|
		\right) \left( \int_\rr \wh{\psi}\left( \tau_1 \right) \wh{f}\left( n,
		\tau - \tau_1
		\right)  d \tau_1 d \tau \right)^2 \right)^{1/2}.
	\end{split}
\end{equation}
%
%
Using the relation
%
%
\begin{equation*}
	\begin{split}
		1 + | \tau - n^{m } |
		& = 1 + | \tau + \tau_1 - n^{m} |
		\\
		& \le 1 + | \tau_1 | + | \tau - \tau_1 - n^{m} |
		\\
		& \le \left( 1 + | \tau_1 | \right)\left( 1 + | \tau - \tau_1 -
		n^{m} | \right),
	\end{split}
\end{equation*}
%
%
we obtain
%
%
\begin{equation*}
	\begin{split}
		\eqref{19b}
		& \le \left( \sum_{n \in \zzdot} |n|^{2s} \right.
		\\
		& \times \left . \int_\rr \left(
		\int_\rr \left( 1 + | \tau_1 | \right)^{1/2} | \wh{\psi}(\tau_1) |
		\left( 1 + | \tau - \tau_1 - n^{m} | \right)^{1/2} \wh{f}\left( n, \tau
		- \tau_1
		\right)d \tau_1
		\right)^2 d \tau \right)^{1/2}
	\end{split}
\end{equation*}
%
%
which by Minkowski's inequality is bounded by
%
%
\begin{equation}
	\label{18a}
	\begin{split}
		& \left( \sum_{n \in \zzdot} |n|^{2s}  \right.
		\\
		& \times \left. \left( \int_\rr \left[ \int_\rr
		\left( 1 + | \tau_{1} | \right) | \wh{\psi}(\tau_1) |^2 \left( 1 + |
		\tau - \tau_1 - n^{m} |
		\right) | \wh{f}\left( n, \tau - \tau_1 \right) |^2 d \tau_1 
		\right]^{1/2} d \tau \right)^2 \right)^{1/2}.
	\end{split}
\end{equation}
%
%
Using the change of variable $\tau - \tau_1 \to \lambda$ gives
%
%
\begin{equation*}
	\begin{split}
		\eqref{18a}
		& = \left( \sum_{n \in \zzdot} |n|^{2s}\right.
		\\
		& \times \left.  \left( \int_\rr \left[
		\int_\rr \left( 1 + | \tau_1 | \right) | \wh{\psi}\left( \tau_1
		\right) |^2 \left( 1 + | \lambda - n^{m} | \right) | \wh{f} \left( n,
		\lambda
		\right)|^2 d \tau_1 \right]^{1/2} d \lambda \right)^2 \right)^{1/2}
		\\
		& =  \left( \sum_{n \in \zzdot} |n|^{2s} \right.
		\\
		& \times \left. \left( \int_\rr \left( 1 + |
		\tau_1 |
		\right)^{1/2} | \wh{\psi}(\tau_1) | d \tau_1 \left[ \int_\rr \left( 1 + |
		\lambda - n^{m} |
		\right) | \wh{f}\left( n, \lambda \right) |^2 d \lambda \right]^{1/2}
		\right)^2 \right)^{1/2}
		\\
		& = c_{\psi} \left( \sum_{n \in \zzdot} |n|^{2s} \left( \left[ \int_\rr
		\left( 1 + | \lambda - n^{m} | \right) | \wh{f}\left( n, \lambda
		\right) |^2 d \lambda
		\right]^{\cancel{1/2}} \right)^{\cancel{2}} \right)^{1/2}
		\\
		& = c_{\psi} \|f\|_{\dot{X}^s},
	\end{split}
\end{equation*}
%
%
concluding the proof. 
\end{proof}

%
\begin{proof}[Proof of Number Theory Lemma]
First note that
%
\begin{equation*}
		| - n^{3} + n_1^3 + n_2^3|
		 = 3 | n | |n_1 | |n_2 |.
\end{equation*}
%
%
Hence, it will be enough to show that for $c \ge 0$
%
%
\begin{equation*}
	\begin{split}
		| n | |n_1 | |n_2 | \gtrsim | n |^{\frac{2 + c}{2}}| n_1
		|^{\frac{2-c}{2}}| n_2 |^{\frac{2-c}{2}}
	\end{split}
\end{equation*}
%
%
or, dividing through on both sides by $|n| | n_1 | | n_2 |$ and rearranging terms
%
%
\begin{equation*}
	\begin{split}
		| n |^{c/2} \lesssim | n_1 |^{c/2} | n_2 |^{c/2}.
	\end{split}
\end{equation*}
%
%
But
%
%
\begin{equation*}
	\begin{split}
		| n |^{c/2} &= | n_1 + n_2 |^{c/2}
		\\
		& \le (| n_1 | + |n_2|)^{c/2} 
		\\
		& \le (2\max\{|
		n_1 |, | n_2 |)^{c/2}
		\\
		& \le (2|
		n_1 | | n_2 |)^{c/2}
		\\
		& = 2^{c/2} | n_1 |^{c/2} | n_2 |^{c/2}
	\end{split}
\end{equation*}
%
%
where the last step follows from the fact that, for $a, b \in \zz$ 
%\cref{1lem:splitting}. \qquad \qed
%
%\subsection{Proof of \cref{1lem:splitting}.} We have
%%
%%
\begin{equation}
	\label{16a}
	\begin{split}
		| a + b | 
		& \le | a | + | b | 
		\\
		& \le 2\left( \max\{| a |, | b | \}\right)
		\\
		& \le 2 |a| |b|.
	\end{split}
\end{equation} 
This concludes the proof.
\end{proof}
%%
%

\chapter{Well Posedness for the KNLS}
%
%
\section{Introduction}
We consider the mixed nonlinear Schr{\"o}dinger and Korteweg-de Vries (KNLS) initial value problem (ivp)
%
%
\begin{gather}
	\label{lKNLS-eq}
	i\p_t u + \p_x^{m} u + \lambda u \p_x u = 0,
	\\
	\label{lKNLS-init-data}
	u(x,0) = u_0(x), \quad x \in \ci, t \in \rr
\end{gather}
%
%
where $m \ge 4$ is an even integer and $\lambda \in \{-1, 1\}$.
%
%
\begin{definition}
	We say that the KNLS ivp \eqref{lKNLS-eq}-\eqref{lKNLS-init-data} is
	\emph{locally well posed} in
	$X$ if 
	\begin{enumerate}
		\item For every $\vp(x) \in
	B_R$ there exists $T>0$ depending on $R$ and a unique function
	\\
	$u \in C([-T, T],
	X)$ satisfying \eqref{lKNLS-eq} for all $t \in [-T, T]$. 
\item The flow map $u_0 \mapsto u(t)$ is locally uniformly continuous. That is, if $u_0
	\in B_R$, $\{u_{0,n}\} \subset B_R$, and 
	$\|u_0 - u_{0, n} \|_{H^{s}(\ci)} \to 0$, then there exists $T >0$ depending
	on $R$ such that $\|u(\cdot, t) - u_{n}(\cdot,t) \|_{X} \to
	0$ for $t \in [-T, T]$. 
	\end{enumerate}
	Otherwise, we say that the KNLS ivp is \emph{ill-posed}.
\end{definition}
%
%
We are now prepared to state the following result.

%%%%%%%%%%%%%%%%%%%%%%%%%%%%%%%%%%%%%%%%%%%%%%%%%%%%%
%
%
%				 Well Posedness Theorem
%
%
%%%%%%%%%%%%%%%%%%%%%%%%%%%%%%%%%%%%%%%%%%%%%%%%%%%%%
%
%
\begin{theorem}
	\label{lthm:main}
	The KNLS is well-posed in $\dot{H}^s(\ci)$ for $s \ge (2-m)/4$.  
\end{theorem}
%
%
%%%%%%%%%%%%%%%%%%%%%%%%%%%%%%%%%%%%%%%%%%%%%%%%%%%%%
%
%
%				Outline
%
%
%%%%%%%%%%%%%%%%%%%%%%%%%%%%%%%%%%%%%%%%%%%%%%%%%%%%%
%
%
\section{Outline of the Proof of Main Theorem}
%
%
%
%
%
We first derive a weak formulation of the KNLS ivp. 
Let $\ci = [0, 2 \pi]$, and use
the following notation for the Fourier transform
%
%
%
%
\begin{equation}
	\label{lfour-trans-pde}
	\begin{split}
		\widehat{f}(n) = \int_{\ci} e^{-ix n} f(x) \, dx.
	\end{split}
\end{equation}
Let $w(x,t) = u \p_x u$. Applying 
the Fourier transform to the KNLS ivp in the space variable we obtain 
%
%
\begin{gather*}
	\p_t \widehat{u}(n, t) = (-1)^{m/2}i n^m \widehat{u}(n, t) + \lambda i  
	\widehat{w} (n, t),
	\\
	\widehat{u} (n,0) = \widehat{\vp}(n)
\end{gather*}
%
%
which is a globally well-defined relation in $t$ 
and $n$. Note that by time reversal, we similarly have 
\begin{gather*}
	-\p_t \widehat{u}(n, -t) = (-1)^{\frac{m}{2}}i n^m \widehat{u}(n, -t) + \lambda i  
	\widehat{w} (n, -t),
\end{gather*}
or
\begin{gather*}
	\p_t \widehat{u}(n, -t) = (-1)^{\frac{m+2}{2}}i n^m \widehat{u}(n, -t) - \lambda i  
	\widehat{w} (n, -t).
\end{gather*}
Since the sign of $\lambda$ plays no role in the proof of local well-posedness,
we now assume $m/2$ to be odd without loss of generality. 
Multiplying \eqref{lfour-trans-pde} by the integrating factor $e^{itn^m}$ then yields
%%
%%
\begin{equation*}
	\begin{split}
		\left[ e^{ it n^m} \widehat{u}(n) \right]_t = i
		 e^{ it n^m} \widehat{w} (n, t).	
	\end{split}
\end{equation*}
%
%
Integrating from $0$ to $t$, we obtain
%
%
\begin{equation*}
	\begin{split}
		\wh{u}(n, t) = \wh{\vp}(n) e^{- it n^m} + i  
		\int_0^t e^{ i(t' - t) n^m} \wh{w}(n, t') \ 
		dt'.
	\end{split}
\end{equation*}
%
%
Therefore, by Fourier inversion 
%
%
\begin{equation}
	\label{lKNLS-integral-form}
	\begin{split}
		u(x,t) & = \sum_{n \in \zz} \wh{\vp}(n) e^{i\left( xn - t n^m 
		\right)} 
		\\
		& + i \sum_{n \in \zz} \int_0^t e^{i\left[ xn + \left( t' - t 
		\right) n^m \right]} \wh{w}(n, t') \ dt'.
	\end{split}
\end{equation}
%
%
Then it is immediate that \eqref{lKNLS-integral-form} is a weaker 
restatement of the Cauchy-problem \eqref{lKNLS-eq}-\eqref{lKNLS-init-data}, 
since by construction any classical solution of the KNLS 
ivp is a solution to \eqref{lKNLS-integral-form}. 
\\
\\
%
%
We now derive an integral 
equation global in $t$ and equivalent to \eqref{lKNLS-integral-form} for $t 
\in [-T, T]$. Let $\psi(t)$ be a cutoff function symmetric about the 
origin such that $\psi(t) = 1$ for $|t| \le T$ and $\text{supp} \, \psi 
= [-2T, 2T ]$. Multiplying the right hand side of expression
$\eqref{lKNLS-integral-form}$ by $\psi(t)$, we obtain
%
%
\begin{equation}
	\begin{split}
		\label{lcutoff-int-eq}
		u(x, t)
		& = \frac{1}{2 \pi} \psi(t) \sum_{n \in \zz} e^{i(xn - t n^{m })} \widehat{\vp}(n) 
		\\
		& + \frac{i }{2 \pi} \psi(t) \int_0^t \sum_{n \in \zz} 
		e^{i\left[ xn + (t - t')n^m \right]} \wh{w}(n, t') \ dt'.
	\end{split}
\end{equation}
%
%
Noting that $e^{i\left( xn + tn^{m } \right)}$ 
does not depend on $t'$, we may rewrite
%
%
\begin{equation}
	\label{lpre-prim-int-form}
	\begin{split}
		& \frac{i }{2 \pi} \psi(t) \int_0^t \sum_{n \in \zz} 
		e^{i\left[ xn + (t - t') n^m \right]} \wh{w}(n, t') \ dt'
		\\
		& = \frac{i}{2 \pi} \psi(t) \sum_{n \in \zz} e^{i\left( xn + t 
		 n^{m } 
		\right)} \int_0^t e^{- it'n^{m }} \wh{w}(n, t') \ dt'.
	\end{split}
\end{equation}
%%
%%
We remark that this is a \emph{global} relation in $t$. Therefore, by Fourier 
inversion
%
%
%
%
%
%
%
\begin{equation*}
	\begin{split}
		\text{rhs of} \; \eqref{lpre-prim-int-form}
		& = \frac{i}{4 \pi^2} \psi (t) \sum_{n \in \zz} e^{i\left( xn + t 
		 n^m
		\right)} \int_0^t \int_\rr e^{it'\left( \tau - n^m \right) }
		\wh{w}(n, \tau) d \tau dt'
		\\
		& = \frac{i}{4 \pi^2} \psi(t) \sum_{n \in \zz} \int_\rr 
		e^{i\left( xn + tn^m \right)} \frac{e^{it\left( \tau - n^m 
		\right)}-1}{\tau - n^m} \wh{w}(n, \tau) d \tau
	\end{split}
\end{equation*}
%
%
where the last step follows from integration. Substituting
into \eqref{lcutoff-int-eq} we obtain
%
%
\begin{equation}
	\begin{split}
		\label{lcutoff-int-eq-2}
		u(x, t)
		& = \frac{1}{2 \pi} \psi(t) \sum_{n \in \zz} e^{i(xn - tn^{m })} \widehat{\vp}(n) 
		\\
		& + \frac{1}{4 \pi^2} \psi(t) \sum_{n \in \zz} \int_\rr
		e^{i(xn + t n^m)} \frac{e^{it(\tau - n^m)}- 1}{\tau - n^m} 
		\wh{w}(n, \tau) \ d \tau.
	\end{split}
\end{equation}
%
%
%
%
%
Next, we localize near the singular curve $\tau =  n^m$.  Multiplying the
summand of the second term of \eqref{lcutoff-int-eq-2} by $1 + \psi(\tau -
n^m) - \psi(\tau -
n^m) $ and
rearranging terms, we have
%
%
\begin{equation*}
	\begin{split}
		 u(x, t)
		& = \frac{1}{2 \pi} \psi(t) \sum_{n \in \zz} e^{i(xn + t n^{m 
		})} \widehat{\vp}(n) 
		\\
		& + \frac{1}{4 \pi^2} \psi(t) \sum_{n \in \zz} \int_\rr e^{ixn}  
		e^{it \tau} \frac{ 1 - \psi(\tau - n^m) 
		}{\tau - n^m} \wh{w}(n, \tau) \ d \tau
		\\
		& - \frac{1}{4 \pi^2} \psi(t) \sum_{n \in \zz} \int _\rr e^{i(xn + 
		t n^m)}
		 \frac{1- \psi(\tau - n^m)}{\tau - n^m} \wh{w}(n, \tau) \ d \tau
		\\
		& + \frac{1}{4 \pi^2} \psi(t) \sum_{n \in \zz} \int_\rr
		e^{i(xn + t n^m)}
		\frac{\psi(\tau - n^m)\left[ e^{it(\tau - n^m)}-1 
		\right]}{\tau - n^m} \wh{w}(n, \tau) \ d \tau
	\end{split}
\end{equation*}
%
%
which by a power series expansion of $[e^{it(\tau - n^m)}-1]$ simplifies  
to
%
%
\begin{align}
	\label{lmain-int-expression-0}
	& u(x, t) 
		\\
		\label{lmain-int-expression-1}
		& = \frac{1}{2 \pi} \psi(t) \sum_{n \in \zz} e^{i(xn + tn^{m 
		})} \widehat{\vp}(n) 
		\\
		\label{lmain-int-expression-2}
		& + \frac{1}{4 \pi^2} \psi(t) \sum_{n\in \zz} \int_\rr e^{ixn}  
		e^{it \tau} \frac{ 1 - \psi(\tau -  n^m) 
		}{\tau -  n^m} \wh{w}(n, \tau) \ d \tau
		\\
		\label{lmain-int-expression-3}
		& - \frac{1}{4 \pi^2} \psi(t) \sum_{n\in \zz} \int_\rr e^{i(xn + 
		t n^m)}
		 \frac{1- \psi(\tau -  n^m)}{\tau -  n^m} \wh{w}(n, \tau) \ d \tau
		\\
		\label{lmain-int-expression-4}
		& + \frac{1}{4 \pi^2} \psi(t) \sum_{k \ge 1} \frac{i^k t^k}{k!}
		\sum_{n \in \zz} \int_\rr e^{i(xn + t n^m )}
		\psi(\tau -  n^m) (\tau -  n^m)^{k-1} \wh{w}(n, \tau)  
		\\
		& \doteq T(u) \notag
\end{align}
%
%
where $T = T_{\vp}$. We now introduce the following spaces. 

\begin{definition}
	Denote $\dot{Y}^s$ to be the space of all
	functions $u$ on $\ci \times \rr$ with
	bounded norm
\begin{equation}
	\label{lY-s-norm}
	\begin{split}
		\|u\|_{\dot{Y}^s} = \|u\|_{\dot{X}^s} + \|n^s \wh{ u}\|_{ \dot{\ell}^2_n L^1_\tau }
	\end{split}
\end{equation}
%
%
%
%
where
%
\begin{equation}
	\label{lX^s-norm}
	\begin{split}
		& \|u\|_{\dot{X}^s}
		= \left ( \sum_{n\in \zz} |n|^{2s} \int_\rr \left ( 1 + | 
		\tau - n^{m } \right ) | \wh{u} ( n, \tau ) |^2
		\right )^{1/2}
	\end{split}
\end{equation}
and
%
%
\begin{equation}
	\label{lE-norm}
	\|n^s \wh{u}\|_{ \dot{\ell}^2_n L^1_\tau } = \left[ \sum_{n \in \zzdot}| n |^{2s} \left(
	\int_{\rr}| \wh{u}(n, \tau) |d \tau \right)^{2} \right]^{1/2}.
\end{equation}
%
%
%
%
\end{definition}
The $\dot{Y}^s$ spaces have the following important property, whose proof
is provided in the appendix.
\begin{lemma}
	\label{llem:cutoff-loc-soln}
	Let $\psi(t)$ be a smooth cutoff function with $\psi(t) =1$ for $t \in [-T, T]$. If
	$\psi(t)u(x,t) \in \dot{Y}^s$, then $u \in C([-T, T], \dot{H}^s(\ci))$.
\end{lemma}
%
%
We will 
show that for initial data $\vp \in \dot{H}^s(\ci)$, $T$ is a contraction on $B_M 
\subset \dot{Y}^s$, where $B_M$ is the ball centered at the origin of radius $M = 
M_{\vp}> 0$, by estimating the $\dot{Y}^s$
norm of \eqref{lmain-int-expression-1}-\eqref{lmain-int-expression-4}. The 
Picard fixed point theorem will
then yield a unique solution to
\eqref{lmain-int-expression-0}-\eqref{lmain-int-expression-4}. An application of
\cref{llem:cutoff-loc-soln} will then imply the existence of a unique, local
solution $u \in C([-T, T], \dot{H}^s(\ci))$ to the KNLS ivp which coincides with the solution to
\eqref{lmain-int-expression-0}-\eqref{lmain-int-expression-4} on the interval $[-T, T]$. Local Lipschitz continuity of the flow map will follow
from estimates used to establish the contraction mapping. %
%
%%%%%%%%%%%%%%%%%%%%%%%%%%%%%%%%%%%%%%%%%%%%%%%%%%%%%
%
%
%			Proof of Theorem	
%
%
%%%%%%%%%%%%%%%%%%%%%%%%%%%%%%%%%%%%%%%%%%%%%%%%%%%%%
%
%
\section{Proof of Main Theorem}
%
%
%
%%%%%%%%%%%%%%%%%%%%%%%%%%%%%%%%%%%%%%%%%%%%%%%%%%%%%
%
%
%		Estimation of Integral Equality Part 1		
%
%
%%%%%%%%%%%%%%%%%%%%%%%%%%%%%%%%%%%%%%%%%%%%%%%%%%%%%
%
%
%
%
\subsection{Estimate for \eqref{lmain-int-expression-1}.}
%
%
Letting $f(x,t) = \psi(t) \sum_{n \in \zz} e^{i(xn + tn^{m})} 
\wh{\vp}(n)$, we have $\wh{f}(n,t) = \psi(t) \wh{\vp}(n) e^{itn^{m}}$,
from which we obtain
%
%
\begin{equation}
	\label{lfourier-trans-calc}
	\begin{split}
		\wh{f}(n, \tau)
		& = \wh{\vp}(n) \int_\rr e^{-it( \tau - n^{m})} 
		\psi(t) \ d t
		= \wh{\psi}(\tau - n^{m}) \wh{\vp}(n).
	\end{split}
\end{equation}
%
%
%
%
%
%
Since $\wh{\psi}(\xi)$ is Schwartz for $|\xi| \ge T$, we see that 
%
%
\begin{equation}
	\begin{split}
	\label{lmain-int1-est}
		\|\eqref{lmain-int-expression-1}\|_{\dot{Y}^s}
		& = \left (  \sum_{n\in \zz} |n|^s \int_\rr \left( 1 + | \tau - n^{m} 
		| \right )
		| \wh{\psi}(\tau - n^{m}) \wh{\vp}(n) |^2 d \tau \right)^{1/2} 
		\\
		& + \left[ \sum_{n \in \zz }\left( 1 + | n | \right)^{2s} \left( \int_{\rr} |
		\wh{\psi}(\tau - n^{m})\wh{\vp}(n) | d \tau
		\right)^{2} \right]^{1/2}
		\\
		& \le c_{\psi}
		\|\vp\|_{\dot{H}^s(\ci)}.
	\end{split}
\end{equation}
%
%
%
%
\subsection{Estimate for \eqref{lmain-int-expression-2}.}
We now need the following lemma, whose proof is provided in the appendix.
%
%
%%%%%%%%%%%%%%%%%%%%%%%%%%%%%%%%%%%%%%%%%%%%%%%%%%%%%
%
%
%			Schwartz Multiplier	
%
%
%%%%%%%%%%%%%%%%%%%%%%%%%%%%%%%%%%%%%%%%%%%%%%%%%%%%%
%
%
\begin{lemma}
\label{llem:schwartz-mult}
	For $\psi \in S(\rr)$,
%
%
\begin{equation}
	\label{lschwartz-mult}
	\begin{split}
		\|\psi f \|_{\dot{X}^s} \le c_{\psi} \|f \|_{\dot{X}^s}.
	\end{split}
\end{equation}
%
%
\end{lemma}
%
%
Hence,
%
%
\begin{equation}
	\label{lmain-int2-est-X-s-part}
	\begin{split}
		\|\eqref{lmain-int-expression-2}\|_{\dot{X}^s} 
		& \lesssim 
		\left( \| \sum_{n \in \zz} e^{ixn} \int_\rr 
		e^{it \tau} \frac{ 1 - \psi (\tau - n^{m} ) 
		}{\tau - n^{m}} \wh{w}(n, \tau) \ 
		d \tau\|_{\dot{X}^s} \right)^{1/2}
		\\
		& =  \left( \sum_{n \in \zz} |n|^{2s} \int_\rr
		(1 + |\tau - n^{m}|) \left | \frac{1 - \psi(\tau - n^{2 
		})}{\tau - n^{m}} 
		\wh{w}(n, \tau) \right |^2 \ d 
		\tau \right)^{1/2}
		\\
		& \le \left( \sum_{n \in \zz} |n|^{2s} \int_{| \tau - n^{m }| \ge 1}
		(1 + |\tau - n^{m}|) \frac{|\wh{w}(n, \tau)|^2 }{|\tau - n^{m }|^2} 
		\ d 
		\tau \right)^{1/2}
		\\
		& \lesssim  \left( \sum_{n \in 
		\zz} |n|^{2s} \int_\rr
		\frac{|\wh{w}(n, \tau) |^2}{1+ |\tau - 
		n^{m}|} 
		 \ d \tau 
		\right)^{1/2}
		\\
		& \lesssim  \|u\|_{\dot{X}^s}^3
	\end{split}
\end{equation}
%
%
where the last two steps follow from the inequality 
%
\begin{equation}
	\label{lone-plus-ineq}
	\begin{split}
		\frac{1}{|\tau - n^{m}| } \le \frac{2}{1 + |\tau - n^{m}| }, 
		\qquad |\tau - n^{m}| \ge 1
	\end{split}
\end{equation}
%
%
and the following bilinear estimate, whose proof we leave for later.
%
%%%%%%%%%%%%%%%%%%%%%%%%%%%%%%%%%%%%%%%%%%%%%%%%%%%%%
%
%
%				 Bilinear Estimates
%
%
%%%%%%%%%%%%%%%%%%%%%%%%%%%%%%%%%%%%%%%%%%%%%%%%%%%%%
%
%
\begin{proposition}
	\label{lprop:prim-bilin-est}
	For any $b \ge 1/2$, $s \ge \frac{b(3-m)}{2}$ we have
	\begin{equation}
		\left( \sum_{n \in \dot{\zz}} |n|^{2s} \int_\rr
		\frac{|\wh{w_{fg}}(n, \tau) |^2}{\left (1+ |\tau - 
		n^{m}| \right )^{2b}} 
		 \ d \tau 
		\right)^{1/2}
		\lesssim \|f\|_{\dot{X}^s} \|g\|_{\dot{X}^s}
	\end{equation}
	where $w_{fg}(x,t)$ = $\p_x(fg)(x,t)$.
\end{proposition}
%\nocite{*}
%\bibliography{/Users/davidkarapetyan/Documents/math/}
Furthermore,
%
%
%
%
\begin{equation}
	\label{lmain-int-expression-2-Y-s-part}
	\begin{split}
		\|\wh{\eqref{lmain-int-expression-2}} \|_{\dot{\ell}^2_n L^1_\tau}
		& \lesssim \left( \| \sum_{n \in \zz} e^{ixn} \int_\rr 
		e^{it \tau} \frac{ 1 - \psi (\tau - n^{m} ) 
		}{\tau - n^{m}} \wh{w}(n, \tau) \ 
		d \tau\|_{\dot{\ell}^2_n L^1_\tau} \right)^{1/2}
		\\
		& = \left[ \sum_{n \in \zz}|n|^{2s} \left(
		\int_{\rr}\frac{1 - \psi(\tau - n^{m})}{\tau - n^{m}} \wh{w}(n, \tau) d
		\tau \right)^{2} \right]^{1/2}
		\\
		& \lesssim \|f\|_{X^s} \|g\|_{X^s}
	\end{split}
\end{equation}
%
%
where the last step follows from the following bilinear estimate.
%
%%%%%%%%%%%%%%%%%%%%%%%%%%%%%%%%%%%%%%%%%%%%%%%%%%%%%
%
%
%				Second trilinear Estimate 
%
%
%%%%%%%%%%%%%%%%%%%%%%%%%%%%%%%%%%%%%%%%%%%%%%%%%%%%%
%
%
\begin{proposition}
\label{lprop:bilinear-estimate2}
For any $s \ge \frac{3-m}{4}$ we have
%
%
\begin{equation}
	\label{ltrilinear-estimate2}
	\begin{split}
		\left( \sum_{n \in \zzdot} |n|^{2s}  \left ( \int_\rr 
		\frac{|\wh{w_{fg}}(n, \tau) |}{1 + | \tau - n^{m } |}
		 \ d\tau \right)^2  \right)^{1/2} \lesssim \|f\|_{\dot{X}^s} \|g\|_{\dot{X}^s}.
	\end{split}
\end{equation}
\end{proposition}
%
%
Combining \eqref{lmain-int2-est-X-s-part} and
\eqref{lmain-int-expression-2-Y-s-part}, we conclude that
%
%
%
%
\begin{equation}
	\label{lmain-int2-est}
	\begin{split}
		\|\eqref{lmain-int-expression-2}\|_{\dot{Y}^s} \le c_{\psi}\|f\|_{\dot{X}^s} \|g\|_{\dot{X}^s}.
	\end{split}
\end{equation}
%
%
\subsection{Estimate for \eqref{lmain-int-expression-3}.}
Letting $$f(x,t) = \psi(t) \sum_{n \in \zzdot} e^{i\left( xn + tn^{m} \right)} 
\int_\rr \frac{1 - \psi\left( \lambda - n^{m} \right)}{\lambda - n^{m}} 
\wh{w} \left( n, \lambda \right) \ d \lambda,$$ we have
%
%
\begin{equation*}
	\begin{split}
		& \wh{f^x}(n, t) = \psi(t) e^{itn^{m}} \int_\rr
		\frac{1 - \psi\left( \lambda - n^{m} \right)}{\lambda - n^{m}} 
		\wh{w}(n, \lambda) \ d \lambda
	\end{split}
\end{equation*}
and
\begin{equation*}
	\begin{split}
		 \wh{f}\left( n, \tau \right)
		 & = \int_\rr e^{-it\left( \tau - n^{m} 
		\right)} \psi(t) \int_\rr \frac{1 - \psi\left( 
		\lambda - n^{m} 
		\right)}{\lambda - n^{m}} \wh{w}(n, \lambda) \ d \lambda d \tau
		\\
		& = \wh{\psi}\left( \tau - n^{m} \right) \int_\rr 
		\frac{1 - \psi\left( 
		\lambda - n^{m} 
		\right)}{\lambda - n^{m}} \wh{w}(n, \lambda) \ d \lambda.
	\end{split}
\end{equation*}
Therefore,
%
%
\begin{equation*}
	\begin{split}
		& \| \eqref{lmain-int-expression-3} \|_{\dot{X}^s} 
		\\
		& = \left( \sum_{n \in \zzdot} |n|^{2s} \int_\rr \left( 1 + | \tau - n^{m
		} \right ) | | \wh{\psi}\left( \tau - n^{m } \right) |^2 \ d \tau
		\right.
		\\
		& \times \left . |
		\int_\rr \frac{1 - \psi\left( \lambda - n^{m } \right)}{\lambda -
		n^{m }} \wh{w}(n, \lambda) \ d \lambda |^2  \right)^{1/2}
		\\
		& \lesssim \left( \sum_{n \in \zzdot} |n|^{2s} | \int_\rr
		\frac{1 - \psi\left( \lambda - n^{m } \right)}{\lambda - n^{m }}
		\wh{w}(n, \lambda) \ d\lambda |^2 \right)^{1/2}
		\\
		& \le \left( \sum_{n \in \zzdot} |n|^{2s}  \left ( \int_\rr
		\frac{1 - \psi\left( \lambda - n^{m } \right)}{|\lambda - n^{m }|}
		|\wh{w}(n, \lambda) | \ d\lambda \right )^2 \right)^{1/2}
		\\
		& \le \left( \sum_{n \in \zzdot} |n|^{2s}  \left ( \int_{| \lambda - 
		n^{m } | \ge 1}
		\frac{|\wh{w}(n, \lambda) | }{|\lambda - n^{m }|}
		\ d\lambda \right )^2 \right)^{1/2}.
	\end{split}
\end{equation*}
%
%
Applying estimate \eqref{lone-plus-ineq} then gives
%
%%
\begin{equation}
	\label{lmain-int3-est-X-s-part}
	\begin{split}
		\| \eqref{lmain-int-expression-3} \|_{\dot{X}^s}
		& \lesssim \left( \sum_{n \in \zzdot} |n|^{2s}  \left ( \int_\rr
		\frac{|\wh{w}(n, \lambda)| }{1 + |\lambda - n^{m }|}
		 \ d\lambda \right )^2 \right)^{1/2}
		 \\
		& \lesssim \|u\|_{\dot{X}^s}^3
	\end{split}
\end{equation}
%
%%
where the last step follows from \cref{lprop:bilinear-estimate2}.
Furthermore, 
%
%
\begin{equation}
	\label{lmain-int-estimate-3-Y-s-part}
	\begin{split}
		\|\eqref{lmain-int-expression-3}\|_{\dot{\ell}^2_n L^1_\tau}
		& = \left[ \sum_{n \in \zzdot} |n|^{2s} \int_{\rr} |
		\wh{\psi}(\tau - n^{m}) |^{2} \left( \int_{\rr}\frac{1 - \psi(\lambda -
		n^{m})}{\lambda - n^{m}} \wh{w}(n, \lambda) d \lambda \right)^{2} d \tau
		\right]^{1/2}
		\\
		& \le c_{\psi} \left[ \sum_{n \in \zzdot} |n|^{2s} \left(
		\int_{\rr} \frac{1 - \psi(\lambda - n^{m})}{\lambda - n^{m}}
		\wh{w}(n, \lambda) d \lambda
		\right)^{2}\right]^{1/2}
		\\
		& \le 2 c_{\psi} \left[ \sum_{n \in \zzdot} |n|^{2s} \left(
		\int_{\rr} \frac{\wh{w}(n, \lambda) }{1 + |\lambda - n^{m}|}
		d \lambda
		\right)^{2}\right]^{1/2}
		\\
		& \lesssim \|f\|_{\dot{X}^s} \|g\|_{\dot{X}^s} 
	\end{split}
\end{equation}
%
%
where the last two steps follow from \eqref{lone-plus-ineq} and
\cref{lprop:bilinear-estimate2}, respectively. Combining
\eqref{lmain-int3-est-X-s-part} and \eqref{lmain-int-estimate-3-Y-s-part}, we
conclude that
%
%
\begin{equation}
	\label{lmain-int3-est}
	\begin{split}
		\|\eqref{lmain-int-expression-3}\|_{\dot{Y}^s} 
		\lesssim \|f\|_{\dot{X}^s} \|g\|_{\dot{X}^s}.
	\end{split}
\end{equation}
%
%
%
\subsection{Estimate for \eqref{lmain-int-expression-4}.}
Note that
%
%
\begin{equation}
	\label{l1n}
	\begin{split}
		\eqref{lmain-int-expression-4} \simeq \sum_{k \ge 1}
		\frac{i^k}{k!}g_k(x,t)
	\end{split}
\end{equation}
%
%
where 
%
%
\begin{equation*}
	\begin{split}
		& g_k(x,t) = t^k \psi(t) \sum_{n \in \zzdot} e^{i\left( xn + tn^{m}
		\right)} h_k(n),
		\\
		& h_k(n) = \int_\rr \psi \left( \tau - n^{m } \right) \cdot \left(
		\tau - n^{m } \right)^{k -1} \wh{w}(n, \tau) \ d \tau.
	\end{split}
\end{equation*}
%
%
Hence
%
%
\begin{equation*}
	\begin{split}
		\wh{g_k^x}(n, t) = t^{k} \psi(t) e^{i t n^{m }} h_k(n)
	\end{split}
\end{equation*}
%
%
which gives
%
%
\begin{equation*}
	\begin{split}
		\wh{g_k}(n, \tau)
		& = h_k(n) \int_\rr e^{-it\left( \tau - n^{m } \right)}
		t^{k}\psi(t) \ dt
		\\
		& = h_k(n) \wh{t^{k}\psi(t)} \left( \tau - n^{m } \right).
	\end{split}
\end{equation*}
%
%
Applying this to \eqref{l1n}, we obtain
%
%
\begin{equation}
	\label{l2n}
	\begin{split}
		\|\eqref{lmain-int-expression-4}\|_{\dot{X}^s} 
		& \simeq \left( \sum_{n \in \zzdot} |n|^{2s} \int_\rr \left( 1 + | \tau -
		n^{m }
		|
		\right) | \wh{\sum_{k \ge 1} \frac{i^k}{k!}g_k(x,t)} |^2 \ d \tau
		\right)^{1/2}
		\\
		& \le \sum_{k \ge 1} \frac{1}{k!}\left( \sum_{n \in \zzdot} |n|^{2s}
		\int_\rr \left( 1 + | \tau - n^{m } | \right) | \wh{g_k}(n, \tau) |^2 \
		d \tau \right)^{1/2}
		\\
		& = \sum_{k \ge 1} \frac{1}{k!} \left( \sum_{n \in \zzdot} |n|^{2s}
		\int_\rr \left( 1 + | \tau - n^{m } | \right) | h_k(n) \wh{t^k
		\psi(t)} \left( \tau - n^{m } \right) |^2 \ d \tau \right)^{1/2}
		\\
		& = \sum_{k \ge 1} \frac{1}{k!} \left( \sum_{n \in \zzdot} |n|^{2s} |
		h_k(n) |^2 \int_\rr \left( 1 + | \tau - n^{m } | \right) | \wh{t^k
		\psi(t)} \left( \tau - n^{m } \right) |^2 \ d \tau \right)^{1/2}.
	\end{split}
\end{equation}
%
%
Notice that for fixed $n$, the change of variable $\tau - n^{m } \to \tau'$
gives
%
%
\begin{equation}
	\label{l3n}
	\begin{split}
		\int_\rr \left( 1 + | \tau - n^{m } | \right) | \wh{t^{k}
		\psi(t)}\left( \tau - n^{m } \right) |^2 \ d \tau
		& = \int_\rr \left( 1 + |\tau'| \right) | \wh{t^k \psi(t)}(\tau') |^2 \
		d \tau'
		\\
		& \le \int_\rr \left( 1 + |\tau'| \right)^2 | \wh{t^k \psi(t)}(\tau')
		|^2 \ d \tau'
		\\
		& \lesssim \int_\rr \left( 1 + | \tau' |^2 \right) | \wh{t^{k}
		\psi(t)}(\tau') |^2 \ d \tau'
		\\
		& = \|t^k \psi(t) \|_{H^1(\rr)}^2.
	\end{split}
\end{equation}
%
%
But
%
%
\begin{equation}
	\label{l4n}
	\begin{split}
		\|t^k \psi(t) \|_{H^1(\rr)}^2
		& = \left( \|t^k \psi(t)\|_{L^2(\rr)} + \|\p_t \left( t^k \psi(t)
		\right)\|_{L^2(\rr)} \right)^2
		\\
		& \lesssim \|t^{k}\psi(t) \|_{L^2(\rr)}^2 + \|\p_t \left (t^{k}
		\psi(t) \right )\|_{L^2(\rr)}^2
		\\
		& \le \|t^k \psi(t) \|_{L^2(\rr)}^2 + \|t^k \p_t \psi(t)
		\|_{L^2(\rr)}^2 + \|k t^{k -1} \psi(t) \|_{L^2(\rr)}^2
		\\
		& = c_{\psi} + c_{\psi}' + k^2 c_{\psi}''
		\\
		& \lesssim k^2.
	\end{split}
\end{equation}
%
%
Hence, applying \eqref{l3n} and \eqref{l4n} to \eqref{l2n}, we obtain
%
%%
\begin{equation}
	\label{l5n}
	\begin{split}
		\|\eqref{lmain-int-expression-4} \|_{\dot{X}^s}
		& \lesssim
		\sum_{k \ge 1} \frac{k}{k!} \left( \sum_{n \in \zzdot} |n|^{2s} | h_k(n) |^2 
		\right)^{1/2}
		\\
		& \le \sum_{k \ge 1} \frac{k}{k!}
		\cdot \sup_{k \ge 1} \left( \sum_{n \in \zzdot} |n|^{2s} | 
		h_k(n) |^2 \right)^{1/2}
		\\
		& = \sum_{k \ge 1} \frac{k}{k!} \cdot \sup_{k \ge 1} 
		\left( \sum_{n \in \zzdot} |n|^{2s} \int_\rr 
		\psi\left( \tau - n^{m } \right) \cdot \left( \tau - n^{m } 
		\right)^{k -1} \wh{w}(n, \tau) \ d \tau \right)^{1/2}.
	\end{split}
\end{equation}
%
%%
Recall that $\text{supp} \, |\psi| \subset [0, T ]$. Pick $T \le 1$. 
Then $| \psi\left( \tau - n^{m } \right) \cdot \left( \tau - n^{m } \right)^{k 
-1} | \le \chi_{| \tau - n^{m } | \le 1}$ for all $k \ge 1$. Hence, \eqref{l5n} gives
%
%%
\begin{equation*}
	\begin{split}
		\|\eqref{lmain-int-expression-4} \|_{\dot{X}^s} 
		& \lesssim \sum_{k \ge 1} \frac{k}{k!} \cdot \left( \sum_{n \in \zzdot} | 
		\int_{| \tau - n^{m}  |\le 1} | \wh{w}(n, \tau) \ d \tau |^2 
		\right)^{1/2}
	\end{split}
\end{equation*}
%
%%
which by the inequality
%
%%
\begin{equation*}
	\begin{split}
		\frac{1 + | \tau - n^{m } |}{1 + | \tau  - n^{m } |} \le 
		\frac{2}{1 + | \tau - n^{m } |}, \qquad | \tau - n^{m }  | \le 1
	\end{split}
\end{equation*}
%
%%
implies
%
%%
\begin{equation}
\label{lmain-int4-est-X-s-part}
	\begin{split}
		\|\eqref{lmain-int-expression-4}\|_{\dot{X}^s}
		& \lesssim \left( \sum_{n \in \zzdot} | \int_{| \tau - n^{m}| \le 1 }
		\frac{\wh{w}(n, \tau)}{1 + | \tau - n^{m } |} \ d \tau |^2 
		\right)^{1/2}
		\\
		& \le \left( \sum_{n \in \zzdot} | \int_\rr
		\frac{\wh{w}(n, \tau)}{1 + | \tau - n^{m } |} \ d \tau |^2 
		\right)^{1/2} \\
		& \le \left( \sum_{n \in \zzdot} \left( \int_\rr 
		\frac{|\wh{w}(n, \tau)|}{1 + | \tau - n^{m } |}  \ d \tau  \right)^2
		\right)^{1/2} \\
		& \lesssim \|u\|_{\dot{X}^s}^3
	\end{split}
\end{equation}
%
%%
where the last step follows from \cref{lprop:bilinear-estimate2}. Similarly,
we have
%
%
\begin{equation}
\label{lmain-int4-est-Y-s-part}
	\begin{split}
		\|\eqref{lmain-int-expression-4}\|_{\dot{\ell}^2_n L^1_\tau}
		& \simeq \left[ \sum_{n \in
		\zzdot}|n|^{2s} \left( \int_{\rr} | \sum_{k \ge 1}
		\wh{\frac{i^{k}}{k!}g_{k}(x,t)(n, \tau)} |d \tau \right)^{2} \right]^{1/2}
		\\
		& \le \sum_{k \ge 1} \frac{1}{k!} \left[ \sum_{n \in \zzdot} (1 + | n
		|)^{2s} \left( \int_{\rr} | \wh{g}(n, \tau) | d \tau \right)^{2}
		\right]^{1/2}
		\\
		& = \sum_{k \ge 1} \frac{1}{k!} \left[ \sum_{n \in \zzdot} (1 + | n
		|)^{2s} | h_{k}(n) |^2 \left( \int_{\rr} | \wh{t^{k} \psi(t)}(\tau -
		n^{m}) |d \tau \right)^{2} \right]^{1/2}
		\\
		& = c_{\psi} \sum_{k \ge 1} \frac{1}{k!} \left[ \sum_{n \in \zzdot} (1 + | n
		|)^{2s} | h_{k}(n) |^2 \right]^{1/2}
		\\
		& \lesssim \|u\|_{\dot{X}^s}^{3}
	\end{split}
\end{equation}
%
%
where the last step follows from the computations starting from \eqref{l5n}
through \eqref{lmain-int4-est-X-s-part}.
Combining \eqref{lmain-int4-est-X-s-part} and \eqref{lmain-int4-est-Y-s-part}, we
have
%
%
\begin{equation}
\label{lmain-int4-est}
	\begin{split}
		\|\eqref{lmain-int-expression-4}\|_{\dot{Y}^s} \lesssim \|u\|_{\dot{X}^s}^{3}.
	\end{split}
\end{equation}
%
%
Collecting estimates \eqref{lmain-int1-est}, \eqref{lmain-int2-est}, 
\eqref{lmain-int3-est}, and \eqref{lmain-int4-est}, and recalling 
\eqref{lmain-int-expression-1}-\eqref{lmain-int-expression-4}, we see that
$$\|Tu\|_{\dot{Y}^s} \le c_\psi \left( \|\vp \|_{\dot{H}^s(\ci)} + \|u\|_{\dot{X}^s}^3 \right )$$ 
which by the inequality $\|u\|_{\dot{X}^s} \le \|u\|_{\dot{Y}^s}$ yields the following.
%%
%%%%%%%%%%%%%%%%%%%%%%%%%%%%%%%%%%%%%%%%%%%%%%%%%%%%%
%
%% Contraction Proposition
%				 
%%%%%%%%%%%%%%%%%%%%%%%%%%%%%%%%%%%%%%%%%%%%%%%%%%%%%%
%%
%%
%
\begin{proposition}
\label{lprop:contraction}
Let $s \ge \frac{3-m}{4}$. Then
%
%%
\begin{equation*}
	\begin{split}
		\|Tu\|_{\dot{Y}^s} \le c_\psi \left( \|\vp \|_{\dot{H}^s(\ci)} + \|u\|_{\dot{Y}^s}^3 
		\right).
	\end{split}
\end{equation*}
%
%%
\end{proposition}
We will now use \cref{lprop:contraction} to prove local well-posedness for the 
KNLS ivp. Let $c = c_{\psi}^{1/2}$. For given $\vp$, we may choose $\psi$ such
that 
%
%%
\begin{equation*}
	\begin{split}
		\|\vp\|_{\dot{H}^s(\ci)} \le \frac{15}{64c^3}.
	\end{split}
\end{equation*}
%
%%
Then if $\|u\|_{\dot{Y}^s} \le \frac{1}{4c}$, we have
%
%%
\begin{equation*}
	\begin{split}
		\|T u \|_{\dot{Y}^s} 
		& \le c^2 \left[ \frac{15}{64c^3} + \left( 
		\frac{1}{4c} \right)^3 \right]
		=  \frac{1}{4c}.
	\end{split}
\end{equation*}
%
%%
Hence, $T=T_{\vp}$ maps the ball $B\left( 0, \frac{1}{4c} \right) \subset \dot{Y}^s$ into 
itself. Next, note that
%
%%
\begin{equation*}
	\begin{split}
		Tu - Tv = \eqref{lmain-int-expression-2} + \eqref{lmain-int-expression-3} 
		+ \eqref{lmain-int-expression-4}
	\end{split}
\end{equation*}
%
%%
where now $w = u | u |^2 - v | v |^{2}$. Rewriting
%
%%
\begin{equation*}
	\begin{split}
		u | u |^{2} - v | v |^{2}
		& = | u |^2 \left( u -v \right) + v\left( | u 
		|^2 - | v |^2
		\right)
		\\
		& = u \bar u \left( u -v \right) + v u \bar u - v v \bar v
		\\
		& = u \bar u \left( u - v \right) + v \bar u\left( u - v \right) + v 
		\bar u v - v v \bar v
		\\
		& = u \bar u \left( u -v \right) + v \bar u\left( u - v \right) + v v 
		\left( \overline{u -v} \right)
	\end{split}
\end{equation*}
%
%%
the triangle inequality and linearity of the Fourier transform then give
%
%%
\begin{equation*}
	\begin{split}
		| \wh{w}(n, \tau) | = | \mathcal{F}(u | u |^2 - v| v |^2) |
		& \le | \wh{u \overline{u} \left (u -v \right )} | +
		| \wh{v \overline{u} (u -v)} | + |\wh{v v 
		(\overline{u-v})}|
		\\
		& \doteq | \wh{w_1} | + | \wh{w_2} | + | \wh{w_3} |
	\end{split}
\end{equation*}
%
%%
where
%
%%
\begin{equation*}
	\begin{split}
		w_1 = u \bar u \left( u -v \right), \qquad w_2 = v \bar u \left( u -v 
		\right), \qquad w_3 = v v \left( \overline{u -v} \right).
	\end{split}
\end{equation*}
%
%%
Hence, $Tu - Tv = \sum_{\ell=1, 2, 3} 
T_\ell(u, v)$, where
\begin{align}
	\label{lmain-int-exp-mod1}
	& \frac{1}{4 \pi^2} \psi(t) \sum_{n\in \zzdot} \int_\rr e^{ixn}  
		e^{it \tau} \frac{ 1 - \psi(\tau - n^{m}) 
		}{\tau - n^{m}} \wh{w_\ell}(n, \tau) \ d \tau
		\\
		\label{lmain-int-exp-mod2}
		- & \frac{1}{4 \pi^2} \psi(t) \sum_{n\in \zzdot} \int_\rr e^{i(xn + 
		tn^{m})}
		 \frac{1- \psi(\tau - n^{m})}{\tau - n^{m}} \wh{w_\ell}(n, \tau) \ d \tau
		\\
		\label{lmain-int-exp-mod3}
		+ & \frac{1}{4 \pi^2} \psi(t) \sum_{k \ge 1} \frac{i^k t^k}{k!}
		\sum_{n \in \zzdot} \int_\rr e^{i(xn + tn^{m} )}
		\psi(\tau - n^{m}) (\tau - n^{m})^{k-1} \wh{w_\ell}(n, \tau)  
		\\
		\doteq & T_\ell(u). \notag
\end{align}
Repeating the arguments used to estimate 
\eqref{lmain-int-expression-2}-\eqref{lmain-int-expression-4}, we obtain
%
%%
\begin{equation*}
	\begin{split}
		& \|T_1\|_{\dot{Y}^s} \le c_\psi \|u -v \|_{\dot{Y}^s} \|u\|^2_{\dot{Y}^s}
		\\
		& \|T_2\|_{\dot{Y}^s} \le c_\psi \|u -v \|_{\dot{Y}^s} \|u\|_{\dot{Y}^s} \|v\|_{\dot{Y}^s}
		\\
		& \|T_3\|_{\dot{Y}^s} \le c_\psi \|u -v \|_{\dot{Y}^s} \|v\|_{\dot{Y}^s}^2.
	\end{split}
\end{equation*}
%
%%
Therefore,
%
%%
\begin{equation}
	\label{l20a}
	\begin{split}
		\|Tu - Tv \|_{\dot{Y}^s} = & \| \sum T_\ell(u, v) \|_{\dot{Y}^s}
		\\
		& \le c_\psi \|u -v \|_{\dot{Y}^s} \left( \|u\|_{\dot{Y}^s}^2 + 
		\|u\|_{\dot{Y}^s} \|v\|_{\dot{Y}^s} + \|v\|_{\dot{Y}^s}^2 \right)
		\\
		& \le c_\psi \|u -v\|_{\dot{Y}^s} \left( \|u\|_{\dot{Y}^s} + \|v\|_{\dot{Y}^s} \right)^2
		\\
		& = c^2 \|u -v\|_{\dot{Y}^s} \left( \|u\|_{\dot{Y}^s} + \|v\|_{\dot{Y}^s} \right)^2.
	\end{split}
\end{equation}
%
%%
If $u, v \in B(0, \frac{1}{4c}) \subset \dot{Y}^s$, it follows that
%
%%
\begin{equation}
	\label{l21a}
	\begin{split}
		\|Tu - Tv \|_{\dot{Y}^s}
		& \le c^2 \|u -v \|_{\dot{Y}^s} \left( \frac{1}{4c} + 
		\frac{1}{4c} \right)^2
		\\
		& = \frac{1}{4} \|u -v \|_{\dot{Y}^s}. 
	\end{split}
\end{equation}
%
%%
We conclude that $T = T_{\vp}$ is a contraction on the ball $B(0, 
\frac{1}{4c}) \subset \dot{Y}^s$. A Picard iteration and application of 
\cref{llem:cutoff-loc-soln} then yield a unique, local
solution to the KNLS ivp \eqref{lKNLS-eq}-\eqref{lKNLS-init-data}.
\begin{definition}
	We say that the flow map $u_0 \mapsto u(t)$ is \emph{locally Lipschitz} in a Banach
	space $X$ if for
	$$u_0, v_0 \in B_R \doteq \{f: \|f\|_X < R\},$$ there exist $C, T>0$
	depending on $R$ such that $\|u(\cdot, t) - v(\cdot, t)
	\|_X \le C \|u_{0} - v_0 \|_{X}$ for $t \in [-T, T]$. We
	say the flow map is \emph{locally uniformly
	continuous} in $X$ if for
	$u_0, v_0 \in B_R$ there exists $T >0$ depending on $R$ such that for
	$t \in [-T, T]$, $\|u(\cdot, t) - v(\cdot, t) \|_{X} \to
	0$ if $\|u_0 - v_0 \|_{H^{s}(\ci)} \to 0$. 
\end{definition}
%
%
Clearly any locally Lipschitz flow map is locally uniformly continuous. 
Next, we shall establish local Lipschitz continuity in $\dot{Y}^s$ of the flow
map. Let $\vp_1, \vp_2 \subset \dot{H}^s(\ci)$ be given. Choose $\psi$ such that
$\vp_1, \vp_2 \subset B(0, \frac{15}{64c^{3}})$.  Then there exist $u_1, u_2 \in
\dot{Y}^s$ such that $u_1 = T_{\vp_1}$, $u_2 = T_{\vp_2}$, and so
%
%
\begin{equation*}
	\begin{split}
		T_{\vp_1}(u) - T_{\vp_2}(v) = \frac{1}{2\pi} \psi(t) \sum_{n \in
		\zzdot}e^{i\left( xn + tn^{m} \right)} \wh{\vp_1 - \vp_2}(n) + \sum_{\ell
		= 1,2,3} T_{\ell}(u).
	\end{split}
\end{equation*}
%
%
Using an argument similar to \eqref{lfourier-trans-calc}-\eqref{lmain-int1-est},
we obtain
%
%
\begin{equation*}
	\begin{split}
		\| \frac{1}{2\pi} \psi(t) \sum_{n \in
		\zzdot}e^{i\left( xn + tn^{m} \right)} \wh{\vp_1 - \vp_2}(n)\|_{\dot{Y}^s}
		\le c_\psi \|\vp_{1} - \vp_{2}\|_{\dot{Y}^s} 
	\end{split}
\end{equation*}
%
%
Hence, \eqref{l20a}-\eqref{l21a} gives
%
%
\begin{equation*}
	\begin{split}
		\sum_{\ell=1,2,3} T_{\ell}(u,v) \le \frac{1}{4}\|u-v\|_{\dot{Y}^s}.
	\end{split}
\end{equation*}
%
%
Hence,
%
%
\begin{equation*}
	\begin{split}
		\|u -v \|_{\dot{Y}^s} = \|T_{\vp_1}(u) - T_{\vp_2}(v) \|_{\dot{Y}^s} \le c_\psi
		\|\vp_{1} - \vp_{2} \|_{\dot{H}^s\left( \ci \right)}\| +
		\frac{1}{4} \|u -v \|_{\dot{Y}^s}
	\end{split}
\end{equation*}
%
%
which implies
%
%
\begin{equation*}
	\begin{split}
		\frac{3}{4} \|u-v\|_{\dot{Y}^s} \le c_\psi \|\vp_1 - \vp_2 \|_{\dot{H}^s(\ci)}
	\end{split}
\end{equation*}
%
%
or
%
%
\begin{equation*}
	\begin{split}
		\|u -v \|_{\dot{Y}^s} \le \frac{4}{3} c_\psi \|\vp_1 - \vp_2 \|_{\dot{H}^s(\ci)}.
	\end{split}
\end{equation*}
%
%
Applying \cref{llem:cutoff-loc-soln}, we then obtain
%
%
	 %
	 %
	 \begin{equation*}
		 \begin{split}
			\|u(\cdot, t) -v(\cdot, t) \|_{\dot{H}^s(\ci)} \le \frac{4}{3} c_\psi \|\vp_1 -
			\vp_2 \|_{\dot{H}^s(\ci)}, \qquad t \in [-T, T].
		 \end{split}
	 \end{equation*}
	 %
	 %
Hence, the flow map of the KNLS ivp is locally Lipschitz continuous in
$\dot{H}^s(\ci)$. This
concludes the proof of \cref{lthm:main}. \qquad \qedsymbol
%
%
%
%
\section{Proof of Bilinear Estimate}
Note first that $|\wh{w_{fg}}(n, \tau) |  = | n\wh{f} *  \wh{g} 
(n, \tau)|$. It follows that
%
%
\begin{equation}
	\label{lnon-lin-rep}
	\begin{split}
		| \wh{w_{fg}}(n, \tau)|
		& = | \sum_{n_1, n_2,}  \int n\wh{f}\left( n_1,  \tau_1 
\right) \wh{g}\left( n_2, \tau_2  
\right) d \tau_1 d \tau_2 |
\\
& = | \sum_{n_1 \neq 0, n_2 \neq 0}  \int n\wh{f}\left( n_1,  \tau_1 
\right) \wh{g}\left( n_2, \tau_2  
\right) d \tau_1 d \tau_2 | \qquad \text{(due to conservation of mass)}
\\
& \le \sum_{n_1 \neq 0, n_2 \neq 0}   \int | n | \times | \wh{f}\left( n_1, \tau_1 
\right) | \times  | \wh{g}\left( n_2, \tau_2 
\right) | \times  d \tau_1 d \tau_2  
\\
& = \sum_{n_1 \neq 0, n_2 \neq 0} \int | n | \times \frac{c_f\left( n_1, \tau_1 
\right)}{|n_1|^s \left( 1 + | \tau_1 - n_1^{m} | \right)^{b}}
\\
& \times \frac{c_{g}\left( n_2, \tau_2 \right)}{|n_2|^s\left( 1 + | \tau_2 -  n_2^{m }| 
\right)^{b}}
  \ d \tau_1 d \tau_2 
\end{split}
\end{equation}
%
%
where $n = n_1 + n_2$, $\tau = \tau_1 + \tau_2$, and 
%
%
\begin{equation*}
	\begin{split}
		c_\sigma(n, \tau) =
		\begin{cases}
			|n|^s \left( 1 + | \tau - n^{m } |  
		\right)^{b} | \wh{\sigma}\left( n, \tau \right) |, \qquad & n \neq 0
		\\
		0, \qquad & n = 0.
	\end{cases}
	\end{split}
\end{equation*}
%
%
For clarity of notation, let  $\sum_{n_1, n_2}$ denote $\sum_{n_1 \neq 0, n_2
\neq 0}$. From our work above, it follows that 
%
%
\begin{equation}
	\label{lconvo-est-starting-pnt}
	\begin{split}
		 & |n|^s \left( 1 + | \tau - n^{m } | \right)^{-b} | \wh{w_{fg}}\left( 
		n, \tau \right) |
		\\
		& \le \left( 1 + | \tau - n^{m } | \right)^{-b}
		\sum_{n_1, n_2} \int \frac{|n|^{s+1}}{|n_1|^s | n_2|^s} 
		\times \frac{c_f(n_1, \tau_1)}{\left( 1 + | \tau_1 - n_1^{m } | 
		\right)^{b}}
		\\
		& \times
		\frac{c_g(n_2, \tau_2)}{\left( 1 + | \tau_2 - n_2^{m } | 
		\right)^{b}}\ d \tau_1 d \tau_2.
	\end{split}
\end{equation}
%
%
Unlike the mNLS, we must use the smoothing properties of the
principal symbol $\tau - n^m$ regardless of the choice of $s$, since the quantity
%
%
\begin{equation}
	\label{lconvo-multiplier}
	\begin{split}
		\frac{|n|^{s+1}}{|n_1|^s |n_2|^s }
	\end{split}
\end{equation}
%
%
blows up in general, due to the presence of the extra power of $|n|$ coming from the derivative on
the nonlinearity. To utilize the smoothing effects of the principal symbol, we
will need the following two lemmas, whose
proofs are provided in the appendix.
%
%
%
\begin{lemma}
	\label{llem:number-theory1}
	Let $n=n_1 + n_2$ and suppose that $n, n_1, n_2\neq
	0$. Then for any integer $c \ge 0$
%
%
\begin{equation}
	\begin{split}
		\label{lnumber-theory1}
		| - n^{3} + n_1^3 + n_2^3| \ge 2^{-c/2} | n |^{\frac{2+c}{2}} | n_{1}
		|^{\frac{2-c}{2}}| n_2 |^{\frac{2-c}{2}}.
	\end{split}
\end{equation}
%
%
\end{lemma}
%
%
\begin{remark}
	In~\cite{Bourgain-Fourier-transfo}, Bourgain obtains the lower bound $n^2$ for
	the left hand side of \eqref{lnumber-theory}. This is too coarse an estimate,
	as we shall see.
\end{remark}
%
%
%
%
\begin{lemma}
	\label{llem:number-theory}
	Let $n=n_1 + n_2$ and suppose that $n, n_1, n_2\neq
	0$. Then for any integer  $m \ge 3$
%
%
\begin{equation}
	\begin{split}
		\label{lnumber-theory}
		| - n^{m} + n_1^{m} + n_2^{m }| \ge b_{m, c } 
		|n|^{c/2} |n_1|^{\frac{m-1-c}{2}} | n_2 |^{\frac{m-1-c}{2}}
		\end{split}
\end{equation}
%
%
where the constant $b_{m,c}$ depends only on $m$ and $c$. 
\end{lemma}
%
%
%
%\begin{remark}
%	The case $-1/2 \le s \le 0$ is delicate, and must be treated differently from
%	the case $s < -1/2$ in order to obtain the optimal well-posedness results.
%	This is the motivation for having two instead of one number theory lemma.
%\end{remark}
%
%
%
Since $$| \tau - n^{m} - \left( \tau_1 - n_1^{m} 
+ \tau_2 - n_2^{m }  \right ) | = | - n^{m} + n_1^{m} +
n_2^{m }|,$$ by \cref{llem:number-theory} and
the pigeonhole principle we must have one of the 
following.
%
%
\begin{align}
	\label{lpigeon-case-1}
	& |\tau - n^{m }| \ge \frac{c_m}{3} |n|^2 | n_1 |^{m-3} 		\\
		\label{lpigeon-case-2}
	    & | \tau_1 - n_1^{m } | \ge \frac{c_m}{3} |n|^2 | n_1 |^{m-3} ,  
		\\
		\label{lpigeon-case-3}
		& | \tau_2 - n_2^{m } | \ge
		\frac{c_m}{3} |n|^2 | n_1 |^{m-3}.  
\end{align}
%
%
By the symmetry of the convolution, it will be enough to consider only
\eqref{lpigeon-case-1} and \eqref{lpigeon-case-2}.
%
%
%
\subsection{Subcase \eqref{lpigeon-case-1}.} 
We shall need the following, whose proof is provided in the appendix.
%
%
\begin{lemma}
\label{llem:splitting}
	For $k \ge 0$ and $a, b \in {\zz}$, we have
%
%
\begin{equation}
	\label{lsplitting}
	\begin{split}
		\left ( 1 + |a +b | \right)^k \le 2^k \left(1 + | a | \right)^k \left(
		1 + | b | \right)^k.
	\end{split}
\end{equation}
%
%
\end{lemma}
%
Applying the lemma, we obtain
%
%
%
%%
\begin{equation*}
	\begin{split}
		\frac{|n|^{s+1}}{|n_1|^s | n_2|^s } 
		& =\frac{|n_1|^{-s} |n_2|^{-s}
		}{|n|^{-s}}
		\\
		& = \frac{| n_1|^{-s} | n - n_1 |^{-s} }{ | n|^{-s-1}} 
		\\
		& \lesssim \frac{|n_1|^{-s} \cancel{|n|^{-s} }|n_1 |
		^{-s}}{ |n|
		^{\cancel{-s}-1}}
		\\
		& = |n| | n_1 |^{-2s}
	\end{split}
\end{equation*}
%
%%
which in conjunction with \eqref{lpigeon-case-1} implies
%
%%
\begin{equation}
	\label{lconvo-deriv-bound}
	\begin{split}
		\frac{|n|^s}{|n_1|^s 
		| n_2|^s}
		\times
		\frac{1}{1 + | \tau -n^{m} |^{b}}
		& \lesssim  |n| |n_{1} |^{-2s} \times |n|^{-2b} |n_{1}|^{-b(m-3)} 
		\\
		& \lesssim 1, \qquad b\ge 1/2, \ s \ge \frac{b(3-m)}{2}.
	\end{split}  
\end{equation}
%
%
\begin{remark}
	Note that when $m=2$, we have $|-n^{m} + n_{1}^{m} + n_{2}^{m}| = 2| n_1 |
	|n_2|$. Applying the pigeonhole principle as before, we seek to bound 
	%
	%
	\begin{equation*}
		\begin{split}
			\frac{| n |^{s+1}}{| n_1 |^s |n_2|^s} \times \frac{1}{(| n_1 | |n_2
			|)^{1/2}} = \frac{| n |^{s+1}}{|n_{1}|^{s + 1/2}| n_2 |^{s+1/2}}.
		\end{split}
	\end{equation*}
	%
	%
	However, this quantity blows up (simply take $n=1$ and $n_2 \to \infty$).
	Hence, the KDV dispersive techniques fail for the case $m=2$. 
\end{remark}
%
Hence, recalling \eqref{lconvo-est-starting-pnt} and applying estimates 
\eqref{lpigeon-case-1} and \eqref{lconvo-deriv-bound}, we obtain
%
%
\begin{equation}
	\label{lnon-lin-rep-with-bound}
	\begin{split}
		& |n|^s \left( 1 + | \tau - n^{m } | \right)^{b} | 
		\wh{w_{fg}}(n, \tau) | 
		\\
		& \lesssim \sum_{n_1,n_2} \int \frac{c_f(n_1, \tau_1)}{\left( 1 + | 
		\tau_1 -  n_1^{m }| \right)^{b}}
		\times \frac{c_g\left( n_2, \tau_2\right)}{\left( 1 + | \tau_2 -n_2^{m }|
		\right)^{b}}
		\\
		& = \wh{C_f C_g}(n, \tau)
	\end{split}
\end{equation}
%
%
where
\begin{equation*}
	\begin{split}
		C_\sigma(x,t) =
		\left[ \frac{c_\sigma(n, \tau)}{\left( 1 + | \tau - n^{m } | 
		\right)^{b}}\right]^\vee .	
	\end{split}
\end{equation*}

%
%
Therefore, from \eqref{lnon-lin-rep-with-bound}, Plancherel, and generalized 
H\"{o}lder, we obtain
%
%
\begin{equation}
	\label{lgen-holder-bound}
	\begin{split}
		& \| |n|^s \left( 1 + | \tau - n^{m } | \right ) ^{b} \wh{w_{fg}}\left( 
		n, \tau \right) \|_{L^2(\ci \times \rr)}
		\\
		& \lesssim \|\wh{C_f C_g }\left( n, \tau \right) 
		\|_{L^2\left( \zzdot \times \rr \right)}
		\\
		& \simeq \|C_f C_g \|_{L^2\left( \ci \times \rr \right)}
		\\
		& \le \|C_f \|_{L^4(\ci \times \rr)} \|C_g \|_{L^4(\ci \times \rr)}.
	\end{split}
\end{equation}
%
We now need the following Fourier multiplier estimate, whose proof can be found
in \cite{Himonas-Misiolek-2001-A-priori-estimates-for-Schrodinger}.
%
\begin{lemma}
	\label{llem:four-mult-est-L4}
	Let $(x, t) \in \ci \times \rr $ and $(n, \tau) \in \zz \times \rr$ be 
	the dual variables. Let $v$ be a positive even integer. Then there is a 
	constant $c_v > 0$ such that
%
%
\begin{equation}
	\label{lfour-mult-est-L4}
	\begin{split}
		\|f\|_{L^4(\ci \times \rr)} \le c_v \|\left( 1 + | \tau - n^v | 
		\right)^\frac{v+1}{4v} \wh{f} \|_{L^2( \zz \times \rr)}
	\end{split}
\end{equation}
for every test function $f(x, t)$. 
%
%
%
%
\end{lemma}
From the lemma, we see that
%
%
\begin{equation}
	\label{lfour-mult-conseq}
	\begin{split}
		\|C_\sigma\|_{L^4(\ci \times \rr)} 
		& \lesssim \|(1 + | \tau - n^m |)^{\frac{m+1}{4m}} \wh{C_\sigma}
		\|_{L^2(\zz \times \rr)}
		\\
		& = \|c_{\sigma} \|_{L^2(\zz \times \rr)} \qquad (\text{Since} \ \frac{m+1}{4m} \le 1/2 )
		\\
		& = \|\sigma \|_{\dot{X}^s}. 
	\end{split}
\end{equation}
%
%
Applying this to \eqref{lgen-holder-bound} we
conclude that
\begin{equation*}
	\begin{split}
		\| |n|^s \left( 1 + | \tau - n^{m } | \right ) ^{-b} \wh{w_{fg}}\left( 
		n, \tau \right) \|_{L^2(\zzdot \times \rr)}
		& \lesssim \|f\|_{\dot{X}^s} \|g\|_{\dot{X}^s}.
	\end{split}
\end{equation*}
%
%
%
\subsection{Subcase \eqref{lpigeon-case-2}.}
Using a similar argument to that in Subcase \eqref{lpigeon-case-1}, we obtain
%
%
\begin{equation}
	\label{l1f}
	\begin{split}
		 & |n|^s \left( 1 + | \tau - n^{m } | \right)^{b} | \wh{w_{fg}}\left( 
		n, \tau \right) |
		\\
		& \lesssim \left( 1 + | \tau - n^{m } | \right)^{b}
		\sum_{n_1, n_2} \int
		c_f(n_1, \tau_1)
		\times
		\frac{c_g(n_2, \tau_2)}{\left( 1 + | \tau_2 - n_2^{m } | 
		\right)^{b}} 
		\\
		& = \left( 1 + | \tau - n^{m } | \right)^{b} \wh{\overset{\sim}{C_f} C_g}.
	\end{split}
\end{equation}
%
%%
where
%
%
\begin{equation*}
	\begin{split}
		\overset{\sim}{C_\sigma}(x,t) = \left[ c_\sigma(n, \tau) \right]^\vee.
	\end{split}
\end{equation*}
%
%
Hence
%
%%
\begin{equation}
	\label{l3f}
	\begin{split}
		& \| |n|^s \left( 1 + | \tau - n^{m } | \right)^{b} \wh{w_{fg}}(n, \tau) 
		\|_{L^2(\zzdot \times \rr)}
		\\
		& \lesssim \|\left( 1 + | \tau - n^{m} | \right)^{b} 
		\wh{\overset{\sim}{C_f} C_g } \|_{L^2(\zzdot \times \rr)}
		\\
		& =  \|\left( 1 + | \tau - n^{m} | \right)^{b} 
		\wh{\overset{\sim}{C_f} C_g } \|_{L^2(\zz \times \rr)}
		\\
		& \lesssim  \|\overset{\sim}{C_f} C_g  \|_{L^{4/3}(\ci \times \rr)}
	\end{split}
\end{equation}
%
%%
where the last step follows by dualizing \cref{llem:four-mult-est-L4}. More
precisely, we have the following.
\begin{corollary}
	\label{lcor:four-mult-est-L4}
	Let $(x, t) \in \ci \times \rr $ and $(n, \tau) \in \zzdot \times \rr$ be 
	the dual variables. Let $v$ be a positive even integer. Then there is a 
	constant $c_v > 0$ such that
%
%
\begin{equation}
	\label{lfour-mult-est-L4*}
	\begin{split}
		\| \left( 1 + | \tau - n^v | 
		\right)^{-\frac{v+1}{4v}}
		\wh{f}\|_{L^2(\zz \times \rr)} \le c_v \|f \|_{L^{4/3}( \ci \times \rr)}.
	\end{split}
\end{equation}
%
%
\end{corollary}
%
Applying H\"{o}lder's inequality to the right hand side of
\eqref{l3f}, we obtain the bound
%
%%
\begin{equation}
	\label{l4f}
	\begin{split}
		\|\overset{\sim}{C_f} \|_{L^2(\ci \times \rr)} \|C_g \|_{L^4\left( \ci 
		\times \rr 
		\right)}. 
	\end{split}
\end{equation}
%
%%
By Plancherel we have
%
%%
%
%%
\begin{equation}
	\label{l5f}
	\begin{split}
		\|\overset{\sim}{C_f} \|_{L^2(\ci \times \rr)}
		& \simeq \|c_f\|_{L^2(\zz \times \rr)}
		\\
		& = \|f \|_{\dot{X}^s}
	\end{split}
\end{equation}
%
%%
while \eqref{lfour-mult-conseq} gives
%
%
\begin{equation}
	\label{l6f}
	\begin{split}
		\|C_g \|_{L^4(\ci \times \rr)} \lesssim \|g\|_{\dot{X}^s}.
	\end{split}
\end{equation}
%
%
We conclude from \eqref{l3f}-\eqref{l6f} that
%
%
\begin{equation*}
	\begin{split}
		\| |n|^s \left( 1 + | \tau - n^{m } | \right)^{-b} \wh{w_{fg}}(n, \tau) 
		 \|_{L^2(\zzdot \times \rr)}
		 \lesssim \|f\|_{\dot{X}^s} \|g\|_{\dot{X}^s}
	\end{split}
\end{equation*}
%
%
which completes the proof.  \qquad \qedsymbol
%
%

\section{Proof of Second Bilinear Estimate}
Recall that for the mNLS, one obtains one trilinear estimate as a corollary of
another. Using this as motivation, let us see if we can obtain
\cref{lprop:bilinear-estimate2} as a corollary of
\cref{lprop:prim-bilin-est}. By
duality, it suffices to show that
%
%%
\begin{equation}
	\label{lduality-est}
	\begin{split}
		\sum_{n \in \zzdot}  |n|^{s}
		a_n \int_{\rr} \frac{|\wh{w_{fg}}(n, \tau)|}{1 
		+ | \tau - n^{m } |} \ d \tau \lesssim \|f\|_{\dot{X}^s} \|g\|_{\dot{X}^s}
		\|a_n \|_{\ell^2}, \qquad s \ge (3-m)/4 
	\end{split}
\end{equation}
%
%%
By the triangle inequality 
and Cauchy-Schwartz,
%
%%
\begin{equation}
	\label{l1m}
	\begin{split}
		& | \sum_{n \in \zzdot} |n|^{s} a_n
		\int_{\rr}\frac{| \wh{w_{fg}}(n, \tau) |}{(1 + | \tau - n^{m } |)} \ d \tau |
		\\
		& \le \sum_{n \in \zzdot} \int_{\rr} \frac{| a_n |}{\left( 1 + 
		| \tau - n^{m } |
		\right)^{1/2 + \eta}} \times \frac{| n|^s  |
		\wh{w_{fg}}(n, \tau) |}{\left( 
		1 + | \tau - n^{m } | \right)^{1/2 - \eta}} \ d \tau
		\\
		& \le \left( \sum_{n \in \zzdot} | a_{n} |^2\int_{\rr} \frac{1}{\left( 1 + |
		\tau - n^{m } | \right)^{1 + 2 \eta}} \ d \tau  
		\right)^{1/2} 
		\left ( \sum_{n \in \zzdot}\int_{\rr} \frac{|n|^{2s} | \wh{w_{fg}}(n, \tau) 
		|^2}{\left( 1 + | \tau - n^{m } | \right)^{1 -2 \eta}}\ d \tau 
		\right)^{1/2}.
	\end{split}
\end{equation}
%
%%
Applying the change of variable $\tau - n^{m }
\mapsto \tau'$ we obtain  
%%

\begin{equation*}
	\begin{split}
		& \left( \sum_{n \in \zzdot} | a_{n} |^2\int_{\rr} \frac{1}{\left( 1 + | \tau -
		n^{m } | \right)^{1 + 2 \eta}} \ d \tau  
		\right)^{1/2} 
		\\
		& = \left ( \sum_{n \in \zzdot}
		| a_n |^2 
		\int_{\rr} \frac{1}{\left( 1 + | \tau' | \right)^{1 + 2 \eta}} \ d 
		\tau \right)^{1/2}
		\\
		& \simeq \|a_n\|_{\ell^2}, \qquad \eta >0.
		\end{split}
\end{equation*}
However, if we assume $\eta >0$, then
we cannot use \cref{lprop:prim-bilin-est} to bound
\begin{equation*}
	\begin{split}
		\left ( \sum_{n \in \zzdot}\int_{\rr} \frac{|n|^{2s} | \wh{w_{fg}}(n, \tau) 
		|^2}{\left( 1 + | \tau - n^{m } | \right)^{1 - 2\eta}}\ d \tau
		\right)^{1/2}. 
	\end{split}
\end{equation*}
%%
%%
\begin{remark}
Hence, unlike the mNLS, we have not been able to obtain a second bilinear
estimate as a corollary from the first. Heuristically, this is due to the
derivative in nonlinearity, which is not present in the mNLS nonlinearity,
affording us the ``wiggle room''  of a 1/4 derivative for the mNLS (i.e. in the case
of the mNLS, its analogue of \cref{lprop:prim-bilin-est} holds for $b \ge
3/8$.)
\end{remark}
%
%
Let us proceed in a different fashion. By duality, it suffices to show
\eqref{lduality-est}. By the symmetry of the convolution, we consider only cases
\eqref{lpigeon-case-1} and \eqref{lpigeon-case-2}.
%
%
\subsection{Case \eqref{lpigeon-case-1}.} In Progress
%
%
%Recalling \eqref{lnon-lin-rep-with-bound}, we have
%\begin{equation}
%	\begin{split}
%		 \int_{\rr} a_n |n|^s \left( 1 + | \tau - n^{m } | \right)^{b} | 
%		\wh{w_{fg}}(n, \tau) | d \tau
%	 & \lesssim \int_{\rr} a_n \wh{C_f C_g}(n, \tau) d \tau
%		\\
%		& \le \|a_n\|_{\ell^2} \|\wh{C_f C_g}(n, \tau)\|_{L^2(\zz \times \rr)}
%		\\
%		& \simeq \|a_n\|_{\ell^2} \|C_f C_g\|_{L^2(\ci \times \rr)}
%		\\
%		& \le \|a_n\|_{\ell^2} \|C_f\|_{L^4(\ci \times \rr)} \|C_g\|_{L^4(\ci \times \rr)}
%	\end{split}
%\end{equation}
%%
%%
%where the last three steps follow from H{\"o}lder's inequality and
%Plancherel. Applying \eqref{lfour-mult-conseq} then completes the proof for this
%case.
%
%
\subsection{Case \eqref{lpigeon-case-2}.} Recalling \eqref{l3f}, we have
%
\begin{equation}
	\begin{split}
		& \sum_{n \neq 0} \int_{\rr} a_n |n|^s \left( 1 + | \tau - n^{m } | \right)^{-1} | 
		\wh{w_{fg}}(n, \tau) | d \tau
		\\
		& \le \sum_{n \neq 0}  \int_{\rr} a_{n} (1+ | \tau - n^{m} |)^{-1} \wh{\overset{\sim}{C_f} C_g} d
		\tau
	\\	
	& = \sum_{n \neq 0} \int_{\rr} a_{n} (1+ | \tau - n^{m} |)^{-5/8} (1 + | \tau - n^{m}
	|)^{-3/8} \wh{\overset{\sim}{C_f} C_g} d
		\tau
		\\
		& \le \|a_{n} (1 + | \tau - n^{m} |)^{-5/8}\|_{L^2(\zz \times \rr)}  \| (1 +
		| \tau - n^{m} |)^{-3/8} \wh{\overset{\sim}{C_f} C_g}  \|_{L^2(\zz \times
		\rr)}
		%\\
		%&\wh{\overset{\sim}{C_f} C_g}(n, \tau) d \tau
		%\\
		%& \le \|a_n\|_{\ell^2} \|\wh{\overset{\sim}{C_f} C_g}(n, \tau)\|_{L^2(\zz \times \rr)}
		%\\
		%& \simeq \|a_n\|_{\ell^2} \|\overset{\sim}{C_f} C_g\|_{L^2(\ci \times \rr)}
		%\\
		%& \le \|a_n\|_{\ell^2} \|\overset{\sim}{C_f}\|_{L^4(\ci \times \rr)} \|C_g\|_{L^4(\ci \times \rr)}
	\end{split}
\end{equation}
%
%
where the last step follows from Cauchy-Schwartz. A change of variable shows
that
%
%
\begin{equation*}
	\begin{split}
		\|a_{n} (1 + | \tau - n^{m} |)^{-5/8}\|_{L^2(\zz \times \rr)} \lesssim
		\|a_{n}\|_{\ell^2}
	\end{split}
\end{equation*}
%
%
while \eqref{l3f}-\eqref{l6f} yields the bound
%
%
\begin{equation*}
	\begin{split}
	\| (1 + | \tau - n^{m} |)^{-3/8} \wh{\overset{\sim}{C_f} C_g}  \|_{L^2(\zz
	\times \rr)} \lesssim \|f\|_{\dot{X}^s} \|g\|_{\dot{X}^s}
	\end{split}
\end{equation*}
%
%
completing the proof. \qquad \qedsymbol

%
%\begin{equation*}
%	\begin{split}
%				\\
%		& \le \|C_f\|_{L^4(\ci \times \rr)} \|C_g\|_{L^4(\ci \times \rr)}.
%	\end{split}
%\end{equation*}
%
%



\chapter{Well Posedness for the mKDV}
%
%
%
%
\section{Introduction}
We consider the modified Korteweg-de Vries (mKDV) initial value problem (ivp)
%
%
\begin{gather}
	\label{mmKDV-eq}
	\p_t u + \p_x^{m} u + \lambda u \p_x u = 0,
	\\
	\label{mmKDV-init-data}
	u(x,0) = u_0(x), \quad x \in \ci, \ t \in \rr.
\end{gather}
%
%
where $m \in \{3, 5, 7,\dots \}$ and $\lambda \in \{-1, 1\}$.
%
%
\begin{definition}
	We say that the mKDV ivp \eqref{mmKDV-eq}-\eqref{mmKDV-init-data} is
	\emph{locally well posed} in
	$X$ if 
	\begin{enumerate}
		\item For every $\vp(x) \in
	B_R$ there exists $T>0$ depending on $R$ and a unique function
	\\
	$u \in C([-T, T],
	X)$ satisfying \eqref{mmKDV-eq} for all $t \in [-T, T]$. 
\item The flow map $u_0 \mapsto u(t)$ is locally uniformly continuous. That is, if $u_0
	\in B_R$, $\{u_{0,n}\} \subset B_R$, and 
	$\|u_0 - u_{0, n} \|_{H^{s}(\ci)} \to 0$, then there exists $T >0$ depending
	on $R$ such that $\|u(\cdot, t) - u_{n}(\cdot,t) \|_{X} \to
	0$ for $t \in [-T, T]$. 
	\end{enumerate}
	%Otherwise, we say that the mKDV ivp is \emph{ill-posed}.
\end{definition}
%
%
We are now prepared to state the following result.
%
%%%%%%%%%%%%%%%%%%%%%%%%%%%%%%%%%%%%%%%%%%%%%%%%%%%%%
%
%
%				 Well Posedness Theorem
%
%
%%%%%%%%%%%%%%%%%%%%%%%%%%%%%%%%%%%%%%%%%%%%%%%%%%%%%
%
%
\begin{theorem}
	\label{mthm:main}
	The mKDV ivp is well-posed in $\dot{H}^s(\ci)$ for $s \ge \frac{1-m}{4}$.  
\end{theorem}
%
%
%%%%%%%%%%%%%%%%%%%%%%%%%%%%%%%%%%%%%%%%%%%%%%%%%%%%%
%
%
%				Outline
%
%
%%%%%%%%%%%%%%%%%%%%%%%%%%%%%%%%%%%%%%%%%%%%%%%%%%%%%
%
%
\section{Outline of the Proof of Main Theorem}
%
%
%
%
%
We first derive a weak formulation of the mKDV ivp. 
Let $\ci = [0, 2 \pi]$, and use
the following notation for the Fourier transform
%
%
%
%
\begin{equation}
	\label{mfour-trans-pde}
	\begin{split}
		\widehat{f}(n) = \int_{\ci} e^{-ix n} f(x) \, dx.
	\end{split}
\end{equation}
Let $w(x,t) = u \p_x u$. Applying 
the Fourier transform to the mKDV ivp in the space variable we obtain 
%
%
\begin{gather*}
	\p_t \widehat{u}(n, t) = (-1)^{\frac{m-3}{2}}i n^m \widehat{u}(n, t) - \lambda i  
	\widehat{w} (n, t),
	\\
	\widehat{u} (n,0) = \widehat{\vp}(n)
\end{gather*}
%
%
which is a globally well-defined relation in $t$ 
and $n$. Note that by time reversal, we similarly have 
\begin{gather*}
	-\p_t \widehat{u}(n, -t) = (-1)^{\frac{m-3}{2}}i n^m \widehat{u}(n, -t) - \lambda i  
	\widehat{w} (n, -t),
\end{gather*}
or
\begin{gather*}
	\p_t \widehat{u}(n, -t) = (-1)^{\frac{m-1}{2}}i n^m \widehat{u}(n, -t) + \lambda i  
	\widehat{w} (n, -t).
\end{gather*}
Since the sign of $\lambda$ plays no role in the proof of local well-posedness,
we now assume $\frac{m-1}{2}$ to be even without loss of generality. 
Multiplying \eqref{mfour-trans-pde} by the integrating factor $e^{itn^m}$ then yields
%%
%%
\begin{equation*}
	\begin{split}
		\left[ e^{ it n^m} \widehat{u}(n) \right]_t = i
		 e^{ it n^m} \widehat{w} (n, t).	
	\end{split}
\end{equation*}
%
%
Integrating from $0$ to $t$, we obtain
%
%
\begin{equation*}
	\begin{split}
		\wh{u}(n, t) = \wh{\vp}(n) e^{- it n^m} + i  
		\int_0^t e^{ i(t' - t) n^m} \wh{w}(n, t') \ 
		dt'.
	\end{split}
\end{equation*}
%
%
Therefore, by Fourier inversion 
%
%
\begin{equation}
	\label{mmKDV-integral-form}
	\begin{split}
		u(x,t) & = \sum_{n \in \zz} \wh{\vp}(n) e^{i\left( xn - t n^m 
		\right)} 
		\\
		& + i \sum_{n \in \zz} \int_0^t e^{i\left[ xn + \left( t' - t 
		\right) n^m \right]} \wh{w}(n, t') \ dt'.
	\end{split}
\end{equation}
%
%
Then it is immediate that \eqref{mmKDV-integral-form} is a weaker 
restatement of the Cauchy-problem \eqref{mmKDV-eq}-\eqref{mmKDV-init-data}, 
since by construction any classical solution of the mKDV 
ivp is a solution to \eqref{mmKDV-integral-form}. 
\\
\\
%
%
We now derive an integral 
equation global in $t$ and equivalent to \eqref{mmKDV-integral-form} for $t 
\in [-T, T]$. Let $\psi(t)$ be a cutoff function symmetric about the 
origin such that $\psi(t) = 1$ for $|t| \le T$ and $\text{supp} \, \psi 
= [-2T, 2T ]$. Multiplying the right hand side of expression
$\eqref{mmKDV-integral-form}$ by $\psi(t)$, we obtain
%
%
\begin{equation}
	\begin{split}
		\label{mcutoff-int-eq}
		u(x, t)
		& = \frac{1}{2 \pi} \psi(t) \sum_{n \in \zz} e^{i(xn - t n^m)} \widehat{\vp}(n) 
		\\
		& + \frac{i }{2 \pi} \psi(t) \int_0^t \sum_{n \in \zz} 
		e^{i\left[ xn + (t - t')n^m \right]} \wh{w}(n, t') \ dt'.
	\end{split}
\end{equation}
%
%
Noting that $e^{i\left( xn + tn^m \right)}$ 
does not depend on $t'$, we may rewrite
%
%
\begin{equation}
	\label{mpre-prim-int-form}
	\begin{split}
		& \frac{i }{2 \pi} \psi(t) \int_0^t \sum_{n \in \zz} 
		e^{i\left[ xn + (t - t') n^m \right]} \wh{w}(n, t') \ dt'
		\\
		& = \frac{i}{2 \pi} \psi(t) \sum_{n \in \zz} e^{i\left( xn + t 
		 n^m 
		\right)} \int_0^t e^{- it'n^m} \wh{w}(n, t') \ dt'.
	\end{split}
\end{equation}
%%
%%
We remark that this is a \emph{global} relation in $t$. Therefore, by Fourier 
inversion
%
%
%
%
%
%
%
\begin{equation*}
	\begin{split}
		\text{rhs of} \; \eqref{mpre-prim-int-form}
		& = \frac{i}{4 \pi^2} \psi (t) \sum_{n \in \zz} e^{i\left( xn + t 
		 n^m
		\right)} \int_0^t \int_\rr e^{it'\left( \tau - n^m \right) }
		\wh{w}(n, \tau) d \tau dt'
		\\
		& = \frac{i}{4 \pi^2} \psi(t) \sum_{n \in \zz} \int_\rr 
		e^{i\left( xn + tn^m \right)} \frac{e^{it\left( \tau - n^m 
		\right)}-1}{\tau - n^m} \wh{w}(n, \tau) d \tau
	\end{split}
\end{equation*}
%
%
where the last step follows from integration. Substituting
into \eqref{mcutoff-int-eq} we obtain
%
%
\begin{equation}
	\begin{split}
		\label{mcutoff-int-eq-2}
		u(x, t)
		& = \frac{1}{2 \pi} \psi(t) \sum_{n \in \zz} e^{i(xn - tn^m)} \widehat{\vp}(n) 
		\\
		& + \frac{1}{4 \pi^2} \psi(t) \sum_{n \in \zz} \int_\rr
		e^{i(xn + t n^m)} \frac{e^{it(\tau - n^m)}- 1}{\tau - n^m} 
		\wh{w}(n, \tau) \ d \tau.
	\end{split}
\end{equation}
%
%
%
%
%
Next, we localize near the singular curve $\tau =  n^m$.  Multiplying the
summand of the second term of \eqref{mcutoff-int-eq-2} by $1 + \psi(\tau -
n^m) - \psi(\tau -
n^m) $ and
rearranging terms, we have
%
%
\begin{equation*}
	\begin{split}
		 u(x, t)
		& = \frac{1}{2 \pi} \psi(t) \sum_{n \in \zz} e^{i(xn + t n^{m 
		})} \widehat{\vp}(n) 
		\\
		& + \frac{1}{4 \pi^2} \psi(t) \sum_{n \in \zz} \int_\rr e^{ixn}  
		e^{it \tau} \frac{ 1 - \psi(\tau - n^m) 
		}{\tau - n^m} \wh{w}(n, \tau) \ d \tau
		\\
		& - \frac{1}{4 \pi^2} \psi(t) \sum_{n \in \zz} \int _\rr e^{i(xn + 
		t n^m)}
		 \frac{1- \psi(\tau - n^m)}{\tau - n^m} \wh{w}(n, \tau) \ d \tau
		\\
		& + \frac{1}{4 \pi^2} \psi(t) \sum_{n \in \zz} \int_\rr
		e^{i(xn + t n^m)}
		\frac{\psi(\tau - n^m)\left[ e^{it(\tau - n^m)}-1 
		\right]}{\tau - n^m} \wh{w}(n, \tau) \ d \tau
	\end{split}
\end{equation*}
%
%
which by a power series expansion of $[e^{it(\tau - n^m)}-1]$ simplifies  
to
%
%
\begin{align}
	\label{mmain-int-expression-0}
	& u(x, t) 
		\\
		\label{mmain-int-expression-1}
		& = \frac{1}{2 \pi} \psi(t) \sum_{n \in \zz} e^{i(xn + tn^{m 
		})} \widehat{\vp}(n) 
		\\
		\label{mmain-int-expression-2}
		& + \frac{1}{4 \pi^2} \psi(t) \sum_{n\in \zz} \int_\rr e^{ixn}  
		e^{it \tau} \frac{ 1 - \psi(\tau -  n^m) 
		}{\tau -  n^m} \wh{w}(n, \tau) \ d \tau
		\\
		\label{mmain-int-expression-3}
		& - \frac{1}{4 \pi^2} \psi(t) \sum_{n\in \zz} \int_\rr e^{i(xn + 
		t n^m)}
		 \frac{1- \psi(\tau -  n^m)}{\tau -  n^m} \wh{w}(n, \tau) \ d \tau
		\\
		\label{mmain-int-expression-4}
		& + \frac{1}{4 \pi^2} \psi(t) \sum_{k \ge 1} \frac{i^k t^k}{k!}
		\sum_{n \in \zz} \int_\rr e^{i(xn + t n^m )}
		\psi(\tau -  n^m) (\tau -  n^m)^{k-1} \wh{w}(n, \tau)  
		\\
		& \doteq T(u) \notag
\end{align}
%
%
where $T = T_{\vp}$. We now introduce the following spaces. 
%
\begin{definition}
	Denote $\dot{Y}^s$ to be the space of all
	functions $u$ on $\ci \times \rr$ with
	bounded norm
\begin{equation}
	\label{mY-s-norm}
	\begin{split}
		\|u\|_{\dot{Y}^s} = \|u\|_{\dot{X}^s} + \||n|^s \wh{ u}\|_{ \dot{\ell}^2_n L^1_\tau }
	\end{split}
\end{equation}
%
%
%
%
where
%
\begin{equation}
	\label{mX^s-norm}
	\begin{split}
		& \|u\|_{\dot{X}^s}
		= \left ( \sum_{n\in \zz} |n|^{2s} \int_\rr \left ( 1 + | 
		\tau - n^m \right ) | \wh{u} ( n, \tau ) |^2
		\right )^{1/2}
	\end{split}
\end{equation}
and
%
%
\begin{equation}
	\label{mE-norm}
	\| | n |^{s} \wh{u}\|_{ \dot{\ell}^2_n L^1_\tau } = \left[ \sum_{n \in \zzdot}| n |^{2s} \left(
	\int_{\rr}| \wh{u}(n, \tau) |d \tau \right)^{2} \right]^{1/2}.
\end{equation}
%
%
%
%
\end{definition}
The $\dot{Y}^s$ spaces have the following important property, whose proof
is provided in the appendix.
\begin{lemma}
	\label{mlem:cutoff-loc-soln}
	Let $\psi(t)$ be a smooth cutoff function with $\psi(t) =1$ for $t \in [-T, T]$. If
	$\psi(t)u(x,t) \in \dot{Y}^s$, then $u \in C([-T, T], \dot{H}^s(\ci))$.
\end{lemma}
%
%
We will 
show that for initial data $\vp \in \dot{H}^s(\ci)$, $T$ is a contraction on $B_M 
\subset \dot{Y}^s$, where $B_M$ is the ball centered at the origin of radius $M = 
M_{\vp}> 0$, by estimating the $\dot{Y}^s$
norm of \eqref{mmain-int-expression-1}-\eqref{mmain-int-expression-4}. The 
Picard fixed point theorem will
then yield a unique solution to
\eqref{mmain-int-expression-0}-\eqref{mmain-int-expression-4}. An application of
\cref{mlem:cutoff-loc-soln} will then imply the existence of a unique, local
solution $u \in C([-T, T], \dot{H}^s(\ci))$ to the mKDV ivp which coincides with the solution to
\eqref{mmain-int-expression-0}-\eqref{mmain-int-expression-4} on the interval $[-T, T]$. Local Lipschitz continuity of the flow map will follow
from estimates used to establish the contraction mapping. %
%
%%%%%%%%%%%%%%%%%%%%%%%%%%%%%%%%%%%%%%%%%%%%%%%%%%%%%
%
%
%			Proof of Theorem	
%
%
%%%%%%%%%%%%%%%%%%%%%%%%%%%%%%%%%%%%%%%%%%%%%%%%%%%%%
%
%
\section{Proof of Main Theorem}
%
%
%
%%%%%%%%%%%%%%%%%%%%%%%%%%%%%%%%%%%%%%%%%%%%%%%%%%%%%
%
%
%		Estimation of Integral Equality Part 1		
%
%
%%%%%%%%%%%%%%%%%%%%%%%%%%%%%%%%%%%%%%%%%%%%%%%%%%%%%
%
%
%
%
\subsection{Estimate for
\ref{mmain-int-expression-1}.}
%
%
Letting $f(x,t) = \psi(t) \sum_{n \in \zz} e^{i(xn + tn^m)} 
\wh{\vp}(n)$, we have $\wh{f}(n,t) = \psi(t) \wh{\vp}(n) e^{itn^m}$,
from which we obtain
%
%
\begin{equation}
	\label{mfourier-trans-calc}
	\begin{split}
		\wh{f}(n, \tau)
		& = \wh{\vp}(n) \int_\rr e^{-it( \tau - n^m)} 
		\psi(t) \ d t
		= \wh{\psi}(\tau - n^m) \wh{\vp}(n).
	\end{split}
\end{equation}
%
%
%
%
%
%
Since $\wh{\psi}(\xi)$ is Schwartz for $|\xi| \ge T$, we see that 
%
%
\begin{equation}
	\begin{split}
	\label{mmain-int1-est}
		\|\eqref{mmain-int-expression-1}\|_{\dot{Y}^s}
		& = \left (  \sum_{n\in \zz} |n|^{2s} \int_\rr \left( 1 + | \tau - n^m 
		| \right )
		| \wh{\psi}(\tau - n^m) \wh{\vp}(n) |^2 d \tau \right)^{1/2} 
		\\
		& + \left[ \sum_{n \in \zz }| n |^{2s} \left( \int_{\rr} |
		\wh{\psi}(\tau - n^m)\wh{\vp}(n) | d \tau
		\right)^{2} \right]^{1/2}
		\\
		& \le c_{\psi}
		\|\vp\|_{\dot{H}^s(\ci)}.
	\end{split}
\end{equation}
%
%
%
%
\subsection{Estimate for
\ref{mmain-int-expression-2}.}
We now need the following lemma, whose proof is provided in the appendix.
%
%
%%%%%%%%%%%%%%%%%%%%%%%%%%%%%%%%%%%%%%%%%%%%%%%%%%%%%
%
%
%			Schwartz Multiplier	
%
%
%%%%%%%%%%%%%%%%%%%%%%%%%%%%%%%%%%%%%%%%%%%%%%%%%%%%%
%
%
\begin{lemma}
\label{mlem:schwartz-mult}
	For $\psi \in S(\rr)$,
%
%
\begin{equation}
	\label{mschwartz-mult}
	\begin{split}
		\|\psi f \|_{\dot{X}^s} \le c_{\psi} \|f \|_{\dot{X}^s}.
	\end{split}
\end{equation}
%
%
\end{lemma}
%
%
Hence,
%
%
\begin{equation}
	\label{mmain-int2-est-X-s-part}
	\begin{split}
		\|\eqref{mmain-int-expression-2}\|_{\dot{X}^s} 
		& \lesssim 
		\left( \| \sum_{n \in \zz} e^{ixn} \int_\rr 
		e^{it \tau} \frac{ 1 - \psi (\tau - n^m ) 
		}{\tau - n^m} \wh{w}(n, \tau) \ 
		d \tau\|_{\dot{X}^s} \right)^{1/2}
		\\
		& =  \left( \sum_{n \in \zz} |n|^{2s} \int_\rr
		(1 + |\tau - n^m|) \left | \frac{1 - \psi(\tau - n^{2 
		})}{\tau - n^m} 
		\wh{w}(n, \tau) \right |^2 \ d 
		\tau \right)^{1/2}
		\\
		& \le \left( \sum_{n \in \zz} |n|^{2s} \int_{| \tau - n^m| \ge 1}
		(1 + |\tau - n^m|) \frac{|\wh{w}(n, \tau)|^2 }{|\tau - n^m|^2} 
		\ d 
		\tau \right)^{1/2}
		\\
		& \lesssim  \left( \sum_{n \in 
		\zz} |n|^{2s} \int_\rr
		\frac{|\wh{w}(n, \tau) |^2}{1+ |\tau - 
		n^m|} 
		 \ d \tau 
		\right)^{1/2}
		\\
		& \lesssim  \|u\|_{\dot{X}^s}^2
	\end{split}
\end{equation}
%
%
where the last two steps follow from the inequality 
%
\begin{equation}
	\label{mone-plus-ineq}
	\begin{split}
		\frac{1}{|\tau - n^m| } \le \frac{2}{1 + |\tau - n^m| }, 
		\qquad |\tau - n^m| \ge 1
	\end{split}
\end{equation}
%
%
and the following bilinear estimate, whose proof we leave for later.
%
%%%%%%%%%%%%%%%%%%%%%%%%%%%%%%%%%%%%%%%%%%%%%%%%%%%%%
%
%
%				 Bilinear Estimates
%
%
%%%%%%%%%%%%%%%%%%%%%%%%%%%%%%%%%%%%%%%%%%%%%%%%%%%%%
%
%
\begin{proposition}
	\label{mprop:prim-bilin-est}
	For any $s \ge \frac{1-m}{4}$ we have
	\begin{equation}
		\label{mprim-bilin-est}
		\left( \sum_{n \in \dot{\zz}} |n|^{2s} \int_\rr
		\frac{|\wh{w_{fg}}(n, \tau) |^2}{1+ |\tau - 
		n^m| } 
		 \ d \tau 
		\right)^{1/2}
		\lesssim \|f\|_{\dot{X}^s} \|g\|_{\dot{X}^s}
	\end{equation}
	where $w_{fg}(x,t)$ = $\p_x(fg)(x,t)$.
\end{proposition}
Furthermore,
%
%
%
%
\begin{equation}
	\label{mmain-int-expression-2-Y-s-part}
	\begin{split}
		\| | n |^{s} \wh{\eqref{mmain-int-expression-2}} \|_{\dot{\ell}^2_n L^1_\tau}
		& \lesssim 
		\left[ \sum_{n \in \zz}|n|^{2s} \left(
		\int_{\rr}\frac{1 - \psi(\tau - n^m)}{\tau - n^m} \wh{w}(n, \tau) d
		\tau \right)^{2} \right]^{1/2}
		\\
		& \lesssim \|f\|_{X^s} \|g\|_{X^s}
	\end{split}
\end{equation}
%
%
where the last step follows from \eqref{mone-plus-ineq} and
the following bilinear estimate.
%
%%%%%%%%%%%%%%%%%%%%%%%%%%%%%%%%%%%%%%%%%%%%%%%%%%%%%
%
%
%				Second trilinear Estimate 
%
%
%%%%%%%%%%%%%%%%%%%%%%%%%%%%%%%%%%%%%%%%%%%%%%%%%%%%%
%
%
\begin{proposition}
\label{mprop:bilinear-estimate2}
For any $s \ge \frac{1-m}{4}$ we have
%
%
\begin{equation}
	\label{mbilinear-estimate2}
	\begin{split}
		\left( \sum_{n \in \zzdot} |n|^{2s}  \left ( \int_\rr 
		\frac{|\wh{w_{fg}}(n, \tau) |}{1 + | \tau - n^m |}
		 \ d\tau \right)^2  \right)^{1/2} \lesssim \|f\|_{\dot{X}^s} \|g\|_{\dot{X}^s}.
	\end{split}
\end{equation}
\end{proposition}
%
%
Combining \eqref{mmain-int2-est-X-s-part} and
\eqref{mmain-int-expression-2-Y-s-part}, we conclude that
%
%
%
%
\begin{equation}
	\label{mmain-int2-est}
	\begin{split}
		\|\eqref{mmain-int-expression-2}\|_{\dot{Y}^s} \le c_{\psi}\|f\|_{\dot{X}^s} \|g\|_{\dot{X}^s}.
	\end{split}
\end{equation}
%
%
\subsection{Estimate for
\ref{mmain-int-expression-3}.}
Letting $$f(x,t) = \psi(t) \sum_{n \in \zzdot} e^{i\left( xn + tn^m \right)} 
\int_\rr \frac{1 - \psi\left( \lambda - n^m \right)}{\lambda - n^m} 
\wh{w} \left( n, \lambda \right) \ d \lambda,$$ we have
%
%
\begin{equation*}
	\begin{split}
		& \wh{f^x}(n, t) = \psi(t) e^{itn^m} \int_\rr
		\frac{1 - \psi\left( \lambda - n^m \right)}{\lambda - n^m} 
		\wh{w}(n, \lambda) \ d \lambda
	\end{split}
\end{equation*}
and
\begin{equation*}
	\begin{split}
		 \wh{f}\left( n, \tau \right)
		 & = \int_\rr e^{-it\left( \tau - n^m 
		\right)} \psi(t) \int_\rr \frac{1 - \psi\left( 
		\lambda - n^m 
		\right)}{\lambda - n^m} \wh{w}(n, \lambda) \ d \lambda d \tau
		\\
		& = \wh{\psi}\left( \tau - n^m \right) \int_\rr 
		\frac{1 - \psi\left( 
		\lambda - n^m 
		\right)}{\lambda - n^m} \wh{w}(n, \lambda) \ d \lambda.
	\end{split}
\end{equation*}
Therefore,
%
%
\begin{equation*}
	\begin{split}
		& \| \eqref{mmain-int-expression-3} \|_{\dot{X}^s} 
		\\
		& = \left( \sum_{n \in \zzdot} |n|^{2s} \int_\rr \left( 1 + | \tau - n^{m
		} \right ) | | \wh{\psi}\left( \tau - n^m \right) |^2 \ d \tau
		\right.
		\\
		& \times \left . |
		\int_\rr \frac{1 - \psi\left( \lambda - n^m \right)}{\lambda -
		n^m} \wh{w}(n, \lambda) \ d \lambda |^2  \right)^{1/2}
		\\
		& \lesssim \left( \sum_{n \in \zzdot} |n|^{2s} | \int_\rr
		\frac{1 - \psi\left( \lambda - n^m \right)}{\lambda - n^m}
		\wh{w}(n, \lambda) \ d\lambda |^2 \right)^{1/2}
		\\
		& \le \left( \sum_{n \in \zzdot} |n|^{2s}  \left ( \int_\rr
		\frac{1 - \psi\left( \lambda - n^m \right)}{|\lambda - n^m|}
		|\wh{w}(n, \lambda) | \ d\lambda \right )^2 \right)^{1/2}
		\\
		& \le \left( \sum_{n \in \zzdot} |n|^{2s}  \left ( \int_{| \lambda - 
		n^m | \ge 1}
		\frac{|\wh{w}(n, \lambda) | }{|\lambda - n^m|}
		\ d\lambda \right )^2 \right)^{1/2}.
	\end{split}
\end{equation*}
%
%
Applying estimate \eqref{mone-plus-ineq} then gives
%
%%
\begin{equation}
	\label{mmain-int3-est-X-s-part}
	\begin{split}
		\| \eqref{mmain-int-expression-3} \|_{\dot{X}^s}
		& \lesssim \left( \sum_{n \in \zzdot} |n|^{2s}  \left ( \int_\rr
		\frac{|\wh{w}(n, \lambda)| }{1 + |\lambda - n^m|}
		 \ d\lambda \right )^2 \right)^{1/2}
		 \\
		& \lesssim \|u\|_{\dot{X}^s}^2
	\end{split}
\end{equation}
%
%%
where the last step follows from \cref{mprop:bilinear-estimate2}.
Furthermore, 
%
%
\begin{equation}
	\label{mmain-int-estimate-3-Y-s-part}
	\begin{split}
		\| | n |^s \wh{\eqref{mmain-int-expression-3}}\|_{\dot{\ell}^2_n L^1_\tau}
		& = \left[ \sum_{n \in \zzdot} |n|^{2s} \int_{\rr} |
		\wh{\psi}(\tau - n^m) |^{2} \left( \int_{\rr}\frac{1 - \psi(\lambda -
		n^m)}{\lambda - n^m} \wh{w}(n, \lambda) d \lambda \right)^{2} d \tau
		\right]^{1/2}
		\\
		& \le c_{\psi} \left[ \sum_{n \in \zzdot} |n|^{2s} \left(
		\int_{\rr} \frac{1 - \psi(\lambda - n^m)}{\lambda - n^m}
		\wh{w}(n, \lambda) d \lambda
		\right)^{2}\right]^{1/2}
		\\
		& \le 2 c_{\psi} \left[ \sum_{n \in \zzdot} |n|^{2s} \left(
		\int_{\rr} \frac{\wh{w}(n, \lambda) }{1 + |\lambda - n^m|}
		d \lambda
		\right)^{2}\right]^{1/2}
		\\
		& \lesssim \|f\|_{\dot{X}^s} \|g\|_{\dot{X}^s} 
	\end{split}
\end{equation}
%
%
where the last two steps follow from \eqref{mone-plus-ineq} and
\cref{mprop:bilinear-estimate2}, respectively. Combining
\eqref{mmain-int3-est-X-s-part} and \eqref{mmain-int-estimate-3-Y-s-part}, we
conclude that
%
%
\begin{equation}
	\label{mmain-int3-est}
	\begin{split}
		\|\eqref{mmain-int-expression-3}\|_{\dot{Y}^s} 
		\lesssim \|f\|_{\dot{X}^s} \|g\|_{\dot{X}^s}.
	\end{split}
\end{equation}
%
%
%
\subsection{Estimate for
\ref{mmain-int-expression-4}.}
Note that
%
%
\begin{equation}
	\label{m1n}
	\begin{split}
		\eqref{mmain-int-expression-4} \simeq \sum_{k \ge 1}
		\frac{i^k}{k!}g_k(x,t)
	\end{split}
\end{equation}
%
%
where 
%
%
\begin{equation*}
	\begin{split}
		& g_k(x,t) = t^k \psi(t) \sum_{n \in \zzdot} e^{i\left( xn + tn^m
		\right)} h_k(n),
		\\
		& h_k(n) = \int_\rr \psi \left( \tau - n^m \right) \cdot \left(
		\tau - n^m \right)^{k -1} \wh{w}(n, \tau) \ d \tau.
	\end{split}
\end{equation*}
%
%
Hence
%
%
\begin{equation*}
	\begin{split}
		\wh{g_k^x}(n, t) = t^{k} \psi(t) e^{i t n^m} h_k(n)
	\end{split}
\end{equation*}
%
%
which gives
%
%
\begin{equation*}
	\begin{split}
		\wh{g_k}(n, \tau)
		& = h_k(n) \int_\rr e^{-it\left( \tau - n^m \right)}
		t^{k}\psi(t) \ dt
		\\
		& = h_k(n) \wh{t^{k}\psi(t)} \left( \tau - n^m \right).
	\end{split}
\end{equation*}
%
%
Applying this to \eqref{m1n}, we obtain
%
%
\begin{equation}
	\label{m2n}
	\begin{split}
		\|\eqref{mmain-int-expression-4}\|_{\dot{X}^s} 
		& \simeq \left( \sum_{n \in \zzdot} |n|^{2s} \int_\rr \left( 1 + | \tau -
		n^m
		|
		\right) | \wh{\sum_{k \ge 1} \frac{i^k}{k!}g_k(x,t)} |^2 \ d \tau
		\right)^{1/2}
		\\
		& \le \sum_{k \ge 1} \frac{1}{k!}\left( \sum_{n \in \zzdot} |n|^{2s}
		\int_\rr \left( 1 + | \tau - n^m | \right) | \wh{g_k}(n, \tau) |^2 \
		d \tau \right)^{1/2}
		\\
		& = \sum_{k \ge 1} \frac{1}{k!} \left( \sum_{n \in \zzdot} |n|^{2s}
		\int_\rr \left( 1 + | \tau - n^m | \right) | h_k(n) \wh{t^k
		\psi(t)} \left( \tau - n^m \right) |^2 \ d \tau \right)^{1/2}
		\\
		& = \sum_{k \ge 1} \frac{1}{k!} \left( \sum_{n \in \zzdot} |n|^{2s} |
		h_k(n) |^2 \int_\rr \left( 1 + | \tau - n^m | \right) | \wh{t^k
		\psi(t)} \left( \tau - n^m \right) |^2 \ d \tau \right)^{1/2}.
	\end{split}
\end{equation}
%
%
Notice that for fixed $n$, the change of variable $\tau - n^m \to \tau'$
gives
%
%
\begin{equation}
	\label{m3n}
	\begin{split}
		\int_\rr \left( 1 + | \tau - n^m | \right) | \wh{t^{k}
		\psi(t)}\left( \tau - n^m \right) |^2 \ d \tau
		& = \int_\rr \left( 1 + |\tau'| \right) | \wh{t^k \psi(t)}(\tau') |^2 \
		d \tau'
		\\
		& \le \int_\rr \left( 1 + |\tau'| \right)^2 | \wh{t^k \psi(t)}(\tau')
		|^2 \ d \tau'
		\\
		& \lesssim \int_\rr \left( 1 + | \tau' |^2 \right) | \wh{t^{k}
		\psi(t)}(\tau') |^2 \ d \tau'
		\\
		& = \|t^k \psi(t) \|_{H^1(\rr)}^2.
	\end{split}
\end{equation}
%
%
But
%
%
\begin{equation}
	\label{m4n}
	\begin{split}
		\|t^k \psi(t) \|_{H^1(\rr)}^2
		& = \left( \|t^k \psi(t)\|_{L^2(\rr)} + \|\p_t \left( t^k \psi(t)
		\right)\|_{L^2(\rr)} \right)^2
		\\
		& \lesssim \|t^{k}\psi(t) \|_{L^2(\rr)}^2 + \|\p_t \left (t^{k}
		\psi(t) \right )\|_{L^2(\rr)}^2
		\\
		& \le \|t^k \psi(t) \|_{L^2(\rr)}^2 + \|t^k \p_t \psi(t)
		\|_{L^2(\rr)}^2 + \|k t^{k -1} \psi(t) \|_{L^2(\rr)}^2
		\\
		& = c_{\psi} + c_{\psi}' + k^2 c_{\psi}''
		\\
		& \lesssim k^2.
	\end{split}
\end{equation}
%
%
Hence, applying \eqref{m3n} and \eqref{m4n} to \eqref{m2n}, we obtain
%
%%
\begin{equation}
	\label{m5n}
	\begin{split}
		\|\eqref{mmain-int-expression-4} \|_{\dot{X}^s}
		& \lesssim
		\sum_{k \ge 1} \frac{k}{k!} \left( \sum_{n \in \zzdot} |n|^{2s} | h_k(n) |^2 
		\right)^{1/2}
		\\
		& \le \sum_{k \ge 1} \frac{k}{k!}
		 \sup_{k \ge 1} \left( \sum_{n \in \zzdot} |n|^{2s} | 
		h_k(n) |^2 \right)^{1/2}
		\\
		& = \sum_{k \ge 1} \frac{k}{k!}  \sup_{k \ge 1} 
		\left( \sum_{n \in \zzdot} |n|^{2s} \int_\rr 
		\psi\left( \tau - n^m \right) \cdot \left( \tau - n^m 
		\right)^{k -1} \wh{w}(n, \tau) \ d \tau \right)^{1/2}.
	\end{split}
\end{equation}
%
%%
Recall that $\text{supp} \, |\psi| \subset [0, T ]$. Pick $T \le 1$. 
Then $| \psi\left( \tau - n^m \right) \cdot \left( \tau - n^m \right)^{k 
-1} | \le \chi_{| \tau - n^m | \le 1}$ for all $k \ge 1$. Hence, \eqref{m5n} gives
%
%%
\begin{equation*}
	\begin{split}
		\|\eqref{mmain-int-expression-4} \|_{\dot{X}^s} 
		& \lesssim \sum_{k \ge 1} \frac{k}{k!}  \left( \sum_{n \in \zzdot} | 
		\int_{| \tau - n^m  |\le 1} | \wh{w}(n, \tau) \ d \tau |^2 
		\right)^{1/2}
	\end{split}
\end{equation*}
%
%%
which by the inequality
%
%%
\begin{equation*}
	\begin{split}
		\frac{1 + | \tau - n^m |}{1 + | \tau  - n^m |} \le 
		\frac{2}{1 + | \tau - n^m |}, \qquad | \tau - n^m  | \le 1
	\end{split}
\end{equation*}
%
%%
implies
%
%%
\begin{equation}
\label{mmain-int4-est-X-s-part}
	\begin{split}
		\|\eqref{mmain-int-expression-4}\|_{\dot{X}^s}
		& \lesssim \left( \sum_{n \in \zzdot} | \int_{| \tau - n^m| \le 1 }
		\frac{\wh{w}(n, \tau)}{1 + | \tau - n^m |} \ d \tau |^2 
		\right)^{1/2}
		\\
		& \le \left( \sum_{n \in \zzdot} | \int_\rr
		\frac{\wh{w}(n, \tau)}{1 + | \tau - n^m |} \ d \tau |^2 
		\right)^{1/2} \\
		& \le \left( \sum_{n \in \zzdot} \left( \int_\rr 
		\frac{|\wh{w}(n, \tau)|}{1 + | \tau - n^m |}  \ d \tau  \right)^2
		\right)^{1/2} \\
		& \lesssim \|u\|_{\dot{X}^s}^2
	\end{split}
\end{equation}
%
%%
where the last step follows from \cref{mprop:bilinear-estimate2}. Similarly,
we have
%
%
\begin{equation}
\label{mmain-int4-est-Y-s-part}
	\begin{split}
		\| n| |^s \wh{\eqref{mmain-int-expression-4}}\|_{\dot{\ell}^2_n L^1_\tau}
		& \simeq \left[ \sum_{n \in
		\zzdot}|n|^{2s} \left( \int_{\rr} | \sum_{k \ge 1}
		\wh{\frac{i^{k}}{k!}g_{k}(x,t)(n, \tau)} |d \tau \right)^{2} \right]^{1/2}
		\\
		& \le \sum_{k \ge 1} \frac{1}{k!} \left[ \sum_{n \in \zzdot} (1 + | n
		|)^{2s} \left( \int_{\rr} | \wh{g}(n, \tau) | d \tau \right)^{2}
		\right]^{1/2}
		\\
		& = \sum_{k \ge 1} \frac{1}{k!} \left[ \sum_{n \in \zzdot} (1 + | n
		|)^{2s} | h_{k}(n) |^2 \left( \int_{\rr} | \wh{t^{k} \psi(t)}(\tau -
		n^m) |d \tau \right)^{2} \right]^{1/2}
		\\
		& = c_{\psi} \sum_{k \ge 1} \frac{1}{k!} \left[ \sum_{n \in \zzdot} (1 + | n
		|)^{2s} | h_{k}(n) |^2 \right]^{1/2}
		\\
		& \lesssim \|u\|_{\dot{X}^s}^2
	\end{split}
\end{equation}
%
%
where the last step follows from the computations starting from \eqref{m5n}
through \eqref{mmain-int4-est-X-s-part}.
Combining \eqref{mmain-int4-est-X-s-part} and \eqref{mmain-int4-est-Y-s-part}, we
have
%
%
\begin{equation}
\label{mmain-int4-est}
	\begin{split}
		\|\eqref{mmain-int-expression-4}\|_{\dot{Y}^s} \lesssim \|u\|_{\dot{X}^s}^2.
	\end{split}
\end{equation}
%
%
Collecting estimates \eqref{mmain-int1-est}, \eqref{mmain-int2-est}, 
\eqref{mmain-int3-est}, and \eqref{mmain-int4-est}, and recalling 
\eqref{mmain-int-expression-1}-\eqref{mmain-int-expression-4}, we see that
$$\|Tu\|_{\dot{Y}^s} \le c_\psi \left( \|\vp \|_{\dot{H}^s(\ci)} + \|u\|_{\dot{X}^s}^2 \right )$$ 
which by the inequality $\|u\|_{\dot{X}^s} \le \|u\|_{\dot{Y}^s}$ yields the following.
%%
%%%%%%%%%%%%%%%%%%%%%%%%%%%%%%%%%%%%%%%%%%%%%%%%%%%%%
%
%% Contraction Proposition
%				 
%%%%%%%%%%%%%%%%%%%%%%%%%%%%%%%%%%%%%%%%%%%%%%%%%%%%%%
%%
%%
%
\begin{proposition}
\label{mprop:contraction}
Let $s \ge \frac{1-m}{4}$. Then
%
%%
\begin{equation*}
	\begin{split}
		\|Tu\|_{\dot{Y}^s} \le c_\psi \left( \|\vp \|_{\dot{H}^s(\ci)} + \|u\|_{\dot{Y}^s}^2 
		\right).
	\end{split}
\end{equation*}
%
%%
\end{proposition}
We will now use \cref{mprop:contraction} to prove local well-posedness for the 
mKDV ivp. Let $c = c_{\psi}$. For given $\vp$, we may choose $\psi$ such
that 
%
%%
\begin{equation*}
	\begin{split}
		\|\vp\|_{\dot{H}^s(\ci)} \le \frac{3}{16c^2}.
	\end{split}
\end{equation*}
%
%%
Then if $\|u\|_{\dot{Y}^s} \le \frac{1}{4c}$, we have
%
%%
\begin{equation*}
	\begin{split}
		\|T u \|_{\dot{Y}^s} 
		& \le c \left[ \frac{3}{16c^2} + \left( 
		\frac{1}{4c} \right)^2 \right]
		=  \frac{1}{4c}.
	\end{split}
\end{equation*}
%
%%
Hence, $T=T_{\vp}$ maps the ball $B\left( 0, \frac{1}{4c} \right) \subset \dot{Y}^s$ into 
itself. Next, note that
%
%%
\begin{equation*}
	\begin{split}
		Tu - Tv = \eqref{mmain-int-expression-2} + \eqref{mmain-int-expression-3} 
		+ \eqref{mmain-int-expression-4}
	\end{split}
\end{equation*}
%
%%
where now $w = \frac{1}{2} \p_x (u^2 - v^2)$. Rewriting
%
%%
\begin{equation*}
	\begin{split}
	\p_x(u^2 - v^2)	
		& = \p_x[(u-v)(u+v)]
		\end{split}
\end{equation*}
%
%%
and repeating the arguments used to estimate 
\eqref{mmain-int-expression-2}-\eqref{mmain-int-expression-4} (in the
bilinear estimates, we now set $f=u-v$ and $g = u+v$), we obtain
%
%%
%%
\begin{equation}
	\label{m20a}
	\begin{split}
		\|Tu - Tv \|_{\dot{Y}^s}  
		& \le c_\psi \|u -v\|_{\dot{Y}^s} \|u + v \|_{\dot{Y}^s}
		\\
		& \le c_\psi \|u -v\|_{\dot{Y}^s} (\|u\|_{\dot{Y}^s}+ \|v \|_{\dot{Y}^s}).
	\end{split}
\end{equation}
%
%%
If $u, v \in B(0, \frac{1}{4c}) \subset \dot{Y}^s$, it follows that
%
%%
\begin{equation}
	\label{m21a}
	\begin{split}
		\|Tu - Tv \|_{\dot{Y}^s}
		& \le c \|u -v \|_{\dot{Y}^s} \left( \frac{1}{4c} + 
		\frac{1}{4c} \right)
		\\
		& = \frac{1}{2} \|u -v \|_{\dot{Y}^s}. 
	\end{split}
\end{equation}
%
%%
We conclude that $T = T_{\vp}$ is a contraction on the ball $B(0, 
\frac{1}{4c}) \subset \dot{Y}^s$. A Picard iteration and application of 
\cref{mlem:cutoff-loc-soln} then yield a unique, local
solution to the mKDV ivp \eqref{mmKDV-eq}-\eqref{mmKDV-init-data}.
\begin{definition}
	We say that the flow map $u_0 \mapsto u(t)$ is \emph{locally Lipschitz} in a Banach
	space $X$ if for
	$$u_0, v_0 \in B_R \doteq \{f: \|f\|_X < R\},$$ there exist $C, T>0$
	depending on $R$ such that $\|u(\cdot, t) - v(\cdot, t)
	\|_X \le C \|u_{0} - v_0 \|_{X}$ for $t \in [-T, T]$. We
	say the flow map is \emph{locally uniformly
	continuous} in $X$ if for
	$u_0, v_0 \in B_R$ there exists $T >0$ depending on $R$ such that for
	$t \in [-T, T]$, $\|u(\cdot, t) - v(\cdot, t) \|_{X} \to
	0$ if $\|u_0 - v_0 \|_{H^{s}(\ci)} \to 0$. 
\end{definition}
%
%
Clearly any locally Lipschitz flow map is locally uniformly continuous. 
Next, we shall establish local Lipschitz continuity in $\dot{Y}^s$ of the flow
map. Let $\vp_1, \vp_2 \subset \dot{H}^s(\ci)$ be given. Choose $\psi$ such that
$\vp_1, \vp_2 \subset B(0, \frac{15}{64c^{3}})$.  Then there exist $u_1, u_2 \in
\dot{Y}^s$ such that $u_1 = T_{\vp_1}$, $u_2 = T_{\vp_2}$, and so
%
%
\begin{equation}
	\label{mgen-1a}
	\begin{split}
		T_{\vp_1}(u) - T_{\vp_2}(v)
		& = \frac{1}{2\pi} \psi(t) \sum_{n \in
		\zzdot}e^{i\left( xn + tn^m \right)} \wh{\vp_1 - \vp_2}(n)
		\\
		& + \eqref{mmain-int-expression-2} + \eqref{mmain-int-expression-3} +
		\eqref{mmain-int-expression-4} 
	\end{split}
\end{equation}
%
%
where $w = \frac{1}{2} \p_x (u^2 - v^2)$. Using an argument similar to \eqref{mfourier-trans-calc}-\eqref{mmain-int1-est},
we obtain
%
%
\begin{equation}
	\label{mgen-2a}
	\begin{split}
		\| \frac{1}{2\pi} \psi(t) \sum_{n \in
		\zzdot}e^{i\left( xn + tn^m \right)} \wh{\vp_1 - \vp_2}(n)\|_{\dot{Y}^s}
		\le c_\psi \|\vp_{1} - \vp_{2}\|_{\dot{Y}^s}.
	\end{split}
\end{equation}
%
%
Therefore, from \eqref{m21a}-\eqref{mgen-2a}, we obtain
%
%
\begin{equation*}
	\begin{split}
		\|u -v \|_{\dot{Y}^s} = \|T_{\vp_1}(u) - T_{\vp_2}(v) \|_{\dot{Y}^s} \le c_\psi
		\|\vp_{1} - \vp_{2} \|_{\dot{H}^s\left( \ci \right)} +
		\frac{1}{2} \|u -v \|_{\dot{Y}^s}
	\end{split}
\end{equation*}
%
%
which implies
%
%
\begin{equation*}
	\begin{split}
		\frac{1}{2} \|u-v\|_{\dot{Y}^s} \le c_\psi \|\vp_1 - \vp_2 \|_{\dot{H}^s(\ci)}
	\end{split}
\end{equation*}
%
%
or
%
%
\begin{equation*}
	\begin{split}
		\|u -v \|_{\dot{Y}^s} \le 2 c_\psi \|\vp_1 - \vp_2 \|_{\dot{H}^s(\ci)}.
	\end{split}
\end{equation*}
%
%
Applying \cref{mlem:cutoff-loc-soln}, we then obtain
%
%
	 %
	 %
	 \begin{equation*}
		 \begin{split}
			\|u(\cdot, t) -v(\cdot, t) \|_{\dot{H}^s(\ci)} \le 2 c_\psi \|\vp_1 -
			\vp_2 \|_{\dot{H}^s(\ci)}, \qquad t \in [-T, T].
		 \end{split}
	 \end{equation*}
	 %
	 %
Hence, the flow map of the mKDV ivp is locally Lipschitz continuous in
$\dot{H}^s(\ci)$. This
concludes the proof of \cref{mthm:main}. \qquad \qedsymbol
%
%
%
%
\section{Proof of First Bilinear Estimate}
Note first that $|\wh{w_{fg}}(n, \tau) |  = | n\wh{f} *  \wh{g} 
(n, \tau)|$. From this and the conservation of mass, it follows that
%
%
\begin{equation}
	\label{mnon-lin-rep}
	\begin{split}
		| \wh{w_{fg}}(n, \tau)|
		& = | \sum_{\substack{n_1 \neq 0, n_2 \neq 0 \\n_1 +n_2 =n}}  \int_{\tau_1 + \tau_2 = \tau}n\wh{f}\left( n_1,  \tau_1 
\right) \wh{g}\left( n_2, \tau_2  
\right) d \tau_1 d \tau_2 |
\\
& = | \sum_{\substack{n_1 \neq0, n_2 \neq 0 \\n_1 + n_2 =n}}  \int_{\tau_1 + \tau_2 = \tau}n\wh{f}\left( n_1,  \tau_1 
\right) \wh{g}\left( n_2, \tau_2  
\right) d \tau_1 d \tau_2 | 
\\
& \le \sum_{\substack{n_1 \neq0, n_2 \neq 0 \\n_1 + n_2 =n}}   \int_{\tau_1 + \tau_2 = \tau}| n | \times | \wh{f}\left( n_1, \tau_1 
\right) | \times  | \wh{g}\left( n_2, \tau_2 
\right) |   d \tau_1 d \tau_2  
\\
& = \sum_{\substack{n_1 \neq0, n_2 \neq 0 \\n_1 + n_2 =n}} \int_{\tau_1 + \tau_2 = \tau}| n | \times \frac{c_f\left( n_1, \tau_1 
\right)}{|n_1|^s \left( 1 + | \tau_1 - n_1^m | \right)^{1/2}}
\\
& \times \frac{c_{g}\left( n_2, \tau_2 \right)}{|n_2|^s\left( 1 + | \tau_2 -  n_2^m| 
\right)^{1/2}}
  \ d \tau_1 d \tau_2 
\end{split}
\end{equation}
%
%
where 
%
%
\begin{equation*}
	\begin{split}
		c_h(n, \tau) =
		\begin{cases}
			|n|^s \left( 1 + | \tau - n^m |  
			\right)^{1/2} | \wh{h}\left( n, \tau \right) |, \qquad & n \neq 0
		\\
		0, \qquad & n = 0.
	\end{cases}
	\end{split}
\end{equation*}
%
%
From our work above, it follows that 
%
%
\begin{equation}
	\label{mconvo-est-starting-pnt}
	\begin{split}
		 & |n|^s \left( 1 + | \tau - n^m | \right)^{-1/2} | \wh{w_{fg}}\left( 
		n, \tau \right) |
		\\
		& \le \left( 1 + | \tau - n^m | \right)^{-1/2}
		\sum_{\substack{n_1 \neq0, n_2 \neq 0 \\n_1 + n_2 =n}} \int_{\tau_1 + \tau_2 = \tau}\frac{|n|^{s+1}}{|n_1|^s | n_2|^s} 
		\times \frac{c_f(n_1, \tau_1)}{\left( 1 + | \tau_1 - n_1^m | 
		\right)^{1/2}}
		\\
		& \times
		\frac{c_g(n_2, \tau_2)}{\left( 1 + | \tau_2 - n_2^m | 
		\right)^{1/2}}\ d \tau_1 d \tau_2.
	\end{split}
\end{equation}
%
%
Unlike the NLS, we must use the smoothing properties of the
principal symbol $\tau - n^m$ regardless of the choice of $s$, since the quantity
%
%
\begin{equation}
	\label{mconvo-multiplier}
	\begin{split}
		\frac{|n|^{s+1}}{|n_1|^s |n_2|^s }
	\end{split}
\end{equation}
%
%
blows up in general, due to the presence of the extra power of $|n|$ coming from the derivative on
the nonlinearity. To utilize the smoothing effects of the principal symbol, we
first note that 
$$| \tau - n^m - \left( \tau_1 - n_1^m 
+ \tau_2 - n_2^m  \right ) | = | - n^m + n_1^m +
n_2^m| \doteq d_m(n_1, n_2).$$ We will need the following two lemmas, whose
proofs are provided in the appendix.
%
%
%
\begin{lemma}
	\label{mlem:number-theory1}
	Let $n=n_1 + n_2$ and suppose that $n, n_1, n_2\neq
	0$. Then for any integer $c \ge 0$
%
%
\begin{equation}
	\begin{split}
		\label{mnumber-theory1}
		d_3(n_1,n_2) \ge 2^{-c/2} | n |^{\frac{2+c}{2}} | n_{1}
		|^{\frac{2-c}{2}}| n_2 |^{\frac{2-c}{2}}.
	\end{split}
\end{equation}
%
%
\end{lemma}
%
%
%
%
%
%
\begin{lemma}
	\label{mlem:number-theory}
	Let $n=n_1 + n_2$ and suppose that $n, n_1, n_2\neq
	0$. Then for any integer  $m \ge 3$
%
%
\begin{equation}
	\begin{split}
		\label{mnumber-theory}
		d_m(n_1,n_2) \ge b_{m, c } 
		|n|^{c/2} |n_1|^{\frac{m-1-c}{2}} | n_2 |^{\frac{m-1-c}{2}}
		\end{split}
\end{equation}
%
%
where the constant $b_{m,c}$ depends only on $m$ and $c$. 
\end{lemma}
%
%
%
%\begin{remark}
%	The case $-1/2 \le s \le 0$ is delicate, and must be treated differently from
%	the case $s < -1/2$ in order to obtain the optimal well-posedness results.
%	This is the motivation for having two instead of one number theory lemma.
%\end{remark}
%%
%
Let us proceed with the case $m=3$ first; we will then generalize to arbitrary
odd $m \ge 3$. By the pigeonhole principle we must have one of the 
following.
%
%
\begin{align}
	\label{mpigeon-case-1}
	& |\tau - n^3| \ge \frac{d_m(n_1, n_2)}{3} 
		 \\
		\label{mpigeon-case-2}
		& | \tau_1 - n_1^3 | \ge \frac{d_m(n_1, n_2)}{3} 
		 \\
		\label{mpigeon-case-3}
		& | \tau_2 - n_2^3 | \ge \frac{d_m(n_1, n_2)}{3}.
		\end{align}
%
%
By the symmetry of the convolution, it will be enough to consider only
\eqref{mpigeon-case-1} and \eqref{mpigeon-case-2}.
%
%
%
\subsection{Case \ref{mpigeon-case-1}.} 
Applying \cref{mlem:number-theory1}, we have, for nonzero $ n, n_1, n_2 $
%
%%
\begin{equation}
	\label{mconvo-deriv-bound}
	\begin{split}
		& \frac{|n|^{s+1}}{|n_1|^s 
		| n_2|^s}
		\times
		\frac{1}{(1 + | \tau -n^3 |)^{1/2}}
		\\
		& \lesssim | n |^{s+1}| n_1 |^{-s}| n_2 |^{-s} \times | n
		|^{-\frac{2+c}{4}}| n_1 |^{-\frac{2-c}{4}}| n_2 |^{-\frac{2-c}{4}} 
		\\
		& = | n |^{\frac{4s +2 -c}{4}} | n_1 |^{\frac{-4s -2 +c}{4}} | n_2
		|^{\frac{-4s -2 +c}{4}}
		\\
		& \le 1, \qquad s \ge -1/2.
	\end{split}  
\end{equation}
%
%
\begin{framed}
\begin{remark}
	\label{mrem:s-val}
	The last line follows from the following reasoning: Set $(4s + 2 -c) = 0$
or, equivalently, $-4s -2 +c = 0$. Then for any $c \ge 0$ such that $c = 4s+2$
the left hand side of
\eqref{mconvo-deriv-bound} is bounded by $1$. Of course such a $c$ exists, as
long as $s \ge -1/2$.
\end{remark}
\end{framed}
%
%
%
Hence, recalling \eqref{mconvo-est-starting-pnt} and applying estimates 
\eqref{mpigeon-case-1} and \eqref{mconvo-deriv-bound}, we obtain
%
%
\begin{equation}
	\label{mnon-lin-rep-with-bound}
	\begin{split}
		& |n|^s \left( 1 + | \tau - n^3 | \right)^{-1/2} | 
		\wh{w_{fg}}(n, \tau) | 
		\\
		& \lesssim \sum_{\substack{n_1 \neq0, n_2 \neq 0 \\n_1 + n_2 =n}} \int_{\tau_1 + \tau_2 = \tau}\frac{c_f(n_1, \tau_1)}{\left( 1 + | 
		\tau_1 -  n_1^3| \right)^{1/2}}
		\times \frac{c_g\left( n_2, \tau_2\right)}{\left( 1 + | \tau_2 -n_2^3|
		\right)^{1/2}}
		\\
		& = \wh{C_f C_g}(n, \tau)
	\end{split}
\end{equation}
%
%
where
\begin{equation*}
	\begin{split}
		C_h(x,t) =
		\left[ \frac{c_h(n, \tau)}{\left( 1 + | \tau - n^3 | 
		\right)^{1/2}}\right]^\vee .	
	\end{split}
\end{equation*}
%
%
%
Therefore, from \eqref{mnon-lin-rep-with-bound}, Plancherel, and generalized 
H\"{o}lder, we obtain
%
%
\begin{equation}
	\label{mgen-holder-bound}
	\begin{split}
		& \| |n|^s \left( 1 + | \tau - n^3 | \right )^{-1/2}  \wh{w_{fg}}\left( 
		n, \tau \right) \|_{L^2(\ci \times \rr)}
		\\
		& \lesssim \|\wh{C_f C_g }\left( n, \tau \right) 
		\|_{L^2\left( \zzdot \times \rr \right)}
		\\
		& \simeq \|C_f C_g \|_{L^2\left( \ci \times \rr \right)}
		\\
		& \le \|C_f \|_{L^4(\ci \times \rr)} \|C_g \|_{L^4(\ci \times \rr)}.
	\end{split}
\end{equation}
%
We now need the following Fourier multiplier estimate, whose proof can be found
in~\cite{Himonas:2001db}.
%
\begin{lemma}
	\label{mlem:four-mult-est-L4}
	Let $(x, t) \in \ci \times \rr $ and $(n, \tau) \in \zz \times \rr$ be 
	the dual variables. Let $v$ be a positive even integer. Then there is a 
	constant $c_v > 0$ such that
%
%
\begin{equation}
	\label{mfour-mult-est-L4}
	\begin{split}
		\|f\|_{L^4(\ci \times \rr)} \le c_v \|\left( 1 + | \tau - n^v | 
		\right)^\frac{v+1}{4v} \wh{f} \|_{L^2( \zz \times \rr)}
	\end{split}
\end{equation}
for every test function $f(x, t)$. 
%
%
%
%
\end{lemma}
From the lemma, we see that
%
%
\begin{equation}
	\label{mfour-mult-conseq}
	\begin{split}
		\|C_h\|_{L^4(\ci \times \rr)} 
		& \lesssim \|(1 + | \tau - n^3 |)^{1/2} \wh{C_h}
		\|_{L^2(\zz \times \rr)}
		\\
		& = \|c_{h} \|_{L^2(\zz \times \rr)} 
		\\
		& = \|h \|_{\dot{X}^s}. 
	\end{split}
\end{equation}
%
%
Applying this to \eqref{mgen-holder-bound} we
conclude that
\begin{equation*}
	\begin{split}
		\| |n|^s \left( 1 + | \tau - n^3 | \right ) ^{-1/2} \wh{w_{fg}}\left( 
		n, \tau \right) \|_{L^2(\zzdot \times \rr)}
		& \lesssim \|f\|_{\dot{X}^s} \|g\|_{\dot{X}^s}.
	\end{split}
\end{equation*}
%
%
%
\subsection{Case \ref{mpigeon-case-2}.}
Applying \cref{mlem:number-theory1}, we have, for nonzero $ n, n_1, n_2 $
\begin{equation}
	\label{mconvo-deriv-bound-2}
	\begin{split}
		& \frac{|n|^{s+1}}{|n_1|^s 
		| n_2|^s}
		\times
		\frac{1}{(1 + | \tau -n^3 |)^{1/2}}
		\\
		& \lesssim | n |^{s+1}| n_1 |^{-s}| n_2 |^{-s} \times | n
		|^{-\frac{2+c}{4}}| n_1 |^{-\frac{2-c}{4}}| n_2 |^{-\frac{2-c}{4}} 
		\\
		& = | n |^{\frac{4s +2 -c}{4}} | n_1 |^{\frac{-4s -2 +c}{4}} | n_2
		|^{\frac{-4s -2 +c}{4}}
		\\
		& \le 1, \qquad s \ge -1/2
	\end{split}  
\end{equation}
%
%
where the last line follows from \cref{mrem:s-val}.
%
%
Hence, recalling \eqref{mconvo-est-starting-pnt} and applying estimate 
\eqref{mconvo-deriv-bound-2}, we obtain
%
%
\begin{equation}
	\label{m1f}
	\begin{split}
		& |n|^s  \left( 1 + | \tau - n^3 | \right)^{-1/2}| \wh{w_{fg}}\left( 
		n, \tau \right) |
		\\
		& \lesssim 
		\left( 1 + | \tau - n^3 | \right)^{-1/2}\sum_{\substack{n_1 \neq0, n_2 \neq 0 \\n_1 + n_2 =n}} \int_{\tau_1 + \tau_2 = \tau}		c_f(n_1, \tau_1)
		\times
		\frac{c_g(n_2, \tau_2)}{\left( 1 + | \tau_2 - n_2^3 | 
		\right)^{1/2}} 
		\\
		& = \left( 1 + | \tau - n^3 | \right)^{-1/2} \wh{\overset{\sim}{C_f} C_g}
	\end{split}
\end{equation}
%
%%
where
%
%
\begin{equation*}
	\begin{split}
		\overset{\sim}{C_h}(x,t) = \left[ c_h(n, \tau) \right]^\vee.
	\end{split}
\end{equation*}
%
%
Hence
%
%%
\begin{equation}
	\label{m3f}
	\begin{split}
		& \| |n|^s \left( 1 + | \tau - n^3 | \right)^{-1/2} \wh{w_{fg}}(n, \tau) 
		\|_{L^2(\zzdot \times \rr)}
		\\
		& \lesssim \|\left( 1 + | \tau - n^3 | \right)^{-1/2} 
		\wh{\overset{\sim}{C_f} C_g } \|_{L^2(\zzdot \times \rr)}
		\\
		& =  \|\left( 1 + | \tau - n^3 | \right)^{-1/2} 
		\wh{\overset{\sim}{C_f} C_g } \|_{L^2(\zz \times \rr)}
		\\
		& \lesssim  \|\overset{\sim}{C_f} C_g  \|_{L^{4/3}(\ci \times \rr)}
	\end{split}
\end{equation}
%
%%
where the last step follows by dualizing \cref{mlem:four-mult-est-L4}. More
precisely, we have the following.
\begin{corollary}
	\label{mcor:four-mult-est-L4}
	Let $(x, t) \in \ci \times \rr $ and $(n, \tau) \in \zz \times \rr$ be 
	the dual variables. Let $v$ be a positive even integer. Then there is a 
	constant $c_v > 0$ such that
%
%
\begin{equation}
	\label{mfour-mult-est-L4*}
	\begin{split}
		\| \left( 1 + | \tau - n^v | 
		\right)^{-\frac{v+1}{4v}}
		\wh{f}\|_{L^2(\zz \times \rr)} \le c_v \|f \|_{L^{4/3}( \ci \times \rr)}.
	\end{split}
\end{equation}
%
%
\end{corollary}
%
Applying H\"{o}lder's inequality to the right hand side of
\eqref{m3f}, we obtain the bound
%
%%
\begin{equation}
	\label{m4f}
	\begin{split}
		\|\overset{\sim}{C_f} \|_{L^2(\ci \times \rr)} \|C_g \|_{L^4\left( \ci 
		\times \rr 
		\right)}. 
	\end{split}
\end{equation}
%
%%
By Plancherel we have
%
%%
%
%%
\begin{equation}
	\label{m5f}
	\begin{split}
		\|\overset{\sim}{C_f} \|_{L^2(\ci \times \rr)}
		& \simeq \|c_f\|_{L^2(\zz \times \rr)}
		\\
		& = \|f \|_{\dot{X}^s}
	\end{split}
\end{equation}
%
%%
while \eqref{mfour-mult-conseq} gives
%
%
\begin{equation}
	\label{m6f}
	\begin{split}
		\|C_g \|_{L^4(\ci \times \rr)} \lesssim \|g\|_{\dot{X}^s}.
	\end{split}
\end{equation}
%
%
We conclude from \eqref{m3f}-\eqref{m6f} that
%
%
\begin{equation*}
	\begin{split}
		\| |n|^s \left( 1 + | \tau - n^3 | \right)^{-1/2} \wh{w_{fg}}(n, \tau) 
		 \|_{L^2(\zzdot \times \rr)}
		 \lesssim \|f\|_{\dot{X}^s} \|g\|_{\dot{X}^s}
	\end{split}
\end{equation*}
%
%
\subsection{Generalizing to arbitrary odd \texorpdfstring{$m >3$}{m > 3}.}
%
%
Since $$| \tau - n^m - \left( \tau_1 - n_1^m 
+ \tau_2 - n_2^m  \right ) | = | - n^m + n_1^m +
n_2^m|,$$ by and
the pigeonhole principle we must have one of the 
following.
%
%
\begin{align}
	\label{mpigeon-case-1-gen}
	& |\tau - n^m| \ge \frac{d(n_1, n_2)}{3} 	\\
		\label{mpigeon-case-2-gen}
		& | \tau_1 - n_1^m | \ge \frac{d(n_1, n_2)}{3},		\\
		\label{mpigeon-case-3-gen}
		& | \tau_2 - n_2^m | \ge \frac{d(n_1, n_2)}{3}.
	\end{align}
%
%
By the symmetry of the convolution, it will be enough to consider only
\eqref{mpigeon-case-1-gen} and \eqref{mpigeon-case-2-gen}.
%
%
%
\subsection{Case \ref{mpigeon-case-1-gen}.}
Applying \cref{mlem:number-theory}, we have, for nonzero $ n, n_1, n_2 $
%
%%
\begin{equation}
	\label{mconvo-deriv-bound-gen-case2}
	\begin{split}
		& \frac{|n|^{s+1}}{|n_1|^s 
		| n_2|^s}
		\times
		\frac{1}{(1 + | \tau -n^m |)^{1/2}}
		\\
		& \lesssim | n |^{s+1}| n_1 |^{-s}| n_2 |^{-s} \times | n
		|^{-\frac{c}{2}}| n_1 |^{-\frac{m-1-c}{4}}| n_2 |^{-\frac{m-1-c}{4}} 
		\\
		& = | n |^{\frac{2s+2 -c}{2}} | n_1 |^{\frac{-4s -m + 1+ c}{4}} | n_2
		|^\frac{-4s -m + 1+ c}{4}
		\\
		& \le 1, \qquad s \ge \frac{1-m}{4}.
	\end{split}  
\end{equation}
%
%
\begin{framed}
\begin{remark}
	\label{mrem:gen-s-val}
	 The last line follows from the following reasoning: Set $(2s + 2 -c) \le
0$ and $-4s -m +1 +c \le 0$. Then we want to find $c \ge 0$ such that $2s +2 \le c \le
4s + m-1$ or 
%
%
\begin{equation}
	\label{malgebra-ineq}
	\begin{split}
		2 \le c - 2s \le 2s + m-1.
	\end{split}
\end{equation}
%
%
Note that $c=0$ satisfies \eqref{malgebra-ineq} for $\frac{1-m}{4} \le s \le
-1$. Furthermore, $c = 4 + 4s$ satisfies \eqref{malgebra-ineq} for $s \ge -1$ ($c$ must be non-negative) and $m \ge 5$. 
\end{remark}
\end{framed}
%
Hence, from \eqref{mconvo-est-starting-pnt} and
\eqref{mconvo-deriv-bound-gen-case2},
we obtain 
\begin{equation}
	\label{mconvo-est-starting-pnt-gen-case2}
	\begin{split}
		 & |n|^s \left( 1 + | \tau - n^m | \right)^{-1/2} | \wh{w_{fg}}\left( 
		n, \tau \right) |
		\\
		& \le \left( 1 + | \tau - n^m | \right)^{-1/2}
		\sum_{\substack{n_1 \neq0, n_2 \neq 0 \\n_1 + n_2 =n}} \int_{\tau_1 + \tau_2 = \tau}\frac{|n|^{s+1}}{|n_1|^s | n_2|^s} 
		\times \frac{c_f(n_1, \tau_1)}{\left( 1 + | \tau_1 - n_1^m | 
		\right)^{1/2}}
		\\
		& \times
		\frac{c_g(n_2, \tau_2)}{\left( 1 + | \tau_2 - n_2^m | 
		\right)^{1/2}}\ d \tau_1 d \tau_2
		\\
		& \lesssim \sum_{\substack{n_1 \neq0, n_2 \neq 0 \\n_1 + n_2 =n}} \int_{\tau_1 + \tau_2 = \tau}\frac{c_f(n_1, \tau_1)}{\left( 1 + | \tau_1 - n_1^m | 
		\right)^{1/2}} \times
		\frac{c_g(n_2, \tau_2)}{\left( 1 + | \tau_2 - n_2^m | 
		\right)^{1/2}}\ d \tau_1 d \tau_2, \qquad s \ge \frac{1-m}{4}
		\\
		& = \wh{{C_f} C_g}.
	\end{split}
\end{equation}
Therefore, from \eqref{mconvo-est-starting-pnt-gen-case2}, Plancherel, and generalized 
H\"{o}lder, we obtain
%
%
\begin{equation}
	\label{mgen-holder-bound-case2}
	\begin{split}
		& \| |n|^s \left( 1 + | \tau - n^m | \right )^{-1/2}  \wh{w_{fg}}\left( 
		n, \tau \right) \|_{L^2(\ci \times \rr)}
		\\
		& \lesssim \|\wh{C_f C_g }\left( n, \tau \right) 
		\|_{L^2\left( \zzdot \times \rr \right)}
		\\
		& \simeq \|C_f C_g \|_{L^2\left( \ci \times \rr \right)}
		\\
		& \le \|C_f \|_{L^4(\ci \times \rr)} \|C_g \|_{L^4(\ci \times \rr)}.
	\end{split}
\end{equation}
%
From \cref{mlem:four-mult-est-L4}, we see that
%
%
\begin{equation}
	\label{mfour-mult-conseq-gen-case2}
	\begin{split}
		\|C_\sigma\|_{L^4(\ci \times \rr)} 
		& \lesssim \|(1 + | \tau - n^m |)^{1/2} \wh{C_\sigma}
		\|_{L^2(\zz \times \rr)}
		\\
		& = \|c_{\sigma} \|_{L^2(\zz \times \rr)} 
		\\
		& = \|\sigma \|_{\dot{X}^s}. 
	\end{split}
\end{equation}
%
%
Applying this to \eqref{mgen-holder-bound-case2} we
conclude that
\begin{equation*}
	\begin{split}
		\| |n|^s \left( 1 + | \tau - n^m | \right ) ^{-1/2} \wh{w_{fg}}\left( 
		n, \tau \right) \|_{L^2(\zzdot \times \rr)}
		& \lesssim \|f\|_{\dot{X}^s} \|g\|_{\dot{X}^s}.
	\end{split}
\end{equation*}
%
%
%
\subsection{Case \ref{mpigeon-case-2-gen}.}
We have for nonzero $ n, n_1, n_2 $
%
%%
\begin{equation}
	\label{mconvo-deriv-bound-gen}
	\begin{split}
		& \frac{|n|^{s+1}}{|n_1|^s 
		| n_2|^s}
		\times
		\frac{1}{(1 + | \tau_1 -n_1^m |)^{1/2}}
		\\
		& \lesssim | n |^{s+1}| n_1 |^{-s}| n_2 |^{-s} \times | n
		|^{-\frac{c}{2}}| n_1 |^{-\frac{m-1-c}{4}}| n_2 |^{-\frac{m-1-c}{4}} 
		\\
		& = | n |^{\frac{2s+2 -c}{2}} | n_1 |^{\frac{-4s -m + 1+ c}{4}} | n_2
		|^\frac{-4s -m + 1+ c}{4}
		\\
		& \le 1, \qquad s \ge \frac{1-m}{4}.
	\end{split}  
\end{equation}
%
%
where the last line follows from \cref{mrem:gen-s-val}.
Hence, from \eqref{mconvo-est-starting-pnt} and \eqref{mconvo-deriv-bound-gen},
we obtain 
\begin{equation}
	\label{mconvo-est-starting-pnt-gen}
	\begin{split}
		 & |n|^s \left( 1 + | \tau - n^m | \right)^{-1/2} | \wh{w_{fg}}\left( 
		n, \tau \right) |
		\\
		& \le \left( 1 + | \tau - n^m | \right)^{-1/2}
		\sum_{\substack{n_1 \neq0, n_2 \neq 0 \\n_1 + n_2 =n}} \int_{\tau_1 + \tau_2 = \tau}\frac{|n|^{s+1}}{|n_1|^s | n_2|^s} 
		\times \frac{c_f(n_1, \tau_1)}{\left( 1 + | \tau_1 - n_1^m | 
		\right)^{1/2}}
		\\
		& \times
		\frac{c_g(n_2, \tau_2)}{\left( 1 + | \tau_2 - n_2^m | 
		\right)^{1/2}}\ d \tau_1 d \tau_2
		\\
		& \lesssim \left( 1 + | \tau - n^m | \right)^{-1/2}
		\sum_{\substack{n_1 \neq0, n_2 \neq 0 \\n_1 + n_2 =n}} \int_{\tau_1 + \tau_2
		= \tau} c_f(n_1, \tau_1) \times
		\frac{c_g(n_2, \tau_2)}{\left( 1 + | \tau_2 - n_2^m | 
		\right)^{1/2}}\ d \tau_1 d \tau_2
		\\
		& = \left( 1 + | \tau - n^m | \right)^{-1/2}
\wh{\overset{\sim}{C_f} C_g}.
	\end{split}
\end{equation}
%
%%
%
%
Hence
%
%%
\begin{equation}
	\label{m3f-gen}
	\begin{split}
		& \| |n|^s \left( 1 + | \tau - n^m | \right)^{-1/2} \wh{w_{fg}}(n, \tau) 
		\|_{L^2(\zzdot \times \rr)}
		\\
		& \lesssim \|\left( 1 + | \tau - n^m | \right)^{-1/2} 
		\wh{\overset{\sim}{C_f} C_g } \|_{L^2(\zzdot \times \rr)}
		\\
		& =  \|\left( 1 + | \tau - n^m | \right)^{-1/2} 
		\wh{\overset{\sim}{C_f} C_g } \|_{L^2(\zz \times \rr)}
		\\
		& \lesssim  \|\overset{\sim}{C_f} C_g  \|_{L^{4/3}(\ci \times \rr)}
	\end{split}
\end{equation}
%
%%
where the last step follows from \cref{mcor:four-mult-est-L4}.
%
%
Applying H\"{o}lder's inequality to the right hand side of
\eqref{m3f-gen}, we obtain the bound
%
%%
\begin{equation}
	\label{m4f-gen}
	\begin{split}
		\|\overset{\sim}{C_f} \|_{L^2(\ci \times \rr)} \|C_g \|_{L^4\left( \ci 
		\times \rr 
		\right)}. 
	\end{split}
\end{equation}
%
%%
By Plancherel we have
%
%%
%
%%
\begin{equation}
	\label{m5f-gen}
	\begin{split}
		\|\overset{\sim}{C_f} \|_{L^2(\ci \times \rr)}
		& \simeq \|c_f\|_{L^2(\zz \times \rr)}
		\\
		& = \|f \|_{\dot{X}^s}
	\end{split}
\end{equation}
%
%%
while \cref{mlem:four-mult-est-L4} gives
%
%
\begin{equation}
	\label{mfour-mult-conseq-gen}
	\begin{split}
		\|C_h\|_{L^4(\ci \times \rr)} 
		& \lesssim \|(1 + | \tau - n^m |)^{1/2} \wh{C_h}
		\|_{L^2(\zz \times \rr)}
		\\
		& = \|c_{h} \|_{L^2(\zz \times \rr)} 
		\\
		& = \|h \|_{\dot{X}^s}. 
	\end{split}
\end{equation}
%
%
We conclude from \eqref{m3f-gen}-\eqref{mfour-mult-conseq-gen} that
%
%
\begin{equation*}
	\begin{split}
		\| |n|^s \left( 1 + | \tau - n^m | \right)^{-1/2} \wh{w_{fg}}(n, \tau) 
		 \|_{L^2(\zzdot \times \rr)}
		 \lesssim \|f\|_{\dot{X}^s} \|g\|_{\dot{X}^s}.
	\end{split}
\end{equation*}
%
%
%
%
\section{Proof of Second Bilinear Estimate}
Recall that for the NLS, one obtains one trilinear estimate as a corollary of
another. Using this as motivation, let us see if we can obtain
\cref{mprop:bilinear-estimate2} as a corollary of
\cref{mprop:prim-bilin-est}. By
duality, it suffices to show that
%
%%
\begin{equation}
	\label{mduality-est}
	\begin{split}
	|	\sum_{n \in \zzdot}  |n|^{s}
		a_n \int_{\rr} \frac{|\wh{w_{fg}}(n, \tau)|}{1 
		+ | \tau - n^m |} \ d \tau | \lesssim \|f\|_{\dot{X}^s} \|g\|_{\dot{X}^s}
		\|a_n \|_{\ell^2}, \qquad s \ge -1/2.
	\end{split}
\end{equation}
%
%%
By the triangle inequality 
and Cauchy-Schwartz,
%
%%
\begin{equation}
	\label{m1m}
	\begin{split}
		& | \sum_{n \in \zzdot} |n|^{s} a_n
		\int_{\rr}\frac{| \wh{w_{fg}}(n, \tau) |}{(1 + | \tau - n^m |)} \ d \tau |
		\\
		& \le \sum_{n \in \zzdot} \int_{\rr} \frac{| a_n |}{\left( 1 + 
		| \tau - n^m |
		\right)^{1/2 + \eta}} \times \frac{| n|^s  |
		\wh{w_{fg}}(n, \tau) |}{\left( 
		1 + | \tau - n^m | \right)^{1/2 - \eta}} \ d \tau
		\\
		& \le \left( \sum_{n \in \zzdot} | a_{n} |^2\int_{\rr} \frac{1}{\left( 1 + |
		\tau - n^m | \right)^{1 + 2 \eta}} \ d \tau  
		\right)^{1/2} 
		\left ( \sum_{n \in \zzdot}\int_{\rr} \frac{|n|^{2s} | \wh{w_{fg}}(n, \tau) 
		|^2}{\left( 1 + | \tau - n^m | \right)^{1 -2 \eta}}\ d \tau 
		\right)^{1/2}.
	\end{split}
\end{equation}
%
%%
Applying the change of variable $\tau - n^m
\mapsto \tau'$ we obtain  
%%
%
\begin{equation*}
	\begin{split}
		& \left( \sum_{n \in \zzdot} | a_{n} |^2\int_{\rr} \frac{1}{\left( 1 + | \tau -
		n^m | \right)^{1 + 2 \eta}} \ d \tau  
		\right)^{1/2} 
		\\
		& = \left ( \sum_{n \in \zzdot}
		| a_n |^2 
		\int_{\rr} \frac{1}{\left( 1 + | \tau' | \right)^{1 + 2 \eta}} \ d 
		\tau \right)^{1/2}
		\\
		& \simeq \|a_n\|_{\ell^2}, \qquad \eta >0.
		\end{split}
\end{equation*}
However, if we assume $\eta >0$, then
we cannot use \cref{mprop:prim-bilin-est} to bound
\begin{equation*}
	\begin{split}
		\left ( \sum_{n \in \zzdot}\int_{\rr} \frac{|n|^{2s} | \wh{w_{fg}}(n, \tau) 
		|^2}{\left( 1 + | \tau - n^m | \right)^{1 - 2\eta}}\ d \tau
		\right)^{1/2}. 
	\end{split}
\end{equation*}
%%
%%
\begin{framed}
\begin{remark}
Hence, unlike the NLS, we have not been able to obtain a second bilinear
estimate as a corollary from the first. Heuristically, this is due to the
derivative in nonlinearity, which is not present in the NLS nonlinearity.
However, one can obtain \eqref{mbilinear-estimate2} for $s>\frac{1-m}{4}$ as a
corollary of \cref{mprop:prim-bilin-est} by using the ideas
above and by modifying the proof of \cref{mprop:prim-bilin-est} slightly (i.e.,
showing that if $b = \frac{1}{2}^-$, then \eqref{mprim-bilin-est} holds for
$s\ge \frac{1-m}{4}^+$). To show that \eqref{mbilinear-estimate2} holds for the
case $s=1/2$, we will have to resort to Kenig-Ponce-Vega~\cite{Kenig:1996yn} techniques.
\end{remark}
\end{framed}
%
%
Proceeding, note that by duality, to prove \cref{mprop:bilinear-estimate2} it
suffices to show \eqref{mduality-est} for $s \ge \frac{1-m}{4}$. By the symmetry of the convolution, we
consider only cases \eqref{mpigeon-case-1} and \eqref{mpigeon-case-2}.
%
%
\subsection{Case \ref{mpigeon-case-1}.} Assume $s \ge \frac{1-m}{4}$. Then from 
\eqref{mconvo-est-starting-pnt-gen-case2} we have
%
%
\begin{equation}
	\label{mgen-smoothing-ineq}
	\begin{split}
		& |n|^s \left( 1 + | \tau - n^m | \right)^{-1/2} | 
		\wh{w_{fg}}(n, \tau) | 
		\\
		& \lesssim \sum_{\substack{n_1 \neq0, n_2 \neq 0 \\n_1 + n_2 =n}} \int_{\tau_1 + \tau_2 = \tau}\frac{c_f(n_1, \tau_1)}{\left( 1 + | 
		\tau_1 -  n_1^m| \right)^{1/2}}
		\times \frac{c_g\left( n_2, \tau_2\right)}{\left( 1 + | \tau_2 -n_2^m|
		\right)^{1/2}}.
	\end{split}
\end{equation}
%
%
From the triangle inequality and \eqref{mgen-smoothing-ineq}, we have
%
%
\begin{equation*}
	\begin{split}
	 |\eqref{mduality-est}|
	& \lesssim \sum_{n \in \zzdot} |a_{n}| \int_{\rr} \sum_{\substack{n_1 \neq 0, n_2 \neq 0
		\\ n_1 +n_2 =n}} \int_{\tau_1 + \tau_2 = \tau} c_f(n_1, \tau_1)
		c_g(n_2, \tau_2)
		\\
		& \times \frac{1}{(1 + | \tau - n^m |)^{1/2}(1 + |
		\tau_{1}-n_{1}^m |)^{1/2}(1 + | \tau-n_{2}^m |^{1/2})} d \tau_1 d \tau_2
		d \tau
	\end{split}
\end{equation*}
%
%
which by Cauchy-Schwartz is bounded by
%
%
\begin{equation}
	\label{m10g}
	\begin{split}
		& \sum_{n \in \zzdot} |a_n| \int_{\rr} \left(  \sum_{\substack{n_1 \neq 0, n_2
		\neq 0 \\n_1 +n_2 =n}} \int_{\tau_1 + \tau_2 = \tau} c_{f}^{2}(n_1, \tau_1)
		c_{g}^{2} (n_2, \tau_2) d \tau_1 d \tau_2 \right)^{1/2} 
		\\
		& \times \left( \sum_{\substack{n_1 \neq 0, n_2 \neq 0 \\n_1 +n_2 =n}}
		\int_{\tau_1 + \tau_2 = \tau} \frac{1}{(1 + | \tau - n^m |)(1 + | \tau_{1}-n_{1}^m |)(1 + |
		\tau_2 -n_{2}^m |)} d \tau_1 d \tau_2
		\right)^{1/2} d \tau.
	\end{split}
\end{equation}
%
%
Applying Cauchy-Schwartz again, \eqref{m10g} is bounded by
%
%
\begin{align}
	\notag
		& \|\left( \sum_{\substack{n_1 \neq 0, n_2 \neq 0 \\n_1 +n_2 =n}}\int_{\tau_1 + \tau_2 = \tau} c_{f}^{2}(n_1, \tau_1)
		c_{g}^{2} (n_2, \tau_2) d \tau_1 d \tau_2 \right)^{1/2} \|_{L^{2}(\zz \times
		\rr)}
		\\
		\notag
		& \times  \|a_{n}
		\left( \sum_{\substack{n_1 \neq 0, n_2 \neq 0 \\n_1 +n_2
		=n}}\int_{\tau_1 + \tau_2 = \tau} \frac{1}{ (1 + | \tau - n^m |)(1 + |
		\tau_{1}-n_{1}^m |)(1 + | \tau_2 -n_{2}^m |)} d \tau_1 d \tau_2
		\right)^{1/2} \|_{L^2(\zz \times \rr)}
		\\
		\notag
		& = \|f\|_{\dot{X}^s} \|g\|_{\dot{X}^s}
		\\
		\label{mholder-term}
		& \times 
		\|a_{n}
		\left( \sum_{\substack{n_1 \neq 0, n_2 \neq 0 \\n_1 +n_2
		=n}}\int_{\tau_1 + \tau_2 = \tau} \frac{1}{ (1 + | \tau - n^m |)(1 + |
		\tau_{1}-n_{1}^m |)(1 + | \tau_2 -n_{2}^m |)} d \tau_1 d \tau_2
		\right)^{1/2} \|_{L^2(\zz \times \rr)}.
\end{align}
%
Applying H{\"o}lder then gives
%
%
\begin{equation*}
	\begin{split}
		& \eqref{mholder-term}
		 \le \| a_{n} \|_{\ell^2}
		\\
		& \times \left( \sup_{n \neq 0} \int_{\rr}
		\sum_{\substack{n_1 \neq 0, n_2 \neq 0 \\n_1 +n_2 =n}} \int_{\tau_1 + \tau_2
		= \tau} \frac{1}{ (1 + | \tau - n^m |)(1 + |
		\tau_{1}-n_{1}^m |)(1 + | \tau_2 -n_{2}^m |)} d \tau_1 d \tau_2 d \tau
		\right)^{1/2}.
	\end{split}
\end{equation*}
%
%
Hence, to complete the proof for case \eqref{mpigeon-case-1}, it will be enough
to show that 
%
%
%
%
\begin{equation*}
	\begin{split}
		 \sup_{n \neq 0} \int_{\rr}
		\sum_{\substack{n_1 \neq 0, n_2 \neq 0 \\n_1 +n_2 =n}} \int_{\tau_1 + \tau_2
		= \tau} \frac{1}{ (1 + | \tau - n^m |)(1 + |
		\tau_{1}-n_{1}^m |)(1 + | \tau_2 -n_{2}^m |)} d \tau_1 d \tau_2 d \tau <\infty
	\end{split}
\end{equation*}
%
%
or, equivalently, that
%
%
\begin{equation}
	\label{m12g}
	\begin{split}
		\sup_{n \neq 0} \sum_{\substack{n_1 \neq 0, n_2 \neq 0 \\n_1 +n_2 =n}} \int_{\rr}
		\int_\rr  \frac{1}{(1 + | \tau - n^m |)(1 + | \tau_1 - n_{1}^m |)(1 + | \tau - \tau_1 -
		n_2^m |)} d \tau_1 d \tau < \infty.
	\end{split}
\end{equation}
%
%
Following Kenig~\cite{Kenig:1996yn}, we now need the following Calculus lemma.
%
%
%%%%%%%%%%%%%%%%%%%%%%%%%%%%%%%%%%%%%%%%%%%%%%%%%%%%%
%
%
%				 Calculus Lemma
%
%
%%%%%%%%%%%%%%%%%%%%%%%%%%%%%%%%%%%%%%%%%%%%%%%%%%%%%
%
%
\begin{lemma}
	\label{mlem:calc}
 %
 %
 \begin{equation}
	 \label{mcalc}
	 \begin{split}
		 \int_{\rr} \frac{1}{(1 + | \theta |)(1 + | a - \theta |)} d \theta \lesssim
		 \frac{\log(2 + | a |)}{1 + | a |}.
	 \end{split}
 \end{equation}
 %
 %
 \end{lemma}
%
%
Applying the lemma with $\theta = \tau_1 - n_1^m$ and $a = \tau - n_1^m -
n_2^m$, we see that
%
%
\begin{equation*}
	\begin{split}
	\int_{\rr}
		\int_\rr  \frac{1}{(1 + | \tau - n^m |)(1 + | \tau - \tau_1 -
		n_2^m |)} d \tau_1 d \tau \lesssim \frac{\log(2 + | \tau - n_{1}^m -
		n_{2}^m |)}{1 + | \tau - n_{1}^m - n_{2}^m |}.
	\end{split}
\end{equation*}
%
%
%
Hence, the left hand side of \eqref{m12g} is bounded by
%
\begin{equation*}
	\begin{split}
		\sup_{n \neq 0} \sum_{\substack{n_1 \neq 0, n_2 \neq 0 \\n_1 +n_2 =n}}
		\int_{\rr} \frac{\log(2 + | \tau - n_{1}^m -
		n_{2}^m |)}{(1 + | \tau - n_{1}^m - n_{2}^m |)(1 + | \tau - n^m |)}
		d \tau	
	\end{split}
\end{equation*}
%
%
or, equivalently, by
%
%
\begin{equation}
	\label{m13g}
	\begin{split}
		\sup_{n \neq 0} \sum_{n_1 \neq 0} \int_{\rr} \frac{\log(2 + | \tau -
		n_{1}^m - (n - n_1)^m |)}{(1 + | \tau - n_{1}^m - (n - n_{1})^m |)(1
		+ | \tau - n^m |)} d \tau.
	\end{split}
\end{equation}
%
%
%
Now, note that 
$$ |\tau - n^m| \ge \frac{d_m(n_1, n_2)}{3} \gtrsim
| n_1 n_2 |^{(m-1)/2},$$ where the right hand side follows from
\cref{mlem:number-theory} with $c=0$. Hence, \eqref{m13g} is bounded by a constant times
%
%
%
%
\begin{equation}
	\label{m14g}
	\begin{split}
		& \sup_{n \neq 0} \sum_{n_1 \neq 0}
		\frac{1}{| n_1 n_2 |^{(\frac{1}{2} + \eta)(m-1)/2}} \int_{\rr} \frac{\log(2 + | \tau - n_{1}^m -
		(n - n_1)^m |)}{(1 + | \tau - n_{1}^m - (n - n_{1})^m
		|)(1 + | \tau - n^m |)^{\frac{1}{2}-\eta}}
		d \tau
		\\
		& \le \sup_{n \neq 0} \sum_{n_1 \neq 0}
		\frac{1}{| n_1 n_2 |^{(\frac{1}{2} + \eta)(m-1)/2}} 	\\
		& \times \sup_{n \neq 0} \sum_{n_1 \neq 0}
		\int_{\rr} \frac{\log(2 + | \tau
		- n_{1}^m - (n - n_1)^m |)}{(1 + | \tau - n_{1}^m - (n - n_{1})^m
		|)(1 + | \tau - n^m |)^{\frac{1}{2}-\eta}}
		d \tau
	\end{split}
\end{equation}
%
for any $\eta >0$.
Observe that for the first sum, the supremum is attained at $n=1$.
%
%
\begin{framed}
\begin{remark}
To see this,
write $n_1 n_2 = n_1(n-n_1) \doteq f(n)$ and note that $|f(n)|$ has a global
minimum at $n=n_1$. Furthermore, $f(n)$ is strictly
increasing (if $n_1>0$) or strictly decreasing (if $n_1 <0$).
\end{remark}
\end{framed}
%
%
%
But then $n_2 = 1 - n_1$, and so $| n_1 n_2 | \gtrsim | n_1 |^2$. Furthermore, we know that 
for any $\ee > 0$, we have $\log (2 + | a |) \le c_{\ee}(1 + | a
|)^{\ee}$. Hence, we bound \eqref{m14g} by
%
%
%
%
\begin{equation*}
	\begin{split}
		c_{\ee}  \sum_{n_1 \neq 0} \frac{1}{|n_1|^{(\frac{1}{2} + \eta)(m-1)}}
		\sup_{n \neq 0} \sum_{n_1 \neq 0} \int_{\rr} \frac{1}{(1 +
		| \tau - n_{1}^m - (n - n_{1})^m |)^{1- \ee}(1 + | \tau - n^m
		|)^{\frac{1}{2}-\eta}} d \tau
	\end{split}
\end{equation*}
%
%
%
which due to the estimate
%
%
\begin{equation}
	\label{m16g}
	\begin{split}
		(1 + | \tau - n^m |)
		& = 1 + \frac{1}{4}| \tau - n^m | + \frac{3}{4}| \tau - n^m |
		\\
		& \ge 1 + \frac{1}{4}| \tau - n^m | + \frac{3}{4} \times
		\frac{1}{3}d(n_1,n_2)
		\\
		& = 1 + \frac{1}{4}| \tau - n^m | + \frac{1}{4}| n^m - n_1^m - (n -
		n_1)^m |
		\\
		& \ge \frac{1}{4}| \tau - n_1^m - (n - n_1)^m |
	\end{split}
\end{equation}
%
%
is bounded by
%
%
\begin{equation}
	\label{m15g}
	\begin{split}
		& 4 c_{\ee} \sum_{n_1 \neq 0} \frac{1}{| n_1 |^{(\frac{1}{2} +
		\eta)(m-1)}} 	\sup_{n \neq 0} \sum_{n_1 \neq 0}	\int_{\rr} \frac{1}{(1 + |
		\tau - n_{1}^m - (n - n_{1})^m |)^{\frac{3}{2}-\ee - \eta}} d \tau
		\\
		& \lesssim \sum_{n_! \neq 0} \frac{1}{| n_1 |^{(\frac{1}{2} +
		\eta)(m-1)}} 		\qquad (\text{for} \ \eta \ \text{sufficiently small})
		\\
		& < \infty, \qquad (\text{since} \ m \ge 3). \qquad \qed
	\end{split}
\end{equation}
%
%
%
%
%
%
\subsection{Case \ref{mpigeon-case-2}.} From the triangle inequality and
\eqref{mgen-smoothing-ineq}, we see that for $s \ge \frac{1-m}{4}$ we have
%
%
%
%
%
%
\begin{equation}
	\begin{split}
		& | \sum_{n \neq 0} \int_{\rr} a_n |n|^s \left( 1 + | \tau - n^m | \right)^{-1} | 
		\wh{w_{fg}}(n, \tau) | d \tau |
		\\
		& \lesssim \sum_{n \neq 0}  \int_{\rr} |a_{n}| (1+ | \tau - n^m |)^{-1} \wh{\overset{\sim}{C_f} C_g} d
		\tau
	\\	
	& = \sum_{n \neq 0} \int_{\rr} |a_{n}| (1+ | \tau - n^m |)^{-5/8} (1 + | \tau - n^m
	|)^{-3/8} \wh{\overset{\sim}{C_f} C_g} d
		\tau
		\\
		& \le \|a_{n} (1 + | \tau - n^m |)^{-5/8}\|_{L^2(\zz \times \rr)}  \| (1 +
		| \tau - n^m |)^{-3/8} \wh{\overset{\sim}{C_f} C_g}  \|_{L^2(\zz \times
		\rr)}
		\end{split}
\end{equation}
%
%
where the last step follows from Cauchy-Schwartz. Applying the change of
variable $\tau - n^{m } \mapsto \tau'$ we obtain  %
%%
\begin{equation*}
	\begin{split}
		\|a_{n} (1 + | \tau - n^m |)^{-5/8}\|_{L^2(\zz \times \rr)} 
		& = \left( \sum_{n \in \zz} | a_{n} |^2\int_{\rr} \frac{1}{\left( 1 + | \tau -
		n^{m } | \right)^{5/4}} \ d \tau  
		\right)^{1/2} 
		\\
		& = \left ( \sum_{n \in \zz}
		| a_n |^2 
		\int_{\rr} \frac{1}{\left( 1 + | \tau' | \right)^{5/4}} \ d 
		\tau' \right)^{1/2}
		\\
		& \simeq \|a_n\|_{\ell^2}
		\end{split}
\end{equation*}
%
%
%
while \eqref{m3f-gen}-\eqref{mfour-mult-conseq-gen} yields the bound
%
%
\begin{equation*}
	\begin{split}
	\| (1 + | \tau - n^m |)^{-3/8} \wh{\overset{\sim}{C_f} C_g}  \|_{L^2(\zz
	\times \rr)} \lesssim \|f\|_{\dot{X}^s} \|g\|_{\dot{X}^s}
	\end{split}
\end{equation*}
%
%
completing the proof. \qquad \qedsymbol
%
%
%
\section{Proofs of Lemmas and Estimates}
\begin{proof}[Proof of \cref{mlem:cutoff-loc-soln}]
%
%
\begin{equation*}
	\begin{split}
		\lim_{t_{n} \to t} \|u(\cdot, t) - u(\cdot, t_{n})\|_{\dot{H}^s(\ci)} 
		& = \lim_{t_{n} \to t} \|\psi(t) u(\cdot, t) - \psi(t_n) u(\cdot,
		t_{n})\|_{\dot{H}^s(\ci)} 
		\\
		& = \lim_{t_n \to t} \left[ \sum_{n \in \zzdot}| n |
		^{2s} | \psi(t)  \wh{u}(n, t) - \psi(t_n) \wh{ u}(n, t_n) |^2 \right]^{1/2}
		\\
		& = \lim_{t_n \to t} \left[ \sum_{n \in \zzdot} | n |^{2s} | \int_{\rr} (e^{it \tau} - e^{it_{n} \tau}) \wh{\psi u}(n,
		\tau) d \tau |^2 \right]^{1/2}.
	\end{split}
\end{equation*}
		It is clear that
		%
		%
		\begin{equation*}
			\begin{split}
				| n |
				^{2s} | \int_{\rr} (e^{it \tau} - e^{it_{n}\tau}) \wh{\psi u}(n, \tau) d \tau |^2 
		& \le 4  | n |^{2s} \left ( \int_{\rr} |\wh{\psi u}(n, \tau)| d \tau
		\right )^2 
	\end{split}
\end{equation*}
and 
%
%
\begin{equation*}
	\begin{split}
 \sum_{n \in \zzdot} | n |^{2s} \left ( \int_{\rr} |\wh{\psi u}(n, \tau)| d \tau
		\right ) ^2 
		& = \| |n |^s \wh{\psi u}\|_{\dot{\ell}_n^2 L_\tau^1}
		\\
		& \le \|\psi u \|_{Y^s}^2 
	\end{split}
\end{equation*}
which is bounded by assumption.
Applying dominated convergence completes the proof. 
\end{proof}
%
%
\begin{proof}[Proof of \cref{mlem:schwartz-mult}]
Note that
%
%
\begin{equation*}
	\begin{split}
		\wh{\psi f}\left( n, \tau \right)
		& = \wh{\psi}(\cdot) * \wh{f}(n,
		\cdot)(\tau)
		= \int_\rr \wh{\psi}(\tau_1) \wh{f} \left( n, \tau - \tau_1 \right) 
		d\tau_1
	\end{split}
\end{equation*}
%
%
and hence
%
%
\begin{equation}
	\label{m1b}
	\begin{split}
		\|\psi f\|_{\dot{X}^s} 
		& = \left( \sum_{n \in \zzdot} |n|^{2s} \int_\rr \left( 1 + | \tau -
		n^{m} | \right) | \int_\rr \wh{\psi}(\tau_1) \wh{f}\left( n, \tau -
		\tau_1
		\right)  d \tau_1 d \tau |^2 \right)^{1/2}
		\\
		& \le \left( \sum_{n \in \zzdot} |n|^{2s} \int_\rr \left( 1 + | \tau -
		n^{m }
		|
		\right) \left( \int_\rr \wh{\psi}\left( \tau_1 \right) \wh{f}\left( n,
		\tau - \tau_1
		\right)  d \tau_1 d \tau \right)^2 \right)^{1/2}.
	\end{split}
\end{equation}
%
%
Using the relation
%
%
\begin{equation*}
	\begin{split}
		1 + | \tau - n^{m } |
		& = 1 + | \tau + \tau_1 - n^{m} |
		\\
		& \le 1 + | \tau_1 | + | \tau - \tau_1 - n^{m} |
		\\
		& \le \left( 1 + | \tau_1 | \right)\left( 1 + | \tau - \tau_1 -
		n^{m} | \right)
	\end{split}
\end{equation*}
%
%
we obtain
%
%
\begin{equation*}
	\begin{split}
		\eqref{m1b}
		& \le \left( \sum_{n \in \zzdot} |n|^{2s} \right.
		\\
		& \times \left . \int_\rr \left(
		\int_\rr \left( 1 + | \tau_1 | \right)^{1/2} | \wh{\psi}(\tau_1) |
		\left( 1 + | \tau - \tau_1 - n^{m} | \right)^{1/2} \wh{f}\left( n, \tau
		- \tau_1
		\right)d \tau_1
		\right)^2 d \tau \right)^{1/2}
	\end{split}
\end{equation*}
%
%
which by Minkowski's inequality is bounded by
%
%
\begin{equation}
	\label{m2b}
	\begin{split}
		& \left( \sum_{n \in \zzdot} |n|^{2s}  \right.
		\\
		& \times \left. \left( \int_\rr \left[ \int_\rr
		\left( 1 + | \tau_{1} | \right) | \wh{\psi}(\tau_1) |^2 \left( 1 + |
		\tau - \tau_1 - n^{m} |
		\right) | \wh{f}\left( n, \tau - \tau_1 \right) |^2 d \tau_1 
		\right]^{1/2} d \tau \right)^2 \right)^{1/2}.
	\end{split}
\end{equation}
%
%
Using the change of variable $\tau - \tau_1 \to \lambda$ gives
%
%
\begin{equation*}
	\begin{split}
		\eqref{m2b}
		& = \left( \sum_{n \in \zzdot} |n|^{2s}\right.
		\\
		& \times \left.  \left( \int_\rr \left[
		\int_\rr \left( 1 + | \tau_1 | \right) | \wh{\psi}\left( \tau_1
		\right) |^2 \left( 1 + | \lambda - n^{m} | \right) | \wh{f} \left( n,
		\lambda
		\right)|^2 d \tau_1 \right]^{1/2} d \lambda \right)^2 \right)^{1/2}
		\\
		& =  \left( \sum_{n \in \zzdot} |n|^{2s} \right.
		\\
		& \times \left. \left( \int_\rr \left( 1 + |
		\tau_1 |
		\right)^{1/2} | \wh{\psi}(\tau_1) | d \tau_1 \left[ \int_\rr \left( 1 + |
		\lambda - n^{m} |
		\right) | \wh{f}\left( n, \lambda \right) |^2 d \lambda \right]^{1/2}
		\right)^2 \right)^{1/2}
		\\
		& = c_{\psi} \left( \sum_{n \in \zzdot} |n|^{2s} \left( \left[ \int_\rr
		\left( 1 + | \lambda - n^{m} | \right) | \wh{f}\left( n, \lambda
		\right) |^2 d \lambda
		\right]^{\cancel{1/2}} \right)^{\cancel{2}} \right)^{1/2}
		\\
		& = c_{\psi} \|f\|_{\dot{X}^s},
	\end{split}
\end{equation*}
%
%
concluding the proof. 
\end{proof}
%
%
%
%
%
%
\begin{proof}[Proof of \cref{mlem:number-theory1}] First note that
%
\begin{equation*}
		| - n^m + n_1^m + n_2^m|
		 = 3 | n | |n_1 | |n_2 |.
\end{equation*}
%
%
Hence, it will be enough to show that for $c \ge 0$
%
%
\begin{equation*}
	\begin{split}
		| n | |n_1 | |n_2 | \gtrsim | n |^{\frac{2 + c}{2}}| n_1
		|^{\frac{2-c}{2}}| n_2 |^{\frac{2-c}{2}}
	\end{split}
\end{equation*}
%
%
or, dividing through on both sides by $|n| | n_1 | | n_2 |$ and rearranging terms
%
%
\begin{equation*}
	\begin{split}
		| n |^{c/2} \lesssim | n_1 |^{c/2} | n_2 |^{c/2}.
	\end{split}
\end{equation*}
%
%
But
%
%
\begin{equation*}
	\begin{split}
		| n |^{c/2} &= | n_1 + n_2 |^{c/2}
		\\
		& \le (| n_1 | + |n_2|)^{c/2} 
		\\
		& \le (2\max\{|
		n_1 |, | n_2 |)^{c/2}
		\\
		& \le (2|
		n_1 | | n_2 |)^{c/2}
		\\
		& = 2^{c/2} | n_1 |^{c/2} | n_2 |^{c/2}
	\end{split}
\end{equation*}
%
which completes the proof. 
\end{proof}
%
%
%
\begin{proof}[Proof of \cref{mlem:number-theory}] Define
%
\begin{equation*}
	\begin{split}
		| - n^{m} + n_1^{m} + n_2^{m }|
		& = | n_{1}^{m} - n^{m} + (n-n_{1})^{m}| 
		\\
		& \doteq f(n).
	\end{split}
\end{equation*}
%
%
For fixed $n_1$, the absolute minima
of $f(n)$ occurs at $n = 1+n_{1}$ ($n = n_1$ is not available by assumption). Next, note that
%
%
\begin{equation*}
	\begin{split}
		f(1+ n_{1}) = | n_{1}^{m} - (1 + n_{1})^m + 1 |
		& = | (1 + n_{1} )^{m} - n_{1}^{m} -1 |.
	\end{split}
\end{equation*}
We now seek a lower bound for the right hand side. By symmetry we may assume
$n_1 >0$ without loss of generality.
%
%
\begin{framed}
\begin{remark}
	By the term ``symmetry'', we mean that
	\begin{equation*}
	\begin{split}
	| [1 + (-n_1)]^m - (-n_1)^m -1 |
	& = | (1 - n_1)^m + n_1^m -1 |
	\\
	& = | (1 + p_1)^m + (-p_1)^m -1 |, \qquad p_1 = -n_1
	\\
	& = | (1 + p_1)^m - (p_1)^m -1 |.
	\end{split}
\end{equation*}
%
%
\end{remark}
\end{framed}
%
%
Then 
%
%
\begin{equation*}
	\begin{split}
	| (1 + n_{1} )^{m} - n_{1}^{m} -1 |
	& = | \sum_{1 \le k \le m-1} c_{k} n_1^{k}|, \qquad \{c_k\} \in
	\mathbb{N}\setminus 0
	 \\
	 & = \sum_{1 \le k \le m-1} c_{k} n_1^{k}
	 \\
	 & \ge c_{m-1}  n_1^{m-1}
	 \\
	 & = c_{m-1}  n_1^{c} n_1^{m-1-c}
	 \\
	 & \gtrsim (1 + n_1)^{c}  n_1^{m-1-c}
	 \\
	 & = n^{c} n_1^{m-1-c}. 
 \end{split}
\end{equation*}
%
%
Since we assumed $n_1 >0$ without loss of generality, it follows that 
%
%
\begin{equation*}
	\begin{split}
		f(n) \gtrsim |n|^{c} | n_1 |^{m-1-c}. 
	\end{split}
\end{equation*}
%
%
But since $f(n)$ is symmetric in $n_1$ and $n_2$, a similar argument shows that
%
%
\begin{equation*}
	\begin{split}
		f(n) \gtrsim |n|^{c} | n_2 |^{m-1-c}. 
	\end{split}
\end{equation*}
%
%
Therefore,
%
%
\begin{equation*}
	\begin{split}
		f(n) \gtrsim | n |^{c}| n_1 |^{\frac{m-1-c}{2}} | n_2 |^{\frac{m-1-c}{2}}
	\end{split}
\end{equation*}
%
%
completing the proof. 
%
\end{proof}
%
%
\begin{proof}[Proof of \cref{mlem:calc}]
%
%
%
By the change of variable $\theta \mapsto a/2 + x$, we have
%
%
\begin{equation*}
	\begin{split}
		\int_{\rr} \frac{1}{(1 + | \theta |)(1 + | a - \theta |)}d \theta
	= \int_{\rr} \frac{1}{(1 + |  a/2 + x |)(1 + | a/2 - x |)}d x.
	\end{split}
\end{equation*}
%
%
Hence, it suffices to show that
%
%
\begin{equation*}
	\begin{split}
		\int_{\rr} \frac{1}{(1 + | a - \theta |)(1 + | a + \theta |)}d \theta
		\lesssim \frac{\log(2 + | a |)}{1 + | a |}.
	\end{split}
\end{equation*}
%
%
Let us leave the case $a = 0$ for last. By symmetry, the cases $a<0$ and $a >0$
are equivalent. Hence, to cover the case $a \neq0$, we may assume
without loss of generality that $a >0$.
%
%
Then
\begin{equation}
	\label{ma1}
	\begin{split}
		& \int_{\rr} \frac{1}{(1 + | a - \theta |)(1 + | a + \theta |)}d \theta
		\\
		& = \int_{| \theta| \le a+1 } \frac{1}{(1 + | a - \theta |)(1 + | a + \theta
		|)}d \theta + \int_{| \theta| \ge a+1 } \frac{1}{(1 + | a - \theta |)(1 + |
		a + \theta |)}d \theta.
	\end{split}
\end{equation}
Estimating the second integral of \eqref{ma1}, we have
\begin{equation*}
	\begin{split}
		& \int_{| \theta| \ge a+1 } \frac{1}{(1 + | a - \theta |)(1 + | a + \theta
		|)}d \theta 
		\\
		& = \int_{\theta \ge a + 1} \frac{1}{(1 + \theta-a)(1 + \theta+a)} d \theta
		+ \int_{\theta \le -a -1} \frac{1}{(1 + \theta - a) (1 + \theta + a)}d \theta
		\\
		& = \frac{1}{2a} \int_{\theta \ge a + 1} \left[ \frac{1}{1 + \theta -a} -
		\frac{1}{1 + \theta+a} \right] d \theta
		+ \frac{1}{2a} \int_{\theta \le -a-1} \left[ \frac{1}{1 + \theta+a}
		-\frac{1}{1 + \theta -a} \right] d \theta
		\\
		& = \frac{1}{a} \log(1+a)
		\\
		& \lesssim \frac{\log(2 + |a|)}{1 + | a |}.
	\end{split}
\end{equation*}
To evaluate the first integral of \eqref{ma1}, we split into the cases $a \le \theta \le
a+1$, $-a \le \theta \le 0$, $0 \le \theta \le a$, and $a \le \theta \le a+1$.
However, note that 
%
%
\begin{equation*}
	\begin{split}
		& \int_{a}^{a+1} \frac{1}{(1 + | a - \theta |)(1 + | a + \theta |)}d \theta =
		\int_{-a-1}^{-a} \frac{1}{(1 + | a - \theta |)(1 + | a + \theta |)}d \theta,
		\\
		& \int_{0}^{a} \frac{1}{(1 + | a - \theta |)(1 + | a + \theta |)}d \theta =
		\int_{-a}^{0} \frac{1}{(1 + | a - \theta |)(1 + | a + \theta |)}d \theta.
	\end{split}
\end{equation*}
%
%
Therefore, we need only consider the cases $a \le \theta \le a+1$ and $0 \le
\theta \le a$.  For the case $a \le \theta \le a+1$ we have
%
%
\begin{equation*}
	\begin{split}
		\int_{a}^{a+1} \frac{1}{(1 + | a-\theta |)(1 + | a + \theta |)}d \theta
		& = \int_{a}^{a+1} \frac{1}{(1 + \theta -a)(1 + a + \theta)}d \theta
		\\
		& = \frac{1}{2a} \int_{a}^{a+1} \left[ \frac{1}{1 + \theta -a} -
		\frac{1}{1 + \theta + a}  \right]d \theta
		\\
		& =\frac{1}{2a} \log\left( \frac{1 + \theta -a}{1 + \theta + a} \right) \Big
		|_a^{a+1}
		\\
		& = \frac{1}{2a} \log\left( \frac{2a+1}{a+1} \right)
		\\
		& \lesssim\frac{\log 2}{2a}
		\\
		& \lesssim \frac{\log(2 + | a |)}{1 + | a |}.
	\end{split}
\end{equation*}
%
%
while for the case $0 \le \theta \le a$
%
%
\begin{equation*}
	\begin{split}
		\int_{0}^{a} \frac{1}{(1 + | a - \theta |)(1 + | a + \theta |)}d \theta
		& = \int_{0}^{a} \frac{1}{(1 +  a - \theta )(1 +  a + \theta )}d \theta
		\\
		& = \frac{1}{2(1 + a)} \int_{0}^{a} \left[ \frac{1}{1 + a - \theta} +
		\frac{1}{1 + a + \theta} \right]d \theta
		\\
		& = \frac{1}{2(1 + a)} \log \left( \frac{1 + a + \theta}{1 + a - \theta}
		\right) \Big |_{0}^{a}
		\\
		& = \frac{\log\left( 1 + 2a \right)}{2\left( 1 + a \right)}
		\\
		& \lesssim \frac{\log(2 + | a |)}{1 + | a |}.
	\end{split}
\end{equation*}
%
%
This completes the proof for the case $a \neq 0$. Lastly, for the case
$a =0$, we use dominated convergence and our preceding work to
conclude that
%
%
\begin{equation*}
	\begin{split}
		\int_{\rr} \frac{1}{(1 + | \theta|)^2} d \theta
		& = \lim_{a \to 0}
		\int_{\rr} \frac{1}{(1 + | a - \theta |)(1 + | a + \theta |)}d \theta
		\\
		& \lesssim \lim_{a \to 0} \frac{\log(2 + | a |)}{1 + | a |}
		\\
		& =  \log 2
		\\
		& = \frac{\log(2 + | 0 |)}{1 + | 0 |}. 
	\end{split}
\end{equation*}
%
This completes the proof. 
\end{proof}
%

\chapter{Well-Posedness for a Modified Nonlinear Schr\"{o}dinger Equation }
%\author{Alex Himonas, David Karapetyan, and Gerson Petronilho}
%\address{Department of Mathematics  \\
%University  of Notre Dame\\
%Notre Dame, IN 46556 }
%\address{Department of Mathematics \\
%University  of Notre Dame\\
%Notre Dame, IN 46556 }
%\address{Departamento de Matemática \\
%Universidade Federal de São
%Carlos \\
%Rodovia Washington Luiz, Km 235, São Carlos, SP,
%13565-905, Brasil}
				  %
				  %
				  %
				  %
				  %
				  %
				  \section{Introduction}
				  We consider the modified nonlinear Schr\"{o}dinger (mNLS) 
				  initial value problem (ivp)
%
%
\begin{gather}
	\label{nmNLS-eq}
	i \p_t u + \p_x^{m} u + \lambda |u|^2 u =0,
		\\
		\label{nmNLS-init-data}
		u(x,0) = \vp(x) \in H^s(\ci), \ \ t \in \rr, \ \ x \in \ci
\end{gather}
%
%
where $m \in \{2,4,6,\dots\}$ and $\lambda \in \{-1, 1\}$. 
%
%
%
%
%
%
%%%%%%%%%%%%%%%%%%%%%%%%%%%%%%%%%%%%%%%%%%%%%%%%%%%%%
%
%
%				Outline
%
%
%%%%%%%%%%%%%%%%%%%%%%%%%%%%%%%%%%%%%%%%%%%%%%%%%%%%%
%
%
%
%
%
%
%
To derive a weak formulation of the mNLS ivp, we first let
$\ci = [0, 2 \pi]$, and use
the following notation for the Fourier transform
%
%
%
%
\begin{equation}
	\label{nfour-trans-pde}
	\begin{split}
    \widehat{f}(n) = \int_{\ci} e^{-ix n} f(x) \, dx.
	\end{split}
\end{equation}
Applying the spatial Fourier transform to the mNLS ivp we obtain
%
%
\begin{gather}
  \p_t \widehat{u}(n, t) = i^{m+1} n^m \widehat{u}(n, t) + \lambda i  
	\widehat{w} (n, t),
  \label{neq-1}
	\\
	\widehat{u} (n,0) = \widehat{\vp}(n)
  \label{neq-1-init-data}.
\end{gather}
%
%
Similarly, we have 
\begin{gather*}
  \p_t [\widehat{u}(n, -t)] =
  i^{m+1} n^m \widehat{u}(n, -t) + \lambda i  
	\widehat{w} (n, -t),
  \\
  \widehat{u} (n,0) = \widehat{\vp}(n)
\end{gather*}
or
\begin{gather}
  \label{neq-2}
  \p_t \widehat{u}(n, -t) = -i^{m+1} n^m \widehat{u}(n, -t) - \lambda i  
	\widehat{w} (n, -t)
  \\
\widehat{u} (n,0) = \widehat{\vp}(n)
  \label{neq-2-init-data}.
\end{gather}
Since \eqref{neq-1}-\eqref{neq-1-init-data} and
\eqref{neq-2}-\eqref{neq-2-init-data} are equivalent initial value problems, we
may without loss of generality restrict our attention to
\eqref{neq-1}-\eqref{neq-1-init-data} when $m \in \left\{ 2, 6, 10, \dots
\right\}$, and \eqref{neq-2}-\eqref{neq-2-init-data} when $m \in \left\{ 4, 8, 12,
\dots
\right\}$. That is, it is enough to consider the initial value problem   
%
%
\begin{gather}
  \label{nhk}
  \p_t \widehat{u}(n, t) = -i n^m \widehat{u}(n, t) + (-1)^{\frac{m+2}{2}} \lambda i  
  \widehat{w} (n, t)	\\
	\widehat{u} (n,0) = \widehat{\vp}(n)
\end{gather}
for $m \in \left\{ 2, 4, 6, \dots \right\}$.
Assume $\lambda = (-1)^{\frac{m+2}{2}}$; we will address the case $\lambda =
(-1)^{\frac{m}{2}}$ later. Then from the above, we obtain 
the ivp
\begin{gather}
  \label{nfour}
  \p_t \widehat{u}(n, t) = -i n^m \widehat{u}(n, t) + i  
  \widehat{w} (n, t)
	\\
  \label{nfour-init-data}
	\widehat{u} (n,0) = \widehat{\vp}(n)
\end{gather}
for $ m \in \left\{ 2, 4, 6, \dots \right\}$.
Multiplying \eqref{nfour} by the integrating factor $e^{itn^m}$ then yields
%%
%%
\begin{equation*}
	\begin{split}
		\left[ e^{ it n^m} \widehat{u}(n) \right]_t = i
		 e^{ it n^m} \widehat{w} (n, t).	
	\end{split}
\end{equation*}
%
%
Integrating from $0$ to $t$, we obtain
%
%
\begin{equation*}
	\begin{split}
		\wh{u}(n, t) = \wh{\vp}(n) e^{- it n^m} + i  
		\int_0^t e^{ i(t' - t) n^m} \wh{w}(n, t') \ 
		dt'.
	\end{split}
\end{equation*}
%
%
Therefore, by Fourier inversion 
%
%
\begin{equation}
	\label{nmNLS-integral-form}
	\begin{split}
		u(x,t) & = \frac{1}{2\pi} \sum_{n \in \zz} \wh{\vp}(n) e^{i\left( xn - t n^m 
		\right)} 
		\\
    & + \frac{i}{2 \pi} \sum_{n \in \zz} \int_0^t e^{i\left[ xn + \left( t' - t 
		\right) n^m \right]} \wh{w}(n, t') \ dt'.
	\end{split}
\end{equation}
%
%
Note that \eqref{nmNLS-integral-form} is a weaker 
restatement of the Cauchy-problem \eqref{nmNLS-eq}-\eqref{nmNLS-init-data}, 
since by construction any classical solution of the mNLS 
ivp is a solution to \eqref{nmNLS-integral-form}. 
%
%
We now derive an integral 
equation global in $t$ and equivalent to \eqref{nmNLS-integral-form} for $t 
\in [-\delta, \delta]$. Let $\psi(t)$ be a cutoff function symmetric about the 
origin such that $\psi(t) = 1$ for $|t| \le 1/2$ and $\text{supp} \, \psi 
= [-1, 1 ]$. Define $\psi_{\delta}(t) = \psi(2t/\delta)$.  Multiplying both sides of expression
$\eqref{nmNLS-integral-form}$ by $\psi_{\delta}(t)$, we obtain
%
%
\begin{equation}
	\begin{split}
		\label{ncutoff-int-eq}
    \psi_{\delta} u(x, t)
		& = \frac{1}{2 \pi} \psi_{\delta}(t) \sum_{n \in \zz} e^{i(xn - t n^{m })} \widehat{\vp}(n) 
		\\
		& + \frac{i }{2 \pi} \psi_{\delta}(t) \int_0^t \sum_{n \in \zz} 
		e^{i\left[ xn + (t - t')n^m \right]} \wh{w}(n, t') \ dt'.
	\end{split}
\end{equation}
%
%
Noting that $e^{i\left( xn + tn^{m } \right)}$ 
does not depend on $t'$, we may rewrite the second term
%
%
\begin{equation}
	\label{npre-prim-int-form}
	\begin{split}
		& \frac{i }{2 \pi} \psi_{\delta}(t) \int_0^t \sum_{n \in \zz} 
		e^{i\left[ xn + (t - t') n^m \right]} \wh{w}(n, t') \ dt'
		\\
		& = \frac{i}{2 \pi} \psi_{\delta}(t) \sum_{n \in \zz} e^{i\left( xn + t 
		 n^{m } 
		\right)} \int_0^t e^{- it'n^{m }} \wh{w}(n, t') \ dt'.
	\end{split}
\end{equation}
%%
%%
We remark that this is a \emph{global} relation in $t$. Therefore, by Fourier 
inversion
%
%
%
%
%
%
%
\begin{equation*}
	\begin{split}
		\text{rhs of} \; \eqref{npre-prim-int-form}
		& = \frac{i}{4 \pi^2} \psi (t) \sum_{n \in \zz} e^{i\left( xn + t 
		 n^m
		\right)} \int_0^t \int_\rr e^{it'\left( \tau - n^m \right) }
		\wh{w}(n, \tau) d \tau dt'
		\\
		& = \frac{1}{4 \pi^2} \psi_{\delta}(t) \sum_{n \in \zz} \int_\rr 
		e^{i\left( xn + tn^m \right)} \frac{e^{it\left( \tau - n^m 
		\right)}-1}{\tau - n^m} \wh{w}(n, \tau) d \tau
	\end{split}
\end{equation*}
%
%
where the last step follows from Fubini and integration. Substituting
into \eqref{ncutoff-int-eq} we obtain
%
%
\begin{equation}
	\begin{split}
		\label{ncutoff-int-eq-2}
    \psi_{\delta} u(x, t)
		& = \frac{1}{2 \pi} \psi_{\delta}(t) \sum_{n \in \zz} e^{i(xn - tn^{m })} \widehat{\vp}(n) 
		\\
		& + \frac{1}{4 \pi^2} \psi_{\delta}(t) \sum_{n \in \zz} \int_\rr
		e^{i(xn + t n^m)} \frac{e^{it(\tau - n^m)}- 1}{\tau - n^m} 
		\wh{w}(n, \tau) \ d \tau.
	\end{split}
\end{equation}
%
%
%
Next, we localize near the singular curve $\tau =  n^m$.  Multiplying the
summand of the second term of \eqref{ncutoff-int-eq-2} by $1 + \psi(\tau -
n^m) - \psi(\tau -
n^m) $ and
rearranging terms, we have
%
%
\begin{equation*}
	\begin{split}
    \psi_{\delta} u(x, t)
		& = \frac{1}{2 \pi} \psi_{\delta}(t) \sum_{n \in \zz} e^{i(xn + t n^{m 
		})} \widehat{\vp}(n) 
		\\
		& + \frac{1}{4 \pi^2} \psi_{\delta}(t) \sum_{n \in \zz} \int_\rr e^{ixn}  
		e^{it \tau} \frac{ 1 - \psi(\tau - n^m) 
		}{\tau - n^m} \wh{w}(n, \tau) \ d \tau
		\\
		& - \frac{1}{4 \pi^2} \psi_{\delta}(t) \sum_{n \in \zz} \int _\rr e^{i(xn + 
		t n^m)}
		 \frac{1- \psi(\tau - n^m)}{\tau - n^m} \wh{w}(n, \tau) \ d \tau
		\\
		& + \frac{1}{4 \pi^2} \psi_{\delta}(t) \sum_{n \in \zz} \int_\rr
		e^{i(xn + t n^m)}
		\frac{\psi(\tau - n^m)\left[ e^{it(\tau - n^m)}-1 
		\right]}{\tau - n^m} \wh{w}(n, \tau) \ d \tau
	\end{split}
\end{equation*}
%
%
which by a power series expansion of $[e^{it(\tau - n^m)}-1]$ simplifies  
to
%
%
\begin{align}
	\label{nmain-int-expression-0}
  & \psi_{\delta} u(x, t) 
		\\
		\label{nmain-int-expression-1}
		& = \frac{1}{2 \pi} \psi_{\delta}(t) \sum_{n \in \zz} e^{i(xn + tn^{m 
		})} \widehat{\vp}(n) 
		\\
		\label{nmain-int-expression-2}
		& + \frac{1}{4 \pi^2} \psi_{\delta}(t) \sum_{n\in \zz} \int_\rr e^{ixn}  
		e^{it \tau} \frac{ 1 - \psi(\tau -  n^m) 
		}{\tau -  n^m} \wh{w}(n, \tau) \ d \tau
		\\
		\label{nmain-int-expression-3}
		& - \frac{1}{4 \pi^2} \psi_{\delta}(t) \sum_{n\in \zz} \int_\rr e^{i(xn + 
		t n^m)}
		 \frac{1- \psi(\tau -  n^m)}{\tau -  n^m} \wh{w}(n, \tau) \ d \tau
		\\
		\label{nmain-int-expression-4}
		& + \frac{1}{4 \pi^2} \psi_{\delta}(t) \sum_{k \ge 1} \frac{i^k t^k}{k!}
		\sum_{n \in \zz} \int_\rr e^{i(xn + t n^m )}
		\psi(\tau -  n^m) (\tau -  n^m)^{k-1} \wh{w}(n, \tau)  
		\\
		& \doteq T(u) \notag
\end{align}
%
%
where $T = T_{\vp, \psi, \delta}$. 
\begin{definition}
  Given a Banach space $X$, denote $B_{X}(R) \doteq \left\{ f: \| f \|_{X} < R
      \right\}$. We say that the mNLS ivp \eqref{nmNLS-eq}-\eqref{nmNLS-init-data} is
	\emph{locally well posed} in
	$H^s(\ci)$ if 
	\begin{enumerate}
    \item For every $\vp(x) \in B_{H^{s}(\ci)}(R)$
      there exists $\delta>0$ depending on $R$ and a unique $u \in C([-\delta,
      \delta], H^s(\ci))$ satisfying
      \eqref{nmain-int-expression-0}-\eqref{nmain-int-expression-4}.
    \item
      The flow map $u_0 \mapsto u(t)$ is uniformly continuous from
      $B_{H^{s}(\ci)}(R)$ 
      to $C(\left[ -\delta, \delta \right], H^s(\ci))$. That is, if
      $\{u_{0,n} \}, \{v_{0,n}\} \subset B_{H^{s}(\ci)}(R)$ such that $\|u_{0,n} -
      v_{0,n} \|_{H^{s}(\ci)} \to 0$, then \\
      $\sup_{t \in [-\delta, \delta]}
      \|u_{n}(\cdot, t) - v_{n}(\cdot, t) \|_{H^s(\ci)} \to 0$.
  \end{enumerate}
	Otherwise, we say that the mNLS ivp is \emph{ill-posed}.
\end{definition}
%
\begin{definition}
  Let $\mathcal{Y}$ be the space of functions $F(\cdot)$ such that
  \begin{enumerate}[(i)]
   \item{$F: \ci \times \rr \to \cc$ }.
   \item{ $F(x, \cdot) \in \mathcal{S}(\rr)$ for each $x \in \ci$}.
   \item{ $F(\cdot, t) \in C^{\infty}(\ci)$for each $t \in \rr$}.
  \end{enumerate}
  Let $Y^{s}$ denote the completion of $\mathcal{Y}$ with
  respect to the norm
  %
  %
  \begin{equation}
	\label{nY-s-norm}
	\begin{split}
		\|u\|_{Y^s} = \|u\|_{X^s} + \|\wh{u}\|_{ \ell^2_n L^1_\tau }
	\end{split}
\end{equation}
  %
    %
    where
\begin{equation}
	\label{nX^s-norm}
	\begin{split}
		& \|u\|_{X^s}
		= \left ( \sum_{n\in \zz} \left (1 + |n| \right )^{2s} \int_\rr \left ( 1 + | 
		\tau - n^{m } | \right ) | \wh{u} ( n, \tau ) |^2
		\right )^{1/2}
	\end{split}
\end{equation}
and
%
%
\begin{equation}
	\label{nE-norm}
	\|\wh{u}\|_{ \ell^2_n L^1_\tau } = \left[ \sum_{n \in \zz}(1 + | n |)^{2s} \left(
	\int_{\rr}| \wh{u}(n, \tau) |d \tau \right)^{2} \right]^{1/2}.
\end{equation}
    %
  \end{definition}

%
The $Y^s$ spaces have the following important property, whose proof
is provided in the appendix.
\begin{lemma}
	\label{nlem:cutoff-loc-soln}
	Let $\psi(t)$ be a smooth cutoff function with $\psi(t) =1$ for $t \in [-1,
  1]$, and define $\psi_{\delta}(t) = \psi(t/\delta)$. If
  $\psi_{\delta}(t)u(x,t) \in Y^s$, then $u \in C([-\delta, \delta], H^s(\ci))$.
\end{lemma}

We are now prepared to state the main result of this paper.
%
%
%
%
%%%%%%%%%%%%%%%%%%%%%%%%%%%%%%%%%%%%%%%%%%%%%%%%%%%%%
%
%
%	Main Result				
%
%
%%%%%%%%%%%%%%%%%%%%%%%%%%%%%%%%%%%%%%%%%%%%%%%%%%%%%
%
%
\begin{theorem}
\label{nthm:main}
The initial value problem 
\eqref{nmNLS-eq}-\eqref{nmNLS-init-data} is locally well-posed in $H^s(\ci)$ for $s \ge
0$, and ill-posed for $s <0$. %
%
\end{theorem} 
%
%
%
%
%
%
%%%%%%%%%%%%%%%%%%%%%%%%%%%%%%%%%%%%%%%%%%%%%%%%%%%%%
%
%
%			Proof of Theorem	
%
%
%%%%%%%%%%%%%%%%%%%%%%%%%%%%%%%%%%%%%%%%%%%%%%%%%%%%%
%
%
\section{Proof of Main Theorem}
%
%
To prove well-posedness for the mNLS ivp we we will 
show that for initial data $\vp \in B_{H^{s}(\ci)}(R)$, $T$ is a contraction on
$B_{Y^{s}}(M_{R})$ by estimating the $Y^s$
norm of \eqref{nmain-int-expression-1}-\eqref{nmain-int-expression-4}. The 
Picard fixed point theorem will
then yield a unique solution to
$u = Tu$. An application of
\cref{nlem:cutoff-loc-soln} will then imply the existence of a unique, local
solution $u \in C([-\delta, \delta], H^s(\ci))$ to
$\psi_{\delta} u = Tu$. Lipschitz continuity of the flow map (and hence, uniform
continuity) will follow from estimates used to establish the contraction
mapping. The proof of ill-posedness will follow via counterexample 
by selecting an appropriate sequence of
solutions to \eqref{nmNLS-eq}-\eqref{nmNLS-init-data}. 
%
\begin{framed}
\begin{remark}
  To cover the case $\lambda = (-1)^{\frac{m}{2}}$ in \eqref{nhk}, we need only
  replace $i$ by $-i$ in
  expressions \eqref{nmain-int-expression-2}-\eqref{nmain-int-expression-4}. This
  does not change their $Y^s$ norms. Hence, we may assume $\lambda =
  (-1)^{\frac{m+2}{2}}$ without loss of generality throughout this paper.
\label{nrem:lambda-arb}
\end{remark}
\end{framed}

%%%%%%%%%%%%%%%%%%%%%%%%%%%%%%%%%%%%%%%%%%%%%%%%%%%%%
%
%
%		Estimation of Integral Equality Part 1		
%
%
%%%%%%%%%%%%%%%%%%%%%%%%%%%%%%%%%%%%%%%%%%%%%%%%%%%%%
%
%
%
%
\subsection{Estimate for \eqref{nmain-int-expression-1}.}
%
%
Letting $f(x,t) = \psi_{\delta}(t) \sum_{n \in \zz} e^{i(xn + tn^{m})} 
\wh{\vp}(n)$, we have $\wh{f}(n,t) = \psi_{\delta}(t) \wh{\vp}(n) e^{itn^{m}}$,
from which we obtain
%
%
\begin{equation}
	\label{nfourier-trans-calc}
	\begin{split}
		\wh{f}(n, \tau)
		& = \wh{\vp}(n) \int_\rr e^{-it( \tau - n^{m})} 
		\psi_{\delta}(t) \ d t
    = \wh{\psi_{\delta}}(\tau - n^{m}) \wh{\vp}(n).
	\end{split}
\end{equation}
%
%
%
%
%
%
Therefore
%
\begin{equation}
	\begin{split}
	\label{nmain-int1-est}
		\|\eqref{nmain-int-expression-1}\|_{Y^s}
		& \simeq \left (  \sum_{n\in \zz} \left (1 + |n| \right )^s \int_\rr \left( 1 + | \tau - n^{m} 
		| \right )
    | \wh{\psi_{\delta}}(\tau - n^{m}) \wh{\vp}(n) |^2 d \tau \right)^{1/2} 
		\\
		& + \left[ \sum_{n \in \zz }\left( 1 + | n | \right)^{2s} \left( \int_{\rr} |
    \wh{\psi_{\delta}}(\tau - n^{m})\wh{\vp}(n) | d \tau
		\right)^{2} \right]^{1/2}
		\\
    & = c_{\psi, \delta}
		\|\vp\|_{H^s(\ci)}.
	\end{split}
\end{equation}
%
%
%
%
\subsection{Estimate for \eqref{nmain-int-expression-2}.}
We now need the following lemma, whose proof is provided in the appendix.
%
%
%%%%%%%%%%%%%%%%%%%%%%%%%%%%%%%%%%%%%%%%%%%%%%%%%%%%%
%
%
%			Schwartz Multiplier	
%
%
%%%%%%%%%%%%%%%%%%%%%%%%%%%%%%%%%%%%%%%%%%%%%%%%%%%%%
%
%
\begin{lemma}
\label{nlem:schwartz-mult}
	For $\psi \in S(\rr)$,
%
%
\begin{equation}
	\label{nschwartz-mult}
	\begin{split}
		\|\psi f \|_{Y^s} \le c_{\psi} \|f \|_{Y^s}.
	\end{split}
\end{equation}
%
%
\end{lemma}
%
%
Hence,
%
%
\begin{equation}
  \label{nyu}
	\begin{split}
		\|\eqref{nmain-int-expression-2}\|_{Y^s} 
    & \le c_{\psi, \delta}
		\| \sum_{n \in \zz} e^{ixn} \int_\rr 
		e^{it \tau} \frac{ 1 - \psi (\tau - n^{m} ) 
		}{\tau - n^{m}} \wh{w}(n, \tau) \ 
		d \tau\|_{Y^s}.
			\end{split}
\end{equation}
%
We first estimate
%
%
\begin{equation}
\label{nmain-int2-est-X-s-part}
\begin{split}
  & \| \sum_{n \in \zz} e^{ixn} \int_\rr 
		e^{it \tau} \frac{ 1 - \psi (\tau - n^{m} ) 
		}{\tau - n^{m}} \wh{w}(n, \tau) \ 
		d \tau\|_{X^s}
		\\
    & = \left( \sum_{n \in \zz} \left (1 + |n| \right )^{2s} \int_\rr
		(1 + |\tau - n^{m}|) \left | \frac{1 - \psi(\tau - n^{2 
		})}{\tau - n^{m}} 
		\wh{w}(n, \tau) \right |^2 \ d 
		\tau \right)^{1/2}
		\\
		& \le \left( \sum_{n \in \zz} \left (1 + |n| \right )^{2s} \int_{| \tau - n^{m }| \ge 1}
		(1 + |\tau - n^{m}|) \frac{|\wh{w}(n, \tau)|^2 }{|\tau - n^{m }|^2} 
		\ d 
		\tau \right)^{1/2}
		\\
		& \lesssim  \left( \sum_{n \in 
		\zz} \left (1 + |n| \right )^{2s} \int_\rr
		\frac{|\wh{w}(n, \tau) |^2}{1+ |\tau - 
		n^{m}|} 
		 \ d \tau 
		\right)^{1/2}
		\\
		& \lesssim  \|u\|_{X^s}^3
\end{split}
\end{equation}
%
%
%
where the last two steps follow from the inequality 
%
\begin{equation}
	\label{none-plus-ineq}
	\begin{split}
		\frac{1}{|\tau - n^{m}| } \le \frac{2}{1 + |\tau - n^{m}| }, 
		\qquad |\tau - n^{m}| \ge 1
	\end{split}
\end{equation}
%
%
and the following trilinear estimate, whose proof we leave for later.
%
%
%%%%%%%%%%%%%%%%%%%%%%%%%%%%%%%%%%%%%%%%%%%%%%%%%%%%%
%
%
%				Proposition
%
%
%%%%%%%%%%%%%%%%%%%%%%%%%%%%%%%%%%%%%%%%%%%%%%%%%%%%%
%
%
\begin{proposition}
\label{nprop:trilinear-est}
	%
	%
	For any $s \ge 0$ and $b \ge 3/8$, we have
	\begin{equation}
		\left( \sum_{n \in \zz} \left (1 + |n| \right )^{2s} \int_\rr
		\frac{|\wh{w_{fgh}}(n, \tau) |^2}{\left (1+ |\tau - 
    n^{m}| \right )^{2b}} 
		 \ d \tau 
		\right)^{1/2}
		\lesssim \|f\|_{X^s} \|g\|_{X^s}\|h\|_{X^s}
	\end{equation}
	where $w_{fgh}(x,t)$ = $fg \bar h (x,t)$.
%
%
%
%
\end{proposition}
%
%
Furthermore,
%
%
%
%
\begin{equation}
	\label{nmain-int-expression-2-Y-s-part}
	\begin{split}
    & \| \wh{\eqref{nmain-int-expression-2}}\|_{\ell^{2}_{n}L^{1}_{\tau}}
		\\
    & \simeq \left[ \sum_{n \in \zz}(1 + | n |)^{2s} \left(
		\int_{\rr}\frac{|1 - \psi(\tau - n^{m})|}{|\tau - n^{m}|} |\wh{w}(n, \tau)| d
		\tau \right)^{2} \right]^{1/2}
		\\
    & \le \left[ \sum_{n \in \zz}(1 + | n |)^{2s} \left(
    \int_{| \tau - n^{m} | \ge 1 }\frac{|\wh{w}(n, \tau)|}{|\tau - n^{m}|}  d
		\tau \right)^{2} \right]^{1/2}
    \\
    & \lesssim \left[ \sum_{n \in \zz}(1 + | n |)^{2s} \left(
    \int_{\rr}\frac{|\wh{w}(n, \tau)|}{1 + |\tau - n^{m}|}  d
		\tau \right)^{2} \right]^{1/2}
    \\
		& \lesssim \|f\|_{X^s} \|g\|_{X^s}\|h\|_{X^s}
	\end{split}
\end{equation}
%
%
where the last two steps follow from \eqref{none-plus-ineq} and the following
corollary to \cref{nprop:trilinear-est}.
%
%
%%%%%%%%%%%%%%%%%%%%%%%%%%%%%%%%%%%%%%%%%%%%%%%%%%%%%
%
%
%				Second trilinear Estimate 
%
%
%%%%%%%%%%%%%%%%%%%%%%%%%%%%%%%%%%%%%%%%%%%%%%%%%%%%%
%
%
\begin{corollary}
\label{ncor:trilinear-estimate2}
	For $s \ge 0$ we have
%
%
\begin{equation}
	\label{ntrilinear-estimate2}
	\begin{split}
		\left( \sum_{n \in \zz} \left (1 + |n| \right )^{2s}  \left ( \int_\rr 
		\frac{|\wh{w_{fgh}}(n, \tau) |}{1 + | \tau - n^{m } |}
		 \ d\tau \right)^2  \right)^{1/2} \lesssim \|f\|_{X^s} \|g\|_{X^s}\|h\|_{X^s}.
	\end{split}
\end{equation}
\end{corollary}
%
%
Combining \eqref{nyu}, \eqref{nmain-int2-est-X-s-part}, and
\eqref{nmain-int-expression-2-Y-s-part}, we conclude that
%
%
%
%
\begin{equation}
	\label{nmain-int2-est}
	\begin{split}
		\|\eqref{nmain-int-expression-2}\|_{Y^s} \le c_{\psi, \delta}\|f\|_{X^s} \|g\|_{X^s}\|h\|_{X^s}.
	\end{split}
\end{equation}
%
%
\subsection{Estimate for \eqref{nmain-int-expression-3}.}
Letting $$f(x,t) = \psi_{\delta}(t) \sum_{n \in \zz} e^{i\left( xn + tn^{m} \right)} 
\int_\rr \frac{1 - \psi\left( \lambda - n^{m} \right)}{\lambda - n^{m}} 
\wh{w} \left( n, \lambda \right) \ d \lambda,$$ we have
%
%
\begin{equation*}
	\begin{split}
		& \wh{f^x}(n, t) = \psi_{\delta}(t) e^{itn^{m}} \int_\rr
		\frac{1 - \psi\left( \lambda - n^{m} \right)}{\lambda - n^{m}} 
		\wh{w}(n, \lambda) \ d \lambda
	\end{split}
\end{equation*}
and
\begin{equation*}
	\begin{split}
		 \wh{f}\left( n, \tau \right)
		 & = \int_\rr e^{-it\left( \tau - n^{m} 
		\right)} \psi_{\delta}(t) \int_\rr \frac{1 - \psi\left( 
		\lambda - n^{m} 
		\right)}{\lambda - n^{m}} \wh{w}(n, \lambda) \ d \lambda d \tau
		\\
    & = \wh{\psi_{\delta}}\left( \tau - n^{m} \right) \int_\rr 
		\frac{1 - \psi\left( 
		\lambda - n^{m} 
		\right)}{\lambda - n^{m}} \wh{w}(n, \lambda) \ d \lambda.
	\end{split}
\end{equation*}
Therefore,
%
%
\begin{equation}
  \label{niu}
	\begin{split}
		& \| \eqref{nmain-int-expression-3} \|_{X^s} 
		\\
		& \simeq \left( \sum_{n \in \zz} \left (1 + |n| \right )^{2s} \int_\rr \left( 1 + | \tau - n^{m
    } \right ) | | \wh{\psi_{\delta}}\left( \tau - n^{m } \right) |^2 \ d \tau
		\right.
		\\
		& \times \left . |
		\int_\rr \frac{1 - \psi\left( \lambda - n^{m } \right)}{\lambda -
		n^{m }} \wh{w}(n, \lambda) \ d \lambda |^2  \right)^{1/2}
		\\
    & \le c_{\psi, \delta}\left( \sum_{n \in \zz} \left (1 + |n| \right )^{2s} | \int_\rr
		\frac{1 - \psi\left( \lambda - n^{m } \right)}{\lambda - n^{m }}
		\wh{w}(n, \lambda) \ d\lambda |^2 \right)^{1/2}
		\\
		& \simeq \left( \sum_{n \in \zz} \left (1 + |n| \right )^{2s}  \left ( \int_\rr
		\frac{1 - \psi\left( \lambda - n^{m } \right)}{|\lambda - n^{m }|}
		|\wh{w}(n, \lambda) | \ d\lambda \right )^2 \right)^{1/2}
		\\
		& \le \left( \sum_{n \in \zz} \left (1 + |n| \right )^{2s}  \left ( \int_{| \lambda - 
		n^{m } | \ge 1}
		\frac{|\wh{w}(n, \lambda) | }{|\lambda - n^{m }|}
		\ d\lambda \right )^2 \right)^{1/2}.
	\end{split}
\end{equation}
%
%
Applying estimate \eqref{none-plus-ineq}, we bound this by
%
%%
\begin{equation}
	\label{nmain-int3-est-X-s-part}
	\begin{split}
		& 2 \left( \sum_{n \in \zz} \left (1 + |n| \right )^{2s}  \left ( \int_\rr
		\frac{|\wh{w}(n, \lambda)| }{1 + |\lambda - n^{m }|}
		 \ d\lambda \right )^2 \right)^{1/2}
		 \\
		& \lesssim \|u\|_{X^s}^3
	\end{split}
\end{equation}
%
%%
where the last step follows from \cref{ncor:trilinear-estimate2}.
Furthermore, 
%
%
\begin{equation}
	\label{nmain-int-estimate-3-Y-s-part}
	\begin{split}
    \|\wh{\eqref{nmain-int-expression-3}}\|_{\ell^{2}_{n}L^{1}_{\tau}}
		& \simeq \left[ \sum_{n \in \zz} (1 + | n |)^{2s} \int_{\rr} |
    \wh{\psi_{\delta}}(\tau - n^{m}) |^{2} \left( \int_{\rr}\frac{1 - \psi(\lambda -
		n^{m})}{\lambda - n^{m}} \wh{w}(n, \lambda) d \lambda \right)^{2} d \tau
		\right]^{1/2}
		\\
		& \le c_{\psi, \delta} \left[ \sum_{n \in \zz} (1 + | n |)^{2s} \left(
		\int_{\rr} \frac{1 - \psi(\lambda - n^{m})}{\lambda - n^{m}}
		\wh{w}(n, \lambda) d \lambda
		\right)^{2}\right]^{1/2}
		\\
		& \le 2 c_{\psi, \delta} \left[ \sum_{n \in \zz} (1 + | n |)^{2s} \left(
		\int_{\rr} \frac{\wh{w}(n, \lambda) }{1 + |\lambda - n^{m}|}
		d \lambda
		\right)^{2}\right]^{1/2}
		\\
		& \lesssim \|f\|_{X^s} \|g\|_{X^s} \|h\|_{X^s}
	\end{split}
\end{equation}
%
%
where the last two steps follow from \eqref{none-plus-ineq} and
\cref{ncor:trilinear-estimate2}, respectively. Combining \eqref{niu},
\eqref{nmain-int3-est-X-s-part}, and \eqref{nmain-int-estimate-3-Y-s-part}, we
conclude that
%
%
\begin{equation}
	\label{nmain-int3-est}
	\begin{split}
		\|\eqref{nmain-int-expression-3}\|_{Y^s} 
    \le c_{\psi, \delta} \|f\|_{X^s} \|g\|_{X^s} \|h\|_{X^s}.
	\end{split}
\end{equation}
%
%
%
\subsection{Estimate for \eqref{nmain-int-expression-4}.}
Note that
%
%
\begin{equation}
	\label{n1n}
	\begin{split}
		\eqref{nmain-int-expression-4} \simeq \sum_{k \ge 1}
		\frac{i^k}{k!}g_k(x,t)
	\end{split}
\end{equation}
%
%
where 
%
%
\begin{equation*}
	\begin{split}
		& g_k(x,t) = t^k \psi_{\delta}(t) \sum_{n \in \zz} e^{i\left( xn + tn^{m}
		\right)} h_k(n),
		\\
		& h_k(n) = \int_\rr \psi \left( \tau - n^{m } \right) \cdot \left(
		\tau - n^{m } \right)^{k -1} \wh{w}(n, \tau) \ d \tau.
	\end{split}
\end{equation*}
%
%
Hence
%
%
\begin{equation*}
	\begin{split}
		\wh{g_k^x}(n, t) = t^{k} \psi_{\delta}(t) e^{i t n^{m }} h_k(n)
	\end{split}
\end{equation*}
%
%
which gives
%
%
\begin{equation*}
	\begin{split}
		\wh{g_k}(n, \tau)
		& = h_k(n) \int_\rr e^{-it\left( \tau - n^{m } \right)}
		t^{k}\psi_{\delta}(t) \ dt
		\\
		& = h_k(n) \wh{t^{k}\psi_{\delta}(t)} \left( \tau - n^{m } \right).
	\end{split}
\end{equation*}
%
%
Applying this to \eqref{n1n}, we obtain
%
%
\begin{equation}
	\label{n2n}
	\begin{split}
		\|\eqref{nmain-int-expression-4}\|_{X^s} 
		& \simeq \left( \sum_{n \in \zz} \left (1 + |n| \right )^{2s} \int_\rr \left( 1 + | \tau -
		n^{m }
		|
		\right) | \wh{\sum_{k \ge 1} \frac{i^k}{k!}g_k(x,t)} |^2 \ d \tau
		\right)^{1/2}
		\\
		& \le \sum_{k \ge 1} \frac{1}{k!}\left( \sum_{n \in \zz} \left (1 + |n| \right )^{2s}
		\int_\rr \left( 1 + | \tau - n^{m } | \right) | \wh{g_k}(n, \tau) |^2 \
		d \tau \right)^{1/2}
		\\
		& = \sum_{k \ge 1} \frac{1}{k!} \left( \sum_{n \in \zz} \left (1 + |n| \right )^{2s}
		\int_\rr \left( 1 + | \tau - n^{m } | \right) | h_k(n) \wh{t^k
		\psi_{\delta}(t)} \left( \tau - n^{m } \right) |^2 \ d \tau \right)^{1/2}
		\\
		& = \sum_{k \ge 1} \frac{1}{k!} \left( \sum_{n \in \zz} \left (1 + |n| \right )^{2s} |
		h_k(n) |^2 \int_\rr \left( 1 + | \tau - n^{m } | \right) | \wh{t^k
		\psi_{\delta}(t)} \left( \tau - n^{m } \right) |^2 \ d \tau \right)^{1/2}.
	\end{split}
\end{equation}
%
%
Notice that for fixed $n$, the change of variable $\tau - n^{m } = \tau'$
gives
%
%
\begin{equation}
	\label{n3n}
	\begin{split}
		\int_\rr \left( 1 + | \tau - n^{m } | \right) | \wh{t^{k}
		\psi_{\delta}(t)}\left( \tau - n^{m } \right) |^2 \ d \tau
		& = \int_\rr \left( 1 + |\tau'| \right) | \wh{t^k \psi_{\delta}(t)}(\tau') |^2 \
		d \tau'
		\\
		& \le \int_\rr \left( 1 + |\tau'| \right)^2 | \wh{t^k \psi_{\delta}(t)}(\tau')
		|^2 \ d \tau'
		\\
		& \lesssim \int_\rr \left( 1 + | \tau' |^2 \right) | \wh{t^{k}
		\psi_{\delta}(t)}(\tau') |^2 \ d \tau'
		\\
		& = \|t^k \psi_{\delta}(t) \|_{H^1(\rr)}^2.
	\end{split}
\end{equation}
%
%
But
%
%
\begin{equation}
	\label{n4n}
	\begin{split}
		\|t^k \psi_{\delta}(t) \|_{H^1(\rr)}^2
		& = \left( \|t^k \psi_{\delta}(t)\|_{L^2(\rr)} + \|\p_t \left( t^k \psi_{\delta}(t)
		\right)\|_{L^2(\rr)} \right)^2
		\\
		& \lesssim \|t^{k}\psi_{\delta}(t) \|_{L^2(\rr)}^2 + \|\p_t \left (t^{k}
		\psi_{\delta}(t) \right )\|_{L^2(\rr)}^2
		\\
		& \le \|t^k \psi_{\delta}(t) \|_{L^2(\rr)}^2 + \|t^k \p_t \psi_{\delta}(t)
		\|_{L^2(\rr)}^2 + \|k t^{k -1} \psi_{\delta}(t) \|_{L^2(\rr)}^2
		\\
		& = c_{\psi, \delta} + c_{\psi, \delta}' + c_{\psi, \delta}''k^2 
		\\
    & \lesssim c_{\psi, \delta} k^2.
	\end{split}
\end{equation}
%
%
Hence, applying \eqref{n3n} and \eqref{n4n} to \eqref{n2n}, we obtain
%
%%
\begin{equation}
	\label{n5n}
	\begin{split}
		\|\eqref{nmain-int-expression-4} \|_{X^s}
		& \lesssim
    c_{\psi, \delta}\sum_{k \ge 1} \frac{k}{k!} \left( \sum_{n \in \zz} \left (1 + |n| \right )^{2s} | h_k(n) |^2 
		\right)^{1/2}
		\\
		& \lesssim \sum_{k \ge 1} \frac{1}{(k-1)!}
		\times \sup_{k \ge 1} \left( \sum_{n \in \zz} \left (1 + |n| \right )^{2s} | 
		h_k(n) |^2 \right)^{1/2}
		\\
		& = \sum_{k \ge 1} \frac{1}{(k-1)!} \times \sup_{k \ge 1} 
		\left( \sum_{n \in \zz} \left (1 + |n| \right )^{2s} |\int_\rr 
		\psi\left( \tau - n^{m } \right) \cdot \left( \tau - n^{m } 
    \right)^{k -1} \wh{w}(n, \tau) \ d \tau|^{2} \right)^{1/2}.
    	\end{split}
\end{equation}
%
%%
Recall that $0 \le \psi \le 1, \text{ supp} \, \psi \subset [-1,1 ]$. 
This implies $| \psi\left( \tau - n^{m } \right) \cdot \left( \tau - n^{m } \right)^{k 
-1} | \le \chi_{| \tau - n^{m } | \le 1}$ for all $k \ge 1$. Hence, we bound the
right hand side of \eqref{n5n} by
%
%%
\begin{equation*}
	\begin{split}
		& \sum_{k \ge 1} \frac{1}{(k-1)!} \times \left( \sum_{n \in \zz} (1 + | n |)^{2s}| 
		\int_{| \tau - n^{m}  |\le 1}  \wh{w}(n, \tau) \ d \tau |^2 
		\right)^{1/2}
    \\
    & = e \left( \sum_{n \in \zz} (1 + | n |)^{2s}| 
		\int_{| \tau - n^{m}  |\le 1}  \wh{w}(n, \tau) \ d \tau |^2 
		\right)^{1/2}
    \\
    & \le e \left[ \sum_{n \in \zz} (1 + | n |)^{2s}\left (  
		\int_{| \tau - n^{m}  |\le 1} | \wh{w}(n, \tau) | \ d \tau \right ) ^2 
		\right]^{1/2}
	\end{split}
\end{equation*}
%
%%
which by the inequality
%
%%
\begin{equation*}
	\begin{split}
		\frac{1 + | \tau - n^{m } |}{1 + | \tau  - n^{m } |} \le 
		\frac{2}{1 + | \tau - n^{m } |}, \qquad | \tau - n^{m }  | \le 1
	\end{split}
\end{equation*}
%
%%
is bounded by 
%
%%
\begin{equation}
\label{nmain-int4-est-X-s-part}
	\begin{split}
		& 2e \left[ \sum_{n \in \zz} (1 + | n |)^{2s}\left ( \int_\rr
		\frac{|\wh{w}(n, \tau)|}{1 + | \tau - n^{m } |} \ d \tau \right ) ^2 
		\right]^{1/2} \\
		& \lesssim \|u\|_{X^s}^3
	\end{split}
\end{equation}
%
%%
where the last step follows from \cref{ncor:trilinear-estimate2}. Similarly,
we have
%
%
\begin{equation}
\label{nmain-int4-est-Y-s-part}
	\begin{split}
    \|\wh{\eqref{nmain-int-expression-4}}\|_{ \ell^2_n L^1_\tau }
		& \simeq \left[ \sum_{n \in
		\zz}(1 + | n |)^{2s} \left( \int_{\rr} | \sum_{k \ge 1}
		\wh{\frac{i^{k}}{k!}g_{k}(x,t)(n, \tau)} |d \tau \right)^{2} \right]^{1/2}
		\\
		& \le \sum_{k \ge 1} \frac{1}{k!} \left[ \sum_{n \in \zz} (1 + | n
    |)^{2s} \left( \int_{\rr} | \wh{g_{k}}(n, \tau) | d \tau \right)^{2}
		\right]^{1/2}
		\\
		& = \sum_{k \ge 1} \frac{1}{k!} \left[ \sum_{n \in \zz} (1 + | n
		|)^{2s} | h_{k}(n) |^2 \left( \int_{\rr} | \wh{t^{k} \psi_{\delta}(t)}(\tau -
		n^{m}) |d \tau \right)^{2} \right]^{1/2}
		\\
		& = c_{\psi, \delta} \sum_{k \ge 1} \frac{1}{k!} \left[ \sum_{n \in \zz} (1 + | n
		|)^{2s} | h_{k}(n) |^2 \right]^{1/2}
		\\
		& \lesssim \|u\|_{X^s}^{3}
	\end{split}
\end{equation}
%
%
where the last step follows from the computations starting from \eqref{n5n}
through \eqref{nmain-int4-est-X-s-part}.
Combining \eqref{nmain-int4-est-X-s-part} and \eqref{nmain-int4-est-Y-s-part}, we
have
%
%
\begin{equation}
\label{nmain-int4-est}
	\begin{split}
    \|\eqref{nmain-int-expression-4}\|_{Y^s} \le c_{\psi, \delta} \|u\|_{X^s}^{3}.
	\end{split}
\end{equation}
%
%
Collecting estimates \eqref{nmain-int1-est}, \eqref{nmain-int2-est}, 
\eqref{nmain-int3-est}, and \eqref{nmain-int4-est}, and recalling 
\eqref{nmain-int-expression-1}-\eqref{nmain-int-expression-4}, we see that
$$\|Tu\|_{Y^s} \le c_{\psi,\delta} \left( \|\vp \|_{H^s(\ci)} + \|u\|_{X^s}^3 \right )$$ 
which by the inequality $\|u\|_{X^s} \le \|u\|_{Y^s}$ yields the following.
%%
%%%%%%%%%%%%%%%%%%%%%%%%%%%%%%%%%%%%%%%%%%%%%%%%%%%%%
%
%% Contraction Proposition
%				 
%%%%%%%%%%%%%%%%%%%%%%%%%%%%%%%%%%%%%%%%%%%%%%%%%%%%%%
%%
%%
%
\begin{proposition}
\label{nprop:contraction}
	Let $s \ge0$. Then
%
%%
\begin{equation*}
	\begin{split}
		\|Tu\|_{Y^s} \le c_{\psi,\delta} \left( \|\vp \|_{H^s(\ci)} + \|u\|_{Y^s}^3 
		\right).
	\end{split}
\end{equation*}
%
%%
\end{proposition}
We will now use \cref{nprop:contraction} to prove local well-posedness for the 
mNLS ivp. Let $c = c_{\psi, \delta}^{1/2}$. For given $\vp$, we may choose $\delta$ such
that 
%
%%
\begin{equation*}
	\begin{split}
		\|\vp\|_{H^s(\ci)} \le \frac{15}{64c^3}.
	\end{split}
\end{equation*}
%
%%
Then if $$\|u\|_{Y^s} \le \frac{1}{4c}$$ we have
%
%%
\begin{equation*}
	\begin{split}
		\|T u \|_{Y^s} 
		& \le c^2 \left[ \frac{15}{64c^3} + \left( 
		\frac{1}{4c} \right)^3 \right]
		=  \frac{1}{4c}.
	\end{split}
\end{equation*}
%
%%
Hence, $T=T_{\vp, \psi, T}$ maps the ball $B_{Y^{s}}\left( \frac{1}{4c} \right)$ into 
itself. Next, note that
%
%%
\begin{equation*}
	\begin{split}
		Tu - Tv = \eqref{nmain-int-expression-2} + \eqref{nmain-int-expression-3} 
		+ \eqref{nmain-int-expression-4}
	\end{split}
\end{equation*}
%
%%
where now $w = u | u |^2 - v | v |^{2}$. Rewriting
%
%%
\begin{equation*}
	\begin{split}
		u | u |^{2} - v | v |^{2}
		& = | u |^2 \left( u -v \right) + v\left( | u 
		|^2 - | v |^2
		\right)
		\\
		& = u \bar u \left( u -v \right) + v u \bar u - v v \bar v
		\\
		& = u \bar u \left( u - v \right) + v \bar u\left( u - v \right) + v 
		\bar u v - v v \bar v
		\\
		& = u \bar u \left( u -v \right) + v \bar u\left( u - v \right) + v v 
		\left( \overline{u -v} \right),
	\end{split}
\end{equation*}
%
%%
the triangle inequality and linearity of the Fourier transform give
%
%%
\begin{equation*}
	\begin{split}
		| \wh{w}(n, \tau) | = | \mathcal{F}(u | u |^2 - v| v |^2) |
		& \le | \wh{u \overline{u} \left (u -v \right )} | +
		| \wh{v \overline{u} (u -v)} | + |\wh{v v 
		(\overline{u-v})}|
		\\
		& = | \wh{w_1} | + | \wh{w_2} | + | \wh{w_3} |
	\end{split}
\end{equation*}
%
%%
where
%
%%
\begin{equation*}
	\begin{split}
		w_1 = u \bar u \left( u -v \right), \qquad w_2 = v \bar u \left( u -v 
		\right), \qquad w_3 = v v \left( \overline{u -v} \right).
	\end{split}
\end{equation*}
%
%%
Hence, $Tu - Tv = \sum_{\ell=1}^{3} 
T_\ell(u, v)$, where
\begin{align}
	\label{nmain-int-exp-mod1}
	& \frac{i}{4 \pi^2} \psi_{\delta}(t) \sum_{n\in \zz} \int_\rr e^{ixn}  
		e^{it \tau} \frac{ 1 - \psi(\tau - n^{m}) 
		}{\tau - n^{m}} \wh{w_\ell}(n, \tau) \ d \tau
		\\
		\label{nmain-int-exp-mod2}
		- & \frac{i}{4 \pi^2} \psi_{\delta}(t) \sum_{n\in \zz} \int_\rr e^{i(xn + 
		tn^{m})}
		 \frac{1- \psi(\tau - n^{m})}{\tau - n^{m}} \wh{w_\ell}(n, \tau) \ d \tau
		\\
		\label{nmain-int-exp-mod3}
		+ & \frac{i}{4 \pi^2} \psi_{\delta}(t) \sum_{k \ge 1} \frac{i^k t^k}{k!}
		\sum_{n \in \zz} \int_\rr e^{i(xn + tn^{m} )}
		\psi(\tau - n^{m}) (\tau - n^{m})^{k-1} \wh{w_\ell}(n, \tau)  
		\\
		\doteq & T_\ell(u). \notag
\end{align}
Repeating the arguments used to estimate 
\eqref{nmain-int-expression-2}-\eqref{nmain-int-expression-4}, we obtain
%
%%
\begin{equation*}
	\begin{split}
    & \|T_1\|_{Y^s} \le c_{\psi,\delta} \|u -v \|_{Y^s} \|u\|^2_{Y^s}
		\\
    & \|T_2\|_{Y^s} \le c_{\psi,\delta} \|u -v \|_{Y^s} \|u\|_{Y^s} \|v\|_{Y^s}
		\\
    & \|T_3\|_{Y^s} \le c_{\psi,\delta} \|u -v \|_{Y^s} \|v\|_{Y^s}^2.
	\end{split}
\end{equation*}
%
%%
Therefore,
%
%%
\begin{equation}
	\label{n20a}
	\begin{split}
    \|Tu - Tv \|_{Y^s} = & \| \sum_{\ell =1}^{3} T_\ell(u, v) \|_{Y^s}
		\\
    & \le c_{\psi,\delta} \|u -v \|_{Y^s} \left( \|u\|_{Y^s}^2 + 
		\|u\|_{Y^s} \|v\|_{Y^s} + \|v\|_{Y^s}^2 \right)
		\\
		& \le c_{\psi,\delta} \|u -v\|_{Y^s} \left( \|u\|_{Y^s} + \|v\|_{Y^s} \right)^2
		\\
		& = c^2 \|u -v\|_{Y^s} \left( \|u\|_{Y^s} + \|v\|_{Y^s} \right)^2.
	\end{split}
\end{equation}
%
%%
If $u, v \in B_{Y^{s}}(\frac{1}{4c})$, it follows that
%
%%
\begin{equation}
	\label{n21a}
	\begin{split}
		\|Tu - Tv \|_{Y^s}
		& \le c^2 \|u -v \|_{Y^s} \left( \frac{1}{4c} + 
		\frac{1}{4c} \right)^2
		\\
		& = \frac{1}{4} \|u -v \|_{Y^s}. 
	\end{split}
\end{equation}
%
%%
We conclude that $T = T_{\vp, \psi, \delta}$ is a contraction on the ball
$B_{Y^{s}}(\frac{1}{4c})$. A Picard iteration, coupled with
\cref{nlem:cutoff-loc-soln} then yields a unique solution $u \in C(\left[ -\delta,
\delta \right], H^{s}(\ci))$ to the integral equation $\psi_{\delta} u = Tu$.\\
\\
We now establish Lipschitz continuity of the flow map from
$B_{H^{s}(\ci)}(\frac{15}{64c^{3}})$ to $C(\left[ -\delta, \delta \right], H^{s}(\ci))$.
Assume $\vp_1, \vp_2
\subset B_{H^s(\ci)}(\frac{15}{64c^{3}})$.
Then by the preceding arguments there exist $u_1, u_2 \in Y^s$ such that 
$u_1 = T_{\vp_1, \psi, \delta}$, $u_2 = T_{\vp_2, \psi, \delta}$, and so
%
%
\begin{equation*}
	\begin{split}
		T_{\vp_1, \psi, \delta}(u) -
    T_{\vp_2, \psi, \delta}(v) = \frac{1}{2\pi} \psi_{\delta}(t) \sum_{n \in
		\zz}e^{i\left( xn + tn^{m} \right)} \wh{\vp_1 - \vp_2}(n) + \sum_{\ell=1
    }^{3} T_{\ell}(u).
	\end{split}
\end{equation*}
%
%
Using an argument similar to \eqref{nfourier-trans-calc}-\eqref{nmain-int1-est},
we obtain
%
%
\begin{equation*}
	\begin{split}
		\| \frac{1}{2\pi} \psi_{\delta}(t) \sum_{n \in
		\zz}e^{i\left( xn + tn^{m} \right)} \wh{\vp_1 - \vp_2}(n)\|_{Y^s}
		\le c_{\psi,\delta} \|\vp_{1} - \vp_{2}\|_{Y^s}.
	\end{split}
\end{equation*}
%
%
Hence, \eqref{n20a}-\eqref{n21a} gives
%
%
\begin{equation*}
	\begin{split}
    \sum_{\ell=1}^{3} T_{\ell}(u,v) \le \frac{1}{4}\|u-v\|_{Y^s}.
	\end{split}
\end{equation*}
%
%
Hence,
%
%
\begin{equation*}
	\begin{split}
    \|\psi_{\delta} u - \psi_{\delta} v \|_{Y^s} = \|T_{\vp_1}(u) - T_{\vp_2}(v) \|_{Y^s} \le c_{\psi,\delta}
		\|\vp_{1} - \vp_{2} \|_{H^{s}\left( \ci \right)}\| +
		\frac{1}{4} \|u -v \|_{Y^s}
	\end{split}
\end{equation*}
%
%
which implies
%
%
\begin{equation*}
	\begin{split}
    \frac{3}{4} \|\psi_{\delta} u- \psi_{\delta} v\|_{Y^s} \le c_{\psi,\delta} \|\vp_1 - \vp_2 \|_{H^s(\ci)}
	\end{split}
\end{equation*}
%
%
or
%
%
\begin{equation*}
	\begin{split}
    \|\psi_{\delta} u - \psi_{\delta} v \|_{Y^s} \le \frac{4}{3} c_{\psi,\delta} \|\vp_1 - \vp_2 \|_{H^{s}(\ci)}.
	\end{split}
\end{equation*}
%
%
Applying \cref{nlem:cutoff-loc-soln}, we then obtain
%
%
	 %
	 %
	 \begin{equation*}
		 \begin{split}
       \sup_{t \in \left[ -\delta, \delta \right]}\|u(\cdot, t) -v(\cdot, t) \|_{H^s(\ci)} \le \frac{4}{3} c_{\psi,\delta} \|\vp_1 -
			\vp_2 \|_{H^{s}(\ci)}, \qquad t \in [-\delta, \delta].
		 \end{split}
	 \end{equation*}
	 %
	 %
   Hence, the flow map of the mNLS ivp is Lipschitz continuous from
   $B_{H^{s}(\ci)}( \frac{15}{64c^{3}})$ to  $C(\left[ -\delta, \delta \right],
   H^{s}(\ci))$. Since this implies uniform continuity of the flow map from
   $B_{H^{s}(\ci)}( \frac{15}{64c^{3}})$ to  $C(\left[ -\delta, \delta \right],
   H^{s}(\ci))$, the proof of \cref{nthm:main} is complete. \qquad
   \qedsymbol
%
%
%
%
\section{Proof of Trilinear Estimate}
%
%
%
%
%
%
Note first that $|\wh{w_{fgh}}(n, \tau) |  = | \wh{f} * ( \wh{g} 
* \wh{\bar h})(n, \tau)|$ and $| \wh{\bar{h}}(n, \tau) | = |\overline{ \wh{\overline{h}} 
}(n, \tau)| = | \wh{h}(-n, -\tau) |$. It follows that
%
%
\begin{equation}
	\label{nnon-lin-rep}
	\begin{split}
		& | \wh{w_{fgh}}(n, \tau)|
    \\
    & = |  \sum_{n_{1} } \int_{\tau_{1}} \sum_{n_{2}}
    \int_{\tau_{2}} \wh{f}\left( n
    -n_1,  \tau - \tau_1 \right) \wh{g}\left( n_{1} - n_2, \tau_{1} - \tau_2  
\right) \wh{\bar h}\left( n_2, \tau_2 \right) d \tau_2 d \tau_1 |
\\
& \le \sum_{n_{1} } \int_{\tau_{1}} \sum_{n_{2}}
\int_{\tau_{2}}  | \wh{f}\left( n - n_1, \tau - \tau_1 
\right) | \times  | \wh{g}\left( n_1 - n_2, \tau_1 - \tau_2 
\right) | \times | \wh{\bar h}\left( n_{2}, \tau_{2} \right) | d \tau_2 d \tau_1
\\
& = \sum_{n_{1} } \int_{\tau_{1}} \sum_{n_{2}}
\int_{\tau_{2}}  | \wh{f}\left( n - n_1, \tau - \tau_1 
     \right) | \times | \wh{g}\left( n_{1} - n_2, \tau_{1} - \tau_2 
\right) | \times | \wh{h}\left( -n_{2}, - \tau_2 \right) | d \tau_2 d \tau_1
\\
& = \sum_{n_{1} } \int_{\tau_{1}} \sum_{n_{2}}
\int_{\tau_{2}} \frac{c_f\left( n - n_1, \tau - \tau_1 
\right)}{\left (1 + |n - n_{1}| \right )^s \left( 1 + | \tau - \tau_1 - (n - n_{1})^{m} | \right)^{b}}
\\
& \times \frac{c_{g}\left( n_1 - n_2, \tau_1 - \tau_2 \right)}{\left (1 + |n_1 -
n_2| \right ) 
^s\left( 1 + | \tau_1 - \tau_2 -  (n_1 - n_2)^{m }| 
\right)^{b}}
 \times \frac{c_{h}\left( -n_{2}, -\tau_2 \right)}{\left (1 + |n_{2}| \right ) ^s\left( 1 + | 
\tau_2 + n_{2}^{m } | \right)^{b}} \ d \tau_2 d \tau_1
\end{split}
\end{equation}
%
%
where
%
%
\begin{equation*}
	\begin{split}
		c_\sigma(n, \tau) = \left (1 + |n| \right ) ^s \left( 1 + | \tau - n^{m } |  
		\right)^{b} | \wh{\sigma}\left( n, \tau \right) | .
	\end{split}
\end{equation*}
%
%
Hence
%
%
\begin{equation}
	\label{nconvo-est-starting-pnt}
	\begin{split}
		 & \left (1 + |n| \right )^s \left( 1 + | \tau - n^{m } | \right)^{-b} | \wh{w_{fgh}}\left( 
		n, \tau \right) |
		\\
		& \le \left( 1 + | \tau - n^{m } | \right)^{-b}
    \sum_{n_{1} } \int_{\tau_{1}} \sum_{n_{2}}
    \int_{\tau_{2}}     \\
    & \frac{\left (1 + |n| \right )^s}{\left (1 +
		|n - n_{1}| \right )^s \left (1 + | n_1 - n_2| \right )^s \left (1 + |n_{2}| \right )^s} 
		\times \frac{c_f(n - n_{1}, \tau - \tau_1)}{\left( 1 + | \tau - \tau_1 - (n - n_{1})^{m } | 
		\right)^{b}}
		\\
		& \times
		\frac{c_g(n_1 - n_2, \tau_1 - \tau_2)}{\left( 1 + | \tau_1 - \tau_2 - (n_1 - n_2)^{m } | 
		\right)^{b}} \times
		\frac{c_h(-n_{2}, -\tau_2)}{\left( 1 + | \tau_2 + n_{2}^{m } | 
		\right)^{b}}\ d \tau_2 d \tau_1 .
	\end{split}
\end{equation}
%
%
For $s \ge 0$, observe that
%
%
\begin{equation}
	\label{nderiv-bound-easy-s}
	\begin{split}
		\frac{\left (1 + |n| \right ) ^s}{\left (1 + |n - n_{1}| \right ) ^s \left (1 + |n_1 - n_2| \right ) ^s \left (1 + |n_2| \right ) ^s} 
		\le 3^{s}
	\end{split}
\end{equation}
%
%
by the following lemma, whose proof is provided in the appendix.
%
%
\begin{lemma}
\label{nlem:splitting}
	For $v \ge 0$ we have
%
%
\begin{equation}
	\label{nsplitting}
	\begin{split}
		\left ( 1 + |a +b + c| \right)^v \le 3^v \left(1 + | a | \right)^v \left(
		1 + | b | \right)^v \left( 1 + | c | \right)^v.
	\end{split}
\end{equation}
%
%
\end{lemma}
%
%
Hence, from \eqref{nconvo-est-starting-pnt} and \eqref{nderiv-bound-easy-s}, we 
obtain
%
\begin{equation*}
	\begin{split}
		& \left (1 + |n| \right )^s \left( 1 +  | \tau - n^{m }  | \right)^{-b} | 
		\wh{w_{fgh}}\left( n, \tau \right) | 
		\\
    & \lesssim  \frac{1}{\left( 1 +
		| \tau - n^{m}| 
		\right)^{b}}  
		\times
    \sum_{n_{1} } \int_{\tau_{1}} \sum_{n_{2}}
    \int_{\tau_{2}} \frac{c_f\left( n - n_{1}, \tau - \tau_1 
		\right)}{\left (1 + |n - n_{1}| \right )^s \left( 1 + | \tau - \tau_1 - (n - n_{1})^{m} |
		\right)^{b}}
		\\
		& \times \frac{c_{g}\left( n_1 - n_2, \tau_1 - \tau_2 \right)}{\left (1 + |n_1 - n_2| \right ) 
		^s\left( 1 + | \tau_1 - \tau_2 -  (n_1 - n_2)^{m }| 
		\right)^{b}}
    \times \frac{c_{h}\left( -n_2, -\tau_2 \right)}{\left (1 + |n_2| \right )
    ^s\left( 1 + | \tau_2 + n_2^{m } | \right)^{b}} \ d \tau_2 d \tau_1 
    \\
		& = \left( 1 + | \tau - n^{m } | \right)^{-b}
		\wh{C_f C_{g} C^+_{h}} \left( n, \tau \right)
	\end{split}
\end{equation*}
%
%
where
%
%
\begin{equation*}
	\begin{split}
		C_\sigma(x, t) = \left[ \frac{c_\sigma\left( n, \tau \right)}{\left( 
		1 + | \tau - n^{m } | \right)^{b}} \right]^\vee,
		\ \ C^+_\sigma(x, t) = \left[ \frac{c_\sigma\left( -n, -\tau \right)}{\left( 
		1 + | \tau + n^{m } | \right)^{b}} \right]^\vee.
	\end{split}
\end{equation*}
%
%
Therefore
%
%
\begin{equation}
	\label{ngen-holder-pre-estimate}
	\begin{split}
		& \| \left( 1 + |n | \right)^s
		\left( 1 + | \tau - n^{m } | \right)^{-b} \wh{w_{fgh}}(n, 
		\tau)		
		\|_{L^2(\zz \times \rr)}
		\\
		& \lesssim \| \left( 1 + | \tau - n^{m } | \right)^{-b}
		\wh{C_f C_{g} C^+_{h}} \|_{L^2(\zz \times \rr)}.
	\end{split}
\end{equation}
%
We now require the following multiplier estimate, whose proof can be found in 
\cite{Himonas-Misiolek-2001-A-priori-estimates-for-Schrodinger}.
%
%
%%%%%%%%%%%%%%%%%%%%%%%%%%%%%%%%%%%%%%%%%%%%%%%%%%%%%
%
%
%			Four Mult Est	
%
%
%%%%%%%%%%%%%%%%%%%%%%%%%%%%%%%%%%%%%%%%%%%%%%%%%%%%%
%
%
%
%
%
%
%
%
\begin{lemma}
	\label{nlem:four-mult-est-L4}
	Let $(x, t) \in \ci \times \rr $ and $(n, \tau) \in \zz \times \rr$ be 
	the dual variables. Let $v$ be a positive even integer. Then there is a 
	constant $c_v > 0$ such that
%
%
\begin{equation}
	\label{nfour-mult-est-L4*}
	\begin{split}
		\| \left( 1 + | \tau - n^v | 
		\right)^{-\frac{v+1}{4v}}
		\wh{f}\|_{L^2(\zz \times \rr)} \le c_v \|f \|_{L^{4/3}( \ci \times \rr)}.
	\end{split}
\end{equation}
%
%
\end{lemma}
%
%
Applying \cref{ncor:four-mult-est-L4} and generalized H\"{o}lder to the 
right-hand-side of \eqref{ngen-holder-pre-estimate} gives
%
%
\begin{equation}
	\label{ngen-holder-piece-1}
	\begin{split}
		\|\left( 1 + | \tau - n^{m } | \right)^{-b} \wh{C_f C_{ 
		g } C^+_{h}}\|_{L^2(\zz \times \rr)}
		& \lesssim  \|C_f C_{g} C^+_{h} \|_{L^{4/3}(\ci \times \rr)}
		\\
		& \le \|C_f \|_{L^4(\ci \times \rr)} \|C_{g}\|_{L^4(\ci \times \rr)} 
		\|C^+_{h}\|_{L^4(\ci \times \rr)}.
	\end{split}
\end{equation}
%
%
Note that a change of variable gives
%
%
\begin{equation*}
	\begin{split}
		C_\sigma^+(x, t)
		& = \sum_{n \in \zz} \int_\rr e^{i(nx +  \tau t)} \frac{c_\sigma\left( -n, -\tau \right)}{\left( 
		1 + | \tau + n^{m } | \right)^{b}} \ d \tau
		\\
		& = - \sum_{n \in \zz} \int_\rr e^{-i(nx +   \tau t )}
		\frac{c_\sigma\left( n, \tau \right)}{\left( 
		1 + | \tau - n^{m } | \right)^{b}} \ d \tau
	\end{split}
\end{equation*}
%
%
and so
%
%
\begin{equation*}
	\begin{split}
		C_\sigma^+(-x, -t) = -C_\sigma(x, t).
	\end{split}
\end{equation*}
%
%
We will now the need the following dual estimate of
\cref{nlem:four-mult-est-L4}.
%
\begin{corollary}
	\label{ncor:four-mult-est-L4}
	Let $(x, t) \in \ci \times \rr $ and $(n, \tau) \in \zz \times \rr$ be 
	the dual variables. Let $v$ be a positive even integer. Then there is a 
	constant $c_v > 0$ such that
%
%
\begin{equation}
	\label{nfour-mult-est-L4}
	\begin{split}
		\|f\|_{L^4(\ci \times \rr)} \le c_v \|\left( 1 + | \tau - n^v | 
		\right)^\frac{v+1}{4v} \wh{f} \|_{L^2( \zz \times \rr)}
	\end{split}
\end{equation}
for every test function $f(x, t)$. 
%
%
%
%
\end{corollary}
%
%
Recalling that $L^4(\ci \times \rr)$ is invariant under the transformation $(x, 
t) \mapsto (-x,-t)$ and applying 
\cref{ncor:four-mult-est-L4}, we obtain
%
%
\begin{equation}
	\label{nC-sig-estimate}
	\begin{split}
		\| C^+_\sigma \|_{L^4(\ci \times \rr)} = \|C_\sigma \|_{L^4(\ci \times \rr)} 
		& \lesssim \|\left( 1 + | \tau - n^{m } | 
		\right)^{(m +1)/4m} \left( 1 + | \tau - n^{m } | 
		\right)^{-b} c_\sigma \|_{L^2(\zz \times \rr)}
		\\
		& = \|\left( 1 + | \tau - n^{m } | 
		\right)^{[m(1 - 4b) + 1]/4m } c_\sigma \|_{L^2(\zz \times \rr)}
		\\
		& \le \|c_\sigma \|_{L^2(\zz \times \rr)}  \qquad (\text{since  } [m(1 - 4b) + 
		1]/4m \le 0)
		\\
		& = \|\sigma\|_{X^s}.
	\end{split}
\end{equation}
%
%
We conclude from \eqref{ngen-holder-pre-estimate}, \eqref{ngen-holder-piece-1}, 
and \eqref{nC-sig-estimate} that
%
%
%
%
\begin{equation*}
	\begin{split}
		\| \left( 1 + |n | \right)^s \left( 1 + | \tau - n^{m} | \right)^{-b} \wh{w_{fgh}} 
		(n, \tau) \|_{L^2(\zz \times \rr)} \lesssim 
		\|f\|_{X^s}\|g\|_{X^s}\|h\|_{X^s}.
	\end{split}
\end{equation*}
%
%
%
%
%
%
%
\section{Proof of \cref{ncor:trilinear-estimate2}.}
By duality, it suffices to show that 
%
%%
\begin{equation*}
	\begin{split}
		| \sum_{n \in \zz} \left (1 + |n| \right )^{s}
		a_n \int_{\rr} \frac{|\wh{w_{fgh}}(n, \tau)|}{1 
		+ | \tau - n^{m } |} \ d \tau | \lesssim \|f\|_{X^s} \|g\|_{X^s} \|h\|_{X^s}
		\|a_n \|_{\ell^2}
	\end{split}
\end{equation*}
%
%%
for $\{a_n\} \in \ell^2$. By the triangle inequality 
and Cauchy-Schwartz,
%
%%
\begin{equation}
	\label{n1m}
	\begin{split}
		& | \sum_{n \in \zz} \left (1 + |n| \right )^{s} a_n
		\int_{\rr}\frac{| \wh{w_{fgh}}(n, \tau) |}{1 + | \tau - n^{m } |} \ d \tau |
		\\
		& \le \sum_{n \in \zz} \int_{\rr} \frac{| a_n |}{\left( 1 + 
		| \tau - n^{m } |
		\right)^{1/2 + \eta}} \cdot \frac{\left( 1 + | n| \right)^s  |
		\wh{w_{fgh}}(n, \tau) |}{\left( 
		1 + | \tau - n^{m } | \right)^{1/2 - \eta}} \ d \tau
		\\
		& \le \left( \sum_{n \in \zz} | a_{n} |^2\int_{\rr} \frac{1}{\left( 1 + | \tau - n^{m } | \right)^{1 + 2 \eta}} \ d \tau  
		\right)^{1/2} 
		\left ( \sum_{n \in \zz}\int_{\rr} \frac{\left (1 + |n| \right )^{2s} | \wh{w_{fgh}}(n, \tau) 
		|^2}{\left( 1 + | \tau - n^{m } | \right)^{1 - 2 \eta}}\ d \tau 
		\right)^{1/2}
	\end{split}
\end{equation}
%
%%
Restrict $\eta \in (0, 1/8)$. Applying the change of variable $\tau - n^{m }
= \tau'$ we obtain  %
%%

\begin{equation*}
	\begin{split}
		& \left( \sum_{n \in \zz} | a_{n} |^2\int_{\rr} \frac{1}{\left( 1 + | \tau -
		n^{m } | \right)^{1 + 2 \eta}} \ d \tau  
		\right)^{1/2} 
		\\
		& = \left ( \sum_{n \in \zz}
		| a_n |^2 
		\int_{\rr} \frac{1}{\left( 1 + | \tau' | \right)^{1 + 2 \eta}} \ d 
		\tau \right)^{1/2}
		\\
		& \simeq \|a_n\|_{\ell^2}
		\end{split}
\end{equation*}
while \cref{nprop:trilinear-est} gives the bound
\begin{equation*}
	\begin{split}
		\left ( \sum_{n \in \zz}\int_{\rr} \frac{\left (1 + |n| \right )^{2s} | \wh{w_{fgh}}(n, \tau) 
		|^2}{\left( 1 + | \tau - n^{m } | \right)^{1 - 2 \eta}}\ d \tau 
		\right)^{1/2} \lesssim \|f\|_{X^s} \|g\|_{X^s} \|h\|_{X^s}
	\end{split}
\end{equation*}
%
%%
completing the proof.
\qquad \qedsymbol
%
%
%
\section{Proof of Ill-Posedness}
We adapt an argument from~\cite{Burq_Gerad_Tzvetkov-An-instability-}. For $s<0$,
$m \in \{4, 8, 12, \dots\}$, set
%
%
%
%
\begin{equation}
	\label{nill-soln}
	\begin{split}
		u_{n}(x,t)=\frac{1}{2 }n^{-s}e^{it\left( n^{m}+\frac{1}{4}n^{-2s}
		\right)}e^{inx}.
	\end{split}
\end{equation}
%
%
Then
%
%
\begin{equation*}
	\begin{split}
		& i \p_t u_{n}
		= -\frac{1}{2 }n^{-s}\left( n^{m}+\frac{1}{4}n^{-2s} \right)e^{it\left(
		n^{m}+ \frac{1}{4}n^{-2s} \right)}e^{inx},
		\\
		& \p_x^{m}u_{n}  = \frac{1}{2 }n^{-s+m}e^{it\left(
		n^{m}+\frac{1}{4}n^{-2s} \right)}e^{inx},
		\\
		& | u_{n} |^{2}u_{n}  = \frac{1}{8 }n^{-3s}e^{it\left(
		n^{m}+\frac{1}{4}n^{-2s} \right)}e^{inx}.
	\end{split}
\end{equation*}
%
%
Hence,
%
%
\begin{equation*}
	\begin{split}
		i \p_t u_{n} + \p_x^{m}u_{n} + | u_{n} |^{2} u_{n}
		=0.
	\end{split}
\end{equation*}
%
%
Therefore, $u_{n}(x,t)$ solves the initial value problem
%
%
\begin{gather*}
	\begin{split}
		i \p_t u + \p_x^m u + | u |^{2}u = 0,
		\\
    u(x,0) = \frac{1}{2 }n^{-s}e^{inx}.
	\end{split}
\end{gather*}
%
%
Next, we show that $u_{n}(\cdot, t) \in H^{s}(\ci), \text{ } s < 0$ for all $t
\in \rr$.  First, we compute
%
%
\begin{equation*}
	\begin{split}
		\|e^{inx}\|_{H^{s}(\ci)}
		& =  \left[ \sum_{\xi \in \zz} \left( 1+ | \xi |
		\right)^{2s} | \wh{e^{inx}}(\xi) |^{2} \right]^{1/2}
		\\
		& =  \left[ \sum_{\xi \in \zz} \left( 1 + | \xi | \right)^{2s} |
		\int_{\ci}e^{ix(n- \xi)}dx |^{2}\right]^{1/2}.
	\end{split}
\end{equation*}
%
%
Noting that
%
\begin{equation*}
	\begin{split}
		\int_{\ci}e^{ix(n - \xi)}dx =
		\begin{cases}0, \qquad & \xi \neq n 
			\\
			2 \pi, \qquad & \xi = n 
		\end{cases}
	\end{split}
\end{equation*}
%
%
we obtain
%
%
\begin{equation}
	\label{noscill-bound}
	\begin{split}
		\|e^{inx}\|_{H^{s}(\ci)} & = 2 \pi (1 + | n |)^{s}
	\end{split}
\end{equation}
%
%
and so
%
%
\begin{equation*}
	\begin{split}
		\|u_{n}(\cdot, t)\|_{H^s{(\ci)}}
    & = \frac{1}{2}|n|^{-s}\|e^{inx}\|_{H^{s}(\ci)}
    \\
    & = \pi |n|^{-s} (1 + | n |)^{s} 
    \\
    & \le \pi, \qquad s < 0.
	\end{split}
\end{equation*}
%
%
Next, let
%
%
\begin{equation*}
	\begin{split}
		u_{k_{n}}(x,t) = k_{n}n^{-s}e^{it\left( n^{m} + k_{n}^2 n^{-2s}
		\right)}e^{inx}.
	\end{split}
\end{equation*}
%
%
Following our preceding computations, one can check that $u_{n, k_{n}}$ is a solution to the ivp
%
%
\begin{equation}
	\label{nfamily-ivp}
	\begin{split}
		i\p_t u + \p_x^{m} + | u |^{2}u = 0,
		\\
		u(x,0) = k_{n}n^{-s}e^{inx}
	\end{split}
\end{equation}
%
%
and satisfies 
%
%
\begin{equation*}
	\begin{split}
		\|u_{n, k_{n}}(\cdot, t)\|_{H^{s}(\ci)} \le 2 \pi |k_{n}|.
	\end{split}
\end{equation*}
%
%
Furthermore, choosing $\{k_{n}\}_{n} \subset (0, 1/2)$ to be a family of
rational numbers converging to $k =1/2$ and recalling \eqref{noscill-bound}, we
have
%
%
\begin{equation*}
	\begin{split}
		\|u(x,0) - u_{n, k_{n}}(x, 0) \|_{H^s(\ci)} 
		& =
		\|\frac{1}{2}n^{-s}e^{inx} - k_{n}n^{-s}e^{inx} \|_{H^s(\ci)}
		\\
		& = | n |^{-s} \|e^{inx}\|_{H^s(\ci)}|\frac{1}{2} - k_{n}|
		\\
    & = 2 \pi |n|^{-s}(1+| n |)^{s} |\frac{1}{2} - k_{n}|
    \\
    & \to 0, \qquad s < 0
	\end{split}
\end{equation*}
%
%
and
%
%
\begin{equation*}
	\begin{split}
		& \|u_{n}(\cdot, t) - u_{n, k_{n}}(\cdot, t) \|_{H^{s}(\ci)}
		\\
		& = \|\frac{1}{2}n^{-s}e^{it\left( n^{m} + \frac{1}{4}n^{-2s}
		\right)}e^{inx} - k_{n}n^{-s}e^{it\left( n^{m} + k_{n}^{2}n^{-2s}
		\right)}e^{inx} \|_{H^{s}(\ci)}
		\\
		& = | n |^{-s} \|e^{it\left( n^{m} + \frac{1}{4}n^{-2s}
		\right)}e^{inx}\left( \frac{1}{2} - k_{n}e^{it\left(
		k_{n}^{2}n^{-2s}-\frac{1}{4}n^{-2s} \right)} \right)\|_{H^{s}(\ci)}
		\\
		& = | n|^{-s} \|e^{inx} \|_{H^{s}(\ci)}
    | \frac{1}{2} - k_{n}e^{it\left(
    k_{n}^{2}n^{-2s} - \frac{1}{4}n^{-2s} \right )}|
		\\
    & = 2 \pi | n |^{-s} (1 + | n |)^{s} | \frac{1}{2} - k_{n}e^{itn^{-2s}\left(
    k_{n}^{2}- \frac{1}{4}\right)} |.
	\end{split}
\end{equation*}
%
%
Hence, in order for uniform continuity of the flow map to hold for $s < 0$, we must have
%
%
\begin{equation*}
	\begin{split}
		\lim_{n \to \infty}  k_{n} e^{itn^{-2s}\left( k_{n}^{2} -
		\frac{1}{4} \right)}  = \frac{1}{2}.
	\end{split}
\end{equation*}
%
%
But setting $k_{n} = \left( \frac{1}{4} + n^{2s + \ee} \right)^{1/2}$ where  $\ee >
0$, we see that 
%
%
\begin{equation*}
	\begin{split}
		\lim_{n \to \infty} k_{n} e^{itn^{-2s}\left( k_{n}^{2} - \frac{1}{4}
		\right)}
    & = \lim_{n \to \infty} \left( \frac{1}{4} + n^{2s + \ee} \right)^{1/2}
    e^{itn^{\ee}} \\ & = \frac{1}{2} \lim_{n \to \infty} e^{itn^{\ee}}, \qquad s
    < 0
    \\
    & \neq \frac{1}{2}.
	\end{split}
\end{equation*}
%
%
In fact, the above limit does not converge at all. This concludes the proof for
the case $m \in \{4, 8, 12, \dots \}$. For the case $m \in \{2, 6, 10, \dots \}$, we take
%
%
\begin{equation*}
	\begin{split}
		u_{n}(x,t) = \frac{1}{2}n^{-s}e^{it\left( -n^{m} + \frac{1}{4}n^{-2s}
		\right)}e^{inx},
		\\ u_{n, k_{n}}(x,t) = k_{n}n^{-s}e^{it\left( -n^{m} + k_{n}^{2}n^{-2s}
		\right)}e^{inx} 
	\end{split}
\end{equation*}
and then repeat the above arguments. \qquad \qedsymbol
%
%
\begin{framed}
\begin{remark}
	Note that this result implies that it will be impossible to use a Picard
	iteration type argument to prove existence and uniqueness of solutions to the
	mNLS ivp for $s<0$, since this technique would imply uniform
	continuity of the flow map.
\end{remark}
\end{framed}
%
%
%\section{Failure of Continuity of the Flow Map}
%We shall prove the following.
%%
%%
%%%%%%%%%%%%%%%%%%%%%%%%%%%%%%%%%%%%%%%%%%%%%%%%%%%%%%
%%
%%
%%				 Failure of Continuity Theorem
%%
%%
%%%%%%%%%%%%%%%%%%%%%%%%%%%%%%%%%%%%%%%%%%%%%%%%%%%%%%
%%
%%
%\begin{theorem}
%The flow map $u_0 \mapsto u(t)$ of the mNLS ivp is not continuous for $s<0$, for
%any $t \neq 0$. More precisely, there exists initial data $u_0 \in L^2(\ci)$ and
%a sequence of intial data $\{u_{0,n} \} \subset L^2(\ci)$ such that
%%
%%
%\begin{equation*}
%	\begin{split}
%		u_{0,n} \to u_{0} \ \ \text{in} \ \ H^s(\ci)
%	\end{split}
%\end{equation*}
%%
%%
%and
%%
%%
%\begin{equation*}
%	\begin{split}
%		u_{n} \to e^{ \frac{it \gamma}{\pi}\left( \alpha^2 - \|u_{0}\|_{L^2(\ci)} 
%		\right)}u(t) \ \ \text{in} \ \ H^s(\ci)
%	\end{split}
%\end{equation*}
%%
%%
%where $\alpha \in \rr \setminus \|u_0\|_{L^2(\ci)}$ and $u$ is the unique solution to the mNLS ivp with
%initial data $u_{0}$.
%\end{theorem}
%%
%%
%{\bf Proof.} Define the trilinear operator
%%
%%
%\begin{equation*}
%	\begin{split}
%		g(u,v,w) \doteq \bar{u} v w.
%	\end{split}
%\end{equation*}
%%
%%
%Following Molinet, we rewrite this as
%%
%%
%\begin{equation*}
%	\begin{split}
%		g(u,v,w) 
%		&= \sum_{k_1, k_2, k_3 \in \zz}
%		\wh{\bar{u}}(k_{1})\wh{v}(k_2)\wh{w}(k_3)e^{i(k_1 + k_2 + k_3)x}
%		\\
%		& = \sum_{k_1, k_3 \in \zz} \wh{\bar{u}}(k_1)\wh{v}(-k_1)
%		\wh{w}(k_3)e^{ik_3 x} + \sum_{k_1, k_2 \in \zz} \wh{\bar{u}}(k_1)
%		\wh{v}(k_2) \wh{w}(-k_1)e^{ik_2x}
%		\\
%		& - \sum_{ k \in \zz} \wh{\bar{u}}(k) \wh{v}(-k)\wh{w}(-k)e^{-ikx}
%		+
%		\sum_{\substack{k_1, k_2, k_3 \in \zz\\ (k_1 + k_2)(k_1 + k_3) \neq 0}}
%		\wh{\bar{u}}(k_1) \wh{v}(k_2)
%		\wh{w}(k_3)e^{i(k_1 + k_2 + k_3)x}.
%	\end{split}
%\end{equation*}
%%
%%
%In particular, if $u=v=w$, we obtain
%%
%%
%\begin{equation*}
%	\begin{split}
%		g(u,u,u) = \frac{1}{\pi} \|u\|_{L^2}^2 u + \Lambda_1(u, u, u) + \Lambda_2
%		(u, u, u)	
%	\end{split}
%\end{equation*}
%%
%%
%where
%%
%%
%\begin{equation*}
%	\begin{split}
%		& \Lambda_1(u, v, w)
%		 = \sum_{\substack{k_1, k_2, k_3 \in \zz\\ (k_1 + k_2)(k_1 + k_3) \neq 0}}
%		\wh{\bar{u}}(k_1) \wh{v}(k_2)
%		\wh{w}(k_3)e^{i(k_1 + k_2 + k_3)x}
%		\\
%		& \Lambda_2(u, v, w) = - \sum_{ k \in \zz} \wh{\bar{u}}(k) \wh{v}(-k)\wh{w}(-k)e^{-ikx}
%	\end{split}
%\end{equation*}
%
%
%
%
%
%


%
\documentclass[12pt,reqno]{amsart}
\usepackage{amssymb}
\usepackage{cancel}  %for cancelling terms explicity on pdf
\usepackage{yhmath}   %makes fourier transform look nicer, among other things
\usepackage{framed}  %for framing remarks, theorems, etc.
%\usepackage{showkeys}  %shows source equation labels on the pdf
\usepackage[margin=3cm]{geometry}  %page layout
%\usepackage[pdftex]{graphicx} %for importing pictures into latex--pdf compilation
\setcounter{secnumdepth}{1} %number only sections, not subsections
\usepackage[shortalphabetic, initials, msc-links]{amsrefs} %for the bibliography; uses cite pkg
\hypersetup{colorlinks=true, linkcolor=blue, citecolor=blue, urlcolor=blue}
\synctex=1
\numberwithin{equation}{section}  %eliminate need for keeping track of counters
\numberwithin{figure}{section}
\setlength{\parindent}{0in} %no indentation of paragraphs after section title
\renewcommand{\baselinestretch}{1.1} %increases vert spacing of text
%
%
\newcommand{\ds}{\displaystyle}
\newcommand{\ts}{\textstyle}
\newcommand{\nin}{\noindent}
\newcommand{\rr}{\mathbb{R}}
\newcommand{\nn}{\mathbb{N}}
\newcommand{\zz}{\mathbb{Z}}
\newcommand{\cc}{\mathbb{C}}
\newcommand{\ci}{\mathbb{T}}
\newcommand{\zzdot}{\dot{\zz}}
\newcommand{\wh}{\widehat}
\newcommand{\p}{\partial}
\newcommand{\ee}{\varepsilon}
\newcommand{\vp}{\varphi}
%
%
\theoremstyle{plain}  
\newtheorem{theorem}{Theorem}
\newtheorem{proposition}{Proposition}
\newtheorem{lemma}{Lemma}
\newtheorem{corollary}{Corollary}
\newtheorem{claim}{Claim}
\newtheorem{conjecture}[subsection]{conjecture}
%
\theoremstyle{definition}
\newtheorem{definition}{Definition}
%
\theoremstyle{remark}
\newtheorem{remark}{Remark}
%
% Goes with hyperref package and \autoref command--now when you type
% \autoref{some-environment}, it typsets the reference with the appropriate
% environment name. For example, \autoref{some-theorem} gives Theorem 2,
% \autoref{some lemma} gives Lemma 2, and so on
\def\makeautorefname#1#2{\expandafter\def\csname#1autorefname\endcsname{#2}}
\makeautorefname{equation}{Equation}
\makeautorefname{footnote}{footnote}
\makeautorefname{item}{item}
\makeautorefname{figure}{Figure}
\makeautorefname{table}{Table}
\makeautorefname{part}{Part}
\makeautorefname{appendix}{Appendix}
\makeautorefname{chapter}{Chapter}
\makeautorefname{section}{Section}
\makeautorefname{subsection}{Section}
\makeautorefname{subsubsection}{Section}
\makeautorefname{paragraph}{Paragraph}
\makeautorefname{subparagraph}{Paragraph}
\makeautorefname{theorem}{Theorem}
\makeautorefname{theo}{Theorem}
\makeautorefname{thm}{Theorem}
\makeautorefname{addendum}{Addendum}
\makeautorefname{add}{Addendum}
\makeautorefname{maintheorem}{Main theorem}
\makeautorefname{corollary}{Corollary}
\makeautorefname{lemma}{Lemma}
\makeautorefname{sublemma}{Sublemma}
\makeautorefname{proposition}{Proposition}
\makeautorefname{property}{Property}
\makeautorefname{scholium}{Scholium}
\makeautorefname{step}{Step}
\makeautorefname{conjecture}{Conjecture}
\makeautorefname{question}{Question}
\makeautorefname{definition}{Definition}
\makeautorefname{notation}{Notation}
\makeautorefname{remark}{Remark}
\makeautorefname{remarks}{Remarks}
\makeautorefname{example}{Example}
\makeautorefname{algorithm}{Algorithm}
\makeautorefname{axiom}{Axiom}
\makeautorefname{case}{Case}
\makeautorefname{claim}{Claim}
\makeautorefname{assumption}{Assumption}
\makeautorefname{conclusion}{Conclusion}
\makeautorefname{condition}{Condition}
\makeautorefname{construction}{Construction}
\makeautorefname{criterion}{Criterion}
\makeautorefname{exercise}{Exercise}
\makeautorefname{problem}{Problem}
\makeautorefname{solution}{Solution}
\makeautorefname{summary}{Summary}
\makeautorefname{operation}{Operation}
\makeautorefname{observation}{Observation}
\makeautorefname{convention}{Convention}
\makeautorefname{warning}{Warning}
\makeautorefname{note}{Note}
\makeautorefname{fact}{Fact}
%
%
\begin{document}
\title{Well-posedness of NLS in Analytic Spaces on the Circle}
\author{A. Himonas, D. Karapetyan, and G. Petronilho}
\date{November 10, 2010}
%
\maketitle
%
%
%
%
%
\section{Introduction}
\label{sec:introduction}
We consider the Cauchy problem for the NLS equation
\begin{gather}
  \label{eqn:nls}
  i\partial_tu+\partial_x^2u+\lambda |u|^2u=0,
  \\
  \label{eqn:nls-data}
  u(x,0)=\varphi(x) \in \mathcal{C}^\omega(\mathbb{T}),
\end{gather}
%
where $\lambda =\pm 1$, and prove the following theorem.
%
\begin{theorem}
  \label{thm:nls-analyt}
  The solution $u(x,t)$  to the Cauchy problem for NLS \eqref{eqn:nls}--
  \eqref{eqn:nls-data}  with   initial data $\varphi(x)$  in  the Gevrey space
  $G^1(\mathbb{T})=\mathcal{C}^\omega(\mathbb{T})$ belongs to the anisotropic
  Gevrey space $G^{1, \rho}( \mathbb{T}\times [-T, T])$,  $\rho \ge 2$, where
  $T>0$ is the lifespan of the solution. Furthermore, $3\sigma$ is sharp in the
  sense that $u(x,t)$ as a function of  time  may fail to be in $G^r$, $1\le
  r<2$, near $0$.
\end{theorem}
%
%
%
%
%%%%%%%%%%%%%%%%%%%%%%%%%%%%%%%%%%%%%%%%%%%%%%%%%%%%%
%
%
%                Analyticity in Space
%
%
%%%%%%%%%%%%%%%%%%%%%%%%%%%%%%%%%%%%%%%%%%%%%%%%%%%%%
%
%
\section{Introduction to the Proof of Spatial Analyticity} 
\label{sec:space-anal}
We follow \cite{Gorsky:2005fk}, and differentiate
\eqref{eqn:nls}-\eqref{eqn:nls-data} $k$-times to obtain the system
%
%
\begin{gather}
  i \p_t u_k + \p^2_x u_k + B_{k}= 0, 
  \label{eqn:nls-system}
  \\
  u_{k}(x,0) = \vp_k(x), \quad k=0,1,2,\ldots
  \label{eqn:nls-system-init-data}
\end{gather}
%
%
where
\begin{gather*}
  u_k \doteq \p_x^k u, \quad \vp_k \doteq \p_x^k \vp, \quad \text{and}
  \quad B_k = B_{k}(u,u) =
  \lambda \sum_{j=0}^{k} \sum_{i=0}^{j} {k \choose j}{j \choose i} u_{k-j} u_{j-i}
  \bar{u_i}
  \label{eqn:system-notat}
\end{gather*}
Rewrite the system in integral form
%
%
\begin{equation}
\begin{split}
  u_k(x,t) = W(t) \vp_{k}(x) - \int_{0}^{t} W(t- \tau)B_{k}(x, \tau) d \tau
  \label{eqn:sys-integral-form}
\end{split}
\end{equation}
%
%
where $W(t) = e^{it \Delta}$. To localize in time, we introduce a cutoff
function $\psi(t) \in C^{\infty}_{0}(-1,1)$ with $0 \le \psi \le 1$ and
$\psi(t) \equiv 1$ for $| t | < 1/2$, and multiply \eqref{eqn:sys-integral-form}
by $\psi(t)$ to obtain
%
%
\begin{align}
\psi(t) u_k(x,t)
& = \psi(t) W(t) \vp_{k}(x) - \psi(t) \int_{0}^{t} W(t-
\tau)B_{k}(x, \tau) d \tau \notag
\\
\label{main-int-expression-1}
    & = \frac{1}{2 \pi} \psi(t) \sum_{n \in \zz} e^{i(xn + tn^{m 
    })} \widehat{\vp_k}(n) 
    \\
    & + \frac{1}{4 \pi^2} \psi(t) \sum_{n\in \zz} \int_\rr e^{ixn}  
    e^{it \tau} \frac{ 1 - \psi(\tau -  n^m) 
    }{\tau -  n^m} \wh{B_k}(n, \tau) \ d \tau
    \label{main-int-expression-2}
    \\
    \label{main-int-expression-3}
    & - \frac{1}{4 \pi^2} \psi(t) \sum_{n\in \zz} \int_\rr e^{i(xn + 
    t n^m)}
     \frac{1- \psi(\tau -  n^m)}{\tau -  n^m} \wh{B_k}(n, \tau) \ d \tau
    \\
    \label{main-int-expression-4}
    & + \frac{1}{4 \pi^2} \psi(t) \sum_{k \ge 1} \frac{i^k t^k}{k!}
    \sum_{n \in \zz} \int_\rr e^{i(xn + t n^m )}
    \psi(\tau -  n^m) (\tau -  n^m)^{k-1} \wh{B_k}(n, \tau)  
    \\
    & \doteq T_k(u_0, u_1,\cdots,u_k) \notag
\end{align}
%
where $T_k = T_{k,\vp_k}$.
%
The strategy for proving spatial analyticity will be to show that there exists a
map constructed from the $\{T_k\}$ which is a
contraction on an appropriate subspace of the following spaces.
\begin{definition}
	Denote $Y^s$ to be the space of all
	functions $u$ on $\ci \times \rr$ with
	bounded norm
\begin{equation}
	\label{Y-s-norm}
	\begin{split}
    \|u\|_{Y^s} = \|u\|_{X^s} + \|u\|_E
  \end{split}
\end{equation}
%
%
%
%
where
%
\begin{equation}
	\label{X^s-norm}
	\begin{split}
		& \|u\|_{X^s}
		= \left ( \sum_{n\in \zz} \left (1 + |n| \right )^{2s} \int_\rr \left ( 1 + | 
		\tau - n^{2} \right ) | \wh{u} ( n, \tau ) |^2
		\right )^{1/2},
    \\
    & \|u\|_E = \left( \sum_{n \in \zz} | n |^{2s} \left(
    \int_{\rr} | \wh{u}(n, \tau) |d \tau
    \right)^{2} \right)^{1/2}.
	\end{split}
\end{equation}
%
%
%
\end{definition}
The $Y^s$ spaces were first introduced by Colliander, Keel, Staffilani, Takaoka,
and Tao \cite{Colliander:2003kx}, and are a natural extension of the Bourgain spaces
\cite{Bourgain-Fourier-transfo-1}, \cite{Bourgain-Fourier-transfo}. Our main
motivation for introducing the $Y^s$ spaces is that, unlike the
Bourgain spaces, they possess the
following important property.
%
%
\begin{lemma}
	\label{lem:cutoff-loc-soln}
	Let $\psi(t)$ be a smooth cutoff function with $\psi(t) =1$ for $t \in [-T, T]$. If
	$\psi(t)u(x,t) \in Y^s$, then $u \in C([-T, T], H^s(\ci))$.
\end{lemma}
%
%
%
%
%%%%%%%%%%%%%%%%%%%%%%%%%%%%%%%%%%%%%%%%%%%%%%%%%%%%%
%
%
%                Spaces for Picard Iteration
%
%
%%%%%%%%%%%%%%%%%%%%%%%%%%%%%%%%%%%%%%%%%%%%%%%%%%%%%
%
%
\section{Introducing Appropriate Spaces for the Picard Iteration}
\label{sec:picard-spaces}
Note that $\vp \in
\mathcal{C}^{\omega}(\ci)$ if and only if
%
%
\begin{equation}
\begin{split}
  \|\vp \|_{H^s(\ci)} \le M_{0} \left( \frac{1}{2 C_0} \right)^{k} k!, \quad k
  \in \mathbb{N}_0
  \label{eqn:analy-condition}
\end{split}
\end{equation}
%
%
for some positive constants $M_0$ and $C_0$. Therefore, if we let
%
%
\begin{equation*}
\begin{split}
  \left\{ \vp_k \right\} \doteq \left( \vp_{0}, \vp_{1}, \vp_{2}, \cdots
  \right), \quad \vp_k = \p_x^k \vp,
\end{split}
\end{equation*}
%
%
and define the norm
%
%
\begin{equation*}
\begin{split}
  \|\left\{ \vp_k \right\} \|_{s} \doteq \sum_{k=0}^\infty \frac{C_0^k}{k!}
  \|\vp_k \|_{H^s(\ci)},
\end{split}
\end{equation*}
%
%
then by \eqref{eqn:analy-condition} we have
%
%
\begin{equation*}
\begin{split}
  \|\left\{ \vp_k \right\} \|_s < \infty.
\end{split}
\end{equation*}
%
%
Therefore, if we denote 
%
%
\begin{equation*}
\begin{split}
  | | |\left\{ u_k \right\} | | | \doteq \sum_{k=0}^\infty \frac{C_0^k}{k!}
  \|u_k \|_{H^s(\ci)}
\end{split}
\end{equation*}
%
%
then a natural space for expressing spatial analyticity is
%
%
\begin{equation*}
\begin{split}
  \mathcal{A}(Y^s) \doteq \left\{ (u_0, u_1, u_2, \cdots ) \doteq \left\{ u_k
  \right\}: u_j \in Y^s, j \in \mathbb{N}_0, \ \text{and}\ | | |\left\{ u_k
  \right\} | | | < \infty \right\}.
\end{split}
\end{equation*}
%
%
In the non-periodic case, such norms have been introduced by Kato and Ogawa
\cite{Kato:2000vn}. Our strategy is to show the existence of solutions $\left\{
u_k \right\}$ to \eqref{eqn:nls-system}-\eqref{eqn:nls-system-init-data}
satisfying
%
%
\begin{equation*}
\begin{split}
  | | |\left\{ u_k \right\} | | | < \infty.
\end{split}
\end{equation*}
%
%
This will imply that there exists $M_1 >0$ such that
%
%
\begin{equation*}
\begin{split}
  \frac{C_0^k}{k!} \|u_k\|_{Y^s} < M_1, \quad k \in \mathbb{N}_0
\end{split}
\end{equation*}
%
%
or
%
%
\begin{equation*}
\begin{split}
  \|u_k\|_{Y^s} \le M_1 \left( \frac{1}{C_0}^k  \right)^k, \quad k \in
  \mathbb{N}_0.
\end{split}
\end{equation*}
%
%
Then by \autoref{lem:cutoff-loc-soln} we will have that
%
%
\begin{equation*}
\begin{split}
  \|\p_x^k u(\cdot, t) \|_{H^s(\ci)} = \|u_k(\cdot, t) \|_{H^s(\ci)} \le \|u_k
  \|_{Y^s} \le c M_1 \left( \frac{1}{C_0} \right)^k k!, \quad k \in
  \mathbb{N}_0,
\end{split}
\end{equation*}
%
%
implying $u(\cdot, t) \in \mathcal{C}^\omega (\ci)$. 
%
%
%
%%%%%%%%%%%%%%%%%%%%%%%%%%%%%%%%%%%%%%%%%%%%%%%%%%%%%
%
%
%                Proof of Spatial Analyticity
%
%
%%%%%%%%%%%%%%%%%%%%%%%%%%%%%%%%%%%%%%%%%%%%%%%%%%%%%
%
%
\section{Proof of Spatial Analyticity} 
\label{sec:proof-spat-anal}
We proceed by estimating the $Y^s$ norms of
\eqref{main-int-expression-1}-\eqref{main-int-expression-4}. This will allow us
to show that $T_k$ is a contraction on $\mathcal{A}(Y^s)$. A Picard iteration
will then complete the proof. 
%
%
\subsection{Estimate for \eqref{main-int-expression-1}.}
Letting $f_k(x,t) = \psi(t) \sum_{n \in \zz} e^{i(xn + tn^{2})} 
\wh{\vp_k}(n)$, we have $\wh{f_k}(n,t) = \psi(t) \wh{\vp_k}(n) e^{itn^{2}}$,
from which we obtain
%
%
\begin{equation*}
  \begin{split}
    \wh{f_k}(n, \tau)
    & = \wh{\vp}(n) \int_\rr e^{-it( \tau - n^{2})} 
    \psi(t) \ d t
    = \wh{\psi}(\tau - n^{2}) \wh{\vp_k}(n).
  \end{split}
\end{equation*}
%
%
Since $\wh{\psi}(\xi)$ is Schwartz for $|\xi| \ge T$, we see that 
%
%
\begin{equation}
  \begin{split}
  \label{main-int1-est}
    \|\eqref{main-int-expression-1}\|_{Y^s}
    & = \left (  \sum_{n\in \zz} \left (1 + |n| \right )^s \int_\rr \left( 1 + | \tau - n^{2} 
    | \right )
    | \wh{\psi}(\tau - n^{2}) \wh{\vp_k}(n) |^2 d \tau \right)^{1/2} 
    \\
    & + \left[ \sum_{n \in \zz }\left( 1 + | n | \right)^{2s} \left( \int_{\rr} |
    \wh{\psi}(\tau - n^{2})\wh{\vp_k}(n) | d \tau
    \right)^{2} \right]^{1/2}
    \\
    & \le c_{\psi}
    \|\vp_k\|_{H^s(\ci)}.
  \end{split}
\end{equation}
%
%
%
%
\subsection{Estimate for \eqref{main-int-expression-2}.}
We now need the following lemma, whose proof is provided in the appendix.
%
%
%%%%%%%%%%%%%%%%%%%%%%%%%%%%%%%%%%%%%%%%%%%%%%%%%%%%%
%
%
%			Schwartz Multiplier	
%
%
%%%%%%%%%%%%%%%%%%%%%%%%%%%%%%%%%%%%%%%%%%%%%%%%%%%%%
%
%
\begin{lemma}
\label{lem:schwartz-mult}
  For $\psi \in S(\rr)$,
%
%
\begin{gather}
  \label{schwartz-mult-piece-1}
    \|\psi f \|_{X^s} \le c_{\psi} \|f \|_{X^s},
    \\
    \label{schwartz-mult-piece-2}
    \|\psi f \|_{E} \le c_{\psi} \|f \|_{E},
  \end{gather}
  and hence
  \begin{gather}
    \label{schwartz-mult}
\|\psi f \|_{Y^s} \le c_{\psi} \|f \|_{Y^s}.
\end{gather}
%
%
\end{lemma}
%
%
Hence,
%
%
\begin{equation}
  \label{main-int2-est-X-s-part}
  \begin{split}
    \|\eqref{main-int-expression-2}\|_{X^s} 
    & \lesssim 
    \left( \| \sum_{n \in \zz} e^{ixn} \int_\rr 
    e^{it \tau} \frac{ 1 - \psi (\tau - n^{2} ) 
    }{\tau - n^{2}} \wh{B_k}(n, \tau) \ 
    d \tau\|_{X^s} \right)^{1/2}
    \\
    & =  \left( \sum_{n \in \zz} \left (1 + |n| \right )^{2s} \int_\rr
    (1 + |\tau - n^{2}|) \left | \frac{1 - \psi(\tau - n^{2 
    })}{\tau - n^{2}} 
    \wh{B_k}(n, \tau) \right |^2 \ d 
    \tau \right)^{1/2}
    \\
    & \le \left( \sum_{n \in \zz} \left (1 + |n| \right )^{2s} \int_{| \tau - n^{2}| \ge 1}
    (1 + |\tau - n^{2}|) \frac{|\wh{B_k}(n, \tau)|^2 }{|\tau - n^{2}|^2} 
    \ d 
    \tau \right)^{1/2}
    \\
    & \lesssim  \left( \sum_{n \in 
    \zz} \left (1 + |n| \right )^{2s} \int_\rr
    \frac{|\wh{B_k}(n, \tau) |^2}{1+ |\tau - 
    n^{2}|} 
     \ d \tau 
    \right)^{1/2}
    \\
    & \lesssim 
    \sum_{j=0}^{k} \sum_{i=0}^{j} {k \choose j}{j \choose i}
    \|u_{k-j}\|_{X^s} \| u_{j-i}\|_{X^s}
    \| u_i \|_{X^s}
  \end{split}
\end{equation}
%
%
where the last two steps follow from the inequality 
%
\begin{equation}
  \label{one-plus-ineq}
  \begin{split}
    \frac{1}{|\tau - n^{2}| } \le \frac{2}{1 + |\tau - n^{2}| }, 
    \qquad |\tau - n^{2}| \ge 1
  \end{split}
\end{equation}
%
%
and the following trilinear estimate, whose proof we leave for later.
%
%
%%%%%%%%%%%%%%%%%%%%%%%%%%%%%%%%%%%%%%%%%%%%%%%%%%%%%
%
%
%				Proposition
%
%
%%%%%%%%%%%%%%%%%%%%%%%%%%%%%%%%%%%%%%%%%%%%%%%%%%%%%
%
%
\begin{proposition}
\label{prop:trilinear-est}
  %
  %
  For any $s \ge 0$ and $b \ge 3/4$, we have
  \begin{equation}
    \begin{split}
    & \left( \sum_{n \in \zz} \left (1 + |n| \right )^{2s} \int_\rr
    \frac{|\wh{{B_k}(f,g,h)}(n, \tau) |^2}{\left (1+ |\tau - 
    n^{2}| \right ) ^b} 
     \ d \tau 
    \right)^{1/2}
    \\
    & \lesssim \sum_{j=0}^{k} \sum_{i=0}^{j} {k \choose j}{j \choose i}
    \|f_{k-j}\|_{X^s} \| g_{j-i}\|_{X^s}
    \| h_i \|_{X^s}
  \end{split}
  \end{equation}
  where $$B_k(f,g,h)(x,t) = \sum_{j=0}^{k} \sum_{i=0}^{j} {k \choose j}{j \choose
  i} f_{k-j} g_{j-i} \bar{h_i}.$$
%
%
%
%
\end{proposition}
%
%
Furthermore,
%
%
%
%
\begin{equation}
  \label{main-int-expression-2-Y-s-part}
  \begin{split}
    \|\eqref{main-int-expression-2} \|_{E}
    & \lesssim \left( \| \sum_{n \in \zz} e^{ixn} \int_\rr 
    e^{it \tau} \frac{ 1 - \psi (\tau - n^{2} ) 
    }{\tau - n^{2}} \wh{{B_k}}(n, \tau) \ 
    d \tau\|_{E} \right)^{1/2}
    \\
    & = \left[ \sum_{n \in \zz}(1 + | n |)^{2s} \left(
    \int_{\rr}\frac{1 - \psi(\tau - n^{2})}{\tau - n^{2}} \wh{{B_k}}(n, \tau) d
    \tau \right)^{2} \right]^{1/2}
    \\
    & \lesssim \sum_{j=0}^{k} \sum_{i=0}^{j} {k \choose j}{j \choose i}
    \|u_{k-j}\|_{X^s} \| u_{j-i}\|_{X^s}
    \| u_i \|_{X^s}
  \end{split}
\end{equation}
%
%
where the last step follows from the following corollary to the preceding trilinear
estimate.
%
%
%%%%%%%%%%%%%%%%%%%%%%%%%%%%%%%%%%%%%%%%%%%%%%%%%%%%%
%
%
%				Second trilinear Estimate 
%
%
%%%%%%%%%%%%%%%%%%%%%%%%%%%%%%%%%%%%%%%%%%%%%%%%%%%%%
%
%
\begin{corollary}
\label{cor:trilinear-estimate2}
  For $s \ge 0$ we have
%
%
\begin{equation}
  \label{trilinear-estimate2}
  \begin{split}
    & \left( \sum_{n \in \zz} \left (1 + |n| \right )^{2s}  \left ( \int_\rr 
    \frac{|\wh{{B_k}(f,g,h)}(n, \tau) |}{1 + | \tau - n^{2} |}
     \ d\tau \right)^2  \right)^{1/2} 
     \\
     & \lesssim \sum_{j=0}^{k} \sum_{i=0}^{j} {k \choose j}{j \choose i}
    \|f_{k-j}\|_{X^s} \| g_{j-i}\|_{X^s}
    \| h_i \|_{X^s}.
  \end{split}
\end{equation}
\end{corollary}
%
%
Combining \eqref{main-int2-est-X-s-part} and
\eqref{main-int-expression-2-Y-s-part}, we conclude that
%
%
%
%
\begin{equation}
  \label{main-int2-est}
  \begin{split}
    \|\eqref{main-int-expression-2}\|_{Y^s} \le c_{\psi}
    \sum_{j=0}^{k} \sum_{i=0}^{j} {k \choose j}{j \choose i}
    \|u_{k-j}\|_{X^s} \| u_{j-i}\|_{X^s}
    \| u_i \|_{X^s}.
  \end{split}
\end{equation}
%
%
\subsection{Estimate for \eqref{main-int-expression-3}.}
Letting $$f_k(x,t) = \psi(t) \sum_{n \in \zz} e^{i\left( xn + tn^{2} \right)} 
\int_\rr \frac{1 - \psi\left( \lambda - n^{2} \right)}{\lambda - n^{2}} 
\wh{{B_k}} \left( n, \lambda \right) \ d \lambda,$$ we have
%
%
\begin{equation*}
  \begin{split}
    & \wh{f_k^x}(n, t) = \psi(t) e^{itn^{2}} \int_\rr
    \frac{1 - \psi\left( \lambda - n^{2} \right)}{\lambda - n^{2}} 
    \wh{{B_k}}(n, \lambda) \ d \lambda
  \end{split}
\end{equation*}
and
\begin{equation*}
  \begin{split}
     \wh{f_k}\left( n, \tau \right)
     & = \int_\rr e^{-it\left( \tau - n^{2} 
    \right)} \psi(t) \int_\rr \frac{1 - \psi\left( 
    \lambda - n^{2} 
    \right)}{\lambda - n^{2}} \wh{{B_k}}(n, \lambda) \ d \lambda d \tau
    \\
    & = \wh{\psi}\left( \tau - n^{2} \right) \int_\rr 
    \frac{1 - \psi\left( 
    \lambda - n^{2} 
    \right)}{\lambda - n^{2}} \wh{{B_k}}(n, \lambda) \ d \lambda.
  \end{split}
\end{equation*}
Therefore,
%
%
\begin{equation*}
  \begin{split}
    & \| \eqref{main-int-expression-3} \|_{X^s} 
    \\
    & = \left( \sum_{n \in \zz} \left (1 + |n| \right )^{2s} \int_\rr \left( 1 + | \tau - n^{m
    } \right ) | | \wh{\psi}\left( \tau - n^{2} \right) |^2 \ d \tau
    \right.
    \\
    & \times \left . |
    \int_\rr \frac{1 - \psi\left( \lambda - n^{2} \right)}{\lambda -
    n^{2}} \wh{{B_k}}(n, \lambda) \ d \lambda |^2  \right)^{1/2}
    \\
    & \lesssim \left( \sum_{n \in \zz} \left (1 + |n| \right )^{2s} | \int_\rr
    \frac{1 - \psi\left( \lambda - n^{2} \right)}{\lambda - n^{2}}
    \wh{{B_k}}(n, \lambda) \ d\lambda |^2 \right)^{1/2}
    \\
    & \le \left( \sum_{n \in \zz} \left (1 + |n| \right )^{2s}  \left ( \int_\rr
    \frac{1 - \psi\left( \lambda - n^{2} \right)}{|\lambda - n^{2}|}
    |\wh{{B_k}}(n, \lambda) | \ d\lambda \right )^2 \right)^{1/2}
    \\
    & \le \left( \sum_{n \in \zz} \left (1 + |n| \right )^{2s}  \left ( \int_{| \lambda - 
    n^{2} | \ge 1}
    \frac{|\wh{{B_k}}(n, \lambda) | }{|\lambda - n^{2}|}
    \ d\lambda \right )^2 \right)^{1/2}.
  \end{split}
\end{equation*}
%
%
Applying estimate \eqref{one-plus-ineq} then gives
%
%%
\begin{equation}
  \label{main-int3-est-X-s-part}
  \begin{split}
    \| \eqref{main-int-expression-3} \|_{X^s}
    & \lesssim \left( \sum_{n \in \zz} \left (1 + |n| \right )^{2s}  \left ( \int_\rr
    \frac{|\wh{{B_k}}(n, \lambda)| }{1 + |\lambda - n^{2}|}
     \ d\lambda \right )^2 \right)^{1/2}
     \\
    & \lesssim \sum_{j=0}^{k} \sum_{i=0}^{j} {k \choose j}{j \choose i}
    \|u_{k-j}\|_{X^s} \| u_{j-i}\|_{X^s}
    \| u_i \|_{X^s}
  \end{split}
\end{equation}
%
%%
where the last step follows from \autoref{cor:trilinear-estimate2}.
Furthermore, 
%
%
\begin{equation}
  \label{main-int-estimate-3-Y-s-part}
  \begin{split}
    \|\eqref{main-int-expression-3}\|_{E}
    & = \left[ \sum_{n \in \zz} (1 + | n |)^{2s} \int_{\rr} |
    \wh{\psi}(\tau - n^{2}) |^{2} \left( \int_{\rr}\frac{1 - \psi(\lambda -
    n^{2})}{\lambda - n^{2}} \wh{{B_k}}(n, \lambda) d \lambda \right)^{2} d \tau
    \right]^{1/2}
    \\
    & \le c_{\psi} \left[ \sum_{n \in \zz} (1 + | n |)^{2s} \left(
    \int_{\rr} \frac{1 - \psi(\lambda - n^{2})}{\lambda - n^{2}}
    \wh{{B_k}}(n, \lambda) d \lambda
    \right)^{2}\right]^{1/2}
    \\
    & \le 2 c_{\psi} \left[ \sum_{n \in \zz} (1 + | n |)^{2s} \left(
    \int_{\rr} \frac{\wh{{B_k}}(n, \lambda) }{1 + |\lambda - n^{2}|}
    d \lambda
    \right)^{2}\right]^{1/2}
    \\
    & \lesssim 
    \sum_{j=0}^{k} \sum_{i=0}^{j} {k \choose j}{j \choose i}
    \|u_{k-j}\|_{X^s} \| u_{j-i}\|_{X^s}
    \| u_i \|_{X^s}
  \end{split}
\end{equation}
%
%
where the last two steps follow from \eqref{one-plus-ineq} and
\autoref{cor:trilinear-estimate2}, respectively. Combining
\eqref{main-int3-est-X-s-part} and \eqref{main-int-estimate-3-Y-s-part}, we
conclude that
%
%
\begin{equation}
  \label{main-int3-est}
  \begin{split}
    \|\eqref{main-int-expression-3}\|_{Y^s} 
    \lesssim \sum_{j=0}^{k} \sum_{i=0}^{j} {k \choose j}{j \choose i}
    \|u_{k-j}\|_{X^s} \| u_{j-i}\|_{X^s}
    \| u_i \|_{X^s}.
  \end{split}
\end{equation}
%
%
%
\subsection{Estimate for \eqref{main-int-expression-4}.}
Note that
%
%
\begin{equation}
  \label{1n}
  \begin{split}
    \eqref{main-int-expression-4} \simeq \sum_{k \ge 1}
    \frac{i^k}{k!}g_k(x,t)
  \end{split}
\end{equation}
%
%
where 
%
%
\begin{equation*}
  \begin{split}
    & g_k(x,t) = t^k \psi(t) \sum_{n \in \zz} e^{i\left( xn + tn^{2}
    \right)} h_k(n),
    \\
    & h_k(n) = \int_\rr \psi \left( \tau - n^{2} \right) \cdot \left(
    \tau - n^{2} \right)^{k -1} \wh{{B_k}}(n, \tau) \ d \tau.
  \end{split}
\end{equation*}
%
%
Hence
%
%
\begin{equation*}
  \begin{split}
    \wh{g_k^x}(n, t) = t^{k} \psi(t) e^{i t n^{2}} h_k(n)
  \end{split}
\end{equation*}
%
%
which gives
%
%
\begin{equation*}
  \begin{split}
    \wh{g_k}(n, \tau)
    & = h_k(n) \int_\rr e^{-it\left( \tau - n^{2} \right)}
    t^{k}\psi(t) \ dt
    \\
    & = h_k(n) \wh{t^{k}\psi(t)} \left( \tau - n^{2} \right).
  \end{split}
\end{equation*}
%
%
Applying this to \eqref{1n}, we obtain
%
%
\begin{equation}
  \label{2n}
  \begin{split}
    \|\eqref{main-int-expression-4}\|_{X^s} 
    & \simeq \left( \sum_{n \in \zz} \left (1 + |n| \right )^{2s} \int_\rr \left( 1 + | \tau -
    n^{2}
    |
    \right) | \wh{\sum_{k \ge 1} \frac{i^k}{k!}g_k(x,t)} |^2 \ d \tau
    \right)^{1/2}
    \\
    & \le \sum_{k \ge 1} \frac{1}{k!}\left( \sum_{n \in \zz} \left (1 + |n| \right )^{2s}
    \int_\rr \left( 1 + | \tau - n^{2} | \right) | \wh{g_k}(n, \tau) |^2 \
    d \tau \right)^{1/2}
    \\
    & = \sum_{k \ge 1} \frac{1}{k!} \left( \sum_{n \in \zz} \left (1 + |n| \right )^{2s}
    \int_\rr \left( 1 + | \tau - n^{2} | \right) | h_k(n) \wh{t^k
    \psi(t)} \left( \tau - n^{2} \right) |^2 \ d \tau \right)^{1/2}
    \\
    & = \sum_{k \ge 1} \frac{1}{k!} \left( \sum_{n \in \zz} \left (1 + |n| \right )^{2s} |
    h_k(n) |^2 \int_\rr \left( 1 + | \tau - n^{2} | \right) | \wh{t^k
    \psi(t)} \left( \tau - n^{2} \right) |^2 \ d \tau \right)^{1/2}.
  \end{split}
\end{equation}
%
%
Notice that for fixed $n$, the change of variable $\tau - n^{2} \to \tau'$
gives
%
%
\begin{equation}
  \label{3n}
  \begin{split}
    \int_\rr \left( 1 + | \tau - n^{2} | \right) | \wh{t^{k}
    \psi(t)}\left( \tau - n^{2} \right) |^2 \ d \tau
    & = \int_\rr \left( 1 + |\tau'| \right) | \wh{t^k \psi(t)}(\tau') |^2 \
    d \tau'
    \\
    & \le \int_\rr \left( 1 + |\tau'| \right)^2 | \wh{t^k \psi(t)}(\tau')
    |^2 \ d \tau'
    \\
    & \lesssim \int_\rr \left( 1 + | \tau' |^2 \right) | \wh{t^{k}
    \psi(t)}(\tau') |^2 \ d \tau'
    \\
    & = \|t^k \psi(t) \|_{H^1(\rr)}^2.
  \end{split}
\end{equation}
%
%
But
%
%
\begin{equation}
  \label{4n}
  \begin{split}
    \|t^k \psi(t) \|_{H^1(\rr)}^2
    & = \left( \|t^k \psi(t)\|_{L^2(\rr)} + \|\p_t \left( t^k \psi(t)
    \right)\|_{L^2(\rr)} \right)^2
    \\
    & \lesssim \|t^{k}\psi(t) \|_{L^2(\rr)}^2 + \|\p_t \left (t^{k}
    \psi(t) \right )\|_{L^2(\rr)}^2
    \\
    & \le \|t^k \psi(t) \|_{L^2(\rr)}^2 + \|t^k \p_t \psi(t)
    \|_{L^2(\rr)}^2 + \|k t^{k -1} \psi(t) \|_{L^2(\rr)}^2
    \\
    & = c_{\psi} + c_{\psi}' + k^2 c_{\psi}''
    \\
    & \lesssim k^2.
  \end{split}
\end{equation}
%
%
Hence, applying \eqref{3n} and \eqref{4n} to \eqref{2n}, we obtain
%
%%
\begin{equation}
  \label{5n}
  \begin{split}
    \|\eqref{main-int-expression-4} \|_{X^s}
    & \lesssim
    \sum_{k \ge 1} \frac{k}{k!} \left( \sum_{n \in \zz} \left (1 + |n| \right )^{2s} | h_k(n) |^2 
    \right)^{1/2}
    \\
    & \le \sum_{k \ge 1} \frac{k}{k!}
    \cdot \sup_{k \ge 1} \left( \sum_{n \in \zz} \left (1 + |n| \right )^{2s} | 
    h_k(n) |^2 \right)^{1/2}
    \\
    & = \sum_{k \ge 1} \frac{k}{k!} \cdot \sup_{k \ge 1} 
    \left( \sum_{n \in \zz} \left (1 + |n| \right )^{2s} \int_\rr 
    \psi\left( \tau - n^{2} \right) \cdot \left( \tau - n^{2} 
    \right)^{k -1} \wh{{B_k}}(n, \tau) \ d \tau \right)^{1/2}.
  \end{split}
\end{equation}
%
%%
Recall that $\text{supp} \, |\psi| \subset [0, T ]$. Pick $T \le 1$. 
Then $| \psi\left( \tau - n^{2} \right) \cdot \left( \tau - n^{2} \right)^{k 
-1} | \le \chi_{| \tau - n^{2} | \le 1}$ for all $k \ge 1$. Hence, \eqref{5n} gives
%
%%
\begin{equation*}
  \begin{split}
    \|\eqref{main-int-expression-4} \|_{X^s} 
    & \lesssim \sum_{k \ge 1} \frac{k}{k!} \cdot \left( \sum_{n \in \zz} | 
    \int_{| \tau - n^{2}  |\le 1} | \wh{{B_k}}(n, \tau) \ d \tau |^2 
    \right)^{1/2}
  \end{split}
\end{equation*}
%
%%
which by the inequality
%
%%
\begin{equation*}
  \begin{split}
    \frac{1 + | \tau - n^{2} |}{1 + | \tau  - n^{2} |} \le 
    \frac{2}{1 + | \tau - n^{2} |}, \qquad | \tau - n^{2}  | \le 1
  \end{split}
\end{equation*}
%
%%
implies
%
%%
\begin{equation}
\label{main-int4-est-X-s-part}
  \begin{split}
    \|\eqref{main-int-expression-4}\|_{X^s}
    & \lesssim \left( \sum_{n \in \zz} | \int_{| \tau - n^{2}| \le 1 }
    \frac{\wh{{B_k}}(n, \tau)}{1 + | \tau - n^{2} |} \ d \tau |^2 
    \right)^{1/2}
    \\
    & \le \left( \sum_{n \in \zz} | \int_\rr
    \frac{\wh{{B_k}}(n, \tau)}{1 + | \tau - n^{2} |} \ d \tau |^2 
    \right)^{1/2} \\
    & \le \left( \sum_{n \in \zz} \left( \int_\rr 
    \frac{|\wh{{B_k}}(n, \tau)|}{1 + | \tau - n^{2} |}  \ d \tau  \right)^2
    \right)^{1/2} \\
    & \lesssim 
    \sum_{j=0}^{k} \sum_{i=0}^{j} {k \choose j}{j \choose i}
    \|u_{k-j}\|_{X^s} \| u_{j-i}\|_{X^s}
    \| u_i \|_{X^s}.
  \end{split}
\end{equation}
%
%%
where the last step follows from \autoref{cor:trilinear-estimate2}. Similarly,
we have
%
%
\begin{equation}
\label{main-int4-est-Y-s-part}
  \begin{split}
    \|\eqref{main-int-expression-4}\|_{E}
    & \simeq \left[ \sum_{n \in
    \zz}(1 + | n |)^{2s} \left( \int_{\rr} | \sum_{k \ge 1}
    \wh{\frac{i^{k}}{k!}g_{k}(x,t)(n, \tau)} |d \tau \right)^{2} \right]^{1/2}
    \\
    & \le \sum_{k \ge 1} \frac{1}{k!} \left[ \sum_{n \in \zz} (1 + | n
    |)^{2s} \left( \int_{\rr} | \wh{g}(n, \tau) | d \tau \right)^{2}
    \right]^{1/2}
    \\
    & = \sum_{k \ge 1} \frac{1}{k!} \left[ \sum_{n \in \zz} (1 + | n
    |)^{2s} | h_{k}(n) |^2 \left( \int_{\rr} | \wh{t^{k} \psi(t)}(\tau -
    n^{2}) |d \tau \right)^{2} \right]^{1/2}
    \\
    & = c_{\psi} \sum_{k \ge 1} \frac{1}{k!} \left[ \sum_{n \in \zz} (1 + | n
    |)^{2s} | h_{k}(n) |^2 \right]^{1/2}
    \\
    & \lesssim 
    \sum_{j=0}^{k} \sum_{i=0}^{j} {k \choose j}{j \choose i}
    \|u_{k-j}\|_{X^s} \| u_{j-i}\|_{X^s}
    \| u_i \|_{X^s}
  \end{split}
\end{equation}
%
%
where the last step follows from the computations starting from \eqref{5n}
through \eqref{main-int4-est-X-s-part}.
Combining \eqref{main-int4-est-X-s-part} and \eqref{main-int4-est-Y-s-part}, we
have
%
%
\begin{equation}
\label{main-int4-est}
  \begin{split}
    \|\eqref{main-int-expression-4}\|_{Y^s} \lesssim 
    \sum_{j=0}^{k} \sum_{i=0}^{j} {k \choose j}{j \choose i}
    \|u_{k-j}\|_{X^s} \| u_{j-i}\|_{X^s}
    \| u_i \|_{X^s}.
  \end{split}
\end{equation}
%
%
Collecting estimates \eqref{main-int1-est}, \eqref{main-int2-est}, 
\eqref{main-int3-est}, and \eqref{main-int4-est}, and recalling 
\eqref{main-int-expression-1}-\eqref{main-int-expression-4}, we see that
$$\|T_k(u_0, u_1, \cdots, u_k)\|_{Y^s} \le c_\psi \left( \|\vp_k \|_{H^s(\ci)} + 
\sum_{j=0}^{k} \sum_{i=0}^{j} {k \choose j}{j \choose i} \|u_{k-j}\|_{X^s} \|
u_{j-i}\|_{X^s} \| u_i \|_{X^s}  \right )$$ 
which by the inequality $\|u\|_{X^s} \le \|u\|_{Y^s}$ yields the following.
%%
%%%%%%%%%%%%%%%%%%%%%%%%%%%%%%%%%%%%%%%%%%%%%%%%%%%%%
%
%% Contraction Proposition
%				 
%%%%%%%%%%%%%%%%%%%%%%%%%%%%%%%%%%%%%%%%%%%%%%%%%%%%%%
%%
%%
%
\begin{proposition}
\label{prop:contraction}
  Let $s \ge0$. Then
%
%%
\begin{equation*}
  \begin{split}
    \|T_k(u_0, u_1, \cdots, u_k)\|_{Y^s} \le c_\psi  \left( \|\vp_k
    \|_{H^s(\ci)} +
    \sum_{j=0}^{k} \sum_{i=0}^{j} {k \choose j}{j \choose i}
    \|u_{k-j}\|_{Y^s} \| u_{j-i}\|_{Y^s}
    \| u_i \|_{Y^s}\right).
  \end{split}
\end{equation*}
%
%%
\end{proposition}
\subsection{Setting Up the Picard Iteration}
We will now use \autoref{prop:contraction} to prove local well-posedness for
analytic initial data for the NLS ivp. We first establish that the map 
%
%
\begin{equation*}
\begin{split}
  \left\{ u_k \right\}_{k=0}^{\infty} \mapsto T\left( \left\{ u_k
  \right\}_{k=0}^{\infty} \right) = \left( T_0(u_0), T_1(u_0, u_1), \cdots \right)
\end{split}
\end{equation*}
%
%
goes from $\mathcal{A}(Y^s)$ to $\mathcal{A}(Y^s)$, and that it is a contraction
in an appropriate ball.
%
Applying \autoref{prop:contraction}, we have the estimate
%
%
\begin{equation}
\begin{split}
  | | |T(\left\{ u_{k} \right\}) | | |
  & = \sum_{k=0}^{\infty} \frac{C_0^k}{k!} \|T_{k}(u_0, u_1, \cdots, u_k)
  \|_{Y^s}
  \\
  & \le c_{\psi} \sum_{k=0}^{\infty} \frac{C_0^k}{k!}\left[ \| \vp_{k}
  \|_{H^s(\ci)} + \sum_{j=0}^{k} \sum_{j=0}^{i} {k \choose j } {j \choose i }
  \|u_{k-j}\|_{Y^s} \| u_{j-1}\|_{Y^s} \|u_{i} \|_{Y^s} \| \right]
  \\
  & = c_{\psi}\left( | | | \left\{ \vp_{k} \right\} | | |_{s} +
  \sum_{k=0}^{\infty} \sum_{j=0}^{k}
  \sum_{j=0}^{i} {k \choose j } {j \choose i }
  \|u_{k-j}\|_{Y^s} \| u_{j-1}\|_{Y^s} \| u_{i} \|_{Y^s}\right).
\end{split}
\label{eqn:map-onto-ball}
\end{equation}
%
%
We now rewrite the second term
%
%
\begin{equation*}
\begin{split}
  & \sum_{k=0}^{\infty} \sum_{j=0}^{k}  \sum_{j=0}^{i} {k \choose j } {j \choose i } \|u_{k-j}\|_{Y^s}
\| u_{j-1}\|_{Y^s} \| u_{i} \|_{Y^s} 
  \\
  & = \sum_{k=0}^{\infty} \sum_{j=0}^{k}  \sum_{j=0}^{i} \frac{k!}{\cancel{j!}(k-j)!}
  \cdot \frac{\cancel{j!}}{i!(j-i)!} \| u_{k-j} \|_{Y^s} \| u_{j-i} \|_{Y^s} \|
  u_{i} \|_{Y^s}
  \\
  & = \sum_{k=0}^{\infty} \sum_{j=0}^{k}  \sum_{j=0}^{i}  k! \frac{\|
  u_{k-j} \|_{Y^s}}{(k-j)!} \frac{\| u_{j-i} \|_{Y^s}}{(j-1)!} \frac{\|
  u_{i} \|_{Y^s}}{i!}
\end{split}
\end{equation*}
%
%
which implies 
%
%
\begin{equation}
\begin{split}
  & \sum_{k=0}^{\infty} \frac{C_0^k}{k!} \sum_{j=0}^{k}  \sum_{j=0}^{i} {k \choose j } {j \choose i } \|u_{k-j}\|_{Y^s}
\| u_{j-1}\|_{Y^s} \| u_{i} \|_{Y^s} 
\\
& = \sum_{k=0}^{\infty} \sum_{j=0}^{k}  \sum_{j=0}^{i} C_{0}^{k} \frac{\|
u_{k-j} \|_{Y^s}}{(k-j)!} \frac{\| u_{j-i} \|_{Y^s}}{(j-i)} \frac{\|
u_{i} \|_{Y^s}}{i!}
\\
& = \sum_{k=0}^{\infty} \sum_{j=0}^{k}  \sum_{j=0}^{i}
\frac{C_{0}^{k-j}}{(k-j)!} \| u_{k-j} \|_{Y^s} \frac{C_{0}^{j-i}}{(j-i)!}\|
u_{j-i} \|_{Y^s} \frac{C_{0}^{i}}{i!} \| u_{i} \|_{Y^s}
\\
& \le \left( \sum_{k=0} \frac{C_0^k}{k!} \| u_{k} \|_{Y^s} \right)^{3}
= | | | \left\{ u_{k} \right\} | | |^3_{s}.
\label{eqn:low-dim-to-high-comp}
\end{split}
\end{equation}
%
%
Substituting into \eqref{eqn:map-onto-ball}, we obtain the relation
%
%
\begin{equation}
\begin{split}
  | | | T(\left\{ u_{k} \right\}) | | |
  \le c_{\psi} \left( | | | \left\{ \vp_{k} \right\} | | |_{s} +
  | | | \left\{ u_{k} \right\} | | |^3_{s} \right ).
\end{split}
\label{eqn:onto-relation}
\end{equation}
%
%
Let $c = c_{\psi}^{1/2}$. For given $\vp$, we may choose $\psi$ such
that 
%
%%
\begin{equation*}
  \begin{split}
    \|\{\vp\}\|_{s} \le \frac{15}{64c^3}.
  \end{split}
\end{equation*}
%
%%
Then if $| | |\{u_k\} | | | \le \frac{1}{4c}$, \eqref{eqn:onto-relation} gives
%
%%
\begin{equation*}
  \begin{split}
    |  | |T (\{u_k \}) | | |
    & \le c^2 \left[ \frac{15}{64c^3} + \left( 
    \frac{1}{4c} \right)^3 \right]
    =  \frac{1}{4c}.
  \end{split}
\end{equation*}
%
%%
Hence, $T=T_{\vp}$ maps the ball $B\left( 0, \frac{1}{4c} \right) \subset
\mathcal{A}(Y^s)$ into 
itself. Next, note that
%
%%
\begin{equation*}
  \begin{split}
    T_k(u_0, u_1,\cdots,u_k) - T_k(v_0, v_1, \cdots, v_k)
    = \eqref{main-int-expression-2} + \eqref{main-int-expression-3} 
    + \eqref{main-int-expression-4}
  \end{split}
\end{equation*}
%
%%
where now we replace $B_k(u,u,u)$ with $B_k(u,u,u) - B_k(v,v,v) = \p_x^k(u | u
|^2 - v | v |^{2})$. Rewriting
%
%%
\begin{equation*}
  \begin{split}
    u | u |^{2} - v | v |^{2}
    & = | u |^2 \left( u -v \right) + v\left( | u 
    |^2 - | v |^2
    \right)
    \\
    & = u \bar u \left( u -v \right) + v u \bar u - v v \bar v
    \\
    & = u \bar u \left( u - v \right) + v \bar u\left( u - v \right) + v 
    \bar u v - v v \bar v
    \\
    & = u \bar u \left( u -v \right) + v \bar u\left( u - v \right) + v v 
    \left( \overline{u -v} \right)
  \end{split}
\end{equation*}
%
%%
the triangle inequality and linearity of $\p_x^k$ and the Fourier transform then give
%
%%
\begin{equation*}
  \begin{split}
    | \mathcal{F}[{B_k}(u,u,u) - B_k(v,v,v)](n, \tau) |
    & \le | \mathcal{F}[\p_x^k[u \overline{u} \left (u -v \right )]] | +
    | \mathcal{F}[\p_x^k[v \overline{u} (u -v)]] | + |\mathcal{F}[\p_x^k[v v 
    (\overline{u-v})]|
    \\
    & \doteq | \wh{{B_k}(u-v, u, u} | + | \wh{{B_k}(u-v, u, v} | + |
    \wh{{B_k}(u-v,v,v)} |.
  \end{split}
\end{equation*}
%
%%
%
%%
Hence, $T_k(u_0, u_1,\cdots,u_k) - T_k(v_0, v_1, \cdots, v_k)
 = L_k(u, u-v,u) + L_k(v,u-v,u) + L_k(v, v, u-v)$, where
\begin{align}
  \label{main-int-exp-mod1}
  & \frac{1}{4 \pi^2} \psi(t) \sum_{n\in \zz} \int_\rr e^{ixn}  
    e^{it \tau} \frac{ 1 - \psi(\tau - n^{2}) 
    }{\tau - n^{2}} \wh{{B_k}(f,g,h)}(n, \tau) \ d \tau
    \\
    \label{main-int-exp-mod2}
    - & \frac{1}{4 \pi^2} \psi(t) \sum_{n\in \zz} \int_\rr e^{i(xn + 
    tn^{2})}
    \frac{1- \psi(\tau - n^{2})}{\tau - n^{2}} \wh{{B_k}(f,g,h)}(n, \tau) \ d \tau
    \\
    \label{main-int-exp-mod3}
    + & \frac{1}{4 \pi^2} \psi(t) \sum_{k \ge 1} \frac{i^k t^k}{k!}
    \sum_{n \in \zz} \int_\rr e^{i(xn + tn^{2} )}
    \psi(\tau - n^{2}) (\tau - n^{2})^{k-1} \wh{{B_k}(f,g,h)}(n, \tau)  
    \\
    \doteq & L_k(f,g,h). \notag
\end{align}
Repeating the arguments used to estimate 
\eqref{main-int-expression-2}-\eqref{main-int-expression-4}, we obtain
%
%%
\begin{equation*}
  \begin{split}
    & \|L_k(f, g, h)\|_{Y^s} \le c_\psi \sum_{j=0}^{k} \sum_{i=0}^{j} {k \choose
    j}{j \choose i} \|f_{k-j}\|_{Y^s} \| g_{j-i}\|_{Y^s} \| h_i \|_{Y^s}.
\end{split}
\end{equation*}
%
which implies
%
%%
\begin{equation}
  \label{20a}
  \begin{split}
    & | | |T_k(\left\{ u_k \right\}) - T_k(\left\{ v_k \right\}) | | |
    \\
    & = \sum_{k=0}^{\infty} \frac{C_0^k}{k!}\|T_k(u_0, u_1, \cdots, u_k) -
    T_k(v_0, v_1, \cdots, v_k) \|_{Y^s}
    \\
    & = \sum_{k=0}^{\infty} \frac{C_0^k}{k!}\left( \| L_k(u, u-v,u) +
    L_k(v,u-v,u) + L_k(v, v, u-v) \|_{Y^s} \right)
    \\
    & \le \sum_{k=0}^{\infty} \frac{C_0^k}{k!}\left( \| L_k(u, u-v,u) \|_{Y^s}
    + \|L_k(v,u-v,u)\|_{Y^s} +
    \|L_k(v, v, u-v) \|_{Y^s}\right)
    \\
    & \lesssim_{\psi} \sum_{k=0}^{\infty} \sum_{j=0}^{k}
    \sum_{i=0}^{j}\frac{C_{0}^{k}}{k!} {k \choose j } {j \choose i}
    \|u_{k-j} -v_{k-j} \|_{Y^s}
    \\
    & \times \left( \|u_{j-i}\|_{Y^s} \|u_{i}\|_{Y^s} +
    \|u_{j-i}\|_{Y^s} \| v_{i} \|_{Y^s} + \| v_{j-i} \|_{Y^s} \| v_i \|_{Y^s}  \right)
    %\\
    %& = c^2 \|u -v\|_{Y^s} \left( \|u\|_{Y^s} + \|v\|_{Y^s} \right)^2.
  \end{split}
\end{equation}
%
%%
Using a computation similar to \eqref{eqn:low-dim-to-high-comp}, we bound this
by
%
%
\begin{equation*}
\begin{split}
  & | | |\left\{ u_{k} \right\} - \left\{ v_{k} \right\}| | |_{s}\left( | | |\left\{
  u_{k} \right\} | | |_{s}^{2} + | | |\left\{ u_{k} \right\} | | |_{s} | |
  | \left\{ v_k \right\} | | |_{s} + | | |\{v_{k}\} | | |_{s}^{2}
  \right)
  \\
  & \le | | |\left\{ u_{k} \right\} - \left\{ v_{k} \right\}| | |_{s}
  \left( | | |\left\{
  u_{k} \right\} | | |_{s}+ | | |\{v_{k}\} | | |_{s} \right)^2.
\end{split}
\end{equation*}
%
%
Hence,  
%
%
\begin{equation*}
\begin{split}
  | | |T_k(\left\{ u_k \right\}) - T_k(\left\{ v_k \right\}) | | |
  \le c_{\psi} | | |\left\{ u_{k} \right\} - \left\{ v_{k} \right\}| | |_{s}
  \left( | | |\left\{
  u_{k} \right\} | | |_{s}+ | | |\{v_{k}\} | | |_{s} \right)^2.
\end{split}
\end{equation*}
%
%
Therefore, if $u, v \in B(0, \frac{1}{4c}) \subset \mathcal{A}(Y^s)$, it follows that
%
%%
\begin{equation}
  \label{21a}
  \begin{split}
    |  | |T\{u_k\} - T\{v_k\} |  | |
    & \le c^2 | | |u -v |  | |_{s} \left( \frac{1}{4c} + 
    \frac{1}{4c} \right)^2
    \\
    & = \frac{1}{4} \|u -v \|_{Y^s}. 
  \end{split}
\end{equation}
%
%%
We conclude that $T = T_{\vp}$ is a contraction on the ball $B(0, 
\frac{1}{4c}) \subset \mathcal{A}(Y^s)$. A Picard iteration, coupled with
\autoref{lem:cutoff-loc-soln} then yields a unique, local
solution in $\mathcal{A}(Y^s)$ to the NLS ivp
\eqref{eqn:nls}-\eqref{eqn:nls-data}. This completes the proof. \qquad
\qedsymbol
%
%
%
%
\section{Proof of \autoref{prop:trilinear-est}.}
%
%
By the triangle inequality, it is enough to show that for any $s \ge 0$ and $b
\ge 3/4$, we have
  \begin{equation}
    \label{trilin-est-simp}
    \begin{split}
    & \left( \sum_{n \in \zz} \left (1 + |n| \right )^{2s} \int_\rr
    \frac{|\wh{{w}_{fgh}}(n, \tau) |^2}{\left (1+ |\tau - 
    n^{2}| \right ) ^b} 
     \ d \tau 
    \right)^{1/2}
    \lesssim \|f\|_{X^s} \| g\|_{X^s}
    \| h \|_{X^s}
  \end{split}
  \end{equation}
  where $w_{fgh}(x,t) = f g \bar{h}$.
%
%
%
%
Note first that $|\wh{{w}_{fgh}}(n, \tau) |  = | \wh{f} * ( \wh{g} 
* \wh{\bar h})(n, \tau)|$ and $| \wh{\bar{h}}(n, \tau) | = |\overline{ \wh{\overline{h}} 
}(n, \tau)| = | \wh{h}(-n, -\tau) |$. It follows that
%
%
\begin{equation}
  \label{non-lin-rep}
  \begin{split}
    | \wh{{w}_{fgh}}(n, \tau)|
    & = | \sum_{n_1 + n_2 + n_3 = n}  \int_{\tau_1 + \tau_2 + \tau_3 = \tau} \wh{f}\left( n_1,  \tau_1 
\right) \wh{g}\left( n_2, \tau_2  
\right) \wh{\bar h}\left( n_3, \tau_3 \right) d \tau_1 d \tau_2 d \tau_3 |
\\
& \le \sum_{n_1 + n_2 + n_3 = n}  \int_{\tau_1 + \tau_2 + \tau_3 = \tau} | \wh{f}\left( n_1, \tau_1 
\right) | \times  | \wh{g}\left( n_2, \tau_2 
\right) | \times | \wh{\bar h}\left( n_3, \tau_3 \right) | d \tau_1 d \tau_2 d 
\tau_3
\\
& \le \sum_{n_1 + n_2 + n_3 = n}  \int_{\tau_1 + \tau_2 + \tau_3 = \tau} | \wh{f}\left( n_1, \tau_1 
\right) | \times | \wh{g}\left( n_2, \tau_2 
\right) | \times | \wh{h}\left( -n_3, - \tau_3 \right) | d \tau_1 d \tau_2 d 
\tau_3
\\
& = \sum_{n_1 + n_2 + n_3 = n} \int_{\tau_1 + \tau_2 + \tau_3 = \tau} \frac{c_f\left( n_1, \tau_1 
\right)}{\left (1 + |n_1| \right )^s \left( 1 + | \tau_1 - n_1^2 | \right)^{b/2}}
\\
& \times \frac{c_{g}\left( n_2, \tau_2 \right)}{\left (1 + |n_2| \right ) 
^s\left( 1 + | \tau_2 -  n_2^2| 
\right)^{b/2}}
 \times \frac{c_{h}\left( -n_3, -\tau_3 \right)}{\left (1 + |n_3| \right ) ^s\left( 1 + | 
\tau_3 + n_3^2 | \right)^{b/2}} \ d \tau_1 d \tau_2 d \tau_3
\end{split}
\end{equation}
%
%
where 
%
%
\begin{equation*}
  \begin{split}
    c_\sigma(n, \tau) = \left (1 + |n| \right ) ^s \left( 1 + | \tau - n^{2} |  
    \right)^{b/2} | \wh{\sigma}\left( n, \tau \right) | .
  \end{split}
\end{equation*}
%
%
Hence
%
%
\begin{equation}
  \label{convo-est-starting-pnt}
  \begin{split}
     & \left (1 + |n| \right )^s \left( 1 + | \tau - n^{2} | \right)^{-b/2} | \wh{{w}_{fgh}}\left( 
    n, \tau \right) |
    \\
    & \le \left( 1 + | \tau - n^{2} | \right)^{-b/2}
    \sum_{n_1 + n_2 + n_3 = n} \int_{\tau_1 + \tau_2 + \tau_3 = \tau} \frac{\left (1 + |n| \right )^s}{\left (1 +
    |n_1| \right )^s \left (1 + | n_2| \right )^s \left (1 + |n_3| \right )^s} 
    \\
    & \times \frac{c_f(n_1, \tau_1)}{\left( 1 + | \tau_1 - n_1^2 | 
    \right)^{b/2}}
    \times
    \frac{c_g(n_2, \tau_2)}{\left( 1 + | \tau_2 - n_2^2 | 
    \right)^{b/2}} \times
    \frac{c_h(-n_3, -\tau_3)}{\left( 1 + | \tau_3 + n_3^2 | 
    \right)^{b/2}}\ d \tau_1 d \tau_2 d \tau_3.
  \end{split}
\end{equation}
%
%
For $s \ge 0$, observe that
%
%
\begin{equation}
  \label{deriv-bound-easy-s}
  \begin{split}
    \frac{\left (1 + |n| \right ) ^s}{\left (1 + |n_1| \right ) ^s \left (1 + |n_2| \right ) ^s \left (1 + |n_3| \right ) ^s} 
    \le 3^{s}
  \end{split}
\end{equation}
%
%
by the following lemma, whose proof is provided in the appendix.
%
%
\begin{lemma}
\label{lem:splitting}
  For $v \ge 0$ and $a, b, c \in \zz$, we have
%
%
\begin{equation}
  \label{splitting}
  \begin{split}
    \left ( 1 + |a +b + c| \right)^v \le 3^v \left(1 + | a | \right)^v \left(
    1 + | b | \right)^v \left( 1 + | c | \right)^v.
  \end{split}
\end{equation}
%
%
\end{lemma}
%
%
Hence, from \eqref{convo-est-starting-pnt} and \eqref{deriv-bound-easy-s}, we 
obtain
%
\begin{equation*}
  \begin{split}
    \left (1 + |n| \right )^s \left( 1 +  | \tau - n^{2}  | \right)^{-b/2} | 
    \wh{{w}_{fgh}}\left( n, \tau \right) | 
    & \lesssim \sum_{n_1 + n_2 + n_3 = n} \int_{\tau_1 + \tau_2 + \tau_3 = \tau} \frac{1}{\left( 1 +
    | \tau - n^{2}| 
    \right)^{b/2}}  
    \\
    & \times
    \sum_{n_1 + n_2 + n_3 = n} \int_{\tau_1 + \tau_2 + \tau_3 = \tau} \frac{c_f\left( n_1, \tau_1 
    \right)}{\left (1 + |n_1| \right )^s \left( 1 + | \tau_1 - n_1^2 |
    \right)^{b/2}}
    \\
    & \times \frac{c_{g}\left( n_2, \tau_2 \right)}{\left (1 + |n_2| \right ) 
    ^s\left( 1 + | \tau_2 -  n_2^2| 
    \right)^{b/2}}
    \\
    & \times \frac{c_{h}\left( -n_3, -\tau_3 \right)}{\left (1 + |n_3| \right ) ^s\left( 1 + | 
    \tau_3 + n_3^2 | \right)^{b/2}} \ d \tau_1 d \tau_2 d \tau_3
    \\
    & = \left( 1 + | \tau - n^{2} | \right)^{-b/2}
    \wh{C_f C_{g} C^+_{h}} \left( n, \tau \right)
  \end{split}
\end{equation*}
%
%
where
%
%
\begin{equation*}
  \begin{split}
    C_\sigma(x, t) = \left[ \frac{c_\sigma\left( n, \tau \right)}{\left( 
    1 + | \tau - n^{2} | \right)^{b/2}} \right]^\vee,
    \ \ C^+_\sigma(x, t) = \left[ \frac{c_\sigma\left( -n, -\tau \right)}{\left( 
    1 + | \tau + n^{2} | \right)^{b/2}} \right]^\vee.
  \end{split}
\end{equation*}
%
%
Therefore
%
%
\begin{equation}
  \label{gen-holder-pre-estimate}
  \begin{split}
    & \| \left( 1 + |n | \right)^s
    \left( 1 + | \tau - n^{2} | \right)^{-b/2} \wh{{w}_{fgh}}(n, 
    \tau)		
    \|_{L^2(\zz \times \rr)}
    \\
    & \lesssim \| \left( 1 + | \tau - n^{2} | \right)^{-b/2}
    \wh{C_f C_{g} C^+_{h}} \|_{L^2(\zz \times \rr)}.
  \end{split}
\end{equation}
%
We now require the following multiplier estimate, whose proof can be found in 
\cite{Himonas-Misiolek-2001-A-priori-estimates-for-Schrodinger}.
%
%
%%%%%%%%%%%%%%%%%%%%%%%%%%%%%%%%%%%%%%%%%%%%%%%%%%%%%
%
%
%			Four Mult Est	
%
%
%%%%%%%%%%%%%%%%%%%%%%%%%%%%%%%%%%%%%%%%%%%%%%%%%%%%%
%
%
%
%
%
%
%
%
\begin{lemma}
  \label{lem:four-mult-est-L4}
  Let $(x, t) \in \ci \times \rr $ and $(n, \tau) \in \zz \times \rr$ be 
  the dual variables. Let $v$ be a positive even integer. Then there is a 
  constant $c_v > 0$ such that
%
%
\begin{equation}
  \label{four-mult-est-L4*}
  \begin{split}
    \| \left( 1 + | \tau - n^v | 
    \right)^{-\frac{v+1}{4v}}
    \wh{f}\|_{L^2(\zz \times \rr)} \le c_v \|f \|_{L^{4/3}( \ci \times \rr)}.
  \end{split}
\end{equation}
%
%
\end{lemma}
%
%
Applying \autoref{cor:four-mult-est-L4} and generalized H\"{o}lder to the 
right-hand-side of \eqref{gen-holder-pre-estimate} gives
%
%
\begin{equation}
  \label{gen-holder-piece-1}
  \begin{split}
    \|\left( 1 + | \tau - n^{2} | \right)^{-b/2} \wh{C_f C_{ 
    g } C^+_{h}}\|_{L^2(\zz \times \rr)}
    & \lesssim  \|C_f C_{g} C^+_{h} \|_{L^{4/3}(\ci \times \rr)}
    \\
    & \le \|C_f \|_{L^4(\ci \times \rr)} \|C_{g}\|_{L^4(\ci \times \rr)} 
    \|C^+_{h}\|_{L^4(\ci \times \rr)}.
  \end{split}
\end{equation}
%
%
Note that a change of variable gives
%
%
\begin{equation*}
  \begin{split}
    C_\sigma^+(x, t)
    & = \sum_{n \in \zz} \int_\rr e^{i(nx +  \tau t)} \frac{c_\sigma\left( -n, -\tau \right)}{\left( 
    1 + | \tau + n^{2} | \right)^{b/2}} \ d \tau
    \\
    & = - \sum_{n \in \zz} \int_\rr e^{-i(nx +   \tau t )}
    \frac{c_\sigma\left( n, \tau \right)}{\left( 
    1 + | \tau - n^{2} | \right)^{b/2}} \ d \tau
  \end{split}
\end{equation*}
%
%
and so
%
%
\begin{equation*}
  \begin{split}
    C_\sigma^+(-x, -t) = -C_\sigma(x, t).
  \end{split}
\end{equation*}
%
%
We will now the need the following dual estimate of
\autoref{lem:four-mult-est-L4}.
%
\begin{corollary}
  \label{cor:four-mult-est-L4}
  Let $(x, t) \in \ci \times \rr $ and $(n, \tau) \in \zz \times \rr$ be 
  the dual variables. Let $v$ be a positive even integer. Then there is a 
  constant $c_v > 0$ such that
%
%
\begin{equation}
  \label{four-mult-est-L4}
  \begin{split}
    \|f\|_{L^4(\ci \times \rr)} \le c_v \|\left( 1 + | \tau - n^v | 
    \right)^\frac{v+1}{4v} \wh{f} \|_{L^2( \zz \times \rr)}
  \end{split}
\end{equation}
for every test function $f(x, t)$. 
%
%
%
%
\end{corollary}
%
%
Recalling that $L^4(\ci \times \rr)$ is invariant under the transformation $(x, 
t) \mapsto (-x,-t)$ and applying 
\autoref{cor:four-mult-est-L4}, we obtain
%
%
\begin{equation}
  \label{C-sig-estimate}
  \begin{split}
    \| C^+_\sigma \|_{L^4(\ci \times \rr)} = \|C_\sigma \|_{L^4(\ci \times \rr)} 
    & \lesssim \|\left( 1 + | \tau - n^{2} | 
    \right)^{(m +1)/4m} \left( 1 + | \tau - n^{2} | 
    \right)^{-b/2} c_\sigma \|_{L^2(\zz \times \rr)}
    \\
    & = \|\left( 1 + | \tau - n^{2} | 
    \right)^{[m(1 - 2b) + 1]/4m } c_\sigma \|_{L^2(\zz \times \rr)}
    \\
    & \le \|c_\sigma \|_{L^2(\zz \times \rr)}  \qquad (\text{since  } [m(1 - 2b) + 
    1]/4m \le 0 )
    \\
    & = \|\sigma\|_{X^s}.
  \end{split}
\end{equation}
%
%
We conclude from \eqref{gen-holder-pre-estimate}, \eqref{gen-holder-piece-1}, 
and \eqref{C-sig-estimate} that
%
%
%
%
\begin{equation*}
  \begin{split}
    \| \left( 1 + |n | \right)^s \left( 1 + | \tau - n^{2} | \right)^{-b/2} \wh{{w}_{fgh}} 
    (n, \tau) \|_{L^2(\zz \times \rr)} \lesssim 
    \|f\|_{X^s}\|g\|_{X^s}\|h\|_{X^s}. \qquad \qed
  \end{split}
\end{equation*}
%
%
%
%
%
%
%
\section{Proof of \autoref{cor:trilinear-estimate2}.}
By the triangle inequality, it is enough to show that
for $s \ge 0$ we have
%
%
\begin{equation}
  \begin{split}
    & \left( \sum_{n \in \zz} \left (1 + |n| \right )^{2s}  \left ( \int_\rr 
    \frac{|\wh{{w}_{fgh}}(n, \tau) |}{1 + | \tau - n^{2} |}
     \ d\tau \right)^2  \right)^{1/2} 
     \lesssim \|f\|_{X^s} \| g\|_{X^s}
    \| h \|_{X^s}.
  \end{split}
\end{equation}
%
Hence, by duality, it suffices to show that 
%
%%
\begin{equation*}
  \begin{split}
    \sum_{n \in \zz} \left (1 + |n| \right )^{s}
    a_n \int_{\rr} \frac{|\wh{{w}_{fgh}}(n, \tau)|}{1 
    + | \tau - n^{2} |} \ d \tau \lesssim \|f\|_{X^s} \|g\|_{X^s} \|h\|_{X^s}
    \|a_n \|_{\ell^2}
  \end{split}
\end{equation*}
%
%%
for $\{a_n\} \in \ell^2$. By the triangle inequality 
and Cauchy-Schwartz,
%
%%
\begin{equation}
  \label{1m}
  \begin{split}
    & | \sum_{n \in \zz} \left (1 + |n| \right )^{s} a_n
    \int_{\rr}\frac{| \wh{{w}_{fgh}}(n, \tau) |}{1 + | \tau - n^{2} |} \ d \tau |
    \\
    & \le \sum_{n \in \zz} \int_{\rr} \frac{| a_n |}{\left( 1 + 
    | \tau - n^{2} |
    \right)^{1/2 + \eta}} \cdot \frac{\left( 1 + | n| \right)^s  |
    \wh{{w}_{fgh}}(n, \tau) |}{\left( 
    1 + | \tau - n^{2} | \right)^{1/2 - \eta}} \ d \tau
    \\
    & \le \left( \sum_{n \in \zz} | a_{n} |^2\int_{\rr} \frac{1}{\left( 1 + | \tau - n^{2} | \right)^{1 + 2 \eta}} \ d \tau  
    \right)^{1/2} 
    \left ( \sum_{n \in \zz}\int_{\rr} \frac{\left (1 + |n| \right )^{2s} | \wh{{w}_{fgh}}(n, \tau) 
    |^2}{\left( 1 + | \tau - n^{2} | \right)^{1 - 2 \eta}}\ d \tau 
    \right)^{1/2}
  \end{split}
\end{equation}
%
%%
Restrict $\eta \in (0, 1/8)$. Applying the change of variable $\tau - n^{2}
\mapsto \tau'$ we obtain  %
%%

\begin{equation*}
  \begin{split}
    & \left( \sum_{n \in \zz} | a_{n} |^2\int_{\rr} \frac{1}{\left( 1 + | \tau -
    n^{2} | \right)^{1 + 2 \eta}} \ d \tau  
    \right)^{1/2} 
    \\
    & = \left ( \sum_{n \in \zz}
    | a_n |^2 
    \int_{\rr} \frac{1}{\left( 1 + | \tau' | \right)^{1 + 2 \eta}} \ d 
    \tau \right)^{1/2}
    \\
    & \simeq \|a_n\|_{\ell^2}
    \end{split}
\end{equation*}
while \eqref{trilin-est-simp} gives the bound
\begin{equation*}
  \begin{split}
    \left ( \sum_{n \in \zz}\int_{\rr} \frac{\left (1 + |n| \right )^{2s} | \wh{{w}_{fgh}}(n, \tau) 
    |^2}{\left( 1 + | \tau - n^{2} | \right)^{1 - 2 \eta}}\ d \tau 
    \right)^{1/2} \lesssim \|f\|_{X^s} \|g\|_{X^s} \|h\|_{X^s}
  \end{split}
\end{equation*}
%
%%
completing the proof.
\qquad \qedsymbol
%
%
\appendix
\section{}
\subsection{Proof of \autoref{lem:cutoff-loc-soln}}
%
%
\begin{equation*}
  \begin{split}
    \lim_{t_{n} \to t} \|u(\cdot, t) - u(\cdot, t_{n})\|_{H^s(\ci)} 
    & = \lim_{t_{n} \to t} \|\psi(t) u(\cdot, t) - \psi(t_n) u(\cdot, t_{n})\|_{H^s(\ci)} 
    \\
    & = \lim_{t_n \to t} \left[ \sum_{n \in \zz}\left( 1 + | n |
    \right)^{2s} | \psi(t)  \wh{u}(n, t) - \psi(t_n) \wh{ u}(n, t_n) |^2 \right]^{1/2}
    \\
    & = \lim_{t_n \to t} \left[ \sum_{n \in \zz} \left( 1 + | n |
    \right)^{2s} | \int_{\rr} (e^{it \tau} - e^{it_{n} \tau}) \wh{\psi u}(n,
    \tau) d \tau |^2 \right]^{1/2}.
  \end{split}
\end{equation*}
    It is clear that
    %
    %
    \begin{equation*}
      \begin{split}
        \left( 1 + | n |
        \right)^{2s} | \int_{\rr} (e^{it \tau} - e^{it_{n}\tau}) \wh{\psi u}(n, \tau) d \tau |^2 
    & \le 4  \left( 1 + | n |
    \right)^{2s} \left ( \int_{\rr} |\wh{\psi u}(n, \tau)| d \tau
    \right )^2 
  \end{split}
\end{equation*}
and 
%
%
\begin{equation*}
  \begin{split}
 \sum_{n \in \zz} \left( 1 + | n |
    \right)^{2s} \left ( \int_{\rr} |\wh{\psi u}(n, \tau)| d \tau
    \right ) ^2 
    \le \|\psi u \|_{Y^s}^2 
  \end{split}
\end{equation*}
which is bounded by assumption.
Applying dominated convergence completes the proof. \qquad \qedsymbol
%
%
\subsection{Proof of \autoref{lem:schwartz-mult}}
Note that
%
%
\begin{equation*}
	\begin{split}
		\wh{\psi f}\left( n, \tau \right)
		& = \wh{\psi}(\cdot) * \wh{f}(n,
		\cdot)(\tau)
		= \int_\rr \wh{\psi}(\tau_1) \wh{f} \left( n, \tau - \tau_1 \right) 
		d\tau_1
	\end{split}
\end{equation*}
%
%
and hence
%
%
\begin{equation}
	\label{1b}
	\begin{split}
		\|\psi f\|_{X^s} 
		& = \left( \sum_{n \in \zz} \left (1 + |n| \right )^{2s} \int_\rr \left( 1 + | \tau -
		n^{m} | \right) | \int_\rr \wh{\psi}(\tau_1) \wh{f}\left( n, \tau -
		\tau_1
		\right)  d \tau_1 d \tau |^2 \right)^{1/2}
		\\
		& \le \left( \sum_{n \in \zz} \left (1 + |n| \right )^{2s} \int_\rr \left( 1 + | \tau -
		n^{m }
		|
		\right) \left( \int_\rr \wh{\psi}\left( \tau_1 \right) \wh{f}\left( n,
		\tau - \tau_1
		\right)  d \tau_1 d \tau \right)^2 \right)^{1/2}.
	\end{split}
\end{equation}
%
%
Using the relation
%
%
\begin{equation*}
	\begin{split}
		1 + | \tau - n^{m } |
		& = 1 + | \tau + \tau_1 - n^{m} |
		\\
		& \le 1 + | \tau_1 | + | \tau - \tau_1 - n^{m} |
		\\
		& \le \left( 1 + | \tau_1 | \right)\left( 1 + | \tau - \tau_1 -
		n^{m} | \right),
	\end{split}
\end{equation*}
%
%
we obtain
%
%
\begin{equation*}
	\begin{split}
		\eqref{1b}
		& \le \left( \sum_{n \in \zz} \left (1 + |n| \right )^{2s} \right.
		\\
		& \times \left . \int_\rr \left(
		\int_\rr \left( 1 + | \tau_1 | \right)^{1/2} | \wh{\psi}(\tau_1) |
		\left( 1 + | \tau - \tau_1 - n^{m} | \right)^{1/2} \wh{f}\left( n, \tau
		- \tau_1
		\right)d \tau_1
		\right)^2 d \tau \right)^{1/2}
	\end{split}
\end{equation*}
%
%
which by Minkowski's inequality is bounded by
%
%
\begin{equation}
	\label{2b}
	\begin{split}
		& \left( \sum_{n \in \zz} \left (1 + |n| \right )^{2s}  \right.
		\\
		& \times \left. \left( \int_\rr \left[ \int_\rr
		\left( 1 + | \tau_{1} | \right) | \wh{\psi}(\tau_1) |^2 \left( 1 + |
		\tau - \tau_1 - n^{m} |
		\right) | \wh{f}\left( n, \tau - \tau_1 \right) |^2 d \tau_1 
		\right]^{1/2} d \tau \right)^2 \right)^{1/2}.
	\end{split}
\end{equation}
%
%
Using the change of variable $\tau - \tau_1 \to \lambda$ gives
%
%
\begin{equation*}
	\begin{split}
		\eqref{2b}
		& = \left( \sum_{n \in \zz} \left (1 + |n| \right )^{2s}\right.
		\\
		& \times \left.  \left( \int_\rr \left[
		\int_\rr \left( 1 + | \tau_1 | \right) | \wh{\psi}\left( \tau_1
		\right) |^2 \left( 1 + | \lambda - n^{m} | \right) | \wh{f} \left( n,
		\lambda
		\right)|^2 d \tau_1 \right]^{1/2} d \lambda \right)^2 \right)^{1/2}
		\\
		& =  \left( \sum_{n \in \zz} \left (1 + |n| \right )^{2s} \right.
		\\
		& \times \left. \left( \int_\rr \left( 1 + |
		\tau_1 |
		\right)^{1/2} | \wh{\psi}(\tau_1) | d \tau_1 \left[ \int_\rr \left( 1 + |
		\lambda - n^{m} |
		\right) | \wh{f}\left( n, \lambda \right) |^2 d \lambda \right]^{1/2}
		\right)^2 \right)^{1/2}
		\\
		& = c_{\psi} \left( \sum_{n \in \zz} \left (1 + |n| \right )^{2s} \left( \left[ \int_\rr
		\left( 1 + | \lambda - n^{m} | \right) | \wh{f}\left( n, \lambda
		\right) |^2 d \lambda
		\right]^{\cancel{1/2}} \right)^{\cancel{2}} \right)^{1/2}
		\\
		& = c_{\psi} \|f\|_{X^s},
	\end{split}
\end{equation*}
%
%
which proves \eqref{schwartz-mult-piece-1}.
Estimating 
%
%
\begin{equation*}
\begin{split}
  \| \psi f \|_{E}^{2}
  & = \sum_{n \in \zz} | n |^{2s} \left( \int_{\rr} |
  \wh{\psi f}(n, \tau)
  | d \tau \right)^{2}
  \\
  & = \sum_{n \in \zz} | n |^{2s} \left( \int_{\rr} \wh{\psi} * \wh{f}
  (n, \tau) d \tau \right)^{2}
  \\
  & \le \| \wh{\psi} \|_{L^1}^{2} \sum_{n \in \zz} | n |^{2s} \left(
  \int_{\rr} \wh{f}(n, \tau) d \tau
  \right)^{2} \quad \text{(Young's Inequality)}
  \\
  & = c_{\psi} \| f \|_{E}^2.
\end{split}
\end{equation*}
%
%
and taking square roots of both sides gives \eqref{schwartz-mult-piece-2}. Combining
\eqref{schwartz-mult-piece-1} and \eqref{schwartz-mult-piece-2} we obtain
\eqref{schwartz-mult}, completing the proof.  \qquad \qedsymbol
%
%
\subsection{Proof of \autoref{lem:splitting}.} We have
%
%
\begin{equation}
	\label{6a}
	\begin{split}
		1 + | a + b + c| 
		& \le 1 + | a | + | b | + | c |
		\\
		& \le 1 + | a | + 1 + | b | + 1 + | c |
		\\
		& \le 3\left( \max\{1+| a |, 1+| b |, 1+ | c | \}\right)
		\\
		& \le 3 \left( 1 + | a | \right)\left( 1 + | b | \right) \left( 1 + |
		c |
		\right), \qquad a, b, c \in \zz.
	\end{split}
\end{equation}
%
%
Raising both sides of expression $\eqref{6a}$ to the $v$ power completes 
the proof. \qquad \qedsymbol 

  




%\nocite{*}
%\bibliography{/Users/davidkarapetyan/math/bib-files/references.bib}
% \bib, bibdiv, biblist are defined by the amsrefs package.
\begin{bibdiv}
\begin{biblist}

\bib{Bourgain-Fourier-transfo-1}{article}{
      author={Bourgain, J.},
       title={Fourier transform restriction phenomena for certain lattice
  subsets and applications to nonlinear evolution equations. {I}.
  {S}chr{\"o}dinger equations},
        date={1993},
        ISSN={1016-443X},
     journal={Geom. Funct. Anal.},
      volume={3},
      number={2},
       pages={107\ndash 156},
         url={http://dx.doi.org/10.1007/BF01896020},
      review={\MR{MR1209299 (95d:35160a)}},
}

\bib{Bourgain-Fourier-transfo}{article}{
      author={Bourgain, J.},
       title={Fourier transform restriction phenomena for certain lattice
  subsets and applications to nonlinear evolution equations. {II}. {T}he
  {K}d{V}-equation},
        date={1993},
        ISSN={1016-443X},
     journal={Geom. Funct. Anal.},
      volume={3},
      number={3},
       pages={209\ndash 262},
         url={http://dx.doi.org/10.1007/BF01895688},
      review={\MR{MR1215780 (95d:35160b)}},
}

\bib{Colliander:2003kx}{article}{
      author={Colliander, J.},
      author={Keel, M.},
      author={Staffilani, G.},
      author={Takaoka, H.},
      author={Tao, T.},
       title={Sharp global well-posedness for {K}d{V} and modified {K}d{V} on
  {$\Bbb R$} and {$\Bbb T$}},
        date={2003},
        ISSN={0894-0347},
     journal={J. Amer. Math. Soc.},
      volume={16},
      number={3},
       pages={705\ndash 749 (electronic)},
         url={http://dx.doi.org/10.1090/S0894-0347-03-00421-1},
      review={\MR{1969209 (2004c:35352)}},
}

\bib{Gorsky:2005fk}{incollection}{
      author={Gorsky, Jennifer},
      author={Alexandrou~Himonas, A.},
       title={On analyticity in space variable of solutions to the {K}d{V}
  equation},
        date={2005},
   booktitle={Geometric analysis of {PDE} and several complex variables},
      series={Contemp. Math.},
      volume={368},
   publisher={Amer. Math. Soc.},
     address={Providence, RI},
       pages={233\ndash 247},
      review={\MR{2126473 (2007g:35210)}},
}

\bib{Himonas-Misiolek-2001-A-priori-estimates-for-Schrodinger}{article}{
      author={Himonas, A.~Alexandrou},
      author={Misiolek, Gerard},
       title={A priori estimates for {S}chr{\"o}dinger type multipliers},
        date={2001},
        ISSN={0019-2082},
     journal={Illinois J. Math.},
      volume={45},
      number={2},
       pages={631\ndash 640},
      review={\MR{MR1878623 (2002j:42018)}},
}

\bib{Kato:2000vn}{article}{
      author={Kato, Keiichi},
      author={Ogawa, Takayoshi},
       title={Analyticity and smoothing effect for the {K}orteweg de {V}ries
  equation with a single point singularity},
        date={2000},
        ISSN={0025-5831},
     journal={Math. Ann.},
      volume={316},
      number={3},
       pages={577\ndash 608},
         url={http://dx.doi.org/10.1007/s002080050345},
      review={\MR{1752786 (2001c:35217)}},
}

\end{biblist}
\end{bibdiv}
\end{document}

\part{Weakly Dispersive Equations}
\chapter{Non-Uniform Dependence and Well-posedness for 
HR}
\section{Introduction}
%
We consider the  initial value problem for
the hyperelastic rod (HR)  equation
%
%
\begin{gather}
\label{hr}
\p_t u
-
\p_t \p_x^2 u
+
3u\p_x u
=
\gamma \big (
2\p_x u \p_x^2 u
+
u \p_x^3 u
\big ),
\\
\label{hr-data} u(x, 0) = u_0 (x),
\quad x \in \ci, \ \text{or} \ \rr \quad t \in \rr,
\end{gather}
%
%
where  $\gamma$  is a  nonzero constant,
and prove that the dependence of solutions on initial data is not uniformly 
continuous in Sobolev spaces $H^s(\ci)$, $s>3/2$.
Thus, we extend the result proved by Olson 
\cite{Olson_2006_Non-uniform-dep} in the periodic
case (for $s\ge 2$ and $\gamma \ne 3$)  to  $s>3/2$ (the entire 
well-posedness range) for HR\@. Furthermore,  motivated by the work of
Himonas  and Kenig \cite{Himonas:2009fk},
we establish non-uniform dependence
in the non-periodic case, where the method of traveling wave solutions used in  
\cite{Olson_2006_Non-uniform-dep} does not seem to work.
%
%

The HR equation was first
derived by Dai in \cite{Dai_1998_Model-equations} as a one-dimensional 
model for finite-length and
small-amplitude axial deformation waves in thin cylindrical
rods composed of a compressible Mooney-Rivlin
material. The derivation relied upon a reductive perturbation technique, 
and took into account the nonlinear dispersion of pulses propagating 
along a rod. It was assumed that each cross-section of the rod is 
subject to a stretching and rotation in space. The solution $u(x,t)$ to the 
HR equation represents the radial stretch relative
to a pre-stressed state, while $\gamma$ is a fixed constant depending upon 
the pre-stress and the material used in
the rod, with values ranging from $- 29.4760$ to $3.4174$.

%
The well-posedness of the HR equation has been studied by several authors. 
In Yin \cite{Yin_2003_On-the-Cauchy-p} and Zhou 
\cite{Zhou_2005_Local-well-pose}, a proof of local well-posedness in Sobolev 
spaces $H^s$,  $s > 3/2$, is described  on the line and the circle, respectively. 
Their approach is to rewrite the HR equation   
in its non-local form, and then to verify the conditions needed to apply 
Kato's semi-group theory \cite{Kato:1975}. 
For details on how this is done for CH on the line, see Rodriguez-Blanco 
\cite{Rodriguez-Blanco_2001_On-the-Cauchy-p}. Blow-up criteria 
is also investigated in \cite{Yin_2003_On-the-Cauchy-p} and 
\cite{Zhou_2005_Local-well-pose}, as well as by Constantin and Strauss 
\cite{Constantin_2000_Stability-of-a-}. 


Setting $\gamma = 0$ gives the celebrated 
BBM equation, which was proposed by 
Benjamin, Bona, and Mahony 
\cite{Benjamin_1972_Model-equations} as a model for 
the unidirectional evolution of long waves.
Solitary-wave solutions to this 
equation are global and orbitally stable (see Benjamin 
\cite{Benjamin_1972_The-stability-o}, 
\cite{Benjamin_1972_Model-equations}, and 
\cite{Constantin_2000_Stability-of-a-}).
For more general $\gamma$, the existence of global 
solutions to HR on the line with constant $H^1$ energy
was proved recently by Mustafa \cite{Mustafa:2007}
using the approach developed by Bressan and 
Constantin in \cite{Bressan_2007_Global-conserva}. Using a vanishing 
viscosity argument, Coclite, 
Holden, and Karlsen \cite{Coclite_2005_Global-weak-sol}
established existence of a strongly continuous semigroup of global 
weak solutions of HR on the line for initial data in $H^1$.
Bendahmane, Coclite, and Karlsen 
\cite{Bendahmane:2006p1301} extended this result to traveling 
wave solutions that are supersonic solitary shockwaves.
For more information on the existence of global solutions to the HR
equation, see Holden and Raynaud \cite{Holden_2007_Global-conserva}
and \cite{Yin_2003_On-the-Cauchy-p}. 

There is a variety of traveling wave solutions to the HR equation that can be 
obtained using various combinations of peaks, cusps, compactons, 
fractal-like waves, and plateaus (see Lenells 
\cite{Lenells_2006_Traveling-waves}). Orbital stability of solitary wave 
solutions was proved in \cite{Constantin_2000_Stability-of-a-}.
Solitary shock wave formation was 
analyzed in Dai and Huo \cite{Dai_2000_Solitary-shock-} using traveling 
wave solutions of the HR equation to derive a system of ordinary differential 
equations, with a vertical singular line in the phase plane corresponding with the 
formation of shock waves. Head-on collisions between two solitary 
waves was investigated in the work of Hui-Hui Dai, 
Shiqiang Dai, and Huo \cite{Dai_2000_Head-on-collisi} using a reductive 
perturbation method coupled with the technique of strained coordinates. 

In this work we study the continuity of the data-to-solution map for the HR 
equation.
Using the method of traveling wave solutions it was shown in  
\cite{Olson_2006_Non-uniform-dep} that the data-to-solution map
$u_0  \mapsto u$ of the periodic HR equation is not uniformly continuous 
from any bounded set in $H^s(\ci)$ into $C([0, T ], H^s(\ci))$ for $s \ge 
2$ and $\gamma \neq 3$. Non-uniform dependence for the non-periodic CH 
equation in 
$H^s(\rr)$ for $s>3/2$ was proven in \cite{Himonas:2009fk} 
using the method of approximate solutions and well-posedness estimates. The 
case $s=1$ for both the line and the circle
was proved earlier by Himonas, Misiolek, and Ponce in 
\cite{Himonas_2007_Non-uniform-con}.
Recently in \cite{Himonas:2010} non-uniform 
continuity of the solution map for the CH equation
on the circle has been proved
for the whole range of Sobolev exponents for which local well-posedness of 
CH is known.

We mention that the continuity of the data-to-solution map  for CH has 
been studied in  \cite{Himonas_2007_Non-uniform-con},
\cite{Himonas_2001_The-Cauchy-prob}, and 
\cite{Himonas:2005}, and for the Euler equations in 
\cite{Himonas_2009_Non-uniform-dep-euler}. Continuity of this map for  
the  Benjamin-Ono equation was studied in  Koch and  Tzvetkov 
\cite{Koch:2005}. For related ill-posedness results, we 
refer the reader to Kenig, Ponce, and  Vega 
\cite{Kenig_2001_On-the-ill-pose}, Christ, Colliander, and Tao 
\cite{Christ_2003_Asymptotics-fre}, and the references 
therein.

Here we consider the initial value problem for the HR equation
in both the periodic and non-periodic cases
and prove non-uniform  continuity of the solution map. 
More precisely, we show the following result:
%
%%%%%%%%%%%%%%%%%%%%%%%%
%
%
%    Theorem:  thm:hr-non-unif-dependence
%
%
%%%%%%%%%%%%%%%%%%%%%
%
\begin{theorem}
\label{thm:hr-non-unif-dependence}
Let $\gamma$ be a nonzero constant. Then 
the data-to-solution map $u(0) \mapsto u(t)$ of the Cauchy-problem
for the HR equation
\eqref{hr}-\eqref{hr-data}
is not uniformly continuous
from any bounded subset of  $H^s$ into $C([-T, T], H^s)$
for $s>3/2$ on the line and circle.
%
\end{theorem}
%
%
%
As we mentioned above, when  $\gamma=0$ the HR equation
becomes the BBM equation.
Bona and Tzvetkov \cite{Bona_2009_Sharp-well-pose} have recently proved  that this equation  
is globally well-posed in  Sobolev spaces $H^s$, if $s \ge 0$,
and that its data-to-solution map is smooth.
%
%
%
Our approach  for proving \cref{thm:hr-non-unif-dependence}  
mirrors  that in Himonas and Kenig \cite{Himonas:2009fk} and 
Himonas, Kenig, and Misiolek \cite{Himonas:2010}.
That is, we will choose 
approximate solutions to the HR equation such that the size of the difference between approximate and actual solutions with 
identical initial data is negligible. Hence, to understand the degree of 
dependence, it will suffice to focus on the behavior of the approximate 
solutions (which will be simple in form), rather than on the behavior of the 
actual solutions. In order for the method to go through, we will 
need well-posedness estimates for  the size of the 
actual solutions to the HR equation, as well a 
lower bound for their lifespan. This will permit us to obtain an upper 
bound for the size of the difference of approximate and actual solutions. 
More precisely, we will need the following well-posedness result  with estimates,  
stated in both the  periodic and non-periodic case:


%%%%%%%%%%%%%%%%%%%%%%%%
%
%            wp of theorem in R and T
%
%%%%%%%%%%%%%%%%%%%%%%%%
%
%
%
%
\begin{theorem}
\label{thm:HR_existence_continuous_dependence}
If   $s>3/2$  then we have:

(i) If $u_0\in H^s$  then  there exists a unique solution to
the Cauchy problem  \eqref{hr}--\eqref{hr-data} in $C([-T, T], H^s)$, where 
the lifespan  $T$ depends on the size
of the initial data $u_0$. Moreover, 
the  lifespan $T$ satisfies the lower bound estimate 
%
%
%
\begin{equation}
\label{Life-span-est}
T
\ge
\frac{1}{2c_s \|u_0\|_{H^s}}.
\end{equation}
%

(ii)
The flow map $u_0 \mapsto u(t)$ is continuous from
bounded sets of $H^s$ into \\ $C([-T, T], H^s)$,
and the solution $u$ satisfies the estimate
%
%
%
\begin{equation}
\label{u_x-Linfty-Hs}
\|
u(t)
\|_ {H^s}
\le
2
\|
u_0
\|_{H^s}, \ \ |t|\le T.
\end{equation}
%
%
%
\end{theorem}
%
%
A proof of existence, uniqueness, and continuous dependence in this 
theorem for $\gamma =1$ (CH) 
is given  by Li and Olver in 
\cite{Li_2000_Well-posedness-} using a regularization method, and in
\cite{Rodriguez-Blanco_2001_On-the-Cauchy-p} using 
Kato's semi-group method \cite{Kato:1975}. As mentioned 
above, proofs of 
existence, uniqueness, and continuous dependence for  HR
have been outlined in \cite{Yin_2003_On-the-Cauchy-p} and 
\cite{Zhou_2005_Local-well-pose} for the line and circle, 
respectively. Both outlines rely upon an application of Kato's semi-group 
method. However, we have not been able to find estimates  
\eqref{Life-span-est} and \eqref{u_x-Linfty-Hs}  in the literature.
Here we shall give a proof of local well-posedness of HR,
including  estimates \eqref{Life-span-est} and \eqref{u_x-Linfty-Hs},
which are key ingredients in our work, 
following an alternative approach used for nonlinear hyperbolic equations
in Taylor \cite{Taylor:1991}.
\\
\\
For the Burgers equation, it is also known that for $s > 3/2$, dependence is not
better than continuous. Furthermore, Kato \cite{Kato:1975}
showed that for $s > 3/2$ the data to solution map $u_{0} \mapsto u(t)$ is not
H\"older continuous from a closed ball in $H^{s}(\rr)$ centered at $0$ and measured
in the $H^{r}(\rr)$ norm, $r < s$, to $C([0, T], H^{r}(\rr))$, where $T$ depends upon
the $H^{s}(\rr)$ radius of the ball. More precisely, for fixed $0 < \gamma < 1$
and fixed constant $c > 0$,
there exist solutions $u, v$ of Burgers with bounded initial data in $H^{s}(\rr)$
(and hence, a common lifespan $T$) and $0 < t_{0} < T$ such that
%
%
\begin{equation*}
\begin{split}
\| u(t_{0}) - v(t_{0}) \|_{H^{r}(\rr)} 
& > c \| u_{0} - v_{0} \|^{\gamma}_{H^{r}(\rr)}.
\end{split}
\end{equation*}
However, for certain general quasi-linear hyperbolic systems, Kato also obtained
uniform continuity of the data to solution map for initial data in Sobolev
spaces with integer index, measured in a weaker Sobolev norm. More recently, Tao
\cite{Tao:2004} obtained Lipschitz continuity of the data to solution map
for the Benjamin-Ono equation for $H^{1}(\rr)$ initial data measured in $L^{2}(\rr)$.
Herr, Ionescu, Kenig, and Koch \cite{Herr:2010p886} have also obtained Lipschitz
continuity in a weaker topology for the Benjamin-Ono with generalized
dispersion. Hence, it is reasonable to ask whether a result similar to these
holds for HR\@. Our main motivation stems from the work of Chen, Liu, and Zhang
\cite{Chen:2011fk} on the
b-family
%
%
\begin{gather}
\p_t u =  -u \p_x u -
\p_{x} (1 - \p_{x}^{2})^{-1} \left[ \frac{b}{2}u^2 +
\frac{3-b}{2} \left( \p_x u \right)^2
\right],
\label{b-family}
\\
u(x,0) = u_0(x), \ x \in \ci \ \text{or} \ \rr, \ t \in \rr
\label{init-cond-iu-b-fam}
\end{gather}
%
%
for which they proved H\"older continuity of the data to solution map from a closed ball $B(0, h)$ in $H^{s}(\rr)$, $s >
3/2$ (measured in the $H^{r}(\rr)$ topology, $r <s$) to $C([0, T], H^{r}(\rr))$, with $T
= T(h)> 0$ and H\"older index $\alpha = \alpha(b, s, r)$ given by 
%
%
\begin{equation*}
\begin{split}
\alpha = 
\begin{cases}
1, \quad & (s, r) \in \Omega_{1}
\\
1, \quad & b=3 \ \ \text{and} \ \ (s, r) \in \Omega_{2}
\\
2(s-1)/(s-r), \quad  & b\neq 3 \ \ \text{and} \ \ (s, r) \in \Omega_{2}
\\
s-r, \quad & (s, r) \in \Omega_{3}
\end{cases}
\end{split}
\end{equation*}
%
%
where
\begin{equation*}
\begin{split}
\Omega_{1} & = \left\{ (s, r): \   s > 3/2, \ 0 \le r \le s-1, \  r \ge 2-s \right\}
\\
\Omega_{2} & = \left\{(s, r): \  3/2 < s < 2, \ 0 < r < 2-s   \right\}
\\
\Omega_{3} & = \left\{ (s,r): \  s > 3/2, \ s-1 < r < s  \right\}.
\end{split}
\end{equation*}
Given this result, and the similarities between the $b$-family and HR (both can
be thought of as weakly dispersive nonlocal perturbations of Burgers), in this
work we study the continuity properties of the
data-to-solution map for the HR equation, expanding upon \cref{thm:hr-non-unif-dependence}. More precisely, following \cite{Chen:2011fk} we show
the following result:
%
%
\begin{theorem}
For $\gamma \neq 0$, the
data to solution map for HR is H\"older continuous on both the line and circle from $B_{H^{s}}(R)$ (in
the topology of $H^{r}$) to $C([0, T], H^{r})$, where $T = T(R)$, for $s >
3/2$, $-1 \le r < s$. More
precisely, consider the following sets 
%
%
\begin{equation*}
\begin{split}
& \Omega_{1} = \left\{ (s, \ r) \in \rr^{2}:
\ s>3/2, \ -1 \le r \le s-1, \  r \ge 2 -s  \right\}
\\
& \Omega_{2} = \left\{ (s, \ r) \in \rr^{2}:
\ 3/2 < s < 3, \ -1 \le r < 2-s \right\}
\\
& \Omega_{3} = \left\{ (s, \ r) \in \rr^{2}:
\  s>3/2, \  s-1 < r < s  \right\}.
\end{split}
\end{equation*}
%
%
\label{thm:main-thm}
\end{theorem}
%
\begin{center}
\begin{tikzpicture}[scale=1.5]
% Draw thin grid lines with color 40% gray + 60% white
% Draw x and y axis lines
\draw [->] (0,0) -- (3,0) node [below] {$s$};
\draw [->] (0,-1) -- (0,3) node [left] {$r$};
\draw [->, dashed] (0,0) -- (3,3);
\draw [->, dashed] (0,-1) -- (3,2);
\draw [->, dashed] (0,2) -- (3,-1);
\draw [->, dashed] (0,-1) -- (3,-1);
\draw [->, dashed] (3/2,-1) -- (3/2, 3);
\fill[color=green, fill opacity=0.3] (1.5, 0.5) -- (3,2) -- (3,0) -- (3,-1);
\fill[color=red, fill opacity=0.3] (1.5, 0.5) -- (1.5,1.5) -- (3,3) -- (3,2);
\fill[color=blue, fill opacity=0.3] (1.5, 0.5) -- (1.5, -1) -- (3, -1);
\foreach \x/\xtext in {1, 2}
\draw[shift={(\x,0)}]  node[below] {$\xtext$};
\foreach \y/\ytext in {-1, 1, 2}
\draw[shift={(0,\y)}]  node[left] {$\ytext$};
\draw (2,1.5) node {$\Omega_{3}$};
\draw (2,0.5) node {$\Omega_{1}$};
\draw (2,-0.5) node {$\Omega_{2}$};
\end{tikzpicture}
\end{center}
%
%
Then for two initial data $u_{0}, v_{0} \in B_{H^{s}}(R)$, there exist unique
corresponding solutions \\ $u(x,t), v(x,t)$ for $0 \le t \le T= T(R)$ to the
HR equation \eqref{hyperelastic-rod-equation} which satisfy 
%
%
\begin{equation*}
\begin{split}
\| u(t) - v(t) \|_{H^{r}} \le C \| u_{0} - v_{0} \|_{H^{r}}^{\alpha(s, r)},
\quad 0
\le t \le T
\end{split}
\end{equation*}
%
%
where 
%
%
\begin{equation*}
\begin{split}
\alpha = 
\begin{cases}
1, \quad & (s,r) \in \Omega_{1} 
\\
2(s-1)/(s-r),  \quad & (s, r) \in \Omega_{2}
\\
s-r, \quad & (s, r) \in \Omega_{3}.
\end{cases}
\end{split}
\end{equation*}
%
%
%%%%%%%%%%%%%%%%%%%%%%%%%%%%%%%%%%%%%%%%%%%%%%%%%%%%%
%
%
%                Main theorem
%
%
%%%%%%%%%%%%%%%%%%%%%%%%%%%%%%%%%%%%%%%%%%%%%%%%%%%%%
%
%
We remark that this result is sharper then the analogue obtained in
\cite{Chen:2011fk} for the $b$-family. We are confident that the techniques
applied here can be applied to sharpen the results obtained in
\cite{Chen:2011fk}.

The document is structured as follows. We first prove 
\cref{thm:hr-non-unif-dependence} on the line and 
then on the circle.
As mentioned above, we begin with two sequences of
appropriate approximate solutions and then 
we construct  actual solutions
coinciding at time zero  with the approximate solutions.
The key step is to show that  the $H^s$-size of
the difference between approximate and actual solutions 
converges to zero (see \cref{applelem:bound_for_difference-of-approx-and-actual-soln}
and \cref{prop:bound_for_difference-of-approx-actual-soln}). 
We then prove \cref{thm:HR_existence_continuous_dependence} 
using a Galerkin-type argument and energy estimates. Using the tools (energy estimates, commutator estimate, and multiplier estimate) used to prove \cref{thm:HR_existence_continuous_dependence}, we will then prove \cref{thm:main-thm}. 
%
%
%	
%
%
%
%
%%%%%%%%%%%%%%%%%%%%%%%%
%
%           Proof of  \cref on the line
%
%%%%%%%%%%%%%%%%%%%%%%%%
%
%
%
%
%
\section{Proof of Non-Uniform Dependence on the Line}
\label{sec:2}
%
%
%
%
We begin by outlining the method of the proof,
as it has been applied for the case $\gamma=1$ in \cite{Himonas:2009fk}.
We will show  that there there exist two sequences of solutions 
$u_n(t)$
and $v_n(t)$ in $C([-T, T], H^s)$ such that
%
%
%
%
\begin{equation}
\label{h-s-bdd}
\| u_n(t)  \|_{H^s}
+
\| v_n(t)  \|_{H^s}
\lesssim
1,
\end{equation}
%
%
%
%
%
\begin{equation}
\label{zero-limit-at-0}
\lim_{n\to\infty}
\|
u_n(0)
-
v_n(0)
\|_{H^s}
=
0,
\end{equation}
%
%
%
%
and
%
%
%
%
\begin{equation}
\label{bdd-away-from-0}
\liminf_{n\to\infty}
\|
u_n(t)
-
v_n(t)
\|_{H^s}
\gtrsim
|\sin ( \gamma t)|,
\quad
| \gamma t|\le 1.
\end{equation}%
%
%
We accomplish this in two steps.
First, we will construct two sequences of approximate solutions
satisfying the above properties.
Then, we will construct two sequences of actual solutions 
coinciding with the approximate solutions at time zero.
The key point of this method is that 
the difference between solutions and approximate solutions
decays rapidly.

%
%
For this method, it is more convenient 
to rewrite the Cauchy problem for the HR equation 
in the following non-local form
%
%
\begin{align}
& \p_t u =  -\gamma u \p_x u -
\Lambda^{-1} \left[ \frac{3-\gamma}{2}u^2 +
\frac{\gamma}{2} \left( \p_x u \right)^2
\right],
\label{apple1'}
\\
&  u(x,0) = u_0(x), \; \; x \in \rr, \; \; t \in \rr
\label{apple2'}
\end{align}
%
%
where 
\begin{equation*}
\Lambda^{-1} = \p_x (1 - \p_x^2)^{-1}.
\end{equation*}
%
%
%
%
%
%

\subsection{Approximate solutions}
Following \cite{Himonas:2009fk}, our approximate solutions
\\ $u^{\omega, \lambda} = u^{\omega,
\lambda}(x,t)$ to \eqref{apple1'}-\eqref{apple2'} will
consist of a low frequency and a high frequency part,
i.e.
%
%
%
%
\begin{equation}
\label{apple1}
u^{\omega,\lambda} = u_\ell + u^h
\end{equation}
%
%
%
%
where $\omega$ is in a bounded set of $\rr$ and $\lambda > 0$. The high frequency part is given by 
%
%
%
%
\begin{equation}
\begin{split}
u^h = u^{h,\omega,\lambda}(x,t) =
\lambda^{-\frac{\delta}{2} -s}
\phi \left (\frac{x}{\lambda^\delta}\right )
\cos(\lambda x - \gamma \omega t)
\end{split}
\end{equation}
%
%
%
%
where $\phi$ is a $C^\infty$ cut-off function such that
%
%
%
%
\begin{equation*}
\phi = \begin{cases}
1, &\text{if $|x|<1$,} \\
0, &\text{if $|x| \ge 2,$} \end{cases}
\end{equation*}
%
%
%
%
and by \cref{thm:HR_existence_continuous_dependence} 
we let the low frequency part $u_\ell = u_{l,
\omega, \lambda}(x,t)$ be the unique solution to the Cauchy problem
%
%
\begin{align}
\label{u-l-apple1'}
& \p_t u_\ell = -\gamma u_\ell \p_x u_\ell -
\Lambda^{-1} \left[ \frac{3-\gamma}{2}(u_\ell)^2 +
\frac{\gamma}{2} \left( \p_x u_\ell \right)^2
\right],
\\
& u_\ell(x,0) = \omega \lambda^{-1} \tilde{\phi} \left(
\frac{x}{\lambda^{\delta}}
\right), \quad x \in \rr, \quad t \in \rr
\label{apple1ah}
\end{align}
%
%
%
%
where $\tilde{\phi}$ is a $C^{\infty}_0(\rr)$ function such that
%
%
%
%
\begin{equation}
\label{apple1ah7}
\tilde{\phi}(x) = 1 \; \;  \text{if} \; \;
x \in \text{supp} \; \phi.
\end{equation}
%
%
%
%
We remark that for $\lambda >>1$ and $\delta < 2$ the approximate solutions 
$u^{\omega, \lambda}$ share a common lifespan $T >> 1$. To see why, we 
first note that the high frequency part $u^{h, \omega, \lambda}$ has 
infinite lifespan by the following, whose 
proof can be found in \cite{Himonas:2009fk}: 
%
%
\begin{lemma}
\label{applea}
Let $\psi \in S(\rr)$, $\alpha \in \rr$. Then for $s \ge 0$ we have
%
%
\begin{equation}
\begin{split}
\lim_{\lambda \to \infty} \lambda^{-\frac{\delta}{2}-s}
\|\psi \left( \frac{x}{\lambda^\delta} \right)\cos(\lambda
x - \alpha) \|_{H^s(\rr)} = \frac{1}{\sqrt
2}\|\psi\|_{L^2(\rr)}.
\label{apple6}
\end{split}
\end{equation}
%
%
Relation \eqref{apple6} remains true if $\cos$ is
replaced by $\sin$.
\end{lemma}
%
%
For the low frequency part $u_{\ell, \omega, \lambda}$, we apply \eqref{Life-span-est} and the estimate
%
%
\begin{equation}
\begin{split}
\label{tildphi}
\|\tilde{\phi}\left( \frac{x}{\lambda^\delta}
\right)\|_{H^{k}(\rr)} \le
\lambda^{\frac{\delta}{2}}\|\tilde{\phi}\|_{H^{k}(\rr)},
\quad k\ge 0
\end{split}
\end{equation}
%
%
to obtain a lower bound for its lifespan
%
\begin{equation*}
\label{lifespan-bound-1}
\begin{split}
T_{\ell, \omega,\lambda} \ge \frac{1}{2 c_s \|u_{\ell, \omega, \lambda}(0)\|_{H^s(\rr)}} = 
\frac{1}{2 c_s |\omega|
\lambda^{\frac{\delta}{2}-1}\|\tilde{\phi}\|_{H^s(\rr)}} >> 1.
\end{split}
\end{equation*}
%
%
Since $\omega$ belongs to a bounded subset of $\rr$, the existence of a 
common lifespan $T >> 1$ follows. 
%
%

Substituting the
approximate solution $u^{\omega, \lambda} = u_\ell + u^h$ into the HR
equation, we see that the error
$E$ of our approximate solution is given by
%
%
\begin{equation*}
E=E_1 + E_2 + \dots + E_8
\end{equation*}
%
%
where
%
%
\begin{equation}
\label{all_errors_together}
\begin{split}
E_1 & = \gamma \lambda^{1 -\frac{\delta}{2}-s}  \left[ u_\ell(x,0) - u_\ell(x,t)
\right] \phi\left(
\frac{x}{\lambda^ \delta}
\right)\sin(\lambda x - \gamma \omega t),
\\
E_2 & = \gamma \lambda^{-\frac{3\delta}{2}-s}
u_\ell(x,t) \cdot \phi'\left( \frac{x}{\lambda^\delta} \right)\cos\left( \lambda
x - \gamma \omega t
\right),
\\
E_3 & = \gamma u^h \p_x u_\ell, \; \; E_4 = \gamma u^h \p_x u^h, \ E_5  = 
\frac{3-\gamma}{2} \Lambda^{-1} \left[  \left( u^h \right)^2 \right], \\
E_6 & = (3- \gamma)\Lambda^{-1}
\left[ u_\ell u^h \right], \  E_7 = \frac{\gamma}{2} \Lambda^{-1} \left[ 
\left(
\p_x u^h \right)^2 \right ], \ E_8 = \gamma \Lambda^{-1} \left[  \p_x u_\ell \p_x u^h \right].
\end{split}
\end{equation}
%
%
%
Next we prove the decay of the error:
%
%
\begin{proposition}
Let $1<\delta<2$. Then for $s > 1$, bounded $\omega$, and
$\lambda >>1$ we are assured the decay of the error $E$ of the
approximate solutions to the HR equation. Specifically
%
%
%
\begin{equation}
\label{E-est}
\|E(t)\|_{H^1(\rr)} \lesssim \lambda^{\frac{\delta}{2} -s}, \quad |t| \le 
T.
\end{equation}
%
%
%
\end{proposition}
%
%
%
\begin{proof}
    It will suffice to estimate the $H^1$ norms of each $E_i$. \vspace{0.25cm}\\
{\bf Estimating the $H^1$ norm of $\hyperref[all_errors_together]{E_1}$.} 
We have
%
%
\begin{equation}
\label{fw-est}
\begin{split}
\|E_1\|_{H^1(\rr)}
& = \| \gamma \lambda^{1 -\frac{\delta}{2}-s} \left[ u_\ell(x,0) - u_\ell(x,t) \right]
\phi\left( \frac{x}{\lambda^\delta}
\right ) \sin (\lambda x - \gamma \omega t )\|_{H^1(\rr)}
\\
& \lesssim \lambda^{1 -\frac{\delta}{2} -s } \|\left[ u_\ell(x,0) - 
u_\ell(x,t)
\right] \phi\left( \frac{x}{\lambda^\delta} \right )
\sin\left( \lambda x - \gamma \omega t
\right) \|_{H^1(\rr)}.
\end{split}
\end{equation}
%
%
Applying the inequality 
%
%
\begin{equation*}
\label{applec}
\|fg\|_{H^1(\rr)} \lesssim \|f\|_{C^1(\rr)} \|g\|_{H^1(\rr)}
\end{equation*}
%
%
%
%
%
%
%
%
to estimate \eqref{fw-est} gives
%
%
\begin{equation}
\begin{split}
\|E_1\|_{H^1(\rr)} \lesssim \lambda^{1 - \frac{\delta}{2} -s } \|\phi
\left( \frac{x}{\lambda^\delta} \right) \sin (\lambda x - \gamma \omega t)
\|_{C^1(\rr)} \|[u_\ell (x,0) - u_\ell (x,t) ] \|_{H^1(\rr)}.
\label{apple14}
\end{split}
\end{equation}
%
%
We now estimate the right-hand side of \eqref{apple14} in pieces. First, 
note that routine computations give
%
%
\begin{equation}
\begin{split}
\|\phi\left( \frac{x}{\lambda^\delta} \right) \sin(\lambda x - \gamma 
\omega t)
\|_{C^1(\rr)}
\lesssim \lambda.
\label{apple15}
\end{split}
\end{equation}
%
%
Next, we observe that the fundamental 
theorem
of calculus and Minkowski's inequality give
%
%
%
%
\begin{equation}
\begin{split}
\|u_\ell(x,t) - u_\ell(x,0)\|_{H^1(\rr)}
& =  \| \int_0^t \p_\tau
u_\ell(x,\tau) \; d \tau \|_{H^1(\rr)}
\le \int_0^t \|\p_\tau u_\ell (x,\tau) \|_{H^1(\rr)} \; d \tau.
\label{apple100}
\end{split}
\end{equation}
%
%
We want to estimate the right-hand side of \eqref{apple100}. Recalling
\eqref{apple1'}, we have
%
%
\begin{equation}
\label{apple101}
\begin{split}
\|\p_\tau u_\ell(x,\tau) \|_{H^1(\rr)}
& \le \|\gamma u_\ell \p_x u_\ell \|_{H^1(\rr)}
+ \|\Lambda^{-1} \left[
\frac{3-\gamma}{2} (u_\ell)^2 + \frac{\gamma}{2} \left( \p_x u_\ell 
\right)^2
\right] \|_{H^1(\rr)}.
\end{split}
\end{equation}
%
%
Applying the algebra property of Sobolev spaces, we obtain
%
%
\begin{equation*}
\begin{split}
\|\gamma u_\ell \p_x u_\ell \|_{H^1(\rr)} &
\lesssim \|u_\ell\|_{H^2(\rr)}^2
\end{split}
\end{equation*}
%
%
which yields 
%
%
\begin{equation}
\begin{split}
\|\gamma u_\ell \p_x u_\ell \|_{H^1(\rr)} \lesssim \lambda^{-2 + \delta}
\label{apple102}
\end{split}
\end{equation}
%
%
%
%
by the following:
%
%
%
%%%%%%%%%%%%%%%%%%%%%%%%%%%%%%%%%%%%%%%%%%%%%%%%%%%%%
%
%
% 				
%
%
%%%%%%%%%%%%%%%%%%%%%%%%%%%%%%%%%%%%%%%%%%%%%%%%%%%%%
%
%
%
\begin{lemma}
\label{appleb}
Let $0<\delta<2$, $\lambda >>1$, with $\omega$ belonging to a bounded
subset of $\rr$. Then the initial value problem
\eqref{u-l-apple1'}-\eqref{apple1ah}
has a unique solution
$u_\ell \in C( [-T,T], H^s(\rr))$ for all $s
> 3/2$ which satisfies
%
%
\begin{equation}
\label{apple10'}
\|u_\ell(t)\|_{H^s(\rr)} \le c_s \lambda^{-1 +
\frac{\delta}{2}}, \quad |t| \le T.
\end{equation}
%
%
\end{lemma}
%
An analogous result can be found in \cite{Himonas:2009fk}. 
%
%
%
Applying the inequality 
%
\begin{equation*}
\begin{split}
\|\Lambda^{-1} f \|_{H^1(\rr)} \le \|f\|_{L^2(\rr)},
\label{apple27}
\end{split}
\end{equation*}
%
%
and the algebra property of Sobolev spaces, we obtain
%
%
\begin{equation*}
\begin{split}
\|\Lambda^{-1} \left[ \frac{3-\gamma}{2}(u_\ell)^2 +
\frac{\gamma}{2}\left( \p_x u_\ell \right)^2 \right] \|_{H^1(\rr)}
& \lesssim \|u_\ell\|_{H^2(\rr)}^2
\end{split}
\end{equation*}
%
%
which by \cref{appleb} gives
%
%
\begin{equation}
\begin{split}
\|\Lambda^{-1} \left[ \frac{3-\gamma}{2}(u_\ell)^2 +
\frac{\gamma}{2}\left( \p_x u_\ell \right)^2 \right] \|_{H^1(\rr)}
\lesssim \lambda^{-2 + \delta}, \quad |t| \le T.
\label{apple104}
\end{split}
\end{equation}
%
%
Substituting \eqref{apple102} and \eqref{apple104} into the right-hand side 
of
\eqref{apple101}, and recalling \eqref{apple100}, we obtain
%
%
\begin{equation}
\begin{split}
\|u_\ell(x,t) - u_\ell(x,0)\|_{H^1(\rr)} \lesssim \lambda^{-2 + \delta}, 
\quad |t| \le T.
\label{apple1055}
\end{split}
\end{equation}
%
Substituting \eqref{apple1055} and \eqref{apple15} into \eqref{apple14}, we obtain
%
%
\begin{equation}
  \label{apple105}
\begin{split}
  \| E_{1} \|_{H^{1}} \lesssim \lambda^{\delta/2 -s}.
\end{split}
\end{equation}
%
%

{\bf Estimating the $H^1$ norm of $\hyperref[all_errors_together]{E_2}$.} Applying  \eqref{applec}, we have
\begin{equation}
	\begin{split}
		\|E_2\|_{H^1(\rr)} 
		& = \gamma \lambda^{-\frac{3 \delta}{2} -s } \|u_\ell(x,t) \cdot
		\phi'\left( \frac{x}{\lambda^\delta} \right) \cos (\lambda x - \gamma \omega t)
		\|_{H^1(\rr)}
		\\
		& \lesssim_{s} \gamma \lambda^{\frac{-3 \delta}{2} -s } \|u_\ell(x,t) \|_{H^1(\rr)}
		\|\phi'\left( \frac{x}{\lambda^\delta} \right )
		\cos(\lambda x - \gamma \omega t 
		\|_{C^1(\rr)}.
		\label{apple18} 
	\end{split}
\end{equation}
We note that
\begin{equation*}
	\begin{split}
		& \|\phi'\left( \frac{x}{\lambda^\delta} \right) \cos(\lambda x - \gamma \omega t)
		\|_{C^1(\rr)}
		\\
		& \le \|\phi' \left( \frac{x}{\lambda^\delta} \right)\|_{L^\infty(\rr)} +
		\lambda \|\phi'\left( \frac{x}{\lambda^\delta} \right)\|_{L^\infty(\rr)}
		+ \lambda^{-\delta} \|\phi''\left( \frac{x}{\lambda^\delta} \right)
		\|_{L^\infty(\rr)}
	\end{split}
\end{equation*}
which gives
\begin{equation}
	\begin{split}
		\|\phi'\left( \frac{x}{\lambda^\delta} \right) \cos(\lambda x - \gamma \omega t)
		\|_{C^1(\rr)} \lesssim \lambda.
		\label{apple19}
	\end{split}
\end{equation}
Applying estimates \eqref{apple19} and \eqref{apple10'} to \eqref{apple18}, we obtain
\begin{equation*}
	\begin{split}
	\label{apple20}
	\|E_2\|_{H^1(\rr)} \lesssim \lambda^{-\delta -s }.
\end{split}
\end{equation*}
%
%
%
%
{\bf Estimating the $H^1$ norm of $\hyperref[all_errors_together]{E_3}$.} 
By  \eqref{applec}, we deduce
\begin{equation}
	\begin{split}
		\|\gamma u^h \p_x u_\ell \| \lesssim \|u^h\|_{C^1(\rr)}
		\|u_\ell\|_{H^1(\rr)}.
		\label{apple21}
	\end{split}
\end{equation}
Now, note that
\begin{equation}
	\begin{split}
		\|u^h\|_{L^\infty(\rr)} 
		& = \lambda^{-\frac{\delta}{2} -s } \|\phi\left( \frac{x}{\lambda^\delta}
		\right) \cos \left( \lambda x - \gamma \omega t \right) \|_{L^\infty(\rr)}
		\\
		& \lesssim \lambda^{-\frac{\delta}{2} -s }
		\label{apple22}
	\end{split}
\end{equation}
and 
\begin{equation}
	\begin{split}
		& \|\p_x u^h \|_{L^\infty(\rr)}
		\\
		& = \lambda^{-\frac{\delta}{2}-s} \|\phi\left(
		\frac{x}{\lambda^\delta}
		\right) \cdot -\lambda \sin(\lambda x - \gamma \omega t) + \lambda^{-\delta}
		\phi'\left( \frac{x}{\lambda^\delta}\right) \cos(\lambda x - \gamma \omega
		t) \|_{L^\infty(\rr)}
		\\
		& \lesssim \lambda^{1 - \frac{\delta}{2} -s }.
		\label{apple23}
	\end{split}
\end{equation}
Therefore, from \eqref{apple22} and \eqref{apple23} it follows that
\begin{equation}
	\begin{split}
		\|u^h\|_{C^1(\rr)} \lesssim \lambda^{-\frac{\delta}{2} -s } + \lambda^{1
		-\frac{\delta}{2} -s}
		\approx \lambda^{1- \frac{\delta}{2} -s}.
		\label{apple24}
	\end{split}
\end{equation}
Substituting estimates \eqref{apple24} and  \eqref{apple10'} into \eqref{apple21} we obtain
\begin{equation}
	\begin{split}
		\|\gamma u^h \p_x u_\ell \|_{H^1(\rr)} \lesssim \lambda^{-s}.
		\label{apple24'}
	\end{split}
\end{equation}
%
{\bf Estimating the $H^1$ norm of $\hyperref[all_errors_together]{E_4}$.} Applying  \eqref{applec} we have
\begin{equation}
	\begin{split}
		\|\gamma u^h \p_x u^h\|_{H^1(\rr)} \lesssim \|u^h\|_{C^1(\rr)}
		\|u^h\|_{H^1(\rr)}.
		\label{apple25}
	\end{split}
\end{equation}
Substituting in \eqref{apple24}, and observing that $\|u^h\|_{H^1(\rr)} \lesssim 1$ for $\lambda >>1$ by 
\cref{applea}, we obtain
\begin{equation}
	\begin{split}
		\|u^h \p_x u^h \|_{H^1(\rr)} \lesssim \lambda^{1-\frac{\delta}{2}-s}.
		\label{apple26}
	\end{split}
\end{equation}
%
%
%
{\bf Estimating the $H^1$ norm of $\hyperref[all_errors_together]{E_5}$.}
Applying \eqref{apple27}, we obtain
\begin{equation}
	\begin{split}
		\|E_5\|_{H^1(\rr)}
		& = \|\Lambda^{-1}\left[ \frac{3-\gamma}{2}(u^h)^2
		\right]\|_{H^1(\rr)}
		\\
		& \lesssim \|u^h\|_{L^\infty(\rr)} \|u^h\|_{L^2(\rr)}.
		\label{apple28}
	\end{split}
\end{equation}
Substituting \eqref{apple6} and \eqref{apple22} into \eqref{apple28}, we conclude that
\begin{equation}
	\begin{split}
		\|E_5\|_{H^1(\rr)} \lesssim \lambda^{-\frac{\delta}{2}-s}.
		\label{apple29}
	\end{split}
\end{equation}
%
%
%
%
%
{\bf Estimating the $H^1$ norm of $\hyperref[all_errors_together]{E_6}$.} Applying \eqref{apple27}, we obtain
\begin{equation}
	\begin{split}
		\|E_6\|_{H^1(\rr)} 
		& = \|\Lambda^{-1} \left[ (3 -\gamma) u_\ell u^h \right]\|_{H^1(\rr)}
		\\
		& \lesssim \|u_\ell\|_{L^2(\rr)} \|u^h\|_{L^\infty(\rr)}
		\label{apple30}
	\end{split}
\end{equation}
which by  \cref{appleb} and \eqref{apple22} reduces to
\begin{equation}
	\begin{split}
		\|E_6\|_{H^1(\rr)} \lesssim \lambda^{-1-s}.
		\label{apple31}
	\end{split}
\end{equation}
%
%
%
%
%
{\bf Estimating the $H^1$ norm of $\hyperref[all_errors_together]{E_7}$.} Applying \eqref{apple27}, we obtain
\begin{equation}
	\begin{split}
		\|E_7\|_{H^1(\rr)} 
		& = \|\Lambda^{-1} \left[ \frac{\gamma}{2}\left( \p_x u \right)^2
		\right]\|_{H^1(\rr)}
		\\
		& \lesssim  \|\p_x u^h\|_{L^\infty(\rr)} \|u^h\|_{H^1(\rr)}.
		\label{apple32}
	\end{split}
\end{equation}
which by  \cref{applea} and \eqref{apple23} reduces to
\begin{equation}
	\begin{split}
		\|E_7\|_{H^1(\rr)} \lesssim \lambda^{1-\frac{\delta}{2}-s}.
		\label{apple33'}
	\end{split}
\end{equation}
%
%
%
%
%
{\bf Estimating the $H^1$ norm of $\hyperref[all_errors_together]{E_8}$.} Applying \eqref{apple27}, we have
\begin{equation}
	\begin{split}
		\|E_8\|_{H^1(\rr)}
		& = \|\Lambda^{-1}\left[ \gamma \p_x u_\ell \p_x u^h \right]\|_{H^1(\rr)}
		\\
		& \lesssim \|u_\ell\|_{H^2(\rr)} \|\p_x u^h\|_{L^\infty(\rr)}
		\label{apple33}
	\end{split}
\end{equation}
which by  \cref{appleb} and \eqref{apple23} reduces to
\begin{equation}
	\begin{split}
		\|E_8\|_{H^1(\rr)} \lesssim \lambda^{-s}.
		\label{apple34}
	\end{split}
\end{equation}
Collecting all our estimates for the $E_i$ and recalling that we have assumed
$1<\delta<2$, we obtain
\begin{equation*}
	\begin{split}
		\|E\|_{H^1(\rr)}
		 \lesssim \lambda^{\frac{\delta}{2} -s }, \qquad \lambda >>1
	\end{split}
\end{equation*}
which completes the proof.
\end{proof}
%
%
%
%
%
\subsection{Construction of solutions}
We wish now to estimate the difference between approximate and actual 
solutions to
the HR i.v.p.\ with common initial data. Let
$u_{\omega,\lambda}(x,t)$ be the unique solution to the HR equation
with initial data $u^{\omega,\lambda}(x,0)$. That is,
$u_{\omega,\lambda}$ solves the initial value problem
\begin{align}
& \p_t u_{\omega,\lambda} = - \gamma u_{\omega,\lambda} \p_x 
u_{\omega,\lambda} - \Lambda^{-1} \left[
\frac{3- \gamma}{2}\left( u_{\omega,\lambda} \right)^2 + 
\frac{\gamma}{2}\left(
\p_x u_{\omega,\lambda} \right)^2
\right], \label{apple50}
\\
& u_{\omega,\lambda}(x, 0) = u^{\omega,\lambda}(x,0) = \omega \lambda^{-1}
\tilde{\phi} \left( \frac{x}{\lambda^\delta} \right)
+ \lambda^{-\frac{\delta}{2} -s}
\phi\left( \frac{x}{\lambda^\delta} \right) \cos(\lambda x).
\label{apple41}
\end{align}
%
%
%
We will now prove that the $H^1(\rr)$ norm of the difference decays: 
%
%%%%%%%%%%%%%%%%%%%%%%%%%%%%%%%%%%
%
%
%
%
%
%    : H^1 bound_for_difference-of-approx-and-actual-soln
%
%
%
%
%
%
%%%%%%%%%%%%%%%%%%%%%%%%%%%%%%%%%
%
%
%
\begin{proposition}
\label{applelem:bound_for_difference-of-approx-and-actual-soln}
%
Let $v = u^{\omega,\lambda} - u_{\omega,\lambda}$, with $\lambda >>1$.
Then, for $s > 1$ and $1<\delta<2$ we have
%
%
\begin{equation} \|
v(t)
\|_{H^1(\rr)}
\lesssim \lambda^{\frac{\delta}{2} -s}, \quad
|t| \le T.
\end{equation}
%
%
\end{proposition}
%
%
\begin{proof} First we observe that $v$ satisfies 
%
%
\begin{equation*}
\begin{split}
\p_t v & = E + \gamma(v \p_x v - v \p_x u^{\omega,\lambda} - 
u^{\omega,\lambda} \p_x v) \\
& + \Lambda^{-1}  \left[ \frac{3-
\gamma}{2}v^2 + \frac{\gamma}{2}\left( \p_x v \right)^2 - \left(
3 - \gamma \right)u^{\omega,\lambda} v -
\gamma \p_x u^{\omega,\lambda} \p_x v \right].
\end{split}
\end{equation*}
%%
It follows immediately that
\begin{equation}
\label{applev-dtv-pseudo-functional-equalityu}
\begin{split}
v(1-\p_x^2)\p_t v &= v(1- \p_x^2)E + v\gamma(1- \p_x^2)(v\p_x v 
- v\p_x u^{\omega,\lambda} -
u^{\omega,\lambda} \p_x v)
\\
&+ v\p_x \left[ \frac{3-\gamma}{2}v^2 + \frac{\gamma}{2}(\p_x v)^2 -
(3-\gamma)u^{\omega,\lambda} v - \gamma \p_x u^{\omega,\lambda} \p_x v \right].
\end{split}
\end{equation}
Applying the relation $v\p_t v = v(1-\p_x^2) \p_t v + v\p_x^2 \p_t v$ to
\eqref{applev-dtv-pseudo-functional-equalityu}, we obtain
\begin{equation}
\label{pre-int}
\begin{split}
v \p_t v &= v(1- \p_x^2)E + v\gamma(1- \p_x^2)(v\p_x v - v\p_x u^{\omega,\lambda} -
u^{\omega,\lambda} \p_x v)
\\
&+ v\p_x \left[ \frac{3-\gamma}{2}v^2 + \frac{\gamma}{2}(\p_x v)^2 -
(3-\gamma)u^{\omega,\lambda} v - \gamma \p_x u^{\omega,\lambda} \p_x v
\right] + v\p_x^2 \p_t v.
\end{split}
\end{equation}
Adding $\p_x v \p_t \p_x v$ to both sides of \eqref{pre-int} and 
integrating gives
\begin{equation}
\label{appleenergy-estu}
\begin{split}
&\frac{1}{2} \frac{d}{dt} \|v\|_{H^1(\rr)}^2  
\\
& =  \int_{\rr} v(1-\p_x^2)E dx
\\
& - \gamma \int_{\rr}  v(1-\p_x^2)(v\p_x u^{\omega,\lambda} + u^{\omega,\lambda} \p_x v) dx
\\
&- \int_{\rr}\left[ \left( 3-\gamma \right)v \p_x\left( u^{\omega,\lambda}v \right) + \gamma v
\p_x \left( \p_x u^{\omega,\lambda} \p_x v \right)\right]dx
\\
&+  \int_{\rr}
\left[ \gamma v \left( 1-\p_x^2 \right)\left( v \p_x v \right) + v
\p_x \left( \frac{3-\gamma}{2} v^2 + \frac{\gamma}{2}\left( \p_x v \right)^2
\right) \right . +  v \p_x^2 \p_t v + \p_x v \p_t \p_x v\bigg]dx.
\end{split}
\end{equation}
Noting that the last integral can be rewritten as 
\begin{equation*}
\begin{split}
\int_{\rr} \left[ \p_x (v^3) - \gamma \p_x (v^2 \p_x^2 v) + \p_x\left( v \p_t
\p_x v
\right) \right]dx  = 0
\end{split}
\end{equation*}
%
we can simplify \eqref{appleenergy-estu} to obtain
%
%
\begin{equation}
\label{appleenergy-est}
\begin{split}
\frac{1}{2} \frac{d}{dt} \|v\|_{H^1(\rr)}^2  
& = 
\int_{\rr}  v(1-\p_x^2)E dx\\
&-
\gamma \int_{\rr} v(1-\p_x^2)(v\p_x u^{\omega,\lambda} + 
u^{\omega,\lambda} \p_x v) dx
\\
&- \int_{\rr}\left[ \left( 3-\gamma \right)v \p_x\left( u^{\omega,\lambda}v 
\right) + \gamma v
\p_x \left( \p_x u^{\omega,\lambda} \p_x v \right)\right]dx.
\end{split}
\end{equation}
%
%
We now estimate the three integrals in the right-hand side of 
\eqref{appleenergy-est}. Integrating by parts and applying Cauchy-Schwartz,  
we obtain
%
%
%
\begin{equation}
\begin{split}
\label{applefirst_piece}
& \left |\int_{\rr} \left [v (1- \p_x^2)E \right ] dx \right |
\lesssim
\|v\|_{H^1(\rr)} \|E\|_{H^1(\rr)}
\end{split}
\end{equation}
for the first integral. Applying Parseval and H\"older gives
%
%
\begin{equation}
\begin{split}
\label{applesecond-piece-final}
& \left | -\gamma \int_{\rr}
\left[ v\left( 1-\p_x^2 \right)\left( v \p_x u^{\omega,\lambda} + 
u^{\omega,\lambda} \p_x v
\right) \right] dx \right |
\\
& \lesssim \left( \|u^{\omega,\lambda}\|_{L^\infty(\rr)}\| + \|\p_x 
u^{\omega,\lambda}
\|_{L^\infty(\rr)} \right .  + \|\p_x^2 u^{\omega,\lambda} 
\|_{L^\infty(\rr)}
\big )\|v\|_{H^1(\rr)}^2
\end{split}
\end{equation}
for the second integral, and applying H\"older gives 
\begin{equation}
\begin{split}
\label{applelast_piece_final}
& \left | -\int_{\rr} \left[ \left( 3-\gamma \right)v
\p_x \left( u^{\omega,\lambda} v \right) + \gamma
v \p_x \left( \p_x u^{\omega,\lambda} \p_x v \right)\right]dx \right |
\\
& \lesssim \big(
\|u^{\omega,\lambda}\|_{L^\infty(\rr)}
+ \|\p_x u^{\omega,\lambda} \|_{L^\infty(\rr)} \big)
\|v\|_{H^1(\rr)}^2
\end{split}
\end{equation}
%
%
%
for the third integral. Combining 
\eqref{applefirst_piece}-\eqref{applelast_piece_final}, we 
obtain
%
%
\begin{equation}
\begin{split}
\label{appleenergy-estimate-best}
\frac{d}{dt} \|v(t)\|_{H^1(\rr)}^2
& \lesssim \left( \|u^{\omega,\lambda}\|_{L^\infty(\rr)} + \|
\p_x u^{\omega,\lambda} \|_{L^\infty(\rr)} + \|\p_x^2 u^{\omega,\lambda} 
\|_{L^\infty (\rr)} \right)
\|v\|_{H^1(\rr)}^2 \\
&+ \|v\|_{H^1(\rr)} \|E\|_{H^1(\rr)}.
\end{split}
\end{equation}
%
%
Assume $\lambda >>1$. A straightforward calculation of derivatives yields
%
%
\begin{equation*}
\begin{split}
\|u^h\|_{L^\infty(\rr)} + \|\p_x u^h\|_{L^\infty(\rr)} + \|\p_x^2
u^h\|_{L^\infty(\rr)} \lesssim \lambda^{- \frac{\delta}{2} - s +2 }.
\label{apple53}
\end{split}
\end{equation*}
%
%
Furthermore, by the Sobolev Imbedding Theorem and \cref{appleb}, we have
%
%
%
%
\begin{equation*}
\begin{split}
\|u_\ell\|_{L^\infty(\rr)} + \|\p_x u_\ell \|_{L^\infty(\rr)} + \|\p_x^2
u_\ell\|_{L^\infty(\rr)}
& \le c_s \|u_\ell\|_{H^3(\rr)} 
\lesssim \lambda^{-1 + \frac{\delta}{2}}, 
\quad |t| \le T.
\label{apple55}
\end{split}
\end{equation*}
%
%
Hence
%
%
\begin{equation}
\begin{split}
\|u^{\omega,\lambda}\|_{L^\infty(\rr)} + \|\p_x 
u^{\omega,\lambda}\|_{L^\infty(\rr)} + \|\p_x^2
u^{\omega,\lambda}\|_{L^\infty(\rr)}
& \lesssim \lambda^{-\rho_s}, \quad |t| \le T
\label{apple56}
\end{split}
\end{equation}
%
%
where $\rho_s = \text{min} \Big\{ \frac{\delta}{2} + s -2, \; 1-
\frac{\delta}{2} \Big\}$.  Note that for $s>1$, we can assure $\rho_s > 0$
by choosing a suitable $1<\delta<2$.
Substituting \eqref{E-est} and \eqref{apple56} into \eqref{appleenergy-estimate-best},
we get
%
%
\begin{equation}
\label{apple58}
\frac{d}{dt} \|v(t)\|_{H(\rr)}^2 \lesssim \lambda^{-\rho_s}
\|v\|_{H^1(\rr)}^2 + \lambda^{-r_s}
\|v \|_{H^1(\rr)}, \quad |t| \le T.
\end{equation}
%
%
Applying Gronwall's inequality completes the proof. 
\end{proof}
%
%
%
%

\subsection{Non-Uniform Dependence for $s>3/2$}
Let $u_{\pm 1,\lambda}$ be solutions to the HR i.v.p.\ with initial 
data $u^{\pm 1,
n}(0)$. We wish to show that the $H^s$ norm of the difference of $u_{\pm 1,
n}$ and the associated approximate solution $u^{\pm 1,\lambda}$
decays as $\lambda \to \infty$. Note that
%
%
\begin{equation*}
\begin{split}
\label{apple62}
\|u^{\pm 1, \lambda}(t)\|_{H^{2s-1}(\rr)}
& \le \|u_{\ell, \pm 1, \lambda}\|_{H^{2s-1}(\rr)} +
\| \lambda^{-\frac{\delta}{2} -s} \phi \left(
\frac{x}{\lambda^\delta} \right) \cos(\lambda x \mp \gamma \omega t)
\|_{H^{2s-1}(\rr)}
\\
& \lesssim \lambda^{s-1}, \quad |t| \le T
\end{split}
\end{equation*}
%
%
where the last step follows from \cref{applea} and \cref{appleb}.
Using \eqref{u_x-Linfty-Hs}, we have 
%
\begin{equation*}
\begin{split}
\|u_{\pm 1,\lambda} (t) \|_{H^{2s-1}(\rr)}
& \lesssim  \|u^{\pm 1,\lambda}(0) \|_{H^{2s-1}(\rr)}, \quad
|t| \le T.
\label{apple60}
\end{split}
\end{equation*}
%
%
%
%
Hence
%
\begin{equation}
\begin{split}
\|u^{\pm 1, \lambda}(t) - u_{\pm 1, \lambda}(t) \|_{H^{2s-1}(\rr)}
\lesssim \lambda^{s-1}, \quad |t| \le T.
\label{apple63}
\end{split}
\end{equation}
%
%
Furthermore, by 
\cref{applelem:bound_for_difference-of-approx-and-actual-soln} 
%
%
\begin{equation}
\begin{split}
\|u^{\pm 1, \lambda}(t) - u_{\pm 1, \lambda} \|_{H^1(\rr)} \lesssim
\lambda^{\frac{\delta}{2} -s}, \quad |t| \le T.
\label{apple64}
\end{split}
\end{equation}
%
%
%
%
%
%
%
%
Interpolating between estimates \eqref{apple63} and \eqref{apple64} using 
the inequality
\begin{equation*}
\label{apple403}
\|\psi \|_{H^s (\rr)} \leq  (\| \psi \|_{H^1 (\rr)} \| \psi
\|_{H^{2s-1}(\rr)})^\frac12
\end{equation*}
%
%
gives
%
%
\begin{equation}
\begin{split}
\|u^{\pm 1, \lambda}(t) - u_{\pm 1, \lambda}(t)
\|_{H^s(\rr)}
\lesssim \lambda^{\frac{\delta -2}{4}}, \quad |t| \le T.
\label{apple65}
\end{split}
\end{equation}
%
%
Next, we will use estimate \eqref{apple65} to prove non-uniform
dependence when $s > 3/2$.

\subsection*{Behavior at time $t=0$.}  We have
%
%
%
%
\begin{equation*}
\begin{split}
\|u_{1,\lambda}(0) - u_{-1,\lambda}(0) \|_{H^s(\rr)} & = \|u^{1,\lambda}(0) 
- u^{-1,\lambda}(0) \|_{H^s(\rr)}
= 2 \lambda^{-1} \| \tilde{\phi}\left( \frac{x}{\lambda^\delta}
\right) \|_{H^s(\rr)}.
\label{apple}
\end{split}
\end{equation*}
%
%
%
%
Applying \eqref{tildphi} and recalling that $1<\delta<2$, we conclude that
%
%
\begin{equation}
\begin{split}
\|u_{1,\lambda}(0) - u_{-1,\lambda}(0) \|_{H^s(\rr)} \le 2
\lambda^{\frac{\delta}{2}-1} \|\tilde{\phi} \|_{H^s(\rr)} \to 0
\; \; \text{as} \; \; \lambda \to \infty.
\label{apple70}
\end{split}
\end{equation}
%
%
%
%
%%%%%%%%%%%%%% Behavior at time  t >0  %%%%%%%%%%%% 
%  
%

\subsection*{Behavior at time  $t>0$.}  Using the reverse triangle inequality, we 
have
%
%
%
%
%
\begin{equation} \label{appleHR-slns-differ-t-pos}
\begin{split}
\|
u_{1,\lambda}(t)
-
u_{- 1,\lambda}(t)
\|_{H^s(\rr)}
&
\ge
\|
u^{1,\lambda}(t)
-
u^{- 1,\lambda}(t)
\|_{H^s(\rr)}
\\
& -
\|
u^{1,\lambda}(t)
-
u_{1,\lambda}(t)
\|_{H^s(\rr)}
\\
& -
\|
-u^{-1,\lambda}(t)
+
u_{-1,\lambda}(t)
\|_{H^s(\rr)}.
\end{split}
\end{equation}
%
%
%
%
Using estimate \eqref{apple65} for the last two terms of 
the right-hand side of \eqref{appleHR-slns-differ-t-pos} 
and letting $\lambda$ go to $\infty$ 
yields
%
%
%
\begin{equation} \label{appleHR-slns-to-ap-est}
\liminf_{n\to\infty}
\|
u_{1,\lambda}(t)
-
u_{- 1,\lambda}(t)
\|_{H^s(\rr)}
\ge
\liminf_{n\to\infty}
\|
u^{1,\lambda}(t)
-
u^{- 1,\lambda}(t)
\|_{H^s(\rr)}.
\end{equation}
%
%
%
%
Using the identity $$
\cos \alpha -\cos \beta
=
-2
\sin(\frac{\alpha + \beta}{2})
\sin(\frac{\alpha - \beta}{2})
$$
gives
%
%
\begin{equation}
\label{apple80}
\begin{split}
u^{1,\lambda}(t)
-
u^{- 1,\lambda}(t)
=
u_{\ell,1,\lambda}(t) - u_{\ell,-1,\lambda}(t) + 
2\lambda^{-\frac{\delta}{2}-s}
\phi\left( \frac{x}{\lambda^\delta} \right)\sin(\lambda x) \sin(\gamma t).
\end{split}
\end{equation}
%
%
%
Hence, applying the reverse triangle inequality to \eqref{apple80}, we 
obtain
%
%
\begin{equation} \label{apple90}
\begin{split}
& \|
u^{1,\lambda}(t)
-
u^{- 1,\lambda}(t)
\|_{H^s(\rr)}
\\
& \ge 2 \lambda^{-\frac{\delta}{2}-s} \|\phi\left(
\frac{x}{\lambda^\delta} \right) \sin(\lambda x) \|_{H^s(\rr)} |\sin \gamma 
t|
- \|u_{\ell,-1,\lambda}(t) - u_{\ell,1,\lambda}(t)\|_{H^s(\rr)}.
\end{split}
\end{equation}
%
%
%
%
Letting $\lambda$ go to $\infty$, we see that \cref{applea}, \cref{appleb}, and \eqref{apple90} give
%
%
%
%
\begin{equation} \label{apple91}
\liminf_{\lambda \to\infty}
\|
u^{1,\lambda}(t)
-
u^{- 1,\lambda}(t)
\|_{H^s(\rr)}
\gtrsim
|\sin \gamma t|, \quad |t| \le T.
\end{equation}
%
%
Combining \eqref{appleHR-slns-to-ap-est} with \eqref{apple91}, and 
recalling that $T >>1$, we obtain \eqref{bdd-away-from-0}. This completes 
the proof of \cref{thm:hr-non-unif-dependence} for the
non-periodic case.  \qed
%
%%%%%%%%%%%%%%%%%%%%%%%%%%%%%%%%%%
%
%
%
%             Proof of  in the Periodic case
%
%
%
%%%%%%%%%%%%%%%%%%%%%%%%%%%%%%%%%%
%
\section{Proof of Non-Uniform Dependence 
on the Circle}
\label{sec:3}

%
Here we follow the proof in \cite{Himonas:2010}. 
Consider the periodic Cauchy problem for the HR equation
%
\begin{align}
& \p_t u = -\gamma u \p_x u  - \Lambda^{-1} \left[ \frac{3 - 
\gamma}{2}u^2 +
\frac{\gamma}{2}(\p_x u)^2 \right] ,
\label{hyperelastic-rod-equation}
\\
& u(x,0) = u_0(x), \; \; x \in \ci, \; \; t \in \rr.  \label{init-cond}
\end{align}
%
%
%
In this case the  approximate solutions are of the form
%
%
\begin{equation}
\label{approx-solutions-form}
u^{\omega,n}(x,t) = \omega n^{-1} + n^{-s} \cos \left( nx - \gamma \omega t
\right), 
\end{equation}
where $n$ is a positive integer and $\omega$ is in a bounded subset of 
$\rr$. We remark that the approximate 
solutions are in $C^\infty(\ci)$ for all $t \in \rr$, and hence have 
infinite lifespan in $H^s(\ci)$ for $s  \ge 0$. Furthermore, for $n>>1$ we 
have 
%
%
\begin{equation}
\label{bound-approx}
\begin{split}
\|u^{\omega,n} \|_{H^s(\ci)} \approx 1	
\end{split}
\end{equation}
%
%
from the inequality
\begin{equation}
\label{1m}
\begin{split}
\|\cos(k(nx-c))\|_{H^s(\ci)} \simeq n^s, \quad k \in \rr \setminus
\{0\}.
\end{split}
\end{equation}
%
%
%
%
Note that for $\gamma=1$ 
one gets the  approximate solutions
used for the CH equation in \cite{Himonas:2010}.
%
%
Substituting the approximate solutions into 
\eqref{hyperelastic-rod-equation}, we obtain the error
%
%
\begin{equation}
\begin{split}
E=
E_1 + E_2 + E_3 \label{57}
\end{split}
\end{equation}
%
%
where
\begin{align}
\label{90u}
& E_1 =
- \frac{\gamma}{2}n^{-2s+1}\sin\left[ 2\left( nx - \gamma \omega t \right)
\right],
\\
\label{90ah}
& E_2 = - \Lambda^{-1} \bigg[ \frac{3-\gamma}{2} \bigg (
n^{-2s+1} \sin\left( 2(nx - \gamma \omega t \right) + 2\omega n^{-s} \sin( 
2(nx - \gamma \omega t))
\bigg )
\bigg ],
\\
& E_3 = \frac{\gamma}{4}
n^{-2s+2} \left [ 1- \cos \left (\frac{nx - \gamma \omega t}{2} \right) 
\right ].
\label{90}
\end{align}
%
%
%
Next we will prove a decaying estimate for the error:
%
%%%%%%%%%%%%%%%%%%%%%%%%%%%%%%%%%
%
%
%
%                      
%
%
%
%
%%%%%%%%%%%%%%%%%%%%%%%%%%%%%%%%
\begin{lemma}
\label{lem:error_of_approx_solution}
Let $u^{\omega,n}$ be an approximate solution to the HR i.v.p., with 
$\sigma \le 1$,  $\omega$ bounded, and $n >> 1$.
Then for the error $E$ we have
%
%
\begin{equation}
\label{total-error-approx-solution}
\begin{split}
\|E(t)\|_{H^\sigma(\ci)} \lesssim n^{-r_s} \ \ \text{where} \ \ r_s = 
\begin{cases}
2(s-1)   & \text{if} \quad s \le 3,\\  s+1  & \text{if} \quad s > 3. \\
\end{cases}
\end{split}
\end{equation}
%
%
%
%
\end{lemma}
%
%
%
%
%
%
%
\begin{proof} It follows from the inequality
%
%
%
%
\begin{equation*}
\begin{split}
\|\Lambda^{-1}f \|_{H^{k}(\ci)} \le
\|f\|_{H^{k-1}(\ci)}
\end{split}
\end{equation*}
and \eqref{1m}.
\end{proof}
%%%%%%%%%%%%%%%%%%%%%%%%%%%%%%%%%
%
%
%
%   Proof of   in periodic case for s between 3/2 and 2
%
%
%
%%%%%%%%%%%%%%%%%%%%%%%%%%%%%%%%%%%
%
%
%
%
We are now prepared to prove a decaying estimate for the difference of 
approximate and actual solutions:
%
%
\begin{proposition}
\label{prop:bound_for_difference-of-approx-actual-soln}
Let $v=u^{\omega,n} - u_{\omega,n}$, $n >>1$,
where $u_{\omega,n}$ denotes a solution to
the Cauchy-problem \eqref{hyperelastic-rod-equation}-\eqref{init-cond} with
initial data $u_0(x) = u^{\omega,n}(x,0)$.
If \ $s > 3/2 $ and $\sigma = 1/2 + \ee$ for a sufficiently
small $\ee = \ee(s) > 0$, then 
%
%
\begin{equation} \label{differ-Hsigma-est} \|
v(t)
\|_{H^\sigma(\ci)}
\lesssim n^{-r_s}, \quad |t| \le T.
\end{equation}
%
%
\end{proposition}
%
%
\begin{proof} The difference $v = u^{\omega,n} - u_{\omega,n}$ satisfies 
the i.v.p
\begin{align}
\label{1.7}
& \p_t v  =  E - \frac{\gamma}{2} \p_x
\left[ \left( u^{\omega,n} + u_{\omega,n} \right)v \right]
\\
\notag & \phantom{\p_t v} - \Lambda^{-1} \left[
\frac{3-\gamma}{2} \left( u^{\omega,n} + u_{\omega,n}
\right) v +
\frac{\gamma}{2}\left( \p_x u^{\omega,n} +
\p_x u_{\omega,n}
\right) \p_x v
\right], \\
& v(x,0)=0.
\end{align}
For any $\sigma \in \rr$ let   $D^\sigma=(1-\p_x^2)^{\sigma/2}$ be the  operator
defined by 
%
$$ \widehat{D^\sigma f}(\xi) \doteq (1 + \xi^2)^{\sigma/2} \widehat{f}(\xi), $$
%
where $ \widehat{f}$ is the Fourier transform
%
$$ \widehat{f}(\xi) =  \int_{\ci} e^{-i \xi x} f(x) \ dx.  $$
%
%
Applying $D^\sigma$ to both sides of \eqref{1.7}, multiplying by
$D^\sigma v$, and integrating, we obtain the
relation
%
%
\begin{equation}
\begin{split}
\frac{1}{2}\frac{d}{dt}\|v(t)\|_{H^\sigma(\ci)}^2
& = \int_{\ci} D^\sigma E \cdot D^\sigma
v \ dx
\\
&-
\frac{\gamma}{2}\int_{\ci} D^\sigma
\p_x \left[ \left( u^{\omega,n} + u_{\omega,n} \right)v
\right]\cdot D^\sigma v \ dx
\\
& -
\frac{3-\gamma}{2}\int_{\ci} D^{\sigma
-2} \p_x \left[ \left( u^{\omega,n} + u_{\omega,n}
\right)v \right] \cdot D^\sigma v \ dx
\\
& - \frac{\gamma}{2}\int_{\ci} D^{\sigma
-2}
\p_x \left[ \left( \p_x u^{\omega,n} + \p_x u_{\omega,n}
\right)\cdot \p_x v \right] \cdot
D^\sigma v \ dx.
\label{X}
\end{split}
\end{equation}
%
%
We now estimate each integral of the right-hand side
of \eqref{X}.
\subsection*{Estimate of Integral 1.} Applying Cauchy-Schwartz, we obtain
%
%
\begin{equation}
\begin{split}
\left |\int_{\ci} D^\sigma E \cdot D^\sigma v \ dx \right |
\le \|E\|_{H^\sigma(\ci)} \|v\|_{H^\sigma(\ci)}.
\label{est_for_1}
\end{split}
\end{equation}
%
%
%
\subsection*{Estimate of Integral 2.} We can rewrite
%
%
\begin{equation}
\begin{split}
-\frac{\gamma}{2} \int_{\ci} D^\sigma \p_x \left[ \left( u^{\omega,n} + 
u_{\omega,n}
\right)v \right] \cdot D^\sigma v \ dx
= & -\frac{\gamma}{2}\int_{\ci} \left[ D^\sigma \p_x , u^{\omega,n} + 
u_{\omega,n}
\right]v \cdot D^\sigma v \ dx
\\
& - \frac{\gamma}{2} \int_{\ci} (u^{\omega,n} + u_{\omega,n})
D^\sigma \p_x v \cdot
D^\sigma v \ dx.
\label{est_for_2}
\end{split}
\end{equation}
%
%
We now estimate \eqref{est_for_2}. Integration 
by parts and Cauchy-Schwartz gives 
%
%
\begin{equation}
\begin{split}
\left | \frac{\gamma}{2} \int_{\ci} (u^{\omega,n} + u_{\omega,n})
D^\sigma \p_x v \cdot
D^\sigma v \ dx \right |
& \lesssim \|\p_x(u^{\omega,n} + u_{\omega,n}) \|_{L^\infty(\ci)}
\|v\|_{H^\sigma(\ci)}^2.
\label{2'}
\end{split}
\end{equation}
%
%
We now need the following result
taken from \cite{Himonas:2010}:
%
\begin{lemma}
\label{cor1}
If $\rho > 3/2$ and $0 \le \sigma + 1 \le \rho$, then
%
%
\begin{equation}
\begin{split}
\|[D^\sigma \p_x ,f]v\|_{L^2} \le C \|f\|_{H^\rho} \|v\|_{H^\sigma}.
\label{15}
\end{split}
\end{equation}
%
%
\end{lemma}
%
Let $\sigma = 1/2 + \ee$ and $\rho = 3/2 + \ee$, where 
$\ee > 0$ is
arbitrarily small. Applying Cauchy-Schwartz and \cref{cor1}, we obtain 
%
%
%
%
%
\begin{equation}
\begin{split}
\left | -\frac{\gamma}{2} \int_{\ci} [D^\sigma \p_x , u^{\omega,n} + 
u_{\omega,n}]v
\cdot D^\sigma v \ dx \right | \lesssim \|u^{\omega,n} +
u_{\omega,n}\|_{H^{\rho}(\ci)} \|v\|_{H^\sigma(\ci)}^2.
\label{7}
\end{split}
\end{equation}
%
%
Combining estimates \eqref{2'} and \eqref{7} we conclude that
%
%
\begin{equation}
\begin{split}
& \left | -\frac{\gamma}{2} \int_{\ci} D^\sigma \p_x \left[ \left( 
u^{\omega,n} + u_{\omega,n}
\right)v \right]  \cdot D^\sigma v \ dx \right |
\\
& \lesssim (\|u^{\omega,n} + u_{\omega,n}\|_{H^{\rho}(\ci)} + \|\p_x 
u^{\omega,n} +
\p_x u_{\omega,n}\|_{L^\infty(\ci)} ) \cdot \|v\|_{H^\sigma(\ci)}^2.
\label{8}
\end{split}
\end{equation}
%
%
%

\subsection*{Estimate of Integral 3.} Using Cauchy-Schwartz, and recalling that
$\sigma = 1/2 + \ee$,  we obtain
%
%
\begin{equation}
\begin{split}
\bigg | -\frac{3-\gamma}{2} \int_{\ci} D^{\sigma -2} \p_x \left[
(u^{\omega,n} + u_{\omega,n})v \right]
\cdot D^\sigma v \ dx \bigg |
\lesssim \|u^{\omega,n} + u_{\omega,n} \|_{L^\infty(\ci)} 
\|v\|_{H^\sigma(\ci)}^2.
\label{9}
\end{split}
\end{equation}
%
%
%
\subsection*{Estimate of Integral 4.}
We will need the following result.
%
\begin{lemma}
  \label{lem:frac-deriv}
For $s > 3/2$, $r \le s$, $s + r \ge 2$, we have
%
%
\begin{equation}
\label{11}
\begin{split}
  \| fg \|_{H^{r-1}} \lesssim \| f \|_{H^{r-1}} \| g \|_{H^{s-1}}.
\end{split}
\end{equation}
%
%
\end{lemma}
%
%
\begin{proof}
For the periodic case we have
%
%
\begin{equation*}
\begin{split}
\| fg\|_{H^{r-1}}^{2}
& \le  \sum_{n \in \zz}  (1 + n^{2})^{r-1}\left [ \sum_{k \in \zz}
| \wh{f}(k) |  | \wh{g}(n - k) | (1 +
k^{2})^{\frac{1-s}{2}} (1 + k^{2})^{\frac{s-1}{2}}
\right ]^{2}.
\end{split}
\end{equation*}
%
Applying Cauchy Schwartz in $k$, we bound this by
%
%
%
\begin{equation*}
\label{np-key-term-iu}
\begin{split}
\| f \|_{H^{s-1}}^{2} \sum_{n \in \zz}  (1 + n^{2})^{r-1}\sum_{k \in \zz} \frac{|
\wh{g}(n - k) |^{2}}{(1 + k^{2})^{s-1}}.
\end{split}
\end{equation*}
%
But by change of variables and Fubini
%
\begin{equation}
\label{opp}
\begin{split}
\sum_{n \in \zz}  (1 + n^{2})^{r-1}\sum_{k \in \zz} \frac{|
\wh{g}(n - k) |^{2}}{(1 + k^{2})^{s-1}}
& = \sum_{k \in \zz}| \wh{g}(k) |^{2} \sum_{n \in \zz}  
\frac{1}{(1 + n^{2})^{s-1}[1 + (n - k)^{2}]^{1-r}}.  
\end{split}
\end{equation}
%
Without loss of generality, we assume $k \ge 0$ and write 
\begin{equation*}
\begin{split}
&  \sum_{n \in \zz}  
\frac{1}{(1 + n^{2})^{s-1}[1 + (n - k)^{2}]^{1-r}}  
\\
& = 
\sum_{0 \le n \le 2k} \frac{1}{(1 + n^{2})^{s-1}[1 + (n - k)^{2}]^{1-r}} 
+ \sum_{n > 2k} \frac{1}{(1 + n^{2})^{s-1}[1 + (n - k)^{2}]^{1-r}}
\\
& + \sum_{n \ge 0} \frac{1}{(1 + n^{2})^{s-1}[1 + (n + k)^{2}]^{1-r}} 
\\
& \doteq I + II + III.
\end{split} 
\end{equation*}
%
We have the estimate
%
%
\begin{equation}
\label{est-tem}
\begin{split}
II 
& \le \sup_{n > 2k} \frac{1}{\left[ 1 + (n-k)^{2} \right]^{1-r}}
\sum_{n > 2k} \frac{1}{(1 + n^{2})^{s-1}} 
\\
& \lesssim (1 + k^{2})^{r-1}, \quad
s > 3/2.
\end{split}
\end{equation}
Similarly
%
%
\begin{equation*}
\begin{split}
III \lesssim (1 + k^{2})^{r-1}, \quad s > 3/2.
\end{split}
\end{equation*}
%
%
%
To estimate $I$, we assume without loss of generality that $k$ is even and write
%
%
\begin{equation*}
\begin{split}
&  I = \sum_{0 \le n \le k/2} \frac{1}{(1 + n^{2})^{s-1}[1 + (n - k)^{2}]^{1-r}} 
+ \sum_{k/2 < n \le 3k/2} \frac{1}{(1 + n^{2})^{s-1}[1 + (n - k)^{2}]^{1-r}} 
\\
& + \sum_{3k/2 < n \le 2k} \frac{1}{(1 + n^{2})^{s-1}[1 + (n - k)^{2}]^{1-r}} 
\\
& \doteq i + ii + iii.
\end{split} 
\end{equation*}
Hence, estimating as in \eqref{est-tem}, we have
%
%
\begin{equation*}
\begin{split}
i, iii \lesssim (1 + k^{2})^{r-1}, \quad
s > 3/2
\end{split}
\end{equation*}
%
and
%
%
\begin{equation*}
\begin{split}
ii & \le \sup_{k/2 \le n \le 3k/2} \frac{1}{\left( 1 + n^{2} \right)^{s-1}}
\sum_{k/2 \le n \le 3k/2} \frac{1}{[1 + (n-k)^{2}]^{1-r}} \\
& \lesssim \frac{1}{(1 + k^{2})^{s-1}}, \quad r \le 1/2.
\end{split}
\end{equation*}
%
%
Therefore, 
%
%
%
\begin{equation*}
\begin{split}
I + II + III & \lesssim (1 + k^{2})^{1-s} + (1 + k^{2})^{r-1}, \quad r \le 1/2, \ s > 3/2
\\
& \lesssim  (1 + k^{2})^{r-1}, \quad r -1 \ge 1-s.
\end{split}
\end{equation*}
%
Applying this estimate to \eqref{opp} and recalling \eqref{np-key-term-iu},
we obtain
%
%
%
%
\begin{equation}
\label{yhh-iu}
\begin{split}
\| f g \|_{H^{r-1}} \lesssim \| f \|_{H^{s-1}} \| g \|_{H^{r-1}},
\quad s > 3/2, \ r \le 1/2, \ s + r \ge 2.
\end{split}
\end{equation}
We now need the following result taken from Taylor \cite{Taylor:2011}.
%
%
%%%%%%%%%%%%%%%%%%%%%%%%%%%%%%%%%%%%%%%%%%%%%%%%%%%%%
%
%
%                
%
%
%%%%%%%%%%%%%%%%%%%%%%%%%%%%%%%%%%%%%%%%%%%%%%%%%%%%%
%
%
\begin{lemma}[Sobolev Interpolation]
For fixed $j \le k, m \le n$ suppose that \\ $T: H^{j} \to H^{m}$ continuously
and $T: H^{k} \to H^{n}$. Then\\ $T: H^{\theta j + (1 - \theta)k} \to H^{\theta
m + (1 - \theta) n}$ continuously for all $\theta \in (0,1]$.
\label{prop:sob-interp-iu}
\end{lemma}
%
To apply Lemma~\ref{prop:sob-interp-iu}, we note that \eqref{yhh-iu}
and the algebra property of the Sobolev space $H^{t}$, $t > 1/2$ imply that for $s > 3/2$
%
%
\begin{equation*}
\begin{split}
\| f g \|_{H^{r-1}} \lesssim \| g \|_{H^{r-1}}, \  \text{where} \ 
r=1/2 \ \text{or} \  r =s, \ \| f \|_{H^{s-1}} =1.
\end{split}
\end{equation*}
%
%
That is, for fixed $f \in H^{s-1}$ with $\| f \|_{H^{s-1}} =1$, the map $g \mapsto
Tg = fg$ is linear continuous from $H^{-1/2}$ to $H^{-1/2}$ and from $H^{s-1}$ to
$H^{s-1}$. Therefore, by Lemma~\ref{prop:sob-interp-iu}, it is continuous from
$H^{\theta (s-1) + (1 - \theta)(-1/2) }$ to $H^{\theta (s-1) + (1 - \theta)(-1/2) }$ for all $\theta \in
[0, 1)$. Setting $\theta = (r-1/2)/(s-1/2)$, $ 1/2 \le r < s$, we obtain that $T$ is
continuous from $H^{r-1}$ to $H^{r-1}$. Since $T$ is also linear from $H^{r-1}$
to $H^{r-1}$, we see that 
%
%
\begin{equation*}
\begin{split}
\| f g \|_{H^{r-1}} \lesssim \| g \|_{H^{r-1}}, \quad 1/2 \le r \le s, \ s > 3/2, \ \| f \|_{H^{s-1}} =1
\end{split}
\end{equation*}
and so for general $f \in H^{s-1}$ we have 
%
\begin{equation}
\label{hhh-iu}
\begin{split}
\| f g \|_{H^{r-1}} \lesssim \|f \|_{H^{s-1}}
\| g \|_{H^{r-1}}, \quad 1/2 \le r \le s, \ s > 3/2. 
\end{split}
\end{equation}
%
Combining \eqref{yhh-iu} and \eqref{hhh-iu} completes the proof in the periodic
case. For the non-periodic case we have
%
%
\begin{equation*}
\begin{split}
\| fg\|_{H^{r-1}}^{2}
\le \int_{\rr}  (1 + \xi^{2})^{r-1}\left [ \int_{\rr}
| \wh{f}(\eta) |  | \wh{g}(\xi - \eta) | (1 +
\eta^{2})^{\frac{1-s}{2}} (1 + \eta^{2})^{\frac{s-1}{2}}
d \eta \right ]^{2} d \xi.
\end{split}
\end{equation*}
%
Applying Cauchy Schwartz in $\eta$, we bound this by
%
%
%
\begin{equation*}
\begin{split}
\| f \|_{H^{s-1}}^{2} \int_{\rr}  (1 + \xi^{2})^{r-1}\int_{\rr} \frac{|
\wh{g}(\xi - \eta) |^{2}}{(1 + \eta^{2})^{s-1}} d \eta d \xi.
\end{split}
\end{equation*}
%
We now wish to bound the integral term. Applying a change of variable, we see it
is equal to
%
\begin{equation*}
\begin{split}
\int_{\rr} (1 + \xi^{2})^{r-1} \int_{\rr}
\frac{| \wh{g}(\eta) |^{2}}{[1 + (\xi - \eta)^{2}]^{s-1}} d \eta d \xi
\end{split}
\end{equation*}
which by Fubini is equal to
%
%
\begin{equation}
\label{int-pre-calc-lem-iu}
\begin{split}
& \int_{\rr} | \wh{g}(\eta) |^{2} \int_{\rr} \frac{1}{\left[
1 + (\xi - \eta)^{2} \right]^{s-1} (1 + \xi^{2})^{1-r}} d \xi d \eta
\\
& \lesssim \int_{\rr} | \wh{g}(\eta) |^{2} \int_{\rr} \frac{1}{\left[
1 + |\xi - \eta| \right]^{2(s-1)} (1 + |\xi|)^{2(1-r)}} d \xi d \eta.
\end{split}
\end{equation}
%
We now need the following lemma: 
%
%
\begin{lemma}
\label{lem:calc}
%
Fix $p, q > 0$ such that $p +q >1$, and let $r =\min\left\{p - \ee_{q}, q -
\ee_{p}, p+q-1 \right\}$, where $\ee_{j} > 0$ is arbitrarily small for $j = 1$
and $\ee_{j} = 0$ for $j \neq 1$. Adopt the notation
$\langle x - \alpha \rangle  \doteq 1 + | x - \alpha |$. Then 
%
\begin{equation*}
\begin{split}
& \int_{\rr} \frac{1}{\langle x - \alpha \rangle ^{p} \langle x -
\beta \rangle
^{q}} d x
\le \frac{c_{r}}{\langle \alpha - \beta \rangle ^{r}}. 
\end{split}
\end{equation*}
\end{lemma}
%
To be able to apply Lemma~\ref{lem:calc} to the integral term in \eqref{int-pre-calc-lem-iu}, 
we must first check
that its conditions are met. Let $ s = 3/2 + \ee$, $r = 1- \delta$, $\ee > 0$, $
\delta \ge 0$ and observe that
%
%
\begin{equation*}
\begin{split}
2(s-1) + 2(1-r)
& = 2(s-r)
\\
& = 2[3/2 + \ee - (1 - \delta)]
\\
& = 2(1/2 + \ee + \delta)
\\
& = 1 + 2 \ee + 2 \delta > 1.
\end{split}
\end{equation*}
%
%
Furthermore, $2(s-1), 2(1-r) > 0$. Hence, Lemma~\ref{lem:calc} is applicable. 
Note that since $s > 3/2$, we see that $2(s-1) \neq 1$. However, it is possible that $2(1-r) =1$; hence we must now separate the cases $r \neq 1/2$ and $r = 1/2$. Suppose $r \neq 1/2$. Then 
%
%
\begin{equation*}
\begin{split}
\min\left\{ 2(s-1), 2(1-r), 2(s-1) + 2(1-r) -1 \right\}
& = \min\left\{ 1 + 2 \ee, 2 \delta, 2\ee + 2 \delta \right\}
\\
& = \min\left\{ 1 + 2 \ee, 2 \delta\right\}
\\
& = 2 \delta, \quad \delta \le 1/2 + \ee.
\end{split}
\end{equation*}
%
If $r = 1/2$, then since $s > 3/2$, we can choose $\eta > 0$ sufficiently small
such that
%
%
\begin{equation*}
\begin{split}
\min\left\{ 2(s-1) -\eta , 2(1-r), 2(1-r) + 2(s-1) - 1  \right\}
& = 1 
\\
& = 2(1 -r)
\\
& = 2\delta.
\end{split}
\end{equation*}
%
Hence, for $0 \le \delta \le 1/2 + \ee$, $\ee >
0$, \eqref{int-pre-calc-lem-iu} is bounded by
\begin{equation*}
\begin{split}
C_{s,r} \int_{\rr}  | \wh{g}(\eta) |^{2} \int_{\rr} \frac{1}{\left( 1
+ |\eta| \right )^{2 \delta}} d \xi d \eta 
& \lesssim
\| g \|_{H^{-\delta}}^{2}
\\
& = \| g \|_{H^{r-1}}^{2}.
\end{split}
\end{equation*}
%
Our restriction on $\delta$ is equivalent to the restriction 
$$1-r \le 1/2 + s - 3/2, \quad r \le 1, \ s > 3/2,$$ or
$$s + r \ge 2,  \quad  r \le 1, \ s > 3/2.$$ Therefore, 
%
%
%
%
\begin{equation*}
\begin{split}
\| f g \|_{H^{r-1}} \lesssim \| f \|_{H^{s-1}} \| g \|_{H^{r-1}},
\quad s + r \ge 2, \ s > 3/2, \ r \le 1.
\end{split}
\end{equation*}
%
%
The remainder of the proof is analogous to that in the periodic case.
\end{proof}
%
%
%
\begin{proof}[Proof of Lemma~\ref{lem:calc}]
%
By the change of variable $x \mapsto x/2 + (\alpha + \beta)/2$, we have
%
%
\begin{equation}
\label{rur}
\begin{split}
\int_{\rr} \frac{1}{\langle x - \alpha \rangle^{p} \langle  x -
\beta
\rangle^{q}}d x
& \simeq \int_{\rr} \frac{1}{\langle x/2 - (\alpha - \beta)/2  \rangle^{p}
\langle  x/2 + (\alpha - \beta)/2 \rangle^{q}} d x
\\
& \lesssim \int_{\rr} \frac{1}{\langle x - (\alpha - \beta)  \rangle^{p}
\langle  x + (\alpha - \beta) \rangle^{q}} d x
\\
& = \int_{\rr} \frac{1}{\langle a - x \rangle ^{p} \langle a + x \rangle
^{q}} d x, \quad a = \alpha - \beta
\end{split}
\end{equation}
%
which for $a =0$ reduces to 
%
%
\begin{equation*}
\begin{split}
\int_{\rr} \frac{1}{\langle x \rangle ^{p+q}} d x 
& = 2 \int_{0}^{\infty} \frac{1}{(1 + x)^{p+q}} d x
\\
& = \frac{2}{p+q -1}.
\end{split}
\end{equation*}
%
%
We now handle the case $a \neq 0$. Note that by the change of variable $x \mapsto
-x$ we may restrict our attention to the case  $a > 0$ without loss of
generality. Split
%
%
\begin{equation*}
\begin{split}
\int_{\rr} \frac{1}{\langle a + x \rangle ^{p} \langle a - x \rangle
^{q}} d x
& = \int_{-2a}^{2a}
\frac{1}{\langle a + x \rangle ^{p} \langle a - x \rangle
^{q}} d x
\\
& + \int_{| x | \ge 2a} 
\frac{1}{\langle a + x \rangle ^{p} \langle a - x \rangle
^{q}} d x
\\
& = I + II.
\end{split}
\end{equation*}
%
%
Then
\begin{equation*}
\begin{split}
I 
& = \int_{0}^{2a}
\frac{1}{\langle a + x \rangle ^{p} \langle a - x \rangle
^{q}} d x + \int_{-2a}^{0}
\frac{1}{\langle a + x \rangle ^{p} \langle a - x \rangle
^{q}} d x.
\end{split}
\end{equation*}
We bound the first term by
\begin{equation*}
\begin{split}
\sup_{0 \le x \le 2a} \frac{1}{\langle a + x \rangle
^{p}} \int_{0}^{2a} \frac{1}{\langle a - x \rangle ^{q}} d x
& = \frac{1}{\langle a \rangle ^{p}} \int_{0}^{2a} \frac{1}{(1 + | a -
x
|)^{q}} d x  
\\
& = \frac{2}{\langle a \rangle ^{p}} \int_{0}^{a} \frac{1}{(1 + a -
x)^{q}} d x
\\
& \lesssim
\begin{cases}
1/{\langle a \rangle ^{p}} \left| 1 - 1/{(1 +
a)^{q -1}} \right|, \quad & q \neq 1
\\
\log(1+a)/{\langle a \rangle^{p} }, \quad & q =1.
\end{cases}
\end{split}
\end{equation*}
%
But
%
%
\begin{equation*}
\begin{split}
\frac{1}{\langle a \rangle ^{p}}\left| 1 - \frac{1}{(1 +
a)^{q -1}} \right|
& \lesssim
\begin{cases}
1/{\langle a \rangle^{p} }, \quad & q > 1
\\
1/{\langle a \rangle ^{p + q -1}}, \quad & q < 1
\end{cases}
\end{split}
\end{equation*}
%
%
and
%
%
\begin{equation*}
\begin{split}
\frac{\log(1 + a)}{\langle a \rangle^{p} } \le  \frac{c_{\ee}}{\langle a
\rangle ^{p - \ee}} \ \text{for any} \ \ee > 0.
\end{split}
\end{equation*}
%
For the second term, we bound by
%
%
\begin{equation*}
\begin{split}
& \sup_{-2a \le x \le 0} \frac{1}{\langle a - x \rangle
^{q}} \int_{-2a}^{0} \frac{1}{\langle a + x \rangle ^{p}} d x
\\
& = \frac{1}{\langle a \rangle ^{q}} \int_{-2a}^{0} \frac{1}{(1 + | a +
x
|)^{p}} d x 
\\
& = \frac{2}{\langle a \rangle ^{q}} \int_{-a}^{0} \frac{1}{(1 + a +
x)^{p}} d x
\\
& \lesssim
\begin{cases}
1/{\langle a \rangle ^{q}} \left| 1 - 1/{(1 +
a)^{p -1}} \right|, \quad & p \neq 1
\\
\log(1+a)/{\langle a \rangle^{q} }, \quad & p =1.
\end{cases}
\end{split}
\end{equation*}
%
But
%
%
\begin{equation*}
\begin{split}
\frac{1}{\langle a \rangle ^{q}}\left| 1 - \frac{1}{(1 +
a)^{p -1}} \right|
& \lesssim
\begin{cases}
1/{\langle a \rangle ^{q}}, \quad & p > 1
\\
1/{\langle a \rangle ^{p + q -1}}, \quad & p < 1
\end{cases}
\end{split}
\end{equation*}
%
%
and
%
%
\begin{equation*}
\begin{split}
\frac{\log(1 + a)}{\langle a \rangle^{q} } \le  \frac{c_{\ee}}{\langle a
\rangle ^{q - \ee}} \ \text{for any} \ \ee > 0.
\end{split}
\end{equation*}
%
%
%
Therefore,
\begin{equation*}
I \le  \frac{c_{p,q, \ee}}{\langle a \rangle ^{\min\left\{ p-\ee_{q}, q -\ee_{p}, p + q-1 \right\}}}.
\end{equation*}
%
%
Also
%
%
\begin{equation*}
\begin{split}
II 
& = \int_{x \ge 2a} \frac{1}{(1 + x - a)^{p} (1 + x +
a)^{q}} d x
\\
& \le \int_{x \ge 2a} \frac{1}{(1 + x -a)^{p+q}} d x
\\
& \simeq \frac{1}{\langle a \rangle^{p+q -1}}, \qquad p + q > 1.
\end{split}
\end{equation*}
%
%
Collecting our estimates for $I$ and $II$ we see that for 
$p, q > 0$ such that $p +q >1$, and $r =\min\left\{p -\ee_{q}, q - \ee_{p}, p+q-1
\right\}$, we have
\begin{align*}
\int_{\rr} \frac{1}{\langle a - x \rangle ^{p} \langle a + x \rangle
^{q}} d x
\le \frac{c_{r}}{\langle a \rangle ^{r}}.
\label{est-2}
\end{align*}
Recalling \eqref{rur}, the proof is complete.
\end{proof}
%
%
Noting that \eqref{11} implies
%
%
%
\begin{equation}
\label{impo}
\begin{split}
\|fg\|_{H^{\sigma - 1}} \le C \|f\|_{H^{\sigma}}
\|g\|_{H^{\sigma -1}}, \quad \sigma > 1/2
\end{split}
\end{equation}
%
%
%
and applying Cauchy-Schwartz and  \eqref{impo}, we obtain
%
%
\begin{equation}
\begin{split}
& \left | -\frac{\gamma}{2} \int_{\ci} D^{\sigma -2 } \p_x \left[
\left( \p_x u^{\omega,n} + \p_x u_{\omega,n} \right) \cdot \p_x v
\right] \cdot D^\sigma v \ dx \right |
\\
& \lesssim \|\p_x u^{\omega,n} + \p_x u_{\omega,n}
\|_{H^\sigma(\ci)} \|v\|_{H^\sigma(\ci)}^2.
\label{12}
\end{split}
\end{equation}
%
%
Collecting estimates \eqref{est_for_1}, \eqref{8}, \eqref{9}, and
\eqref{12}, and applying the Sobolev Imbedding Theorem, we deduce
%
%
\begin{equation}
\begin{split}
\frac{1}{2}\frac{d}{dt} \|v\|_{H^\sigma(\ci)}^2
& \lesssim
\|u^{\omega,n} + u_{\omega,n}\|_{H^{\rho}(\ci)} \|v\|_{H^\sigma(\ci)}^2
+ \|E\|_{H^\sigma(\ci)}
\|v\|_{H^\sigma(\ci)}.
\label{10}
\end{split}
\end{equation}
%
%
It follows from \eqref{Life-span-est} and 
\eqref{bound-approx} that the solutions $u_{\omega,n}$ have a common 
lifespan $T$. Hence, applying the triangle inequality, 
\eqref{u_x-Linfty-Hs}, and \eqref{1m} we obtain  
%
%
\begin{equation}
\|u^{\omega,n} + u_{\omega,n}\|_{H^\rho(\ci)} \lesssim n^{\rho -s}, 
\quad |t| \le T.
\label{3r}
\end{equation}
%
%
%
%
%
%
%proof of existence relies on an argument similar to that used on the line to prove 
%the existence of a common lifespan for the approximate solutions).
Using \cref{lem:error_of_approx_solution} and
substituting \eqref{total-error-approx-solution} and \eqref{3r}
into \eqref{10}, we obtain
%
%
\begin{equation}
\begin{split}
\frac{1}{2}\frac{d}{dt}\|v\|_{H^\sigma(\ci)}^2 \lesssim n^{\rho - s}
\|v\|_{H^\sigma(\ci)}^2 + n^{-r_s}\|v\|_{H^\sigma(\ci)}, \quad |t| \le T.
\label{200r}
\end{split}
\end{equation}
%
%
Applying Gronwall's inequality gives \eqref{differ-Hsigma-est}, concluding 
the proof. \qed 
%
%
%
%
%%%%%%%%%%%%%%%%%%%%%%%%%%%%%%%%%%%%%%%%%
%
% Non-Uniform Dependence for $3/2<s<2$
%
%
%%%%%%%%%%%%%%%%%%%%%%%%%%%%%%%%%%%%%%%%%

\subsection*{Non-Uniform Dependence for $s > 3/2$.}
%
%
%
Let $u_{\pm 1, n}$ be solutions to the HR i.v.p.\ with common initial data 
$u^{\pm 1,
n}(0)$, respectively.
We wish to show that the $H^s$ norm of the difference of $u_{\pm 1,
n}$ and the associated approximate solution $u^{\pm 1, n}$ decays.
We assume
$s > 3/2 $ and $\sigma = 1/2 + \ee$ \ for a sufficiently small
$\ee= \ee(s) > 0$. 
Then by \cref{prop:bound_for_difference-of-approx-actual-soln} we 
have
%
%
\begin{equation}
\begin{split}
\|u^{\pm 1, n}(t) - u_{\pm 1, n} (t) \|_{H^\sigma (\ci)} \lesssim n^{-r_s}
\label{6h}, \quad |t| \le T.
\end{split}
\end{equation}
%
%
Furthermore, by \eqref{1m} we obtain
%
%
\begin{equation}
\begin{split}
\|u^{\pm 1, n} (t) \|_{H^{2s - \sigma} (\ci)}
& \lesssim n^{s-\sigma}
\label{4}
\end{split}
\end{equation}
%
%
while \eqref{u_x-Linfty-Hs} and \eqref{4} give
\begin{equation}
\begin{split}
\|u_{\pm 1, n} (t) \|_{H^{2s - \sigma}(\ci)}
& \lesssim n^{s- \sigma}, \quad |t| \le T.
\label{final-est-Hk-norm-sol}
\end{split}
\end{equation}
%
%
%
%
%
%
%
Therefore, \eqref{4}, \eqref{final-est-Hk-norm-sol}, and the triangle
inequality yield
%
%
\begin{equation}
\begin{split}
\|u^{\pm 1, n} (t) - u_{\pm 1, n}(t)\|_{H^{2s - \sigma}(\ci)}
\lesssim n^{s-\sigma}, \quad |t| \le T.
\label{5h}
\end{split}
\end{equation}
%
%
%
%
Interpolating between estimates \eqref{6h} and \eqref{5h} using the 
inequality
%
\begin{equation*}
\|\psi \|_{H^s (\ci)} \leq  (\| \psi \|_{H^\sigma (\ci)} \| \psi
\|_{H^{2s-\sigma}(\ci)})^\frac12
\end{equation*}
%
%
we obtain
%
%
\begin{equation}
\begin{split}
\|u^{\pm 1,n}(t) - u_{\pm 1, n}(t) \|_{H^s (\ci)} \lesssim
n^{-\ee(s)/2}, \quad |t| \le T.
\label{10h}
\end{split}
\end{equation}
%
%
The remainder of the proof of non-uniform dependence on the circle is
analogous to that on the real line. 
\end{proof}
%
%
%
%
%
\section{Well-Posedness for HR in the Periodic Case}
%
%
%
%
We will now prove well-posedness for the periodic case, after which we will
provide the necessary details to extend the argument to the non-periodic case.
\subsection{Existence.}
\label{existence}
Here we will prove the existence of a solution to the HR i.v.p.\
and inequalities
\eqref{Life-span-est} and \eqref{u_x-Linfty-Hs}.  We begin by mollifying the HR equation, so that we may apply the following ODE
theorem, taken from Dieudonn\`e \cite{Dieudonne:1969}: 
%
\begin{theorem}
\label{ode_theorem}
Let  $Y$  be a Banach space, $X\subset Y$ be an open subset,
$J \subset \rr$, and $f: J \times X\to Y$ a continuously differentiable
map.  Then for any $t_{0} \in J$ and $x_{0} \in X$ there exists an
open ball $I \subset J$ and a unique differentiable mapping $u:I
\to Y$ such that for all $t \in I$,  $u'(t) = f(t, u)$
and $u(t_{0}) = x_{0}$.
\end{theorem}
%
To see why we cannot apply the Banach Space ODE Theorem to the HR equation as
is, let $u=x^{-1/2} \chi_{[0,1]}$. Then $u
\in L^{2}$ but $u\p_x u \notin L^{2}$. Hence, returning to the general case, we see
that the HR equation as is can not be thought as an ODE on the space $H^s$. To
deal with this problem we will replace the i.v.p \eqref{hr}--\eqref{hr-data} by  
\begin{gather}
\label{hr-moli}
\p_t  u_\ee =
-\gamma J_\ee u_\ee \partial_x  J_\ee  u_\ee - \p_x (1-\p_x^2)^{-1} 
\left[\frac{3-\gamma}{2}u^2 + \frac{\gamma}{2}(\p_x u)^2 \right],
\\
\label{hr-moli-data} 
u_\ee(x, 0) = u_0 (x),
\end{gather}
%
where $J_\ee$ is defined as follows: Pick a non-negative $j(x) \in
\mathcal{S}(\rr)$ and let
\begin{equation*}
\begin{split}
j_\ee(x) = \frac{1}{\ee}j\left( \frac{x}{\ee} \right).
\end{split}
\end{equation*}
We then define $J_\ee$ to be the ``Friedrichs mollifier''
\begin{equation}
\begin{split}
J_\ee f(x) = j_\ee * f(x), \quad \ee>0.
\end{split}
\end{equation}
%
%
Notice that the map $f \mapsto J_{\ee} f$ is a bounded map from $H^s(\ci)$
to $H^s(\ci)$.  In order to apply the ODE Theorem, we will also need to
show that it is a continuously differentiable map:
%
%
%
\begin{lemma}
Let $f_\ee:H^s(\ci) \to H^s(\ci)$ be given by 
\begin{equation}
\label{f_ep}
f_{\ee}(u) = -\gamma  J_\varepsilon u \partial_x J_\varepsilon u
- \p_x (1-\p_x^2)^{-1} \left
[\frac{3-\gamma}{2}u^2 + \frac{\gamma}{2}(\p_x u)^2 \right ].
\end{equation}
Then $f_\ee$  is a continuously differentiable map.
\end{lemma}
%
%
\begin{proof} We explicitly calculate the derivative of $f_\ee$ at an
arbitrary $w \in H^s(\ci)$:
\begin{equation*}
\begin{split}
[Df_{\ee}(u)](w)
=
& -\gamma (J_\varepsilon w \cdot \partial_x J_\varepsilon u +
J_\varepsilon u \cdot \partial_x J_\varepsilon w)
\\
& - (1-\p_x ^2)^{-1}
\p_x \left [(3-\gamma)w u + \gamma\p_x w \p_x u \right ].
\end{split}
\end{equation*}
Let $w_n \xrightarrow{H^s(\ci)} w$. Then it is easy to check that
%
\begin{equation}
\begin{split}
& -\gamma (J_\varepsilon w_n \cdot \partial_x J_\varepsilon u 
+ J_\varepsilon u \cdot \partial_x J_\varepsilon w_n)
+ (1-\p_x ^2)^{-1}
\p_x \left [(3-\gamma)w_n u + \gamma\p_x w_n \p_x u \right ]
\\
& \xrightarrow{H^s(\ci)} 
-\gamma (J_\varepsilon w \cdot \partial_x J_\varepsilon u 
+ J_\varepsilon u \cdot \partial_x J_\varepsilon w) + (1-\p_x ^2)^{-1}
\p_x \left [(3-\gamma)w u + \gamma\p_x w \p_x u \right ].
\end{split}
\end{equation}
This concludes the proof. 
\end{proof}
Hence, by \cref{ode_theorem}, for each $\ee > 0$ there exists a
unique solution $u_\ee \in C(I, H^s(\ci))$ satisfying the Cauchy-problem
\eqref{hr-moli}-\eqref{hr-moli-data}. Next, we analyze the size and
lifespan of the family $\{u_\ee\}$ of solutions.
%%%%%%%%%%%%%%%%%%%%%%%%
%
%     Estimates  for Life-span and Sobolev norm of $u_\ee$
%
%%%%%%%%%%%%%%%%%%%%%%%%
%
%
\subsection{Estimates  for Life-span and Sobolev norm of $u_\ee$.}
%
We will show that there is a lower bound  $T$
for $T_\ee$, which is  independent of $\ee\in(0, 1]$.
This is based on the following differential
inequality for the solution $u_\ee$:
%
\begin{equation} 
\label{B-diff-ineq}
\frac 12
\frac{d}{dt}
\|u_\ee(t)\|_{H^{s}(\ci)}^2
\le
c_s
\|u_\ee(t)\|_{H^{s}(\ci)}^3,
\quad
|t| \le T_\ee.
\end{equation}
%
%
We will prove this inequality  by
following the approach used for quasilinear symmetric
hyperbolic systems in Taylor \cite{Taylor:1991}. In what follows we will suppress the
$t$ parameter for the sake of clarity.
%
For any $s\in \ci$ let   $D^s=(1-\p_x^2)^{s/2}$ be the  operator
defined by 
%
$$ \widehat{D^s f}(\xi) \doteq (1 + \xi^2)^{s/2} \widehat{f}(\xi), $$
%
where $ \widehat{f}$ is the Fourier transform
%
$$ \widehat{f}(\xi) =  \int_{\ci} e^{-i \xi x} f(x) \ dx.  $$
%
Applying the operator $D^s$ to  both sides of  \eqref{hr-moli},
then  multiplying the resulting equation by $D^s J_\ee u_\ee$
and integrating it for $x\in\ci$ gives
%
\begin{equation} 
\begin{split}
\label{B-moli-int}
\frac 12
\frac{d}{dt} \|u_\ee \|_{H^s}^2
=
&-
\gamma \int_{\ci}  D^s(J_\ee u_\ee \partial_x J_\ee u_\ee) \cdot
D^s J_\ee u_\ee  \  dx
\\
&- \frac{3 -\gamma}{2} \int_{\ci} D^{s-2} \p_x (u_{\ee}^2) 
\cdot D^s J_\ee u_{\ee} \ dx
\\
&- \frac{\gamma}{2} \int_{\ci}  D^{s-2} \p_x (\p_x u_\ee)^2
\cdot D^s J_\ee u_\ee  \ dx.
\end{split}
\end{equation}
%
We will estimate the right hand side of \eqref{B-moli-int} in parts. In
what follows next we use the fact that  $D^s$ and $J_\ee$ commute and
that  $J_\ee$ satisfies the properties 
%
\begin{equation} 
\label{J-e-inner-prod-property}
(J_\ee f, g)_{L^2(\ci)}=( f, J_\ee g)_{L^2(\ci)}
\end{equation}
%
and
%
\begin{equation} 
\label{Je-u-Hs}
\| J_\ee u \|_{H^s(\ci)}
\le
\|  u \|_{H^s(\ci)}.
\end{equation}
%
%%%%%%%%%%%% Burgers term energy estimate %%%%%%%%%%%%
%
%
%
\noindent
Letting 
%
\begin{equation} 
\label{v-Je-ue}
v=J_\ee u_\ee
\end{equation}
%
%
we have
%
\begin{equation} 
\begin{split}
\label{B-moli-int-v}
& -  \gamma \int_{\ci}   D^s (J_{\ee} u_{\ee} \p_x J_\ee u_\ee)
\cdot D^s
J_{\ee}u_\ee \ dx  
\\
& = - \gamma \int_\ci
D^s(v \partial_x v) \cdot   D^s v \ dx
\\
& = - \gamma \int_\ci
\left [ 
D^s(v\p_x v)  -  v D^s (\p_xv)
\right ] 
D^s v \ dx - \gamma \int_\ci
v D^s (\p_xv)
D^s v \ dx.
\end{split}
\end{equation}
%
%
%
We now estimate \eqref{B-moli-int-v} in parts. Applying the Cauchy-Schwarz inequality gives
%
\begin{equation} 
\label{int1-est-calc2}
\begin{split}
& \Big|
- \gamma \int_\ci
\big[ 
D^s(v\p_x v)  -  v D^s (\p_xv)
\big]
D^s v   \, dx
\Big|
\\
& \lesssim
\cdot \|
D^s(v\p_x v)  -  v D^s (\p_xv)
\|_{L^2(\ci)}
\|
D^s v 
\|_{L^2(\ci)}
\\
& = 
  \|
D^s(v\p_x v)  -  v D^s (\p_xv)
\|_{L^2(\ci)}
\|
v
\|_{H^s(\ci)}
\\
&\le c_s \| \p_x v \|_{L^\infty(\ci)} 
\| v \|_{H^s(\ci)}^2,
\end{split}
\end{equation}
%
where the last step follows from 
%
\begin{equation} 
\label{int1-est-calc3}
\| D^s(v\p_x v)  -  v D^s (\p_xv) \|_{L^2(\ci)}
\le
c_s    \| \p_x v \|_{L^\infty(\ci)} 
\| v \|_{H^s(\ci)},
\end{equation}
which we prove below by using the following Kato-Ponce commutator 
estimate:  
\begin{lemma} 
\label{KP-lemma}
[Kato-Ponce]
If  $s>0$ then there is $c_s>0$ such that 
%
\begin{equation} 
\label{KP-com-est}
\| D^{s} \big(fg) -  f D^s g\|_{L^2(\ci)}
\le
c_s\big(
\| D^{s}f \|_{L^2(\ci)}    \| g \|_{L^\infty(\ci)} 
+
\| \p_xf \|_{L^\infty(\ci)}    \| D^{s-1}g \|_{L^2(\ci)}   
\big).
\end{equation}
%
\end{lemma}
%
%
In fact, applying  this estimate with $f=v$ and $g=\p_xv$ gives 
%
\begin{equation} 
\label{int1-est-calc4}
\begin{split}
& \| D^s(v\p_x v)  -  v D^s (\p_xv) \|_{L^2(\ci)}
\\
& \le
{c_s} \big(
\| D^{s}v \|_{L^2(\ci)}    \| \p_x v \|_{L^\infty(\ci)} 
+
\| \p_xv \|_{L^\infty(\ci)}    \| D^{s-1}\p_x v \|_{L^2(\ci)}   
\big)
\\
& \lesssim {c_s}    \| \p_x v \|_{L^\infty(\ci)} 
\| v \|_{H^s(\ci)}, 
\end{split}
\end{equation}
%
which  is the desired estimate  \eqref{int1-est-calc3}.
Next, we have
%
%
%
\begin{equation} 
\label{int1-est-calc5}
\begin{split}
\Big|
-\gamma \int_\ci
v D^s (\p_x v)
\cdot  D^s v \ dx
\Big|
& \simeq 
 \Big|
\int_\ci
v \p_x\left(D^s v\right)^2  dx
\Big|
\\
& \simeq
\Big | \int_\ci
\p_x v \, (D^s v)^2 \ dx
\Big|
\\
& \le
\int_\ci
\Big | \p_x v \, (D^s v)^2   
\Big| \ dx
\\
& \le
\| \p_x v \|_{L^\infty(\ci)} 
\| v \|_{H^s(\ci)}^2.
\end{split}
\end{equation}
%
%
%
Combining inequalities  \eqref{int1-est-calc2} and
\eqref{int1-est-calc5} and applying the Sobolev Imbedding Theorem, we
have
%
\begin{equation} 
\label{burgers_est'}
\begin{split}
\Big|
-\gamma \int_\ci
D^s(v \partial_x v) \cdot   D^s v \, dx  
\Big|
&\lesssim_{s}
\| \p_x v \|_{L^\infty(\ci)} 
\|  v \|_{H^s(\ci)}^2
\\
& \le  \| v \|_{C^1(\ci)} \| v \|_{H^s(\ci)}^2
\\
& \lesssim_{s}  \| v \|_{H^s(\ci)}^3
\\
& \le  \| u_\ee \|_{H^s(\ci)}^3.
\end{split}
\end{equation}
%
Next we estimate
\begin{equation}
\begin{split}
\left | - \frac{3 -\gamma}{2} \int_\ci D^{s-2} \p_x u_\ee^2 \cdot
D^s J_\ee u_\ee \; \right | dx 
& \lesssim  \int_\ci \left |
D^{s-2} \p_x u_\ee^2 \cdot D^s J_\ee u_\ee \; dx \right | 
\\
& \le 
\|D^{s-2} \p_x u_\ee^2 \|_{L^2(\ci)} 
\|D^s J_\ee u_\ee \|_{L^2(\ci)}
\\
& \le 
\|D^{s-1} u_\ee^2 \|_{L^2(\ci)} 
\|D^s u_\ee \|_{L^2(\ci)}
\\
& \lesssim \| u_\ee^2 \|_{H^s(\ci)} \| u_\ee \|_{H^s(\ci)}.
\end{split}
\end{equation}
%
%
Applying the algebra property, we obtain
%
\begin{equation}
\label{hl1}
\begin{split}
\left | - \frac{3 -\gamma}{2} \int_\ci D^{s-2} \p_x u_\ee^2 \cdot
D^s J_\ee u_\ee \; dx \right |
\lesssim_{s} \| u_\ee \|_{H^s(\ci)}^3.
\end{split}
\end{equation}
%
%
We also have
\begin{equation}
\begin{split}
\left |- \frac{\gamma}{2} \int_\ci D^{s-2} \p_x (\p_x u_\ee)^2 \cdot
D^s J_\ee u_\ee \; dx \right |
& \lesssim  \int_\ci \left | D^{s-2} \p_x (\p_x u_\ee)^2 \right |
\cdot \left |D^s J_\ee u_\ee \right | \; dx
\\
& \le 
\| D^{s-1} (\p_x u_\ee)^2 \|_{L^2(\ci)}
\| D^s J_\ee u_\ee \|_{L^2(\ci)}
\\
& \lesssim \|(\p_x u_\ee)^2 \|_{H^{s-1}(\ci)}
\| J_\ee u_\ee \|_{H^{s-1}(\ci)} 
\\
& \lesssim \|(\p_x u_\ee)^2 \|_{H^{s-1}(\ci)} \| u_\ee \|_{H^{s-1}(\ci)} 
\end{split}
\end{equation}
and applying the algebra property yields
\begin{equation}
\label{hl2}
\begin{split}
\left | - \frac{\gamma}{2} \int_\ci D^{s-2} (\p_x u_\ee)^2 \cdot
D^s J_\ee u_\ee \; dx \right |
& \lesssim_{s} \| \p_x u_\ee \|_{H^{s-1}(\ci)}^2 \| u_\ee \|_{H^s(\ci)} 
\\
& \le \|u_\ee\|_{H^s(\ci)}^3.
\end{split}
\end{equation}
%
Combining \eqref{burgers_est'}, \eqref{hl1}, and \eqref{hl2}, we obtain
\eqref{B-diff-ineq}.
%%%%%%%%%%%%%%%%%%%%%%%%%%%%%%%%%%%
%  
%           Lifespan for CH  solution    
% 
%%%%%%%%%%%%%%%%%%%%%%%%%%%%%%%%%%%
%
%
%   
%
\noindent
\subsection{Lifespan estimate of $u_\ee$.} To derive an explicit formula for
$T_\ee$ we proceed as follows.  Letting  $y(t)=
\|u_\ee(t)\|_{H^s(\ci)}^2$ inequality  \eqref{B-diff-ineq} takes the
form
%
\begin{equation} 
\label{energy-y-ineq}
\frac 12
y^{-3/2}\frac{dy}{dt}
\le
c_s,
\qquad
y(0)=y_0=  \|u_0\|_{H^s(\ci)}^2.
\end{equation}
%
Suppose $t$ is non-negative. Integrating  \eqref{energy-y-ineq} from  0  to $t$ gives
%
\begin{equation} 
\label{energy-y-ineq-calc1}
\frac{1}{\sqrt{y_0}}  - \frac{1}{\sqrt{y(t)}} 
\le
c_s t.
\end{equation}
%
%
Replacing $y(t)$ with   $\|u_\ee(t)\|_{H^s(\ci)}^2$  and solving for  $\|u_\ee(t)\|_{H^s(\ci)}$
we obtain the formula
%
\begin{equation} 
\label{norm-u(t)-formula}
\|u_\ee(t)\|_{H^s(\ci)}
\le
\frac{\|u_0\|_{H^s(\ci)}}{1-c_s\|u_0\|_{H^s(\ci)} t}, \quad t\ge
0.
\end{equation}
%
Now, from \eqref{norm-u(t)-formula} we see that  $\|u_\ee(t)\|_{H^s(\ci)}$ is finite  if 
%
\begin{equation*} 
\label{Lifespan-calc1}
c_s    \|u_0\|_{H^s(\ci)} t<1,
\end{equation*}
%
or
%
\begin{equation} 
t
<
\frac{1}{c_s \|u_0\|_{H^s(\ci)}}.
\end{equation}
%
Similarly, 
by integrating  \eqref{energy-y-ineq} from  $-t$ to $0$ gives
\begin{equation} 
\label{norm-u(t)-formula-prime}
\|u_\ee(t)\|_{H^s(\ci)}
\le
\frac{\|u_0\|_{H^s(\ci)}}{1+c_s\|u_0\|_{H^s(\ci)} t}, \quad t \le 0
\end{equation}
from which it follows that $\|u_\ee(t)\|_{H^s(\ci)}$ is finite  if 
%
\begin{equation} 
t
>
\frac{-1}{c_s \|u_0\|_{H^s(\ci)}}.
\end{equation}
Therefore, the  solution  $u_\ee(t)$ to the mollified CH Cauchy
problem exists for $|t| <T_0$, where
%
\begin{equation} 
\label{CH-Lifespan}
T_0
=
\frac{1}{c_s \|u_0\|_{H^s(\ci)}}.
\end{equation}
%
%%%%%%%%%%%%%%%%%%%%%%%%%%%%%%%%%%%
%  
%            Norm of   
% 
%%%%%%%%%%%%%%%%%%%%%%%%%%%%%%%%%%%
%
%
%   
%
\noindent
\subsection{Size of the solution estimate} If we choose  $T=\frac12 T_0$, that is
%
\begin{equation} 
\label{T-def}
T
=
\frac{1}{2 c_s \|u_0\|_{H^s(\ci)}},
\end{equation}
%
then for $|t| \le T$, estimates \eqref{norm-u(t)-formula} and
\eqref{norm-u(t)-formula-prime} imply 
%
\begin{equation*} 
\label{u(t)-u(0)-bound}
\|u_\ee(t)\|_{H^s(\ci)}
\le
\frac{\|u_0\|_{H^s(\ci)}}{1-(c_s\|u_0\|_{H^s(\ci)})/(2 c_s \|u_0\|_{H^s(\ci)})},
\end{equation*}
%
or 
%
\begin{equation} 
\|u_\ee(t)\|_{H^s(\ci)}
\le
2 \|u_0\|_{H^s(\ci)},
\quad 
|t| \le T.
\end{equation}
%
Thus we have obtained a lower bound for $T_\ee$ and an upper bound for
$\|u_\ee(t)\|_{H^s(\ci)}$ independent of $\ee\in (0, 1]$. The following
lemma summarizes these results and provides an estimate for the
$H^{s-1}(\ci)$ norm of $\p_t u_\ee(t)$:
%
%
\begin{lemma}
\label{hr_wp}
Let  $u_0(x) \in  H^s(\ci)$, $s >3/2$. Then for any $\ee\in (0, 1]$
the i.v.p.\ for the mollified HR equation 
%
\begin{equation} 
\label{hr-moli-2}
\partial_t  u_\ee 
=
-\gamma (J_\ee u_\ee \partial_x  J_\ee  u_\ee) - \p_x (1-\p_x^2)^{-1} \left
[\frac{3-\gamma}{2}(u_\ee)^2 + \frac{\gamma}{2}(\p_x u_\ee)^2
\right ], 
\end{equation} 
%
\begin{equation} 
\label{burgers-moli-data-2} 
u_\ee(x, 0) = u_0 (x),
\end{equation}
%
has a unique solution $u_\ee( t)\in C([-T, T]; H^s(\ci))$. 
In particular,
%
\begin{equation} 
\label{life-est}
T
=
\frac{1}{2 c_s \|u_0\|_{H^s(\ci)}}
\end{equation}
%
is independent of $\ee$ and
is a lower bound for the lifespan of $u_\ee( t)$ and
%
\begin{equation}
\label{u-e-Hs-bound}
\|u_\ee(t)\|_{H^s(\ci)}
\le
2 \|u_0 \|_{H^s(\ci)},
\quad
|t| \le T.
\end{equation}
%
Furthermore,  $u_\ee( t)\in C^1([T, T]; H^{s-1}(\ci))$ and 
satisfies
\begin{equation}
\label{dt-u-e-Hs-bound}
\|\p_t u_\ee(t)\|_{H^{s-1}(\ci)}
\lesssim
\|u_0 \|_{H^s(\ci)}^2,
\quad
|t| \le T.
\end{equation}
% 
Here  $c_s$ is a constant depending only on $s$.
\end{lemma}
%
%
\begin{proof} It suffices to prove  \eqref{dt-u-e-Hs-bound}.
Using equation \eqref{hr-moli-2}, for any $t\in [-T, T]$ we have
%
\begin{equation*}
\begin{split}
& \| \partial_t u_\varepsilon(t) \|_{H^{s-1}(\ci)}  
\\
& = 
\| -\gamma (J_\ee u_\ee \partial_x  J_\ee  u_\ee) -
\p_x (1-\p_x^2)^{-1} \left [\frac{3-\gamma}{2} (u_\ee)^2 +
\frac{\gamma}{2}(\p_x u_\ee)^2 \right ] \|_{H^{s-1}(\ci)}
\\
& \lesssim  
\| J_\ee u_\ee \partial_x  J_\ee  u_\ee \|_{H^{s-1}(\ci)}
+ \|\p_x (1-\p_x^2)^{-1} (u_\ee)^2 \|_{H^{s-1}(\ci)}
\\
& + \| \p_x (1-\p_x^2)^{-1}(\p_x u_\ee)^2\|_{H^{s-1}(\ci)}.
\end{split}
\end{equation*}
We break this into three parts:
\begin{equation}
\label{bixi}
\begin{split}
\| J_\ee u_\ee \p_x J_\ee u_\ee \|_{H^{s-1}(\ci)}
& = 
\frac{1}{2}\|\p_x[(J_\varepsilon u_\varepsilon
)^2]\|_{H^{s-1}(\ci)}
\\
& \lesssim \|(J_\varepsilon u_\varepsilon )^2\|_{H^s(\ci)}.
\end{split}
\end{equation}
Applying the algebra property of Sobolev spaces and estimate
\eqref{u-e-Hs-bound} to \eqref{bixi} gives 
%
\begin{equation}
\label{deriv1}
\begin{split}
\|J_\ee u_\ee \p_x J_\ee u_\ee  
\|_{H^{s-1}(\ci)}
& \lesssim
\|J_\varepsilon u_\varepsilon \|_{H^s(\ci)}^2
\\
&\lesssim
\| u_\varepsilon \|_{H^s(\ci)}^2
\\
&\lesssim
\|u_0\|_{H^s(\ci)}^2.
\end{split}
\end{equation}
We also have
\begin{equation*}
\begin{split}
\|\p_x (1-\p_x^2)^{-1} (u_\ee)^2\|_{H^{s-1}(\ci)}
& \le \| (u_\ee)^2\|_{H^{s-1}(\ci)}
\end{split}
\end{equation*}
which by the algebra property and estimate \eqref{u-e-Hs-bound}
gives
\begin{equation}
\begin{split}
\label{deriv2}
\|\p_x (1-\p_x^2)^{-1} (u_\ee)^2\|_{H^{s-1}(\ci)}
& \lesssim \|u_\ee\|^2_{H^s(\ci)} 
\\
& \lesssim  \|u_0\|^2_{H^s(\ci)}.
\end{split}
\end{equation}
Similarly,
\begin{equation}
\begin{split}
\label{deriv3}
\|\p_x (1-\p_x^2)^{-1} (\p_x u_\ee)^2\|_{H^{s-1}(\ci)}
& \lesssim \|\p_x u_\ee\|^2_{H^{s-1}(\ci)} 
\\
& \lesssim  \|u_\ee \|^2_{H^s(\ci)}
\\
& \lesssim \|u_0\|^2_{H^s(\ci)}.
\end{split}
\end{equation}
Combining \eqref{deriv1}, \eqref{deriv2}, and \eqref{deriv3}, we
obtain \eqref{dt-u-e-Hs-bound}. 
\end{proof}
%%%%%%%%%%%%%%%%%%%%%%%%
%
%     Choosing  a convergent subsequence
%
%%%%%%%%%%%%%%%%%%%%%%%%
\subsection{Choosing  a convergent subsequence.}
%
Next we shall show that  the family $\{ u_\ee\}$ has a convergent subsequence
whose limit $u$ solves the hyperelastic rod i.v.p. 
Let
$$
I= [-T, T].
$$
By \cref{hr_wp} we have 
%
\begin{equation}
\label{C-1-fam}
\{u_\ee\}\subset C(I, H^s(\ci))\cap C^1(I, H^{s-1}(\ci))
\end{equation}
%
and bounded. Since $I$ is compact, we have  
%
\begin{equation}
\label{Lip-1-fam}
\{u_\ee\}\subset L^{\infty}(I, H^s(\ci))\cap C^1(I,
H^{s-1}(\ci)).
\end{equation}
%
Now, by the Riesz Lemma, we can identify $H^s(\rr)$ with
$(H^s(\rr))^*$, where for $w, \psi \in H^s(\rr)$ the duality is
defined by 
\begin{equation*}
T_w(\psi) = <w, \psi>_{H^s(\rr)}.
\end{equation*}
Hence, by the Riesz Representation Theorem it follows that we can
identify \\ $L^\infty(I, H^s(\ci)) $ with the dual space of $L^1(I,
H^{s}(\ci)$, where for $v\in L^\infty(I, H^s(\ci)) $ and $ \phi \in
L^1(I, H^{s}(\ci))$ the duality is defined by  
%
\begin{equation}
T_v(\phi) = \int_I <v (t), \phi (t)>_{H^s(\rr)} dt  = \int_I
\int_{\rr}
\widehat{v}(\xi, t) \overline{\widehat{\phi}}(\xi, t) \cdot (1
+ \xi^2)^s \ d \xi dt.
\end{equation}
%
Next, we recall Aloaglu's Theorem:
\begin{theorem}
If $X$ is a normed vector space,
the closed unit ball $B^* = \{f \in X^* : \|f\| \le
1\}$ in $X^*$ is compact in the $weak^*$ topology.
\end{theorem}
Therefore the bounded family $\{u_\ee\}$ is compact 
in the weak$^*$ topology of \\
$L^\infty(I, H^s(\ci))$. More precisely,
there is a sequence  $\{ u_{\ee_n} \}$ converging
weakly to a $ u\in L^{\infty}(I, H^s(\ci))$.
That is,
%
\begin{equation}
\label{weak-conv}
\lim_{n\to \infty} T_{u_{\ee_n}}(\phi)  =  T_u (\phi) 
\; \; \ 		
\text{for all} \ \;\;  \phi \in L^1(I, H^{s}(\ci)).
\end{equation}
%
In order to show that  $u$ solves the HR i.v.p.\ we need to 
obtain a stronger  convergence for  $u_{\ee_n}$ so that 
we can take the limit in the mollified HR equation.
In fact we will prove that 
%
\begin{equation}
\label{strong-conv}
u_{\ee_n}\longrightarrow u
\quad
 \ \text{in} \ \,\,   C(I, H^{s-\sigma}(\ci)),\ \text{for any} \
\, 0 < \sigma <
1.
\end{equation}
%
For this we will need the following interpolation  result:
%%%%%%%%%%%%%%%%%%%%%%%%%%%
%
%
%                 Interpolation Lemma
%
%
%%%%%%%%%%%%%%%%%%%%%%%%%%%
\begin{lemma}
\label{interpolation-lem}
(Interpolation)     Let  $s > \frac{3}{2}$.
If $v \in C(I, H^s(\ci)) \cap C^1(I, H^{s-1}(\ci))$
then $v \in C^\sigma (I, H^{s- \sigma}(\ci))$ for  $0 < \sigma < 1$.
\end{lemma}
%
\begin{proof} We have
\begin{equation*}
\begin{split}
& \frac{\|v(t) - v(t')\|^2_{H^{s - \sigma}}}{|t - t'|^{2\sigma}}
\\
& = 
\sum_{\xi \in \zz} (1 + \xi^2)^{s- \sigma} 
\frac{|\hat{v}(\xi, t) - \hat{v}(\xi, t')|^2}{|t-t'|^{2\sigma}} d\xi\\
& = \sum_{\xi \in \zz} (1+\xi^2)^s 
\bigg(\frac{1}{(1+ \xi^2)|t - t'|^2} \bigg)^\sigma |\hat{v}(\xi, t)- \hat{v}(\xi, t')|^2 d\xi\\
& \leq \sum_{\xi \in \zz}(1+\xi^2)^s \bigg( 1 + \frac{1}{(1+\xi^2)|t-t'|^2} \bigg)
|\hat{v}(\xi,t) - \hat{v}(\xi,t')|^2 d\xi \\
& \leq \sum_{\xi \in \zz} (1+ \xi^2)^s |\hat{v}(\xi, t)- \hat{v}(\xi, t')|^2 d\xi
+ \sum_{\xi \in \zz} (1+ \xi^2)^{s-1} \frac{|\hat{v}(\xi, t) - \hat{v} (\xi, t')|^2}{|t-t'|^2} \\
& \leq  \sup_{t \in I} \|v(t)\|_{H^s(\ci)}^2 + \sup_{t \in I}
\| \partial_t v(t) \|_{H^{s-1}(\ci)}^2
\\
& < \infty
\end{split}
\end{equation*}
%
which completes the proof.
\end{proof}
%
Next, using this lemma we will show that the family $\{u_\ee\}$ is
equicontinuous in $C(I, H^{s-\sigma}(\ci))$, $0 < \sigma < 1$. We
will follow this by proving that there exists a sub-family
$\{u_{\ee_n} \}$ that is precompact in $C(I,
H^{s-\sigma}(\ci))$. These two facts, in conjunction with Ascoli's
Theorem, will yield
\begin{equation}
\label{strong-conv2}
u_{\ee_n} \to u \; \; \text{in} \; \; C(I,H^{s-\sigma}(\ci)),
\quad
0 < \sigma < 1.
\end{equation}
%%%%%%%%%%%%%%%%%%%%%%
%
%
%       Equicontinuity
%
%
%%%%%%%%%%%%%%%%%%%%%%
%
\subsection{Equicontinuity of $\{u_\ee\}_\ee$  in
$C(I,H^{s-\sigma}(\ci))$.} Applying  \cref{interpolation-lem} gives 
%
\begin{equation}
\label{equic-1}
\sup_{t \neq t'} \frac {\|u_\ee(t) - u_\ee(t') \|_{H^{s -
\sigma}(\ci)}}{|t - t'|^\sigma} < \infty
\end{equation}
%
or
%
\begin{equation}
\label{equic-2}
\|u_\ee(t) - u_\ee(t') \|_{H^{s - \sigma}(\ci)}< c|t - t'|^\sigma, 
 \ \text{for all} \  \,\,  t, t'\in I,
\end{equation}
%
which shows that  the family  $\{u_\ee\}$ is equicontinuous in 
$C(I, H^{s-\sigma}(\ci))$.
%
%
%%%%%%%%%%%%%%%%%%%%%%
%
%
%      PreCompactness
%
%
%%%%%%%%%%%%%%%%%%%%%%%%%%
%
%
%
%
%		
\subsection{Precompactness of $\{u_\ee(t)\}$ in $H^{s-\sigma}(\ci))$.}
Now recall that
\begin{equation}
\label{compact-1}
\|u_\ee(t)\|_{H^{s}(\ci)}
\le
2 \|u_0 \|_{H^s(\ci)}, \,
\quad
t\in I.
\end{equation}
%
By Kondrachov's Theorem, the inclusion $H^s(\ci) \subset H^{s-
\sigma }(\ci)$ is compact. By \eqref{compact-1},
it follows that $\{u_\ee(t)\}$ is precompact in $H^{s-\sigma}(\ci)$.
%
%
%
%
We are now in a position to apply Ascoli's Theorem: 
\begin{theorem}
\label{Ascoli}
(Ascoli)  Let $X$ be a Banach space, $I$ be a compact metric space,
and $C(I,X)$  be the set of continuous functions $f: I\longrightarrow X$.
Suppose $S \subset C(I,X)$  has the following properties:
%
\begin{itemize}
\item[(1)]   $S$ is  equicontinuous.
\item[(2)]  For each $x \in M$ that the set $S(x) = \{f(x)\}$  is  precompact in $X$.
\end{itemize} 
%
Then $S$  is  precompact  in  $C(I,X)$.
\end{theorem}
Compiling our previous results on equicontinuity and precompactness
and applying \cref{Ascoli}, we
conclude that there exists a subfamily $\left\{ u_{\ee_n} \right\}$
such that
\begin{equation}
\label{strong-conv-of-u_ep}
u_{\ee_n} \to u \; \; \text{in} \; \; C(I, H^{s-\sigma}(\ci)).
\end{equation}
%
%
%
%%%%%%%%%%%%%%%%%%%%%%%%%%%%%%%%%
%
%
%     Verifying that the limit $u$ solves Burgers equation
%
%
%%%%%%%%%%%%%%%%%%%%%%%%%%%%%%%%%
\subsection{Verifying that the limit $u$ solves the HR equation.} 
We shall need the following. 
\begin{proposition}
\label{prop:1aa}
\begin{align}
    \label{tri}
& J_{\ee_n} u_{\ee_n} \to  u \ \ \text{in} \ \
C(I, H^{s-\sigma}(\ci)),
\\
\label{0dd}
& J_{\ee_n} \p_x u_{\ee_n} \to  \p_x u \ \
\text{in} \ \ C(I, H^{s-\sigma-1}(\ci)).
\end{align}
\end{proposition}
\begin{proof} Note that
\begin{equation}
\begin{split}
& \| u -  J_{\ee_n} u_{\ee_n}
\|_{C(I, H^{s-\sigma}(\ci))}
\\
&= \| u -  J_{\ee_n} u_{\ee_n} \pm 
u_{\ee_n} \|_{C(I, H^{s-\sigma}(\ci))}
\\
& = \| u -  u_{\ee_n}
\|_{C(I,H^{s-\sigma}(\ci))} + \| (I - J_{\ee_n})
u_{\ee_n} \|_{C(I, H^{s-\sigma}(\ci))}.
\label{1bb}
\end{split}
\end{equation}
Applying the estimates
\begin{equation*}
\begin{split}
& \|I-J_{\ee_n} \|_{L(H^{s-\sigma}(\ci), H^{s -
\sigma}(\ci))} = o(1),
\\
& \|u_{\ee_n}\|_{H^{s-\sigma}(\ci)} \le 2
\|u_0\|_{H^{s-\sigma}(\ci)}
\end{split}
\end{equation*}
to \eqref{1bb} gives
\begin{equation}
\label{2bb}
\begin{split}
\| u -  J_{\ee_n} u_{\ee_n}\|_{H^{s-\sigma}(\ci)}
\le \left( \| u -  u_{\ee_n}
\|_{C(I, H^{s-\sigma}(\ci))} + o(1) \cdot \|u_0
\|_{H^{s-\sigma}(\ci)} \right).
\end{split}
\end{equation}
Letting $\ee_n \to 0$ in \eqref{2bb} and applying
\eqref{strong-conv-of-u_ep} gives \eqref{tri}. Furthermore, note that
%
%
\begin{equation*}
\begin{split}
\|\p_x u - J_\ee \p_x u_{\ee_n} \|_{C(I,
H^{s-\sigma-1}))}  
& = \|\p_x u - \p_x J_\ee u_{\ee_n} \|_{C(I,
H^{s-\sigma-1}(\ci))} 
\\
& \le \|u - J_\ee u_{\ee_n} \|_{C(I,
H^{s-\sigma}(\ci))}.
\end{split}
\end{equation*}
Applying \eqref{tri} completes the proof of \cref{prop:1aa}. 
\end{proof}
%
Observe that \cref{prop:1aa} implies
\begin{equation}
\begin{split}
\label{burgers_and_nonlocal_conv}
&  J_{\varepsilon_n} u_{\varepsilon_n} 
\cdot J_{\varepsilon_n} \p_x u_{\varepsilon_n} 
\to  u \partial_x u \; \; 
\text{in} \; \;
C(I, H^{s-\sigma-1}(\ci)). 
\end{split}
\end{equation}
%
Furthermore, since $\|\p_x (1-\p_x^2)^{-1}\|_{L(H^s(\ci), H^s(\ci))}
\le 1$ for all $s \in \rr$, it follows immediately from
\eqref{strong-conv-of-u_ep} that
\begin{equation}
\begin{split}
& \p_x(1- \p_x^2)^{-1} \left( \frac{3-\gamma}{2}
(u_{\ee_n})^2
+ \frac{\gamma}{2} (\p_x u_{\ee_n})^2 \right )
\\
& \to
\p_x(1- \p_x^2)^{-1} \left( \frac{3-\gamma}{2} u^2
+ \frac{\gamma}{2} (\p_x u)^2 \right ) \ \
\text{in} \ \ C(I, H^{s-\sigma-1}(\ci)).
\label{non-local-convergence}
\end{split}
\end{equation}
Combining \eqref{burgers_and_nonlocal_conv} and
\eqref{non-local-convergence}, and applying the Sobolev Imbedding
Theorem, we deduce 
\begin{equation}
\begin{split}
& -\gamma (J_{\ee_n} u_{\ee_n} \cdot J_{\ee_n} \p_x
u_{\ee_n}) - \p_x(1- \p_x^2)^{-1} \left( \frac{3-\gamma}{2}
(u_{\ee_n})^2
+ \frac{\gamma}{2} (\p_x u_{\ee_n})^2 \right )
\\
\to & -\gamma u \p_x u -
\p_x(1- \p_x^2)^{-1} \left( \frac{3-\gamma}{2} u^2
+ \frac{\gamma}{2} (\p_x u)^2 \right ) \ \
\text{in} \ \ C(I, C(\ci)).
\label{loc-non-loc-tog}
\end{split}
\end{equation}
Furthermore, we note that the convergence  
%
\begin{equation}
\label{weak-conv-2}
T_{u_{\ee_n}}(\phi)  \longrightarrow  T_u(\phi) \;
\ \text{for all} \  \;  \phi \in L^1(I, H^{s}(\ci))
\end{equation}
%
implies
%
\begin{equation}
u_{\ee_n}  \longrightarrow  u
\quad
\ \text{in} \   \,\,
\mathcal{D}'(I\times \ci)
\end{equation}
%
and so
%
\begin{equation}
\label{distib-conv-2}
\p_tu_{\ee_n}  \longrightarrow  \p_tu
\quad
\ \text{in} \   \,\, \mathcal{D}'(I\times \ci).
\end{equation}
%
Since for all $n$ we have 
%
\begin{equation}
\p_tu_{\ee_n} 
=
-\gamma (J_{\varepsilon_n} u_{\varepsilon_n}  \cdot
J_{\varepsilon_n}\partial_x u_{\varepsilon_n}) - \p_x (1-
\p_x^2)^{-1} \left
[\frac{3-\gamma}{2}(u_\ee)^2 + \frac{\gamma}{2}(\p_x u_\ee)^2 \right ] 
\end{equation}
%
by the uniqueness  of the limit in \eqref{loc-non-loc-tog}
we must have
%
\begin{equation}
\label{1000y}
\partial_t u =- \gamma u \partial_x u- \p_x (1- \p_x^2)^{-1} \left
[\frac{3-\gamma}{2}u^2 + \frac{\gamma}{2}(\p_x u)^2 \right ].
\end{equation}
%
Thus we have constructed a solution $u \in L^\infty(I, H^s(\ci))$
to the HR i.v.p. In fact, $u \in L^{\infty}(I, H^{s}(\ci)) \cap \text{Lip}(I,
H^{s-1})$. To see this, we observe that $u_{\ee_{n}}(t) \to u(t)$ strongly in $H^{s}$, and so
%
%
\begin{gather*}
    u_{\ee_{n}}(t_{1}) \to u(t_{1})    \ \text{in} \ H^{s-1}
    \\
    u_{\ee_{n}}(t_{2})  \to u(t_{2})  \ \text{in} \ H^{s-1}.
\end{gather*}
%
%
Applying this in conjunction with the triangle inequality  we obtain
%
%
\begin{equation*}
\begin{split}
    \| u(t_{1}) - u(t_{2}) \|_{H^{s-1}} \le \lim_{n \to \infty} \| u_{\ee_{n}}(t_{1}) - u_{\ee_{n}}(t_{2}) \|_{H^{s-1}} \le C | t_{1} - t_{2} |.
\end{split}
\end{equation*}
%
%
where the last step follows from \eqref{dt-u-e-Hs-bound} and the mean value theorem.
%
%
%%%%%%%%%%%%%%%%%%%%%%%%%%
%
%
%Proof that  $u \in C(I, H^s(\ci)) \bigcap C^1(I, H^{s-1}(\ci))$.
%
%
%
%%%%%%%%%%%%%%%%%%%%%%%%%%
\subsection{Proof that $u \in C(I, H^s(\ci))$.} 
We first outline our strategy. Since \\
$u \in L^\infty(I, H^s(\ci)) \bigcap \text{Lip}(I, H^{s-1})$, it is a
continuous function from $I$ to $H^s(\ci)$ with respect to the weak
topology on $H^{s}$; that is, for $\{t_n\} \subset I$ such that $t_n \to t$, we
have
\begin{equation}
\begin{split}
<u(t_n), \ v>_{H^s(\ci)} \ \longrightarrow \
<u(t), \ v>_{H^s(\ci)}, \quad \forall
v \in H^s(\ci).
\label{1ff}
\end{split}
\end{equation}
Indeed, let $\{t_{n}\}$ be a bounded sequence in $I$ converging to $t$. Then
since $u \in L^\infty(I, H^s(\ci))$ and $H^{s}$ is a separable Hilbert space,
there exists a subsequence $t_{n_{j}}$ such that
%
%
\begin{equation*}
\begin{split}
    u(t_{n_{j}}) \rightharpoonup w(t)  \ \text{in} \ H^{s}.
\end{split}
\end{equation*}
%
%
But
\begin{equation*}
\begin{split}
    u(t_{n_{j}}) \to u(t)  \ \text{in} \ H^{s-1}
\end{split}
\end{equation*}
so by the uniqueness of the limit we must have
$w(t) = u(t)$. Next, note that
\begin{equation}
\begin{split}
\|u(t) - u(t_n) \|_{H^s(\ci)}^2
& = <u(t) - u(t_n), \ u(t) -
u(t_n)>_{H^s(\ci)}
\\
& = \|u(t)\|_{H^s(\ci)}^2 + \|u(t_n)\|_{H^s(\ci)}^2
\\
& - <u(t_n), \
u(t) >_{H^s(\ci)} - <u(t), u(t_n)>_{H^s(\ci)}.
\label{2ff}
\end{split}
\end{equation}
Applying \eqref{1ff} and \eqref{2ff}, we see that
\begin{equation}
\begin{split}
\lim_{n \to \infty} \|u(t) - u(t_n)\|_{H^s(\ci)}^2 = \left[ \lim_{n
\to \infty} \|u(t_n)\|_{H^s(\ci)}^2
\right] - \|u(t)\|_{H^s(\ci)}^2.
\label{3ff}
\end{split}
\end{equation}
Hence, by \eqref{3ff}, to prove that $u \in C(I, H^s(\ci))$, it will be
enough to show that the map $t \mapsto \|u(t)\|_{H^s(\ci)}$ is a continuous
function of $t$. However, this will follow from the energy
estimate
\begin{equation}
\label{en-est-u}
\frac{1}{2} \frac{d}{dt} \|u(t)\|_{H^s(\ci)}^2
\le c_s \|u(t)\|_{H^s(\ci)}^3, \quad |t| \le T
\end{equation}
which we now derive. Applying $D^s$ to both sides of
\eqref{1000y}, multiplying the
resulting equation by $D^s u$, and integrating for $x\in \ci$, we obtain
\begin{equation}
\begin{split}
\label{bound-int}
\frac 12
\frac{d}{dt} \|u \|_{H^s}^2
=
&-
\gamma \int_{\ci}   D^s (u \p_x u) \cdot
D^s u \  dx
\\
&- \frac{3 -\gamma}{2} \int_{\ci}  D^{s-2} \p_x (u^2) 
\cdot D^s u \ dx
\\
&- \frac{\gamma}{2} \int_{\ci}   D^{s-2} \p_x (\p_x u)^2
\cdot D^s u \ dx.
\end{split}
\end{equation}
First we estimate
\begin{equation}
\begin{split}
\left | - \frac{3 -\gamma}{2} \int_\ci D^{s-2} \p_x (u^2) \cdot
D^s u \; dx \right |
& \lesssim 
\int_\ci \left |
D^{s-2} \p_x (u^2) \cdot D^s u \right | dx 
\\
& \le 
\|D^{s-2} \p_x (u^2) \|_{L^2(\ci)} 
\|D^s u \|_{L^2(\ci)}
\\
& \le 
\|D^{s-1} (u^2) \|_{L^2(\ci)} 
\|D^s u \|_{L^2(\ci)}
\\
& = 
\| u^2 \|_{H^{s-1}(\ci)} \| u \|_{H^s(\ci)}
\\
& \le
\| u^2 \|_{H^s(\ci)} \| u
\|_{H^s(\ci)}.
\end{split}
\end{equation}
%
%
Applying the algebra property, we obtain
%
\begin{equation}
\label{hl1-prime}
\begin{split}
\left | -\frac{3 -\gamma}{2} \int_\ci D^{s-2} \p_x u^2 \cdot
D^s u \; dx \right |
\lesssim_{s}  \| u \|_{H^s(\ci)}^3.
\end{split}
\end{equation}
%
%
We also have
\begin{equation}
\begin{split}
\left | -\frac{\gamma}{2} \int_\ci D^{s-2} \p_x (\p_x u)^2 \cdot
D^s u \; dx \right |
& \lesssim 
\int_\ci \left | D^{s-2} \p_x (\p_x u)^2 
\cdot D^s u \right | \; dx
\\
& \le 
\| D^{s-2} \p_x (\p_x u)^2 \|_{L^2(\ci)}
\| D^s u \|_{L^2(\ci)}
\\
&  \le  \|(\p_x u)^2
\|_{H^{s-1}(\ci)} \| u \|_{H^s(\ci)} 
\end{split}
\end{equation}
and applying the algebra property yields
\begin{equation}
\label{hl2-prime}
\left | -\frac{\gamma}{2} \int_\ci D^{s-2} (\p_x u)^2 \cdot
D^s u \; dx \right |
\lesssim_{s}  \|u\|_{H^s(\ci)}^3.
\end{equation}
It remains to estimate 
\begin{equation*}
- \gamma \int_{\ci} \left [  D^s (u \p_x u) \cdot
D^s u \right ]  dx.
\end{equation*}
We have
\begin{equation} 
\begin{split}
\label{B-moli-int-v'}
-  \gamma \int_{\ci} \left [D^s (u \p_x u) \cdot D^s
u \right ] \ dx
= &- \gamma  \int_\ci
\left [ D^s(u \partial_x u) \cdot   D^s u \right ] \ dx
\\
=& - \gamma \int_\ci
\big[ 
D^s(u\p_x u)  -  u D^s (\p_xu)
\big] \cdot
D^s u   \ dx
\\
&
- \gamma \int_\ci
u D^s (\p_xu) \cdot
D^su \ dx.
\end{split}
\end{equation}
%
%
%
We now estimate \eqref{B-moli-int-v'} in parts. Applying the Cauchy-Schwarz inequality gives
%
\begin{equation*} 
\begin{split}
& \Big|
- \gamma \int_\ci
\big[ 
D^s(u\p_x u)  -  u D^s (\p_xu)
\big] \cdot
D^s u   \, dx
\Big|
\\
& \lesssim
 \|
D^s(u\p_x u)  -  u D^s (\p_xu)
\|_{L^2(\ci)}
\|
D^s u 
\|_{L^2(\ci)}
\\
& =
 \| D^s(u\p_x u)  -  u D^s (\p_xu)
\|_{L^2(\ci)}
\|
u
\|_{H^s(\ci)}
\end{split}
\end{equation*}
Applying \eqref{int1-est-calc3}, we obtain
\begin{equation*}
\begin{split}
\Big|
- \gamma \int_\ci
\big[ 
D^s(u\p_x u)  -  u D^s (\p_xu)
\big]
D^s u   \, dx
\Big|
&\lesssim_{s}
\| \p_x u \|_{L^\infty(\ci)} 
\| u \|_{H^s(\ci)}^2.
\end{split}
\end{equation*}
%\label{int1-est-calc2'}
%
Next, we apply Cauchy-Schwartz and the Sobolev Imbedding Theorem to deduce 
%
%
%
\begin{equation} 
\label{int1-est-calc5'}
\begin{split}
\Big|
\int_\ci
u D^s (\p_x u)
\cdot  D^s u  dx
\Big|
& \simeq 
 \Big|
\int_\ci
u \p_x\left(D^s u\right)^2  \ dx
\Big|
\\
& \le
 \int_\ci \Big |
\p_x u \, (D^s u)^2  
\Big| \ dx
\\
& \le
\| \p_x u \|_{L^\infty(\ci)} 
\| u \|_{H^s(\ci)}^2.
\\
& \lesssim_{s}  \|u\|_{H^s(\ci)}^3.
\end{split}
\end{equation}
%
%
%
Combining \eqref{hl1-prime}, \eqref{hl2-prime},
and \eqref{int1-est-calc5'}, we obtain \eqref{en-est-u}, as desired.
Letting  $y(t)=  \|u(t)\|_{H^s(\ci)}^2$ inequality \eqref{en-est-u}
takes the form
%
\begin{equation} 
\label{energy-y-ineq'}
\frac 12
y^{-3/2}\frac{dy}{dt}
\le
c_s,
\qquad
y(0)=y_0=  \|u_0\|_{H^s(\ci)}^2.
\end{equation}
%
Suppose $t$ is non-negative. Then integrating  \eqref{energy-y-ineq'}
from  0 to $t$ gives
%
\begin{equation*} 
\frac{1}{\sqrt{y_0}}  - \frac{1}{\sqrt{y(t)}} 
\le 
c_s t.
\end{equation*}
%
%
Replacing $y(t)$ with   $\|u(t)\|_{H^s(\ci)}^2$  and solving for  $\|u(t)\|_{H^s(\ci)}$
we obtain the formula
%
\begin{equation} 
\label{norm-u(t)-formula'}
\|u(t)\|_{H^s(\ci)}
\le
\frac{\|u_0\|_{H^s(\ci)}}{1-c_s\|u_0\|_{H^s(\ci)} t}.
\end{equation}
%
Now, note that our solution $u$ inherits the common lifespan $T$ of the family
$\{u_\ee\}$; that is, $u$ has lifespan
\begin{equation*}
T
=
\frac{1}{2 c_s \|u_0\|_{H^s(\ci)}}.
\end{equation*}
Substituting into \eqref{norm-u(t)-formula'} we obtain	
%
\begin{equation*} 
\label{u(t)-u(0)-bound'}
\|u(t)\|_{H^s(\ci)}
\le
\frac{\|u_0\|_{H^s(\ci)}}{1-(c_s\|u_0\|_{H^s(\ci)})/(2 c_s \|u_0\|_{H^s(\ci)})},
\end{equation*}
%
which simplifies to 
%
\begin{equation*}
\|u(t)\|_{H^s(\ci)}
\le
2 \|u_0\|_{H^s(\ci)},
\quad 
0\le t \le T.
\end{equation*}
Similarly, for negative $t$, we have
\begin{equation*}
\|u(t)\|_{H^s(\ci)}
\le
2 \|u_0\|_{H^s(\ci)},
\quad 
-T \le t < 0.
\end{equation*}
Hence,
\begin{equation}
\label{uniform_bound_for_u}
\|u(t)\|_{H^s(\ci)}
\le
2 \|u_0\|_{H^s(\ci)},
\quad 
|t| \le T.
\end{equation}
%
Derivating the left hand side of \eqref{en-est-u} and simplifying, we obtain
\begin{equation}
\label{en-est-u-simplified}
\frac{d}{dt} \|u(t)\|_{H^s(\ci)} \le c_s \|u(t)\|_{H^s(\ci)}^2, \quad |t| \le T.
\end{equation}
Since $\|u(t)\|_{H^s(\ci)}$
is uniformly bounded for $|t| \le T$ by
\eqref{uniform_bound_for_u}, we conclude from
\eqref{en-est-u-simplified} that the map $t \mapsto
\|u(t)\|_{H^s(\ci)}$ is Lipschitz continuous in $t$, for $|t| \le T$.
Therefore, by \eqref{3ff}, $u \in C(I, H^s(\ci))$. \qed
%
%
%
%
%
\subsection{Uniqueness.}
%
%
Let $u,\omega \in C(I, H^s(\ci)), \ s > 3/2$ be two solutions to the
Cauchy-problem \eqref{hyperelastic-rod-equation}-\eqref{init-cond} with
common initial data. Let $v=u-w$; since
\begin{align*}
\p_t u 
& = - \gamma u \p_x u - D^{-2} \p_x \left[ \frac{3-\gamma}{2} u^2 +
\frac{\gamma}{2}\left( \p_x u \right)^2 \right]
\\
\p_t w & = -\gamma w \p_x w - D^{-2} \p_x \left[
\frac{3-\gamma}{2} w^2 + \frac{\gamma}{2}(\p_x w)^2 
\right]
\end{align*}
we subtract the two equations to obtain 
\begin{equation*}
\begin{split}
\p_t v
= -\frac{\gamma}{2} \p_x [v(u + w)] - D^{-2} \p_x \left\{
\frac{3-\gamma}{2}[v(u+w)] + \frac{\gamma}{2}[\p_x v \cdot \p_x (u+w)]
\right\}
\end{split}
\end{equation*}
and hence
\begin{equation}
\begin{split}
D^\sigma \p_t v = -\frac{\gamma}{2} D^\sigma \p_x [v(u+w)] - D^{\sigma -2} \p_x
\left\{ \frac{3-\gamma}{2} [v(u+w)] + \frac{\gamma}{2} [\p_x v
\cdot \p_x
(u+w)]
\right\}.
\label{1v}
\end{split}
\end{equation}
Multiplying both sides of \eqref{1v} by $D^\sigma v$ and integrating, we obtain
\begin{equation}
\begin{split}
\frac{1}{2} \frac{d}{dt} \|v\|_{H^\sigma(\ci)}^2
& =  \overbrace{-\frac{\gamma}{2} \int_{\ci} D^\sigma \p_x [v(u+w)] \cdot
D^\sigma v \ dx}^i
\\
& \overbrace{- \frac{3-\gamma}{2} \int_{\ci}  D^{\sigma -2}
\p_x[v(u+w)] \cdot
D^\sigma v \ dx}^{ii} 
\\
& - \overbrace{\frac{\gamma}{2} \int_{\ci} D^{\sigma -2} \p_x [ \p_x v
\cdot \p_x (u+w)]\cdot D^\sigma v \ dx }^{iii}.
\label{2v}
\end{split}
\end{equation}
We will estimate (\hyperref[2v]{ii}) first.
Applying Cauchy-Schwartz, we have 
\begin{equation*}
\begin{split}
|ii|
& \lesssim  \|D^{\sigma -2}
\p_x [v(u+w)] \cdot D^\sigma
v  \|_{L^1(\ci)}
\\
& \le   \|D^{\sigma -2} \p_x [v(u+w)]
\|_{L^2(\ci)} \|v\|_{H^\sigma(\ci)}
\\
& \le \|v(u+w)\|_{H^{\sigma -1}(\ci)} \|v\|_{H^\sigma(\ci)}
\end{split}
\end{equation*}
which by the algebra property and the Sobolev
Imbedding Theorem gives
\begin{equation}
\begin{split}
    |ii| \lesssim_{s} \|u+w\|_{H^{\sigma -1}(\ci)} \|v\|_{H^\sigma(\ci)}^2.
\label{3v}
\end{split}
\end{equation}
To estimate (\hyperref[2v]{iii}) we first apply
Cauchy-Schwartz and the Sobolev Imbedding Theorem:
\begin{equation*}
\begin{split}
|iii| & \lesssim  \|D^{\sigma -2} \p_x
[\p_x v \cdot \p_x (u+w)] \cdot D^\sigma v  \|_{L^1(\ci)} 
\\
& \le   \|D^{\sigma -2} \p_x
[\p_x v \cdot \p_x (u+w)] \|_{L^2(\ci)}
\|v\|_{H^\sigma(\ci)}
\\
& \le 
\|[\p_x v \cdot \p_x (u+w)] \|_{H^{\sigma -1}(\ci)}
\|v\|_{H^\sigma(\ci)}.
\end{split}
\end{equation*}
Restrict $\sigma > 1/2$. Then applying \eqref{impo}, we conclude
that
\begin{equation}
\begin{split}
|iii|
& \lesssim 
\|\p_x(u+w) \|_{H^{\sigma}(\ci)}
\|\p_x v\|_{H^{\sigma -1}(\ci)} \|v\|_{H^\sigma(\ci)}
\\
& \lesssim_{s} \|u+w \|_{H^{\sigma + 1}(\ci)}
\|v\|_{H^\sigma(\ci)}^2.
\label{3'v}
\end{split}
\end{equation}
It remains to estimate (\hyperref[2v]{i}).
Proceeding, we rewrite
\begin{equation}
\begin{split}
|i| & =
\left |
-\frac{\gamma}{2} \int_{\ci} \left[ D^\sigma \p_x, \ u+w \right]v \cdot
D^\sigma v \ dx - \frac{\gamma}{2} \int_{\ci} (u+w) D^\sigma
\p_x v \cdot D^\sigma v\ dx
\right | 
\\
& \lesssim \left |  \int_{\ci} \left[ D^\sigma \p_x, \ u+w \right]v \cdot
D^\sigma v \ dx \right |
+ \left |  \int_{\ci} (u+w) D^\sigma \p_x v
\cdot D^\sigma v\
dx \right |.
\label{4v}
\end{split}
\end{equation}
We now estimate \eqref{4v} in pieces. Observe that by integrating by parts
and applying Cauchy-Schwartz we have
\begin{equation}
\begin{split}
\left | \frac{\gamma}{2}\int_{\ci} (u+w) D^\sigma \p_x v \cdot
D^\sigma v \ dx \right |
& \simeq \left |  \int_{\ci} \p_x (u+w) D^\sigma v
\cdot D^\sigma v \ dx \right |
\\
& \lesssim \|\p_x (u+w) D^\sigma v \|_{L^2(\ci)} \|D^\sigma
v\|_{L^2(\ci)}
\\
& \lesssim \|\p_x (u+w)\|_{L^\infty(\ci)}
\|v\|_{H^\sigma(\ci)}^2.
\label{4'v}
\end{split}
\end{equation}
To estimate the remaining piece of \eqref{4v}, we choose $\ 3/2 < \rho
< s,  \ 1/2< \sigma <\rho -1$ and obtain
\begin{equation}
\begin{split}
\left | -\frac{\gamma}{2} \int_{\ci} [D^\sigma \p_x, \ u+w] v
\cdot D^\sigma v \ dx \right |
& \lesssim \|[D^\sigma \p_x, \ u+w]v\|_{L^2(\ci)}
\|v\|_{H^\sigma(\ci)} \\
& \lesssim \|u+w\|_{H^\rho(\ci)} \|v\|_{H^\sigma(\ci)}^2.
\label{7v}
\end{split}
\end{equation}
Combining \eqref{4'v} and \eqref{7v} and applying the Sobolev Imbedding
Theorem, we obtain the estimate
\begin{equation}
\begin{split}
|i| \lesssim \|u+w\|_{H^\rho(\ci)} \|v\|_{H^\sigma(\ci)}^2.
\label{8v}
\end{split}
\end{equation}
Recall \eqref{2v}. Grouping \eqref{3v}, \eqref{3'v}, and \eqref{8v}, and applying
the Sobolev Imbedding Theorem, we see that 
\begin{equation}
\begin{split}
\frac{1}{2} \frac{d}{dt}
\|v\|_{H^\sigma(\ci)}^2 \lesssim \|u+w\|_{H^\rho(\ci)}
\|v\|_{H^\sigma(\ci)}^2.
\label{9v}
\end{split}
\end{equation}
By Gronwall's inequality, \eqref{9v} gives
\begin{equation}
\label{10lv}
\begin{split}
\|v\|_{H^\sigma(\ci)}
& \lesssim e^{\int_0^t \|u+w\|_{H^{\rho}}}
\|v_0\|_{H^\sigma(\ci)}, \qquad |t| \le T.
\end{split}
\end{equation}
First, note that $v_0 = u_0 - w_0 = 0$; secondly, $\|u + w \|_{H^\rho}
\le \|u + w \|_{H^s(\ci)} < \infty$ for $|t| \le T$ by
the triangle inequality and \eqref{u_x-Linfty-Hs}. Hence, from
\eqref{10lv} we obtain
\begin{equation*}
\begin{split}
\|v\|_{H^\sigma(\ci)}
& \lesssim \|v_0\|_{H^\sigma(\ci)}, \quad |t| \le T	
\\
& = 0.
\end{split}
\end{equation*}
We conclude that solutions to the HR i.v.p.\ with initial data in
$H^s(\ci)$ are unique for $s > 3/2$.  \qed
%
%
%
%
\subsection{Continuous Dependence}
Let $\left\{ u_{0, n} \right\}_n \subset H^s(\ci)$ be a uniformly bounded
sequence converging to $u_0$ in $H^s(\ci)$.
Consider solutions $u $, $u^\ee$, $u^\ee_n$, and $u_n$ to the 
Cauchy-problem
\eqref{hyperelastic-rod-equation}-\eqref{init-cond}
with associated initial data $u_0$, $J_\ee u_0$,
$J_\ee u_{0,n}$, and $u_{0,n}$, respectively, where $J_\ee$ is the operator 
defined by
\begin{equation}
\label{0'u}
\begin{split}
J_\ee f(x) = j_\ee * f(x), \quad \ee>0.
\end{split}
\end{equation}
%
%
Here
\begin{equation}
\begin{split}
j_\ee (x) = \frac{1}{2 \pi}\sum_{\xi \in \zz}
\widehat{j}(\ee \xi) e^{i \xi x}, \quad \ee > 0
\label{parseval-def}
\end{split}
\end{equation}
where $\widehat{j}(\xi) \in \mathcal{S}(\rr)$ is chosen such that 
%
\begin{equation}
\label{0u}
\begin{split}
	 0 \le \widehat{j}(\xi) \le 1  \ \ \text{and} \ \
 \widehat{j}(\xi) = 1 \ \ \text{if} \ \ |\xi| \le 1.
\end{split}
\end{equation}
%
%
%
%
%
%
%
%
%
%
We remark that it follows immediately from \eqref{parseval-def} that
\begin{equation}
\begin{split}
	\widehat{j_\ee}(\xi)  = \widehat{j }(\ee \xi), \quad \ee > 0.
\label{widehat-def}
\end{split}
\end{equation}
This will prove
crucial later on.
%
Next, applying
the triangle inequality, we obtain
%
%
\begin{equation*}
\begin{split}
\|u - u_n\|_{H^s(\ci)}
& \le \|u - u^\ee\|_{H^s(\ci)}
+ \|u^\ee - u^{\ee}_n \|_{H^s(\ci) }
+  \|u^{\ee}_n - u_n \|_{H^s(\ci)}.
\end{split}
\end{equation*}
%
%
Let $\eta > 0$. To prove continuous dependence, it will be enough to show that 
we can find $\ee > 0$ and $N \in \mathbb{N}$ such that for all $n > N$ 
\begin{align}
	 \|u(t) - u^\ee(t)\|_{H^s(\ci)}
	& < \eta/3, \quad |t| \le T,
\label{enough_to_prove1}
\\
  \|u^\ee(t) - u^{\ee}_n(t)
\|_{H^s(\ci)} & < \eta/3, \quad |t| \le T,
\label{enough_to_prove2}
\\
  \|u^{\ee}_n(t) - u_n(t) \|_{H^s(\ci)} & < \eta/3, \quad |t| \le T.
\label{enough_to_prove3}
\end{align}
%
%
The proof of \eqref{enough_to_prove3} will be analogous to that of 
\eqref{enough_to_prove1}, so we will omit the details.
%
%
%
%

{\bf Proof of \eqref{enough_to_prove1}.}
Consider two solutions $u $ and $u^\ee$ to the Cauchy-problem
\eqref{hyperelastic-rod-equation}-\eqref{init-cond}
with associated initial data $u_0$ and
$J_\ee u_0$, respectively. Set $v= u -u^\ee $. Then $v$ solves the
Cauchy-problem
\begin{align}
\label{4u}
& \p_t v  =  - \gamma (v \p_x v + v \p_x u^\ee + u^\ee \p_x v)  \\
& \phantom{\p_t v  =} - D^{-2} \p_x \left\{ \left (\frac{3-\gamma}{2} \right )(v^2 +
2u^\ee v) + \frac{\gamma}{2}\left[ (\p_x v)^2 + 2 \p_x u^\ee \p_x v \right]
\right\}, \notag
\\
& v(0) = (I- J_\ee)u_0.
\label{5u}
\end{align}
Applying the operator $D^s$ to both sides of \eqref{4u}, then multiplying by
$D^s v$ and integrating gives
%
%
\begin{equation}
\begin{split}
\frac{1}{2}\frac{d}{dt} \|v\|_{H^s(\ci)} = A + B
\label{6u}
\end{split}
\end{equation}
%
%
where
%
%
\begin{equation}
\begin{split}
A
 =  & -\gamma \int_{\ci} D^s(v \p_x v) \cdot D^s v \
dx
\\
& - \frac{3- \gamma}{2} \int_\ci D^{s-2} \p_x (v^2) \cdot D^s v
\ dx
\\
& - \frac{\gamma}{2}\int_\ci D^{s-2} \p_x (\p_x v)^2 \cdot D^s
v \ dx
\label{7u}
\end{split}
\end{equation}
%
%
and
%
%
\begin{equation}
\begin{split}
B = & -\gamma \int_\ci D^s (v \p_x u^\ee ) \cdot D^s v \
dx \\
& -\gamma \int_\ci D^s (u^\ee \p_x v) \cdot D^s v \
dx
\\
& - \ ( 3- \gamma) \int_\ci D^{s-2} \p_x (u^\ee v) \cdot D^s
v \ dx
\\
& -\gamma \int_\ci D^{s-2} \p_x
(\p_x u^\ee \cdot \p_x v) \cdot D^s v \
dx.
\label{8u}
\end{split}
\end{equation}
%
%
Using estimates analogous to those in 
\eqref{B-moli-int-v}-\eqref{hl2}, we 
obtain 
%
%
\begin{equation}
\begin{split}
|A| \lesssim \|v\|_{H^s(\ci)}^3, \quad |t| \le T.
\label{8'u}
\end{split}
\end{equation}
%
%
%
Next we estimate $B$ in parts:
%
%
%
%
%

{\bf Estimate of Integral 1.} We can rewrite
%
%
\begin{equation}
\begin{split}
-\gamma \int_\ci D^s (v \p_x u^\ee ) \cdot D^s v \
dx = & -\gamma \int_\ci \left[ D^s(v \p_x u^\ee) - v D^s
\p_x u^\ee \right] \cdot D^s v \ dx
\\
& -  \gamma \int_\ci v D^s \p_x u^\ee \cdot D^s v \ dx.
\label{1wap'}
\end{split}
\end{equation}
%
%
%
%
%
%
Applying Cauchy-Schwartz, the Kato-Ponce estimate \eqref{KP-com-est}, and the Sobolev 
Imbedding Theorem, we obtain 
%
%
%
%
\begin{equation}
\begin{split}
| -\gamma \int_\ci \left[ D^s(v \p_x u^\ee ) - v D^s
\p_x u^\ee \right] \cdot D^s v \ dx |
\lesssim \|u^\ee \|_{H^s(\ci)} \|v\|_{H^s(\ci)}^2.
\label{2wap}
\end{split}
\end{equation}
%
%
For the remaining integral of \eqref{1wap'}, Cauchy-Schwartz, H\"older, and the Sobolev 
Imbedding Theorem give
%
%
%
\begin{equation}
\begin{split}
  | - \gamma \int_\ci v D^s \p_x u^{\ee} \cdot D^s v \ dx |
\lesssim \|u^\ee \|_{H^{s+1}(\ci)} \|v\|_{H^{s-1}(\ci)}
\|v\|_{H^s(\ci)}.
\label{3wap}
\end{split}
\end{equation}
%
%
Combining estimates \eqref{2wap} and \eqref{3wap} we conclude that
%
%
\begin{equation}
\begin{split}
\left | -\gamma \int_\ci D^s (v \p_x u^\ee ) \cdot D^s v \
dx \right | \lesssim \|u^\ee \|_{H^s(\ci)} \|v\|_{H^s(\ci)}^2 + \|u^\ee 
\|_{H^{s+1}(\ci)} \|v\|_{H^{s-1}(\ci)}
\|v\|_{H^s(\ci)}.
\label{4wap}
\end{split}
\end{equation}
%
%
%
{\bf Estimate of Integral 2}. We can rewrite
%
%
\begin{equation}
\begin{split}
-\gamma \int_\ci D^s (u^\ee \p_x v) \cdot D^s v \
dx
= & -\gamma \int_\ci \left[ D^s(u^\ee \p_x v) - u^\ee D^s
\p_x v \right] \cdot D^s v \ dx
\\
& -  \gamma \int_\ci u^\ee D^s \p_x v \cdot D^s v \ dx.
\label{1wa'}
\end{split}
\end{equation}
%
%
Applying Cauchy-Schwartz, the Kato-Ponce estimate \eqref{KP-com-est}, and 
the Sobolev \\ Imbedding Theorem to 
the first integral, we obtain
%
%
%
%
\begin{equation}
\begin{split}
| -\gamma \int_\ci \left[ D^s(u^\ee \p_x v) - u^\ee D^s
\p_x v \right] \cdot D^s v \ dx |
\lesssim \|u^\ee \|_{H^s(\ci)} \|v\|_{H^s(\ci)}^2.
\label{2wa}
\end{split}
\end{equation}
%
%
For the remaining integral of \eqref{1wa'}, integration by parts, 
Cauchy-Schwartz, and the Sobolev Imbedding Theorem give
%
%
\begin{equation}
\begin{split}
| - \gamma \int_\ci u^\ee D^s \p_x v \cdot D^s v \ dx |
\lesssim \|u^\ee \|_{H^s(\ci)} \|v\|_{H^s(\ci)}^2.
\label{3wa}
\end{split}
\end{equation}
%
%
Combining estimates \eqref{2wa} and \eqref{3wa} we conclude that
%
%
\begin{equation}
\begin{split}
\left | -\gamma \int_\ci D^s (u^\ee \p_x v) \cdot D^s v \
dx \right |
 \lesssim \|u^\ee \|_{H^s(\ci)} \|v\|_{H^s(\ci)}^2.
\label{4wa}
\end{split}
\end{equation}
%
%
{\bf Estimate of Integral 3.} Applying Cauchy-Schwartz, the 
algebra property of Sobolev spaces, and the Sobolev Imbedding Theorem gives
%
%
\begin{equation}
\begin{split}
\left |- \ ( 3- \gamma) \int_\ci D^{s-2} \p_x (u^\ee v) \cdot D^s
v \ dx \right |  \lesssim \|u^\ee\|_{H^{s}(\ci)} \|v\|_{H^{s}(\ci)}^2.
\label{13u}
\end{split}
\end{equation}
%
%
%
%
%
%
{\bf Estimate of Integral 4.} Applying Cauchy-Schwartz, the 
algebra property of Sobolev spaces, and the Sobolev Imbedding Theorem, we 
obtain
%
%
\begin{equation*}
\begin{split}
\left |-\gamma \int_\ci D^{s-2} \p_x
(\p_x u^\ee \cdot \p_x v) \cdot D^s v \
dx \right |
 \lesssim \|u^\ee\|_{H^s(\ci)} \|v\|_{H^s(\ci)}^2.
\end{split}
\end{equation*}
%
%
Hence, collecting our estimates for integrals 1-4, we obtain 
%
%
\begin{equation}
\begin{split}
|B| & \lesssim
\|u^\ee\|_{H^s(\ci)}
\|v\|_{H^s(\ci)}^2 + \|u^\ee\|_{H^{s+1}(\ci)}
\|v\|_{H^{s-1}(\ci)} \|v\|_{H^s(\ci)}.
\label{14u}
\end{split}
\end{equation}
%
%
Combining estimates \eqref{8'u} and \eqref{14u} and recalling
\eqref{6u}, we obtain
%
%
\begin{equation*}
\begin{split}
\frac{1}{2}\frac{d}{dt}\|v\|_{H^{s}(\ci)}^2
& \lesssim \|v\|_{H^s(\ci)}^3 + \|u^\ee\|_{H^s(\ci)}
\|v\|_{H^s(\ci)}^2
 + \|u^\ee\|_{H^{s+1}(\ci)}
\|v\|_{H^{s-1}(\ci)} \|v\|_{H^s(\ci)}
\end{split}
\end{equation*}
%
%
which simplifies to 
\begin{equation}
\begin{split}
\frac{d}{dt}\|v\|_{H^{s}(\ci)}
& \lesssim \|v\|_{H^s(\ci)}^2 + 
\|v\|_{H^s(\ci)}
+ \ee^{-1}
\|v\|_{H^{s-1}(\ci)} 
\label{15u}
\end{split}
\end{equation}
by differentiating the left-hand side and applying the following lemma:
%
%
%
\begin{lemma}
\label{lem5r}
For $r \ge s > 3/2$ and $0 < \ee <1$, 
%
%
\begin{equation}
\begin{split}
\|u^{\ee} (t) \|_{H^r(\ci)} \lesssim \ee^{s-r}.
\label{700r}
\end{split}
\end{equation}
%
%
\end{lemma}
%
%
\begin{proof} Recalling the construction of $J_\ee$ in 
\eqref{0'u}-\eqref{widehat-def},  we have
%
%
\begin{equation}
   \label{schwartz}
	|\widehat{J_\ee u_0}(\xi)| = |\widehat{j_\ee}(\xi) \widehat{u_0}(\xi)|
	= |\widehat{j }(\ee \xi) \widehat{u_0}(\xi)| 
\le 
  \begin{cases}
    | \wh{u_{0}}(\xi) |, \quad & | \xi | \le 1/\ee
\\
| \ee \xi |^{s-r} | \wh{u_{0}}(\xi) |, \quad  & | \xi | \ge 1/\ee.
\end{cases}
\end{equation}
%
%
%
%
%
Applying \eqref{u_x-Linfty-Hs} and \eqref{schwartz}, the result follows.
\end{proof}
%
%
%
We now aim to prove decay for the $\ee^{-1}
\|v\|_{H^{s-1}(\ci)} $ term in \eqref{15u}. To do so, we 
will first obtain an estimate for
$\|v\|_{H^\sigma(\ci)}$ for suitably chosen $\sigma < s-1$. Then, 
interpolating between $\|v\|_{H^\sigma(\ci)}$
and $\|v\|_{H^s(\ci)}$, we will show that 
$\|v\|_{H^{s-1}(\ci)}$ experiences $o(\ee)$ decay. This will imply
$o(1)$ decay of $\ee^{-1}
\|v\|_{H^{s-1}(\ci)} $.
%
%
\begin{proposition} \label{prop:6r}
For $\sigma$ such that $\sigma > 1/2$ and $\sigma + 1 \le s$, we have
%
%
\begin{equation}
\begin{split}
\|v\|_{H^{\sigma}(\ci)} = o(\ee^{s- \sigma }), \quad |t| \le T.
\end{split}
\end{equation}
%
%
\end{proposition}
%
%
%
\begin{proof}
Recall that $v$ solves the Cauchy-problem \eqref{4u}-\eqref{5u}.
Applying $D^\sigma$ to both sides of \eqref{4u}, then multiplying by
$D^\sigma v$ and integrating, we obtain 
%
%
\begin{equation*}
\begin{split}
\frac{1}{2}\frac{d}{dt}\|v(t)\|_{H^\sigma(\ci)}^2
= & - \frac{\gamma}{2}\int_{\ci} D^\sigma
\p_x \left[ \left( u + u^\ee \right)v
\right]\cdot D^\sigma v \ dx
\\
& - \frac{3-\gamma}{2}\int_{\ci} D^{\sigma
-2} \p_x \left[ \left( u + u^\ee
\right)v \right] \cdot D^\sigma v \ dx
\\
& - \frac{\gamma}{2}\int_{\ci} D^{\sigma
-2}
\p_x \left[ \left( \p_x u + \p_x u^\ee
\right)\cdot \p_x v \right] \cdot
D^\sigma v \ dx.
\end{split}
\end{equation*}
%
%
Repeating calculations \eqref{X}-\eqref{12}, with $E$ set to zero,
$u^{\omega,n}$ replaced by $u$, $u_{\omega,n}$ replaced by $u^\ee$, and
$\sigma$ and $\rho$ chosen such that
%
%
%
\begin{equation*}
\label{size_of_sigma}
 \sigma > 1/2,
 \quad 
 \text{and}
 \quad
 \sigma + 1 \le \rho \le s 
\end{equation*}
%
%
yields
%
%
\begin{equation*}
\begin{split}
\frac{1}{2}\frac{d}{dt} \|v\|_{H^\sigma(\ci)}^2
& \lesssim
(\|u^{\ee} + u\|_{H^{\rho}(\ci)} +
\|\p_x(u^{\ee} + u) \|_{H^\sigma(\ci)})
\cdot \|v\|_{H^\sigma(\ci)}^2.
\end{split}
\end{equation*}
%
%
By the Sobolev Imbedding Theorem, it follows that 
%
%
\begin{equation}
\begin{split}
\frac{1}{2}\frac{d}{dt} \|v\|_{H^{\sigma}(\ci)}^2
& \lesssim
\|u^{\ee}
+ u\|_{H^{s}(\ci)} \cdot \|v\|_{H^{\sigma}(\ci)}^2.
\label{10x}
\end{split}
\end{equation}
%
%
Hence, applying the triangle inequality, \eqref{u_x-Linfty-Hs}, and the estimate
%
%
\begin{equation}
\begin{split}
	\|J_\ee f\|_{H^s(\ci)} \le \|f\|_{H^s(\ci)}
\label{lem100u}
\end{split}
\end{equation}
%
%
%
%
to the right-hand side of \eqref{10x} yields
%
%
%
%
%
\begin{equation*}
\begin{split}
\label{12x}
\frac{1}{2}\frac{d}{dt} \|v\|_{H^{\sigma}(\ci)}^2
& \le
C \|v\|_{H^{\sigma}(\ci)}^2
\end{split}
\end{equation*}
%
%
where $C = C(\|u_0\|_{H^s(\ci)})$. Gronwall's inequality then gives
%
%
%
\begin{equation*}
\label{conc-lemma}
\begin{split}
\|v\|_{H^{\sigma}(\ci)}
 \le e^{C t}\|v(0)\|_{H^{\sigma}(\ci)}
 = e^{C t}\|u_0 - J_\ee u_0 \|_{H^{\sigma}(\ci)} = o(\ee^{s-r})
\end{split}
\end{equation*}
%
%
%
%
%
where the last step follows from the operator norm estimate provided below. 
\end{proof}
%
%
\begin{lemma}
\label{lem4r}
For $r \le s$ and $\ee>0$
%
%
\begin{equation}
\label{0r}
\begin{split}
\|I - J_\ee\|_{L(H^s(\ci), H^r(\ci))} = o(\ee^{s-r}).
\end{split}
\end{equation}
%
%
\end{lemma}
%
%
\begin{proof}
Let $u \in H^s(\ci)$ and $r, s \in \rr$ such that $r \le s$. 
Recalling the construction of $J_\ee$ in
\eqref{0'u}-\eqref{widehat-def}, we have
%
%
\begin{align}
\label{1r}
& \|u - J_\ee u\|_{H^r(\ci)}^2 = \sum_{\xi \in \zz} | [1- \widehat{j}(\ee 
\xi)] \cdot \widehat{u}(\xi) |^2
(1+\xi^2)^r, \ \ \text{and}
\\
& |1 - \widehat{j}(\ee \xi)| \le |\ee \xi |^{s-r}, \quad 
\xi \in \rr, \ \ee > 0.
\label{2r}
\end{align}
%
%
Applying \eqref{2r} to \eqref{1r} we obtain
%
%
\begin{equation*}
\label{2pr}
\begin{split}
\|u - J_\ee u\|_{H^r(\ci)}
\lesssim \ee^{s-r}
\end{split}
\end{equation*}
%
%
%
%
while a dominated convergence argument gives
%
%
\begin{equation*}
\label{o1}
\begin{split}
\|u - J_\ee u \|_{H^s(\ci) } & = o(1).
\end{split}
\end{equation*}
%
%
%
%
Applying the interpolation estimate 
%
%
\begin{equation}
\begin{split}
\|f\|_{H^{k_2}(\ci)} \le
\|f\|_{H^{k_1}(\ci)}^{(s-k_2)/(s -k_1 )}
\|f\|_{H^s(\ci)}^{1 - (s-k_2)/(s -k_1 )}, \quad k_1 < k_2 \le s
\label{16u}
\end{split}
\end{equation}
%
%
%
%
%
%
completes the proof. 
%
\end{proof}
We now return to analyzing the $\ee^{-1}
\|v\|_{H^{s-1}(\ci)} $ term of \eqref{15u}.
Applying \eqref{16u} and \cref{prop:6r}, 
we obtain
%
%
%
%
$$
\|v\|_{H^{s-1}(\ci)}  \lesssim o(\ee) 
\|v\|_{H^s(\ci)}^{1-
1/(s- \sigma)}.
$$

%
%
Note
that $\|v(t)\|_{H^s(\ci)}$ is uniformly bounded for all $\ee > 0$. More 
precisely, by
the triangle inequality,  \eqref{lem100u}, and \eqref{u_x-Linfty-Hs},
we have
%
%
\begin{equation}
	\label{bound-no-ep}
\begin{split}
\|v(t) \|_{H^s(\ci)}
\le 4 \|u_0\|_{H^s(\ci)}, \quad |t| \le T.
\end{split}
\end{equation}
%
%
Hence
\begin{equation*}
\|v\|_{H^{s-1}(\ci)} = o(\ee)
\end{equation*}
%
%
which implies
\begin{equation}
	\label{e-decay}
	\ee^{-1} \|v\|_{H^{s-1}(\ci)} = o(1).
\end{equation}
%
%
Substituting \eqref{e-decay} into \eqref{15u}, we obtain
%
%
\begin{equation}
\begin{split}
\frac{d}{dt} \|v\|_{H^s(\ci)} \lesssim
\|v\|_{H^s(\ci)}^2 + \|v\|_{H^s(\ci)} + o(1).
\label{202x}
\end{split}
\end{equation}
%
%
Letting $y(t) = \|v(t)\|_{H^s(\ci)}$, we can factor the right-hand side to 
obtain
%
%
\begin{equation}
	\label{y-express}
	\begin{split}
		\frac{dy}{dt} \lesssim (y-\alpha)(y-\beta)	
	\end{split}
\end{equation}
%
%
where
%
%
\begin{equation}
	\begin{split}
		\alpha = \frac{-1 + \sqrt{1-o(1)}}{2} \qquad \text{and} \qquad
		\beta = \frac{-1 - \sqrt{1-o(1)}}{2}.
	\end{split}
\end{equation}
%
%
Rewriting \eqref{y-express} yields
%
%
\begin{equation*}
	\begin{split}
		\left( \frac{1}{y-\alpha} - \frac{1}{y-\beta} 
		\right) \frac{dy}{dt} \lesssim \sqrt{1- o(1)} \approx 1.
	\end{split}
\end{equation*}
%
%
Noting that $1/(y - \alpha) - 1/(y - \beta)$ is positive, and integrating 
from $0$ to $t$, we obtain 
%
%
\begin{equation*}
	\begin{split}
		\ln \left (\frac{y(t) - \alpha}{y(t) - \beta} \cdot 
		\frac{y(0) - \beta}{y(0) - \alpha} \right ) \le ct.	
	\end{split}
\end{equation*}
%
%
Exponentiating both sides and rearranging gives
%
%
\begin{equation*}
	\begin{split}
		\frac{y(t) - \alpha}{y(t) - \beta} \le e^{ct} \cdot
		\frac{y(0) - \alpha}{y(0) - \beta}	
	\end{split}
\end{equation*}
%
%
which implies
%
%
\begin{equation*}
	\begin{split}
		y(t) 
		& \le e^{ct} \cdot \frac{\left [y(0) - \alpha \right ]
		\left [y(t) - \beta \right ]}{y(0) - 
		\beta} + \alpha
		 \lesssim \left [y(0) - \alpha \right ] \left [y(t) - \beta \right ] + \alpha, \quad |t| \le T
	\end{split}
\end{equation*}
%
%
where the last step follows from the fact that $1/2 \le -\beta \le 1$.  Substituting back in $\|v\|_{H^s(\ci)}$, we obtain
\begin{equation}
	\label{key-decay-ineq}
	\begin{split}
		\|v\|_{H^s(\ci)}  \lesssim \left [\|v(0)\|_{H^s(\ci)} - 
		\alpha \right ] \left [\|v\|_{H^s(\ci)} - \beta \right ] + \alpha.
	\end{split}
\end{equation}
Noting that $\|v\|_{H^s(\ci)}$ is uniformly bounded in $\ee$ by 
\eqref{bound-no-ep}, $\alpha \to 0$, and
%
%
\begin{equation*}
\label{303''qx}
\begin{split}
\|v(0)\|_{H^s(\ci)} = \|u_0 - J_\ee u_0 \|_{H^s(\ci)} \to 0 \end{split}
\end{equation*}
by  \cref{lem4r}, we conclude from \eqref{key-decay-ineq} that
%
%
\begin{equation}
\label{304qx}
\begin{split}
\|v(t)\|_{H^s(\ci)} = 
\|u(t) - u^\ee(t) \|_{H^s(\ci)}= o(1), \quad |t| \le T.
\end{split}
\end{equation}
%
%
Choosing $\ee$ sufficiently small gives $\|v(t)\|_{H^s(\ci)} < \eta/3$, 
completing the proof of \eqref{enough_to_prove1}. \qed
%
%
%
%
%
%
\begin{proof}[Proof of \eqref{enough_to_prove2}] Let $v = u^\ee_n - u^\ee$. Then 
$v$ solves the Cauchy problem
\begin{align}
\label{4qu}
& \p_t v  =  -\gamma (v \p_x v + v \p_x u^\ee + u^\ee \p_x v)  \\
& \phantom{\p_t v  =} - D^{-2} \p_x \left\{\left (\frac{3-\gamma}{2} \right )(v^2 +
2u^\ee v) + \frac{\gamma}{2}\left[ (\p_x v)^2 + 2 \p_x u^\ee \p_x v \right]
\right\}, \notag
\\
& v(0) =J_\ee(u_{0,n} - u_0).
\label{5qu}
\end{align}
Applying the operator $D^s$ to both sides of \eqref{4qu}, multiplying by
$D^s$ and integrating, and estimating as in \eqref{8'u}-\eqref{14u}, we 
obtain
%
%
\begin{equation*}
\begin{split}
\frac{1}{2}\frac{d}{dt}\|v\|_{H^{s}(\ci)}^2
& \lesssim \|v\|_{H^s(\ci)}^3 + \|u^\ee\|_{H^s(\ci)}
\|v\|_{H^s(\ci)}^2
 + \|u^\ee\|_{H^{s+1}(\ci)}
\|v\|_{H^{s-1}(\ci)} \|v\|_{H^s(\ci)}
\end{split}
\end{equation*}
%
%
which by differentiating the left-hand side and applying \cref{lem5r} to 
the right-hand side simplifies to
\begin{equation}
\begin{split}
\frac{d}{dt}\|v\|_{H^{s}(\ci)}
& \lesssim \|v\|_{H^s(\ci)}^2 + \|v\|_{H^s(\ci)}
+ \ee^{-1}
\|v\|_{H^{s-1}(\ci)}.
\label{15qu}
\end{split}
\end{equation}
%
%
We now aim to control of the growth the $\ee^{-1}
\|v\|_{H^{s-1}(\ci)}$ term of \eqref{15qu}. To do so, we will need an estimate for
$\|v\|_{H^{s-1}(\ci)}$, which we will obtain using interpolation. First, 
we will need the following:
%
%
%
%
\begin{proposition}
\label{prop:left}
For $\sigma$ such that $1/2 < \sigma < 1$ and $\sigma + 1 \le s$, 
%
%
\begin{equation}
\label{prop:6rq}
\begin{split}
\|v\|_{H^{\sigma}(\ci)} = \|u^\ee_n - u^\ee\|_{H^\sigma(\ci)}
\lesssim \|u_0 - u_{0,n} \|_{H^s(\ci)}, \quad |t| \le T.
\end{split}
\end{equation}
%
%
\end{proposition}
%
%
%
\begin{proof}
Repeating calculations \eqref{X}-\eqref{12}, with $E$ set to zero, 
$u^{\omega,n}$
replaced by $u^\ee_n$, $u_{\omega,n}$ replaced by $u^\ee$, and $\sigma$ and 
$\rho$ chosen such that
%
%
\begin{equation}
\label{size_of_sigma'}
\begin{split}
	& 1/2 < \sigma < 1 \ \ \text{and} \ \  \sigma + 1 \le \rho \le s
\end{split}
\end{equation}
%
%
yields
%
%
\begin{equation*}
\begin{split}
\frac{1}{2}\frac{d}{dt} \|v\|_{H^\sigma(\ci)}^2
& \lesssim
(\|u^{\ee}_n + u^\ee \|_{H^{\rho}(\ci)} +
\|\p_x(u^{\ee}_n + u^\ee) \|_{H^\sigma(\ci)})
\cdot \|v\|_{H^\sigma(\ci)}^2.
\end{split}
\end{equation*}
%
%
Since $u_{0,n} \to u_{0}$ in $H^s(\ci)$, it follows that 
%
\begin{equation}
\begin{split}
\label{12qx}
\frac{1}{2}\frac{d}{dt} \|v\|_{H^{\sigma}(\ci)}^2
& \le
C \|v\|_{H^{\sigma}(\ci)}^2.
\end{split}
\end{equation}
%
%
where $C = C(\|u_0\|_{H^s(\ci)})$. 
Applying Gronwall's inequality to \eqref{12qx}, we obtain
%
%
\begin{equation*}
\begin{split}
\|v\|_{H^{\sigma}(\ci)}
& \le e^{C t}\|v(0)\|_{H^{\sigma}(\ci)}
= e^{C t}\|u^\ee(0) - u^\ee_n(0) \|_{H^{\sigma}(\ci)} \le e^{C t} \|u_0 - 
u_{0,n}\|_{H^\sigma(\ci)}
\end{split}
\end{equation*}
%
%
concluding the proof. 
\end{proof}
%
%
%

We now return to analyzing the $\ee^{-1}
\|v\|_{H^{s-1}(\ci)}$ term of \eqref{15qu}.
Applying the 
interpolation estimate \eqref{16u} and
\cref{prop:left} gives
%
%
%
%
\begin{equation}
\begin{split}
\label{200qx}
\|v\|_{H^{s-1}(\ci)} 
& \lesssim  
\|u_0-u_{0,n}\|_{H^s(\ci)}^{1/(s-\sigma)}\|v\|_{H^s(\ci)}^{1- 
1/(s-\sigma)}.
\end{split}
\end{equation}
%
%
%
Note that the triangle inequality, \eqref{u_x-Linfty-Hs},
and \eqref{lem100u} 
imply that $\|v\|_{H^s(\ci)}$ is uniformly bounded in $n$ \emph{and} $\ee$. 
That is
%
%
\begin{equation*}
\begin{split}
	\|v\|_{H^s(\ci)} \lesssim \|u_0 \|_{H^s(\ci)}, \quad |t| \le T.
\label{growth_v}
\end{split}
\end{equation*}
%
%
Hence, \eqref{200qx} gives
%
%
\begin{equation}
\begin{split}
\label{200qxr}
\|v\|_{H^{s-1}(\ci)} 
& \lesssim  
\|u_0-u_{0,n}\|_{H^s(\ci)}^{1/(s-\sigma)}.
\end{split}
\end{equation}

Fix $\ee, \rho > 0$. Since $\|u_0 -
u_{0,n} \|_{H^s(\ci)} \to 0$, we
can find $N \in \mathbb{N}$ such that for all $n > N$
%
%
\begin{equation*}
\begin{split}
\ee^{-1} \|u_0-u_{0,n}\|_{H^s(\ci)}^{1/(s-\sigma)}
& < \rho
\label{uniform_n}
\end{split}
\end{equation*}
%
%
which by \eqref{200qxr} implies
%
%
\begin{equation}
	\label{end-decay}
	\begin{split}
		\ee^{-1} \|v\|_{H^{s-1}(\ci)} \lesssim \rho.
	\end{split}
\end{equation}
%
%
%
%
Since $\rho$ can be chosen to be arbitrarily small, the remainder of the 
proof is analogous to that of \eqref{enough_to_prove1}. 
\end{proof}
%
%
%
%
%
%
%
\section{Extending Well-Posedness for HR to the Non-Periodic Case}
\label{sec:defs}
The method will be analogous to that of the periodic case, with two major
modifications. First, we must choose a different mollifier $J_\ee$ in the
proof of continuous dependence. Pick a
function $j(x) \in \mathcal{S}(\rr)$ such that
\begin{equation*}
\begin{split}
& 0 \le \widehat{j}(\xi) \le 1,
\\
& \widehat{j}(\xi) = 1 \ \ \text{if} \ \ |\xi| \le 1.
\end{split}
\end{equation*}
Letting
\begin{equation*}
\begin{split}
j_\ee(x) = \frac{1}{\ee} j \left (\frac{x}{\ee} \right )
\end{split}
\end{equation*}
it can be verified that 
\begin{equation*}
\begin{split}
\widehat{j_\ee}(\xi) = \widehat{j }(\ee \xi), \quad \ee > 0.
\end{split}
\end{equation*}
We then define $J_\ee$ to be the ``Friedrichs mollifier''
\begin{equation*}
\begin{split}
J_\ee f(x) = j_\ee * f(x), \quad \ee>0.
\end{split}
\end{equation*}
Given this construction, the proofs of \cref{lem5r} and \cref{lem4r} for the non-periodic case will be
analogous to those in the periodic case.
Secondly, in the proof of existence, we will have difficulties in arranging
that the solutions $\{u_\ee\}$ to the mollified HR i.v.p.\ converge in $C(I,
H^{s- \sigma}(\rr))$, $0 < \sigma < 1$ to a candidate solution $u$ of the HR
i.v.p. We will get around this by considering the family $\left\{ \varphi
u_\ee \right\}$ instead, where $\vp \in S(\rr)$.
%
%
%
%
We divide our work into three parts:
\subsection{Existence.}
Mirroring the argument in the periodic case, we see that the bounded
family $\{u_\ee\}$ is compact in the weak* topology of $L^\infty(I,
H^{s}(\rr))$. More precisely, there is a sequence  $\{ u_{\ee_n} \}$
converging weak* to a $ u\in L^{\infty}(I, H^s(\rr))$; that is 
%
\begin{equation*}
\label{hhweak-conv}
\lim_{n\to \infty} T_{u_{\ee_n}}(\varphi)  =  T_u (\varphi) 
\; \;		
\ \text{for all} \  \;\;  \varphi \in L^1(I, H^{s}(\rr))
\end{equation*}
where
\begin{equation}
T_v(\varphi) = \int_I <v (t), \varphi (t)>_{H^s(\rr)} dt  = \int_I
\int_\rr
\widehat{v}(\xi, t) \bar{\widehat{\varphi}} (\xi, t) \cdot (1 +
\xi^2)^s \ d \xi \; dt.
\end{equation}
%
Similarly, $\left\{ \p_x u_{\ee_n} \right\}$ is compact in the
weak* topology of $L^\infty(I, H^{s-1}(\rr))$ and converges weak*
to $\p_x u$. Hence, for any $k \in \mathbb{N}$, we have
\begin{align}
\label{base-weak}
& (u_{\ee_n})^k \xrightarrow{\text{weak*}} u^k \ \
\text{on} \ \
L^\infty(I, H^s(\rr)),
\\
\label{base-weak-2}
& (\p_x u_{\ee_n})^k \xrightarrow{\text{weak*}} (\p_x u)^k
\ \ \text{on} \ \
L^\infty(I, H^{s-1}(\rr)). 
\end{align}
In order to show that $u$ solves the HR i.v.p., it would
suffice to obtain a stronger convergence for  $u_{\ee_n}$ so that 
we could take the limit in the mollified HR equation. However,
this is difficult, and unnecessary. Rather, our approach will be to
show that for any pseudo-differential operator
$P \in \Psi^0$ and arbitrary $\vp \in S(\rr)$, $k \in
\mathbb{N}$, $0< \sigma < 1$, we have
%
%
\begin{align}
\label{hhstrong-conv}
& \varphi P [(u_{\ee_n})^k] \longrightarrow \varphi P [u^k]  
\quad
\ \text{in} \  \,\,   C(I, H^{s-\sigma}(\rr)), \ \,
\\
\label{hhstrong-conv-next}
& \varphi P [(\p_x u_{\ee_n})^k] \longrightarrow \varphi P
[(\p_x u)^k]  
\quad
\ \text{in} \  \,\,   C(I, H^{s-\sigma -1}(\rr)), \ \ 
\end{align}
%
which will then be applied to a rewritten version of the HR
i.v.p. Our focus will be on proving \eqref{hhstrong-conv}; since the proof of
\eqref{hhstrong-conv-next} is similar, we will omit the
details. First, we will need the following
interpolation result:
%%%%%%%%%%%%%%%%%%%%%%%%%%%
%
%
%                 Interpolation Lemma
%
%
%%%%%%%%%%%%%%%%%%%%%%%%%%%
\begin{lemma}
\label{hhinterpolation-lem}
(Interpolation)     Let  $s > \frac{3}{2}$.
If $v \in C(I, H^s(\rr)) \cap C^1(I, H^{s-1}(\rr))$
then $v \in C^\sigma (I, H^{s- \sigma}(\rr))$ for  $0 < \sigma < 1$.
\end{lemma}
%
\begin{proof} It is analogous to the proof in the periodic case.
\end{proof}
Fix $k \in \mathbb{N}$. Using \cref{hhinterpolation-lem}, we
will show that the family
\begin{equation*}
\begin{split}
\{\varphi P[(u_\ee)^k]\}_\ee
\end{split}
\end{equation*}
is equicontinuous in $C(I, H^{s-\sigma}(\rr))$ 
for $0 < \sigma < 1$ and $\varphi \in \mathcal{S}(\rr)$.
We will follow this by proving that
there exists a sub-family $\{\varphi P[(u_{\ee_n}(t))^k]\}_n$
that is precompact in $H^{s-\sigma}(\rr)$ for $\sigma > 0$. 
These two facts, in conjunction with Ascoli's Theorem, will
yield
\begin{equation*}
\label{hhstrong-conv2}
\varphi P[(u_\ee)^k] \to \tilde{u}
\; \; \text{in} \; \; C(I,H^{s-\sigma}(\rr))
\end{equation*}
for $0 < \sigma < 1$.
We will then show that $\tilde{u} = \varphi P[u^k]$, from which it will
follow that
\begin{equation*}
\label{hhphiplus}
\begin{split}
\varphi P[(u_\ee)^k] \to \varphi P[u^k]
\; \; \text{in} \; \; C(I,H^{s-\sigma}(\rr)).
\end{split}
\end{equation*}
%%%%%%%%%%%%%%%%%%%%%%
%
%
%       Equicontinuity
%
%
%%%%%%%%%%%%%%%%%%%%%%
%
\subsection{Equicontinuity of $\{ \varphi P [(u_\ee)^k]\}_\ee$  in $C(I,
H^{s-\sigma}(\rr))$}
%
%
Since $\varphi \in \mathcal{S}(\rr)$, the map $u \mapsto \vp u$
is a bounded linear function on $H^s(\rr)$, for arbitrary $s \in
\rr$, where  
\begin{equation}
\begin{split}
\|\varphi u\|_{H^s(\rr)} \le C(s, \varphi)
\|u\|_{H^s(\rr)}, \quad \forall s\in \rr.
\label{hhschwartz-estimate}
\end{split}
\end{equation}
Furthermore, $$P: H^s(\rr) \to H^s(\rr)$$ is bounded and linear,
with 
\begin{equation}
\label{operator-normaa}
\|P\|_{L(H^s(\rr), H^s(\rr))} \le 1.
\end{equation}
Hence, the map 
\begin{equation}
\label{the-map}
\begin{split}
& T: H^s(\rr) \to H^s(\rr),
\\
& T(u) = \vp P u 
\end{split}
\end{equation}
is bounded and linear, with 
\begin{equation}
\begin{split}
\|T\|_{L(H^s(\rr), H^s(\rr))} \le C(s, \vp).
\label{op-norm-product}
\end{split}
\end{equation}
Therefore, applying \cref{hhinterpolation-lem} gives 
%
\begin{equation*}
\begin{split}
\label{hhequic-1}
& \sup_{t \neq t'} \frac {\| \varphi P [(u_\ee(t))^k] - \varphi
P [(u_\ee(t'))^k] \|_{H^{s -
\sigma  }(\rr)}}{|t - t'|}
\\
& \le \sup_{t \neq t'}  \frac {\|\vp P \|_{L(H^{s-\sigma}(\rr),
H^{s-\sigma}(\rr))} \cdot \|   [u_\ee(t)]^k  - 
[u_\ee(t')]^k \|_{H^{s -
\sigma }(\rr)}}{|t - t'|}
\\
& \le C(s, \vp) \cdot \sup_{t \neq t'}  \frac {\|   [u_\ee(t)]^k  - 
[u_\ee(t')]^k \|_{H^{s -
\sigma }(\rr)}}{|t - t'|}
\\
&< c
\end{split}
\end{equation*}
%
or
%
\begin{equation*}
\label{hhequic-2}
\|\varphi P [(u_\ee(t))^k] - \varphi
P [(u_\ee(t'))^k \|_{H^{s - \sigma }(\rr)}< c|t -
t'|, 
\ \text{for all} \   \,\,  t, t'\in I,
\end{equation*}
%
which shows that  the family  $\{\varphi P [(u_\ee)^k]\}_\ee$ is
equicontinuous in $C(I, H^{s-\sigma }(\rr))$.  
%
%
%%%%%%%%%%%%%%%%%%%%%%
%
%
%      PreCompactness
%
%
%%%%%%%%%%%%%%%%%%%%%%%%%%
%
%
%
%
%		
\subsection{Precompactness of $\{\varphi P [(u_\ee(t))^k]\}_\ee$ in
$H^{s-\sigma  }(\rr)$}
Applying the algebra property of Sobolev
Spaces, and recalling \eqref{the-map}-\eqref{op-norm-product}, we have
\begin{equation}
\begin{split}
\label{hhcompact-1}
\|\varphi P [(u_\ee(t))^k]\|_{H^{s}(\rr)}
& \le  C(s, \vp) \cdot \|[u_\ee(t)]^k\|_{H^{s}(\rr)}
\\
& \le C(s, \vp) \cdot \|u_\ee(t)\|^k_{H^{s}(\rr)}.
\end{split}
\end{equation}
%
Letting $|t| \le T$, we now apply \cref{hr_wp} to
\eqref{hhcompact-1} to obtain
\begin{equation*}
\begin{split}
\|\varphi P [(u_\ee(t))^k]\|_{H^{s}(\rr)}
\le 2^k C(s, \vp) \cdot  \|u_0 \|^k_{H^s(\rr)} < \infty.
\end{split}
\end{equation*}
Therefore, by Rellich's Theorem, the family $\left\{
\varphi P [(u_\ee(t))^k] \right\}_\ee$ is
precompact in $H^{s- \sigma }(\rr)$ for all $\sigma > 0$ and $|t| \le T$. 
\\
\\
Hence, compiling our previous results on equicontinuity and precompactness
and applying Ascoli's Theorem, we
conclude that we can find $\tilde{u}$ and a subfamily 
\\ $\left\{
\varphi P [(u_{\ee_n})^k]
\right\}_n$ such that
\begin{equation}
\label{hhstrong-conv-of-u_ep}
\varphi P [(u_{\ee_n})^k] \to \tilde{u}
\; \; \text{in} \; \; C(I, H^{s-\sigma}(\rr)).
\end{equation}
%
%
We would now like to find out what $\tilde{u}$ is:
%
%
%
\begin{lemma}
\label{hhlem:crit-conv}
For arbitrary $k \in \mathbb{N}$,
\begin{equation}
\begin{split}
\varphi P [(u_{\ee_n})^k] \xrightarrow{weak^*}
\varphi P [u^k] \ \ \text{on} \ \ L^\infty(I,
H^{s-\sigma}(\rr)).
\label{hhcrit-conv-est}
\end{split}
\end{equation}
\end{lemma}
\begin{proof} 
Fix $k \in \mathbb{N}$ and recall that the operators 
\begin{equation*}
\begin{split}
& T_\varphi: H^s(\rr) \to H^s(\rr)\\
& T_\varphi u = \varphi u
\end{split}
\end{equation*}
and 
\begin{equation*}
\begin{split}
P:H^s(\rr) \to H^s(\rr)
\end{split}
\end{equation*}
are continuous; therefore 
\begin{equation*}
\begin{split}
T_\vp P: H^s(\rr) \to H^s(\rr)
\end{split}
\end{equation*}
continuously. Since $H^{s}$, it follws that the adjoint operator $(T_\varphi P)^*$
exists and
\begin{equation*}
(T_\varphi P)^*: H^s(\rr) \to H^s(\rr) 
\end{equation*}
continuously. Therefore, applying \eqref{base-weak}, we conclude that
\begin{equation}
\label{widpseudo}
\begin{split}
& \int_I <\varphi P[u^k] - \varphi
P [(u_{\ee_n})^k],\  f>_{H^{s-\sigma }(\rr)} dt
\\
&= \int_I <u^k - 
(u_{\ee_n})^k, \ (T_\vp P)^* f>_{H^{s-\sigma }(\rr)} \to 0
\end{split}
\end{equation}
completing the proof. 
\end{proof}
%
%
Now, recalling \eqref{hhstrong-conv-of-u_ep} and applying \cref{hhlem:crit-conv}, we obtain
\begin{equation}
\begin{split}
\vp P [(u_{\ee_n})^k] \to \vp P [u^k] \ \ \text{in}  \ \ C(I,
H^{s-\sigma}(\rr))
\label{hhvp_u_ep_conv}
\end{split}
\end{equation}
for arbitrary $k \in \mathbb{N}$.  Using precisely the same
strategy we used to prove \eqref{hhvp_u_ep_conv} (applied now to
the family $\{ \vp P [(\p_x u_{\ee})^k] \}_\ee$), one can also show
\begin{equation}
\begin{split}
\vp P [ (\p_x u_{\ee_n})^k] \to \vp P [(\p_x u)^k] \ \ \text{in}  \ \ C(I,
H^{s-\sigma -1 }(\rr)).
\end{split}
\end{equation}
We summarize our result below:
\begin{theorem}
\label{hhthm:crit1}
Let $P \in \Psi^0$ be a pseudo-differential operator. Then for
arbitrary $k \in \mathbb{N}$, 
\begin{equation}
\begin{split}
& \vp P [(u_{\ee_n})^k] \to \vp P [u^k] \ \ \text{in}  \ \ C(I,
H^{s-\sigma }(\rr)),
\\
& 
\vp P [(\p_x u_{\ee_n})^k] \to \vp P [(\p_x u)^k] \ \
\text{in}  \ \ C(I,
H^{s-\sigma -1}(\rr)).
\end{split}
\end{equation}
\end{theorem}
\subsection{Verifying that the weak* limit $u$ solves the HR equation.} 
We recall the mollified HR i.v.p
\begin{align}
& \p_t u_{\ee_n}  = -\gamma(J_{\ee_n} u_{\ee_n} \cdot \p_x
J_{\ee_n} u_{\ee_n}) - \p_x(1- \p_x^2)^{-1} \left( \frac{3-\gamma}{2} u^2
+ \frac{\gamma}{2} (\p_x u)^2 \right ) 
\label{hh1gr}
\\
& u(x,0) = u_0(x).
\label{hh2gr}
\end{align}
Multiplying both sides of \eqref{hh1gr} by $\varphi$ and rewriting,
we obtain
\begin{equation}
\label{hh3}
\begin{split}
\p_t(u_{\ee_n} \varphi) = -\gamma \vp (J_{\ee_n} u_{\ee_n} \cdot
J_{\ee_n} \p_x u_{\ee_n}) - \p_x(1- \p_x^2)^{-1} \left( \frac{3-\gamma}{2} u^2
+ \frac{\gamma}{2} (\p_x u)^2 \right ).
\end{split}
\end{equation}
The following lemma will play a crucial role in our proof of the
existence of a solution to the HR i.v.p.
\begin{lemma}
\label{hhlem:cc}
For $\vp \in \mathcal{S}(\rr)$ such that
$\vp^\frac{1}{2} \in \mathcal{S}(\rr)$, we have
\begin{equation}
\begin{split}
\label{hhburgers_and_nonlocal_conv}
& \vp (J_{\varepsilon_n} u_{\varepsilon_n} 
\cdot J_{\varepsilon_n}\partial_x u_{\varepsilon_n}) 
\to \vp u \partial_x u \; \; 
\text{in} \; \;
C(I, H^{s-\sigma-1}(\rr)). 
\end{split}
\end{equation}
\end{lemma}
%
\begin{proof} We will need a couple of propositions:
\begin{proposition}
For arbitrary $\vp \in \mathcal{S}(\rr)$
\label{hhprop:1aa}
\begin{equation}
\begin{split}
\vp J_{\ee_n} u_{\ee_n} \to \vp u \ \ \text{in} \ \
C(I, H^{s-\sigma}(\rr)).
\label{hh}
\end{split}
\end{equation}
\end{proposition}
\begin{proof}
Note that
\begin{equation}
\begin{split}
& \|\vp u - \vp J_{\ee_n} u_{\ee_n}
\|_{C(I, H^{s-\sigma}(\rr))}
\\
&= \|\vp u - \vp J_{\ee_n} u_{\ee_n} \pm \vp
u_{\ee_n} \|_{C(I, H^{s-\sigma}(\rr))}
\\
& = \|\vp u - \vp u_{\ee_n}
\|_{C(I, H^{s-\sigma}(\rr))} + \|\vp (I - J_{\ee_n})
u_{\ee_n} \|_{C(I, H^{s-\sigma}(\rr))}.
\label{hh1bb}
\end{split}
\end{equation}
Applying \eqref{hhschwartz-estimate} and the estimates
\begin{equation*}
\begin{split}
& \|I-J_{\ee_n} \|_{L(H^{s-\sigma}(\rr), H^{s -
\sigma}(\rr))} = o(1),
\\
& \|u_{\ee_n}\|_{H^{s-\sigma}(\rr)} \le 2
\|u_0\|_{H^{s-\sigma}(\rr)}
\end{split}
\end{equation*}
to \eqref{hh1bb} gives
\begin{equation}
\label{hh2bb}
\begin{split}
\|\vp u - \vp J_{\ee_n} u_{\ee_n}\|_{H^{s-\sigma}(\rr)}
\le \left( \|\vp u - \vp u_{\ee_n}
\|_{C(I, H^{s-\sigma}(\rr))} + C(s, \vp) \cdot o(1) \cdot \|u_0
\|_{H^{s-\sigma}(\rr)} \right).
\end{split}
\end{equation}
Letting $\ee \to 0$ in \eqref{hh2bb} and applying \cref{hhthm:crit1} completes
the proof.
\end{proof}
%
%
\begin{proposition}
\label{hhprop:dd}
For arbitrary $ \vp \in \mathcal{S}(\rr)$,
\begin{equation}
\begin{split}
\vp J_{\ee_n} \p_x u_{\ee_n} \to \vp u \ \
\text{in} \ \ C(I, H^{s-\sigma-1}(\rr)).
\label{hh0dd}
\end{split}
\end{equation}
\end{proposition}
\begin{proof} The result follows from \cref{hhthm:crit1}.
The proof is nearly identical to that of
\cref{hhprop:1aa}, with $s-1$ substituted for $s$
and $\p_x u_{\ee_n}$ substituted for $u_{\ee_n}$. 
\end{proof}
%
%
We now have enough tools to prove \cref{hhlem:cc}. Restrict the
choice of $\vp$ such that $\vp^\frac{1}{2} \in S(\rr)$
(such Schwartz functions exist; as an example, take the square
of the Gaussian). Using this fact, and applying \cref{hhprop:1aa} and \cref{hhprop:dd}, we conclude that
\begin{equation*}
\begin{split}
\vp J_{\ee_n} u_{\ee_n} \p_x J_{\ee_n} u_{\ee_n} 
& = \vp^\frac{1}{2} J_{\ee_n} u_{\ee_n} \cdot
\vp^\frac{1}{2} \p_x J_{\ee_n} u_{\ee_n}
\\
& \to \vp^\frac{1}{2} u \cdot \vp^\frac{1}{2} \p_x u = \vp
u \p_x u
\end{split}
\end{equation*}
completing the proof of \cref{hhlem:cc}. 
\end{proof}
%
%
%
%
By \cref{hhthm:crit1} it follows immediately that
\begin{equation}
\begin{split}
& \vp \p_x(1- \p_x^2)^{-1} \left( \frac{3-\gamma}{2}
(u_{\ee_n})^2
+ \frac{\gamma}{2} (\p_x u_{\ee_n})^2 \right )
\\
& \to
\vp \p_x(1- \p_x^2)^{-1} \left( \frac{3-\gamma}{2} u^2
+ \frac{\gamma}{2} (\p_x u)^2 \right ) \ \
\text{in} \ \ C(I, H^{s-\sigma-1}(\rr)).
\label{non-localii-convergence}
\end{split}
\end{equation}
Combining \eqref{hhburgers_and_nonlocal_conv} and
\eqref{non-localii-convergence}, and applying the Sobolev Imbedding
Theorem, we deduce 
\begin{equation}
\begin{split}
& -\gamma \vp (J_{\ee_n} u_{\ee_n} \cdot J_{\ee_n} \p_x
u_{\ee_n}) -
\vp \p_x(1- \p_x^2)^{-1} \left( \frac{3-\gamma}{2}
(u_{\ee_n})^2
+ \frac{\gamma}{2} (\p_x u_{\ee_n})^2 \right )
\\
\to & -\gamma \vp u \p_x u -
\vp \p_x(1- \p_x^2)^{-1} \left( \frac{3-\gamma}{2} u^2
+ \frac{\gamma}{2} (\p_x u)^2 \right ) \ \
\text{in} \ \ C(I, C(\rr)).
\label{llloc-non-loc-tog}
\end{split}
\end{equation}
%
Next, we note that the convergence  
%
\begin{equation}
\label{hhweak-conv-2}
T_{\vp u_{\ee_n}}(f)  \longrightarrow  T_{\vp u} (f) \;
\ \text{for all} \  \;  f \in L^1(I, H^{-s}(\rr))
\end{equation}
%
can be restated as 
%
\begin{equation}
\vp u_{\ee_n}  \longrightarrow  \vp u
\quad
\ \text{in} \  \,\,
\mathcal{D}'(I\times \rr).
\end{equation}
%
This implies 
%
\begin{equation}
\label{hhdistib-conv-2}
\p_t(\vp u_{\ee_n})  \longrightarrow  \p_t (\vp u)
\quad
\ \text{in} \   \,\, \mathcal{D}'(I\times \rr).
\end{equation}
%
Since for all $n$ we have 
%
\begin{equation}
\begin{split}
\p_t (\vp u_{\ee_n})
= & -\gamma \vp
(J_{\varepsilon_n} u_{\varepsilon_n}  \cdot
J_{\varepsilon_n}\partial_x u_{\varepsilon_n})
\\
& -
\vp \p_x(1- \p_x^2)^{-1} \left( \frac{3-\gamma}{2} (u_{\ee_n})^2
+ \frac{\gamma}{2} (\p_x u_{\ee_n})^2 \right )
\end{split}
\end{equation}
%
it follows from \eqref{hhdistib-conv-2} and the uniqueness of the
limit in \eqref{llloc-non-loc-tog} that
\begin{equation}
\begin{split}
\p_t (\vp u)
= & -\gamma \vp
u \p_x u - \vp \p_x(1- \p_x^2)^{-1} \left( \frac{3-\gamma}{2} u^2
+ \frac{\gamma}{2} (\p_x u)^2 \right ).
\label{hhadone}
\end{split}
\end{equation}
Further restricting $\vp \in \mathcal{S}(\rr)$ to be nonzero in
$\rr$, we
can divide both sides of \eqref{hhadone} by $\vp$ to obtain
\begin{equation}
\label{hh2yy}
\begin{split}
\p_t  u
= & -\gamma
u \p_x u - \p_x(1- \p_x^2)^{-1} \left( \frac{3-\gamma}{2} u^2
+ \frac{\gamma}{2} (\p_x u)^2 \right ).
\end{split}
\end{equation}
Thus we have constructed a solution $u \in L^\infty(I, H^s(\rr))$
to the HR i.v.p. In fact, $u \in C(I, H^s(\rr))$; the proof is analogous to that in the periodic case.
\subsection{Uniqueness.} The proof is analogous to that in the periodic case.
\subsection{Continuous Dependence.} The proof is analogous to the proof in
the periodic case, with one important caveat. Recall how we defined the operator
$J_\ee$. By construction, the proofs of \cref{lem5r} and \cref{lem4r} for the non-periodic case will be
analogous to those in the periodic case. Hence, how we
construct the mollifier $J_\ee$ plays a critical role in the proofs of
well-posedness for the HR i.v.p.\ in both the periodic and non-periodic cases. %
%
%%%%%%%%%%%%%%%%%%%%%%%%%%%%%%%%%%%%%%%%%%%%%%%%%%%%%
%
%
%				BBM
%
%
%%%%%%%%%%%%%%%%%%%%%%%%%%%%%%%%%%%%%%%%%%%%%%%%%%%%%
%
%
%
%
%

%
%
\appendix
\part{Appendix}
\section{Proofs of Lemmas and Estimates}
\begin{proof}[Proof of \eqref{schwartz-bound}] We first observe that
\begin{equation*}
  \begin{split}
    \wh{\psi_1}(0) = \int_\ci \psi_1 \ dt \le 2 \pi 
  \end{split}
\end{equation*}
%
%
while for $k \ge 0$, we use integration by parts to express the 
the Fourier transform of the periodic extension of $\psi_1$ 
in the following form
%
%
\begin{equation*}
  \begin{split}
    \wh{\psi_1}(\lambda) 
    & = \int_\ci e^{-i \lambda t} \psi_1 (t) \ dt
    = \frac{1}{\left( i \lambda \right)^k } \int_\ci e^{-i \lambda t} 
    \p_t^k \psi_1(t) \ dt
  \end{split}
\end{equation*}
%
%
which in conjunction with the estimate $\left( 1 + |\lambda|  \right)^{k } \le |2 \lambda|^k$ for $\lambda \in \mathbb{N}$ 
gives
%
%
\begin{equation*}
  \begin{split}
    \left( 1 + |\lambda| \right)^k |\wh{\psi_1}(\lambda)|
    & \le |2 \lambda|^k | \int_\ci e^{-i \lambda t} \psi_1(t) \ dt |
    \\
    & \le 2 ^k | \int_\ci e^{-i \lambda t} \p_t^k \psi_1(t) \ dt|
    \\
    & \le 2^k \int_\ci | \p_t^k \psi(t) | \ dt 
    \\
    & \le 2^k \cdot 2 \pi  \|\p_t^k \psi_1(t) \|_{L^\infty(\ci)}
    \\
    & = c_k  
  \end{split}
\end{equation*}
%
which completes the proof.
\end{proof}
%
%
%
\begin{proof}[Proof of \cref{cor:four-mult-est-dual}]
By duality,
%
%
\begin{equation}
  \label{1g}
  \begin{split}
    \|f\|_{L^{\frac{4}{3}}(\ci^2)}
    & = \sup_{g \in L^4(\ci^2)} 
    \frac{|\ \int_{\ci^2} f \bar{g} \ dx dt|}{\|g\|_{L^4(\ci^2)}}
    \\
    & \ge \frac{|\ \int_{\ci^2} f \bar{g} \ dx 
    dt|}{\|g\|_{L^4(\ci^2)}}, \qquad g \in L^4(\ci^2)
    \\
    &  =   \frac{|\sum_{m,n} \wh{f}(m,n) 
    \bar{\wh{g}}\left( m,n \right)| }{ 4 \pi^2 \|g\|_{L^4(\ci^2)}}
  \end{split}
\end{equation}
%
%
where the last step follows from Parseval's theorem. Choose $g$ such that
%
%
\begin{equation*}
  \begin{split}
    \wh{g}(m,n) = \left( 1 + |n - m^2| \right)^{-3/4} \wh{f}(m,n).
  \end{split}
\end{equation*}
%
%
Then by \cref{lem:four-mult-est},
%
%
\begin{equation*}
  \begin{split}
    \|g\|_{L^4(\ci^2)}
    & \le \|\left( 1 + |n-m^2| \right)^{-3/8} 
    \wh{g}(m,n) \|_{\ell^2(\zz^2)}
    \\
    & = \|\left( 1 + |n-m^2| \right)^{-3/8} 
    \left( 1 + |n - m^2| \right)^{-3/4} \wh{f}(m,n)  \|_{\ell^2(\zz^2)}
    \\
    & \lesssim \|f\|_{\ell^2(\ci^2)} < \infty.
  \end{split}
\end{equation*}
%
%
Therefore, substituting into \eqref{1g}, we obtain
%
%
\begin{equation*}
  \begin{split}
    \|f\|_{L^{4/3}(\ci^2)}
    & \ge \frac{\sum_{m,n} |\wh{f}(m,n)|^2 \left( 
    1 + |n-m^2| \right)^{-3/4}}{4 \pi^2 \|g\|_{L^4(\ci^2)}}
    \\
    & \ge \frac{\sum_{m,n} |\wh{f}(m,n)|^2 \left( 
    1 + |n-m^2| \right)^{-3/4}}{4 \pi^2 c \|\left( 1 + |n - m^2| 
    \right)^{3/8} \wh{g}(n,m) \|_{\ell^2(\zz^2)}}
    \\
    & = \frac{ \sum_{m,n} |\wh{f}(m,n)|^2 \left( 1 + |n - m| 
    \right)^{-3/4}}{4 \pi^2 c \left( \sum_{n,m} |\wh{f}(n,m)|^2 \left( 
    1 + |n-m^2| \right)^{-3/4} \right)^{1/2}}
    \\
    & \simeq \|\left( 1 + |n-m^2| \right)^{-3/8} \wh{f}(m,n) \|_{\ell^2(\zz^2)}
  \end{split}
\end{equation*}
%
%
completing the proof. 
\end{proof}
%

%
\begin{proof}[Proof of \cref{1lem:cutoff-loc-soln}]
%
%
\begin{equation*}
	\begin{split}
		\lim_{t_{n} \to t} \|u(\cdot, t) - u(\cdot, t_{n})\|_{\dot{H}^s(\ci)} 
		& = \lim_{t_{n} \to t} \|\psi(t) u(\cdot, t) - \psi(t_n) u(\cdot,
		t_{n})\|_{\dot{H}^s(\ci)} 
		\\
		& = \lim_{t_n \to t} \left[ \sum_{n \in \zzdot}| n |
		^{2s} | \psi(t)  \wh{u}(n, t) - \psi(t_n) \wh{ u}(n, t_n) |^2 \right]^{1/2}
		\\
		& = \lim_{t_n \to t} \left[ \sum_{n \in \zzdot} | n |^{2s} | \int_{\rr} (e^{it \tau} - e^{it_{n} \tau}) \wh{\psi u}(n,
		\tau) d \tau |^2 \right]^{1/2}.
	\end{split}
\end{equation*}
		It is clear that
		%
		%
		\begin{equation*}
			\begin{split}
				| n |
				^{2s} | \int_{\rr} (e^{it \tau} - e^{it_{n}\tau}) \wh{\psi u}(n, \tau) d \tau |^2 
		& \le 4  | n |^{2s} \left ( \int_{\rr} |\wh{\psi u}(n, \tau)| d \tau
		\right )^2 
	\end{split}
\end{equation*}
and 
%
%
\begin{equation*}
	\begin{split}
 \sum_{n \in \zzdot} | n |^{2s} \left ( \int_{\rr} |\wh{\psi u}(n, \tau)| d \tau
		\right ) ^2 
		& = \| |n |^s \wh{\psi u}\|_{\dot{\ell}_n^2 L_\tau^1}
		\\
		& \le \|\psi u \|_{Y^s}^2 
	\end{split}
\end{equation*}
which is bounded by assumption.
Applying dominated convergence completes the proof. 
\end{proof}
%
%
\begin{proof}[Proof of \cref{1lem:schwartz-mult}]
Note that
%
%
\begin{equation*}
	\begin{split}
		\wh{\psi f}\left( n, \tau \right)
		& = \wh{\psi}(\cdot) * \wh{f}(n,
		\cdot)(\tau)
		= \int_\rr \wh{\psi}(\tau_1) \wh{f} \left( n, \tau - \tau_1 \right) 
		d\tau_1
	\end{split}
\end{equation*}
%
%
and hence
%
%
\begin{equation}
	\label{11b}
	\begin{split}
		\|\psi f\|_{\dot{X}^s} 
		& = \left( \sum_{n \in \zzdot} |n|^{2s} \int_\rr \left( 1 + | \tau -
		n^{m} | \right) | \int_\rr \wh{\psi}(\tau_1) \wh{f}\left( n, \tau -
		\tau_1
		\right)  d \tau_1 d \tau |^2 \right)^{1/2}
		\\
		& \le \left( \sum_{n \in \zzdot} |n|^{2s} \int_\rr \left( 1 + | \tau -
		n^{m }
		|
		\right) \left( \int_\rr \wh{\psi}\left( \tau_1 \right) \wh{f}\left( n,
		\tau - \tau_1
		\right)  d \tau_1 d \tau \right)^2 \right)^{1/2}.
	\end{split}
\end{equation}
%
%
Using the relation
%
%
\begin{equation*}
	\begin{split}
		1 + | \tau - n^{m } |
		& = 1 + | \tau + \tau_1 - n^{m} |
		\\
		& \le 1 + | \tau_1 | + | \tau - \tau_1 - n^{m} |
		\\
		& \le \left( 1 + | \tau_1 | \right)\left( 1 + | \tau - \tau_1 -
		n^{m} | \right)
	\end{split}
\end{equation*}
%
%
we obtain
%
%
\begin{equation*}
	\begin{split}
		\eqref{11b}
		& \le \left( \sum_{n \in \zzdot} |n|^{2s} \right.
		\\
		& \times \left . \int_\rr \left(
		\int_\rr \left( 1 + | \tau_1 | \right)^{1/2} | \wh{\psi}(\tau_1) |
		\left( 1 + | \tau - \tau_1 - n^{m} | \right)^{1/2} \wh{f}\left( n, \tau
		- \tau_1
		\right)d \tau_1
		\right)^2 d \tau \right)^{1/2}
	\end{split}
\end{equation*}
%
%
which by Minkowski's inequality is bounded by
%
%
\begin{equation}
	\label{12b}
	\begin{split}
		& \left( \sum_{n \in \zzdot} |n|^{2s}  \right.
		\\
		& \times \left. \left( \int_\rr \left[ \int_\rr
		\left( 1 + | \tau_{1} | \right) | \wh{\psi}(\tau_1) |^2 \left( 1 + |
		\tau - \tau_1 - n^{m} |
		\right) | \wh{f}\left( n, \tau - \tau_1 \right) |^2 d \tau_1 
		\right]^{1/2} d \tau \right)^2 \right)^{1/2}.
	\end{split}
\end{equation}
%
%
Using the change of variable $\tau - \tau_1 \to \lambda$ gives
%
%
\begin{equation*}
	\begin{split}
		\eqref{12b}
		& = \left( \sum_{n \in \zzdot} |n|^{2s}\right.
		\\
		& \times \left.  \left( \int_\rr \left[
		\int_\rr \left( 1 + | \tau_1 | \right) | \wh{\psi}\left( \tau_1
		\right) |^2 \left( 1 + | \lambda - n^{m} | \right) | \wh{f} \left( n,
		\lambda
		\right)|^2 d \tau_1 \right]^{1/2} d \lambda \right)^2 \right)^{1/2}
		\\
		& =  \left( \sum_{n \in \zzdot} |n|^{2s} \right.
		\\
		& \times \left. \left( \int_\rr \left( 1 + |
		\tau_1 |
		\right)^{1/2} | \wh{\psi}(\tau_1) | d \tau_1 \left[ \int_\rr \left( 1 + |
		\lambda - n^{m} |
		\right) | \wh{f}\left( n, \lambda \right) |^2 d \lambda \right]^{1/2}
		\right)^2 \right)^{1/2}
		\\
		& = c_{\psi} \left( \sum_{n \in \zzdot} |n|^{2s} \left( \left[ \int_\rr
		\left( 1 + | \lambda - n^{m} | \right) | \wh{f}\left( n, \lambda
		\right) |^2 d \lambda
		\right]^{\cancel{1/2}} \right)^{\cancel{2}} \right)^{1/2}
		\\
		& = c_{\psi} \|f\|_{\dot{X}^s},
	\end{split}
\end{equation*}
%
%
concluding the proof. 
\end{proof}
%
%
%
%
%
%
\begin{proof}[Proof of \cref{1lem:number-theory1}]
First note that
%
\begin{equation*}
		| - n^m + n_1^m + n_2^m|
		 = 3 | n | |n_1 | |n_2 |.
\end{equation*}
%
%
Hence, it will be enough to show that for $c \ge 0$
%
%
\begin{equation*}
	\begin{split}
		| n | |n_1 | |n_2 | \gtrsim | n |^{\frac{2 + c}{2}}| n_1
		|^{\frac{2-c}{2}}| n_2 |^{\frac{2-c}{2}}
	\end{split}
\end{equation*}
%
%
or, dividing through on both sides by $|n| | n_1 | | n_2 |$ and rearranging terms
%
%
\begin{equation*}
	\begin{split}
		| n |^{c/2} \lesssim | n_1 |^{c/2} | n_2 |^{c/2}.
	\end{split}
\end{equation*}
%
%
But
%
%
\begin{equation*}
	\begin{split}
		| n |^{c/2} &= | n_1 + n_2 |^{c/2}
		\\
		& \le (| n_1 | + |n_2|)^{c/2} 
		\\
		& \le (2\max\{|
		n_1 |, | n_2 |)^{c/2}
		\\
		& \le (2|
		n_1 | | n_2 |)^{c/2}
		\\
		& = 2^{c/2} | n_1 |^{c/2} | n_2 |^{c/2}
	\end{split}
\end{equation*}
completing the proof.
\end{proof}
%
%
%
%
\begin{proof}[Proof of \cref{1lem:number-theory}] Define
%
\begin{equation*}
	\begin{split}
		| - n^{m} + n_1^{m} + n_2^{m }|
		& = | n_{1}^{m} - n^{m} + (n-n_{1})^{m}| 
		\\
		& \doteq f(n).
	\end{split}
\end{equation*}
%
%
For fixed $n_1$, the absolute minima
of $f(n)$ occurs at $n = 1+n_{1}$ ($n = n_1$ is not available by assumption). Next, note that
%
%
\begin{equation*}
	\begin{split}
		f(1+ n_{1}) = | n_{1}^{m} - (1 + n_{1})^m + 1 |
		& = | (1 + n_{1} )^{m} - n_{1}^{m} -1 |.
	\end{split}
\end{equation*}
We now seek a lower bound for the right hand side. By symmetry we may assume
$n_1 >0$ without loss of generality.
%
%
\begin{framed}
\begin{remark}
	By the term ``symmetry'', we mean that
	\begin{equation*}
	\begin{split}
	| [1 + (-n_1)]^m - (-n_1)^m -1 |
	& = | (1 - n_1)^m + n_1^m -1 |
	\\
	& = | (1 + p_1)^m + (-p_1)^m -1 |, \qquad p_1 = -n_1
	\\
	& = | (1 + p_1)^m - (p_1)^m -1 |.
	\end{split}
\end{equation*}
%
%
\end{remark}
\end{framed}
%
%
Then 
%
%
\begin{equation*}
	\begin{split}
	| (1 + n_{1} )^{m} - n_{1}^{m} -1 |
	& = | \sum_{1 \le k \le m-1} c_{k} n_1^{k}|, \qquad \{c_k\} \in
	\mathbb{N}\setminus 0
	 \\
	 & = \sum_{1 \le k \le m-1} c_{k} n_1^{k}
	 \\
	 & \ge c_{m-1}  n_1^{m-1}
	 \\
	 & = c_{m-1}  n_1^{c} n_1^{m-1-c}
	 \\
	 & \gtrsim (1 + n_1)^{c}  n_1^{m-1-c}
	 \\
	 & = n^{c} n_1^{m-1-c}. 
 \end{split}
\end{equation*}
%
%
Since we assumed $n_1 >0$ without loss of generality, it follows that 
%
%
\begin{equation*}
	\begin{split}
		f(n) \gtrsim |n|^{c} | n_1 |^{m-1-c}. 
	\end{split}
\end{equation*}
%
%
But since $f(n)$ is symmetric in $n_1$ and $n_2$, a similar argument shows that
%
%
\begin{equation*}
	\begin{split}
		f(n) \gtrsim |n|^{c} | n_2 |^{m-1-c}. 
	\end{split}
\end{equation*}
%
%
Therefore,
%
%
\begin{equation*}
	\begin{split}
		f(n) \gtrsim | n |^{c}| n_1 |^{\frac{m-1-c}{2}} | n_2 |^{\frac{m-1-c}{2}}
	\end{split}
\end{equation*}
%
%
completing the proof. 
\end{proof}
%
%
\begin{proof}[Proof of \cref{1lem:calc}]
%
%
%
By the change of variable $\theta \mapsto a/2 + x$, we have
%
%
\begin{equation*}
	\begin{split}
		\int_{\rr} \frac{1}{(1 + | \theta |)(1 + | a - \theta |)}d \theta
	= \int_{\rr} \frac{1}{(1 + |  a/2 + x |)(1 + | a/2 - x |)}d x.
	\end{split}
\end{equation*}
%
%
Hence, it suffices to show that
%
%
\begin{equation*}
	\begin{split}
		\int_{\rr} \frac{1}{(1 + | a - \theta |)(1 + | a + \theta |)}d \theta
		\lesssim \frac{\log(2 + | a |)}{1 + | a |}.
	\end{split}
\end{equation*}
%
%
Let us leave the case $a = 0$ for last. By symmetry, the cases $a<0$ and $a >0$
are equivalent. Hence, to cover the case $a \neq0$, we may assume
without loss of generality that $a >0$.
%
%
Then
\begin{equation}
	\label{1a1}
	\begin{split}
		& \int_{\rr} \frac{1}{(1 + | a - \theta |)(1 + | a + \theta |)}d \theta
		\\
		& = \int_{| \theta| \le a+1 } \frac{1}{(1 + | a - \theta |)(1 + | a + \theta
		|)}d \theta + \int_{| \theta| \ge a+1 } \frac{1}{(1 + | a - \theta |)(1 + |
		a + \theta |)}d \theta.
	\end{split}
\end{equation}
Estimating the second integral of \eqref{1a1}, we have
\begin{equation*}
	\begin{split}
		& \int_{| \theta| \ge a+1 } \frac{1}{(1 + | a - \theta |)(1 + | a + \theta
		|)}d \theta 
		\\
		& = \int_{\theta \ge a + 1} \frac{1}{(1 + \theta-a)(1 + \theta+a)} d \theta
		+ \int_{\theta \le -a -1} \frac{1}{(1 + \theta - a) (1 + \theta + a)}d \theta
		\\
		& = \frac{1}{2a} \int_{\theta \ge a + 1} \left[ \frac{1}{1 + \theta -a} -
		\frac{1}{1 + \theta+a} \right] d \theta
		+ \frac{1}{2a} \int_{\theta \le -a-1} \left[ \frac{1}{1 + \theta+a}
		-\frac{1}{1 + \theta -a} \right] d \theta
		\\
		& = \frac{1}{a} \log(1+a)
		\\
		& \lesssim \frac{\log(2 + |a|)}{1 + | a |}.
	\end{split}
\end{equation*}
To evaluate the first integral of \eqref{1a1}, we split into the cases $a \le \theta \le
a+1$, $-a \le \theta \le 0$, $0 \le \theta \le a$, and $a \le \theta \le a+1$.
However, note that 
%
%
\begin{equation*}
	\begin{split}
		& \int_{a}^{a+1} \frac{1}{(1 + | a - \theta |)(1 + | a + \theta |)}d \theta =
		\int_{-a-1}^{-a} \frac{1}{(1 + | a - \theta |)(1 + | a + \theta |)}d \theta,
		\\
		& \int_{0}^{a} \frac{1}{(1 + | a - \theta |)(1 + | a + \theta |)}d \theta =
		\int_{-a}^{0} \frac{1}{(1 + | a - \theta |)(1 + | a + \theta |)}d \theta.
	\end{split}
\end{equation*}
%
%
Therefore, we need only consider the cases $a \le \theta \le a+1$ and $0 \le
\theta \le a$. For the case $a \le \theta \le a+1$, we have have
%
%
\begin{equation*}
	\begin{split}
		\int_{a}^{a+1} \frac{1}{(1 + | a-\theta |)(1 + | a + \theta |)}d \theta
		& = \int_{a}^{a+1} \frac{1}{(1 + \theta -a)(1 + a + \theta)}d \theta
		\\
		& = \frac{1}{2a} \int_{a}^{a+1} \left[ \frac{1}{1 + \theta -a} -
		\frac{1}{1 + \theta + a}  \right]d \theta
		\\
		& =\frac{1}{2a} \log\left( \frac{1 + \theta -a}{1 + \theta + a} \right) \Big
		|_a^{a+1}
		\\
		& = \frac{1}{2a} \log\left( \frac{2a+1}{a+1} \right)
		\\
		& \lesssim\frac{\log 2}{2a}
		\\
		& \lesssim \frac{\log(2 + | a |)}{1 + | a |}.
	\end{split}
\end{equation*}
%
%
while for the case $0 \le \theta \le a$, we have
%
%
\begin{equation*}
	\begin{split}
		\int_{0}^{a} \frac{1}{(1 + | a - \theta |)(1 + | a + \theta |)}d \theta
		& = \int_{0}^{a} \frac{1}{(1 +  a - \theta )(1 +  a + \theta )}d \theta
		\\
		& = \frac{1}{2(1 + a)} \int_{0}^{a} \left[ \frac{1}{1 + a - \theta} +
		\frac{1}{1 + a + \theta} \right]d \theta
		\\
		& = \frac{1}{2(1 + a)} \log \left( \frac{1 + a + \theta}{1 + a - \theta}
		\right) \Big |_{0}^{a}
		\\
		& = \frac{\log\left( 1 + 2a \right)}{2\left( 1 + a \right)}
		\\
		& \lesssim \frac{\log(2 + | a |)}{1 + | a |}.
	\end{split}
\end{equation*}
%
%
This completes the proof for the case $a \neq 0$. Lastly, for the case
$a =0$, we use dominated convergence and our preceding work to
conclude that
%
%
\begin{equation*}
	\begin{split}
		\int_{\rr} \frac{1}{(1 + | \theta|)^2} d \theta
		& = \lim_{a \to 0}
		\int_{\rr} \frac{1}{(1 + | a - \theta |)(1 + | a + \theta |)}d \theta
		\\
		& \lesssim \lim_{a \to 0} \frac{\log(2 + | a |)}{1 + | a |}
		\\
		& =  \log 2
		\\
		& = \frac{\log(2 + | 0 |)}{1 + | 0 |} 
	\end{split}
\end{equation*}
%
which completes the proof.
%
\end{proof}
%
\begin{proof}[Conservation of the $L_x^2$ norm] 
We have
%
%
\begin{equation*}
	\begin{split}
		\frac{d}{dt} \int_\ci | u |^2  dx
		& = \int_\ci \frac{d}{dt} | u |^2  dx
		\\
		& = \int_\ci \frac{d}{dt} \left( u \overline{u} \right)  dx
		\\
		& = \int_\ci \left( u \p_t \overline{u} + \overline{u} \p_t u \right) dx
		\\
		& = \int_\ci \left( u \overline{\p_t u} + \overline{u} \p_t u \right)dx.
	\end{split}
\end{equation*}
%
%
Substituting in $\p_t u = i\left( \p_x^{m} u + | u |^2 u \right)$ we obtain
%
%
\begin{equation*}
	\begin{split}
		& \int_{\ci} \left\{ u\left[ -i\left( \p_x^{m} \overline{u} + | u |^2
		\overline{u} \right) \right] + \overline{u}\left[ i\left( \p_x^{m} u + | u
		|^2 u \right) \right] \right\}dx
		\\
		& = \int_\ci \left[ -iu \p_x^{m} \overline{u} - i| u |^4 + i \overline{u}
		\p_x^{m} u + i | u |^4 \right]dx
		\\
		& = i \int_{\ci}\left( \overline{u} \p_x^{m} u - u \p_x^{m } \overline{u}
		\right)dx.
	\end{split}
\end{equation*}
%
%
Integrating by parts $m/2$ times and using
the spatial periodicity of $u$, the right
hand side simplifies to
%
%
\begin{equation*}
	\begin{split}
		i \int_\ci \left( \p_x^{m/2} \overline{u} \p_x^{m/2} u - \p_x^{m/2} u
		\p_x^{m/2 } 
		\overline{u} \right) dx = 0.
	\end{split}
\end{equation*}
%
%
Therefore, the $L_x^2(\ci)$ norm of solutions to the dNLS is conserved. 
\end{proof}

\begin{proof}[Proof of \cref{1lem:cutoff-loc-soln}]
%
%
\begin{equation*}
	\begin{split}
		\lim_{t_{n} \to t} \|u(\cdot, t) - u(\cdot, t_{n})\|_{\dot{H}^s(\ci)} 
		& = \lim_{t_{n} \to t} \|\psi(t) u(\cdot, t) - \psi(t_n) u(\cdot,
		t_{n})\|_{\dot{H}^s(\ci)} 
		\\
		& = \lim_{t_n \to t} \left[ \sum_{n \in \zzdot}| n |
		^{2s} | \psi(t)  \wh{u}(n, t) - \psi(t_n) \wh{ u}(n, t_n) |^2 \right]^{1/2}
		\\
		& = \lim_{t_n \to t} \left[ \sum_{n \in \zzdot} | n |^{2s} | \int_{\rr} (e^{it \tau} - e^{it_{n} \tau}) \wh{\psi u}(n,
		\tau) d \tau |^2 \right]^{1/2}.
	\end{split}
\end{equation*}
		It is clear that
		%
		%
		\begin{equation*}
			\begin{split}
				| n |
				^{2s} | \int_{\rr} (e^{it \tau} - e^{it_{n}\tau}) \wh{\psi u}(n, \tau) d \tau |^2 
		& \le 4  | n |^{2s} \left ( \int_{\rr} |\wh{\psi u}(n, \tau)| d \tau
		\right )^2 
	\end{split}
\end{equation*}
and 
%
%
\begin{equation*}
	\begin{split}
 \sum_{n \in \zzdot} | n |^{2s} \left ( \int_{\rr} |\wh{\psi u}(n, \tau)| d \tau
		\right ) ^2 
		& = \|\wh{\psi u}\|_{\dot{\ell}_n^2 L_\tau^1}
		\\
		& \le \|\psi u \|_{Y^s}^2 
	\end{split}
\end{equation*}
which is bounded by assumption.
Applying dominated convergence completes the proof. 
\end{proof}
%
%
\begin{proof}[Proof of \cref{2lem:schwartz-mult}]
Note that
%
%
\begin{equation*}
	\begin{split}
		\wh{\psi f}\left( n, \tau \right)
		& = \wh{\psi}(\cdot) * \wh{f}(n,
		\cdot)(\tau)
		= \int_\rr \wh{\psi}(\tau_1) \wh{f} \left( n, \tau - \tau_1 \right) 
		d\tau_1
	\end{split}
\end{equation*}
%
%
and hence
%
%
\begin{equation}
	\label{19b}
	\begin{split}
		\|\psi f\|_{\dot{X}^s} 
		& = \left( \sum_{n \in \zzdot} |n|^{2s} \int_\rr \left( 1 + | \tau -
		n^{m} | \right) | \int_\rr \wh{\psi}(\tau_1) \wh{f}\left( n, \tau -
		\tau_1
		\right)  d \tau_1 d \tau |^2 \right)^{1/2}
		\\
		& \le \left( \sum_{n \in \zzdot} |n|^{2s} \int_\rr \left( 1 + | \tau -
		n^{m }
		|
		\right) \left( \int_\rr \wh{\psi}\left( \tau_1 \right) \wh{f}\left( n,
		\tau - \tau_1
		\right)  d \tau_1 d \tau \right)^2 \right)^{1/2}.
	\end{split}
\end{equation}
%
%
Using the relation
%
%
\begin{equation*}
	\begin{split}
		1 + | \tau - n^{m } |
		& = 1 + | \tau + \tau_1 - n^{m} |
		\\
		& \le 1 + | \tau_1 | + | \tau - \tau_1 - n^{m} |
		\\
		& \le \left( 1 + | \tau_1 | \right)\left( 1 + | \tau - \tau_1 -
		n^{m} | \right),
	\end{split}
\end{equation*}
%
%
we obtain
%
%
\begin{equation*}
	\begin{split}
		\eqref{19b}
		& \le \left( \sum_{n \in \zzdot} |n|^{2s} \right.
		\\
		& \times \left . \int_\rr \left(
		\int_\rr \left( 1 + | \tau_1 | \right)^{1/2} | \wh{\psi}(\tau_1) |
		\left( 1 + | \tau - \tau_1 - n^{m} | \right)^{1/2} \wh{f}\left( n, \tau
		- \tau_1
		\right)d \tau_1
		\right)^2 d \tau \right)^{1/2}
	\end{split}
\end{equation*}
%
%
which by Minkowski's inequality is bounded by
%
%
\begin{equation}
	\label{18a}
	\begin{split}
		& \left( \sum_{n \in \zzdot} |n|^{2s}  \right.
		\\
		& \times \left. \left( \int_\rr \left[ \int_\rr
		\left( 1 + | \tau_{1} | \right) | \wh{\psi}(\tau_1) |^2 \left( 1 + |
		\tau - \tau_1 - n^{m} |
		\right) | \wh{f}\left( n, \tau - \tau_1 \right) |^2 d \tau_1 
		\right]^{1/2} d \tau \right)^2 \right)^{1/2}.
	\end{split}
\end{equation}
%
%
Using the change of variable $\tau - \tau_1 \to \lambda$ gives
%
%
\begin{equation*}
	\begin{split}
		\eqref{18a}
		& = \left( \sum_{n \in \zzdot} |n|^{2s}\right.
		\\
		& \times \left.  \left( \int_\rr \left[
		\int_\rr \left( 1 + | \tau_1 | \right) | \wh{\psi}\left( \tau_1
		\right) |^2 \left( 1 + | \lambda - n^{m} | \right) | \wh{f} \left( n,
		\lambda
		\right)|^2 d \tau_1 \right]^{1/2} d \lambda \right)^2 \right)^{1/2}
		\\
		& =  \left( \sum_{n \in \zzdot} |n|^{2s} \right.
		\\
		& \times \left. \left( \int_\rr \left( 1 + |
		\tau_1 |
		\right)^{1/2} | \wh{\psi}(\tau_1) | d \tau_1 \left[ \int_\rr \left( 1 + |
		\lambda - n^{m} |
		\right) | \wh{f}\left( n, \lambda \right) |^2 d \lambda \right]^{1/2}
		\right)^2 \right)^{1/2}
		\\
		& = c_{\psi} \left( \sum_{n \in \zzdot} |n|^{2s} \left( \left[ \int_\rr
		\left( 1 + | \lambda - n^{m} | \right) | \wh{f}\left( n, \lambda
		\right) |^2 d \lambda
		\right]^{\cancel{1/2}} \right)^{\cancel{2}} \right)^{1/2}
		\\
		& = c_{\psi} \|f\|_{\dot{X}^s},
	\end{split}
\end{equation*}
%
%
concluding the proof. 
\end{proof}

%
\begin{proof}[Proof of \cref{1lem:number-theory}]
First note that
%
\begin{equation*}
		| - n^{3} + n_1^3 + n_2^3|
		 = 3 | n | |n_1 | |n_2 |.
\end{equation*}
%
%
Hence, it will be enough to show that for $c \ge 0$
%
%
\begin{equation*}
	\begin{split}
		| n | |n_1 | |n_2 | \gtrsim | n |^{\frac{2 + c}{2}}| n_1
		|^{\frac{2-c}{2}}| n_2 |^{\frac{2-c}{2}}
	\end{split}
\end{equation*}
%
%
or, dividing through on both sides by $|n| | n_1 | | n_2 |$ and rearranging terms
%
%
\begin{equation*}
	\begin{split}
		| n |^{c/2} \lesssim | n_1 |^{c/2} | n_2 |^{c/2}.
	\end{split}
\end{equation*}
%
%
But
%
%
\begin{equation*}
	\begin{split}
		| n |^{c/2} &= | n_1 + n_2 |^{c/2}
		\\
		& \le (| n_1 | + |n_2|)^{c/2} 
		\\
		& \le (2\max\{|
		n_1 |, | n_2 |)^{c/2}
		\\
		& \le (2|
		n_1 | | n_2 |)^{c/2}
		\\
		& = 2^{c/2} | n_1 |^{c/2} | n_2 |^{c/2}
	\end{split}
\end{equation*}
%
%
where the last step follows from the fact that, for $a, b \in \zz$ 
%\cref{1lem:splitting}. \qquad \qed
%
%\subsection{Proof of \cref{1lem:splitting}.} We have
%%
%%
\begin{equation}
	\label{16a}
	\begin{split}
		| a + b | 
		& \le | a | + | b | 
		\\
		& \le 2\left( \max\{| a |, | b | \}\right)
		\\
		& \le 2 |a| |b|.
	\end{split}
\end{equation} 
This concludes the proof.
\end{proof}
%%
%

%
%
\begin{proof}[Proof of \cref{llem:number-theory}] Define
%
\begin{equation*}
	\begin{split}
		| - n^{m} + n_1^{m} + n_2^{m }|
		& = | n_{1}^{m} - n^{m} + (n-n_{1})^{m}| 
		\\
		& \doteq f(n).
	\end{split}
\end{equation*}
%
%
Then the absolute minima
of $f(n)$ occur only on the line $n = 1+n_{1}$ of lattice points 
(the line $n = n_1$ is not available by assumption). Next, note that
%
%
\begin{equation*}
	\begin{split}
		f(1+ n_{1}) = | n_{1}^{m} - (1 + n_{1})^m + 1 |
		& = | (1 + n_{1} )^{m} - n_{1}^{m} -1 |
		\\
		& = | \sum_{1 \le k \le m-1} c_{k} n_1^{k}|, \qquad \{c_k\} \in \mathbb{N}
		\setminus 0.
	\end{split}
\end{equation*}
If $n_1 = -1$, then clearly $| (1 + n_{1})^m - n_1^m -1 | = 2 \ge 0 = |n|^2
|n_1|^{m-3}$. Hence, by symmetry, and the fact that $n_1 \neq 0$ by assumption,
we may further assume
$n_1 >0$ without loss of generality.
Then 
%
%
\begin{equation*}
	\begin{split}
	  | \sum_{1 \le k \le m-1} c_{k} n_1^{k}|
	 & = \sum_{1 \le k \le m-1} |c_{k}| |n_1|^{k}
	 \\
	 & = |n_1| \sum_{0 \le \ell \le m-2} |c_{\ell}| |n_1|^{k}, \qquad \{c_\ell\}
	 \subset \mathbb{N} \setminus 0
	 \\
	 & \ge c_{m-2}|n_1| | n_1|^{m-2}
	 \\
	 & = c_{m-2}| n_1 |^2 | n_1 |^{m-3}
	 \\
	 & \ge \frac{c_{m-2}}{4} (1 + | n_1 |)^2 | n_1|^{m-3}
	 \\
	 & \simeq n^2 | n_1 |^{m-3}
	\end{split}
\end{equation*}
%
%
completing the proof. 
\end{proof}
%
%
%
\begin{proof}[Proof of \cref{llem:splitting}] We have
%
%
\begin{equation}
	\label{l6a}
	\begin{split}
		1 + | a + b | 
		& \le 1 + | a | + | b | 
		\\
		& \le 1 + | a | + 1 + | b | 
		\\
		& \le 2\left( \max\{1+| a |, 1+| b | \}\right)
		\\
		& \le 2 \left( 1 + | a | \right)\left( 1 + | b | \right), \qquad a, b \in {\zz}.
	\end{split}
\end{equation}
%
%
Raising both sides of expression $\eqref{l6a}$ to the $k$ power completes 
the proof. 
\end{proof}



\begin{proof}[Proof of \cref{nlem:cutoff-loc-soln}]
%
%
\begin{equation}
  \label{ndm}
	\begin{split}
		\lim_{t_{k} \to t} \|u(\cdot, t) - u(\cdot, t_{k})\|_{H^s(\ci)} 
    & = \lim_{t_{k} \to t} \|\psi_{\delta}(t) u(\cdot, t) - \psi_{\delta}(t_{k}) u(\cdot, t_{k})\|_{H^s(\ci)} 
		\\
		& = \lim_{t_{k} \to t} \left[ \sum_{n}\left( 1 + | n |
    \right)^{2s} | \psi_{\delta}(t)  \wh{u}(n, t) - \psi_{\delta}(t_{k}) \wh{ u}(n, t_{k}) |^2 \right]^{1/2}
		\\
		& = \lim_{t_{k} \to t} \left[ \sum_{n} \left( 1 + | n |
    \right)^{2s} | \int_{\rr} (e^{it \tau} - e^{it_{k} \tau})
    \wh{\psi_{\delta} u}(n,
		\tau) d \tau |^2 \right]^{1/2}.
	\end{split}
\end{equation}
First note that
%
%
%
%
\begin{equation*}
\begin{split}
& \lim_{t_{k} \to t}  | \int_{\rr} (e^{it \tau} - e^{it_{k} \tau})
    \wh{\psi_{\delta} u}(n,
		\tau) d \tau |^2 
    \\
    = 
     & \lim_{t_{k} \to t}  \int_{\rr} (e^{it \tau} - e^{it_{k} \tau})
    \wh{\psi_{\delta} u}(n,
    \tau) d \tau \times \lim_{t_{k} \to t} \overline{\int_{\rr} (e^{it \tau} - e^{it_{k} \tau})
    \wh{\psi_{\delta} u}(n,
    \tau) d \tau }  
    \\
    = 
    &  \lim_{t_{k} \to t}  \int_{\rr} (e^{it \tau} - e^{it_{k} \tau})
    \wh{\psi_{\delta} u}(n,
    \tau) d \tau \times \lim_{t_{k} \to t} \int_{\rr} (e^{-it \tau} - e^{-it_{k} \tau})
    \overline{\wh{\psi_{\delta} u}}(n,
    \tau) d \tau.   
    \end{split}
\end{equation*}
%
%
But for fixed $n$ 
%
%
\begin{equation*}
\begin{split}
|(e^{it \tau} - e^{it_{k} \tau})  
    \wh{\psi_{\delta} u}(n, \tau) | \le 2 |\wh{\psi_{\delta} u(n, \tau)} |
\end{split}
\end{equation*}
%
%
and
%
%
%
\begin{equation*}
\begin{split}
  \int_{\rr} |2 \wh{\psi_{\delta} u(n, \tau)} | d \tau < \infty.
\end{split}
\end{equation*}
%
%
Hence, by dominated convergence
%
%
\begin{equation*}
\begin{split}
\lim_{t_{k} \to t}  \int_{\rr} (e^{it \tau} - e^{it_{k} \tau})
    \wh{\psi_{\delta} u}(n,
    \tau) d \tau =  \int_{\rr} \lim_{t_{k} \to t} (e^{it \tau} - e^{it_{k} \tau})
    \wh{\psi_{\delta} u}(n,
    \tau) d \tau = 0. 
\end{split}
\end{equation*}
%
%
Similarly, 
%
%
%
\begin{equation*}
\begin{split}
\lim_{t_{k} \to t} \int_{\rr} (e^{-it \tau} - e^{-it_{k} \tau})
    \overline{\wh{\psi_{\delta} u}}(n,
    \tau) d \tau  =\int_{\rr}  \lim_{t_{k} \to t} (e^{-it \tau} - e^{-it_{k} \tau})
    \overline{\wh{\psi_{\delta} u}}(n,
    \tau) d \tau  = 0.
\end{split}
\end{equation*}
%
%
Hence
%
%
%
\begin{equation}
  \label{ngh}
\begin{split}
  \lim_{t_{k} \to t} | \int_{\rr} (e^{it \tau} - e^{it_{k} \tau})
    \wh{\psi_{\delta} u}(n,
		\tau) d \tau |^2 = 0.
\end{split}
\end{equation}
%
%
		Furthermore,
    %
    %
    \begin{equation*}
    \begin{split}
      (1 + | n |)^{2s} | \int_{\rr} \left( e^{it\tau} - e^{it_{k} \tau} \right)
      \wh{\psi_{\delta}u}(n, \tau) d \tau|^{2} \le 4 (1 + | n |)^{2s} \left(
      \int_{\rr} | \wh{\psi_{\delta} u}(n, \tau)  | d \tau
      \right)^{2}
    \end{split}
    \end{equation*}
    %
    %
    and
		%
		%
		\begin{equation*}
			\begin{split}
         \sum_{n}  \left( 1 + | n |
        \right)^{2s} \left ( \int_{\rr} |\wh{\psi_{\delta} u}(n, \tau)| d \tau
        \right )^2  
        & = \|\wh{\psi_{\delta} u}\|_{\ell^{2}_{n}L^{1}_{\tau}}^2
		\\
		& \le \|\psi_{\delta} u \|_{X_{s,b}}^2 
	\end{split}
\end{equation*}
which is bounded by assumption. Therefore, applying dominated convergence and
\eqref{ngh}, we
obtain 
%
%
\begin{equation*}
\begin{split}
  \text{rhs of \eqref{ndm}} = \left[ \sum_{n} \left( 1 + | n |
    \right)^{2s} \lim_{t_{k} \to t} | \int_{\rr} (e^{it \tau} - e^{it_{k} \tau})
    \wh{\psi_{\delta} u}(n,
		\tau) d \tau |^2 \right]^{1/2} = 0
\end{split}
\end{equation*}
%
%
completing the proof. 
\end{proof}
%
%
\begin{proof}[Proof of \cref{nlem:schwartz-mult}]
Note that
%
%
\begin{equation*}
	\begin{split}
		\wh{\psi f}\left( n, \tau \right)
		& = \wh{\psi}(\cdot) * \wh{f}(n,
		\cdot)(\tau)
		= \int_\rr \wh{\psi}(\tau_1) \wh{f} \left( n, \tau - \tau_1 \right) 
		d\tau_1
	\end{split}
\end{equation*}
%
%
and hence
%
%
\begin{equation}
	\label{n1b}
	\begin{split}
		\|\psi f\|_{X^s} 
		& = \left( \sum_{n \in \zz} \left (1 + |n| \right )^{2s} \int_\rr \left( 1 + | \tau -
		n^{m} | \right) | \int_\rr \wh{\psi}(\tau_1) \wh{f}\left( n, \tau -
		\tau_1
		\right)  d \tau_1 d \tau |^2 \right)^{1/2}
		\\
		& \le \left( \sum_{n \in \zz} \left (1 + |n| \right )^{2s} \int_\rr \left( 1 + | \tau -
		n^{m }
		|
		\right) \left( \int_\rr |\wh{\psi}\left( \tau_1 \right) | |\wh{f}\left( n,
		\tau - \tau_1
		\right) |  d \tau_1 d \tau \right)^2 \right)^{1/2}.
	\end{split}
\end{equation}
%
%
Using the relation
%
%
\begin{equation*}
	\begin{split}
		1 + | \tau - n^{m } |
    & = 1 + | \tau - \tau_1 + \tau_{1} - n^{m} |
		\\
		& \le 1 + | \tau_1 | + | \tau - \tau_1 - n^{m} |
		\\
		& \le \left( 1 + | \tau_1 | \right)\left( 1 + | \tau - \tau_1 -
		n^{m} | \right)
	\end{split}
\end{equation*}
%
%
we obtain
%
%
\begin{equation*}
	\begin{split}
		\eqref{n1b}
		& \le \left( \sum_{n \in \zz} \left (1 + |n| \right )^{2s} \right.
		\\
		& \times \left . \int_\rr \left(
		\int_\rr \left( 1 + | \tau_1 | \right)^{1/2} | \wh{\psi}(\tau_1) |
		\left( 1 + | \tau - \tau_1 - n^{m} | \right)^{1/2} \wh{f}\left( n, \tau
		- \tau_1
		\right)d \tau_1
		\right)^2 d \tau \right)^{1/2}
	\end{split}
\end{equation*}
%
%
which by Minkowski's inequality is bounded by
%
%
\begin{equation}
	\label{n2b}
	\begin{split}
		& \left( \sum_{n \in \zz} \left (1 + |n| \right )^{2s}  \right.
		\\
		& \times \left. \left( \int_\rr \left[ \int_\rr
		\left( 1 + | \tau_{1} | \right) | \wh{\psi}(\tau_1) |^2 \left( 1 + |
		\tau - \tau_1 - n^{m} |
		\right) | \wh{f}\left( n, \tau - \tau_1 \right) |^2 d \tau 
    \right]^{1/2} d \tau_{1} \right)^2 \right)^{1/2}.
	\end{split}
\end{equation}
%
%
Using the change of variable $\tau - \tau_1 = \lambda$ gives
%
%
\begin{equation*}
	\begin{split}
		\eqref{n2b}
		& = \left( \sum_{n \in \zz} \left (1 + |n| \right )^{2s}\right.
		\\
		& \times \left.  \left( \int_\rr \left[
		\int_\rr \left( 1 + | \tau_1 | \right) | \wh{\psi}\left( \tau_1
		\right) |^2 \left( 1 + | \lambda - n^{m} | \right) | \wh{f} \left( n,
		\lambda
    \right)|^2 d \lambda \right]^{1/2} d \tau_{1} \right)^2 \right)^{1/2}
		\\
		& =  \left( \sum_{n \in \zz} \left (1 + |n| \right )^{2s} \right.
		\\
		& \times \left. \left( \int_\rr \left( 1 + |
		\tau_1 |
		\right)^{1/2} | \wh{\psi}(\tau_1) | d \tau_1 \left[ \int_\rr \left( 1 + |
		\lambda - n^{m} |
		\right) | \wh{f}\left( n, \lambda \right) |^2 d \lambda \right]^{1/2}
		\right)^2 \right)^{1/2}
		\\
		& = c_{\psi} \left( \sum_{n \in \zz} \left (1 + |n| \right )^{2s} \left( \left[ \int_\rr
		\left( 1 + | \lambda - n^{m} | \right) | \wh{f}\left( n, \lambda
		\right) |^2 d \lambda
		\right]^{\cancel{1/2}} \right)^{\cancel{2}} \right)^{1/2}
		\\
		& = c_{\psi} \|f\|_{X^s}.
	\end{split}
\end{equation*}
%
Also, by Young's inequality we have the estimate 
%
%
\begin{equation*}
\begin{split}
  \|\wh{\psi f}\|_{\ell^{2}_{n} L^{1}_{\tau}} 
  & = \left[ \sum_{n \in \zz} \left (1 + |n| \right )^{2s} \left (
  \int_{\rr} | \wh{\psi}(\cdot) * \wh{f}(n, \cdot)(\tau) | d \tau  \right ) ^2 \right]^{1/2}
  \\
  & \le  \left[ \sum_{n \in \zz} (1 + | n |)^{2s} \left( \int_{\rr} |
    \wh{\psi}(\tau) | d \tau  \times \int_{\rr} | \wh{f}(n, \tau) | d \tau
    \right)^{2}\right]^{1/2}
  \\
  & = c_{\psi} \| \wh{f} \|_{\ell^{2}_{n} L^{1}_{\tau}}
\end{split}
\end{equation*}
%
%
%
concluding the proof. 
\end{proof}
%
%
\begin{proof}[Proof of \cref{nlem:splitting}] We have
%
%
\begin{equation}
	\label{n6a}
	\begin{split}
		1 + | a + b + c| 
		& \le 1 + | a | + | b | + | c |
		\\
		& \le 1 + | a | + 1 + | b | + 1 + | c |
		\\
		& \le 3\left( \max\{1+| a |, 1+| b |, 1+ | c | \}\right)
		\\
		& \le 3 \left( 1 + | a | \right)\left( 1 + | b | \right) \left( 1 + |
		c |
		\right).
	\end{split}
\end{equation}
%
%
Raising both sides of expression $\eqref{n6a}$ to the $v$ power completes 
the proof. 
\end{proof}
%
%
%
\section{Remarks about mNLS and Related Equations}
%
\begin{proof}[Why \cref{nprop:trilinear-est} fails when dealing
with nonlinearity $\frac{1}{3} \p_x u^3$]
%
%
%
Recalling \eqref{nnon-lin-rep}, we have
\begin{equation}
	\begin{split}
		| \wh{w_{fgh}}(n, \tau)|
    & = | \sum_{n_1, n_2, n_3 = n}  \int_{\tau_{1} + \tau_{2} + \tau_{3} = \tau}
    n \wh{f}\left( n_1,  \tau_1 
\right) \wh{g}\left( n_2, \tau_2  
\right) \wh{h}\left( n_3, \tau_3 \right) d \tau_1 d \tau_2 d \tau_3 |
\\
& \le \sum_{n_1, n_2, n_3 = n}  \int_{\tau_{1} + \tau_{2} + \tau_{3} = \tau}
| n | \times | \wh{f}\left( n_1, \tau_1 
\right) | \times  | \wh{g}\left( n_2, \tau_2 
\right) | \times | \wh{ h}\left( n_3, \tau_3 \right) | d \tau_1 d \tau_2 d 
\tau_3
\\
& = \sum_{n_1, n_2, n_3 = n}  \int_{\tau_{1} + \tau_{2} + \tau_{3} = \tau} \frac{| n |c_f\left( n_1, \tau_1 
\right)}{\left (1 + |n_1| \right )^s \left( 1 + | \tau_1 - n_1^{m} | \right)^{b}}
\\
& \times \frac{c_{g}\left( n_2, \tau_2 \right)}{\left (1 + |n_2| \right ) 
^s\left( 1 + | \tau_2 -  n_2^{m }| 
\right)^{b}}
 \times \frac{c_{h}\left( n_3, \tau_3 \right)}{\left (1 + |n_3| \right ) ^s\left( 1 + | 
\tau_3 - n_3^{m } | \right)^{b}} \ d \tau_1 d \tau_2 d \tau_3
\end{split}
\end{equation}
where 
%
%
\begin{equation*}
	\begin{split}
		c_\sigma(n, \tau) = \left (1 + |n| \right ) ^s \left( 1 + | \tau - n^{m } |  
		\right)^{b} | \wh{\sigma}\left( n, \tau \right) | .
	\end{split}
\end{equation*}
%
%
Hence
%
%
\begin{equation*}
	\begin{split}
		 & \left (1 + |n| \right )^s \left( 1 + | \tau - n^{m } | \right)^{-b} | \wh{w_{fgh}}\left( 
		n, \tau \right) |
		\\
		& \le \left( 1 + | \tau - n^{m } | \right)^{-b}
		\sum_{n_1, n_2, n_3 = n}  \int_{\tau_{1} + \tau_{2} + \tau_{3} = \tau}
    \frac{\left (1 + |n| \right )^s}{\left (1 +
		|n_1| \right )^s \left (1 + | n_2| \right )^s \left (1 + |n_3| \right )^s} 
		\\
    & \times \frac{c_f(n_1, \tau_1)}{\left( 1 + | \tau_1 - n_1^{m } | 
		\right)^{b}}
		\times
		\frac{c_g(n_2, \tau_2)}{\left( 1 + | \tau_2 - n_2^{m } | 
		\right)^{b}} \times
		\frac{c_h(n_3, \tau_3)}{\left( 1 + | \tau_3 - n_3^{m } | 
		\right)^{b}}\ d \tau_1 d \tau_2 d \tau_3.
	\end{split}
\end{equation*}
%
%
For $s \ge 0$, observe that the quantity 
%
%
\begin{equation}
	\label{nunbounded-quan}
	\begin{split}
		\frac{| n | \left (1 + |n| \right ) ^s}{\left (1 + |n_1| \right ) ^s \left (1 + |n_2| \right ) ^s \left (1 + |n_3| \right ) ^s} 
	\end{split}
\end{equation}
is unbounded (take $n_1 = n_2 = 0$ and $n_3$ arbitrarily large). Hence
we hope that the  principal symbol $\tau - n^m$ offers enough
decay to give control of \eqref{nunbounded-quan}. Following the Bourgain
approach we consider the quantity
%
%
\begin{equation*}
	\begin{split}
		| \tau - n^{m} - \left( \tau_{1} - n_{1}^m + \tau_{2} - n_{2}^m +
		\tau_{3} - n_{3}^m \right) | = |n_{1}^m + n_2^m + n_3^m - n^m|
	\end{split}
\end{equation*}
%
%
and seek a lower bound that is a function of $n$. No such bound exists (this
becomes evident if one sets $n_1 = n_2$). In the case of the KDV, to prove well-posedness we need to bound
\begin{equation}
	\label{nKDV-bound-term}
	\begin{split}
		\frac{| n | \left (1 + |n| \right ) ^s}{\left (1 + |n_1| \right ) ^s \left (1 + |n_2| \right ) ^s} 
	\end{split}
\end{equation}
where $n_1 + n_2 = n$. 
Consider 
%
%
\begin{equation}
  \label{nah}
	\begin{split}
		| \tau - n^{3} - \left( \tau_{1} - n_{1}^3 + \tau_{2} - n_{2}^3 \right) | = |n_{1}^3 + n_2^3 - n^3|
	\end{split}
\end{equation}
where  $\tau_1 + \tau_2 = \tau$. Unlike the mNLS with derivative nonlinearity
considered above, in the case of the KDV we can use the conservation of mass to
\emph{exclude} the pathological cases $n_1=0$ and $n_2=0$, allowing us to obtain
a lower bound of $|n|^{2}$ for \eqref{nah}.
By the pigeonhole principle, we then have three
cases, each with the lower bound
%
%
\begin{equation*}
	\begin{split}
		\frac{1}{| \tau_{i} - n_{i}^{m} |^{b}} \gtrsim \frac{1}{|n|^{b}}	
	\end{split}
\end{equation*}
%
%
where $\tau_0 =\tau, n_0 = n$. 
Hence, we must set $b \ge 1$ to offset the $|n|$ in the numerator of 
\eqref{nKDV-bound-term}.
Lastly, in the case of the mNLS, we are able to bound 
\begin{equation*}
	\begin{split}
    \frac{\left (1 + |n| \right ) ^s}{\left (1 + |n_1| \right ) ^s \left (1 +
    |n_2| \right ) ^s \left (1 + |n_3| \right ) ^s}, \quad s \ge 0 
	\end{split}
\end{equation*}
without relying on any potential smoothing from the principal symbol.
This is due to the
absence of a $|n|$ term in the
numerator. A consequence is that we have more freedom in how
we choose $b$. In fact, we can choose $b$ all the way down to $3/8$, but no
lower, since we must have $b \ge 3/8$ in order to be able to apply
\cref{ncor:four-mult-est-L4}. 
\end{proof}
%
%
\begin{proof}[Conservation of the $L_x^2$ norm.] 
We have
%
%
\begin{equation*}
	\begin{split}
		\frac{d}{dt} \int_\ci | u |^2  dx
		& = \int_\ci \frac{d}{dt} | u |^2  dx
		\\
		& = \int_\ci \frac{d}{dt} \left( u \overline{u} \right)  dx
		\\
		& = \int_\ci \left( u \p_t \overline{u} + \overline{u} \p_t u \right) dx
		\\
		& = \int_\ci \left( u \overline{\p_t u} + \overline{u} \p_t u \right)dx.
	\end{split}
\end{equation*}
%
%
Substituting in $\p_t u = i\left( \p_x^{m} u + | u |^2 u \right)$ we obtain
%
%
\begin{equation*}
	\begin{split}
		& \int_{\ci} \left\{ u\left[ -i\left( \p_x^{m} \overline{u} + | u |^2
		\overline{u} \right) \right] + \overline{u}\left[ i\left( \p_x^{m} u + | u
		|^2 u \right) \right] \right\}dx
		\\
		& = \int_\ci \left[ -iu \p_x^{m} \overline{u} - i| u |^4 + i \overline{u}
		\p_x^{m} u + i | u |^4 \right]dx
		\\
		& = i \int_{\ci}\left( \overline{u} \p_x^{m} u - u \p_x^{m } \overline{u}
		\right)dx.
	\end{split}
\end{equation*}
%
%
Integrating by parts $m/2$ times and using
the spatial periodicity of $u$, the right
hand side simplifies to
%
%
\begin{equation*}
	\begin{split}
    i (-1)^{m/2}\int_\ci \left( \p_x^{m/2} \overline{u} \p_x^{m/2} u - \p_x^{m/2} u
		\p_x^{m/2 } 
		\overline{u} \right) dx = 0.
	\end{split}
\end{equation*}
%
%
Therefore, the $L_x^2(\ci)$ norm of solutions to the mNLS is conserved. 
\end{proof}
%
%
\begin{proof}[Why Assuming Mean Initial Data is Problematic]
Recall the NLS ivp
%
%
\begin{equation*}
	\begin{split}
		&i \p_t u = \p_x^2 u - | u |^2 u,
		\\
		& u(x,0) = \vp(x).
	\end{split}
\end{equation*}
%
%
This is equivalent to the ivp
%
%
\begin{gather*}
		 i \p_t [u - \wh{\vp}(0)]
		  = -\p_x^2 [u - \wh{\vp}(0)] - [u -
		\wh{\vp}(0)][u - \wh{\vp}(0)][\bar{u} - \bar{\wh{\vp}}(0)]
		\\
		- 2| u |^2
		\wh{\vp}(0) + \bar{u}\left[ \wh{\vp}(0) \right]^2 - u^{2}
		\bar{\wh{\vp}}(0) + 2 u | \wh{\vp}(0) |^2 - | \wh{\vp}(0) |^2
		\wh{\vp}(0),
		\\
		u(x,0) = \vp(x) - \wh{\vp}(0)
\end{gather*}
or
%
%
\begin{gather*}
		 i \p_t u 
		  = -\p_x^2 u - [u -
		\wh{\vp}(0)][u - \wh{\vp}(0)][\bar{u} - \bar{\wh{\vp}}(0)]
		\\
		- 2| u |^2
		\wh{\vp}(0) + \bar{u}\left[ \wh{\vp}(0) \right]^2 - u^{2}
		\bar{\wh{\vp}}(0) + 2 u | \wh{\vp}(0) |^2 - \boxed{| \wh{\vp}(0) |^2
		\wh{\vp}(0)},
		\\
		u(x,0) = \vp(x) - \wh{\vp}(0).
\end{gather*}
%
The boxed term is problematic. 
\end{proof}
%
\section{Classical Well-Posedness for the mNLS}
%
%
%%%%%%%%%%%%%%%%%%%%%%%%%%%%%%%%%%%%%%%%%%%%%%%%%%%%%
%
%
%			Alternate WP Theorem	
%
%
%%%%%%%%%%%%%%%%%%%%%%%%%%%%%%%%%%%%%%%%%%%%%%%%%%%%%
%
%
%
%
We have introduced the spaces $Y_s$ in part because well-posedness
in $H^s(\ci)$ for the mNLS becomes problematic as $s$ becomes small. 
On the other hand, for $s > 1/2$, well-posedness in $H^s(\ci)$ is a direct 
consequence of the algebra property of Sobolev spaces and the fact that the operator 
$e^{it \p_x^2}$ isometrically preserves Sobolev spaces. Stated more 
precisely, we have the following result:
%
%
\begin{proposition}
  Let $B_R \doteq \{f \in H^s : \|f\|_{H^s} < R \}$.
  Then the generalized mNLS ivp
\begin{gather}
  \label{general-mNLS-eq}
    i \p_t v = - \p_x^2 u - \lambda |u|^{\alpha -1} u, \ \ \alpha > 
    1, \lambda > 1
    \\
    \label{general-mNLS-init-data}
    u(x,0) = \vp(x), \ \ t \in \rr, \ \ x \in \ci \ \text{or} \ \rr
\end{gather}
  is locally well-posed in $H^s$ for $s > 1/2$ for 
  sufficiently small initial data $\vp \in B_R$, where the lifespan $T$ 
  satisfies 
%
%
\begin{equation*}
  \begin{split}
    T < 1/c
  \end{split}
\end{equation*}
%
%
for some constant $c = c(s, \lambda, \alpha, R, \vp)$.
\end{proposition}
%
%
\begin{proof} We will only provide a proof on the circle; the case on 
the line is nearly identical. The key ingredient
will be to establish that $L$ is a 
contraction on $C([-T, T], B_R)$. For the sake of clarity, we let $H^s_x 
= H^s_x(\ci)$. Let $e^{it \p_x^2}: \mathcal{E}'(\ci) \to 
\mathcal{E}'(\ci)$ be an operator defined by  
%
%
\begin{equation}
  \label{unit-op}
  \begin{split}
    e^{it \p_x^2} f(x) = \left[ e^{(-1)^j i t n^2} \wh{f}(n)
    \right]^{\vee} = 
    \sum_{n \in \zz} e^{i(nx + (-1)^j it n^2)} \wh{f}(n).
  \end{split}
\end{equation}
%
%
First, note that $e^{it \p_x^2}$ 
is unitary on $H^s(\ci)$; that is
%
%
\begin{equation}
  \label{unitary-op}
  \begin{split}
    \|e^{it \p_x^2} f \|_{H^s_x} & = \sum_{n \in \zz} |e^{(-1)^j it n^2} 
    \wh{f}(n)|^2 (1 + n^2)^s  
    \\
    & = \sum_{n \in \zz} |\wh{f}(n)|^2 (1 + n^2)^s 
    \\
    & = \|f\|_{H^s_x}.
  \end{split}
\end{equation}

Rewriting \eqref{general-mNLS-eq}-\eqref{general-mNLS-init-data} in its 
integral form
%
%
\begin{equation}
  \label{mNLS-int-form-with-op}
  \begin{split}
    u(x,t) = e^{it \p_x^2} \vp + i \lambda \int_0^t e^{i(t - 
    t')\p_x^2} |u|^{\alpha -1} u(x, t') \ dt' 
  \end{split}
\end{equation}
%
%
and applying the triangle inequality, Minkowski's inequality, and 
\eqref{unitary-op}, we obtain
%
%
\begin{equation}
  \label{bound-for-L}
  \begin{split}
    & \|Lu\|_{L^\infty_t[-T, T] H^s_x}
    \\
    & \le \|e^{t \p_x^2}
    \vp\|_{L^\infty_t[-T, T] H^s_x} + \|i \lambda \int_0^t e^{i(t - 
    t')\p_x^2} |u|^{\alpha -1} u(x, t') \ dt' \|_{L^\infty_t[-T, T] 
    H^s_x} 
    \\
    & \le \|\vp\|_{H^s_x} + |\lambda| \int_0^T \|e^{i(t 
    -t')\p_x^2} |u|^{\alpha -1} u \|_{L^\infty_t[-T, T] H^s_x} \ 
    dt'
    \\
    & = \|\vp\|_{H^s} + T |\lambda| \|u^\alpha \|_{L^\infty_t[-T, T] H^s_x}.
  \end{split}
\end{equation}
%
%
We now need the following lemma, whose proof can be found in Taylor 
\cite{Taylor_1991_Pseudodifferent}:
%
%
%
\begin{lemma}
  \label{lem:algebra-prop}
  The Sobolev space $H^s$ is an algebra for $s>1/2$. More precisely, 
%
%
\begin{equation}
  \label{algebra-prop}
  \begin{split}
    \|fg\|_{H^s} \le c_s \|f\|_{H^s} \|g\|_{H^s}.
  \end{split}
\end{equation}
%
%
%
\end{lemma}
%
%
Applying \cref{lem:algebra-prop} to estimate \eqref{bound-for-L} gives
%
%
%
%
\begin{equation}
  \label{Tu-space-bound}
  \begin{split}
    \|Lu\|_{L^\infty_t[-T, T] H^s_x}
    & \le \|\vp\|_{H^s_x} + Tc_s | \lambda| \|u\|_{L^\infty_t[-T, T] 
    H^s_x}^\alpha
  \end{split}
\end{equation}
%
%
and since $u \in C([-T, T], B_R)$ a priori, it follows
that for sufficiently small $\vp$ and $T = T(s, \lambda, \alpha, R, \vp)$ we must 
have $Lu \in L^\infty([-T, T], B_R)$. To improve the regularity of 
$Lu$, let $\{t_n\} \subset [-T, T]$ and suppose that $t_n \to t \in [-T, 
T]$. Then
%
%
\begin{equation}
  \label{befo-dom}
  \begin{split}
    & \lim_{n \to \infty} \|Lu(\cdot, t) - Lu(\cdot, t_n)\|_{H^s_x} 
    \\
    & = \lim_{n \to \infty} \| \left \{ i \lambda \int_0^{t - t_n} e^{i(t  
    - t') \p_x^2} \left [|u|^{\alpha -1}u( \cdot, t') \right ]
    \ dt'\right \} \|_{H^s_x}
    \\
    & \le |\lambda|
    \lim_{n \to \infty}  \left \{  \int_0^{t - t_n} \| e^{i(t  
    - t') \p_x^2} \left [ |u|^{\alpha -1}u( \cdot, t') \right ]  
    \|_{H^s_x} \ dt' \right \}
    \\
    & = |\lambda|
    \lim_{n \to \infty}  \left[  \int_\rr \chi_{[0, t-t_n]}
    \| u^{\alpha}(\cdot, t') \|_{H^s_x} \ dt' \right ]
  \end{split}
\end{equation}
%
%
where the last step follows from \eqref{unitary-op}. 
By the algebra property and our a priori
assumption $u \in C([-T, T], B_R )$, we have 
%
%
\begin{equation*}
  \begin{split}
    \chi_{[0,t-t_n]}	\|u^\alpha(\cdot, t')\|_{H^s_x}
    \lesssim \chi_{[0,t-t_n]} 	\|u(\cdot, t')\|_{H^s_x}^\alpha 
    \le \chi_{[0,T]}\|u\|^\alpha_{L^\infty_t(\rr) H^s_x} \in 
    L^1_t(\rr).
    \end{split}
\end{equation*}
%
%
%
%
Hence, applying dominated 
convergence to \eqref{befo-dom}, we may pass the limit inside the integral,  
giving
%
\begin{equation*}
  \begin{split}
    \lim_{n \to \infty} \|Lu(\cdot, t)  - Lu(\cdot, t_n)\|_{H^s_x} 
    \le |\lambda|
    \int_\rr \lim_{n \to \infty} \left [ \chi_{[0, t-t_n]}
    \| u^{\alpha}(\cdot, t') \|_{H^s_x} \ dt' \right ]
    = 0
  \end{split}
\end{equation*}
%
%
which implies $Lu \in C([-T, T], B_R)$. Furthermore, for 
$u, v \in C([-T, T], B_R)$, we have
%
%
\begin{equation*}
  \begin{split}
    & \|Lu-Lv\|_{L^\infty_t[-T, T] H^s_x}
    \\
    & = \|i \lambda \int_0^t e^{i(t 
    -t')\p_x^2} (|u|^{\alpha - 1}u -|v|^{\alpha -1} v ) \ dt'
    \|_{L^\infty_t[-T, T] H^s_x}
    \\
    & \le |\lambda| \int_0^T \||u^{\alpha-1 }| u - | v^{\alpha - 1}| v
    \|_{L^\infty_t[-T, T] H^s_x} \ dt'
    \\
    & = |\lambda| T \cdot \|(u-v)(|u^{\alpha -1}| + |v^{\alpha -1}|) 
    + |u^{\alpha -1}|v 
    + u |v^{\alpha -1}| \|_{L^\infty_t[-T, T] H^s_x}
  \end{split}
\end{equation*}
%
%
which by the triangle inequality and algebra property simplifies to
%
%
\begin{equation}
  \label{L-contract}
  \begin{split}
    & \|Lu-Lv\|_{L^\infty_t[-T, T] H^s_x}
    \\
    & \le  T c_s |\lambda| \cdot \big [ \|u-v\|_{L^\infty_t[-T, T]
    H^s_x}(\|u\|^{\alpha -1}_{L^\infty_t[-T, T] H^s_x} +
    \|v\|^{\alpha -1}_{L^\infty_t[-T, T] H^s_x})
    \\
    & + \|u\|^{\alpha-1}_
    {L^\infty_t[-T, T] H^s_x} \|v\|_{L^\infty_t[-T, T] H^s_x}
    + \|u\|_
    {L^\infty_t[-T, T] H^s_x} \|v\|^{\alpha -1}_
    {L^\infty_t[-T, T] H^s_x} \big ]
    \\
    & \le T c_s |\lambda| \cdot \left[  2R^{\alpha -1} 
    \|u -v\|_{L^\infty_t[-T, T] H^s_x} + 2R^{\alpha} \right]
    \\
    & \le T c' \|u -v \|_{L^\infty_t[-T, T] 
    H^s_x}
  \end{split}
\end{equation}
%
%
where $c' = c'(s, \lambda, \alpha, R)$ is a constant.
Since it was established earlier that $T = T(s, \lambda, \alpha, R, \vp)$,  
we conclude that for $$T < 1/c,  \qquad c = c(s, \lambda, \alpha, R, 
\vp),$$ $L$ is a 
contraction on $C([-T, T], B_R)$. 
\end{proof}




\section{Well-Posedness for HR in the Periodic Case}
%
%
%
%
We will now prove well-posedness for the periodic case, after which we will
provide the necessary details to extend the argument to the non-periodic case.
\subsection{Existence.}
\label{existence}
Here we will prove the existence of a solution to the HR i.v.p. and inequalities
\eqref{Life-span-est} and \eqref{u_x-Linfty-Hs}.  We begin by mollifying the HR equation, so that we may apply the following ODE
theorem: 
%
\begin{theorem}
	\label{ode_theorem}
	Let  $Y$  be a Banach space, $X\subset Y$ be an open subset,
	$I' \subset \rr$, and $f: I' \times X\to Y$ a continuously differentiable
	map.  Then for any $t_{0} \in I'$ and $x_{0} \in X$ there exists an
	open ball $I \subset I'$ and a unique differentiable mapping $u:I
	\to Y$ such that for all $t \in I$,  $u'(t) = f(t, u)$
	and $u(t_{0}) = x_{0}.$
\end{theorem}
%
To see why we cannot apply the Banach Space ODE Theorem to the HR equation as is,
we use a counterexample. Let $u=x^{-1/2} \chi_{[0,1]}$ and $s=0$. Then $u \in H^s$ but
$u\p_x u \notin H^s$. Hence, returning to the general case, we see that the
HR equation as is can not be thought as an ODE on the space $H^s$.  To
deal with this problem we will replace the i.v.p \eqref{hr}--\eqref{hr-data} by  
\begin{equation}
	\label{hr-moli}
	\p_t  u_\ee =
	-\gamma J_\ee u_\ee \partial_x  J_\ee  u_\ee - \p_x (1-\p_x^2)^{-1} 
	\left [\frac{3-\gamma}{2}u^2 + \frac{\gamma}{2}(\p_x u)^2 \right ],
\end{equation} 
%
\begin{equation} 
	\label{hr-moli-data} 
	u_\ee(x, 0) = u_0 (x),
\end{equation}
%
where $J_\ee$ is defined as follows: Pick a non-negative $j(x) \in
\mathcal{S}(\rr)$ and let
\begin{equation*}
	\begin{split}
		j_\ee(x) = \frac{1}{\ee}j\left( \frac{x}{\ee} \right).
	\end{split}
\end{equation*}
	We then define $J_\ee$ to be the ``Friedrichs mollifier''
	\begin{equation}
		\begin{split}
			J_\ee f(x) = j_\ee * f(x), \quad \ee>0.
		\end{split}
	\end{equation}
%
%
Notice that the right hand side of \eqref{hr-moli} is a map from $H^s(\ci)$
to $H^s(\ci)$.  In order to apply the ODE Theorem, we will also need to
show that it is a continuously differentiable map:
%
%
%
\begin{lemma}
	Let $f_\ee:H^s(\ci) \to H^s(\ci)$ be given by 
	\begin{equation}
		\label{f_ep}
		f_{\ee}(u) = -\gamma  J_\varepsilon u \partial_x J_\varepsilon u
		- \p_x (1-\p_x^2)^{-1} \left
		[\frac{3-\gamma}{2}u^2 + \frac{\gamma}{2}(\p_x u)^2 \right ].
	\end{equation}
	Then $f_\ee$  is a continuously differentiable map.
\end{lemma}
%
%
\subsection{ Proof.} We explicitly calculate the derivative of $f_\ee$ at an
arbitrary $w \in H^s(\ci)$:
\begin{equation*}
	\begin{split}
		[Df_{\ee}(u)](w)
		=
		& -\gamma (J_\varepsilon w \cdot \partial_x J_\varepsilon u +
		J_\varepsilon u \cdot \partial_x J_\varepsilon w)
		\\
		& - (1-\p_x ^2)^{-1}
		\p_x \left [(3-\gamma)w u + \gamma\p_x w \p_x u \right ].
	\end{split}
\end{equation*}
Let $w_n \xrightarrow{H^s(\ci)} w$. Then it is easy to check that
%
\begin{equation}
	\begin{split}
		& -\gamma (J_\varepsilon w_n \cdot \partial_x J_\varepsilon u 
		+ J_\varepsilon u \cdot \partial_x J_\varepsilon w_n)
		+ (1-\p_x ^2)^{-1}
		\p_x \left [(3-\gamma)w_n u + \gamma\p_x w_n \p_x u \right ]
		\\
		& \xrightarrow{H^s(\ci)} 
		 -\gamma (J_\varepsilon w \cdot \partial_x J_\varepsilon u 
		+ J_\varepsilon u \cdot \partial_x J_\varepsilon w) + (1-\p_x ^2)^{-1}
		\p_x \left [(3-\gamma)w u + \gamma\p_x w \p_x u \right ].
	\end{split}
\end{equation}
This concludes the proof. $\quad \square$
Hence, by Theorem \ref{ode_theorem}, for each $\ee > 0$ there exists a
unique solution $u_\ee \in C(I, H^s(\ci))$ satisfying the Cauchy-problem
\eqref{hr-moli}-\eqref{hr-moli-data}. Next, we analyze the size and
lifespan of the family $\{u_\ee\}$ of solutions.
%%%%%%%%%%%%%%%%%%%%%%%%
%
%     Estimates  for Life-span and Sobolev norm of $u_\ee$
%
%%%%%%%%%%%%%%%%%%%%%%%%
%
%
\subsection{ Estimates  for Life-span and Sobolev norm of $u_\ee$.}
%
We will show that there is a lower bound  $T$
for $T_\ee$, which is  independent of $\ee\in(0, 1]$.
This is based on the following differential
inequality for the solution $u_\ee$:
%
\begin{equation} 
	\label{B-diff-ineq}
	\frac 12
	\frac{d}{dt}
	\|u_\ee(t)\|_{H^{s}(\ci)}^2
	\le
	c_s
	\|u_\ee(t)\|_{H^{s}(\ci)}^3,
	\quad
	|t| \le T_\ee.
\end{equation}
%
%
We will prove this inequality  by
following the approach used for quasilinear symmetric
hyperbolic systems in Taylor \cite{Taylor_1991_Pseudodifferent}. In what follows we will suppress the
$t$ parameter for the sake of clarity.
%
For any $s\in \ci$ let   $D^s=(1-\p_x^2)^{s/2}$ be the  operator
defined by 
%
$$ \widehat{D^s f}(\xi) \doteq (1 + \xi^2)^{s/2} \widehat{f}(\xi), $$
%
where $ \widehat{f}$ is the Fourier transform
%
$$ \widehat{f}(\xi) =  \frac{1}{2\pi}\int_{\ci} e^{-i \xi x} f(x) \ dx.  $$
%
Applying the operator $D^s$ to  both sides of  \eqref{hr-moli},
then  multiplying the resulting equation by $D^s J_\ee u_\ee$
and integrating it for $x\in\ci$ gives
%
\begin{equation} 
	\begin{split}
		\label{B-moli-int}
		\frac 12
		\frac{d}{dt} \|u_\ee \|_{H^s}^2
		=
		&-
		\gamma \int_{\ci}  D^s(J_\ee u_\ee \partial_x J_\ee u_\ee) \cdot
		D^s J_\ee u_\ee  \  dx
		\\
		&- \frac{3 -\gamma}{2} \int_{\ci} D^{s-2} \p_x (u_{\ee}^2) 
		\cdot D^s J_\ee u_{\ee} \ dx
		\\
		&- \frac{\gamma}{2} \int_{\ci}  D^{s-2} \p_x (\p_x u_\ee)^2
		\cdot D^s J_\ee u_\ee  \ dx.
	\end{split}
\end{equation}
%
We will estimate the right hand side of \eqref{B-moli-int} in parts. In
what follows next we use the fact that  $D^s$ and $J_\ee$ commute and
that  $J_\ee$ satisfies the properties 
%
\begin{equation} 
	\label{J-e-inner-prod-property}
	(J_\ee f, g)_{L^2(\ci)}=( f, J_\ee g)_{L^2(\ci)}
\end{equation}
%
and
%
\begin{equation} 
	\label{Je-u-Hs}
	\| J_\ee u \|_{H^s(\ci)}
	\le
	\|  u \|_{H^s(\ci)}.
\end{equation}
%
%%%%%%%%%%%% Burgers term energy estimate %%%%%%%%%%%%
%
%
%
\noindent
Letting 
%
\begin{equation} 
	\label{v-Je-ue}
	v=J_\ee u_\ee
\end{equation}
%
%
we have
%
\begin{equation} 
	\begin{split}
		\label{B-moli-int-v}
		& -  \gamma \int_{\ci}   D^s (J_{\ee} u_{\ee} \p_x J_\ee u_\ee)
		 \cdot D^s
		J_{\ee}u_\ee \ dx  
		\\
		& = - \gamma \int_\ci
		 D^s(v \partial_x v) \cdot   D^s v \ dx
		\\
		& = - \gamma \int_\ci
		\left [ 
		D^s(v\p_x v)  -  v D^s (\p_xv)
		\right ] 
		D^s v \ dx - \gamma \int_\ci
		v D^s (\p_xv)
		D^s v \ dx.
	\end{split}
\end{equation}
%
%
%
We now estimate \eqref{B-moli-int-v} in parts. Applying the Cauchy-Schwarz inequality gives
%
\begin{equation} 
	\label{int1-est-calc2}
	\begin{split}
		& \Big|
		- \gamma \int_\ci
		\big[ 
		D^s(v\p_x v)  -  v D^s (\p_xv)
		\big]
		D^s v   \, dx
		\Big|
		\\
		& \le
		|\gamma| \cdot \|
		D^s(v\p_x v)  -  v D^s (\p_xv)
		\|_{L^2(\ci)}
		\|
		D^s v 
		\|_{L^2(\ci)}
		\\
		&\le
		|\gamma| \cdot \|
		D^s(v\p_x v)  -  v D^s (\p_xv)
		\|_{L^2(\ci)}
		\|
		v
		\|_{H^s(\ci)}
		\\
		&\le c_s \| \p_x v \|_{L^\infty(\ci)} 
		\| v \|_{H^s(\ci)}^2,
	\end{split}
\end{equation}
%
where the last step follows from 
%
\begin{equation} 
	\label{int1-est-calc3}
	\| D^s(v\p_x v)  -  v D^s (\p_xv) \|_{L^2(\ci)}
	\le
	2 c_s^{\prime}    \| \p_x v \|_{L^\infty(\ci)} 
	\| v \|_{H^s(\ci)},
\end{equation}
which we prove below by using the following Kato-Ponce commutator 
estimate:  
\begin{lemma} 
	\label{KP-lemma}
	[Kato-Ponce]
	If  $s>0$ then there is $c_s^{\prime}>0$ such that 
	%
	\begin{equation} 
		\label{KP-com-est}
		\| D^{s} \big(fg) -  f D^s g\|_{L^2(\ci)}
		\le
		c_s^{\prime}\big(
		\| D^{s}f \|_{L^2(\ci)}    \| g \|_{L^\infty(\ci)} 
		+
		\| \p_xf \|_{L^\infty(\ci)}    \| D^{s-1}g \|_{L^2(\ci)}   
		\big).
	\end{equation}
	%
	\end{lemma}
	%
	%
	In fact, applying  this estimate with $f=v$ and $g=\p_xv$ gives 
	%
	\begin{equation} 
		\label{int1-est-calc4}
		\begin{split}
			& \| D^s(v\p_x v)  -  v D^s (\p_xv) \|_{L^2(\ci)}
			\\
			& \le
			{c_s}^\prime \big(
			\| D^{s}v \|_{L^2(\ci)}    \| \p_x v \|_{L^\infty(\ci)} 
			+
			\| \p_xv \|_{L^\infty(\ci)}    \| D^{s-1}\p_x v \|_{L^2(\ci)}   
			\big)
			\\
			& \le
			2{c_s}^\prime    \| \p_x v \|_{L^\infty(\ci)} 
			\| v \|_{H^s(\ci)}, 
		\end{split}
	\end{equation}
	%
	which  is the desired estimate  \eqref{int1-est-calc3}.
	Next, we have
	%
	%
	%
	\begin{equation} 
		\label{int1-est-calc5}
		\begin{split}
			\Big|
			-\gamma \int_\ci
			v D^s (\p_x v)
			\cdot  D^s v \ dx
			\Big|
			& =
			\left | \frac{\gamma}{2} \right | \cdot \Big|
			\int_\ci
			v \p_x\left(D^s v\right)^2  dx
			\Big|
			\\
			& =
			\left | \frac{\gamma}{2} \right | \cdot \Big | \int_\ci
			\p_x v \, (D^s v)^2 \ dx
			\Big|
			\\
			& \le
			\left | \frac{\gamma}{2} \right |  \cdot \int_\ci
			\Big | \p_x v \, (D^s v)^2   
			\Big| \ dx
			\\
			& \lesssim
			\| \p_x v \|_{L^\infty(\ci)} 
			\| v \|_{H^s(\ci)}^2.
		\end{split}
	\end{equation}
	%
	%
	%
	Combining inequalities  \eqref{int1-est-calc2} and
	\eqref{int1-est-calc5} and applying the Sobolev Imbedding Theorem, we
	have
	%
	\begin{equation} 
		\label{burgers_est'}
		\begin{split}
			\Big|
			-\gamma \int_\ci
			D^s(v \partial_x v) \cdot   D^s v \, dx  
			\Big|
			&\le
			{c_s}^\prime
			\| \p_x v \|_{L^\infty(\ci)} 
			\|  v \|_{H^s(\ci)}^2
			\\
			& \le {c_s}^\prime \| v \|_{C^1(\ci)} \| v \|_{H^s(\ci)}^2
			\\
			& \le {c_s}^{\prime \prime} \| v \|_{H^s(\ci)}^3
			\\
			& \le {c_s}^{\prime \prime} \| u_\ee \|_{H^s(\ci)}^3.
		\end{split}
	\end{equation}
	%
	Next we estimate
	\begin{equation}
		\begin{split}
			\left | - \frac{3 -\gamma}{2} \int_\ci D^{s-2} \p_x u_\ee^2 \cdot
			D^s J_\ee u_\ee \; dx \right |
			& \le \left | \frac{3- \gamma}{2} \right | \int_\ci \left |
			D^{s-2} \p_x u_\ee^2 \cdot D^s J_\ee u_\ee \; dx \right | 
			\\
			& \le \left | \frac{3- \gamma}{2} \right |
			\|D^{s-2} \p_x u_\ee^2 \|_{L^2(\ci)} 
			\|D^s J_\ee u_\ee \|_{L^2(\ci)}
			\\
			& \le \left | \frac{3- \gamma}{2} \right |
			\|D^{s-1} u_\ee^2 \|_{L^2(\ci)} 
			\|D^s u_\ee \|_{L^2(\ci)}
			\\
			& \lesssim \| u_\ee^2 \|_{H^s(\ci)} \| u_\ee \|_{H^s(\ci)}.
		\end{split}
	\end{equation}
	%
	%
	Applying the algebra property, we obtain
	%
	\begin{equation}
		\label{hl1}
		\begin{split}
			\left | - \frac{3 -\gamma}{2} \int_\ci D^{s-2} \p_x u_\ee^2 \cdot
			D^s J_\ee u_\ee \; dx \right |
			\lesssim \| u_\ee \|_{H^s(\ci)}^3.
		\end{split}
	\end{equation}
	%
	%
	We also have
	\begin{equation}
		\begin{split}
			\left |- \frac{\gamma}{2} \int_\ci D^{s-2} \p_x (\p_x u_\ee)^2 \cdot
			D^s J_\ee u_\ee \; dx \right |
			& \le \left | \frac{\gamma}{2} \right | \int_\ci \left | D^{s-2} \p_x (\p_x u_\ee)^2 \right |
			\cdot \left |D^s J_\ee u_\ee \right | \; dx
			\\
			& \le \left | \frac{\gamma}{2} \right |
			\| D^{s-1} (\p_x u_\ee)^2 \|_{L^2(\ci)}
			\| D^s J_\ee u_\ee \|_{L^2(\ci)}
			\\
			& \lesssim \|(\p_x u_\ee)^2 \|_{H^{s-1}(\ci)}
			\| J_\ee u_\ee \|_{H^{s-1}(\ci)} 
			\\
			& \lesssim \|(\p_x u_\ee)^2 \|_{H^{s-1}(\ci)} \| u_\ee \|_{H^{s-1}(\ci)} 
		\end{split}
	\end{equation}
	and applying the algebra property yields
	\begin{equation}
		\label{hl2}
		\begin{split}
		\left | - \frac{\gamma}{2} \int_\ci D^{s-2} (\p_x u_\ee)^2 \cdot
		D^s J_\ee u_\ee \; dx \right |
		& \lesssim \| \p_x u_\ee \|_{H^{s-1}(\ci)}^2 \| u_\ee \|_{H^s(\ci)} 
		\\
		& \lesssim \|u_\ee\|_{H^s(\ci)}^3.
	\end{split}
	\end{equation}
	%
	Combining \eqref{burgers_est'}, \eqref{hl1}, and \eqref{hl2}, we obtain
	\eqref{B-diff-ineq}.
	%%%%%%%%%%%%%%%%%%%%%%%%%%%%%%%%%%%
	%  
	%           Lifespan for CH  solution    
	% 
	%%%%%%%%%%%%%%%%%%%%%%%%%%%%%%%%%%%
	%
	%
	%   
	%
	\noindent
	\subsection{  Lifespan estimate of $u_\ee$.} To derive an explicit formula for
	$T_\ee$ we proceed as follows.  Letting  $y(t)=
	\|u_\ee(t)\|_{H^s(\ci)}^2$ inequality  \eqref{B-diff-ineq} takes the
	form
	%
	\begin{equation} 
		\label{energy-y-ineq}
		\frac 12
		y^{-3/2}\frac{dy}{dt}
		\le
		c_s,
		\qquad
		y(0)=y_0=  \|u_0\|_{H^s(\ci)}^2.
	\end{equation}
	%
	Suppose $t$ is non-negative. Integrating  \eqref{energy-y-ineq} from  0  to $t$ gives
	%
	\begin{equation} 
		\label{energy-y-ineq-calc1}
		\frac{1}{\sqrt{y_0}}  - \frac{1}{\sqrt{y(t)}} 
		\le
		c_s t.
	\end{equation}
	%
	%
	Replacing $y(t)$ with   $\|u_\ee(t)\|_{H^s(\ci)}^2$  and solving for  $\|u_\ee(t)\|_{H^s(\ci)}$
	we obtain the formula
	%
	\begin{equation} 
		\label{norm-u(t)-formula}
		\|u_\ee(t)\|_{H^s(\ci)}
		\le
		\frac{ \|u_0\|_{H^s(\ci)}}{1-c_s\|u_0\|_{H^s(\ci)} t}, \quad t\ge
		0.
	\end{equation}
	%
	Now, from \eqref{norm-u(t)-formula} we see that  $\|u_\ee(t)\|_{H^s(\ci)}$ is finite  if 
	%
	\begin{equation*} 
		\label{Lifespan-calc1}
		c_s    \|u_0\|_{H^s(\ci)} t<1,
	\end{equation*}
	%
	or
	%
	\begin{equation} 
		t
		<
		\frac{1}{ c_s \|u_0\|_{H^s(\ci)}}.
	\end{equation}
	%
	Similarly, if $t$ is negative, then 
	\begin{equation} 
		\label{norm-u(t)-formula-prime}
		\|u_\ee(t)\|_{H^s(\ci)}
		\le
		\frac{ \|u_0\|_{H^s(\ci)}}{1+c_s\|u_0\|_{H^s(\ci)} t}, \quad t < 0.
	\end{equation}
	from which it follows that $\|u_\ee(t)\|_{H^s(\ci)}$ is finite  if 
	%
	\begin{equation} 
		t
		>
		 \frac{-1}{ c_s \|u_0\|_{H^s(\ci)}}.
	\end{equation}
	Therefore, the  solution  $u_\ee(t)$ to the mollified CH Cauchy
	problem exists for $|t| <T_0$, where
	%
	\begin{equation} 
		\label{CH-Lifespan}
		T_0
		=
		\frac{1}{ c_s \|u_0\|_{H^s(\ci)}}.
	\end{equation}
	%
	%%%%%%%%%%%%%%%%%%%%%%%%%%%%%%%%%%%
	%  
	%            Norm of   
	% 
	%%%%%%%%%%%%%%%%%%%%%%%%%%%%%%%%%%%
	%
	%
	%   
	%
	\noindent
	\subsection{  Size of the solution estimate.} If we choose  $T=\frac12 T_0$, that is
	%
	\begin{equation} 
		\label{T-def}
		T
		=
		\frac{1}{2 c_s \|u_0\|_{H^s(\ci)}},
	\end{equation}
	%
	then for $|t| \le T$, estimates \eqref{norm-u(t)-formula} and
	\eqref{norm-u(t)-formula-prime} imply 
	%
	\begin{equation*} 
		\label{u(t)-u(0)-bound}
		\|u_\ee(t)\|_{H^s(\ci)}
		\le
		\frac{ \|u_0\|_{H^s(\ci)}}{1-(c_s\|u_0\|_{H^s(\ci)})/(2 c_s \|u_0\|_{H^s(\ci)})},
	\end{equation*}
	%
	or 
	%
	\begin{equation} 
		\|u_\ee(t)\|_{H^s(\ci)}
		\le
		  2 \|u_0\|_{H^s(\ci)},
		\quad 
		|t| \le T.
	\end{equation}
	%
	Thus we have obtained a lower bound for $T_\ee$ and an upper bound for
	$\|u_\ee(t)\|_{H^s(\ci)}$ independent of $\ee\in (0, 1]$. The following
	lemma summarizes these results and provides an estimate for the
	$H^{s-1}(\ci)$ norm of $\p_t u_\ee(t)$:
	%
	%
	\begin{lemma}
		\label{hr_wp}
		Let  $u_0(x) \in  H^s(\ci)$, $s >3/2$. Then for any $\ee\in (0, 1]$
		the i.v.p. for the mollified HR equation 
		%
		\begin{equation} 
			\label{hr-moli-2}
			\partial_t  u_\ee 
			=
			-\gamma (J_\ee u_\ee \partial_x  J_\ee  u_\ee) - \p_x (1-\p_x^2)^{-1} \left
			[\frac{3-\gamma}{2}(u_\ee)^2 + \frac{\gamma}{2}(\p_x u_\ee)^2
			\right ], 
		\end{equation} 
		%
		\begin{equation} 
			\label{burgers-moli-data-2} 
			u_\ee(x, 0) = u_0 (x),
		\end{equation}
		%
		has a unique solution $u_\ee( t)\in C([-T, T]; H^s(\ci))$. 
		In particular,
		%
		\begin{equation} 
			\label{life-est}
			T
			=
			\frac{1}{2 c_s \|u_0\|_{H^s(\ci)}},
		\end{equation}
		%
		is independent of $\ee$ and
		is a lower bound for the lifespan of $u_\ee( t)$ and
		%
		\begin{equation}
			\label{u-e-Hs-bound}
			\|u_\ee(t)\|_{H^s(\ci)}
			\le
			2 \|u_0 \|_{H^s(\ci)},
			\quad
			|t| \le T.
		\end{equation}
		%
		Furthermore,  $u_\ee( t)\in C^1([T, T]; H^{s-1}(\ci))$ and 
		satisfies
		\begin{equation}
			\label{dt-u-e-Hs-bound}
			\|\p_t u_\ee(t)\|_{H^{s-1}(\ci)}
			\lesssim
			\|u_0 \|_{H^s(\ci)}^2,
			\quad
			|t| \le T.
		\end{equation}
		% 
		Here  $c_s$ is a constant depending only on $s$.
	\end{lemma}
	%
	%
	\subsection{ Proof.}  It suffices to prove  \eqref{dt-u-e-Hs-bound}.
	Using equation \eqref{hr-moli-2}, for any $t\in [-T, T]$ we have
	%
	\begin{equation*}
		\begin{split}
			& \| \partial_t u_\varepsilon(t) \|_{H^{s-1}(\ci)}  
			\\
			& = 
			\| -\gamma (J_\ee u_\ee \partial_x  J_\ee  u_\ee) -
			\p_x (1-\p_x^2)^{-1} \left [\frac{3-\gamma}{2} (u_\ee)^2 +
			\frac{\gamma}{2}(\p_x u_\ee)^2 \right ] \|_{H^{s-1}(\ci)}
			\\
			& \lesssim  
			\| J_\ee u_\ee \partial_x  J_\ee  u_\ee \|_{H^{s-1}(\ci)}
			+ \|\p_x (1-\p_x^2)^{-1} (u_\ee)^2 \|_{H^{s-1}(\ci)}
			\\
			& + \| \p_x (1-\p_x^2)^{-1}(\p_x u_\ee)^2\|_{H^{s-1}(\ci)}.
			\end{split}
		\end{equation*}
		We break this into three parts:
		\begin{equation}
			\label{bixi}
			\begin{split}
				\| J_\ee u_\ee \p_x J_\ee u_\ee \|_{H^{s-1}(\ci)}
				& = 
				\frac{1}{2}\|\p_x[(J_\varepsilon u_\varepsilon
				)^2]\|_{H^{s-1}(\ci)}
				\\
				& \lesssim \|(J_\varepsilon u_\varepsilon )^2\|_{H^s(\ci)}.
			\end{split}
		\end{equation}
		Applying the algebra property of Sobolev spaces and estimate
		\eqref{u-e-Hs-bound} to \eqref{bixi} gives 
		%
		\begin{equation}
			\label{deriv1}
			\begin{split}
				\|J_\ee u_\ee \p_x J_\ee u_\ee  
				\|_{H^{s-1}(\ci)}
				& \lesssim
				\|J_\varepsilon u_\varepsilon \|_{H^s(\ci)}^2
				\\
				&\lesssim
				\| u_\varepsilon \|_{H^s(\ci)}^2
				\\
				&\lesssim
				\|u_0\|_{H^s(\ci)}^2.
			\end{split}
		\end{equation}
		We also have
		\begin{equation*}
			\begin{split}
				\|\p_x (1-\p_x^2)^{-1} (u_\ee)^2\|_{H^{s-1}(\ci)}
				& \le \| (u_\ee)^2\|_{H^{s-1}(\ci)}
				\end{split}
		\end{equation*}
		which by the algebra property and estimate \eqref{u-e-Hs-bound}
		gives
		\begin{equation}
			\begin{split}
				\label{deriv2}
				\|\p_x (1-\p_x^2)^{-1} (u_\ee)^2\|_{H^{s-1}(\ci)}
				& \lesssim \|u_\ee\|^2_{H^s(\ci)} 
				\\
				& \lesssim  \|u_0\|^2_{H^s(\ci)}.
			\end{split}
		\end{equation}
		Similarly,
		\begin{equation}
			\begin{split}
				\label{deriv3}
				\|\p_x (1-\p_x^2)^{-1} (\p_x u_\ee)^2\|_{H^{s-1}(\ci)}
				& \lesssim \|\p_x u_\ee\|^2_{H^{s-1}(\ci)} 
				\\
				& \lesssim  \|u_\ee \|^2_{H^s(\ci)}
				\\
				& \lesssim \|u_0\|^2_{H^s(\ci)}.
			\end{split}
		\end{equation}
		Combining \eqref{deriv1}, \eqref{deriv2}, and \eqref{deriv3}, we
		obtain \eqref{dt-u-e-Hs-bound}. $\qquad \Box$
		%%%%%%%%%%%%%%%%%%%%%%%%
		%
		%     Choosing  a convergent subsequence
		%
		%%%%%%%%%%%%%%%%%%%%%%%%
		\subsection{ Choosing  a convergent subsequence.}
		%
		Next we shall show that  the family $\{ u_\ee\}$ has a convergent subsequence
		whose limit $u$ solves the Hyperelastic i.v.p. 
		Let
		$$
		I= [-T, T].
		$$
		By Lemma \ref{hr_wp} we have 
		%
		\begin{equation}
			\label{C-1-fam}
			\{u_\ee\}\subset C(I, H^s(\ci))\cap C^1(I, H^{s-1}(\ci))
		\end{equation}
		%
		and bounded. Since $I$ is compact, we have  
		%
		\begin{equation}
			\label{Lip-1-fam}
			\{u_\ee\}\subset L^{\infty}(I, H^s(\ci))\cap C^1(I,
			H^{s-1}(\ci)).
		\end{equation}
		%
		Now, by the Riesz Lemma, we can identify $H^s(\rr)$ with
		$(H^s(\rr))^*$, where for $w, \psi \in H^s(\rr)$ the duality is
		defined by 
		\begin{equation*}
			T_w(\psi) = <w, \psi>_{H^s(\rr)}.
		\end{equation*}
		Hence, by the Riesz Representation Theorem it follows that we can
		identify \\ $L^\infty(I, H^s(\ci)) $ with the dual space of $L^1(I,
		H^{s}(\ci)$, where for $v\in L^\infty(I, H^s(\ci)) $ and $ \phi \in
		L^1(I, H^{s}(\ci))$ the duality is defined by  
		%
		\begin{equation}
			T_v(\phi) = \int_I <v (t), \phi (t)>_{H^s(\rr)} dt  = \int_I
			 \int_{\rr}
			 \widehat{v}(\xi, t) \overline{\widehat{\phi}}(\xi, t) \cdot (1
			 + \xi^2)^s \ d \xi dt.
		\end{equation}
		%
		Next, we recall Aloaglu's Theorem:
		\begin{theorem}
			If $X$ is a normed vector space,
			the closed unit ball $B^* = \{f \in X^* : \|f\| \le
			1\}$ in $X^*$ is compact in the $weak^*$ topology.
		\end{theorem}
		Therefore the bounded family $\{u_\ee\}$ is compact 
		in the weak$^*$ topology of \\
		$L^\infty(I, H^s(\ci))$. More precisely,
		there is a sequence  $\{ u_{\ee_n} \}$ converging
		weakly to a $ u\in L^{\infty}(I, H^s(\ci))$;
		that is 
		%
		\begin{equation}
			\label{weak-conv}
			\lim_{n\to \infty} T_{u_{\ee_n}}(\phi)  =  T_u (\phi) 
			\; \;		
			\text{ for all } \;\;  \phi \in L^1(I, H^{s}(\ci)).
		\end{equation}
		%
		In order to show that  $u$ solves the HR i.v.p. we need to 
		obtain a stronger  convergence for  $u_{\ee_n}$ so that 
		we can take the limit in the mollified HR equation.
		In fact we will prove that 
		%
		\begin{equation}
			\label{strong-conv}
			u_{\ee_n}\longrightarrow u
			\quad
			\text{ in } \,\,   C(I, H^{s-\sigma}(\ci)),\ \text{for any} \
			\, 0 < \sigma <
			1.
		\end{equation}
		%
		For this we will need the following interpolation  result:
		%%%%%%%%%%%%%%%%%%%%%%%%%%%
		%
		%
		%                 Interpolation Lemma
		%
		%
		%%%%%%%%%%%%%%%%%%%%%%%%%%%
		\begin{lemma}
			\label{interpolation-lem}
			(Interpolation)     Let  $s > \frac{3}{2}$.
			If $v \in C(I, H^s(\ci)) \cap C^1(I, H^{s-1}(\ci))$
			then $v \in C^\sigma (I, H^{s- \sigma}(\ci))$ for  $0 < \sigma < 1$.
		\end{lemma}
		%
		\subsection{ Proof.}  We have
		\begin{equation*}
			\begin{split}
				& \frac{\|v(t) - v(t')\|^2_{H^{s - \sigma}}}{|t - t'|^{2\sigma}}
				\\
				& = 
				\sum_{\xi \in \zz} (1 + \xi^2)^{s- \sigma} 
				\frac{|\hat{v}(\xi, t) - \hat{v}(\xi, t')|^2}{|t-t'|^{2\sigma}} d\xi\\
				& = \sum_{\xi \in \zz} (1+\xi^2)^s 
				\bigg(\frac{1}{(1+ \xi^2)|t - t'|^2} \bigg)^\sigma |\hat{v}(\xi, t)- \hat{v}(\xi, t')|^2 d\xi\\
				& \leq \sum_{\xi \in \zz}(1+\xi^2)^s \bigg( 1 + \frac{1}{(1+\xi^2)|t-t'|^2} \bigg)
				|\hat{v}(\xi,t) - \hat{v}(\xi,t')|^2 d\xi \\
				& \leq \sum_{\xi \in \zz} (1+ \xi^2)^s |\hat{v}(\xi, t)- \hat{v}(\xi, t')|^2 d\xi
				+ \sum_{\xi \in \zz} (1+ \xi^2)^{s-1} \frac{|\hat{v}(\xi, t) - \hat{v} (\xi, t')|^2}{|t-t'|^2} \\
				& \leq  \sup_t \|v(t)\|_{H^s(\ci)}^2 + \sup_t
				\| \partial_t v(t) \|_{H^{s-1}(\ci)}^2
				\\
				& < \infty.
				\\
			\end{split}
		\end{equation*}
		%
		%
		Next, using this lemma we will show that the family $\{u_\ee\}$ is
		equicontinuous in $C(I, H^{s-\sigma}(\ci))$, $0 < \sigma < 1$. We
		will follow this by proving that there exists a sub-family
		$\{u_{\ee_n} \}$ that is precompact in $C(I,
		H^{s-\sigma}(\ci))$. These two facts, in conjunction with Ascoli's
		Theorem, will yield
		\begin{equation}
			\label{strong-conv2}
			u_{\ee_n} \to u \; \; \text{in} \; \; C(I,H^{s-\sigma}(\ci)),
			\quad
			0 < \sigma < 1.
		\end{equation}
		%%%%%%%%%%%%%%%%%%%%%%
		%
		%
		%       Equicontinuity
		%
		%
		%%%%%%%%%%%%%%%%%%%%%%
		%
		\subsection{  Equicontinuity of $\{u_\ee\}_\ee$  in
		$C(I,H^{s-\sigma}(\ci))$.} Applying  Lemma \ref{interpolation-lem} gives 
		%
		\begin{equation}
			\label{equic-1}
			\sup_{t \neq t'} \frac { \|u_\ee(t) - u_\ee(t') \|_{H^{s -
			\sigma}(\ci)}}{|t - t'|^\sigma} < c<\infty
		\end{equation}
		%
		or
		%
		\begin{equation}
			\label{equic-2}
			\|u_\ee(t) - u_\ee(t') \|_{H^{s - \sigma}(\ci)}< c|t - t'|^\sigma, 
			\text{ for all }  \,\,  t, t'\in I,
		\end{equation}
		%
		which shows that  the family  $\{u_\ee\}$ is equicontinuous in 
		$C(I, H^{s-\sigma}(\ci))$. $\qquad \Box$
		%
		%
		%%%%%%%%%%%%%%%%%%%%%%
		%
		%
		%      PreCompactness
		%
		%
		%%%%%%%%%%%%%%%%%%%%%%%%%%
		%
		%
		%
		%
		%		
		\subsection{ Precompactness of $\{u_\ee(t)\}$ in $H^{s-\sigma}(\ci))$.}
		Now recall that
		\begin{equation}
			\label{compact-1}
			\|u_\ee(t)\|_{H^{s}(\ci)}
			\le
			2 \|u_0 \|_{H^s(\ci)}, \,
			\quad
			t\in I.
		\end{equation}
		%
		By Kondrachov's Theorem, the inclusion $H^s(\ci) \subset H^{s-
		\sigma }(\ci)$ is compact. By \eqref{compact-1},
		it follows that $\{u_\ee(t)\}$ is precompact in $H^{s-\sigma}(\ci)$.
		$\quad \Box$
		%
		%
		%
		%
		We are now in a position to apply Ascoli's Theorem: 
		\begin{theorem}
			\label{Ascoli}
			(Ascoli)  Let $X$ be a Banach space, $I$ be a compact metric space,
			and $C(I,X)$  be the set of continuous functions $f: I\longrightarrow X$.
			Suppose $S \subset C(I,X)$  has the following properties:
			%
			\begin{itemize}
				\item[(1)]   $S$ is  equicontinuous.
				\item[(2)]  For each $x \in M$ that the set $S(x) = \{f(x)\}$  is  precompact in $X$.
			\end{itemize} 
			%
			Then $S$  is  precompact  in  $C(I,X)$.
		\end{theorem}
		Compiling our previous results on equicontinuity and precompactness
		and applying Theorem \ref{Ascoli}, we
		conclude that there exists a subfamily $\left\{ u_{\ee_n} \right\}$
		such that
		\begin{equation}
			\label{strong-conv-of-u_ep}
			u_{\ee_n} \to u \; \; \text{in} \; \; C(I, H^{s-\sigma}(\ci)).
		\end{equation}
		%
		%
		%
		%%%%%%%%%%%%%%%%%%%%%%%%%%%%%%%%%
		%
		%
		%     Verifying that the limit $u$ solves Burgers equation
		%
		%
		%%%%%%%%%%%%%%%%%%%%%%%%%%%%%%%%%
		\subsection{ Verifying that the limit $u$ solves the HR equation.} 
		The following lemma will play a crucial role in our proof of the
		existence of a solution to the HR i.v.p.
		\begin{lemma}
			\label{lem:cc}
			We have
			\begin{equation}
				\begin{split}
					\label{burgers_and_nonlocal_conv}
				&  J_{\varepsilon_n} u_{\varepsilon_n} 
				\cdot J_{\varepsilon_n} \p_x u_{\varepsilon_n} 
				\to  u \partial_x u \; \; 
				\text{in} \; \;
				C(I, H^{s-\sigma-1}(\ci)). 
			\end{split}
			\end{equation}
		\end{lemma}
		%
		\subsection{ Proof.} It is implied by the following propositions:
		\begin{proposition}
			\label{prop:1aa}
			\begin{equation}
				\begin{split}
					 J_{\ee_n} u_{\ee_n} \to  u \ \ \text{in} \ \
					C(I, H^{s-\sigma}(\ci)).
					\label{}
				\end{split}
			\end{equation}
		\end{proposition}
			\subsection{ Proof.} Note that
			\begin{equation}
				\begin{split}
					& \| u -  J_{\ee_n} u_{\ee_n}
					\|_{C(I, H^{s-\sigma}(\ci))}
					\\
					&= \| u -  J_{\ee_n} u_{\ee_n} \pm 
					u_{\ee_n} \|_{C(I, H^{s-\sigma}(\ci))}
					\\
					& = \| u -  u_{\ee_n}
					\|_{C(I,H^{s-\sigma}(\ci))} + \| (I - J_{\ee_n})
					u_{\ee_n} \|_{C(I, H^{s-\sigma}(\ci))}.
					\label{1bb}
				\end{split}
			\end{equation}
			Applying the estimates
			\begin{equation*}
				\begin{split}
					& \|I-J_{\ee_n} \|_{L(H^{s-\sigma}(\ci), H^{s -
					\sigma}(\ci))} = o(1),
					\\
					& \|u_{\ee_n}\|_{H^{s-\sigma}(\ci)} \le 2
					\|u_0\|_{H^{s-\sigma}(\ci)}
				\end{split}
			\end{equation*}
			to \eqref{1bb} gives
			\begin{equation}
				\label{2bb}
				\begin{split}
					\| u -  J_{\ee_n} u_{\ee_n}\|_{H^{s-\sigma}(\ci)}
					\le \left( \| u -  u_{\ee_n}
					\|_{C(I, H^{s-\sigma}(\ci))} + o(1) \cdot \|u_0
					\|_{H^{s-\sigma}(\ci)} \right).
				\end{split}
			\end{equation}
			Letting $\ee_n \to 0$ in \eqref{2bb} and applying
			\eqref{strong-conv-of-u_ep} completes the proof. $\quad \Box$
			%
			%
			\begin{proposition}
				\label{prop:dd}
				\begin{equation}
					\begin{split}
						 J_{\ee_n} \p_x u_{\ee_n} \to  \p_x u \ \
						\text{in} \ \ C(I, H^{s-\sigma-1}(\ci)).
						\label{0dd}
					\end{split}
				\end{equation}
			\end{proposition}
			\subsection{ Proof.} 
			\begin{equation*}
				\begin{split}
					\|\p_x u - J_\ee \p_x u_{\ee_n} \|_{C(I,
					H^{s-\sigma-1}))}  
					& = \|\p_x u - \p_x J_\ee u_{\ee_n} \|_{C(I,
					H^{s-\sigma-1}(\ci))} 
					\\
					& \le \|u - J_\ee u_{\ee_n} \|_{C(I,
					H^{s-\sigma}(\ci))}.
				\end{split}
			\end{equation*}
			Applying Proposition \ref{prop:1aa} completes the proof. $\quad
			\Box$
			%
			This completes the proof of Lemma \ref{lem:cc}. $\quad \Box$
		%
		Note that since $\|\p_x (1-\p_x^2)^{-1}\|_{L(H^s(\ci), H^s(\ci))}
		\le 1$ for all $s \in \rr$, it follows immediately from
		\eqref{strong-conv-of-u_ep} that
		\begin{equation}
			\begin{split}
				& \p_x(1- \p_x^2)^{-1} \left( \frac{3-\gamma}{2}
				(u_{\ee_n})^2
				 + \frac{\gamma}{2} (\p_x u_{\ee_n})^2 \right )
				 \\
				 & \to
				 \p_x(1- \p_x^2)^{-1} \left( \frac{3-\gamma}{2} u^2
				 + \frac{\gamma}{2} (\p_x u)^2 \right ) \ \
				 \text{in} \ \ C(I, H^{s-\sigma-1}(\ci)).
				\label{non-local-convergence}
			\end{split}
		\end{equation}
		Combining \eqref{burgers_and_nonlocal_conv} and
		\eqref{non-local-convergence}, and applying the Sobolev Imbedding
		Theorem, we deduce 
		\begin{equation}
			\begin{split}
				& -\gamma (J_{\ee_n} u_{\ee_n} \cdot J_{\ee_n} \p_x
				u_{\ee_n}) - \p_x(1- \p_x^2)^{-1} \left( \frac{3-\gamma}{2}
				(u_{\ee_n})^2
				 + \frac{\gamma}{2} (\p_x u_{\ee_n})^2 \right )
				 \\
				 \to & -\gamma u \p_x u -
				 \p_x(1- \p_x^2)^{-1} \left( \frac{3-\gamma}{2} u^2
				 + \frac{\gamma}{2} (\p_x u)^2 \right ) \ \
				 \text{in} \ \ C(I, C(\ci)).
				\label{loc-non-loc-tog}
			\end{split}
		\end{equation}
		Furthermore, we note that the convergence  
		%
		\begin{equation}
			\label{weak-conv-2}
			T_{u_{\ee_n}}(\phi)  \longrightarrow  T_u(\phi) \;
			\text{ for all } \;  \phi \in L^1(I, H^{s}(\ci))
		\end{equation}
		%
		can be restated as 
		%
		\begin{equation}
			u_{\ee_n}  \longrightarrow  u
			\quad
			\text{ in }  \,\,
			\mathcal{D}'(I\times \ci).
		\end{equation}
		%
		This implies 
		%
		\begin{equation}
			\label{distib-conv-2}
			\p_tu_{\ee_n}  \longrightarrow  \p_tu
			\quad
			\text{ in }  \,\, \mathcal{D}'(I\times \ci).
		\end{equation}
		%
		Since for all $n$ we have 
		%
		\begin{equation}
			\p_tu_{\ee_n} 
			=
			-\gamma (J_{\varepsilon_n} u_{\varepsilon_n}  \cdot
			J_{\varepsilon_n}\partial_x u_{\varepsilon_n}) - \p_x (1-
			\p_x^2)^{-1} \left
			[\frac{3-\gamma}{2}(u_\ee)^2 + \frac{\gamma}{2}(\p_x u_\ee)^2 \right ] 
		\end{equation}
		%
		by the uniqueness  of the limit in \eqref{loc-non-loc-tog}
		we must have
		%
		\begin{equation}
			\label{1000y}
			\partial_t u =- \gamma u \partial_x u- \p_x (1- \p_x^2)^{-1} \left
			[\frac{3-\gamma}{2}u^2 + \frac{\gamma}{2}(\p_x u)^2 \right ].
		\end{equation}
		%
		Thus we have constructed a solution $u \in L^\infty(I, H^s(\ci))$
		to the HR i.v.p. $\qquad \Box$
		It remains to prove that $u \in C(I, H^s(\ci)).$
		%%%%%%%%%%%%%%%%%%%%%%%%%%
%
%
%Proof that  $u \in C(I, H^s(\ci)) \bigcap C^1(I, H^{s-1}(\ci))$.
%
%
%
%%%%%%%%%%%%%%%%%%%%%%%%%%
\subsection{ Proof that $u \in C(I, H^s(\ci))$.} 
We first outline our strategy. Since \\
$u \in L^\infty(I, H^s(\ci))$, it is a
continuous function from $I$ to $H^s(\ci)$ with respect to the weak
topology on $I$; that is, for $\{t_n\} \subset I$ such that $t_n \to t$, we
have
\begin{equation}
	\begin{split}
		<u(t_n), \ v>_{H^s(\ci)} \ \longrightarrow \
		<u(t), \ v>_{H^s(\ci)}, \quad \forall
		v \in H^s(\ci).
		\label{1ff}
	\end{split}
\end{equation}
Next, note that
\begin{equation}
	\begin{split}
		\|u(t) - u(t_n) \|_{H^s(\ci)}^2
		& = <u(t) - u(t_n), \ u(t) -
		u(t_n)>_{H^s(\ci)}
		\\
		& = \|u(t)\|_{H^s(\ci)}^2 + \|u(t_n)\|_{H^s(\ci)}^2
		\\
		& - <u(t_n), \
		u(t) >_{H^s(\ci)} - <u(t), u(t_n)>_{H^s(\ci)}.
		\label{2ff}
	\end{split}
\end{equation}
Applying \eqref{1ff} and \eqref{2ff}, we see that
\begin{equation}
	\begin{split}
		\lim_{n \to \infty} \|u(t) - u(t_n)\|_{H^s(\ci)}^2 = \left[ \lim_{n
		\to \infty} \|u(t_n)\|_{H^s(\ci)}^2
		\right] - \|u(t)\|_{H^s(\ci)}^2.
		\label{3ff}
	\end{split}
\end{equation}
Hence, by \eqref{3ff}, to prove that $u \in C(I, H^s(\ci))$, it will be
enough to show that the map $t \mapsto \|u(t)\|_{H^s(\ci)}$ is a continuous
function of $t$. However, this will follow from the energy
estimate
		\begin{equation}
			\label{en-est-u}
			\frac{1}{2} \frac{d}{dt} \|u(t)\|_{H^s(\ci)}^2
			\le c_s \|u(t)\|_{H^s(\ci)}^3, \quad |t| \le T
		\end{equation}
		which we now derive. Applying $D^s$ to both sides of
		\eqref{1000y}, multiplying the
		resulting equation by $D^s u$, and integrating for $x\in \ci$, we obtain
		\begin{equation}
			\begin{split}
				\label{bound-int}
				\frac 12
				\frac{d}{dt} \|u \|_{H^s}^2
				=
				&-
				\gamma \int_{\ci}   D^s (u \p_x u) \cdot
				D^s u \  dx
				\\
				&- \frac{3 -\gamma}{2} \int_{\ci}  D^{s-2} \p_x (u^2) 
				\cdot D^s u \ dx
				\\
				&- \frac{\gamma}{2} \int_{\ci}   D^{s-2} \p_x (\p_x u)^2
				\cdot D^s u \ dx.
			\end{split}
		\end{equation}
		First we estimate
	\begin{equation}
		\begin{split}
			\left | - \frac{3 -\gamma}{2} \int_\ci D^{s-2} \p_x (u^2) \cdot
			D^s u \; dx \right |
			& \le \left | \frac{3 -\gamma}{2} \right |
			\int_\ci \left |
			D^{s-2} \p_x (u^2) \cdot D^s u \right | dx 
			\\
			& \le \left | \frac{3 -\gamma}{2} \right |
			\|D^{s-2} \p_x (u^2) \|_{L^2(\ci)} 
			\|D^s u \|_{L^2(\ci)}
			\\
			& \le \left | \frac{3 -\gamma}{2} \right |
			\|D^{s-1} (u^2) \|_{L^2(\ci)} 
			\|D^s u \|_{L^2(\ci)}
			\\
			& = \left | \frac{3 -\gamma}{2} \right |
			\| u^2 \|_{H^{s-1}(\ci)} \| u \|_{H^s(\ci)}
			\\
			& \le
			\left | \frac{3 -\gamma}{2} \right | \| u^2 \|_{H^s(\ci)} \| u
			\|_{H^s(\ci)}.
		\end{split}
	\end{equation}
	%
	%
	Applying the algebra property, we obtain
	%
	\begin{equation}
		\label{hl1-prime}
		\begin{split}
			\left | -\frac{3 -\gamma}{2} \int_\ci D^{s-2} \p_x u^2 \cdot
			D^s u \; dx \right |
			\le c_s' \| u \|_{H^s(\ci)}^3.
		\end{split}
	\end{equation}
	%
	%
	We also have
	\begin{equation}
		\begin{split}
			\left | -\frac{\gamma}{2} \int_\ci D^{s-2} \p_x (\p_x u)^2 \cdot
			D^s u \; dx \right |
			& \le \left | \frac{\gamma}{2} \right |
			\int_\ci \left | D^{s-2} \p_x (\p_x u)^2 
			\cdot D^s u \right | \; dx
			\\
			& \le \left | \frac{\gamma}{2} \right |
			\| D^{s-2} \p_x (\p_x u)^2 \|_{L^2(\ci)}
			\| D^s u \|_{L^2(\ci)}
			\\
			&  \le \left | \frac{\gamma}{2} \right | \|(\p_x u)^2
			\|_{H^{s-1}(\ci)} \| u \|_{H^s(\ci)} 
		\end{split}
	\end{equation}
	and applying the algebra property yields
	\begin{equation}
		\label{hl2-prime}
		\left | -\frac{\gamma}{2} \int_\ci D^{s-2} (\p_x u)^2 \cdot
		D^s u \; dx \right |
		\le c_s'' \|u\|_{H^s(\ci)}^3.
	\end{equation}
	It remains to estimate 
	\begin{equation*}
		- \gamma \int_{\ci} \left [  D^s (u \p_x u) \cdot
		D^s u \right ]  dx
	\end{equation*}
	We have
	\begin{equation} 
	\begin{split}
		\label{B-moli-int-v'}
		-  \gamma \int_{\ci} \left [D^s (u \p_x u) \cdot D^s
		u \right ] \ dx
		= &- \gamma  \int_\ci
		\left [ D^s(u \partial_x u) \cdot   D^s u \right ] \ dx
		\\
		=& - \gamma \int_\ci
		\big[ 
		D^s(u\p_x u)  -  u D^s (\p_xu)
		\big] \cdot
		D^s u   \ dx
		\\
		&
		- \gamma \int_\ci
		u D^s (\p_xu) \cdot
		D^su \ dx.
	\end{split}
\end{equation}
%
%
%
We now estimate \eqref{B-moli-int-v'} in parts. Applying the Cauchy-Schwarz inequality gives
%
\begin{equation*} 
	\begin{split}
		& \Big|
		- \gamma \int_\ci
		\big[ 
		D^s(u\p_x u)  -  u D^s (\p_xu)
		\big] \cdot
		D^s u   \, dx
		\Big|
		\\
		& \le
		|\gamma| \cdot \|
		D^s(u\p_x u)  -  u D^s (\p_xu)
		\|_{L^2(\ci)}
		\|
		D^s u 
		\|_{L^2(\ci)}
		\\
		& =
		|\gamma| \cdot \| D^s(u\p_x u)  -  u D^s (\p_xu)
		\|_{L^2(\ci)}
		\|
		u
		\|_{H^s(\ci)}
			\end{split}
\end{equation*}
Applying \eqref{int1-est-calc3}, we obtain
\begin{equation*}
\begin{split}
		\Big|
		- \gamma \int_\ci
		\big[ 
		D^s(u\p_x u)  -  u D^s (\p_xu)
		\big]
		D^s u   \, dx
		\Big|
		&\le
		 c_s'''   \| \p_x u \|_{L^\infty(\ci)} 
		\| u \|_{H^s(\ci)}^2.
	\end{split}
\end{equation*}
%\label{int1-est-calc2'}
%
Next, we apply Cauchy-Schwartz and the Sobolev Imbedding Theorem to deduce 
	%
	%
	%
	\begin{equation} 
		\label{int1-est-calc5'}
		\begin{split}
			\Big|
			\int_\ci
			\left [u D^s (\p_x u)
			\cdot  D^s u \right ] dx
			\Big|
			& =
			\frac{1}{2} \Big|
			   \int_\ci
			\left [u \p_x\left(D^s u\right)^2 \right ] \ dx
			\Big|
			\\
			& \le
			\frac{1}{2} \int_\ci \Big |
			\left [\p_x u \, (D^s u)^2  \right ] 
			\Big| \ dx
			\\
			& \le
			\frac{1}{2}
			\| \p_x u \|_{L^\infty(\ci)} 
			\| u \|_{H^s(\ci)}^2.
			\\
			& \le c_s'''' \|u\|_{H^s(\ci)}^3.
		\end{split}
	\end{equation}
	%
	%
	%
	Combining \eqref{hl1-prime}, \eqref{hl2-prime},
	and \eqref{int1-est-calc5'}, we obtain \eqref{en-est-u}, as desired.
	\subsection{ Size of the solution}. 
	Letting  $y(t)=  \|u(t)\|_{H^s(\ci)}^2$ inequality \eqref{en-est-u}
	takes the form
	%
	\begin{equation} 
		\label{energy-y-ineq'}
		\frac 12
		y^{-3/2}\frac{dy}{dt}
		\le
		c_s,
		\qquad
		y(0)=y_0=  \|u_0\|_{H^s(\ci)}^2.
	\end{equation}
	%
	Suppose $t$ is non-negative. Then integrating  \eqref{energy-y-ineq'}
	from  0 to $t$ gives
	%
	\begin{equation*} 
		\frac{1}{\sqrt{y_0}}  - \frac{1}{\sqrt{y(t)}} 
		\le 
		c_s t.
	\end{equation*}
	%
	%
	Replacing $y(t)$ with   $\|u(t)\|_{H^s(\ci)}^2$  and solving for  $\|u(t)\|_{H^s(\ci)}$
	we obtain the formula
	%
	\begin{equation} 
		\label{norm-u(t)-formula'}
		\|u(t)\|_{H^s(\ci)}
		\le
		\frac{ \|u_0\|_{H^s(\ci)}}{1-c_s\|u_0\|_{H^s(\ci)} t}.
	\end{equation}
	%
	Now, note that our solution $u$ inherits the common lifespan $T$ of the family
	$\{u_\ee\}$; that is, $u$ has lifespan
	\begin{equation*}
		T
		=
		\frac{1}{2 c_s \|u_0\|_{H^s(\ci)}}.
	\end{equation*}
	Substituting into \eqref{norm-u(t)-formula'} we obtain	
	%
	\begin{equation*} 
		\label{u(t)-u(0)-bound'}
		\|u(t)\|_{H^s(\ci)}
		\le
		\frac{ \|u_0\|_{H^s(\ci)}}{1-(c_s\|u_0\|_{H^s(\ci)})/(2 c_s \|u_0\|_{H^s(\ci)})},
	\end{equation*}
	%
	which simplifies to 
	%
	\begin{equation*}
		\|u(t)\|_{H^s(\ci)}
		\le
		2 \|u_0\|_{H^s(\ci)},
		\quad 
		0\le t \le T.
	\end{equation*}
	Similarly, for negative $t$, we have
	\begin{equation*}
		\|u(t)\|_{H^s(\ci)}
		\le
		2 \|u_0\|_{H^s(\ci)},
		\quad 
		-T \le t < 0.
	\end{equation*}
	Hence,
	\begin{equation}
		\label{uniform_bound_for_u}
		\|u(t)\|_{H^s(\ci)}
		\le
		2 \|u_0\|_{H^s(\ci)},
		\quad 
		|t| \le T.
	\end{equation}
		%
		\subsection{ Space of the solution}.
	Derivating the left hand side of \eqref{en-est-u} and simplifying, we obtain
	\begin{equation}
		\label{en-est-u-simplified}
	\frac{d}{dt} \|u(t)\|_{H^s(\ci)} \le c_s \|u(t)\|_{H^s(\ci)}^2, \quad |t| \le T.
	\end{equation}
	Since $\|u(t)\|_{H^s(\ci)}$
	is uniformly bounded for $|t| \le T$ by
	\eqref{uniform_bound_for_u}, we conclude from
	\eqref{en-est-u-simplified} that the map $t \mapsto
	\|u(t)\|_{H^s(\ci)}$ is Lipschitz continuous in $t$, for $|t| \le T$.
	Therefore, by \eqref{3ff}, $u \in C(I, H^s(\ci))$. 
	%
	%
	%
	%
	%
	\subsection{Uniqueness.}
	%
	%
	Let $u,\omega \in C(I, H^s(\ci)), \ s > 3/2$ be two solutions to the
	Cauchy-problem \eqref{hyperelastic-rod-equation}-\eqref{init-cond} with
	common initial data. Let $v=u-w$; since
	\begin{align*}
		\p_t u 
		& = - \gamma u \p_x u - D^{-2} \p_x \left[ \frac{3-\gamma}{2} u^2 +
		\frac{\gamma}{2}\left( \p_x u \right)^2 \right]
		\\
		\p_t w & = -\gamma w \p_x w - D^{-2} \p_x \left[
		\frac{3-\gamma}{2} w^2 + \frac{\gamma}{2}(\p_x w)^2 
		\right]
	\end{align*}
	we subtract the two equations to obtain 
	\begin{equation*}
		\begin{split}
			\p_t v
			= -\frac{\gamma}{2} \p_x [v(u + w)] - D^{-2} \p_x \left\{
			\frac{3-\gamma}{2}[v(u+w)] + \frac{\gamma}{2}[\p_x v \cdot \p_x (u+w)]
			\right\}
		\end{split}
	\end{equation*}
	and hence
	\begin{equation}
		\begin{split}
			D^\sigma \p_t v = -\frac{\gamma}{2} D^\sigma \p_x [v(u+w)] - D^{\sigma -2} \p_x
			\left\{ \frac{3-\gamma}{2} [v(u+w)] + \frac{\gamma}{2} [\p_x v
			\cdot \p_x
			(u+w)]
			\right\}.
			\label{1v}
		\end{split}
	\end{equation}
	Multiplying both sides of \eqref{1v} by $D^\sigma v$ and integrating, we obtain
	\begin{equation}
		\begin{split}
			\frac{1}{2} \frac{d}{dt} \|v\|_{H^\sigma(\ci)}^2
			& =  \overbrace{-\frac{\gamma}{2} \int_{\ci} D^\sigma \p_x [v(u+w)] \cdot
			D^\sigma v \ dx}^i
			\\
			& \overbrace{- \frac{3-\gamma}{2} \int_{\ci}  D^{\sigma -2}
			\p_x[v(u+w)] \cdot
			D^\sigma v \ dx}^{ii} 
			\\
			& - \overbrace{\frac{\gamma}{2} \int_{\ci} D^{\sigma -2} \p_x [ \p_x v
			\cdot \p_x (u+w)]\cdot D^\sigma v \ dx }^{iii}.
			\label{2v}
		\end{split}
	\end{equation}
	We will estimate (\hyperref[2v]{ii}) first.
	Applying Cauchy-Schwartz, we have 
	\begin{equation*}
		\begin{split}
			|ii|
			& \le \left | \frac{3-\gamma}{2} \right | \|D^{\sigma -2}
			\p_x [v(u+w)] \cdot D^\sigma
			v  \|_{L^1(\ci)}
			\\
			 & \le  \left | \frac{3-\gamma}{2} \right | \|D^{\sigma -2} \p_x [v(u+w)]
			\|_{L^2(\ci)} \|v\|_{H^\sigma(\ci)}
			\\
			& \lesssim \|v(u+w)\|_{H^{\sigma -1}(\ci)} \|v\|_{H^\sigma(\ci)}
		\end{split}
	\end{equation*}
	which by the algebra property and the Sobolev
	Imbedding Theorem gives
\begin{equation}
		\begin{split}
		|ii| \lesssim \|u+w\|_{H^{\sigma -1}(\ci)} \|v\|_{H^\sigma(\ci)}^2.
			\label{3v}
		\end{split}
	\end{equation}
	To estimate (\hyperref[2v]{iii}) we first apply
	Cauchy-Schwartz and the Sobolev Imbedding Theorem:
	\begin{equation*}
		\begin{split}
		|iii| & \le	\left | \frac{\gamma}{2} \right | \|D^{\sigma -2} \p_x
			[\p_x v \cdot \p_x (u+w)] \cdot D^\sigma v  \|_{L^1(\ci)} 
			\\
			& \le  \left | \frac{\gamma}{2} \right | \|D^{\sigma -2} \p_x
			[\p_x v \cdot \p_x (u+w)] \|_{L^2(\ci)}
			\|v\|_{H^\sigma(\ci)}
			\\
			& \le \left |\frac{\gamma}{2} \right|
			\|[\p_x v \cdot \p_x (u+w)] \|_{H^{\sigma -1}(\ci)}
			\|v\|_{H^\sigma(\ci)}.
		\end{split}
	\end{equation*}
	Restrict $1/2 < \sigma < 1$. Then applying Lemma \ref{impo}, we conclude
	that
	\begin{equation}
		\begin{split}
			|iii|
			& \le C \left | \frac{\gamma}{2} \right |
			\|\p_x(u+w) \|_{H^{\sigma}(\ci)}
			\|\p_x v\|_{H^{\sigma -1}(\ci)} \|v\|_{H^\sigma(\ci)}
			\\
			& \lesssim \|u+w \|_{H^{\sigma + 1}(\ci)}
			\|v\|_{H^\sigma(\ci)}^2.
			\label{3'v}
		\end{split}
	\end{equation}
	It remains to estimate (\hyperref[2v]{i}).
	Proceeding, we rewrite
	\begin{equation}
		\begin{split}
			|i| & =
			\left |
			-\frac{\gamma}{2} \int_{\ci} \left[ D^\sigma \p_x, \ u+w \right]v \cdot
			D^\sigma v \ dx - \frac{\gamma}{2} \int_{\ci} (u+w) D^\sigma
			\p_x v \cdot D^\sigma v\ dx
			\right | 
			\\
			& \le \left |
			-\frac{\gamma}{2} \int_{\ci} \left[ D^\sigma \p_x, \ u+w \right]v \cdot
			D^\sigma v \ dx \right |
			+ \left | \frac{\gamma}{2} \int_{\ci} (u+w) D^\sigma \p_x v
			\cdot D^\sigma v\
			dx \right |.
			\label{4v}
		\end{split}
	\end{equation}
	We now estimate \eqref{4v} in pieces. Observe that by integrating by parts
	and applying Cauchy-Schwartz we have
	\begin{equation}
		\begin{split}
			\left | \frac{\gamma}{2}\int_{\ci} (u+w) D^\sigma \p_x v \cdot
			D^\sigma v \ dx \right |
			& = \left | -\frac{\gamma}{2} \int_{\ci} \p_x (u+w) D^\sigma v
			\cdot D^\sigma v \ dx \right |
			\\
			& \lesssim \|\p_x (u+w) D^\sigma v \|_{L^2(\ci)} \|D^\sigma
			v\|_{L^2(\ci)}
			\\
			& \lesssim \|\p_x (u+w)\|_{L^\infty(\ci)}
			\|v\|_{H^\sigma(\ci)}^2.
			\label{4'v}
		\end{split}
	\end{equation}
	To estimate the remaining piece of \eqref{4v}, we recall first that we
	have the restriction $1/2 < \sigma < 1$. However, this will not prevent
	us from applying Corollary \ref{cor1}; in fact, choosing $\ 3/2 < \rho
	< s,  \ 1/2< \sigma <\min\{1, \ \rho -1 \}$, we obtain
	\begin{equation}
		\begin{split}
			\left | -\frac{\gamma}{2} \int_{\ci} [D^\sigma \p_x, \ u+w] v
			\cdot D^\sigma v \ dx \right |
			& \le \left | \frac{\gamma}{2} \right| \int_{\ci} \left |
			[D^\sigma \p_x, \ u+w] v
			\cdot D^\sigma v \right | dx 
			\\
			& \lesssim \|[D^\sigma \p_x, \ u+w]v\|_{L^2(\ci)}
			\|v\|_{H^\sigma(\ci)} \\
			& \lesssim \|u+w\|_{H^\rho(\ci)} \|v\|_{H^\sigma(\ci)}^2.
			\label{7v}
		\end{split}
	\end{equation}
	Combining \eqref{4'v} and \eqref{7v} and applying the Sobolev Imbedding
	Theorem, we obtain the estimate
	\begin{equation}
		\begin{split}
			|i| \lesssim \|u+w\|_{H^\rho(\ci)} \|v\|_{H^\sigma(\ci)}^2.
			\label{8v}
		\end{split}
	\end{equation}
	Recall \eqref{2v}. Grouping \eqref{3v}, \eqref{3'v}, and \eqref{8v}, and applying
	the Sobolev Imbedding Theorem, we see that 
	\begin{equation}
		\begin{split}
			\frac{1}{2} \frac{d}{dt}
			\|v\|_{H^\sigma(\ci)}^2 \lesssim \|u+w\|_{H^\rho(\ci)}
			\|v\|_{H^\sigma(\ci)}^2.
			\label{9v}
		\end{split}
	\end{equation}
	By Gronwall's inequality, \eqref{9v} gives
	\begin{equation}
		\label{10lv}
		\begin{split}
			\|v\|_{H^\sigma(\ci)}
			& \lesssim e^{\int_0^t \|u+w\|_{H^{\rho}}}
			\|v_0\|_{H^\sigma(\ci)}, \qquad |t| \le T.
		\end{split}
	\end{equation}
	First, note that $v_0 = u_0 - w_0 = 0$; secondly, $\|u + w \|_{H^\rho}
	\le \|u + w \|_{H^s(\ci)} < \infty$ for $|t| \le T$ by
	the triangle inequality and \eqref{u_x-Linfty-Hs}. Hence, from
	\eqref{10lv} we obtain
	\begin{equation*}
		\begin{split}
			\|v\|_{H^\sigma(\ci)}
			& \lesssim \|v_0\|_{H^\sigma(\ci)}, \quad |t| \le T	
			\\
			& = 0.
		\end{split}
	\end{equation*}
	We conclude that solutions to the HR i.v.p. with initial data in
	$H^s(\ci)$ are unique for $s > 3/2$.  $\qquad
	\Box$
	%
	%
	%
	%
	\subsection{Continuous Dependence.}
	Let $\left\{ u_{0, n} \right\}_n \subset H^s(\ci)$ be a uniformly bounded
sequence converging to $u_0$ in $H^s(\ci)$.
Consider solutions $u $, $u^\ee$, $u^\ee_n$, and $u_n$ to the Cauchy-problem
\eqref{hyperelastic-rod-equation}-\eqref{init-cond}
with associated initial data $u_0$, $J_\ee u_0$,
$J_\ee u_{0,n}$, and $u_{0,n}$, respectively, where $J_\ee$ is defined as follows: Pick a function $\widehat{j}(\xi) \in \mathcal{S}(\rr)$ such that
	\begin{equation}
		\label{0u}
		\begin{split}
			& 0 \le \widehat{j}(\xi) \le 1,
			\\
			& \widehat{j}(\xi) = 1 \ \ \text{if} \ \ |\xi| \le 1.
		\end{split}
	\end{equation}
	Since $\sum_{-M}^M \widehat{j}(\ee \xi) e^{i \xi x}$ converges uniformly as $M \to
	\infty$, we can let
	\begin{equation}
		\begin{split}
			j_\ee (x) = \frac{1}{2 \pi}\sum_{\xi \in \zz}
			\widehat{j}(\ee \xi) e^{i \xi x}, \quad \ee > 0
			\label{parseval-def}
		\end{split}
	\end{equation}
	which is equivalent to stating that we can find $\left\{ j_\ee
	\right\} \subset \mathcal{S}(\ci)$ such that
	\begin{equation}
		\begin{split}
			\widehat{j_\ee} = \widehat{j }(\ee \xi), \quad \ee > 0.
			\label{widehat-def}
		\end{split}
	\end{equation}
	We then define $J_\ee$ to be the ``Friedrichs mollifier''
	\begin{equation}
		\label{0'u}
		\begin{split}
			J_\ee f(x) = j_\ee * f(x), \quad \ee>0.
		\end{split}
	\end{equation}
We remark that we have constructed the operator $J_\ee$ in this manner in
order that inequality \eqref{widehat-def} is satisfied; this will prove
crucial later on.
%
Applying
the triangle inequality, we have
\begin{equation*}
	\begin{split}
		\|u - u_n\|_{H^s(\ci)}
		& \le \|u - u^\ee\|_{H^s(\ci)}
		+ \|u^\ee - u^{\ee}_n \|_{H^s(\ci) }
		+  \|u^{\ee}_n - u_n \|_{H^s(\ci)}.
	\end{split}
\end{equation*}
Therefore, to prove continuous dependence, it will be enough to show 
\begin{align}
	& \lim_{\substack{n\to \infty \\ \ee \to 0}} \|u - u^\ee\|_{H^s(\ci)}
	=0,
	\label{enough_to_prove1}
	\\
	& \lim_{\substack{n\to \infty \\ \ee \to 0}} \|u^\ee - u^{\ee}_n
	\|_{H^s(\ci)} = 0,
	\label{enough_to_prove2}
	\\
	& \lim_{\substack{n\to \infty \\ \ee \to 0}}
	\|u^{\ee}_n - u_n \|_{H^s(\ci)} =0
	\label{enough_to_prove3}
\end{align}
where we define 
\medskip
	\begin{equation}
		\label{lim-not}
		\begin{split}
			\lim_{\substack{n\to \infty \\ \ee \to 0}} (\cdot) \doteq \lim_{\ee \to
			\infty} [\lim_{n \to \infty} (\cdot )].
		\end{split}
	\end{equation}
\subsection{ Proof of \eqref{enough_to_prove1}.}
		Consider two solutions $u $ and $u^\ee$ to the Cauchy-problem
	\eqref{hyperelastic-rod-equation}-\eqref{init-cond}
	with associated initial data $u_0$ and
	$J_\ee u_0$, respectively. Set $v= u -u^\ee $. Then $v$ solves the
	Cauchy-problem
	\begin{align}
		\label{4u}
		\p_t v 
		& =  - \gamma (v \p_x v + v \p_x u^\ee + u^\ee \p_x v)  
		\\
		& - D^{-2} \p_x \left\{ \left (\frac{3-\gamma}{2} \right )(v^2 +
		2u^\ee v) + \frac{\gamma}{2}\left[ (\p_x v)^2 + 2 \p_x u^\ee \p_x v \right]
		\right\}, \notag
		\\
		& v(0) = (I- J_\ee)u_0.
		\label{5u}
	\end{align}
	Applying the operator $D^s$ to both sides of \eqref{4u}, multiplying by
	$D^s v$ and integrating, we have
	\begin{equation}
		\begin{split}
			\frac{1}{2}\frac{d}{dt} \|v\|_{H^s(\ci)} = A + B
			\label{6u}
		\end{split}
	\end{equation}
	where
	\begin{equation}
		\begin{split}
			A
			& =  -\gamma \int_{\ci} D^s(v \p_x v) \cdot D^s v \
			dx
			- \frac{3- \gamma}{2} \int_\ci D^{s-2} \p_x (v^2) \cdot D^s v
			\ dx
			\\
			& - \frac{\gamma}{2}\int_\ci D^{s-2} \p_x (\p_x v)^2 \cdot D^s
			v \ dx
			\label{7u}
		\end{split}
	\end{equation}
	and
	\begin{equation}
		\begin{split}
			B 
			= & \ \overbrace{-\gamma \int_\ci D^s (v \p_x u^\ee ) \cdot D^s v \
			 dx}^{(i)} \ \overbrace{-\gamma \int_\ci D^s (u^\ee \p_x v) \cdot D^s v \
			 dx}^{(ii)}
			  \\
			  & \ \overbrace{- \ ( 3- \gamma) \int_\ci D^{s-2} \p_x (u^\ee v) \cdot D^s
			 v \ dx}^{(iii)}
			 \\
			 & \overbrace{-\gamma \int_\ci D^{s-2} \p_x
			(\p_x u^\ee \cdot \p_x v) \cdot D^s v \
			dx}^{(iv)}.
			\label{8u}
		\end{split}
	\end{equation}
	We now provide estimates for $A$ and $B$:
	\subsection{ A.} 
	Recalling the proof of \eqref{en-est-u}, with $u$ replaced by
	$v$ gives 
	\begin{equation}
		\begin{split}
			|A| \lesssim \|v\|_{H^s(\ci)}^3, \quad |t| \le T.
			\label{8'u}
		\end{split}
	\end{equation}
%
	\subsection{ B.} We now estimate in parts:
	%
	%
	%
		%
	%
\subsection{ Estimate of (\hyperref[8u]{i}).} 
We can rewrite
	\begin{equation}
		\begin{split}
			(i)
			= & -\gamma \int_\ci \left[ D^s(v \p_x u^\ee) - v D^s
			\p_x u^\ee \right] \cdot D^s v \ dx
			\\
			& -  \gamma \int_\ci v D^s \p_x u^\ee \cdot D^s v \ dx.
			\label{1wap'}
		\end{split}
	\end{equation}
	Estimating in parts, we have
	\begin{equation}
		\begin{split}
			& |-\gamma \int_\ci \left[ D^s(v \p_x u^\ee) - v D^s
			\p_x u^\ee \right] \cdot D^s v \ dx |
			\\
			& \le |\gamma| \int_\ci |\left[ D^s(v \p_x u^\ee ) - v D^s
			\p_x u^\ee \right] \cdot D^s v| \ dx
			\\
			& \le |\gamma| \cdot \|D^s (v \p_x u^\ee) - v D^s \p_x u^\ee
			\|_{L^2(\ci)} \|v\|_{H^s(\ci)}.
			\label{1wap}
		\end{split}
	\end{equation}
	Applying the Kato-Ponce estimate \eqref{KP-com-est} to \eqref{1wap}, we
	obtain
	\begin{equation*}
		\begin{split}
			& | -\gamma \int_\ci \left[ D^s(v \p_x u^\ee) - v D^s
			\p_x u^\ee \right] \cdot D^s v \ dx |
			\\
			& \le c_s |\gamma| \cdot ( \|D^s v \|_{L^2(\ci)} \|\p_x
			u^\ee\|_{L^\infty(\ci)} + \|\p_x v \|_{L^\infty(\ci)} \|D^{s-1}
			\p_x u^\ee \|_{L^2(\ci)}) \cdot \|v\|_{H^s(\ci)}
		\end{split}
	\end{equation*}
	which by the Sobolev Imbedding Theorem simplifies to
	\begin{equation}
		\begin{split}
			| -\gamma \int_\ci \left[ D^s(v \p_x u^\ee ) - v D^s
			\p_x u^\ee \right] \cdot D^s v \ dx |
			\lesssim \|u^\ee \|_{H^s(\ci)} \|v\|_{H^s(\ci)}^2.
			\label{2wap}
		\end{split}
	\end{equation}
	For the remaining piece of \eqref{1wap'}, we have
	\begin{equation*}
		\begin{split}
			| - \gamma \int_\ci v D^s \p_x u^\ee \cdot D^s v \ dx |
			& \le |\gamma| \int_\ci |v D^s \p_x u^\ee \cdot D^s v | \ dx
			\\
			& \le |\gamma| \cdot \|v\|_{L^\infty(\ci)} \|D^s \p_x u^\ee
			\|_{L^2(\ci)} \|D^s v\|_{L^2(\ci)}
		\end{split}
	\end{equation*}
	which by the Sobolev Imbedding Theorem gives 
	\begin{equation}
		\begin{split}
			| - \gamma \int_\ci u^\ee D^s \p_x v \cdot D^s v \ dx |
			\lesssim \|u^\ee \|_{H^{s+1}(\ci)} \|v\|_{H^{s-1}(\ci)}
			\|v\|_{H^s(\ci)}.
			\label{3wap}
		\end{split}
	\end{equation}
	Combining estimates \eqref{2wap} and \eqref{3wap} we conclude that
	\begin{equation}
		\begin{split}
			|(i)| \lesssim \|u^\ee \|_{H^s(\ci)} \|v\|_{H^s(\ci)}^2 + 
			\|u^\ee \|_{H^{s+1}(\ci)} \|v\|_{H^{s-1}(\ci)}
			\|v\|_{H^s(\ci)}.
			\label{4wap}
		\end{split}
	\end{equation}
%
\subsection{ Estimate of (\hyperref[8u]{ii}).} We can rewrite
	\begin{equation}
		\begin{split}
			(ii)
			= & -\gamma \int_\ci \left[ D^s(u^\ee \p_x v) - u^\ee D^s
			\p_x v \right] \cdot D^s v \ dx
			\\
			& -  \gamma \int_\ci u^\ee D^s \p_x v \cdot D^s v \ dx.
			\label{1wa'}
		\end{split}
	\end{equation}
	Estimating in parts, we have
	\begin{equation}
		\begin{split}
			& |-\gamma \int_\ci \left[ D^s(u^\ee \p_x v) - u^\ee D^s
			\p_x v \right] \cdot D^s v \ dx |
			\\
			& \le |\gamma| \int_\ci |\left[ D^s(u^\ee \p_x v) - u^\ee D^s
			\p_x v \right] \cdot D^s v| \ dx
			\\
			& \le |\gamma| \cdot \|D^s (u^\ee \p_x v) - u^\ee D^s \p_x v
			\|_{L^2(\ci)} \|v\|_{H^s(\ci)}.
			\label{1wa}
		\end{split}
	\end{equation}
	Applying the Kato-Ponce estimate \eqref{KP-com-est} to \eqref{1wa}, we
	obtain
	\begin{equation*}
		\begin{split}
			& | -\gamma \int_\ci \left[ D^s(u^\ee \p_x v) - u^\ee D^s
			\p_x v \right] \cdot D^s v \ dx |
			\\
			& \le c_s |\gamma| \cdot ( \|D^s u^\ee \|_{L^2(\ci)} \|\p_x
			v\|_{L^\infty(\ci)} + \|\p_x u^\ee \|_{L^\infty(\ci)} \|D^{s-1}
			\p_x v \|_{L^2(\ci)}) \cdot \|v\|_{H^s(\ci)}
		\end{split}
	\end{equation*}
	which by the Sobolev Imbedding Theorem simplifies to
	\begin{equation}
		\begin{split}
			| -\gamma \int_\ci \left[ D^s(u^\ee \p_x v) - u^\ee D^s
			\p_x v \right] \cdot D^s v \ dx |
			\lesssim \|u^\ee \|_{H^s(\ci)} \|v\|_{H^s(\ci)}^2.
			\label{2wa}
		\end{split}
	\end{equation}
	For the remaining piece of \eqref{1wa'}, we have
	\begin{equation*}
		\begin{split}
			| - \gamma \int_\ci u^\ee D^s \p_x v \cdot D^s v \ dx |
			& = \left | -\frac{\gamma}{2} \int_\ci u^\ee \p_x (D^s v)^2 \
			dx \right |
			\\
			& = \left | \frac{\gamma}{2} \int_\ci \p_x u^\ee (D^s v)^2 \ dx
			\right |
			\\
			& \le \left | \frac{\gamma}{2} \right | \int_\ci |\p_x u^\ee
			(D^s v)^2 | dx
			\\
			& \le \left | \frac{\gamma}{2} \right | \|\p_x u^\ee
			\|_{L^\infty(\ci)} \|v\|_{H^s(\ci)}^2
		\end{split}
	\end{equation*}
	and applying the Sobolev Imbedding Theorem gives
	\begin{equation}
		\begin{split}
			| - \gamma \int_\ci u^\ee D^s \p_x v \cdot D^s v \ dx |
			\lesssim \|u^\ee \|_{H^s(\ci)} \|v\|_{H^s(\ci)}^2.
			\label{3wa}
		\end{split}
	\end{equation}
	Combining estimates \eqref{2wa} and \eqref{3wa} we conclude that
	\begin{equation}
		\begin{split}
			|(ii)| \lesssim \|u^\ee \|_{H^s(\ci)} \|v\|_{H^s(\ci)}^2.
			\label{4wa}
		\end{split}
	\end{equation}
\subsection{ Estimate of (\hyperref[8u]{iii}).} We have
	\begin{equation}
		\begin{split}
			|(iii)|
			& \le |3-\gamma| \int_{\ci} |D^{s-2} \p_x (u^\ee v) \cdot D^s v
			\ | \ dx
			\\
			& \le |3- \gamma| \cdot  \|D^{s-2}\p_x (u^\ee v)
			\|_{L^2(\ci)} \cdot \|v\|_{H^s(\ci)}
			\\
			& \le |3- \gamma| \cdot  \| u^\ee v \|_{H^{s -1}(\ci)} \cdot \|v\|_{H^s(\ci)}
			\label{12u}
		\end{split}
	\end{equation}
	and applying the algebra property and the Sobolev Imbedding Theorem gives
	\begin{equation}
		\begin{split}
			|(iii)| & \lesssim \|u^\ee\|_{H^{s-1}(\ci)} \|v\|_{H^{s-1}(\ci)}
			\|v\|_{H^s(\ci)}
			\\
			& \lesssim \|u^\ee\|_{H^{s}(\ci)} \|v\|_{H^{s}(\ci)}^2.
			\label{13u}
		\end{split}
	\end{equation}
	%
	%
	%
	%
	\subsection{ Estimate of (\hyperref[8u]{iv}).} We have
	\begin{equation*}
		\begin{split}
			|(iv)|
			& \le |\gamma| \cdot \|D^{s-2} \p_x (\p_x u^\ee \cdot \p_x v)
			\|_{L^2(\ci)} \|D^s v\|_{L^2(\ci)}
			\\
			& \le |\gamma| \cdot \|\p_x u^\ee \cdot \p_x v \|_{H^{s-1}(\ci)}
			\|v\|_{H^s(\ci)}
		\end{split}
	\end{equation*}
	and applying the algebra property gives
	\begin{equation*}
		\begin{split}
			|(iv)|
			& \le |\gamma| \cdot \|\p_x u^\ee \|_{H^{s-1}(\ci)} \|\p_x v
			\|_{H^{s-1}(\ci)} \|v\|_{H^s(\ci)}
			\\
			& \lesssim \|u^\ee\|_{H^s(\ci)} \|v\|_{H^s(\ci)}^2.
		\end{split}
	\end{equation*}
	Hence, collecting our estimates for (\hyperref[8u]{i}),
	(\hyperref[8u]{ii}), (\hyperref[8u]{iii}), and (\hyperref[8u]{iv})
	yields
		\begin{equation}
		\begin{split}
			|B| 
			& \lesssim
			\|u^\ee\|_{H^s(\ci)}
			\|v\|_{H^s(\ci)}^2 + \|u^\ee\|_{H^{s+1}(\ci)}
			\|v\|_{H^{s-1}(\ci)} \|v\|_{H^s(\ci)}.
			\label{14u}
		\end{split}
	\end{equation}
	Combining estimates \eqref{8'u} and \eqref{14u} and recalling
	\eqref{6u}, we obtain
	\begin{equation}
		\begin{split}
			\frac{1}{2}\frac{d}{dt}\|v\|_{H^{s}(\ci)}^2
			& \le c_s(\|v\|_{H^s(\ci)}^3 + \|u^\ee\|_{H^s(\ci)}
			\|v\|_{H^s(\ci)}^2
			\\
			& + \|u^\ee\|_{H^{s+1}(\ci)}
			\|v\|_{H^{s-1}(\ci)} \|v\|_{H^s(\ci)})
			\label{15u}
		\end{split}
	\end{equation}
	where $c_s$ is a constant depending only on $s$.
	Note that the first two terms in the parentheses on the right hand side
	of \eqref{15u} will offer us little trouble;
	it is the third term that requires special care (due to the
	$\|u^\ee\|_{H^{s+1}(\ci)}$ factor, which becomes increasingly large as
	$\ee$ decreases). More precisely:
	%
	%
	%
	\begin{remark}
	\label{lem5r}
	For $r \ge s > 3/2$ and $0 < \ee <<1$, 
	\begin{equation}
		\begin{split}
			\|u^{\ee} (t) \|_{H^r(\ci)} \le C \, \ee^{s-r}
			\label{700r}
		\end{split}
	\end{equation}
	where $C = C(r, \|u_0\|_{H^s(\ci)})$.
\end{remark}	
\subsection{ Proof.} By part (iii) of Theorem
\ref{thm:HR_existence_continous_dependence}, proved in Section
\ref{existence}, we have
\begin{equation}
	\begin{split}
		\|u^\ee(t) \|_{H^r(\ci)}^2
		& \le C' \|u^\ee (0)\|_{H^r(\ci)}^2
		\\
		& = C' \|J_\ee u_0\|_{H^r(\ci)}^2
		\\
		& = C' \sum_{\xi \in \zz} |\widehat{j_\ee} (\xi) \widehat{u_0}(\xi)
		|^2 \cdot (1 + \xi^2)^r
		\label{0qr}
	\end{split}
\end{equation}
Recall how we chose the mollifier $J_\ee$; it will now play a fundamental role. Since
\eqref{widehat-def} holds by construction, \eqref{0qr} gives 
\begin{equation}
	\begin{split}
		\|u^\ee(t) \|_{H^r(\ci)}^2
		& = C' \sum_{\xi \in \zz} |\widehat{j }(\ee \xi)|^2 \cdot (1 +
		\xi^2)^{r-s} \cdot |\widehat{u_0}(\xi)|^2 \cdot (1 + \xi^2)^s
		\\
		& = C'|\widehat{u_0}(0)|^2 +
		C' \sum_{\xi \in \zz \setminus {0}} |\widehat{j }(\ee \xi)|^2 \cdot (1 +
		\xi^2)^{r-s} \cdot |\widehat{u_0}(\xi)|^2 \cdot (1 + \xi^2)^s.
		\label{1qr}
	\end{split}
\end{equation}
Assume $r \ge s$. Since $\widehat{j }(\xi) \in \mathcal{S}(\rr)$, 
\begin{equation}
	\label{schwartz}
	\begin{split}
		|\widehat{j }(\ee \xi)| \le c_r |\ee \xi |^{s-r}, \quad \xi \neq 0.
	\end{split}
\end{equation}
Applying \eqref{schwartz} to \eqref{1qr}, we obtain
\begin{equation}
	\label{calc_ue}
	\begin{split}
		\|u^\ee (t)\|_{H^r(\ci)}^2 
		& \le C' |\widehat{u_0}(0) |^2 + c_r \sum_{\xi \in \zz \setminus
		{0}} |\ee \xi |^{2(s-r)} \cdot (1 + \xi^2)^{r-s}
		|\widehat{u_0}(\xi) |^2 \cdot (1 + \xi^2)^s
		\\
		& \le C' |\widehat{u_0}(0) |^2 + 2^{r-s} c_r \ee^{2(s-r)}
		\sum_{\xi \in \zz \setminus {0}} |\widehat{u_0}(\xi)|^2 \cdot (1 +
		\xi^2)^s
		\\
		& \le C' \|u_0\|_{H^s(\ci)}^2 + 2^{r-s} c_r \ee^{2(s-r)}
		\|u_0\|_{H^s(\ci)}^2
		\\
		& = (C' + 2^{r-s} c_r \ee^{2(s-r)}) \cdot \|u_0\|^2_{H^s(\ci)}.
	\end{split}
\end{equation}
Assuming $0 < \ee <<1$, we conclude from \eqref{calc_ue} that 
\begin{equation*}
	\begin{split}
		\|u^\ee(t)\|_{H^s(\ci)} \le C \ee^{s-r}
	\end{split}
\end{equation*}
where $C = C(r, \|u_0\|_{H^s(\ci)})$. $\qquad \Box$
	We remark that, despite the blowup of $\|u^\ee \|_{H^{s+1}(\ci)}$
	as $\ee$ becomes small, our difficulties would have been
	amplified if we had originally taken $v=w-u$ for some arbitrary
	solution $w$ to the HR
	i.v.p with initial data $w_0 \in H^s(\ci)$, for then we would be dealing with
	\eqref{15u}, with $w$ substituted in for $u^\ee$. However, note that 
	$\|w\|_{H^{s+1}(\ci)}$ might not even be bounded, whereas $\|u^\ee
	\|_{H^{s+1}(\ci)}$ is always bounded, for any $\ee > 0$.
	%
	%
	In light of the blowup of $\|u^\ee \|_{H^{s+1}(\ci)}$,
	our strategy in tackling
	\eqref{15u} will be as follows. First, we will obtain an estimate for
	$\|v\|_{H^\sigma(\ci)}$ for suitably chosen $\sigma < s-1$. Then, we
	will use this estimate to interpolate between $\|v\|_{H^\sigma(\ci)}$
	and $\|v\|_{H^s(\ci)}$,
	yielding an estimate for $\|v\|_{H^{s-1}(\ci)}$ which will allow us to control the growth of
	$\|u^\ee\|_{H^{s+1}(\ci)}$. 
	%
	%
	%
	%
\begin{lemma} 
	\label{lem6r}
	For $\sigma$ such that $1/2 < \sigma < 1$ and $\sigma + 1 \le s$, we have
	\begin{equation}
	\begin{split}
		\|v\|_{H^{\sigma}(\ci)} \le C \cdot o(\ee^{s- \sigma }), \qquad |t| \le T
	\end{split}
\end{equation}
where $C=C(\|u_0\|_{H^s(\ci)})$.
\end{lemma}
%
%
%
\subsection{ Proof.}
Recall that $v$ solves the Cauchy-problem \eqref{4u}-\eqref{5u}.
Applying $D^\sigma$ to both sides of \eqref{4u}, multiplying by
$D^\sigma v$, and integrating, we obtain the
relation
\begin{equation*}
	\begin{split}
		\frac{1}{2}\frac{d}{dt}\|v(t)\|_{H^\sigma(\ci)}^2
		= & - \frac{\gamma}{2}\int_{\ci} D^\sigma
		\p_x \left[ \left( u + u^\ee \right)v
		\right]\cdot D^\sigma v \ dx
		\\
		& - \frac{3-\gamma}{2}\int_{\ci} D^{\sigma
		-2} \p_x \left[ \left( u + u^\ee
		\right)v \right] \cdot D^\sigma v \ dx
		\\
		& - \frac{\gamma}{2}\int_{\ci} D^{\sigma
		-2}
		\p_x \left[ \left( \p_x u + \p_x u^\ee
		\right)\cdot \p_x v \right] \cdot
		D^\sigma v \ dx.
	\end{split}
\end{equation*}
Repeating calculations \eqref{X}-\eqref{12}, with $E$ set to zero,
$u^{\omega,n}$ replaced by $u$, $u_{\omega,n}$ replaced by $u^\ee$, and
$\sigma$ and $\rho$ chosen such that
%
\begin{equation}
	\label{size_of_sigma}
	\begin{split}
	& 1/2 < \sigma < 1,
	\\
	& \sigma + 1 \le \rho \le s 
	\end{split}
\end{equation}
yields
 \begin{equation*}
	\begin{split}
		\frac{1}{2}\frac{d}{dt} \|v\|_{H^\sigma(\ci)}^2
		& \le
		c_s' (\|u^{\ee} + u\|_{H^{\rho}(\ci)} +
		\|\p_x(u^{\ee} + u) \|_{H^\sigma(\ci)})
		\cdot \|v\|_{H^\sigma(\ci)}^2.
	\end{split}
\end{equation*}
\medskip
By the Sobolev Imbedding Theorem, it follows that 
\begin{equation}
	\begin{split}
		\frac{1}{2}\frac{d}{dt} \|v\|_{H^{\sigma}(\ci)}^2
		& \le
		c_s \cdot \|u^{\ee}
		+ u\|_{H^{s}(\ci)}\cdot \|v\|_{H^{\sigma}(\ci)}^2.
		\label{10x}
	\end{split}
\end{equation}
Hence, applying the triangle inequality and
part (iii) of Theorem \ref{thm:HR_existence_continous_dependence} (proved
in Section \ref{existence}) to \eqref{10x} yields
%
\begin{equation}
	\begin{split}
		\label{11x}
		\frac{1}{2}\frac{d}{dt} \|v\|_{H^{\sigma}(\ci)}^2
		& \le
		c_s (\|u^{\ee}(0)\|_{H^{s}(\ci)}
		+ \|u(0)\|_{H^{s}(\ci)})\cdot \|v\|_{H^{\sigma}(\ci)}^2
		\\
		& = c_s (\|J_\ee u_0\|_{H^{s}(\ci)}
		+ \|u_0\|_{H^{s}(\ci)})\cdot \|v\|_{H^{\sigma}(\ci)}^2.
	\end{split}
\end{equation}
We now need the following:
\begin{proposition}
	\label{lem3r}
	For arbitrary $u \in L^2(\ci)$,
	\begin{equation}
		\begin{split}
			\|J_\ee u\|_{H^s(\ci)} \le \|u\|_{H^s(\ci)}.
			\label{lem100u}
		\end{split}
	\end{equation}
\end{proposition}
%
%
%
%
\subsection{ Proof.}
\begin{equation*}
	\begin{split}
		\|J_\ee u\|_{H^s(\ci)} 
		& = \left[\sum_{\xi \in \zz} |\widehat{j_\ee * u}(\xi) |^2
		(1+\xi^2)^s \right ]^{1/2}
		\\
		& = \left [ \sum_{\xi \in \zz} |\widehat{j_\ee} (\xi) \widehat{u}(\xi) |^2
		(1+ \xi^2)^s \right ]^{1/2}
		\\
		& = \left [ \sum_{\xi \in \zz} |\widehat{j}(\ee \xi)
		\widehat{u}(\xi)|^2 ( 1+ \xi^2)^s \right ]^{1/2}
	\end{split}
\end{equation*}
and since $|\widehat{j }(\ee \xi) | \le 1$ by \eqref{0u}, the result
follows. $\qquad \Box$
Using estimate \eqref{11x}, and applying Proposition \ref{lem3r}, 
we obtain the critical estimate 
\begin{equation}
	\begin{split}
		\label{12x}
		\frac{1}{2}\frac{d}{dt} \|v\|_{H^{\sigma}(\ci)}^2
		& \le
	 C \|v\|_{H^{\sigma}(\ci)}^2
\end{split}
\end{equation}
where $C = C(\|u_0\|_{H^s(\ci)})$. Differentiating the left hand side of
\eqref{12x} and simplifying, we obtain
\begin{equation}
	\begin{split}
		\frac{d}{dt}\|v\|_{H^{\sigma}(\ci)} \le C \|v\|_{H^{\sigma}(\ci)}.
		\label{100x}
	\end{split}
\end{equation}
Let $y(t) = \|v\|_{H^{\sigma}(\ci)}$. Then \eqref{100x} gives
\begin{equation*}
	\begin{split}
		\frac{1}{y(t)}\frac{dy}{dt} \le C.
	\end{split}
\end{equation*}
Hence,
\begin{equation*}
	\begin{split}
		\int_0^t \frac{1}{y(\tau)} \frac{dy}{d \tau}
		\le \int_0^t C \ d \tau, \qquad |t| \le T
	\end{split}
\end{equation*}
from which we obtain
\begin{equation}
	\begin{split}
		\ln |y(t) | - \ln |y(0)| \le C t.
		\label{101x}
	\end{split}
\end{equation}
Simplifying \eqref{101x}, we have
\begin{equation*}
	\begin{split}
		\ln \left |\frac{y(t)}{y(0)} \right | \le C t
	\end{split}
\end{equation*}
Since $y(t)$ is non-negative for all $t \in \rr$, this yields the estimate
\begin{equation*}
	\begin{split}
		y(t) \le y(0) e^{C t}, \qquad |t| \le T.
	\end{split}
\end{equation*}
Substituting back in $\|v\|_{H^{\sigma}(\ci)}$ for $y$, we get
\begin{equation}
	\label{conc-lemma}
	\begin{split}
		\|v\|_{H^{\sigma}(\ci)}
		& \le e^{C t}\|v(0)\|_{H^{\sigma}(\ci)}
		\\
		& = e^{C t}\|u(0) - u^\ee(0) \|_{H^{\sigma}(\ci)}
		\\
		& = e^{C t}\|u_0 - J_\ee u_0 \|_{H^{\sigma}(\ci)}.
	\end{split}
\end{equation}
We now require a critical operator norm estimate which will play an
important role later on.
\begin{proposition}
	\label{lem4r}
	For $r \le s$ and $\ee>0$
	\begin{equation}
	\label{0r}
		\begin{split}
			\|I - J_\ee\|_{L(H^s(\ci), H^r(\ci))} \le o(\ee^{s-r}).
		\end{split}
	\end{equation}
\end{proposition}
Using Proposition \ref{lem4r}, we conclude from estimate \eqref{conc-lemma} that
\begin{equation*}
	\begin{split}
		\|v\|_{H^{\sigma}(\ci)} \le C \cdot o(\ee^{s - \sigma}), \qquad |t|
		\le T
	\end{split}
\end{equation*}
where $C=C(\|u_0\|_{H^s(\ci)})$, completing the proof of Lemma \ref{lem6r}.
$\qquad \Box$
%
\subsection{ Proof of Proposition \ref{lem4r}.}
Pick an arbitrary $u \in H^s(\ci)$ such that $\|u\|_{H^s(\ci)} = 1$, and $r, s \in \rr$ such that $r \le s$. Using the fact that
$\widehat{j_\ee}(\xi) = \widehat{j}(\ee \xi)$ by construction, we have 
\begin{equation}
	\begin{split}
		\|u - J_\ee u\|_{H^r(\ci)}^2 
		& = \sum_{\xi \in \zz} |\widehat{u}(\xi) - \widehat{j_\ee * u}(\xi) |^2
		(1+\xi^2)^r
		\\
		& = \sum_{\xi \in \zz} |\widehat{u}(\xi) - \widehat{j_\ee}(\xi)
		\widehat{u}(\xi) |^2 (1+\xi^2)^r
		\\
		& = \sum_{\xi \in \zz} | [1- \widehat{j_\ee}(\xi] \cdot \widehat{u}(\xi) |^2
		(1+\xi^2)^r
		\\
		& = \sum_{\xi \in \zz} | [1- \widehat{j}(\ee \xi)] \cdot \widehat{u}(\xi) |^2
		(1+\xi^2)^r.
		\label{1r}
	\end{split}
\end{equation}
Assume $r \le s$. Then by construction (see \ref{0u}) we have
\begin{equation*}
	\begin{split}
		|1 - \widehat{j } (\xi) | \le |\xi|^{s-r}
	\end{split}
\end{equation*}
for all $\xi \in \rr$; hence
\begin{equation}
	\begin{split}
		|1 - \widehat{ j }(\ee \xi)| \le |\ee \xi |^{s-r}, \quad \forall
		\xi \in \rr, \ \ee > 0.
		\label{2r}
	\end{split}
\end{equation}
Applying \eqref{2r} to \eqref{1r} and recalling that $r \le s$, we obtain
\begin{equation}
	\label{2pr}
	\begin{split}
	\|u - J_\ee u\|_{H^r(\ci)}^2 
	& \le \sum_{\xi \in \zz}  |\ee \xi |^{2(s-r)}
	|\widehat{u}(\xi)|^2 (1 + \xi^2)^r
	\\
	& = \ee^{2(s-r)} \sum_{\xi \in \zz} |\widehat{u}(\xi)|^2  \cdot (\xi^2)^{s-r}
	(1 + \xi^2)^{r-s} (1 + \xi^2)^{s}
	\\
	& \le \ee^{2(s-r)}
	\sum_{\xi \in \zz} |\widehat{u}(\xi)|^2 (1 + \xi^2)^s
	\\
	& =  \ee^{2(s-r)}.
	\end{split}
\end{equation}
Furthermore,
\begin{equation*}
	\begin{split}
		& |[1- \widehat{j_\ee}(\xi)] \cdot \widehat{u}(\xi)|^2 (1 + \xi^2)^r \le
		|\widehat{u}(\xi)|^2 (1 + \xi^2)^r, \quad \ee > 0, \ \text{and}
		\\
		& \sum_{\xi \in \zz} |\widehat{u}(\xi)|^2 (1 + \xi^2)^r < \infty;
	\end{split}
\end{equation*}
therefore, by the dominated convergence theorem for series
\begin{equation}
	\label{o1}
	\begin{split}
		\lim_{\ee \to 0} \|u - J_\ee u \|_{H^r }^2 
		& = \lim_{\ee \to 0} \sum_{\xi \in \zz} |[1-\widehat{j_\ee}(\xi)]
		\widehat{u}(\xi) |^2 (1 + \xi^2)^r
		\\
		& = \lim_{\ee \to 0} \sum_{\xi \in \zz} |[1-\widehat{j}(\ee \xi)]
		\widehat{u}(\xi) |^2 (1 + \xi^2)^r
		\\
		& = \sum_{\xi \in \zz} \lim_{\ee \to 0} |[1-\widehat{j}(\ee \xi)]
		\widehat{u}(\xi) |^2 (1 + \xi^2)^r
		\\
		& = 0.
	\end{split}
\end{equation}
To complete the proof of Proposition \ref{lem4r}, we take note of the following interpolation result:
\begin{remark}
	\label{lem2r}
	For $\sigma < r \le s$ and arbitrary $u \in L^2(\ci)$,
	\begin{equation}
		\begin{split}
			\|u\|_{H^{r}(\ci)} \le
			\|u\|_{H^\sigma(\ci)}^{(r-s)/(\sigma -s)}
			\|u\|_{H^s(\ci)}^{1 - (r-s)/(\sigma -s)}.
			\label{16u}
		\end{split}
	\end{equation}
\end{remark}
%
%
%
%
\subsection{ Proof.} Assuming $u \in L^2(\ci)$ and $\sigma < r \le s$,
we rewrite and apply Holder's inequality:
\begin{equation*}
	\begin{split}
		&\|u\|_{H^{r}(\ci)}^2
		\\
		& = \sum_{\xi \in \zz} |\widehat{u}(\xi)|^2 (1 + \xi^2)^{r}
		\\
		& = \sum_{\xi \in \zz}
		\left [|\widehat{u}(\xi)|^2 (1 + \xi^2)^\sigma \right ]^{(r-s)/(\sigma -s)}
		\cdot \left [ |\widehat{u}(\xi )
		|^2 (1+ \xi^2)^s \right ] ^{1 - (r-s)/(\sigma -s)} 
		\\
		& \le \|\left[ |\widehat{u}(\xi)|^2 (1 + \xi^2)^\sigma
		\right]^{(r-s)/(\sigma -s)} \|_{l^{(\sigma -s)/(r-s)}(\zz)}
		\\
		& \cdot \|\left[ |\widehat{u}(\xi)|^2 (1 + \xi^2)^\sigma
		\right]^{1- (r-s)/(\sigma -s)} \|_{l^{1/[1 -(\sigma -s)/(r-s)]}(\zz)}
		\\
		& = \|v\|_{H^\sigma(\ci)}^{2(r-s)/(\sigma -s)}
		\|v\|_{H^s(\ci)}^{2[1 - (r-s)/(\sigma -s)]}
	\end{split}
\end{equation*}
from which the result follows. 
Assume without loss of generality that $s > 0$. Applying Remark \ref{lem2r}, and estimates \eqref{2pr} and \eqref{o1}, we
see that for $r>0$ 
\begin{equation*}
	\begin{split}
		\|u - J_\ee u \|_{H^r(\ci)}
		& \le \|u - J_\ee u
		\|_{L^2(\ci)}^{(s-r)/s} \|u - J_\ee u \|_{H^s(\ci)}^{1 -
		(s-r)/s}
		\\
		& = \left( \ee^{s} \right)^{(s-r)/s} \cdot o(1)
		\\
		& = o(\ee^{s-r})
	\end{split}
\end{equation*}
Similarly, for $r < 0$
\begin{equation*}
	\begin{split}
		\|u - J_\ee u \|_{H^r(\ci)}^2
		& \le \|u - J_\ee u
		\|_{H^\sigma(\ci)}^{(r-s)/(\sigma - s)} \|u - J_\ee u \|_{H^s(\ci)}^{1 -
		(r-s)/(\sigma -s)}
		\\
		& = \left( \ee^{s-\sigma} \right)^{(r-s)/(\sigma -s)} \cdot o(1)
		\\
		& = o(\ee^{s-r})
	\end{split}
\end{equation*}
Lastly, for the case $r=0$, we note that \eqref{o1} implies $\|u - J_\ee u
\|_{H^r(\ci)} \le o(1)$ for all $r \le s$. Hence, the proof of Proposition
\ref{lem4r} is complete.  $\quad \Box$
%
%
%
%
%
%
%
%
%
%
%
%
%
%
%
%
%
%
%
%
%
%\subsection{ Proof.} By \eqref{uniform_bound_for_u}, we have
%\begin{equation}
%	\begin{split}
%		\|u^\ee(t, \cdot \|_{H^r(\ci)}^2
%		& \le C \|u^\ee (0, \cdot)
%		\|_{H^r(\ci)}^2
%		\\
%		& = \|J_\ee u_0 \|_{H^r(\ci)}^2
%		\\
%		& = \sum_{\xi \in \zz} |\widehat{j_\ee}(\xi) \widehat{u_0}(\xi) |^2
%		\cdot (1 + \xi^2)^r
%		\\
%		& \le \sum_{\xi \in \zz} |[1-\widehat{ j_\ee}(\xi)] \widehat{u_0}(\xi) |^2
%		\cdot (1 + \xi^2)^r
%		\\
%		& + \sum_{\xi \in \zz} |\widehat{j_\ee}(\xi) \widehat{u_0}(\xi) |^2
%		\cdot (1 + \xi^2)^r.
%		\label{1q}
%	\end{split}
%\end{equation}
%Since $J_\ee u_0$ is smooth, by \eqref{uniform_bound_for_u} we have
%\begin{equation*}
%	\begin{split}
%		\sum_{\xi \in \zz} |\widehat{j_\ee}(\xi) \widehat{u_0}(\xi) |^2
%		\cdot (1 + \xi^2)^r
%		= \|J_\ee u_0\|_{H^r(\ci)} = C_r
%	\end{split}
%\end{equation*}
%for all $r \ge 3/2$.
%
%
%
%
%
%
%
%
We now return to analyzing
\eqref{15u}. Applying Remark \ref{lem5r}, Remark \ref{lem2r}, and Lemma
\ref{lem6r}, we have
\begin{equation}
	\begin{split}
		\label{200x}
		\|u^\ee \|_{H^{s+1}(\ci)} \|v \|_{H^{s-1}(\ci)} \|v\|_{H^s(\ci)}
		& \le C''' \ee^{-1} \cdot \|v\|_{H^\sigma(\ci)}^{1/(s-\sigma)}
		\|v\|_{H^s(\ci)}^{2 - 1/(s- \sigma)}
		\\
		& \le C'' \ee^{-1} \cdot o(\ee^{s- \sigma})^{1/(s-\sigma)}
		\|v\|_{H^s(\ci)}^{2- 1/(s-\sigma)}
		\\
		& \le C'' \cdot o(1) \cdot \|v\|_{H^s(\ci)}^{2- 1/(s-\sigma)}
	\end{split}
\end{equation}
where we stress that $C'' = C''(\|u_0\|_{H^s(\ci)})$ does not depend on $\ee$.
Hence, $\|v\|_{H^{s-1}(\ci)}$ has proved to be sufficient to control the
growth of $\|u^\ee \|_{H^{s+1}(\ci)}$. For the remaining terms of
\eqref{15u}, we leave $\|v\|_{H^s(\ci)}^3$ as is, and note that by Remark \ref{lem5r}
\begin{equation}
	\begin{split}
		\|u^\ee\|_{H^s(\ci)} \|v\|_{H^s(\ci)}^2 \le C'
		\cdot \|v\|_{H^s(\ci)}^2
		\label{u-ep-bound}
	\end{split}
\end{equation}
where $C' = C'(\|u_0\|_{H^s(\ci)})$. Hence, applying \eqref{u-ep-bound} and \eqref{200x} to \eqref{15u}, we obtain
\begin{equation}
	\begin{split}
		\frac{1}{2} \frac{d}{dt} \|v\|_{H^s(\ci)}^2 \le C (
		\|v\|_{H^s(\ci)}^3 + \|v\|_{H^s(\ci)}^2 + \ee ^{-1} o(\ee) \|v\|_{H^s(\ci)}^{2-
		1/(s- \sigma)}).
		\label{201x}
	\end{split}
\end{equation}
where $C=C(\|u_0\|_{H^s(\ci)}$ does not depend on $\ee$. We also remark 
that $\|v(t)\|_{H^s(\ci)}$ is uniformly bounded for all $\ee > 0$, since by
the triangle inequality, Proposition \ref{lem3r}, and part (iii) of Theorem
\ref{thm:HR_existence_continous_dependence} we have
\begin{equation*}
	\begin{split}
		\|v(t) \|_{H^s(\ci)}
		& = \|u - u^\ee \|_{H^s(\ci)}
		\\
		& \le \|u \|_{H^s(\ci)} + \|u^\ee \|_{H^s(\ci)}
		\\
		& \le 2( \|u_0\|_{H^s(\ci)} + \|J_\ee u_0\|_{H^s(\ci)})
		\\
		& \le 4 \|u_0\|_{H^s(\ci)}.
	\end{split}
\end{equation*}
Hence, \eqref{201x} gives
\begin{equation}
	\begin{split}
		\lim_{\ee \to 0} \frac{1}{2} \frac{d}{dt} \|v\|_{H^s(\ci)}^2 \le
		 \lim_{\ee \to 0} C (
		\|v\|_{H^s(\ci)}^3 + \|v\|_{H^s(\ci)}^2).
		\label{202x}
	\end{split}
\end{equation}
Differentiating the left hand side of
\eqref{202x} and simplifying, it follows that
\begin{equation*}
	\begin{split}
		\lim_{\ee \to 0}\frac{d}{dt} \|v\|_{H^s(\ci)} \le
		\lim_{\ee \to 0} C (\|v\|_{H^s(\ci)}^2 +
		\|v\|_{H^s(\ci)}).
	\end{split}
\end{equation*}
Letting $y = \|v\|_{H^s(\ci)}$ and rearranging, we obtain
\begin{equation*}
	\begin{split}
		\lim_{\ee \to 0} \ \frac{1}{y(y+1)} \frac{dy}{dt} \le C	
	\end{split}
\end{equation*}
which can be rewritten as
\begin{equation*}
	\begin{split}
		\lim_{\ee \to 0}
		\left( \frac{1}{y} - \frac{1}{y+1} \right)\frac{dy}{dt} \le C
	\end{split}
\end{equation*}
implying
\begin{equation}
	\label{est-int'}
	\begin{split}
		\lim_{\ee \to 0} \left [
\int_0^t \frac{1}{y} \frac{dy}{d \tau} \ d \tau
		- \int_0^t \frac{1}{y+1} \frac{dy}{d \tau} \ d \tau \right ]
		\le \int_0^t C \ d \tau, \quad |t| \le T.
	\end{split}
\end{equation}
Hence \eqref{est-int'} gives 
\begin{equation}
	\begin{split}
	\lim_{\ee \to 0} 
	\left [ \ln \left | \frac{y(t)}{y(0)}
	\cdot \frac{y(0) + 1}{y(t) + 1} \right | \right ] \le C t.
		\label{301''qx}
	\end{split}
\end{equation}
Exponentiating both sides of \eqref{301''qx}, and noting that $f(x) = e^x$
is a continuous function on $\rr$, we must have
\begin{equation*}
	\begin{split}
		\lim_{\ee \to 0}  \
		\left | \frac{y(t)}{y(0)} \cdot \frac{y(0) + 1}{y(t) + 1} \right | \le e^{C t}.
	\end{split}
\end{equation*}
Rearranging, and recalling that $y(t) = \|v(t)\|_{H^s(\ci)} \ge 0$, we obtain
\begin{equation*}
	\begin{split}
		\lim_{\ee \to 0} \frac{y(t)}{y(t) + 1}
		\le \lim_{\ee \to 0} \frac{e^{C t} \cdot y(0)}{y(0) + 1} \le
		\lim_{\ee \to 0} e^{C t} \cdot y(0).
	\end{split}
\end{equation*}
Substituting back in $\|v(t)\|_{H^s(\ci)}$ for $y(t)$ gives
\begin{equation}
	\begin{split}
		\lim_{\ee \to 0}	\frac{\|v(t)\|_{H^s(\ci)}}{\|v(t)\|_{H^s(\ci)} + 1}  \le
		\lim_{\ee \to 0} e^{C t} \cdot \|v(0)\|_{H^s(\ci)}.
		\label{303'qx}
	\end{split}
\end{equation}
Since 
\begin{equation*}
	\label{303''qx}
	\begin{split}
		\|v(0)\|_{H^s(\ci)} = \|u_0 - J_\ee u_0 \|_{H^s(\ci)} \le
		\|u_0\|_{H^s(\ci)} \cdot o(1)
	\end{split}
\end{equation*}
by Proposition \ref{lem4r}, we conclude from \eqref{303'qx} that
\begin{equation}
	\label{304qx}
	\begin{split}
		\lim_{\ee \to 0} \|v(t)\|_{H^s(\ci)} = \lim_{\ee \to 0}
		\|u^\ee(t) - u(t)\|_{H^s(\ci)}= 0, \qquad |t| \le T,
	\end{split}
\end{equation}
and since the family $\left\{ u^\ee - u \right\}_\ee$ does not depend on $n$,
the proof of \eqref{enough_to_prove1} is complete. 
%
%
%
%
%
%
%
%
\subsection{ Proof of \eqref{enough_to_prove2}.} 
Let $v = u^\ee_n - u^\ee$. Then $v$ solves the Cauchy problem
\begin{align}
		\label{4qu}
		\p_t v 
		& =  -\gamma (v \p_x v + v \p_x u^\ee + u^\ee \p_x v)  
		\\
		& - D^{-2} \p_x \left\{ \left (\frac{3-\gamma}{2} \right )(v^2 +
		2u^\ee v) + \frac{\gamma}{2}\left[ (\p_x v)^2 + 2 \p_x u^\ee \p_x v \right]
		\right\}, \notag
		\\
		& v(0) =J_\ee(u_{0,n} - u_0).
		\label{5qu}
	\end{align}
Applying the operator $D^s$ to both sides of \eqref{4qu}, multiplying by
	$D^s$ and integrating, we have
	\begin{equation}
		\begin{split}
			\frac{1}{2}\frac{d}{dt} \|v\|_{H^s(\ci)} = A + B
			\label{6qu}
		\end{split}
	\end{equation}
	where
	\begin{equation}
		\begin{split}
			A
			& =  -\gamma \int_{\ci} D^s(v \p_x v) \cdot D^s v \
			dx
			- \frac{3- \gamma}{2} \int_\ci D^{s-2} \p_x (v^2) \cdot D^s v
			\ dx
			\\
			& - \frac{\gamma}{2}\int_\ci D^{s-2} \p_x (\p_x v)^2 \cdot D^s
			v \ dx
			\label{7qu}
		\end{split}
	\end{equation}
	and
	\begin{equation}
		\begin{split}
			B 
			 = &  \overbrace{-\gamma \int_\ci D^s (u^\ee \p_x v) \cdot D^s v \
			 dx}^{(i)}
			 \ \overbrace{-\gamma \int_\ci D^s (v \p_x u^\ee ) \cdot D^s v \
			 dx}^{(ii)}
			 \\
			  & \overbrace{- \ ( 3- \gamma) \int_\ci D^{s-2} \p_x (u^\ee v) \cdot D^s
			 v \ dx}^{(iii)}
			 \\
			 & \overbrace{-\gamma \int_\ci D^{s-2} \p_x
			(\p_x u^\ee \cdot \p_x v) \cdot D^s v \
			dx}^{(iv)}.
			\label{8qu}
		\end{split}
	\end{equation}
	Estimating as in \eqref{8'u}-\eqref{14u}, we obtain
	\begin{equation}
		\begin{split}
			\frac{1}{2}\frac{d}{dt}\|v\|_{H^{s}(\ci)}^2
			& \le c_s(\|v\|_{H^s(\ci)}^3 + \|u^\ee\|_{H^s(\ci)}
			\|v\|_{H^s(\ci)}^2
			\\
			& + \|u^\ee\|_{H^{s+1}(\ci)}
			\|v\|_{H^{s-1}(\ci)} \|v\|_{H^s(\ci)}).
			\label{15qu}
		\end{split}
	\end{equation}
	We now aim to control the growth of $\|u^\ee\|_{H^{s+1}(\ci)}$ by
	$\|v\|_{H^{s-1}(\ci)}$. To do so, we will need an estimate for
	$\|v\|_{H^{s-1}(\ci)}$, which we will obtain through the following lemma:
%
%
%
%
\begin{lemma}
	\label{lem:left}
	For $\sigma$ such that $1/2 < \sigma < 1$ and $\sigma + 1 \le s$, we have
	\begin{equation}
	\label{lem6rq}
	\begin{split}
		\|v\|_{H^{\sigma}(\ci)} = 
		\|u^\ee_n - u^\ee\|_{H^\sigma(\ci)}
		\le C \cdot o(\ee^{s- \sigma }) + \|u_0 - u_{0,n} \|_{H^s(\ci)}, \qquad |t| \le T
	\end{split}
\end{equation}
where $C=C(\|u_0\|_{H^s(\ci)})$.
\end{lemma}
%
%
%
\subsection{ Proof.}
Repeating calculations \eqref{X}-\eqref{12}, with $E$ set to zero, $u^{\omega,n}$
replaced by $u^\ee_n$, $u_{\omega,n}$ replaced by $u^\ee$, and $\sigma$ and $\rho$ chosen such that
\begin{equation}
	\label{size_of_sigma'}
	\begin{split}
	& 1/2 < \sigma < 1,
	\\
	& \sigma + 1 \le \rho \le s 
	\end{split}
\end{equation}
yields
 \begin{equation*}
	\begin{split}
		\frac{1}{2}\frac{d}{dt} \|v\|_{H^\sigma(\ci)}^2
		& \le
		C'' (\|u^{\ee}_n + u^\ee \|_{H^{\rho}(\ci)} +
		\|\p_x(u^{\ee}_n + u^\ee) \|_{H^\sigma(\ci)})
		\cdot \|v\|_{H^\sigma(\ci)}^2.
	\end{split}
\end{equation*}
\medskip
It follows that 
\begin{equation}
	\begin{split}
		\frac{1}{2}\frac{d}{dt} \|v\|_{H^{\sigma}(\ci)}^2
		& \le
		C'' \cdot \|u^{\ee}_n
		+ u^\ee\|_{H^{s}(\ci)}\cdot \|v\|_{H^{\sigma}(\ci)}^2.
		\label{10qx}
	\end{split}
\end{equation}
Applying the triangle inequality and
part (iii) of Theorem \ref{thm:HR_existence_continous_dependence} (proved in
Section \ref{existence})
to \eqref{10qx} yields
%
\begin{equation}
	\begin{split}
		\label{11qx}
		\frac{1}{2}\frac{d}{dt} \|v\|_{H^{\sigma}(\ci)}^2
		& \le
		C' (\|u^{\ee}_n(0)\|_{H^{s}(\ci)}
		+ \|u^\ee(0)\|_{H^{s}(\ci)})\cdot \|v\|_{H^{\sigma}(\ci)}^2
		\\
		& = C' (\|J_\ee u_{0,n}\|_{H^{s}(\ci)}
		+ \|J_\ee u_0\|_{H^{s}(\ci)})\cdot \|v\|_{H^{\sigma}(\ci)}^2.
	\end{split}
\end{equation}
Note that the family $\left\{ u_{0,n} \right\}_n$ is uniformly bounded in
$H^s(\ci)$. Hence, applying Lemma \ref{lem3r} to \eqref{11qx} we obtain the critical estimate 
\begin{equation}
	\begin{split}
		\label{12qx}
		\frac{1}{2}\frac{d}{dt} \|v\|_{H^{\sigma}(\ci)}^2
		& \le
	C \|v\|_{H^{\sigma}(\ci)}^2
\end{split}
\end{equation}
with $C = C(\|u_0\|_{H^s(\ci)}, \ R)$, where
\begin{equation}
	\label{r-def}
	R = \inf \left\{ R' \in \rr:\ \{u_{0,n}\} \subset B_{H^s(\ci)}(R',0)
	\right\}.
\end{equation}
Differentiating the left hand side of \eqref{12qx} and simplifying, we
obtain
\begin{equation}
	\begin{split}
		\frac{d}{dt}\|v\|_{H^{\sigma}(\ci)} \le C \|v\|_{H^{\sigma}(\ci)}.
		\label{100qx}
	\end{split}
\end{equation}
Let $y(t) = \|v\|_{H^{\sigma}(\ci)}$. Then \eqref{100qx} gives
\begin{equation*}
	\begin{split}
		\frac{1}{y(t)}\frac{dy}{dt} \le C.
	\end{split}
\end{equation*}
Hence,
\begin{equation*}
	\begin{split}
		\int_0^t \frac{1}{y(\tau)} \frac{dy}{d \tau}
		\le \int_0^t C \ d \tau, \qquad |t| \le T
	\end{split}
\end{equation*}
from which we obtain
\begin{equation}
	\begin{split}
		\ln |y(t) | - \ln |y(0)| \le C t.
		\label{101qx}
	\end{split}
\end{equation}
Simplifying \eqref{101qx}, we have
\begin{equation*}
	\begin{split}
		\ln \left |\frac{y(t)}{y(0)} \right | \le C t
	\end{split}
\end{equation*}
which yields the estimate
\begin{equation*}
	\begin{split}
		y(t) \le y(0) e^{C t}, \qquad |t| \le T.
	\end{split}
\end{equation*}
Substituting back in $\|v\|_{H^{\sigma}(\ci)}$ for $y$, we get
\begin{equation*}
	\begin{split}
		\|v\|_{H^{\sigma}(\ci)}
		& \le e^{C t}\|v(0)\|_{H^{\sigma}(\ci)}
		\\
		& = e^{C t}\|u^\ee(0) - u^\ee_n(0) \|_{H^{\sigma}(\ci)}.
	\end{split}
\end{equation*}
To conclude the proof, we apply the following:
\begin{proposition}
		\label{lem11r}
	For $r \le s$,
	\begin{equation}
		\begin{split}
			\|u^\ee(0) - u_n^\ee (0) \|_{H^r(\ci)} \le C
			\cdot o(\ee^{s-r}) + \|u_0 - u_{0,n} \|_{H^s(\ci)}
			\label{3w}
		\end{split}
	\end{equation}
	where $C=C(\|u_0\|_{H^s(\ci)})$ does not depend on $n$.
\end{proposition}
%
%
Recalling \eqref{r-def}, we deduce by Proposition \ref{lem11r}
\begin{equation*}
	\begin{split}
		\|v\|_{H^{\sigma}(\ci)} \le C \cdot o(\ee^{s - \sigma}) + \|u_0 -
		u_{0,n} \|_{H^s(\ci)} \qquad |t| \le T
	\end{split}
\end{equation*}
where $C=C(\|u_0\|_{H^s(\ci)}), \ R)$ does not depend
on $n$, completing the proof of Lemma \ref{lem:left}. $\qquad \Box$
%
%
\subsection{ Proof of Proposition \ref{lem11r}.} We write
\medskip
\begin{equation}
	\begin{split}
		\|u^\ee(0) - u_n^\ee (0) \|_{H^r(\ci)} 
		& = \|J_\ee u_0 - J_\ee u_{0,n} \|_{H^r(\ci)}
		\\
		& \le \|J_\ee u_0 - u_0 \|_{H^r(\ci)} + \| u_0 - u_{0,n}
		\|_{H^r(\ci)}
		\\
		& + \|u_{0,n} - J_\ee u_{0,n} \|_{H^r(\ci)}
		\\
		& \le \|I - J_\ee\|_{L(H^s(\ci), H^r(\ci))} \|u_0\|_{H^s(\ci)}
		\\
		& +
		\|u_0 - u_{0,n} \|_{H^r(\ci)} + 
		\|I - J_\ee\|_{L(H^s(\ci), H^r(\ci))} \|u_{0,n}\|_{H^s(\ci)}.
		\label{4w}
	\end{split}
\end{equation}
Applying Proposition \ref{lem4r} to \eqref{4w}, and recalling that the family
$\left\{ u_{0,n} \right\}_n$ belongs to a bounded subset of
$H^s(\ci)$, we have
\medskip
\begin{equation}
	\label{finito}
	\begin{split}
		\|u^\ee(0) - u_n^\ee (0) \|_{H^r(\ci)} 
		& \le
		C' \cdot o(\ee^{s-r}) \cdot \|u_0\|_{H^s(\ci)}
		 \\
		 & + \|u_0 - u_{0,n} \|_{H^r(\ci)} + C' \cdot o (\ee^{s-r}) \cdot
		 \|u_{0,n}\|_{H^s(\ci)}
		 \\
		 & \le
		 C' \cdot o(\ee^{s-r}) \cdot \|u_0\|_{H^s(\ci)}
		 \\
		 & + \|u_0 - u_{0,n} \|_{H^s(\ci)} + C' \cdot o (\ee^{s-r}) \cdot
		 R
	\end{split}
\end{equation}
where $R$ is defined as in \ref{r-def}. The result follows immediately from
\eqref{finito}. $\qquad \Box$
We are now prepared to interpolate. Recall \eqref{15qu}. Applying Remark \ref{lem5r}, Remark \ref{lem2r}, and
Proposition \ref{lem11r} gives
\begin{equation*}
	\begin{split}
		& \|u^\ee \|_{H^{s+1}(\ci)} \|v\|_{H^{s-1}(\ci)} \|v\|_{H^s
		(\ci)}
		\\
		&\le C' \ee^{-1} \cdot \|v\|_{H^\sigma(\ci)}^{1/(s-\sigma)}
		\|v\|_{H^s(\ci)}^{2 - 1/(s- \sigma)}
		\\
		& \le C' \ee^{-1} \cdot \Big [C \cdot o(\ee^{s- \sigma}) + \|u_0 -
		u_{0,n}\|_{H^s(\ci)} \Big ]^{1/(s-\sigma)}
		\cdot \|v\|_{H^s(\ci)}^{2- 1/(s-\sigma)}
	\end{split}
\end{equation*}
from which we obtain
\begin{equation}
	\begin{split}
		\label{200qx}
		\|u^\ee\|_{H^{s+1}(\ci)} \|v\|_{H^{s-1}(\ci)} \|v \|_{H^s(\ci)}
		& \lesssim  o(1) + \ee^{-1}
		\|u_0-u_{0,n}\|_{H^s(\ci)}^{1/(s-\sigma)}\|v\|_{H^s(\ci)}^{2- 1/(s-\sigma)}.
	\end{split}
\end{equation}
We wish to control the growth of the second term of the
right hand side of \eqref{200qx}.
First, note that the triangle inequality, part (iii) of Theorem
\ref{thm:HR_existence_continous_dependence} and Proposition \ref{lem3r} imply
\begin{equation}
	\begin{split}
		\|v\|_{H^s(\ci)} & = \|u^\ee_n - u^\ee \|_{H^s(\ci)} 
		\\
		& \le \|u^\ee_n\|_{H^s(\ci)} + \|u^\ee \|_{H^s(\ci)}  
		\\
		& \le 2\left[  \|J_\ee u_{0,n}\|_{H^s(\ci)} + \|J_\ee u_0 \|_{H^s(\ci)} 
		 \right]
		\\
		& \le 2 \left[ \|u_{0,n} \|_{H^s(\ci)} + \|u_0 \|_{H^s(\ci)} 
		\right], \qquad |t| \le T
		\label{growth_v}
	\end{split}
\end{equation}
and since $\{u_{0,n}\}_n$ belongs to a bounded subset of
$H^s(\ci)$, we see from \eqref{growth_v} that $\|v \|_{H^s(\ci)}$ is
uniformly bounded in $n$ \emph{and} $\ee$.  Secondly, since $\|u_0 -
u_{0,n} \|_{H^s(\ci)} \to 0$ uniformly in $n$, then for any given $\ee$ we
can chose a family $\{N_j\} $ such that
\begin{equation}
	\begin{split}
		\|u_0 - u_{0,n} \|_{H^s(\ci)} \lesssim
		\frac{\ee^{(s-\sigma)}}{2^{j(s -\sigma)}}, \quad n >
		N_j.
		\label{uniform_n}
	\end{split}
\end{equation}
Thirdly, by Remark \ref{lem5r}, we have 
\begin{equation}
	\label{u-ee-bound}
	\|u^\ee \|_{H^s(\ci)} \le C(\|u_0\|_{H^s(\ci)}), \quad \forall \ee > 0.
\end{equation}
Applying \eqref{200qx} to \eqref{15qu} in light of 
\eqref{growth_v}, \eqref{uniform_n}, and \eqref{u-ee-bound}, we obtain
\begin{equation*}
		\begin{split}
			\lim_{n \to \infty }
			\frac{1}{2}\frac{d}{dt}\|v\|_{H^{s}(\ci)}^2
			& \le
			C \lim_{n \to \infty} \Big [\|v\|_{H^s(\ci)}^3 +
			\|v\|_{H^s(\ci)}^2 + o(1)\Big ]
		\end{split}
	\end{equation*}
	for every $\ee > 0$, where $C = C(\|u_0\|_{H^s(\ci)}, \ R)$ with
	$R$ defined as in \eqref{r-def}; hence we have
\begin{equation}
		\begin{split}
			\lim_{\substack{n \to \infty \\ \ee \to 0} }
			\frac{1}{2}\frac{d}{dt}\|v\|_{H^{s}(\ci)}^2
			& \le C
			\lim_{\substack{n \to \infty \\ \ee \to 0}}
			\Big [\|v\|_{H^s(\ci)}^3 + 
			\|v\|_{H^s(\ci)}^2 \Big ].
			\label{15qx}
		\end{split}
	\end{equation}
	We differentiate the left hand side of \eqref{15qx} and obtain
\begin{equation*}
	\begin{split}
		\lim_{\substack{n \to \infty \\ \ee \to 0}}\frac{d}{dt}
		\|v\|_{H^s(\ci)} \le C
		\lim_{\substack{n \to \infty \\ \ee \to 0}} \left [\|v\|_{H^s(\ci)}^2 +
		\|v\|_{H^s(\ci)} \right ].
	\end{split}
\end{equation*}
Letting $y = \|v\|_{H^s(\ci)}$ and rearranging gives
\begin{equation*}
	\begin{split}
		\lim_{\substack{n \to \infty \\ \ee \to 0} } \ \frac{1}{y(y+1)} \frac{dy}{dt}
		\le	C
	\end{split}
\end{equation*}
which can be rewritten as
\begin{equation*}
	\begin{split}
		\lim_{\substack{n \to \infty \\ \ee \to 0} }
		\left( \frac{1}{y} - \frac{1}{y+1} \right)\frac{dy}{dt} \le C 
	\end{split}
\end{equation*}
implying
\begin{equation}
	\label{est-int}
	\begin{split}
		\lim_{\substack{n \to \infty \\ \ee \to 0} } \left [
\int_0^t \frac{1}{y} \frac{dy}{d \tau} \ d \tau
		- \int_0^t \frac{1}{y+1} \frac{dy}{d \tau} \ d \tau \right ]
		\le \int_0^t C \ d \tau, \quad |t| \le T.
	\end{split}
\end{equation}
Recalling that $y(t) = \|v(t)\|_{H^s(\ci)} > 0$, \eqref{est-int} gives 
\begin{equation}
	\begin{split}
	\lim_{\substack{n \to \infty \\ \ee \to 0} }
	\left [ \ln \left ( \frac{y(t)}{y(0)}
	\cdot \frac{y(0) + 1}{y(t) + 1} \right ) \right ] \le C t.
		\label{301'qx}
	\end{split}
\end{equation}
Exponentiating both sides of \eqref{301'qx}, and noting that $f(x) = e^x$
is a continuous function on $\rr$, we must have
\begin{equation*}
	\begin{split}
		\lim_{\substack{n \to \infty \\ \ee \to 0} } \
		\frac{y(t)}{y(0)} \cdot \frac{y(0) + 1}{y(t) + 1} \le e^{C t}.
	\end{split}
\end{equation*}
Rearranging, we obtain
\begin{equation*}
	\begin{split}
		\lim_{\substack{n \to \infty \\ \ee \to 0}} \frac{y(t)}{y(t) + 1}
		\le \lim_{\substack{n \to \infty \\ \ee \to 0}} \frac{e^{C t} \cdot y(0)}{y(0) + 1} \le
		\lim_{\substack{n \to \infty \\ \ee \to 0}} e^{C t} \cdot y(0).
	\end{split}
\end{equation*}
Substituting back in $\|v(t)\|_{H^s(\ci)}$ for $y(t)$ gives
\begin{equation}
	\begin{split}
		\lim_{\substack{n \to \infty \\ \ee \to 0}}	\frac{\|v(t)\|_{H^s(\ci)}}{\|v(t)\|_{H^s(\ci)} + 1}  \le
		\lim_{\substack{n \to \infty \\ \ee \to 0}} e^{C t} \cdot \|v(0)\|_{H^s(\ci)}.
		\label{303qx}
	\end{split}
\end{equation}
Since by Proposition \ref{lem3r} 
\begin{equation*}
	\begin{split}
	\lim_{\substack{n \to \infty \\ \ee \to 0} }
	\|v(0)\|_{H^s(\ci)}
	& = \lim_{\substack{n \to \infty \\ \ee \to 0} }
	\|J_\ee u_{0,n} - J_\ee u_0 \|_{H^s(\ci)} 
	\\
	& \le \lim_{n \to \infty } \|u_{0,n} - u_0 \|_{H^s(\ci)}
	\\
	& = 0
	\end{split}
\end{equation*}
we deduce from \eqref{303qx} that
\begin{equation*}
	\begin{split}
		\lim_{\substack{n \to \infty \\ \ee \to 0}} \|v(t)\|_{H^s(\ci)} = 0, \qquad |t| \le T
	\end{split}
\end{equation*}
completing the proof of \eqref{enough_to_prove2}. $\quad \Box$
%
%
%
\subsection{ Proof of \eqref{enough_to_prove3}.} 
Let $v = u_n - u^\ee_n$. Then $v$ solves the Cauchy problem
\begin{align}
		\label{a4qu}
		\p_t v 
		& =  -\gamma (v \p_x v + v \p_x u^\ee_n + u^\ee_n \p_x v)  
		\\
		& - D^{-2} \p_x \left\{ \left (\frac{3-\gamma}{2} \right )(v^2 +
		2u^\ee_n v) + \frac{\gamma}{2}\left[ (\p_x v)^2 + 2 \p_x u^\ee_n \p_x v \right]
		\right\}, \notag
		\\
		& v(0) = (I- J_\ee)u_{0,n}.
		\label{a5qu}
	\end{align}
Applying the operator $D^s$ to both sides of \eqref{a4qu}, multiplying by
	$D^s$ and integrating, we have
	\begin{equation}
		\begin{split}
			\frac{1}{2}\frac{d}{dt} \|v\|_{H^s(\ci)} = A + B
			\label{a6qu}
		\end{split}
	\end{equation}
	where
	\begin{equation}
		\begin{split}
			A
			& =  -\gamma \int_{\ci} D^s(v \p_x v) \cdot D^s v \
			dx
			- \frac{3- \gamma}{2} \int_\ci D^{s-2} \p_x (v^2) \cdot D^s v
			\ dx
			\\
			& - \frac{\gamma}{2}\int_\ci D^{s-2} \p_x (\p_x v)^2 \cdot D^s
			v \ dx
			\label{a7qu}
		\end{split}
	\end{equation}
	and
	\begin{equation}
		\begin{split}
			B 
			 = &  \overbrace{-\gamma \int_\ci D^s (u^\ee_n \p_x v) \cdot D^s v \
			 dx}^{(i)}
			 \ \overbrace{-\gamma \int_\ci D^s (v \p_x u^\ee_n ) \cdot D^s v \
			 dx}^{(ii)}
			 \\
			  & \overbrace{- \ ( 3- \gamma) \int_\ci D^{s-2} \p_x (u^\ee_n v) \cdot D^s
			 v \ dx}^{(iii)}
			 \\
			 & \overbrace{-\gamma \int_\ci D^{s-2} \p_x
			(\p_x u^\ee_n \cdot \p_x v) \cdot D^s v \
			dx}^{(iv)}.
			\label{a8qu}
		\end{split}
	\end{equation}
	Estimating as in \eqref{8'u}-\eqref{14u}, we obtain
	\begin{equation}
		\begin{split}
			\frac{1}{2}\frac{d}{dt}\|v\|_{H^{s}(\ci)}^2
			& \le c_s(\|v\|_{H^s(\ci)}^3 + \|u^\ee_n\|_{H^s(\ci)}
			\|v\|_{H^s(\ci)}^2
			\\
			& + \|u^\ee_n\|_{H^{s+1}(\ci)}
			\|v\|_{H^{s-1}(\ci)} \|v\|_{H^s(\ci)}).
			\label{a15qu}
		\end{split}
	\end{equation}
	Note that the first two terms in parentheses on the right hand side
	of \eqref{a15qu} will offer us little trouble;
	it is the third term that requires special care (due to the
	$\|u^\ee_n\|_{H^{s+1}(\ci)}$ factor, which becomes increasingly large as
	$\ee$ decreases). More precisely:
	%
	%
	%
	\begin{remark}
	\label{lem5r'}
	For $r \ge s > 3/2$ and $0 < \ee <<1$ 
	\begin{equation}
		\begin{split}
			\|u^\ee_n (t, \cdot) \|_{H^r(\ci)} \le C \, \ee^{s-r}
			\label{700r'}
		\end{split}
	\end{equation}
	for all $n \in \mathbb{N}$, with $C = C(r, R)$, where $R$ is defined as
	in \eqref{r-def}.
\end{remark}
\subsection{ Proof.} By part (iii) of Theorem
\ref{thm:HR_existence_continous_dependence}, proved in Section
\ref{existence}, we have
\begin{equation}
	\begin{split}
		\|u^\ee_n \|_{H^r(\ci)}^2
		& \le C' \|u^\ee_n (0)\|_{H^r(\ci)}^2
		\\
		& = C' \|J_\ee u_{0,n}\|_{H^r(\ci)}^2
		\\
		& = C' \sum_{\xi \in \zz} |\widehat{j_\ee} (\xi) \widehat{u_{0,n}}(\xi)
		|^2 \cdot (1 + \xi^2)^r
		\\
		& = C' \sum_{\xi \in \zz} |\widehat{j }(\ee \xi)|^2 \cdot (1 +
		\xi^2)^{r-s} \cdot |\widehat{u_{0,n}}(\xi)|^2 \cdot (1 + \xi^2)^s
		\\
		& = C'|\widehat{u_{0,n}}(0)|^2 +
		C' \sum_{\xi \in \zz \setminus {0}} |\widehat{j }(\ee \xi)|^2 \cdot (1 +
		\xi^2)^{r-s} \cdot |\widehat{u_{0,n}}(\xi)|^2 \cdot (1 + \xi^2)^s.
		\label{1qr'}
	\end{split}
\end{equation}
Assume $r \ge s$. Since $\widehat{j }(\xi) \in \mathcal{S}(\rr)$, 
\begin{equation}
	\label{schwartz'}
	\begin{split}
		|\widehat{j }(\ee \xi)| \le c_r |\ee \xi |^{s-r}, \quad \xi \neq 0.
	\end{split}
\end{equation}
Applying \eqref{schwartz'} to \eqref{1qr'}, we obtain
\begin{equation}
	\label{calc_ue'}
	\begin{split}
		\|u^\ee_n \|_{H^r(\ci)}^2 
		& \le C' |\widehat{u_{0,n}}(0) |^2 + c_r \sum_{\xi \in \zz \setminus
		{0}} |\ee \xi |^{2(s-r)} \cdot (1 + \xi^2)^{r-s}
		|\widehat{u_{0,n}}(\xi) |^2 \cdot (1 + \xi^2)^s
		\\
		& \le C' |\widehat{u_{0,n}}(0) |^2 + 2^{r-s} c_r \ee^{2(s-r)}
		\sum_{\xi \in \zz \setminus {0}} |\widehat{u_{0,n}}(\xi)|^2 \cdot (1 +
		\xi^2)^s
		\\
		& \le C' \|u_{0,n}\|_{H^s(\ci)}^2 + 2^{r-s} c_r \ee^{2(s-r)}
		\|u_{0,n}\|_{H^s(\ci)}^2
		\\
		& = (C' + 2^{r-s} c_r \ee^{2(s-r)}) \cdot \|u_{0,n}\|^2_{H^s(\ci)}.
	\end{split}
\end{equation}
Assuming $0 < \ee <<1$, and noting that
\begin{equation*}
	\begin{split}
		\|u_{0,n}\|_{H^s(\ci)} \le R, \quad \forall n \in \mathbb{N}
	\end{split}
\end{equation*}
where $R$ is defined as in \eqref{r-def},
we conclude from \eqref{calc_ue'} that 
\begin{equation*}
	\begin{split}
		\|u^\ee_n\|_{H^s(\ci)} \le C \ee^{s-r}
	\end{split}
\end{equation*}
where $C = C(r, R)$. $\qquad \Box$
%
In light of Remark \ref{lem5r'}, we now aim to control the growth of
$\|u^\ee_n\|_{H^{s+1}(\ci)}$ by $\|v\|_{H^{s-1}(\ci)}$. As before, we will
first obtain an estimate for $\|v\|_{H^\sigma(\ci)}$ for suitably chosen
$\sigma < s-1$. Then, we will use this estimate to interpolate between
$\|v\|_{H^\sigma(\ci)}$ and $\|v\|_{H^s(\ci)}$, yielding an estimate for
$\|v\|_{H^{s-1}(\ci)}$ which will allow us to control the growth of
$\|u^\ee_n\|_{H^{s+1}(\ci)}$. 
%
%
%
%
\begin{proposition}
	\label{prop:180}
If $\sigma$ is chosen appropriately in the range $1/2 < \sigma < 1$ and
$\sigma + 1 < s$, then for all $n \in \mathbb{N}$ 
	\begin{equation}
	\label{alem6rq}
	\begin{split}
		\|v\|_{H^{\sigma}(\ci)} = 
		\|u_n - u^\ee_n\|_{H^\sigma(\ci)}
		\le C \cdot o(\ee^{s- \sigma }), \qquad |t| \le T
	\end{split}
\end{equation}
with $C = C(R)$, where $R$ is defined as in \eqref{r-def}.
\end{proposition}
%
%
%
\subsection{ Proof.}
Recall that $v$ solves the Cauchy-problem \eqref{a4qu}-\eqref{a5qu}.
Applying $D^\sigma$ to both sides of \eqref{a4qu}, multiplying by
$D^\sigma v$, and integrating, we obtain the
relation
\begin{equation*}
	\begin{split}
		\frac{1}{2}\frac{d}{dt}\|v(t)\|_{H^\sigma(\ci)}^2
		= & - \frac{\gamma}{2}\int_{\ci} D^\sigma
		\p_x \left[ \left( u_n + u^\ee_n \right)v
		\right]\cdot D^\sigma v \ dx
		\\
		& - \frac{3-\gamma}{2}\int_{\ci} D^{\sigma
		-2} \p_x \left[ \left( u_n + u^\ee_n
		\right)v \right] \cdot D^\sigma v \ dx
		\\
		& - \frac{\gamma}{2}\int_{\ci} D^{\sigma
		-2}
		\p_x \left[ \left( \p_x u_n + \p_x u^\ee_n
		\right)\cdot \p_x v \right] \cdot
		D^\sigma v \ dx.
	\end{split}
\end{equation*}
Repeating calculations \eqref{X}-\eqref{12}, with $E$ set to zero,
$u^{\omega,n}$ replaced by $u$, $u_{\omega,n}$ replaced by $u^\ee$, and
$\sigma$ and $\rho$ chosen such that
%
\begin{equation}
	\begin{split}
	& 1/2 < \sigma < 1,
	\\
	& \sigma + 1 \le \rho \le s 
	\end{split}
\end{equation}
yields
 \begin{equation*}
	\begin{split}
		\frac{1}{2}\frac{d}{dt} \|v\|_{H^\sigma(\ci)}^2
		& \le
		C'' (\|u_n + u^\ee_n \|_{H^{\rho}(\ci)} +
		\|\p_x(u_n + u^\ee_n) \|_{H^\sigma(\ci)})
		\cdot \|v\|_{H^\sigma(\ci)}^2.
	\end{split}
\end{equation*}
\medskip
It follows that 
\begin{equation}
	\begin{split}
		\frac{1}{2}\frac{d}{dt} \|v\|_{H^{\sigma}(\ci)}^2
		& \le
		C'' \cdot \|u_n
		+ u^\ee_n\|_{H^{s}(\ci)}\cdot \|v\|_{H^{\sigma}(\ci)}^2.
		\label{a10qx}
	\end{split}
\end{equation}
Applying the triangle inequality and
part (iii) of Theorem \ref{thm:HR_existence_continous_dependence} (proved in
Section \ref{existence})
to \eqref{a10qx} yields
%
\begin{equation}
	\begin{split}
		\label{a11qx}
		\frac{1}{2}\frac{d}{dt} \|v\|_{H^{\sigma}(\ci)}^2
		& \le
		C' (\|u_n(0)\|_{H^{s}(\ci)}
		+ \|u^\ee_n(0)\|_{H^{s}(\ci)})\cdot \|v\|_{H^{\sigma}(\ci)}^2
		\\
		& = C' (\|u_{0,n}\|_{H^{s}(\ci)}
		+ \|J_\ee u_{0,n}\|_{H^{s}(\ci)})\cdot \|v\|_{H^{\sigma}(\ci)}^2.
	\end{split}
\end{equation}
Note that the family $\left\{ u_{0,n} \right\}_n$ is uniformly bounded in
$H^s(\ci)$. Hence, applying Proposition \ref{lem3r} to \eqref{a11qx} we obtain the critical estimate 
\begin{equation}
	\begin{split}
		\label{a12qx}
		\frac{1}{2}\frac{d}{dt} \|v\|_{H^{\sigma}(\ci)}^2
		& \le
	C \|v\|_{H^{\sigma}(\ci)}^2
\end{split}
\end{equation}
with $C = C(R)$, where $R$ is defined as in \eqref{r-def}. Note that $C$
does not depend on $n$ or $\ee$. Differentiating
the left hand side of \eqref{a12qx} and simplifying, we obtain
\begin{equation}
	\begin{split}
		\frac{d}{dt}\|v\|_{H^{\sigma}(\ci)} \le C \|v\|_{H^{\sigma}(\ci)}.
		\label{a100qx}
	\end{split}
\end{equation}
Let $y(t) = \|v\|_{H^{\sigma}(\ci)}$. Then \eqref{a100qx} gives
\begin{equation*}
	\begin{split}
		\frac{1}{y(t)}\frac{dy}{dt} \le C.
	\end{split}
\end{equation*}
Hence,
\begin{equation*}
	\begin{split}
		\int_0^t \frac{1}{y(\tau)} \frac{dy}{d \tau}
		\le \int_0^t C \ d \tau, \qquad |t| \le T
	\end{split}
\end{equation*}
from which we obtain
\begin{equation}
	\begin{split}
		\ln |y(t) | - \ln |y(0)| \le C t.
		\label{a101qx}
	\end{split}
\end{equation}
Simplifying \eqref{a101qx}, we have
\begin{equation*}
	\begin{split}
		\ln \left |\frac{y(t)}{y(0)} \right | \le C t
	\end{split}
\end{equation*}
which yields the estimate
\begin{equation*}
	\begin{split}
		y(t) \le y(0) e^{C t}, \qquad |t| \le T.
	\end{split}
\end{equation*}
Substituting back in $\|v\|_{H^{\sigma}(\ci)}$ for $y$, we get
\begin{equation}
	\label{vsig-est}
	\begin{split}
		\|v\|_{H^{\sigma}(\ci)}
		& \le e^{C t}\|v(0)\|_{H^{\sigma}(\ci)}
		\\
		& = e^{C t}\|u_n(0) - u^\ee_n(0) \|_{H^{\sigma}(\ci)}
		\\
		& = e^{C t}\|u_{0,n} - J_\ee u_{0,n}\|_{H^{\sigma}(\ci)}.
	\end{split}
\end{equation}
Applying Proposition \ref{lem4r} to \eqref{vsig-est}, we obtain 
\begin{equation}
	\label{almost}
	\begin{split}
		\|v\|_{H^\sigma (\ci)} \le e^{Ct} \|u_{0,n}\|_{H^\sigma(\ci)} \cdot
		o(\ee^{s-\sigma})
	\end{split}
\end{equation}
and since $\|u_{0,n}\|_{H^s(\ci)} \le R$ for all $n \in \mathbb{N}$, where
$R$ is defined as in \eqref{r-def}, we conclude from estimate \eqref{almost} that
\begin{equation*}
	\begin{split}
		\|v\|_{H^\sigma(\ci)} \le C(R) \cdot o(\ee^{s-\sigma})
	\end{split}
\end{equation*}
completing the proof. $\quad \Box$
We are now prepared to interpolate. Recall \eqref{a15qu}. Applying Remark
\ref{lem2r}, Remark \ref{lem5r'}, and
Proposition \ref{prop:180} gives
\begin{equation*}
	\begin{split}
		& \|u^\ee_n \|_{H^{s+1}(\ci)} \|v\|_{H^{s-1}(\ci)} \|v\|_{H^s
		(\ci)}
		\\
		&\le C'' \ee^{-1} \cdot \|v\|_{H^\sigma(\ci)}^{1/(s-\sigma)}
		\|v\|_{H^s(\ci)}^{2 - 1/(s- \sigma)}
		\\
		& \le C'' \ee^{-1} \cdot \Big [C' \cdot o(\ee^{s- \sigma})\Big ]^{1/(s-\sigma)}
		\cdot \|v\|_{H^s(\ci)}^{2- 1/(s-\sigma)}
	\end{split}
\end{equation*}
from which we obtain
\begin{equation}
	\begin{split}
		\label{a200qx}
		\|u^\ee_n\|_{H^{s+1}(\ci)} \|v\|_{H^{s-1}(\ci)} \|v \|_{H^s(\ci)}
		& \le  C \cdot o(1) \cdot \|v\|_{H^s(\ci)}^{2- 1/(s-\sigma)}.
	\end{split}
\end{equation}
where $C=C(R)$ does not depend on $\ee$ or $n$. We wish to control the growth of the right hand side of \eqref{a200qx}.
First, note that the triangle inequality, part (iii) of Theorem
\ref{thm:HR_existence_continous_dependence}, and Proposition \ref{lem3r} imply
\begin{equation}
	\begin{split}
		\|v\|_{H^s(\ci)} & = \|u_n - u^\ee_n \|_{H^s(\ci)} 
		\\
		& \le \|u_n \|_{H^s(\ci)} + \|u^\ee_n\|_{H^s(\ci)}
		\\
		& \le 2\left[ \|u_{0,n} \|_{H^s(\ci)} + \|J_\ee u_{0,n}
		\|_{H^s(\ci)} \right]
		\\
		& \le 4 \|u_{0,n} \|_{H^s(\ci)}, \qquad |t| \le T
		\label{agrowth_v}
	\end{split}
\end{equation}
and since $\{u_{0,n}\}_n$ belongs to a bounded subset of
$H^s(\ci)$, we see from \eqref{agrowth_v} that $\|v \|_{H^s(\ci)}$ is
uniformly bounded in $n$ \emph{and} $\ee$.  Secondly, by Remark \ref{lem5r'}, we have 
\begin{equation}
	\label{au-ee-bound}
	\|u^\ee_n \|_{H^s(\ci)} \le C(R), \ \ \text{for all} \ \ 0 < \ee <<1, \ n \in
	\mathbb{N}.
\end{equation}
Applying \eqref{a200qx}, \eqref{agrowth_v}, and \eqref{au-ee-bound}
to \eqref{a15qu}, it follows that 
\begin{equation*}
	\label{lim-est-in}
		\begin{split}
			\lim_{n \to \infty }
			\frac{1}{2}\frac{d}{dt}\|v\|_{H^{s}(\ci)}^2
			& \le
			C \lim_{n \to \infty} \Big [\|v\|_{H^s(\ci)}^3 +
			\|v\|_{H^s(\ci)}^2 + o(1)\Big ]
		\end{split}
	\end{equation*}
	for every $0 < \ee <<1$, where $C = C(\|u_0\|_{H^s(\ci)}, \ R)$.
	Therefore
	\begin{equation}
		\begin{split}
			\lim_{\substack{n \to \infty \\ \ee \to 0} }
			\frac{1}{2}\frac{d}{dt}\|v\|_{H^{s}(\ci)}^2
			& \le C
			\lim_{\substack{n \to \infty \\ \ee \to 0}}
			\Big [\|v\|_{H^s(\ci)}^3 + 
			\|v\|_{H^s(\ci)}^2 \Big ].
			\label{a15qx}
		\end{split}
	\end{equation}
	We differentiate the left hand side of \eqref{a15qx} and obtain
\begin{equation*}
	\begin{split}
		\lim_{\substack{n \to \infty \\ \ee \to 0}}\frac{d}{dt}
		\|v\|_{H^s(\ci)} \le C
		\lim_{\substack{n \to \infty \\ \ee \to 0}} \left [\|v\|_{H^s(\ci)}^2 +
		\|v\|_{H^s(\ci)} \right ].
	\end{split}
\end{equation*}
Letting $y = \|v\|_{H^s(\ci)}$ and rearranging gives
\begin{equation*}
	\begin{split}
		\lim_{\substack{n \to \infty \\ \ee \to 0} } \ \frac{1}{y(y+1)} \frac{dy}{dt}
		\le	C
	\end{split}
\end{equation*}
which can be rewritten as
\begin{equation*}
	\begin{split}
		\lim_{\substack{n \to \infty \\ \ee \to 0} }
		\left( \frac{1}{y} - \frac{1}{y+1} \right)\frac{dy}{dt} \le C 
	\end{split}
\end{equation*}
implying
\begin{equation}
	\label{aest-int}
	\begin{split}
		\lim_{\substack{n \to \infty \\ \ee \to 0} } \left [
\int_0^t \frac{1}{y} \frac{dy}{d \tau} \ d \tau
		- \int_0^t \frac{1}{y+1} \frac{dy}{d \tau} \ d \tau \right ]
		\le \int_0^t C \ d \tau, \quad |t| \le T.
	\end{split}
\end{equation}
Hence \eqref{aest-int} gives 
\begin{equation}
	\begin{split}
	\lim_{\substack{n \to \infty \\ \ee \to 0} }	\left [ \ln \left | \frac{y(t)}{y(0)}
	\cdot \frac{y(0) + 1}{y(t) + 1} \right | \right ] \le C t.
		\label{20b}
	\end{split}
\end{equation}
Exponentiating both sides of \eqref{20b}, and noting that $f(x) = e^x$
is a continuous function on $\rr$, we must have
\begin{equation*}
	\begin{split}
		\lim_{\substack{n \to \infty \\ \ee \to 0} }	
		\left |
		\frac{y(t)}{y(0)} \cdot \frac{y(0) + 1}{y(t) + 1} \right | \le e^{C t}.
	\end{split}
\end{equation*}
Recalling that $y(t) = \|v(t)\|_{H^s(\ci)} \ge 0$, we obtain
\begin{equation*}
	\begin{split}
		\lim_{\substack{n \to \infty \\ \ee \to 0} }	
		\frac{y(t)}{y(0)} \cdot \frac{y(0) + 1}{y(t) + 1} \le e^{C t}.
	\end{split}
\end{equation*}
Rearranging, it follows that 
\begin{equation*}
	\begin{split}
		\lim_{\substack{n \to \infty \\ \ee \to 0}} \frac{y(t)}{y(t) + 1}
		\le \lim_{\substack{n \to \infty \\ \ee \to 0}} \frac{e^{C t} \cdot y(0)}{y(0) + 1} \le
		\lim_{\substack{n \to \infty \\ \ee \to 0}} e^{C t} \cdot y(0).
	\end{split}
\end{equation*}
Substituting back in $\|v(t)\|_{H^s(\ci)}$ for $y(t)$ gives
\begin{equation}
	\begin{split}
		\lim_{\substack{n \to \infty \\ \ee \to 0}}	\frac{\|v(t)\|_{H^s(\ci)}}{\|v(t)\|_{H^s(\ci)} + 1}  \le
		\lim_{\substack{n \to \infty \\ \ee \to 0}} e^{C t} \cdot \|v(0)\|_{H^s(\ci)}.
		\label{a303qx}
	\end{split}
\end{equation}
Since by Proposition \ref{lem4r} 
\begin{equation*}
	\begin{split}
	\lim_{\substack{n \to \infty \\ \ee \to 0} }
	\|v(0)\|_{H^s(\ci)}
	& = \lim_{\substack{n \to \infty \\ \ee \to 0} }
	\|u_{0,n} - J_\ee u_{0,n} \|_{H^s(\ci)} 
	\\
	& \le  \lim_{\substack{n \to \infty \\ \ee \to 0}}
	\left [ \|u_{0,n}\|_{H^s(\ci)} \cdot o(1) \right ]
	\\
	& = \lim_{\ee \to 0} \left [ \|u_0\|_{H^s(\ci)} \cdot o(1) \right ]
	\\
	& = 0
	\end{split}
\end{equation*}
we deduce from \eqref{a303qx} that
\begin{equation*}
	\begin{split}
		\lim_{\substack{n \to \infty \\ \ee \to 0}} \|v(t)\|_{H^s(\ci)} = 0, \qquad |t| \le T
	\end{split}
\end{equation*}
completing the proof of \eqref{enough_to_prove3}. $\quad \Box$
%
%
%
\section{Extending Well-Posedness for HR to the Non-Periodic Case}
\label{sec:defs}
The method will be analogous to that of the periodic case, with two major
modifications. First, we must choose a different mollifier $J_\ee$ in the
proof of continuous dependence. Pick a
function $j(x) \in \mathcal{S}(\rr)$ such that
\begin{equation*}
		\begin{split}
			& 0 \le \widehat{j}(\xi) \le 1,
			\\
			& \widehat{j}(\xi) = 1 \ \ \text{if} \ \ |\xi| \le 1.
		\end{split}
	\end{equation*}
Letting
\begin{equation*}
	\begin{split}
		j_\ee(x) = \frac{1}{\ee} j \left (\frac{x}{\ee} \right )
	\end{split}
\end{equation*}
it can be verified that 
		\begin{equation*}
		\begin{split}
			\widehat{j_\ee}(\xi) = \widehat{j }(\ee \xi), \quad \ee > 0.
		\end{split}
	\end{equation*}
We then define $J_\ee$ to be the ``Friedrichs mollifier''
	\begin{equation*}
		\begin{split}
			J_\ee f(x) = j_\ee * f(x), \quad \ee>0.
		\end{split}
	\end{equation*}
Given this construction, the proofs of Remark \ref{lem5r}, Remark
\ref{lem5r'}, and Proposition \ref{lem4r} for the non-periodic case will be
analogous to those in the periodic case.
Secondly, in the proof of existence, we will have difficulties in arranging
that the solutions $\{u_\ee\}$ to the mollified HR i.v.p. converge in $C(I,
H^{s- \sigma}(\rr))$, $0 < \sigma < 1$ to a candidate solution $u$ of the HR
i.v.p. We will get around this by considering the family $\left\{ \varphi
u_\ee \right\}$ instead.
%
%
%
%
We divide our work into three parts:
\subsection{Existence.}
Mirroring the argument in the periodic case, we see that the bounded
family $\{u_\ee\}$ is compact in the weak* topology of $L^\infty(I,
H^{s}(\rr))$. More precisely, there is a sequence  $\{ u_{\ee_n} \}$
converging weak* to a $ u\in L^{\infty}(I, H^s(\rr))$; that is 
		%
		\begin{equation*}
			\label{hhweak-conv}
			\lim_{n\to \infty} T_{u_{\ee_n}}(\varphi)  =  T_u (\varphi) 
			\; \;		
			\text{ for all } \;\;  \varphi \in L^1(I, H^{s}(\rr))
		\end{equation*}
		where
		\begin{equation}
			T_v(\varphi) = \int_I <v (t), \varphi (t)>_{H^s(\rr)} dt  = \int_I
			 \int_\rr
			 \widehat{v}(\xi, t) \bar{\widehat{\varphi}} (\xi, t) \cdot (1 +
			 \xi^2)^s \ d \xi \; dt.
		\end{equation}
		%
		Similarly, $\left\{ \p_x u_{\ee_n} \right\}$ is compact in the
		weak* topology of $L^\infty(I, H^{s-1}(\rr))$ and converges weak*
		to $\p_x u$. Hence, for any $k \in \mathbb{N}$, we have
		\begin{align}
			\label{base-weak}
				& (u_{\ee_n})^k \xrightarrow{\text{weak*}} u^k \ \
				\text{on} \ \
				L^\infty(I, H^s(\rr)),
				\\
				\label{base-weak-2}
				& (\p_x u_{\ee_n})^k \xrightarrow{\text{weak*}} (\p_x u)^k
				\ \ \text{on} \ \
				L^\infty(I, H^{s-1}(\rr)). 
		\end{align}
		In order to show that $u$ solves the HR i.v.p., it would
		suffice to obtain a stronger convergence for  $u_{\ee_n}$ so that 
		we could take the limit in the mollified HR equation. However,
		this is difficult, and unnecessary. Rather, our approach will be to
		show that for any pseudo-differential operator
		$P \in \Psi^0$ and arbitrary $\vp \in S(\rr)$, $k \in
		\mathbb{N}$, $0< \sigma < 1$, we have
		%
		%
			\begin{align}
			\label{hhstrong-conv}
			& \varphi P [(u_{\ee_n})^k] \longrightarrow \varphi P [u^k]  
			\quad
			\text{ in } \,\,   C(I, H^{s-\sigma}(\rr)), \ \,
			\\
			\label{hhstrong-conv-next}
			& \varphi P [(\p_x u_{\ee_n})^k] \longrightarrow \varphi P
			[(\p_x u)^k]  
			\quad
			\text{ in } \,\,   C(I, H^{s-\sigma -1}(\rr)), \ \ 
		\end{align}
		%
		which will then be applied to a rewritten version of the HR
		i.v.p. Our focus will be on proving \eqref{hhstrong-conv}; since the proof of
		\eqref{hhstrong-conv-next} is similar, we will omit the
		details. First, we will need the following
		interpolation result:
		%%%%%%%%%%%%%%%%%%%%%%%%%%%
		%
		%
		%                 Interpolation Lemma
		%
		%
		%%%%%%%%%%%%%%%%%%%%%%%%%%%
		\begin{lemma}
			\label{hhinterpolation-lem}
			(Interpolation)     Let  $s > \frac{3}{2}$.
			If $v \in C(I, H^s(\rr)) \cap C^1(I, H^{s-1}(\rr))$
			then $v \in C^\sigma (I, H^{s- \sigma}(\rr))$ for  $0 < \sigma < 1$.
		\end{lemma}
		%
		\subsection{ Proof.} It is analogous to the proof in the periodic case.
		$\quad \Box$
		Fix $k \in \mathbb{N}$. Using Lemma \ref{hhinterpolation-lem}, we
		will show that the family
		\begin{equation*}
			\begin{split}
			 \{\varphi P[(u_\ee)^k]\}_\ee
		\end{split}
	\end{equation*}
		is equicontinuous in $C(I, H^{s-\sigma}(\rr))$ 
		for $0 < \sigma < 1$ and $\varphi = \varphi(x) \in \mathcal{S}(\rr)$.
		We will follow this by proving that
		there exists a sub-family $\{\varphi P[(u_{\ee_n}(t))^k]\}_n$
		that is precompact in $H^{s-\sigma}(\rr)$ for $\sigma > 0$. 
		These two facts, in conjunction with Ascoli's Theorem, will
		yield
		\begin{equation*}
			\label{hhstrong-conv2}
			\varphi P[(u_\ee)^k] \to \tilde{u}
			\; \; \text{in} \; \; C(I,H^{s-\sigma}(\rr))
		\end{equation*}
		for $0 < \sigma < 1$.
		We will then show that $\tilde{u} = \varphi P[u^k]$, from which it will
		follow that
		\begin{equation*}
			\label{hhphiplus}
			\begin{split}
				\varphi P[(u_\ee)^k] \to \varphi P[u^k]
				\; \; \text{in} \; \; C(I,H^{s-\sigma}(\rr)).
			\end{split}
		\end{equation*}
		%%%%%%%%%%%%%%%%%%%%%%
		%
		%
		%       Equicontinuity
		%
		%
		%%%%%%%%%%%%%%%%%%%%%%
		%
		\subsection{  Equicontinuity of $\{ \varphi P [(u_\ee)^k]\}_\ee$  in $C(I,
		H^{s-\sigma}(\rr)$}).
		%
		%
		Since $\varphi \in \mathcal{S}(\rr)$, the map $u \mapsto \vp u$
		is a bounded linear function on $H^s(\rr)$, for arbitrary $s \in
		\rr$, where  
		\begin{equation}
			\begin{split}
				\|\varphi u\|_{H^s(\rr)} \le C(s, \varphi)
				\|u\|_{H^s(\rr)}, \quad \forall s\in \rr.
				\label{hhschwartz-estimate}
			\end{split}
		\end{equation}
		Furthermore, $$P: H^s(\rr) \to H^s(\rr)$$ is bounded and linear,
		with 
		\begin{equation}
			\label{operator-normaa}
			\|P\|_{L(H^s(\rr), H^s(\rr))} \le 1.
		\end{equation}
		Hence, the map 
		\begin{equation}
			\label{the-map}
			\begin{split}
			& T: H^s(\rr) \to H^s(\rr),
			\\
			& T(u) = \vp P u 
		\end{split}
	\end{equation}
	is bounded and linear, with 
	\begin{equation}
		\begin{split}
			\|T\|_{L(H^s(\rr), H^s(\rr))} \le C(s, \vp).
			\label{op-norm-product}
		\end{split}
	\end{equation}
	Therefore, applying Lemma
		\ref{hhinterpolation-lem} gives 
		%
		\begin{equation*}
			\begin{split}
			\label{hhequic-1}
			& \sup_{t \neq t'} \frac {\| \varphi P [(u_\ee(t))^k] - \varphi
			P [(u_\ee(t'))^k] \|_{H^{s -
			\sigma  }(\rr)}}{|t - t'|}
			\\
			& \le \sup_{t \neq t'}  \frac { \|\vp P \|_{L(H^{s-\sigma}(\rr),
			H^{s-\sigma}(\rr))} \cdot \|   [u_\ee(t)]^k  - 
			[u_\ee(t')]^k \|_{H^{s -
			\sigma }(\rr)}}{|t - t'|}
			\\
			& \le C(s, \vp) \cdot \sup_{t \neq t'}  \frac { \|   [u_\ee(t)]^k  - 
			[u_\ee(t')]^k \|_{H^{s -
			\sigma }(\rr)}}{|t - t'|}
			\\
			&< c
		\end{split}
		\end{equation*}
		%
		or
		%
		\begin{equation*}
			\label{hhequic-2}
			\|\varphi P [(u_\ee(t))^k] - \varphi
			P [(u_\ee(t'))^k \|_{H^{s - \sigma }(\rr)}< c|t -
			t'|, 
			\text{ for all }  \,\,  t, t'\in I,
		\end{equation*}
		%
		which shows that  the family  $\{\varphi P [(u_\ee)^k]\}_\ee$ is
		equicontinuous in $C(I, H^{s-\sigma }(\rr))$.  $\quad \Box$
		%
		%
		%%%%%%%%%%%%%%%%%%%%%%
		%
		%
		%      PreCompactness
		%
		%
		%%%%%%%%%%%%%%%%%%%%%%%%%%
		%
		%
		%
		%
		%		
		\subsection{ Precompactness of $\{\varphi P [(u_\ee(t))^k]\}_\ee$ in
		$H^{s-\sigma  }(\rr)$}.
		Applying the algebra property of Sobolev
		Spaces, and recalling \eqref{the-map}-\eqref{op-norm-product}, we have
		\begin{equation}
			\begin{split}
			\label{hhcompact-1}
			 \|\varphi P [(u_\ee(t))^k]\|_{H^{s}(\rr)}
			& \le  C(s, \vp) \cdot \|[u_\ee(t)]^k\|_{H^{s}(\rr)}
			\\
			& \le C(s, \vp) \cdot \|u_\ee(t)\|^k_{H^{s}(\rr)}.
			\end{split}
		\end{equation}
		%
		Letting $|t| \le T$, we now apply Lemma \ref{hr_wp} to
		\eqref{hhcompact-1} to obtain
		\begin{equation*}
			\begin{split}
			\|\varphi P [(u_\ee(t))^k]\|_{H^{s}(\rr)}
			\le 2^k C(s, \vp) \cdot  \|u_0 \|^k_{H^s(\rr)} < \infty.
			\end{split}
		\end{equation*}
		Therefore, by Reillich's Theorem, the family $\left\{
		\varphi P [(u_\ee(t))^k] \right\}_\ee$ is
		precompact in $H^{s- \sigma }(\rr)$ for all $\sigma > 0$ and $|t| \le T$. $\quad
		\Box$ 
		Hence, compiling our previous results on equicontinuity and precompactness
		and applying Ascoli's Theorem, we
		conclude that we can find $\tilde{u}$ and a subfamily 
		\\ $\left\{
		\varphi P [(u_{\ee_n})^k]
		\right\}_n$ such that
		\begin{equation}
			\label{hhstrong-conv-of-u_ep}
			\varphi P [(u_{\ee_n})^k] \to \tilde{u}
			\; \; \text{in} \; \; C(I, H^{s-\sigma}(\rr)).
		\end{equation}
		%
		%
		We would now like to find out what $\tilde{u}$ is:
		%
		%
		%
		\begin{lemma}
			\label{hhlem:crit-conv}
			For arbitrary $k \in \mathbb{N}$,
			\begin{equation}
				\begin{split}
					\varphi P [(u_{\ee_n})^k] \xrightarrow{weak^*}
					\varphi P [u^k] \ \ \text{on} \ \ L^\infty(I,
					H^{s-\sigma}(\rr)).
					\label{hhcrit-conv-est}
				\end{split}
			\end{equation}
		\end{lemma}
		\subsection{ Proof.} 
		Fix $k \in \mathbb{N}$ and recall that the operators 
		\begin{equation*}
			\begin{split}
			 & T_\varphi: H^s(\rr) \to H^s(\rr)\\
			 & T_\varphi u = \varphi u
		\end{split}
	\end{equation*}
and 
\begin{equation*}
	\begin{split}
		P:H^s(\rr) \to H^s(\rr)
	\end{split}
\end{equation*}
	are continuous; therefore 
	\begin{equation*}
		\begin{split}
			T_\vp P: H^s(\rr) \to H^s(\rr)
		\end{split}
	\end{equation*}
	continuously. Hence, its adjoint  $(T_\varphi P)^*$
	exists and
		\begin{equation*}
			(T_\varphi P)^*: H^s(\rr) \to H^s(\rr) 
		\end{equation*}
		continuously. Therefore, applying \eqref{base-weak}, we conclude that
		\begin{equation}
			\label{widpseudo}
			\begin{split}
				& \int_I <\varphi P[u^k] - \varphi
				P [(u_{\ee_n})^k],\  f>_{H^{s-\sigma }(\rr)} dt
				\\
				&= \int_I <u^k - 
				 (u_{\ee_n})^k, \ (T_\vp P)^* f>_{H^{s-\sigma }(\rr)} \to 0
			\end{split}
		\end{equation}
		completing the proof. $\quad \Box$
		%
		%
		Now, recalling \eqref{hhstrong-conv-of-u_ep} and applying Lemma
		\ref{hhlem:crit-conv}, we obtain
			\begin{equation}
			\begin{split}
				\vp P [(u_{\ee_n})^k] \to \vp P [u^k] \ \ \text{in}  \ \ C(I,
				H^{s-\sigma}(\rr))
				\label{hhvp_u_ep_conv}
			\end{split}
		\end{equation}
		for arbitrary $k \in \mathbb{N}$.  Using precisely the same
		strategy we used to prove \eqref{hhvp_u_ep_conv} (applied now to
		the family $\{ \vp P [(\p_x u_{\ee})^k] \}_\ee$), one can also show
	\begin{equation}
			\begin{split}
			\vp P [ (\p_x u_{\ee_n})^k] \to \vp P [(\p_x u)^k] \ \ \text{in}  \ \ C(I,
				H^{s-\sigma -1 }(\rr)).
			\end{split}
		\end{equation}
		We summarize our result below:
		\begin{theorem}
		\label{hhthm:crit1}
		Let $P \in \Psi^0$ be a pseudo-differential operator. Then for
		arbitrary $k \in \mathbb{N}$, 
			\begin{equation}
			\begin{split}
				& \vp P [(u_{\ee_n})^k] \to \vp P [u^k] \ \ \text{in}  \ \ C(I,
				H^{s-\sigma }(\rr)),
				\\
				& 
				\vp P [(\p_x u_{\ee_n})^k] \to \vp P [(\p_x u)^k] \ \
				\text{in}  \ \ C(I,
				H^{s-\sigma -1}(\rr)).
				\label{hhdx_vp_u_ep_conv}
			\end{split}
		\end{equation}
	\end{theorem}
		\subsection{ Verifying that the weak* limit $u$ solves the HR equation.} 
		We recall the mollified HR i.v.p
		\begin{align}
			& \p_t u_{\ee_n}  = -\gamma(J_{\ee_n} u_{\ee_n} \cdot \p_x
			J_{\ee_n} u_{\ee_n})
			\label{hh1gr}
			\\
			& u(x,0) = u_0(x).
			\label{hh2gr}
		\end{align}
		Multiplying both sides of \eqref{hh1gr} by $\varphi$ and rewriting,
		we obtain
		\begin{equation}
			\label{hh3}
			\begin{split}
				\p_t(u_{\ee_n} \varphi) = -\gamma \vp (J_{\ee_n} u_{\ee_n} \cdot
				J_{\ee_n} \p_x u_{\ee_n}).
			\end{split}
		\end{equation}
		The following lemma will play a crucial role in our proof of the
		existence of a solution to the HR i.v.p.
		\begin{lemma}
			\label{hhlem:cc}
			For $\vp \in \mathcal{S}(\rr)$ such that
			$\vp^\frac{1}{2} \in \mathcal{S}(\rr)$, we have
			\begin{equation}
				\begin{split}
					\label{hhburgers_and_nonlocal_conv}
				& \vp (J_{\varepsilon_n} u_{\varepsilon_n} 
				\cdot J_{\varepsilon_n}\partial_x u_{\varepsilon_n}) 
				\to \vp u \partial_x u \; \; 
				\text{in} \; \;
				C(I, H^{s-\sigma-1}(\rr)). 
			\end{split}
			\end{equation}
		\end{lemma}
		%
		\subsection{ Proof.} We will need a couple of propositions:
		\begin{proposition}
			For arbitrary $\vp \in \mathcal{S}(\rr)$
			\label{hhprop:1aa}
			\begin{equation}
				\begin{split}
					\vp J_{\ee_n} u_{\ee_n} \to \vp u \ \ \text{in} \ \
					C(I, H^{s-\sigma}(\rr)).
					\label{hh}
				\end{split}
			\end{equation}
		\end{proposition}
			\subsection{ Proof.} Note that
			\begin{equation}
				\begin{split}
					& \|\vp u - \vp J_{\ee_n} u_{\ee_n}
					\|_{C(I, H^{s-\sigma}(\rr))}
					\\
					&= \|\vp u - \vp J_{\ee_n} u_{\ee_n} \pm \vp
					u_{\ee_n} \|_{C(I, H^{s-\sigma}(\rr))}
					\\
					& = \|\vp u - \vp u_{\ee_n}
					\|_{C(I, H^{s-\sigma}(\rr))} + \|\vp (I - J_{\ee_n})
					u_{\ee_n} \|_{C(I, H^{s-\sigma}(\rr))}.
					\label{hh1bb}
				\end{split}
			\end{equation}
			Applying \eqref{hhschwartz-estimate} and the estimates
			\begin{equation*}
				\begin{split}
					& \|I-J_{\ee_n} \|_{L(H^{s-\sigma}(\rr), H^{s -
					\sigma}(\rr))} = o(1),
					\\
					& \|u_{\ee_n}\|_{H^{s-\sigma}(\rr)} \le 2
					\|u_0\|_{H^{s-\sigma}(\rr)}
				\end{split}
			\end{equation*}
			to \eqref{hh1bb} gives
			\begin{equation}
				\label{hh2bb}
				\begin{split}
					\|\vp u - \vp J_{\ee_n} u_{\ee_n}\|_{H^{s-\sigma}(\rr)}
					\le \left( \|\vp u - \vp u_{\ee_n}
					\|_{C(I, H^{s-\sigma}(\rr))} + C(s, \vp) \cdot o(1) \cdot \|u_0
					\|_{H^{s-\sigma}(\rr)} \right).
				\end{split}
			\end{equation}
			Letting $\ee \to 0$ in \eqref{hh2bb} and applying Theorem
			\ref{hhthm:crit1} completes the proof. $\quad \Box$
			%
			%
			\begin{proposition}
				\label{hhprop:dd}
				For arbitrary $ \vp \in \mathcal{S}(\rr)$,
				\begin{equation}
					\begin{split}
						\vp J_{\ee_n} \p_x u_{\ee_n} \to \vp u \ \
						\text{in} \ \ C(I, H^{s-\sigma-1}(\rr)).
						\label{hh0dd}
					\end{split}
				\end{equation}
			\end{proposition}
			\subsection{ Proof.} The result follows from Theorem \ref{hhthm:crit1}.
			The proof is nearly identical to that of
			Proposition \ref{hhprop:1aa}, with $s-1$ substituted for $s$
			and $\p_x u_{\ee_n}$ substituted for $u_{\ee_n}$. $\quad \Box$
			%
			%
			We now have enough tools to prove Lemma \ref{hhlem:cc}. Restrict the
			choice of $\vp$ such that $\vp^\frac{1}{2} \in S(\rr)$
			(Such Schwartz functions exist; as an example, take the square
			of the Gaussian). Using this fact, and applying Proposition
			\ref{hhprop:1aa} and Proposition \ref{hhprop:dd}, we conclude that
			\begin{equation*}
				\begin{split}
					\vp J_{\ee_n} u_{\ee_n} \p_x J_{\ee_n} u_{\ee_n} 
					& = \vp^\frac{1}{2} J_{\ee_n} u_{\ee_n} \cdot
					\vp^\frac{1}{2} \p_x J_{\ee_n} u_{\ee_n}
					\\
					& \to \vp^\frac{1}{2} u \cdot \vp^\frac{1}{2} \p_x u = \vp
					u \p_x u
				\end{split}
			\end{equation*}
			completing the proof of Lemma \ref{hhlem:cc}. $\quad \Box$
%
%
%
%
By Theorem \ref{hhthm:crit1} it follows immediately that
		\begin{equation}
			\begin{split}
				& \vp \p_x(1- \p_x^2)^{-1} \left( \frac{3-\gamma}{2}
				(u_{\ee_n})^2
				 + \frac{\gamma}{2} (\p_x u_{\ee_n})^2 \right )
				 \\
				 & \to
				 \vp \p_x(1- \p_x^2)^{-1} \left( \frac{3-\gamma}{2} u^2
				 + \frac{\gamma}{2} (\p_x u)^2 \right ) \ \
				 \text{in} \ \ C(I, H^{s-\sigma-1}(\rr)).
				\label{llnon-local-convergence}
			\end{split}
		\end{equation}
		Combining \eqref{hhburgers_and_nonlocal_conv} and
		\eqref{llnon-local-convergence}, and applying the Sobolev Imbedding
		Theorem, we deduce 
		\begin{equation}
			\begin{split}
				& -\gamma \vp (J_{\ee_n} u_{\ee_n} \cdot J_{\ee_n} \p_x
				u_{\ee_n}) -
				\vp \p_x(1- \p_x^2)^{-1} \left( \frac{3-\gamma}{2}
				(u_{\ee_n})^2
				 + \frac{\gamma}{2} (\p_x u_{\ee_n})^2 \right )
				 \\
				 \to & -\gamma \vp u \p_x u -
				 \vp \p_x(1- \p_x^2)^{-1} \left( \frac{3-\gamma}{2} u^2
				 + \frac{\gamma}{2} (\p_x u)^2 \right ) \ \
				 \text{in} \ \ C(I, C(\rr)).
				\label{llloc-non-loc-tog}
			\end{split}
		\end{equation}
		%
		Next, we note that the convergence  
		%
		\begin{equation}
			\label{hhweak-conv-2}
			T_{\vp u_{\ee_n}}(f)  \longrightarrow  T_{\vp u} (f) \;
			\text{ for all } \;  f \in L^1(I, H^{-s}(\rr))
		\end{equation}
		%
		can be restated as 
		%
		\begin{equation}
			\vp u_{\ee_n}  \longrightarrow  \vp u
			\quad
			\text{ in }  \,\,
			\mathcal{D}'(I\times \rr).
		\end{equation}
		%
		This implies 
		%
		\begin{equation}
			\label{hhdistib-conv-2}
			\p_t(\vp u_{\ee_n})  \longrightarrow  \p_t (\vp u)
			\quad
			\text{ in }  \,\, \mathcal{D}'(I\times \rr).
		\end{equation}
		%
		Since for all $n$ we have 
		%
		\begin{equation}
			\begin{split}
			 \p_t (\vp u_{\ee_n})
			 = & -\gamma \vp
			(J_{\varepsilon_n} u_{\varepsilon_n}  \cdot
			J_{\varepsilon_n}\partial_x u_{\varepsilon_n})
			\\
			& -
			\vp \p_x(1- \p_x^2)^{-1} \left( \frac{3-\gamma}{2} (u_{\ee_n})^2
			 + \frac{\gamma}{2} (\p_x u_{\ee_n})^2 \right )
		 \end{split}
		\end{equation}
		%
		it follows from \eqref{hhdistib-conv-2} and the uniqueness of the
		limit in \eqref{llloc-non-loc-tog} that
		\begin{equation}
			\begin{split}
			 \p_t (\vp u)
			 = & -\gamma \vp
			u \p_x u - \vp \p_x(1- \p_x^2)^{-1} \left( \frac{3-\gamma}{2} u^2
			 + \frac{\gamma}{2} (\p_x u)^2 \right )
			\label{hhadone}
			\end{split}
		\end{equation}
		Further restricting $\vp \in \mathcal{S}(\rr)$ to be nonzero in
		$\rr$, we
		can divide both sides of \eqref{hhadone} by $\vp$ to obtain
		\begin{equation}
			\label{hh2yy}
			\begin{split}
			 \p_t  u
			 = & -\gamma
			u \p_x u - \p_x(1- \p_x^2)^{-1} \left( \frac{3-\gamma}{2} u^2
			 + \frac{\gamma}{2} (\p_x u)^2 \right ).
			\end{split}
		\end{equation}
		Thus we have constructed a solution $u \in L^\infty(I, H^s(\rr))$
		to the HR i.v.p. 
\subsection{Uniqueness.} The proof is analogous to that in the periodic case.
\subsection{Continuous Dependence.} The proof is analogous to the proof in
the periodic case, with one important caveat. Recall the introduction to Section
\ref{sec:defs} in the appendix; specifically, how we defined the operator
$J_\ee$. By construction, the proofs of Remark \ref{lem5r}, Remark
\ref{lem5r'}, and Proposition \ref{lem4r} for the non-periodic case will be
analogous to those in the periodic case. Hence, how we
construct the mollifier $J_\ee$ plays a critical role in the proofs of
well-posedness for the HR i.v.p. in both the periodic and non-periodic cases. %
%

%
%
\backmatter
\bibliography{/Users/davidkarapetyan/math/bib-files/references}
%
\end{document}
