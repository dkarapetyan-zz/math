\documentclass[12pt,reqno]{amsart}
\usepackage{amssymb}
\usepackage{cancel}  %for cancelling terms explicity on pdf
\usepackage{yhmath}   %makes fourier transform look nicer, among other things
\usepackage[alphabetic, msc-links]{amsrefs} %for the bibliography; uses cite pkg
%\usepackage[notcite, notref]{showkeys}
\usepackage[margin=3cm]{geometry}  %page layout
%\usepackage[pdftex]{graphicx} %for importing pictures into latex--pdf compilation
\setcounter{secnumdepth}{1} %number only sections, not subsections
\hypersetup{colorlinks=true,
linkcolor=blue,
citecolor=blue,
urlcolor=blue,
}
\synctex=1
\numberwithin{equation}{section}  %eliminate need for keeping track of counters
\numberwithin{figure}{section}
\setlength{\parindent}{0in} %no indentation of paragraphs after section title
\renewcommand{\baselinestretch}{1.1} %increases vert spacing of text
%
\newcommand{\ds}{\displaystyle}
\newcommand{\ts}{\textstyle}
\newcommand{\nin}{\noindent}
\newcommand{\rr}{\mathbb{R}}
\newcommand{\nn}{\mathbb{N}}
\newcommand{\zz}{\mathbb{Z}}
\newcommand{\cc}{\mathbb{C}}
\newcommand{\ci}{\mathbb{T}}
\newcommand{\zzdot}{\dot{\zz}}
\newcommand{\wh}{\widehat}
\newcommand{\p}{\partial}
\newcommand{\ee}{\varepsilon}
\newcommand{\vp}{\varphi}
\newcommand{\quod}{\qquad \qedsymbol}
%
%
\theoremstyle{plain}  
\newtheorem{theorem}{Theorem}
\newtheorem{proposition}{Proposition}
\newtheorem{lemma}{Lemma}
\newtheorem{corollary}{Corollary}
\newtheorem{claim}{Claim}
\newtheorem{conjecture}[subsection]{conjecture}
%
\theoremstyle{definition}
\newtheorem{definition}{Definition}
%
\theoremstyle{remark}
\newtheorem{remark}{Remark}
%
%
%
\def\makeautorefname#1#2{\expandafter\def\csname#1autorefname\endcsname{#2}}
\makeautorefname{equation}{Equation}
\makeautorefname{footnote}{footnote}
\makeautorefname{item}{item}
\makeautorefname{figure}{Figure}
\makeautorefname{table}{Table}
\makeautorefname{part}{Part}
\makeautorefname{appendix}{Appendix}
\makeautorefname{chapter}{Chapter}
\makeautorefname{section}{Section}
\makeautorefname{subsection}{Section}
\makeautorefname{subsubsection}{Section}
\makeautorefname{paragraph}{Paragraph}
\makeautorefname{subparagraph}{Paragraph}
\makeautorefname{theorem}{Theorem}
\makeautorefname{theo}{Theorem}
\makeautorefname{thm}{Theorem}
\makeautorefname{addendum}{Addendum}
\makeautorefname{add}{Addendum}
\makeautorefname{maintheorem}{Main theorem}
\makeautorefname{corollary}{Corollary}
\makeautorefname{lemma}{Lemma}
\makeautorefname{sublemma}{Sublemma}
\makeautorefname{proposition}{Proposition}
\makeautorefname{property}{Property}
\makeautorefname{scholium}{Scholium}
\makeautorefname{step}{Step}
\makeautorefname{conjecture}{Conjecture}
\makeautorefname{question}{Question}
\makeautorefname{definition}{Definition}
\makeautorefname{notation}{Notation}
\makeautorefname{remark}{Remark}
\makeautorefname{remarks}{Remarks}
\makeautorefname{example}{Example}
\makeautorefname{algorithm}{Algorithm}
\makeautorefname{axiom}{Axiom}
\makeautorefname{case}{Case}
\makeautorefname{claim}{Claim}
\makeautorefname{assumption}{Assumption}
\makeautorefname{conclusion}{Conclusion}
\makeautorefname{condition}{Condition}
\makeautorefname{construction}{Construction}
\makeautorefname{criterion}{Criterion}
\makeautorefname{exercise}{Exercise}
\makeautorefname{problem}{Problem}
\makeautorefname{solution}{Solution}
\makeautorefname{summary}{Summary}
\makeautorefname{operation}{Operation}
\makeautorefname{observation}{Observation}
\makeautorefname{convention}{Convention}
\makeautorefname{warning}{Warning}
\makeautorefname{note}{Note}
\makeautorefname{fact}{Fact}
%
\begin{document}
%\begin{titlepage}
\title{Report on ``A model containing both the Camassa-Holm and
Degasperis-Procesi equations''}
\author{David Karapetyan \\ University of Notre Dame}
\address{Department of Mathematics  \\
         University  of Notre Dame\\
         Notre Dame, IN 46556 }
				  \date{08/04/10}
				  %
				  \maketitle
				  %
				  %
				  \parindent0in
				  \parskip0.1in
				  %
				  %\end{titlepage}
				  %
				  %
				  %
				  \setcounter{equation}{0}
				  The paper in question deals with a generalized 
				  Camassa-Holm and Degasperis-Procesi (gCD) initial 
				  value problem%
%
\begin{gather}
	\label{gch}
	u_t - u_{xxt} + 2ku_x + mu u_x=au_x u_{xx} + buu_{xxx}, 		\\
		\label{gch-init-data}
		u(x,0) = u_0(x), \quad x \in \rr,
\end{gather}
%
%
where $a,b,k,m$ are constants. Note that for $m=4, a=3, b=1$ one obtains the
Degasperis-Procesi equation
\cite{Degasperis-Procesi-1999-Asymptotic-integrability}, while $k =0, m=3, a=2,
b=1$ yields the Camassa Holm equation
\cite{Camassa-Holm-1993-An-integrable-shallow-water}. The main result consists of two parts: a proof of local existence and uniqueness of solutions $u(x,t) \in C([0, T], H^s(\rr)) \cap 
C^1([0, T], H^{s-1}(\rr))$, $s > 3/2$
given initial data $u_0 \in H^s$, 
where $T = T(\|u_0\|_{H^s})$, and a proof of the existence of weak 
solutions $u(x,t) \in L^2([0,T], H^s(\rr))$ for $1 < s \le 3/2$ if $u_0 \in 
H^s(\rr)$ and $u_{0x} \in L^\infty(\rr)$. The authors prove well-posedness
for $s>3/2$ by verifying that equation \eqref{gch} satisfies the
requirements of Kato's semi-group theory
\cite{Kato-1975-Quasi-linear-equations-of-evolution}. For the proof of the existence of weak
solutions for $1 <s \le 3/2$, the authors consider \eqref{gch}, with initial data given by a mollification of
\eqref{gch-init-data}
%
	\begin{gather}
	\label{gch-regular}
	u_t - u_{xxt} + 2ku_x + mu u_x=au_x u_{xx} + buu_{xxx}, 		\\
		\label{gch-init-data-moll}
		u(x,0) = u_{\ee 0} (x), \quad x \in \rr.
	\end{gather}
	They then show that there exists $T>0$ such that if
	$\|u_{0x}\|_{L^\infty} <\infty$, then the family
$\{\p_x u_{\ee} \}$ is uniformly bounded in $L^\infty$ for $t \in [0, T]$. 
Using this fact, they are then able to bound the $H^r$ norms of $u_{\ee}$ and
$u_{\ee t}$, for $r \in (0, s-1)$. An application of Aubin's and Aloaglu's
compactness lemmas then completes the proof.
\\
\\
Equation \eqref{gch} first appeared in the work
of Constantin and Lannes
\cite{Constantin-Lannes-2009-The-hydrodynamical-relevance-of-the-Camassa-Holm}
on the Camassa-Holm \cite{Camassa-Holm-1993-An-integrable-shallow-water} and
Degasperis-Procesi \cite{Degasperis-Procesi-1999-Asymptotic-integrability}
equations. 
Unlike the Camassa-Holm and Degasperis-Procesi equations, \eqref{gch} is not
integrable. However, for $a =2b$, the $H^1$ norm of
solutions are conserved, as in the Camassa-Holm equation. For the general case, the authors show that, for initial data $u_0 \in H^s, s \ge4$, the
following holds 
%
%
\begin{equation}
	\label{semi-conservation-law}
	\begin{split}
		\|u\|_{H^1}^2 \le \int_{\rr}\left( u_0^2 + u_{0x}^2 \right)dx + | a - 2b
		|\int_{0}^t \|u_x\|_{L^\infty} \|u\|_{H^1}^2 d \tau
	\end{split}
\end{equation}
%
%
which plays an important role in the proof of existence for weak solutions for
$1 < s \le 3/2$. While the techniques are not new, the argumentation is clear,
and proper credit is given to other authors where needed. Since the
gCD ivp \eqref{gch}-\eqref{gch-init-data} subsumes both the Camassa-Holm
and Degasperis-Procesi equations, it is important in its own right.
Therefore, the reader recommends this paper for publication, pending the
following comments and suggestions: %
%
\begin{enumerate}
	\item Three theorems are stated, but it is clear that the main results of 
		the paper are Theorem 1 and Theorem 3. Theorem 2 should be renamed as a
		lemma or proposition, as it is used to prove Theorem 3 and is not important
		enough in its own right. 
		\item Due to the spacing, it is unclear at times when a proof is complete.
			A qed symbol at the end of a proof or adjusting spacing would
			improve the clarity of the article. 
		\item In the section containing the proofs of Theorem 2 and Theorem 3,
			it would aid the reader if lemmas were introduced at the precise moment
			they are needed, rather than being introduced en masse at the beginning of
			the section.
\end{enumerate}
	% \bib, bibdiv, biblist are defined by the amsrefs package.
\begin{bibdiv}
\begin{biblist}

\bib{Camassa-Holm-1993-An-integrable-shallow-water}{article}{
      author={Camassa, Roberto},
      author={Holm, Darryl~D.},
       title={An integrable shallow water equation with peaked solitons},
        date={1993},
        ISSN={0031-9007},
     journal={Phys. Rev. Lett.},
      volume={71},
      number={11},
       pages={1661\ndash 1664},
         url={http://dx.doi.org/10.1103/PhysRevLett.71.1661},
      review={\MR{MR1234453 (94f:35121)}},
}

\bib{Constantin-Lannes-2009-The-hydrodynamical-relevance-of-the-Camassa-Holm}{%
article}{
      author={Constantin, Adrian},
      author={Lannes, David},
       title={The hydrodynamical relevance of the {C}amassa-{H}olm and
  {D}egasperis-{P}rocesi equations},
        date={2009},
        ISSN={0003-9527},
     journal={Arch. Ration. Mech. Anal.},
      volume={192},
      number={1},
       pages={165\ndash 186},
         url={http://dx.doi.org/10.1007/s00205-008-0128-2},
      review={\MR{MR2481064 (2010f:35334)}},
}

\bib{Degasperis-Procesi-1999-Asymptotic-integrability}{incollection}{
      author={Degasperis, A.},
      author={Procesi, M.},
       title={Asymptotic integrability},
        date={1999},
   booktitle={Symmetry and perturbation theory ({R}ome, 1998)},
   publisher={World Sci. Publ., River Edge, NJ},
       pages={23\ndash 37},
      review={\MR{MR1844104 (2002f:37112)}},
}

\bib{Kato-1975-Quasi-linear-equations-of-evolution}{incollection}{
      author={Kato, Tosio},
       title={Quasi-linear equations of evolution, with applications to partial
  differential equations},
        date={1975},
   booktitle={Spectral theory and differential equations ({P}roc. {S}ympos.,
  {D}undee, 1974; dedicated to {K}onrad {J}{\"o}rgens)},
   publisher={Springer},
     address={Berlin},
       pages={25\ndash 70. Lecture Notes in Math., Vol. 448},
      review={\MR{MR0407477 (53 \#11252)}},
}

\end{biblist}
\end{bibdiv}	
%\bibliography{/Users/davidkarapetyan/Documents/math/bib-files/review.bib}


				  \end{document}


