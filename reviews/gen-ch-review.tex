%
\documentclass[12pt,reqno]{amsart}
\usepackage{amscd}
\usepackage{amsfonts}
\usepackage{amsmath}
\usepackage{amssymb}
\usepackage{amsthm}
\usepackage{appendix}
\usepackage{fancyhdr}
\usepackage{latexsym}
\usepackage{cancel}
\usepackage{amsxtra}
\synctex=1
\usepackage[colorlinks=true, pdfstartview=fitv, linkcolor=blue,
citecolor=blue, urlcolor=blue]{hyperref}
\input epsf
\input texdraw
\input txdtools.tex
\input xy
\xyoption{all}
%%%%%%%%%%%%%%%%%%%%%%
\usepackage{color}
\definecolor{red}{rgb}{1.00, 0.00, 0.00}
\definecolor{darkgreen}{rgb}{0.00, 1.00, 0.00}
\definecolor{blue}{rgb}{0.00, 0.00, 1.00}
\definecolor{cyan}{rgb}{0.00, 1.00, 1.00}
\definecolor{magenta}{rgb}{1.00, 0.00, 1.00}
\definecolor{deepskyblue}{rgb}{0.00, 0.75, 1.00}
\definecolor{darkgreen}{rgb}{0.00, 0.39, 0.00}
\definecolor{springgreen}{rgb}{0.00, 1.00, 0.50}
\definecolor{darkorange}{rgb}{1.00, 0.55, 0.00}
\definecolor{orangered}{rgb}{1.00, 0.27, 0.00}
\definecolor{deeppink}{rgb}{1.00, 0.08, 0.57}
\definecolor{darkviolet}{rgb}{0.58, 0.00, 0.82}
\definecolor{saddlebrown}{rgb}{0.54, 0.27, 0.07}
\definecolor{black}{rgb}{0.00, 0.00, 0.00}
\definecolor{dark-magenta}{rgb}{.5,0,.5}
\definecolor{myblack}{rgb}{0,0,0}
\definecolor{darkgray}{gray}{0.5}
\definecolor{lightgray}{gray}{0.75}
%%%%%%%%%%%%%%%%%%%%%%
%%%%%%%%%%%%%%%%%%%%%%%%%%%%
%  for importing pictures  %
%%%%%%%%%%%%%%%%%%%%%%%%%%%%
\usepackage[pdftex]{graphicx}
\usepackage{epstopdf}
% \usepackage{graphicx}
%% page setup %%
\setlength{\textheight}{20.8truecm}
\setlength{\textwidth}{14.8truecm}
\marginparwidth  0truecm
\oddsidemargin   01truecm
\evensidemargin  01truecm
\marginparsep    0truecm
\renewcommand{\baselinestretch}{1.1}
%% new commands %%
\newcommand{\tf}{\tilde{f}}
\newcommand{\ti}{\tilde}
\newcommand{\bigno}{\bigskip\noindent}
\newcommand{\ds}{\displaystyle}
\newcommand{\medno}{\medskip\noindent}
\newcommand{\smallno}{\smallskip\noindent}
\newcommand{\nin}{\noindent}
\newcommand{\ts}{\textstyle}
\newcommand{\rr}{\mathbb{R}}
\newcommand{\p}{\partial}
\newcommand{\zz}{\mathbb{Z}}
\newcommand{\cc}{\mathbb{C}}
\newcommand{\ci}{\mathbb{T}}
\newcommand{\ee}{\varepsilon}
\newcommand{\vp}{\varphi}
\def\autorefer #1\par{\noindent\hangindent=\parindent\hangafter=1 #1\par}
%% equation numbers %%
\renewcommand{\theequation}{\thesection.\arabic{equation}}
%% new environments %%
%\swapnumbers
\theoremstyle{plain}  % default
\newtheorem{theorem}{Theorem}
\newtheorem{proposition}{Proposition}
\newtheorem{lemma}{Lemma}
\newtheorem{corollary}{Corollary}
\newtheorem{claim}{Claim}
\newtheorem{remark}{Remark}
\newtheorem{conjecture}[subsection]{conjecture}
\newtheorem{definition}{Definition}
\def\makeautorefname#1#2{\expandafter\def\csname#1autorefname\endcsname{#2}}
\makeautorefname{equation}{Equation}
\makeautorefname{footnote}{footnote}
\makeautorefname{item}{item}
\makeautorefname{figure}{Figure}
\makeautorefname{table}{Table}
\makeautorefname{part}{Part}
\makeautorefname{appendix}{Appendix}
\makeautorefname{chapter}{Chapter}
\makeautorefname{section}{Section}
\makeautorefname{subsection}{Section}
\makeautorefname{subsubsection}{Section}
\makeautorefname{paragraph}{Paragraph}
\makeautorefname{subparagraph}{Paragraph}
\makeautorefname{theorem}{Theorem}
\makeautorefname{theo}{Theorem}
\makeautorefname{thm}{Theorem}
\makeautorefname{addendum}{Addendum}
\makeautorefname{addend}{Addendum}
\makeautorefname{add}{Addendum}
\makeautorefname{maintheorem}{Main theorem}
\makeautorefname{mainthm}{Main theorem}
\makeautorefname{corollary}{Corollary}
\makeautorefname{corol}{Corollary}
\makeautorefname{coro}{Corollary}
\makeautorefname{cor}{Corollary}
\makeautorefname{lemma}{Lemma}
\makeautorefname{lemm}{Lemma}
\makeautorefname{lem}{Lemma}
\makeautorefname{sublemma}{Sublemma}
\makeautorefname{sublem}{Sublemma}
\makeautorefname{subl}{Sublemma}
\makeautorefname{proposition}{Proposition}
\makeautorefname{proposit}{Proposition}
\makeautorefname{propos}{Proposition}
\makeautorefname{propo}{Proposition}
\makeautorefname{prop}{Proposition}
\makeautorefname{proposition}{Proposition}
\makeautorefname{property}{Property}
\makeautorefname{proper}{Property}
\makeautorefname{scholium}{Scholium}
\makeautorefname{step}{Step}
\makeautorefname{conjecture}{Conjecture}
\makeautorefname{conject}{Conjecture}
\makeautorefname{conj}{Conjecture}
\makeautorefname{question}{Question}
\makeautorefname{questn}{Question}
\makeautorefname{quest}{Question}
\makeautorefname{ques}{Question}
\makeautorefname{qn}{Question}
\makeautorefname{definition}{Definition}
\makeautorefname{defin}{Definition}
\makeautorefname{defi}{Definition}
\makeautorefname{def}{Definition}
\makeautorefname{dfn}{Definition}
\makeautorefname{notation}{Notation}
\makeautorefname{nota}{Notation}
\makeautorefname{notn}{Notation}
\makeautorefname{remark}{Remark}
\makeautorefname{rema}{Remark}
\makeautorefname{rem}{Remark}
\makeautorefname{rmk}{Remark}
\makeautorefname{rk}{Remark}
\makeautorefname{remarks}{Remarks}
\makeautorefname{rems}{Remarks}
\makeautorefname{rmks}{Remarks}
\makeautorefname{rks}{Remarks}
\makeautorefname{example}{Example}
\makeautorefname{examp}{Example}
\makeautorefname{exmp}{Example}
\makeautorefname{exam}{Example}
\makeautorefname{exa}{Example}
\makeautorefname{algorithm}{Algorith}
\makeautorefname{algo}{Algorith}
\makeautorefname{alg}{Algorith}
\makeautorefname{axiom}{Axiom}
\makeautorefname{axi}{Axiom}
\makeautorefname{ax}{Axiom}
\makeautorefname{case}{Case}
\makeautorefname{claim}{Claim}
\makeautorefname{clm}{Claim}
\makeautorefname{assumption}{Assumption}
\makeautorefname{assumpt}{Assumption}
\makeautorefname{conclusion}{Conclusion}
\makeautorefname{concl}{Conclusion}
\makeautorefname{conc}{Conclusion}
\makeautorefname{condition}{Condition}
\makeautorefname{condit}{Condition}
\makeautorefname{cond}{Condition}
\makeautorefname{construction}{Construction}
\makeautorefname{construct}{Construction}
\makeautorefname{const}{Construction}
\makeautorefname{cons}{Construction}
\makeautorefname{criterion}{Criterion}
\makeautorefname{criter}{Criterion}
\makeautorefname{crit}{Criterion}
\makeautorefname{exercise}{Exercise}
\makeautorefname{exer}{Exercise}
\makeautorefname{exe}{Exercise}
\makeautorefname{problem}{Problem}
\makeautorefname{problm}{Problem}
\makeautorefname{probm}{Problem}
\makeautorefname{prob}{Problem}
\makeautorefname{solution}{Solution}
\makeautorefname{soln}{Solution}
\makeautorefname{sol}{Solution}
\makeautorefname{summary}{Summary}
\makeautorefname{summ}{Summary}
\makeautorefname{sum}{Summary}
\makeautorefname{operation}{Operation}
\makeautorefname{oper}{Operation}
\makeautorefname{observation}{Observation}
\makeautorefname{observn}{Observation}
\makeautorefname{obser}{Observation}
\makeautorefname{obs}{Observation}
\makeautorefname{ob}{Observation}
\makeautorefname{convention}{Convention}
\makeautorefname{convent}{Convention}
\makeautorefname{conv}{Convention}
\makeautorefname{cvn}{Convention}
\makeautorefname{warning}{Warning}
\makeautorefname{warn}{Warning}
\makeautorefname{note}{Note}
\makeautorefname{fact}{Fact}
%
\begin{document}
%\begin{titlepage}
\title{Report on ``The local well-posedness and existence of weak solutions  
for a generalized Camassa-Holm equation''}
\author{David Karapetyan \\ University of Notre Dame}
\address{Department of Mathematics  \\
         University  of Notre Dame\\
         Notre Dame, IN 46556 }
				  \date{11/18/09}
				  %
				  \maketitle
				  %
				  %
				  \parindent0in
				  \parskip0.1in
				  %
				  %\end{titlepage}
				  %
				  %
				  %
				  \setcounter{equation}{0}
				  The paper in question deals with a generalized 
				  Camassa-Holm (gCH) initial 
				  value problem%
%
\begin{gather}
	\label{gch}
		u_t - u_{xxt} + 2k u_x + au^m = \big( nu^{n-1} 
		\frac{u_x^2}{2} + u^n u_{xx} \big)_x + \beta \p_x \left 
		[(u_x)^{2N-1} \right ],
		\\
		\label{gch-init-data}
		u(x,0) = u_0(x), \quad x \in \rr,
\end{gather}
%
%
where $m \ge 1$, $n\ge1$, and $N \ge 1$ are natural numbers, and $a$, $k$, and 
$\beta \ge 0$ are real constants. The main result consists of two parts: a proof of local existence and uniqueness of solutions $u(x,t) \in C([0, T], H^s(\rr)) \cap 
C^1([0, T], H^{s-1}(\rr))$, $s > 3/2$, for gCH 
given initial data $u_0 \in H^s$, 
where $T = T(\|u_0\|_{H^s})$, and a proof of the existence of weak 
solutions $u(x,t) \in L^2([0,T], H^s(\rr))$ for $1 < s \le 3/2$ if $u_0 \in 
H^s(\rr)$ and $u_{0x} \in L^\infty(\rr)$. The method of proof relies upon an elliptic 
regularization of \eqref{gch}, with initial data given by a mollification 
of \eqref{gch-init-data}
%
%
	\begin{gather}
	\label{gch-regular}
	u_t - u_{xxt} + 2k u_x + au^m + \ee u_{xxxxt} = \big( nu^{n-1} 
		\frac{u_x^2}{2} + u^n u_{xx} \big)_x + \beta \p_x \left 
		[(u_x)^{2N-1} \right ],
		\\
		\label{gch-init-data-moll}
		u(x,0) = u_{\ee 0} (x), \quad x \in \rr.
\end{gather}
Using a Banach fixed point argument, the authors prove the existence of a 
unique solution $u_\ee(x,t) \in C^\infty([0, \infty], 
H^\infty(\rr))$ to the i.v.p 
\eqref{gch-regular}-\eqref{gch-init-data-moll}. They then prove the 
existence of a uniform bound for the family 
$\{u_\ee\}$ for $0 \le t < T$, which is then used to show that the solutions $\{u_\ee
\}$ are Cauchy in the space $C([0,T], H^s(\rr)) \cap C^1(\left[0,T \right], 
H^{s-1}(\rr))$. Taking $\ee \to 0$ in \eqref{gch-regular}, the main result 
follows.

The motivation for the paper is the work
of Hakkaev and Kirchev \cite{Hakkaev_Kirchev-Local-well-pose} and 
Tian and Song \cite{Tian_Song-New-peaked-soli} on different generalized 
versions of the Camassa-Holm. The version studied in 
\cite{Tian_Song-New-peaked-soli}
%
%
\begin{equation*}
	\begin{split}
		u_t + 2ku_x -u_{xxt} + au^m u_x = 2u_x u_{xx} + u u_{xxx}
	\end{split}
\end{equation*}
%
%
with $a > 0$, $k \in \rr$, and $m \in \mathbb{N}$ has a higher order nonlinear term  
than the Camassa-Holm, giving rise to new traveling wave, peakon, and 
solitary wave solutions. In particular, compacton solutions arise. 
Given this background on a related generalized Camassa-Holm equation, it is unclear 
what the additional merits of gCH are; in particular,  
what purpose does the modified non-linear dissipative term $\beta \p_x 
[(u_x)^{2N-1}]$
serve, other than to create a new generalized Camassa-Holm equation? This 
is a key detail, and should be emphasized in a revision of the 
introduction. Other comments and suggestions for improving the paper:
%
%
\begin{enumerate}
	\item Five theorems are stated, but it is clear that the main result of 
		the paper is a proof of local existence and uniqueness of solutions 
		for gCH for $s > 3/2$ 
		and the existence of weak solutions for $1 < s \le 3/2$. This 
		should be stated immediately after the introduction. 
	\item An outline of the proof of the main result is in order.
		For example, it is not stated clearly 
		that theorems 1 and 4
		serve merely as tools to prove Theorem 5 (local well-posedness for $s 
		> 3/2$), while Theorem 2 is a tool to prove Theorem 3 (existence of weak 
		solutions for $1 < s \le 3/2$). These preparatory theorems should be 
		renamed as propositions or lemmas, as they are not the focal 
		points of the paper. Grouping theorems 1,4, and 5 into one section   
		dealing with local well-posedness for $s >3/2$, and theorems 2 and 
		3 into a different section dealing with local well-posedness for 
		lower Sobolev indexes would improve the flow and clarity of the 
		paper.
	\item Rather than stating a series of theorems and lemmas en masse 
		before proceeding with the proof of the main result, it would be 
		clearer to begin with the proof of the main result, and then 
		introduce theorems and lemmas when needed. 
	\item The results of this paper and the techniques used to obtain them
		mirror those of Li and Olver 
		\cite{Li_2000_Well-posedness-} and 
		\cite{Hakkaev_Kirchev-Local-well-pose}, 
		yet not enough credit is given. In particular, the organization of 
		the proofs of all five theorems is very similar to the organization
		of the proofs found 
		in the references above. 
	\item The spacing between equations and paragraphs can be reduced to 
		make the paper more readable.

\end{enumerate}
		
		
	\begin{thebibliography}{1}
\newcommand{\enquote}[1]{``#1''}

\bibitem{Hakkaev_Kirchev-Local-well-pose}
\textsc{S.~Hakkaev and K.~Kirchev}.
\newblock \enquote{Local well-posedness and orbital stability of solitary wave
  solutions for the generalized {C}amassa-{H}olm equation.}
\newblock \emph{Comm. Partial Differential Equations}, \textbf{30} (2005), no.
  4-6, 761--781.

\bibitem{Li_2000_Well-posedness-}
\textsc{Y.~A. Li and P.~J. Olver}.
\newblock \enquote{Well-posedness and blow-up solutions for an integrable
  nonlinearly dispersive model wave equation.}
\newblock \emph{J. Differential Equations}, \textbf{162} (2000), no.~1, 27--63.

\bibitem{Tian_Song-New-peaked-soli}
\textsc{L.~Tian and X.~Song}.
\newblock \enquote{New peaked solitary wave solutions of the generalized
  {C}amassa-{H}olm equation.}
\newblock \emph{Chaos Solitons Fractals}, \textbf{19} (2004), no.~3, 621--637.

\end{thebibliography}	
				  
			  
				  

				  





				  \end{document}


